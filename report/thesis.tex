\documentclass[
    % final, % any \todo{<x>} commands are removed
    % print, % removes link color for print version
    % german % to switch language to German
]{thesis}

% A lot of important packages are loaded in the thesis.cls file, check first whether they are already loaded if you need a package.
% \usepackage{...}

% Change this to your needs.
\newcommand{\university}{Otto-von-Guericke University Magdeburg}
\newcommand{\school}{Faculty of Computer Science}
\newcommand{\logo}{\includegraphics[trim=0mm 0mm 50mm 0mm,clip,height=3cm]{ovgu}}
\newcommand{\advisorone}{Marco Filax}
\newcommand{\departmentone}{Chair of Software Engineering}
\newcommand{\universityone}{Otto-von-Guericke University, Magdeburg}
\newcommand{\advisortwo}{Konstantin Kirchheim}
\newcommand{\departmenttwo}{Chair of Software Engineering}
\newcommand{\universitytwo}{Otto-von-Guericke University, Magdeburg}
\newcommand{\supervisorone}{Prof.Dr.Frank Ortmeier}
\newcommand{\sdepartmentone}{Chair of Software Engineering}
\newcommand{\suniversityone}{Otto-von-Guericke University, Magdeburg}
\newcommand{\supervisortwo}{xxx}
\newcommand{\sdepartmenttwo}{xxx xxx}
\newcommand{\suniversitytwo}{Otto-von-Guericke University, Magdeburg}
\newcommand{\advisorthree}{}
\newcommand{\departmentthree}{}
\newcommand{\universitythree}{}
\newcommand{\thesiskind}{Master's Thesis}
\newcommand{\theforename}{Kartik}
\newcommand{\thesurname}{Prabhu}
\newcommand{\thetitle}{Synth2Real : 3D-Furniture Reconstruction in Ersatz Environment \\ (S2R:3D-FREE)}
\newcommand{\thedate}{October 15, 2021}
\newcommand{\theyear}{2021}
\newcommand{\signatureplace}{Magdeburg}
\newcommand{\signaturedate}{15th October 2021}

% Here, you can define acronyms for consistent usage in your text like this:
% \acro{key}{XYZ}{Correctly Capitalized Acronym}{all lowercase acronym}
% cite with \gls{key}, \Gls{key} (at the start of a sentence), \glspl{key} (plural form), \gls{key} (plural at start of sentence)
%% Only the acronyms you actually use in the text are shown in the list of acronyms.
%\newcommand{\acro}[4]{\newacronym[description=#3]{#1}{#2}{#4}}
%\newcommand{\makeacronyms}{
%    \acro{3dfree}){S2R:3DFREE}{Synth2Real : 3D-Furniture Reconstruction in Ersatz Environment}
%    \acro{mse}{MSE}{Mean Squared Error}
%    \acro{fid}{FID}{Fréchet Inception Distance}
%    % ...
%}

\newacronym{mse}{MSE}{Mean Squared Error}
\newacronym{free}{S2R:3DFREE}{Synth2Real : 3D-Furniture Reconstruction in Ersatz Environment}
\newacronym{fid}{FID}{Fréchet Inception Distance}
\newacronym{front}{3DFRONT}{3D Furnished Rooms with layOuts and semaNTics}
\newacronym{ai2thor}{AI2THOR}{Artificial Intelligence to The House Of inteRactions}
\newacronym{front}{3DFRONT}{3D Furnished Rooms with layOuts and semaNTics}
\newacronym{rnn}{RNN}{Recurrent Neural Networks}
\newacronym{cnn}{CNN}{Convolutional Neural Networks}
\newacronym{vgg}{VGG}{Visual Geometry Group}
\newacronym{resnet}{ResNet}{Residual neural network}
\newacronym{iou}{IoU}{Intersection over Union}
\newacronym{hdrp}{HDRP}{High Definition Render Pipeline}
\newacronym{urp}{URP}{Universal Render Pipeline}
\newacronym{s2rv1}{S2R\_V1}{S2R:3DFREE Version 1}
\newacronym{s2rv2}{S2R\_V2}{S2R:3DFREE Version 2}
\newacronym{tsne}{t-SNE}{t-Distributed Stochastic Neighbor Embedding }
\newacronym{f1}{F1 Score}{Dice Coefficient/Dice Score/F Score}
% For some theses a list of symbols can be nice. For example usage, see below.
\newcommand{\category}[3]{\newglossaryentry{_#2}{type=symbol, name={#3}, description={}, sort={#1}}}
\newcommand{\sym}[5]{\newglossaryentry{#3}{type=symbol, name={\ensuremath{#4}}, sort={#2}, description={#5}, parent=_#1}}
%\newcommand{\makesymbols}{
%    \category{1}{general}{General}
%    \category{2}{logic}{First-Order and Dynamic Logic}
%    \sym{general}{1}{subst}{\alpha[x \backslash y]}{Substitution of $x$ in $\alpha$ with $y$}
%    \sym{general}{2}{power}{\mathcal{P}(A)}{Power set of a set $A$}
%    \sym{general}{3}{restrict}{f|_{A'}}{Restriction of a function $f \colon A \to B$ to a smaller domain $A' \subseteq A$}
%    \sym{logic}{1}{pand}{A \pand B}{Formalization of the sentence ``\emph{P} and \emph{Q}''}
%    % ...
%}

% These are custom commands (or macros). They are useful for example for mathematical notations that you use often.
\newcommand{\todots}{\todo{[...]}} % mark as todo
\newcommand{\eg}{e.g.,\ } % abbreviates "for example, ..."
\newcommand{\ie}{i.e.,\ } % abbreviates "that is, ..."
\newcommand{\cf}[1]{(cf.~\autoref{#1})} % abbreviates "(confer ...)"
\newcommand{\myparagraph}[1]{\noindent \textit{\biolinum #1}.} % starts a little subparagraph
\newcommand{\m}[1]{\ensuremath{\mathit{#1}}} % write some text both in- and outside of math mode
\newcommand{\pand}{\wedge} % logical conjunction
\newcommand{\por}{\vee} % logical disjunction
\newcommand{\pnot}{\neg} % logical negation
\newcommand{\pequals}{\leftrightarrow} % logical equality
\newcommand{\pimplies}{\rightarrow} % logical implication
\newcommand{\defeq}{\vcentcolon=} % defining equals sign
\newcommand{\RQ}[1]{\ensuremath{\text{RQ}_{#1}}} % research question

% document
\hypersetup{pdfauthor={\theforename\ \thesurname}, pdftitle={\thetitle}}
\makeindex
\makeglossaries
\makeacronyms
\makesymbols

\begin{document}

\frontmatter
\renewcommand{\proofname}{\itshape\biolinum{Proof}} % only relevant if you use the proof environment
\newcommand{\theauthor}{\theforename\ \thesurname}
\newcommand{\theauthorrev}{\thesurname,\ \theforename}
\graphicspath{{files/}}
\pagenumbering{roman}

% the first page - you can change this by setting the commands at the beginning of this file
\begin{titlepage}
    \thispagestyle{empty}
    \begin{center}
        {\university}\\[0.4cm]
        {\school}\\[2.0cm]
        \hbox{}\hfill
        \begin{minipage}[t]{\textwidth}
            \begin{center}
                \logo
            \end{center}
        \end{minipage}
        \hfill\hbox{}
        \ \\[0.4cm]
        {\large \thesiskind \\[1cm]}
        {\LARGE\bfseries\biolinum \thetitle \\[1cm]}
        {\iftoggle{german}{Autor}{Author}:}\\[0.4cm]
        {\large \theauthor}\\[0.8cm]
        {\large\thedate}\\[0.8cm]

    \vspace{2cm}
    \renewcommand{\arraystretch}{.9}
    \begin{tabular}{cc}
        \multicolumn{2}{c}{\small Advisers:} \\[1mm]
%        {\small Supervisor} & {\small Supervisor} \\
        {\large \advisorone} & {\large \advisortwo} \\[2mm]
        {\small \departmentone} 			 & {\small \departmentone} \\
        {\small \universityone} 			 & {\small \universitytwo} 	\\
%        {\small \myunistreet} 		 & {\small \myunistreet} \\
%        {\small \myunizipcity}		 & {\small \myunizipcity}
    \end{tabular}

    \vspace{1cm}

    \begin{tabular}{cc}
        \multicolumn{2}{c}{\small Supervisors:} \\[1mm]
%        {\small Supervisor} & {\small Supervisor} \\
        {\large \supervisorone} & {\large \supervisortwo} \\[2mm]
        {\small \sdepartmentone} 			 & {\small \sdepartmenttwo} \\
        {\small \suniversityone} 			 & {\small \suniversitytwo} 	\\
%        {\small \myunistreet} 		 & {\small \myunistreet} \\
%        {\small \myunizipcity}		 & {\small \myunizipcity}
    \end{tabular}

    \renewcommand{\arraystretch}{1}
    \end{center}
    \vspace*{\fill}
%    {\iftoggle{german}{Betreuer}{Advisors}:}\\[0.3cm]
%    \ifdefempty{\advisorone}{}{{\large\advisorone}\\[0.1cm]}
%    \ifdefempty{\departmentone}{}{{\departmentone}\\[0.1cm]}
%    \ifdefempty{\universityone}{}{{\universityone}\\[0.3cm]}
%    \ifdefempty{\advisortwo}{}{{\large\advisortwo}\\[0.1cm]}
%    \ifdefempty{\departmenttwo}{}{{\departmenttwo}\\[0.1cm]}
%    \ifdefempty{\universitytwo}{}{{\universitytwo}\\[0.3cm]}
%    \ifdefempty{\advisorthree}{}{{\large\advisorthree}\\[0.1cm]}
%    \ifdefempty{\departmentthree}{}{{\departmentthree}\\[0.1cm]}
%    \ifdefempty{\universitythree}{}{{\universitythree}}
\end{titlepage}

% the second page
\thispagestyle{empty}
\vspace*{\fill}
\begin{minipage}{15.0cm}
    \textbf{\theauthorrev:}\\
    \emph{\thetitle\\}
    \thesiskind, \university, \theyear.
\end{minipage}
\newpage

\iftoggle{german}{\chapter*{Zusammenfassung}}{\chapter*{Abstract}}

The field of Deep Learning is growing exponentially and has tremendous applications in every domain.
The key for training deep learning models is a large dataset.
The real-world dataset is cost-ineffective, time-consuming, difficult to obtain, and are very limited.
Synthetic data are easier to create and automate.
This thesis aims to check if images created using game engines(Unity) are photorealistic and valuable in  3D reconstruction tasks using Deep Neural Networks.
To achieve this, we contribute the following:
(a)A Unity-based application to create synthetic images of furniture in an indoor environment.
(b)We use the new synthetic dataset as a benchmark for the 3D reconstruction task.
(c) A comprehensive study on domain adaptation and domain randomization.

We conducted a user survey with the proposed synthetic dataset, a real dataset, and seven other proclaimed photorealistic synthetic datasets to check the photorealism.
We see that the proposed dataset got the most votes to be 'Real' among the automated datasets.
We compare the performance of baseline models with transfer learning and mixed training to check the influence of the synthetic dataset.
We check domain randomization by creating a synthetic chair dataset with different parameters like light, textures, camera position, etc.
The domain gap between real and synthetic datasets is visualized using T-SNE and quantitatively measured with the FID score.
Finally, these experimental results prove that the proposed dataset enhances the performances of Deep Neural Net models for the 3D reconstruction task.

\textbf{ \emph{keywords}}: \emph{Synthetic dataset, 3D reconstruction, Domain adaptation, Domain randomisation, Mixed training}

%Abstracts typically follow the same structure.
%You start by describing the research area as well as the general and the specific problem you are focusing on.
%Then, you outline how you approach the problem in terms of concepts and evaluations.
%Finally, you close with the most interesting insights that you gained and why they are relevant for the research area.

\blankpage

\iftoggle{german}{\chapter*{Danksagungen}}{\chapter*{Acknowledgments}}

Throughout this thesis work, I have received a great deal of guidance, supervision, and support.

First and foremost, I would like to express my supervisors Marco Filax and Konstantin Kirchheim,
for their expert guidance, constant support, encouragement, and patience right from the beginning of the thesis.
I would especially like to thank them for accepting the thesis proposal and helping me fine-tune the research question,
and making sure that I do not divert from the topic at hand.
The discussions we had throughout the thesis helped me think differently and tackle and make tough decisions in a limited time interval.

I would also like to extend my gratitude to Prof.Dr.Frank Ortmeier for accepting my thesis research topic and taking me under his banner.
I sincerely thank him for providing an excellent research environment and all the required resources to complete the thesis.
The active research under the Chair of Software Engineering Workgroup urged me to take up the thesis under the same department.

I would also like to thank my family for their constant support during these challenging times.
Their life advice and casual talks always calmed me and re-energized whenever required.
Last but not least, to all my friends, there were challenging situations during the pandemic, and it was all thanks to them the period passed like a breeze.
Thank you for teaching me to enjoy life!

% table of contents
\blankpage
{\parskip 0pt \pdfbookmark{\contentsname}{\contentsname}\chapterheadfont \tableofcontents}
\blankpage
% remove this if you do not want a list of figures, tables, or listings
\listoffigures
\listoftables
%\lstlistoflistings
% remove this if you do not want a list of symbols or acronyms
\glsaddall[types={symbol}]
\printglossary[type=acronym,
    title=\iftoggle{german}{Abkürzungsverzeichnis}{List of Acronyms},
    toctitle=\iftoggle{german}{Abkürzungsverzeichnis}{List of Acronyms}]
\printglossary[type=symbol,style=supergroup]

\mainmatter
\pagenumbering{arabic}
% Here comes the main text of your thesis.
% Change this as you like - every thesis looks different, this is only an example.
\chapter{\iftoggle{german}{Einführung}{Introduction}}\label{ch:introduction}

As the Deep Learning field goes deeper and broader, data on large scales has become an essential requirement.
The performance of computer vision based on deep learning models heavily depends on the consistency of labeled images.
In practice, collecting data manually is time-consuming, expensive, and needs much effort.
As a solution, we can train models on a synthetic dataset and deploy the models to real-world problems.
Unfortunately, with supervised learning, the model does not generalize when the distributions of the dataset diverge.
Hence, a model trained on a synthetic dataset may not perform consistently with real-world data.

Synthetic data can be defined as data created artificially using computer simulations or algorithms and not from real-world events.
These data can be created in abundance and hence are frontrunners for Deep Learning training.
Synthetic approaches are also means of getting data that might be difficult to collect and annotate in the real world.
Furthermore, companies have realized that synthetic data are necessary not just for the task that is inconvenient to annotate,
but also for improving the performance of existing models.
In a research article published by Gartner~\cite{gartnerreport}, they claim that by 2030, synthetic datasets will overshadow a real dataset in all domains.
The progression of synthetic data generation is predicted to be as shown in \autoref{fig:Gartner research on synthetic dataset}.

\begin{figure}
    \centering
    \includegraphics[width=1.0\textwidth]{/Users/apple/OVGU/Thesis/code/3dReconstruction/report/images/intro/Gartner-chart_2}
    \caption{A graph showing how synthetic dataset will evolve overtime. The research graph shows that synthetic dataset will overshadow the existing real dataset~\cite{gartnerreport}.
    Source: Research article from Gartner on "Synthetic Data Is the Future of AI"~\cite{gartnerreport}}
    \label{fig:Gartner research on synthetic dataset}
\end{figure}

Deep Learning has made leaps and bounds in solving 2D-image tasks like classification, segmentation, object detection, etc.
It has now entered the realm of Three-Dimension.
An image is a 3D objects projects onto a 2D surface.
While doing so, some data from the higher dimension is being lost in the lower dimension.
This inverse ability to reconstruct 3D objects from 2D images has a wide-scale application in computer vision and robotics.
Tasks like human body shapes and face reconstructions~\cite{deng2019accurate,Guo20173DFaceNetRD,9210569,richardson20163d,Richardson2017LearningDF},
3D-Scene reconstructions~\cite{Denninger20203DSR,Song2017SemanticSC,LiSilhouetteAssisted3O,Shin20193DSR}, generic object mesh reconstruction with textures are some of the advancements we had over recent years.
Here comes the challenge;3D data is not easy to collect.
3D data is not just limited but also noisy and expensive to collect.
The results of models depend on the consistency of the labeled data.
In this thesis, we discuss how game engines can contribute to generating quality-labeled datasets for 3D reconstruction tasks.


Deep Learning has made leaps and bounds in solving 2D-image tasks like classification, segmentation, object detection, and many more.
It has now entered the realm of Three-Dimension.
An image is a 3D objects projects onto a 2D surface.
While doing so, some data from the higher dimension is lost in, the lower dimension.
This inverse ability to reconstruct 3D objects from 2D images has wide applications in computer vision and robotics.
Tasks like human body shapes and face reconstructions~\cite{deng2019accurate,Guo20173DFaceNetRD,9210569,richardson20163d,Richardson2017LearningDF},
3D-Scene reconstructions~\cite{Denninger20203DSR,Song2017SemanticSC,LiSilhouetteAssisted3O,Shin20193DSR},
generic object mesh reconstruction with textures are some of the advancements we had over recent years.
Here comes the challenge;3D data is not easy to collect.
3D data is not just limited but also noisy and expensive to collect.

\section{Domain Adaptation vs. Domain randomization} \label{sec:da vs dr}

As mentioned earlier, the success of Convolutional Neural Networks depends highly on the training data.
As it so happens, not every task will have abundant data to train.
One solution for this problem is using a pre-trained network of similar problems and using transfer learning.
The pre-trained network is then trained over with the limited data available for the problem at hand.
In this case, some layers get overwritten by fine-tuning the features learned from the larger dataset using a small annotated dataset of different domains.
Domain adaptation is a field of study in machine learning where training data distribution differs from testing or target distribution.
One shallow approach is, as mentioned above,
re-weighting the network with testing data to adapt to the problem domain~\cite{Li2017PredictionRF}.
”Deep domain adaptation methods leverage deep networks to learn more transferable representations by embedding domain adaptation in the pipeline of
deep learning”~\cite{DBLP:journals/corr/abs-1802-03601}.
For this thesis, domain adaptation is relevant as the synthetic dataset might not represent the real or target domain, leading to a drop in performance.

Domain randomization, on the other hand, is improving data quality such that the target domain distribution is included in the training domain.
This is mainly used to bridge the 'reality gap' as termed in~\cite{tobin2017domain}, between simulated environment and real images.
This is achieved by randomizing the rendering of objects in the synthetic data.
The key concept here is that by rendering objects with large variance, the real-world distribution will appear to be just a variation in the dataset.
Some features that can be randomized in a synthetic dataset are textures, brightness, shadows, camera settings, and object pose.
Domain randomization will be further discussed in \autoref{ch:concept}.

Domain randomization, on the other hand, is improving data quality such that the training domain includes that the target domain distribution.
The ’reality gap’~\cite{tobin2017domain} gets negated with this approach between simulated environment and natural images.
This is achievable by randomizing the rendering of objects in the synthetic data.
The key concept here is that the real-world distribution will appear to be just a variation in the dataset by rendering objects with significant variance.
Textures, brightness, shadows, camera settings, and object pose are some of the features that can be randomized.
In \autoref{ch:concept} we will discuss domain randomization.


\section{Volumetric representations of 3D shapes} \label{sec:Volumetric representation}
The universal representation of images in computer graphics is via pixels.
However, unlike 2D, 3D data can be represented in various formats like voxels, mesh and point clouds.
\autoref{fig:3d representation} presents the representation of these three formats.
Each of the representations has its own sets of advantages and disadvantages.
Voxels, which is short for Volumetric Pixels, is a 3D grid of pixels of constant size.
As indicated in~\cite{li2016fpnn}, the main advantage of voxels is that Convolution Neural Networks can be easily applied to 3-Dimensions as in 2-Dimensions.
Since most of the 3D geometry is surface-based, it can be wasteful and computationally expensive. Mesh is composed of vertices, edges, and faces in 3D space that indicates the formation of 3D objects. This form of representation is far more compact at a granular level, depending on the resolution. On the other hand, point clouds are collections of 3D points with (x,y,z) coordinates on the surface of the object. The collection of points determines detailing the 3D object representation.
In both these cases, \gls{cnn} is not directly applicable.
In this thesis, we opt for voxel-based models as the focus is on the performance of synthetic data rather than the model or the 3d representation itself.

\begin{figure}
    \centering
    \includegraphics[width=1.0\textwidth]{/Users/apple/OVGU/Thesis/code/3dReconstruction/report/images/intro/3drepresentation}
    \caption{3D representation of Standford bunny model.(left to right) Point cloud, voxel and mesh~\cite{Hoang2019ADL}
    \label{fig:3d representation}}
\end{figure}

\section{GameEngines}\label{sec:gameengines}
~\cite{10.5555/983334} defines a \emph{Game engine} as  “A series of modules and interfaces that allow a development team to focus on product gameplay content,
rather than technical content” Gold [2004].
In layman’s terms, game engines are software used for the development of games.
Unreal Engine, Unity, GameMaker, Amazon Lumberyard, and CryEngine are popular game engines in modern times.
Over time, the graphics in each game engine have improved to an extent where visually, we cannot differentiate between reality and graphics.
These use object-oriented techniques, which help in modularity.
Also, they are GUI-oriented and hence are user-friendly and accessible even to a novice programmer.
The critical component in game engines that we will be focusing on is the ’Rendering engine’.
3D-Rendering is the means of converting 3D data into 2D images.
Rendering can either be real-time or offline/pre-rendering.
Game engines aim to make photorealism at the highest level with the least possible rendering time.


\section{Background and motivation}\label{sec:Background and motivation}

3D reconstruction is of great significance to understand a scene and to cross the 2D realm.
However, the task is not trivial, considering the input is still a Two-dimensional RGB image.
It is even more challenging since obtaining new data is expensive and time-consuming.
The 3D reconstruction field has datasets like ShapeNet~\cite{chang2015shapenet}, 3D-Front~\cite{Fu20203DFRONT3F}.
Nevertheless, there are no corresponding real images to check if the model will work in real-world scenarios.
Also, searching for real-world examples or creating a set-up scene and then labeling the images manually can lead to human mistakes resulting in wrong labels.
A solution to this problem is automating the task of generating synthetic data which can produce pixel-perfect annotations.
Automation can solve the human-in-the-loop problem, and while time gets saved, it facilitates the generation of a large quantity of data.
However, quantity should not be the only focus while creating the data;
we should also give priority to the quality of the data.
Game engines have come a long way from just supporting 2D to supporting games with photorealistic rendering.
Even though game engines have grown to be successful, few tools use them as the base framework.
As 3D models are available, we can import these models into the game engine and create an ersatz environment to generate a photorealistic dataset.
Further, we can check if these photorealistic synthetic data are useful for 3D reconstruction tasks.

\section{Goal of this Thesis}\label{sec:goal}

The following we will answer the following research questions:
\begin{enumerate}
    \item Are the game engines an excellent medium to create photorealistic synthetic datasets?

    The photorealism of the images will be evaluated with a survey to compare images from the real dataset and generate datasets, and other proclaimed natural-looking datasets.

    \item Can Ersatz environment from a game engine like Unity replace real-world data for training in 3d reconstruction tasks?

    Using the dataset created with the Unity game engine, a single-view image to 3d reconstruction of furniture will be trained and tested on a real image dataset.

    \item To what extent can the performance of model Pre-trained with real dataset be improved with Synthetic dataset from game engine, with the size of synthetic dataset being ten times that of the real dataset?
%    \todo: dont forget to change times

    We use the dataset created using the Unity game engine to train a single-view image to 3d reconstruction of furniture and further fine-tune it with the real dataset to know if the performance is enhanced.
    Further an ablation study on chair dataset to check the effects domain randomization parameters.
    This experiment will also answer whether augmenting images with domain randomization with real data during training improves the performance?

%    \item \todo{(optional: Can the photorealism of synthetic image further be enhanced using domain adaptation/GAN techniques?)}
\end{enumerate}

\section{Contributions of this Thesis}\label{sec:contributions}
In answering the research questions formulated in \autoref{sec:goal}, this thesis contributes the following:

\begin{itemize}
    \item A Unity-based framework to create a synthetic dataset with domain randomization.
    The tool supports both automated and manual image creation.
    \item A survey of photorealism to check whether Game engines produce quality synthetic datasets comparable with other proclaimed benchmarked photorealistic indoor datasets
    \item Provide a new dataset related to 3d reconstruction tasks or Synthetic to real domain adaptation tasks.
    The dataset includes G-buffers like RGB, normals, depth maps, and semantic segmentation images.
    \item Provide a focused synthetic chair dataset created with different parameters of domain randomization.
    A case study was conducted to check the influence of each component of domain randomization.
    \item A case study to understand the impact of mixed training.
    Synthetic and real image datasets are mixed in different ratios to check the performance of the 3D reconstruction task.
    \item Fine-tuning models trained on a synthetic dataset with a small real dataset.
\end{itemize}


\section{Structure of this Thesis}\label{sec:Structure of thesis}

The rest of this thesis is structured as follows:

\begin{itemize}
    \item In \autoref{ch:related_work}, we visit some related work on synthetic data generation and benchmark 3d reconstruction networks.
    \item \autoref{ch:concept} deals with concepts and design choices.
    We will discuss in detail the choices for the dataset and model selection for 3d reconstruction.
    \item In \autoref{ch:implementation},discusses the 3DScene tool developed using the Unity game engine to create an ersatz environment.
    On the Deep learning side, we will discuss the pipeline design for the 3D reconstruction task.
    \item We dedicate \autoref{ch:evaluation} to reviewing and discussing evaluation results obtained by comparing the synthetic and real dataset from the survey conducted and Deep learning based on the 3D reconstruction task.
    \item Finally, in \autoref{ch:conclusion}, we conclude this thesis with the study results, highlight few limitations and discuss future improvements.
\end{itemize}
\chapter{\iftoggle{german}{Verwandte Arbeiten}{Related Work}}\label{ch:related_work}


\section{Indoor datasets}\label{ss:indoor scenes}

Indoor scene dataset has been in the rise with increasing interest in scene processing understanding ~\cite{dai2017scannet,Silberman2012IndoorSA,Xiao2013SUN3DAD,Hua2016SceneNNAS,Armeni20163DSP,chang2017matterport3d,Handa2016UnderstandingRI,InteriorNet18,li2021openrooms,zheng2020structured3d,Roberts2020HypersimAP,McCormac:etal:ICCV2017}.
Synthetic dataset is not something new in the world of machine learning.
As researchers realised the disadvantages of real dataset,focus was shifted to synthetic dataset.
While ~\cite{dai2017scannet,} are real-world datasets, ~\cite{Fu20203DFRONT3F,Handa2016UnderstandingRI,McCormac:etal:ICCV2017,Roberts2020HypersimAP} are synthetically produced.
The real-world dataset are gathered from live scans.
The synthetic dataset can either bet manually configured by a professional or automated by a programmer.

\subsubsection{Indoor synthetic datasets}
Alibaba group introduced 3D-FRONT~\cite{Fu20203DFRONT3F} which stands for 3D Furnished Rooms with layOuts and semaNTics dataset which comprises of
synthetic indoor scenes designed under the supervision of professionals.
It consists of 18,968 rooms and 13,151 textured furnitures.
SceneNet~\cite{McCormac:etal:ICCV2017} is a large collection of photorealistic images and trajectories.
This is discussed in detail in ~\ref{ss:SceneNet} section.

SunCG ~\cite{Song2017SemanticSC} was a key dataset for scene understanding.
The dataset contained over 45,000 variations of scenes with realistic room layout created manually.
Each scene was semantically labeld and also provided with volumetric ground truth data.
The datset has also been used for task like depth estimation, semantic scene completion, SLAM, indoor navigation, etc.
Unfortunately due to legal issues \footnote{https://futurism.com/tech-suing-facebook-princeton-data} the dataset has been made publically unavailable which has left a void in the field.

Structure3D ~\cite{zheng2020structured3d} is another impressive synthetic data for indoor scenes which introduced their own photorealistic renderer.
The dataset comprises 21,835 rooms in 3,500 scenes and 196k 2D-images rendered with photo-realism.
But the CAD models of the 3D furnitures which is used to populate the scenes is not made available to public.
And hence 3D reconstruction related tasks can not be performed.
It is also demonstrated that with combination of synthetic and real dataset deep learning task for room layout estimation improved performance on benchamark datasets.
This dataset is more focused on room layout estimation and not 3D reconstruction, but with few modification, it can be mapped to 3d furntiture reconstuction tasks.

Openrooms~\cite{li2021openrooms} use Scannet~\cite{dai2017scannet} as their layout foundation, retrieve corresponding models from shapenet~\cite{chang2015shapenet}
and then replace the CAD model with retrieved model with proper alignment.
They further add reflectance and illumination to compose photo-realistic images.
As of August 2021, only the dataset has been made public and not the generation tool or the CAD models.
Though the underlying concept of Openrooms is to convert existing scans into photo-realistic synthetic images, in terms of the output images we consider them our counterpart, as the framework can produce normals, depth maps, instancs segmentation and masks same as we do.

Hypermism ~\cite{Roberts2020HypersimAP} is Apple's repository for holistic indoor scene understanding.
It is a collection of synthetic scenes created with the help of professional artist.
~\cite{Evermotion} was the starting point for the dataset for which assets were purchased from ~\cite{TurboSquid}.
The dataset includes images, 3D assets, semantic instance segmentations, and a disentangled image representation with diffused lighting and shading.
Even though the 3d triangle meshes for each asset is available online, we have to purchase them to create custom dataset.
They also admit that the costs to generate the dataset is expensive \{approximately \$57K ~\cite{Roberts2020HypersimAP}\}.

InteriorNet ~\cite{InteriorNet18} claims to be a photo-realistic indoor scene simulator with realistic lighting and scenes which change over time.
The image dataset includes rgb, depth and semantic segmentations.
Along with images, they also provide synthesized realistic trajectories at video-frame rate with various motion patterns.
The simulator also supports scenes from ~\cite{McCormac:etal:ICCV2017} and ~\cite{Song2017SemanticSC} along with their own database.

Another simulated framework for visual research is House Of inteRactions (THOR) introduced in AI2-Thor ~\cite{kolve2019ai2thor}.
This is again a Agent focused photo-realistic dataset with the key factor being actionable objects so that agents can interact with the objects or manipulate them.
The underlying renderer for this framework is Unity game engine.
RoboThor ~\cite{Deitke2020RoboTHORAO} is built upon AI2-Thor which consists of real scenes and its corresponding synthetic equivalent.
This helps in the study of behavior of agents in real world when trained on synthetic data.

Habitat: A Platform for Embodied AI Research ~\cite{savva2019habitat}, is a photorealistic 3D simulation which can be used for training virtual agents for tasks like navigation, question answering, intruction foloowing.
The paper introduces Habitat-Sim which renders scenes from Matterport3d ~\cite{chang2017matterport3d}, Gibson ~\cite{xia2018gibson}, Replica ~\cite{Straub2019TheRD} and some other datasets.
The focus of the simulator is providing the agent with sensor data and allowing additional sensors as plugins.
At the foundation level Habitat-sim uses Magnum graphics
middleware library~\footnote{https://magnum.graphics/} which supports cross-platform on various hardware configuration.


\subsubsection{Tools to create synthetic datasets}

BlenderProc: Reducing the Reality Gap with Photorealistic Rendering ~\cite{denninger2019blenderproc}

NVIDIA developed a Deep learning Dataset Synthesizer (NDDS) ~\cite{to2018ndds} in the form of a plugin for Unreal Engine 4(UE4).
The plugin can synthesize images,per-pixel segmentation, depth, object 3D pose, 2D/3D bounding box, keypoints, and custom stencils.
It even supports domain randomisation of objects, lighting, camera position, poses and textures.
Leveraging the asynchronous-multithreaded frames, data was generated at high rates(50–100 Hz) for Falling Things (FAT) ~\cite{tremblay2018falling} dataset.

SynthDet: An end-to-end object detection pipeline using synthetic data

UnrealCV ~\cite{qiu2017unrealcv}

\todo{A table to explain diffs between various methods( only if 3D-FREE has more advantage)}

\section{State of the art for 3d-reconstruction}\label{ss:state_of_the_art}
Voxel based, mesh based
Pix2vox,OctNet,Mesh rcnn,Pixel2Mesh,Occupancy networks, etc

\section{Domain adaptation}\label{ss:domain_adaptation}
Discuss other methods used to mitigate domain shift
Example: distance learning, subspace matching

\subsection{GAN based style transfer}\label{ss:gan_based_styletransfer}


\chapter{\iftoggle{german}{Konzept}{Concept}}\label{ch:concept}

In this chapter, we discuss the critical datasets used to create a new synthetic dataset and the reason for their selection.
Furthermore, how we combine multiple datasets to our advantage of creating new datasets in the Ersatz environment.
The rationale in choosing Unity as a framework to build the pipeline \autoref{sec:unity-based-pipeline}.
We will also discuss domain randomization  \autoref{subsec:domain-randomisation-with-unity-engine} and each of its parameters.
In \autoref{sec:s2r:3d-free-a-pix3d-based-synthetic-dataset}, we will introduce the proposed \gls{free} dataset used for the 3D-reconstruction task.

In the Deep Learning domain(\autoref{sec:3D reconstruction}), we explore what 3D-reconstruction is and how to achieve it.
In \autoref{subsec:pix2vox-and-pix2vox++}, the state-of-the-art models for 3D-Reconstruction on voxel representation and reason for selecting the sibling models as baselines are discussed.

\section{Pix3D: A large-scale benchmark}\label{sec:pix3d}
~\cite{pix3d}introduces a large-scale benchmark for 2D-3D alignment.
The raw images from web search engines were collected, and labeled key points were used to align the 2D images with the corresponding 3D shapes.
The 3D models extend the IKEA dataset~\cite{Lim2013ParsingIO}, a collection of high-quality IKEA furniture.
The dataset also provides a binary mask and the key points for the object under observation.
To add more images to the IKEA dataset, the authors of~\cite{pix3d} conducted a manual web search on Google, Bing, and Baidu, using the IKEA model name as keywords.
This search led to around 104,220 images which were further filtered by removing irrelevant images with the help of Amazon Mechanical Turk (AMT) workers.
After this manual experiment, the total images in Pix3D, only 14,600 images were selected for the 219 Ikea models.
For our synthetic to real dataset experiment, we chose to select only furniture classes from Pix3D, leaving out \”misc and tools\” classes, which were significantly less in comparison.


\subsection{Disadvantages of Pix3D}\label{subsec:disadvantages-of-pix3d}

The distribution of models in Pix3D is as shown in \autoref{fig:pix3d_histogram}.
As we can see, the dataset distribution is uneven across classes, and more than 50\% of classes have less than 1000 images.

Though Pix3D set a benchmark for 2D-3D alignment, here are few disadvantages of using this real dataset.
\begin{enumerate}
    \item For Deep Learning approaches, we need large-scale data, and 14,600 might not be sufficient.
    \item The orientation of an object is not randomized.
    \item The dataset does not provide 2.5D information (i.e., depth maps and normals), which can be crucial for 3D learning.
\end{enumerate}

\begin{figure}[!ht]
    \resizebox{0.49\textwidth}{6cm}{%% Creator: Matplotlib, PGF backend
%%
%% To include the figure in your LaTeX document, write
%%   \input{<filename>.pgf}
%%
%% Make sure the required packages are loaded in your preamble
%%   \usepackage{pgf}
%%
%% Figures using additional raster images can only be included by \input if
%% they are in the same directory as the main LaTeX file. For loading figures
%% from other directories you can use the `import` package
%%   \usepackage{import}
%%
%% and then include the figures with
%%   \import{<path to file>}{<filename>.pgf}
%%
%% Matplotlib used the following preamble
%%   \usepackage{fontspec}
%%   \setmainfont{DejaVuSerif.ttf}[Path=\detokenize{/Users/apple/opt/anaconda3/envs/kaolin/lib/python3.7/site-packages/matplotlib/mpl-data/fonts/ttf/}]
%%   \setsansfont{DejaVuSans.ttf}[Path=\detokenize{/Users/apple/opt/anaconda3/envs/kaolin/lib/python3.7/site-packages/matplotlib/mpl-data/fonts/ttf/}]
%%   \setmonofont{DejaVuSansMono.ttf}[Path=\detokenize{/Users/apple/opt/anaconda3/envs/kaolin/lib/python3.7/site-packages/matplotlib/mpl-data/fonts/ttf/}]
%%
\begingroup%
\makeatletter%
\begin{pgfpicture}%
\pgfpathrectangle{\pgfpointorigin}{\pgfqpoint{5.752068in}{4.226888in}}%
\pgfusepath{use as bounding box, clip}%
\begin{pgfscope}%
\pgfsetbuttcap%
\pgfsetmiterjoin%
\definecolor{currentfill}{rgb}{1.000000,1.000000,1.000000}%
\pgfsetfillcolor{currentfill}%
\pgfsetlinewidth{0.000000pt}%
\definecolor{currentstroke}{rgb}{1.000000,1.000000,1.000000}%
\pgfsetstrokecolor{currentstroke}%
\pgfsetdash{}{0pt}%
\pgfpathmoveto{\pgfqpoint{0.000000in}{0.000000in}}%
\pgfpathlineto{\pgfqpoint{5.752068in}{0.000000in}}%
\pgfpathlineto{\pgfqpoint{5.752068in}{4.226888in}}%
\pgfpathlineto{\pgfqpoint{0.000000in}{4.226888in}}%
\pgfpathclose%
\pgfusepath{fill}%
\end{pgfscope}%
\begin{pgfscope}%
\pgfsetbuttcap%
\pgfsetmiterjoin%
\definecolor{currentfill}{rgb}{1.000000,1.000000,1.000000}%
\pgfsetfillcolor{currentfill}%
\pgfsetlinewidth{0.000000pt}%
\definecolor{currentstroke}{rgb}{0.000000,0.000000,0.000000}%
\pgfsetstrokecolor{currentstroke}%
\pgfsetstrokeopacity{0.000000}%
\pgfsetdash{}{0pt}%
\pgfpathmoveto{\pgfqpoint{0.692068in}{0.385400in}}%
\pgfpathlineto{\pgfqpoint{5.652068in}{0.385400in}}%
\pgfpathlineto{\pgfqpoint{5.652068in}{4.081400in}}%
\pgfpathlineto{\pgfqpoint{0.692068in}{4.081400in}}%
\pgfpathclose%
\pgfusepath{fill}%
\end{pgfscope}%
\begin{pgfscope}%
\pgfpathrectangle{\pgfqpoint{0.692068in}{0.385400in}}{\pgfqpoint{4.960000in}{3.696000in}}%
\pgfusepath{clip}%
\pgfsetbuttcap%
\pgfsetmiterjoin%
\definecolor{currentfill}{rgb}{0.121569,0.466667,0.705882}%
\pgfsetfillcolor{currentfill}%
\pgfsetlinewidth{0.000000pt}%
\definecolor{currentstroke}{rgb}{0.000000,0.000000,0.000000}%
\pgfsetstrokecolor{currentstroke}%
\pgfsetstrokeopacity{0.000000}%
\pgfsetdash{}{0pt}%
\pgfpathmoveto{\pgfqpoint{0.917523in}{0.385400in}}%
\pgfpathlineto{\pgfqpoint{1.448004in}{0.385400in}}%
\pgfpathlineto{\pgfqpoint{1.448004in}{0.608208in}}%
\pgfpathlineto{\pgfqpoint{0.917523in}{0.608208in}}%
\pgfpathclose%
\pgfusepath{fill}%
\end{pgfscope}%
\begin{pgfscope}%
\pgfpathrectangle{\pgfqpoint{0.692068in}{0.385400in}}{\pgfqpoint{4.960000in}{3.696000in}}%
\pgfusepath{clip}%
\pgfsetbuttcap%
\pgfsetmiterjoin%
\definecolor{currentfill}{rgb}{0.121569,0.466667,0.705882}%
\pgfsetfillcolor{currentfill}%
\pgfsetlinewidth{0.000000pt}%
\definecolor{currentstroke}{rgb}{0.000000,0.000000,0.000000}%
\pgfsetstrokecolor{currentstroke}%
\pgfsetstrokeopacity{0.000000}%
\pgfsetdash{}{0pt}%
\pgfpathmoveto{\pgfqpoint{1.580624in}{0.385400in}}%
\pgfpathlineto{\pgfqpoint{2.111105in}{0.385400in}}%
\pgfpathlineto{\pgfqpoint{2.111105in}{1.296804in}}%
\pgfpathlineto{\pgfqpoint{1.580624in}{1.296804in}}%
\pgfpathclose%
\pgfusepath{fill}%
\end{pgfscope}%
\begin{pgfscope}%
\pgfpathrectangle{\pgfqpoint{0.692068in}{0.385400in}}{\pgfqpoint{4.960000in}{3.696000in}}%
\pgfusepath{clip}%
\pgfsetbuttcap%
\pgfsetmiterjoin%
\definecolor{currentfill}{rgb}{0.121569,0.466667,0.705882}%
\pgfsetfillcolor{currentfill}%
\pgfsetlinewidth{0.000000pt}%
\definecolor{currentstroke}{rgb}{0.000000,0.000000,0.000000}%
\pgfsetstrokecolor{currentstroke}%
\pgfsetstrokeopacity{0.000000}%
\pgfsetdash{}{0pt}%
\pgfpathmoveto{\pgfqpoint{2.243726in}{0.385400in}}%
\pgfpathlineto{\pgfqpoint{2.774207in}{0.385400in}}%
\pgfpathlineto{\pgfqpoint{2.774207in}{1.028151in}}%
\pgfpathlineto{\pgfqpoint{2.243726in}{1.028151in}}%
\pgfpathclose%
\pgfusepath{fill}%
\end{pgfscope}%
\begin{pgfscope}%
\pgfpathrectangle{\pgfqpoint{0.692068in}{0.385400in}}{\pgfqpoint{4.960000in}{3.696000in}}%
\pgfusepath{clip}%
\pgfsetbuttcap%
\pgfsetmiterjoin%
\definecolor{currentfill}{rgb}{0.121569,0.466667,0.705882}%
\pgfsetfillcolor{currentfill}%
\pgfsetlinewidth{0.000000pt}%
\definecolor{currentstroke}{rgb}{0.000000,0.000000,0.000000}%
\pgfsetstrokecolor{currentstroke}%
\pgfsetstrokeopacity{0.000000}%
\pgfsetdash{}{0pt}%
\pgfpathmoveto{\pgfqpoint{2.906827in}{0.385400in}}%
\pgfpathlineto{\pgfqpoint{3.437309in}{0.385400in}}%
\pgfpathlineto{\pgfqpoint{3.437309in}{3.905400in}}%
\pgfpathlineto{\pgfqpoint{2.906827in}{3.905400in}}%
\pgfpathclose%
\pgfusepath{fill}%
\end{pgfscope}%
\begin{pgfscope}%
\pgfpathrectangle{\pgfqpoint{0.692068in}{0.385400in}}{\pgfqpoint{4.960000in}{3.696000in}}%
\pgfusepath{clip}%
\pgfsetbuttcap%
\pgfsetmiterjoin%
\definecolor{currentfill}{rgb}{0.121569,0.466667,0.705882}%
\pgfsetfillcolor{currentfill}%
\pgfsetlinewidth{0.000000pt}%
\definecolor{currentstroke}{rgb}{0.000000,0.000000,0.000000}%
\pgfsetstrokecolor{currentstroke}%
\pgfsetstrokeopacity{0.000000}%
\pgfsetdash{}{0pt}%
\pgfpathmoveto{\pgfqpoint{3.569929in}{0.385400in}}%
\pgfpathlineto{\pgfqpoint{4.100410in}{0.385400in}}%
\pgfpathlineto{\pgfqpoint{4.100410in}{0.716403in}}%
\pgfpathlineto{\pgfqpoint{3.569929in}{0.716403in}}%
\pgfpathclose%
\pgfusepath{fill}%
\end{pgfscope}%
\begin{pgfscope}%
\pgfpathrectangle{\pgfqpoint{0.692068in}{0.385400in}}{\pgfqpoint{4.960000in}{3.696000in}}%
\pgfusepath{clip}%
\pgfsetbuttcap%
\pgfsetmiterjoin%
\definecolor{currentfill}{rgb}{0.121569,0.466667,0.705882}%
\pgfsetfillcolor{currentfill}%
\pgfsetlinewidth{0.000000pt}%
\definecolor{currentstroke}{rgb}{0.000000,0.000000,0.000000}%
\pgfsetstrokecolor{currentstroke}%
\pgfsetstrokeopacity{0.000000}%
\pgfsetdash{}{0pt}%
\pgfpathmoveto{\pgfqpoint{4.233031in}{0.385400in}}%
\pgfpathlineto{\pgfqpoint{4.763512in}{0.385400in}}%
\pgfpathlineto{\pgfqpoint{4.763512in}{2.171532in}}%
\pgfpathlineto{\pgfqpoint{4.233031in}{2.171532in}}%
\pgfpathclose%
\pgfusepath{fill}%
\end{pgfscope}%
\begin{pgfscope}%
\pgfpathrectangle{\pgfqpoint{0.692068in}{0.385400in}}{\pgfqpoint{4.960000in}{3.696000in}}%
\pgfusepath{clip}%
\pgfsetbuttcap%
\pgfsetmiterjoin%
\definecolor{currentfill}{rgb}{0.121569,0.466667,0.705882}%
\pgfsetfillcolor{currentfill}%
\pgfsetlinewidth{0.000000pt}%
\definecolor{currentstroke}{rgb}{0.000000,0.000000,0.000000}%
\pgfsetstrokecolor{currentstroke}%
\pgfsetstrokeopacity{0.000000}%
\pgfsetdash{}{0pt}%
\pgfpathmoveto{\pgfqpoint{4.896132in}{0.385400in}}%
\pgfpathlineto{\pgfqpoint{5.426614in}{0.385400in}}%
\pgfpathlineto{\pgfqpoint{5.426614in}{2.100013in}}%
\pgfpathlineto{\pgfqpoint{4.896132in}{2.100013in}}%
\pgfpathclose%
\pgfusepath{fill}%
\end{pgfscope}%
\begin{pgfscope}%
\pgfsetbuttcap%
\pgfsetroundjoin%
\definecolor{currentfill}{rgb}{0.000000,0.000000,0.000000}%
\pgfsetfillcolor{currentfill}%
\pgfsetlinewidth{0.803000pt}%
\definecolor{currentstroke}{rgb}{0.000000,0.000000,0.000000}%
\pgfsetstrokecolor{currentstroke}%
\pgfsetdash{}{0pt}%
\pgfsys@defobject{currentmarker}{\pgfqpoint{0.000000in}{-0.048611in}}{\pgfqpoint{0.000000in}{0.000000in}}{%
\pgfpathmoveto{\pgfqpoint{0.000000in}{0.000000in}}%
\pgfpathlineto{\pgfqpoint{0.000000in}{-0.048611in}}%
\pgfusepath{stroke,fill}%
}%
\begin{pgfscope}%
\pgfsys@transformshift{1.182763in}{0.385400in}%
\pgfsys@useobject{currentmarker}{}%
\end{pgfscope}%
\end{pgfscope}%
\begin{pgfscope}%
\definecolor{textcolor}{rgb}{0.000000,0.000000,0.000000}%
\pgfsetstrokecolor{textcolor}%
\pgfsetfillcolor{textcolor}%
\pgftext[x=1.182763in,y=0.288178in,,top]{\color{textcolor}\sffamily\fontsize{14.000000}{16.800000}\selectfont wardrobe}%
\end{pgfscope}%
\begin{pgfscope}%
\pgfsetbuttcap%
\pgfsetroundjoin%
\definecolor{currentfill}{rgb}{0.000000,0.000000,0.000000}%
\pgfsetfillcolor{currentfill}%
\pgfsetlinewidth{0.803000pt}%
\definecolor{currentstroke}{rgb}{0.000000,0.000000,0.000000}%
\pgfsetstrokecolor{currentstroke}%
\pgfsetdash{}{0pt}%
\pgfsys@defobject{currentmarker}{\pgfqpoint{0.000000in}{-0.048611in}}{\pgfqpoint{0.000000in}{0.000000in}}{%
\pgfpathmoveto{\pgfqpoint{0.000000in}{0.000000in}}%
\pgfpathlineto{\pgfqpoint{0.000000in}{-0.048611in}}%
\pgfusepath{stroke,fill}%
}%
\begin{pgfscope}%
\pgfsys@transformshift{1.845865in}{0.385400in}%
\pgfsys@useobject{currentmarker}{}%
\end{pgfscope}%
\end{pgfscope}%
\begin{pgfscope}%
\definecolor{textcolor}{rgb}{0.000000,0.000000,0.000000}%
\pgfsetstrokecolor{textcolor}%
\pgfsetfillcolor{textcolor}%
\pgftext[x=1.845865in,y=0.288178in,,top]{\color{textcolor}\sffamily\fontsize{14.000000}{16.800000}\selectfont bed}%
\end{pgfscope}%
\begin{pgfscope}%
\pgfsetbuttcap%
\pgfsetroundjoin%
\definecolor{currentfill}{rgb}{0.000000,0.000000,0.000000}%
\pgfsetfillcolor{currentfill}%
\pgfsetlinewidth{0.803000pt}%
\definecolor{currentstroke}{rgb}{0.000000,0.000000,0.000000}%
\pgfsetstrokecolor{currentstroke}%
\pgfsetdash{}{0pt}%
\pgfsys@defobject{currentmarker}{\pgfqpoint{0.000000in}{-0.048611in}}{\pgfqpoint{0.000000in}{0.000000in}}{%
\pgfpathmoveto{\pgfqpoint{0.000000in}{0.000000in}}%
\pgfpathlineto{\pgfqpoint{0.000000in}{-0.048611in}}%
\pgfusepath{stroke,fill}%
}%
\begin{pgfscope}%
\pgfsys@transformshift{2.508966in}{0.385400in}%
\pgfsys@useobject{currentmarker}{}%
\end{pgfscope}%
\end{pgfscope}%
\begin{pgfscope}%
\definecolor{textcolor}{rgb}{0.000000,0.000000,0.000000}%
\pgfsetstrokecolor{textcolor}%
\pgfsetfillcolor{textcolor}%
\pgftext[x=2.508966in,y=0.288178in,,top]{\color{textcolor}\sffamily\fontsize{14.000000}{16.800000}\selectfont desk}%
\end{pgfscope}%
\begin{pgfscope}%
\pgfsetbuttcap%
\pgfsetroundjoin%
\definecolor{currentfill}{rgb}{0.000000,0.000000,0.000000}%
\pgfsetfillcolor{currentfill}%
\pgfsetlinewidth{0.803000pt}%
\definecolor{currentstroke}{rgb}{0.000000,0.000000,0.000000}%
\pgfsetstrokecolor{currentstroke}%
\pgfsetdash{}{0pt}%
\pgfsys@defobject{currentmarker}{\pgfqpoint{0.000000in}{-0.048611in}}{\pgfqpoint{0.000000in}{0.000000in}}{%
\pgfpathmoveto{\pgfqpoint{0.000000in}{0.000000in}}%
\pgfpathlineto{\pgfqpoint{0.000000in}{-0.048611in}}%
\pgfusepath{stroke,fill}%
}%
\begin{pgfscope}%
\pgfsys@transformshift{3.172068in}{0.385400in}%
\pgfsys@useobject{currentmarker}{}%
\end{pgfscope}%
\end{pgfscope}%
\begin{pgfscope}%
\definecolor{textcolor}{rgb}{0.000000,0.000000,0.000000}%
\pgfsetstrokecolor{textcolor}%
\pgfsetfillcolor{textcolor}%
\pgftext[x=3.172068in,y=0.288178in,,top]{\color{textcolor}\sffamily\fontsize{14.000000}{16.800000}\selectfont chair}%
\end{pgfscope}%
\begin{pgfscope}%
\pgfsetbuttcap%
\pgfsetroundjoin%
\definecolor{currentfill}{rgb}{0.000000,0.000000,0.000000}%
\pgfsetfillcolor{currentfill}%
\pgfsetlinewidth{0.803000pt}%
\definecolor{currentstroke}{rgb}{0.000000,0.000000,0.000000}%
\pgfsetstrokecolor{currentstroke}%
\pgfsetdash{}{0pt}%
\pgfsys@defobject{currentmarker}{\pgfqpoint{0.000000in}{-0.048611in}}{\pgfqpoint{0.000000in}{0.000000in}}{%
\pgfpathmoveto{\pgfqpoint{0.000000in}{0.000000in}}%
\pgfpathlineto{\pgfqpoint{0.000000in}{-0.048611in}}%
\pgfusepath{stroke,fill}%
}%
\begin{pgfscope}%
\pgfsys@transformshift{3.835170in}{0.385400in}%
\pgfsys@useobject{currentmarker}{}%
\end{pgfscope}%
\end{pgfscope}%
\begin{pgfscope}%
\definecolor{textcolor}{rgb}{0.000000,0.000000,0.000000}%
\pgfsetstrokecolor{textcolor}%
\pgfsetfillcolor{textcolor}%
\pgftext[x=3.835170in,y=0.288178in,,top]{\color{textcolor}\sffamily\fontsize{14.000000}{16.800000}\selectfont bookcase}%
\end{pgfscope}%
\begin{pgfscope}%
\pgfsetbuttcap%
\pgfsetroundjoin%
\definecolor{currentfill}{rgb}{0.000000,0.000000,0.000000}%
\pgfsetfillcolor{currentfill}%
\pgfsetlinewidth{0.803000pt}%
\definecolor{currentstroke}{rgb}{0.000000,0.000000,0.000000}%
\pgfsetstrokecolor{currentstroke}%
\pgfsetdash{}{0pt}%
\pgfsys@defobject{currentmarker}{\pgfqpoint{0.000000in}{-0.048611in}}{\pgfqpoint{0.000000in}{0.000000in}}{%
\pgfpathmoveto{\pgfqpoint{0.000000in}{0.000000in}}%
\pgfpathlineto{\pgfqpoint{0.000000in}{-0.048611in}}%
\pgfusepath{stroke,fill}%
}%
\begin{pgfscope}%
\pgfsys@transformshift{4.498271in}{0.385400in}%
\pgfsys@useobject{currentmarker}{}%
\end{pgfscope}%
\end{pgfscope}%
\begin{pgfscope}%
\definecolor{textcolor}{rgb}{0.000000,0.000000,0.000000}%
\pgfsetstrokecolor{textcolor}%
\pgfsetfillcolor{textcolor}%
\pgftext[x=4.498271in,y=0.288178in,,top]{\color{textcolor}\sffamily\fontsize{14.000000}{16.800000}\selectfont sofa}%
\end{pgfscope}%
\begin{pgfscope}%
\pgfsetbuttcap%
\pgfsetroundjoin%
\definecolor{currentfill}{rgb}{0.000000,0.000000,0.000000}%
\pgfsetfillcolor{currentfill}%
\pgfsetlinewidth{0.803000pt}%
\definecolor{currentstroke}{rgb}{0.000000,0.000000,0.000000}%
\pgfsetstrokecolor{currentstroke}%
\pgfsetdash{}{0pt}%
\pgfsys@defobject{currentmarker}{\pgfqpoint{0.000000in}{-0.048611in}}{\pgfqpoint{0.000000in}{0.000000in}}{%
\pgfpathmoveto{\pgfqpoint{0.000000in}{0.000000in}}%
\pgfpathlineto{\pgfqpoint{0.000000in}{-0.048611in}}%
\pgfusepath{stroke,fill}%
}%
\begin{pgfscope}%
\pgfsys@transformshift{5.161373in}{0.385400in}%
\pgfsys@useobject{currentmarker}{}%
\end{pgfscope}%
\end{pgfscope}%
\begin{pgfscope}%
\definecolor{textcolor}{rgb}{0.000000,0.000000,0.000000}%
\pgfsetstrokecolor{textcolor}%
\pgfsetfillcolor{textcolor}%
\pgftext[x=5.161373in,y=0.288178in,,top]{\color{textcolor}\sffamily\fontsize{14.000000}{16.800000}\selectfont table}%
\end{pgfscope}%
\begin{pgfscope}%
\pgfsetbuttcap%
\pgfsetroundjoin%
\definecolor{currentfill}{rgb}{0.000000,0.000000,0.000000}%
\pgfsetfillcolor{currentfill}%
\pgfsetlinewidth{0.803000pt}%
\definecolor{currentstroke}{rgb}{0.000000,0.000000,0.000000}%
\pgfsetstrokecolor{currentstroke}%
\pgfsetdash{}{0pt}%
\pgfsys@defobject{currentmarker}{\pgfqpoint{-0.048611in}{0.000000in}}{\pgfqpoint{-0.000000in}{0.000000in}}{%
\pgfpathmoveto{\pgfqpoint{-0.000000in}{0.000000in}}%
\pgfpathlineto{\pgfqpoint{-0.048611in}{0.000000in}}%
\pgfusepath{stroke,fill}%
}%
\begin{pgfscope}%
\pgfsys@transformshift{0.692068in}{0.385400in}%
\pgfsys@useobject{currentmarker}{}%
\end{pgfscope}%
\end{pgfscope}%
\begin{pgfscope}%
\definecolor{textcolor}{rgb}{0.000000,0.000000,0.000000}%
\pgfsetstrokecolor{textcolor}%
\pgfsetfillcolor{textcolor}%
\pgftext[x=0.471134in, y=0.311534in, left, base]{\color{textcolor}\sffamily\fontsize{14.000000}{16.800000}\selectfont 0}%
\end{pgfscope}%
\begin{pgfscope}%
\pgfsetbuttcap%
\pgfsetroundjoin%
\definecolor{currentfill}{rgb}{0.000000,0.000000,0.000000}%
\pgfsetfillcolor{currentfill}%
\pgfsetlinewidth{0.803000pt}%
\definecolor{currentstroke}{rgb}{0.000000,0.000000,0.000000}%
\pgfsetstrokecolor{currentstroke}%
\pgfsetdash{}{0pt}%
\pgfsys@defobject{currentmarker}{\pgfqpoint{-0.048611in}{0.000000in}}{\pgfqpoint{-0.000000in}{0.000000in}}{%
\pgfpathmoveto{\pgfqpoint{-0.000000in}{0.000000in}}%
\pgfpathlineto{\pgfqpoint{-0.048611in}{0.000000in}}%
\pgfusepath{stroke,fill}%
}%
\begin{pgfscope}%
\pgfsys@transformshift{0.692068in}{0.843853in}%
\pgfsys@useobject{currentmarker}{}%
\end{pgfscope}%
\end{pgfscope}%
\begin{pgfscope}%
\definecolor{textcolor}{rgb}{0.000000,0.000000,0.000000}%
\pgfsetstrokecolor{textcolor}%
\pgfsetfillcolor{textcolor}%
\pgftext[x=0.223711in, y=0.769987in, left, base]{\color{textcolor}\sffamily\fontsize{14.000000}{16.800000}\selectfont 500}%
\end{pgfscope}%
\begin{pgfscope}%
\pgfsetbuttcap%
\pgfsetroundjoin%
\definecolor{currentfill}{rgb}{0.000000,0.000000,0.000000}%
\pgfsetfillcolor{currentfill}%
\pgfsetlinewidth{0.803000pt}%
\definecolor{currentstroke}{rgb}{0.000000,0.000000,0.000000}%
\pgfsetstrokecolor{currentstroke}%
\pgfsetdash{}{0pt}%
\pgfsys@defobject{currentmarker}{\pgfqpoint{-0.048611in}{0.000000in}}{\pgfqpoint{-0.000000in}{0.000000in}}{%
\pgfpathmoveto{\pgfqpoint{-0.000000in}{0.000000in}}%
\pgfpathlineto{\pgfqpoint{-0.048611in}{0.000000in}}%
\pgfusepath{stroke,fill}%
}%
\begin{pgfscope}%
\pgfsys@transformshift{0.692068in}{1.302306in}%
\pgfsys@useobject{currentmarker}{}%
\end{pgfscope}%
\end{pgfscope}%
\begin{pgfscope}%
\definecolor{textcolor}{rgb}{0.000000,0.000000,0.000000}%
\pgfsetstrokecolor{textcolor}%
\pgfsetfillcolor{textcolor}%
\pgftext[x=0.100000in, y=1.228440in, left, base]{\color{textcolor}\sffamily\fontsize{14.000000}{16.800000}\selectfont 1000}%
\end{pgfscope}%
\begin{pgfscope}%
\pgfsetbuttcap%
\pgfsetroundjoin%
\definecolor{currentfill}{rgb}{0.000000,0.000000,0.000000}%
\pgfsetfillcolor{currentfill}%
\pgfsetlinewidth{0.803000pt}%
\definecolor{currentstroke}{rgb}{0.000000,0.000000,0.000000}%
\pgfsetstrokecolor{currentstroke}%
\pgfsetdash{}{0pt}%
\pgfsys@defobject{currentmarker}{\pgfqpoint{-0.048611in}{0.000000in}}{\pgfqpoint{-0.000000in}{0.000000in}}{%
\pgfpathmoveto{\pgfqpoint{-0.000000in}{0.000000in}}%
\pgfpathlineto{\pgfqpoint{-0.048611in}{0.000000in}}%
\pgfusepath{stroke,fill}%
}%
\begin{pgfscope}%
\pgfsys@transformshift{0.692068in}{1.760758in}%
\pgfsys@useobject{currentmarker}{}%
\end{pgfscope}%
\end{pgfscope}%
\begin{pgfscope}%
\definecolor{textcolor}{rgb}{0.000000,0.000000,0.000000}%
\pgfsetstrokecolor{textcolor}%
\pgfsetfillcolor{textcolor}%
\pgftext[x=0.100000in, y=1.686892in, left, base]{\color{textcolor}\sffamily\fontsize{14.000000}{16.800000}\selectfont 1500}%
\end{pgfscope}%
\begin{pgfscope}%
\pgfsetbuttcap%
\pgfsetroundjoin%
\definecolor{currentfill}{rgb}{0.000000,0.000000,0.000000}%
\pgfsetfillcolor{currentfill}%
\pgfsetlinewidth{0.803000pt}%
\definecolor{currentstroke}{rgb}{0.000000,0.000000,0.000000}%
\pgfsetstrokecolor{currentstroke}%
\pgfsetdash{}{0pt}%
\pgfsys@defobject{currentmarker}{\pgfqpoint{-0.048611in}{0.000000in}}{\pgfqpoint{-0.000000in}{0.000000in}}{%
\pgfpathmoveto{\pgfqpoint{-0.000000in}{0.000000in}}%
\pgfpathlineto{\pgfqpoint{-0.048611in}{0.000000in}}%
\pgfusepath{stroke,fill}%
}%
\begin{pgfscope}%
\pgfsys@transformshift{0.692068in}{2.219211in}%
\pgfsys@useobject{currentmarker}{}%
\end{pgfscope}%
\end{pgfscope}%
\begin{pgfscope}%
\definecolor{textcolor}{rgb}{0.000000,0.000000,0.000000}%
\pgfsetstrokecolor{textcolor}%
\pgfsetfillcolor{textcolor}%
\pgftext[x=0.100000in, y=2.145345in, left, base]{\color{textcolor}\sffamily\fontsize{14.000000}{16.800000}\selectfont 2000}%
\end{pgfscope}%
\begin{pgfscope}%
\pgfsetbuttcap%
\pgfsetroundjoin%
\definecolor{currentfill}{rgb}{0.000000,0.000000,0.000000}%
\pgfsetfillcolor{currentfill}%
\pgfsetlinewidth{0.803000pt}%
\definecolor{currentstroke}{rgb}{0.000000,0.000000,0.000000}%
\pgfsetstrokecolor{currentstroke}%
\pgfsetdash{}{0pt}%
\pgfsys@defobject{currentmarker}{\pgfqpoint{-0.048611in}{0.000000in}}{\pgfqpoint{-0.000000in}{0.000000in}}{%
\pgfpathmoveto{\pgfqpoint{-0.000000in}{0.000000in}}%
\pgfpathlineto{\pgfqpoint{-0.048611in}{0.000000in}}%
\pgfusepath{stroke,fill}%
}%
\begin{pgfscope}%
\pgfsys@transformshift{0.692068in}{2.677664in}%
\pgfsys@useobject{currentmarker}{}%
\end{pgfscope}%
\end{pgfscope}%
\begin{pgfscope}%
\definecolor{textcolor}{rgb}{0.000000,0.000000,0.000000}%
\pgfsetstrokecolor{textcolor}%
\pgfsetfillcolor{textcolor}%
\pgftext[x=0.100000in, y=2.603798in, left, base]{\color{textcolor}\sffamily\fontsize{14.000000}{16.800000}\selectfont 2500}%
\end{pgfscope}%
\begin{pgfscope}%
\pgfsetbuttcap%
\pgfsetroundjoin%
\definecolor{currentfill}{rgb}{0.000000,0.000000,0.000000}%
\pgfsetfillcolor{currentfill}%
\pgfsetlinewidth{0.803000pt}%
\definecolor{currentstroke}{rgb}{0.000000,0.000000,0.000000}%
\pgfsetstrokecolor{currentstroke}%
\pgfsetdash{}{0pt}%
\pgfsys@defobject{currentmarker}{\pgfqpoint{-0.048611in}{0.000000in}}{\pgfqpoint{-0.000000in}{0.000000in}}{%
\pgfpathmoveto{\pgfqpoint{-0.000000in}{0.000000in}}%
\pgfpathlineto{\pgfqpoint{-0.048611in}{0.000000in}}%
\pgfusepath{stroke,fill}%
}%
\begin{pgfscope}%
\pgfsys@transformshift{0.692068in}{3.136117in}%
\pgfsys@useobject{currentmarker}{}%
\end{pgfscope}%
\end{pgfscope}%
\begin{pgfscope}%
\definecolor{textcolor}{rgb}{0.000000,0.000000,0.000000}%
\pgfsetstrokecolor{textcolor}%
\pgfsetfillcolor{textcolor}%
\pgftext[x=0.100000in, y=3.062250in, left, base]{\color{textcolor}\sffamily\fontsize{14.000000}{16.800000}\selectfont 3000}%
\end{pgfscope}%
\begin{pgfscope}%
\pgfsetbuttcap%
\pgfsetroundjoin%
\definecolor{currentfill}{rgb}{0.000000,0.000000,0.000000}%
\pgfsetfillcolor{currentfill}%
\pgfsetlinewidth{0.803000pt}%
\definecolor{currentstroke}{rgb}{0.000000,0.000000,0.000000}%
\pgfsetstrokecolor{currentstroke}%
\pgfsetdash{}{0pt}%
\pgfsys@defobject{currentmarker}{\pgfqpoint{-0.048611in}{0.000000in}}{\pgfqpoint{-0.000000in}{0.000000in}}{%
\pgfpathmoveto{\pgfqpoint{-0.000000in}{0.000000in}}%
\pgfpathlineto{\pgfqpoint{-0.048611in}{0.000000in}}%
\pgfusepath{stroke,fill}%
}%
\begin{pgfscope}%
\pgfsys@transformshift{0.692068in}{3.594569in}%
\pgfsys@useobject{currentmarker}{}%
\end{pgfscope}%
\end{pgfscope}%
\begin{pgfscope}%
\definecolor{textcolor}{rgb}{0.000000,0.000000,0.000000}%
\pgfsetstrokecolor{textcolor}%
\pgfsetfillcolor{textcolor}%
\pgftext[x=0.100000in, y=3.520703in, left, base]{\color{textcolor}\sffamily\fontsize{14.000000}{16.800000}\selectfont 3500}%
\end{pgfscope}%
\begin{pgfscope}%
\pgfsetbuttcap%
\pgfsetroundjoin%
\definecolor{currentfill}{rgb}{0.000000,0.000000,0.000000}%
\pgfsetfillcolor{currentfill}%
\pgfsetlinewidth{0.803000pt}%
\definecolor{currentstroke}{rgb}{0.000000,0.000000,0.000000}%
\pgfsetstrokecolor{currentstroke}%
\pgfsetdash{}{0pt}%
\pgfsys@defobject{currentmarker}{\pgfqpoint{-0.048611in}{0.000000in}}{\pgfqpoint{-0.000000in}{0.000000in}}{%
\pgfpathmoveto{\pgfqpoint{-0.000000in}{0.000000in}}%
\pgfpathlineto{\pgfqpoint{-0.048611in}{0.000000in}}%
\pgfusepath{stroke,fill}%
}%
\begin{pgfscope}%
\pgfsys@transformshift{0.692068in}{4.053022in}%
\pgfsys@useobject{currentmarker}{}%
\end{pgfscope}%
\end{pgfscope}%
\begin{pgfscope}%
\definecolor{textcolor}{rgb}{0.000000,0.000000,0.000000}%
\pgfsetstrokecolor{textcolor}%
\pgfsetfillcolor{textcolor}%
\pgftext[x=0.100000in, y=3.979156in, left, base]{\color{textcolor}\sffamily\fontsize{14.000000}{16.800000}\selectfont 4000}%
\end{pgfscope}%
\begin{pgfscope}%
\pgfsetrectcap%
\pgfsetmiterjoin%
\pgfsetlinewidth{0.803000pt}%
\definecolor{currentstroke}{rgb}{0.000000,0.000000,0.000000}%
\pgfsetstrokecolor{currentstroke}%
\pgfsetdash{}{0pt}%
\pgfpathmoveto{\pgfqpoint{0.692068in}{0.385400in}}%
\pgfpathlineto{\pgfqpoint{0.692068in}{4.081400in}}%
\pgfusepath{stroke}%
\end{pgfscope}%
\begin{pgfscope}%
\pgfsetrectcap%
\pgfsetmiterjoin%
\pgfsetlinewidth{0.803000pt}%
\definecolor{currentstroke}{rgb}{0.000000,0.000000,0.000000}%
\pgfsetstrokecolor{currentstroke}%
\pgfsetdash{}{0pt}%
\pgfpathmoveto{\pgfqpoint{5.652068in}{0.385400in}}%
\pgfpathlineto{\pgfqpoint{5.652068in}{4.081400in}}%
\pgfusepath{stroke}%
\end{pgfscope}%
\begin{pgfscope}%
\pgfsetrectcap%
\pgfsetmiterjoin%
\pgfsetlinewidth{0.803000pt}%
\definecolor{currentstroke}{rgb}{0.000000,0.000000,0.000000}%
\pgfsetstrokecolor{currentstroke}%
\pgfsetdash{}{0pt}%
\pgfpathmoveto{\pgfqpoint{0.692068in}{0.385400in}}%
\pgfpathlineto{\pgfqpoint{5.652068in}{0.385400in}}%
\pgfusepath{stroke}%
\end{pgfscope}%
\begin{pgfscope}%
\pgfsetrectcap%
\pgfsetmiterjoin%
\pgfsetlinewidth{0.803000pt}%
\definecolor{currentstroke}{rgb}{0.000000,0.000000,0.000000}%
\pgfsetstrokecolor{currentstroke}%
\pgfsetdash{}{0pt}%
\pgfpathmoveto{\pgfqpoint{0.692068in}{4.081400in}}%
\pgfpathlineto{\pgfqpoint{5.652068in}{4.081400in}}%
\pgfusepath{stroke}%
\end{pgfscope}%
\end{pgfpicture}%
\makeatother%
\endgroup%
}
    \resizebox{0.49\textwidth}{6cm}{%% Creator: Matplotlib, PGF backend
%%
%% To include the figure in your LaTeX document, write
%%   \input{<filename>.pgf}
%%
%% Make sure the required packages are loaded in your preamble
%%   \usepackage{pgf}
%%
%% Figures using additional raster images can only be included by \input if
%% they are in the same directory as the main LaTeX file. For loading figures
%% from other directories you can use the `import` package
%%   \usepackage{import}
%%
%% and then include the figures with
%%   \import{<path to file>}{<filename>.pgf}
%%
%% Matplotlib used the following preamble
%%   \usepackage{fontspec}
%%   \setmainfont{DejaVuSerif.ttf}[Path=\detokenize{/Users/apple/opt/anaconda3/envs/kaolin/lib/python3.7/site-packages/matplotlib/mpl-data/fonts/ttf/}]
%%   \setsansfont{DejaVuSans.ttf}[Path=\detokenize{/Users/apple/opt/anaconda3/envs/kaolin/lib/python3.7/site-packages/matplotlib/mpl-data/fonts/ttf/}]
%%   \setmonofont{DejaVuSansMono.ttf}[Path=\detokenize{/Users/apple/opt/anaconda3/envs/kaolin/lib/python3.7/site-packages/matplotlib/mpl-data/fonts/ttf/}]
%%
\begingroup%
\makeatletter%
\begin{pgfpicture}%
\pgfpathrectangle{\pgfpointorigin}{\pgfqpoint{5.628357in}{4.181400in}}%
\pgfusepath{use as bounding box, clip}%
\begin{pgfscope}%
\pgfsetbuttcap%
\pgfsetmiterjoin%
\definecolor{currentfill}{rgb}{1.000000,1.000000,1.000000}%
\pgfsetfillcolor{currentfill}%
\pgfsetlinewidth{0.000000pt}%
\definecolor{currentstroke}{rgb}{1.000000,1.000000,1.000000}%
\pgfsetstrokecolor{currentstroke}%
\pgfsetdash{}{0pt}%
\pgfpathmoveto{\pgfqpoint{-0.000000in}{0.000000in}}%
\pgfpathlineto{\pgfqpoint{5.628357in}{0.000000in}}%
\pgfpathlineto{\pgfqpoint{5.628357in}{4.181400in}}%
\pgfpathlineto{\pgfqpoint{-0.000000in}{4.181400in}}%
\pgfpathclose%
\pgfusepath{fill}%
\end{pgfscope}%
\begin{pgfscope}%
\pgfsetbuttcap%
\pgfsetmiterjoin%
\definecolor{currentfill}{rgb}{1.000000,1.000000,1.000000}%
\pgfsetfillcolor{currentfill}%
\pgfsetlinewidth{0.000000pt}%
\definecolor{currentstroke}{rgb}{0.000000,0.000000,0.000000}%
\pgfsetstrokecolor{currentstroke}%
\pgfsetstrokeopacity{0.000000}%
\pgfsetdash{}{0pt}%
\pgfpathmoveto{\pgfqpoint{0.568357in}{0.385400in}}%
\pgfpathlineto{\pgfqpoint{5.528357in}{0.385400in}}%
\pgfpathlineto{\pgfqpoint{5.528357in}{4.081400in}}%
\pgfpathlineto{\pgfqpoint{0.568357in}{4.081400in}}%
\pgfpathclose%
\pgfusepath{fill}%
\end{pgfscope}%
\begin{pgfscope}%
\pgfpathrectangle{\pgfqpoint{0.568357in}{0.385400in}}{\pgfqpoint{4.960000in}{3.696000in}}%
\pgfusepath{clip}%
\pgfsetbuttcap%
\pgfsetmiterjoin%
\definecolor{currentfill}{rgb}{0.121569,0.466667,0.705882}%
\pgfsetfillcolor{currentfill}%
\pgfsetlinewidth{0.000000pt}%
\definecolor{currentstroke}{rgb}{0.000000,0.000000,0.000000}%
\pgfsetstrokecolor{currentstroke}%
\pgfsetstrokeopacity{0.000000}%
\pgfsetdash{}{0pt}%
\pgfpathmoveto{\pgfqpoint{0.793811in}{0.385400in}}%
\pgfpathlineto{\pgfqpoint{1.324292in}{0.385400in}}%
\pgfpathlineto{\pgfqpoint{1.324292in}{0.559815in}}%
\pgfpathlineto{\pgfqpoint{0.793811in}{0.559815in}}%
\pgfpathclose%
\pgfusepath{fill}%
\end{pgfscope}%
\begin{pgfscope}%
\pgfpathrectangle{\pgfqpoint{0.568357in}{0.385400in}}{\pgfqpoint{4.960000in}{3.696000in}}%
\pgfusepath{clip}%
\pgfsetbuttcap%
\pgfsetmiterjoin%
\definecolor{currentfill}{rgb}{0.121569,0.466667,0.705882}%
\pgfsetfillcolor{currentfill}%
\pgfsetlinewidth{0.000000pt}%
\definecolor{currentstroke}{rgb}{0.000000,0.000000,0.000000}%
\pgfsetstrokecolor{currentstroke}%
\pgfsetstrokeopacity{0.000000}%
\pgfsetdash{}{0pt}%
\pgfpathmoveto{\pgfqpoint{1.456913in}{0.385400in}}%
\pgfpathlineto{\pgfqpoint{1.987394in}{0.385400in}}%
\pgfpathlineto{\pgfqpoint{1.987394in}{0.718373in}}%
\pgfpathlineto{\pgfqpoint{1.456913in}{0.718373in}}%
\pgfpathclose%
\pgfusepath{fill}%
\end{pgfscope}%
\begin{pgfscope}%
\pgfpathrectangle{\pgfqpoint{0.568357in}{0.385400in}}{\pgfqpoint{4.960000in}{3.696000in}}%
\pgfusepath{clip}%
\pgfsetbuttcap%
\pgfsetmiterjoin%
\definecolor{currentfill}{rgb}{0.121569,0.466667,0.705882}%
\pgfsetfillcolor{currentfill}%
\pgfsetlinewidth{0.000000pt}%
\definecolor{currentstroke}{rgb}{0.000000,0.000000,0.000000}%
\pgfsetstrokecolor{currentstroke}%
\pgfsetstrokeopacity{0.000000}%
\pgfsetdash{}{0pt}%
\pgfpathmoveto{\pgfqpoint{2.120014in}{0.385400in}}%
\pgfpathlineto{\pgfqpoint{2.650496in}{0.385400in}}%
\pgfpathlineto{\pgfqpoint{2.650496in}{0.765941in}}%
\pgfpathlineto{\pgfqpoint{2.120014in}{0.765941in}}%
\pgfpathclose%
\pgfusepath{fill}%
\end{pgfscope}%
\begin{pgfscope}%
\pgfpathrectangle{\pgfqpoint{0.568357in}{0.385400in}}{\pgfqpoint{4.960000in}{3.696000in}}%
\pgfusepath{clip}%
\pgfsetbuttcap%
\pgfsetmiterjoin%
\definecolor{currentfill}{rgb}{0.121569,0.466667,0.705882}%
\pgfsetfillcolor{currentfill}%
\pgfsetlinewidth{0.000000pt}%
\definecolor{currentstroke}{rgb}{0.000000,0.000000,0.000000}%
\pgfsetstrokecolor{currentstroke}%
\pgfsetstrokeopacity{0.000000}%
\pgfsetdash{}{0pt}%
\pgfpathmoveto{\pgfqpoint{2.783116in}{0.385400in}}%
\pgfpathlineto{\pgfqpoint{3.313597in}{0.385400in}}%
\pgfpathlineto{\pgfqpoint{3.313597in}{3.905400in}}%
\pgfpathlineto{\pgfqpoint{2.783116in}{3.905400in}}%
\pgfpathclose%
\pgfusepath{fill}%
\end{pgfscope}%
\begin{pgfscope}%
\pgfpathrectangle{\pgfqpoint{0.568357in}{0.385400in}}{\pgfqpoint{4.960000in}{3.696000in}}%
\pgfusepath{clip}%
\pgfsetbuttcap%
\pgfsetmiterjoin%
\definecolor{currentfill}{rgb}{0.121569,0.466667,0.705882}%
\pgfsetfillcolor{currentfill}%
\pgfsetlinewidth{0.000000pt}%
\definecolor{currentstroke}{rgb}{0.000000,0.000000,0.000000}%
\pgfsetstrokecolor{currentstroke}%
\pgfsetstrokeopacity{0.000000}%
\pgfsetdash{}{0pt}%
\pgfpathmoveto{\pgfqpoint{3.446218in}{0.385400in}}%
\pgfpathlineto{\pgfqpoint{3.976699in}{0.385400in}}%
\pgfpathlineto{\pgfqpoint{3.976699in}{0.670806in}}%
\pgfpathlineto{\pgfqpoint{3.446218in}{0.670806in}}%
\pgfpathclose%
\pgfusepath{fill}%
\end{pgfscope}%
\begin{pgfscope}%
\pgfpathrectangle{\pgfqpoint{0.568357in}{0.385400in}}{\pgfqpoint{4.960000in}{3.696000in}}%
\pgfusepath{clip}%
\pgfsetbuttcap%
\pgfsetmiterjoin%
\definecolor{currentfill}{rgb}{0.121569,0.466667,0.705882}%
\pgfsetfillcolor{currentfill}%
\pgfsetlinewidth{0.000000pt}%
\definecolor{currentstroke}{rgb}{0.000000,0.000000,0.000000}%
\pgfsetstrokecolor{currentstroke}%
\pgfsetstrokeopacity{0.000000}%
\pgfsetdash{}{0pt}%
\pgfpathmoveto{\pgfqpoint{4.109319in}{0.385400in}}%
\pgfpathlineto{\pgfqpoint{4.639801in}{0.385400in}}%
\pgfpathlineto{\pgfqpoint{4.639801in}{0.718373in}}%
\pgfpathlineto{\pgfqpoint{4.109319in}{0.718373in}}%
\pgfpathclose%
\pgfusepath{fill}%
\end{pgfscope}%
\begin{pgfscope}%
\pgfpathrectangle{\pgfqpoint{0.568357in}{0.385400in}}{\pgfqpoint{4.960000in}{3.696000in}}%
\pgfusepath{clip}%
\pgfsetbuttcap%
\pgfsetmiterjoin%
\definecolor{currentfill}{rgb}{0.121569,0.466667,0.705882}%
\pgfsetfillcolor{currentfill}%
\pgfsetlinewidth{0.000000pt}%
\definecolor{currentstroke}{rgb}{0.000000,0.000000,0.000000}%
\pgfsetstrokecolor{currentstroke}%
\pgfsetstrokeopacity{0.000000}%
\pgfsetdash{}{0pt}%
\pgfpathmoveto{\pgfqpoint{4.772421in}{0.385400in}}%
\pgfpathlineto{\pgfqpoint{5.302902in}{0.385400in}}%
\pgfpathlineto{\pgfqpoint{5.302902in}{1.400175in}}%
\pgfpathlineto{\pgfqpoint{4.772421in}{1.400175in}}%
\pgfpathclose%
\pgfusepath{fill}%
\end{pgfscope}%
\begin{pgfscope}%
\pgfsetbuttcap%
\pgfsetroundjoin%
\definecolor{currentfill}{rgb}{0.000000,0.000000,0.000000}%
\pgfsetfillcolor{currentfill}%
\pgfsetlinewidth{0.803000pt}%
\definecolor{currentstroke}{rgb}{0.000000,0.000000,0.000000}%
\pgfsetstrokecolor{currentstroke}%
\pgfsetdash{}{0pt}%
\pgfsys@defobject{currentmarker}{\pgfqpoint{0.000000in}{-0.048611in}}{\pgfqpoint{0.000000in}{0.000000in}}{%
\pgfpathmoveto{\pgfqpoint{0.000000in}{0.000000in}}%
\pgfpathlineto{\pgfqpoint{0.000000in}{-0.048611in}}%
\pgfusepath{stroke,fill}%
}%
\begin{pgfscope}%
\pgfsys@transformshift{1.059052in}{0.385400in}%
\pgfsys@useobject{currentmarker}{}%
\end{pgfscope}%
\end{pgfscope}%
\begin{pgfscope}%
\definecolor{textcolor}{rgb}{0.000000,0.000000,0.000000}%
\pgfsetstrokecolor{textcolor}%
\pgfsetfillcolor{textcolor}%
\pgftext[x=1.059052in,y=0.288178in,,top]{\color{textcolor}\sffamily\fontsize{14.000000}{16.800000}\selectfont wardrobe}%
\end{pgfscope}%
\begin{pgfscope}%
\pgfsetbuttcap%
\pgfsetroundjoin%
\definecolor{currentfill}{rgb}{0.000000,0.000000,0.000000}%
\pgfsetfillcolor{currentfill}%
\pgfsetlinewidth{0.803000pt}%
\definecolor{currentstroke}{rgb}{0.000000,0.000000,0.000000}%
\pgfsetstrokecolor{currentstroke}%
\pgfsetdash{}{0pt}%
\pgfsys@defobject{currentmarker}{\pgfqpoint{0.000000in}{-0.048611in}}{\pgfqpoint{0.000000in}{0.000000in}}{%
\pgfpathmoveto{\pgfqpoint{0.000000in}{0.000000in}}%
\pgfpathlineto{\pgfqpoint{0.000000in}{-0.048611in}}%
\pgfusepath{stroke,fill}%
}%
\begin{pgfscope}%
\pgfsys@transformshift{1.722153in}{0.385400in}%
\pgfsys@useobject{currentmarker}{}%
\end{pgfscope}%
\end{pgfscope}%
\begin{pgfscope}%
\definecolor{textcolor}{rgb}{0.000000,0.000000,0.000000}%
\pgfsetstrokecolor{textcolor}%
\pgfsetfillcolor{textcolor}%
\pgftext[x=1.722153in,y=0.288178in,,top]{\color{textcolor}\sffamily\fontsize{14.000000}{16.800000}\selectfont bed}%
\end{pgfscope}%
\begin{pgfscope}%
\pgfsetbuttcap%
\pgfsetroundjoin%
\definecolor{currentfill}{rgb}{0.000000,0.000000,0.000000}%
\pgfsetfillcolor{currentfill}%
\pgfsetlinewidth{0.803000pt}%
\definecolor{currentstroke}{rgb}{0.000000,0.000000,0.000000}%
\pgfsetstrokecolor{currentstroke}%
\pgfsetdash{}{0pt}%
\pgfsys@defobject{currentmarker}{\pgfqpoint{0.000000in}{-0.048611in}}{\pgfqpoint{0.000000in}{0.000000in}}{%
\pgfpathmoveto{\pgfqpoint{0.000000in}{0.000000in}}%
\pgfpathlineto{\pgfqpoint{0.000000in}{-0.048611in}}%
\pgfusepath{stroke,fill}%
}%
\begin{pgfscope}%
\pgfsys@transformshift{2.385255in}{0.385400in}%
\pgfsys@useobject{currentmarker}{}%
\end{pgfscope}%
\end{pgfscope}%
\begin{pgfscope}%
\definecolor{textcolor}{rgb}{0.000000,0.000000,0.000000}%
\pgfsetstrokecolor{textcolor}%
\pgfsetfillcolor{textcolor}%
\pgftext[x=2.385255in,y=0.288178in,,top]{\color{textcolor}\sffamily\fontsize{14.000000}{16.800000}\selectfont desk}%
\end{pgfscope}%
\begin{pgfscope}%
\pgfsetbuttcap%
\pgfsetroundjoin%
\definecolor{currentfill}{rgb}{0.000000,0.000000,0.000000}%
\pgfsetfillcolor{currentfill}%
\pgfsetlinewidth{0.803000pt}%
\definecolor{currentstroke}{rgb}{0.000000,0.000000,0.000000}%
\pgfsetstrokecolor{currentstroke}%
\pgfsetdash{}{0pt}%
\pgfsys@defobject{currentmarker}{\pgfqpoint{0.000000in}{-0.048611in}}{\pgfqpoint{0.000000in}{0.000000in}}{%
\pgfpathmoveto{\pgfqpoint{0.000000in}{0.000000in}}%
\pgfpathlineto{\pgfqpoint{0.000000in}{-0.048611in}}%
\pgfusepath{stroke,fill}%
}%
\begin{pgfscope}%
\pgfsys@transformshift{3.048357in}{0.385400in}%
\pgfsys@useobject{currentmarker}{}%
\end{pgfscope}%
\end{pgfscope}%
\begin{pgfscope}%
\definecolor{textcolor}{rgb}{0.000000,0.000000,0.000000}%
\pgfsetstrokecolor{textcolor}%
\pgfsetfillcolor{textcolor}%
\pgftext[x=3.048357in,y=0.288178in,,top]{\color{textcolor}\sffamily\fontsize{14.000000}{16.800000}\selectfont chair}%
\end{pgfscope}%
\begin{pgfscope}%
\pgfsetbuttcap%
\pgfsetroundjoin%
\definecolor{currentfill}{rgb}{0.000000,0.000000,0.000000}%
\pgfsetfillcolor{currentfill}%
\pgfsetlinewidth{0.803000pt}%
\definecolor{currentstroke}{rgb}{0.000000,0.000000,0.000000}%
\pgfsetstrokecolor{currentstroke}%
\pgfsetdash{}{0pt}%
\pgfsys@defobject{currentmarker}{\pgfqpoint{0.000000in}{-0.048611in}}{\pgfqpoint{0.000000in}{0.000000in}}{%
\pgfpathmoveto{\pgfqpoint{0.000000in}{0.000000in}}%
\pgfpathlineto{\pgfqpoint{0.000000in}{-0.048611in}}%
\pgfusepath{stroke,fill}%
}%
\begin{pgfscope}%
\pgfsys@transformshift{3.711458in}{0.385400in}%
\pgfsys@useobject{currentmarker}{}%
\end{pgfscope}%
\end{pgfscope}%
\begin{pgfscope}%
\definecolor{textcolor}{rgb}{0.000000,0.000000,0.000000}%
\pgfsetstrokecolor{textcolor}%
\pgfsetfillcolor{textcolor}%
\pgftext[x=3.711458in,y=0.288178in,,top]{\color{textcolor}\sffamily\fontsize{14.000000}{16.800000}\selectfont bookcase}%
\end{pgfscope}%
\begin{pgfscope}%
\pgfsetbuttcap%
\pgfsetroundjoin%
\definecolor{currentfill}{rgb}{0.000000,0.000000,0.000000}%
\pgfsetfillcolor{currentfill}%
\pgfsetlinewidth{0.803000pt}%
\definecolor{currentstroke}{rgb}{0.000000,0.000000,0.000000}%
\pgfsetstrokecolor{currentstroke}%
\pgfsetdash{}{0pt}%
\pgfsys@defobject{currentmarker}{\pgfqpoint{0.000000in}{-0.048611in}}{\pgfqpoint{0.000000in}{0.000000in}}{%
\pgfpathmoveto{\pgfqpoint{0.000000in}{0.000000in}}%
\pgfpathlineto{\pgfqpoint{0.000000in}{-0.048611in}}%
\pgfusepath{stroke,fill}%
}%
\begin{pgfscope}%
\pgfsys@transformshift{4.374560in}{0.385400in}%
\pgfsys@useobject{currentmarker}{}%
\end{pgfscope}%
\end{pgfscope}%
\begin{pgfscope}%
\definecolor{textcolor}{rgb}{0.000000,0.000000,0.000000}%
\pgfsetstrokecolor{textcolor}%
\pgfsetfillcolor{textcolor}%
\pgftext[x=4.374560in,y=0.288178in,,top]{\color{textcolor}\sffamily\fontsize{14.000000}{16.800000}\selectfont sofa}%
\end{pgfscope}%
\begin{pgfscope}%
\pgfsetbuttcap%
\pgfsetroundjoin%
\definecolor{currentfill}{rgb}{0.000000,0.000000,0.000000}%
\pgfsetfillcolor{currentfill}%
\pgfsetlinewidth{0.803000pt}%
\definecolor{currentstroke}{rgb}{0.000000,0.000000,0.000000}%
\pgfsetstrokecolor{currentstroke}%
\pgfsetdash{}{0pt}%
\pgfsys@defobject{currentmarker}{\pgfqpoint{0.000000in}{-0.048611in}}{\pgfqpoint{0.000000in}{0.000000in}}{%
\pgfpathmoveto{\pgfqpoint{0.000000in}{0.000000in}}%
\pgfpathlineto{\pgfqpoint{0.000000in}{-0.048611in}}%
\pgfusepath{stroke,fill}%
}%
\begin{pgfscope}%
\pgfsys@transformshift{5.037661in}{0.385400in}%
\pgfsys@useobject{currentmarker}{}%
\end{pgfscope}%
\end{pgfscope}%
\begin{pgfscope}%
\definecolor{textcolor}{rgb}{0.000000,0.000000,0.000000}%
\pgfsetstrokecolor{textcolor}%
\pgfsetfillcolor{textcolor}%
\pgftext[x=5.037661in,y=0.288178in,,top]{\color{textcolor}\sffamily\fontsize{14.000000}{16.800000}\selectfont table}%
\end{pgfscope}%
\begin{pgfscope}%
\pgfsetbuttcap%
\pgfsetroundjoin%
\definecolor{currentfill}{rgb}{0.000000,0.000000,0.000000}%
\pgfsetfillcolor{currentfill}%
\pgfsetlinewidth{0.803000pt}%
\definecolor{currentstroke}{rgb}{0.000000,0.000000,0.000000}%
\pgfsetstrokecolor{currentstroke}%
\pgfsetdash{}{0pt}%
\pgfsys@defobject{currentmarker}{\pgfqpoint{-0.048611in}{0.000000in}}{\pgfqpoint{-0.000000in}{0.000000in}}{%
\pgfpathmoveto{\pgfqpoint{-0.000000in}{0.000000in}}%
\pgfpathlineto{\pgfqpoint{-0.048611in}{0.000000in}}%
\pgfusepath{stroke,fill}%
}%
\begin{pgfscope}%
\pgfsys@transformshift{0.568357in}{0.385400in}%
\pgfsys@useobject{currentmarker}{}%
\end{pgfscope}%
\end{pgfscope}%
\begin{pgfscope}%
\definecolor{textcolor}{rgb}{0.000000,0.000000,0.000000}%
\pgfsetstrokecolor{textcolor}%
\pgfsetfillcolor{textcolor}%
\pgftext[x=0.347423in, y=0.311534in, left, base]{\color{textcolor}\sffamily\fontsize{14.000000}{16.800000}\selectfont 0}%
\end{pgfscope}%
\begin{pgfscope}%
\pgfsetbuttcap%
\pgfsetroundjoin%
\definecolor{currentfill}{rgb}{0.000000,0.000000,0.000000}%
\pgfsetfillcolor{currentfill}%
\pgfsetlinewidth{0.803000pt}%
\definecolor{currentstroke}{rgb}{0.000000,0.000000,0.000000}%
\pgfsetstrokecolor{currentstroke}%
\pgfsetdash{}{0pt}%
\pgfsys@defobject{currentmarker}{\pgfqpoint{-0.048611in}{0.000000in}}{\pgfqpoint{-0.000000in}{0.000000in}}{%
\pgfpathmoveto{\pgfqpoint{-0.000000in}{0.000000in}}%
\pgfpathlineto{\pgfqpoint{-0.048611in}{0.000000in}}%
\pgfusepath{stroke,fill}%
}%
\begin{pgfscope}%
\pgfsys@transformshift{0.568357in}{1.178193in}%
\pgfsys@useobject{currentmarker}{}%
\end{pgfscope}%
\end{pgfscope}%
\begin{pgfscope}%
\definecolor{textcolor}{rgb}{0.000000,0.000000,0.000000}%
\pgfsetstrokecolor{textcolor}%
\pgfsetfillcolor{textcolor}%
\pgftext[x=0.223712in, y=1.104327in, left, base]{\color{textcolor}\sffamily\fontsize{14.000000}{16.800000}\selectfont 50}%
\end{pgfscope}%
\begin{pgfscope}%
\pgfsetbuttcap%
\pgfsetroundjoin%
\definecolor{currentfill}{rgb}{0.000000,0.000000,0.000000}%
\pgfsetfillcolor{currentfill}%
\pgfsetlinewidth{0.803000pt}%
\definecolor{currentstroke}{rgb}{0.000000,0.000000,0.000000}%
\pgfsetstrokecolor{currentstroke}%
\pgfsetdash{}{0pt}%
\pgfsys@defobject{currentmarker}{\pgfqpoint{-0.048611in}{0.000000in}}{\pgfqpoint{-0.000000in}{0.000000in}}{%
\pgfpathmoveto{\pgfqpoint{-0.000000in}{0.000000in}}%
\pgfpathlineto{\pgfqpoint{-0.048611in}{0.000000in}}%
\pgfusepath{stroke,fill}%
}%
\begin{pgfscope}%
\pgfsys@transformshift{0.568357in}{1.970986in}%
\pgfsys@useobject{currentmarker}{}%
\end{pgfscope}%
\end{pgfscope}%
\begin{pgfscope}%
\definecolor{textcolor}{rgb}{0.000000,0.000000,0.000000}%
\pgfsetstrokecolor{textcolor}%
\pgfsetfillcolor{textcolor}%
\pgftext[x=0.100000in, y=1.897120in, left, base]{\color{textcolor}\sffamily\fontsize{14.000000}{16.800000}\selectfont 100}%
\end{pgfscope}%
\begin{pgfscope}%
\pgfsetbuttcap%
\pgfsetroundjoin%
\definecolor{currentfill}{rgb}{0.000000,0.000000,0.000000}%
\pgfsetfillcolor{currentfill}%
\pgfsetlinewidth{0.803000pt}%
\definecolor{currentstroke}{rgb}{0.000000,0.000000,0.000000}%
\pgfsetstrokecolor{currentstroke}%
\pgfsetdash{}{0pt}%
\pgfsys@defobject{currentmarker}{\pgfqpoint{-0.048611in}{0.000000in}}{\pgfqpoint{-0.000000in}{0.000000in}}{%
\pgfpathmoveto{\pgfqpoint{-0.000000in}{0.000000in}}%
\pgfpathlineto{\pgfqpoint{-0.048611in}{0.000000in}}%
\pgfusepath{stroke,fill}%
}%
\begin{pgfscope}%
\pgfsys@transformshift{0.568357in}{2.763779in}%
\pgfsys@useobject{currentmarker}{}%
\end{pgfscope}%
\end{pgfscope}%
\begin{pgfscope}%
\definecolor{textcolor}{rgb}{0.000000,0.000000,0.000000}%
\pgfsetstrokecolor{textcolor}%
\pgfsetfillcolor{textcolor}%
\pgftext[x=0.100000in, y=2.689913in, left, base]{\color{textcolor}\sffamily\fontsize{14.000000}{16.800000}\selectfont 150}%
\end{pgfscope}%
\begin{pgfscope}%
\pgfsetbuttcap%
\pgfsetroundjoin%
\definecolor{currentfill}{rgb}{0.000000,0.000000,0.000000}%
\pgfsetfillcolor{currentfill}%
\pgfsetlinewidth{0.803000pt}%
\definecolor{currentstroke}{rgb}{0.000000,0.000000,0.000000}%
\pgfsetstrokecolor{currentstroke}%
\pgfsetdash{}{0pt}%
\pgfsys@defobject{currentmarker}{\pgfqpoint{-0.048611in}{0.000000in}}{\pgfqpoint{-0.000000in}{0.000000in}}{%
\pgfpathmoveto{\pgfqpoint{-0.000000in}{0.000000in}}%
\pgfpathlineto{\pgfqpoint{-0.048611in}{0.000000in}}%
\pgfusepath{stroke,fill}%
}%
\begin{pgfscope}%
\pgfsys@transformshift{0.568357in}{3.556571in}%
\pgfsys@useobject{currentmarker}{}%
\end{pgfscope}%
\end{pgfscope}%
\begin{pgfscope}%
\definecolor{textcolor}{rgb}{0.000000,0.000000,0.000000}%
\pgfsetstrokecolor{textcolor}%
\pgfsetfillcolor{textcolor}%
\pgftext[x=0.100000in, y=3.482705in, left, base]{\color{textcolor}\sffamily\fontsize{14.000000}{16.800000}\selectfont 200}%
\end{pgfscope}%
\begin{pgfscope}%
\pgfsetrectcap%
\pgfsetmiterjoin%
\pgfsetlinewidth{0.803000pt}%
\definecolor{currentstroke}{rgb}{0.000000,0.000000,0.000000}%
\pgfsetstrokecolor{currentstroke}%
\pgfsetdash{}{0pt}%
\pgfpathmoveto{\pgfqpoint{0.568357in}{0.385400in}}%
\pgfpathlineto{\pgfqpoint{0.568357in}{4.081400in}}%
\pgfusepath{stroke}%
\end{pgfscope}%
\begin{pgfscope}%
\pgfsetrectcap%
\pgfsetmiterjoin%
\pgfsetlinewidth{0.803000pt}%
\definecolor{currentstroke}{rgb}{0.000000,0.000000,0.000000}%
\pgfsetstrokecolor{currentstroke}%
\pgfsetdash{}{0pt}%
\pgfpathmoveto{\pgfqpoint{5.528357in}{0.385400in}}%
\pgfpathlineto{\pgfqpoint{5.528357in}{4.081400in}}%
\pgfusepath{stroke}%
\end{pgfscope}%
\begin{pgfscope}%
\pgfsetrectcap%
\pgfsetmiterjoin%
\pgfsetlinewidth{0.803000pt}%
\definecolor{currentstroke}{rgb}{0.000000,0.000000,0.000000}%
\pgfsetstrokecolor{currentstroke}%
\pgfsetdash{}{0pt}%
\pgfpathmoveto{\pgfqpoint{0.568357in}{0.385400in}}%
\pgfpathlineto{\pgfqpoint{5.528357in}{0.385400in}}%
\pgfusepath{stroke}%
\end{pgfscope}%
\begin{pgfscope}%
\pgfsetrectcap%
\pgfsetmiterjoin%
\pgfsetlinewidth{0.803000pt}%
\definecolor{currentstroke}{rgb}{0.000000,0.000000,0.000000}%
\pgfsetstrokecolor{currentstroke}%
\pgfsetdash{}{0pt}%
\pgfpathmoveto{\pgfqpoint{0.568357in}{4.081400in}}%
\pgfpathlineto{\pgfqpoint{5.528357in}{4.081400in}}%
\pgfusepath{stroke}%
\end{pgfscope}%
\end{pgfpicture}%
\makeatother%
\endgroup%
}
    \caption{Distribution of Pix3D~\cite{pix3d} images(left), unique models(right)}
    \label{fig:pix3d_histogram}
\end{figure}

\subsection{Why Pix3D?}\label{subsec:why-pix3d?}
Pix3D is a collection of indoor scenes with a complex background, varying light conditions with shadows, reflective surfaces, and even varying occlusion levels.
Each image comprises a collection of objects in the scene, but only one object from the category is annotated.
It is a perfect example of having limited real-world data.
Since 3D models are available for each piece of furniture, synthetic data can be generated in abundance from those models.
Samples from Pix2D are as shown in \autoref{fig:Pix3D samples}.

\begin{figure}[!ht]
    \centering
    \subfloat[][]{\includegraphics[width=.4\textwidth]{/Users/apple/OVGU/Thesis/code/3dReconstruction/report/images/pix3d/pix3d_5}}\quad
    \subfloat[][]{\includegraphics[width=.4\textwidth]{/Users/apple/OVGU/Thesis/code/3dReconstruction/report/images/pix3d/pix3d_2}}\\
    \subfloat[][]{\includegraphics[width=.4\textwidth]{/Users/apple/OVGU/Thesis/code/3dReconstruction/report/images/pix3d/pix3d_6}}\quad
    \subfloat[][]{\includegraphics[width=.4\textwidth]{/Users/apple/OVGU/Thesis/code/3dReconstruction/report/images/pix3d/pix3d_4}}
    \caption{Sample RGB images from pix3D}
    \label{fig:Pix3D samples}
\end{figure}

\section{Role of SceneNet}\label{sec:role-of-scenenet}
SceneNet~\cite{McCormac:etal:ICCV2017} is an extensive collection of photorealistic indoor scene trajectories.
The dataset provides images and videos of indoor scenes that can be used for tasks like SLAM,
semantic and instance segmentation, object detection, and further enhanced for other vision problems like optical flow depth and pose estimation ~\cite{McCormac:etal:ICCV2017}).
They use ShapeNet~\cite{chang2015shapenet} models to occupy 57 indoor scenes, giving the scene unlimited configurations.
Unfortunately, there is no mapping from scene to 3D model for tasks like 3D reconstructions.
In our approach, we utilize the scene provided by SceneNet as a layout for our indoor scenes.
Each scene includes initial Shapenet models with furniture placement.
We then replace the class under observation with a corresponding model from Pix3D.

\section{Unity-based pipeline}\label{sec:unity-based-pipeline}
To create the synthetic dataset, we use Unity Game Engine.
The reasons for selecting Unity as our platform for the application are:
\begin{enumerate}
    \item Cross-platform game engine and hence usable on any Operating system.
    \item The basic version is available for free.
    \item Provides all the necessary tools to create an ersatz environment with well-maintained documentation.
    \item An active developer community.
\end{enumerate}

There is no official comparison between Unreal Engine and Unity engine to have a deciding factor.
The selection of the Unity engine was a purely individual choice and ease of use.
However, both these game engines can create realistic-looking scenes that are usable for synthetic dataset generation.
With the Unity game engine, the available scenes from the Scenenet and 3D models from Pix3D can be imported to form an ersatz environment.
Further domain randomization can be applied to create a dataset of photorealistic images and 2.5D data like normals, depth maps, masks.

\subsection{Domain Randomization with Unity Engine}\label{subsec:domain-randomisation-with-unity-engine}
In this section, we discuss what kind of domain randomization we apply for dataset creation.

\subsubsection{Scenes from SceneNet}\label{subsubsec:scenes-from-scenenet}
As discussed in \autoref{sec:role-of-scenenet}, SceneNet provides a wide variety of indoor scenes, including bedrooms, bathrooms, living rooms, kitchens, and offices.
We did not consider the bathroom for our case, as the categories under observations are furniture which we rarely see in the bathroom.
For the rest of the scenes, we have a pre-determined setup given by SceneNet.
We use 25 scenes in total as basic rooms to insert our target object.
Moreover, we feel this should be sufficient to check if Unity can produce a helpful dataset with further randomization.

\subsubsection{Camera Viewpoints}
The position and orientation of the camera give us the most randomization in this setup.
The user can choose the minimum and maximum distance to place the camera from the target object.
A random point is selected in this range as the position for the camera.
The camera is oriented such that it always looks at the target object, irrespective of its position.
With this, we achieved different backgrounds for the target object, and the target object appears to be indifferent positions and orientations.

\subsubsection{Lighting and shadows}
Lighting plays a vital role in photorealism.
If the lighting is not set up correctly, the synthetic dataset will never seem to be photorealistic.
Unity offers a wide variety of lighting like global light, which acts like sunlight, and various indoor lighting systems.
Ideally, we should make the luminous objects like lamps, chandeliers, bulbs, etc.,
the source of light for indoor scenes, but we observed that the room does not light up uniformly,
making it less photorealistic.
Hence we use some pre-determined lighting settings, discussed further in implementation(\autoref{sec:3d-scene-framework}).

\subsubsection{Randomized textures}
SceneNet~\cite{McCormac:etal:ICCV2017} also provide textures for different categories in the scene.
We further increase the texture database by adding more textures from ambientCG.com, which provides free licenses.
We randomly allocate textures to each object, ensuring that each category has the same texture to make the scene more uniform.

\subsubsection{Replacing target objects}
To further randomize the scene, the category of target objects is replaced by the object under observation.
When more than one object of the same category is present, we randomize the object to be replaced, further randomizing the captured data.
The target object is scaled such that the least dimension of the target object matches the least dimension of the category object in the scene.
For example, if the length of the category object in the scene is the most petite amount length, width, and height, then the target object is scaled to match this length.
The rescaling makes the target object blend in with the scene.

\section{\gls{free}, a Pix3D based synthetic dataset}\label{sec:s2r:3d-free-a-pix3d-based-synthetic-dataset}

\gls{free} dataset, which stands for Synth2Real: 3-Dimensional Furniture Reconstruction from Ersatz Environment, combines SceneNet and Pix3D dataset.
We utilize the availability of 3D models of rooms and pieces of furniture from these two datasets to create an ersatz environment using Unity as our framework.
We randomize the indoor scenes from SceneNet~\cite{McCormac:etal:ICCV2017} and textures provided by them with some additional complex textures.
The other option would have been to use the ShapeNet as the target model, but we would not have a real dataset to compare to this case.
Ideally, a model trained on ShapeNet should also perform well with a real dataset like Pix3D, but we decided to have a synthetic dataset based on Pix3D itself for better comparison.

\section{3D reconstruction pipeline}\label{sec:3D reconstruction pipeline}
We create a pipeline for processing the 3D reconstruction task.
The backbone of the pipeline is the base model and the dataset being used to train.
In this section, we discuss the model used as a base and the rationale behind its selection.

\subsection{Pix2Vox and Pix2Vox++}\label{subsec:pix2vox-and-pix2vox++}
The architecture of pix2vox~\cite{Xie_2019} is as shown in \autoref{fig:architectures}(a).
The network consists of 4 modules: Encoder, Decoder, Merger, and Refiner.
Merger plays a significant part when it comes to the multi-view reconstruction of 3d objects.
As we focus on single-view 3d reconstruction, the merger module will not influence the output significantly.
As an encoder, pix2vox has utilized \gls{vgg}16 network~\cite{simonyan2015deep} pre-trained on ImageNet~\cite{Deng2009ImageNetAL}.
The decoder is an expansive network that converts 2D embeddings to a 3D voxel grid.
The refiner is an auto-encoder that takes the 3D output from the merger and produces a more refined final output.
A second paper based on pix2vox was titled Pix2Vox++~\cite{Xie_2020}.
The extended work replaces \gls{vgg} encoder~\cite{simonyan2015deep} with \gls{resnet}~\cite{He2016DeepRL}.


\begin{figure}[!ht]
    \centering
    \subfloat[][]{\includegraphics[angle=-90,width=0.45\linewidth]{/Users/apple/OVGU/Thesis/code/3dReconstruction/report/images/concept/pix2vox}}\quad
    \subfloat[][]{\includegraphics[angle=-90,width=0.45\linewidth]{/Users/apple/OVGU/Thesis/code/3dReconstruction/report/images/concept/pix2voxpp}}\\
    \caption{Network architectures used as a baselines (a)Pix2Vox (b)Pix2Vox++}
    \label{fig:architectures}
\end{figure}

\subsection{Why Pix2Vox?}\label{subsec:why-pix2vox?}
Pix2Vox has been used as a baseline by most of the research-oriented to 3D reconstruction.
This network is one of the few networks to be tested on the Pix3D dataset.
According to the survey conducted by~\cite{Han2021ImageBased3O}, the performance of pix2vox~\cite{Xie_2019}
is significantly higher compared to previous work(\cite{DBLP:journals/corr/TulsianiZEM17,tatarchenko2016multiview,richter2018matryoshka,gwak2017weakly,8265323}), as shown in \autoref{fig:survey on 3d reconstruction}.
At the same time, this comparison was made on 3d reconstruction of the ShapeNet dataset since Pix3D was not available when previous work was published.
From our survey, only CoReNet~\cite{popov2020corenet} had a slight gain in performance compared to pix2vox.
When trained on ShapeNet and tested with Pix3D, CoReNet gave a result of 29.7\% \gls{iou} while Pix2Vox gave a result of 28.8\% \gls{iou}  and Pix2Vox++ a result of 29.2\% \gls{iou}\@.
Since the difference in the performance was not significant, we decided to stick with the baseline model.
Another reason for selecting the pix2vox model is that the backbone of the architecture is pre-trained with ImageNet.
Hence, the embeddings generated from this encoder can help visualize the domain space of both Pix3D (real images)  and \gls{free}(synthetic images).
As mentioned above, for pix2vox++, the \gls{resnet} is the backbone encoder with 25\% lesser parameters and 5\% lesser inference time than \gls{vgg}\@.
In addition, the author even demonstrated that Pix2vox++ performs 1.5\% better than pix2vox.
The architecture of pix2vox++ is as shown in \autoref{fig:architectures}(b).
Furthermore, the focus of this thesis is not to check which is the best model to reconstruct the furnitures, but to check if game engines can produce photorealistic images usable for 3d reconstruction.
Hence the selection of the model was not of utmost importance.
However, since the two architectures are relatable, it would be interesting to compare the results for the 3D Reconstruction task.

%\begin{figure}
%    \centering
%    \includegraphics[width=\textwidth]{/Users/apple/OVGU/Thesis/code/3dReconstruction/report/images/concept/pix2vox}
%    \caption{Network architecture for pix2vox~\cite{Xie_2019}}
%    \label{fig:pix2vox architecture}
%\end{figure}
%
%\begin{figure}
%    \centering
%    \includegraphics[width=\textwidth]{/Users/apple/OVGU/Thesis/code/3dReconstruction/report/images/concept/pix2voxpp}
%    \caption{Network architecture for pix2vox++~\cite{Xie_2020}}
%    \label{fig:pix2voxpp architecture}
%\end{figure}

\begin{figure}[ht]
    \centering
    \includegraphics[width=\textwidth]{/Users/apple/OVGU/Thesis/code/3dReconstruction/report/images/concept/survey}
    \caption{A survey conducted by~\cite{Han2021ImageBased3O}, demonstrates that Pix2Vox is considerably a good 3D reconstruction model.
    The values are from 3D reconstruction of ShapeNet~\cite{chang2015shapenet} since Pix3D was not published by then.}
    \label{fig:survey on 3d reconstruction}
\end{figure}

%\section{Domain adaptation with Gan \todo{if we decide to go with this approach}}

\section{Domain space}\label{sec:domain-space}

Since this thesis includes image datasets from different domains, it is essential to understand the domain space of these datasets.
This section discusses how we intend to visualize the domain space and a quantitative measure to indicate the difference between the domains.

\subsection{Visualizing with \gls{tsne}}\label{subsec:visualizing-with-tsne}
\gls{tsne} stands for t-Distributed Stochastic Neighbor Embedding~\cite{vanDerMaaten2008}; it is a tool to visualize a high dimensional space in a low dimension space while retaining the information of high dimension as much as possible.
In our case, the images are embedded using a \gls{vgg}16 network with an output dimension of 1000.
This multi-dimensional output is converted to two-dimension for better visualization of the embedding space.

In this segment, we briefly explain how \gls{tsne} works.

\begin{enumerate}
    \item \textbf{Step 1}: Calculate the similarity between points in high dimensional space using a joint probability distribution.
        To achieve this, the Euclidean distances between each point is transformed into conditional probability distribution as in \autoref{eq:equation1}, which represents similarity between a pair of points.
        The conditional probability of a point $x_i$ with point $x_j$ to be located next to it, is represented by a Gaussian with center at $x_i$.
        Normalising is done to handle the different cluster densities.
        Conditional probability is converted to Joint probability using \autoref{eq:equation2}.
    \begin{equation}
        p_{j \mid i}=\frac{\exp \left(-\left\|x_{i}-x_{j}\right\|^{2} / 2 \sigma_{i}^{2}\right)}{\sum_{k \neq i} \exp \left(-\left\|x_{i}-x_{k}\right\|^{2} / 2 \sigma_{i}^{2}\right)}
        \label{eq:equation1}
    \end{equation}

    \begin{equation}
        p_{i j}=\frac{p_{j \mid i}+p_{i \mid j}}{2 n}
        \label{eq:equation2}
    \end{equation}

    \item \textbf{Step 2}: Using a random point dataset on a lower dimension with same number of points in higher dimension, a Joint probability distribution is created but with a Student-t distribution instead of Gaussian.
    This is represented by \autoref{eq:equation3}.
    t-distribution makes sure that the projection in low dimension is not concentrated on a single point.
    \begin{equation}
        q_{ij} = \frac{(1+\left \| y_{i}-y_{j} \right \|^{2})^{-1}}{\sum _{k\neq l} (1+\left \| y_{k}-y_{l} \right \|^{2})^{-1}}
        \label{eq:equation3}
    \end{equation}

    \item \textbf{Step 3}: Learn to represent the high dimension space in low dimension by making the Joint distribution of the lower dimension as close as possible to that of the higher dimension.
    For this Kullback-Leiber divergence\cite{Joyce2011} is used to compare the similarity between the two distributions as in \autoref{eq:equation4}.
    Using gradient descent, the cost function based on KL-divergence (\autoref{eq:equation5}) is minimized to make the points fall in place.

    \begin{equation}
        D_{\mathrm{KL}}(P \| Q)=\sum_{x \in \mathcal{X}} P(x) \log \left(\frac{P(x)}{Q(x)}\right)
        \label{eq:equation4}
    \end{equation}

    \begin{equation}
        C=K L(P \| Q)=\sum_{i} \sum_{j} p_{i j} \log \frac{p_{i j}}{q_{i j}}
        \label{eq:equation5}
    \end{equation}
\end{enumerate}

Perplexity is a parameter that informs the algorithm on how much attention to give the local and global features.
The publication~\cite{vanDerMaaten2008} indicates the values between 5 and 50 to be a good initialization.
An article on \gls{tsne} interpretation~\cite{wattenberg2016how} informs us to have a perplexity value lesser than the total number of points.
    It also notes that the cluster distance may not be as the intuition suggests, i.e., greater distance may not necessarily mean good separation, or lesser distance does not inevitably mean closer similarity for well-separated clusters.
    However, \gls{tsne} can still visualize if the clusters are overlapping or have a shared embedding space or not.
    Hence, we will use \gls{tsne} as a medium to visualize the embedding space without giving importance to the distance or the scale of the plots.

%\subsection{Maximum Mean Discrepancy(MMD)}
\subsection{Fr\'echet Inception Distance(FID))}\label{subsec:fr'echet-inception-distance)}
For quantitative measure, we use \gls{fid}
\gls{fid} \cite{Heusel2017GANsTB} was specifically formulated to evaluate the performance of Generative Adversarial Networks \cite{Goodfellow2014GenerativeAN}.
The core idea behind \gls{fid} was to evaluate the generated synthetic images using the statistics collected from both synthetic and real images.
Lower the value of \gls{fid} better the quality of the images.

The embeddings of the images are calculated by passing it through an Inception v3 model~\cite{Szegedy2016RethinkingTI} pre-trained on ImageNet~\cite{Deng2009ImageNetAL}.
The output layer is removed, and the penultimate pooling layer activations are used to get the embedding vector.
This vector has a dimension of 2048.
The images from both synthetic and real datasets are passed through the model, and the obtained statistics is used to calculated the \gls{fid} as in \autoref{eq:fid}.

\begin{equation}
    d^{2}\left((\boldsymbol{\mu}_{r}, \boldsymbol{C}_{r}),\left(\boldsymbol{\mu}_{s}, \boldsymbol{C}_{s}\right)\right)=\left\|\boldsymbol{\mu}_{r}-\boldsymbol{\mu}_{s}\right\|^{2}+\operatorname{Tr}\left(\boldsymbol{C}_{r}+\boldsymbol{C}_{s}-2\left(\boldsymbol{C}_{w} {C}_{w}\right)^{1 / 2}\right)
    \label{eq:fid}
\end{equation}

In the above \autoref{eq:fid}, $\mu_r$ and $\mu_s$ are feature-wise mean for real and synthetic images, $C_r$ and $C_s$ are respective covariance matrices.
$Tr$ is the Trace linear algebra operation, which is the sum of main diagonal elements.




\chapter{\iftoggle{german}{Implementierung}{Implementation}}\label{ch:implementation}

In this chapter we discuss the implementations of the all the framework and how it was acheived.
In~\ref{sec:3d-scene-framework}, we go through the implementation of each of the modes of operation.
Further, we discuss each component that can be used for domain randomization like scenes from SceneNet, camera viewpoints, textures, lighting conditions, and replacement of the target object.
We then explain how Ml-Image Synthesis library is integrated with the cameras to take the snapshots.
Some sample images are added for each of the components for the readers to better grasp its importance.

in~\ref{sec:3d-reconstruction-framework}, we discuss the Deep Learning framework used to train and evaluate the 3D reconstruction task.
\section{3D-Scene framework}\label{sec:3d-scene-framework}.
We discuss the underlying libraries and hyperparameters used for the evaluation of the models.

3D-Scene is a Unity-based application used in editor mode written in CSharp programming language.
The First step of the application is to import existing .obj files for 3D models, along with 3D models of rooms.
These rooms and furniture models can be randomly textured and placed.
In the final step, the main camera is used for taking snapshots of the scene.
Along with RGB images, depth maps, normals, semantic segmentation are also saved.
The pipeline is as shown in figure~\ref{fig:pipeline process}.
For an automation-based pipeline, the process loops through each step, while in the manual pipeline the user decides which block needs to be executed.

\begin{figure}[!ht]
    \centering
    \includegraphics[width=\textwidth]{/Users/apple/OVGU/Thesis/code/3dReconstruction/report/images/concept/process}
    \caption{Automated pipeline for image generation with different blocks and external libraries.}
    \label{fig:pipeline process}
\end{figure}

\subsection{Modes of operations}\label{subsec:modes-of-operations}
The creation of a dataset can either be automated or by manual intervention.
Automated images may not give us the perfect images which we expect.
There can be some bad lighting, unexpected intersections with other objects, unforeseen camera angles, etc.
To facilitate ease of use for the user, the application has 3 key pipelines.
After configuring the parameters to create an ersatz environment using Unity,
the user can select any of this pipeline to either get automatic snapshots or manually snaps for the models.

The different modes of operation are as follows:
\begin{enumerate}
\item Single Room pipeline
\item Manual Room pipeline
\item Multi Objects pipeline
\end{enumerate}

The single Room pipeline is used to create the dataset with objects in the center of an empty room.
A room path can either be provided, else a default room is imported.
Manual Room pipeline randomizes a furnished room and then replaces the category under observation.
The selection of objects to be replaced is also random.
In this mode, the user has control over taking the snap of the scene.
The user can randomize model and room textures, randomize camera position or manually set it and randomize the lighting conditions.
Once the user is satisfied with the view of the scene, images can be saved with a click of the Snap button.
In Multi Objects pipeline, all the process mentioned in the Manual Room pipeline is automated.
The user won’t have control over any of the processes, while the program snaps images at random.
Another version of the pipeline which is similar to the Single Room pipeline is the Multi-threaded Single Room pipeline.
As the name suggests, multiple rooms are created based on the number of categories and for each room, the Single Room pipeline is applied in parallel.
This is an attempt to let the tool perform faster with multi-threading.
The multi-process uses Coroutines which is still a sequential operation and hence this pipeline needs some work.
To our estimate, even the Multi-threaded pipeline runs 1.5 times faster than Single room pipeline.

\subsection{Scenes from SceneNet}\label{subsec:scenes-from-scenenet}
For the rooms in our ersatz environments, we utilize scenes provided by SceneNet~\cite{McCormac:etal:ICCV2017}.
SceneNet has made 25 rooms available to the public.
We only use scenes of type Bedroom(11), Kitchen(1), Living room(6), and office(7).
Figure~\ref{fig:Scene Types} shows different types of scenes utilized to generate a synthetic dataset.
These scenes are further modified by adding few more objects of categories present in Pix3D so that we get more variations in the dataset.
The distribution of furniture matching the category of Pix3D in the scenes is as shown in figure~\ref{fig:distribution of scenes}.
Each scene is replaced for every snap we take for the dataset we create at random.

\begin{figure}[!ht]
    \centering
    \subfloat[][]{\includegraphics[width=.35\textwidth, height = .35\textwidth]{/Users/apple/OVGU/Thesis/code/3dReconstruction/report/images/implementation/scenenet_scenes/scene_bedroom}}\quad
    \subfloat[][]{\includegraphics[width=.35\textwidth, height = .35\textwidth]{/Users/apple/OVGU/Thesis/code/3dReconstruction/report/images/implementation/scenenet_scenes/scene_livingroom}}\\
    \subfloat[][]{\includegraphics[width=.35\textwidth, height = .35\textwidth]{/Users/apple/OVGU/Thesis/code/3dReconstruction/report/images/implementation/scenenet_scenes/scene_kitchen}}\quad
    \subfloat[][]{\includegraphics[width=.35\textwidth, height = .35\textwidth]{/Users/apple/OVGU/Thesis/code/3dReconstruction/report/images/implementation/scenenet_scenes/scene_office}}
    \caption{The top view of sample scene layouts from SceneNet. Types: (a)Bedroom, (b)LivingRoom, (c)Kitchen and (d)Office}
    \label{fig:Scene Types}
\end{figure}


\begin{figure}[!ht]
    \resizebox{0.49\textwidth}{6cm}{%% Creator: Matplotlib, PGF backend
%%
%% To include the figure in your LaTeX document, write
%%   \input{<filename>.pgf}
%%
%% Make sure the required packages are loaded in your preamble
%%   \usepackage{pgf}
%%
%% Figures using additional raster images can only be included by \input if
%% they are in the same directory as the main LaTeX file. For loading figures
%% from other directories you can use the `import` package
%%   \usepackage{import}
%%
%% and then include the figures with
%%   \import{<path to file>}{<filename>.pgf}
%%
%% Matplotlib used the following preamble
%%   \usepackage{fontspec}
%%   \setmainfont{DejaVuSerif.ttf}[Path=\detokenize{/Users/apple/opt/anaconda3/envs/kaolin/lib/python3.7/site-packages/matplotlib/mpl-data/fonts/ttf/}]
%%   \setsansfont{DejaVuSans.ttf}[Path=\detokenize{/Users/apple/opt/anaconda3/envs/kaolin/lib/python3.7/site-packages/matplotlib/mpl-data/fonts/ttf/}]
%%   \setmonofont{DejaVuSansMono.ttf}[Path=\detokenize{/Users/apple/opt/anaconda3/envs/kaolin/lib/python3.7/site-packages/matplotlib/mpl-data/fonts/ttf/}]
%%
\begingroup%
\makeatletter%
\begin{pgfpicture}%
\pgfpathrectangle{\pgfpointorigin}{\pgfqpoint{5.504645in}{4.391361in}}%
\pgfusepath{use as bounding box, clip}%
\begin{pgfscope}%
\pgfsetbuttcap%
\pgfsetmiterjoin%
\definecolor{currentfill}{rgb}{1.000000,1.000000,1.000000}%
\pgfsetfillcolor{currentfill}%
\pgfsetlinewidth{0.000000pt}%
\definecolor{currentstroke}{rgb}{1.000000,1.000000,1.000000}%
\pgfsetstrokecolor{currentstroke}%
\pgfsetdash{}{0pt}%
\pgfpathmoveto{\pgfqpoint{0.000000in}{0.000000in}}%
\pgfpathlineto{\pgfqpoint{5.504645in}{0.000000in}}%
\pgfpathlineto{\pgfqpoint{5.504645in}{4.391361in}}%
\pgfpathlineto{\pgfqpoint{0.000000in}{4.391361in}}%
\pgfpathclose%
\pgfusepath{fill}%
\end{pgfscope}%
\begin{pgfscope}%
\pgfsetbuttcap%
\pgfsetmiterjoin%
\definecolor{currentfill}{rgb}{1.000000,1.000000,1.000000}%
\pgfsetfillcolor{currentfill}%
\pgfsetlinewidth{0.000000pt}%
\definecolor{currentstroke}{rgb}{0.000000,0.000000,0.000000}%
\pgfsetstrokecolor{currentstroke}%
\pgfsetstrokeopacity{0.000000}%
\pgfsetdash{}{0pt}%
\pgfpathmoveto{\pgfqpoint{0.444645in}{0.385400in}}%
\pgfpathlineto{\pgfqpoint{5.404645in}{0.385400in}}%
\pgfpathlineto{\pgfqpoint{5.404645in}{4.081400in}}%
\pgfpathlineto{\pgfqpoint{0.444645in}{4.081400in}}%
\pgfpathclose%
\pgfusepath{fill}%
\end{pgfscope}%
\begin{pgfscope}%
\pgfpathrectangle{\pgfqpoint{0.444645in}{0.385400in}}{\pgfqpoint{4.960000in}{3.696000in}}%
\pgfusepath{clip}%
\pgfsetbuttcap%
\pgfsetmiterjoin%
\definecolor{currentfill}{rgb}{0.121569,0.466667,0.705882}%
\pgfsetfillcolor{currentfill}%
\pgfsetlinewidth{0.000000pt}%
\definecolor{currentstroke}{rgb}{0.000000,0.000000,0.000000}%
\pgfsetstrokecolor{currentstroke}%
\pgfsetstrokeopacity{0.000000}%
\pgfsetdash{}{0pt}%
\pgfpathmoveto{\pgfqpoint{0.670100in}{0.385400in}}%
\pgfpathlineto{\pgfqpoint{1.619382in}{0.385400in}}%
\pgfpathlineto{\pgfqpoint{1.619382in}{2.625400in}}%
\pgfpathlineto{\pgfqpoint{0.670100in}{2.625400in}}%
\pgfpathclose%
\pgfusepath{fill}%
\end{pgfscope}%
\begin{pgfscope}%
\pgfpathrectangle{\pgfqpoint{0.444645in}{0.385400in}}{\pgfqpoint{4.960000in}{3.696000in}}%
\pgfusepath{clip}%
\pgfsetbuttcap%
\pgfsetmiterjoin%
\definecolor{currentfill}{rgb}{0.121569,0.466667,0.705882}%
\pgfsetfillcolor{currentfill}%
\pgfsetlinewidth{0.000000pt}%
\definecolor{currentstroke}{rgb}{0.000000,0.000000,0.000000}%
\pgfsetstrokecolor{currentstroke}%
\pgfsetstrokeopacity{0.000000}%
\pgfsetdash{}{0pt}%
\pgfpathmoveto{\pgfqpoint{1.856703in}{0.385400in}}%
\pgfpathlineto{\pgfqpoint{2.805985in}{0.385400in}}%
\pgfpathlineto{\pgfqpoint{2.805985in}{0.705400in}}%
\pgfpathlineto{\pgfqpoint{1.856703in}{0.705400in}}%
\pgfpathclose%
\pgfusepath{fill}%
\end{pgfscope}%
\begin{pgfscope}%
\pgfpathrectangle{\pgfqpoint{0.444645in}{0.385400in}}{\pgfqpoint{4.960000in}{3.696000in}}%
\pgfusepath{clip}%
\pgfsetbuttcap%
\pgfsetmiterjoin%
\definecolor{currentfill}{rgb}{0.121569,0.466667,0.705882}%
\pgfsetfillcolor{currentfill}%
\pgfsetlinewidth{0.000000pt}%
\definecolor{currentstroke}{rgb}{0.000000,0.000000,0.000000}%
\pgfsetstrokecolor{currentstroke}%
\pgfsetstrokeopacity{0.000000}%
\pgfsetdash{}{0pt}%
\pgfpathmoveto{\pgfqpoint{3.043305in}{0.385400in}}%
\pgfpathlineto{\pgfqpoint{3.992588in}{0.385400in}}%
\pgfpathlineto{\pgfqpoint{3.992588in}{3.905400in}}%
\pgfpathlineto{\pgfqpoint{3.043305in}{3.905400in}}%
\pgfpathclose%
\pgfusepath{fill}%
\end{pgfscope}%
\begin{pgfscope}%
\pgfpathrectangle{\pgfqpoint{0.444645in}{0.385400in}}{\pgfqpoint{4.960000in}{3.696000in}}%
\pgfusepath{clip}%
\pgfsetbuttcap%
\pgfsetmiterjoin%
\definecolor{currentfill}{rgb}{0.121569,0.466667,0.705882}%
\pgfsetfillcolor{currentfill}%
\pgfsetlinewidth{0.000000pt}%
\definecolor{currentstroke}{rgb}{0.000000,0.000000,0.000000}%
\pgfsetstrokecolor{currentstroke}%
\pgfsetstrokeopacity{0.000000}%
\pgfsetdash{}{0pt}%
\pgfpathmoveto{\pgfqpoint{4.229908in}{0.385400in}}%
\pgfpathlineto{\pgfqpoint{5.179191in}{0.385400in}}%
\pgfpathlineto{\pgfqpoint{5.179191in}{2.305400in}}%
\pgfpathlineto{\pgfqpoint{4.229908in}{2.305400in}}%
\pgfpathclose%
\pgfusepath{fill}%
\end{pgfscope}%
\begin{pgfscope}%
\pgfsetbuttcap%
\pgfsetroundjoin%
\definecolor{currentfill}{rgb}{0.000000,0.000000,0.000000}%
\pgfsetfillcolor{currentfill}%
\pgfsetlinewidth{0.803000pt}%
\definecolor{currentstroke}{rgb}{0.000000,0.000000,0.000000}%
\pgfsetstrokecolor{currentstroke}%
\pgfsetdash{}{0pt}%
\pgfsys@defobject{currentmarker}{\pgfqpoint{0.000000in}{-0.048611in}}{\pgfqpoint{0.000000in}{0.000000in}}{%
\pgfpathmoveto{\pgfqpoint{0.000000in}{0.000000in}}%
\pgfpathlineto{\pgfqpoint{0.000000in}{-0.048611in}}%
\pgfusepath{stroke,fill}%
}%
\begin{pgfscope}%
\pgfsys@transformshift{1.144741in}{0.385400in}%
\pgfsys@useobject{currentmarker}{}%
\end{pgfscope}%
\end{pgfscope}%
\begin{pgfscope}%
\definecolor{textcolor}{rgb}{0.000000,0.000000,0.000000}%
\pgfsetstrokecolor{textcolor}%
\pgfsetfillcolor{textcolor}%
\pgftext[x=1.144741in,y=0.288178in,,top]{\color{textcolor}\sffamily\fontsize{14.000000}{16.800000}\selectfont LivingRoom}%
\end{pgfscope}%
\begin{pgfscope}%
\pgfsetbuttcap%
\pgfsetroundjoin%
\definecolor{currentfill}{rgb}{0.000000,0.000000,0.000000}%
\pgfsetfillcolor{currentfill}%
\pgfsetlinewidth{0.803000pt}%
\definecolor{currentstroke}{rgb}{0.000000,0.000000,0.000000}%
\pgfsetstrokecolor{currentstroke}%
\pgfsetdash{}{0pt}%
\pgfsys@defobject{currentmarker}{\pgfqpoint{0.000000in}{-0.048611in}}{\pgfqpoint{0.000000in}{0.000000in}}{%
\pgfpathmoveto{\pgfqpoint{0.000000in}{0.000000in}}%
\pgfpathlineto{\pgfqpoint{0.000000in}{-0.048611in}}%
\pgfusepath{stroke,fill}%
}%
\begin{pgfscope}%
\pgfsys@transformshift{2.331344in}{0.385400in}%
\pgfsys@useobject{currentmarker}{}%
\end{pgfscope}%
\end{pgfscope}%
\begin{pgfscope}%
\definecolor{textcolor}{rgb}{0.000000,0.000000,0.000000}%
\pgfsetstrokecolor{textcolor}%
\pgfsetfillcolor{textcolor}%
\pgftext[x=2.331344in,y=0.288178in,,top]{\color{textcolor}\sffamily\fontsize{14.000000}{16.800000}\selectfont Kitchen}%
\end{pgfscope}%
\begin{pgfscope}%
\pgfsetbuttcap%
\pgfsetroundjoin%
\definecolor{currentfill}{rgb}{0.000000,0.000000,0.000000}%
\pgfsetfillcolor{currentfill}%
\pgfsetlinewidth{0.803000pt}%
\definecolor{currentstroke}{rgb}{0.000000,0.000000,0.000000}%
\pgfsetstrokecolor{currentstroke}%
\pgfsetdash{}{0pt}%
\pgfsys@defobject{currentmarker}{\pgfqpoint{0.000000in}{-0.048611in}}{\pgfqpoint{0.000000in}{0.000000in}}{%
\pgfpathmoveto{\pgfqpoint{0.000000in}{0.000000in}}%
\pgfpathlineto{\pgfqpoint{0.000000in}{-0.048611in}}%
\pgfusepath{stroke,fill}%
}%
\begin{pgfscope}%
\pgfsys@transformshift{3.517947in}{0.385400in}%
\pgfsys@useobject{currentmarker}{}%
\end{pgfscope}%
\end{pgfscope}%
\begin{pgfscope}%
\definecolor{textcolor}{rgb}{0.000000,0.000000,0.000000}%
\pgfsetstrokecolor{textcolor}%
\pgfsetfillcolor{textcolor}%
\pgftext[x=3.517947in,y=0.288178in,,top]{\color{textcolor}\sffamily\fontsize{14.000000}{16.800000}\selectfont Bedroom}%
\end{pgfscope}%
\begin{pgfscope}%
\pgfsetbuttcap%
\pgfsetroundjoin%
\definecolor{currentfill}{rgb}{0.000000,0.000000,0.000000}%
\pgfsetfillcolor{currentfill}%
\pgfsetlinewidth{0.803000pt}%
\definecolor{currentstroke}{rgb}{0.000000,0.000000,0.000000}%
\pgfsetstrokecolor{currentstroke}%
\pgfsetdash{}{0pt}%
\pgfsys@defobject{currentmarker}{\pgfqpoint{0.000000in}{-0.048611in}}{\pgfqpoint{0.000000in}{0.000000in}}{%
\pgfpathmoveto{\pgfqpoint{0.000000in}{0.000000in}}%
\pgfpathlineto{\pgfqpoint{0.000000in}{-0.048611in}}%
\pgfusepath{stroke,fill}%
}%
\begin{pgfscope}%
\pgfsys@transformshift{4.704549in}{0.385400in}%
\pgfsys@useobject{currentmarker}{}%
\end{pgfscope}%
\end{pgfscope}%
\begin{pgfscope}%
\definecolor{textcolor}{rgb}{0.000000,0.000000,0.000000}%
\pgfsetstrokecolor{textcolor}%
\pgfsetfillcolor{textcolor}%
\pgftext[x=4.704549in,y=0.288178in,,top]{\color{textcolor}\sffamily\fontsize{14.000000}{16.800000}\selectfont Office}%
\end{pgfscope}%
\begin{pgfscope}%
\pgfsetbuttcap%
\pgfsetroundjoin%
\definecolor{currentfill}{rgb}{0.000000,0.000000,0.000000}%
\pgfsetfillcolor{currentfill}%
\pgfsetlinewidth{0.803000pt}%
\definecolor{currentstroke}{rgb}{0.000000,0.000000,0.000000}%
\pgfsetstrokecolor{currentstroke}%
\pgfsetdash{}{0pt}%
\pgfsys@defobject{currentmarker}{\pgfqpoint{-0.048611in}{0.000000in}}{\pgfqpoint{-0.000000in}{0.000000in}}{%
\pgfpathmoveto{\pgfqpoint{-0.000000in}{0.000000in}}%
\pgfpathlineto{\pgfqpoint{-0.048611in}{0.000000in}}%
\pgfusepath{stroke,fill}%
}%
\begin{pgfscope}%
\pgfsys@transformshift{0.444645in}{0.385400in}%
\pgfsys@useobject{currentmarker}{}%
\end{pgfscope}%
\end{pgfscope}%
\begin{pgfscope}%
\definecolor{textcolor}{rgb}{0.000000,0.000000,0.000000}%
\pgfsetstrokecolor{textcolor}%
\pgfsetfillcolor{textcolor}%
\pgftext[x=0.223711in, y=0.311534in, left, base]{\color{textcolor}\sffamily\fontsize{14.000000}{16.800000}\selectfont 0}%
\end{pgfscope}%
\begin{pgfscope}%
\pgfsetbuttcap%
\pgfsetroundjoin%
\definecolor{currentfill}{rgb}{0.000000,0.000000,0.000000}%
\pgfsetfillcolor{currentfill}%
\pgfsetlinewidth{0.803000pt}%
\definecolor{currentstroke}{rgb}{0.000000,0.000000,0.000000}%
\pgfsetstrokecolor{currentstroke}%
\pgfsetdash{}{0pt}%
\pgfsys@defobject{currentmarker}{\pgfqpoint{-0.048611in}{0.000000in}}{\pgfqpoint{-0.000000in}{0.000000in}}{%
\pgfpathmoveto{\pgfqpoint{-0.000000in}{0.000000in}}%
\pgfpathlineto{\pgfqpoint{-0.048611in}{0.000000in}}%
\pgfusepath{stroke,fill}%
}%
\begin{pgfscope}%
\pgfsys@transformshift{0.444645in}{1.025400in}%
\pgfsys@useobject{currentmarker}{}%
\end{pgfscope}%
\end{pgfscope}%
\begin{pgfscope}%
\definecolor{textcolor}{rgb}{0.000000,0.000000,0.000000}%
\pgfsetstrokecolor{textcolor}%
\pgfsetfillcolor{textcolor}%
\pgftext[x=0.223711in, y=0.951534in, left, base]{\color{textcolor}\sffamily\fontsize{14.000000}{16.800000}\selectfont 2}%
\end{pgfscope}%
\begin{pgfscope}%
\pgfsetbuttcap%
\pgfsetroundjoin%
\definecolor{currentfill}{rgb}{0.000000,0.000000,0.000000}%
\pgfsetfillcolor{currentfill}%
\pgfsetlinewidth{0.803000pt}%
\definecolor{currentstroke}{rgb}{0.000000,0.000000,0.000000}%
\pgfsetstrokecolor{currentstroke}%
\pgfsetdash{}{0pt}%
\pgfsys@defobject{currentmarker}{\pgfqpoint{-0.048611in}{0.000000in}}{\pgfqpoint{-0.000000in}{0.000000in}}{%
\pgfpathmoveto{\pgfqpoint{-0.000000in}{0.000000in}}%
\pgfpathlineto{\pgfqpoint{-0.048611in}{0.000000in}}%
\pgfusepath{stroke,fill}%
}%
\begin{pgfscope}%
\pgfsys@transformshift{0.444645in}{1.665400in}%
\pgfsys@useobject{currentmarker}{}%
\end{pgfscope}%
\end{pgfscope}%
\begin{pgfscope}%
\definecolor{textcolor}{rgb}{0.000000,0.000000,0.000000}%
\pgfsetstrokecolor{textcolor}%
\pgfsetfillcolor{textcolor}%
\pgftext[x=0.223711in, y=1.591534in, left, base]{\color{textcolor}\sffamily\fontsize{14.000000}{16.800000}\selectfont 4}%
\end{pgfscope}%
\begin{pgfscope}%
\pgfsetbuttcap%
\pgfsetroundjoin%
\definecolor{currentfill}{rgb}{0.000000,0.000000,0.000000}%
\pgfsetfillcolor{currentfill}%
\pgfsetlinewidth{0.803000pt}%
\definecolor{currentstroke}{rgb}{0.000000,0.000000,0.000000}%
\pgfsetstrokecolor{currentstroke}%
\pgfsetdash{}{0pt}%
\pgfsys@defobject{currentmarker}{\pgfqpoint{-0.048611in}{0.000000in}}{\pgfqpoint{-0.000000in}{0.000000in}}{%
\pgfpathmoveto{\pgfqpoint{-0.000000in}{0.000000in}}%
\pgfpathlineto{\pgfqpoint{-0.048611in}{0.000000in}}%
\pgfusepath{stroke,fill}%
}%
\begin{pgfscope}%
\pgfsys@transformshift{0.444645in}{2.305400in}%
\pgfsys@useobject{currentmarker}{}%
\end{pgfscope}%
\end{pgfscope}%
\begin{pgfscope}%
\definecolor{textcolor}{rgb}{0.000000,0.000000,0.000000}%
\pgfsetstrokecolor{textcolor}%
\pgfsetfillcolor{textcolor}%
\pgftext[x=0.223711in, y=2.231534in, left, base]{\color{textcolor}\sffamily\fontsize{14.000000}{16.800000}\selectfont 6}%
\end{pgfscope}%
\begin{pgfscope}%
\pgfsetbuttcap%
\pgfsetroundjoin%
\definecolor{currentfill}{rgb}{0.000000,0.000000,0.000000}%
\pgfsetfillcolor{currentfill}%
\pgfsetlinewidth{0.803000pt}%
\definecolor{currentstroke}{rgb}{0.000000,0.000000,0.000000}%
\pgfsetstrokecolor{currentstroke}%
\pgfsetdash{}{0pt}%
\pgfsys@defobject{currentmarker}{\pgfqpoint{-0.048611in}{0.000000in}}{\pgfqpoint{-0.000000in}{0.000000in}}{%
\pgfpathmoveto{\pgfqpoint{-0.000000in}{0.000000in}}%
\pgfpathlineto{\pgfqpoint{-0.048611in}{0.000000in}}%
\pgfusepath{stroke,fill}%
}%
\begin{pgfscope}%
\pgfsys@transformshift{0.444645in}{2.945400in}%
\pgfsys@useobject{currentmarker}{}%
\end{pgfscope}%
\end{pgfscope}%
\begin{pgfscope}%
\definecolor{textcolor}{rgb}{0.000000,0.000000,0.000000}%
\pgfsetstrokecolor{textcolor}%
\pgfsetfillcolor{textcolor}%
\pgftext[x=0.223711in, y=2.871534in, left, base]{\color{textcolor}\sffamily\fontsize{14.000000}{16.800000}\selectfont 8}%
\end{pgfscope}%
\begin{pgfscope}%
\pgfsetbuttcap%
\pgfsetroundjoin%
\definecolor{currentfill}{rgb}{0.000000,0.000000,0.000000}%
\pgfsetfillcolor{currentfill}%
\pgfsetlinewidth{0.803000pt}%
\definecolor{currentstroke}{rgb}{0.000000,0.000000,0.000000}%
\pgfsetstrokecolor{currentstroke}%
\pgfsetdash{}{0pt}%
\pgfsys@defobject{currentmarker}{\pgfqpoint{-0.048611in}{0.000000in}}{\pgfqpoint{-0.000000in}{0.000000in}}{%
\pgfpathmoveto{\pgfqpoint{-0.000000in}{0.000000in}}%
\pgfpathlineto{\pgfqpoint{-0.048611in}{0.000000in}}%
\pgfusepath{stroke,fill}%
}%
\begin{pgfscope}%
\pgfsys@transformshift{0.444645in}{3.585400in}%
\pgfsys@useobject{currentmarker}{}%
\end{pgfscope}%
\end{pgfscope}%
\begin{pgfscope}%
\definecolor{textcolor}{rgb}{0.000000,0.000000,0.000000}%
\pgfsetstrokecolor{textcolor}%
\pgfsetfillcolor{textcolor}%
\pgftext[x=0.100000in, y=3.511534in, left, base]{\color{textcolor}\sffamily\fontsize{14.000000}{16.800000}\selectfont 10}%
\end{pgfscope}%
\begin{pgfscope}%
\pgfsetrectcap%
\pgfsetmiterjoin%
\pgfsetlinewidth{0.803000pt}%
\definecolor{currentstroke}{rgb}{0.000000,0.000000,0.000000}%
\pgfsetstrokecolor{currentstroke}%
\pgfsetdash{}{0pt}%
\pgfpathmoveto{\pgfqpoint{0.444645in}{0.385400in}}%
\pgfpathlineto{\pgfqpoint{0.444645in}{4.081400in}}%
\pgfusepath{stroke}%
\end{pgfscope}%
\begin{pgfscope}%
\pgfsetrectcap%
\pgfsetmiterjoin%
\pgfsetlinewidth{0.803000pt}%
\definecolor{currentstroke}{rgb}{0.000000,0.000000,0.000000}%
\pgfsetstrokecolor{currentstroke}%
\pgfsetdash{}{0pt}%
\pgfpathmoveto{\pgfqpoint{5.404645in}{0.385400in}}%
\pgfpathlineto{\pgfqpoint{5.404645in}{4.081400in}}%
\pgfusepath{stroke}%
\end{pgfscope}%
\begin{pgfscope}%
\pgfsetrectcap%
\pgfsetmiterjoin%
\pgfsetlinewidth{0.803000pt}%
\definecolor{currentstroke}{rgb}{0.000000,0.000000,0.000000}%
\pgfsetstrokecolor{currentstroke}%
\pgfsetdash{}{0pt}%
\pgfpathmoveto{\pgfqpoint{0.444645in}{0.385400in}}%
\pgfpathlineto{\pgfqpoint{5.404645in}{0.385400in}}%
\pgfusepath{stroke}%
\end{pgfscope}%
\begin{pgfscope}%
\pgfsetrectcap%
\pgfsetmiterjoin%
\pgfsetlinewidth{0.803000pt}%
\definecolor{currentstroke}{rgb}{0.000000,0.000000,0.000000}%
\pgfsetstrokecolor{currentstroke}%
\pgfsetdash{}{0pt}%
\pgfpathmoveto{\pgfqpoint{0.444645in}{4.081400in}}%
\pgfpathlineto{\pgfqpoint{5.404645in}{4.081400in}}%
\pgfusepath{stroke}%
\end{pgfscope}%
\begin{pgfscope}%
\definecolor{textcolor}{rgb}{0.000000,0.000000,0.000000}%
\pgfsetstrokecolor{textcolor}%
\pgfsetfillcolor{textcolor}%
\pgftext[x=2.924645in,y=4.164734in,,base]{\color{textcolor}\sffamily\fontsize{12.000000}{14.400000}\selectfont Type of scenes}%
\end{pgfscope}%
\end{pgfpicture}%
\makeatother%
\endgroup%
}
    \resizebox{0.49\textwidth}{6cm}{%% Creator: Matplotlib, PGF backend
%%
%% To include the figure in your LaTeX document, write
%%   \input{<filename>.pgf}
%%
%% Make sure the required packages are loaded in your preamble
%%   \usepackage{pgf}
%%
%% Figures using additional raster images can only be included by \input if
%% they are in the same directory as the main LaTeX file. For loading figures
%% from other directories you can use the `import` package
%%   \usepackage{import}
%%
%% and then include the figures with
%%   \import{<path to file>}{<filename>.pgf}
%%
%% Matplotlib used the following preamble
%%   \usepackage{fontspec}
%%   \setmainfont{DejaVuSerif.ttf}[Path=\detokenize{/Users/apple/opt/anaconda3/envs/kaolin/lib/python3.7/site-packages/matplotlib/mpl-data/fonts/ttf/}]
%%   \setsansfont{DejaVuSans.ttf}[Path=\detokenize{/Users/apple/opt/anaconda3/envs/kaolin/lib/python3.7/site-packages/matplotlib/mpl-data/fonts/ttf/}]
%%   \setmonofont{DejaVuSansMono.ttf}[Path=\detokenize{/Users/apple/opt/anaconda3/envs/kaolin/lib/python3.7/site-packages/matplotlib/mpl-data/fonts/ttf/}]
%%
\begingroup%
\makeatletter%
\begin{pgfpicture}%
\pgfpathrectangle{\pgfpointorigin}{\pgfqpoint{5.522318in}{4.337596in}}%
\pgfusepath{use as bounding box, clip}%
\begin{pgfscope}%
\pgfsetbuttcap%
\pgfsetmiterjoin%
\definecolor{currentfill}{rgb}{1.000000,1.000000,1.000000}%
\pgfsetfillcolor{currentfill}%
\pgfsetlinewidth{0.000000pt}%
\definecolor{currentstroke}{rgb}{1.000000,1.000000,1.000000}%
\pgfsetstrokecolor{currentstroke}%
\pgfsetdash{}{0pt}%
\pgfpathmoveto{\pgfqpoint{0.000000in}{0.000000in}}%
\pgfpathlineto{\pgfqpoint{5.522318in}{0.000000in}}%
\pgfpathlineto{\pgfqpoint{5.522318in}{4.337596in}}%
\pgfpathlineto{\pgfqpoint{0.000000in}{4.337596in}}%
\pgfpathclose%
\pgfusepath{fill}%
\end{pgfscope}%
\begin{pgfscope}%
\pgfsetbuttcap%
\pgfsetmiterjoin%
\definecolor{currentfill}{rgb}{1.000000,1.000000,1.000000}%
\pgfsetfillcolor{currentfill}%
\pgfsetlinewidth{0.000000pt}%
\definecolor{currentstroke}{rgb}{0.000000,0.000000,0.000000}%
\pgfsetstrokecolor{currentstroke}%
\pgfsetstrokeopacity{0.000000}%
\pgfsetdash{}{0pt}%
\pgfpathmoveto{\pgfqpoint{0.462318in}{0.331635in}}%
\pgfpathlineto{\pgfqpoint{5.422318in}{0.331635in}}%
\pgfpathlineto{\pgfqpoint{5.422318in}{4.027635in}}%
\pgfpathlineto{\pgfqpoint{0.462318in}{4.027635in}}%
\pgfpathclose%
\pgfusepath{fill}%
\end{pgfscope}%
\begin{pgfscope}%
\pgfpathrectangle{\pgfqpoint{0.462318in}{0.331635in}}{\pgfqpoint{4.960000in}{3.696000in}}%
\pgfusepath{clip}%
\pgfsetbuttcap%
\pgfsetmiterjoin%
\definecolor{currentfill}{rgb}{0.000000,0.000000,1.000000}%
\pgfsetfillcolor{currentfill}%
\pgfsetlinewidth{0.000000pt}%
\definecolor{currentstroke}{rgb}{0.000000,0.000000,0.000000}%
\pgfsetstrokecolor{currentstroke}%
\pgfsetstrokeopacity{0.000000}%
\pgfsetdash{}{0pt}%
\pgfpathmoveto{\pgfqpoint{0.687773in}{0.331635in}}%
\pgfpathlineto{\pgfqpoint{1.218254in}{0.331635in}}%
\pgfpathlineto{\pgfqpoint{1.218254in}{3.851635in}}%
\pgfpathlineto{\pgfqpoint{0.687773in}{3.851635in}}%
\pgfpathclose%
\pgfusepath{fill}%
\end{pgfscope}%
\begin{pgfscope}%
\pgfpathrectangle{\pgfqpoint{0.462318in}{0.331635in}}{\pgfqpoint{4.960000in}{3.696000in}}%
\pgfusepath{clip}%
\pgfsetbuttcap%
\pgfsetmiterjoin%
\definecolor{currentfill}{rgb}{0.000000,0.000000,1.000000}%
\pgfsetfillcolor{currentfill}%
\pgfsetlinewidth{0.000000pt}%
\definecolor{currentstroke}{rgb}{0.000000,0.000000,0.000000}%
\pgfsetstrokecolor{currentstroke}%
\pgfsetstrokeopacity{0.000000}%
\pgfsetdash{}{0pt}%
\pgfpathmoveto{\pgfqpoint{1.350874in}{0.331635in}}%
\pgfpathlineto{\pgfqpoint{1.881356in}{0.331635in}}%
\pgfpathlineto{\pgfqpoint{1.881356in}{1.079246in}}%
\pgfpathlineto{\pgfqpoint{1.350874in}{1.079246in}}%
\pgfpathclose%
\pgfusepath{fill}%
\end{pgfscope}%
\begin{pgfscope}%
\pgfpathrectangle{\pgfqpoint{0.462318in}{0.331635in}}{\pgfqpoint{4.960000in}{3.696000in}}%
\pgfusepath{clip}%
\pgfsetbuttcap%
\pgfsetmiterjoin%
\definecolor{currentfill}{rgb}{0.000000,0.000000,1.000000}%
\pgfsetfillcolor{currentfill}%
\pgfsetlinewidth{0.000000pt}%
\definecolor{currentstroke}{rgb}{0.000000,0.000000,0.000000}%
\pgfsetstrokecolor{currentstroke}%
\pgfsetstrokeopacity{0.000000}%
\pgfsetdash{}{0pt}%
\pgfpathmoveto{\pgfqpoint{2.013976in}{0.331635in}}%
\pgfpathlineto{\pgfqpoint{2.544457in}{0.331635in}}%
\pgfpathlineto{\pgfqpoint{2.544457in}{1.484201in}}%
\pgfpathlineto{\pgfqpoint{2.013976in}{1.484201in}}%
\pgfpathclose%
\pgfusepath{fill}%
\end{pgfscope}%
\begin{pgfscope}%
\pgfpathrectangle{\pgfqpoint{0.462318in}{0.331635in}}{\pgfqpoint{4.960000in}{3.696000in}}%
\pgfusepath{clip}%
\pgfsetbuttcap%
\pgfsetmiterjoin%
\definecolor{currentfill}{rgb}{0.000000,0.000000,1.000000}%
\pgfsetfillcolor{currentfill}%
\pgfsetlinewidth{0.000000pt}%
\definecolor{currentstroke}{rgb}{0.000000,0.000000,0.000000}%
\pgfsetstrokecolor{currentstroke}%
\pgfsetstrokeopacity{0.000000}%
\pgfsetdash{}{0pt}%
\pgfpathmoveto{\pgfqpoint{2.677078in}{0.331635in}}%
\pgfpathlineto{\pgfqpoint{3.207559in}{0.331635in}}%
\pgfpathlineto{\pgfqpoint{3.207559in}{0.954644in}}%
\pgfpathlineto{\pgfqpoint{2.677078in}{0.954644in}}%
\pgfpathclose%
\pgfusepath{fill}%
\end{pgfscope}%
\begin{pgfscope}%
\pgfpathrectangle{\pgfqpoint{0.462318in}{0.331635in}}{\pgfqpoint{4.960000in}{3.696000in}}%
\pgfusepath{clip}%
\pgfsetbuttcap%
\pgfsetmiterjoin%
\definecolor{currentfill}{rgb}{0.000000,0.000000,1.000000}%
\pgfsetfillcolor{currentfill}%
\pgfsetlinewidth{0.000000pt}%
\definecolor{currentstroke}{rgb}{0.000000,0.000000,0.000000}%
\pgfsetstrokecolor{currentstroke}%
\pgfsetstrokeopacity{0.000000}%
\pgfsetdash{}{0pt}%
\pgfpathmoveto{\pgfqpoint{3.340179in}{0.331635in}}%
\pgfpathlineto{\pgfqpoint{3.870660in}{0.331635in}}%
\pgfpathlineto{\pgfqpoint{3.870660in}{1.048095in}}%
\pgfpathlineto{\pgfqpoint{3.340179in}{1.048095in}}%
\pgfpathclose%
\pgfusepath{fill}%
\end{pgfscope}%
\begin{pgfscope}%
\pgfpathrectangle{\pgfqpoint{0.462318in}{0.331635in}}{\pgfqpoint{4.960000in}{3.696000in}}%
\pgfusepath{clip}%
\pgfsetbuttcap%
\pgfsetmiterjoin%
\definecolor{currentfill}{rgb}{0.000000,0.000000,1.000000}%
\pgfsetfillcolor{currentfill}%
\pgfsetlinewidth{0.000000pt}%
\definecolor{currentstroke}{rgb}{0.000000,0.000000,0.000000}%
\pgfsetstrokecolor{currentstroke}%
\pgfsetstrokeopacity{0.000000}%
\pgfsetdash{}{0pt}%
\pgfpathmoveto{\pgfqpoint{4.003281in}{0.331635in}}%
\pgfpathlineto{\pgfqpoint{4.533762in}{0.331635in}}%
\pgfpathlineto{\pgfqpoint{4.533762in}{1.390750in}}%
\pgfpathlineto{\pgfqpoint{4.003281in}{1.390750in}}%
\pgfpathclose%
\pgfusepath{fill}%
\end{pgfscope}%
\begin{pgfscope}%
\pgfpathrectangle{\pgfqpoint{0.462318in}{0.331635in}}{\pgfqpoint{4.960000in}{3.696000in}}%
\pgfusepath{clip}%
\pgfsetbuttcap%
\pgfsetmiterjoin%
\definecolor{currentfill}{rgb}{0.000000,0.000000,1.000000}%
\pgfsetfillcolor{currentfill}%
\pgfsetlinewidth{0.000000pt}%
\definecolor{currentstroke}{rgb}{0.000000,0.000000,0.000000}%
\pgfsetstrokecolor{currentstroke}%
\pgfsetstrokeopacity{0.000000}%
\pgfsetdash{}{0pt}%
\pgfpathmoveto{\pgfqpoint{4.666382in}{0.331635in}}%
\pgfpathlineto{\pgfqpoint{5.196864in}{0.331635in}}%
\pgfpathlineto{\pgfqpoint{5.196864in}{0.674290in}}%
\pgfpathlineto{\pgfqpoint{4.666382in}{0.674290in}}%
\pgfpathclose%
\pgfusepath{fill}%
\end{pgfscope}%
\begin{pgfscope}%
\pgfsetbuttcap%
\pgfsetroundjoin%
\definecolor{currentfill}{rgb}{0.000000,0.000000,0.000000}%
\pgfsetfillcolor{currentfill}%
\pgfsetlinewidth{0.803000pt}%
\definecolor{currentstroke}{rgb}{0.000000,0.000000,0.000000}%
\pgfsetstrokecolor{currentstroke}%
\pgfsetdash{}{0pt}%
\pgfsys@defobject{currentmarker}{\pgfqpoint{0.000000in}{-0.048611in}}{\pgfqpoint{0.000000in}{0.000000in}}{%
\pgfpathmoveto{\pgfqpoint{0.000000in}{0.000000in}}%
\pgfpathlineto{\pgfqpoint{0.000000in}{-0.048611in}}%
\pgfusepath{stroke,fill}%
}%
\begin{pgfscope}%
\pgfsys@transformshift{0.953013in}{0.331635in}%
\pgfsys@useobject{currentmarker}{}%
\end{pgfscope}%
\end{pgfscope}%
\begin{pgfscope}%
\definecolor{textcolor}{rgb}{0.000000,0.000000,0.000000}%
\pgfsetstrokecolor{textcolor}%
\pgfsetfillcolor{textcolor}%
\pgftext[x=0.953013in,y=0.234413in,,top]{\color{textcolor}\sffamily\fontsize{10.000000}{12.000000}\selectfont chair}%
\end{pgfscope}%
\begin{pgfscope}%
\pgfsetbuttcap%
\pgfsetroundjoin%
\definecolor{currentfill}{rgb}{0.000000,0.000000,0.000000}%
\pgfsetfillcolor{currentfill}%
\pgfsetlinewidth{0.803000pt}%
\definecolor{currentstroke}{rgb}{0.000000,0.000000,0.000000}%
\pgfsetstrokecolor{currentstroke}%
\pgfsetdash{}{0pt}%
\pgfsys@defobject{currentmarker}{\pgfqpoint{0.000000in}{-0.048611in}}{\pgfqpoint{0.000000in}{0.000000in}}{%
\pgfpathmoveto{\pgfqpoint{0.000000in}{0.000000in}}%
\pgfpathlineto{\pgfqpoint{0.000000in}{-0.048611in}}%
\pgfusepath{stroke,fill}%
}%
\begin{pgfscope}%
\pgfsys@transformshift{1.616115in}{0.331635in}%
\pgfsys@useobject{currentmarker}{}%
\end{pgfscope}%
\end{pgfscope}%
\begin{pgfscope}%
\definecolor{textcolor}{rgb}{0.000000,0.000000,0.000000}%
\pgfsetstrokecolor{textcolor}%
\pgfsetfillcolor{textcolor}%
\pgftext[x=1.616115in,y=0.234413in,,top]{\color{textcolor}\sffamily\fontsize{10.000000}{12.000000}\selectfont bed}%
\end{pgfscope}%
\begin{pgfscope}%
\pgfsetbuttcap%
\pgfsetroundjoin%
\definecolor{currentfill}{rgb}{0.000000,0.000000,0.000000}%
\pgfsetfillcolor{currentfill}%
\pgfsetlinewidth{0.803000pt}%
\definecolor{currentstroke}{rgb}{0.000000,0.000000,0.000000}%
\pgfsetstrokecolor{currentstroke}%
\pgfsetdash{}{0pt}%
\pgfsys@defobject{currentmarker}{\pgfqpoint{0.000000in}{-0.048611in}}{\pgfqpoint{0.000000in}{0.000000in}}{%
\pgfpathmoveto{\pgfqpoint{0.000000in}{0.000000in}}%
\pgfpathlineto{\pgfqpoint{0.000000in}{-0.048611in}}%
\pgfusepath{stroke,fill}%
}%
\begin{pgfscope}%
\pgfsys@transformshift{2.279217in}{0.331635in}%
\pgfsys@useobject{currentmarker}{}%
\end{pgfscope}%
\end{pgfscope}%
\begin{pgfscope}%
\definecolor{textcolor}{rgb}{0.000000,0.000000,0.000000}%
\pgfsetstrokecolor{textcolor}%
\pgfsetfillcolor{textcolor}%
\pgftext[x=2.279217in,y=0.234413in,,top]{\color{textcolor}\sffamily\fontsize{10.000000}{12.000000}\selectfont desk}%
\end{pgfscope}%
\begin{pgfscope}%
\pgfsetbuttcap%
\pgfsetroundjoin%
\definecolor{currentfill}{rgb}{0.000000,0.000000,0.000000}%
\pgfsetfillcolor{currentfill}%
\pgfsetlinewidth{0.803000pt}%
\definecolor{currentstroke}{rgb}{0.000000,0.000000,0.000000}%
\pgfsetstrokecolor{currentstroke}%
\pgfsetdash{}{0pt}%
\pgfsys@defobject{currentmarker}{\pgfqpoint{0.000000in}{-0.048611in}}{\pgfqpoint{0.000000in}{0.000000in}}{%
\pgfpathmoveto{\pgfqpoint{0.000000in}{0.000000in}}%
\pgfpathlineto{\pgfqpoint{0.000000in}{-0.048611in}}%
\pgfusepath{stroke,fill}%
}%
\begin{pgfscope}%
\pgfsys@transformshift{2.942318in}{0.331635in}%
\pgfsys@useobject{currentmarker}{}%
\end{pgfscope}%
\end{pgfscope}%
\begin{pgfscope}%
\definecolor{textcolor}{rgb}{0.000000,0.000000,0.000000}%
\pgfsetstrokecolor{textcolor}%
\pgfsetfillcolor{textcolor}%
\pgftext[x=2.942318in,y=0.234413in,,top]{\color{textcolor}\sffamily\fontsize{10.000000}{12.000000}\selectfont bookcase}%
\end{pgfscope}%
\begin{pgfscope}%
\pgfsetbuttcap%
\pgfsetroundjoin%
\definecolor{currentfill}{rgb}{0.000000,0.000000,0.000000}%
\pgfsetfillcolor{currentfill}%
\pgfsetlinewidth{0.803000pt}%
\definecolor{currentstroke}{rgb}{0.000000,0.000000,0.000000}%
\pgfsetstrokecolor{currentstroke}%
\pgfsetdash{}{0pt}%
\pgfsys@defobject{currentmarker}{\pgfqpoint{0.000000in}{-0.048611in}}{\pgfqpoint{0.000000in}{0.000000in}}{%
\pgfpathmoveto{\pgfqpoint{0.000000in}{0.000000in}}%
\pgfpathlineto{\pgfqpoint{0.000000in}{-0.048611in}}%
\pgfusepath{stroke,fill}%
}%
\begin{pgfscope}%
\pgfsys@transformshift{3.605420in}{0.331635in}%
\pgfsys@useobject{currentmarker}{}%
\end{pgfscope}%
\end{pgfscope}%
\begin{pgfscope}%
\definecolor{textcolor}{rgb}{0.000000,0.000000,0.000000}%
\pgfsetstrokecolor{textcolor}%
\pgfsetfillcolor{textcolor}%
\pgftext[x=3.605420in,y=0.234413in,,top]{\color{textcolor}\sffamily\fontsize{10.000000}{12.000000}\selectfont sofa}%
\end{pgfscope}%
\begin{pgfscope}%
\pgfsetbuttcap%
\pgfsetroundjoin%
\definecolor{currentfill}{rgb}{0.000000,0.000000,0.000000}%
\pgfsetfillcolor{currentfill}%
\pgfsetlinewidth{0.803000pt}%
\definecolor{currentstroke}{rgb}{0.000000,0.000000,0.000000}%
\pgfsetstrokecolor{currentstroke}%
\pgfsetdash{}{0pt}%
\pgfsys@defobject{currentmarker}{\pgfqpoint{0.000000in}{-0.048611in}}{\pgfqpoint{0.000000in}{0.000000in}}{%
\pgfpathmoveto{\pgfqpoint{0.000000in}{0.000000in}}%
\pgfpathlineto{\pgfqpoint{0.000000in}{-0.048611in}}%
\pgfusepath{stroke,fill}%
}%
\begin{pgfscope}%
\pgfsys@transformshift{4.268521in}{0.331635in}%
\pgfsys@useobject{currentmarker}{}%
\end{pgfscope}%
\end{pgfscope}%
\begin{pgfscope}%
\definecolor{textcolor}{rgb}{0.000000,0.000000,0.000000}%
\pgfsetstrokecolor{textcolor}%
\pgfsetfillcolor{textcolor}%
\pgftext[x=4.268521in,y=0.234413in,,top]{\color{textcolor}\sffamily\fontsize{10.000000}{12.000000}\selectfont table}%
\end{pgfscope}%
\begin{pgfscope}%
\pgfsetbuttcap%
\pgfsetroundjoin%
\definecolor{currentfill}{rgb}{0.000000,0.000000,0.000000}%
\pgfsetfillcolor{currentfill}%
\pgfsetlinewidth{0.803000pt}%
\definecolor{currentstroke}{rgb}{0.000000,0.000000,0.000000}%
\pgfsetstrokecolor{currentstroke}%
\pgfsetdash{}{0pt}%
\pgfsys@defobject{currentmarker}{\pgfqpoint{0.000000in}{-0.048611in}}{\pgfqpoint{0.000000in}{0.000000in}}{%
\pgfpathmoveto{\pgfqpoint{0.000000in}{0.000000in}}%
\pgfpathlineto{\pgfqpoint{0.000000in}{-0.048611in}}%
\pgfusepath{stroke,fill}%
}%
\begin{pgfscope}%
\pgfsys@transformshift{4.931623in}{0.331635in}%
\pgfsys@useobject{currentmarker}{}%
\end{pgfscope}%
\end{pgfscope}%
\begin{pgfscope}%
\definecolor{textcolor}{rgb}{0.000000,0.000000,0.000000}%
\pgfsetstrokecolor{textcolor}%
\pgfsetfillcolor{textcolor}%
\pgftext[x=4.931623in,y=0.234413in,,top]{\color{textcolor}\sffamily\fontsize{10.000000}{12.000000}\selectfont wardrobe}%
\end{pgfscope}%
\begin{pgfscope}%
\pgfsetbuttcap%
\pgfsetroundjoin%
\definecolor{currentfill}{rgb}{0.000000,0.000000,0.000000}%
\pgfsetfillcolor{currentfill}%
\pgfsetlinewidth{0.803000pt}%
\definecolor{currentstroke}{rgb}{0.000000,0.000000,0.000000}%
\pgfsetstrokecolor{currentstroke}%
\pgfsetdash{}{0pt}%
\pgfsys@defobject{currentmarker}{\pgfqpoint{-0.048611in}{0.000000in}}{\pgfqpoint{-0.000000in}{0.000000in}}{%
\pgfpathmoveto{\pgfqpoint{-0.000000in}{0.000000in}}%
\pgfpathlineto{\pgfqpoint{-0.048611in}{0.000000in}}%
\pgfusepath{stroke,fill}%
}%
\begin{pgfscope}%
\pgfsys@transformshift{0.462318in}{0.331635in}%
\pgfsys@useobject{currentmarker}{}%
\end{pgfscope}%
\end{pgfscope}%
\begin{pgfscope}%
\definecolor{textcolor}{rgb}{0.000000,0.000000,0.000000}%
\pgfsetstrokecolor{textcolor}%
\pgfsetfillcolor{textcolor}%
\pgftext[x=0.276731in, y=0.278873in, left, base]{\color{textcolor}\sffamily\fontsize{10.000000}{12.000000}\selectfont 0}%
\end{pgfscope}%
\begin{pgfscope}%
\pgfsetbuttcap%
\pgfsetroundjoin%
\definecolor{currentfill}{rgb}{0.000000,0.000000,0.000000}%
\pgfsetfillcolor{currentfill}%
\pgfsetlinewidth{0.803000pt}%
\definecolor{currentstroke}{rgb}{0.000000,0.000000,0.000000}%
\pgfsetstrokecolor{currentstroke}%
\pgfsetdash{}{0pt}%
\pgfsys@defobject{currentmarker}{\pgfqpoint{-0.048611in}{0.000000in}}{\pgfqpoint{-0.000000in}{0.000000in}}{%
\pgfpathmoveto{\pgfqpoint{-0.000000in}{0.000000in}}%
\pgfpathlineto{\pgfqpoint{-0.048611in}{0.000000in}}%
\pgfusepath{stroke,fill}%
}%
\begin{pgfscope}%
\pgfsys@transformshift{0.462318in}{0.954644in}%
\pgfsys@useobject{currentmarker}{}%
\end{pgfscope}%
\end{pgfscope}%
\begin{pgfscope}%
\definecolor{textcolor}{rgb}{0.000000,0.000000,0.000000}%
\pgfsetstrokecolor{textcolor}%
\pgfsetfillcolor{textcolor}%
\pgftext[x=0.188365in, y=0.901882in, left, base]{\color{textcolor}\sffamily\fontsize{10.000000}{12.000000}\selectfont 20}%
\end{pgfscope}%
\begin{pgfscope}%
\pgfsetbuttcap%
\pgfsetroundjoin%
\definecolor{currentfill}{rgb}{0.000000,0.000000,0.000000}%
\pgfsetfillcolor{currentfill}%
\pgfsetlinewidth{0.803000pt}%
\definecolor{currentstroke}{rgb}{0.000000,0.000000,0.000000}%
\pgfsetstrokecolor{currentstroke}%
\pgfsetdash{}{0pt}%
\pgfsys@defobject{currentmarker}{\pgfqpoint{-0.048611in}{0.000000in}}{\pgfqpoint{-0.000000in}{0.000000in}}{%
\pgfpathmoveto{\pgfqpoint{-0.000000in}{0.000000in}}%
\pgfpathlineto{\pgfqpoint{-0.048611in}{0.000000in}}%
\pgfusepath{stroke,fill}%
}%
\begin{pgfscope}%
\pgfsys@transformshift{0.462318in}{1.577653in}%
\pgfsys@useobject{currentmarker}{}%
\end{pgfscope}%
\end{pgfscope}%
\begin{pgfscope}%
\definecolor{textcolor}{rgb}{0.000000,0.000000,0.000000}%
\pgfsetstrokecolor{textcolor}%
\pgfsetfillcolor{textcolor}%
\pgftext[x=0.188365in, y=1.524891in, left, base]{\color{textcolor}\sffamily\fontsize{10.000000}{12.000000}\selectfont 40}%
\end{pgfscope}%
\begin{pgfscope}%
\pgfsetbuttcap%
\pgfsetroundjoin%
\definecolor{currentfill}{rgb}{0.000000,0.000000,0.000000}%
\pgfsetfillcolor{currentfill}%
\pgfsetlinewidth{0.803000pt}%
\definecolor{currentstroke}{rgb}{0.000000,0.000000,0.000000}%
\pgfsetstrokecolor{currentstroke}%
\pgfsetdash{}{0pt}%
\pgfsys@defobject{currentmarker}{\pgfqpoint{-0.048611in}{0.000000in}}{\pgfqpoint{-0.000000in}{0.000000in}}{%
\pgfpathmoveto{\pgfqpoint{-0.000000in}{0.000000in}}%
\pgfpathlineto{\pgfqpoint{-0.048611in}{0.000000in}}%
\pgfusepath{stroke,fill}%
}%
\begin{pgfscope}%
\pgfsys@transformshift{0.462318in}{2.200662in}%
\pgfsys@useobject{currentmarker}{}%
\end{pgfscope}%
\end{pgfscope}%
\begin{pgfscope}%
\definecolor{textcolor}{rgb}{0.000000,0.000000,0.000000}%
\pgfsetstrokecolor{textcolor}%
\pgfsetfillcolor{textcolor}%
\pgftext[x=0.188365in, y=2.147900in, left, base]{\color{textcolor}\sffamily\fontsize{10.000000}{12.000000}\selectfont 60}%
\end{pgfscope}%
\begin{pgfscope}%
\pgfsetbuttcap%
\pgfsetroundjoin%
\definecolor{currentfill}{rgb}{0.000000,0.000000,0.000000}%
\pgfsetfillcolor{currentfill}%
\pgfsetlinewidth{0.803000pt}%
\definecolor{currentstroke}{rgb}{0.000000,0.000000,0.000000}%
\pgfsetstrokecolor{currentstroke}%
\pgfsetdash{}{0pt}%
\pgfsys@defobject{currentmarker}{\pgfqpoint{-0.048611in}{0.000000in}}{\pgfqpoint{-0.000000in}{0.000000in}}{%
\pgfpathmoveto{\pgfqpoint{-0.000000in}{0.000000in}}%
\pgfpathlineto{\pgfqpoint{-0.048611in}{0.000000in}}%
\pgfusepath{stroke,fill}%
}%
\begin{pgfscope}%
\pgfsys@transformshift{0.462318in}{2.823670in}%
\pgfsys@useobject{currentmarker}{}%
\end{pgfscope}%
\end{pgfscope}%
\begin{pgfscope}%
\definecolor{textcolor}{rgb}{0.000000,0.000000,0.000000}%
\pgfsetstrokecolor{textcolor}%
\pgfsetfillcolor{textcolor}%
\pgftext[x=0.188365in, y=2.770909in, left, base]{\color{textcolor}\sffamily\fontsize{10.000000}{12.000000}\selectfont 80}%
\end{pgfscope}%
\begin{pgfscope}%
\pgfsetbuttcap%
\pgfsetroundjoin%
\definecolor{currentfill}{rgb}{0.000000,0.000000,0.000000}%
\pgfsetfillcolor{currentfill}%
\pgfsetlinewidth{0.803000pt}%
\definecolor{currentstroke}{rgb}{0.000000,0.000000,0.000000}%
\pgfsetstrokecolor{currentstroke}%
\pgfsetdash{}{0pt}%
\pgfsys@defobject{currentmarker}{\pgfqpoint{-0.048611in}{0.000000in}}{\pgfqpoint{-0.000000in}{0.000000in}}{%
\pgfpathmoveto{\pgfqpoint{-0.000000in}{0.000000in}}%
\pgfpathlineto{\pgfqpoint{-0.048611in}{0.000000in}}%
\pgfusepath{stroke,fill}%
}%
\begin{pgfscope}%
\pgfsys@transformshift{0.462318in}{3.446679in}%
\pgfsys@useobject{currentmarker}{}%
\end{pgfscope}%
\end{pgfscope}%
\begin{pgfscope}%
\definecolor{textcolor}{rgb}{0.000000,0.000000,0.000000}%
\pgfsetstrokecolor{textcolor}%
\pgfsetfillcolor{textcolor}%
\pgftext[x=0.100000in, y=3.393918in, left, base]{\color{textcolor}\sffamily\fontsize{10.000000}{12.000000}\selectfont 100}%
\end{pgfscope}%
\begin{pgfscope}%
\pgfsetrectcap%
\pgfsetmiterjoin%
\pgfsetlinewidth{0.803000pt}%
\definecolor{currentstroke}{rgb}{0.000000,0.000000,0.000000}%
\pgfsetstrokecolor{currentstroke}%
\pgfsetdash{}{0pt}%
\pgfpathmoveto{\pgfqpoint{0.462318in}{0.331635in}}%
\pgfpathlineto{\pgfqpoint{0.462318in}{4.027635in}}%
\pgfusepath{stroke}%
\end{pgfscope}%
\begin{pgfscope}%
\pgfsetrectcap%
\pgfsetmiterjoin%
\pgfsetlinewidth{0.803000pt}%
\definecolor{currentstroke}{rgb}{0.000000,0.000000,0.000000}%
\pgfsetstrokecolor{currentstroke}%
\pgfsetdash{}{0pt}%
\pgfpathmoveto{\pgfqpoint{5.422318in}{0.331635in}}%
\pgfpathlineto{\pgfqpoint{5.422318in}{4.027635in}}%
\pgfusepath{stroke}%
\end{pgfscope}%
\begin{pgfscope}%
\pgfsetrectcap%
\pgfsetmiterjoin%
\pgfsetlinewidth{0.803000pt}%
\definecolor{currentstroke}{rgb}{0.000000,0.000000,0.000000}%
\pgfsetstrokecolor{currentstroke}%
\pgfsetdash{}{0pt}%
\pgfpathmoveto{\pgfqpoint{0.462318in}{0.331635in}}%
\pgfpathlineto{\pgfqpoint{5.422318in}{0.331635in}}%
\pgfusepath{stroke}%
\end{pgfscope}%
\begin{pgfscope}%
\pgfsetrectcap%
\pgfsetmiterjoin%
\pgfsetlinewidth{0.803000pt}%
\definecolor{currentstroke}{rgb}{0.000000,0.000000,0.000000}%
\pgfsetstrokecolor{currentstroke}%
\pgfsetdash{}{0pt}%
\pgfpathmoveto{\pgfqpoint{0.462318in}{4.027635in}}%
\pgfpathlineto{\pgfqpoint{5.422318in}{4.027635in}}%
\pgfusepath{stroke}%
\end{pgfscope}%
\begin{pgfscope}%
\definecolor{textcolor}{rgb}{0.000000,0.000000,0.000000}%
\pgfsetstrokecolor{textcolor}%
\pgfsetfillcolor{textcolor}%
\pgftext[x=2.942318in,y=4.110968in,,base]{\color{textcolor}\sffamily\fontsize{12.000000}{14.400000}\selectfont Pix3D Categories in SceneNet}%
\end{pgfscope}%
\end{pgfpicture}%
\makeatother%
\endgroup%
}
    \caption{(Left)Distribution of types of scenes, (Right) Distribution of objects matching the categories of pix3d in scenes}
    \label{fig:distribution of scenes}
\end{figure}


\subsection{Camera ViewPoints}\label{subsec:camera-viewpoints}

The distance of the camera from the target object is a configurable entity.
The user can set the minimum and maximum distance to the target object for the camera to be placed.
The program will then randomly select a point within this range.
Similarly, the height of the camera can be configured by the user.
The camera is programed to always look towards the target object and will be changed if the object is not in the frame.
This is achieved by passing a ray and determining if the center of the target object is visible.
To make the frame realistic, we avoid unrelatable views by applying a constraint on the camera posiiton and make sure that the camera is in front of the target object within an angle of 60 degrees.
Figure~\ref{fig:Camera viewpoints} shows samples of different camera view points on a constant object and scene.

\begin{figure}
    \centering
    \includegraphics[width=.4\textwidth, height = .3\textwidth,valign=m]{/Users/apple/OVGU/Thesis/code/3dReconstruction/report/images/implementation/randomisation/camera1}
    \includegraphics[width=.4\textwidth, height = .3\textwidth,valign=m]{/Users/apple/OVGU/Thesis/code/3dReconstruction/report/images/implementation/randomisation/camera2}\\
    \vspace{0.1cm}
    \includegraphics[width=.4\textwidth, height = .3\textwidth,valign=m]{/Users/apple/OVGU/Thesis/code/3dReconstruction/report/images/implementation/randomisation/camera3}
    \includegraphics[width=.4\textwidth, height = .3\textwidth,valign=m]{/Users/apple/OVGU/Thesis/code/3dReconstruction/report/images/implementation/randomisation/camera4}\\
    \caption{Sample images with different camera viewpoints of same object with a constant scene.}
    \label{fig:Camera viewpoints}
\end{figure}

\subsection{Lightings and Shadows}\label{subsec:lightings-and-shadows}

We consider lighting to be a key component of photorealism.
The shadows formed with different lighting conditions enhance the photorealism of the images.
For this purpose, we use 3 types of lights offered in Unity.

\begin{itemize}
    \item Point light
    \item Spotlight
    \item Sunlight
\end{itemize}

Point lights act as indoor lights for the room the range of which can be varied.
We pre-define 6 light patterns for a room with a cuboid shape.
Knowing the bounds of the room we decide how to place the lights by randomly select 1 to 6 lights with corresponding light patterns on the ceiling.
For the spotlight, only a single light is placed a meter above the target object.
This light gives a variation that focuses only on the target object.
Both point and spot-light have color variation along with varying brightness.
Sunlight is the default light settings in Unity.
This is specifically effective when the room has windows forming more realistic shadows.
In this case, we modulate the brightness, and the angle at which the rays are emitted, which simulates different times of the day.
Additional to these different types of light, we give ceilings the property of light with limited brightness as we observed that the entire room does not light up for a single light source.
This is similar to what BlenderProc~\cite{denninger2019blenderproc} do for all scenarios.
Samples of different lighting conditions and shadows with varying colors are displayed in figure~\ref{fig:Lighting and shadows}.

Another randomized entity is the skybox.
Skybox acts as the outdoor scene for a given room.
For each of the pipelines, skybox changes for every snap taken.
This simulates different outdoor places and weather scenarios, thus providing more randomization for rooms that have open doors and windows.
Samples for different skyboxes are shown in figure~\ref{fig:skybox samples}.

%\begin{figure}
%    \centering
%    \includegraphics[width=.3\textwidth, height = .3\textwidth,valign=m]{/Users/apple/OVGU/Thesis/code/3dReconstruction/report/images/implementation/randomisation/lighting1}
%    \includegraphics[width=.3\textwidth, height = .3\textwidth,valign=m]{/Users/apple/OVGU/Thesis/code/3dReconstruction/report/images/implementation/randomisation/lighting2}\\
%    \vspace{0.1cm}
%    \includegraphics[width=.3\textwidth, height = .3\textwidth,valign=m]{/Users/apple/OVGU/Thesis/code/3dReconstruction/report/images/implementation/randomisation/lighting3}
%    \includegraphics[width=.3\textwidth, height = .3\textwidth,valign=m]{/Users/apple/OVGU/Thesis/code/3dReconstruction/report/images/implementation/randomisation/lighting4}\\
%    \caption{Sample images with different lighting and shadows conditions}
%    \label{fig:Lighting and shadows}
%\end{figure}

\begin{figure}
    \centering
        \includegraphics[width=.24\linewidth,valign=m]{/Users/apple/OVGU/Thesis/code/3dReconstruction/report/images/implementation/randomisation/lighting1}
        \includegraphics[width=.24\linewidth,valign=m]{/Users/apple/OVGU/Thesis/code/3dReconstruction/report/images/implementation/randomisation/lighting2}
        \includegraphics[width=.24\linewidth,valign=m]{/Users/apple/OVGU/Thesis/code/3dReconstruction/report/images/implementation/randomisation/lighting3}
        \includegraphics[width=.24\linewidth,valign=m]{/Users/apple/OVGU/Thesis/code/3dReconstruction/report/images/implementation/randomisation/lighting4}\\
    \vspace{0.1cm}
        \includegraphics[width=.24\linewidth,valign=m]{/Users/apple/OVGU/Thesis/code/3dReconstruction/report/images/implementation/randomisation/lighting5}
        \includegraphics[width=.24\linewidth,valign=m]{/Users/apple/OVGU/Thesis/code/3dReconstruction/report/images/implementation/randomisation/lighting6}
        \includegraphics[width=.24\linewidth,valign=m]{/Users/apple/OVGU/Thesis/code/3dReconstruction/report/images/implementation/randomisation/lighting7}
        \includegraphics[width=.24\linewidth,valign=m]{/Users/apple/OVGU/Thesis/code/3dReconstruction/report/images/implementation/randomisation/lighting8}\\
    \caption{Sample images with different lighting and shadows conditions.First row is samples for light with different intensity and direction. Second row is differnt color for light.}
    \label{fig:Lighting and shadows}
\end{figure}

\subsection{Randomised Texture}\label{subsec:randomised-texture}

The objects in the scene are renamed to standard objects seen in day-to-day life.
The textures are grouped together to these names as folder names.
A total of 982 texture images with 58 regular object categories, some of which are just JPG or PNG images, while there are few textures from cctextures.com with more details like normals, displacement, and roughness.
This makes the textures more realistic.
The distribution of the top 40 texture categories used for randomization of scenes is as shown in figure~\ref{fig:Distribution of textures}, with categories of Pix3d a having higher number of images.
Each scene is randomized for every snap taken of the target object.
Samples of texture randomization of background in the scene with target object in the focus are shown in figure~\ref{fig:Texture Randomisation}.
Randomizing the outdoor environment is important in cases where we have open doors and windows.
Unity provides a wrapper for scenes called skyboxes.
This is something of a globe around the room under observation.
This gives us a varying outdoor environment for the scene with the scenery at the horizon.
We randomize 10 skyboxes to achieve this.
Samples of the skybox are shown in figure~\ref{fig:skybox samples}.


\begin{figure}
    \centering
    \includegraphics[width=.4\textwidth, height = .3\textwidth,valign=m]{/Users/apple/OVGU/Thesis/code/3dReconstruction/report/images/implementation/randomisation/background_texture1}
    \includegraphics[width=.4\textwidth, height = .3\textwidth,valign=m]{/Users/apple/OVGU/Thesis/code/3dReconstruction/report/images/implementation/randomisation/background_texture2}\\
    \vspace{0.1cm}
    \includegraphics[width=.4\textwidth, height = .3\textwidth,valign=m]{/Users/apple/OVGU/Thesis/code/3dReconstruction/report/images/implementation/randomisation/background_texture3}
    \includegraphics[width=.4\textwidth, height = .3\textwidth,valign=m]{/Users/apple/OVGU/Thesis/code/3dReconstruction/report/images/implementation/randomisation/background_texture4}\\
    \caption{Sample images with different textures for same scene.}
    \label{fig:Texture Randomisation}
\end{figure}


%\begin{figure}[!ht]
%        \centering
%        \includegraphics[width=\textwidth,valign=m]{/Users/apple/OVGU/Thesis/code/3dReconstruction/report/images/implementation/scenenet_scenes/distribution_of_textures}
%        \caption{Distribution of textures used on scenes. The categories of pix3d(target furnitures) have higher number of images.}
%        \label{fig:Distribution of textures}
%\end{figure}

\begin{figure}[!ht]
    \centering
    \resizebox{\textwidth}{!}{%% Creator: Matplotlib, PGF backend
%%
%% To include the figure in your LaTeX document, write
%%   \input{<filename>.pgf}
%%
%% Make sure the required packages are loaded in your preamble
%%   \usepackage{pgf}
%%
%% Figures using additional raster images can only be included by \input if
%% they are in the same directory as the main LaTeX file. For loading figures
%% from other directories you can use the `import` package
%%   \usepackage{import}
%%
%% and then include the figures with
%%   \import{<path to file>}{<filename>.pgf}
%%
%% Matplotlib used the following preamble
%%   \usepackage{fontspec}
%%   \setmainfont{DejaVuSerif.ttf}[Path=\detokenize{/Users/apple/opt/anaconda3/envs/kaolin/lib/python3.7/site-packages/matplotlib/mpl-data/fonts/ttf/}]
%%   \setsansfont{DejaVuSans.ttf}[Path=\detokenize{/Users/apple/opt/anaconda3/envs/kaolin/lib/python3.7/site-packages/matplotlib/mpl-data/fonts/ttf/}]
%%   \setmonofont{DejaVuSansMono.ttf}[Path=\detokenize{/Users/apple/opt/anaconda3/envs/kaolin/lib/python3.7/site-packages/matplotlib/mpl-data/fonts/ttf/}]
%%
\begingroup%
\makeatletter%
\begin{pgfpicture}%
\pgfpathrectangle{\pgfpointorigin}{\pgfqpoint{5.522318in}{4.872128in}}%
\pgfusepath{use as bounding box, clip}%
\begin{pgfscope}%
\pgfsetbuttcap%
\pgfsetmiterjoin%
\definecolor{currentfill}{rgb}{1.000000,1.000000,1.000000}%
\pgfsetfillcolor{currentfill}%
\pgfsetlinewidth{0.000000pt}%
\definecolor{currentstroke}{rgb}{1.000000,1.000000,1.000000}%
\pgfsetstrokecolor{currentstroke}%
\pgfsetdash{}{0pt}%
\pgfpathmoveto{\pgfqpoint{0.000000in}{0.000000in}}%
\pgfpathlineto{\pgfqpoint{5.522318in}{0.000000in}}%
\pgfpathlineto{\pgfqpoint{5.522318in}{4.872128in}}%
\pgfpathlineto{\pgfqpoint{0.000000in}{4.872128in}}%
\pgfpathclose%
\pgfusepath{fill}%
\end{pgfscope}%
\begin{pgfscope}%
\pgfsetbuttcap%
\pgfsetmiterjoin%
\definecolor{currentfill}{rgb}{1.000000,1.000000,1.000000}%
\pgfsetfillcolor{currentfill}%
\pgfsetlinewidth{0.000000pt}%
\definecolor{currentstroke}{rgb}{0.000000,0.000000,0.000000}%
\pgfsetstrokecolor{currentstroke}%
\pgfsetstrokeopacity{0.000000}%
\pgfsetdash{}{0pt}%
\pgfpathmoveto{\pgfqpoint{0.462318in}{0.866167in}}%
\pgfpathlineto{\pgfqpoint{5.422318in}{0.866167in}}%
\pgfpathlineto{\pgfqpoint{5.422318in}{4.562168in}}%
\pgfpathlineto{\pgfqpoint{0.462318in}{4.562168in}}%
\pgfpathclose%
\pgfusepath{fill}%
\end{pgfscope}%
\begin{pgfscope}%
\pgfpathrectangle{\pgfqpoint{0.462318in}{0.866167in}}{\pgfqpoint{4.960000in}{3.696000in}}%
\pgfusepath{clip}%
\pgfsetbuttcap%
\pgfsetmiterjoin%
\definecolor{currentfill}{rgb}{0.121569,0.466667,0.705882}%
\pgfsetfillcolor{currentfill}%
\pgfsetlinewidth{0.000000pt}%
\definecolor{currentstroke}{rgb}{0.000000,0.000000,0.000000}%
\pgfsetstrokecolor{currentstroke}%
\pgfsetstrokeopacity{0.000000}%
\pgfsetdash{}{0pt}%
\pgfpathmoveto{\pgfqpoint{0.687773in}{0.866167in}}%
\pgfpathlineto{\pgfqpoint{0.778408in}{0.866167in}}%
\pgfpathlineto{\pgfqpoint{0.778408in}{4.386167in}}%
\pgfpathlineto{\pgfqpoint{0.687773in}{4.386167in}}%
\pgfpathclose%
\pgfusepath{fill}%
\end{pgfscope}%
\begin{pgfscope}%
\pgfpathrectangle{\pgfqpoint{0.462318in}{0.866167in}}{\pgfqpoint{4.960000in}{3.696000in}}%
\pgfusepath{clip}%
\pgfsetbuttcap%
\pgfsetmiterjoin%
\definecolor{currentfill}{rgb}{0.121569,0.466667,0.705882}%
\pgfsetfillcolor{currentfill}%
\pgfsetlinewidth{0.000000pt}%
\definecolor{currentstroke}{rgb}{0.000000,0.000000,0.000000}%
\pgfsetstrokecolor{currentstroke}%
\pgfsetstrokeopacity{0.000000}%
\pgfsetdash{}{0pt}%
\pgfpathmoveto{\pgfqpoint{0.801066in}{0.866167in}}%
\pgfpathlineto{\pgfqpoint{0.891701in}{0.866167in}}%
\pgfpathlineto{\pgfqpoint{0.891701in}{4.023281in}}%
\pgfpathlineto{\pgfqpoint{0.801066in}{4.023281in}}%
\pgfpathclose%
\pgfusepath{fill}%
\end{pgfscope}%
\begin{pgfscope}%
\pgfpathrectangle{\pgfqpoint{0.462318in}{0.866167in}}{\pgfqpoint{4.960000in}{3.696000in}}%
\pgfusepath{clip}%
\pgfsetbuttcap%
\pgfsetmiterjoin%
\definecolor{currentfill}{rgb}{0.121569,0.466667,0.705882}%
\pgfsetfillcolor{currentfill}%
\pgfsetlinewidth{0.000000pt}%
\definecolor{currentstroke}{rgb}{0.000000,0.000000,0.000000}%
\pgfsetstrokecolor{currentstroke}%
\pgfsetstrokeopacity{0.000000}%
\pgfsetdash{}{0pt}%
\pgfpathmoveto{\pgfqpoint{0.914360in}{0.866167in}}%
\pgfpathlineto{\pgfqpoint{1.004995in}{0.866167in}}%
\pgfpathlineto{\pgfqpoint{1.004995in}{2.825755in}}%
\pgfpathlineto{\pgfqpoint{0.914360in}{2.825755in}}%
\pgfpathclose%
\pgfusepath{fill}%
\end{pgfscope}%
\begin{pgfscope}%
\pgfpathrectangle{\pgfqpoint{0.462318in}{0.866167in}}{\pgfqpoint{4.960000in}{3.696000in}}%
\pgfusepath{clip}%
\pgfsetbuttcap%
\pgfsetmiterjoin%
\definecolor{currentfill}{rgb}{0.121569,0.466667,0.705882}%
\pgfsetfillcolor{currentfill}%
\pgfsetlinewidth{0.000000pt}%
\definecolor{currentstroke}{rgb}{0.000000,0.000000,0.000000}%
\pgfsetstrokecolor{currentstroke}%
\pgfsetstrokeopacity{0.000000}%
\pgfsetdash{}{0pt}%
\pgfpathmoveto{\pgfqpoint{1.027654in}{0.866167in}}%
\pgfpathlineto{\pgfqpoint{1.118289in}{0.866167in}}%
\pgfpathlineto{\pgfqpoint{1.118289in}{2.716889in}}%
\pgfpathlineto{\pgfqpoint{1.027654in}{2.716889in}}%
\pgfpathclose%
\pgfusepath{fill}%
\end{pgfscope}%
\begin{pgfscope}%
\pgfpathrectangle{\pgfqpoint{0.462318in}{0.866167in}}{\pgfqpoint{4.960000in}{3.696000in}}%
\pgfusepath{clip}%
\pgfsetbuttcap%
\pgfsetmiterjoin%
\definecolor{currentfill}{rgb}{0.121569,0.466667,0.705882}%
\pgfsetfillcolor{currentfill}%
\pgfsetlinewidth{0.000000pt}%
\definecolor{currentstroke}{rgb}{0.000000,0.000000,0.000000}%
\pgfsetstrokecolor{currentstroke}%
\pgfsetstrokeopacity{0.000000}%
\pgfsetdash{}{0pt}%
\pgfpathmoveto{\pgfqpoint{1.140948in}{0.866167in}}%
\pgfpathlineto{\pgfqpoint{1.231583in}{0.866167in}}%
\pgfpathlineto{\pgfqpoint{1.231583in}{2.716889in}}%
\pgfpathlineto{\pgfqpoint{1.140948in}{2.716889in}}%
\pgfpathclose%
\pgfusepath{fill}%
\end{pgfscope}%
\begin{pgfscope}%
\pgfpathrectangle{\pgfqpoint{0.462318in}{0.866167in}}{\pgfqpoint{4.960000in}{3.696000in}}%
\pgfusepath{clip}%
\pgfsetbuttcap%
\pgfsetmiterjoin%
\definecolor{currentfill}{rgb}{0.121569,0.466667,0.705882}%
\pgfsetfillcolor{currentfill}%
\pgfsetlinewidth{0.000000pt}%
\definecolor{currentstroke}{rgb}{0.000000,0.000000,0.000000}%
\pgfsetstrokecolor{currentstroke}%
\pgfsetstrokeopacity{0.000000}%
\pgfsetdash{}{0pt}%
\pgfpathmoveto{\pgfqpoint{1.254241in}{0.866167in}}%
\pgfpathlineto{\pgfqpoint{1.344876in}{0.866167in}}%
\pgfpathlineto{\pgfqpoint{1.344876in}{2.716889in}}%
\pgfpathlineto{\pgfqpoint{1.254241in}{2.716889in}}%
\pgfpathclose%
\pgfusepath{fill}%
\end{pgfscope}%
\begin{pgfscope}%
\pgfpathrectangle{\pgfqpoint{0.462318in}{0.866167in}}{\pgfqpoint{4.960000in}{3.696000in}}%
\pgfusepath{clip}%
\pgfsetbuttcap%
\pgfsetmiterjoin%
\definecolor{currentfill}{rgb}{0.121569,0.466667,0.705882}%
\pgfsetfillcolor{currentfill}%
\pgfsetlinewidth{0.000000pt}%
\definecolor{currentstroke}{rgb}{0.000000,0.000000,0.000000}%
\pgfsetstrokecolor{currentstroke}%
\pgfsetstrokeopacity{0.000000}%
\pgfsetdash{}{0pt}%
\pgfpathmoveto{\pgfqpoint{1.367535in}{0.866167in}}%
\pgfpathlineto{\pgfqpoint{1.458170in}{0.866167in}}%
\pgfpathlineto{\pgfqpoint{1.458170in}{2.571735in}}%
\pgfpathlineto{\pgfqpoint{1.367535in}{2.571735in}}%
\pgfpathclose%
\pgfusepath{fill}%
\end{pgfscope}%
\begin{pgfscope}%
\pgfpathrectangle{\pgfqpoint{0.462318in}{0.866167in}}{\pgfqpoint{4.960000in}{3.696000in}}%
\pgfusepath{clip}%
\pgfsetbuttcap%
\pgfsetmiterjoin%
\definecolor{currentfill}{rgb}{0.121569,0.466667,0.705882}%
\pgfsetfillcolor{currentfill}%
\pgfsetlinewidth{0.000000pt}%
\definecolor{currentstroke}{rgb}{0.000000,0.000000,0.000000}%
\pgfsetstrokecolor{currentstroke}%
\pgfsetstrokeopacity{0.000000}%
\pgfsetdash{}{0pt}%
\pgfpathmoveto{\pgfqpoint{1.480829in}{0.866167in}}%
\pgfpathlineto{\pgfqpoint{1.571464in}{0.866167in}}%
\pgfpathlineto{\pgfqpoint{1.571464in}{2.354003in}}%
\pgfpathlineto{\pgfqpoint{1.480829in}{2.354003in}}%
\pgfpathclose%
\pgfusepath{fill}%
\end{pgfscope}%
\begin{pgfscope}%
\pgfpathrectangle{\pgfqpoint{0.462318in}{0.866167in}}{\pgfqpoint{4.960000in}{3.696000in}}%
\pgfusepath{clip}%
\pgfsetbuttcap%
\pgfsetmiterjoin%
\definecolor{currentfill}{rgb}{0.121569,0.466667,0.705882}%
\pgfsetfillcolor{currentfill}%
\pgfsetlinewidth{0.000000pt}%
\definecolor{currentstroke}{rgb}{0.000000,0.000000,0.000000}%
\pgfsetstrokecolor{currentstroke}%
\pgfsetstrokeopacity{0.000000}%
\pgfsetdash{}{0pt}%
\pgfpathmoveto{\pgfqpoint{1.594123in}{0.866167in}}%
\pgfpathlineto{\pgfqpoint{1.684758in}{0.866167in}}%
\pgfpathlineto{\pgfqpoint{1.684758in}{2.208848in}}%
\pgfpathlineto{\pgfqpoint{1.594123in}{2.208848in}}%
\pgfpathclose%
\pgfusepath{fill}%
\end{pgfscope}%
\begin{pgfscope}%
\pgfpathrectangle{\pgfqpoint{0.462318in}{0.866167in}}{\pgfqpoint{4.960000in}{3.696000in}}%
\pgfusepath{clip}%
\pgfsetbuttcap%
\pgfsetmiterjoin%
\definecolor{currentfill}{rgb}{0.121569,0.466667,0.705882}%
\pgfsetfillcolor{currentfill}%
\pgfsetlinewidth{0.000000pt}%
\definecolor{currentstroke}{rgb}{0.000000,0.000000,0.000000}%
\pgfsetstrokecolor{currentstroke}%
\pgfsetstrokeopacity{0.000000}%
\pgfsetdash{}{0pt}%
\pgfpathmoveto{\pgfqpoint{1.707416in}{0.866167in}}%
\pgfpathlineto{\pgfqpoint{1.798051in}{0.866167in}}%
\pgfpathlineto{\pgfqpoint{1.798051in}{1.918539in}}%
\pgfpathlineto{\pgfqpoint{1.707416in}{1.918539in}}%
\pgfpathclose%
\pgfusepath{fill}%
\end{pgfscope}%
\begin{pgfscope}%
\pgfpathrectangle{\pgfqpoint{0.462318in}{0.866167in}}{\pgfqpoint{4.960000in}{3.696000in}}%
\pgfusepath{clip}%
\pgfsetbuttcap%
\pgfsetmiterjoin%
\definecolor{currentfill}{rgb}{0.121569,0.466667,0.705882}%
\pgfsetfillcolor{currentfill}%
\pgfsetlinewidth{0.000000pt}%
\definecolor{currentstroke}{rgb}{0.000000,0.000000,0.000000}%
\pgfsetstrokecolor{currentstroke}%
\pgfsetstrokeopacity{0.000000}%
\pgfsetdash{}{0pt}%
\pgfpathmoveto{\pgfqpoint{1.820710in}{0.866167in}}%
\pgfpathlineto{\pgfqpoint{1.911345in}{0.866167in}}%
\pgfpathlineto{\pgfqpoint{1.911345in}{1.882250in}}%
\pgfpathlineto{\pgfqpoint{1.820710in}{1.882250in}}%
\pgfpathclose%
\pgfusepath{fill}%
\end{pgfscope}%
\begin{pgfscope}%
\pgfpathrectangle{\pgfqpoint{0.462318in}{0.866167in}}{\pgfqpoint{4.960000in}{3.696000in}}%
\pgfusepath{clip}%
\pgfsetbuttcap%
\pgfsetmiterjoin%
\definecolor{currentfill}{rgb}{0.121569,0.466667,0.705882}%
\pgfsetfillcolor{currentfill}%
\pgfsetlinewidth{0.000000pt}%
\definecolor{currentstroke}{rgb}{0.000000,0.000000,0.000000}%
\pgfsetstrokecolor{currentstroke}%
\pgfsetstrokeopacity{0.000000}%
\pgfsetdash{}{0pt}%
\pgfpathmoveto{\pgfqpoint{1.934004in}{0.866167in}}%
\pgfpathlineto{\pgfqpoint{2.024639in}{0.866167in}}%
\pgfpathlineto{\pgfqpoint{2.024639in}{1.882250in}}%
\pgfpathlineto{\pgfqpoint{1.934004in}{1.882250in}}%
\pgfpathclose%
\pgfusepath{fill}%
\end{pgfscope}%
\begin{pgfscope}%
\pgfpathrectangle{\pgfqpoint{0.462318in}{0.866167in}}{\pgfqpoint{4.960000in}{3.696000in}}%
\pgfusepath{clip}%
\pgfsetbuttcap%
\pgfsetmiterjoin%
\definecolor{currentfill}{rgb}{0.121569,0.466667,0.705882}%
\pgfsetfillcolor{currentfill}%
\pgfsetlinewidth{0.000000pt}%
\definecolor{currentstroke}{rgb}{0.000000,0.000000,0.000000}%
\pgfsetstrokecolor{currentstroke}%
\pgfsetstrokeopacity{0.000000}%
\pgfsetdash{}{0pt}%
\pgfpathmoveto{\pgfqpoint{2.047298in}{0.866167in}}%
\pgfpathlineto{\pgfqpoint{2.137933in}{0.866167in}}%
\pgfpathlineto{\pgfqpoint{2.137933in}{1.845961in}}%
\pgfpathlineto{\pgfqpoint{2.047298in}{1.845961in}}%
\pgfpathclose%
\pgfusepath{fill}%
\end{pgfscope}%
\begin{pgfscope}%
\pgfpathrectangle{\pgfqpoint{0.462318in}{0.866167in}}{\pgfqpoint{4.960000in}{3.696000in}}%
\pgfusepath{clip}%
\pgfsetbuttcap%
\pgfsetmiterjoin%
\definecolor{currentfill}{rgb}{0.121569,0.466667,0.705882}%
\pgfsetfillcolor{currentfill}%
\pgfsetlinewidth{0.000000pt}%
\definecolor{currentstroke}{rgb}{0.000000,0.000000,0.000000}%
\pgfsetstrokecolor{currentstroke}%
\pgfsetstrokeopacity{0.000000}%
\pgfsetdash{}{0pt}%
\pgfpathmoveto{\pgfqpoint{2.160591in}{0.866167in}}%
\pgfpathlineto{\pgfqpoint{2.251226in}{0.866167in}}%
\pgfpathlineto{\pgfqpoint{2.251226in}{1.664518in}}%
\pgfpathlineto{\pgfqpoint{2.160591in}{1.664518in}}%
\pgfpathclose%
\pgfusepath{fill}%
\end{pgfscope}%
\begin{pgfscope}%
\pgfpathrectangle{\pgfqpoint{0.462318in}{0.866167in}}{\pgfqpoint{4.960000in}{3.696000in}}%
\pgfusepath{clip}%
\pgfsetbuttcap%
\pgfsetmiterjoin%
\definecolor{currentfill}{rgb}{0.121569,0.466667,0.705882}%
\pgfsetfillcolor{currentfill}%
\pgfsetlinewidth{0.000000pt}%
\definecolor{currentstroke}{rgb}{0.000000,0.000000,0.000000}%
\pgfsetstrokecolor{currentstroke}%
\pgfsetstrokeopacity{0.000000}%
\pgfsetdash{}{0pt}%
\pgfpathmoveto{\pgfqpoint{2.273885in}{0.866167in}}%
\pgfpathlineto{\pgfqpoint{2.364520in}{0.866167in}}%
\pgfpathlineto{\pgfqpoint{2.364520in}{1.664518in}}%
\pgfpathlineto{\pgfqpoint{2.273885in}{1.664518in}}%
\pgfpathclose%
\pgfusepath{fill}%
\end{pgfscope}%
\begin{pgfscope}%
\pgfpathrectangle{\pgfqpoint{0.462318in}{0.866167in}}{\pgfqpoint{4.960000in}{3.696000in}}%
\pgfusepath{clip}%
\pgfsetbuttcap%
\pgfsetmiterjoin%
\definecolor{currentfill}{rgb}{0.121569,0.466667,0.705882}%
\pgfsetfillcolor{currentfill}%
\pgfsetlinewidth{0.000000pt}%
\definecolor{currentstroke}{rgb}{0.000000,0.000000,0.000000}%
\pgfsetstrokecolor{currentstroke}%
\pgfsetstrokeopacity{0.000000}%
\pgfsetdash{}{0pt}%
\pgfpathmoveto{\pgfqpoint{2.387179in}{0.866167in}}%
\pgfpathlineto{\pgfqpoint{2.477814in}{0.866167in}}%
\pgfpathlineto{\pgfqpoint{2.477814in}{1.591941in}}%
\pgfpathlineto{\pgfqpoint{2.387179in}{1.591941in}}%
\pgfpathclose%
\pgfusepath{fill}%
\end{pgfscope}%
\begin{pgfscope}%
\pgfpathrectangle{\pgfqpoint{0.462318in}{0.866167in}}{\pgfqpoint{4.960000in}{3.696000in}}%
\pgfusepath{clip}%
\pgfsetbuttcap%
\pgfsetmiterjoin%
\definecolor{currentfill}{rgb}{0.121569,0.466667,0.705882}%
\pgfsetfillcolor{currentfill}%
\pgfsetlinewidth{0.000000pt}%
\definecolor{currentstroke}{rgb}{0.000000,0.000000,0.000000}%
\pgfsetstrokecolor{currentstroke}%
\pgfsetstrokeopacity{0.000000}%
\pgfsetdash{}{0pt}%
\pgfpathmoveto{\pgfqpoint{2.500473in}{0.866167in}}%
\pgfpathlineto{\pgfqpoint{2.591108in}{0.866167in}}%
\pgfpathlineto{\pgfqpoint{2.591108in}{1.446786in}}%
\pgfpathlineto{\pgfqpoint{2.500473in}{1.446786in}}%
\pgfpathclose%
\pgfusepath{fill}%
\end{pgfscope}%
\begin{pgfscope}%
\pgfpathrectangle{\pgfqpoint{0.462318in}{0.866167in}}{\pgfqpoint{4.960000in}{3.696000in}}%
\pgfusepath{clip}%
\pgfsetbuttcap%
\pgfsetmiterjoin%
\definecolor{currentfill}{rgb}{0.121569,0.466667,0.705882}%
\pgfsetfillcolor{currentfill}%
\pgfsetlinewidth{0.000000pt}%
\definecolor{currentstroke}{rgb}{0.000000,0.000000,0.000000}%
\pgfsetstrokecolor{currentstroke}%
\pgfsetstrokeopacity{0.000000}%
\pgfsetdash{}{0pt}%
\pgfpathmoveto{\pgfqpoint{2.613766in}{0.866167in}}%
\pgfpathlineto{\pgfqpoint{2.704401in}{0.866167in}}%
\pgfpathlineto{\pgfqpoint{2.704401in}{1.446786in}}%
\pgfpathlineto{\pgfqpoint{2.613766in}{1.446786in}}%
\pgfpathclose%
\pgfusepath{fill}%
\end{pgfscope}%
\begin{pgfscope}%
\pgfpathrectangle{\pgfqpoint{0.462318in}{0.866167in}}{\pgfqpoint{4.960000in}{3.696000in}}%
\pgfusepath{clip}%
\pgfsetbuttcap%
\pgfsetmiterjoin%
\definecolor{currentfill}{rgb}{0.121569,0.466667,0.705882}%
\pgfsetfillcolor{currentfill}%
\pgfsetlinewidth{0.000000pt}%
\definecolor{currentstroke}{rgb}{0.000000,0.000000,0.000000}%
\pgfsetstrokecolor{currentstroke}%
\pgfsetstrokeopacity{0.000000}%
\pgfsetdash{}{0pt}%
\pgfpathmoveto{\pgfqpoint{2.727060in}{0.866167in}}%
\pgfpathlineto{\pgfqpoint{2.817695in}{0.866167in}}%
\pgfpathlineto{\pgfqpoint{2.817695in}{1.446786in}}%
\pgfpathlineto{\pgfqpoint{2.727060in}{1.446786in}}%
\pgfpathclose%
\pgfusepath{fill}%
\end{pgfscope}%
\begin{pgfscope}%
\pgfpathrectangle{\pgfqpoint{0.462318in}{0.866167in}}{\pgfqpoint{4.960000in}{3.696000in}}%
\pgfusepath{clip}%
\pgfsetbuttcap%
\pgfsetmiterjoin%
\definecolor{currentfill}{rgb}{0.121569,0.466667,0.705882}%
\pgfsetfillcolor{currentfill}%
\pgfsetlinewidth{0.000000pt}%
\definecolor{currentstroke}{rgb}{0.000000,0.000000,0.000000}%
\pgfsetstrokecolor{currentstroke}%
\pgfsetstrokeopacity{0.000000}%
\pgfsetdash{}{0pt}%
\pgfpathmoveto{\pgfqpoint{2.840354in}{0.866167in}}%
\pgfpathlineto{\pgfqpoint{2.930989in}{0.866167in}}%
\pgfpathlineto{\pgfqpoint{2.930989in}{1.410497in}}%
\pgfpathlineto{\pgfqpoint{2.840354in}{1.410497in}}%
\pgfpathclose%
\pgfusepath{fill}%
\end{pgfscope}%
\begin{pgfscope}%
\pgfpathrectangle{\pgfqpoint{0.462318in}{0.866167in}}{\pgfqpoint{4.960000in}{3.696000in}}%
\pgfusepath{clip}%
\pgfsetbuttcap%
\pgfsetmiterjoin%
\definecolor{currentfill}{rgb}{0.121569,0.466667,0.705882}%
\pgfsetfillcolor{currentfill}%
\pgfsetlinewidth{0.000000pt}%
\definecolor{currentstroke}{rgb}{0.000000,0.000000,0.000000}%
\pgfsetstrokecolor{currentstroke}%
\pgfsetstrokeopacity{0.000000}%
\pgfsetdash{}{0pt}%
\pgfpathmoveto{\pgfqpoint{2.953648in}{0.866167in}}%
\pgfpathlineto{\pgfqpoint{3.044283in}{0.866167in}}%
\pgfpathlineto{\pgfqpoint{3.044283in}{1.374209in}}%
\pgfpathlineto{\pgfqpoint{2.953648in}{1.374209in}}%
\pgfpathclose%
\pgfusepath{fill}%
\end{pgfscope}%
\begin{pgfscope}%
\pgfpathrectangle{\pgfqpoint{0.462318in}{0.866167in}}{\pgfqpoint{4.960000in}{3.696000in}}%
\pgfusepath{clip}%
\pgfsetbuttcap%
\pgfsetmiterjoin%
\definecolor{currentfill}{rgb}{0.121569,0.466667,0.705882}%
\pgfsetfillcolor{currentfill}%
\pgfsetlinewidth{0.000000pt}%
\definecolor{currentstroke}{rgb}{0.000000,0.000000,0.000000}%
\pgfsetstrokecolor{currentstroke}%
\pgfsetstrokeopacity{0.000000}%
\pgfsetdash{}{0pt}%
\pgfpathmoveto{\pgfqpoint{3.066941in}{0.866167in}}%
\pgfpathlineto{\pgfqpoint{3.157576in}{0.866167in}}%
\pgfpathlineto{\pgfqpoint{3.157576in}{1.374209in}}%
\pgfpathlineto{\pgfqpoint{3.066941in}{1.374209in}}%
\pgfpathclose%
\pgfusepath{fill}%
\end{pgfscope}%
\begin{pgfscope}%
\pgfpathrectangle{\pgfqpoint{0.462318in}{0.866167in}}{\pgfqpoint{4.960000in}{3.696000in}}%
\pgfusepath{clip}%
\pgfsetbuttcap%
\pgfsetmiterjoin%
\definecolor{currentfill}{rgb}{0.121569,0.466667,0.705882}%
\pgfsetfillcolor{currentfill}%
\pgfsetlinewidth{0.000000pt}%
\definecolor{currentstroke}{rgb}{0.000000,0.000000,0.000000}%
\pgfsetstrokecolor{currentstroke}%
\pgfsetstrokeopacity{0.000000}%
\pgfsetdash{}{0pt}%
\pgfpathmoveto{\pgfqpoint{3.180235in}{0.866167in}}%
\pgfpathlineto{\pgfqpoint{3.270870in}{0.866167in}}%
\pgfpathlineto{\pgfqpoint{3.270870in}{1.374209in}}%
\pgfpathlineto{\pgfqpoint{3.180235in}{1.374209in}}%
\pgfpathclose%
\pgfusepath{fill}%
\end{pgfscope}%
\begin{pgfscope}%
\pgfpathrectangle{\pgfqpoint{0.462318in}{0.866167in}}{\pgfqpoint{4.960000in}{3.696000in}}%
\pgfusepath{clip}%
\pgfsetbuttcap%
\pgfsetmiterjoin%
\definecolor{currentfill}{rgb}{0.121569,0.466667,0.705882}%
\pgfsetfillcolor{currentfill}%
\pgfsetlinewidth{0.000000pt}%
\definecolor{currentstroke}{rgb}{0.000000,0.000000,0.000000}%
\pgfsetstrokecolor{currentstroke}%
\pgfsetstrokeopacity{0.000000}%
\pgfsetdash{}{0pt}%
\pgfpathmoveto{\pgfqpoint{3.293529in}{0.866167in}}%
\pgfpathlineto{\pgfqpoint{3.384164in}{0.866167in}}%
\pgfpathlineto{\pgfqpoint{3.384164in}{1.301631in}}%
\pgfpathlineto{\pgfqpoint{3.293529in}{1.301631in}}%
\pgfpathclose%
\pgfusepath{fill}%
\end{pgfscope}%
\begin{pgfscope}%
\pgfpathrectangle{\pgfqpoint{0.462318in}{0.866167in}}{\pgfqpoint{4.960000in}{3.696000in}}%
\pgfusepath{clip}%
\pgfsetbuttcap%
\pgfsetmiterjoin%
\definecolor{currentfill}{rgb}{0.121569,0.466667,0.705882}%
\pgfsetfillcolor{currentfill}%
\pgfsetlinewidth{0.000000pt}%
\definecolor{currentstroke}{rgb}{0.000000,0.000000,0.000000}%
\pgfsetstrokecolor{currentstroke}%
\pgfsetstrokeopacity{0.000000}%
\pgfsetdash{}{0pt}%
\pgfpathmoveto{\pgfqpoint{3.406823in}{0.866167in}}%
\pgfpathlineto{\pgfqpoint{3.497458in}{0.866167in}}%
\pgfpathlineto{\pgfqpoint{3.497458in}{1.301631in}}%
\pgfpathlineto{\pgfqpoint{3.406823in}{1.301631in}}%
\pgfpathclose%
\pgfusepath{fill}%
\end{pgfscope}%
\begin{pgfscope}%
\pgfpathrectangle{\pgfqpoint{0.462318in}{0.866167in}}{\pgfqpoint{4.960000in}{3.696000in}}%
\pgfusepath{clip}%
\pgfsetbuttcap%
\pgfsetmiterjoin%
\definecolor{currentfill}{rgb}{0.121569,0.466667,0.705882}%
\pgfsetfillcolor{currentfill}%
\pgfsetlinewidth{0.000000pt}%
\definecolor{currentstroke}{rgb}{0.000000,0.000000,0.000000}%
\pgfsetstrokecolor{currentstroke}%
\pgfsetstrokeopacity{0.000000}%
\pgfsetdash{}{0pt}%
\pgfpathmoveto{\pgfqpoint{3.520116in}{0.866167in}}%
\pgfpathlineto{\pgfqpoint{3.610751in}{0.866167in}}%
\pgfpathlineto{\pgfqpoint{3.610751in}{1.301631in}}%
\pgfpathlineto{\pgfqpoint{3.520116in}{1.301631in}}%
\pgfpathclose%
\pgfusepath{fill}%
\end{pgfscope}%
\begin{pgfscope}%
\pgfpathrectangle{\pgfqpoint{0.462318in}{0.866167in}}{\pgfqpoint{4.960000in}{3.696000in}}%
\pgfusepath{clip}%
\pgfsetbuttcap%
\pgfsetmiterjoin%
\definecolor{currentfill}{rgb}{0.121569,0.466667,0.705882}%
\pgfsetfillcolor{currentfill}%
\pgfsetlinewidth{0.000000pt}%
\definecolor{currentstroke}{rgb}{0.000000,0.000000,0.000000}%
\pgfsetstrokecolor{currentstroke}%
\pgfsetstrokeopacity{0.000000}%
\pgfsetdash{}{0pt}%
\pgfpathmoveto{\pgfqpoint{3.633410in}{0.866167in}}%
\pgfpathlineto{\pgfqpoint{3.724045in}{0.866167in}}%
\pgfpathlineto{\pgfqpoint{3.724045in}{1.156477in}}%
\pgfpathlineto{\pgfqpoint{3.633410in}{1.156477in}}%
\pgfpathclose%
\pgfusepath{fill}%
\end{pgfscope}%
\begin{pgfscope}%
\pgfpathrectangle{\pgfqpoint{0.462318in}{0.866167in}}{\pgfqpoint{4.960000in}{3.696000in}}%
\pgfusepath{clip}%
\pgfsetbuttcap%
\pgfsetmiterjoin%
\definecolor{currentfill}{rgb}{0.121569,0.466667,0.705882}%
\pgfsetfillcolor{currentfill}%
\pgfsetlinewidth{0.000000pt}%
\definecolor{currentstroke}{rgb}{0.000000,0.000000,0.000000}%
\pgfsetstrokecolor{currentstroke}%
\pgfsetstrokeopacity{0.000000}%
\pgfsetdash{}{0pt}%
\pgfpathmoveto{\pgfqpoint{3.746704in}{0.866167in}}%
\pgfpathlineto{\pgfqpoint{3.837339in}{0.866167in}}%
\pgfpathlineto{\pgfqpoint{3.837339in}{1.156477in}}%
\pgfpathlineto{\pgfqpoint{3.746704in}{1.156477in}}%
\pgfpathclose%
\pgfusepath{fill}%
\end{pgfscope}%
\begin{pgfscope}%
\pgfpathrectangle{\pgfqpoint{0.462318in}{0.866167in}}{\pgfqpoint{4.960000in}{3.696000in}}%
\pgfusepath{clip}%
\pgfsetbuttcap%
\pgfsetmiterjoin%
\definecolor{currentfill}{rgb}{0.121569,0.466667,0.705882}%
\pgfsetfillcolor{currentfill}%
\pgfsetlinewidth{0.000000pt}%
\definecolor{currentstroke}{rgb}{0.000000,0.000000,0.000000}%
\pgfsetstrokecolor{currentstroke}%
\pgfsetstrokeopacity{0.000000}%
\pgfsetdash{}{0pt}%
\pgfpathmoveto{\pgfqpoint{3.859998in}{0.866167in}}%
\pgfpathlineto{\pgfqpoint{3.950632in}{0.866167in}}%
\pgfpathlineto{\pgfqpoint{3.950632in}{1.120188in}}%
\pgfpathlineto{\pgfqpoint{3.859998in}{1.120188in}}%
\pgfpathclose%
\pgfusepath{fill}%
\end{pgfscope}%
\begin{pgfscope}%
\pgfpathrectangle{\pgfqpoint{0.462318in}{0.866167in}}{\pgfqpoint{4.960000in}{3.696000in}}%
\pgfusepath{clip}%
\pgfsetbuttcap%
\pgfsetmiterjoin%
\definecolor{currentfill}{rgb}{0.121569,0.466667,0.705882}%
\pgfsetfillcolor{currentfill}%
\pgfsetlinewidth{0.000000pt}%
\definecolor{currentstroke}{rgb}{0.000000,0.000000,0.000000}%
\pgfsetstrokecolor{currentstroke}%
\pgfsetstrokeopacity{0.000000}%
\pgfsetdash{}{0pt}%
\pgfpathmoveto{\pgfqpoint{3.973291in}{0.866167in}}%
\pgfpathlineto{\pgfqpoint{4.063926in}{0.866167in}}%
\pgfpathlineto{\pgfqpoint{4.063926in}{1.120188in}}%
\pgfpathlineto{\pgfqpoint{3.973291in}{1.120188in}}%
\pgfpathclose%
\pgfusepath{fill}%
\end{pgfscope}%
\begin{pgfscope}%
\pgfpathrectangle{\pgfqpoint{0.462318in}{0.866167in}}{\pgfqpoint{4.960000in}{3.696000in}}%
\pgfusepath{clip}%
\pgfsetbuttcap%
\pgfsetmiterjoin%
\definecolor{currentfill}{rgb}{0.121569,0.466667,0.705882}%
\pgfsetfillcolor{currentfill}%
\pgfsetlinewidth{0.000000pt}%
\definecolor{currentstroke}{rgb}{0.000000,0.000000,0.000000}%
\pgfsetstrokecolor{currentstroke}%
\pgfsetstrokeopacity{0.000000}%
\pgfsetdash{}{0pt}%
\pgfpathmoveto{\pgfqpoint{4.086585in}{0.866167in}}%
\pgfpathlineto{\pgfqpoint{4.177220in}{0.866167in}}%
\pgfpathlineto{\pgfqpoint{4.177220in}{1.120188in}}%
\pgfpathlineto{\pgfqpoint{4.086585in}{1.120188in}}%
\pgfpathclose%
\pgfusepath{fill}%
\end{pgfscope}%
\begin{pgfscope}%
\pgfpathrectangle{\pgfqpoint{0.462318in}{0.866167in}}{\pgfqpoint{4.960000in}{3.696000in}}%
\pgfusepath{clip}%
\pgfsetbuttcap%
\pgfsetmiterjoin%
\definecolor{currentfill}{rgb}{0.121569,0.466667,0.705882}%
\pgfsetfillcolor{currentfill}%
\pgfsetlinewidth{0.000000pt}%
\definecolor{currentstroke}{rgb}{0.000000,0.000000,0.000000}%
\pgfsetstrokecolor{currentstroke}%
\pgfsetstrokeopacity{0.000000}%
\pgfsetdash{}{0pt}%
\pgfpathmoveto{\pgfqpoint{4.199879in}{0.866167in}}%
\pgfpathlineto{\pgfqpoint{4.290514in}{0.866167in}}%
\pgfpathlineto{\pgfqpoint{4.290514in}{1.120188in}}%
\pgfpathlineto{\pgfqpoint{4.199879in}{1.120188in}}%
\pgfpathclose%
\pgfusepath{fill}%
\end{pgfscope}%
\begin{pgfscope}%
\pgfpathrectangle{\pgfqpoint{0.462318in}{0.866167in}}{\pgfqpoint{4.960000in}{3.696000in}}%
\pgfusepath{clip}%
\pgfsetbuttcap%
\pgfsetmiterjoin%
\definecolor{currentfill}{rgb}{0.121569,0.466667,0.705882}%
\pgfsetfillcolor{currentfill}%
\pgfsetlinewidth{0.000000pt}%
\definecolor{currentstroke}{rgb}{0.000000,0.000000,0.000000}%
\pgfsetstrokecolor{currentstroke}%
\pgfsetstrokeopacity{0.000000}%
\pgfsetdash{}{0pt}%
\pgfpathmoveto{\pgfqpoint{4.313172in}{0.866167in}}%
\pgfpathlineto{\pgfqpoint{4.403807in}{0.866167in}}%
\pgfpathlineto{\pgfqpoint{4.403807in}{1.083899in}}%
\pgfpathlineto{\pgfqpoint{4.313172in}{1.083899in}}%
\pgfpathclose%
\pgfusepath{fill}%
\end{pgfscope}%
\begin{pgfscope}%
\pgfpathrectangle{\pgfqpoint{0.462318in}{0.866167in}}{\pgfqpoint{4.960000in}{3.696000in}}%
\pgfusepath{clip}%
\pgfsetbuttcap%
\pgfsetmiterjoin%
\definecolor{currentfill}{rgb}{0.121569,0.466667,0.705882}%
\pgfsetfillcolor{currentfill}%
\pgfsetlinewidth{0.000000pt}%
\definecolor{currentstroke}{rgb}{0.000000,0.000000,0.000000}%
\pgfsetstrokecolor{currentstroke}%
\pgfsetstrokeopacity{0.000000}%
\pgfsetdash{}{0pt}%
\pgfpathmoveto{\pgfqpoint{4.426466in}{0.866167in}}%
\pgfpathlineto{\pgfqpoint{4.517101in}{0.866167in}}%
\pgfpathlineto{\pgfqpoint{4.517101in}{1.083899in}}%
\pgfpathlineto{\pgfqpoint{4.426466in}{1.083899in}}%
\pgfpathclose%
\pgfusepath{fill}%
\end{pgfscope}%
\begin{pgfscope}%
\pgfpathrectangle{\pgfqpoint{0.462318in}{0.866167in}}{\pgfqpoint{4.960000in}{3.696000in}}%
\pgfusepath{clip}%
\pgfsetbuttcap%
\pgfsetmiterjoin%
\definecolor{currentfill}{rgb}{0.121569,0.466667,0.705882}%
\pgfsetfillcolor{currentfill}%
\pgfsetlinewidth{0.000000pt}%
\definecolor{currentstroke}{rgb}{0.000000,0.000000,0.000000}%
\pgfsetstrokecolor{currentstroke}%
\pgfsetstrokeopacity{0.000000}%
\pgfsetdash{}{0pt}%
\pgfpathmoveto{\pgfqpoint{4.539760in}{0.866167in}}%
\pgfpathlineto{\pgfqpoint{4.630395in}{0.866167in}}%
\pgfpathlineto{\pgfqpoint{4.630395in}{1.083899in}}%
\pgfpathlineto{\pgfqpoint{4.539760in}{1.083899in}}%
\pgfpathclose%
\pgfusepath{fill}%
\end{pgfscope}%
\begin{pgfscope}%
\pgfpathrectangle{\pgfqpoint{0.462318in}{0.866167in}}{\pgfqpoint{4.960000in}{3.696000in}}%
\pgfusepath{clip}%
\pgfsetbuttcap%
\pgfsetmiterjoin%
\definecolor{currentfill}{rgb}{0.121569,0.466667,0.705882}%
\pgfsetfillcolor{currentfill}%
\pgfsetlinewidth{0.000000pt}%
\definecolor{currentstroke}{rgb}{0.000000,0.000000,0.000000}%
\pgfsetstrokecolor{currentstroke}%
\pgfsetstrokeopacity{0.000000}%
\pgfsetdash{}{0pt}%
\pgfpathmoveto{\pgfqpoint{4.653054in}{0.866167in}}%
\pgfpathlineto{\pgfqpoint{4.743689in}{0.866167in}}%
\pgfpathlineto{\pgfqpoint{4.743689in}{1.083899in}}%
\pgfpathlineto{\pgfqpoint{4.653054in}{1.083899in}}%
\pgfpathclose%
\pgfusepath{fill}%
\end{pgfscope}%
\begin{pgfscope}%
\pgfpathrectangle{\pgfqpoint{0.462318in}{0.866167in}}{\pgfqpoint{4.960000in}{3.696000in}}%
\pgfusepath{clip}%
\pgfsetbuttcap%
\pgfsetmiterjoin%
\definecolor{currentfill}{rgb}{0.121569,0.466667,0.705882}%
\pgfsetfillcolor{currentfill}%
\pgfsetlinewidth{0.000000pt}%
\definecolor{currentstroke}{rgb}{0.000000,0.000000,0.000000}%
\pgfsetstrokecolor{currentstroke}%
\pgfsetstrokeopacity{0.000000}%
\pgfsetdash{}{0pt}%
\pgfpathmoveto{\pgfqpoint{4.766347in}{0.866167in}}%
\pgfpathlineto{\pgfqpoint{4.856982in}{0.866167in}}%
\pgfpathlineto{\pgfqpoint{4.856982in}{1.083899in}}%
\pgfpathlineto{\pgfqpoint{4.766347in}{1.083899in}}%
\pgfpathclose%
\pgfusepath{fill}%
\end{pgfscope}%
\begin{pgfscope}%
\pgfpathrectangle{\pgfqpoint{0.462318in}{0.866167in}}{\pgfqpoint{4.960000in}{3.696000in}}%
\pgfusepath{clip}%
\pgfsetbuttcap%
\pgfsetmiterjoin%
\definecolor{currentfill}{rgb}{0.121569,0.466667,0.705882}%
\pgfsetfillcolor{currentfill}%
\pgfsetlinewidth{0.000000pt}%
\definecolor{currentstroke}{rgb}{0.000000,0.000000,0.000000}%
\pgfsetstrokecolor{currentstroke}%
\pgfsetstrokeopacity{0.000000}%
\pgfsetdash{}{0pt}%
\pgfpathmoveto{\pgfqpoint{4.879641in}{0.866167in}}%
\pgfpathlineto{\pgfqpoint{4.970276in}{0.866167in}}%
\pgfpathlineto{\pgfqpoint{4.970276in}{1.083899in}}%
\pgfpathlineto{\pgfqpoint{4.879641in}{1.083899in}}%
\pgfpathclose%
\pgfusepath{fill}%
\end{pgfscope}%
\begin{pgfscope}%
\pgfpathrectangle{\pgfqpoint{0.462318in}{0.866167in}}{\pgfqpoint{4.960000in}{3.696000in}}%
\pgfusepath{clip}%
\pgfsetbuttcap%
\pgfsetmiterjoin%
\definecolor{currentfill}{rgb}{0.121569,0.466667,0.705882}%
\pgfsetfillcolor{currentfill}%
\pgfsetlinewidth{0.000000pt}%
\definecolor{currentstroke}{rgb}{0.000000,0.000000,0.000000}%
\pgfsetstrokecolor{currentstroke}%
\pgfsetstrokeopacity{0.000000}%
\pgfsetdash{}{0pt}%
\pgfpathmoveto{\pgfqpoint{4.992935in}{0.866167in}}%
\pgfpathlineto{\pgfqpoint{5.083570in}{0.866167in}}%
\pgfpathlineto{\pgfqpoint{5.083570in}{1.083899in}}%
\pgfpathlineto{\pgfqpoint{4.992935in}{1.083899in}}%
\pgfpathclose%
\pgfusepath{fill}%
\end{pgfscope}%
\begin{pgfscope}%
\pgfpathrectangle{\pgfqpoint{0.462318in}{0.866167in}}{\pgfqpoint{4.960000in}{3.696000in}}%
\pgfusepath{clip}%
\pgfsetbuttcap%
\pgfsetmiterjoin%
\definecolor{currentfill}{rgb}{0.121569,0.466667,0.705882}%
\pgfsetfillcolor{currentfill}%
\pgfsetlinewidth{0.000000pt}%
\definecolor{currentstroke}{rgb}{0.000000,0.000000,0.000000}%
\pgfsetstrokecolor{currentstroke}%
\pgfsetstrokeopacity{0.000000}%
\pgfsetdash{}{0pt}%
\pgfpathmoveto{\pgfqpoint{5.106229in}{0.866167in}}%
\pgfpathlineto{\pgfqpoint{5.196864in}{0.866167in}}%
\pgfpathlineto{\pgfqpoint{5.196864in}{1.047611in}}%
\pgfpathlineto{\pgfqpoint{5.106229in}{1.047611in}}%
\pgfpathclose%
\pgfusepath{fill}%
\end{pgfscope}%
\begin{pgfscope}%
\pgfsetbuttcap%
\pgfsetroundjoin%
\definecolor{currentfill}{rgb}{0.000000,0.000000,0.000000}%
\pgfsetfillcolor{currentfill}%
\pgfsetlinewidth{0.803000pt}%
\definecolor{currentstroke}{rgb}{0.000000,0.000000,0.000000}%
\pgfsetstrokecolor{currentstroke}%
\pgfsetdash{}{0pt}%
\pgfsys@defobject{currentmarker}{\pgfqpoint{0.000000in}{-0.048611in}}{\pgfqpoint{0.000000in}{0.000000in}}{%
\pgfpathmoveto{\pgfqpoint{0.000000in}{0.000000in}}%
\pgfpathlineto{\pgfqpoint{0.000000in}{-0.048611in}}%
\pgfusepath{stroke,fill}%
}%
\begin{pgfscope}%
\pgfsys@transformshift{0.733090in}{0.866167in}%
\pgfsys@useobject{currentmarker}{}%
\end{pgfscope}%
\end{pgfscope}%
\begin{pgfscope}%
\definecolor{textcolor}{rgb}{0.000000,0.000000,0.000000}%
\pgfsetstrokecolor{textcolor}%
\pgfsetfillcolor{textcolor}%
\pgftext[x=0.771407in, y=0.493066in, left, base,rotate=90.000000]{\color{textcolor}\sffamily\fontsize{10.000000}{12.000000}\selectfont wall}%
\end{pgfscope}%
\begin{pgfscope}%
\pgfsetbuttcap%
\pgfsetroundjoin%
\definecolor{currentfill}{rgb}{0.000000,0.000000,0.000000}%
\pgfsetfillcolor{currentfill}%
\pgfsetlinewidth{0.803000pt}%
\definecolor{currentstroke}{rgb}{0.000000,0.000000,0.000000}%
\pgfsetstrokecolor{currentstroke}%
\pgfsetdash{}{0pt}%
\pgfsys@defobject{currentmarker}{\pgfqpoint{0.000000in}{-0.048611in}}{\pgfqpoint{0.000000in}{0.000000in}}{%
\pgfpathmoveto{\pgfqpoint{0.000000in}{0.000000in}}%
\pgfpathlineto{\pgfqpoint{0.000000in}{-0.048611in}}%
\pgfusepath{stroke,fill}%
}%
\begin{pgfscope}%
\pgfsys@transformshift{0.846384in}{0.866167in}%
\pgfsys@useobject{currentmarker}{}%
\end{pgfscope}%
\end{pgfscope}%
\begin{pgfscope}%
\definecolor{textcolor}{rgb}{0.000000,0.000000,0.000000}%
\pgfsetstrokecolor{textcolor}%
\pgfsetfillcolor{textcolor}%
\pgftext[x=0.884701in, y=0.454411in, left, base,rotate=90.000000]{\color{textcolor}\sffamily\fontsize{10.000000}{12.000000}\selectfont floor}%
\end{pgfscope}%
\begin{pgfscope}%
\pgfsetbuttcap%
\pgfsetroundjoin%
\definecolor{currentfill}{rgb}{0.000000,0.000000,0.000000}%
\pgfsetfillcolor{currentfill}%
\pgfsetlinewidth{0.803000pt}%
\definecolor{currentstroke}{rgb}{0.000000,0.000000,0.000000}%
\pgfsetstrokecolor{currentstroke}%
\pgfsetdash{}{0pt}%
\pgfsys@defobject{currentmarker}{\pgfqpoint{0.000000in}{-0.048611in}}{\pgfqpoint{0.000000in}{0.000000in}}{%
\pgfpathmoveto{\pgfqpoint{0.000000in}{0.000000in}}%
\pgfpathlineto{\pgfqpoint{0.000000in}{-0.048611in}}%
\pgfusepath{stroke,fill}%
}%
\begin{pgfscope}%
\pgfsys@transformshift{0.959678in}{0.866167in}%
\pgfsys@useobject{currentmarker}{}%
\end{pgfscope}%
\end{pgfscope}%
\begin{pgfscope}%
\definecolor{textcolor}{rgb}{0.000000,0.000000,0.000000}%
\pgfsetstrokecolor{textcolor}%
\pgfsetfillcolor{textcolor}%
\pgftext[x=0.997994in, y=0.417179in, left, base,rotate=90.000000]{\color{textcolor}\sffamily\fontsize{10.000000}{12.000000}\selectfont table}%
\end{pgfscope}%
\begin{pgfscope}%
\pgfsetbuttcap%
\pgfsetroundjoin%
\definecolor{currentfill}{rgb}{0.000000,0.000000,0.000000}%
\pgfsetfillcolor{currentfill}%
\pgfsetlinewidth{0.803000pt}%
\definecolor{currentstroke}{rgb}{0.000000,0.000000,0.000000}%
\pgfsetstrokecolor{currentstroke}%
\pgfsetdash{}{0pt}%
\pgfsys@defobject{currentmarker}{\pgfqpoint{0.000000in}{-0.048611in}}{\pgfqpoint{0.000000in}{0.000000in}}{%
\pgfpathmoveto{\pgfqpoint{0.000000in}{0.000000in}}%
\pgfpathlineto{\pgfqpoint{0.000000in}{-0.048611in}}%
\pgfusepath{stroke,fill}%
}%
\begin{pgfscope}%
\pgfsys@transformshift{1.072971in}{0.866167in}%
\pgfsys@useobject{currentmarker}{}%
\end{pgfscope}%
\end{pgfscope}%
\begin{pgfscope}%
\definecolor{textcolor}{rgb}{0.000000,0.000000,0.000000}%
\pgfsetstrokecolor{textcolor}%
\pgfsetfillcolor{textcolor}%
\pgftext[x=1.111288in, y=0.442543in, left, base,rotate=90.000000]{\color{textcolor}\sffamily\fontsize{10.000000}{12.000000}\selectfont desk}%
\end{pgfscope}%
\begin{pgfscope}%
\pgfsetbuttcap%
\pgfsetroundjoin%
\definecolor{currentfill}{rgb}{0.000000,0.000000,0.000000}%
\pgfsetfillcolor{currentfill}%
\pgfsetlinewidth{0.803000pt}%
\definecolor{currentstroke}{rgb}{0.000000,0.000000,0.000000}%
\pgfsetstrokecolor{currentstroke}%
\pgfsetdash{}{0pt}%
\pgfsys@defobject{currentmarker}{\pgfqpoint{0.000000in}{-0.048611in}}{\pgfqpoint{0.000000in}{0.000000in}}{%
\pgfpathmoveto{\pgfqpoint{0.000000in}{0.000000in}}%
\pgfpathlineto{\pgfqpoint{0.000000in}{-0.048611in}}%
\pgfusepath{stroke,fill}%
}%
\begin{pgfscope}%
\pgfsys@transformshift{1.186265in}{0.866167in}%
\pgfsys@useobject{currentmarker}{}%
\end{pgfscope}%
\end{pgfscope}%
\begin{pgfscope}%
\definecolor{textcolor}{rgb}{0.000000,0.000000,0.000000}%
\pgfsetstrokecolor{textcolor}%
\pgfsetfillcolor{textcolor}%
\pgftext[x=1.224582in, y=0.111122in, left, base,rotate=90.000000]{\color{textcolor}\sffamily\fontsize{10.000000}{12.000000}\selectfont bookcase}%
\end{pgfscope}%
\begin{pgfscope}%
\pgfsetbuttcap%
\pgfsetroundjoin%
\definecolor{currentfill}{rgb}{0.000000,0.000000,0.000000}%
\pgfsetfillcolor{currentfill}%
\pgfsetlinewidth{0.803000pt}%
\definecolor{currentstroke}{rgb}{0.000000,0.000000,0.000000}%
\pgfsetstrokecolor{currentstroke}%
\pgfsetdash{}{0pt}%
\pgfsys@defobject{currentmarker}{\pgfqpoint{0.000000in}{-0.048611in}}{\pgfqpoint{0.000000in}{0.000000in}}{%
\pgfpathmoveto{\pgfqpoint{0.000000in}{0.000000in}}%
\pgfpathlineto{\pgfqpoint{0.000000in}{-0.048611in}}%
\pgfusepath{stroke,fill}%
}%
\begin{pgfscope}%
\pgfsys@transformshift{1.299559in}{0.866167in}%
\pgfsys@useobject{currentmarker}{}%
\end{pgfscope}%
\end{pgfscope}%
\begin{pgfscope}%
\definecolor{textcolor}{rgb}{0.000000,0.000000,0.000000}%
\pgfsetstrokecolor{textcolor}%
\pgfsetfillcolor{textcolor}%
\pgftext[x=1.337876in, y=0.114784in, left, base,rotate=90.000000]{\color{textcolor}\sffamily\fontsize{10.000000}{12.000000}\selectfont wardrobe}%
\end{pgfscope}%
\begin{pgfscope}%
\pgfsetbuttcap%
\pgfsetroundjoin%
\definecolor{currentfill}{rgb}{0.000000,0.000000,0.000000}%
\pgfsetfillcolor{currentfill}%
\pgfsetlinewidth{0.803000pt}%
\definecolor{currentstroke}{rgb}{0.000000,0.000000,0.000000}%
\pgfsetstrokecolor{currentstroke}%
\pgfsetdash{}{0pt}%
\pgfsys@defobject{currentmarker}{\pgfqpoint{0.000000in}{-0.048611in}}{\pgfqpoint{0.000000in}{0.000000in}}{%
\pgfpathmoveto{\pgfqpoint{0.000000in}{0.000000in}}%
\pgfpathlineto{\pgfqpoint{0.000000in}{-0.048611in}}%
\pgfusepath{stroke,fill}%
}%
\begin{pgfscope}%
\pgfsys@transformshift{1.412853in}{0.866167in}%
\pgfsys@useobject{currentmarker}{}%
\end{pgfscope}%
\end{pgfscope}%
\begin{pgfscope}%
\definecolor{textcolor}{rgb}{0.000000,0.000000,0.000000}%
\pgfsetstrokecolor{textcolor}%
\pgfsetfillcolor{textcolor}%
\pgftext[x=1.451169in, y=0.199826in, left, base,rotate=90.000000]{\color{textcolor}\sffamily\fontsize{10.000000}{12.000000}\selectfont painting}%
\end{pgfscope}%
\begin{pgfscope}%
\pgfsetbuttcap%
\pgfsetroundjoin%
\definecolor{currentfill}{rgb}{0.000000,0.000000,0.000000}%
\pgfsetfillcolor{currentfill}%
\pgfsetlinewidth{0.803000pt}%
\definecolor{currentstroke}{rgb}{0.000000,0.000000,0.000000}%
\pgfsetstrokecolor{currentstroke}%
\pgfsetdash{}{0pt}%
\pgfsys@defobject{currentmarker}{\pgfqpoint{0.000000in}{-0.048611in}}{\pgfqpoint{0.000000in}{0.000000in}}{%
\pgfpathmoveto{\pgfqpoint{0.000000in}{0.000000in}}%
\pgfpathlineto{\pgfqpoint{0.000000in}{-0.048611in}}%
\pgfusepath{stroke,fill}%
}%
\begin{pgfscope}%
\pgfsys@transformshift{1.526146in}{0.866167in}%
\pgfsys@useobject{currentmarker}{}%
\end{pgfscope}%
\end{pgfscope}%
\begin{pgfscope}%
\definecolor{textcolor}{rgb}{0.000000,0.000000,0.000000}%
\pgfsetstrokecolor{textcolor}%
\pgfsetfillcolor{textcolor}%
\pgftext[x=1.564463in, y=0.322304in, left, base,rotate=90.000000]{\color{textcolor}\sffamily\fontsize{10.000000}{12.000000}\selectfont carpet}%
\end{pgfscope}%
\begin{pgfscope}%
\pgfsetbuttcap%
\pgfsetroundjoin%
\definecolor{currentfill}{rgb}{0.000000,0.000000,0.000000}%
\pgfsetfillcolor{currentfill}%
\pgfsetlinewidth{0.803000pt}%
\definecolor{currentstroke}{rgb}{0.000000,0.000000,0.000000}%
\pgfsetstrokecolor{currentstroke}%
\pgfsetdash{}{0pt}%
\pgfsys@defobject{currentmarker}{\pgfqpoint{0.000000in}{-0.048611in}}{\pgfqpoint{0.000000in}{0.000000in}}{%
\pgfpathmoveto{\pgfqpoint{0.000000in}{0.000000in}}%
\pgfpathlineto{\pgfqpoint{0.000000in}{-0.048611in}}%
\pgfusepath{stroke,fill}%
}%
\begin{pgfscope}%
\pgfsys@transformshift{1.639440in}{0.866167in}%
\pgfsys@useobject{currentmarker}{}%
\end{pgfscope}%
\end{pgfscope}%
\begin{pgfscope}%
\definecolor{textcolor}{rgb}{0.000000,0.000000,0.000000}%
\pgfsetstrokecolor{textcolor}%
\pgfsetfillcolor{textcolor}%
\pgftext[x=1.677757in, y=0.423757in, left, base,rotate=90.000000]{\color{textcolor}\sffamily\fontsize{10.000000}{12.000000}\selectfont chair}%
\end{pgfscope}%
\begin{pgfscope}%
\pgfsetbuttcap%
\pgfsetroundjoin%
\definecolor{currentfill}{rgb}{0.000000,0.000000,0.000000}%
\pgfsetfillcolor{currentfill}%
\pgfsetlinewidth{0.803000pt}%
\definecolor{currentstroke}{rgb}{0.000000,0.000000,0.000000}%
\pgfsetstrokecolor{currentstroke}%
\pgfsetdash{}{0pt}%
\pgfsys@defobject{currentmarker}{\pgfqpoint{0.000000in}{-0.048611in}}{\pgfqpoint{0.000000in}{0.000000in}}{%
\pgfpathmoveto{\pgfqpoint{0.000000in}{0.000000in}}%
\pgfpathlineto{\pgfqpoint{0.000000in}{-0.048611in}}%
\pgfusepath{stroke,fill}%
}%
\begin{pgfscope}%
\pgfsys@transformshift{1.752734in}{0.866167in}%
\pgfsys@useobject{currentmarker}{}%
\end{pgfscope}%
\end{pgfscope}%
\begin{pgfscope}%
\definecolor{textcolor}{rgb}{0.000000,0.000000,0.000000}%
\pgfsetstrokecolor{textcolor}%
\pgfsetfillcolor{textcolor}%
\pgftext[x=1.791051in, y=0.632294in, left, base,rotate=90.000000]{\color{textcolor}\sffamily\fontsize{10.000000}{12.000000}\selectfont tv}%
\end{pgfscope}%
\begin{pgfscope}%
\pgfsetbuttcap%
\pgfsetroundjoin%
\definecolor{currentfill}{rgb}{0.000000,0.000000,0.000000}%
\pgfsetfillcolor{currentfill}%
\pgfsetlinewidth{0.803000pt}%
\definecolor{currentstroke}{rgb}{0.000000,0.000000,0.000000}%
\pgfsetstrokecolor{currentstroke}%
\pgfsetdash{}{0pt}%
\pgfsys@defobject{currentmarker}{\pgfqpoint{0.000000in}{-0.048611in}}{\pgfqpoint{0.000000in}{0.000000in}}{%
\pgfpathmoveto{\pgfqpoint{0.000000in}{0.000000in}}%
\pgfpathlineto{\pgfqpoint{0.000000in}{-0.048611in}}%
\pgfusepath{stroke,fill}%
}%
\begin{pgfscope}%
\pgfsys@transformshift{1.866028in}{0.866167in}%
\pgfsys@useobject{currentmarker}{}%
\end{pgfscope}%
\end{pgfscope}%
\begin{pgfscope}%
\definecolor{textcolor}{rgb}{0.000000,0.000000,0.000000}%
\pgfsetstrokecolor{textcolor}%
\pgfsetfillcolor{textcolor}%
\pgftext[x=1.904344in, y=0.168766in, left, base,rotate=90.000000]{\color{textcolor}\sffamily\fontsize{10.000000}{12.000000}\selectfont furniture}%
\end{pgfscope}%
\begin{pgfscope}%
\pgfsetbuttcap%
\pgfsetroundjoin%
\definecolor{currentfill}{rgb}{0.000000,0.000000,0.000000}%
\pgfsetfillcolor{currentfill}%
\pgfsetlinewidth{0.803000pt}%
\definecolor{currentstroke}{rgb}{0.000000,0.000000,0.000000}%
\pgfsetstrokecolor{currentstroke}%
\pgfsetdash{}{0pt}%
\pgfsys@defobject{currentmarker}{\pgfqpoint{0.000000in}{-0.048611in}}{\pgfqpoint{0.000000in}{0.000000in}}{%
\pgfpathmoveto{\pgfqpoint{0.000000in}{0.000000in}}%
\pgfpathlineto{\pgfqpoint{0.000000in}{-0.048611in}}%
\pgfusepath{stroke,fill}%
}%
\begin{pgfscope}%
\pgfsys@transformshift{1.979321in}{0.866167in}%
\pgfsys@useobject{currentmarker}{}%
\end{pgfscope}%
\end{pgfscope}%
\begin{pgfscope}%
\definecolor{textcolor}{rgb}{0.000000,0.000000,0.000000}%
\pgfsetstrokecolor{textcolor}%
\pgfsetfillcolor{textcolor}%
\pgftext[x=2.017638in, y=0.100271in, left, base,rotate=90.000000]{\color{textcolor}\sffamily\fontsize{10.000000}{12.000000}\selectfont television}%
\end{pgfscope}%
\begin{pgfscope}%
\pgfsetbuttcap%
\pgfsetroundjoin%
\definecolor{currentfill}{rgb}{0.000000,0.000000,0.000000}%
\pgfsetfillcolor{currentfill}%
\pgfsetlinewidth{0.803000pt}%
\definecolor{currentstroke}{rgb}{0.000000,0.000000,0.000000}%
\pgfsetstrokecolor{currentstroke}%
\pgfsetdash{}{0pt}%
\pgfsys@defobject{currentmarker}{\pgfqpoint{0.000000in}{-0.048611in}}{\pgfqpoint{0.000000in}{0.000000in}}{%
\pgfpathmoveto{\pgfqpoint{0.000000in}{0.000000in}}%
\pgfpathlineto{\pgfqpoint{0.000000in}{-0.048611in}}%
\pgfusepath{stroke,fill}%
}%
\begin{pgfscope}%
\pgfsys@transformshift{2.092615in}{0.866167in}%
\pgfsys@useobject{currentmarker}{}%
\end{pgfscope}%
\end{pgfscope}%
\begin{pgfscope}%
\definecolor{textcolor}{rgb}{0.000000,0.000000,0.000000}%
\pgfsetstrokecolor{textcolor}%
\pgfsetfillcolor{textcolor}%
\pgftext[x=2.130932in, y=0.507172in, left, base,rotate=90.000000]{\color{textcolor}\sffamily\fontsize{10.000000}{12.000000}\selectfont bed}%
\end{pgfscope}%
\begin{pgfscope}%
\pgfsetbuttcap%
\pgfsetroundjoin%
\definecolor{currentfill}{rgb}{0.000000,0.000000,0.000000}%
\pgfsetfillcolor{currentfill}%
\pgfsetlinewidth{0.803000pt}%
\definecolor{currentstroke}{rgb}{0.000000,0.000000,0.000000}%
\pgfsetstrokecolor{currentstroke}%
\pgfsetdash{}{0pt}%
\pgfsys@defobject{currentmarker}{\pgfqpoint{0.000000in}{-0.048611in}}{\pgfqpoint{0.000000in}{0.000000in}}{%
\pgfpathmoveto{\pgfqpoint{0.000000in}{0.000000in}}%
\pgfpathlineto{\pgfqpoint{0.000000in}{-0.048611in}}%
\pgfusepath{stroke,fill}%
}%
\begin{pgfscope}%
\pgfsys@transformshift{2.205909in}{0.866167in}%
\pgfsys@useobject{currentmarker}{}%
\end{pgfscope}%
\end{pgfscope}%
\begin{pgfscope}%
\definecolor{textcolor}{rgb}{0.000000,0.000000,0.000000}%
\pgfsetstrokecolor{textcolor}%
\pgfsetfillcolor{textcolor}%
\pgftext[x=2.244226in, y=0.366452in, left, base,rotate=90.000000]{\color{textcolor}\sffamily\fontsize{10.000000}{12.000000}\selectfont pillow}%
\end{pgfscope}%
\begin{pgfscope}%
\pgfsetbuttcap%
\pgfsetroundjoin%
\definecolor{currentfill}{rgb}{0.000000,0.000000,0.000000}%
\pgfsetfillcolor{currentfill}%
\pgfsetlinewidth{0.803000pt}%
\definecolor{currentstroke}{rgb}{0.000000,0.000000,0.000000}%
\pgfsetstrokecolor{currentstroke}%
\pgfsetdash{}{0pt}%
\pgfsys@defobject{currentmarker}{\pgfqpoint{0.000000in}{-0.048611in}}{\pgfqpoint{0.000000in}{0.000000in}}{%
\pgfpathmoveto{\pgfqpoint{0.000000in}{0.000000in}}%
\pgfpathlineto{\pgfqpoint{0.000000in}{-0.048611in}}%
\pgfusepath{stroke,fill}%
}%
\begin{pgfscope}%
\pgfsys@transformshift{2.319203in}{0.866167in}%
\pgfsys@useobject{currentmarker}{}%
\end{pgfscope}%
\end{pgfscope}%
\begin{pgfscope}%
\definecolor{textcolor}{rgb}{0.000000,0.000000,0.000000}%
\pgfsetstrokecolor{textcolor}%
\pgfsetfillcolor{textcolor}%
\pgftext[x=2.357519in, y=0.433998in, left, base,rotate=90.000000]{\color{textcolor}\sffamily\fontsize{10.000000}{12.000000}\selectfont deco}%
\end{pgfscope}%
\begin{pgfscope}%
\pgfsetbuttcap%
\pgfsetroundjoin%
\definecolor{currentfill}{rgb}{0.000000,0.000000,0.000000}%
\pgfsetfillcolor{currentfill}%
\pgfsetlinewidth{0.803000pt}%
\definecolor{currentstroke}{rgb}{0.000000,0.000000,0.000000}%
\pgfsetstrokecolor{currentstroke}%
\pgfsetdash{}{0pt}%
\pgfsys@defobject{currentmarker}{\pgfqpoint{0.000000in}{-0.048611in}}{\pgfqpoint{0.000000in}{0.000000in}}{%
\pgfpathmoveto{\pgfqpoint{0.000000in}{0.000000in}}%
\pgfpathlineto{\pgfqpoint{0.000000in}{-0.048611in}}%
\pgfusepath{stroke,fill}%
}%
\begin{pgfscope}%
\pgfsys@transformshift{2.432496in}{0.866167in}%
\pgfsys@useobject{currentmarker}{}%
\end{pgfscope}%
\end{pgfscope}%
\begin{pgfscope}%
\definecolor{textcolor}{rgb}{0.000000,0.000000,0.000000}%
\pgfsetstrokecolor{textcolor}%
\pgfsetfillcolor{textcolor}%
\pgftext[x=2.470813in, y=0.137842in, left, base,rotate=90.000000]{\color{textcolor}\sffamily\fontsize{10.000000}{12.000000}\selectfont unknown}%
\end{pgfscope}%
\begin{pgfscope}%
\pgfsetbuttcap%
\pgfsetroundjoin%
\definecolor{currentfill}{rgb}{0.000000,0.000000,0.000000}%
\pgfsetfillcolor{currentfill}%
\pgfsetlinewidth{0.803000pt}%
\definecolor{currentstroke}{rgb}{0.000000,0.000000,0.000000}%
\pgfsetstrokecolor{currentstroke}%
\pgfsetdash{}{0pt}%
\pgfsys@defobject{currentmarker}{\pgfqpoint{0.000000in}{-0.048611in}}{\pgfqpoint{0.000000in}{0.000000in}}{%
\pgfpathmoveto{\pgfqpoint{0.000000in}{0.000000in}}%
\pgfpathlineto{\pgfqpoint{0.000000in}{-0.048611in}}%
\pgfusepath{stroke,fill}%
}%
\begin{pgfscope}%
\pgfsys@transformshift{2.545790in}{0.866167in}%
\pgfsys@useobject{currentmarker}{}%
\end{pgfscope}%
\end{pgfscope}%
\begin{pgfscope}%
\definecolor{textcolor}{rgb}{0.000000,0.000000,0.000000}%
\pgfsetstrokecolor{textcolor}%
\pgfsetfillcolor{textcolor}%
\pgftext[x=2.584107in, y=0.426538in, left, base,rotate=90.000000]{\color{textcolor}\sffamily\fontsize{10.000000}{12.000000}\selectfont cloth}%
\end{pgfscope}%
\begin{pgfscope}%
\pgfsetbuttcap%
\pgfsetroundjoin%
\definecolor{currentfill}{rgb}{0.000000,0.000000,0.000000}%
\pgfsetfillcolor{currentfill}%
\pgfsetlinewidth{0.803000pt}%
\definecolor{currentstroke}{rgb}{0.000000,0.000000,0.000000}%
\pgfsetstrokecolor{currentstroke}%
\pgfsetdash{}{0pt}%
\pgfsys@defobject{currentmarker}{\pgfqpoint{0.000000in}{-0.048611in}}{\pgfqpoint{0.000000in}{0.000000in}}{%
\pgfpathmoveto{\pgfqpoint{0.000000in}{0.000000in}}%
\pgfpathlineto{\pgfqpoint{0.000000in}{-0.048611in}}%
\pgfusepath{stroke,fill}%
}%
\begin{pgfscope}%
\pgfsys@transformshift{2.659084in}{0.866167in}%
\pgfsys@useobject{currentmarker}{}%
\end{pgfscope}%
\end{pgfscope}%
\begin{pgfscope}%
\definecolor{textcolor}{rgb}{0.000000,0.000000,0.000000}%
\pgfsetstrokecolor{textcolor}%
\pgfsetfillcolor{textcolor}%
\pgftext[x=2.697401in, y=0.477604in, left, base,rotate=90.000000]{\color{textcolor}\sffamily\fontsize{10.000000}{12.000000}\selectfont sofa}%
\end{pgfscope}%
\begin{pgfscope}%
\pgfsetbuttcap%
\pgfsetroundjoin%
\definecolor{currentfill}{rgb}{0.000000,0.000000,0.000000}%
\pgfsetfillcolor{currentfill}%
\pgfsetlinewidth{0.803000pt}%
\definecolor{currentstroke}{rgb}{0.000000,0.000000,0.000000}%
\pgfsetstrokecolor{currentstroke}%
\pgfsetdash{}{0pt}%
\pgfsys@defobject{currentmarker}{\pgfqpoint{0.000000in}{-0.048611in}}{\pgfqpoint{0.000000in}{0.000000in}}{%
\pgfpathmoveto{\pgfqpoint{0.000000in}{0.000000in}}%
\pgfpathlineto{\pgfqpoint{0.000000in}{-0.048611in}}%
\pgfusepath{stroke,fill}%
}%
\begin{pgfscope}%
\pgfsys@transformshift{2.772378in}{0.866167in}%
\pgfsys@useobject{currentmarker}{}%
\end{pgfscope}%
\end{pgfscope}%
\begin{pgfscope}%
\definecolor{textcolor}{rgb}{0.000000,0.000000,0.000000}%
\pgfsetstrokecolor{textcolor}%
\pgfsetfillcolor{textcolor}%
\pgftext[x=2.810694in, y=0.443831in, left, base,rotate=90.000000]{\color{textcolor}\sffamily\fontsize{10.000000}{12.000000}\selectfont vase}%
\end{pgfscope}%
\begin{pgfscope}%
\pgfsetbuttcap%
\pgfsetroundjoin%
\definecolor{currentfill}{rgb}{0.000000,0.000000,0.000000}%
\pgfsetfillcolor{currentfill}%
\pgfsetlinewidth{0.803000pt}%
\definecolor{currentstroke}{rgb}{0.000000,0.000000,0.000000}%
\pgfsetstrokecolor{currentstroke}%
\pgfsetdash{}{0pt}%
\pgfsys@defobject{currentmarker}{\pgfqpoint{0.000000in}{-0.048611in}}{\pgfqpoint{0.000000in}{0.000000in}}{%
\pgfpathmoveto{\pgfqpoint{0.000000in}{0.000000in}}%
\pgfpathlineto{\pgfqpoint{0.000000in}{-0.048611in}}%
\pgfusepath{stroke,fill}%
}%
\begin{pgfscope}%
\pgfsys@transformshift{2.885671in}{0.866167in}%
\pgfsys@useobject{currentmarker}{}%
\end{pgfscope}%
\end{pgfscope}%
\begin{pgfscope}%
\definecolor{textcolor}{rgb}{0.000000,0.000000,0.000000}%
\pgfsetstrokecolor{textcolor}%
\pgfsetfillcolor{textcolor}%
\pgftext[x=2.923988in, y=0.430404in, left, base,rotate=90.000000]{\color{textcolor}\sffamily\fontsize{10.000000}{12.000000}\selectfont book}%
\end{pgfscope}%
\begin{pgfscope}%
\pgfsetbuttcap%
\pgfsetroundjoin%
\definecolor{currentfill}{rgb}{0.000000,0.000000,0.000000}%
\pgfsetfillcolor{currentfill}%
\pgfsetlinewidth{0.803000pt}%
\definecolor{currentstroke}{rgb}{0.000000,0.000000,0.000000}%
\pgfsetstrokecolor{currentstroke}%
\pgfsetdash{}{0pt}%
\pgfsys@defobject{currentmarker}{\pgfqpoint{0.000000in}{-0.048611in}}{\pgfqpoint{0.000000in}{0.000000in}}{%
\pgfpathmoveto{\pgfqpoint{0.000000in}{0.000000in}}%
\pgfpathlineto{\pgfqpoint{0.000000in}{-0.048611in}}%
\pgfusepath{stroke,fill}%
}%
\begin{pgfscope}%
\pgfsys@transformshift{2.998965in}{0.866167in}%
\pgfsys@useobject{currentmarker}{}%
\end{pgfscope}%
\end{pgfscope}%
\begin{pgfscope}%
\definecolor{textcolor}{rgb}{0.000000,0.000000,0.000000}%
\pgfsetstrokecolor{textcolor}%
\pgfsetfillcolor{textcolor}%
\pgftext[x=3.037282in, y=0.281274in, left, base,rotate=90.000000]{\color{textcolor}\sffamily\fontsize{10.000000}{12.000000}\selectfont curtain}%
\end{pgfscope}%
\begin{pgfscope}%
\pgfsetbuttcap%
\pgfsetroundjoin%
\definecolor{currentfill}{rgb}{0.000000,0.000000,0.000000}%
\pgfsetfillcolor{currentfill}%
\pgfsetlinewidth{0.803000pt}%
\definecolor{currentstroke}{rgb}{0.000000,0.000000,0.000000}%
\pgfsetstrokecolor{currentstroke}%
\pgfsetdash{}{0pt}%
\pgfsys@defobject{currentmarker}{\pgfqpoint{0.000000in}{-0.048611in}}{\pgfqpoint{0.000000in}{0.000000in}}{%
\pgfpathmoveto{\pgfqpoint{0.000000in}{0.000000in}}%
\pgfpathlineto{\pgfqpoint{0.000000in}{-0.048611in}}%
\pgfusepath{stroke,fill}%
}%
\begin{pgfscope}%
\pgfsys@transformshift{3.112259in}{0.866167in}%
\pgfsys@useobject{currentmarker}{}%
\end{pgfscope}%
\end{pgfscope}%
\begin{pgfscope}%
\definecolor{textcolor}{rgb}{0.000000,0.000000,0.000000}%
\pgfsetstrokecolor{textcolor}%
\pgfsetfillcolor{textcolor}%
\pgftext[x=3.150575in, y=0.315318in, left, base,rotate=90.000000]{\color{textcolor}\sffamily\fontsize{10.000000}{12.000000}\selectfont plastic}%
\end{pgfscope}%
\begin{pgfscope}%
\pgfsetbuttcap%
\pgfsetroundjoin%
\definecolor{currentfill}{rgb}{0.000000,0.000000,0.000000}%
\pgfsetfillcolor{currentfill}%
\pgfsetlinewidth{0.803000pt}%
\definecolor{currentstroke}{rgb}{0.000000,0.000000,0.000000}%
\pgfsetstrokecolor{currentstroke}%
\pgfsetdash{}{0pt}%
\pgfsys@defobject{currentmarker}{\pgfqpoint{0.000000in}{-0.048611in}}{\pgfqpoint{0.000000in}{0.000000in}}{%
\pgfpathmoveto{\pgfqpoint{0.000000in}{0.000000in}}%
\pgfpathlineto{\pgfqpoint{0.000000in}{-0.048611in}}%
\pgfusepath{stroke,fill}%
}%
\begin{pgfscope}%
\pgfsys@transformshift{3.225553in}{0.866167in}%
\pgfsys@useobject{currentmarker}{}%
\end{pgfscope}%
\end{pgfscope}%
\begin{pgfscope}%
\definecolor{textcolor}{rgb}{0.000000,0.000000,0.000000}%
\pgfsetstrokecolor{textcolor}%
\pgfsetfillcolor{textcolor}%
\pgftext[x=3.263869in, y=0.370657in, left, base,rotate=90.000000]{\color{textcolor}\sffamily\fontsize{10.000000}{12.000000}\selectfont duvet}%
\end{pgfscope}%
\begin{pgfscope}%
\pgfsetbuttcap%
\pgfsetroundjoin%
\definecolor{currentfill}{rgb}{0.000000,0.000000,0.000000}%
\pgfsetfillcolor{currentfill}%
\pgfsetlinewidth{0.803000pt}%
\definecolor{currentstroke}{rgb}{0.000000,0.000000,0.000000}%
\pgfsetstrokecolor{currentstroke}%
\pgfsetdash{}{0pt}%
\pgfsys@defobject{currentmarker}{\pgfqpoint{0.000000in}{-0.048611in}}{\pgfqpoint{0.000000in}{0.000000in}}{%
\pgfpathmoveto{\pgfqpoint{0.000000in}{0.000000in}}%
\pgfpathlineto{\pgfqpoint{0.000000in}{-0.048611in}}%
\pgfusepath{stroke,fill}%
}%
\begin{pgfscope}%
\pgfsys@transformshift{3.338846in}{0.866167in}%
\pgfsys@useobject{currentmarker}{}%
\end{pgfscope}%
\end{pgfscope}%
\begin{pgfscope}%
\definecolor{textcolor}{rgb}{0.000000,0.000000,0.000000}%
\pgfsetstrokecolor{textcolor}%
\pgfsetfillcolor{textcolor}%
\pgftext[x=3.377163in, y=0.435626in, left, base,rotate=90.000000]{\color{textcolor}\sffamily\fontsize{10.000000}{12.000000}\selectfont shelf}%
\end{pgfscope}%
\begin{pgfscope}%
\pgfsetbuttcap%
\pgfsetroundjoin%
\definecolor{currentfill}{rgb}{0.000000,0.000000,0.000000}%
\pgfsetfillcolor{currentfill}%
\pgfsetlinewidth{0.803000pt}%
\definecolor{currentstroke}{rgb}{0.000000,0.000000,0.000000}%
\pgfsetstrokecolor{currentstroke}%
\pgfsetdash{}{0pt}%
\pgfsys@defobject{currentmarker}{\pgfqpoint{0.000000in}{-0.048611in}}{\pgfqpoint{0.000000in}{0.000000in}}{%
\pgfpathmoveto{\pgfqpoint{0.000000in}{0.000000in}}%
\pgfpathlineto{\pgfqpoint{0.000000in}{-0.048611in}}%
\pgfusepath{stroke,fill}%
}%
\begin{pgfscope}%
\pgfsys@transformshift{3.452140in}{0.866167in}%
\pgfsys@useobject{currentmarker}{}%
\end{pgfscope}%
\end{pgfscope}%
\begin{pgfscope}%
\definecolor{textcolor}{rgb}{0.000000,0.000000,0.000000}%
\pgfsetstrokecolor{textcolor}%
\pgfsetfillcolor{textcolor}%
\pgftext[x=3.490457in, y=0.214678in, left, base,rotate=90.000000]{\color{textcolor}\sffamily\fontsize{10.000000}{12.000000}\selectfont ceramic}%
\end{pgfscope}%
\begin{pgfscope}%
\pgfsetbuttcap%
\pgfsetroundjoin%
\definecolor{currentfill}{rgb}{0.000000,0.000000,0.000000}%
\pgfsetfillcolor{currentfill}%
\pgfsetlinewidth{0.803000pt}%
\definecolor{currentstroke}{rgb}{0.000000,0.000000,0.000000}%
\pgfsetstrokecolor{currentstroke}%
\pgfsetdash{}{0pt}%
\pgfsys@defobject{currentmarker}{\pgfqpoint{0.000000in}{-0.048611in}}{\pgfqpoint{0.000000in}{0.000000in}}{%
\pgfpathmoveto{\pgfqpoint{0.000000in}{0.000000in}}%
\pgfpathlineto{\pgfqpoint{0.000000in}{-0.048611in}}%
\pgfusepath{stroke,fill}%
}%
\begin{pgfscope}%
\pgfsys@transformshift{3.565434in}{0.866167in}%
\pgfsys@useobject{currentmarker}{}%
\end{pgfscope}%
\end{pgfscope}%
\begin{pgfscope}%
\definecolor{textcolor}{rgb}{0.000000,0.000000,0.000000}%
\pgfsetstrokecolor{textcolor}%
\pgfsetfillcolor{textcolor}%
\pgftext[x=3.603750in, y=0.421791in, left, base,rotate=90.000000]{\color{textcolor}\sffamily\fontsize{10.000000}{12.000000}\selectfont lamp}%
\end{pgfscope}%
\begin{pgfscope}%
\pgfsetbuttcap%
\pgfsetroundjoin%
\definecolor{currentfill}{rgb}{0.000000,0.000000,0.000000}%
\pgfsetfillcolor{currentfill}%
\pgfsetlinewidth{0.803000pt}%
\definecolor{currentstroke}{rgb}{0.000000,0.000000,0.000000}%
\pgfsetstrokecolor{currentstroke}%
\pgfsetdash{}{0pt}%
\pgfsys@defobject{currentmarker}{\pgfqpoint{0.000000in}{-0.048611in}}{\pgfqpoint{0.000000in}{0.000000in}}{%
\pgfpathmoveto{\pgfqpoint{0.000000in}{0.000000in}}%
\pgfpathlineto{\pgfqpoint{0.000000in}{-0.048611in}}%
\pgfusepath{stroke,fill}%
}%
\begin{pgfscope}%
\pgfsys@transformshift{3.678728in}{0.866167in}%
\pgfsys@useobject{currentmarker}{}%
\end{pgfscope}%
\end{pgfscope}%
\begin{pgfscope}%
\definecolor{textcolor}{rgb}{0.000000,0.000000,0.000000}%
\pgfsetstrokecolor{textcolor}%
\pgfsetfillcolor{textcolor}%
\pgftext[x=3.717044in, y=0.414602in, left, base,rotate=90.000000]{\color{textcolor}\sffamily\fontsize{10.000000}{12.000000}\selectfont plant}%
\end{pgfscope}%
\begin{pgfscope}%
\pgfsetbuttcap%
\pgfsetroundjoin%
\definecolor{currentfill}{rgb}{0.000000,0.000000,0.000000}%
\pgfsetfillcolor{currentfill}%
\pgfsetlinewidth{0.803000pt}%
\definecolor{currentstroke}{rgb}{0.000000,0.000000,0.000000}%
\pgfsetstrokecolor{currentstroke}%
\pgfsetdash{}{0pt}%
\pgfsys@defobject{currentmarker}{\pgfqpoint{0.000000in}{-0.048611in}}{\pgfqpoint{0.000000in}{0.000000in}}{%
\pgfpathmoveto{\pgfqpoint{0.000000in}{0.000000in}}%
\pgfpathlineto{\pgfqpoint{0.000000in}{-0.048611in}}%
\pgfusepath{stroke,fill}%
}%
\begin{pgfscope}%
\pgfsys@transformshift{3.792021in}{0.866167in}%
\pgfsys@useobject{currentmarker}{}%
\end{pgfscope}%
\end{pgfscope}%
\begin{pgfscope}%
\definecolor{textcolor}{rgb}{0.000000,0.000000,0.000000}%
\pgfsetstrokecolor{textcolor}%
\pgfsetfillcolor{textcolor}%
\pgftext[x=3.830338in, y=0.242008in, left, base,rotate=90.000000]{\color{textcolor}\sffamily\fontsize{10.000000}{12.000000}\selectfont window}%
\end{pgfscope}%
\begin{pgfscope}%
\pgfsetbuttcap%
\pgfsetroundjoin%
\definecolor{currentfill}{rgb}{0.000000,0.000000,0.000000}%
\pgfsetfillcolor{currentfill}%
\pgfsetlinewidth{0.803000pt}%
\definecolor{currentstroke}{rgb}{0.000000,0.000000,0.000000}%
\pgfsetstrokecolor{currentstroke}%
\pgfsetdash{}{0pt}%
\pgfsys@defobject{currentmarker}{\pgfqpoint{0.000000in}{-0.048611in}}{\pgfqpoint{0.000000in}{0.000000in}}{%
\pgfpathmoveto{\pgfqpoint{0.000000in}{0.000000in}}%
\pgfpathlineto{\pgfqpoint{0.000000in}{-0.048611in}}%
\pgfusepath{stroke,fill}%
}%
\begin{pgfscope}%
\pgfsys@transformshift{3.905315in}{0.866167in}%
\pgfsys@useobject{currentmarker}{}%
\end{pgfscope}%
\end{pgfscope}%
\begin{pgfscope}%
\definecolor{textcolor}{rgb}{0.000000,0.000000,0.000000}%
\pgfsetstrokecolor{textcolor}%
\pgfsetfillcolor{textcolor}%
\pgftext[x=3.943632in, y=0.124753in, left, base,rotate=90.000000]{\color{textcolor}\sffamily\fontsize{10.000000}{12.000000}\selectfont keyboard}%
\end{pgfscope}%
\begin{pgfscope}%
\pgfsetbuttcap%
\pgfsetroundjoin%
\definecolor{currentfill}{rgb}{0.000000,0.000000,0.000000}%
\pgfsetfillcolor{currentfill}%
\pgfsetlinewidth{0.803000pt}%
\definecolor{currentstroke}{rgb}{0.000000,0.000000,0.000000}%
\pgfsetstrokecolor{currentstroke}%
\pgfsetdash{}{0pt}%
\pgfsys@defobject{currentmarker}{\pgfqpoint{0.000000in}{-0.048611in}}{\pgfqpoint{0.000000in}{0.000000in}}{%
\pgfpathmoveto{\pgfqpoint{0.000000in}{0.000000in}}%
\pgfpathlineto{\pgfqpoint{0.000000in}{-0.048611in}}%
\pgfusepath{stroke,fill}%
}%
\begin{pgfscope}%
\pgfsys@transformshift{4.018609in}{0.866167in}%
\pgfsys@useobject{currentmarker}{}%
\end{pgfscope}%
\end{pgfscope}%
\begin{pgfscope}%
\definecolor{textcolor}{rgb}{0.000000,0.000000,0.000000}%
\pgfsetstrokecolor{textcolor}%
\pgfsetfillcolor{textcolor}%
\pgftext[x=4.056925in, y=0.282427in, left, base,rotate=90.000000]{\color{textcolor}\sffamily\fontsize{10.000000}{12.000000}\selectfont drawer}%
\end{pgfscope}%
\begin{pgfscope}%
\pgfsetbuttcap%
\pgfsetroundjoin%
\definecolor{currentfill}{rgb}{0.000000,0.000000,0.000000}%
\pgfsetfillcolor{currentfill}%
\pgfsetlinewidth{0.803000pt}%
\definecolor{currentstroke}{rgb}{0.000000,0.000000,0.000000}%
\pgfsetstrokecolor{currentstroke}%
\pgfsetdash{}{0pt}%
\pgfsys@defobject{currentmarker}{\pgfqpoint{0.000000in}{-0.048611in}}{\pgfqpoint{0.000000in}{0.000000in}}{%
\pgfpathmoveto{\pgfqpoint{0.000000in}{0.000000in}}%
\pgfpathlineto{\pgfqpoint{0.000000in}{-0.048611in}}%
\pgfusepath{stroke,fill}%
}%
\begin{pgfscope}%
\pgfsys@transformshift{4.131902in}{0.866167in}%
\pgfsys@useobject{currentmarker}{}%
\end{pgfscope}%
\end{pgfscope}%
\begin{pgfscope}%
\definecolor{textcolor}{rgb}{0.000000,0.000000,0.000000}%
\pgfsetstrokecolor{textcolor}%
\pgfsetfillcolor{textcolor}%
\pgftext[x=4.170219in, y=0.362587in, left, base,rotate=90.000000]{\color{textcolor}\sffamily\fontsize{10.000000}{12.000000}\selectfont fridge}%
\end{pgfscope}%
\begin{pgfscope}%
\pgfsetbuttcap%
\pgfsetroundjoin%
\definecolor{currentfill}{rgb}{0.000000,0.000000,0.000000}%
\pgfsetfillcolor{currentfill}%
\pgfsetlinewidth{0.803000pt}%
\definecolor{currentstroke}{rgb}{0.000000,0.000000,0.000000}%
\pgfsetstrokecolor{currentstroke}%
\pgfsetdash{}{0pt}%
\pgfsys@defobject{currentmarker}{\pgfqpoint{0.000000in}{-0.048611in}}{\pgfqpoint{0.000000in}{0.000000in}}{%
\pgfpathmoveto{\pgfqpoint{0.000000in}{0.000000in}}%
\pgfpathlineto{\pgfqpoint{0.000000in}{-0.048611in}}%
\pgfusepath{stroke,fill}%
}%
\begin{pgfscope}%
\pgfsys@transformshift{4.245196in}{0.866167in}%
\pgfsys@useobject{currentmarker}{}%
\end{pgfscope}%
\end{pgfscope}%
\begin{pgfscope}%
\definecolor{textcolor}{rgb}{0.000000,0.000000,0.000000}%
\pgfsetstrokecolor{textcolor}%
\pgfsetfillcolor{textcolor}%
\pgftext[x=4.283513in, y=0.453733in, left, base,rotate=90.000000]{\color{textcolor}\sffamily\fontsize{10.000000}{12.000000}\selectfont door}%
\end{pgfscope}%
\begin{pgfscope}%
\pgfsetbuttcap%
\pgfsetroundjoin%
\definecolor{currentfill}{rgb}{0.000000,0.000000,0.000000}%
\pgfsetfillcolor{currentfill}%
\pgfsetlinewidth{0.803000pt}%
\definecolor{currentstroke}{rgb}{0.000000,0.000000,0.000000}%
\pgfsetstrokecolor{currentstroke}%
\pgfsetdash{}{0pt}%
\pgfsys@defobject{currentmarker}{\pgfqpoint{0.000000in}{-0.048611in}}{\pgfqpoint{0.000000in}{0.000000in}}{%
\pgfpathmoveto{\pgfqpoint{0.000000in}{0.000000in}}%
\pgfpathlineto{\pgfqpoint{0.000000in}{-0.048611in}}%
\pgfusepath{stroke,fill}%
}%
\begin{pgfscope}%
\pgfsys@transformshift{4.358490in}{0.866167in}%
\pgfsys@useobject{currentmarker}{}%
\end{pgfscope}%
\end{pgfscope}%
\begin{pgfscope}%
\definecolor{textcolor}{rgb}{0.000000,0.000000,0.000000}%
\pgfsetstrokecolor{textcolor}%
\pgfsetfillcolor{textcolor}%
\pgftext[x=4.396807in, y=0.517887in, left, base,rotate=90.000000]{\color{textcolor}\sffamily\fontsize{10.000000}{12.000000}\selectfont box}%
\end{pgfscope}%
\begin{pgfscope}%
\pgfsetbuttcap%
\pgfsetroundjoin%
\definecolor{currentfill}{rgb}{0.000000,0.000000,0.000000}%
\pgfsetfillcolor{currentfill}%
\pgfsetlinewidth{0.803000pt}%
\definecolor{currentstroke}{rgb}{0.000000,0.000000,0.000000}%
\pgfsetstrokecolor{currentstroke}%
\pgfsetdash{}{0pt}%
\pgfsys@defobject{currentmarker}{\pgfqpoint{0.000000in}{-0.048611in}}{\pgfqpoint{0.000000in}{0.000000in}}{%
\pgfpathmoveto{\pgfqpoint{0.000000in}{0.000000in}}%
\pgfpathlineto{\pgfqpoint{0.000000in}{-0.048611in}}%
\pgfusepath{stroke,fill}%
}%
\begin{pgfscope}%
\pgfsys@transformshift{4.471784in}{0.866167in}%
\pgfsys@useobject{currentmarker}{}%
\end{pgfscope}%
\end{pgfscope}%
\begin{pgfscope}%
\definecolor{textcolor}{rgb}{0.000000,0.000000,0.000000}%
\pgfsetstrokecolor{textcolor}%
\pgfsetfillcolor{textcolor}%
\pgftext[x=4.510100in, y=0.307926in, left, base,rotate=90.000000]{\color{textcolor}\sffamily\fontsize{10.000000}{12.000000}\selectfont basket}%
\end{pgfscope}%
\begin{pgfscope}%
\pgfsetbuttcap%
\pgfsetroundjoin%
\definecolor{currentfill}{rgb}{0.000000,0.000000,0.000000}%
\pgfsetfillcolor{currentfill}%
\pgfsetlinewidth{0.803000pt}%
\definecolor{currentstroke}{rgb}{0.000000,0.000000,0.000000}%
\pgfsetstrokecolor{currentstroke}%
\pgfsetdash{}{0pt}%
\pgfsys@defobject{currentmarker}{\pgfqpoint{0.000000in}{-0.048611in}}{\pgfqpoint{0.000000in}{0.000000in}}{%
\pgfpathmoveto{\pgfqpoint{0.000000in}{0.000000in}}%
\pgfpathlineto{\pgfqpoint{0.000000in}{-0.048611in}}%
\pgfusepath{stroke,fill}%
}%
\begin{pgfscope}%
\pgfsys@transformshift{4.585077in}{0.866167in}%
\pgfsys@useobject{currentmarker}{}%
\end{pgfscope}%
\end{pgfscope}%
\begin{pgfscope}%
\definecolor{textcolor}{rgb}{0.000000,0.000000,0.000000}%
\pgfsetstrokecolor{textcolor}%
\pgfsetfillcolor{textcolor}%
\pgftext[x=4.623394in, y=0.232582in, left, base,rotate=90.000000]{\color{textcolor}\sffamily\fontsize{10.000000}{12.000000}\selectfont cushion}%
\end{pgfscope}%
\begin{pgfscope}%
\pgfsetbuttcap%
\pgfsetroundjoin%
\definecolor{currentfill}{rgb}{0.000000,0.000000,0.000000}%
\pgfsetfillcolor{currentfill}%
\pgfsetlinewidth{0.803000pt}%
\definecolor{currentstroke}{rgb}{0.000000,0.000000,0.000000}%
\pgfsetstrokecolor{currentstroke}%
\pgfsetdash{}{0pt}%
\pgfsys@defobject{currentmarker}{\pgfqpoint{0.000000in}{-0.048611in}}{\pgfqpoint{0.000000in}{0.000000in}}{%
\pgfpathmoveto{\pgfqpoint{0.000000in}{0.000000in}}%
\pgfpathlineto{\pgfqpoint{0.000000in}{-0.048611in}}%
\pgfusepath{stroke,fill}%
}%
\begin{pgfscope}%
\pgfsys@transformshift{4.698371in}{0.866167in}%
\pgfsys@useobject{currentmarker}{}%
\end{pgfscope}%
\end{pgfscope}%
\begin{pgfscope}%
\definecolor{textcolor}{rgb}{0.000000,0.000000,0.000000}%
\pgfsetstrokecolor{textcolor}%
\pgfsetfillcolor{textcolor}%
\pgftext[x=4.736688in, y=0.417179in, left, base,rotate=90.000000]{\color{textcolor}\sffamily\fontsize{10.000000}{12.000000}\selectfont plate}%
\end{pgfscope}%
\begin{pgfscope}%
\pgfsetbuttcap%
\pgfsetroundjoin%
\definecolor{currentfill}{rgb}{0.000000,0.000000,0.000000}%
\pgfsetfillcolor{currentfill}%
\pgfsetlinewidth{0.803000pt}%
\definecolor{currentstroke}{rgb}{0.000000,0.000000,0.000000}%
\pgfsetstrokecolor{currentstroke}%
\pgfsetdash{}{0pt}%
\pgfsys@defobject{currentmarker}{\pgfqpoint{0.000000in}{-0.048611in}}{\pgfqpoint{0.000000in}{0.000000in}}{%
\pgfpathmoveto{\pgfqpoint{0.000000in}{0.000000in}}%
\pgfpathlineto{\pgfqpoint{0.000000in}{-0.048611in}}%
\pgfusepath{stroke,fill}%
}%
\begin{pgfscope}%
\pgfsys@transformshift{4.811665in}{0.866167in}%
\pgfsys@useobject{currentmarker}{}%
\end{pgfscope}%
\end{pgfscope}%
\begin{pgfscope}%
\definecolor{textcolor}{rgb}{0.000000,0.000000,0.000000}%
\pgfsetstrokecolor{textcolor}%
\pgfsetfillcolor{textcolor}%
\pgftext[x=4.849982in, y=0.364960in, left, base,rotate=90.000000]{\color{textcolor}\sffamily\fontsize{10.000000}{12.000000}\selectfont paper}%
\end{pgfscope}%
\begin{pgfscope}%
\pgfsetbuttcap%
\pgfsetroundjoin%
\definecolor{currentfill}{rgb}{0.000000,0.000000,0.000000}%
\pgfsetfillcolor{currentfill}%
\pgfsetlinewidth{0.803000pt}%
\definecolor{currentstroke}{rgb}{0.000000,0.000000,0.000000}%
\pgfsetstrokecolor{currentstroke}%
\pgfsetdash{}{0pt}%
\pgfsys@defobject{currentmarker}{\pgfqpoint{0.000000in}{-0.048611in}}{\pgfqpoint{0.000000in}{0.000000in}}{%
\pgfpathmoveto{\pgfqpoint{0.000000in}{0.000000in}}%
\pgfpathlineto{\pgfqpoint{0.000000in}{-0.048611in}}%
\pgfusepath{stroke,fill}%
}%
\begin{pgfscope}%
\pgfsys@transformshift{4.924959in}{0.866167in}%
\pgfsys@useobject{currentmarker}{}%
\end{pgfscope}%
\end{pgfscope}%
\begin{pgfscope}%
\definecolor{textcolor}{rgb}{0.000000,0.000000,0.000000}%
\pgfsetstrokecolor{textcolor}%
\pgfsetfillcolor{textcolor}%
\pgftext[x=4.963275in, y=0.307248in, left, base,rotate=90.000000]{\color{textcolor}\sffamily\fontsize{10.000000}{12.000000}\selectfont candle}%
\end{pgfscope}%
\begin{pgfscope}%
\pgfsetbuttcap%
\pgfsetroundjoin%
\definecolor{currentfill}{rgb}{0.000000,0.000000,0.000000}%
\pgfsetfillcolor{currentfill}%
\pgfsetlinewidth{0.803000pt}%
\definecolor{currentstroke}{rgb}{0.000000,0.000000,0.000000}%
\pgfsetstrokecolor{currentstroke}%
\pgfsetdash{}{0pt}%
\pgfsys@defobject{currentmarker}{\pgfqpoint{0.000000in}{-0.048611in}}{\pgfqpoint{0.000000in}{0.000000in}}{%
\pgfpathmoveto{\pgfqpoint{0.000000in}{0.000000in}}%
\pgfpathlineto{\pgfqpoint{0.000000in}{-0.048611in}}%
\pgfusepath{stroke,fill}%
}%
\begin{pgfscope}%
\pgfsys@transformshift{5.038252in}{0.866167in}%
\pgfsys@useobject{currentmarker}{}%
\end{pgfscope}%
\end{pgfscope}%
\begin{pgfscope}%
\definecolor{textcolor}{rgb}{0.000000,0.000000,0.000000}%
\pgfsetstrokecolor{textcolor}%
\pgfsetfillcolor{textcolor}%
\pgftext[x=5.076569in, y=0.100000in, left, base,rotate=90.000000]{\color{textcolor}\sffamily\fontsize{10.000000}{12.000000}\selectfont ventilator}%
\end{pgfscope}%
\begin{pgfscope}%
\pgfsetbuttcap%
\pgfsetroundjoin%
\definecolor{currentfill}{rgb}{0.000000,0.000000,0.000000}%
\pgfsetfillcolor{currentfill}%
\pgfsetlinewidth{0.803000pt}%
\definecolor{currentstroke}{rgb}{0.000000,0.000000,0.000000}%
\pgfsetstrokecolor{currentstroke}%
\pgfsetdash{}{0pt}%
\pgfsys@defobject{currentmarker}{\pgfqpoint{0.000000in}{-0.048611in}}{\pgfqpoint{0.000000in}{0.000000in}}{%
\pgfpathmoveto{\pgfqpoint{0.000000in}{0.000000in}}%
\pgfpathlineto{\pgfqpoint{0.000000in}{-0.048611in}}%
\pgfusepath{stroke,fill}%
}%
\begin{pgfscope}%
\pgfsys@transformshift{5.151546in}{0.866167in}%
\pgfsys@useobject{currentmarker}{}%
\end{pgfscope}%
\end{pgfscope}%
\begin{pgfscope}%
\definecolor{textcolor}{rgb}{0.000000,0.000000,0.000000}%
\pgfsetstrokecolor{textcolor}%
\pgfsetfillcolor{textcolor}%
\pgftext[x=5.189863in, y=0.433591in, left, base,rotate=90.000000]{\color{textcolor}\sffamily\fontsize{10.000000}{12.000000}\selectfont stool}%
\end{pgfscope}%
\begin{pgfscope}%
\pgfsetbuttcap%
\pgfsetroundjoin%
\definecolor{currentfill}{rgb}{0.000000,0.000000,0.000000}%
\pgfsetfillcolor{currentfill}%
\pgfsetlinewidth{0.803000pt}%
\definecolor{currentstroke}{rgb}{0.000000,0.000000,0.000000}%
\pgfsetstrokecolor{currentstroke}%
\pgfsetdash{}{0pt}%
\pgfsys@defobject{currentmarker}{\pgfqpoint{-0.048611in}{0.000000in}}{\pgfqpoint{-0.000000in}{0.000000in}}{%
\pgfpathmoveto{\pgfqpoint{-0.000000in}{0.000000in}}%
\pgfpathlineto{\pgfqpoint{-0.048611in}{0.000000in}}%
\pgfusepath{stroke,fill}%
}%
\begin{pgfscope}%
\pgfsys@transformshift{0.462318in}{0.866167in}%
\pgfsys@useobject{currentmarker}{}%
\end{pgfscope}%
\end{pgfscope}%
\begin{pgfscope}%
\definecolor{textcolor}{rgb}{0.000000,0.000000,0.000000}%
\pgfsetstrokecolor{textcolor}%
\pgfsetfillcolor{textcolor}%
\pgftext[x=0.276731in, y=0.813406in, left, base]{\color{textcolor}\sffamily\fontsize{10.000000}{12.000000}\selectfont 0}%
\end{pgfscope}%
\begin{pgfscope}%
\pgfsetbuttcap%
\pgfsetroundjoin%
\definecolor{currentfill}{rgb}{0.000000,0.000000,0.000000}%
\pgfsetfillcolor{currentfill}%
\pgfsetlinewidth{0.803000pt}%
\definecolor{currentstroke}{rgb}{0.000000,0.000000,0.000000}%
\pgfsetstrokecolor{currentstroke}%
\pgfsetdash{}{0pt}%
\pgfsys@defobject{currentmarker}{\pgfqpoint{-0.048611in}{0.000000in}}{\pgfqpoint{-0.000000in}{0.000000in}}{%
\pgfpathmoveto{\pgfqpoint{-0.000000in}{0.000000in}}%
\pgfpathlineto{\pgfqpoint{-0.048611in}{0.000000in}}%
\pgfusepath{stroke,fill}%
}%
\begin{pgfscope}%
\pgfsys@transformshift{0.462318in}{1.591941in}%
\pgfsys@useobject{currentmarker}{}%
\end{pgfscope}%
\end{pgfscope}%
\begin{pgfscope}%
\definecolor{textcolor}{rgb}{0.000000,0.000000,0.000000}%
\pgfsetstrokecolor{textcolor}%
\pgfsetfillcolor{textcolor}%
\pgftext[x=0.188365in, y=1.539179in, left, base]{\color{textcolor}\sffamily\fontsize{10.000000}{12.000000}\selectfont 20}%
\end{pgfscope}%
\begin{pgfscope}%
\pgfsetbuttcap%
\pgfsetroundjoin%
\definecolor{currentfill}{rgb}{0.000000,0.000000,0.000000}%
\pgfsetfillcolor{currentfill}%
\pgfsetlinewidth{0.803000pt}%
\definecolor{currentstroke}{rgb}{0.000000,0.000000,0.000000}%
\pgfsetstrokecolor{currentstroke}%
\pgfsetdash{}{0pt}%
\pgfsys@defobject{currentmarker}{\pgfqpoint{-0.048611in}{0.000000in}}{\pgfqpoint{-0.000000in}{0.000000in}}{%
\pgfpathmoveto{\pgfqpoint{-0.000000in}{0.000000in}}%
\pgfpathlineto{\pgfqpoint{-0.048611in}{0.000000in}}%
\pgfusepath{stroke,fill}%
}%
\begin{pgfscope}%
\pgfsys@transformshift{0.462318in}{2.317714in}%
\pgfsys@useobject{currentmarker}{}%
\end{pgfscope}%
\end{pgfscope}%
\begin{pgfscope}%
\definecolor{textcolor}{rgb}{0.000000,0.000000,0.000000}%
\pgfsetstrokecolor{textcolor}%
\pgfsetfillcolor{textcolor}%
\pgftext[x=0.188365in, y=2.264952in, left, base]{\color{textcolor}\sffamily\fontsize{10.000000}{12.000000}\selectfont 40}%
\end{pgfscope}%
\begin{pgfscope}%
\pgfsetbuttcap%
\pgfsetroundjoin%
\definecolor{currentfill}{rgb}{0.000000,0.000000,0.000000}%
\pgfsetfillcolor{currentfill}%
\pgfsetlinewidth{0.803000pt}%
\definecolor{currentstroke}{rgb}{0.000000,0.000000,0.000000}%
\pgfsetstrokecolor{currentstroke}%
\pgfsetdash{}{0pt}%
\pgfsys@defobject{currentmarker}{\pgfqpoint{-0.048611in}{0.000000in}}{\pgfqpoint{-0.000000in}{0.000000in}}{%
\pgfpathmoveto{\pgfqpoint{-0.000000in}{0.000000in}}%
\pgfpathlineto{\pgfqpoint{-0.048611in}{0.000000in}}%
\pgfusepath{stroke,fill}%
}%
\begin{pgfscope}%
\pgfsys@transformshift{0.462318in}{3.043487in}%
\pgfsys@useobject{currentmarker}{}%
\end{pgfscope}%
\end{pgfscope}%
\begin{pgfscope}%
\definecolor{textcolor}{rgb}{0.000000,0.000000,0.000000}%
\pgfsetstrokecolor{textcolor}%
\pgfsetfillcolor{textcolor}%
\pgftext[x=0.188365in, y=2.990726in, left, base]{\color{textcolor}\sffamily\fontsize{10.000000}{12.000000}\selectfont 60}%
\end{pgfscope}%
\begin{pgfscope}%
\pgfsetbuttcap%
\pgfsetroundjoin%
\definecolor{currentfill}{rgb}{0.000000,0.000000,0.000000}%
\pgfsetfillcolor{currentfill}%
\pgfsetlinewidth{0.803000pt}%
\definecolor{currentstroke}{rgb}{0.000000,0.000000,0.000000}%
\pgfsetstrokecolor{currentstroke}%
\pgfsetdash{}{0pt}%
\pgfsys@defobject{currentmarker}{\pgfqpoint{-0.048611in}{0.000000in}}{\pgfqpoint{-0.000000in}{0.000000in}}{%
\pgfpathmoveto{\pgfqpoint{-0.000000in}{0.000000in}}%
\pgfpathlineto{\pgfqpoint{-0.048611in}{0.000000in}}%
\pgfusepath{stroke,fill}%
}%
\begin{pgfscope}%
\pgfsys@transformshift{0.462318in}{3.769260in}%
\pgfsys@useobject{currentmarker}{}%
\end{pgfscope}%
\end{pgfscope}%
\begin{pgfscope}%
\definecolor{textcolor}{rgb}{0.000000,0.000000,0.000000}%
\pgfsetstrokecolor{textcolor}%
\pgfsetfillcolor{textcolor}%
\pgftext[x=0.188365in, y=3.716499in, left, base]{\color{textcolor}\sffamily\fontsize{10.000000}{12.000000}\selectfont 80}%
\end{pgfscope}%
\begin{pgfscope}%
\pgfsetbuttcap%
\pgfsetroundjoin%
\definecolor{currentfill}{rgb}{0.000000,0.000000,0.000000}%
\pgfsetfillcolor{currentfill}%
\pgfsetlinewidth{0.803000pt}%
\definecolor{currentstroke}{rgb}{0.000000,0.000000,0.000000}%
\pgfsetstrokecolor{currentstroke}%
\pgfsetdash{}{0pt}%
\pgfsys@defobject{currentmarker}{\pgfqpoint{-0.048611in}{0.000000in}}{\pgfqpoint{-0.000000in}{0.000000in}}{%
\pgfpathmoveto{\pgfqpoint{-0.000000in}{0.000000in}}%
\pgfpathlineto{\pgfqpoint{-0.048611in}{0.000000in}}%
\pgfusepath{stroke,fill}%
}%
\begin{pgfscope}%
\pgfsys@transformshift{0.462318in}{4.495033in}%
\pgfsys@useobject{currentmarker}{}%
\end{pgfscope}%
\end{pgfscope}%
\begin{pgfscope}%
\definecolor{textcolor}{rgb}{0.000000,0.000000,0.000000}%
\pgfsetstrokecolor{textcolor}%
\pgfsetfillcolor{textcolor}%
\pgftext[x=0.100000in, y=4.442272in, left, base]{\color{textcolor}\sffamily\fontsize{10.000000}{12.000000}\selectfont 100}%
\end{pgfscope}%
\begin{pgfscope}%
\pgfsetrectcap%
\pgfsetmiterjoin%
\pgfsetlinewidth{0.803000pt}%
\definecolor{currentstroke}{rgb}{0.000000,0.000000,0.000000}%
\pgfsetstrokecolor{currentstroke}%
\pgfsetdash{}{0pt}%
\pgfpathmoveto{\pgfqpoint{0.462318in}{0.866167in}}%
\pgfpathlineto{\pgfqpoint{0.462318in}{4.562167in}}%
\pgfusepath{stroke}%
\end{pgfscope}%
\begin{pgfscope}%
\pgfsetrectcap%
\pgfsetmiterjoin%
\pgfsetlinewidth{0.803000pt}%
\definecolor{currentstroke}{rgb}{0.000000,0.000000,0.000000}%
\pgfsetstrokecolor{currentstroke}%
\pgfsetdash{}{0pt}%
\pgfpathmoveto{\pgfqpoint{5.422318in}{0.866167in}}%
\pgfpathlineto{\pgfqpoint{5.422318in}{4.562167in}}%
\pgfusepath{stroke}%
\end{pgfscope}%
\begin{pgfscope}%
\pgfsetrectcap%
\pgfsetmiterjoin%
\pgfsetlinewidth{0.803000pt}%
\definecolor{currentstroke}{rgb}{0.000000,0.000000,0.000000}%
\pgfsetstrokecolor{currentstroke}%
\pgfsetdash{}{0pt}%
\pgfpathmoveto{\pgfqpoint{0.462318in}{0.866167in}}%
\pgfpathlineto{\pgfqpoint{5.422318in}{0.866167in}}%
\pgfusepath{stroke}%
\end{pgfscope}%
\begin{pgfscope}%
\pgfsetrectcap%
\pgfsetmiterjoin%
\pgfsetlinewidth{0.803000pt}%
\definecolor{currentstroke}{rgb}{0.000000,0.000000,0.000000}%
\pgfsetstrokecolor{currentstroke}%
\pgfsetdash{}{0pt}%
\pgfpathmoveto{\pgfqpoint{0.462318in}{4.562168in}}%
\pgfpathlineto{\pgfqpoint{5.422318in}{4.562168in}}%
\pgfusepath{stroke}%
\end{pgfscope}%
\begin{pgfscope}%
\definecolor{textcolor}{rgb}{0.000000,0.000000,0.000000}%
\pgfsetstrokecolor{textcolor}%
\pgfsetfillcolor{textcolor}%
\pgftext[x=2.942318in,y=4.645501in,,base]{\color{textcolor}\sffamily\fontsize{12.000000}{14.400000}\selectfont Distribution of textures}%
\end{pgfscope}%
\end{pgfpicture}%
\makeatother%
\endgroup%
}
    \caption{Distribution of textures used on scenes. The categories of pix3d(target furniture) have higher number of images.}
    \label{fig:Distribution of textures}
\end{figure}

\begin{figure}
    \centering
        \includegraphics[width=.4\textwidth, height = .3\textwidth,valign=m]{/Users/apple/OVGU/Thesis/code/3dReconstruction/report/images/implementation/randomisation/skybox_1}
        \includegraphics[width=.4\textwidth, height = .3\textwidth,valign=m]{/Users/apple/OVGU/Thesis/code/3dReconstruction/report/images/implementation/randomisation/skybox_2} \\
        \vspace{0.1cm}
        \includegraphics[width=.4\textwidth, height = .3\textwidth,valign=m]{/Users/apple/OVGU/Thesis/code/3dReconstruction/report/images/implementation/randomisation/skybox_3}
        \includegraphics[width=.4\textwidth, height = .3\textwidth,valign=m]{/Users/apple/OVGU/Thesis/code/3dReconstruction/report/images/implementation/randomisation/skybox_4}\\
    \caption{Samples for different skyboxes which change the outdoor environment for the scenes. In the figure we see an open window with changing skybox.}
    \label{fig:skybox samples}
\end{figure}

\subsection{Replacing target Objects}\label{subsec:replacing-target-objects}

Once the room is set up using Scenenet~\cite{McCormac:etal:ICCV2017} which contains objects from ShapeNet~\cite{chang2015shapenet}, the objects are renamed such that it matches the classes for the Pix3D dataset.
Since a given scene can have more than one category under observation, a single model is chosen randomly out of all the present models.
The chosen model is replaced with a model in Pix3D of the same category.
In case the replaced model is intersecting with any other models in the scene, we make it invisible.
Further, the camera checks if the center of replaced model is in the frame.
If not, the camera viewpoint is changed until it satisfies the condition.
After the snapshot is taken, the scene is reset to the original and the next model is imported.
Samples for replacing a target object from an original scene from Scenenet is shown in figure~\ref{fig:replace model}.

\begin{figure}
    \centering
        \includegraphics[width=.4\textwidth, height = .3\textwidth,valign=m]{/Users/apple/OVGU/Thesis/code/3dReconstruction/report/images/implementation/randomisation/replace_1-1}
        \includegraphics[width=.4\textwidth, height = .3\textwidth,valign=m]{/Users/apple/OVGU/Thesis/code/3dReconstruction/report/images/implementation/randomisation/replace_1-2} \\
        \vspace{0.1cm}
        \includegraphics[width=.4\textwidth, height = .3\textwidth,valign=m]{/Users/apple/OVGU/Thesis/code/3dReconstruction/report/images/implementation/randomisation/replace_2-1}
        \includegraphics[width=.4\textwidth, height = .3\textwidth,valign=m]{/Users/apple/OVGU/Thesis/code/3dReconstruction/report/images/implementation/randomisation/replace_2-2}\\
    \caption{Samples for object replacement. The Left column shows a scene from SceneNet, while the right column shows an object being replaced in original scene.}
    \label{fig:replace model}
\end{figure}

\subsection{Snapshots with ML-ImageSynthesis}\label{subsec:snapshots-with-ml-imagesynthesis}

ML-ImageSynthesis~\cite{imagesynthesis} is a library provided by Unity to support synthetic dataset creation.
It supports the following G-buffers Image segmentation, Object categorization, Optical flow, Depth, Normals, etc.
G-buffers is basically a collective term to represent light properties that are aggregated to give the final rendering.
In image segmentation, each object is assigned a unique color.
In object categorization, each category of the objects is assigned a single color.
For optical flow, each pixel is colored depending on Unity's per-pixel Motion Vectors with respect to the camera.
Depth is as the word suggests the distance of each pixel from the camera encoded in 8-bit channels of the PNG image.
Normals are color encoding based on the orientation of object with respect to the camera.
For the snapshot, each of these properties is assigned a hidden camera, and has each one overrides rendering scene with custom shaders to generate corresponding output.
These camera outputs can be seen in editor mode by changing the display window.
Samples of G-buffer collected for the dataset are as shown in figure~\ref{fig:G-buffers-samples}.

\begin{figure}
    \centering
        \includegraphics[width=.19\linewidth]{/Users/apple/OVGU/Thesis/code/3dReconstruction/report/images/implementation/gbuffers/0_img}
        \includegraphics[width=.19\linewidth]{/Users/apple/OVGU/Thesis/code/3dReconstruction/report/images/implementation/gbuffers/0_depth}
        \includegraphics[width=.19\linewidth]{/Users/apple/OVGU/Thesis/code/3dReconstruction/report/images/implementation/gbuffers/0_id}
        \includegraphics[width=.19\linewidth]{/Users/apple/OVGU/Thesis/code/3dReconstruction/report/images/implementation/gbuffers/0_layer}
        \includegraphics[width=.19\linewidth]{/Users/apple/OVGU/Thesis/code/3dReconstruction/report/images/implementation/gbuffers/0_normals}\\
        \vspace{0.1cm}
        \includegraphics[width=.19\linewidth]{/Users/apple/OVGU/Thesis/code/3dReconstruction/report/images/implementation/gbuffers/1_img}
        \includegraphics[width=.19\linewidth]{/Users/apple/OVGU/Thesis/code/3dReconstruction/report/images/implementation/gbuffers/1_depth}
        \includegraphics[width=.19\linewidth]{/Users/apple/OVGU/Thesis/code/3dReconstruction/report/images/implementation/gbuffers/1_id}
        \includegraphics[width=.19\linewidth]{/Users/apple/OVGU/Thesis/code/3dReconstruction/report/images/implementation/gbuffers/1_layer}
        \includegraphics[width=.19\linewidth]{/Users/apple/OVGU/Thesis/code/3dReconstruction/report/images/implementation/gbuffers/1_normals}\\
        \vspace{0.1cm}

        \includegraphics[width=.19\linewidth]{/Users/apple/OVGU/Thesis/code/3dReconstruction/report/images/implementation/gbuffers/2_img}
        \includegraphics[width=.19\linewidth]{/Users/apple/OVGU/Thesis/code/3dReconstruction/report/images/implementation/gbuffers/2_depth}
        \includegraphics[width=.19\linewidth]{/Users/apple/OVGU/Thesis/code/3dReconstruction/report/images/implementation/gbuffers/2_id}
        \includegraphics[width=.19\linewidth]{/Users/apple/OVGU/Thesis/code/3dReconstruction/report/images/implementation/gbuffers/2_layer}
        \includegraphics[width=.19\linewidth]{/Users/apple/OVGU/Thesis/code/3dReconstruction/report/images/implementation/gbuffers/2_normals}\\
        \vspace{0.1cm}
        \includegraphics[width=.19\linewidth]{/Users/apple/OVGU/Thesis/code/3dReconstruction/report/images/implementation/gbuffers/3_img}
        \includegraphics[width=.19\linewidth]{/Users/apple/OVGU/Thesis/code/3dReconstruction/report/images/implementation/gbuffers/3_depth}
        \includegraphics[width=.19\linewidth]{/Users/apple/OVGU/Thesis/code/3dReconstruction/report/images/implementation/gbuffers/3_id}
        \includegraphics[width=.19\linewidth]{/Users/apple/OVGU/Thesis/code/3dReconstruction/report/images/implementation/gbuffers/3_layer}
        \includegraphics[width=.19\linewidth]{/Users/apple/OVGU/Thesis/code/3dReconstruction/report/images/implementation/gbuffers/3_normals}\\
    \vspace{0.1cm}

        \includegraphics[width=.19\linewidth]{/Users/apple/OVGU/Thesis/code/3dReconstruction/report/images/implementation/gbuffers/4_img}
        \includegraphics[width=.19\linewidth]{/Users/apple/OVGU/Thesis/code/3dReconstruction/report/images/implementation/gbuffers/4_depth}
        \includegraphics[width=.19\linewidth]{/Users/apple/OVGU/Thesis/code/3dReconstruction/report/images/implementation/gbuffers/4_id}
        \includegraphics[width=.19\linewidth]{/Users/apple/OVGU/Thesis/code/3dReconstruction/report/images/implementation/gbuffers/4_layer}
        \includegraphics[width=.19\linewidth]{/Users/apple/OVGU/Thesis/code/3dReconstruction/report/images/implementation/gbuffers/4_normals}\\
    \vspace{0.1cm}

        \includegraphics[width=.19\linewidth]{/Users/apple/OVGU/Thesis/code/3dReconstruction/report/images/implementation/gbuffers/5_img}
        \includegraphics[width=.19\linewidth]{/Users/apple/OVGU/Thesis/code/3dReconstruction/report/images/implementation/gbuffers/5_depth}
        \includegraphics[width=.19\linewidth]{/Users/apple/OVGU/Thesis/code/3dReconstruction/report/images/implementation/gbuffers/5_id}
        \includegraphics[width=.19\linewidth]{/Users/apple/OVGU/Thesis/code/3dReconstruction/report/images/implementation/gbuffers/5_layer}
        \includegraphics[width=.19\linewidth]{/Users/apple/OVGU/Thesis/code/3dReconstruction/report/images/implementation/gbuffers/5_normals}\\
    \vspace{0.1cm}

        \includegraphics[width=.19\linewidth]{/Users/apple/OVGU/Thesis/code/3dReconstruction/report/images/implementation/gbuffers/6_img}
        \includegraphics[width=.19\linewidth]{/Users/apple/OVGU/Thesis/code/3dReconstruction/report/images/implementation/gbuffers/6_depth}
        \includegraphics[width=.19\linewidth]{/Users/apple/OVGU/Thesis/code/3dReconstruction/report/images/implementation/gbuffers/6_id}
        \includegraphics[width=.19\linewidth]{/Users/apple/OVGU/Thesis/code/3dReconstruction/report/images/implementation/gbuffers/6_layer}
        \includegraphics[width=.19\linewidth]{/Users/apple/OVGU/Thesis/code/3dReconstruction/report/images/implementation/gbuffers/6_normals}\\
    \caption{Samples for G-buffers collected from ImageSynthesis as part of the dataset. In the figure,
        (From left to right) RGB image, Depth map, Instance segmentation, Semantic Segmentation and Normal map}
    \label{fig:G-buffers-samples}
\end{figure}

\subsection{Demo for the application}\label{subsec:demo}

The researchers are provided source code for the Unity-based application~\footnote{https://github.com/kartikprabhu20/3dScene}, which was used to create the synthetic datasets.
The application allows the users to provide paths for the 3D models, 3D rooms, textures, and output.
The user can also select the category of furniture for which images are to be generated and the quantity per category.
For the camera, the user can decide upon the height of the camera position, minimum and maximum distance from the target model between which a viewpoint will be randomly chosen.
The modes are discussed in~\ref{subsec:modes-of-operations}, which can be select once all the data is configured.
For the manual mode, the user can randomize one or more parameters and take the snapshot as desired.
Figure~\ref{fig:demo1} shows the configuration page for the demo application, figure~\ref{fig:demo2} shows the application in action in manual mode.

\begin{figure}
    \centering
    \includegraphics[width=\textwidth]{/Users/apple/OVGU/Thesis/code/3dReconstruction/report/images/implementation/demo/demo}
    \caption{A screenshot of the Unity based application developed for proof of concept to create synthetic dataset.}
    \label{fig:demo1}
\end{figure}

\begin{figure}
    \centering
    \includegraphics[width=\textwidth]{/Users/apple/OVGU/Thesis/code/3dReconstruction/report/images/implementation/demo/demo2}
    \caption{A screenshot of the Unity based application in action, the user is able to configure the pipeline using GUI and take snapshots either automatically or manually.}
    \label{fig:demo2}
\end{figure}

\section{3d-Reconstruction framework}\label{sec:3d-reconstruction-framework}

In this section, we discuss the pipeline used for the 3D reconstruction~\footnote{https://github.com/kartikprabhu20/3dReconstruction} tasks using Deep Learning.
The discussion will revolve around the implementation of models and the training parameters.
The code for the pipeline was written in Python 3.7.9.
The backbone of this pipeline is pytorch 1.7.1~\cite{NEURIPS2019_9015}.

The key features of this pipeline are as follows:
\begin{enumerate}
    \item Allows users to configure the parameter using a config file which included dataset paths, root paths, output path, parameters for 2D augmentations, etc.
    \item Allows users to select train, validate, test with real data, an empty image test.
    \item Allows users to select the pipeline type(3d-reconstruction, cyclegan~\cite{CycleGAN2017}, autoencoder)
    \item Allows users to select the dataset(Pix3d, \gls(free), Mixed), with a further option to add new datasets.
    \item Allows users to collect graphs, images per epoch using tensorboard~\cite{tensorflow2015-whitepaper}.
\end{enumerate}

\subsection{Preprocessing}\label{subsec:preprocessing}
As preprocessing for testing, images were resized to 224$\times$224, and then center cropped to 128$\times$128.
The images were then normalized using the parameters used for ImageNet~\cite{Deng2009ImageNetAL} Mean=[0.485, 0.456, 0.406] and STD=[0.229, 0.224, 0.225].
This was also done for training images followed by 2D-augmentations which included ColorJitter, RandomNoise, RandomFlip and, RandomPermuteRGB.
All the images were then transformed to a tensor to make them compatible with deep learning models.

\subsection{Modelling}\label{subsec:modelling}
For models we used the architectures proposed in ~\cite{Xie_2019} and ~\cite{Xie_2020}.
The backbones of these 2 models are \gls{vgg} and \gls{resnet} respectively.
These models were obtained from pytorch's model zoo, pretrained on ImageNet.
Rest of the architecture were built using basic models using pytorch.
The weights for the models were initialised using Glorot Initializer~\cite{Glorot2010UnderstandingTD}.

\subsection{Training}\label{subsec:training}

For training, the Deep learning models for the 3D reconstruction task the GPU used was NVIDIA DGX-1 with 1 node,
with 512 GB RAM, and 8x NVIDIA Tesla V100 GPU cards.
The hyperparameters were initially fine-tuned with a grid search approach and then trained on parameters on which the baselines were published.
The difference was not significant, hence we decided to follow the hyperparameters used by the authors~\cite{Xie_2019}.
The only visible difference being the batch size, where we used the maximum possible value that would fit a 32GB memory of the used GPU.
The rest of the hyperparameters are as shown in table~\ref{tab:hyperparameter}.
A more extensive hyperparameter optimization can be done in the future for better performance.
The model took up to 96 hours to train.


The users could decide when to write the training values in the tensorboard in the config file.
The tensorboard not only saves the loss values for each epoch but also the images of the 3d models for the configured epochs.
Tensorboard also has the capability to save 3D models, but we observed that the interface freezes and also occupies a larger storge.
Hence, we convert the 3D models into 2D images for visualization using Matplotlib.

\begin{table}[ht]
    \centering
    \begin{tabular}{|c |c |}
        \hline
        Hyperparameter & Value \\ [0.5ex]
        \hline\hline
        Optimizer & Adam(\beta_1 =.9, \beta_2=.999)\\
        \hline
        Weight Initializer & Glorot \\
        \hline
        Batch Size & 128  \\
        \hline
        Learning Rate & 0.001 \\
        \hline
        Epochs & 400\\
        \hline
        Loss & Binary Crossentropy\\
        \hline
    \end{tabular}
    \caption{Hyperparameters used for training 3d reconstruction models.}
    \label{tab:hyperparameter}
\end{table}

\subsection{Evaluation}\label{subsec:evaluation}

For evaluating the models we use \gls{iou} as used in the original paper~\cite{Xie_2019}.
In the validation step, we compute the average ~\gls{iou} for each of the batches and store the best epoch value.
In case where the value increases, we replace the best value and save the model state as the best model.
No thresholding is done during the validation step to apply binary cross-entropy.

For the testing step, we use the test split from the real dataset(Pix3D).
In this step, we compute the average of each sample with different threshold values and select the best average value as our result.
The threshold values used are 0.05, 0.075, .1, .2, .3, .4, .5, .6, .7.
The average \gls{iou }values for each of the categories are also noted for real data test.




\chapter{\iftoggle{german}{Evaluierung}{Evaluation}}\label{ch:evaluation}

\todo{
    \begin{enumerate}
        \item baseline comparison, pix3d pix2vox, pix2vox++ classwise)
        \item values per category
        \item baseline different version of dataset with different threshold
        \item ablation values per model
        \item !!!!!finetuning with different datasize (pending)
        \item !!!!!mixed training with different real datasize(pending)
        \item output diagrams
        \item training graphs
    \end{enumerate}
}

In this chapter we will conduct experiments that will help us find solutions for the research questions discussed in~\ref{sec:goal}.
Section~\ref{sec:a-survey-on-photorealism} will contain the survey results conducted to check the photorealism of the proposed synthetic dataset.
In this section we will also compare the ratings given to other proclaimed photorealistic dataset and check whether \gls{free} dataset compares to those datasets.
We will further evaluate the datasets using a T-SNE as qualitative measure and MSE \& FID as quantitative measure which will indicate the domain gaps with respect to the real dataset.

In section~\ref{sec:datasets}, we describe different datasets specifically generated to evaluate our baseline models and check randomisation parameters.
Section~\ref{sec:baseline} evaluates baseline models with real and synthetic versions of datasets.
Section~\ref{sec:fine-tuning} further evaluates models pretrained on synthetic dataset by fine-tuning them using real dataset.
In section~\ref{sec:ablation-study-on-chairs} we will evaluate models on chair dataset with different parameters of randomisation and also mixed training for these individual datasets.

\section{A survey on photorealism}\label{sec:a-survey-on-photorealism}
The participants were given minimalistic information about the intention behind the survey.
The goal of the survey was to analyse if humans have the same perception for photographic and computer generated images.
No time limit was set for the survey, and it was open to everyone.
A total of 72 participants responded to the survey.
The survey was created using Google forms and the link was distributed.
The participants either used a mobile phone or a desktop to respond to the survey.
A total of 9 datasets were used in the survey.
\gls{front}~\cite{Fu20203DFRONT3F}, Hypersim~\cite{Roberts2020HypersimAP}, InteriorNet~\cite{InteriorNet18}, SceneNet~\cite{McCormac:etal:ICCV2017}, BlenderProc~\cite{denninger2019blenderproc},
\gls{ai2thor}~\cite{kolve2019ai2thor}, Openroom~\cite{li2021openrooms}, Pix3D~\cite{pix3d} and proposed \gls{free} dataset.
Only Pix3D was a real dataset while all others are synthetic dataset proclaimed to be photorealistic.

The survey was composed of 3 sections.
\begin{enumerate}
    \item Section 1: Decide if the image is real or not real.

    In this section there were a total of 27 images, 3 each from the above mentioned datasets.
    Each image had only 2 options to select: "Real" or "Not real"
    This approach eliminated any ambiguous perception towards the images.

    \item Section 2: Rate the image on scale of 1 to 10 in terms of realism (1 -> least real, 10 -> most real).

    In this section, the participant used a likert scale to rate the images based on photorealism.
    Similar to section 1, there were 27 questions of 3 images per dataset.

    \item Section 3: Rank the images from 1 to 9 (1 -> Most real, 9 -> Least real).

    In this section, the participant had only 3 questions, with each question having an image from each of the dataset arranged in a 3\x3 grid format.
    The users were asked to rank them in the increasing order of the photorealism.
\end{enumerate}

\subsection{Survey results}\label{subsec:survey-results}
In this segment, we discuss the results of survey collected from participants.

\subsubsection{Section 1: Real or Not}
In section 1, the participants had only 2 options to select from independently.
Figure~\ref{fig:question1}, shows that the real dataset Pix3D~\cite{pix3d} was rightly recognised as real.
77\% of the real images that belonged to Pix3D were recognised to be real.
This shows that the participants were not convinced even with the real images as 23\% of the images were still recognized are not real.
Among the synthetic datasets, Hyeperism~\cite{Roberts2020HypersimAP} got the best results of 59.7\% identified as real.
AI2THOR had the least amount of images recognised as real with just 5\% positive responses.
The proposed \gls{free} dataset had 8\% of images identified as real, which shows that images generated using the automated Unity framework needs some improvement.
Suppose we have a threshold of 50\%, we see that the datasets for which the images selected as Not real below the threshold value belong to datasets which are automated and not created by professionals manually.
As mentioned in~\ref{subsec:indoor-synthetic-datasets}, Openrooms, SceneNet are Blenderproc are datasets obtained from automation.
We consider \gls{free} dataset to be automated and belog to this category.
Among the automated images, Openrooms have got most vote of confidence with 37\% recognised as real images.
Even though \gls{free} was least recognised as real among the automated tools, it had better percetage than AI2THOR which has Unity based frameworm to generate images and was manually configured by professional by taking in reference of real world images.

\begin{figure}
    \centering
    \resizebox{\textwidth}{!}{%% Creator: Matplotlib, PGF backend
%%
%% To include the figure in your LaTeX document, write
%%   \input{<filename>.pgf}
%%
%% Make sure the required packages are loaded in your preamble
%%   \usepackage{pgf}
%%
%% Figures using additional raster images can only be included by \input if
%% they are in the same directory as the main LaTeX file. For loading figures
%% from other directories you can use the `import` package
%%   \usepackage{import}
%%
%% and then include the figures with
%%   \import{<path to file>}{<filename>.pgf}
%%
%% Matplotlib used the following preamble
%%   \usepackage{fontspec}
%%   \setmainfont{DejaVuSerif.ttf}[Path=\detokenize{/Users/apple/opt/anaconda3/envs/kaolin/lib/python3.7/site-packages/matplotlib/mpl-data/fonts/ttf/}]
%%   \setsansfont{DejaVuSans.ttf}[Path=\detokenize{/Users/apple/opt/anaconda3/envs/kaolin/lib/python3.7/site-packages/matplotlib/mpl-data/fonts/ttf/}]
%%   \setmonofont{DejaVuSansMono.ttf}[Path=\detokenize{/Users/apple/opt/anaconda3/envs/kaolin/lib/python3.7/site-packages/matplotlib/mpl-data/fonts/ttf/}]
%%
\begingroup%
\makeatletter%
\begin{pgfpicture}%
\pgfpathrectangle{\pgfpointorigin}{\pgfqpoint{9.190000in}{5.000000in}}%
\pgfusepath{use as bounding box, clip}%
\begin{pgfscope}%
\pgfsetbuttcap%
\pgfsetmiterjoin%
\definecolor{currentfill}{rgb}{1.000000,1.000000,1.000000}%
\pgfsetfillcolor{currentfill}%
\pgfsetlinewidth{0.000000pt}%
\definecolor{currentstroke}{rgb}{1.000000,1.000000,1.000000}%
\pgfsetstrokecolor{currentstroke}%
\pgfsetdash{}{0pt}%
\pgfpathmoveto{\pgfqpoint{0.000000in}{0.000000in}}%
\pgfpathlineto{\pgfqpoint{9.190000in}{0.000000in}}%
\pgfpathlineto{\pgfqpoint{9.190000in}{5.000000in}}%
\pgfpathlineto{\pgfqpoint{0.000000in}{5.000000in}}%
\pgfpathclose%
\pgfusepath{fill}%
\end{pgfscope}%
\begin{pgfscope}%
\pgfsetbuttcap%
\pgfsetmiterjoin%
\definecolor{currentfill}{rgb}{1.000000,1.000000,1.000000}%
\pgfsetfillcolor{currentfill}%
\pgfsetlinewidth{0.000000pt}%
\definecolor{currentstroke}{rgb}{0.000000,0.000000,0.000000}%
\pgfsetstrokecolor{currentstroke}%
\pgfsetstrokeopacity{0.000000}%
\pgfsetdash{}{0pt}%
\pgfpathmoveto{\pgfqpoint{1.148750in}{0.550000in}}%
\pgfpathlineto{\pgfqpoint{8.271000in}{0.550000in}}%
\pgfpathlineto{\pgfqpoint{8.271000in}{4.400000in}}%
\pgfpathlineto{\pgfqpoint{1.148750in}{4.400000in}}%
\pgfpathclose%
\pgfusepath{fill}%
\end{pgfscope}%
\begin{pgfscope}%
\pgfpathrectangle{\pgfqpoint{1.148750in}{0.550000in}}{\pgfqpoint{7.122250in}{3.850000in}}%
\pgfusepath{clip}%
\pgfsetbuttcap%
\pgfsetmiterjoin%
\definecolor{currentfill}{rgb}{0.248058,0.667205,0.350250}%
\pgfsetfillcolor{currentfill}%
\pgfsetfillopacity{0.500000}%
\pgfsetlinewidth{0.000000pt}%
\definecolor{currentstroke}{rgb}{0.000000,0.000000,0.000000}%
\pgfsetstrokecolor{currentstroke}%
\pgfsetstrokeopacity{0.500000}%
\pgfsetdash{}{0pt}%
\pgfpathmoveto{\pgfqpoint{1.148750in}{4.225000in}}%
\pgfpathlineto{\pgfqpoint{6.655304in}{4.225000in}}%
\pgfpathlineto{\pgfqpoint{6.655304in}{4.019118in}}%
\pgfpathlineto{\pgfqpoint{1.148750in}{4.019118in}}%
\pgfpathclose%
\pgfusepath{fill}%
\end{pgfscope}%
\begin{pgfscope}%
\pgfpathrectangle{\pgfqpoint{1.148750in}{0.550000in}}{\pgfqpoint{7.122250in}{3.850000in}}%
\pgfusepath{clip}%
\pgfsetbuttcap%
\pgfsetmiterjoin%
\definecolor{currentfill}{rgb}{0.248058,0.667205,0.350250}%
\pgfsetfillcolor{currentfill}%
\pgfsetfillopacity{0.500000}%
\pgfsetlinewidth{0.000000pt}%
\definecolor{currentstroke}{rgb}{0.000000,0.000000,0.000000}%
\pgfsetstrokecolor{currentstroke}%
\pgfsetstrokeopacity{0.500000}%
\pgfsetdash{}{0pt}%
\pgfpathmoveto{\pgfqpoint{1.148750in}{3.813235in}}%
\pgfpathlineto{\pgfqpoint{5.402316in}{3.813235in}}%
\pgfpathlineto{\pgfqpoint{5.402316in}{3.607353in}}%
\pgfpathlineto{\pgfqpoint{1.148750in}{3.607353in}}%
\pgfpathclose%
\pgfusepath{fill}%
\end{pgfscope}%
\begin{pgfscope}%
\pgfpathrectangle{\pgfqpoint{1.148750in}{0.550000in}}{\pgfqpoint{7.122250in}{3.850000in}}%
\pgfusepath{clip}%
\pgfsetbuttcap%
\pgfsetmiterjoin%
\definecolor{currentfill}{rgb}{0.248058,0.667205,0.350250}%
\pgfsetfillcolor{currentfill}%
\pgfsetfillopacity{0.500000}%
\pgfsetlinewidth{0.000000pt}%
\definecolor{currentstroke}{rgb}{0.000000,0.000000,0.000000}%
\pgfsetstrokecolor{currentstroke}%
\pgfsetstrokeopacity{0.500000}%
\pgfsetdash{}{0pt}%
\pgfpathmoveto{\pgfqpoint{1.148750in}{3.401471in}}%
\pgfpathlineto{\pgfqpoint{4.775822in}{3.401471in}}%
\pgfpathlineto{\pgfqpoint{4.775822in}{3.195588in}}%
\pgfpathlineto{\pgfqpoint{1.148750in}{3.195588in}}%
\pgfpathclose%
\pgfusepath{fill}%
\end{pgfscope}%
\begin{pgfscope}%
\pgfpathrectangle{\pgfqpoint{1.148750in}{0.550000in}}{\pgfqpoint{7.122250in}{3.850000in}}%
\pgfusepath{clip}%
\pgfsetbuttcap%
\pgfsetmiterjoin%
\definecolor{currentfill}{rgb}{0.248058,0.667205,0.350250}%
\pgfsetfillcolor{currentfill}%
\pgfsetlinewidth{0.000000pt}%
\definecolor{currentstroke}{rgb}{0.000000,0.000000,0.000000}%
\pgfsetstrokecolor{currentstroke}%
\pgfsetstrokeopacity{0.000000}%
\pgfsetdash{}{0pt}%
\pgfpathmoveto{\pgfqpoint{1.148750in}{2.989706in}}%
\pgfpathlineto{\pgfqpoint{3.786620in}{2.989706in}}%
\pgfpathlineto{\pgfqpoint{3.786620in}{2.783824in}}%
\pgfpathlineto{\pgfqpoint{1.148750in}{2.783824in}}%
\pgfpathclose%
\pgfusepath{fill}%
\end{pgfscope}%
\begin{pgfscope}%
\pgfpathrectangle{\pgfqpoint{1.148750in}{0.550000in}}{\pgfqpoint{7.122250in}{3.850000in}}%
\pgfusepath{clip}%
\pgfsetbuttcap%
\pgfsetmiterjoin%
\definecolor{currentfill}{rgb}{0.248058,0.667205,0.350250}%
\pgfsetfillcolor{currentfill}%
\pgfsetlinewidth{0.000000pt}%
\definecolor{currentstroke}{rgb}{0.000000,0.000000,0.000000}%
\pgfsetstrokecolor{currentstroke}%
\pgfsetstrokeopacity{0.000000}%
\pgfsetdash{}{0pt}%
\pgfpathmoveto{\pgfqpoint{1.148750in}{2.577941in}}%
\pgfpathlineto{\pgfqpoint{3.160126in}{2.577941in}}%
\pgfpathlineto{\pgfqpoint{3.160126in}{2.372059in}}%
\pgfpathlineto{\pgfqpoint{1.148750in}{2.372059in}}%
\pgfpathclose%
\pgfusepath{fill}%
\end{pgfscope}%
\begin{pgfscope}%
\pgfpathrectangle{\pgfqpoint{1.148750in}{0.550000in}}{\pgfqpoint{7.122250in}{3.850000in}}%
\pgfusepath{clip}%
\pgfsetbuttcap%
\pgfsetmiterjoin%
\definecolor{currentfill}{rgb}{0.248058,0.667205,0.350250}%
\pgfsetfillcolor{currentfill}%
\pgfsetfillopacity{0.500000}%
\pgfsetlinewidth{0.000000pt}%
\definecolor{currentstroke}{rgb}{0.000000,0.000000,0.000000}%
\pgfsetstrokecolor{currentstroke}%
\pgfsetstrokeopacity{0.500000}%
\pgfsetdash{}{0pt}%
\pgfpathmoveto{\pgfqpoint{1.148750in}{2.166176in}}%
\pgfpathlineto{\pgfqpoint{3.028233in}{2.166176in}}%
\pgfpathlineto{\pgfqpoint{3.028233in}{1.960294in}}%
\pgfpathlineto{\pgfqpoint{1.148750in}{1.960294in}}%
\pgfpathclose%
\pgfusepath{fill}%
\end{pgfscope}%
\begin{pgfscope}%
\pgfpathrectangle{\pgfqpoint{1.148750in}{0.550000in}}{\pgfqpoint{7.122250in}{3.850000in}}%
\pgfusepath{clip}%
\pgfsetbuttcap%
\pgfsetmiterjoin%
\definecolor{currentfill}{rgb}{0.248058,0.667205,0.350250}%
\pgfsetfillcolor{currentfill}%
\pgfsetlinewidth{0.000000pt}%
\definecolor{currentstroke}{rgb}{0.000000,0.000000,0.000000}%
\pgfsetstrokecolor{currentstroke}%
\pgfsetstrokeopacity{0.000000}%
\pgfsetdash{}{0pt}%
\pgfpathmoveto{\pgfqpoint{1.148750in}{1.754412in}}%
\pgfpathlineto{\pgfqpoint{2.467685in}{1.754412in}}%
\pgfpathlineto{\pgfqpoint{2.467685in}{1.548529in}}%
\pgfpathlineto{\pgfqpoint{1.148750in}{1.548529in}}%
\pgfpathclose%
\pgfusepath{fill}%
\end{pgfscope}%
\begin{pgfscope}%
\pgfpathrectangle{\pgfqpoint{1.148750in}{0.550000in}}{\pgfqpoint{7.122250in}{3.850000in}}%
\pgfusepath{clip}%
\pgfsetbuttcap%
\pgfsetmiterjoin%
\definecolor{currentfill}{rgb}{0.248058,0.667205,0.350250}%
\pgfsetfillcolor{currentfill}%
\pgfsetlinewidth{0.000000pt}%
\definecolor{currentstroke}{rgb}{0.000000,0.000000,0.000000}%
\pgfsetstrokecolor{currentstroke}%
\pgfsetstrokeopacity{0.000000}%
\pgfsetdash{}{0pt}%
\pgfpathmoveto{\pgfqpoint{1.148750in}{1.342647in}}%
\pgfpathlineto{\pgfqpoint{1.709297in}{1.342647in}}%
\pgfpathlineto{\pgfqpoint{1.709297in}{1.136765in}}%
\pgfpathlineto{\pgfqpoint{1.148750in}{1.136765in}}%
\pgfpathclose%
\pgfusepath{fill}%
\end{pgfscope}%
\begin{pgfscope}%
\pgfpathrectangle{\pgfqpoint{1.148750in}{0.550000in}}{\pgfqpoint{7.122250in}{3.850000in}}%
\pgfusepath{clip}%
\pgfsetbuttcap%
\pgfsetmiterjoin%
\definecolor{currentfill}{rgb}{0.248058,0.667205,0.350250}%
\pgfsetfillcolor{currentfill}%
\pgfsetfillopacity{0.500000}%
\pgfsetlinewidth{0.000000pt}%
\definecolor{currentstroke}{rgb}{0.000000,0.000000,0.000000}%
\pgfsetstrokecolor{currentstroke}%
\pgfsetstrokeopacity{0.500000}%
\pgfsetdash{}{0pt}%
\pgfpathmoveto{\pgfqpoint{1.148750in}{0.930882in}}%
\pgfpathlineto{\pgfqpoint{1.511457in}{0.930882in}}%
\pgfpathlineto{\pgfqpoint{1.511457in}{0.725000in}}%
\pgfpathlineto{\pgfqpoint{1.148750in}{0.725000in}}%
\pgfpathclose%
\pgfusepath{fill}%
\end{pgfscope}%
\begin{pgfscope}%
\pgfpathrectangle{\pgfqpoint{1.148750in}{0.550000in}}{\pgfqpoint{7.122250in}{3.850000in}}%
\pgfusepath{clip}%
\pgfsetbuttcap%
\pgfsetmiterjoin%
\definecolor{currentfill}{rgb}{0.898885,0.305498,0.206767}%
\pgfsetfillcolor{currentfill}%
\pgfsetfillopacity{0.500000}%
\pgfsetlinewidth{0.000000pt}%
\definecolor{currentstroke}{rgb}{0.000000,0.000000,0.000000}%
\pgfsetstrokecolor{currentstroke}%
\pgfsetstrokeopacity{0.500000}%
\pgfsetdash{}{0pt}%
\pgfpathmoveto{\pgfqpoint{6.655304in}{4.225000in}}%
\pgfpathlineto{\pgfqpoint{8.271000in}{4.225000in}}%
\pgfpathlineto{\pgfqpoint{8.271000in}{4.019118in}}%
\pgfpathlineto{\pgfqpoint{6.655304in}{4.019118in}}%
\pgfpathclose%
\pgfusepath{fill}%
\end{pgfscope}%
\begin{pgfscope}%
\pgfpathrectangle{\pgfqpoint{1.148750in}{0.550000in}}{\pgfqpoint{7.122250in}{3.850000in}}%
\pgfusepath{clip}%
\pgfsetbuttcap%
\pgfsetmiterjoin%
\definecolor{currentfill}{rgb}{0.898885,0.305498,0.206767}%
\pgfsetfillcolor{currentfill}%
\pgfsetfillopacity{0.500000}%
\pgfsetlinewidth{0.000000pt}%
\definecolor{currentstroke}{rgb}{0.000000,0.000000,0.000000}%
\pgfsetstrokecolor{currentstroke}%
\pgfsetstrokeopacity{0.500000}%
\pgfsetdash{}{0pt}%
\pgfpathmoveto{\pgfqpoint{5.402316in}{3.813235in}}%
\pgfpathlineto{\pgfqpoint{8.271000in}{3.813235in}}%
\pgfpathlineto{\pgfqpoint{8.271000in}{3.607353in}}%
\pgfpathlineto{\pgfqpoint{5.402316in}{3.607353in}}%
\pgfpathclose%
\pgfusepath{fill}%
\end{pgfscope}%
\begin{pgfscope}%
\pgfpathrectangle{\pgfqpoint{1.148750in}{0.550000in}}{\pgfqpoint{7.122250in}{3.850000in}}%
\pgfusepath{clip}%
\pgfsetbuttcap%
\pgfsetmiterjoin%
\definecolor{currentfill}{rgb}{0.898885,0.305498,0.206767}%
\pgfsetfillcolor{currentfill}%
\pgfsetfillopacity{0.500000}%
\pgfsetlinewidth{0.000000pt}%
\definecolor{currentstroke}{rgb}{0.000000,0.000000,0.000000}%
\pgfsetstrokecolor{currentstroke}%
\pgfsetstrokeopacity{0.500000}%
\pgfsetdash{}{0pt}%
\pgfpathmoveto{\pgfqpoint{4.775822in}{3.401471in}}%
\pgfpathlineto{\pgfqpoint{8.271000in}{3.401471in}}%
\pgfpathlineto{\pgfqpoint{8.271000in}{3.195588in}}%
\pgfpathlineto{\pgfqpoint{4.775822in}{3.195588in}}%
\pgfpathclose%
\pgfusepath{fill}%
\end{pgfscope}%
\begin{pgfscope}%
\pgfpathrectangle{\pgfqpoint{1.148750in}{0.550000in}}{\pgfqpoint{7.122250in}{3.850000in}}%
\pgfusepath{clip}%
\pgfsetbuttcap%
\pgfsetmiterjoin%
\definecolor{currentfill}{rgb}{0.898885,0.305498,0.206767}%
\pgfsetfillcolor{currentfill}%
\pgfsetlinewidth{0.000000pt}%
\definecolor{currentstroke}{rgb}{0.000000,0.000000,0.000000}%
\pgfsetstrokecolor{currentstroke}%
\pgfsetstrokeopacity{0.000000}%
\pgfsetdash{}{0pt}%
\pgfpathmoveto{\pgfqpoint{3.786620in}{2.989706in}}%
\pgfpathlineto{\pgfqpoint{8.271000in}{2.989706in}}%
\pgfpathlineto{\pgfqpoint{8.271000in}{2.783824in}}%
\pgfpathlineto{\pgfqpoint{3.786620in}{2.783824in}}%
\pgfpathclose%
\pgfusepath{fill}%
\end{pgfscope}%
\begin{pgfscope}%
\pgfpathrectangle{\pgfqpoint{1.148750in}{0.550000in}}{\pgfqpoint{7.122250in}{3.850000in}}%
\pgfusepath{clip}%
\pgfsetbuttcap%
\pgfsetmiterjoin%
\definecolor{currentfill}{rgb}{0.898885,0.305498,0.206767}%
\pgfsetfillcolor{currentfill}%
\pgfsetlinewidth{0.000000pt}%
\definecolor{currentstroke}{rgb}{0.000000,0.000000,0.000000}%
\pgfsetstrokecolor{currentstroke}%
\pgfsetstrokeopacity{0.000000}%
\pgfsetdash{}{0pt}%
\pgfpathmoveto{\pgfqpoint{3.160126in}{2.577941in}}%
\pgfpathlineto{\pgfqpoint{8.271000in}{2.577941in}}%
\pgfpathlineto{\pgfqpoint{8.271000in}{2.372059in}}%
\pgfpathlineto{\pgfqpoint{3.160126in}{2.372059in}}%
\pgfpathclose%
\pgfusepath{fill}%
\end{pgfscope}%
\begin{pgfscope}%
\pgfpathrectangle{\pgfqpoint{1.148750in}{0.550000in}}{\pgfqpoint{7.122250in}{3.850000in}}%
\pgfusepath{clip}%
\pgfsetbuttcap%
\pgfsetmiterjoin%
\definecolor{currentfill}{rgb}{0.898885,0.305498,0.206767}%
\pgfsetfillcolor{currentfill}%
\pgfsetfillopacity{0.500000}%
\pgfsetlinewidth{0.000000pt}%
\definecolor{currentstroke}{rgb}{0.000000,0.000000,0.000000}%
\pgfsetstrokecolor{currentstroke}%
\pgfsetstrokeopacity{0.500000}%
\pgfsetdash{}{0pt}%
\pgfpathmoveto{\pgfqpoint{3.028233in}{2.166176in}}%
\pgfpathlineto{\pgfqpoint{8.271000in}{2.166176in}}%
\pgfpathlineto{\pgfqpoint{8.271000in}{1.960294in}}%
\pgfpathlineto{\pgfqpoint{3.028233in}{1.960294in}}%
\pgfpathclose%
\pgfusepath{fill}%
\end{pgfscope}%
\begin{pgfscope}%
\pgfpathrectangle{\pgfqpoint{1.148750in}{0.550000in}}{\pgfqpoint{7.122250in}{3.850000in}}%
\pgfusepath{clip}%
\pgfsetbuttcap%
\pgfsetmiterjoin%
\definecolor{currentfill}{rgb}{0.898885,0.305498,0.206767}%
\pgfsetfillcolor{currentfill}%
\pgfsetlinewidth{0.000000pt}%
\definecolor{currentstroke}{rgb}{0.000000,0.000000,0.000000}%
\pgfsetstrokecolor{currentstroke}%
\pgfsetstrokeopacity{0.000000}%
\pgfsetdash{}{0pt}%
\pgfpathmoveto{\pgfqpoint{2.467685in}{1.754412in}}%
\pgfpathlineto{\pgfqpoint{8.271000in}{1.754412in}}%
\pgfpathlineto{\pgfqpoint{8.271000in}{1.548529in}}%
\pgfpathlineto{\pgfqpoint{2.467685in}{1.548529in}}%
\pgfpathclose%
\pgfusepath{fill}%
\end{pgfscope}%
\begin{pgfscope}%
\pgfpathrectangle{\pgfqpoint{1.148750in}{0.550000in}}{\pgfqpoint{7.122250in}{3.850000in}}%
\pgfusepath{clip}%
\pgfsetbuttcap%
\pgfsetmiterjoin%
\definecolor{currentfill}{rgb}{0.898885,0.305498,0.206767}%
\pgfsetfillcolor{currentfill}%
\pgfsetlinewidth{0.000000pt}%
\definecolor{currentstroke}{rgb}{0.000000,0.000000,0.000000}%
\pgfsetstrokecolor{currentstroke}%
\pgfsetstrokeopacity{0.000000}%
\pgfsetdash{}{0pt}%
\pgfpathmoveto{\pgfqpoint{1.709297in}{1.342647in}}%
\pgfpathlineto{\pgfqpoint{8.271000in}{1.342647in}}%
\pgfpathlineto{\pgfqpoint{8.271000in}{1.136765in}}%
\pgfpathlineto{\pgfqpoint{1.709297in}{1.136765in}}%
\pgfpathclose%
\pgfusepath{fill}%
\end{pgfscope}%
\begin{pgfscope}%
\pgfpathrectangle{\pgfqpoint{1.148750in}{0.550000in}}{\pgfqpoint{7.122250in}{3.850000in}}%
\pgfusepath{clip}%
\pgfsetbuttcap%
\pgfsetmiterjoin%
\definecolor{currentfill}{rgb}{0.898885,0.305498,0.206767}%
\pgfsetfillcolor{currentfill}%
\pgfsetfillopacity{0.500000}%
\pgfsetlinewidth{0.000000pt}%
\definecolor{currentstroke}{rgb}{0.000000,0.000000,0.000000}%
\pgfsetstrokecolor{currentstroke}%
\pgfsetstrokeopacity{0.500000}%
\pgfsetdash{}{0pt}%
\pgfpathmoveto{\pgfqpoint{1.511457in}{0.930882in}}%
\pgfpathlineto{\pgfqpoint{8.271000in}{0.930882in}}%
\pgfpathlineto{\pgfqpoint{8.271000in}{0.725000in}}%
\pgfpathlineto{\pgfqpoint{1.511457in}{0.725000in}}%
\pgfpathclose%
\pgfusepath{fill}%
\end{pgfscope}%
\begin{pgfscope}%
\pgfsetbuttcap%
\pgfsetroundjoin%
\definecolor{currentfill}{rgb}{0.000000,0.000000,0.000000}%
\pgfsetfillcolor{currentfill}%
\pgfsetlinewidth{0.803000pt}%
\definecolor{currentstroke}{rgb}{0.000000,0.000000,0.000000}%
\pgfsetstrokecolor{currentstroke}%
\pgfsetdash{}{0pt}%
\pgfsys@defobject{currentmarker}{\pgfqpoint{-0.048611in}{0.000000in}}{\pgfqpoint{-0.000000in}{0.000000in}}{%
\pgfpathmoveto{\pgfqpoint{-0.000000in}{0.000000in}}%
\pgfpathlineto{\pgfqpoint{-0.048611in}{0.000000in}}%
\pgfusepath{stroke,fill}%
}%
\begin{pgfscope}%
\pgfsys@transformshift{1.148750in}{4.122059in}%
\pgfsys@useobject{currentmarker}{}%
\end{pgfscope}%
\end{pgfscope}%
\begin{pgfscope}%
\definecolor{textcolor}{rgb}{0.000000,0.000000,0.000000}%
\pgfsetstrokecolor{textcolor}%
\pgfsetfillcolor{textcolor}%
\pgftext[x=0.654731in, y=4.069297in, left, base]{\color{textcolor}\sffamily\fontsize{10.000000}{12.000000}\selectfont Pix3D}%
\end{pgfscope}%
\begin{pgfscope}%
\pgfsetbuttcap%
\pgfsetroundjoin%
\definecolor{currentfill}{rgb}{0.000000,0.000000,0.000000}%
\pgfsetfillcolor{currentfill}%
\pgfsetlinewidth{0.803000pt}%
\definecolor{currentstroke}{rgb}{0.000000,0.000000,0.000000}%
\pgfsetstrokecolor{currentstroke}%
\pgfsetdash{}{0pt}%
\pgfsys@defobject{currentmarker}{\pgfqpoint{-0.048611in}{0.000000in}}{\pgfqpoint{-0.000000in}{0.000000in}}{%
\pgfpathmoveto{\pgfqpoint{-0.000000in}{0.000000in}}%
\pgfpathlineto{\pgfqpoint{-0.048611in}{0.000000in}}%
\pgfusepath{stroke,fill}%
}%
\begin{pgfscope}%
\pgfsys@transformshift{1.148750in}{3.710294in}%
\pgfsys@useobject{currentmarker}{}%
\end{pgfscope}%
\end{pgfscope}%
\begin{pgfscope}%
\definecolor{textcolor}{rgb}{0.000000,0.000000,0.000000}%
\pgfsetstrokecolor{textcolor}%
\pgfsetfillcolor{textcolor}%
\pgftext[x=0.387940in, y=3.657533in, left, base]{\color{textcolor}\sffamily\fontsize{10.000000}{12.000000}\selectfont Hyperism}%
\end{pgfscope}%
\begin{pgfscope}%
\pgfsetbuttcap%
\pgfsetroundjoin%
\definecolor{currentfill}{rgb}{0.000000,0.000000,0.000000}%
\pgfsetfillcolor{currentfill}%
\pgfsetlinewidth{0.803000pt}%
\definecolor{currentstroke}{rgb}{0.000000,0.000000,0.000000}%
\pgfsetstrokecolor{currentstroke}%
\pgfsetdash{}{0pt}%
\pgfsys@defobject{currentmarker}{\pgfqpoint{-0.048611in}{0.000000in}}{\pgfqpoint{-0.000000in}{0.000000in}}{%
\pgfpathmoveto{\pgfqpoint{-0.000000in}{0.000000in}}%
\pgfpathlineto{\pgfqpoint{-0.048611in}{0.000000in}}%
\pgfusepath{stroke,fill}%
}%
\begin{pgfscope}%
\pgfsys@transformshift{1.148750in}{3.298529in}%
\pgfsys@useobject{currentmarker}{}%
\end{pgfscope}%
\end{pgfscope}%
\begin{pgfscope}%
\definecolor{textcolor}{rgb}{0.000000,0.000000,0.000000}%
\pgfsetstrokecolor{textcolor}%
\pgfsetfillcolor{textcolor}%
\pgftext[x=0.301067in, y=3.245768in, left, base]{\color{textcolor}\sffamily\fontsize{10.000000}{12.000000}\selectfont InteriorNet}%
\end{pgfscope}%
\begin{pgfscope}%
\pgfsetbuttcap%
\pgfsetroundjoin%
\definecolor{currentfill}{rgb}{0.000000,0.000000,0.000000}%
\pgfsetfillcolor{currentfill}%
\pgfsetlinewidth{0.803000pt}%
\definecolor{currentstroke}{rgb}{0.000000,0.000000,0.000000}%
\pgfsetstrokecolor{currentstroke}%
\pgfsetdash{}{0pt}%
\pgfsys@defobject{currentmarker}{\pgfqpoint{-0.048611in}{0.000000in}}{\pgfqpoint{-0.000000in}{0.000000in}}{%
\pgfpathmoveto{\pgfqpoint{-0.000000in}{0.000000in}}%
\pgfpathlineto{\pgfqpoint{-0.048611in}{0.000000in}}%
\pgfusepath{stroke,fill}%
}%
\begin{pgfscope}%
\pgfsys@transformshift{1.148750in}{2.886765in}%
\pgfsys@useobject{currentmarker}{}%
\end{pgfscope}%
\end{pgfscope}%
\begin{pgfscope}%
\definecolor{textcolor}{rgb}{0.000000,0.000000,0.000000}%
\pgfsetstrokecolor{textcolor}%
\pgfsetfillcolor{textcolor}%
\pgftext[x=0.212701in, y=2.834003in, left, base]{\color{textcolor}\sffamily\fontsize{10.000000}{12.000000}\selectfont OpenRooms}%
\end{pgfscope}%
\begin{pgfscope}%
\pgfsetbuttcap%
\pgfsetroundjoin%
\definecolor{currentfill}{rgb}{0.000000,0.000000,0.000000}%
\pgfsetfillcolor{currentfill}%
\pgfsetlinewidth{0.803000pt}%
\definecolor{currentstroke}{rgb}{0.000000,0.000000,0.000000}%
\pgfsetstrokecolor{currentstroke}%
\pgfsetdash{}{0pt}%
\pgfsys@defobject{currentmarker}{\pgfqpoint{-0.048611in}{0.000000in}}{\pgfqpoint{-0.000000in}{0.000000in}}{%
\pgfpathmoveto{\pgfqpoint{-0.000000in}{0.000000in}}%
\pgfpathlineto{\pgfqpoint{-0.048611in}{0.000000in}}%
\pgfusepath{stroke,fill}%
}%
\begin{pgfscope}%
\pgfsys@transformshift{1.148750in}{2.475000in}%
\pgfsys@useobject{currentmarker}{}%
\end{pgfscope}%
\end{pgfscope}%
\begin{pgfscope}%
\definecolor{textcolor}{rgb}{0.000000,0.000000,0.000000}%
\pgfsetstrokecolor{textcolor}%
\pgfsetfillcolor{textcolor}%
\pgftext[x=0.209921in, y=2.422238in, left, base]{\color{textcolor}\sffamily\fontsize{10.000000}{12.000000}\selectfont Blenderproc}%
\end{pgfscope}%
\begin{pgfscope}%
\pgfsetbuttcap%
\pgfsetroundjoin%
\definecolor{currentfill}{rgb}{0.000000,0.000000,0.000000}%
\pgfsetfillcolor{currentfill}%
\pgfsetlinewidth{0.803000pt}%
\definecolor{currentstroke}{rgb}{0.000000,0.000000,0.000000}%
\pgfsetstrokecolor{currentstroke}%
\pgfsetdash{}{0pt}%
\pgfsys@defobject{currentmarker}{\pgfqpoint{-0.048611in}{0.000000in}}{\pgfqpoint{-0.000000in}{0.000000in}}{%
\pgfpathmoveto{\pgfqpoint{-0.000000in}{0.000000in}}%
\pgfpathlineto{\pgfqpoint{-0.048611in}{0.000000in}}%
\pgfusepath{stroke,fill}%
}%
\begin{pgfscope}%
\pgfsys@transformshift{1.148750in}{2.063235in}%
\pgfsys@useobject{currentmarker}{}%
\end{pgfscope}%
\end{pgfscope}%
\begin{pgfscope}%
\definecolor{textcolor}{rgb}{0.000000,0.000000,0.000000}%
\pgfsetstrokecolor{textcolor}%
\pgfsetfillcolor{textcolor}%
\pgftext[x=0.381769in, y=2.010474in, left, base]{\color{textcolor}\sffamily\fontsize{10.000000}{12.000000}\selectfont 3DFRONT}%
\end{pgfscope}%
\begin{pgfscope}%
\pgfsetbuttcap%
\pgfsetroundjoin%
\definecolor{currentfill}{rgb}{0.000000,0.000000,0.000000}%
\pgfsetfillcolor{currentfill}%
\pgfsetlinewidth{0.803000pt}%
\definecolor{currentstroke}{rgb}{0.000000,0.000000,0.000000}%
\pgfsetstrokecolor{currentstroke}%
\pgfsetdash{}{0pt}%
\pgfsys@defobject{currentmarker}{\pgfqpoint{-0.048611in}{0.000000in}}{\pgfqpoint{-0.000000in}{0.000000in}}{%
\pgfpathmoveto{\pgfqpoint{-0.000000in}{0.000000in}}%
\pgfpathlineto{\pgfqpoint{-0.048611in}{0.000000in}}%
\pgfusepath{stroke,fill}%
}%
\begin{pgfscope}%
\pgfsys@transformshift{1.148750in}{1.651471in}%
\pgfsys@useobject{currentmarker}{}%
\end{pgfscope}%
\end{pgfscope}%
\begin{pgfscope}%
\definecolor{textcolor}{rgb}{0.000000,0.000000,0.000000}%
\pgfsetstrokecolor{textcolor}%
\pgfsetfillcolor{textcolor}%
\pgftext[x=0.384278in, y=1.598709in, left, base]{\color{textcolor}\sffamily\fontsize{10.000000}{12.000000}\selectfont SceneNet}%
\end{pgfscope}%
\begin{pgfscope}%
\pgfsetbuttcap%
\pgfsetroundjoin%
\definecolor{currentfill}{rgb}{0.000000,0.000000,0.000000}%
\pgfsetfillcolor{currentfill}%
\pgfsetlinewidth{0.803000pt}%
\definecolor{currentstroke}{rgb}{0.000000,0.000000,0.000000}%
\pgfsetstrokecolor{currentstroke}%
\pgfsetdash{}{0pt}%
\pgfsys@defobject{currentmarker}{\pgfqpoint{-0.048611in}{0.000000in}}{\pgfqpoint{-0.000000in}{0.000000in}}{%
\pgfpathmoveto{\pgfqpoint{-0.000000in}{0.000000in}}%
\pgfpathlineto{\pgfqpoint{-0.048611in}{0.000000in}}%
\pgfusepath{stroke,fill}%
}%
\begin{pgfscope}%
\pgfsys@transformshift{1.148750in}{1.239706in}%
\pgfsys@useobject{currentmarker}{}%
\end{pgfscope}%
\end{pgfscope}%
\begin{pgfscope}%
\definecolor{textcolor}{rgb}{0.000000,0.000000,0.000000}%
\pgfsetstrokecolor{textcolor}%
\pgfsetfillcolor{textcolor}%
\pgftext[x=0.188762in, y=1.186944in, left, base]{\color{textcolor}\sffamily\fontsize{10.000000}{12.000000}\selectfont S2R:3DFREE}%
\end{pgfscope}%
\begin{pgfscope}%
\pgfsetbuttcap%
\pgfsetroundjoin%
\definecolor{currentfill}{rgb}{0.000000,0.000000,0.000000}%
\pgfsetfillcolor{currentfill}%
\pgfsetlinewidth{0.803000pt}%
\definecolor{currentstroke}{rgb}{0.000000,0.000000,0.000000}%
\pgfsetstrokecolor{currentstroke}%
\pgfsetdash{}{0pt}%
\pgfsys@defobject{currentmarker}{\pgfqpoint{-0.048611in}{0.000000in}}{\pgfqpoint{-0.000000in}{0.000000in}}{%
\pgfpathmoveto{\pgfqpoint{-0.000000in}{0.000000in}}%
\pgfpathlineto{\pgfqpoint{-0.048611in}{0.000000in}}%
\pgfusepath{stroke,fill}%
}%
\begin{pgfscope}%
\pgfsys@transformshift{1.148750in}{0.827941in}%
\pgfsys@useobject{currentmarker}{}%
\end{pgfscope}%
\end{pgfscope}%
\begin{pgfscope}%
\definecolor{textcolor}{rgb}{0.000000,0.000000,0.000000}%
\pgfsetstrokecolor{textcolor}%
\pgfsetfillcolor{textcolor}%
\pgftext[x=0.432089in, y=0.775180in, left, base]{\color{textcolor}\sffamily\fontsize{10.000000}{12.000000}\selectfont AI2THOR}%
\end{pgfscope}%
\begin{pgfscope}%
\definecolor{textcolor}{rgb}{0.000000,0.000000,0.000000}%
\pgfsetstrokecolor{textcolor}%
\pgfsetfillcolor{textcolor}%
\pgftext[x=0.133206in,y=2.475000in,,bottom,rotate=90.000000]{\color{textcolor}\sffamily\fontsize{10.000000}{12.000000}\selectfont Datasets}%
\end{pgfscope}%
\begin{pgfscope}%
\pgfpathrectangle{\pgfqpoint{1.148750in}{0.550000in}}{\pgfqpoint{7.122250in}{3.850000in}}%
\pgfusepath{clip}%
\pgfsetbuttcap%
\pgfsetroundjoin%
\pgfsetlinewidth{1.505625pt}%
\definecolor{currentstroke}{rgb}{0.000000,0.000000,0.000000}%
\pgfsetstrokecolor{currentstroke}%
\pgfsetstrokeopacity{0.200000}%
\pgfsetdash{{5.550000pt}{2.400000pt}}{0.000000pt}%
\pgfpathmoveto{\pgfqpoint{4.709875in}{0.550000in}}%
\pgfpathlineto{\pgfqpoint{4.709875in}{4.400000in}}%
\pgfusepath{stroke}%
\end{pgfscope}%
\begin{pgfscope}%
\pgfsetrectcap%
\pgfsetmiterjoin%
\pgfsetlinewidth{0.803000pt}%
\definecolor{currentstroke}{rgb}{0.000000,0.000000,0.000000}%
\pgfsetstrokecolor{currentstroke}%
\pgfsetdash{}{0pt}%
\pgfpathmoveto{\pgfqpoint{1.148750in}{0.550000in}}%
\pgfpathlineto{\pgfqpoint{1.148750in}{4.400000in}}%
\pgfusepath{stroke}%
\end{pgfscope}%
\begin{pgfscope}%
\pgfsetrectcap%
\pgfsetmiterjoin%
\pgfsetlinewidth{0.803000pt}%
\definecolor{currentstroke}{rgb}{0.000000,0.000000,0.000000}%
\pgfsetstrokecolor{currentstroke}%
\pgfsetdash{}{0pt}%
\pgfpathmoveto{\pgfqpoint{8.271000in}{0.550000in}}%
\pgfpathlineto{\pgfqpoint{8.271000in}{4.400000in}}%
\pgfusepath{stroke}%
\end{pgfscope}%
\begin{pgfscope}%
\pgfsetrectcap%
\pgfsetmiterjoin%
\pgfsetlinewidth{0.803000pt}%
\definecolor{currentstroke}{rgb}{0.000000,0.000000,0.000000}%
\pgfsetstrokecolor{currentstroke}%
\pgfsetdash{}{0pt}%
\pgfpathmoveto{\pgfqpoint{1.148750in}{0.550000in}}%
\pgfpathlineto{\pgfqpoint{8.271000in}{0.550000in}}%
\pgfusepath{stroke}%
\end{pgfscope}%
\begin{pgfscope}%
\pgfsetrectcap%
\pgfsetmiterjoin%
\pgfsetlinewidth{0.803000pt}%
\definecolor{currentstroke}{rgb}{0.000000,0.000000,0.000000}%
\pgfsetstrokecolor{currentstroke}%
\pgfsetdash{}{0pt}%
\pgfpathmoveto{\pgfqpoint{1.148750in}{4.400000in}}%
\pgfpathlineto{\pgfqpoint{8.271000in}{4.400000in}}%
\pgfusepath{stroke}%
\end{pgfscope}%
\begin{pgfscope}%
\definecolor{textcolor}{rgb}{1.000000,1.000000,1.000000}%
\pgfsetstrokecolor{textcolor}%
\pgfsetfillcolor{textcolor}%
\pgftext[x=3.902027in,y=4.122059in,,]{\color{textcolor}\sffamily\fontsize{10.000000}{12.000000}\selectfont 167}%
\end{pgfscope}%
\begin{pgfscope}%
\definecolor{textcolor}{rgb}{1.000000,1.000000,1.000000}%
\pgfsetstrokecolor{textcolor}%
\pgfsetfillcolor{textcolor}%
\pgftext[x=3.275533in,y=3.710294in,,]{\color{textcolor}\sffamily\fontsize{10.000000}{12.000000}\selectfont 129}%
\end{pgfscope}%
\begin{pgfscope}%
\definecolor{textcolor}{rgb}{1.000000,1.000000,1.000000}%
\pgfsetstrokecolor{textcolor}%
\pgfsetfillcolor{textcolor}%
\pgftext[x=2.962286in,y=3.298529in,,]{\color{textcolor}\sffamily\fontsize{10.000000}{12.000000}\selectfont 110}%
\end{pgfscope}%
\begin{pgfscope}%
\definecolor{textcolor}{rgb}{1.000000,1.000000,1.000000}%
\pgfsetstrokecolor{textcolor}%
\pgfsetfillcolor{textcolor}%
\pgftext[x=2.467685in,y=2.886765in,,]{\color{textcolor}\sffamily\fontsize{10.000000}{12.000000}\selectfont 80}%
\end{pgfscope}%
\begin{pgfscope}%
\definecolor{textcolor}{rgb}{1.000000,1.000000,1.000000}%
\pgfsetstrokecolor{textcolor}%
\pgfsetfillcolor{textcolor}%
\pgftext[x=2.154438in,y=2.475000in,,]{\color{textcolor}\sffamily\fontsize{10.000000}{12.000000}\selectfont 61}%
\end{pgfscope}%
\begin{pgfscope}%
\definecolor{textcolor}{rgb}{1.000000,1.000000,1.000000}%
\pgfsetstrokecolor{textcolor}%
\pgfsetfillcolor{textcolor}%
\pgftext[x=2.088491in,y=2.063235in,,]{\color{textcolor}\sffamily\fontsize{10.000000}{12.000000}\selectfont 57}%
\end{pgfscope}%
\begin{pgfscope}%
\definecolor{textcolor}{rgb}{1.000000,1.000000,1.000000}%
\pgfsetstrokecolor{textcolor}%
\pgfsetfillcolor{textcolor}%
\pgftext[x=1.808218in,y=1.651471in,,]{\color{textcolor}\sffamily\fontsize{10.000000}{12.000000}\selectfont 40}%
\end{pgfscope}%
\begin{pgfscope}%
\definecolor{textcolor}{rgb}{1.000000,1.000000,1.000000}%
\pgfsetstrokecolor{textcolor}%
\pgfsetfillcolor{textcolor}%
\pgftext[x=1.429024in,y=1.239706in,,]{\color{textcolor}\sffamily\fontsize{10.000000}{12.000000}\selectfont 17}%
\end{pgfscope}%
\begin{pgfscope}%
\definecolor{textcolor}{rgb}{1.000000,1.000000,1.000000}%
\pgfsetstrokecolor{textcolor}%
\pgfsetfillcolor{textcolor}%
\pgftext[x=1.330104in,y=0.827941in,,]{\color{textcolor}\sffamily\fontsize{10.000000}{12.000000}\selectfont 11}%
\end{pgfscope}%
\begin{pgfscope}%
\definecolor{textcolor}{rgb}{1.000000,1.000000,1.000000}%
\pgfsetstrokecolor{textcolor}%
\pgfsetfillcolor{textcolor}%
\pgftext[x=7.463152in,y=4.122059in,,]{\color{textcolor}\sffamily\fontsize{10.000000}{12.000000}\selectfont 49}%
\end{pgfscope}%
\begin{pgfscope}%
\definecolor{textcolor}{rgb}{1.000000,1.000000,1.000000}%
\pgfsetstrokecolor{textcolor}%
\pgfsetfillcolor{textcolor}%
\pgftext[x=6.836658in,y=3.710294in,,]{\color{textcolor}\sffamily\fontsize{10.000000}{12.000000}\selectfont 87}%
\end{pgfscope}%
\begin{pgfscope}%
\definecolor{textcolor}{rgb}{1.000000,1.000000,1.000000}%
\pgfsetstrokecolor{textcolor}%
\pgfsetfillcolor{textcolor}%
\pgftext[x=6.523411in,y=3.298529in,,]{\color{textcolor}\sffamily\fontsize{10.000000}{12.000000}\selectfont 106}%
\end{pgfscope}%
\begin{pgfscope}%
\definecolor{textcolor}{rgb}{1.000000,1.000000,1.000000}%
\pgfsetstrokecolor{textcolor}%
\pgfsetfillcolor{textcolor}%
\pgftext[x=6.028810in,y=2.886765in,,]{\color{textcolor}\sffamily\fontsize{10.000000}{12.000000}\selectfont 136}%
\end{pgfscope}%
\begin{pgfscope}%
\definecolor{textcolor}{rgb}{1.000000,1.000000,1.000000}%
\pgfsetstrokecolor{textcolor}%
\pgfsetfillcolor{textcolor}%
\pgftext[x=5.715563in,y=2.475000in,,]{\color{textcolor}\sffamily\fontsize{10.000000}{12.000000}\selectfont 155}%
\end{pgfscope}%
\begin{pgfscope}%
\definecolor{textcolor}{rgb}{1.000000,1.000000,1.000000}%
\pgfsetstrokecolor{textcolor}%
\pgfsetfillcolor{textcolor}%
\pgftext[x=5.649616in,y=2.063235in,,]{\color{textcolor}\sffamily\fontsize{10.000000}{12.000000}\selectfont 159}%
\end{pgfscope}%
\begin{pgfscope}%
\definecolor{textcolor}{rgb}{1.000000,1.000000,1.000000}%
\pgfsetstrokecolor{textcolor}%
\pgfsetfillcolor{textcolor}%
\pgftext[x=5.369343in,y=1.651471in,,]{\color{textcolor}\sffamily\fontsize{10.000000}{12.000000}\selectfont 176}%
\end{pgfscope}%
\begin{pgfscope}%
\definecolor{textcolor}{rgb}{1.000000,1.000000,1.000000}%
\pgfsetstrokecolor{textcolor}%
\pgfsetfillcolor{textcolor}%
\pgftext[x=4.990149in,y=1.239706in,,]{\color{textcolor}\sffamily\fontsize{10.000000}{12.000000}\selectfont 199}%
\end{pgfscope}%
\begin{pgfscope}%
\definecolor{textcolor}{rgb}{1.000000,1.000000,1.000000}%
\pgfsetstrokecolor{textcolor}%
\pgfsetfillcolor{textcolor}%
\pgftext[x=4.891229in,y=0.827941in,,]{\color{textcolor}\sffamily\fontsize{10.000000}{12.000000}\selectfont 205}%
\end{pgfscope}%
\begin{pgfscope}%
\pgfsetbuttcap%
\pgfsetmiterjoin%
\definecolor{currentfill}{rgb}{1.000000,1.000000,1.000000}%
\pgfsetfillcolor{currentfill}%
\pgfsetfillopacity{0.800000}%
\pgfsetlinewidth{1.003750pt}%
\definecolor{currentstroke}{rgb}{0.800000,0.800000,0.800000}%
\pgfsetstrokecolor{currentstroke}%
\pgfsetstrokeopacity{0.800000}%
\pgfsetdash{}{0pt}%
\pgfpathmoveto{\pgfqpoint{1.229736in}{4.457847in}}%
\pgfpathlineto{\pgfqpoint{2.893579in}{4.457847in}}%
\pgfpathquadraticcurveto{\pgfqpoint{2.916717in}{4.457847in}}{\pgfqpoint{2.916717in}{4.480986in}}%
\pgfpathlineto{\pgfqpoint{2.916717in}{4.639230in}}%
\pgfpathquadraticcurveto{\pgfqpoint{2.916717in}{4.662369in}}{\pgfqpoint{2.893579in}{4.662369in}}%
\pgfpathlineto{\pgfqpoint{1.229736in}{4.662369in}}%
\pgfpathquadraticcurveto{\pgfqpoint{1.206597in}{4.662369in}}{\pgfqpoint{1.206597in}{4.639230in}}%
\pgfpathlineto{\pgfqpoint{1.206597in}{4.480986in}}%
\pgfpathquadraticcurveto{\pgfqpoint{1.206597in}{4.457847in}}{\pgfqpoint{1.229736in}{4.457847in}}%
\pgfpathclose%
\pgfusepath{stroke,fill}%
\end{pgfscope}%
\begin{pgfscope}%
\pgfsetbuttcap%
\pgfsetmiterjoin%
\definecolor{currentfill}{rgb}{0.248058,0.667205,0.350250}%
\pgfsetfillcolor{currentfill}%
\pgfsetfillopacity{0.500000}%
\pgfsetlinewidth{0.000000pt}%
\definecolor{currentstroke}{rgb}{0.000000,0.000000,0.000000}%
\pgfsetstrokecolor{currentstroke}%
\pgfsetstrokeopacity{0.500000}%
\pgfsetdash{}{0pt}%
\pgfpathmoveto{\pgfqpoint{1.252875in}{4.528190in}}%
\pgfpathlineto{\pgfqpoint{1.484264in}{4.528190in}}%
\pgfpathlineto{\pgfqpoint{1.484264in}{4.609176in}}%
\pgfpathlineto{\pgfqpoint{1.252875in}{4.609176in}}%
\pgfpathclose%
\pgfusepath{fill}%
\end{pgfscope}%
\begin{pgfscope}%
\definecolor{textcolor}{rgb}{0.000000,0.000000,0.000000}%
\pgfsetstrokecolor{textcolor}%
\pgfsetfillcolor{textcolor}%
\pgftext[x=1.576819in,y=4.528190in,left,base]{\color{textcolor}\sffamily\fontsize{8.330000}{9.996000}\selectfont Real}%
\end{pgfscope}%
\begin{pgfscope}%
\pgfsetbuttcap%
\pgfsetmiterjoin%
\definecolor{currentfill}{rgb}{0.898885,0.305498,0.206767}%
\pgfsetfillcolor{currentfill}%
\pgfsetfillopacity{0.500000}%
\pgfsetlinewidth{0.000000pt}%
\definecolor{currentstroke}{rgb}{0.000000,0.000000,0.000000}%
\pgfsetstrokecolor{currentstroke}%
\pgfsetstrokeopacity{0.500000}%
\pgfsetdash{}{0pt}%
\pgfpathmoveto{\pgfqpoint{2.057618in}{4.528190in}}%
\pgfpathlineto{\pgfqpoint{2.289007in}{4.528190in}}%
\pgfpathlineto{\pgfqpoint{2.289007in}{4.609176in}}%
\pgfpathlineto{\pgfqpoint{2.057618in}{4.609176in}}%
\pgfpathclose%
\pgfusepath{fill}%
\end{pgfscope}%
\begin{pgfscope}%
\definecolor{textcolor}{rgb}{0.000000,0.000000,0.000000}%
\pgfsetstrokecolor{textcolor}%
\pgfsetfillcolor{textcolor}%
\pgftext[x=2.381562in,y=4.528190in,left,base]{\color{textcolor}\sffamily\fontsize{8.330000}{9.996000}\selectfont Not Real}%
\end{pgfscope}%
\end{pgfpicture}%
\makeatother%
\endgroup%
}
    \caption{The figure represents distribution for Section 1 of survey. The participants were asked to distinguish if the image was 'Real' or 'Nor Real'.}
    \label{fig:question1}
\end{figure}

\subsubsection{Section 2: Likert Scale}
In section 2, the participants could select ratings from 1 to 10 (1 being the least photorealistic).
The distribution of values for each dataset can be seen in figure~\ref{fig:question2} and the average ratings are as seen in figure~\ref{fig:question2_2}.
If we consider the scale 1, which is least rating that can be given to the image, AI2THOR has most number of votes.
Suppose we have a cut off at scale 2 and 3, Openrooms and \gls{free} are the least photorealistic respectively.
Interestingly Interiornet has least number of scale 1 instead of Pix3D which is the real dataset.
However, Pix3D has the highest number of perfect score(10) among all the datasets.
Coming to the averages, we again see the datasets created from automated pipeline (Blenderproc, SceneNet, Openrooms,\gls{free}), have least average ratings,
while the manually created datasets(Hyperism, \gls{front}, InteriorNet) have higher average ratings.
Pix3D has highest of the average ratings closely followed by InteriorNet.
Even though \gls{free} has the least average, it is still comparable to other automated pipelines and even Unity based AI2THOR dataset.

\begin{figure}
    \centering
    \resizebox{\textwidth}{!}{%% Creator: Matplotlib, PGF backend
%%
%% To include the figure in your LaTeX document, write
%%   \input{<filename>.pgf}
%%
%% Make sure the required packages are loaded in your preamble
%%   \usepackage{pgf}
%%
%% Figures using additional raster images can only be included by \input if
%% they are in the same directory as the main LaTeX file. For loading figures
%% from other directories you can use the `import` package
%%   \usepackage{import}
%%
%% and then include the figures with
%%   \import{<path to file>}{<filename>.pgf}
%%
%% Matplotlib used the following preamble
%%   \usepackage{fontspec}
%%   \setmainfont{DejaVuSerif.ttf}[Path=\detokenize{/Users/apple/opt/anaconda3/envs/kaolin/lib/python3.7/site-packages/matplotlib/mpl-data/fonts/ttf/}]
%%   \setsansfont{DejaVuSans.ttf}[Path=\detokenize{/Users/apple/opt/anaconda3/envs/kaolin/lib/python3.7/site-packages/matplotlib/mpl-data/fonts/ttf/}]
%%   \setmonofont{DejaVuSansMono.ttf}[Path=\detokenize{/Users/apple/opt/anaconda3/envs/kaolin/lib/python3.7/site-packages/matplotlib/mpl-data/fonts/ttf/}]
%%
\begingroup%
\makeatletter%
\begin{pgfpicture}%
\pgfpathrectangle{\pgfpointorigin}{\pgfqpoint{8.472206in}{4.360980in}}%
\pgfusepath{use as bounding box, clip}%
\begin{pgfscope}%
\pgfsetbuttcap%
\pgfsetmiterjoin%
\definecolor{currentfill}{rgb}{1.000000,1.000000,1.000000}%
\pgfsetfillcolor{currentfill}%
\pgfsetlinewidth{0.000000pt}%
\definecolor{currentstroke}{rgb}{1.000000,1.000000,1.000000}%
\pgfsetstrokecolor{currentstroke}%
\pgfsetdash{}{0pt}%
\pgfpathmoveto{\pgfqpoint{0.000000in}{0.000000in}}%
\pgfpathlineto{\pgfqpoint{8.472206in}{0.000000in}}%
\pgfpathlineto{\pgfqpoint{8.472206in}{4.360980in}}%
\pgfpathlineto{\pgfqpoint{0.000000in}{4.360980in}}%
\pgfpathclose%
\pgfusepath{fill}%
\end{pgfscope}%
\begin{pgfscope}%
\pgfsetbuttcap%
\pgfsetmiterjoin%
\definecolor{currentfill}{rgb}{1.000000,1.000000,1.000000}%
\pgfsetfillcolor{currentfill}%
\pgfsetlinewidth{0.000000pt}%
\definecolor{currentstroke}{rgb}{0.000000,0.000000,0.000000}%
\pgfsetstrokecolor{currentstroke}%
\pgfsetstrokeopacity{0.000000}%
\pgfsetdash{}{0pt}%
\pgfpathmoveto{\pgfqpoint{1.249956in}{0.148611in}}%
\pgfpathlineto{\pgfqpoint{8.372206in}{0.148611in}}%
\pgfpathlineto{\pgfqpoint{8.372206in}{3.998611in}}%
\pgfpathlineto{\pgfqpoint{1.249956in}{3.998611in}}%
\pgfpathclose%
\pgfusepath{fill}%
\end{pgfscope}%
\begin{pgfscope}%
\pgfpathrectangle{\pgfqpoint{1.249956in}{0.148611in}}{\pgfqpoint{7.122250in}{3.850000in}}%
\pgfusepath{clip}%
\pgfsetbuttcap%
\pgfsetmiterjoin%
\definecolor{currentfill}{rgb}{0.898885,0.305498,0.206767}%
\pgfsetfillcolor{currentfill}%
\pgfsetfillopacity{0.500000}%
\pgfsetlinewidth{0.000000pt}%
\definecolor{currentstroke}{rgb}{0.000000,0.000000,0.000000}%
\pgfsetstrokecolor{currentstroke}%
\pgfsetstrokeopacity{0.500000}%
\pgfsetdash{}{0pt}%
\pgfpathmoveto{\pgfqpoint{1.249956in}{3.823611in}}%
\pgfpathlineto{\pgfqpoint{2.502945in}{3.823611in}}%
\pgfpathlineto{\pgfqpoint{2.502945in}{3.617729in}}%
\pgfpathlineto{\pgfqpoint{1.249956in}{3.617729in}}%
\pgfpathclose%
\pgfusepath{fill}%
\end{pgfscope}%
\begin{pgfscope}%
\pgfpathrectangle{\pgfqpoint{1.249956in}{0.148611in}}{\pgfqpoint{7.122250in}{3.850000in}}%
\pgfusepath{clip}%
\pgfsetbuttcap%
\pgfsetmiterjoin%
\definecolor{currentfill}{rgb}{0.898885,0.305498,0.206767}%
\pgfsetfillcolor{currentfill}%
\pgfsetfillopacity{0.500000}%
\pgfsetlinewidth{0.000000pt}%
\definecolor{currentstroke}{rgb}{0.000000,0.000000,0.000000}%
\pgfsetstrokecolor{currentstroke}%
\pgfsetstrokeopacity{0.500000}%
\pgfsetdash{}{0pt}%
\pgfpathmoveto{\pgfqpoint{1.249956in}{3.411846in}}%
\pgfpathlineto{\pgfqpoint{3.294306in}{3.411846in}}%
\pgfpathlineto{\pgfqpoint{3.294306in}{3.205964in}}%
\pgfpathlineto{\pgfqpoint{1.249956in}{3.205964in}}%
\pgfpathclose%
\pgfusepath{fill}%
\end{pgfscope}%
\begin{pgfscope}%
\pgfpathrectangle{\pgfqpoint{1.249956in}{0.148611in}}{\pgfqpoint{7.122250in}{3.850000in}}%
\pgfusepath{clip}%
\pgfsetbuttcap%
\pgfsetmiterjoin%
\definecolor{currentfill}{rgb}{0.898885,0.305498,0.206767}%
\pgfsetfillcolor{currentfill}%
\pgfsetlinewidth{0.000000pt}%
\definecolor{currentstroke}{rgb}{0.000000,0.000000,0.000000}%
\pgfsetstrokecolor{currentstroke}%
\pgfsetstrokeopacity{0.000000}%
\pgfsetdash{}{0pt}%
\pgfpathmoveto{\pgfqpoint{1.249956in}{3.000082in}}%
\pgfpathlineto{\pgfqpoint{2.700785in}{3.000082in}}%
\pgfpathlineto{\pgfqpoint{2.700785in}{2.794199in}}%
\pgfpathlineto{\pgfqpoint{1.249956in}{2.794199in}}%
\pgfpathclose%
\pgfusepath{fill}%
\end{pgfscope}%
\begin{pgfscope}%
\pgfpathrectangle{\pgfqpoint{1.249956in}{0.148611in}}{\pgfqpoint{7.122250in}{3.850000in}}%
\pgfusepath{clip}%
\pgfsetbuttcap%
\pgfsetmiterjoin%
\definecolor{currentfill}{rgb}{0.898885,0.305498,0.206767}%
\pgfsetfillcolor{currentfill}%
\pgfsetfillopacity{0.500000}%
\pgfsetlinewidth{0.000000pt}%
\definecolor{currentstroke}{rgb}{0.000000,0.000000,0.000000}%
\pgfsetstrokecolor{currentstroke}%
\pgfsetstrokeopacity{0.500000}%
\pgfsetdash{}{0pt}%
\pgfpathmoveto{\pgfqpoint{1.249956in}{2.588317in}}%
\pgfpathlineto{\pgfqpoint{1.843477in}{2.588317in}}%
\pgfpathlineto{\pgfqpoint{1.843477in}{2.382435in}}%
\pgfpathlineto{\pgfqpoint{1.249956in}{2.382435in}}%
\pgfpathclose%
\pgfusepath{fill}%
\end{pgfscope}%
\begin{pgfscope}%
\pgfpathrectangle{\pgfqpoint{1.249956in}{0.148611in}}{\pgfqpoint{7.122250in}{3.850000in}}%
\pgfusepath{clip}%
\pgfsetbuttcap%
\pgfsetmiterjoin%
\definecolor{currentfill}{rgb}{0.898885,0.305498,0.206767}%
\pgfsetfillcolor{currentfill}%
\pgfsetfillopacity{0.500000}%
\pgfsetlinewidth{0.000000pt}%
\definecolor{currentstroke}{rgb}{0.000000,0.000000,0.000000}%
\pgfsetstrokecolor{currentstroke}%
\pgfsetstrokeopacity{0.500000}%
\pgfsetdash{}{0pt}%
\pgfpathmoveto{\pgfqpoint{1.249956in}{2.176552in}}%
\pgfpathlineto{\pgfqpoint{1.612664in}{2.176552in}}%
\pgfpathlineto{\pgfqpoint{1.612664in}{1.970670in}}%
\pgfpathlineto{\pgfqpoint{1.249956in}{1.970670in}}%
\pgfpathclose%
\pgfusepath{fill}%
\end{pgfscope}%
\begin{pgfscope}%
\pgfpathrectangle{\pgfqpoint{1.249956in}{0.148611in}}{\pgfqpoint{7.122250in}{3.850000in}}%
\pgfusepath{clip}%
\pgfsetbuttcap%
\pgfsetmiterjoin%
\definecolor{currentfill}{rgb}{0.898885,0.305498,0.206767}%
\pgfsetfillcolor{currentfill}%
\pgfsetlinewidth{0.000000pt}%
\definecolor{currentstroke}{rgb}{0.000000,0.000000,0.000000}%
\pgfsetstrokecolor{currentstroke}%
\pgfsetstrokeopacity{0.000000}%
\pgfsetdash{}{0pt}%
\pgfpathmoveto{\pgfqpoint{1.249956in}{1.764788in}}%
\pgfpathlineto{\pgfqpoint{3.096466in}{1.764788in}}%
\pgfpathlineto{\pgfqpoint{3.096466in}{1.558905in}}%
\pgfpathlineto{\pgfqpoint{1.249956in}{1.558905in}}%
\pgfpathclose%
\pgfusepath{fill}%
\end{pgfscope}%
\begin{pgfscope}%
\pgfpathrectangle{\pgfqpoint{1.249956in}{0.148611in}}{\pgfqpoint{7.122250in}{3.850000in}}%
\pgfusepath{clip}%
\pgfsetbuttcap%
\pgfsetmiterjoin%
\definecolor{currentfill}{rgb}{0.898885,0.305498,0.206767}%
\pgfsetfillcolor{currentfill}%
\pgfsetfillopacity{0.500000}%
\pgfsetlinewidth{0.000000pt}%
\definecolor{currentstroke}{rgb}{0.000000,0.000000,0.000000}%
\pgfsetstrokecolor{currentstroke}%
\pgfsetstrokeopacity{0.500000}%
\pgfsetdash{}{0pt}%
\pgfpathmoveto{\pgfqpoint{1.249956in}{1.353023in}}%
\pgfpathlineto{\pgfqpoint{1.678610in}{1.353023in}}%
\pgfpathlineto{\pgfqpoint{1.678610in}{1.147141in}}%
\pgfpathlineto{\pgfqpoint{1.249956in}{1.147141in}}%
\pgfpathclose%
\pgfusepath{fill}%
\end{pgfscope}%
\begin{pgfscope}%
\pgfpathrectangle{\pgfqpoint{1.249956in}{0.148611in}}{\pgfqpoint{7.122250in}{3.850000in}}%
\pgfusepath{clip}%
\pgfsetbuttcap%
\pgfsetmiterjoin%
\definecolor{currentfill}{rgb}{0.898885,0.305498,0.206767}%
\pgfsetfillcolor{currentfill}%
\pgfsetlinewidth{0.000000pt}%
\definecolor{currentstroke}{rgb}{0.000000,0.000000,0.000000}%
\pgfsetstrokecolor{currentstroke}%
\pgfsetstrokeopacity{0.000000}%
\pgfsetdash{}{0pt}%
\pgfpathmoveto{\pgfqpoint{1.249956in}{0.941258in}}%
\pgfpathlineto{\pgfqpoint{3.162412in}{0.941258in}}%
\pgfpathlineto{\pgfqpoint{3.162412in}{0.735376in}}%
\pgfpathlineto{\pgfqpoint{1.249956in}{0.735376in}}%
\pgfpathclose%
\pgfusepath{fill}%
\end{pgfscope}%
\begin{pgfscope}%
\pgfpathrectangle{\pgfqpoint{1.249956in}{0.148611in}}{\pgfqpoint{7.122250in}{3.850000in}}%
\pgfusepath{clip}%
\pgfsetbuttcap%
\pgfsetmiterjoin%
\definecolor{currentfill}{rgb}{0.898885,0.305498,0.206767}%
\pgfsetfillcolor{currentfill}%
\pgfsetlinewidth{0.000000pt}%
\definecolor{currentstroke}{rgb}{0.000000,0.000000,0.000000}%
\pgfsetstrokecolor{currentstroke}%
\pgfsetstrokeopacity{0.000000}%
\pgfsetdash{}{0pt}%
\pgfpathmoveto{\pgfqpoint{1.249956in}{0.529493in}}%
\pgfpathlineto{\pgfqpoint{2.700785in}{0.529493in}}%
\pgfpathlineto{\pgfqpoint{2.700785in}{0.323611in}}%
\pgfpathlineto{\pgfqpoint{1.249956in}{0.323611in}}%
\pgfpathclose%
\pgfusepath{fill}%
\end{pgfscope}%
\begin{pgfscope}%
\pgfpathrectangle{\pgfqpoint{1.249956in}{0.148611in}}{\pgfqpoint{7.122250in}{3.850000in}}%
\pgfusepath{clip}%
\pgfsetbuttcap%
\pgfsetmiterjoin%
\definecolor{currentfill}{rgb}{0.966551,0.497424,0.295040}%
\pgfsetfillcolor{currentfill}%
\pgfsetfillopacity{0.500000}%
\pgfsetlinewidth{0.000000pt}%
\definecolor{currentstroke}{rgb}{0.000000,0.000000,0.000000}%
\pgfsetstrokecolor{currentstroke}%
\pgfsetstrokeopacity{0.500000}%
\pgfsetdash{}{0pt}%
\pgfpathmoveto{\pgfqpoint{2.502945in}{3.823611in}}%
\pgfpathlineto{\pgfqpoint{3.195386in}{3.823611in}}%
\pgfpathlineto{\pgfqpoint{3.195386in}{3.617729in}}%
\pgfpathlineto{\pgfqpoint{2.502945in}{3.617729in}}%
\pgfpathclose%
\pgfusepath{fill}%
\end{pgfscope}%
\begin{pgfscope}%
\pgfpathrectangle{\pgfqpoint{1.249956in}{0.148611in}}{\pgfqpoint{7.122250in}{3.850000in}}%
\pgfusepath{clip}%
\pgfsetbuttcap%
\pgfsetmiterjoin%
\definecolor{currentfill}{rgb}{0.966551,0.497424,0.295040}%
\pgfsetfillcolor{currentfill}%
\pgfsetfillopacity{0.500000}%
\pgfsetlinewidth{0.000000pt}%
\definecolor{currentstroke}{rgb}{0.000000,0.000000,0.000000}%
\pgfsetstrokecolor{currentstroke}%
\pgfsetstrokeopacity{0.500000}%
\pgfsetdash{}{0pt}%
\pgfpathmoveto{\pgfqpoint{3.294306in}{3.411846in}}%
\pgfpathlineto{\pgfqpoint{4.217561in}{3.411846in}}%
\pgfpathlineto{\pgfqpoint{4.217561in}{3.205964in}}%
\pgfpathlineto{\pgfqpoint{3.294306in}{3.205964in}}%
\pgfpathclose%
\pgfusepath{fill}%
\end{pgfscope}%
\begin{pgfscope}%
\pgfpathrectangle{\pgfqpoint{1.249956in}{0.148611in}}{\pgfqpoint{7.122250in}{3.850000in}}%
\pgfusepath{clip}%
\pgfsetbuttcap%
\pgfsetmiterjoin%
\definecolor{currentfill}{rgb}{0.966551,0.497424,0.295040}%
\pgfsetfillcolor{currentfill}%
\pgfsetlinewidth{0.000000pt}%
\definecolor{currentstroke}{rgb}{0.000000,0.000000,0.000000}%
\pgfsetstrokecolor{currentstroke}%
\pgfsetstrokeopacity{0.000000}%
\pgfsetdash{}{0pt}%
\pgfpathmoveto{\pgfqpoint{2.700785in}{3.000082in}}%
\pgfpathlineto{\pgfqpoint{3.689986in}{3.000082in}}%
\pgfpathlineto{\pgfqpoint{3.689986in}{2.794199in}}%
\pgfpathlineto{\pgfqpoint{2.700785in}{2.794199in}}%
\pgfpathclose%
\pgfusepath{fill}%
\end{pgfscope}%
\begin{pgfscope}%
\pgfpathrectangle{\pgfqpoint{1.249956in}{0.148611in}}{\pgfqpoint{7.122250in}{3.850000in}}%
\pgfusepath{clip}%
\pgfsetbuttcap%
\pgfsetmiterjoin%
\definecolor{currentfill}{rgb}{0.966551,0.497424,0.295040}%
\pgfsetfillcolor{currentfill}%
\pgfsetfillopacity{0.500000}%
\pgfsetlinewidth{0.000000pt}%
\definecolor{currentstroke}{rgb}{0.000000,0.000000,0.000000}%
\pgfsetstrokecolor{currentstroke}%
\pgfsetstrokeopacity{0.500000}%
\pgfsetdash{}{0pt}%
\pgfpathmoveto{\pgfqpoint{1.843477in}{2.588317in}}%
\pgfpathlineto{\pgfqpoint{2.568892in}{2.588317in}}%
\pgfpathlineto{\pgfqpoint{2.568892in}{2.382435in}}%
\pgfpathlineto{\pgfqpoint{1.843477in}{2.382435in}}%
\pgfpathclose%
\pgfusepath{fill}%
\end{pgfscope}%
\begin{pgfscope}%
\pgfpathrectangle{\pgfqpoint{1.249956in}{0.148611in}}{\pgfqpoint{7.122250in}{3.850000in}}%
\pgfusepath{clip}%
\pgfsetbuttcap%
\pgfsetmiterjoin%
\definecolor{currentfill}{rgb}{0.966551,0.497424,0.295040}%
\pgfsetfillcolor{currentfill}%
\pgfsetfillopacity{0.500000}%
\pgfsetlinewidth{0.000000pt}%
\definecolor{currentstroke}{rgb}{0.000000,0.000000,0.000000}%
\pgfsetstrokecolor{currentstroke}%
\pgfsetstrokeopacity{0.500000}%
\pgfsetdash{}{0pt}%
\pgfpathmoveto{\pgfqpoint{1.612664in}{2.176552in}}%
\pgfpathlineto{\pgfqpoint{1.876451in}{2.176552in}}%
\pgfpathlineto{\pgfqpoint{1.876451in}{1.970670in}}%
\pgfpathlineto{\pgfqpoint{1.612664in}{1.970670in}}%
\pgfpathclose%
\pgfusepath{fill}%
\end{pgfscope}%
\begin{pgfscope}%
\pgfpathrectangle{\pgfqpoint{1.249956in}{0.148611in}}{\pgfqpoint{7.122250in}{3.850000in}}%
\pgfusepath{clip}%
\pgfsetbuttcap%
\pgfsetmiterjoin%
\definecolor{currentfill}{rgb}{0.966551,0.497424,0.295040}%
\pgfsetfillcolor{currentfill}%
\pgfsetlinewidth{0.000000pt}%
\definecolor{currentstroke}{rgb}{0.000000,0.000000,0.000000}%
\pgfsetstrokecolor{currentstroke}%
\pgfsetstrokeopacity{0.000000}%
\pgfsetdash{}{0pt}%
\pgfpathmoveto{\pgfqpoint{3.096466in}{1.764788in}}%
\pgfpathlineto{\pgfqpoint{4.481348in}{1.764788in}}%
\pgfpathlineto{\pgfqpoint{4.481348in}{1.558905in}}%
\pgfpathlineto{\pgfqpoint{3.096466in}{1.558905in}}%
\pgfpathclose%
\pgfusepath{fill}%
\end{pgfscope}%
\begin{pgfscope}%
\pgfpathrectangle{\pgfqpoint{1.249956in}{0.148611in}}{\pgfqpoint{7.122250in}{3.850000in}}%
\pgfusepath{clip}%
\pgfsetbuttcap%
\pgfsetmiterjoin%
\definecolor{currentfill}{rgb}{0.966551,0.497424,0.295040}%
\pgfsetfillcolor{currentfill}%
\pgfsetfillopacity{0.500000}%
\pgfsetlinewidth{0.000000pt}%
\definecolor{currentstroke}{rgb}{0.000000,0.000000,0.000000}%
\pgfsetstrokecolor{currentstroke}%
\pgfsetstrokeopacity{0.500000}%
\pgfsetdash{}{0pt}%
\pgfpathmoveto{\pgfqpoint{1.678610in}{1.353023in}}%
\pgfpathlineto{\pgfqpoint{1.909424in}{1.353023in}}%
\pgfpathlineto{\pgfqpoint{1.909424in}{1.147141in}}%
\pgfpathlineto{\pgfqpoint{1.678610in}{1.147141in}}%
\pgfpathclose%
\pgfusepath{fill}%
\end{pgfscope}%
\begin{pgfscope}%
\pgfpathrectangle{\pgfqpoint{1.249956in}{0.148611in}}{\pgfqpoint{7.122250in}{3.850000in}}%
\pgfusepath{clip}%
\pgfsetbuttcap%
\pgfsetmiterjoin%
\definecolor{currentfill}{rgb}{0.966551,0.497424,0.295040}%
\pgfsetfillcolor{currentfill}%
\pgfsetlinewidth{0.000000pt}%
\definecolor{currentstroke}{rgb}{0.000000,0.000000,0.000000}%
\pgfsetstrokecolor{currentstroke}%
\pgfsetstrokeopacity{0.000000}%
\pgfsetdash{}{0pt}%
\pgfpathmoveto{\pgfqpoint{3.162412in}{0.941258in}}%
\pgfpathlineto{\pgfqpoint{4.316481in}{0.941258in}}%
\pgfpathlineto{\pgfqpoint{4.316481in}{0.735376in}}%
\pgfpathlineto{\pgfqpoint{3.162412in}{0.735376in}}%
\pgfpathclose%
\pgfusepath{fill}%
\end{pgfscope}%
\begin{pgfscope}%
\pgfpathrectangle{\pgfqpoint{1.249956in}{0.148611in}}{\pgfqpoint{7.122250in}{3.850000in}}%
\pgfusepath{clip}%
\pgfsetbuttcap%
\pgfsetmiterjoin%
\definecolor{currentfill}{rgb}{0.966551,0.497424,0.295040}%
\pgfsetfillcolor{currentfill}%
\pgfsetlinewidth{0.000000pt}%
\definecolor{currentstroke}{rgb}{0.000000,0.000000,0.000000}%
\pgfsetstrokecolor{currentstroke}%
\pgfsetstrokeopacity{0.000000}%
\pgfsetdash{}{0pt}%
\pgfpathmoveto{\pgfqpoint{2.700785in}{0.529493in}}%
\pgfpathlineto{\pgfqpoint{3.393226in}{0.529493in}}%
\pgfpathlineto{\pgfqpoint{3.393226in}{0.323611in}}%
\pgfpathlineto{\pgfqpoint{2.700785in}{0.323611in}}%
\pgfpathclose%
\pgfusepath{fill}%
\end{pgfscope}%
\begin{pgfscope}%
\pgfpathrectangle{\pgfqpoint{1.249956in}{0.148611in}}{\pgfqpoint{7.122250in}{3.850000in}}%
\pgfusepath{clip}%
\pgfsetbuttcap%
\pgfsetmiterjoin%
\definecolor{currentfill}{rgb}{0.992388,0.693887,0.390081}%
\pgfsetfillcolor{currentfill}%
\pgfsetfillopacity{0.500000}%
\pgfsetlinewidth{0.000000pt}%
\definecolor{currentstroke}{rgb}{0.000000,0.000000,0.000000}%
\pgfsetstrokecolor{currentstroke}%
\pgfsetstrokeopacity{0.500000}%
\pgfsetdash{}{0pt}%
\pgfpathmoveto{\pgfqpoint{3.195386in}{3.823611in}}%
\pgfpathlineto{\pgfqpoint{4.052694in}{3.823611in}}%
\pgfpathlineto{\pgfqpoint{4.052694in}{3.617729in}}%
\pgfpathlineto{\pgfqpoint{3.195386in}{3.617729in}}%
\pgfpathclose%
\pgfusepath{fill}%
\end{pgfscope}%
\begin{pgfscope}%
\pgfpathrectangle{\pgfqpoint{1.249956in}{0.148611in}}{\pgfqpoint{7.122250in}{3.850000in}}%
\pgfusepath{clip}%
\pgfsetbuttcap%
\pgfsetmiterjoin%
\definecolor{currentfill}{rgb}{0.992388,0.693887,0.390081}%
\pgfsetfillcolor{currentfill}%
\pgfsetfillopacity{0.500000}%
\pgfsetlinewidth{0.000000pt}%
\definecolor{currentstroke}{rgb}{0.000000,0.000000,0.000000}%
\pgfsetstrokecolor{currentstroke}%
\pgfsetstrokeopacity{0.500000}%
\pgfsetdash{}{0pt}%
\pgfpathmoveto{\pgfqpoint{4.217561in}{3.411846in}}%
\pgfpathlineto{\pgfqpoint{5.239735in}{3.411846in}}%
\pgfpathlineto{\pgfqpoint{5.239735in}{3.205964in}}%
\pgfpathlineto{\pgfqpoint{4.217561in}{3.205964in}}%
\pgfpathclose%
\pgfusepath{fill}%
\end{pgfscope}%
\begin{pgfscope}%
\pgfpathrectangle{\pgfqpoint{1.249956in}{0.148611in}}{\pgfqpoint{7.122250in}{3.850000in}}%
\pgfusepath{clip}%
\pgfsetbuttcap%
\pgfsetmiterjoin%
\definecolor{currentfill}{rgb}{0.992388,0.693887,0.390081}%
\pgfsetfillcolor{currentfill}%
\pgfsetlinewidth{0.000000pt}%
\definecolor{currentstroke}{rgb}{0.000000,0.000000,0.000000}%
\pgfsetstrokecolor{currentstroke}%
\pgfsetstrokeopacity{0.000000}%
\pgfsetdash{}{0pt}%
\pgfpathmoveto{\pgfqpoint{3.689986in}{3.000082in}}%
\pgfpathlineto{\pgfqpoint{4.712161in}{3.000082in}}%
\pgfpathlineto{\pgfqpoint{4.712161in}{2.794199in}}%
\pgfpathlineto{\pgfqpoint{3.689986in}{2.794199in}}%
\pgfpathclose%
\pgfusepath{fill}%
\end{pgfscope}%
\begin{pgfscope}%
\pgfpathrectangle{\pgfqpoint{1.249956in}{0.148611in}}{\pgfqpoint{7.122250in}{3.850000in}}%
\pgfusepath{clip}%
\pgfsetbuttcap%
\pgfsetmiterjoin%
\definecolor{currentfill}{rgb}{0.992388,0.693887,0.390081}%
\pgfsetfillcolor{currentfill}%
\pgfsetfillopacity{0.500000}%
\pgfsetlinewidth{0.000000pt}%
\definecolor{currentstroke}{rgb}{0.000000,0.000000,0.000000}%
\pgfsetstrokecolor{currentstroke}%
\pgfsetstrokeopacity{0.500000}%
\pgfsetdash{}{0pt}%
\pgfpathmoveto{\pgfqpoint{2.568892in}{2.588317in}}%
\pgfpathlineto{\pgfqpoint{3.492146in}{2.588317in}}%
\pgfpathlineto{\pgfqpoint{3.492146in}{2.382435in}}%
\pgfpathlineto{\pgfqpoint{2.568892in}{2.382435in}}%
\pgfpathclose%
\pgfusepath{fill}%
\end{pgfscope}%
\begin{pgfscope}%
\pgfpathrectangle{\pgfqpoint{1.249956in}{0.148611in}}{\pgfqpoint{7.122250in}{3.850000in}}%
\pgfusepath{clip}%
\pgfsetbuttcap%
\pgfsetmiterjoin%
\definecolor{currentfill}{rgb}{0.992388,0.693887,0.390081}%
\pgfsetfillcolor{currentfill}%
\pgfsetfillopacity{0.500000}%
\pgfsetlinewidth{0.000000pt}%
\definecolor{currentstroke}{rgb}{0.000000,0.000000,0.000000}%
\pgfsetstrokecolor{currentstroke}%
\pgfsetstrokeopacity{0.500000}%
\pgfsetdash{}{0pt}%
\pgfpathmoveto{\pgfqpoint{1.876451in}{2.176552in}}%
\pgfpathlineto{\pgfqpoint{2.140238in}{2.176552in}}%
\pgfpathlineto{\pgfqpoint{2.140238in}{1.970670in}}%
\pgfpathlineto{\pgfqpoint{1.876451in}{1.970670in}}%
\pgfpathclose%
\pgfusepath{fill}%
\end{pgfscope}%
\begin{pgfscope}%
\pgfpathrectangle{\pgfqpoint{1.249956in}{0.148611in}}{\pgfqpoint{7.122250in}{3.850000in}}%
\pgfusepath{clip}%
\pgfsetbuttcap%
\pgfsetmiterjoin%
\definecolor{currentfill}{rgb}{0.992388,0.693887,0.390081}%
\pgfsetfillcolor{currentfill}%
\pgfsetlinewidth{0.000000pt}%
\definecolor{currentstroke}{rgb}{0.000000,0.000000,0.000000}%
\pgfsetstrokecolor{currentstroke}%
\pgfsetstrokeopacity{0.000000}%
\pgfsetdash{}{0pt}%
\pgfpathmoveto{\pgfqpoint{4.481348in}{1.764788in}}%
\pgfpathlineto{\pgfqpoint{5.404602in}{1.764788in}}%
\pgfpathlineto{\pgfqpoint{5.404602in}{1.558905in}}%
\pgfpathlineto{\pgfqpoint{4.481348in}{1.558905in}}%
\pgfpathclose%
\pgfusepath{fill}%
\end{pgfscope}%
\begin{pgfscope}%
\pgfpathrectangle{\pgfqpoint{1.249956in}{0.148611in}}{\pgfqpoint{7.122250in}{3.850000in}}%
\pgfusepath{clip}%
\pgfsetbuttcap%
\pgfsetmiterjoin%
\definecolor{currentfill}{rgb}{0.992388,0.693887,0.390081}%
\pgfsetfillcolor{currentfill}%
\pgfsetfillopacity{0.500000}%
\pgfsetlinewidth{0.000000pt}%
\definecolor{currentstroke}{rgb}{0.000000,0.000000,0.000000}%
\pgfsetstrokecolor{currentstroke}%
\pgfsetstrokeopacity{0.500000}%
\pgfsetdash{}{0pt}%
\pgfpathmoveto{\pgfqpoint{1.909424in}{1.353023in}}%
\pgfpathlineto{\pgfqpoint{2.140238in}{1.353023in}}%
\pgfpathlineto{\pgfqpoint{2.140238in}{1.147141in}}%
\pgfpathlineto{\pgfqpoint{1.909424in}{1.147141in}}%
\pgfpathclose%
\pgfusepath{fill}%
\end{pgfscope}%
\begin{pgfscope}%
\pgfpathrectangle{\pgfqpoint{1.249956in}{0.148611in}}{\pgfqpoint{7.122250in}{3.850000in}}%
\pgfusepath{clip}%
\pgfsetbuttcap%
\pgfsetmiterjoin%
\definecolor{currentfill}{rgb}{0.992388,0.693887,0.390081}%
\pgfsetfillcolor{currentfill}%
\pgfsetlinewidth{0.000000pt}%
\definecolor{currentstroke}{rgb}{0.000000,0.000000,0.000000}%
\pgfsetstrokecolor{currentstroke}%
\pgfsetstrokeopacity{0.000000}%
\pgfsetdash{}{0pt}%
\pgfpathmoveto{\pgfqpoint{4.316481in}{0.941258in}}%
\pgfpathlineto{\pgfqpoint{5.932176in}{0.941258in}}%
\pgfpathlineto{\pgfqpoint{5.932176in}{0.735376in}}%
\pgfpathlineto{\pgfqpoint{4.316481in}{0.735376in}}%
\pgfpathclose%
\pgfusepath{fill}%
\end{pgfscope}%
\begin{pgfscope}%
\pgfpathrectangle{\pgfqpoint{1.249956in}{0.148611in}}{\pgfqpoint{7.122250in}{3.850000in}}%
\pgfusepath{clip}%
\pgfsetbuttcap%
\pgfsetmiterjoin%
\definecolor{currentfill}{rgb}{0.992388,0.693887,0.390081}%
\pgfsetfillcolor{currentfill}%
\pgfsetlinewidth{0.000000pt}%
\definecolor{currentstroke}{rgb}{0.000000,0.000000,0.000000}%
\pgfsetstrokecolor{currentstroke}%
\pgfsetstrokeopacity{0.000000}%
\pgfsetdash{}{0pt}%
\pgfpathmoveto{\pgfqpoint{3.393226in}{0.529493in}}%
\pgfpathlineto{\pgfqpoint{4.052694in}{0.529493in}}%
\pgfpathlineto{\pgfqpoint{4.052694in}{0.323611in}}%
\pgfpathlineto{\pgfqpoint{3.393226in}{0.323611in}}%
\pgfpathclose%
\pgfusepath{fill}%
\end{pgfscope}%
\begin{pgfscope}%
\pgfpathrectangle{\pgfqpoint{1.249956in}{0.148611in}}{\pgfqpoint{7.122250in}{3.850000in}}%
\pgfusepath{clip}%
\pgfsetbuttcap%
\pgfsetmiterjoin%
\definecolor{currentfill}{rgb}{0.995463,0.847674,0.519262}%
\pgfsetfillcolor{currentfill}%
\pgfsetfillopacity{0.500000}%
\pgfsetlinewidth{0.000000pt}%
\definecolor{currentstroke}{rgb}{0.000000,0.000000,0.000000}%
\pgfsetstrokecolor{currentstroke}%
\pgfsetstrokeopacity{0.500000}%
\pgfsetdash{}{0pt}%
\pgfpathmoveto{\pgfqpoint{4.052694in}{3.823611in}}%
\pgfpathlineto{\pgfqpoint{4.942975in}{3.823611in}}%
\pgfpathlineto{\pgfqpoint{4.942975in}{3.617729in}}%
\pgfpathlineto{\pgfqpoint{4.052694in}{3.617729in}}%
\pgfpathclose%
\pgfusepath{fill}%
\end{pgfscope}%
\begin{pgfscope}%
\pgfpathrectangle{\pgfqpoint{1.249956in}{0.148611in}}{\pgfqpoint{7.122250in}{3.850000in}}%
\pgfusepath{clip}%
\pgfsetbuttcap%
\pgfsetmiterjoin%
\definecolor{currentfill}{rgb}{0.995463,0.847674,0.519262}%
\pgfsetfillcolor{currentfill}%
\pgfsetfillopacity{0.500000}%
\pgfsetlinewidth{0.000000pt}%
\definecolor{currentstroke}{rgb}{0.000000,0.000000,0.000000}%
\pgfsetstrokecolor{currentstroke}%
\pgfsetstrokeopacity{0.500000}%
\pgfsetdash{}{0pt}%
\pgfpathmoveto{\pgfqpoint{5.239735in}{3.411846in}}%
\pgfpathlineto{\pgfqpoint{6.360830in}{3.411846in}}%
\pgfpathlineto{\pgfqpoint{6.360830in}{3.205964in}}%
\pgfpathlineto{\pgfqpoint{5.239735in}{3.205964in}}%
\pgfpathclose%
\pgfusepath{fill}%
\end{pgfscope}%
\begin{pgfscope}%
\pgfpathrectangle{\pgfqpoint{1.249956in}{0.148611in}}{\pgfqpoint{7.122250in}{3.850000in}}%
\pgfusepath{clip}%
\pgfsetbuttcap%
\pgfsetmiterjoin%
\definecolor{currentfill}{rgb}{0.995463,0.847674,0.519262}%
\pgfsetfillcolor{currentfill}%
\pgfsetlinewidth{0.000000pt}%
\definecolor{currentstroke}{rgb}{0.000000,0.000000,0.000000}%
\pgfsetstrokecolor{currentstroke}%
\pgfsetstrokeopacity{0.000000}%
\pgfsetdash{}{0pt}%
\pgfpathmoveto{\pgfqpoint{4.712161in}{3.000082in}}%
\pgfpathlineto{\pgfqpoint{5.371629in}{3.000082in}}%
\pgfpathlineto{\pgfqpoint{5.371629in}{2.794199in}}%
\pgfpathlineto{\pgfqpoint{4.712161in}{2.794199in}}%
\pgfpathclose%
\pgfusepath{fill}%
\end{pgfscope}%
\begin{pgfscope}%
\pgfpathrectangle{\pgfqpoint{1.249956in}{0.148611in}}{\pgfqpoint{7.122250in}{3.850000in}}%
\pgfusepath{clip}%
\pgfsetbuttcap%
\pgfsetmiterjoin%
\definecolor{currentfill}{rgb}{0.995463,0.847674,0.519262}%
\pgfsetfillcolor{currentfill}%
\pgfsetfillopacity{0.500000}%
\pgfsetlinewidth{0.000000pt}%
\definecolor{currentstroke}{rgb}{0.000000,0.000000,0.000000}%
\pgfsetstrokecolor{currentstroke}%
\pgfsetstrokeopacity{0.500000}%
\pgfsetdash{}{0pt}%
\pgfpathmoveto{\pgfqpoint{3.492146in}{2.588317in}}%
\pgfpathlineto{\pgfqpoint{4.514321in}{2.588317in}}%
\pgfpathlineto{\pgfqpoint{4.514321in}{2.382435in}}%
\pgfpathlineto{\pgfqpoint{3.492146in}{2.382435in}}%
\pgfpathclose%
\pgfusepath{fill}%
\end{pgfscope}%
\begin{pgfscope}%
\pgfpathrectangle{\pgfqpoint{1.249956in}{0.148611in}}{\pgfqpoint{7.122250in}{3.850000in}}%
\pgfusepath{clip}%
\pgfsetbuttcap%
\pgfsetmiterjoin%
\definecolor{currentfill}{rgb}{0.995463,0.847674,0.519262}%
\pgfsetfillcolor{currentfill}%
\pgfsetfillopacity{0.500000}%
\pgfsetlinewidth{0.000000pt}%
\definecolor{currentstroke}{rgb}{0.000000,0.000000,0.000000}%
\pgfsetstrokecolor{currentstroke}%
\pgfsetstrokeopacity{0.500000}%
\pgfsetdash{}{0pt}%
\pgfpathmoveto{\pgfqpoint{2.140238in}{2.176552in}}%
\pgfpathlineto{\pgfqpoint{2.898625in}{2.176552in}}%
\pgfpathlineto{\pgfqpoint{2.898625in}{1.970670in}}%
\pgfpathlineto{\pgfqpoint{2.140238in}{1.970670in}}%
\pgfpathclose%
\pgfusepath{fill}%
\end{pgfscope}%
\begin{pgfscope}%
\pgfpathrectangle{\pgfqpoint{1.249956in}{0.148611in}}{\pgfqpoint{7.122250in}{3.850000in}}%
\pgfusepath{clip}%
\pgfsetbuttcap%
\pgfsetmiterjoin%
\definecolor{currentfill}{rgb}{0.995463,0.847674,0.519262}%
\pgfsetfillcolor{currentfill}%
\pgfsetlinewidth{0.000000pt}%
\definecolor{currentstroke}{rgb}{0.000000,0.000000,0.000000}%
\pgfsetstrokecolor{currentstroke}%
\pgfsetstrokeopacity{0.000000}%
\pgfsetdash{}{0pt}%
\pgfpathmoveto{\pgfqpoint{5.404602in}{1.764788in}}%
\pgfpathlineto{\pgfqpoint{6.492724in}{1.764788in}}%
\pgfpathlineto{\pgfqpoint{6.492724in}{1.558905in}}%
\pgfpathlineto{\pgfqpoint{5.404602in}{1.558905in}}%
\pgfpathclose%
\pgfusepath{fill}%
\end{pgfscope}%
\begin{pgfscope}%
\pgfpathrectangle{\pgfqpoint{1.249956in}{0.148611in}}{\pgfqpoint{7.122250in}{3.850000in}}%
\pgfusepath{clip}%
\pgfsetbuttcap%
\pgfsetmiterjoin%
\definecolor{currentfill}{rgb}{0.995463,0.847674,0.519262}%
\pgfsetfillcolor{currentfill}%
\pgfsetfillopacity{0.500000}%
\pgfsetlinewidth{0.000000pt}%
\definecolor{currentstroke}{rgb}{0.000000,0.000000,0.000000}%
\pgfsetstrokecolor{currentstroke}%
\pgfsetstrokeopacity{0.500000}%
\pgfsetdash{}{0pt}%
\pgfpathmoveto{\pgfqpoint{2.140238in}{1.353023in}}%
\pgfpathlineto{\pgfqpoint{2.568892in}{1.353023in}}%
\pgfpathlineto{\pgfqpoint{2.568892in}{1.147141in}}%
\pgfpathlineto{\pgfqpoint{2.140238in}{1.147141in}}%
\pgfpathclose%
\pgfusepath{fill}%
\end{pgfscope}%
\begin{pgfscope}%
\pgfpathrectangle{\pgfqpoint{1.249956in}{0.148611in}}{\pgfqpoint{7.122250in}{3.850000in}}%
\pgfusepath{clip}%
\pgfsetbuttcap%
\pgfsetmiterjoin%
\definecolor{currentfill}{rgb}{0.995463,0.847674,0.519262}%
\pgfsetfillcolor{currentfill}%
\pgfsetlinewidth{0.000000pt}%
\definecolor{currentstroke}{rgb}{0.000000,0.000000,0.000000}%
\pgfsetstrokecolor{currentstroke}%
\pgfsetstrokeopacity{0.000000}%
\pgfsetdash{}{0pt}%
\pgfpathmoveto{\pgfqpoint{5.932176in}{0.941258in}}%
\pgfpathlineto{\pgfqpoint{6.921378in}{0.941258in}}%
\pgfpathlineto{\pgfqpoint{6.921378in}{0.735376in}}%
\pgfpathlineto{\pgfqpoint{5.932176in}{0.735376in}}%
\pgfpathclose%
\pgfusepath{fill}%
\end{pgfscope}%
\begin{pgfscope}%
\pgfpathrectangle{\pgfqpoint{1.249956in}{0.148611in}}{\pgfqpoint{7.122250in}{3.850000in}}%
\pgfusepath{clip}%
\pgfsetbuttcap%
\pgfsetmiterjoin%
\definecolor{currentfill}{rgb}{0.995463,0.847674,0.519262}%
\pgfsetfillcolor{currentfill}%
\pgfsetlinewidth{0.000000pt}%
\definecolor{currentstroke}{rgb}{0.000000,0.000000,0.000000}%
\pgfsetstrokecolor{currentstroke}%
\pgfsetstrokeopacity{0.000000}%
\pgfsetdash{}{0pt}%
\pgfpathmoveto{\pgfqpoint{4.052694in}{0.529493in}}%
\pgfpathlineto{\pgfqpoint{5.008922in}{0.529493in}}%
\pgfpathlineto{\pgfqpoint{5.008922in}{0.323611in}}%
\pgfpathlineto{\pgfqpoint{4.052694in}{0.323611in}}%
\pgfpathclose%
\pgfusepath{fill}%
\end{pgfscope}%
\begin{pgfscope}%
\pgfpathrectangle{\pgfqpoint{1.249956in}{0.148611in}}{\pgfqpoint{7.122250in}{3.850000in}}%
\pgfusepath{clip}%
\pgfsetbuttcap%
\pgfsetmiterjoin%
\definecolor{currentfill}{rgb}{0.998539,0.954710,0.673049}%
\pgfsetfillcolor{currentfill}%
\pgfsetfillopacity{0.500000}%
\pgfsetlinewidth{0.000000pt}%
\definecolor{currentstroke}{rgb}{0.000000,0.000000,0.000000}%
\pgfsetstrokecolor{currentstroke}%
\pgfsetstrokeopacity{0.500000}%
\pgfsetdash{}{0pt}%
\pgfpathmoveto{\pgfqpoint{4.942975in}{3.823611in}}%
\pgfpathlineto{\pgfqpoint{5.371629in}{3.823611in}}%
\pgfpathlineto{\pgfqpoint{5.371629in}{3.617729in}}%
\pgfpathlineto{\pgfqpoint{4.942975in}{3.617729in}}%
\pgfpathclose%
\pgfusepath{fill}%
\end{pgfscope}%
\begin{pgfscope}%
\pgfpathrectangle{\pgfqpoint{1.249956in}{0.148611in}}{\pgfqpoint{7.122250in}{3.850000in}}%
\pgfusepath{clip}%
\pgfsetbuttcap%
\pgfsetmiterjoin%
\definecolor{currentfill}{rgb}{0.998539,0.954710,0.673049}%
\pgfsetfillcolor{currentfill}%
\pgfsetfillopacity{0.500000}%
\pgfsetlinewidth{0.000000pt}%
\definecolor{currentstroke}{rgb}{0.000000,0.000000,0.000000}%
\pgfsetstrokecolor{currentstroke}%
\pgfsetstrokeopacity{0.500000}%
\pgfsetdash{}{0pt}%
\pgfpathmoveto{\pgfqpoint{6.360830in}{3.411846in}}%
\pgfpathlineto{\pgfqpoint{6.855431in}{3.411846in}}%
\pgfpathlineto{\pgfqpoint{6.855431in}{3.205964in}}%
\pgfpathlineto{\pgfqpoint{6.360830in}{3.205964in}}%
\pgfpathclose%
\pgfusepath{fill}%
\end{pgfscope}%
\begin{pgfscope}%
\pgfpathrectangle{\pgfqpoint{1.249956in}{0.148611in}}{\pgfqpoint{7.122250in}{3.850000in}}%
\pgfusepath{clip}%
\pgfsetbuttcap%
\pgfsetmiterjoin%
\definecolor{currentfill}{rgb}{0.998539,0.954710,0.673049}%
\pgfsetfillcolor{currentfill}%
\pgfsetlinewidth{0.000000pt}%
\definecolor{currentstroke}{rgb}{0.000000,0.000000,0.000000}%
\pgfsetstrokecolor{currentstroke}%
\pgfsetstrokeopacity{0.000000}%
\pgfsetdash{}{0pt}%
\pgfpathmoveto{\pgfqpoint{5.371629in}{3.000082in}}%
\pgfpathlineto{\pgfqpoint{6.195963in}{3.000082in}}%
\pgfpathlineto{\pgfqpoint{6.195963in}{2.794199in}}%
\pgfpathlineto{\pgfqpoint{5.371629in}{2.794199in}}%
\pgfpathclose%
\pgfusepath{fill}%
\end{pgfscope}%
\begin{pgfscope}%
\pgfpathrectangle{\pgfqpoint{1.249956in}{0.148611in}}{\pgfqpoint{7.122250in}{3.850000in}}%
\pgfusepath{clip}%
\pgfsetbuttcap%
\pgfsetmiterjoin%
\definecolor{currentfill}{rgb}{0.998539,0.954710,0.673049}%
\pgfsetfillcolor{currentfill}%
\pgfsetfillopacity{0.500000}%
\pgfsetlinewidth{0.000000pt}%
\definecolor{currentstroke}{rgb}{0.000000,0.000000,0.000000}%
\pgfsetstrokecolor{currentstroke}%
\pgfsetstrokeopacity{0.500000}%
\pgfsetdash{}{0pt}%
\pgfpathmoveto{\pgfqpoint{4.514321in}{2.588317in}}%
\pgfpathlineto{\pgfqpoint{5.239735in}{2.588317in}}%
\pgfpathlineto{\pgfqpoint{5.239735in}{2.382435in}}%
\pgfpathlineto{\pgfqpoint{4.514321in}{2.382435in}}%
\pgfpathclose%
\pgfusepath{fill}%
\end{pgfscope}%
\begin{pgfscope}%
\pgfpathrectangle{\pgfqpoint{1.249956in}{0.148611in}}{\pgfqpoint{7.122250in}{3.850000in}}%
\pgfusepath{clip}%
\pgfsetbuttcap%
\pgfsetmiterjoin%
\definecolor{currentfill}{rgb}{0.998539,0.954710,0.673049}%
\pgfsetfillcolor{currentfill}%
\pgfsetfillopacity{0.500000}%
\pgfsetlinewidth{0.000000pt}%
\definecolor{currentstroke}{rgb}{0.000000,0.000000,0.000000}%
\pgfsetstrokecolor{currentstroke}%
\pgfsetstrokeopacity{0.500000}%
\pgfsetdash{}{0pt}%
\pgfpathmoveto{\pgfqpoint{2.898625in}{2.176552in}}%
\pgfpathlineto{\pgfqpoint{3.426199in}{2.176552in}}%
\pgfpathlineto{\pgfqpoint{3.426199in}{1.970670in}}%
\pgfpathlineto{\pgfqpoint{2.898625in}{1.970670in}}%
\pgfpathclose%
\pgfusepath{fill}%
\end{pgfscope}%
\begin{pgfscope}%
\pgfpathrectangle{\pgfqpoint{1.249956in}{0.148611in}}{\pgfqpoint{7.122250in}{3.850000in}}%
\pgfusepath{clip}%
\pgfsetbuttcap%
\pgfsetmiterjoin%
\definecolor{currentfill}{rgb}{0.998539,0.954710,0.673049}%
\pgfsetfillcolor{currentfill}%
\pgfsetlinewidth{0.000000pt}%
\definecolor{currentstroke}{rgb}{0.000000,0.000000,0.000000}%
\pgfsetstrokecolor{currentstroke}%
\pgfsetstrokeopacity{0.000000}%
\pgfsetdash{}{0pt}%
\pgfpathmoveto{\pgfqpoint{6.492724in}{1.764788in}}%
\pgfpathlineto{\pgfqpoint{7.119218in}{1.764788in}}%
\pgfpathlineto{\pgfqpoint{7.119218in}{1.558905in}}%
\pgfpathlineto{\pgfqpoint{6.492724in}{1.558905in}}%
\pgfpathclose%
\pgfusepath{fill}%
\end{pgfscope}%
\begin{pgfscope}%
\pgfpathrectangle{\pgfqpoint{1.249956in}{0.148611in}}{\pgfqpoint{7.122250in}{3.850000in}}%
\pgfusepath{clip}%
\pgfsetbuttcap%
\pgfsetmiterjoin%
\definecolor{currentfill}{rgb}{0.998539,0.954710,0.673049}%
\pgfsetfillcolor{currentfill}%
\pgfsetfillopacity{0.500000}%
\pgfsetlinewidth{0.000000pt}%
\definecolor{currentstroke}{rgb}{0.000000,0.000000,0.000000}%
\pgfsetstrokecolor{currentstroke}%
\pgfsetstrokeopacity{0.500000}%
\pgfsetdash{}{0pt}%
\pgfpathmoveto{\pgfqpoint{2.568892in}{1.353023in}}%
\pgfpathlineto{\pgfqpoint{3.162412in}{1.353023in}}%
\pgfpathlineto{\pgfqpoint{3.162412in}{1.147141in}}%
\pgfpathlineto{\pgfqpoint{2.568892in}{1.147141in}}%
\pgfpathclose%
\pgfusepath{fill}%
\end{pgfscope}%
\begin{pgfscope}%
\pgfpathrectangle{\pgfqpoint{1.249956in}{0.148611in}}{\pgfqpoint{7.122250in}{3.850000in}}%
\pgfusepath{clip}%
\pgfsetbuttcap%
\pgfsetmiterjoin%
\definecolor{currentfill}{rgb}{0.998539,0.954710,0.673049}%
\pgfsetfillcolor{currentfill}%
\pgfsetlinewidth{0.000000pt}%
\definecolor{currentstroke}{rgb}{0.000000,0.000000,0.000000}%
\pgfsetstrokecolor{currentstroke}%
\pgfsetstrokeopacity{0.000000}%
\pgfsetdash{}{0pt}%
\pgfpathmoveto{\pgfqpoint{6.921378in}{0.941258in}}%
\pgfpathlineto{\pgfqpoint{7.350032in}{0.941258in}}%
\pgfpathlineto{\pgfqpoint{7.350032in}{0.735376in}}%
\pgfpathlineto{\pgfqpoint{6.921378in}{0.735376in}}%
\pgfpathclose%
\pgfusepath{fill}%
\end{pgfscope}%
\begin{pgfscope}%
\pgfpathrectangle{\pgfqpoint{1.249956in}{0.148611in}}{\pgfqpoint{7.122250in}{3.850000in}}%
\pgfusepath{clip}%
\pgfsetbuttcap%
\pgfsetmiterjoin%
\definecolor{currentfill}{rgb}{0.998539,0.954710,0.673049}%
\pgfsetfillcolor{currentfill}%
\pgfsetlinewidth{0.000000pt}%
\definecolor{currentstroke}{rgb}{0.000000,0.000000,0.000000}%
\pgfsetstrokecolor{currentstroke}%
\pgfsetstrokeopacity{0.000000}%
\pgfsetdash{}{0pt}%
\pgfpathmoveto{\pgfqpoint{5.008922in}{0.529493in}}%
\pgfpathlineto{\pgfqpoint{5.536496in}{0.529493in}}%
\pgfpathlineto{\pgfqpoint{5.536496in}{0.323611in}}%
\pgfpathlineto{\pgfqpoint{5.008922in}{0.323611in}}%
\pgfpathclose%
\pgfusepath{fill}%
\end{pgfscope}%
\begin{pgfscope}%
\pgfpathrectangle{\pgfqpoint{1.249956in}{0.148611in}}{\pgfqpoint{7.122250in}{3.850000in}}%
\pgfusepath{clip}%
\pgfsetbuttcap%
\pgfsetmiterjoin%
\definecolor{currentfill}{rgb}{0.944483,0.976624,0.673049}%
\pgfsetfillcolor{currentfill}%
\pgfsetfillopacity{0.500000}%
\pgfsetlinewidth{0.000000pt}%
\definecolor{currentstroke}{rgb}{0.000000,0.000000,0.000000}%
\pgfsetstrokecolor{currentstroke}%
\pgfsetstrokeopacity{0.500000}%
\pgfsetdash{}{0pt}%
\pgfpathmoveto{\pgfqpoint{5.371629in}{3.823611in}}%
\pgfpathlineto{\pgfqpoint{6.393804in}{3.823611in}}%
\pgfpathlineto{\pgfqpoint{6.393804in}{3.617729in}}%
\pgfpathlineto{\pgfqpoint{5.371629in}{3.617729in}}%
\pgfpathclose%
\pgfusepath{fill}%
\end{pgfscope}%
\begin{pgfscope}%
\pgfpathrectangle{\pgfqpoint{1.249956in}{0.148611in}}{\pgfqpoint{7.122250in}{3.850000in}}%
\pgfusepath{clip}%
\pgfsetbuttcap%
\pgfsetmiterjoin%
\definecolor{currentfill}{rgb}{0.944483,0.976624,0.673049}%
\pgfsetfillcolor{currentfill}%
\pgfsetfillopacity{0.500000}%
\pgfsetlinewidth{0.000000pt}%
\definecolor{currentstroke}{rgb}{0.000000,0.000000,0.000000}%
\pgfsetstrokecolor{currentstroke}%
\pgfsetstrokeopacity{0.500000}%
\pgfsetdash{}{0pt}%
\pgfpathmoveto{\pgfqpoint{6.855431in}{3.411846in}}%
\pgfpathlineto{\pgfqpoint{7.251111in}{3.411846in}}%
\pgfpathlineto{\pgfqpoint{7.251111in}{3.205964in}}%
\pgfpathlineto{\pgfqpoint{6.855431in}{3.205964in}}%
\pgfpathclose%
\pgfusepath{fill}%
\end{pgfscope}%
\begin{pgfscope}%
\pgfpathrectangle{\pgfqpoint{1.249956in}{0.148611in}}{\pgfqpoint{7.122250in}{3.850000in}}%
\pgfusepath{clip}%
\pgfsetbuttcap%
\pgfsetmiterjoin%
\definecolor{currentfill}{rgb}{0.944483,0.976624,0.673049}%
\pgfsetfillcolor{currentfill}%
\pgfsetlinewidth{0.000000pt}%
\definecolor{currentstroke}{rgb}{0.000000,0.000000,0.000000}%
\pgfsetstrokecolor{currentstroke}%
\pgfsetstrokeopacity{0.000000}%
\pgfsetdash{}{0pt}%
\pgfpathmoveto{\pgfqpoint{6.195963in}{3.000082in}}%
\pgfpathlineto{\pgfqpoint{6.690564in}{3.000082in}}%
\pgfpathlineto{\pgfqpoint{6.690564in}{2.794199in}}%
\pgfpathlineto{\pgfqpoint{6.195963in}{2.794199in}}%
\pgfpathclose%
\pgfusepath{fill}%
\end{pgfscope}%
\begin{pgfscope}%
\pgfpathrectangle{\pgfqpoint{1.249956in}{0.148611in}}{\pgfqpoint{7.122250in}{3.850000in}}%
\pgfusepath{clip}%
\pgfsetbuttcap%
\pgfsetmiterjoin%
\definecolor{currentfill}{rgb}{0.944483,0.976624,0.673049}%
\pgfsetfillcolor{currentfill}%
\pgfsetfillopacity{0.500000}%
\pgfsetlinewidth{0.000000pt}%
\definecolor{currentstroke}{rgb}{0.000000,0.000000,0.000000}%
\pgfsetstrokecolor{currentstroke}%
\pgfsetstrokeopacity{0.500000}%
\pgfsetdash{}{0pt}%
\pgfpathmoveto{\pgfqpoint{5.239735in}{2.588317in}}%
\pgfpathlineto{\pgfqpoint{5.701363in}{2.588317in}}%
\pgfpathlineto{\pgfqpoint{5.701363in}{2.382435in}}%
\pgfpathlineto{\pgfqpoint{5.239735in}{2.382435in}}%
\pgfpathclose%
\pgfusepath{fill}%
\end{pgfscope}%
\begin{pgfscope}%
\pgfpathrectangle{\pgfqpoint{1.249956in}{0.148611in}}{\pgfqpoint{7.122250in}{3.850000in}}%
\pgfusepath{clip}%
\pgfsetbuttcap%
\pgfsetmiterjoin%
\definecolor{currentfill}{rgb}{0.944483,0.976624,0.673049}%
\pgfsetfillcolor{currentfill}%
\pgfsetfillopacity{0.500000}%
\pgfsetlinewidth{0.000000pt}%
\definecolor{currentstroke}{rgb}{0.000000,0.000000,0.000000}%
\pgfsetstrokecolor{currentstroke}%
\pgfsetstrokeopacity{0.500000}%
\pgfsetdash{}{0pt}%
\pgfpathmoveto{\pgfqpoint{3.426199in}{2.176552in}}%
\pgfpathlineto{\pgfqpoint{4.052694in}{2.176552in}}%
\pgfpathlineto{\pgfqpoint{4.052694in}{1.970670in}}%
\pgfpathlineto{\pgfqpoint{3.426199in}{1.970670in}}%
\pgfpathclose%
\pgfusepath{fill}%
\end{pgfscope}%
\begin{pgfscope}%
\pgfpathrectangle{\pgfqpoint{1.249956in}{0.148611in}}{\pgfqpoint{7.122250in}{3.850000in}}%
\pgfusepath{clip}%
\pgfsetbuttcap%
\pgfsetmiterjoin%
\definecolor{currentfill}{rgb}{0.944483,0.976624,0.673049}%
\pgfsetfillcolor{currentfill}%
\pgfsetlinewidth{0.000000pt}%
\definecolor{currentstroke}{rgb}{0.000000,0.000000,0.000000}%
\pgfsetstrokecolor{currentstroke}%
\pgfsetstrokeopacity{0.000000}%
\pgfsetdash{}{0pt}%
\pgfpathmoveto{\pgfqpoint{7.119218in}{1.764788in}}%
\pgfpathlineto{\pgfqpoint{7.712739in}{1.764788in}}%
\pgfpathlineto{\pgfqpoint{7.712739in}{1.558905in}}%
\pgfpathlineto{\pgfqpoint{7.119218in}{1.558905in}}%
\pgfpathclose%
\pgfusepath{fill}%
\end{pgfscope}%
\begin{pgfscope}%
\pgfpathrectangle{\pgfqpoint{1.249956in}{0.148611in}}{\pgfqpoint{7.122250in}{3.850000in}}%
\pgfusepath{clip}%
\pgfsetbuttcap%
\pgfsetmiterjoin%
\definecolor{currentfill}{rgb}{0.944483,0.976624,0.673049}%
\pgfsetfillcolor{currentfill}%
\pgfsetfillopacity{0.500000}%
\pgfsetlinewidth{0.000000pt}%
\definecolor{currentstroke}{rgb}{0.000000,0.000000,0.000000}%
\pgfsetstrokecolor{currentstroke}%
\pgfsetstrokeopacity{0.500000}%
\pgfsetdash{}{0pt}%
\pgfpathmoveto{\pgfqpoint{3.162412in}{1.353023in}}%
\pgfpathlineto{\pgfqpoint{3.788907in}{1.353023in}}%
\pgfpathlineto{\pgfqpoint{3.788907in}{1.147141in}}%
\pgfpathlineto{\pgfqpoint{3.162412in}{1.147141in}}%
\pgfpathclose%
\pgfusepath{fill}%
\end{pgfscope}%
\begin{pgfscope}%
\pgfpathrectangle{\pgfqpoint{1.249956in}{0.148611in}}{\pgfqpoint{7.122250in}{3.850000in}}%
\pgfusepath{clip}%
\pgfsetbuttcap%
\pgfsetmiterjoin%
\definecolor{currentfill}{rgb}{0.944483,0.976624,0.673049}%
\pgfsetfillcolor{currentfill}%
\pgfsetlinewidth{0.000000pt}%
\definecolor{currentstroke}{rgb}{0.000000,0.000000,0.000000}%
\pgfsetstrokecolor{currentstroke}%
\pgfsetstrokeopacity{0.000000}%
\pgfsetdash{}{0pt}%
\pgfpathmoveto{\pgfqpoint{7.350032in}{0.941258in}}%
\pgfpathlineto{\pgfqpoint{7.877606in}{0.941258in}}%
\pgfpathlineto{\pgfqpoint{7.877606in}{0.735376in}}%
\pgfpathlineto{\pgfqpoint{7.350032in}{0.735376in}}%
\pgfpathclose%
\pgfusepath{fill}%
\end{pgfscope}%
\begin{pgfscope}%
\pgfpathrectangle{\pgfqpoint{1.249956in}{0.148611in}}{\pgfqpoint{7.122250in}{3.850000in}}%
\pgfusepath{clip}%
\pgfsetbuttcap%
\pgfsetmiterjoin%
\definecolor{currentfill}{rgb}{0.944483,0.976624,0.673049}%
\pgfsetfillcolor{currentfill}%
\pgfsetlinewidth{0.000000pt}%
\definecolor{currentstroke}{rgb}{0.000000,0.000000,0.000000}%
\pgfsetstrokecolor{currentstroke}%
\pgfsetstrokeopacity{0.000000}%
\pgfsetdash{}{0pt}%
\pgfpathmoveto{\pgfqpoint{5.536496in}{0.529493in}}%
\pgfpathlineto{\pgfqpoint{6.162990in}{0.529493in}}%
\pgfpathlineto{\pgfqpoint{6.162990in}{0.323611in}}%
\pgfpathlineto{\pgfqpoint{5.536496in}{0.323611in}}%
\pgfpathclose%
\pgfusepath{fill}%
\end{pgfscope}%
\begin{pgfscope}%
\pgfpathrectangle{\pgfqpoint{1.249956in}{0.148611in}}{\pgfqpoint{7.122250in}{3.850000in}}%
\pgfusepath{clip}%
\pgfsetbuttcap%
\pgfsetmiterjoin%
\definecolor{currentfill}{rgb}{0.819608,0.923722,0.524798}%
\pgfsetfillcolor{currentfill}%
\pgfsetfillopacity{0.500000}%
\pgfsetlinewidth{0.000000pt}%
\definecolor{currentstroke}{rgb}{0.000000,0.000000,0.000000}%
\pgfsetstrokecolor{currentstroke}%
\pgfsetstrokeopacity{0.500000}%
\pgfsetdash{}{0pt}%
\pgfpathmoveto{\pgfqpoint{6.393804in}{3.823611in}}%
\pgfpathlineto{\pgfqpoint{7.119218in}{3.823611in}}%
\pgfpathlineto{\pgfqpoint{7.119218in}{3.617729in}}%
\pgfpathlineto{\pgfqpoint{6.393804in}{3.617729in}}%
\pgfpathclose%
\pgfusepath{fill}%
\end{pgfscope}%
\begin{pgfscope}%
\pgfpathrectangle{\pgfqpoint{1.249956in}{0.148611in}}{\pgfqpoint{7.122250in}{3.850000in}}%
\pgfusepath{clip}%
\pgfsetbuttcap%
\pgfsetmiterjoin%
\definecolor{currentfill}{rgb}{0.819608,0.923722,0.524798}%
\pgfsetfillcolor{currentfill}%
\pgfsetfillopacity{0.500000}%
\pgfsetlinewidth{0.000000pt}%
\definecolor{currentstroke}{rgb}{0.000000,0.000000,0.000000}%
\pgfsetstrokecolor{currentstroke}%
\pgfsetstrokeopacity{0.500000}%
\pgfsetdash{}{0pt}%
\pgfpathmoveto{\pgfqpoint{7.251111in}{3.411846in}}%
\pgfpathlineto{\pgfqpoint{7.613819in}{3.411846in}}%
\pgfpathlineto{\pgfqpoint{7.613819in}{3.205964in}}%
\pgfpathlineto{\pgfqpoint{7.251111in}{3.205964in}}%
\pgfpathclose%
\pgfusepath{fill}%
\end{pgfscope}%
\begin{pgfscope}%
\pgfpathrectangle{\pgfqpoint{1.249956in}{0.148611in}}{\pgfqpoint{7.122250in}{3.850000in}}%
\pgfusepath{clip}%
\pgfsetbuttcap%
\pgfsetmiterjoin%
\definecolor{currentfill}{rgb}{0.819608,0.923722,0.524798}%
\pgfsetfillcolor{currentfill}%
\pgfsetlinewidth{0.000000pt}%
\definecolor{currentstroke}{rgb}{0.000000,0.000000,0.000000}%
\pgfsetstrokecolor{currentstroke}%
\pgfsetstrokeopacity{0.000000}%
\pgfsetdash{}{0pt}%
\pgfpathmoveto{\pgfqpoint{6.690564in}{3.000082in}}%
\pgfpathlineto{\pgfqpoint{7.119218in}{3.000082in}}%
\pgfpathlineto{\pgfqpoint{7.119218in}{2.794199in}}%
\pgfpathlineto{\pgfqpoint{6.690564in}{2.794199in}}%
\pgfpathclose%
\pgfusepath{fill}%
\end{pgfscope}%
\begin{pgfscope}%
\pgfpathrectangle{\pgfqpoint{1.249956in}{0.148611in}}{\pgfqpoint{7.122250in}{3.850000in}}%
\pgfusepath{clip}%
\pgfsetbuttcap%
\pgfsetmiterjoin%
\definecolor{currentfill}{rgb}{0.819608,0.923722,0.524798}%
\pgfsetfillcolor{currentfill}%
\pgfsetfillopacity{0.500000}%
\pgfsetlinewidth{0.000000pt}%
\definecolor{currentstroke}{rgb}{0.000000,0.000000,0.000000}%
\pgfsetstrokecolor{currentstroke}%
\pgfsetstrokeopacity{0.500000}%
\pgfsetdash{}{0pt}%
\pgfpathmoveto{\pgfqpoint{5.701363in}{2.588317in}}%
\pgfpathlineto{\pgfqpoint{6.459750in}{2.588317in}}%
\pgfpathlineto{\pgfqpoint{6.459750in}{2.382435in}}%
\pgfpathlineto{\pgfqpoint{5.701363in}{2.382435in}}%
\pgfpathclose%
\pgfusepath{fill}%
\end{pgfscope}%
\begin{pgfscope}%
\pgfpathrectangle{\pgfqpoint{1.249956in}{0.148611in}}{\pgfqpoint{7.122250in}{3.850000in}}%
\pgfusepath{clip}%
\pgfsetbuttcap%
\pgfsetmiterjoin%
\definecolor{currentfill}{rgb}{0.819608,0.923722,0.524798}%
\pgfsetfillcolor{currentfill}%
\pgfsetfillopacity{0.500000}%
\pgfsetlinewidth{0.000000pt}%
\definecolor{currentstroke}{rgb}{0.000000,0.000000,0.000000}%
\pgfsetstrokecolor{currentstroke}%
\pgfsetstrokeopacity{0.500000}%
\pgfsetdash{}{0pt}%
\pgfpathmoveto{\pgfqpoint{4.052694in}{2.176552in}}%
\pgfpathlineto{\pgfqpoint{4.712161in}{2.176552in}}%
\pgfpathlineto{\pgfqpoint{4.712161in}{1.970670in}}%
\pgfpathlineto{\pgfqpoint{4.052694in}{1.970670in}}%
\pgfpathclose%
\pgfusepath{fill}%
\end{pgfscope}%
\begin{pgfscope}%
\pgfpathrectangle{\pgfqpoint{1.249956in}{0.148611in}}{\pgfqpoint{7.122250in}{3.850000in}}%
\pgfusepath{clip}%
\pgfsetbuttcap%
\pgfsetmiterjoin%
\definecolor{currentfill}{rgb}{0.819608,0.923722,0.524798}%
\pgfsetfillcolor{currentfill}%
\pgfsetlinewidth{0.000000pt}%
\definecolor{currentstroke}{rgb}{0.000000,0.000000,0.000000}%
\pgfsetstrokecolor{currentstroke}%
\pgfsetstrokeopacity{0.000000}%
\pgfsetdash{}{0pt}%
\pgfpathmoveto{\pgfqpoint{7.712739in}{1.764788in}}%
\pgfpathlineto{\pgfqpoint{7.976526in}{1.764788in}}%
\pgfpathlineto{\pgfqpoint{7.976526in}{1.558905in}}%
\pgfpathlineto{\pgfqpoint{7.712739in}{1.558905in}}%
\pgfpathclose%
\pgfusepath{fill}%
\end{pgfscope}%
\begin{pgfscope}%
\pgfpathrectangle{\pgfqpoint{1.249956in}{0.148611in}}{\pgfqpoint{7.122250in}{3.850000in}}%
\pgfusepath{clip}%
\pgfsetbuttcap%
\pgfsetmiterjoin%
\definecolor{currentfill}{rgb}{0.819608,0.923722,0.524798}%
\pgfsetfillcolor{currentfill}%
\pgfsetfillopacity{0.500000}%
\pgfsetlinewidth{0.000000pt}%
\definecolor{currentstroke}{rgb}{0.000000,0.000000,0.000000}%
\pgfsetstrokecolor{currentstroke}%
\pgfsetstrokeopacity{0.500000}%
\pgfsetdash{}{0pt}%
\pgfpathmoveto{\pgfqpoint{3.788907in}{1.353023in}}%
\pgfpathlineto{\pgfqpoint{4.349454in}{1.353023in}}%
\pgfpathlineto{\pgfqpoint{4.349454in}{1.147141in}}%
\pgfpathlineto{\pgfqpoint{3.788907in}{1.147141in}}%
\pgfpathclose%
\pgfusepath{fill}%
\end{pgfscope}%
\begin{pgfscope}%
\pgfpathrectangle{\pgfqpoint{1.249956in}{0.148611in}}{\pgfqpoint{7.122250in}{3.850000in}}%
\pgfusepath{clip}%
\pgfsetbuttcap%
\pgfsetmiterjoin%
\definecolor{currentfill}{rgb}{0.819608,0.923722,0.524798}%
\pgfsetfillcolor{currentfill}%
\pgfsetlinewidth{0.000000pt}%
\definecolor{currentstroke}{rgb}{0.000000,0.000000,0.000000}%
\pgfsetstrokecolor{currentstroke}%
\pgfsetstrokeopacity{0.000000}%
\pgfsetdash{}{0pt}%
\pgfpathmoveto{\pgfqpoint{7.877606in}{0.941258in}}%
\pgfpathlineto{\pgfqpoint{8.141393in}{0.941258in}}%
\pgfpathlineto{\pgfqpoint{8.141393in}{0.735376in}}%
\pgfpathlineto{\pgfqpoint{7.877606in}{0.735376in}}%
\pgfpathclose%
\pgfusepath{fill}%
\end{pgfscope}%
\begin{pgfscope}%
\pgfpathrectangle{\pgfqpoint{1.249956in}{0.148611in}}{\pgfqpoint{7.122250in}{3.850000in}}%
\pgfusepath{clip}%
\pgfsetbuttcap%
\pgfsetmiterjoin%
\definecolor{currentfill}{rgb}{0.819608,0.923722,0.524798}%
\pgfsetfillcolor{currentfill}%
\pgfsetlinewidth{0.000000pt}%
\definecolor{currentstroke}{rgb}{0.000000,0.000000,0.000000}%
\pgfsetstrokecolor{currentstroke}%
\pgfsetstrokeopacity{0.000000}%
\pgfsetdash{}{0pt}%
\pgfpathmoveto{\pgfqpoint{6.162990in}{0.529493in}}%
\pgfpathlineto{\pgfqpoint{6.921378in}{0.529493in}}%
\pgfpathlineto{\pgfqpoint{6.921378in}{0.323611in}}%
\pgfpathlineto{\pgfqpoint{6.162990in}{0.323611in}}%
\pgfpathclose%
\pgfusepath{fill}%
\end{pgfscope}%
\begin{pgfscope}%
\pgfpathrectangle{\pgfqpoint{1.249956in}{0.148611in}}{\pgfqpoint{7.122250in}{3.850000in}}%
\pgfusepath{clip}%
\pgfsetbuttcap%
\pgfsetmiterjoin%
\definecolor{currentfill}{rgb}{0.662745,0.856055,0.423299}%
\pgfsetfillcolor{currentfill}%
\pgfsetfillopacity{0.500000}%
\pgfsetlinewidth{0.000000pt}%
\definecolor{currentstroke}{rgb}{0.000000,0.000000,0.000000}%
\pgfsetstrokecolor{currentstroke}%
\pgfsetstrokeopacity{0.500000}%
\pgfsetdash{}{0pt}%
\pgfpathmoveto{\pgfqpoint{7.119218in}{3.823611in}}%
\pgfpathlineto{\pgfqpoint{7.646792in}{3.823611in}}%
\pgfpathlineto{\pgfqpoint{7.646792in}{3.617729in}}%
\pgfpathlineto{\pgfqpoint{7.119218in}{3.617729in}}%
\pgfpathclose%
\pgfusepath{fill}%
\end{pgfscope}%
\begin{pgfscope}%
\pgfpathrectangle{\pgfqpoint{1.249956in}{0.148611in}}{\pgfqpoint{7.122250in}{3.850000in}}%
\pgfusepath{clip}%
\pgfsetbuttcap%
\pgfsetmiterjoin%
\definecolor{currentfill}{rgb}{0.662745,0.856055,0.423299}%
\pgfsetfillcolor{currentfill}%
\pgfsetfillopacity{0.500000}%
\pgfsetlinewidth{0.000000pt}%
\definecolor{currentstroke}{rgb}{0.000000,0.000000,0.000000}%
\pgfsetstrokecolor{currentstroke}%
\pgfsetstrokeopacity{0.500000}%
\pgfsetdash{}{0pt}%
\pgfpathmoveto{\pgfqpoint{7.613819in}{3.411846in}}%
\pgfpathlineto{\pgfqpoint{7.943552in}{3.411846in}}%
\pgfpathlineto{\pgfqpoint{7.943552in}{3.205964in}}%
\pgfpathlineto{\pgfqpoint{7.613819in}{3.205964in}}%
\pgfpathclose%
\pgfusepath{fill}%
\end{pgfscope}%
\begin{pgfscope}%
\pgfpathrectangle{\pgfqpoint{1.249956in}{0.148611in}}{\pgfqpoint{7.122250in}{3.850000in}}%
\pgfusepath{clip}%
\pgfsetbuttcap%
\pgfsetmiterjoin%
\definecolor{currentfill}{rgb}{0.662745,0.856055,0.423299}%
\pgfsetfillcolor{currentfill}%
\pgfsetlinewidth{0.000000pt}%
\definecolor{currentstroke}{rgb}{0.000000,0.000000,0.000000}%
\pgfsetstrokecolor{currentstroke}%
\pgfsetstrokeopacity{0.000000}%
\pgfsetdash{}{0pt}%
\pgfpathmoveto{\pgfqpoint{7.119218in}{3.000082in}}%
\pgfpathlineto{\pgfqpoint{7.547872in}{3.000082in}}%
\pgfpathlineto{\pgfqpoint{7.547872in}{2.794199in}}%
\pgfpathlineto{\pgfqpoint{7.119218in}{2.794199in}}%
\pgfpathclose%
\pgfusepath{fill}%
\end{pgfscope}%
\begin{pgfscope}%
\pgfpathrectangle{\pgfqpoint{1.249956in}{0.148611in}}{\pgfqpoint{7.122250in}{3.850000in}}%
\pgfusepath{clip}%
\pgfsetbuttcap%
\pgfsetmiterjoin%
\definecolor{currentfill}{rgb}{0.662745,0.856055,0.423299}%
\pgfsetfillcolor{currentfill}%
\pgfsetfillopacity{0.500000}%
\pgfsetlinewidth{0.000000pt}%
\definecolor{currentstroke}{rgb}{0.000000,0.000000,0.000000}%
\pgfsetstrokecolor{currentstroke}%
\pgfsetstrokeopacity{0.500000}%
\pgfsetdash{}{0pt}%
\pgfpathmoveto{\pgfqpoint{6.459750in}{2.588317in}}%
\pgfpathlineto{\pgfqpoint{7.284085in}{2.588317in}}%
\pgfpathlineto{\pgfqpoint{7.284085in}{2.382435in}}%
\pgfpathlineto{\pgfqpoint{6.459750in}{2.382435in}}%
\pgfpathclose%
\pgfusepath{fill}%
\end{pgfscope}%
\begin{pgfscope}%
\pgfpathrectangle{\pgfqpoint{1.249956in}{0.148611in}}{\pgfqpoint{7.122250in}{3.850000in}}%
\pgfusepath{clip}%
\pgfsetbuttcap%
\pgfsetmiterjoin%
\definecolor{currentfill}{rgb}{0.662745,0.856055,0.423299}%
\pgfsetfillcolor{currentfill}%
\pgfsetfillopacity{0.500000}%
\pgfsetlinewidth{0.000000pt}%
\definecolor{currentstroke}{rgb}{0.000000,0.000000,0.000000}%
\pgfsetstrokecolor{currentstroke}%
\pgfsetstrokeopacity{0.500000}%
\pgfsetdash{}{0pt}%
\pgfpathmoveto{\pgfqpoint{4.712161in}{2.176552in}}%
\pgfpathlineto{\pgfqpoint{5.800283in}{2.176552in}}%
\pgfpathlineto{\pgfqpoint{5.800283in}{1.970670in}}%
\pgfpathlineto{\pgfqpoint{4.712161in}{1.970670in}}%
\pgfpathclose%
\pgfusepath{fill}%
\end{pgfscope}%
\begin{pgfscope}%
\pgfpathrectangle{\pgfqpoint{1.249956in}{0.148611in}}{\pgfqpoint{7.122250in}{3.850000in}}%
\pgfusepath{clip}%
\pgfsetbuttcap%
\pgfsetmiterjoin%
\definecolor{currentfill}{rgb}{0.662745,0.856055,0.423299}%
\pgfsetfillcolor{currentfill}%
\pgfsetlinewidth{0.000000pt}%
\definecolor{currentstroke}{rgb}{0.000000,0.000000,0.000000}%
\pgfsetstrokecolor{currentstroke}%
\pgfsetstrokeopacity{0.000000}%
\pgfsetdash{}{0pt}%
\pgfpathmoveto{\pgfqpoint{7.976526in}{1.764788in}}%
\pgfpathlineto{\pgfqpoint{8.108419in}{1.764788in}}%
\pgfpathlineto{\pgfqpoint{8.108419in}{1.558905in}}%
\pgfpathlineto{\pgfqpoint{7.976526in}{1.558905in}}%
\pgfpathclose%
\pgfusepath{fill}%
\end{pgfscope}%
\begin{pgfscope}%
\pgfpathrectangle{\pgfqpoint{1.249956in}{0.148611in}}{\pgfqpoint{7.122250in}{3.850000in}}%
\pgfusepath{clip}%
\pgfsetbuttcap%
\pgfsetmiterjoin%
\definecolor{currentfill}{rgb}{0.662745,0.856055,0.423299}%
\pgfsetfillcolor{currentfill}%
\pgfsetfillopacity{0.500000}%
\pgfsetlinewidth{0.000000pt}%
\definecolor{currentstroke}{rgb}{0.000000,0.000000,0.000000}%
\pgfsetstrokecolor{currentstroke}%
\pgfsetstrokeopacity{0.500000}%
\pgfsetdash{}{0pt}%
\pgfpathmoveto{\pgfqpoint{4.349454in}{1.353023in}}%
\pgfpathlineto{\pgfqpoint{6.130017in}{1.353023in}}%
\pgfpathlineto{\pgfqpoint{6.130017in}{1.147141in}}%
\pgfpathlineto{\pgfqpoint{4.349454in}{1.147141in}}%
\pgfpathclose%
\pgfusepath{fill}%
\end{pgfscope}%
\begin{pgfscope}%
\pgfpathrectangle{\pgfqpoint{1.249956in}{0.148611in}}{\pgfqpoint{7.122250in}{3.850000in}}%
\pgfusepath{clip}%
\pgfsetbuttcap%
\pgfsetmiterjoin%
\definecolor{currentfill}{rgb}{0.662745,0.856055,0.423299}%
\pgfsetfillcolor{currentfill}%
\pgfsetlinewidth{0.000000pt}%
\definecolor{currentstroke}{rgb}{0.000000,0.000000,0.000000}%
\pgfsetstrokecolor{currentstroke}%
\pgfsetstrokeopacity{0.000000}%
\pgfsetdash{}{0pt}%
\pgfpathmoveto{\pgfqpoint{8.141393in}{0.941258in}}%
\pgfpathlineto{\pgfqpoint{8.207339in}{0.941258in}}%
\pgfpathlineto{\pgfqpoint{8.207339in}{0.735376in}}%
\pgfpathlineto{\pgfqpoint{8.141393in}{0.735376in}}%
\pgfpathclose%
\pgfusepath{fill}%
\end{pgfscope}%
\begin{pgfscope}%
\pgfpathrectangle{\pgfqpoint{1.249956in}{0.148611in}}{\pgfqpoint{7.122250in}{3.850000in}}%
\pgfusepath{clip}%
\pgfsetbuttcap%
\pgfsetmiterjoin%
\definecolor{currentfill}{rgb}{0.662745,0.856055,0.423299}%
\pgfsetfillcolor{currentfill}%
\pgfsetlinewidth{0.000000pt}%
\definecolor{currentstroke}{rgb}{0.000000,0.000000,0.000000}%
\pgfsetstrokecolor{currentstroke}%
\pgfsetstrokeopacity{0.000000}%
\pgfsetdash{}{0pt}%
\pgfpathmoveto{\pgfqpoint{6.921378in}{0.529493in}}%
\pgfpathlineto{\pgfqpoint{7.646792in}{0.529493in}}%
\pgfpathlineto{\pgfqpoint{7.646792in}{0.323611in}}%
\pgfpathlineto{\pgfqpoint{6.921378in}{0.323611in}}%
\pgfpathclose%
\pgfusepath{fill}%
\end{pgfscope}%
\begin{pgfscope}%
\pgfpathrectangle{\pgfqpoint{1.249956in}{0.148611in}}{\pgfqpoint{7.122250in}{3.850000in}}%
\pgfusepath{clip}%
\pgfsetbuttcap%
\pgfsetmiterjoin%
\definecolor{currentfill}{rgb}{0.468897,0.771319,0.395771}%
\pgfsetfillcolor{currentfill}%
\pgfsetfillopacity{0.500000}%
\pgfsetlinewidth{0.000000pt}%
\definecolor{currentstroke}{rgb}{0.000000,0.000000,0.000000}%
\pgfsetstrokecolor{currentstroke}%
\pgfsetstrokeopacity{0.500000}%
\pgfsetdash{}{0pt}%
\pgfpathmoveto{\pgfqpoint{7.646792in}{3.823611in}}%
\pgfpathlineto{\pgfqpoint{8.174366in}{3.823611in}}%
\pgfpathlineto{\pgfqpoint{8.174366in}{3.617729in}}%
\pgfpathlineto{\pgfqpoint{7.646792in}{3.617729in}}%
\pgfpathclose%
\pgfusepath{fill}%
\end{pgfscope}%
\begin{pgfscope}%
\pgfpathrectangle{\pgfqpoint{1.249956in}{0.148611in}}{\pgfqpoint{7.122250in}{3.850000in}}%
\pgfusepath{clip}%
\pgfsetbuttcap%
\pgfsetmiterjoin%
\definecolor{currentfill}{rgb}{0.468897,0.771319,0.395771}%
\pgfsetfillcolor{currentfill}%
\pgfsetfillopacity{0.500000}%
\pgfsetlinewidth{0.000000pt}%
\definecolor{currentstroke}{rgb}{0.000000,0.000000,0.000000}%
\pgfsetstrokecolor{currentstroke}%
\pgfsetstrokeopacity{0.500000}%
\pgfsetdash{}{0pt}%
\pgfpathmoveto{\pgfqpoint{7.943552in}{3.411846in}}%
\pgfpathlineto{\pgfqpoint{8.141393in}{3.411846in}}%
\pgfpathlineto{\pgfqpoint{8.141393in}{3.205964in}}%
\pgfpathlineto{\pgfqpoint{7.943552in}{3.205964in}}%
\pgfpathclose%
\pgfusepath{fill}%
\end{pgfscope}%
\begin{pgfscope}%
\pgfpathrectangle{\pgfqpoint{1.249956in}{0.148611in}}{\pgfqpoint{7.122250in}{3.850000in}}%
\pgfusepath{clip}%
\pgfsetbuttcap%
\pgfsetmiterjoin%
\definecolor{currentfill}{rgb}{0.468897,0.771319,0.395771}%
\pgfsetfillcolor{currentfill}%
\pgfsetlinewidth{0.000000pt}%
\definecolor{currentstroke}{rgb}{0.000000,0.000000,0.000000}%
\pgfsetstrokecolor{currentstroke}%
\pgfsetstrokeopacity{0.000000}%
\pgfsetdash{}{0pt}%
\pgfpathmoveto{\pgfqpoint{7.547872in}{3.000082in}}%
\pgfpathlineto{\pgfqpoint{7.910579in}{3.000082in}}%
\pgfpathlineto{\pgfqpoint{7.910579in}{2.794199in}}%
\pgfpathlineto{\pgfqpoint{7.547872in}{2.794199in}}%
\pgfpathclose%
\pgfusepath{fill}%
\end{pgfscope}%
\begin{pgfscope}%
\pgfpathrectangle{\pgfqpoint{1.249956in}{0.148611in}}{\pgfqpoint{7.122250in}{3.850000in}}%
\pgfusepath{clip}%
\pgfsetbuttcap%
\pgfsetmiterjoin%
\definecolor{currentfill}{rgb}{0.468897,0.771319,0.395771}%
\pgfsetfillcolor{currentfill}%
\pgfsetfillopacity{0.500000}%
\pgfsetlinewidth{0.000000pt}%
\definecolor{currentstroke}{rgb}{0.000000,0.000000,0.000000}%
\pgfsetstrokecolor{currentstroke}%
\pgfsetstrokeopacity{0.500000}%
\pgfsetdash{}{0pt}%
\pgfpathmoveto{\pgfqpoint{7.284085in}{2.588317in}}%
\pgfpathlineto{\pgfqpoint{7.943552in}{2.588317in}}%
\pgfpathlineto{\pgfqpoint{7.943552in}{2.382435in}}%
\pgfpathlineto{\pgfqpoint{7.284085in}{2.382435in}}%
\pgfpathclose%
\pgfusepath{fill}%
\end{pgfscope}%
\begin{pgfscope}%
\pgfpathrectangle{\pgfqpoint{1.249956in}{0.148611in}}{\pgfqpoint{7.122250in}{3.850000in}}%
\pgfusepath{clip}%
\pgfsetbuttcap%
\pgfsetmiterjoin%
\definecolor{currentfill}{rgb}{0.468897,0.771319,0.395771}%
\pgfsetfillcolor{currentfill}%
\pgfsetfillopacity{0.500000}%
\pgfsetlinewidth{0.000000pt}%
\definecolor{currentstroke}{rgb}{0.000000,0.000000,0.000000}%
\pgfsetstrokecolor{currentstroke}%
\pgfsetstrokeopacity{0.500000}%
\pgfsetdash{}{0pt}%
\pgfpathmoveto{\pgfqpoint{5.800283in}{2.176552in}}%
\pgfpathlineto{\pgfqpoint{7.284085in}{2.176552in}}%
\pgfpathlineto{\pgfqpoint{7.284085in}{1.970670in}}%
\pgfpathlineto{\pgfqpoint{5.800283in}{1.970670in}}%
\pgfpathclose%
\pgfusepath{fill}%
\end{pgfscope}%
\begin{pgfscope}%
\pgfpathrectangle{\pgfqpoint{1.249956in}{0.148611in}}{\pgfqpoint{7.122250in}{3.850000in}}%
\pgfusepath{clip}%
\pgfsetbuttcap%
\pgfsetmiterjoin%
\definecolor{currentfill}{rgb}{0.468897,0.771319,0.395771}%
\pgfsetfillcolor{currentfill}%
\pgfsetlinewidth{0.000000pt}%
\definecolor{currentstroke}{rgb}{0.000000,0.000000,0.000000}%
\pgfsetstrokecolor{currentstroke}%
\pgfsetstrokeopacity{0.000000}%
\pgfsetdash{}{0pt}%
\pgfpathmoveto{\pgfqpoint{8.108419in}{1.764788in}}%
\pgfpathlineto{\pgfqpoint{8.207339in}{1.764788in}}%
\pgfpathlineto{\pgfqpoint{8.207339in}{1.558905in}}%
\pgfpathlineto{\pgfqpoint{8.108419in}{1.558905in}}%
\pgfpathclose%
\pgfusepath{fill}%
\end{pgfscope}%
\begin{pgfscope}%
\pgfpathrectangle{\pgfqpoint{1.249956in}{0.148611in}}{\pgfqpoint{7.122250in}{3.850000in}}%
\pgfusepath{clip}%
\pgfsetbuttcap%
\pgfsetmiterjoin%
\definecolor{currentfill}{rgb}{0.468897,0.771319,0.395771}%
\pgfsetfillcolor{currentfill}%
\pgfsetfillopacity{0.500000}%
\pgfsetlinewidth{0.000000pt}%
\definecolor{currentstroke}{rgb}{0.000000,0.000000,0.000000}%
\pgfsetstrokecolor{currentstroke}%
\pgfsetstrokeopacity{0.500000}%
\pgfsetdash{}{0pt}%
\pgfpathmoveto{\pgfqpoint{6.130017in}{1.353023in}}%
\pgfpathlineto{\pgfqpoint{7.218138in}{1.353023in}}%
\pgfpathlineto{\pgfqpoint{7.218138in}{1.147141in}}%
\pgfpathlineto{\pgfqpoint{6.130017in}{1.147141in}}%
\pgfpathclose%
\pgfusepath{fill}%
\end{pgfscope}%
\begin{pgfscope}%
\pgfpathrectangle{\pgfqpoint{1.249956in}{0.148611in}}{\pgfqpoint{7.122250in}{3.850000in}}%
\pgfusepath{clip}%
\pgfsetbuttcap%
\pgfsetmiterjoin%
\definecolor{currentfill}{rgb}{0.468897,0.771319,0.395771}%
\pgfsetfillcolor{currentfill}%
\pgfsetlinewidth{0.000000pt}%
\definecolor{currentstroke}{rgb}{0.000000,0.000000,0.000000}%
\pgfsetstrokecolor{currentstroke}%
\pgfsetstrokeopacity{0.000000}%
\pgfsetdash{}{0pt}%
\pgfpathmoveto{\pgfqpoint{8.207339in}{0.941258in}}%
\pgfpathlineto{\pgfqpoint{8.273286in}{0.941258in}}%
\pgfpathlineto{\pgfqpoint{8.273286in}{0.735376in}}%
\pgfpathlineto{\pgfqpoint{8.207339in}{0.735376in}}%
\pgfpathclose%
\pgfusepath{fill}%
\end{pgfscope}%
\begin{pgfscope}%
\pgfpathrectangle{\pgfqpoint{1.249956in}{0.148611in}}{\pgfqpoint{7.122250in}{3.850000in}}%
\pgfusepath{clip}%
\pgfsetbuttcap%
\pgfsetmiterjoin%
\definecolor{currentfill}{rgb}{0.468897,0.771319,0.395771}%
\pgfsetfillcolor{currentfill}%
\pgfsetlinewidth{0.000000pt}%
\definecolor{currentstroke}{rgb}{0.000000,0.000000,0.000000}%
\pgfsetstrokecolor{currentstroke}%
\pgfsetstrokeopacity{0.000000}%
\pgfsetdash{}{0pt}%
\pgfpathmoveto{\pgfqpoint{7.646792in}{0.529493in}}%
\pgfpathlineto{\pgfqpoint{8.009499in}{0.529493in}}%
\pgfpathlineto{\pgfqpoint{8.009499in}{0.323611in}}%
\pgfpathlineto{\pgfqpoint{7.646792in}{0.323611in}}%
\pgfpathclose%
\pgfusepath{fill}%
\end{pgfscope}%
\begin{pgfscope}%
\pgfpathrectangle{\pgfqpoint{1.249956in}{0.148611in}}{\pgfqpoint{7.122250in}{3.850000in}}%
\pgfusepath{clip}%
\pgfsetbuttcap%
\pgfsetmiterjoin%
\definecolor{currentfill}{rgb}{0.248058,0.667205,0.350250}%
\pgfsetfillcolor{currentfill}%
\pgfsetfillopacity{0.500000}%
\pgfsetlinewidth{0.000000pt}%
\definecolor{currentstroke}{rgb}{0.000000,0.000000,0.000000}%
\pgfsetstrokecolor{currentstroke}%
\pgfsetstrokeopacity{0.500000}%
\pgfsetdash{}{0pt}%
\pgfpathmoveto{\pgfqpoint{8.174366in}{3.823611in}}%
\pgfpathlineto{\pgfqpoint{8.372206in}{3.823611in}}%
\pgfpathlineto{\pgfqpoint{8.372206in}{3.617729in}}%
\pgfpathlineto{\pgfqpoint{8.174366in}{3.617729in}}%
\pgfpathclose%
\pgfusepath{fill}%
\end{pgfscope}%
\begin{pgfscope}%
\pgfpathrectangle{\pgfqpoint{1.249956in}{0.148611in}}{\pgfqpoint{7.122250in}{3.850000in}}%
\pgfusepath{clip}%
\pgfsetbuttcap%
\pgfsetmiterjoin%
\definecolor{currentfill}{rgb}{0.248058,0.667205,0.350250}%
\pgfsetfillcolor{currentfill}%
\pgfsetfillopacity{0.500000}%
\pgfsetlinewidth{0.000000pt}%
\definecolor{currentstroke}{rgb}{0.000000,0.000000,0.000000}%
\pgfsetstrokecolor{currentstroke}%
\pgfsetstrokeopacity{0.500000}%
\pgfsetdash{}{0pt}%
\pgfpathmoveto{\pgfqpoint{8.141393in}{3.411846in}}%
\pgfpathlineto{\pgfqpoint{8.372206in}{3.411846in}}%
\pgfpathlineto{\pgfqpoint{8.372206in}{3.205964in}}%
\pgfpathlineto{\pgfqpoint{8.141393in}{3.205964in}}%
\pgfpathclose%
\pgfusepath{fill}%
\end{pgfscope}%
\begin{pgfscope}%
\pgfpathrectangle{\pgfqpoint{1.249956in}{0.148611in}}{\pgfqpoint{7.122250in}{3.850000in}}%
\pgfusepath{clip}%
\pgfsetbuttcap%
\pgfsetmiterjoin%
\definecolor{currentfill}{rgb}{0.248058,0.667205,0.350250}%
\pgfsetfillcolor{currentfill}%
\pgfsetlinewidth{0.000000pt}%
\definecolor{currentstroke}{rgb}{0.000000,0.000000,0.000000}%
\pgfsetstrokecolor{currentstroke}%
\pgfsetstrokeopacity{0.000000}%
\pgfsetdash{}{0pt}%
\pgfpathmoveto{\pgfqpoint{7.910579in}{3.000082in}}%
\pgfpathlineto{\pgfqpoint{8.372206in}{3.000082in}}%
\pgfpathlineto{\pgfqpoint{8.372206in}{2.794199in}}%
\pgfpathlineto{\pgfqpoint{7.910579in}{2.794199in}}%
\pgfpathclose%
\pgfusepath{fill}%
\end{pgfscope}%
\begin{pgfscope}%
\pgfpathrectangle{\pgfqpoint{1.249956in}{0.148611in}}{\pgfqpoint{7.122250in}{3.850000in}}%
\pgfusepath{clip}%
\pgfsetbuttcap%
\pgfsetmiterjoin%
\definecolor{currentfill}{rgb}{0.248058,0.667205,0.350250}%
\pgfsetfillcolor{currentfill}%
\pgfsetfillopacity{0.500000}%
\pgfsetlinewidth{0.000000pt}%
\definecolor{currentstroke}{rgb}{0.000000,0.000000,0.000000}%
\pgfsetstrokecolor{currentstroke}%
\pgfsetstrokeopacity{0.500000}%
\pgfsetdash{}{0pt}%
\pgfpathmoveto{\pgfqpoint{7.943552in}{2.588317in}}%
\pgfpathlineto{\pgfqpoint{8.372206in}{2.588317in}}%
\pgfpathlineto{\pgfqpoint{8.372206in}{2.382435in}}%
\pgfpathlineto{\pgfqpoint{7.943552in}{2.382435in}}%
\pgfpathclose%
\pgfusepath{fill}%
\end{pgfscope}%
\begin{pgfscope}%
\pgfpathrectangle{\pgfqpoint{1.249956in}{0.148611in}}{\pgfqpoint{7.122250in}{3.850000in}}%
\pgfusepath{clip}%
\pgfsetbuttcap%
\pgfsetmiterjoin%
\definecolor{currentfill}{rgb}{0.248058,0.667205,0.350250}%
\pgfsetfillcolor{currentfill}%
\pgfsetfillopacity{0.500000}%
\pgfsetlinewidth{0.000000pt}%
\definecolor{currentstroke}{rgb}{0.000000,0.000000,0.000000}%
\pgfsetstrokecolor{currentstroke}%
\pgfsetstrokeopacity{0.500000}%
\pgfsetdash{}{0pt}%
\pgfpathmoveto{\pgfqpoint{7.284085in}{2.176552in}}%
\pgfpathlineto{\pgfqpoint{8.372206in}{2.176552in}}%
\pgfpathlineto{\pgfqpoint{8.372206in}{1.970670in}}%
\pgfpathlineto{\pgfqpoint{7.284085in}{1.970670in}}%
\pgfpathclose%
\pgfusepath{fill}%
\end{pgfscope}%
\begin{pgfscope}%
\pgfpathrectangle{\pgfqpoint{1.249956in}{0.148611in}}{\pgfqpoint{7.122250in}{3.850000in}}%
\pgfusepath{clip}%
\pgfsetbuttcap%
\pgfsetmiterjoin%
\definecolor{currentfill}{rgb}{0.248058,0.667205,0.350250}%
\pgfsetfillcolor{currentfill}%
\pgfsetlinewidth{0.000000pt}%
\definecolor{currentstroke}{rgb}{0.000000,0.000000,0.000000}%
\pgfsetstrokecolor{currentstroke}%
\pgfsetstrokeopacity{0.000000}%
\pgfsetdash{}{0pt}%
\pgfpathmoveto{\pgfqpoint{8.207339in}{1.764788in}}%
\pgfpathlineto{\pgfqpoint{8.372206in}{1.764788in}}%
\pgfpathlineto{\pgfqpoint{8.372206in}{1.558905in}}%
\pgfpathlineto{\pgfqpoint{8.207339in}{1.558905in}}%
\pgfpathclose%
\pgfusepath{fill}%
\end{pgfscope}%
\begin{pgfscope}%
\pgfpathrectangle{\pgfqpoint{1.249956in}{0.148611in}}{\pgfqpoint{7.122250in}{3.850000in}}%
\pgfusepath{clip}%
\pgfsetbuttcap%
\pgfsetmiterjoin%
\definecolor{currentfill}{rgb}{0.248058,0.667205,0.350250}%
\pgfsetfillcolor{currentfill}%
\pgfsetfillopacity{0.500000}%
\pgfsetlinewidth{0.000000pt}%
\definecolor{currentstroke}{rgb}{0.000000,0.000000,0.000000}%
\pgfsetstrokecolor{currentstroke}%
\pgfsetstrokeopacity{0.500000}%
\pgfsetdash{}{0pt}%
\pgfpathmoveto{\pgfqpoint{7.218138in}{1.353023in}}%
\pgfpathlineto{\pgfqpoint{8.372206in}{1.353023in}}%
\pgfpathlineto{\pgfqpoint{8.372206in}{1.147141in}}%
\pgfpathlineto{\pgfqpoint{7.218138in}{1.147141in}}%
\pgfpathclose%
\pgfusepath{fill}%
\end{pgfscope}%
\begin{pgfscope}%
\pgfpathrectangle{\pgfqpoint{1.249956in}{0.148611in}}{\pgfqpoint{7.122250in}{3.850000in}}%
\pgfusepath{clip}%
\pgfsetbuttcap%
\pgfsetmiterjoin%
\definecolor{currentfill}{rgb}{0.248058,0.667205,0.350250}%
\pgfsetfillcolor{currentfill}%
\pgfsetlinewidth{0.000000pt}%
\definecolor{currentstroke}{rgb}{0.000000,0.000000,0.000000}%
\pgfsetstrokecolor{currentstroke}%
\pgfsetstrokeopacity{0.000000}%
\pgfsetdash{}{0pt}%
\pgfpathmoveto{\pgfqpoint{8.273286in}{0.941258in}}%
\pgfpathlineto{\pgfqpoint{8.372206in}{0.941258in}}%
\pgfpathlineto{\pgfqpoint{8.372206in}{0.735376in}}%
\pgfpathlineto{\pgfqpoint{8.273286in}{0.735376in}}%
\pgfpathclose%
\pgfusepath{fill}%
\end{pgfscope}%
\begin{pgfscope}%
\pgfpathrectangle{\pgfqpoint{1.249956in}{0.148611in}}{\pgfqpoint{7.122250in}{3.850000in}}%
\pgfusepath{clip}%
\pgfsetbuttcap%
\pgfsetmiterjoin%
\definecolor{currentfill}{rgb}{0.248058,0.667205,0.350250}%
\pgfsetfillcolor{currentfill}%
\pgfsetlinewidth{0.000000pt}%
\definecolor{currentstroke}{rgb}{0.000000,0.000000,0.000000}%
\pgfsetstrokecolor{currentstroke}%
\pgfsetstrokeopacity{0.000000}%
\pgfsetdash{}{0pt}%
\pgfpathmoveto{\pgfqpoint{8.009499in}{0.529493in}}%
\pgfpathlineto{\pgfqpoint{8.372206in}{0.529493in}}%
\pgfpathlineto{\pgfqpoint{8.372206in}{0.323611in}}%
\pgfpathlineto{\pgfqpoint{8.009499in}{0.323611in}}%
\pgfpathclose%
\pgfusepath{fill}%
\end{pgfscope}%
\begin{pgfscope}%
\pgfsetbuttcap%
\pgfsetroundjoin%
\definecolor{currentfill}{rgb}{0.000000,0.000000,0.000000}%
\pgfsetfillcolor{currentfill}%
\pgfsetlinewidth{0.803000pt}%
\definecolor{currentstroke}{rgb}{0.000000,0.000000,0.000000}%
\pgfsetstrokecolor{currentstroke}%
\pgfsetdash{}{0pt}%
\pgfsys@defobject{currentmarker}{\pgfqpoint{-0.048611in}{0.000000in}}{\pgfqpoint{-0.000000in}{0.000000in}}{%
\pgfpathmoveto{\pgfqpoint{-0.000000in}{0.000000in}}%
\pgfpathlineto{\pgfqpoint{-0.048611in}{0.000000in}}%
\pgfusepath{stroke,fill}%
}%
\begin{pgfscope}%
\pgfsys@transformshift{1.249956in}{3.720670in}%
\pgfsys@useobject{currentmarker}{}%
\end{pgfscope}%
\end{pgfscope}%
\begin{pgfscope}%
\definecolor{textcolor}{rgb}{0.000000,0.000000,0.000000}%
\pgfsetstrokecolor{textcolor}%
\pgfsetfillcolor{textcolor}%
\pgftext[x=0.482975in, y=3.667908in, left, base]{\color{textcolor}\sffamily\fontsize{10.000000}{12.000000}\selectfont 3DFRONT}%
\end{pgfscope}%
\begin{pgfscope}%
\pgfsetbuttcap%
\pgfsetroundjoin%
\definecolor{currentfill}{rgb}{0.000000,0.000000,0.000000}%
\pgfsetfillcolor{currentfill}%
\pgfsetlinewidth{0.803000pt}%
\definecolor{currentstroke}{rgb}{0.000000,0.000000,0.000000}%
\pgfsetstrokecolor{currentstroke}%
\pgfsetdash{}{0pt}%
\pgfsys@defobject{currentmarker}{\pgfqpoint{-0.048611in}{0.000000in}}{\pgfqpoint{-0.000000in}{0.000000in}}{%
\pgfpathmoveto{\pgfqpoint{-0.000000in}{0.000000in}}%
\pgfpathlineto{\pgfqpoint{-0.048611in}{0.000000in}}%
\pgfusepath{stroke,fill}%
}%
\begin{pgfscope}%
\pgfsys@transformshift{1.249956in}{3.308905in}%
\pgfsys@useobject{currentmarker}{}%
\end{pgfscope}%
\end{pgfscope}%
\begin{pgfscope}%
\definecolor{textcolor}{rgb}{0.000000,0.000000,0.000000}%
\pgfsetstrokecolor{textcolor}%
\pgfsetfillcolor{textcolor}%
\pgftext[x=0.533295in, y=3.256144in, left, base]{\color{textcolor}\sffamily\fontsize{10.000000}{12.000000}\selectfont AI2THOR}%
\end{pgfscope}%
\begin{pgfscope}%
\pgfsetbuttcap%
\pgfsetroundjoin%
\definecolor{currentfill}{rgb}{0.000000,0.000000,0.000000}%
\pgfsetfillcolor{currentfill}%
\pgfsetlinewidth{0.803000pt}%
\definecolor{currentstroke}{rgb}{0.000000,0.000000,0.000000}%
\pgfsetstrokecolor{currentstroke}%
\pgfsetdash{}{0pt}%
\pgfsys@defobject{currentmarker}{\pgfqpoint{-0.048611in}{0.000000in}}{\pgfqpoint{-0.000000in}{0.000000in}}{%
\pgfpathmoveto{\pgfqpoint{-0.000000in}{0.000000in}}%
\pgfpathlineto{\pgfqpoint{-0.048611in}{0.000000in}}%
\pgfusepath{stroke,fill}%
}%
\begin{pgfscope}%
\pgfsys@transformshift{1.249956in}{2.897141in}%
\pgfsys@useobject{currentmarker}{}%
\end{pgfscope}%
\end{pgfscope}%
\begin{pgfscope}%
\definecolor{textcolor}{rgb}{0.000000,0.000000,0.000000}%
\pgfsetstrokecolor{textcolor}%
\pgfsetfillcolor{textcolor}%
\pgftext[x=0.311127in, y=2.844379in, left, base]{\color{textcolor}\sffamily\fontsize{10.000000}{12.000000}\selectfont Blenderproc}%
\end{pgfscope}%
\begin{pgfscope}%
\pgfsetbuttcap%
\pgfsetroundjoin%
\definecolor{currentfill}{rgb}{0.000000,0.000000,0.000000}%
\pgfsetfillcolor{currentfill}%
\pgfsetlinewidth{0.803000pt}%
\definecolor{currentstroke}{rgb}{0.000000,0.000000,0.000000}%
\pgfsetstrokecolor{currentstroke}%
\pgfsetdash{}{0pt}%
\pgfsys@defobject{currentmarker}{\pgfqpoint{-0.048611in}{0.000000in}}{\pgfqpoint{-0.000000in}{0.000000in}}{%
\pgfpathmoveto{\pgfqpoint{-0.000000in}{0.000000in}}%
\pgfpathlineto{\pgfqpoint{-0.048611in}{0.000000in}}%
\pgfusepath{stroke,fill}%
}%
\begin{pgfscope}%
\pgfsys@transformshift{1.249956in}{2.485376in}%
\pgfsys@useobject{currentmarker}{}%
\end{pgfscope}%
\end{pgfscope}%
\begin{pgfscope}%
\definecolor{textcolor}{rgb}{0.000000,0.000000,0.000000}%
\pgfsetstrokecolor{textcolor}%
\pgfsetfillcolor{textcolor}%
\pgftext[x=0.489146in, y=2.432614in, left, base]{\color{textcolor}\sffamily\fontsize{10.000000}{12.000000}\selectfont Hyperism}%
\end{pgfscope}%
\begin{pgfscope}%
\pgfsetbuttcap%
\pgfsetroundjoin%
\definecolor{currentfill}{rgb}{0.000000,0.000000,0.000000}%
\pgfsetfillcolor{currentfill}%
\pgfsetlinewidth{0.803000pt}%
\definecolor{currentstroke}{rgb}{0.000000,0.000000,0.000000}%
\pgfsetstrokecolor{currentstroke}%
\pgfsetdash{}{0pt}%
\pgfsys@defobject{currentmarker}{\pgfqpoint{-0.048611in}{0.000000in}}{\pgfqpoint{-0.000000in}{0.000000in}}{%
\pgfpathmoveto{\pgfqpoint{-0.000000in}{0.000000in}}%
\pgfpathlineto{\pgfqpoint{-0.048611in}{0.000000in}}%
\pgfusepath{stroke,fill}%
}%
\begin{pgfscope}%
\pgfsys@transformshift{1.249956in}{2.073611in}%
\pgfsys@useobject{currentmarker}{}%
\end{pgfscope}%
\end{pgfscope}%
\begin{pgfscope}%
\definecolor{textcolor}{rgb}{0.000000,0.000000,0.000000}%
\pgfsetstrokecolor{textcolor}%
\pgfsetfillcolor{textcolor}%
\pgftext[x=0.402273in, y=2.020850in, left, base]{\color{textcolor}\sffamily\fontsize{10.000000}{12.000000}\selectfont InteriorNet}%
\end{pgfscope}%
\begin{pgfscope}%
\pgfsetbuttcap%
\pgfsetroundjoin%
\definecolor{currentfill}{rgb}{0.000000,0.000000,0.000000}%
\pgfsetfillcolor{currentfill}%
\pgfsetlinewidth{0.803000pt}%
\definecolor{currentstroke}{rgb}{0.000000,0.000000,0.000000}%
\pgfsetstrokecolor{currentstroke}%
\pgfsetdash{}{0pt}%
\pgfsys@defobject{currentmarker}{\pgfqpoint{-0.048611in}{0.000000in}}{\pgfqpoint{-0.000000in}{0.000000in}}{%
\pgfpathmoveto{\pgfqpoint{-0.000000in}{0.000000in}}%
\pgfpathlineto{\pgfqpoint{-0.048611in}{0.000000in}}%
\pgfusepath{stroke,fill}%
}%
\begin{pgfscope}%
\pgfsys@transformshift{1.249956in}{1.661846in}%
\pgfsys@useobject{currentmarker}{}%
\end{pgfscope}%
\end{pgfscope}%
\begin{pgfscope}%
\definecolor{textcolor}{rgb}{0.000000,0.000000,0.000000}%
\pgfsetstrokecolor{textcolor}%
\pgfsetfillcolor{textcolor}%
\pgftext[x=0.313908in, y=1.609085in, left, base]{\color{textcolor}\sffamily\fontsize{10.000000}{12.000000}\selectfont OpenRooms}%
\end{pgfscope}%
\begin{pgfscope}%
\pgfsetbuttcap%
\pgfsetroundjoin%
\definecolor{currentfill}{rgb}{0.000000,0.000000,0.000000}%
\pgfsetfillcolor{currentfill}%
\pgfsetlinewidth{0.803000pt}%
\definecolor{currentstroke}{rgb}{0.000000,0.000000,0.000000}%
\pgfsetstrokecolor{currentstroke}%
\pgfsetdash{}{0pt}%
\pgfsys@defobject{currentmarker}{\pgfqpoint{-0.048611in}{0.000000in}}{\pgfqpoint{-0.000000in}{0.000000in}}{%
\pgfpathmoveto{\pgfqpoint{-0.000000in}{0.000000in}}%
\pgfpathlineto{\pgfqpoint{-0.048611in}{0.000000in}}%
\pgfusepath{stroke,fill}%
}%
\begin{pgfscope}%
\pgfsys@transformshift{1.249956in}{1.250082in}%
\pgfsys@useobject{currentmarker}{}%
\end{pgfscope}%
\end{pgfscope}%
\begin{pgfscope}%
\definecolor{textcolor}{rgb}{0.000000,0.000000,0.000000}%
\pgfsetstrokecolor{textcolor}%
\pgfsetfillcolor{textcolor}%
\pgftext[x=0.755938in, y=1.197320in, left, base]{\color{textcolor}\sffamily\fontsize{10.000000}{12.000000}\selectfont Pix3D}%
\end{pgfscope}%
\begin{pgfscope}%
\pgfsetbuttcap%
\pgfsetroundjoin%
\definecolor{currentfill}{rgb}{0.000000,0.000000,0.000000}%
\pgfsetfillcolor{currentfill}%
\pgfsetlinewidth{0.803000pt}%
\definecolor{currentstroke}{rgb}{0.000000,0.000000,0.000000}%
\pgfsetstrokecolor{currentstroke}%
\pgfsetdash{}{0pt}%
\pgfsys@defobject{currentmarker}{\pgfqpoint{-0.048611in}{0.000000in}}{\pgfqpoint{-0.000000in}{0.000000in}}{%
\pgfpathmoveto{\pgfqpoint{-0.000000in}{0.000000in}}%
\pgfpathlineto{\pgfqpoint{-0.048611in}{0.000000in}}%
\pgfusepath{stroke,fill}%
}%
\begin{pgfscope}%
\pgfsys@transformshift{1.249956in}{0.838317in}%
\pgfsys@useobject{currentmarker}{}%
\end{pgfscope}%
\end{pgfscope}%
\begin{pgfscope}%
\definecolor{textcolor}{rgb}{0.000000,0.000000,0.000000}%
\pgfsetstrokecolor{textcolor}%
\pgfsetfillcolor{textcolor}%
\pgftext[x=0.289968in, y=0.785555in, left, base]{\color{textcolor}\sffamily\fontsize{10.000000}{12.000000}\selectfont S2R:3DFREE}%
\end{pgfscope}%
\begin{pgfscope}%
\pgfsetbuttcap%
\pgfsetroundjoin%
\definecolor{currentfill}{rgb}{0.000000,0.000000,0.000000}%
\pgfsetfillcolor{currentfill}%
\pgfsetlinewidth{0.803000pt}%
\definecolor{currentstroke}{rgb}{0.000000,0.000000,0.000000}%
\pgfsetstrokecolor{currentstroke}%
\pgfsetdash{}{0pt}%
\pgfsys@defobject{currentmarker}{\pgfqpoint{-0.048611in}{0.000000in}}{\pgfqpoint{-0.000000in}{0.000000in}}{%
\pgfpathmoveto{\pgfqpoint{-0.000000in}{0.000000in}}%
\pgfpathlineto{\pgfqpoint{-0.048611in}{0.000000in}}%
\pgfusepath{stroke,fill}%
}%
\begin{pgfscope}%
\pgfsys@transformshift{1.249956in}{0.426552in}%
\pgfsys@useobject{currentmarker}{}%
\end{pgfscope}%
\end{pgfscope}%
\begin{pgfscope}%
\definecolor{textcolor}{rgb}{0.000000,0.000000,0.000000}%
\pgfsetstrokecolor{textcolor}%
\pgfsetfillcolor{textcolor}%
\pgftext[x=0.485484in, y=0.373791in, left, base]{\color{textcolor}\sffamily\fontsize{10.000000}{12.000000}\selectfont SceneNet}%
\end{pgfscope}%
\begin{pgfscope}%
\definecolor{textcolor}{rgb}{0.000000,0.000000,0.000000}%
\pgfsetstrokecolor{textcolor}%
\pgfsetfillcolor{textcolor}%
\pgftext[x=0.234413in,y=2.073611in,,bottom,rotate=90.000000]{\color{textcolor}\sffamily\fontsize{10.000000}{12.000000}\selectfont Datasets}%
\end{pgfscope}%
\begin{pgfscope}%
\pgfsetrectcap%
\pgfsetmiterjoin%
\pgfsetlinewidth{0.803000pt}%
\definecolor{currentstroke}{rgb}{0.000000,0.000000,0.000000}%
\pgfsetstrokecolor{currentstroke}%
\pgfsetdash{}{0pt}%
\pgfpathmoveto{\pgfqpoint{1.249956in}{0.148611in}}%
\pgfpathlineto{\pgfqpoint{1.249956in}{3.998611in}}%
\pgfusepath{stroke}%
\end{pgfscope}%
\begin{pgfscope}%
\pgfsetrectcap%
\pgfsetmiterjoin%
\pgfsetlinewidth{0.803000pt}%
\definecolor{currentstroke}{rgb}{0.000000,0.000000,0.000000}%
\pgfsetstrokecolor{currentstroke}%
\pgfsetdash{}{0pt}%
\pgfpathmoveto{\pgfqpoint{8.372206in}{0.148611in}}%
\pgfpathlineto{\pgfqpoint{8.372206in}{3.998611in}}%
\pgfusepath{stroke}%
\end{pgfscope}%
\begin{pgfscope}%
\pgfsetrectcap%
\pgfsetmiterjoin%
\pgfsetlinewidth{0.803000pt}%
\definecolor{currentstroke}{rgb}{0.000000,0.000000,0.000000}%
\pgfsetstrokecolor{currentstroke}%
\pgfsetdash{}{0pt}%
\pgfpathmoveto{\pgfqpoint{1.249956in}{0.148611in}}%
\pgfpathlineto{\pgfqpoint{8.372206in}{0.148611in}}%
\pgfusepath{stroke}%
\end{pgfscope}%
\begin{pgfscope}%
\pgfsetrectcap%
\pgfsetmiterjoin%
\pgfsetlinewidth{0.803000pt}%
\definecolor{currentstroke}{rgb}{0.000000,0.000000,0.000000}%
\pgfsetstrokecolor{currentstroke}%
\pgfsetdash{}{0pt}%
\pgfpathmoveto{\pgfqpoint{1.249956in}{3.998611in}}%
\pgfpathlineto{\pgfqpoint{8.372206in}{3.998611in}}%
\pgfusepath{stroke}%
\end{pgfscope}%
\begin{pgfscope}%
\definecolor{textcolor}{rgb}{1.000000,1.000000,1.000000}%
\pgfsetstrokecolor{textcolor}%
\pgfsetfillcolor{textcolor}%
\pgftext[x=1.876451in,y=3.720670in,,]{\color{textcolor}\sffamily\fontsize{10.000000}{12.000000}\selectfont 38}%
\end{pgfscope}%
\begin{pgfscope}%
\definecolor{textcolor}{rgb}{1.000000,1.000000,1.000000}%
\pgfsetstrokecolor{textcolor}%
\pgfsetfillcolor{textcolor}%
\pgftext[x=2.272131in,y=3.308905in,,]{\color{textcolor}\sffamily\fontsize{10.000000}{12.000000}\selectfont 62}%
\end{pgfscope}%
\begin{pgfscope}%
\definecolor{textcolor}{rgb}{1.000000,1.000000,1.000000}%
\pgfsetstrokecolor{textcolor}%
\pgfsetfillcolor{textcolor}%
\pgftext[x=1.975371in,y=2.897141in,,]{\color{textcolor}\sffamily\fontsize{10.000000}{12.000000}\selectfont 44}%
\end{pgfscope}%
\begin{pgfscope}%
\definecolor{textcolor}{rgb}{1.000000,1.000000,1.000000}%
\pgfsetstrokecolor{textcolor}%
\pgfsetfillcolor{textcolor}%
\pgftext[x=1.546717in,y=2.485376in,,]{\color{textcolor}\sffamily\fontsize{10.000000}{12.000000}\selectfont 18}%
\end{pgfscope}%
\begin{pgfscope}%
\definecolor{textcolor}{rgb}{1.000000,1.000000,1.000000}%
\pgfsetstrokecolor{textcolor}%
\pgfsetfillcolor{textcolor}%
\pgftext[x=1.431310in,y=2.073611in,,]{\color{textcolor}\sffamily\fontsize{10.000000}{12.000000}\selectfont 11}%
\end{pgfscope}%
\begin{pgfscope}%
\definecolor{textcolor}{rgb}{1.000000,1.000000,1.000000}%
\pgfsetstrokecolor{textcolor}%
\pgfsetfillcolor{textcolor}%
\pgftext[x=2.173211in,y=1.661846in,,]{\color{textcolor}\sffamily\fontsize{10.000000}{12.000000}\selectfont 56}%
\end{pgfscope}%
\begin{pgfscope}%
\definecolor{textcolor}{rgb}{1.000000,1.000000,1.000000}%
\pgfsetstrokecolor{textcolor}%
\pgfsetfillcolor{textcolor}%
\pgftext[x=1.464283in,y=1.250082in,,]{\color{textcolor}\sffamily\fontsize{10.000000}{12.000000}\selectfont 13}%
\end{pgfscope}%
\begin{pgfscope}%
\definecolor{textcolor}{rgb}{1.000000,1.000000,1.000000}%
\pgfsetstrokecolor{textcolor}%
\pgfsetfillcolor{textcolor}%
\pgftext[x=2.206184in,y=0.838317in,,]{\color{textcolor}\sffamily\fontsize{10.000000}{12.000000}\selectfont 58}%
\end{pgfscope}%
\begin{pgfscope}%
\definecolor{textcolor}{rgb}{1.000000,1.000000,1.000000}%
\pgfsetstrokecolor{textcolor}%
\pgfsetfillcolor{textcolor}%
\pgftext[x=1.975371in,y=0.426552in,,]{\color{textcolor}\sffamily\fontsize{10.000000}{12.000000}\selectfont 44}%
\end{pgfscope}%
\begin{pgfscope}%
\definecolor{textcolor}{rgb}{1.000000,1.000000,1.000000}%
\pgfsetstrokecolor{textcolor}%
\pgfsetfillcolor{textcolor}%
\pgftext[x=2.849165in,y=3.720670in,,]{\color{textcolor}\sffamily\fontsize{10.000000}{12.000000}\selectfont 21}%
\end{pgfscope}%
\begin{pgfscope}%
\definecolor{textcolor}{rgb}{1.000000,1.000000,1.000000}%
\pgfsetstrokecolor{textcolor}%
\pgfsetfillcolor{textcolor}%
\pgftext[x=3.755933in,y=3.308905in,,]{\color{textcolor}\sffamily\fontsize{10.000000}{12.000000}\selectfont 28}%
\end{pgfscope}%
\begin{pgfscope}%
\definecolor{textcolor}{rgb}{1.000000,1.000000,1.000000}%
\pgfsetstrokecolor{textcolor}%
\pgfsetfillcolor{textcolor}%
\pgftext[x=3.195386in,y=2.897141in,,]{\color{textcolor}\sffamily\fontsize{10.000000}{12.000000}\selectfont 30}%
\end{pgfscope}%
\begin{pgfscope}%
\definecolor{textcolor}{rgb}{1.000000,1.000000,1.000000}%
\pgfsetstrokecolor{textcolor}%
\pgfsetfillcolor{textcolor}%
\pgftext[x=2.206184in,y=2.485376in,,]{\color{textcolor}\sffamily\fontsize{10.000000}{12.000000}\selectfont 22}%
\end{pgfscope}%
\begin{pgfscope}%
\definecolor{textcolor}{rgb}{1.000000,1.000000,1.000000}%
\pgfsetstrokecolor{textcolor}%
\pgfsetfillcolor{textcolor}%
\pgftext[x=1.744557in,y=2.073611in,,]{\color{textcolor}\sffamily\fontsize{10.000000}{12.000000}\selectfont 8}%
\end{pgfscope}%
\begin{pgfscope}%
\definecolor{textcolor}{rgb}{1.000000,1.000000,1.000000}%
\pgfsetstrokecolor{textcolor}%
\pgfsetfillcolor{textcolor}%
\pgftext[x=3.788907in,y=1.661846in,,]{\color{textcolor}\sffamily\fontsize{10.000000}{12.000000}\selectfont 42}%
\end{pgfscope}%
\begin{pgfscope}%
\definecolor{textcolor}{rgb}{1.000000,1.000000,1.000000}%
\pgfsetstrokecolor{textcolor}%
\pgfsetfillcolor{textcolor}%
\pgftext[x=1.794017in,y=1.250082in,,]{\color{textcolor}\sffamily\fontsize{10.000000}{12.000000}\selectfont 7}%
\end{pgfscope}%
\begin{pgfscope}%
\definecolor{textcolor}{rgb}{1.000000,1.000000,1.000000}%
\pgfsetstrokecolor{textcolor}%
\pgfsetfillcolor{textcolor}%
\pgftext[x=3.739447in,y=0.838317in,,]{\color{textcolor}\sffamily\fontsize{10.000000}{12.000000}\selectfont 35}%
\end{pgfscope}%
\begin{pgfscope}%
\definecolor{textcolor}{rgb}{1.000000,1.000000,1.000000}%
\pgfsetstrokecolor{textcolor}%
\pgfsetfillcolor{textcolor}%
\pgftext[x=3.047006in,y=0.426552in,,]{\color{textcolor}\sffamily\fontsize{10.000000}{12.000000}\selectfont 21}%
\end{pgfscope}%
\begin{pgfscope}%
\definecolor{textcolor}{rgb}{1.000000,1.000000,1.000000}%
\pgfsetstrokecolor{textcolor}%
\pgfsetfillcolor{textcolor}%
\pgftext[x=3.624040in,y=3.720670in,,]{\color{textcolor}\sffamily\fontsize{10.000000}{12.000000}\selectfont 26}%
\end{pgfscope}%
\begin{pgfscope}%
\definecolor{textcolor}{rgb}{1.000000,1.000000,1.000000}%
\pgfsetstrokecolor{textcolor}%
\pgfsetfillcolor{textcolor}%
\pgftext[x=4.728648in,y=3.308905in,,]{\color{textcolor}\sffamily\fontsize{10.000000}{12.000000}\selectfont 31}%
\end{pgfscope}%
\begin{pgfscope}%
\definecolor{textcolor}{rgb}{1.000000,1.000000,1.000000}%
\pgfsetstrokecolor{textcolor}%
\pgfsetfillcolor{textcolor}%
\pgftext[x=4.201074in,y=2.897141in,,]{\color{textcolor}\sffamily\fontsize{10.000000}{12.000000}\selectfont 31}%
\end{pgfscope}%
\begin{pgfscope}%
\definecolor{textcolor}{rgb}{1.000000,1.000000,1.000000}%
\pgfsetstrokecolor{textcolor}%
\pgfsetfillcolor{textcolor}%
\pgftext[x=3.030519in,y=2.485376in,,]{\color{textcolor}\sffamily\fontsize{10.000000}{12.000000}\selectfont 28}%
\end{pgfscope}%
\begin{pgfscope}%
\definecolor{textcolor}{rgb}{1.000000,1.000000,1.000000}%
\pgfsetstrokecolor{textcolor}%
\pgfsetfillcolor{textcolor}%
\pgftext[x=2.008344in,y=2.073611in,,]{\color{textcolor}\sffamily\fontsize{10.000000}{12.000000}\selectfont 8}%
\end{pgfscope}%
\begin{pgfscope}%
\definecolor{textcolor}{rgb}{1.000000,1.000000,1.000000}%
\pgfsetstrokecolor{textcolor}%
\pgfsetfillcolor{textcolor}%
\pgftext[x=4.942975in,y=1.661846in,,]{\color{textcolor}\sffamily\fontsize{10.000000}{12.000000}\selectfont 28}%
\end{pgfscope}%
\begin{pgfscope}%
\definecolor{textcolor}{rgb}{1.000000,1.000000,1.000000}%
\pgfsetstrokecolor{textcolor}%
\pgfsetfillcolor{textcolor}%
\pgftext[x=2.024831in,y=1.250082in,,]{\color{textcolor}\sffamily\fontsize{10.000000}{12.000000}\selectfont 7}%
\end{pgfscope}%
\begin{pgfscope}%
\definecolor{textcolor}{rgb}{1.000000,1.000000,1.000000}%
\pgfsetstrokecolor{textcolor}%
\pgfsetfillcolor{textcolor}%
\pgftext[x=5.124328in,y=0.838317in,,]{\color{textcolor}\sffamily\fontsize{10.000000}{12.000000}\selectfont 49}%
\end{pgfscope}%
\begin{pgfscope}%
\definecolor{textcolor}{rgb}{1.000000,1.000000,1.000000}%
\pgfsetstrokecolor{textcolor}%
\pgfsetfillcolor{textcolor}%
\pgftext[x=3.722960in,y=0.426552in,,]{\color{textcolor}\sffamily\fontsize{10.000000}{12.000000}\selectfont 20}%
\end{pgfscope}%
\begin{pgfscope}%
\definecolor{textcolor}{rgb}{1.000000,1.000000,1.000000}%
\pgfsetstrokecolor{textcolor}%
\pgfsetfillcolor{textcolor}%
\pgftext[x=4.497834in,y=3.720670in,,]{\color{textcolor}\sffamily\fontsize{10.000000}{12.000000}\selectfont 27}%
\end{pgfscope}%
\begin{pgfscope}%
\definecolor{textcolor}{rgb}{1.000000,1.000000,1.000000}%
\pgfsetstrokecolor{textcolor}%
\pgfsetfillcolor{textcolor}%
\pgftext[x=5.800283in,y=3.308905in,,]{\color{textcolor}\sffamily\fontsize{10.000000}{12.000000}\selectfont 34}%
\end{pgfscope}%
\begin{pgfscope}%
\definecolor{textcolor}{rgb}{1.000000,1.000000,1.000000}%
\pgfsetstrokecolor{textcolor}%
\pgfsetfillcolor{textcolor}%
\pgftext[x=5.041895in,y=2.897141in,,]{\color{textcolor}\sffamily\fontsize{10.000000}{12.000000}\selectfont 20}%
\end{pgfscope}%
\begin{pgfscope}%
\definecolor{textcolor}{rgb}{1.000000,1.000000,1.000000}%
\pgfsetstrokecolor{textcolor}%
\pgfsetfillcolor{textcolor}%
\pgftext[x=4.003234in,y=2.485376in,,]{\color{textcolor}\sffamily\fontsize{10.000000}{12.000000}\selectfont 31}%
\end{pgfscope}%
\begin{pgfscope}%
\definecolor{textcolor}{rgb}{1.000000,1.000000,1.000000}%
\pgfsetstrokecolor{textcolor}%
\pgfsetfillcolor{textcolor}%
\pgftext[x=2.519432in,y=2.073611in,,]{\color{textcolor}\sffamily\fontsize{10.000000}{12.000000}\selectfont 23}%
\end{pgfscope}%
\begin{pgfscope}%
\definecolor{textcolor}{rgb}{1.000000,1.000000,1.000000}%
\pgfsetstrokecolor{textcolor}%
\pgfsetfillcolor{textcolor}%
\pgftext[x=5.948663in,y=1.661846in,,]{\color{textcolor}\sffamily\fontsize{10.000000}{12.000000}\selectfont 33}%
\end{pgfscope}%
\begin{pgfscope}%
\definecolor{textcolor}{rgb}{1.000000,1.000000,1.000000}%
\pgfsetstrokecolor{textcolor}%
\pgfsetfillcolor{textcolor}%
\pgftext[x=2.354565in,y=1.250082in,,]{\color{textcolor}\sffamily\fontsize{10.000000}{12.000000}\selectfont 13}%
\end{pgfscope}%
\begin{pgfscope}%
\definecolor{textcolor}{rgb}{1.000000,1.000000,1.000000}%
\pgfsetstrokecolor{textcolor}%
\pgfsetfillcolor{textcolor}%
\pgftext[x=6.426777in,y=0.838317in,,]{\color{textcolor}\sffamily\fontsize{10.000000}{12.000000}\selectfont 30}%
\end{pgfscope}%
\begin{pgfscope}%
\definecolor{textcolor}{rgb}{1.000000,1.000000,1.000000}%
\pgfsetstrokecolor{textcolor}%
\pgfsetfillcolor{textcolor}%
\pgftext[x=4.530808in,y=0.426552in,,]{\color{textcolor}\sffamily\fontsize{10.000000}{12.000000}\selectfont 29}%
\end{pgfscope}%
\begin{pgfscope}%
\definecolor{textcolor}{rgb}{0.662745,0.662745,0.662745}%
\pgfsetstrokecolor{textcolor}%
\pgfsetfillcolor{textcolor}%
\pgftext[x=5.157302in,y=3.720670in,,]{\color{textcolor}\sffamily\fontsize{10.000000}{12.000000}\selectfont 13}%
\end{pgfscope}%
\begin{pgfscope}%
\definecolor{textcolor}{rgb}{0.662745,0.662745,0.662745}%
\pgfsetstrokecolor{textcolor}%
\pgfsetfillcolor{textcolor}%
\pgftext[x=6.608131in,y=3.308905in,,]{\color{textcolor}\sffamily\fontsize{10.000000}{12.000000}\selectfont 15}%
\end{pgfscope}%
\begin{pgfscope}%
\definecolor{textcolor}{rgb}{0.662745,0.662745,0.662745}%
\pgfsetstrokecolor{textcolor}%
\pgfsetfillcolor{textcolor}%
\pgftext[x=5.783796in,y=2.897141in,,]{\color{textcolor}\sffamily\fontsize{10.000000}{12.000000}\selectfont 25}%
\end{pgfscope}%
\begin{pgfscope}%
\definecolor{textcolor}{rgb}{0.662745,0.662745,0.662745}%
\pgfsetstrokecolor{textcolor}%
\pgfsetfillcolor{textcolor}%
\pgftext[x=4.877028in,y=2.485376in,,]{\color{textcolor}\sffamily\fontsize{10.000000}{12.000000}\selectfont 22}%
\end{pgfscope}%
\begin{pgfscope}%
\definecolor{textcolor}{rgb}{0.662745,0.662745,0.662745}%
\pgfsetstrokecolor{textcolor}%
\pgfsetfillcolor{textcolor}%
\pgftext[x=3.162412in,y=2.073611in,,]{\color{textcolor}\sffamily\fontsize{10.000000}{12.000000}\selectfont 16}%
\end{pgfscope}%
\begin{pgfscope}%
\definecolor{textcolor}{rgb}{0.662745,0.662745,0.662745}%
\pgfsetstrokecolor{textcolor}%
\pgfsetfillcolor{textcolor}%
\pgftext[x=6.805971in,y=1.661846in,,]{\color{textcolor}\sffamily\fontsize{10.000000}{12.000000}\selectfont 19}%
\end{pgfscope}%
\begin{pgfscope}%
\definecolor{textcolor}{rgb}{0.662745,0.662745,0.662745}%
\pgfsetstrokecolor{textcolor}%
\pgfsetfillcolor{textcolor}%
\pgftext[x=2.865652in,y=1.250082in,,]{\color{textcolor}\sffamily\fontsize{10.000000}{12.000000}\selectfont 18}%
\end{pgfscope}%
\begin{pgfscope}%
\definecolor{textcolor}{rgb}{0.662745,0.662745,0.662745}%
\pgfsetstrokecolor{textcolor}%
\pgfsetfillcolor{textcolor}%
\pgftext[x=7.135705in,y=0.838317in,,]{\color{textcolor}\sffamily\fontsize{10.000000}{12.000000}\selectfont 13}%
\end{pgfscope}%
\begin{pgfscope}%
\definecolor{textcolor}{rgb}{0.662745,0.662745,0.662745}%
\pgfsetstrokecolor{textcolor}%
\pgfsetfillcolor{textcolor}%
\pgftext[x=5.272709in,y=0.426552in,,]{\color{textcolor}\sffamily\fontsize{10.000000}{12.000000}\selectfont 16}%
\end{pgfscope}%
\begin{pgfscope}%
\definecolor{textcolor}{rgb}{0.662745,0.662745,0.662745}%
\pgfsetstrokecolor{textcolor}%
\pgfsetfillcolor{textcolor}%
\pgftext[x=5.882716in,y=3.720670in,,]{\color{textcolor}\sffamily\fontsize{10.000000}{12.000000}\selectfont 31}%
\end{pgfscope}%
\begin{pgfscope}%
\definecolor{textcolor}{rgb}{0.662745,0.662745,0.662745}%
\pgfsetstrokecolor{textcolor}%
\pgfsetfillcolor{textcolor}%
\pgftext[x=7.053271in,y=3.308905in,,]{\color{textcolor}\sffamily\fontsize{10.000000}{12.000000}\selectfont 12}%
\end{pgfscope}%
\begin{pgfscope}%
\definecolor{textcolor}{rgb}{0.662745,0.662745,0.662745}%
\pgfsetstrokecolor{textcolor}%
\pgfsetfillcolor{textcolor}%
\pgftext[x=6.443264in,y=2.897141in,,]{\color{textcolor}\sffamily\fontsize{10.000000}{12.000000}\selectfont 15}%
\end{pgfscope}%
\begin{pgfscope}%
\definecolor{textcolor}{rgb}{0.662745,0.662745,0.662745}%
\pgfsetstrokecolor{textcolor}%
\pgfsetfillcolor{textcolor}%
\pgftext[x=5.470549in,y=2.485376in,,]{\color{textcolor}\sffamily\fontsize{10.000000}{12.000000}\selectfont 14}%
\end{pgfscope}%
\begin{pgfscope}%
\definecolor{textcolor}{rgb}{0.662745,0.662745,0.662745}%
\pgfsetstrokecolor{textcolor}%
\pgfsetfillcolor{textcolor}%
\pgftext[x=3.739447in,y=2.073611in,,]{\color{textcolor}\sffamily\fontsize{10.000000}{12.000000}\selectfont 19}%
\end{pgfscope}%
\begin{pgfscope}%
\definecolor{textcolor}{rgb}{0.662745,0.662745,0.662745}%
\pgfsetstrokecolor{textcolor}%
\pgfsetfillcolor{textcolor}%
\pgftext[x=7.415978in,y=1.661846in,,]{\color{textcolor}\sffamily\fontsize{10.000000}{12.000000}\selectfont 18}%
\end{pgfscope}%
\begin{pgfscope}%
\definecolor{textcolor}{rgb}{0.662745,0.662745,0.662745}%
\pgfsetstrokecolor{textcolor}%
\pgfsetfillcolor{textcolor}%
\pgftext[x=3.475660in,y=1.250082in,,]{\color{textcolor}\sffamily\fontsize{10.000000}{12.000000}\selectfont 19}%
\end{pgfscope}%
\begin{pgfscope}%
\definecolor{textcolor}{rgb}{0.662745,0.662745,0.662745}%
\pgfsetstrokecolor{textcolor}%
\pgfsetfillcolor{textcolor}%
\pgftext[x=7.613819in,y=0.838317in,,]{\color{textcolor}\sffamily\fontsize{10.000000}{12.000000}\selectfont 16}%
\end{pgfscope}%
\begin{pgfscope}%
\definecolor{textcolor}{rgb}{0.662745,0.662745,0.662745}%
\pgfsetstrokecolor{textcolor}%
\pgfsetfillcolor{textcolor}%
\pgftext[x=5.849743in,y=0.426552in,,]{\color{textcolor}\sffamily\fontsize{10.000000}{12.000000}\selectfont 19}%
\end{pgfscope}%
\begin{pgfscope}%
\definecolor{textcolor}{rgb}{1.000000,1.000000,1.000000}%
\pgfsetstrokecolor{textcolor}%
\pgfsetfillcolor{textcolor}%
\pgftext[x=6.756511in,y=3.720670in,,]{\color{textcolor}\sffamily\fontsize{10.000000}{12.000000}\selectfont 22}%
\end{pgfscope}%
\begin{pgfscope}%
\definecolor{textcolor}{rgb}{1.000000,1.000000,1.000000}%
\pgfsetstrokecolor{textcolor}%
\pgfsetfillcolor{textcolor}%
\pgftext[x=7.432465in,y=3.308905in,,]{\color{textcolor}\sffamily\fontsize{10.000000}{12.000000}\selectfont 11}%
\end{pgfscope}%
\begin{pgfscope}%
\definecolor{textcolor}{rgb}{1.000000,1.000000,1.000000}%
\pgfsetstrokecolor{textcolor}%
\pgfsetfillcolor{textcolor}%
\pgftext[x=6.904891in,y=2.897141in,,]{\color{textcolor}\sffamily\fontsize{10.000000}{12.000000}\selectfont 13}%
\end{pgfscope}%
\begin{pgfscope}%
\definecolor{textcolor}{rgb}{1.000000,1.000000,1.000000}%
\pgfsetstrokecolor{textcolor}%
\pgfsetfillcolor{textcolor}%
\pgftext[x=6.080557in,y=2.485376in,,]{\color{textcolor}\sffamily\fontsize{10.000000}{12.000000}\selectfont 23}%
\end{pgfscope}%
\begin{pgfscope}%
\definecolor{textcolor}{rgb}{1.000000,1.000000,1.000000}%
\pgfsetstrokecolor{textcolor}%
\pgfsetfillcolor{textcolor}%
\pgftext[x=4.382427in,y=2.073611in,,]{\color{textcolor}\sffamily\fontsize{10.000000}{12.000000}\selectfont 20}%
\end{pgfscope}%
\begin{pgfscope}%
\definecolor{textcolor}{rgb}{1.000000,1.000000,1.000000}%
\pgfsetstrokecolor{textcolor}%
\pgfsetfillcolor{textcolor}%
\pgftext[x=7.844632in,y=1.661846in,,]{\color{textcolor}\sffamily\fontsize{10.000000}{12.000000}\selectfont 8}%
\end{pgfscope}%
\begin{pgfscope}%
\definecolor{textcolor}{rgb}{1.000000,1.000000,1.000000}%
\pgfsetstrokecolor{textcolor}%
\pgfsetfillcolor{textcolor}%
\pgftext[x=4.069180in,y=1.250082in,,]{\color{textcolor}\sffamily\fontsize{10.000000}{12.000000}\selectfont 17}%
\end{pgfscope}%
\begin{pgfscope}%
\definecolor{textcolor}{rgb}{1.000000,1.000000,1.000000}%
\pgfsetstrokecolor{textcolor}%
\pgfsetfillcolor{textcolor}%
\pgftext[x=8.009499in,y=0.838317in,,]{\color{textcolor}\sffamily\fontsize{10.000000}{12.000000}\selectfont 8}%
\end{pgfscope}%
\begin{pgfscope}%
\definecolor{textcolor}{rgb}{1.000000,1.000000,1.000000}%
\pgfsetstrokecolor{textcolor}%
\pgfsetfillcolor{textcolor}%
\pgftext[x=6.542184in,y=0.426552in,,]{\color{textcolor}\sffamily\fontsize{10.000000}{12.000000}\selectfont 23}%
\end{pgfscope}%
\begin{pgfscope}%
\definecolor{textcolor}{rgb}{1.000000,1.000000,1.000000}%
\pgfsetstrokecolor{textcolor}%
\pgfsetfillcolor{textcolor}%
\pgftext[x=7.383005in,y=3.720670in,,]{\color{textcolor}\sffamily\fontsize{10.000000}{12.000000}\selectfont 16}%
\end{pgfscope}%
\begin{pgfscope}%
\definecolor{textcolor}{rgb}{1.000000,1.000000,1.000000}%
\pgfsetstrokecolor{textcolor}%
\pgfsetfillcolor{textcolor}%
\pgftext[x=7.778686in,y=3.308905in,,]{\color{textcolor}\sffamily\fontsize{10.000000}{12.000000}\selectfont 10}%
\end{pgfscope}%
\begin{pgfscope}%
\definecolor{textcolor}{rgb}{1.000000,1.000000,1.000000}%
\pgfsetstrokecolor{textcolor}%
\pgfsetfillcolor{textcolor}%
\pgftext[x=7.333545in,y=2.897141in,,]{\color{textcolor}\sffamily\fontsize{10.000000}{12.000000}\selectfont 13}%
\end{pgfscope}%
\begin{pgfscope}%
\definecolor{textcolor}{rgb}{1.000000,1.000000,1.000000}%
\pgfsetstrokecolor{textcolor}%
\pgfsetfillcolor{textcolor}%
\pgftext[x=6.871918in,y=2.485376in,,]{\color{textcolor}\sffamily\fontsize{10.000000}{12.000000}\selectfont 25}%
\end{pgfscope}%
\begin{pgfscope}%
\definecolor{textcolor}{rgb}{1.000000,1.000000,1.000000}%
\pgfsetstrokecolor{textcolor}%
\pgfsetfillcolor{textcolor}%
\pgftext[x=5.256222in,y=2.073611in,,]{\color{textcolor}\sffamily\fontsize{10.000000}{12.000000}\selectfont 33}%
\end{pgfscope}%
\begin{pgfscope}%
\definecolor{textcolor}{rgb}{1.000000,1.000000,1.000000}%
\pgfsetstrokecolor{textcolor}%
\pgfsetfillcolor{textcolor}%
\pgftext[x=8.042473in,y=1.661846in,,]{\color{textcolor}\sffamily\fontsize{10.000000}{12.000000}\selectfont 4}%
\end{pgfscope}%
\begin{pgfscope}%
\definecolor{textcolor}{rgb}{1.000000,1.000000,1.000000}%
\pgfsetstrokecolor{textcolor}%
\pgfsetfillcolor{textcolor}%
\pgftext[x=5.239735in,y=1.250082in,,]{\color{textcolor}\sffamily\fontsize{10.000000}{12.000000}\selectfont 54}%
\end{pgfscope}%
\begin{pgfscope}%
\definecolor{textcolor}{rgb}{1.000000,1.000000,1.000000}%
\pgfsetstrokecolor{textcolor}%
\pgfsetfillcolor{textcolor}%
\pgftext[x=8.174366in,y=0.838317in,,]{\color{textcolor}\sffamily\fontsize{10.000000}{12.000000}\selectfont 2}%
\end{pgfscope}%
\begin{pgfscope}%
\definecolor{textcolor}{rgb}{1.000000,1.000000,1.000000}%
\pgfsetstrokecolor{textcolor}%
\pgfsetfillcolor{textcolor}%
\pgftext[x=7.284085in,y=0.426552in,,]{\color{textcolor}\sffamily\fontsize{10.000000}{12.000000}\selectfont 22}%
\end{pgfscope}%
\begin{pgfscope}%
\definecolor{textcolor}{rgb}{1.000000,1.000000,1.000000}%
\pgfsetstrokecolor{textcolor}%
\pgfsetfillcolor{textcolor}%
\pgftext[x=7.910579in,y=3.720670in,,]{\color{textcolor}\sffamily\fontsize{10.000000}{12.000000}\selectfont 16}%
\end{pgfscope}%
\begin{pgfscope}%
\definecolor{textcolor}{rgb}{1.000000,1.000000,1.000000}%
\pgfsetstrokecolor{textcolor}%
\pgfsetfillcolor{textcolor}%
\pgftext[x=8.042473in,y=3.308905in,,]{\color{textcolor}\sffamily\fontsize{10.000000}{12.000000}\selectfont 6}%
\end{pgfscope}%
\begin{pgfscope}%
\definecolor{textcolor}{rgb}{1.000000,1.000000,1.000000}%
\pgfsetstrokecolor{textcolor}%
\pgfsetfillcolor{textcolor}%
\pgftext[x=7.729225in,y=2.897141in,,]{\color{textcolor}\sffamily\fontsize{10.000000}{12.000000}\selectfont 11}%
\end{pgfscope}%
\begin{pgfscope}%
\definecolor{textcolor}{rgb}{1.000000,1.000000,1.000000}%
\pgfsetstrokecolor{textcolor}%
\pgfsetfillcolor{textcolor}%
\pgftext[x=7.613819in,y=2.485376in,,]{\color{textcolor}\sffamily\fontsize{10.000000}{12.000000}\selectfont 20}%
\end{pgfscope}%
\begin{pgfscope}%
\definecolor{textcolor}{rgb}{1.000000,1.000000,1.000000}%
\pgfsetstrokecolor{textcolor}%
\pgfsetfillcolor{textcolor}%
\pgftext[x=6.542184in,y=2.073611in,,]{\color{textcolor}\sffamily\fontsize{10.000000}{12.000000}\selectfont 45}%
\end{pgfscope}%
\begin{pgfscope}%
\definecolor{textcolor}{rgb}{1.000000,1.000000,1.000000}%
\pgfsetstrokecolor{textcolor}%
\pgfsetfillcolor{textcolor}%
\pgftext[x=8.157879in,y=1.661846in,,]{\color{textcolor}\sffamily\fontsize{10.000000}{12.000000}\selectfont 3}%
\end{pgfscope}%
\begin{pgfscope}%
\definecolor{textcolor}{rgb}{1.000000,1.000000,1.000000}%
\pgfsetstrokecolor{textcolor}%
\pgfsetfillcolor{textcolor}%
\pgftext[x=6.674077in,y=1.250082in,,]{\color{textcolor}\sffamily\fontsize{10.000000}{12.000000}\selectfont 33}%
\end{pgfscope}%
\begin{pgfscope}%
\definecolor{textcolor}{rgb}{1.000000,1.000000,1.000000}%
\pgfsetstrokecolor{textcolor}%
\pgfsetfillcolor{textcolor}%
\pgftext[x=8.240313in,y=0.838317in,,]{\color{textcolor}\sffamily\fontsize{10.000000}{12.000000}\selectfont 2}%
\end{pgfscope}%
\begin{pgfscope}%
\definecolor{textcolor}{rgb}{1.000000,1.000000,1.000000}%
\pgfsetstrokecolor{textcolor}%
\pgfsetfillcolor{textcolor}%
\pgftext[x=7.828146in,y=0.426552in,,]{\color{textcolor}\sffamily\fontsize{10.000000}{12.000000}\selectfont 11}%
\end{pgfscope}%
\begin{pgfscope}%
\definecolor{textcolor}{rgb}{1.000000,1.000000,1.000000}%
\pgfsetstrokecolor{textcolor}%
\pgfsetfillcolor{textcolor}%
\pgftext[x=8.273286in,y=3.720670in,,]{\color{textcolor}\sffamily\fontsize{10.000000}{12.000000}\selectfont 6}%
\end{pgfscope}%
\begin{pgfscope}%
\definecolor{textcolor}{rgb}{1.000000,1.000000,1.000000}%
\pgfsetstrokecolor{textcolor}%
\pgfsetfillcolor{textcolor}%
\pgftext[x=8.256800in,y=3.308905in,,]{\color{textcolor}\sffamily\fontsize{10.000000}{12.000000}\selectfont 7}%
\end{pgfscope}%
\begin{pgfscope}%
\definecolor{textcolor}{rgb}{1.000000,1.000000,1.000000}%
\pgfsetstrokecolor{textcolor}%
\pgfsetfillcolor{textcolor}%
\pgftext[x=8.141393in,y=2.897141in,,]{\color{textcolor}\sffamily\fontsize{10.000000}{12.000000}\selectfont 14}%
\end{pgfscope}%
\begin{pgfscope}%
\definecolor{textcolor}{rgb}{1.000000,1.000000,1.000000}%
\pgfsetstrokecolor{textcolor}%
\pgfsetfillcolor{textcolor}%
\pgftext[x=8.157879in,y=2.485376in,,]{\color{textcolor}\sffamily\fontsize{10.000000}{12.000000}\selectfont 13}%
\end{pgfscope}%
\begin{pgfscope}%
\definecolor{textcolor}{rgb}{1.000000,1.000000,1.000000}%
\pgfsetstrokecolor{textcolor}%
\pgfsetfillcolor{textcolor}%
\pgftext[x=7.828146in,y=2.073611in,,]{\color{textcolor}\sffamily\fontsize{10.000000}{12.000000}\selectfont 33}%
\end{pgfscope}%
\begin{pgfscope}%
\definecolor{textcolor}{rgb}{1.000000,1.000000,1.000000}%
\pgfsetstrokecolor{textcolor}%
\pgfsetfillcolor{textcolor}%
\pgftext[x=8.289773in,y=1.661846in,,]{\color{textcolor}\sffamily\fontsize{10.000000}{12.000000}\selectfont 5}%
\end{pgfscope}%
\begin{pgfscope}%
\definecolor{textcolor}{rgb}{1.000000,1.000000,1.000000}%
\pgfsetstrokecolor{textcolor}%
\pgfsetfillcolor{textcolor}%
\pgftext[x=7.795172in,y=1.250082in,,]{\color{textcolor}\sffamily\fontsize{10.000000}{12.000000}\selectfont 35}%
\end{pgfscope}%
\begin{pgfscope}%
\definecolor{textcolor}{rgb}{1.000000,1.000000,1.000000}%
\pgfsetstrokecolor{textcolor}%
\pgfsetfillcolor{textcolor}%
\pgftext[x=8.322746in,y=0.838317in,,]{\color{textcolor}\sffamily\fontsize{10.000000}{12.000000}\selectfont 3}%
\end{pgfscope}%
\begin{pgfscope}%
\definecolor{textcolor}{rgb}{1.000000,1.000000,1.000000}%
\pgfsetstrokecolor{textcolor}%
\pgfsetfillcolor{textcolor}%
\pgftext[x=8.190853in,y=0.426552in,,]{\color{textcolor}\sffamily\fontsize{10.000000}{12.000000}\selectfont 11}%
\end{pgfscope}%
\begin{pgfscope}%
\pgfsetbuttcap%
\pgfsetmiterjoin%
\definecolor{currentfill}{rgb}{1.000000,1.000000,1.000000}%
\pgfsetfillcolor{currentfill}%
\pgfsetfillopacity{0.800000}%
\pgfsetlinewidth{1.003750pt}%
\definecolor{currentstroke}{rgb}{0.800000,0.800000,0.800000}%
\pgfsetstrokecolor{currentstroke}%
\pgfsetstrokeopacity{0.800000}%
\pgfsetdash{}{0pt}%
\pgfpathmoveto{\pgfqpoint{1.330943in}{4.056458in}}%
\pgfpathlineto{\pgfqpoint{7.508857in}{4.056458in}}%
\pgfpathquadraticcurveto{\pgfqpoint{7.531995in}{4.056458in}}{\pgfqpoint{7.531995in}{4.079597in}}%
\pgfpathlineto{\pgfqpoint{7.531995in}{4.237841in}}%
\pgfpathquadraticcurveto{\pgfqpoint{7.531995in}{4.260980in}}{\pgfqpoint{7.508857in}{4.260980in}}%
\pgfpathlineto{\pgfqpoint{1.330943in}{4.260980in}}%
\pgfpathquadraticcurveto{\pgfqpoint{1.307804in}{4.260980in}}{\pgfqpoint{1.307804in}{4.237841in}}%
\pgfpathlineto{\pgfqpoint{1.307804in}{4.079597in}}%
\pgfpathquadraticcurveto{\pgfqpoint{1.307804in}{4.056458in}}{\pgfqpoint{1.330943in}{4.056458in}}%
\pgfpathclose%
\pgfusepath{stroke,fill}%
\end{pgfscope}%
\begin{pgfscope}%
\pgfsetbuttcap%
\pgfsetmiterjoin%
\definecolor{currentfill}{rgb}{0.898885,0.305498,0.206767}%
\pgfsetfillcolor{currentfill}%
\pgfsetfillopacity{0.500000}%
\pgfsetlinewidth{0.000000pt}%
\definecolor{currentstroke}{rgb}{0.000000,0.000000,0.000000}%
\pgfsetstrokecolor{currentstroke}%
\pgfsetstrokeopacity{0.500000}%
\pgfsetdash{}{0pt}%
\pgfpathmoveto{\pgfqpoint{1.354081in}{4.126801in}}%
\pgfpathlineto{\pgfqpoint{1.585470in}{4.126801in}}%
\pgfpathlineto{\pgfqpoint{1.585470in}{4.207787in}}%
\pgfpathlineto{\pgfqpoint{1.354081in}{4.207787in}}%
\pgfpathclose%
\pgfusepath{fill}%
\end{pgfscope}%
\begin{pgfscope}%
\definecolor{textcolor}{rgb}{0.000000,0.000000,0.000000}%
\pgfsetstrokecolor{textcolor}%
\pgfsetfillcolor{textcolor}%
\pgftext[x=1.678026in,y=4.126801in,left,base]{\color{textcolor}\sffamily\fontsize{8.330000}{9.996000}\selectfont 1}%
\end{pgfscope}%
\begin{pgfscope}%
\pgfsetbuttcap%
\pgfsetmiterjoin%
\definecolor{currentfill}{rgb}{0.966551,0.497424,0.295040}%
\pgfsetfillcolor{currentfill}%
\pgfsetfillopacity{0.500000}%
\pgfsetlinewidth{0.000000pt}%
\definecolor{currentstroke}{rgb}{0.000000,0.000000,0.000000}%
\pgfsetstrokecolor{currentstroke}%
\pgfsetstrokeopacity{0.500000}%
\pgfsetdash{}{0pt}%
\pgfpathmoveto{\pgfqpoint{1.983023in}{4.126801in}}%
\pgfpathlineto{\pgfqpoint{2.214412in}{4.126801in}}%
\pgfpathlineto{\pgfqpoint{2.214412in}{4.207787in}}%
\pgfpathlineto{\pgfqpoint{1.983023in}{4.207787in}}%
\pgfpathclose%
\pgfusepath{fill}%
\end{pgfscope}%
\begin{pgfscope}%
\definecolor{textcolor}{rgb}{0.000000,0.000000,0.000000}%
\pgfsetstrokecolor{textcolor}%
\pgfsetfillcolor{textcolor}%
\pgftext[x=2.306968in,y=4.126801in,left,base]{\color{textcolor}\sffamily\fontsize{8.330000}{9.996000}\selectfont 2}%
\end{pgfscope}%
\begin{pgfscope}%
\pgfsetbuttcap%
\pgfsetmiterjoin%
\definecolor{currentfill}{rgb}{0.992388,0.693887,0.390081}%
\pgfsetfillcolor{currentfill}%
\pgfsetfillopacity{0.500000}%
\pgfsetlinewidth{0.000000pt}%
\definecolor{currentstroke}{rgb}{0.000000,0.000000,0.000000}%
\pgfsetstrokecolor{currentstroke}%
\pgfsetstrokeopacity{0.500000}%
\pgfsetdash{}{0pt}%
\pgfpathmoveto{\pgfqpoint{2.611965in}{4.126801in}}%
\pgfpathlineto{\pgfqpoint{2.843354in}{4.126801in}}%
\pgfpathlineto{\pgfqpoint{2.843354in}{4.207787in}}%
\pgfpathlineto{\pgfqpoint{2.611965in}{4.207787in}}%
\pgfpathclose%
\pgfusepath{fill}%
\end{pgfscope}%
\begin{pgfscope}%
\definecolor{textcolor}{rgb}{0.000000,0.000000,0.000000}%
\pgfsetstrokecolor{textcolor}%
\pgfsetfillcolor{textcolor}%
\pgftext[x=2.935909in,y=4.126801in,left,base]{\color{textcolor}\sffamily\fontsize{8.330000}{9.996000}\selectfont 3}%
\end{pgfscope}%
\begin{pgfscope}%
\pgfsetbuttcap%
\pgfsetmiterjoin%
\definecolor{currentfill}{rgb}{0.995463,0.847674,0.519262}%
\pgfsetfillcolor{currentfill}%
\pgfsetfillopacity{0.500000}%
\pgfsetlinewidth{0.000000pt}%
\definecolor{currentstroke}{rgb}{0.000000,0.000000,0.000000}%
\pgfsetstrokecolor{currentstroke}%
\pgfsetstrokeopacity{0.500000}%
\pgfsetdash{}{0pt}%
\pgfpathmoveto{\pgfqpoint{3.240906in}{4.126801in}}%
\pgfpathlineto{\pgfqpoint{3.472295in}{4.126801in}}%
\pgfpathlineto{\pgfqpoint{3.472295in}{4.207787in}}%
\pgfpathlineto{\pgfqpoint{3.240906in}{4.207787in}}%
\pgfpathclose%
\pgfusepath{fill}%
\end{pgfscope}%
\begin{pgfscope}%
\definecolor{textcolor}{rgb}{0.000000,0.000000,0.000000}%
\pgfsetstrokecolor{textcolor}%
\pgfsetfillcolor{textcolor}%
\pgftext[x=3.564851in,y=4.126801in,left,base]{\color{textcolor}\sffamily\fontsize{8.330000}{9.996000}\selectfont 4}%
\end{pgfscope}%
\begin{pgfscope}%
\pgfsetbuttcap%
\pgfsetmiterjoin%
\definecolor{currentfill}{rgb}{0.998539,0.954710,0.673049}%
\pgfsetfillcolor{currentfill}%
\pgfsetfillopacity{0.500000}%
\pgfsetlinewidth{0.000000pt}%
\definecolor{currentstroke}{rgb}{0.000000,0.000000,0.000000}%
\pgfsetstrokecolor{currentstroke}%
\pgfsetstrokeopacity{0.500000}%
\pgfsetdash{}{0pt}%
\pgfpathmoveto{\pgfqpoint{3.869848in}{4.126801in}}%
\pgfpathlineto{\pgfqpoint{4.101237in}{4.126801in}}%
\pgfpathlineto{\pgfqpoint{4.101237in}{4.207787in}}%
\pgfpathlineto{\pgfqpoint{3.869848in}{4.207787in}}%
\pgfpathclose%
\pgfusepath{fill}%
\end{pgfscope}%
\begin{pgfscope}%
\definecolor{textcolor}{rgb}{0.000000,0.000000,0.000000}%
\pgfsetstrokecolor{textcolor}%
\pgfsetfillcolor{textcolor}%
\pgftext[x=4.193792in,y=4.126801in,left,base]{\color{textcolor}\sffamily\fontsize{8.330000}{9.996000}\selectfont 5}%
\end{pgfscope}%
\begin{pgfscope}%
\pgfsetbuttcap%
\pgfsetmiterjoin%
\definecolor{currentfill}{rgb}{0.944483,0.976624,0.673049}%
\pgfsetfillcolor{currentfill}%
\pgfsetfillopacity{0.500000}%
\pgfsetlinewidth{0.000000pt}%
\definecolor{currentstroke}{rgb}{0.000000,0.000000,0.000000}%
\pgfsetstrokecolor{currentstroke}%
\pgfsetstrokeopacity{0.500000}%
\pgfsetdash{}{0pt}%
\pgfpathmoveto{\pgfqpoint{4.498790in}{4.126801in}}%
\pgfpathlineto{\pgfqpoint{4.730179in}{4.126801in}}%
\pgfpathlineto{\pgfqpoint{4.730179in}{4.207787in}}%
\pgfpathlineto{\pgfqpoint{4.498790in}{4.207787in}}%
\pgfpathclose%
\pgfusepath{fill}%
\end{pgfscope}%
\begin{pgfscope}%
\definecolor{textcolor}{rgb}{0.000000,0.000000,0.000000}%
\pgfsetstrokecolor{textcolor}%
\pgfsetfillcolor{textcolor}%
\pgftext[x=4.822734in,y=4.126801in,left,base]{\color{textcolor}\sffamily\fontsize{8.330000}{9.996000}\selectfont 6}%
\end{pgfscope}%
\begin{pgfscope}%
\pgfsetbuttcap%
\pgfsetmiterjoin%
\definecolor{currentfill}{rgb}{0.819608,0.923722,0.524798}%
\pgfsetfillcolor{currentfill}%
\pgfsetfillopacity{0.500000}%
\pgfsetlinewidth{0.000000pt}%
\definecolor{currentstroke}{rgb}{0.000000,0.000000,0.000000}%
\pgfsetstrokecolor{currentstroke}%
\pgfsetstrokeopacity{0.500000}%
\pgfsetdash{}{0pt}%
\pgfpathmoveto{\pgfqpoint{5.127731in}{4.126801in}}%
\pgfpathlineto{\pgfqpoint{5.359120in}{4.126801in}}%
\pgfpathlineto{\pgfqpoint{5.359120in}{4.207787in}}%
\pgfpathlineto{\pgfqpoint{5.127731in}{4.207787in}}%
\pgfpathclose%
\pgfusepath{fill}%
\end{pgfscope}%
\begin{pgfscope}%
\definecolor{textcolor}{rgb}{0.000000,0.000000,0.000000}%
\pgfsetstrokecolor{textcolor}%
\pgfsetfillcolor{textcolor}%
\pgftext[x=5.451676in,y=4.126801in,left,base]{\color{textcolor}\sffamily\fontsize{8.330000}{9.996000}\selectfont 7}%
\end{pgfscope}%
\begin{pgfscope}%
\pgfsetbuttcap%
\pgfsetmiterjoin%
\definecolor{currentfill}{rgb}{0.662745,0.856055,0.423299}%
\pgfsetfillcolor{currentfill}%
\pgfsetfillopacity{0.500000}%
\pgfsetlinewidth{0.000000pt}%
\definecolor{currentstroke}{rgb}{0.000000,0.000000,0.000000}%
\pgfsetstrokecolor{currentstroke}%
\pgfsetstrokeopacity{0.500000}%
\pgfsetdash{}{0pt}%
\pgfpathmoveto{\pgfqpoint{5.756673in}{4.126801in}}%
\pgfpathlineto{\pgfqpoint{5.988062in}{4.126801in}}%
\pgfpathlineto{\pgfqpoint{5.988062in}{4.207787in}}%
\pgfpathlineto{\pgfqpoint{5.756673in}{4.207787in}}%
\pgfpathclose%
\pgfusepath{fill}%
\end{pgfscope}%
\begin{pgfscope}%
\definecolor{textcolor}{rgb}{0.000000,0.000000,0.000000}%
\pgfsetstrokecolor{textcolor}%
\pgfsetfillcolor{textcolor}%
\pgftext[x=6.080617in,y=4.126801in,left,base]{\color{textcolor}\sffamily\fontsize{8.330000}{9.996000}\selectfont 8}%
\end{pgfscope}%
\begin{pgfscope}%
\pgfsetbuttcap%
\pgfsetmiterjoin%
\definecolor{currentfill}{rgb}{0.468897,0.771319,0.395771}%
\pgfsetfillcolor{currentfill}%
\pgfsetfillopacity{0.500000}%
\pgfsetlinewidth{0.000000pt}%
\definecolor{currentstroke}{rgb}{0.000000,0.000000,0.000000}%
\pgfsetstrokecolor{currentstroke}%
\pgfsetstrokeopacity{0.500000}%
\pgfsetdash{}{0pt}%
\pgfpathmoveto{\pgfqpoint{6.385615in}{4.126801in}}%
\pgfpathlineto{\pgfqpoint{6.617004in}{4.126801in}}%
\pgfpathlineto{\pgfqpoint{6.617004in}{4.207787in}}%
\pgfpathlineto{\pgfqpoint{6.385615in}{4.207787in}}%
\pgfpathclose%
\pgfusepath{fill}%
\end{pgfscope}%
\begin{pgfscope}%
\definecolor{textcolor}{rgb}{0.000000,0.000000,0.000000}%
\pgfsetstrokecolor{textcolor}%
\pgfsetfillcolor{textcolor}%
\pgftext[x=6.709559in,y=4.126801in,left,base]{\color{textcolor}\sffamily\fontsize{8.330000}{9.996000}\selectfont 9}%
\end{pgfscope}%
\begin{pgfscope}%
\pgfsetbuttcap%
\pgfsetmiterjoin%
\definecolor{currentfill}{rgb}{0.248058,0.667205,0.350250}%
\pgfsetfillcolor{currentfill}%
\pgfsetfillopacity{0.500000}%
\pgfsetlinewidth{0.000000pt}%
\definecolor{currentstroke}{rgb}{0.000000,0.000000,0.000000}%
\pgfsetstrokecolor{currentstroke}%
\pgfsetstrokeopacity{0.500000}%
\pgfsetdash{}{0pt}%
\pgfpathmoveto{\pgfqpoint{7.014556in}{4.126801in}}%
\pgfpathlineto{\pgfqpoint{7.245945in}{4.126801in}}%
\pgfpathlineto{\pgfqpoint{7.245945in}{4.207787in}}%
\pgfpathlineto{\pgfqpoint{7.014556in}{4.207787in}}%
\pgfpathclose%
\pgfusepath{fill}%
\end{pgfscope}%
\begin{pgfscope}%
\definecolor{textcolor}{rgb}{0.000000,0.000000,0.000000}%
\pgfsetstrokecolor{textcolor}%
\pgfsetfillcolor{textcolor}%
\pgftext[x=7.338501in,y=4.126801in,left,base]{\color{textcolor}\sffamily\fontsize{8.330000}{9.996000}\selectfont 10}%
\end{pgfscope}%
\end{pgfpicture}%
\makeatother%
\endgroup%
}
    \caption{The figure represents distribution for Section 2 of survey. The participants were asked to rate the image based on photorealism(1 being the least photorealistic).}
    \label{fig:question2}
\end{figure}

\begin{figure}
    \centering
    \resizebox{\textwidth}{!}{%% Creator: Matplotlib, PGF backend
%%
%% To include the figure in your LaTeX document, write
%%   \input{<filename>.pgf}
%%
%% Make sure the required packages are loaded in your preamble
%%   \usepackage{pgf}
%%
%% Figures using additional raster images can only be included by \input if
%% they are in the same directory as the main LaTeX file. For loading figures
%% from other directories you can use the `import` package
%%   \usepackage{import}
%%
%% and then include the figures with
%%   \import{<path to file>}{<filename>.pgf}
%%
%% Matplotlib used the following preamble
%%   \usepackage{fontspec}
%%   \setmainfont{DejaVuSerif.ttf}[Path=\detokenize{/Users/apple/opt/anaconda3/envs/kaolin/lib/python3.7/site-packages/matplotlib/mpl-data/fonts/ttf/}]
%%   \setsansfont{DejaVuSans.ttf}[Path=\detokenize{/Users/apple/opt/anaconda3/envs/kaolin/lib/python3.7/site-packages/matplotlib/mpl-data/fonts/ttf/}]
%%   \setmonofont{DejaVuSansMono.ttf}[Path=\detokenize{/Users/apple/opt/anaconda3/envs/kaolin/lib/python3.7/site-packages/matplotlib/mpl-data/fonts/ttf/}]
%%
\begingroup%
\makeatletter%
\begin{pgfpicture}%
\pgfpathrectangle{\pgfpointorigin}{\pgfqpoint{5.535556in}{4.888302in}}%
\pgfusepath{use as bounding box, clip}%
\begin{pgfscope}%
\pgfsetbuttcap%
\pgfsetmiterjoin%
\definecolor{currentfill}{rgb}{1.000000,1.000000,1.000000}%
\pgfsetfillcolor{currentfill}%
\pgfsetlinewidth{0.000000pt}%
\definecolor{currentstroke}{rgb}{1.000000,1.000000,1.000000}%
\pgfsetstrokecolor{currentstroke}%
\pgfsetdash{}{0pt}%
\pgfpathmoveto{\pgfqpoint{0.000000in}{0.000000in}}%
\pgfpathlineto{\pgfqpoint{5.535556in}{0.000000in}}%
\pgfpathlineto{\pgfqpoint{5.535556in}{4.888302in}}%
\pgfpathlineto{\pgfqpoint{0.000000in}{4.888302in}}%
\pgfpathclose%
\pgfusepath{fill}%
\end{pgfscope}%
\begin{pgfscope}%
\pgfsetbuttcap%
\pgfsetmiterjoin%
\definecolor{currentfill}{rgb}{1.000000,1.000000,1.000000}%
\pgfsetfillcolor{currentfill}%
\pgfsetlinewidth{0.000000pt}%
\definecolor{currentstroke}{rgb}{0.000000,0.000000,0.000000}%
\pgfsetstrokecolor{currentstroke}%
\pgfsetstrokeopacity{0.000000}%
\pgfsetdash{}{0pt}%
\pgfpathmoveto{\pgfqpoint{0.475556in}{1.092302in}}%
\pgfpathlineto{\pgfqpoint{5.435556in}{1.092302in}}%
\pgfpathlineto{\pgfqpoint{5.435556in}{4.788302in}}%
\pgfpathlineto{\pgfqpoint{0.475556in}{4.788302in}}%
\pgfpathclose%
\pgfusepath{fill}%
\end{pgfscope}%
\begin{pgfscope}%
\pgfpathrectangle{\pgfqpoint{0.475556in}{1.092302in}}{\pgfqpoint{4.960000in}{3.696000in}}%
\pgfusepath{clip}%
\pgfsetbuttcap%
\pgfsetmiterjoin%
\definecolor{currentfill}{rgb}{0.121569,0.466667,0.705882}%
\pgfsetfillcolor{currentfill}%
\pgfsetfillopacity{0.500000}%
\pgfsetlinewidth{0.000000pt}%
\definecolor{currentstroke}{rgb}{0.000000,0.000000,0.000000}%
\pgfsetstrokecolor{currentstroke}%
\pgfsetstrokeopacity{0.500000}%
\pgfsetdash{}{0pt}%
\pgfpathmoveto{\pgfqpoint{0.701010in}{1.092302in}}%
\pgfpathlineto{\pgfqpoint{1.110928in}{1.092302in}}%
\pgfpathlineto{\pgfqpoint{1.110928in}{4.612302in}}%
\pgfpathlineto{\pgfqpoint{0.701010in}{4.612302in}}%
\pgfpathclose%
\pgfusepath{fill}%
\end{pgfscope}%
\begin{pgfscope}%
\pgfpathrectangle{\pgfqpoint{0.475556in}{1.092302in}}{\pgfqpoint{4.960000in}{3.696000in}}%
\pgfusepath{clip}%
\pgfsetbuttcap%
\pgfsetmiterjoin%
\definecolor{currentfill}{rgb}{0.121569,0.466667,0.705882}%
\pgfsetfillcolor{currentfill}%
\pgfsetfillopacity{0.500000}%
\pgfsetlinewidth{0.000000pt}%
\definecolor{currentstroke}{rgb}{0.000000,0.000000,0.000000}%
\pgfsetstrokecolor{currentstroke}%
\pgfsetstrokeopacity{0.500000}%
\pgfsetdash{}{0pt}%
\pgfpathmoveto{\pgfqpoint{1.213407in}{1.092302in}}%
\pgfpathlineto{\pgfqpoint{1.623324in}{1.092302in}}%
\pgfpathlineto{\pgfqpoint{1.623324in}{4.551370in}}%
\pgfpathlineto{\pgfqpoint{1.213407in}{4.551370in}}%
\pgfpathclose%
\pgfusepath{fill}%
\end{pgfscope}%
\begin{pgfscope}%
\pgfpathrectangle{\pgfqpoint{0.475556in}{1.092302in}}{\pgfqpoint{4.960000in}{3.696000in}}%
\pgfusepath{clip}%
\pgfsetbuttcap%
\pgfsetmiterjoin%
\definecolor{currentfill}{rgb}{0.121569,0.466667,0.705882}%
\pgfsetfillcolor{currentfill}%
\pgfsetfillopacity{0.500000}%
\pgfsetlinewidth{0.000000pt}%
\definecolor{currentstroke}{rgb}{0.000000,0.000000,0.000000}%
\pgfsetstrokecolor{currentstroke}%
\pgfsetstrokeopacity{0.500000}%
\pgfsetdash{}{0pt}%
\pgfpathmoveto{\pgfqpoint{1.725804in}{1.092302in}}%
\pgfpathlineto{\pgfqpoint{2.135721in}{1.092302in}}%
\pgfpathlineto{\pgfqpoint{2.135721in}{3.752222in}}%
\pgfpathlineto{\pgfqpoint{1.725804in}{3.752222in}}%
\pgfpathclose%
\pgfusepath{fill}%
\end{pgfscope}%
\begin{pgfscope}%
\pgfpathrectangle{\pgfqpoint{0.475556in}{1.092302in}}{\pgfqpoint{4.960000in}{3.696000in}}%
\pgfusepath{clip}%
\pgfsetbuttcap%
\pgfsetmiterjoin%
\definecolor{currentfill}{rgb}{0.121569,0.466667,0.705882}%
\pgfsetfillcolor{currentfill}%
\pgfsetfillopacity{0.500000}%
\pgfsetlinewidth{0.000000pt}%
\definecolor{currentstroke}{rgb}{0.000000,0.000000,0.000000}%
\pgfsetstrokecolor{currentstroke}%
\pgfsetstrokeopacity{0.500000}%
\pgfsetdash{}{0pt}%
\pgfpathmoveto{\pgfqpoint{2.238200in}{1.092302in}}%
\pgfpathlineto{\pgfqpoint{2.648118in}{1.092302in}}%
\pgfpathlineto{\pgfqpoint{2.648118in}{3.442875in}}%
\pgfpathlineto{\pgfqpoint{2.238200in}{3.442875in}}%
\pgfpathclose%
\pgfusepath{fill}%
\end{pgfscope}%
\begin{pgfscope}%
\pgfpathrectangle{\pgfqpoint{0.475556in}{1.092302in}}{\pgfqpoint{4.960000in}{3.696000in}}%
\pgfusepath{clip}%
\pgfsetbuttcap%
\pgfsetmiterjoin%
\definecolor{currentfill}{rgb}{0.121569,0.466667,0.705882}%
\pgfsetfillcolor{currentfill}%
\pgfsetlinewidth{0.000000pt}%
\definecolor{currentstroke}{rgb}{0.000000,0.000000,0.000000}%
\pgfsetstrokecolor{currentstroke}%
\pgfsetstrokeopacity{0.000000}%
\pgfsetdash{}{0pt}%
\pgfpathmoveto{\pgfqpoint{2.750597in}{1.092302in}}%
\pgfpathlineto{\pgfqpoint{3.160515in}{1.092302in}}%
\pgfpathlineto{\pgfqpoint{3.160515in}{3.440531in}}%
\pgfpathlineto{\pgfqpoint{2.750597in}{3.440531in}}%
\pgfpathclose%
\pgfusepath{fill}%
\end{pgfscope}%
\begin{pgfscope}%
\pgfpathrectangle{\pgfqpoint{0.475556in}{1.092302in}}{\pgfqpoint{4.960000in}{3.696000in}}%
\pgfusepath{clip}%
\pgfsetbuttcap%
\pgfsetmiterjoin%
\definecolor{currentfill}{rgb}{0.121569,0.466667,0.705882}%
\pgfsetfillcolor{currentfill}%
\pgfsetlinewidth{0.000000pt}%
\definecolor{currentstroke}{rgb}{0.000000,0.000000,0.000000}%
\pgfsetstrokecolor{currentstroke}%
\pgfsetstrokeopacity{0.000000}%
\pgfsetdash{}{0pt}%
\pgfpathmoveto{\pgfqpoint{3.262994in}{1.092302in}}%
\pgfpathlineto{\pgfqpoint{3.672911in}{1.092302in}}%
\pgfpathlineto{\pgfqpoint{3.672911in}{3.262422in}}%
\pgfpathlineto{\pgfqpoint{3.262994in}{3.262422in}}%
\pgfpathclose%
\pgfusepath{fill}%
\end{pgfscope}%
\begin{pgfscope}%
\pgfpathrectangle{\pgfqpoint{0.475556in}{1.092302in}}{\pgfqpoint{4.960000in}{3.696000in}}%
\pgfusepath{clip}%
\pgfsetbuttcap%
\pgfsetmiterjoin%
\definecolor{currentfill}{rgb}{0.121569,0.466667,0.705882}%
\pgfsetfillcolor{currentfill}%
\pgfsetfillopacity{0.500000}%
\pgfsetlinewidth{0.000000pt}%
\definecolor{currentstroke}{rgb}{0.000000,0.000000,0.000000}%
\pgfsetstrokecolor{currentstroke}%
\pgfsetstrokeopacity{0.500000}%
\pgfsetdash{}{0pt}%
\pgfpathmoveto{\pgfqpoint{3.775391in}{1.092302in}}%
\pgfpathlineto{\pgfqpoint{4.185308in}{1.092302in}}%
\pgfpathlineto{\pgfqpoint{4.185308in}{2.908547in}}%
\pgfpathlineto{\pgfqpoint{3.775391in}{2.908547in}}%
\pgfpathclose%
\pgfusepath{fill}%
\end{pgfscope}%
\begin{pgfscope}%
\pgfpathrectangle{\pgfqpoint{0.475556in}{1.092302in}}{\pgfqpoint{4.960000in}{3.696000in}}%
\pgfusepath{clip}%
\pgfsetbuttcap%
\pgfsetmiterjoin%
\definecolor{currentfill}{rgb}{0.121569,0.466667,0.705882}%
\pgfsetfillcolor{currentfill}%
\pgfsetlinewidth{0.000000pt}%
\definecolor{currentstroke}{rgb}{0.000000,0.000000,0.000000}%
\pgfsetstrokecolor{currentstroke}%
\pgfsetstrokeopacity{0.000000}%
\pgfsetdash{}{0pt}%
\pgfpathmoveto{\pgfqpoint{4.287787in}{1.092302in}}%
\pgfpathlineto{\pgfqpoint{4.697705in}{1.092302in}}%
\pgfpathlineto{\pgfqpoint{4.697705in}{2.789027in}}%
\pgfpathlineto{\pgfqpoint{4.287787in}{2.789027in}}%
\pgfpathclose%
\pgfusepath{fill}%
\end{pgfscope}%
\begin{pgfscope}%
\pgfpathrectangle{\pgfqpoint{0.475556in}{1.092302in}}{\pgfqpoint{4.960000in}{3.696000in}}%
\pgfusepath{clip}%
\pgfsetbuttcap%
\pgfsetmiterjoin%
\definecolor{currentfill}{rgb}{0.121569,0.466667,0.705882}%
\pgfsetfillcolor{currentfill}%
\pgfsetlinewidth{0.000000pt}%
\definecolor{currentstroke}{rgb}{0.000000,0.000000,0.000000}%
\pgfsetstrokecolor{currentstroke}%
\pgfsetstrokeopacity{0.000000}%
\pgfsetdash{}{0pt}%
\pgfpathmoveto{\pgfqpoint{4.800184in}{1.092302in}}%
\pgfpathlineto{\pgfqpoint{5.210101in}{1.092302in}}%
\pgfpathlineto{\pgfqpoint{5.210101in}{2.676537in}}%
\pgfpathlineto{\pgfqpoint{4.800184in}{2.676537in}}%
\pgfpathclose%
\pgfusepath{fill}%
\end{pgfscope}%
\begin{pgfscope}%
\pgfsetbuttcap%
\pgfsetroundjoin%
\definecolor{currentfill}{rgb}{0.000000,0.000000,0.000000}%
\pgfsetfillcolor{currentfill}%
\pgfsetlinewidth{0.803000pt}%
\definecolor{currentstroke}{rgb}{0.000000,0.000000,0.000000}%
\pgfsetstrokecolor{currentstroke}%
\pgfsetdash{}{0pt}%
\pgfsys@defobject{currentmarker}{\pgfqpoint{0.000000in}{-0.048611in}}{\pgfqpoint{0.000000in}{0.000000in}}{%
\pgfpathmoveto{\pgfqpoint{0.000000in}{0.000000in}}%
\pgfpathlineto{\pgfqpoint{0.000000in}{-0.048611in}}%
\pgfusepath{stroke,fill}%
}%
\begin{pgfscope}%
\pgfsys@transformshift{0.905969in}{1.092302in}%
\pgfsys@useobject{currentmarker}{}%
\end{pgfscope}%
\end{pgfscope}%
\begin{pgfscope}%
\definecolor{textcolor}{rgb}{0.000000,0.000000,0.000000}%
\pgfsetstrokecolor{textcolor}%
\pgfsetfillcolor{textcolor}%
\pgftext[x=0.792774in, y=0.639887in, left, base,rotate=45.000000]{\color{textcolor}\sffamily\fontsize{10.000000}{12.000000}\selectfont Pix3D}%
\end{pgfscope}%
\begin{pgfscope}%
\pgfsetbuttcap%
\pgfsetroundjoin%
\definecolor{currentfill}{rgb}{0.000000,0.000000,0.000000}%
\pgfsetfillcolor{currentfill}%
\pgfsetlinewidth{0.803000pt}%
\definecolor{currentstroke}{rgb}{0.000000,0.000000,0.000000}%
\pgfsetstrokecolor{currentstroke}%
\pgfsetdash{}{0pt}%
\pgfsys@defobject{currentmarker}{\pgfqpoint{0.000000in}{-0.048611in}}{\pgfqpoint{0.000000in}{0.000000in}}{%
\pgfpathmoveto{\pgfqpoint{0.000000in}{0.000000in}}%
\pgfpathlineto{\pgfqpoint{0.000000in}{-0.048611in}}%
\pgfusepath{stroke,fill}%
}%
\begin{pgfscope}%
\pgfsys@transformshift{1.418366in}{1.092302in}%
\pgfsys@useobject{currentmarker}{}%
\end{pgfscope}%
\end{pgfscope}%
\begin{pgfscope}%
\definecolor{textcolor}{rgb}{0.000000,0.000000,0.000000}%
\pgfsetstrokecolor{textcolor}%
\pgfsetfillcolor{textcolor}%
\pgftext[x=1.180132in, y=0.389808in, left, base,rotate=45.000000]{\color{textcolor}\sffamily\fontsize{10.000000}{12.000000}\selectfont InteriorNet}%
\end{pgfscope}%
\begin{pgfscope}%
\pgfsetbuttcap%
\pgfsetroundjoin%
\definecolor{currentfill}{rgb}{0.000000,0.000000,0.000000}%
\pgfsetfillcolor{currentfill}%
\pgfsetlinewidth{0.803000pt}%
\definecolor{currentstroke}{rgb}{0.000000,0.000000,0.000000}%
\pgfsetstrokecolor{currentstroke}%
\pgfsetdash{}{0pt}%
\pgfsys@defobject{currentmarker}{\pgfqpoint{0.000000in}{-0.048611in}}{\pgfqpoint{0.000000in}{0.000000in}}{%
\pgfpathmoveto{\pgfqpoint{0.000000in}{0.000000in}}%
\pgfpathlineto{\pgfqpoint{0.000000in}{-0.048611in}}%
\pgfusepath{stroke,fill}%
}%
\begin{pgfscope}%
\pgfsys@transformshift{1.930762in}{1.092302in}%
\pgfsys@useobject{currentmarker}{}%
\end{pgfscope}%
\end{pgfscope}%
\begin{pgfscope}%
\definecolor{textcolor}{rgb}{0.000000,0.000000,0.000000}%
\pgfsetstrokecolor{textcolor}%
\pgfsetfillcolor{textcolor}%
\pgftext[x=1.723243in, y=0.451237in, left, base,rotate=45.000000]{\color{textcolor}\sffamily\fontsize{10.000000}{12.000000}\selectfont Hyperism}%
\end{pgfscope}%
\begin{pgfscope}%
\pgfsetbuttcap%
\pgfsetroundjoin%
\definecolor{currentfill}{rgb}{0.000000,0.000000,0.000000}%
\pgfsetfillcolor{currentfill}%
\pgfsetlinewidth{0.803000pt}%
\definecolor{currentstroke}{rgb}{0.000000,0.000000,0.000000}%
\pgfsetstrokecolor{currentstroke}%
\pgfsetdash{}{0pt}%
\pgfsys@defobject{currentmarker}{\pgfqpoint{0.000000in}{-0.048611in}}{\pgfqpoint{0.000000in}{0.000000in}}{%
\pgfpathmoveto{\pgfqpoint{0.000000in}{0.000000in}}%
\pgfpathlineto{\pgfqpoint{0.000000in}{-0.048611in}}%
\pgfusepath{stroke,fill}%
}%
\begin{pgfscope}%
\pgfsys@transformshift{2.443159in}{1.092302in}%
\pgfsys@useobject{currentmarker}{}%
\end{pgfscope}%
\end{pgfscope}%
\begin{pgfscope}%
\definecolor{textcolor}{rgb}{0.000000,0.000000,0.000000}%
\pgfsetstrokecolor{textcolor}%
\pgfsetfillcolor{textcolor}%
\pgftext[x=2.233457in, y=0.446873in, left, base,rotate=45.000000]{\color{textcolor}\sffamily\fontsize{10.000000}{12.000000}\selectfont 3DFRONT}%
\end{pgfscope}%
\begin{pgfscope}%
\pgfsetbuttcap%
\pgfsetroundjoin%
\definecolor{currentfill}{rgb}{0.000000,0.000000,0.000000}%
\pgfsetfillcolor{currentfill}%
\pgfsetlinewidth{0.803000pt}%
\definecolor{currentstroke}{rgb}{0.000000,0.000000,0.000000}%
\pgfsetstrokecolor{currentstroke}%
\pgfsetdash{}{0pt}%
\pgfsys@defobject{currentmarker}{\pgfqpoint{0.000000in}{-0.048611in}}{\pgfqpoint{0.000000in}{0.000000in}}{%
\pgfpathmoveto{\pgfqpoint{0.000000in}{0.000000in}}%
\pgfpathlineto{\pgfqpoint{0.000000in}{-0.048611in}}%
\pgfusepath{stroke,fill}%
}%
\begin{pgfscope}%
\pgfsys@transformshift{2.955556in}{1.092302in}%
\pgfsys@useobject{currentmarker}{}%
\end{pgfscope}%
\end{pgfscope}%
\begin{pgfscope}%
\definecolor{textcolor}{rgb}{0.000000,0.000000,0.000000}%
\pgfsetstrokecolor{textcolor}%
\pgfsetfillcolor{textcolor}%
\pgftext[x=2.746741in, y=0.448647in, left, base,rotate=45.000000]{\color{textcolor}\sffamily\fontsize{10.000000}{12.000000}\selectfont SceneNet}%
\end{pgfscope}%
\begin{pgfscope}%
\pgfsetbuttcap%
\pgfsetroundjoin%
\definecolor{currentfill}{rgb}{0.000000,0.000000,0.000000}%
\pgfsetfillcolor{currentfill}%
\pgfsetlinewidth{0.803000pt}%
\definecolor{currentstroke}{rgb}{0.000000,0.000000,0.000000}%
\pgfsetstrokecolor{currentstroke}%
\pgfsetdash{}{0pt}%
\pgfsys@defobject{currentmarker}{\pgfqpoint{0.000000in}{-0.048611in}}{\pgfqpoint{0.000000in}{0.000000in}}{%
\pgfpathmoveto{\pgfqpoint{0.000000in}{0.000000in}}%
\pgfpathlineto{\pgfqpoint{0.000000in}{-0.048611in}}%
\pgfusepath{stroke,fill}%
}%
\begin{pgfscope}%
\pgfsys@transformshift{3.467953in}{1.092302in}%
\pgfsys@useobject{currentmarker}{}%
\end{pgfscope}%
\end{pgfscope}%
\begin{pgfscope}%
\definecolor{textcolor}{rgb}{0.000000,0.000000,0.000000}%
\pgfsetstrokecolor{textcolor}%
\pgfsetfillcolor{textcolor}%
\pgftext[x=3.197494in, y=0.325358in, left, base,rotate=45.000000]{\color{textcolor}\sffamily\fontsize{10.000000}{12.000000}\selectfont Blenderproc}%
\end{pgfscope}%
\begin{pgfscope}%
\pgfsetbuttcap%
\pgfsetroundjoin%
\definecolor{currentfill}{rgb}{0.000000,0.000000,0.000000}%
\pgfsetfillcolor{currentfill}%
\pgfsetlinewidth{0.803000pt}%
\definecolor{currentstroke}{rgb}{0.000000,0.000000,0.000000}%
\pgfsetstrokecolor{currentstroke}%
\pgfsetdash{}{0pt}%
\pgfsys@defobject{currentmarker}{\pgfqpoint{0.000000in}{-0.048611in}}{\pgfqpoint{0.000000in}{0.000000in}}{%
\pgfpathmoveto{\pgfqpoint{0.000000in}{0.000000in}}%
\pgfpathlineto{\pgfqpoint{0.000000in}{-0.048611in}}%
\pgfusepath{stroke,fill}%
}%
\begin{pgfscope}%
\pgfsys@transformshift{3.980349in}{1.092302in}%
\pgfsys@useobject{currentmarker}{}%
\end{pgfscope}%
\end{pgfscope}%
\begin{pgfscope}%
\definecolor{textcolor}{rgb}{0.000000,0.000000,0.000000}%
\pgfsetstrokecolor{textcolor}%
\pgfsetfillcolor{textcolor}%
\pgftext[x=3.788438in, y=0.482454in, left, base,rotate=45.000000]{\color{textcolor}\sffamily\fontsize{10.000000}{12.000000}\selectfont AI2THOR}%
\end{pgfscope}%
\begin{pgfscope}%
\pgfsetbuttcap%
\pgfsetroundjoin%
\definecolor{currentfill}{rgb}{0.000000,0.000000,0.000000}%
\pgfsetfillcolor{currentfill}%
\pgfsetlinewidth{0.803000pt}%
\definecolor{currentstroke}{rgb}{0.000000,0.000000,0.000000}%
\pgfsetstrokecolor{currentstroke}%
\pgfsetdash{}{0pt}%
\pgfsys@defobject{currentmarker}{\pgfqpoint{0.000000in}{-0.048611in}}{\pgfqpoint{0.000000in}{0.000000in}}{%
\pgfpathmoveto{\pgfqpoint{0.000000in}{0.000000in}}%
\pgfpathlineto{\pgfqpoint{0.000000in}{-0.048611in}}%
\pgfusepath{stroke,fill}%
}%
\begin{pgfscope}%
\pgfsys@transformshift{4.492746in}{1.092302in}%
\pgfsys@useobject{currentmarker}{}%
\end{pgfscope}%
\end{pgfscope}%
\begin{pgfscope}%
\definecolor{textcolor}{rgb}{0.000000,0.000000,0.000000}%
\pgfsetstrokecolor{textcolor}%
\pgfsetfillcolor{textcolor}%
\pgftext[x=4.223270in, y=0.327324in, left, base,rotate=45.000000]{\color{textcolor}\sffamily\fontsize{10.000000}{12.000000}\selectfont OpenRooms}%
\end{pgfscope}%
\begin{pgfscope}%
\pgfsetbuttcap%
\pgfsetroundjoin%
\definecolor{currentfill}{rgb}{0.000000,0.000000,0.000000}%
\pgfsetfillcolor{currentfill}%
\pgfsetlinewidth{0.803000pt}%
\definecolor{currentstroke}{rgb}{0.000000,0.000000,0.000000}%
\pgfsetstrokecolor{currentstroke}%
\pgfsetdash{}{0pt}%
\pgfsys@defobject{currentmarker}{\pgfqpoint{0.000000in}{-0.048611in}}{\pgfqpoint{0.000000in}{0.000000in}}{%
\pgfpathmoveto{\pgfqpoint{0.000000in}{0.000000in}}%
\pgfpathlineto{\pgfqpoint{0.000000in}{-0.048611in}}%
\pgfusepath{stroke,fill}%
}%
\begin{pgfscope}%
\pgfsys@transformshift{5.005143in}{1.092302in}%
\pgfsys@useobject{currentmarker}{}%
\end{pgfscope}%
\end{pgfscope}%
\begin{pgfscope}%
\definecolor{textcolor}{rgb}{0.000000,0.000000,0.000000}%
\pgfsetstrokecolor{textcolor}%
\pgfsetfillcolor{textcolor}%
\pgftext[x=4.727203in, y=0.310396in, left, base,rotate=45.000000]{\color{textcolor}\sffamily\fontsize{10.000000}{12.000000}\selectfont S2R:3DFREE}%
\end{pgfscope}%
\begin{pgfscope}%
\definecolor{textcolor}{rgb}{0.000000,0.000000,0.000000}%
\pgfsetstrokecolor{textcolor}%
\pgfsetfillcolor{textcolor}%
\pgftext[x=2.955556in,y=0.234413in,,top]{\color{textcolor}\sffamily\fontsize{10.000000}{12.000000}\selectfont Datasets}%
\end{pgfscope}%
\begin{pgfscope}%
\pgfsetbuttcap%
\pgfsetroundjoin%
\definecolor{currentfill}{rgb}{0.000000,0.000000,0.000000}%
\pgfsetfillcolor{currentfill}%
\pgfsetlinewidth{0.803000pt}%
\definecolor{currentstroke}{rgb}{0.000000,0.000000,0.000000}%
\pgfsetstrokecolor{currentstroke}%
\pgfsetdash{}{0pt}%
\pgfsys@defobject{currentmarker}{\pgfqpoint{-0.048611in}{0.000000in}}{\pgfqpoint{-0.000000in}{0.000000in}}{%
\pgfpathmoveto{\pgfqpoint{-0.000000in}{0.000000in}}%
\pgfpathlineto{\pgfqpoint{-0.048611in}{0.000000in}}%
\pgfusepath{stroke,fill}%
}%
\begin{pgfscope}%
\pgfsys@transformshift{0.475556in}{1.092302in}%
\pgfsys@useobject{currentmarker}{}%
\end{pgfscope}%
\end{pgfscope}%
\begin{pgfscope}%
\definecolor{textcolor}{rgb}{0.000000,0.000000,0.000000}%
\pgfsetstrokecolor{textcolor}%
\pgfsetfillcolor{textcolor}%
\pgftext[x=0.289968in, y=1.039541in, left, base]{\color{textcolor}\sffamily\fontsize{10.000000}{12.000000}\selectfont 0}%
\end{pgfscope}%
\begin{pgfscope}%
\pgfsetbuttcap%
\pgfsetroundjoin%
\definecolor{currentfill}{rgb}{0.000000,0.000000,0.000000}%
\pgfsetfillcolor{currentfill}%
\pgfsetlinewidth{0.803000pt}%
\definecolor{currentstroke}{rgb}{0.000000,0.000000,0.000000}%
\pgfsetstrokecolor{currentstroke}%
\pgfsetdash{}{0pt}%
\pgfsys@defobject{currentmarker}{\pgfqpoint{-0.048611in}{0.000000in}}{\pgfqpoint{-0.000000in}{0.000000in}}{%
\pgfpathmoveto{\pgfqpoint{-0.000000in}{0.000000in}}%
\pgfpathlineto{\pgfqpoint{-0.048611in}{0.000000in}}%
\pgfusepath{stroke,fill}%
}%
\begin{pgfscope}%
\pgfsys@transformshift{0.475556in}{1.598507in}%
\pgfsys@useobject{currentmarker}{}%
\end{pgfscope}%
\end{pgfscope}%
\begin{pgfscope}%
\definecolor{textcolor}{rgb}{0.000000,0.000000,0.000000}%
\pgfsetstrokecolor{textcolor}%
\pgfsetfillcolor{textcolor}%
\pgftext[x=0.289968in, y=1.545746in, left, base]{\color{textcolor}\sffamily\fontsize{10.000000}{12.000000}\selectfont 1}%
\end{pgfscope}%
\begin{pgfscope}%
\pgfsetbuttcap%
\pgfsetroundjoin%
\definecolor{currentfill}{rgb}{0.000000,0.000000,0.000000}%
\pgfsetfillcolor{currentfill}%
\pgfsetlinewidth{0.803000pt}%
\definecolor{currentstroke}{rgb}{0.000000,0.000000,0.000000}%
\pgfsetstrokecolor{currentstroke}%
\pgfsetdash{}{0pt}%
\pgfsys@defobject{currentmarker}{\pgfqpoint{-0.048611in}{0.000000in}}{\pgfqpoint{-0.000000in}{0.000000in}}{%
\pgfpathmoveto{\pgfqpoint{-0.000000in}{0.000000in}}%
\pgfpathlineto{\pgfqpoint{-0.048611in}{0.000000in}}%
\pgfusepath{stroke,fill}%
}%
\begin{pgfscope}%
\pgfsys@transformshift{0.475556in}{2.104712in}%
\pgfsys@useobject{currentmarker}{}%
\end{pgfscope}%
\end{pgfscope}%
\begin{pgfscope}%
\definecolor{textcolor}{rgb}{0.000000,0.000000,0.000000}%
\pgfsetstrokecolor{textcolor}%
\pgfsetfillcolor{textcolor}%
\pgftext[x=0.289968in, y=2.051951in, left, base]{\color{textcolor}\sffamily\fontsize{10.000000}{12.000000}\selectfont 2}%
\end{pgfscope}%
\begin{pgfscope}%
\pgfsetbuttcap%
\pgfsetroundjoin%
\definecolor{currentfill}{rgb}{0.000000,0.000000,0.000000}%
\pgfsetfillcolor{currentfill}%
\pgfsetlinewidth{0.803000pt}%
\definecolor{currentstroke}{rgb}{0.000000,0.000000,0.000000}%
\pgfsetstrokecolor{currentstroke}%
\pgfsetdash{}{0pt}%
\pgfsys@defobject{currentmarker}{\pgfqpoint{-0.048611in}{0.000000in}}{\pgfqpoint{-0.000000in}{0.000000in}}{%
\pgfpathmoveto{\pgfqpoint{-0.000000in}{0.000000in}}%
\pgfpathlineto{\pgfqpoint{-0.048611in}{0.000000in}}%
\pgfusepath{stroke,fill}%
}%
\begin{pgfscope}%
\pgfsys@transformshift{0.475556in}{2.610917in}%
\pgfsys@useobject{currentmarker}{}%
\end{pgfscope}%
\end{pgfscope}%
\begin{pgfscope}%
\definecolor{textcolor}{rgb}{0.000000,0.000000,0.000000}%
\pgfsetstrokecolor{textcolor}%
\pgfsetfillcolor{textcolor}%
\pgftext[x=0.289968in, y=2.558156in, left, base]{\color{textcolor}\sffamily\fontsize{10.000000}{12.000000}\selectfont 3}%
\end{pgfscope}%
\begin{pgfscope}%
\pgfsetbuttcap%
\pgfsetroundjoin%
\definecolor{currentfill}{rgb}{0.000000,0.000000,0.000000}%
\pgfsetfillcolor{currentfill}%
\pgfsetlinewidth{0.803000pt}%
\definecolor{currentstroke}{rgb}{0.000000,0.000000,0.000000}%
\pgfsetstrokecolor{currentstroke}%
\pgfsetdash{}{0pt}%
\pgfsys@defobject{currentmarker}{\pgfqpoint{-0.048611in}{0.000000in}}{\pgfqpoint{-0.000000in}{0.000000in}}{%
\pgfpathmoveto{\pgfqpoint{-0.000000in}{0.000000in}}%
\pgfpathlineto{\pgfqpoint{-0.048611in}{0.000000in}}%
\pgfusepath{stroke,fill}%
}%
\begin{pgfscope}%
\pgfsys@transformshift{0.475556in}{3.117123in}%
\pgfsys@useobject{currentmarker}{}%
\end{pgfscope}%
\end{pgfscope}%
\begin{pgfscope}%
\definecolor{textcolor}{rgb}{0.000000,0.000000,0.000000}%
\pgfsetstrokecolor{textcolor}%
\pgfsetfillcolor{textcolor}%
\pgftext[x=0.289968in, y=3.064361in, left, base]{\color{textcolor}\sffamily\fontsize{10.000000}{12.000000}\selectfont 4}%
\end{pgfscope}%
\begin{pgfscope}%
\pgfsetbuttcap%
\pgfsetroundjoin%
\definecolor{currentfill}{rgb}{0.000000,0.000000,0.000000}%
\pgfsetfillcolor{currentfill}%
\pgfsetlinewidth{0.803000pt}%
\definecolor{currentstroke}{rgb}{0.000000,0.000000,0.000000}%
\pgfsetstrokecolor{currentstroke}%
\pgfsetdash{}{0pt}%
\pgfsys@defobject{currentmarker}{\pgfqpoint{-0.048611in}{0.000000in}}{\pgfqpoint{-0.000000in}{0.000000in}}{%
\pgfpathmoveto{\pgfqpoint{-0.000000in}{0.000000in}}%
\pgfpathlineto{\pgfqpoint{-0.048611in}{0.000000in}}%
\pgfusepath{stroke,fill}%
}%
\begin{pgfscope}%
\pgfsys@transformshift{0.475556in}{3.623328in}%
\pgfsys@useobject{currentmarker}{}%
\end{pgfscope}%
\end{pgfscope}%
\begin{pgfscope}%
\definecolor{textcolor}{rgb}{0.000000,0.000000,0.000000}%
\pgfsetstrokecolor{textcolor}%
\pgfsetfillcolor{textcolor}%
\pgftext[x=0.289968in, y=3.570566in, left, base]{\color{textcolor}\sffamily\fontsize{10.000000}{12.000000}\selectfont 5}%
\end{pgfscope}%
\begin{pgfscope}%
\pgfsetbuttcap%
\pgfsetroundjoin%
\definecolor{currentfill}{rgb}{0.000000,0.000000,0.000000}%
\pgfsetfillcolor{currentfill}%
\pgfsetlinewidth{0.803000pt}%
\definecolor{currentstroke}{rgb}{0.000000,0.000000,0.000000}%
\pgfsetstrokecolor{currentstroke}%
\pgfsetdash{}{0pt}%
\pgfsys@defobject{currentmarker}{\pgfqpoint{-0.048611in}{0.000000in}}{\pgfqpoint{-0.000000in}{0.000000in}}{%
\pgfpathmoveto{\pgfqpoint{-0.000000in}{0.000000in}}%
\pgfpathlineto{\pgfqpoint{-0.048611in}{0.000000in}}%
\pgfusepath{stroke,fill}%
}%
\begin{pgfscope}%
\pgfsys@transformshift{0.475556in}{4.129533in}%
\pgfsys@useobject{currentmarker}{}%
\end{pgfscope}%
\end{pgfscope}%
\begin{pgfscope}%
\definecolor{textcolor}{rgb}{0.000000,0.000000,0.000000}%
\pgfsetstrokecolor{textcolor}%
\pgfsetfillcolor{textcolor}%
\pgftext[x=0.289968in, y=4.076771in, left, base]{\color{textcolor}\sffamily\fontsize{10.000000}{12.000000}\selectfont 6}%
\end{pgfscope}%
\begin{pgfscope}%
\pgfsetbuttcap%
\pgfsetroundjoin%
\definecolor{currentfill}{rgb}{0.000000,0.000000,0.000000}%
\pgfsetfillcolor{currentfill}%
\pgfsetlinewidth{0.803000pt}%
\definecolor{currentstroke}{rgb}{0.000000,0.000000,0.000000}%
\pgfsetstrokecolor{currentstroke}%
\pgfsetdash{}{0pt}%
\pgfsys@defobject{currentmarker}{\pgfqpoint{-0.048611in}{0.000000in}}{\pgfqpoint{-0.000000in}{0.000000in}}{%
\pgfpathmoveto{\pgfqpoint{-0.000000in}{0.000000in}}%
\pgfpathlineto{\pgfqpoint{-0.048611in}{0.000000in}}%
\pgfusepath{stroke,fill}%
}%
\begin{pgfscope}%
\pgfsys@transformshift{0.475556in}{4.635738in}%
\pgfsys@useobject{currentmarker}{}%
\end{pgfscope}%
\end{pgfscope}%
\begin{pgfscope}%
\definecolor{textcolor}{rgb}{0.000000,0.000000,0.000000}%
\pgfsetstrokecolor{textcolor}%
\pgfsetfillcolor{textcolor}%
\pgftext[x=0.289968in, y=4.582976in, left, base]{\color{textcolor}\sffamily\fontsize{10.000000}{12.000000}\selectfont 7}%
\end{pgfscope}%
\begin{pgfscope}%
\definecolor{textcolor}{rgb}{0.000000,0.000000,0.000000}%
\pgfsetstrokecolor{textcolor}%
\pgfsetfillcolor{textcolor}%
\pgftext[x=0.234413in,y=2.940302in,,bottom,rotate=90.000000]{\color{textcolor}\sffamily\fontsize{10.000000}{12.000000}\selectfont Average ratings}%
\end{pgfscope}%
\begin{pgfscope}%
\pgfsetrectcap%
\pgfsetmiterjoin%
\pgfsetlinewidth{0.803000pt}%
\definecolor{currentstroke}{rgb}{0.000000,0.000000,0.000000}%
\pgfsetstrokecolor{currentstroke}%
\pgfsetdash{}{0pt}%
\pgfpathmoveto{\pgfqpoint{0.475556in}{1.092302in}}%
\pgfpathlineto{\pgfqpoint{0.475556in}{4.788302in}}%
\pgfusepath{stroke}%
\end{pgfscope}%
\begin{pgfscope}%
\pgfsetrectcap%
\pgfsetmiterjoin%
\pgfsetlinewidth{0.803000pt}%
\definecolor{currentstroke}{rgb}{0.000000,0.000000,0.000000}%
\pgfsetstrokecolor{currentstroke}%
\pgfsetdash{}{0pt}%
\pgfpathmoveto{\pgfqpoint{5.435556in}{1.092302in}}%
\pgfpathlineto{\pgfqpoint{5.435556in}{4.788302in}}%
\pgfusepath{stroke}%
\end{pgfscope}%
\begin{pgfscope}%
\pgfsetrectcap%
\pgfsetmiterjoin%
\pgfsetlinewidth{0.803000pt}%
\definecolor{currentstroke}{rgb}{0.000000,0.000000,0.000000}%
\pgfsetstrokecolor{currentstroke}%
\pgfsetdash{}{0pt}%
\pgfpathmoveto{\pgfqpoint{0.475556in}{1.092302in}}%
\pgfpathlineto{\pgfqpoint{5.435556in}{1.092302in}}%
\pgfusepath{stroke}%
\end{pgfscope}%
\begin{pgfscope}%
\pgfsetrectcap%
\pgfsetmiterjoin%
\pgfsetlinewidth{0.803000pt}%
\definecolor{currentstroke}{rgb}{0.000000,0.000000,0.000000}%
\pgfsetstrokecolor{currentstroke}%
\pgfsetdash{}{0pt}%
\pgfpathmoveto{\pgfqpoint{0.475556in}{4.788302in}}%
\pgfpathlineto{\pgfqpoint{5.435556in}{4.788302in}}%
\pgfusepath{stroke}%
\end{pgfscope}%
\end{pgfpicture}%
\makeatother%
\endgroup%
}
    \caption{The figure represents average rating given by the participant to each of the datasets in section 2 of the survey.}
    \label{fig:question2_2}
\end{figure}

\subsubsection{Section 3: Rank by comparison}
In section 3 of the survey, the users compared all 9 datasets and ranked them according to their photorealism(1 being the best rank).
Figures~\ref{fig:question3} and~\ref{fig:question3_2} show distribution and average ranking for each of the datasets.
The real dataset Pix3D got the highest number of votes for rank 1, while \gls{free} got least.
But if we have a threshold of 5, meaning frequency of votes being in top 5 ranks, then among the automated group, \gls{free} occurs most times followed by SceneNet, Openrooms and Blenderproc.
This is significant because out of 9 datasets, 4 are automated, and these four have the least average rankings.
But \gls{free} breaks the boundary to be in top 5 most times.

\begin{figure}
    \centering
    \resizebox{\textwidth}{!}{%% Creator: Matplotlib, PGF backend
%%
%% To include the figure in your LaTeX document, write
%%   \input{<filename>.pgf}
%%
%% Make sure the required packages are loaded in your preamble
%%   \usepackage{pgf}
%%
%% Figures using additional raster images can only be included by \input if
%% they are in the same directory as the main LaTeX file. For loading figures
%% from other directories you can use the `import` package
%%   \usepackage{import}
%%
%% and then include the figures with
%%   \import{<path to file>}{<filename>.pgf}
%%
%% Matplotlib used the following preamble
%%   \usepackage{fontspec}
%%   \setmainfont{DejaVuSerif.ttf}[Path=\detokenize{/Users/apple/opt/anaconda3/envs/kaolin/lib/python3.7/site-packages/matplotlib/mpl-data/fonts/ttf/}]
%%   \setsansfont{DejaVuSans.ttf}[Path=\detokenize{/Users/apple/opt/anaconda3/envs/kaolin/lib/python3.7/site-packages/matplotlib/mpl-data/fonts/ttf/}]
%%   \setmonofont{DejaVuSansMono.ttf}[Path=\detokenize{/Users/apple/opt/anaconda3/envs/kaolin/lib/python3.7/site-packages/matplotlib/mpl-data/fonts/ttf/}]
%%
\begingroup%
\makeatletter%
\begin{pgfpicture}%
\pgfpathrectangle{\pgfpointorigin}{\pgfqpoint{8.499670in}{4.360980in}}%
\pgfusepath{use as bounding box, clip}%
\begin{pgfscope}%
\pgfsetbuttcap%
\pgfsetmiterjoin%
\definecolor{currentfill}{rgb}{1.000000,1.000000,1.000000}%
\pgfsetfillcolor{currentfill}%
\pgfsetlinewidth{0.000000pt}%
\definecolor{currentstroke}{rgb}{1.000000,1.000000,1.000000}%
\pgfsetstrokecolor{currentstroke}%
\pgfsetdash{}{0pt}%
\pgfpathmoveto{\pgfqpoint{0.000000in}{0.000000in}}%
\pgfpathlineto{\pgfqpoint{8.499670in}{0.000000in}}%
\pgfpathlineto{\pgfqpoint{8.499670in}{4.360980in}}%
\pgfpathlineto{\pgfqpoint{0.000000in}{4.360980in}}%
\pgfpathclose%
\pgfusepath{fill}%
\end{pgfscope}%
\begin{pgfscope}%
\pgfsetbuttcap%
\pgfsetmiterjoin%
\definecolor{currentfill}{rgb}{1.000000,1.000000,1.000000}%
\pgfsetfillcolor{currentfill}%
\pgfsetlinewidth{0.000000pt}%
\definecolor{currentstroke}{rgb}{0.000000,0.000000,0.000000}%
\pgfsetstrokecolor{currentstroke}%
\pgfsetstrokeopacity{0.000000}%
\pgfsetdash{}{0pt}%
\pgfpathmoveto{\pgfqpoint{1.249956in}{0.148611in}}%
\pgfpathlineto{\pgfqpoint{8.372206in}{0.148611in}}%
\pgfpathlineto{\pgfqpoint{8.372206in}{3.998611in}}%
\pgfpathlineto{\pgfqpoint{1.249956in}{3.998611in}}%
\pgfpathclose%
\pgfusepath{fill}%
\end{pgfscope}%
\begin{pgfscope}%
\pgfpathrectangle{\pgfqpoint{1.249956in}{0.148611in}}{\pgfqpoint{7.122250in}{3.850000in}}%
\pgfusepath{clip}%
\pgfsetbuttcap%
\pgfsetmiterjoin%
\definecolor{currentfill}{rgb}{0.248058,0.667205,0.350250}%
\pgfsetfillcolor{currentfill}%
\pgfsetfillopacity{0.500000}%
\pgfsetlinewidth{0.000000pt}%
\definecolor{currentstroke}{rgb}{0.000000,0.000000,0.000000}%
\pgfsetstrokecolor{currentstroke}%
\pgfsetstrokeopacity{0.500000}%
\pgfsetdash{}{0pt}%
\pgfpathmoveto{\pgfqpoint{1.249956in}{3.823611in}}%
\pgfpathlineto{\pgfqpoint{1.651210in}{3.823611in}}%
\pgfpathlineto{\pgfqpoint{1.651210in}{3.617729in}}%
\pgfpathlineto{\pgfqpoint{1.249956in}{3.617729in}}%
\pgfpathclose%
\pgfusepath{fill}%
\end{pgfscope}%
\begin{pgfscope}%
\pgfpathrectangle{\pgfqpoint{1.249956in}{0.148611in}}{\pgfqpoint{7.122250in}{3.850000in}}%
\pgfusepath{clip}%
\pgfsetbuttcap%
\pgfsetmiterjoin%
\definecolor{currentfill}{rgb}{0.248058,0.667205,0.350250}%
\pgfsetfillcolor{currentfill}%
\pgfsetfillopacity{0.500000}%
\pgfsetlinewidth{0.000000pt}%
\definecolor{currentstroke}{rgb}{0.000000,0.000000,0.000000}%
\pgfsetstrokecolor{currentstroke}%
\pgfsetstrokeopacity{0.500000}%
\pgfsetdash{}{0pt}%
\pgfpathmoveto{\pgfqpoint{1.249956in}{3.411846in}}%
\pgfpathlineto{\pgfqpoint{1.718085in}{3.411846in}}%
\pgfpathlineto{\pgfqpoint{1.718085in}{3.205964in}}%
\pgfpathlineto{\pgfqpoint{1.249956in}{3.205964in}}%
\pgfpathclose%
\pgfusepath{fill}%
\end{pgfscope}%
\begin{pgfscope}%
\pgfpathrectangle{\pgfqpoint{1.249956in}{0.148611in}}{\pgfqpoint{7.122250in}{3.850000in}}%
\pgfusepath{clip}%
\pgfsetbuttcap%
\pgfsetmiterjoin%
\definecolor{currentfill}{rgb}{0.248058,0.667205,0.350250}%
\pgfsetfillcolor{currentfill}%
\pgfsetlinewidth{0.000000pt}%
\definecolor{currentstroke}{rgb}{0.000000,0.000000,0.000000}%
\pgfsetstrokecolor{currentstroke}%
\pgfsetstrokeopacity{0.000000}%
\pgfsetdash{}{0pt}%
\pgfpathmoveto{\pgfqpoint{1.249956in}{3.000082in}}%
\pgfpathlineto{\pgfqpoint{1.517459in}{3.000082in}}%
\pgfpathlineto{\pgfqpoint{1.517459in}{2.794199in}}%
\pgfpathlineto{\pgfqpoint{1.249956in}{2.794199in}}%
\pgfpathclose%
\pgfusepath{fill}%
\end{pgfscope}%
\begin{pgfscope}%
\pgfpathrectangle{\pgfqpoint{1.249956in}{0.148611in}}{\pgfqpoint{7.122250in}{3.850000in}}%
\pgfusepath{clip}%
\pgfsetbuttcap%
\pgfsetmiterjoin%
\definecolor{currentfill}{rgb}{0.248058,0.667205,0.350250}%
\pgfsetfillcolor{currentfill}%
\pgfsetfillopacity{0.500000}%
\pgfsetlinewidth{0.000000pt}%
\definecolor{currentstroke}{rgb}{0.000000,0.000000,0.000000}%
\pgfsetstrokecolor{currentstroke}%
\pgfsetstrokeopacity{0.500000}%
\pgfsetdash{}{0pt}%
\pgfpathmoveto{\pgfqpoint{1.249956in}{2.588317in}}%
\pgfpathlineto{\pgfqpoint{1.718085in}{2.588317in}}%
\pgfpathlineto{\pgfqpoint{1.718085in}{2.382435in}}%
\pgfpathlineto{\pgfqpoint{1.249956in}{2.382435in}}%
\pgfpathclose%
\pgfusepath{fill}%
\end{pgfscope}%
\begin{pgfscope}%
\pgfpathrectangle{\pgfqpoint{1.249956in}{0.148611in}}{\pgfqpoint{7.122250in}{3.850000in}}%
\pgfusepath{clip}%
\pgfsetbuttcap%
\pgfsetmiterjoin%
\definecolor{currentfill}{rgb}{0.248058,0.667205,0.350250}%
\pgfsetfillcolor{currentfill}%
\pgfsetfillopacity{0.500000}%
\pgfsetlinewidth{0.000000pt}%
\definecolor{currentstroke}{rgb}{0.000000,0.000000,0.000000}%
\pgfsetstrokecolor{currentstroke}%
\pgfsetstrokeopacity{0.500000}%
\pgfsetdash{}{0pt}%
\pgfpathmoveto{\pgfqpoint{1.249956in}{2.176552in}}%
\pgfpathlineto{\pgfqpoint{2.854970in}{2.176552in}}%
\pgfpathlineto{\pgfqpoint{2.854970in}{1.970670in}}%
\pgfpathlineto{\pgfqpoint{1.249956in}{1.970670in}}%
\pgfpathclose%
\pgfusepath{fill}%
\end{pgfscope}%
\begin{pgfscope}%
\pgfpathrectangle{\pgfqpoint{1.249956in}{0.148611in}}{\pgfqpoint{7.122250in}{3.850000in}}%
\pgfusepath{clip}%
\pgfsetbuttcap%
\pgfsetmiterjoin%
\definecolor{currentfill}{rgb}{0.248058,0.667205,0.350250}%
\pgfsetfillcolor{currentfill}%
\pgfsetlinewidth{0.000000pt}%
\definecolor{currentstroke}{rgb}{0.000000,0.000000,0.000000}%
\pgfsetstrokecolor{currentstroke}%
\pgfsetstrokeopacity{0.000000}%
\pgfsetdash{}{0pt}%
\pgfpathmoveto{\pgfqpoint{1.249956in}{1.764788in}}%
\pgfpathlineto{\pgfqpoint{1.417145in}{1.764788in}}%
\pgfpathlineto{\pgfqpoint{1.417145in}{1.558905in}}%
\pgfpathlineto{\pgfqpoint{1.249956in}{1.558905in}}%
\pgfpathclose%
\pgfusepath{fill}%
\end{pgfscope}%
\begin{pgfscope}%
\pgfpathrectangle{\pgfqpoint{1.249956in}{0.148611in}}{\pgfqpoint{7.122250in}{3.850000in}}%
\pgfusepath{clip}%
\pgfsetbuttcap%
\pgfsetmiterjoin%
\definecolor{currentfill}{rgb}{0.248058,0.667205,0.350250}%
\pgfsetfillcolor{currentfill}%
\pgfsetfillopacity{0.500000}%
\pgfsetlinewidth{0.000000pt}%
\definecolor{currentstroke}{rgb}{0.000000,0.000000,0.000000}%
\pgfsetstrokecolor{currentstroke}%
\pgfsetstrokeopacity{0.500000}%
\pgfsetdash{}{0pt}%
\pgfpathmoveto{\pgfqpoint{1.249956in}{1.353023in}}%
\pgfpathlineto{\pgfqpoint{4.593736in}{1.353023in}}%
\pgfpathlineto{\pgfqpoint{4.593736in}{1.147141in}}%
\pgfpathlineto{\pgfqpoint{1.249956in}{1.147141in}}%
\pgfpathclose%
\pgfusepath{fill}%
\end{pgfscope}%
\begin{pgfscope}%
\pgfpathrectangle{\pgfqpoint{1.249956in}{0.148611in}}{\pgfqpoint{7.122250in}{3.850000in}}%
\pgfusepath{clip}%
\pgfsetbuttcap%
\pgfsetmiterjoin%
\definecolor{currentfill}{rgb}{0.248058,0.667205,0.350250}%
\pgfsetfillcolor{currentfill}%
\pgfsetlinewidth{0.000000pt}%
\definecolor{currentstroke}{rgb}{0.000000,0.000000,0.000000}%
\pgfsetstrokecolor{currentstroke}%
\pgfsetstrokeopacity{0.000000}%
\pgfsetdash{}{0pt}%
\pgfpathmoveto{\pgfqpoint{1.249956in}{0.941258in}}%
\pgfpathlineto{\pgfqpoint{1.316832in}{0.941258in}}%
\pgfpathlineto{\pgfqpoint{1.316832in}{0.735376in}}%
\pgfpathlineto{\pgfqpoint{1.249956in}{0.735376in}}%
\pgfpathclose%
\pgfusepath{fill}%
\end{pgfscope}%
\begin{pgfscope}%
\pgfpathrectangle{\pgfqpoint{1.249956in}{0.148611in}}{\pgfqpoint{7.122250in}{3.850000in}}%
\pgfusepath{clip}%
\pgfsetbuttcap%
\pgfsetmiterjoin%
\definecolor{currentfill}{rgb}{0.248058,0.667205,0.350250}%
\pgfsetfillcolor{currentfill}%
\pgfsetlinewidth{0.000000pt}%
\definecolor{currentstroke}{rgb}{0.000000,0.000000,0.000000}%
\pgfsetstrokecolor{currentstroke}%
\pgfsetstrokeopacity{0.000000}%
\pgfsetdash{}{0pt}%
\pgfpathmoveto{\pgfqpoint{1.249956in}{0.529493in}}%
\pgfpathlineto{\pgfqpoint{1.584334in}{0.529493in}}%
\pgfpathlineto{\pgfqpoint{1.584334in}{0.323611in}}%
\pgfpathlineto{\pgfqpoint{1.249956in}{0.323611in}}%
\pgfpathclose%
\pgfusepath{fill}%
\end{pgfscope}%
\begin{pgfscope}%
\pgfpathrectangle{\pgfqpoint{1.249956in}{0.148611in}}{\pgfqpoint{7.122250in}{3.850000in}}%
\pgfusepath{clip}%
\pgfsetbuttcap%
\pgfsetmiterjoin%
\definecolor{currentfill}{rgb}{0.488581,0.779931,0.397924}%
\pgfsetfillcolor{currentfill}%
\pgfsetfillopacity{0.500000}%
\pgfsetlinewidth{0.000000pt}%
\definecolor{currentstroke}{rgb}{0.000000,0.000000,0.000000}%
\pgfsetstrokecolor{currentstroke}%
\pgfsetstrokeopacity{0.500000}%
\pgfsetdash{}{0pt}%
\pgfpathmoveto{\pgfqpoint{1.651210in}{3.823611in}}%
\pgfpathlineto{\pgfqpoint{2.186215in}{3.823611in}}%
\pgfpathlineto{\pgfqpoint{2.186215in}{3.617729in}}%
\pgfpathlineto{\pgfqpoint{1.651210in}{3.617729in}}%
\pgfpathclose%
\pgfusepath{fill}%
\end{pgfscope}%
\begin{pgfscope}%
\pgfpathrectangle{\pgfqpoint{1.249956in}{0.148611in}}{\pgfqpoint{7.122250in}{3.850000in}}%
\pgfusepath{clip}%
\pgfsetbuttcap%
\pgfsetmiterjoin%
\definecolor{currentfill}{rgb}{0.488581,0.779931,0.397924}%
\pgfsetfillcolor{currentfill}%
\pgfsetfillopacity{0.500000}%
\pgfsetlinewidth{0.000000pt}%
\definecolor{currentstroke}{rgb}{0.000000,0.000000,0.000000}%
\pgfsetstrokecolor{currentstroke}%
\pgfsetstrokeopacity{0.500000}%
\pgfsetdash{}{0pt}%
\pgfpathmoveto{\pgfqpoint{1.718085in}{3.411846in}}%
\pgfpathlineto{\pgfqpoint{2.219652in}{3.411846in}}%
\pgfpathlineto{\pgfqpoint{2.219652in}{3.205964in}}%
\pgfpathlineto{\pgfqpoint{1.718085in}{3.205964in}}%
\pgfpathclose%
\pgfusepath{fill}%
\end{pgfscope}%
\begin{pgfscope}%
\pgfpathrectangle{\pgfqpoint{1.249956in}{0.148611in}}{\pgfqpoint{7.122250in}{3.850000in}}%
\pgfusepath{clip}%
\pgfsetbuttcap%
\pgfsetmiterjoin%
\definecolor{currentfill}{rgb}{0.488581,0.779931,0.397924}%
\pgfsetfillcolor{currentfill}%
\pgfsetlinewidth{0.000000pt}%
\definecolor{currentstroke}{rgb}{0.000000,0.000000,0.000000}%
\pgfsetstrokecolor{currentstroke}%
\pgfsetstrokeopacity{0.000000}%
\pgfsetdash{}{0pt}%
\pgfpathmoveto{\pgfqpoint{1.517459in}{3.000082in}}%
\pgfpathlineto{\pgfqpoint{1.885274in}{3.000082in}}%
\pgfpathlineto{\pgfqpoint{1.885274in}{2.794199in}}%
\pgfpathlineto{\pgfqpoint{1.517459in}{2.794199in}}%
\pgfpathclose%
\pgfusepath{fill}%
\end{pgfscope}%
\begin{pgfscope}%
\pgfpathrectangle{\pgfqpoint{1.249956in}{0.148611in}}{\pgfqpoint{7.122250in}{3.850000in}}%
\pgfusepath{clip}%
\pgfsetbuttcap%
\pgfsetmiterjoin%
\definecolor{currentfill}{rgb}{0.488581,0.779931,0.397924}%
\pgfsetfillcolor{currentfill}%
\pgfsetfillopacity{0.500000}%
\pgfsetlinewidth{0.000000pt}%
\definecolor{currentstroke}{rgb}{0.000000,0.000000,0.000000}%
\pgfsetstrokecolor{currentstroke}%
\pgfsetstrokeopacity{0.500000}%
\pgfsetdash{}{0pt}%
\pgfpathmoveto{\pgfqpoint{1.718085in}{2.588317in}}%
\pgfpathlineto{\pgfqpoint{3.155911in}{2.588317in}}%
\pgfpathlineto{\pgfqpoint{3.155911in}{2.382435in}}%
\pgfpathlineto{\pgfqpoint{1.718085in}{2.382435in}}%
\pgfpathclose%
\pgfusepath{fill}%
\end{pgfscope}%
\begin{pgfscope}%
\pgfpathrectangle{\pgfqpoint{1.249956in}{0.148611in}}{\pgfqpoint{7.122250in}{3.850000in}}%
\pgfusepath{clip}%
\pgfsetbuttcap%
\pgfsetmiterjoin%
\definecolor{currentfill}{rgb}{0.488581,0.779931,0.397924}%
\pgfsetfillcolor{currentfill}%
\pgfsetfillopacity{0.500000}%
\pgfsetlinewidth{0.000000pt}%
\definecolor{currentstroke}{rgb}{0.000000,0.000000,0.000000}%
\pgfsetstrokecolor{currentstroke}%
\pgfsetstrokeopacity{0.500000}%
\pgfsetdash{}{0pt}%
\pgfpathmoveto{\pgfqpoint{2.854970in}{2.176552in}}%
\pgfpathlineto{\pgfqpoint{5.061865in}{2.176552in}}%
\pgfpathlineto{\pgfqpoint{5.061865in}{1.970670in}}%
\pgfpathlineto{\pgfqpoint{2.854970in}{1.970670in}}%
\pgfpathclose%
\pgfusepath{fill}%
\end{pgfscope}%
\begin{pgfscope}%
\pgfpathrectangle{\pgfqpoint{1.249956in}{0.148611in}}{\pgfqpoint{7.122250in}{3.850000in}}%
\pgfusepath{clip}%
\pgfsetbuttcap%
\pgfsetmiterjoin%
\definecolor{currentfill}{rgb}{0.488581,0.779931,0.397924}%
\pgfsetfillcolor{currentfill}%
\pgfsetlinewidth{0.000000pt}%
\definecolor{currentstroke}{rgb}{0.000000,0.000000,0.000000}%
\pgfsetstrokecolor{currentstroke}%
\pgfsetstrokeopacity{0.000000}%
\pgfsetdash{}{0pt}%
\pgfpathmoveto{\pgfqpoint{1.417145in}{1.764788in}}%
\pgfpathlineto{\pgfqpoint{1.617772in}{1.764788in}}%
\pgfpathlineto{\pgfqpoint{1.617772in}{1.558905in}}%
\pgfpathlineto{\pgfqpoint{1.417145in}{1.558905in}}%
\pgfpathclose%
\pgfusepath{fill}%
\end{pgfscope}%
\begin{pgfscope}%
\pgfpathrectangle{\pgfqpoint{1.249956in}{0.148611in}}{\pgfqpoint{7.122250in}{3.850000in}}%
\pgfusepath{clip}%
\pgfsetbuttcap%
\pgfsetmiterjoin%
\definecolor{currentfill}{rgb}{0.488581,0.779931,0.397924}%
\pgfsetfillcolor{currentfill}%
\pgfsetfillopacity{0.500000}%
\pgfsetlinewidth{0.000000pt}%
\definecolor{currentstroke}{rgb}{0.000000,0.000000,0.000000}%
\pgfsetstrokecolor{currentstroke}%
\pgfsetstrokeopacity{0.500000}%
\pgfsetdash{}{0pt}%
\pgfpathmoveto{\pgfqpoint{4.593736in}{1.353023in}}%
\pgfpathlineto{\pgfqpoint{5.797496in}{1.353023in}}%
\pgfpathlineto{\pgfqpoint{5.797496in}{1.147141in}}%
\pgfpathlineto{\pgfqpoint{4.593736in}{1.147141in}}%
\pgfpathclose%
\pgfusepath{fill}%
\end{pgfscope}%
\begin{pgfscope}%
\pgfpathrectangle{\pgfqpoint{1.249956in}{0.148611in}}{\pgfqpoint{7.122250in}{3.850000in}}%
\pgfusepath{clip}%
\pgfsetbuttcap%
\pgfsetmiterjoin%
\definecolor{currentfill}{rgb}{0.488581,0.779931,0.397924}%
\pgfsetfillcolor{currentfill}%
\pgfsetlinewidth{0.000000pt}%
\definecolor{currentstroke}{rgb}{0.000000,0.000000,0.000000}%
\pgfsetstrokecolor{currentstroke}%
\pgfsetstrokeopacity{0.000000}%
\pgfsetdash{}{0pt}%
\pgfpathmoveto{\pgfqpoint{1.316832in}{0.941258in}}%
\pgfpathlineto{\pgfqpoint{1.617772in}{0.941258in}}%
\pgfpathlineto{\pgfqpoint{1.617772in}{0.735376in}}%
\pgfpathlineto{\pgfqpoint{1.316832in}{0.735376in}}%
\pgfpathclose%
\pgfusepath{fill}%
\end{pgfscope}%
\begin{pgfscope}%
\pgfpathrectangle{\pgfqpoint{1.249956in}{0.148611in}}{\pgfqpoint{7.122250in}{3.850000in}}%
\pgfusepath{clip}%
\pgfsetbuttcap%
\pgfsetmiterjoin%
\definecolor{currentfill}{rgb}{0.488581,0.779931,0.397924}%
\pgfsetfillcolor{currentfill}%
\pgfsetlinewidth{0.000000pt}%
\definecolor{currentstroke}{rgb}{0.000000,0.000000,0.000000}%
\pgfsetstrokecolor{currentstroke}%
\pgfsetstrokeopacity{0.000000}%
\pgfsetdash{}{0pt}%
\pgfpathmoveto{\pgfqpoint{1.584334in}{0.529493in}}%
\pgfpathlineto{\pgfqpoint{1.952150in}{0.529493in}}%
\pgfpathlineto{\pgfqpoint{1.952150in}{0.323611in}}%
\pgfpathlineto{\pgfqpoint{1.584334in}{0.323611in}}%
\pgfpathclose%
\pgfusepath{fill}%
\end{pgfscope}%
\begin{pgfscope}%
\pgfpathrectangle{\pgfqpoint{1.249956in}{0.148611in}}{\pgfqpoint{7.122250in}{3.850000in}}%
\pgfusepath{clip}%
\pgfsetbuttcap%
\pgfsetmiterjoin%
\definecolor{currentfill}{rgb}{0.701961,0.872972,0.448674}%
\pgfsetfillcolor{currentfill}%
\pgfsetfillopacity{0.500000}%
\pgfsetlinewidth{0.000000pt}%
\definecolor{currentstroke}{rgb}{0.000000,0.000000,0.000000}%
\pgfsetstrokecolor{currentstroke}%
\pgfsetstrokeopacity{0.500000}%
\pgfsetdash{}{0pt}%
\pgfpathmoveto{\pgfqpoint{2.186215in}{3.823611in}}%
\pgfpathlineto{\pgfqpoint{2.955284in}{3.823611in}}%
\pgfpathlineto{\pgfqpoint{2.955284in}{3.617729in}}%
\pgfpathlineto{\pgfqpoint{2.186215in}{3.617729in}}%
\pgfpathclose%
\pgfusepath{fill}%
\end{pgfscope}%
\begin{pgfscope}%
\pgfpathrectangle{\pgfqpoint{1.249956in}{0.148611in}}{\pgfqpoint{7.122250in}{3.850000in}}%
\pgfusepath{clip}%
\pgfsetbuttcap%
\pgfsetmiterjoin%
\definecolor{currentfill}{rgb}{0.701961,0.872972,0.448674}%
\pgfsetfillcolor{currentfill}%
\pgfsetfillopacity{0.500000}%
\pgfsetlinewidth{0.000000pt}%
\definecolor{currentstroke}{rgb}{0.000000,0.000000,0.000000}%
\pgfsetstrokecolor{currentstroke}%
\pgfsetstrokeopacity{0.500000}%
\pgfsetdash{}{0pt}%
\pgfpathmoveto{\pgfqpoint{2.219652in}{3.411846in}}%
\pgfpathlineto{\pgfqpoint{3.055597in}{3.411846in}}%
\pgfpathlineto{\pgfqpoint{3.055597in}{3.205964in}}%
\pgfpathlineto{\pgfqpoint{2.219652in}{3.205964in}}%
\pgfpathclose%
\pgfusepath{fill}%
\end{pgfscope}%
\begin{pgfscope}%
\pgfpathrectangle{\pgfqpoint{1.249956in}{0.148611in}}{\pgfqpoint{7.122250in}{3.850000in}}%
\pgfusepath{clip}%
\pgfsetbuttcap%
\pgfsetmiterjoin%
\definecolor{currentfill}{rgb}{0.701961,0.872972,0.448674}%
\pgfsetfillcolor{currentfill}%
\pgfsetlinewidth{0.000000pt}%
\definecolor{currentstroke}{rgb}{0.000000,0.000000,0.000000}%
\pgfsetstrokecolor{currentstroke}%
\pgfsetstrokeopacity{0.000000}%
\pgfsetdash{}{0pt}%
\pgfpathmoveto{\pgfqpoint{1.885274in}{3.000082in}}%
\pgfpathlineto{\pgfqpoint{2.353404in}{3.000082in}}%
\pgfpathlineto{\pgfqpoint{2.353404in}{2.794199in}}%
\pgfpathlineto{\pgfqpoint{1.885274in}{2.794199in}}%
\pgfpathclose%
\pgfusepath{fill}%
\end{pgfscope}%
\begin{pgfscope}%
\pgfpathrectangle{\pgfqpoint{1.249956in}{0.148611in}}{\pgfqpoint{7.122250in}{3.850000in}}%
\pgfusepath{clip}%
\pgfsetbuttcap%
\pgfsetmiterjoin%
\definecolor{currentfill}{rgb}{0.701961,0.872972,0.448674}%
\pgfsetfillcolor{currentfill}%
\pgfsetfillopacity{0.500000}%
\pgfsetlinewidth{0.000000pt}%
\definecolor{currentstroke}{rgb}{0.000000,0.000000,0.000000}%
\pgfsetstrokecolor{currentstroke}%
\pgfsetstrokeopacity{0.500000}%
\pgfsetdash{}{0pt}%
\pgfpathmoveto{\pgfqpoint{3.155911in}{2.588317in}}%
\pgfpathlineto{\pgfqpoint{4.760925in}{2.588317in}}%
\pgfpathlineto{\pgfqpoint{4.760925in}{2.382435in}}%
\pgfpathlineto{\pgfqpoint{3.155911in}{2.382435in}}%
\pgfpathclose%
\pgfusepath{fill}%
\end{pgfscope}%
\begin{pgfscope}%
\pgfpathrectangle{\pgfqpoint{1.249956in}{0.148611in}}{\pgfqpoint{7.122250in}{3.850000in}}%
\pgfusepath{clip}%
\pgfsetbuttcap%
\pgfsetmiterjoin%
\definecolor{currentfill}{rgb}{0.701961,0.872972,0.448674}%
\pgfsetfillcolor{currentfill}%
\pgfsetfillopacity{0.500000}%
\pgfsetlinewidth{0.000000pt}%
\definecolor{currentstroke}{rgb}{0.000000,0.000000,0.000000}%
\pgfsetstrokecolor{currentstroke}%
\pgfsetstrokeopacity{0.500000}%
\pgfsetdash{}{0pt}%
\pgfpathmoveto{\pgfqpoint{5.061865in}{2.176552in}}%
\pgfpathlineto{\pgfqpoint{5.931247in}{2.176552in}}%
\pgfpathlineto{\pgfqpoint{5.931247in}{1.970670in}}%
\pgfpathlineto{\pgfqpoint{5.061865in}{1.970670in}}%
\pgfpathclose%
\pgfusepath{fill}%
\end{pgfscope}%
\begin{pgfscope}%
\pgfpathrectangle{\pgfqpoint{1.249956in}{0.148611in}}{\pgfqpoint{7.122250in}{3.850000in}}%
\pgfusepath{clip}%
\pgfsetbuttcap%
\pgfsetmiterjoin%
\definecolor{currentfill}{rgb}{0.701961,0.872972,0.448674}%
\pgfsetfillcolor{currentfill}%
\pgfsetlinewidth{0.000000pt}%
\definecolor{currentstroke}{rgb}{0.000000,0.000000,0.000000}%
\pgfsetstrokecolor{currentstroke}%
\pgfsetstrokeopacity{0.000000}%
\pgfsetdash{}{0pt}%
\pgfpathmoveto{\pgfqpoint{1.617772in}{1.764788in}}%
\pgfpathlineto{\pgfqpoint{2.052463in}{1.764788in}}%
\pgfpathlineto{\pgfqpoint{2.052463in}{1.558905in}}%
\pgfpathlineto{\pgfqpoint{1.617772in}{1.558905in}}%
\pgfpathclose%
\pgfusepath{fill}%
\end{pgfscope}%
\begin{pgfscope}%
\pgfpathrectangle{\pgfqpoint{1.249956in}{0.148611in}}{\pgfqpoint{7.122250in}{3.850000in}}%
\pgfusepath{clip}%
\pgfsetbuttcap%
\pgfsetmiterjoin%
\definecolor{currentfill}{rgb}{0.701961,0.872972,0.448674}%
\pgfsetfillcolor{currentfill}%
\pgfsetfillopacity{0.500000}%
\pgfsetlinewidth{0.000000pt}%
\definecolor{currentstroke}{rgb}{0.000000,0.000000,0.000000}%
\pgfsetstrokecolor{currentstroke}%
\pgfsetstrokeopacity{0.500000}%
\pgfsetdash{}{0pt}%
\pgfpathmoveto{\pgfqpoint{5.797496in}{1.353023in}}%
\pgfpathlineto{\pgfqpoint{6.700317in}{1.353023in}}%
\pgfpathlineto{\pgfqpoint{6.700317in}{1.147141in}}%
\pgfpathlineto{\pgfqpoint{5.797496in}{1.147141in}}%
\pgfpathclose%
\pgfusepath{fill}%
\end{pgfscope}%
\begin{pgfscope}%
\pgfpathrectangle{\pgfqpoint{1.249956in}{0.148611in}}{\pgfqpoint{7.122250in}{3.850000in}}%
\pgfusepath{clip}%
\pgfsetbuttcap%
\pgfsetmiterjoin%
\definecolor{currentfill}{rgb}{0.701961,0.872972,0.448674}%
\pgfsetfillcolor{currentfill}%
\pgfsetlinewidth{0.000000pt}%
\definecolor{currentstroke}{rgb}{0.000000,0.000000,0.000000}%
\pgfsetstrokecolor{currentstroke}%
\pgfsetstrokeopacity{0.000000}%
\pgfsetdash{}{0pt}%
\pgfpathmoveto{\pgfqpoint{1.617772in}{0.941258in}}%
\pgfpathlineto{\pgfqpoint{2.186215in}{0.941258in}}%
\pgfpathlineto{\pgfqpoint{2.186215in}{0.735376in}}%
\pgfpathlineto{\pgfqpoint{1.617772in}{0.735376in}}%
\pgfpathclose%
\pgfusepath{fill}%
\end{pgfscope}%
\begin{pgfscope}%
\pgfpathrectangle{\pgfqpoint{1.249956in}{0.148611in}}{\pgfqpoint{7.122250in}{3.850000in}}%
\pgfusepath{clip}%
\pgfsetbuttcap%
\pgfsetmiterjoin%
\definecolor{currentfill}{rgb}{0.701961,0.872972,0.448674}%
\pgfsetfillcolor{currentfill}%
\pgfsetlinewidth{0.000000pt}%
\definecolor{currentstroke}{rgb}{0.000000,0.000000,0.000000}%
\pgfsetstrokecolor{currentstroke}%
\pgfsetstrokeopacity{0.000000}%
\pgfsetdash{}{0pt}%
\pgfpathmoveto{\pgfqpoint{1.952150in}{0.529493in}}%
\pgfpathlineto{\pgfqpoint{2.620906in}{0.529493in}}%
\pgfpathlineto{\pgfqpoint{2.620906in}{0.323611in}}%
\pgfpathlineto{\pgfqpoint{1.952150in}{0.323611in}}%
\pgfpathclose%
\pgfusepath{fill}%
\end{pgfscope}%
\begin{pgfscope}%
\pgfpathrectangle{\pgfqpoint{1.249956in}{0.148611in}}{\pgfqpoint{7.122250in}{3.850000in}}%
\pgfusepath{clip}%
\pgfsetbuttcap%
\pgfsetmiterjoin%
\definecolor{currentfill}{rgb}{0.868512,0.944637,0.569089}%
\pgfsetfillcolor{currentfill}%
\pgfsetfillopacity{0.500000}%
\pgfsetlinewidth{0.000000pt}%
\definecolor{currentstroke}{rgb}{0.000000,0.000000,0.000000}%
\pgfsetstrokecolor{currentstroke}%
\pgfsetstrokeopacity{0.500000}%
\pgfsetdash{}{0pt}%
\pgfpathmoveto{\pgfqpoint{2.955284in}{3.823611in}}%
\pgfpathlineto{\pgfqpoint{3.690915in}{3.823611in}}%
\pgfpathlineto{\pgfqpoint{3.690915in}{3.617729in}}%
\pgfpathlineto{\pgfqpoint{2.955284in}{3.617729in}}%
\pgfpathclose%
\pgfusepath{fill}%
\end{pgfscope}%
\begin{pgfscope}%
\pgfpathrectangle{\pgfqpoint{1.249956in}{0.148611in}}{\pgfqpoint{7.122250in}{3.850000in}}%
\pgfusepath{clip}%
\pgfsetbuttcap%
\pgfsetmiterjoin%
\definecolor{currentfill}{rgb}{0.868512,0.944637,0.569089}%
\pgfsetfillcolor{currentfill}%
\pgfsetfillopacity{0.500000}%
\pgfsetlinewidth{0.000000pt}%
\definecolor{currentstroke}{rgb}{0.000000,0.000000,0.000000}%
\pgfsetstrokecolor{currentstroke}%
\pgfsetstrokeopacity{0.500000}%
\pgfsetdash{}{0pt}%
\pgfpathmoveto{\pgfqpoint{3.055597in}{3.411846in}}%
\pgfpathlineto{\pgfqpoint{4.326233in}{3.411846in}}%
\pgfpathlineto{\pgfqpoint{4.326233in}{3.205964in}}%
\pgfpathlineto{\pgfqpoint{3.055597in}{3.205964in}}%
\pgfpathclose%
\pgfusepath{fill}%
\end{pgfscope}%
\begin{pgfscope}%
\pgfpathrectangle{\pgfqpoint{1.249956in}{0.148611in}}{\pgfqpoint{7.122250in}{3.850000in}}%
\pgfusepath{clip}%
\pgfsetbuttcap%
\pgfsetmiterjoin%
\definecolor{currentfill}{rgb}{0.868512,0.944637,0.569089}%
\pgfsetfillcolor{currentfill}%
\pgfsetlinewidth{0.000000pt}%
\definecolor{currentstroke}{rgb}{0.000000,0.000000,0.000000}%
\pgfsetstrokecolor{currentstroke}%
\pgfsetstrokeopacity{0.000000}%
\pgfsetdash{}{0pt}%
\pgfpathmoveto{\pgfqpoint{2.353404in}{3.000082in}}%
\pgfpathlineto{\pgfqpoint{2.754657in}{3.000082in}}%
\pgfpathlineto{\pgfqpoint{2.754657in}{2.794199in}}%
\pgfpathlineto{\pgfqpoint{2.353404in}{2.794199in}}%
\pgfpathclose%
\pgfusepath{fill}%
\end{pgfscope}%
\begin{pgfscope}%
\pgfpathrectangle{\pgfqpoint{1.249956in}{0.148611in}}{\pgfqpoint{7.122250in}{3.850000in}}%
\pgfusepath{clip}%
\pgfsetbuttcap%
\pgfsetmiterjoin%
\definecolor{currentfill}{rgb}{0.868512,0.944637,0.569089}%
\pgfsetfillcolor{currentfill}%
\pgfsetfillopacity{0.500000}%
\pgfsetlinewidth{0.000000pt}%
\definecolor{currentstroke}{rgb}{0.000000,0.000000,0.000000}%
\pgfsetstrokecolor{currentstroke}%
\pgfsetstrokeopacity{0.500000}%
\pgfsetdash{}{0pt}%
\pgfpathmoveto{\pgfqpoint{4.760925in}{2.588317in}}%
\pgfpathlineto{\pgfqpoint{6.064999in}{2.588317in}}%
\pgfpathlineto{\pgfqpoint{6.064999in}{2.382435in}}%
\pgfpathlineto{\pgfqpoint{4.760925in}{2.382435in}}%
\pgfpathclose%
\pgfusepath{fill}%
\end{pgfscope}%
\begin{pgfscope}%
\pgfpathrectangle{\pgfqpoint{1.249956in}{0.148611in}}{\pgfqpoint{7.122250in}{3.850000in}}%
\pgfusepath{clip}%
\pgfsetbuttcap%
\pgfsetmiterjoin%
\definecolor{currentfill}{rgb}{0.868512,0.944637,0.569089}%
\pgfsetfillcolor{currentfill}%
\pgfsetfillopacity{0.500000}%
\pgfsetlinewidth{0.000000pt}%
\definecolor{currentstroke}{rgb}{0.000000,0.000000,0.000000}%
\pgfsetstrokecolor{currentstroke}%
\pgfsetstrokeopacity{0.500000}%
\pgfsetdash{}{0pt}%
\pgfpathmoveto{\pgfqpoint{5.931247in}{2.176552in}}%
\pgfpathlineto{\pgfqpoint{6.800630in}{2.176552in}}%
\pgfpathlineto{\pgfqpoint{6.800630in}{1.970670in}}%
\pgfpathlineto{\pgfqpoint{5.931247in}{1.970670in}}%
\pgfpathclose%
\pgfusepath{fill}%
\end{pgfscope}%
\begin{pgfscope}%
\pgfpathrectangle{\pgfqpoint{1.249956in}{0.148611in}}{\pgfqpoint{7.122250in}{3.850000in}}%
\pgfusepath{clip}%
\pgfsetbuttcap%
\pgfsetmiterjoin%
\definecolor{currentfill}{rgb}{0.868512,0.944637,0.569089}%
\pgfsetfillcolor{currentfill}%
\pgfsetlinewidth{0.000000pt}%
\definecolor{currentstroke}{rgb}{0.000000,0.000000,0.000000}%
\pgfsetstrokecolor{currentstroke}%
\pgfsetstrokeopacity{0.000000}%
\pgfsetdash{}{0pt}%
\pgfpathmoveto{\pgfqpoint{2.052463in}{1.764788in}}%
\pgfpathlineto{\pgfqpoint{2.721219in}{1.764788in}}%
\pgfpathlineto{\pgfqpoint{2.721219in}{1.558905in}}%
\pgfpathlineto{\pgfqpoint{2.052463in}{1.558905in}}%
\pgfpathclose%
\pgfusepath{fill}%
\end{pgfscope}%
\begin{pgfscope}%
\pgfpathrectangle{\pgfqpoint{1.249956in}{0.148611in}}{\pgfqpoint{7.122250in}{3.850000in}}%
\pgfusepath{clip}%
\pgfsetbuttcap%
\pgfsetmiterjoin%
\definecolor{currentfill}{rgb}{0.868512,0.944637,0.569089}%
\pgfsetfillcolor{currentfill}%
\pgfsetfillopacity{0.500000}%
\pgfsetlinewidth{0.000000pt}%
\definecolor{currentstroke}{rgb}{0.000000,0.000000,0.000000}%
\pgfsetstrokecolor{currentstroke}%
\pgfsetstrokeopacity{0.500000}%
\pgfsetdash{}{0pt}%
\pgfpathmoveto{\pgfqpoint{6.700317in}{1.353023in}}%
\pgfpathlineto{\pgfqpoint{7.034695in}{1.353023in}}%
\pgfpathlineto{\pgfqpoint{7.034695in}{1.147141in}}%
\pgfpathlineto{\pgfqpoint{6.700317in}{1.147141in}}%
\pgfpathclose%
\pgfusepath{fill}%
\end{pgfscope}%
\begin{pgfscope}%
\pgfpathrectangle{\pgfqpoint{1.249956in}{0.148611in}}{\pgfqpoint{7.122250in}{3.850000in}}%
\pgfusepath{clip}%
\pgfsetbuttcap%
\pgfsetmiterjoin%
\definecolor{currentfill}{rgb}{0.868512,0.944637,0.569089}%
\pgfsetfillcolor{currentfill}%
\pgfsetlinewidth{0.000000pt}%
\definecolor{currentstroke}{rgb}{0.000000,0.000000,0.000000}%
\pgfsetstrokecolor{currentstroke}%
\pgfsetstrokeopacity{0.000000}%
\pgfsetdash{}{0pt}%
\pgfpathmoveto{\pgfqpoint{2.186215in}{0.941258in}}%
\pgfpathlineto{\pgfqpoint{3.155911in}{0.941258in}}%
\pgfpathlineto{\pgfqpoint{3.155911in}{0.735376in}}%
\pgfpathlineto{\pgfqpoint{2.186215in}{0.735376in}}%
\pgfpathclose%
\pgfusepath{fill}%
\end{pgfscope}%
\begin{pgfscope}%
\pgfpathrectangle{\pgfqpoint{1.249956in}{0.148611in}}{\pgfqpoint{7.122250in}{3.850000in}}%
\pgfusepath{clip}%
\pgfsetbuttcap%
\pgfsetmiterjoin%
\definecolor{currentfill}{rgb}{0.868512,0.944637,0.569089}%
\pgfsetfillcolor{currentfill}%
\pgfsetlinewidth{0.000000pt}%
\definecolor{currentstroke}{rgb}{0.000000,0.000000,0.000000}%
\pgfsetstrokecolor{currentstroke}%
\pgfsetstrokeopacity{0.000000}%
\pgfsetdash{}{0pt}%
\pgfpathmoveto{\pgfqpoint{2.620906in}{0.529493in}}%
\pgfpathlineto{\pgfqpoint{3.189348in}{0.529493in}}%
\pgfpathlineto{\pgfqpoint{3.189348in}{0.323611in}}%
\pgfpathlineto{\pgfqpoint{2.620906in}{0.323611in}}%
\pgfpathclose%
\pgfusepath{fill}%
\end{pgfscope}%
\begin{pgfscope}%
\pgfpathrectangle{\pgfqpoint{1.249956in}{0.148611in}}{\pgfqpoint{7.122250in}{3.850000in}}%
\pgfusepath{clip}%
\pgfsetbuttcap%
\pgfsetmiterjoin%
\definecolor{currentfill}{rgb}{0.997078,0.998770,0.745021}%
\pgfsetfillcolor{currentfill}%
\pgfsetfillopacity{0.500000}%
\pgfsetlinewidth{0.000000pt}%
\definecolor{currentstroke}{rgb}{0.000000,0.000000,0.000000}%
\pgfsetstrokecolor{currentstroke}%
\pgfsetstrokeopacity{0.500000}%
\pgfsetdash{}{0pt}%
\pgfpathmoveto{\pgfqpoint{3.690915in}{3.823611in}}%
\pgfpathlineto{\pgfqpoint{4.861238in}{3.823611in}}%
\pgfpathlineto{\pgfqpoint{4.861238in}{3.617729in}}%
\pgfpathlineto{\pgfqpoint{3.690915in}{3.617729in}}%
\pgfpathclose%
\pgfusepath{fill}%
\end{pgfscope}%
\begin{pgfscope}%
\pgfpathrectangle{\pgfqpoint{1.249956in}{0.148611in}}{\pgfqpoint{7.122250in}{3.850000in}}%
\pgfusepath{clip}%
\pgfsetbuttcap%
\pgfsetmiterjoin%
\definecolor{currentfill}{rgb}{0.997078,0.998770,0.745021}%
\pgfsetfillcolor{currentfill}%
\pgfsetfillopacity{0.500000}%
\pgfsetlinewidth{0.000000pt}%
\definecolor{currentstroke}{rgb}{0.000000,0.000000,0.000000}%
\pgfsetstrokecolor{currentstroke}%
\pgfsetstrokeopacity{0.500000}%
\pgfsetdash{}{0pt}%
\pgfpathmoveto{\pgfqpoint{4.326233in}{3.411846in}}%
\pgfpathlineto{\pgfqpoint{5.162178in}{3.411846in}}%
\pgfpathlineto{\pgfqpoint{5.162178in}{3.205964in}}%
\pgfpathlineto{\pgfqpoint{4.326233in}{3.205964in}}%
\pgfpathclose%
\pgfusepath{fill}%
\end{pgfscope}%
\begin{pgfscope}%
\pgfpathrectangle{\pgfqpoint{1.249956in}{0.148611in}}{\pgfqpoint{7.122250in}{3.850000in}}%
\pgfusepath{clip}%
\pgfsetbuttcap%
\pgfsetmiterjoin%
\definecolor{currentfill}{rgb}{0.997078,0.998770,0.745021}%
\pgfsetfillcolor{currentfill}%
\pgfsetlinewidth{0.000000pt}%
\definecolor{currentstroke}{rgb}{0.000000,0.000000,0.000000}%
\pgfsetstrokecolor{currentstroke}%
\pgfsetstrokeopacity{0.000000}%
\pgfsetdash{}{0pt}%
\pgfpathmoveto{\pgfqpoint{2.754657in}{3.000082in}}%
\pgfpathlineto{\pgfqpoint{3.389975in}{3.000082in}}%
\pgfpathlineto{\pgfqpoint{3.389975in}{2.794199in}}%
\pgfpathlineto{\pgfqpoint{2.754657in}{2.794199in}}%
\pgfpathclose%
\pgfusepath{fill}%
\end{pgfscope}%
\begin{pgfscope}%
\pgfpathrectangle{\pgfqpoint{1.249956in}{0.148611in}}{\pgfqpoint{7.122250in}{3.850000in}}%
\pgfusepath{clip}%
\pgfsetbuttcap%
\pgfsetmiterjoin%
\definecolor{currentfill}{rgb}{0.997078,0.998770,0.745021}%
\pgfsetfillcolor{currentfill}%
\pgfsetfillopacity{0.500000}%
\pgfsetlinewidth{0.000000pt}%
\definecolor{currentstroke}{rgb}{0.000000,0.000000,0.000000}%
\pgfsetstrokecolor{currentstroke}%
\pgfsetstrokeopacity{0.500000}%
\pgfsetdash{}{0pt}%
\pgfpathmoveto{\pgfqpoint{6.064999in}{2.588317in}}%
\pgfpathlineto{\pgfqpoint{6.733755in}{2.588317in}}%
\pgfpathlineto{\pgfqpoint{6.733755in}{2.382435in}}%
\pgfpathlineto{\pgfqpoint{6.064999in}{2.382435in}}%
\pgfpathclose%
\pgfusepath{fill}%
\end{pgfscope}%
\begin{pgfscope}%
\pgfpathrectangle{\pgfqpoint{1.249956in}{0.148611in}}{\pgfqpoint{7.122250in}{3.850000in}}%
\pgfusepath{clip}%
\pgfsetbuttcap%
\pgfsetmiterjoin%
\definecolor{currentfill}{rgb}{0.997078,0.998770,0.745021}%
\pgfsetfillcolor{currentfill}%
\pgfsetfillopacity{0.500000}%
\pgfsetlinewidth{0.000000pt}%
\definecolor{currentstroke}{rgb}{0.000000,0.000000,0.000000}%
\pgfsetstrokecolor{currentstroke}%
\pgfsetstrokeopacity{0.500000}%
\pgfsetdash{}{0pt}%
\pgfpathmoveto{\pgfqpoint{6.800630in}{2.176552in}}%
\pgfpathlineto{\pgfqpoint{7.335635in}{2.176552in}}%
\pgfpathlineto{\pgfqpoint{7.335635in}{1.970670in}}%
\pgfpathlineto{\pgfqpoint{6.800630in}{1.970670in}}%
\pgfpathclose%
\pgfusepath{fill}%
\end{pgfscope}%
\begin{pgfscope}%
\pgfpathrectangle{\pgfqpoint{1.249956in}{0.148611in}}{\pgfqpoint{7.122250in}{3.850000in}}%
\pgfusepath{clip}%
\pgfsetbuttcap%
\pgfsetmiterjoin%
\definecolor{currentfill}{rgb}{0.997078,0.998770,0.745021}%
\pgfsetfillcolor{currentfill}%
\pgfsetlinewidth{0.000000pt}%
\definecolor{currentstroke}{rgb}{0.000000,0.000000,0.000000}%
\pgfsetstrokecolor{currentstroke}%
\pgfsetstrokeopacity{0.000000}%
\pgfsetdash{}{0pt}%
\pgfpathmoveto{\pgfqpoint{2.721219in}{1.764788in}}%
\pgfpathlineto{\pgfqpoint{3.557164in}{1.764788in}}%
\pgfpathlineto{\pgfqpoint{3.557164in}{1.558905in}}%
\pgfpathlineto{\pgfqpoint{2.721219in}{1.558905in}}%
\pgfpathclose%
\pgfusepath{fill}%
\end{pgfscope}%
\begin{pgfscope}%
\pgfpathrectangle{\pgfqpoint{1.249956in}{0.148611in}}{\pgfqpoint{7.122250in}{3.850000in}}%
\pgfusepath{clip}%
\pgfsetbuttcap%
\pgfsetmiterjoin%
\definecolor{currentfill}{rgb}{0.997078,0.998770,0.745021}%
\pgfsetfillcolor{currentfill}%
\pgfsetfillopacity{0.500000}%
\pgfsetlinewidth{0.000000pt}%
\definecolor{currentstroke}{rgb}{0.000000,0.000000,0.000000}%
\pgfsetstrokecolor{currentstroke}%
\pgfsetstrokeopacity{0.500000}%
\pgfsetdash{}{0pt}%
\pgfpathmoveto{\pgfqpoint{7.034695in}{1.353023in}}%
\pgfpathlineto{\pgfqpoint{7.703451in}{1.353023in}}%
\pgfpathlineto{\pgfqpoint{7.703451in}{1.147141in}}%
\pgfpathlineto{\pgfqpoint{7.034695in}{1.147141in}}%
\pgfpathclose%
\pgfusepath{fill}%
\end{pgfscope}%
\begin{pgfscope}%
\pgfpathrectangle{\pgfqpoint{1.249956in}{0.148611in}}{\pgfqpoint{7.122250in}{3.850000in}}%
\pgfusepath{clip}%
\pgfsetbuttcap%
\pgfsetmiterjoin%
\definecolor{currentfill}{rgb}{0.997078,0.998770,0.745021}%
\pgfsetfillcolor{currentfill}%
\pgfsetlinewidth{0.000000pt}%
\definecolor{currentstroke}{rgb}{0.000000,0.000000,0.000000}%
\pgfsetstrokecolor{currentstroke}%
\pgfsetstrokeopacity{0.000000}%
\pgfsetdash{}{0pt}%
\pgfpathmoveto{\pgfqpoint{3.155911in}{0.941258in}}%
\pgfpathlineto{\pgfqpoint{4.259358in}{0.941258in}}%
\pgfpathlineto{\pgfqpoint{4.259358in}{0.735376in}}%
\pgfpathlineto{\pgfqpoint{3.155911in}{0.735376in}}%
\pgfpathclose%
\pgfusepath{fill}%
\end{pgfscope}%
\begin{pgfscope}%
\pgfpathrectangle{\pgfqpoint{1.249956in}{0.148611in}}{\pgfqpoint{7.122250in}{3.850000in}}%
\pgfusepath{clip}%
\pgfsetbuttcap%
\pgfsetmiterjoin%
\definecolor{currentfill}{rgb}{0.997078,0.998770,0.745021}%
\pgfsetfillcolor{currentfill}%
\pgfsetlinewidth{0.000000pt}%
\definecolor{currentstroke}{rgb}{0.000000,0.000000,0.000000}%
\pgfsetstrokecolor{currentstroke}%
\pgfsetstrokeopacity{0.000000}%
\pgfsetdash{}{0pt}%
\pgfpathmoveto{\pgfqpoint{3.189348in}{0.529493in}}%
\pgfpathlineto{\pgfqpoint{3.858104in}{0.529493in}}%
\pgfpathlineto{\pgfqpoint{3.858104in}{0.323611in}}%
\pgfpathlineto{\pgfqpoint{3.189348in}{0.323611in}}%
\pgfpathclose%
\pgfusepath{fill}%
\end{pgfscope}%
\begin{pgfscope}%
\pgfpathrectangle{\pgfqpoint{1.249956in}{0.148611in}}{\pgfqpoint{7.122250in}{3.850000in}}%
\pgfusepath{clip}%
\pgfsetbuttcap%
\pgfsetmiterjoin%
\definecolor{currentfill}{rgb}{0.996540,0.892734,0.569089}%
\pgfsetfillcolor{currentfill}%
\pgfsetfillopacity{0.500000}%
\pgfsetlinewidth{0.000000pt}%
\definecolor{currentstroke}{rgb}{0.000000,0.000000,0.000000}%
\pgfsetstrokecolor{currentstroke}%
\pgfsetstrokeopacity{0.500000}%
\pgfsetdash{}{0pt}%
\pgfpathmoveto{\pgfqpoint{4.861238in}{3.823611in}}%
\pgfpathlineto{\pgfqpoint{5.730621in}{3.823611in}}%
\pgfpathlineto{\pgfqpoint{5.730621in}{3.617729in}}%
\pgfpathlineto{\pgfqpoint{4.861238in}{3.617729in}}%
\pgfpathclose%
\pgfusepath{fill}%
\end{pgfscope}%
\begin{pgfscope}%
\pgfpathrectangle{\pgfqpoint{1.249956in}{0.148611in}}{\pgfqpoint{7.122250in}{3.850000in}}%
\pgfusepath{clip}%
\pgfsetbuttcap%
\pgfsetmiterjoin%
\definecolor{currentfill}{rgb}{0.996540,0.892734,0.569089}%
\pgfsetfillcolor{currentfill}%
\pgfsetfillopacity{0.500000}%
\pgfsetlinewidth{0.000000pt}%
\definecolor{currentstroke}{rgb}{0.000000,0.000000,0.000000}%
\pgfsetstrokecolor{currentstroke}%
\pgfsetstrokeopacity{0.500000}%
\pgfsetdash{}{0pt}%
\pgfpathmoveto{\pgfqpoint{5.162178in}{3.411846in}}%
\pgfpathlineto{\pgfqpoint{6.399377in}{3.411846in}}%
\pgfpathlineto{\pgfqpoint{6.399377in}{3.205964in}}%
\pgfpathlineto{\pgfqpoint{5.162178in}{3.205964in}}%
\pgfpathclose%
\pgfusepath{fill}%
\end{pgfscope}%
\begin{pgfscope}%
\pgfpathrectangle{\pgfqpoint{1.249956in}{0.148611in}}{\pgfqpoint{7.122250in}{3.850000in}}%
\pgfusepath{clip}%
\pgfsetbuttcap%
\pgfsetmiterjoin%
\definecolor{currentfill}{rgb}{0.996540,0.892734,0.569089}%
\pgfsetfillcolor{currentfill}%
\pgfsetlinewidth{0.000000pt}%
\definecolor{currentstroke}{rgb}{0.000000,0.000000,0.000000}%
\pgfsetstrokecolor{currentstroke}%
\pgfsetstrokeopacity{0.000000}%
\pgfsetdash{}{0pt}%
\pgfpathmoveto{\pgfqpoint{3.389975in}{3.000082in}}%
\pgfpathlineto{\pgfqpoint{3.891542in}{3.000082in}}%
\pgfpathlineto{\pgfqpoint{3.891542in}{2.794199in}}%
\pgfpathlineto{\pgfqpoint{3.389975in}{2.794199in}}%
\pgfpathclose%
\pgfusepath{fill}%
\end{pgfscope}%
\begin{pgfscope}%
\pgfpathrectangle{\pgfqpoint{1.249956in}{0.148611in}}{\pgfqpoint{7.122250in}{3.850000in}}%
\pgfusepath{clip}%
\pgfsetbuttcap%
\pgfsetmiterjoin%
\definecolor{currentfill}{rgb}{0.996540,0.892734,0.569089}%
\pgfsetfillcolor{currentfill}%
\pgfsetfillopacity{0.500000}%
\pgfsetlinewidth{0.000000pt}%
\definecolor{currentstroke}{rgb}{0.000000,0.000000,0.000000}%
\pgfsetstrokecolor{currentstroke}%
\pgfsetstrokeopacity{0.500000}%
\pgfsetdash{}{0pt}%
\pgfpathmoveto{\pgfqpoint{6.733755in}{2.588317in}}%
\pgfpathlineto{\pgfqpoint{7.502824in}{2.588317in}}%
\pgfpathlineto{\pgfqpoint{7.502824in}{2.382435in}}%
\pgfpathlineto{\pgfqpoint{6.733755in}{2.382435in}}%
\pgfpathclose%
\pgfusepath{fill}%
\end{pgfscope}%
\begin{pgfscope}%
\pgfpathrectangle{\pgfqpoint{1.249956in}{0.148611in}}{\pgfqpoint{7.122250in}{3.850000in}}%
\pgfusepath{clip}%
\pgfsetbuttcap%
\pgfsetmiterjoin%
\definecolor{currentfill}{rgb}{0.996540,0.892734,0.569089}%
\pgfsetfillcolor{currentfill}%
\pgfsetfillopacity{0.500000}%
\pgfsetlinewidth{0.000000pt}%
\definecolor{currentstroke}{rgb}{0.000000,0.000000,0.000000}%
\pgfsetstrokecolor{currentstroke}%
\pgfsetstrokeopacity{0.500000}%
\pgfsetdash{}{0pt}%
\pgfpathmoveto{\pgfqpoint{7.335635in}{2.176552in}}%
\pgfpathlineto{\pgfqpoint{7.736888in}{2.176552in}}%
\pgfpathlineto{\pgfqpoint{7.736888in}{1.970670in}}%
\pgfpathlineto{\pgfqpoint{7.335635in}{1.970670in}}%
\pgfpathclose%
\pgfusepath{fill}%
\end{pgfscope}%
\begin{pgfscope}%
\pgfpathrectangle{\pgfqpoint{1.249956in}{0.148611in}}{\pgfqpoint{7.122250in}{3.850000in}}%
\pgfusepath{clip}%
\pgfsetbuttcap%
\pgfsetmiterjoin%
\definecolor{currentfill}{rgb}{0.996540,0.892734,0.569089}%
\pgfsetfillcolor{currentfill}%
\pgfsetlinewidth{0.000000pt}%
\definecolor{currentstroke}{rgb}{0.000000,0.000000,0.000000}%
\pgfsetstrokecolor{currentstroke}%
\pgfsetstrokeopacity{0.000000}%
\pgfsetdash{}{0pt}%
\pgfpathmoveto{\pgfqpoint{3.557164in}{1.764788in}}%
\pgfpathlineto{\pgfqpoint{4.426547in}{1.764788in}}%
\pgfpathlineto{\pgfqpoint{4.426547in}{1.558905in}}%
\pgfpathlineto{\pgfqpoint{3.557164in}{1.558905in}}%
\pgfpathclose%
\pgfusepath{fill}%
\end{pgfscope}%
\begin{pgfscope}%
\pgfpathrectangle{\pgfqpoint{1.249956in}{0.148611in}}{\pgfqpoint{7.122250in}{3.850000in}}%
\pgfusepath{clip}%
\pgfsetbuttcap%
\pgfsetmiterjoin%
\definecolor{currentfill}{rgb}{0.996540,0.892734,0.569089}%
\pgfsetfillcolor{currentfill}%
\pgfsetfillopacity{0.500000}%
\pgfsetlinewidth{0.000000pt}%
\definecolor{currentstroke}{rgb}{0.000000,0.000000,0.000000}%
\pgfsetstrokecolor{currentstroke}%
\pgfsetstrokeopacity{0.500000}%
\pgfsetdash{}{0pt}%
\pgfpathmoveto{\pgfqpoint{7.703451in}{1.353023in}}%
\pgfpathlineto{\pgfqpoint{8.071266in}{1.353023in}}%
\pgfpathlineto{\pgfqpoint{8.071266in}{1.147141in}}%
\pgfpathlineto{\pgfqpoint{7.703451in}{1.147141in}}%
\pgfpathclose%
\pgfusepath{fill}%
\end{pgfscope}%
\begin{pgfscope}%
\pgfpathrectangle{\pgfqpoint{1.249956in}{0.148611in}}{\pgfqpoint{7.122250in}{3.850000in}}%
\pgfusepath{clip}%
\pgfsetbuttcap%
\pgfsetmiterjoin%
\definecolor{currentfill}{rgb}{0.996540,0.892734,0.569089}%
\pgfsetfillcolor{currentfill}%
\pgfsetlinewidth{0.000000pt}%
\definecolor{currentstroke}{rgb}{0.000000,0.000000,0.000000}%
\pgfsetstrokecolor{currentstroke}%
\pgfsetstrokeopacity{0.000000}%
\pgfsetdash{}{0pt}%
\pgfpathmoveto{\pgfqpoint{4.259358in}{0.941258in}}%
\pgfpathlineto{\pgfqpoint{5.295929in}{0.941258in}}%
\pgfpathlineto{\pgfqpoint{5.295929in}{0.735376in}}%
\pgfpathlineto{\pgfqpoint{4.259358in}{0.735376in}}%
\pgfpathclose%
\pgfusepath{fill}%
\end{pgfscope}%
\begin{pgfscope}%
\pgfpathrectangle{\pgfqpoint{1.249956in}{0.148611in}}{\pgfqpoint{7.122250in}{3.850000in}}%
\pgfusepath{clip}%
\pgfsetbuttcap%
\pgfsetmiterjoin%
\definecolor{currentfill}{rgb}{0.996540,0.892734,0.569089}%
\pgfsetfillcolor{currentfill}%
\pgfsetlinewidth{0.000000pt}%
\definecolor{currentstroke}{rgb}{0.000000,0.000000,0.000000}%
\pgfsetstrokecolor{currentstroke}%
\pgfsetstrokeopacity{0.000000}%
\pgfsetdash{}{0pt}%
\pgfpathmoveto{\pgfqpoint{3.858104in}{0.529493in}}%
\pgfpathlineto{\pgfqpoint{4.928114in}{0.529493in}}%
\pgfpathlineto{\pgfqpoint{4.928114in}{0.323611in}}%
\pgfpathlineto{\pgfqpoint{3.858104in}{0.323611in}}%
\pgfpathclose%
\pgfusepath{fill}%
\end{pgfscope}%
\begin{pgfscope}%
\pgfpathrectangle{\pgfqpoint{1.249956in}{0.148611in}}{\pgfqpoint{7.122250in}{3.850000in}}%
\pgfusepath{clip}%
\pgfsetbuttcap%
\pgfsetmiterjoin%
\definecolor{currentfill}{rgb}{0.993156,0.732334,0.422376}%
\pgfsetfillcolor{currentfill}%
\pgfsetfillopacity{0.500000}%
\pgfsetlinewidth{0.000000pt}%
\definecolor{currentstroke}{rgb}{0.000000,0.000000,0.000000}%
\pgfsetstrokecolor{currentstroke}%
\pgfsetstrokeopacity{0.500000}%
\pgfsetdash{}{0pt}%
\pgfpathmoveto{\pgfqpoint{5.730621in}{3.823611in}}%
\pgfpathlineto{\pgfqpoint{6.499690in}{3.823611in}}%
\pgfpathlineto{\pgfqpoint{6.499690in}{3.617729in}}%
\pgfpathlineto{\pgfqpoint{5.730621in}{3.617729in}}%
\pgfpathclose%
\pgfusepath{fill}%
\end{pgfscope}%
\begin{pgfscope}%
\pgfpathrectangle{\pgfqpoint{1.249956in}{0.148611in}}{\pgfqpoint{7.122250in}{3.850000in}}%
\pgfusepath{clip}%
\pgfsetbuttcap%
\pgfsetmiterjoin%
\definecolor{currentfill}{rgb}{0.993156,0.732334,0.422376}%
\pgfsetfillcolor{currentfill}%
\pgfsetfillopacity{0.500000}%
\pgfsetlinewidth{0.000000pt}%
\definecolor{currentstroke}{rgb}{0.000000,0.000000,0.000000}%
\pgfsetstrokecolor{currentstroke}%
\pgfsetstrokeopacity{0.500000}%
\pgfsetdash{}{0pt}%
\pgfpathmoveto{\pgfqpoint{6.399377in}{3.411846in}}%
\pgfpathlineto{\pgfqpoint{7.502824in}{3.411846in}}%
\pgfpathlineto{\pgfqpoint{7.502824in}{3.205964in}}%
\pgfpathlineto{\pgfqpoint{6.399377in}{3.205964in}}%
\pgfpathclose%
\pgfusepath{fill}%
\end{pgfscope}%
\begin{pgfscope}%
\pgfpathrectangle{\pgfqpoint{1.249956in}{0.148611in}}{\pgfqpoint{7.122250in}{3.850000in}}%
\pgfusepath{clip}%
\pgfsetbuttcap%
\pgfsetmiterjoin%
\definecolor{currentfill}{rgb}{0.993156,0.732334,0.422376}%
\pgfsetfillcolor{currentfill}%
\pgfsetlinewidth{0.000000pt}%
\definecolor{currentstroke}{rgb}{0.000000,0.000000,0.000000}%
\pgfsetstrokecolor{currentstroke}%
\pgfsetstrokeopacity{0.000000}%
\pgfsetdash{}{0pt}%
\pgfpathmoveto{\pgfqpoint{3.891542in}{3.000082in}}%
\pgfpathlineto{\pgfqpoint{5.295929in}{3.000082in}}%
\pgfpathlineto{\pgfqpoint{5.295929in}{2.794199in}}%
\pgfpathlineto{\pgfqpoint{3.891542in}{2.794199in}}%
\pgfpathclose%
\pgfusepath{fill}%
\end{pgfscope}%
\begin{pgfscope}%
\pgfpathrectangle{\pgfqpoint{1.249956in}{0.148611in}}{\pgfqpoint{7.122250in}{3.850000in}}%
\pgfusepath{clip}%
\pgfsetbuttcap%
\pgfsetmiterjoin%
\definecolor{currentfill}{rgb}{0.993156,0.732334,0.422376}%
\pgfsetfillcolor{currentfill}%
\pgfsetfillopacity{0.500000}%
\pgfsetlinewidth{0.000000pt}%
\definecolor{currentstroke}{rgb}{0.000000,0.000000,0.000000}%
\pgfsetstrokecolor{currentstroke}%
\pgfsetstrokeopacity{0.500000}%
\pgfsetdash{}{0pt}%
\pgfpathmoveto{\pgfqpoint{7.502824in}{2.588317in}}%
\pgfpathlineto{\pgfqpoint{8.037828in}{2.588317in}}%
\pgfpathlineto{\pgfqpoint{8.037828in}{2.382435in}}%
\pgfpathlineto{\pgfqpoint{7.502824in}{2.382435in}}%
\pgfpathclose%
\pgfusepath{fill}%
\end{pgfscope}%
\begin{pgfscope}%
\pgfpathrectangle{\pgfqpoint{1.249956in}{0.148611in}}{\pgfqpoint{7.122250in}{3.850000in}}%
\pgfusepath{clip}%
\pgfsetbuttcap%
\pgfsetmiterjoin%
\definecolor{currentfill}{rgb}{0.993156,0.732334,0.422376}%
\pgfsetfillcolor{currentfill}%
\pgfsetfillopacity{0.500000}%
\pgfsetlinewidth{0.000000pt}%
\definecolor{currentstroke}{rgb}{0.000000,0.000000,0.000000}%
\pgfsetstrokecolor{currentstroke}%
\pgfsetstrokeopacity{0.500000}%
\pgfsetdash{}{0pt}%
\pgfpathmoveto{\pgfqpoint{7.736888in}{2.176552in}}%
\pgfpathlineto{\pgfqpoint{8.037828in}{2.176552in}}%
\pgfpathlineto{\pgfqpoint{8.037828in}{1.970670in}}%
\pgfpathlineto{\pgfqpoint{7.736888in}{1.970670in}}%
\pgfpathclose%
\pgfusepath{fill}%
\end{pgfscope}%
\begin{pgfscope}%
\pgfpathrectangle{\pgfqpoint{1.249956in}{0.148611in}}{\pgfqpoint{7.122250in}{3.850000in}}%
\pgfusepath{clip}%
\pgfsetbuttcap%
\pgfsetmiterjoin%
\definecolor{currentfill}{rgb}{0.993156,0.732334,0.422376}%
\pgfsetfillcolor{currentfill}%
\pgfsetlinewidth{0.000000pt}%
\definecolor{currentstroke}{rgb}{0.000000,0.000000,0.000000}%
\pgfsetstrokecolor{currentstroke}%
\pgfsetstrokeopacity{0.000000}%
\pgfsetdash{}{0pt}%
\pgfpathmoveto{\pgfqpoint{4.426547in}{1.764788in}}%
\pgfpathlineto{\pgfqpoint{5.496556in}{1.764788in}}%
\pgfpathlineto{\pgfqpoint{5.496556in}{1.558905in}}%
\pgfpathlineto{\pgfqpoint{4.426547in}{1.558905in}}%
\pgfpathclose%
\pgfusepath{fill}%
\end{pgfscope}%
\begin{pgfscope}%
\pgfpathrectangle{\pgfqpoint{1.249956in}{0.148611in}}{\pgfqpoint{7.122250in}{3.850000in}}%
\pgfusepath{clip}%
\pgfsetbuttcap%
\pgfsetmiterjoin%
\definecolor{currentfill}{rgb}{0.993156,0.732334,0.422376}%
\pgfsetfillcolor{currentfill}%
\pgfsetfillopacity{0.500000}%
\pgfsetlinewidth{0.000000pt}%
\definecolor{currentstroke}{rgb}{0.000000,0.000000,0.000000}%
\pgfsetstrokecolor{currentstroke}%
\pgfsetstrokeopacity{0.500000}%
\pgfsetdash{}{0pt}%
\pgfpathmoveto{\pgfqpoint{8.071266in}{1.353023in}}%
\pgfpathlineto{\pgfqpoint{8.171580in}{1.353023in}}%
\pgfpathlineto{\pgfqpoint{8.171580in}{1.147141in}}%
\pgfpathlineto{\pgfqpoint{8.071266in}{1.147141in}}%
\pgfpathclose%
\pgfusepath{fill}%
\end{pgfscope}%
\begin{pgfscope}%
\pgfpathrectangle{\pgfqpoint{1.249956in}{0.148611in}}{\pgfqpoint{7.122250in}{3.850000in}}%
\pgfusepath{clip}%
\pgfsetbuttcap%
\pgfsetmiterjoin%
\definecolor{currentfill}{rgb}{0.993156,0.732334,0.422376}%
\pgfsetfillcolor{currentfill}%
\pgfsetlinewidth{0.000000pt}%
\definecolor{currentstroke}{rgb}{0.000000,0.000000,0.000000}%
\pgfsetstrokecolor{currentstroke}%
\pgfsetstrokeopacity{0.000000}%
\pgfsetdash{}{0pt}%
\pgfpathmoveto{\pgfqpoint{5.295929in}{0.941258in}}%
\pgfpathlineto{\pgfqpoint{6.299063in}{0.941258in}}%
\pgfpathlineto{\pgfqpoint{6.299063in}{0.735376in}}%
\pgfpathlineto{\pgfqpoint{5.295929in}{0.735376in}}%
\pgfpathclose%
\pgfusepath{fill}%
\end{pgfscope}%
\begin{pgfscope}%
\pgfpathrectangle{\pgfqpoint{1.249956in}{0.148611in}}{\pgfqpoint{7.122250in}{3.850000in}}%
\pgfusepath{clip}%
\pgfsetbuttcap%
\pgfsetmiterjoin%
\definecolor{currentfill}{rgb}{0.993156,0.732334,0.422376}%
\pgfsetfillcolor{currentfill}%
\pgfsetlinewidth{0.000000pt}%
\definecolor{currentstroke}{rgb}{0.000000,0.000000,0.000000}%
\pgfsetstrokecolor{currentstroke}%
\pgfsetstrokeopacity{0.000000}%
\pgfsetdash{}{0pt}%
\pgfpathmoveto{\pgfqpoint{4.928114in}{0.529493in}}%
\pgfpathlineto{\pgfqpoint{5.764059in}{0.529493in}}%
\pgfpathlineto{\pgfqpoint{5.764059in}{0.323611in}}%
\pgfpathlineto{\pgfqpoint{4.928114in}{0.323611in}}%
\pgfpathclose%
\pgfusepath{fill}%
\end{pgfscope}%
\begin{pgfscope}%
\pgfpathrectangle{\pgfqpoint{1.249956in}{0.148611in}}{\pgfqpoint{7.122250in}{3.850000in}}%
\pgfusepath{clip}%
\pgfsetbuttcap%
\pgfsetmiterjoin%
\definecolor{currentfill}{rgb}{0.969319,0.517416,0.304268}%
\pgfsetfillcolor{currentfill}%
\pgfsetfillopacity{0.500000}%
\pgfsetlinewidth{0.000000pt}%
\definecolor{currentstroke}{rgb}{0.000000,0.000000,0.000000}%
\pgfsetstrokecolor{currentstroke}%
\pgfsetstrokeopacity{0.500000}%
\pgfsetdash{}{0pt}%
\pgfpathmoveto{\pgfqpoint{6.499690in}{3.823611in}}%
\pgfpathlineto{\pgfqpoint{7.201884in}{3.823611in}}%
\pgfpathlineto{\pgfqpoint{7.201884in}{3.617729in}}%
\pgfpathlineto{\pgfqpoint{6.499690in}{3.617729in}}%
\pgfpathclose%
\pgfusepath{fill}%
\end{pgfscope}%
\begin{pgfscope}%
\pgfpathrectangle{\pgfqpoint{1.249956in}{0.148611in}}{\pgfqpoint{7.122250in}{3.850000in}}%
\pgfusepath{clip}%
\pgfsetbuttcap%
\pgfsetmiterjoin%
\definecolor{currentfill}{rgb}{0.969319,0.517416,0.304268}%
\pgfsetfillcolor{currentfill}%
\pgfsetfillopacity{0.500000}%
\pgfsetlinewidth{0.000000pt}%
\definecolor{currentstroke}{rgb}{0.000000,0.000000,0.000000}%
\pgfsetstrokecolor{currentstroke}%
\pgfsetstrokeopacity{0.500000}%
\pgfsetdash{}{0pt}%
\pgfpathmoveto{\pgfqpoint{7.502824in}{3.411846in}}%
\pgfpathlineto{\pgfqpoint{7.970953in}{3.411846in}}%
\pgfpathlineto{\pgfqpoint{7.970953in}{3.205964in}}%
\pgfpathlineto{\pgfqpoint{7.502824in}{3.205964in}}%
\pgfpathclose%
\pgfusepath{fill}%
\end{pgfscope}%
\begin{pgfscope}%
\pgfpathrectangle{\pgfqpoint{1.249956in}{0.148611in}}{\pgfqpoint{7.122250in}{3.850000in}}%
\pgfusepath{clip}%
\pgfsetbuttcap%
\pgfsetmiterjoin%
\definecolor{currentfill}{rgb}{0.969319,0.517416,0.304268}%
\pgfsetfillcolor{currentfill}%
\pgfsetlinewidth{0.000000pt}%
\definecolor{currentstroke}{rgb}{0.000000,0.000000,0.000000}%
\pgfsetstrokecolor{currentstroke}%
\pgfsetstrokeopacity{0.000000}%
\pgfsetdash{}{0pt}%
\pgfpathmoveto{\pgfqpoint{5.295929in}{3.000082in}}%
\pgfpathlineto{\pgfqpoint{6.867506in}{3.000082in}}%
\pgfpathlineto{\pgfqpoint{6.867506in}{2.794199in}}%
\pgfpathlineto{\pgfqpoint{5.295929in}{2.794199in}}%
\pgfpathclose%
\pgfusepath{fill}%
\end{pgfscope}%
\begin{pgfscope}%
\pgfpathrectangle{\pgfqpoint{1.249956in}{0.148611in}}{\pgfqpoint{7.122250in}{3.850000in}}%
\pgfusepath{clip}%
\pgfsetbuttcap%
\pgfsetmiterjoin%
\definecolor{currentfill}{rgb}{0.969319,0.517416,0.304268}%
\pgfsetfillcolor{currentfill}%
\pgfsetfillopacity{0.500000}%
\pgfsetlinewidth{0.000000pt}%
\definecolor{currentstroke}{rgb}{0.000000,0.000000,0.000000}%
\pgfsetstrokecolor{currentstroke}%
\pgfsetstrokeopacity{0.500000}%
\pgfsetdash{}{0pt}%
\pgfpathmoveto{\pgfqpoint{8.037828in}{2.588317in}}%
\pgfpathlineto{\pgfqpoint{8.338769in}{2.588317in}}%
\pgfpathlineto{\pgfqpoint{8.338769in}{2.382435in}}%
\pgfpathlineto{\pgfqpoint{8.037828in}{2.382435in}}%
\pgfpathclose%
\pgfusepath{fill}%
\end{pgfscope}%
\begin{pgfscope}%
\pgfpathrectangle{\pgfqpoint{1.249956in}{0.148611in}}{\pgfqpoint{7.122250in}{3.850000in}}%
\pgfusepath{clip}%
\pgfsetbuttcap%
\pgfsetmiterjoin%
\definecolor{currentfill}{rgb}{0.969319,0.517416,0.304268}%
\pgfsetfillcolor{currentfill}%
\pgfsetfillopacity{0.500000}%
\pgfsetlinewidth{0.000000pt}%
\definecolor{currentstroke}{rgb}{0.000000,0.000000,0.000000}%
\pgfsetstrokecolor{currentstroke}%
\pgfsetstrokeopacity{0.500000}%
\pgfsetdash{}{0pt}%
\pgfpathmoveto{\pgfqpoint{8.037828in}{2.176552in}}%
\pgfpathlineto{\pgfqpoint{8.104704in}{2.176552in}}%
\pgfpathlineto{\pgfqpoint{8.104704in}{1.970670in}}%
\pgfpathlineto{\pgfqpoint{8.037828in}{1.970670in}}%
\pgfpathclose%
\pgfusepath{fill}%
\end{pgfscope}%
\begin{pgfscope}%
\pgfpathrectangle{\pgfqpoint{1.249956in}{0.148611in}}{\pgfqpoint{7.122250in}{3.850000in}}%
\pgfusepath{clip}%
\pgfsetbuttcap%
\pgfsetmiterjoin%
\definecolor{currentfill}{rgb}{0.969319,0.517416,0.304268}%
\pgfsetfillcolor{currentfill}%
\pgfsetlinewidth{0.000000pt}%
\definecolor{currentstroke}{rgb}{0.000000,0.000000,0.000000}%
\pgfsetstrokecolor{currentstroke}%
\pgfsetstrokeopacity{0.000000}%
\pgfsetdash{}{0pt}%
\pgfpathmoveto{\pgfqpoint{5.496556in}{1.764788in}}%
\pgfpathlineto{\pgfqpoint{7.034695in}{1.764788in}}%
\pgfpathlineto{\pgfqpoint{7.034695in}{1.558905in}}%
\pgfpathlineto{\pgfqpoint{5.496556in}{1.558905in}}%
\pgfpathclose%
\pgfusepath{fill}%
\end{pgfscope}%
\begin{pgfscope}%
\pgfpathrectangle{\pgfqpoint{1.249956in}{0.148611in}}{\pgfqpoint{7.122250in}{3.850000in}}%
\pgfusepath{clip}%
\pgfsetbuttcap%
\pgfsetmiterjoin%
\definecolor{currentfill}{rgb}{0.969319,0.517416,0.304268}%
\pgfsetfillcolor{currentfill}%
\pgfsetfillopacity{0.500000}%
\pgfsetlinewidth{0.000000pt}%
\definecolor{currentstroke}{rgb}{0.000000,0.000000,0.000000}%
\pgfsetstrokecolor{currentstroke}%
\pgfsetstrokeopacity{0.500000}%
\pgfsetdash{}{0pt}%
\pgfpathmoveto{\pgfqpoint{8.171580in}{1.353023in}}%
\pgfpathlineto{\pgfqpoint{8.305331in}{1.353023in}}%
\pgfpathlineto{\pgfqpoint{8.305331in}{1.147141in}}%
\pgfpathlineto{\pgfqpoint{8.171580in}{1.147141in}}%
\pgfpathclose%
\pgfusepath{fill}%
\end{pgfscope}%
\begin{pgfscope}%
\pgfpathrectangle{\pgfqpoint{1.249956in}{0.148611in}}{\pgfqpoint{7.122250in}{3.850000in}}%
\pgfusepath{clip}%
\pgfsetbuttcap%
\pgfsetmiterjoin%
\definecolor{currentfill}{rgb}{0.969319,0.517416,0.304268}%
\pgfsetfillcolor{currentfill}%
\pgfsetlinewidth{0.000000pt}%
\definecolor{currentstroke}{rgb}{0.000000,0.000000,0.000000}%
\pgfsetstrokecolor{currentstroke}%
\pgfsetstrokeopacity{0.000000}%
\pgfsetdash{}{0pt}%
\pgfpathmoveto{\pgfqpoint{6.299063in}{0.941258in}}%
\pgfpathlineto{\pgfqpoint{7.569699in}{0.941258in}}%
\pgfpathlineto{\pgfqpoint{7.569699in}{0.735376in}}%
\pgfpathlineto{\pgfqpoint{6.299063in}{0.735376in}}%
\pgfpathclose%
\pgfusepath{fill}%
\end{pgfscope}%
\begin{pgfscope}%
\pgfpathrectangle{\pgfqpoint{1.249956in}{0.148611in}}{\pgfqpoint{7.122250in}{3.850000in}}%
\pgfusepath{clip}%
\pgfsetbuttcap%
\pgfsetmiterjoin%
\definecolor{currentfill}{rgb}{0.969319,0.517416,0.304268}%
\pgfsetfillcolor{currentfill}%
\pgfsetlinewidth{0.000000pt}%
\definecolor{currentstroke}{rgb}{0.000000,0.000000,0.000000}%
\pgfsetstrokecolor{currentstroke}%
\pgfsetstrokeopacity{0.000000}%
\pgfsetdash{}{0pt}%
\pgfpathmoveto{\pgfqpoint{5.764059in}{0.529493in}}%
\pgfpathlineto{\pgfqpoint{6.834068in}{0.529493in}}%
\pgfpathlineto{\pgfqpoint{6.834068in}{0.323611in}}%
\pgfpathlineto{\pgfqpoint{5.764059in}{0.323611in}}%
\pgfpathclose%
\pgfusepath{fill}%
\end{pgfscope}%
\begin{pgfscope}%
\pgfpathrectangle{\pgfqpoint{1.249956in}{0.148611in}}{\pgfqpoint{7.122250in}{3.850000in}}%
\pgfusepath{clip}%
\pgfsetbuttcap%
\pgfsetmiterjoin%
\definecolor{currentfill}{rgb}{0.898885,0.305498,0.206767}%
\pgfsetfillcolor{currentfill}%
\pgfsetfillopacity{0.500000}%
\pgfsetlinewidth{0.000000pt}%
\definecolor{currentstroke}{rgb}{0.000000,0.000000,0.000000}%
\pgfsetstrokecolor{currentstroke}%
\pgfsetstrokeopacity{0.500000}%
\pgfsetdash{}{0pt}%
\pgfpathmoveto{\pgfqpoint{7.201884in}{3.823611in}}%
\pgfpathlineto{\pgfqpoint{8.372206in}{3.823611in}}%
\pgfpathlineto{\pgfqpoint{8.372206in}{3.617729in}}%
\pgfpathlineto{\pgfqpoint{7.201884in}{3.617729in}}%
\pgfpathclose%
\pgfusepath{fill}%
\end{pgfscope}%
\begin{pgfscope}%
\pgfpathrectangle{\pgfqpoint{1.249956in}{0.148611in}}{\pgfqpoint{7.122250in}{3.850000in}}%
\pgfusepath{clip}%
\pgfsetbuttcap%
\pgfsetmiterjoin%
\definecolor{currentfill}{rgb}{0.898885,0.305498,0.206767}%
\pgfsetfillcolor{currentfill}%
\pgfsetfillopacity{0.500000}%
\pgfsetlinewidth{0.000000pt}%
\definecolor{currentstroke}{rgb}{0.000000,0.000000,0.000000}%
\pgfsetstrokecolor{currentstroke}%
\pgfsetstrokeopacity{0.500000}%
\pgfsetdash{}{0pt}%
\pgfpathmoveto{\pgfqpoint{7.970953in}{3.411846in}}%
\pgfpathlineto{\pgfqpoint{8.372206in}{3.411846in}}%
\pgfpathlineto{\pgfqpoint{8.372206in}{3.205964in}}%
\pgfpathlineto{\pgfqpoint{7.970953in}{3.205964in}}%
\pgfpathclose%
\pgfusepath{fill}%
\end{pgfscope}%
\begin{pgfscope}%
\pgfpathrectangle{\pgfqpoint{1.249956in}{0.148611in}}{\pgfqpoint{7.122250in}{3.850000in}}%
\pgfusepath{clip}%
\pgfsetbuttcap%
\pgfsetmiterjoin%
\definecolor{currentfill}{rgb}{0.898885,0.305498,0.206767}%
\pgfsetfillcolor{currentfill}%
\pgfsetlinewidth{0.000000pt}%
\definecolor{currentstroke}{rgb}{0.000000,0.000000,0.000000}%
\pgfsetstrokecolor{currentstroke}%
\pgfsetstrokeopacity{0.000000}%
\pgfsetdash{}{0pt}%
\pgfpathmoveto{\pgfqpoint{6.867506in}{3.000082in}}%
\pgfpathlineto{\pgfqpoint{8.372206in}{3.000082in}}%
\pgfpathlineto{\pgfqpoint{8.372206in}{2.794199in}}%
\pgfpathlineto{\pgfqpoint{6.867506in}{2.794199in}}%
\pgfpathclose%
\pgfusepath{fill}%
\end{pgfscope}%
\begin{pgfscope}%
\pgfpathrectangle{\pgfqpoint{1.249956in}{0.148611in}}{\pgfqpoint{7.122250in}{3.850000in}}%
\pgfusepath{clip}%
\pgfsetbuttcap%
\pgfsetmiterjoin%
\definecolor{currentfill}{rgb}{0.898885,0.305498,0.206767}%
\pgfsetfillcolor{currentfill}%
\pgfsetfillopacity{0.500000}%
\pgfsetlinewidth{0.000000pt}%
\definecolor{currentstroke}{rgb}{0.000000,0.000000,0.000000}%
\pgfsetstrokecolor{currentstroke}%
\pgfsetstrokeopacity{0.500000}%
\pgfsetdash{}{0pt}%
\pgfpathmoveto{\pgfqpoint{8.338769in}{2.588317in}}%
\pgfpathlineto{\pgfqpoint{8.372206in}{2.588317in}}%
\pgfpathlineto{\pgfqpoint{8.372206in}{2.382435in}}%
\pgfpathlineto{\pgfqpoint{8.338769in}{2.382435in}}%
\pgfpathclose%
\pgfusepath{fill}%
\end{pgfscope}%
\begin{pgfscope}%
\pgfpathrectangle{\pgfqpoint{1.249956in}{0.148611in}}{\pgfqpoint{7.122250in}{3.850000in}}%
\pgfusepath{clip}%
\pgfsetbuttcap%
\pgfsetmiterjoin%
\definecolor{currentfill}{rgb}{0.898885,0.305498,0.206767}%
\pgfsetfillcolor{currentfill}%
\pgfsetfillopacity{0.500000}%
\pgfsetlinewidth{0.000000pt}%
\definecolor{currentstroke}{rgb}{0.000000,0.000000,0.000000}%
\pgfsetstrokecolor{currentstroke}%
\pgfsetstrokeopacity{0.500000}%
\pgfsetdash{}{0pt}%
\pgfpathmoveto{\pgfqpoint{8.104704in}{2.176552in}}%
\pgfpathlineto{\pgfqpoint{8.372206in}{2.176552in}}%
\pgfpathlineto{\pgfqpoint{8.372206in}{1.970670in}}%
\pgfpathlineto{\pgfqpoint{8.104704in}{1.970670in}}%
\pgfpathclose%
\pgfusepath{fill}%
\end{pgfscope}%
\begin{pgfscope}%
\pgfpathrectangle{\pgfqpoint{1.249956in}{0.148611in}}{\pgfqpoint{7.122250in}{3.850000in}}%
\pgfusepath{clip}%
\pgfsetbuttcap%
\pgfsetmiterjoin%
\definecolor{currentfill}{rgb}{0.898885,0.305498,0.206767}%
\pgfsetfillcolor{currentfill}%
\pgfsetlinewidth{0.000000pt}%
\definecolor{currentstroke}{rgb}{0.000000,0.000000,0.000000}%
\pgfsetstrokecolor{currentstroke}%
\pgfsetstrokeopacity{0.000000}%
\pgfsetdash{}{0pt}%
\pgfpathmoveto{\pgfqpoint{7.034695in}{1.764788in}}%
\pgfpathlineto{\pgfqpoint{8.372206in}{1.764788in}}%
\pgfpathlineto{\pgfqpoint{8.372206in}{1.558905in}}%
\pgfpathlineto{\pgfqpoint{7.034695in}{1.558905in}}%
\pgfpathclose%
\pgfusepath{fill}%
\end{pgfscope}%
\begin{pgfscope}%
\pgfpathrectangle{\pgfqpoint{1.249956in}{0.148611in}}{\pgfqpoint{7.122250in}{3.850000in}}%
\pgfusepath{clip}%
\pgfsetbuttcap%
\pgfsetmiterjoin%
\definecolor{currentfill}{rgb}{0.898885,0.305498,0.206767}%
\pgfsetfillcolor{currentfill}%
\pgfsetfillopacity{0.500000}%
\pgfsetlinewidth{0.000000pt}%
\definecolor{currentstroke}{rgb}{0.000000,0.000000,0.000000}%
\pgfsetstrokecolor{currentstroke}%
\pgfsetstrokeopacity{0.500000}%
\pgfsetdash{}{0pt}%
\pgfpathmoveto{\pgfqpoint{8.305331in}{1.353023in}}%
\pgfpathlineto{\pgfqpoint{8.372206in}{1.353023in}}%
\pgfpathlineto{\pgfqpoint{8.372206in}{1.147141in}}%
\pgfpathlineto{\pgfqpoint{8.305331in}{1.147141in}}%
\pgfpathclose%
\pgfusepath{fill}%
\end{pgfscope}%
\begin{pgfscope}%
\pgfpathrectangle{\pgfqpoint{1.249956in}{0.148611in}}{\pgfqpoint{7.122250in}{3.850000in}}%
\pgfusepath{clip}%
\pgfsetbuttcap%
\pgfsetmiterjoin%
\definecolor{currentfill}{rgb}{0.898885,0.305498,0.206767}%
\pgfsetfillcolor{currentfill}%
\pgfsetlinewidth{0.000000pt}%
\definecolor{currentstroke}{rgb}{0.000000,0.000000,0.000000}%
\pgfsetstrokecolor{currentstroke}%
\pgfsetstrokeopacity{0.000000}%
\pgfsetdash{}{0pt}%
\pgfpathmoveto{\pgfqpoint{7.569699in}{0.941258in}}%
\pgfpathlineto{\pgfqpoint{8.372206in}{0.941258in}}%
\pgfpathlineto{\pgfqpoint{8.372206in}{0.735376in}}%
\pgfpathlineto{\pgfqpoint{7.569699in}{0.735376in}}%
\pgfpathclose%
\pgfusepath{fill}%
\end{pgfscope}%
\begin{pgfscope}%
\pgfpathrectangle{\pgfqpoint{1.249956in}{0.148611in}}{\pgfqpoint{7.122250in}{3.850000in}}%
\pgfusepath{clip}%
\pgfsetbuttcap%
\pgfsetmiterjoin%
\definecolor{currentfill}{rgb}{0.898885,0.305498,0.206767}%
\pgfsetfillcolor{currentfill}%
\pgfsetlinewidth{0.000000pt}%
\definecolor{currentstroke}{rgb}{0.000000,0.000000,0.000000}%
\pgfsetstrokecolor{currentstroke}%
\pgfsetstrokeopacity{0.000000}%
\pgfsetdash{}{0pt}%
\pgfpathmoveto{\pgfqpoint{6.834068in}{0.529493in}}%
\pgfpathlineto{\pgfqpoint{8.372206in}{0.529493in}}%
\pgfpathlineto{\pgfqpoint{8.372206in}{0.323611in}}%
\pgfpathlineto{\pgfqpoint{6.834068in}{0.323611in}}%
\pgfpathclose%
\pgfusepath{fill}%
\end{pgfscope}%
\begin{pgfscope}%
\pgfsetbuttcap%
\pgfsetroundjoin%
\definecolor{currentfill}{rgb}{0.000000,0.000000,0.000000}%
\pgfsetfillcolor{currentfill}%
\pgfsetlinewidth{0.803000pt}%
\definecolor{currentstroke}{rgb}{0.000000,0.000000,0.000000}%
\pgfsetstrokecolor{currentstroke}%
\pgfsetdash{}{0pt}%
\pgfsys@defobject{currentmarker}{\pgfqpoint{-0.048611in}{0.000000in}}{\pgfqpoint{-0.000000in}{0.000000in}}{%
\pgfpathmoveto{\pgfqpoint{-0.000000in}{0.000000in}}%
\pgfpathlineto{\pgfqpoint{-0.048611in}{0.000000in}}%
\pgfusepath{stroke,fill}%
}%
\begin{pgfscope}%
\pgfsys@transformshift{1.249956in}{3.720670in}%
\pgfsys@useobject{currentmarker}{}%
\end{pgfscope}%
\end{pgfscope}%
\begin{pgfscope}%
\definecolor{textcolor}{rgb}{0.000000,0.000000,0.000000}%
\pgfsetstrokecolor{textcolor}%
\pgfsetfillcolor{textcolor}%
\pgftext[x=0.482975in, y=3.667908in, left, base]{\color{textcolor}\sffamily\fontsize{10.000000}{12.000000}\selectfont 3DFRONT}%
\end{pgfscope}%
\begin{pgfscope}%
\pgfsetbuttcap%
\pgfsetroundjoin%
\definecolor{currentfill}{rgb}{0.000000,0.000000,0.000000}%
\pgfsetfillcolor{currentfill}%
\pgfsetlinewidth{0.803000pt}%
\definecolor{currentstroke}{rgb}{0.000000,0.000000,0.000000}%
\pgfsetstrokecolor{currentstroke}%
\pgfsetdash{}{0pt}%
\pgfsys@defobject{currentmarker}{\pgfqpoint{-0.048611in}{0.000000in}}{\pgfqpoint{-0.000000in}{0.000000in}}{%
\pgfpathmoveto{\pgfqpoint{-0.000000in}{0.000000in}}%
\pgfpathlineto{\pgfqpoint{-0.048611in}{0.000000in}}%
\pgfusepath{stroke,fill}%
}%
\begin{pgfscope}%
\pgfsys@transformshift{1.249956in}{3.308905in}%
\pgfsys@useobject{currentmarker}{}%
\end{pgfscope}%
\end{pgfscope}%
\begin{pgfscope}%
\definecolor{textcolor}{rgb}{0.000000,0.000000,0.000000}%
\pgfsetstrokecolor{textcolor}%
\pgfsetfillcolor{textcolor}%
\pgftext[x=0.533295in, y=3.256144in, left, base]{\color{textcolor}\sffamily\fontsize{10.000000}{12.000000}\selectfont AI2THOR}%
\end{pgfscope}%
\begin{pgfscope}%
\pgfsetbuttcap%
\pgfsetroundjoin%
\definecolor{currentfill}{rgb}{0.000000,0.000000,0.000000}%
\pgfsetfillcolor{currentfill}%
\pgfsetlinewidth{0.803000pt}%
\definecolor{currentstroke}{rgb}{0.000000,0.000000,0.000000}%
\pgfsetstrokecolor{currentstroke}%
\pgfsetdash{}{0pt}%
\pgfsys@defobject{currentmarker}{\pgfqpoint{-0.048611in}{0.000000in}}{\pgfqpoint{-0.000000in}{0.000000in}}{%
\pgfpathmoveto{\pgfqpoint{-0.000000in}{0.000000in}}%
\pgfpathlineto{\pgfqpoint{-0.048611in}{0.000000in}}%
\pgfusepath{stroke,fill}%
}%
\begin{pgfscope}%
\pgfsys@transformshift{1.249956in}{2.897141in}%
\pgfsys@useobject{currentmarker}{}%
\end{pgfscope}%
\end{pgfscope}%
\begin{pgfscope}%
\definecolor{textcolor}{rgb}{0.000000,0.000000,0.000000}%
\pgfsetstrokecolor{textcolor}%
\pgfsetfillcolor{textcolor}%
\pgftext[x=0.311127in, y=2.844379in, left, base]{\color{textcolor}\sffamily\fontsize{10.000000}{12.000000}\selectfont Blenderproc}%
\end{pgfscope}%
\begin{pgfscope}%
\pgfsetbuttcap%
\pgfsetroundjoin%
\definecolor{currentfill}{rgb}{0.000000,0.000000,0.000000}%
\pgfsetfillcolor{currentfill}%
\pgfsetlinewidth{0.803000pt}%
\definecolor{currentstroke}{rgb}{0.000000,0.000000,0.000000}%
\pgfsetstrokecolor{currentstroke}%
\pgfsetdash{}{0pt}%
\pgfsys@defobject{currentmarker}{\pgfqpoint{-0.048611in}{0.000000in}}{\pgfqpoint{-0.000000in}{0.000000in}}{%
\pgfpathmoveto{\pgfqpoint{-0.000000in}{0.000000in}}%
\pgfpathlineto{\pgfqpoint{-0.048611in}{0.000000in}}%
\pgfusepath{stroke,fill}%
}%
\begin{pgfscope}%
\pgfsys@transformshift{1.249956in}{2.485376in}%
\pgfsys@useobject{currentmarker}{}%
\end{pgfscope}%
\end{pgfscope}%
\begin{pgfscope}%
\definecolor{textcolor}{rgb}{0.000000,0.000000,0.000000}%
\pgfsetstrokecolor{textcolor}%
\pgfsetfillcolor{textcolor}%
\pgftext[x=0.489146in, y=2.432614in, left, base]{\color{textcolor}\sffamily\fontsize{10.000000}{12.000000}\selectfont Hyperism}%
\end{pgfscope}%
\begin{pgfscope}%
\pgfsetbuttcap%
\pgfsetroundjoin%
\definecolor{currentfill}{rgb}{0.000000,0.000000,0.000000}%
\pgfsetfillcolor{currentfill}%
\pgfsetlinewidth{0.803000pt}%
\definecolor{currentstroke}{rgb}{0.000000,0.000000,0.000000}%
\pgfsetstrokecolor{currentstroke}%
\pgfsetdash{}{0pt}%
\pgfsys@defobject{currentmarker}{\pgfqpoint{-0.048611in}{0.000000in}}{\pgfqpoint{-0.000000in}{0.000000in}}{%
\pgfpathmoveto{\pgfqpoint{-0.000000in}{0.000000in}}%
\pgfpathlineto{\pgfqpoint{-0.048611in}{0.000000in}}%
\pgfusepath{stroke,fill}%
}%
\begin{pgfscope}%
\pgfsys@transformshift{1.249956in}{2.073611in}%
\pgfsys@useobject{currentmarker}{}%
\end{pgfscope}%
\end{pgfscope}%
\begin{pgfscope}%
\definecolor{textcolor}{rgb}{0.000000,0.000000,0.000000}%
\pgfsetstrokecolor{textcolor}%
\pgfsetfillcolor{textcolor}%
\pgftext[x=0.402273in, y=2.020850in, left, base]{\color{textcolor}\sffamily\fontsize{10.000000}{12.000000}\selectfont InteriorNet}%
\end{pgfscope}%
\begin{pgfscope}%
\pgfsetbuttcap%
\pgfsetroundjoin%
\definecolor{currentfill}{rgb}{0.000000,0.000000,0.000000}%
\pgfsetfillcolor{currentfill}%
\pgfsetlinewidth{0.803000pt}%
\definecolor{currentstroke}{rgb}{0.000000,0.000000,0.000000}%
\pgfsetstrokecolor{currentstroke}%
\pgfsetdash{}{0pt}%
\pgfsys@defobject{currentmarker}{\pgfqpoint{-0.048611in}{0.000000in}}{\pgfqpoint{-0.000000in}{0.000000in}}{%
\pgfpathmoveto{\pgfqpoint{-0.000000in}{0.000000in}}%
\pgfpathlineto{\pgfqpoint{-0.048611in}{0.000000in}}%
\pgfusepath{stroke,fill}%
}%
\begin{pgfscope}%
\pgfsys@transformshift{1.249956in}{1.661846in}%
\pgfsys@useobject{currentmarker}{}%
\end{pgfscope}%
\end{pgfscope}%
\begin{pgfscope}%
\definecolor{textcolor}{rgb}{0.000000,0.000000,0.000000}%
\pgfsetstrokecolor{textcolor}%
\pgfsetfillcolor{textcolor}%
\pgftext[x=0.313908in, y=1.609085in, left, base]{\color{textcolor}\sffamily\fontsize{10.000000}{12.000000}\selectfont OpenRooms}%
\end{pgfscope}%
\begin{pgfscope}%
\pgfsetbuttcap%
\pgfsetroundjoin%
\definecolor{currentfill}{rgb}{0.000000,0.000000,0.000000}%
\pgfsetfillcolor{currentfill}%
\pgfsetlinewidth{0.803000pt}%
\definecolor{currentstroke}{rgb}{0.000000,0.000000,0.000000}%
\pgfsetstrokecolor{currentstroke}%
\pgfsetdash{}{0pt}%
\pgfsys@defobject{currentmarker}{\pgfqpoint{-0.048611in}{0.000000in}}{\pgfqpoint{-0.000000in}{0.000000in}}{%
\pgfpathmoveto{\pgfqpoint{-0.000000in}{0.000000in}}%
\pgfpathlineto{\pgfqpoint{-0.048611in}{0.000000in}}%
\pgfusepath{stroke,fill}%
}%
\begin{pgfscope}%
\pgfsys@transformshift{1.249956in}{1.250082in}%
\pgfsys@useobject{currentmarker}{}%
\end{pgfscope}%
\end{pgfscope}%
\begin{pgfscope}%
\definecolor{textcolor}{rgb}{0.000000,0.000000,0.000000}%
\pgfsetstrokecolor{textcolor}%
\pgfsetfillcolor{textcolor}%
\pgftext[x=0.755938in, y=1.197320in, left, base]{\color{textcolor}\sffamily\fontsize{10.000000}{12.000000}\selectfont Pix3D}%
\end{pgfscope}%
\begin{pgfscope}%
\pgfsetbuttcap%
\pgfsetroundjoin%
\definecolor{currentfill}{rgb}{0.000000,0.000000,0.000000}%
\pgfsetfillcolor{currentfill}%
\pgfsetlinewidth{0.803000pt}%
\definecolor{currentstroke}{rgb}{0.000000,0.000000,0.000000}%
\pgfsetstrokecolor{currentstroke}%
\pgfsetdash{}{0pt}%
\pgfsys@defobject{currentmarker}{\pgfqpoint{-0.048611in}{0.000000in}}{\pgfqpoint{-0.000000in}{0.000000in}}{%
\pgfpathmoveto{\pgfqpoint{-0.000000in}{0.000000in}}%
\pgfpathlineto{\pgfqpoint{-0.048611in}{0.000000in}}%
\pgfusepath{stroke,fill}%
}%
\begin{pgfscope}%
\pgfsys@transformshift{1.249956in}{0.838317in}%
\pgfsys@useobject{currentmarker}{}%
\end{pgfscope}%
\end{pgfscope}%
\begin{pgfscope}%
\definecolor{textcolor}{rgb}{0.000000,0.000000,0.000000}%
\pgfsetstrokecolor{textcolor}%
\pgfsetfillcolor{textcolor}%
\pgftext[x=0.289968in, y=0.785555in, left, base]{\color{textcolor}\sffamily\fontsize{10.000000}{12.000000}\selectfont S2R:3DFREE}%
\end{pgfscope}%
\begin{pgfscope}%
\pgfsetbuttcap%
\pgfsetroundjoin%
\definecolor{currentfill}{rgb}{0.000000,0.000000,0.000000}%
\pgfsetfillcolor{currentfill}%
\pgfsetlinewidth{0.803000pt}%
\definecolor{currentstroke}{rgb}{0.000000,0.000000,0.000000}%
\pgfsetstrokecolor{currentstroke}%
\pgfsetdash{}{0pt}%
\pgfsys@defobject{currentmarker}{\pgfqpoint{-0.048611in}{0.000000in}}{\pgfqpoint{-0.000000in}{0.000000in}}{%
\pgfpathmoveto{\pgfqpoint{-0.000000in}{0.000000in}}%
\pgfpathlineto{\pgfqpoint{-0.048611in}{0.000000in}}%
\pgfusepath{stroke,fill}%
}%
\begin{pgfscope}%
\pgfsys@transformshift{1.249956in}{0.426552in}%
\pgfsys@useobject{currentmarker}{}%
\end{pgfscope}%
\end{pgfscope}%
\begin{pgfscope}%
\definecolor{textcolor}{rgb}{0.000000,0.000000,0.000000}%
\pgfsetstrokecolor{textcolor}%
\pgfsetfillcolor{textcolor}%
\pgftext[x=0.485484in, y=0.373791in, left, base]{\color{textcolor}\sffamily\fontsize{10.000000}{12.000000}\selectfont SceneNet}%
\end{pgfscope}%
\begin{pgfscope}%
\definecolor{textcolor}{rgb}{0.000000,0.000000,0.000000}%
\pgfsetstrokecolor{textcolor}%
\pgfsetfillcolor{textcolor}%
\pgftext[x=0.234413in,y=2.073611in,,bottom,rotate=90.000000]{\color{textcolor}\sffamily\fontsize{10.000000}{12.000000}\selectfont Datasets}%
\end{pgfscope}%
\begin{pgfscope}%
\pgfpathrectangle{\pgfqpoint{1.249956in}{0.148611in}}{\pgfqpoint{7.122250in}{3.850000in}}%
\pgfusepath{clip}%
\pgfsetbuttcap%
\pgfsetroundjoin%
\pgfsetlinewidth{1.505625pt}%
\definecolor{currentstroke}{rgb}{0.000000,0.000000,0.000000}%
\pgfsetstrokecolor{currentstroke}%
\pgfsetstrokeopacity{0.200000}%
\pgfsetdash{{5.550000pt}{2.400000pt}}{0.000000pt}%
\pgfpathmoveto{\pgfqpoint{4.259358in}{0.148611in}}%
\pgfpathlineto{\pgfqpoint{4.259358in}{3.998611in}}%
\pgfusepath{stroke}%
\end{pgfscope}%
\begin{pgfscope}%
\pgfsetrectcap%
\pgfsetmiterjoin%
\pgfsetlinewidth{0.803000pt}%
\definecolor{currentstroke}{rgb}{0.000000,0.000000,0.000000}%
\pgfsetstrokecolor{currentstroke}%
\pgfsetdash{}{0pt}%
\pgfpathmoveto{\pgfqpoint{1.249956in}{0.148611in}}%
\pgfpathlineto{\pgfqpoint{1.249956in}{3.998611in}}%
\pgfusepath{stroke}%
\end{pgfscope}%
\begin{pgfscope}%
\pgfsetrectcap%
\pgfsetmiterjoin%
\pgfsetlinewidth{0.803000pt}%
\definecolor{currentstroke}{rgb}{0.000000,0.000000,0.000000}%
\pgfsetstrokecolor{currentstroke}%
\pgfsetdash{}{0pt}%
\pgfpathmoveto{\pgfqpoint{8.372206in}{0.148611in}}%
\pgfpathlineto{\pgfqpoint{8.372206in}{3.998611in}}%
\pgfusepath{stroke}%
\end{pgfscope}%
\begin{pgfscope}%
\pgfsetrectcap%
\pgfsetmiterjoin%
\pgfsetlinewidth{0.803000pt}%
\definecolor{currentstroke}{rgb}{0.000000,0.000000,0.000000}%
\pgfsetstrokecolor{currentstroke}%
\pgfsetdash{}{0pt}%
\pgfpathmoveto{\pgfqpoint{1.249956in}{0.148611in}}%
\pgfpathlineto{\pgfqpoint{8.372206in}{0.148611in}}%
\pgfusepath{stroke}%
\end{pgfscope}%
\begin{pgfscope}%
\pgfsetrectcap%
\pgfsetmiterjoin%
\pgfsetlinewidth{0.803000pt}%
\definecolor{currentstroke}{rgb}{0.000000,0.000000,0.000000}%
\pgfsetstrokecolor{currentstroke}%
\pgfsetdash{}{0pt}%
\pgfpathmoveto{\pgfqpoint{1.249956in}{3.998611in}}%
\pgfpathlineto{\pgfqpoint{8.372206in}{3.998611in}}%
\pgfusepath{stroke}%
\end{pgfscope}%
\begin{pgfscope}%
\definecolor{textcolor}{rgb}{1.000000,1.000000,1.000000}%
\pgfsetstrokecolor{textcolor}%
\pgfsetfillcolor{textcolor}%
\pgftext[x=1.450583in,y=3.720670in,,]{\color{textcolor}\sffamily\fontsize{10.000000}{12.000000}\selectfont 12}%
\end{pgfscope}%
\begin{pgfscope}%
\definecolor{textcolor}{rgb}{1.000000,1.000000,1.000000}%
\pgfsetstrokecolor{textcolor}%
\pgfsetfillcolor{textcolor}%
\pgftext[x=1.484021in,y=3.308905in,,]{\color{textcolor}\sffamily\fontsize{10.000000}{12.000000}\selectfont 14}%
\end{pgfscope}%
\begin{pgfscope}%
\definecolor{textcolor}{rgb}{1.000000,1.000000,1.000000}%
\pgfsetstrokecolor{textcolor}%
\pgfsetfillcolor{textcolor}%
\pgftext[x=1.383708in,y=2.897141in,,]{\color{textcolor}\sffamily\fontsize{10.000000}{12.000000}\selectfont 8}%
\end{pgfscope}%
\begin{pgfscope}%
\definecolor{textcolor}{rgb}{1.000000,1.000000,1.000000}%
\pgfsetstrokecolor{textcolor}%
\pgfsetfillcolor{textcolor}%
\pgftext[x=1.484021in,y=2.485376in,,]{\color{textcolor}\sffamily\fontsize{10.000000}{12.000000}\selectfont 14}%
\end{pgfscope}%
\begin{pgfscope}%
\definecolor{textcolor}{rgb}{1.000000,1.000000,1.000000}%
\pgfsetstrokecolor{textcolor}%
\pgfsetfillcolor{textcolor}%
\pgftext[x=2.052463in,y=2.073611in,,]{\color{textcolor}\sffamily\fontsize{10.000000}{12.000000}\selectfont 48}%
\end{pgfscope}%
\begin{pgfscope}%
\definecolor{textcolor}{rgb}{1.000000,1.000000,1.000000}%
\pgfsetstrokecolor{textcolor}%
\pgfsetfillcolor{textcolor}%
\pgftext[x=1.333551in,y=1.661846in,,]{\color{textcolor}\sffamily\fontsize{10.000000}{12.000000}\selectfont 5}%
\end{pgfscope}%
\begin{pgfscope}%
\definecolor{textcolor}{rgb}{1.000000,1.000000,1.000000}%
\pgfsetstrokecolor{textcolor}%
\pgfsetfillcolor{textcolor}%
\pgftext[x=2.921846in,y=1.250082in,,]{\color{textcolor}\sffamily\fontsize{10.000000}{12.000000}\selectfont 100}%
\end{pgfscope}%
\begin{pgfscope}%
\definecolor{textcolor}{rgb}{1.000000,1.000000,1.000000}%
\pgfsetstrokecolor{textcolor}%
\pgfsetfillcolor{textcolor}%
\pgftext[x=1.283394in,y=0.838317in,,]{\color{textcolor}\sffamily\fontsize{10.000000}{12.000000}\selectfont 2}%
\end{pgfscope}%
\begin{pgfscope}%
\definecolor{textcolor}{rgb}{1.000000,1.000000,1.000000}%
\pgfsetstrokecolor{textcolor}%
\pgfsetfillcolor{textcolor}%
\pgftext[x=1.417145in,y=0.426552in,,]{\color{textcolor}\sffamily\fontsize{10.000000}{12.000000}\selectfont 10}%
\end{pgfscope}%
\begin{pgfscope}%
\definecolor{textcolor}{rgb}{1.000000,1.000000,1.000000}%
\pgfsetstrokecolor{textcolor}%
\pgfsetfillcolor{textcolor}%
\pgftext[x=1.918712in,y=3.720670in,,]{\color{textcolor}\sffamily\fontsize{10.000000}{12.000000}\selectfont 16}%
\end{pgfscope}%
\begin{pgfscope}%
\definecolor{textcolor}{rgb}{1.000000,1.000000,1.000000}%
\pgfsetstrokecolor{textcolor}%
\pgfsetfillcolor{textcolor}%
\pgftext[x=1.968869in,y=3.308905in,,]{\color{textcolor}\sffamily\fontsize{10.000000}{12.000000}\selectfont 15}%
\end{pgfscope}%
\begin{pgfscope}%
\definecolor{textcolor}{rgb}{1.000000,1.000000,1.000000}%
\pgfsetstrokecolor{textcolor}%
\pgfsetfillcolor{textcolor}%
\pgftext[x=1.701367in,y=2.897141in,,]{\color{textcolor}\sffamily\fontsize{10.000000}{12.000000}\selectfont 11}%
\end{pgfscope}%
\begin{pgfscope}%
\definecolor{textcolor}{rgb}{1.000000,1.000000,1.000000}%
\pgfsetstrokecolor{textcolor}%
\pgfsetfillcolor{textcolor}%
\pgftext[x=2.436998in,y=2.485376in,,]{\color{textcolor}\sffamily\fontsize{10.000000}{12.000000}\selectfont 43}%
\end{pgfscope}%
\begin{pgfscope}%
\definecolor{textcolor}{rgb}{1.000000,1.000000,1.000000}%
\pgfsetstrokecolor{textcolor}%
\pgfsetfillcolor{textcolor}%
\pgftext[x=3.958418in,y=2.073611in,,]{\color{textcolor}\sffamily\fontsize{10.000000}{12.000000}\selectfont 66}%
\end{pgfscope}%
\begin{pgfscope}%
\definecolor{textcolor}{rgb}{1.000000,1.000000,1.000000}%
\pgfsetstrokecolor{textcolor}%
\pgfsetfillcolor{textcolor}%
\pgftext[x=1.517459in,y=1.661846in,,]{\color{textcolor}\sffamily\fontsize{10.000000}{12.000000}\selectfont 6}%
\end{pgfscope}%
\begin{pgfscope}%
\definecolor{textcolor}{rgb}{1.000000,1.000000,1.000000}%
\pgfsetstrokecolor{textcolor}%
\pgfsetfillcolor{textcolor}%
\pgftext[x=5.195616in,y=1.250082in,,]{\color{textcolor}\sffamily\fontsize{10.000000}{12.000000}\selectfont 36}%
\end{pgfscope}%
\begin{pgfscope}%
\definecolor{textcolor}{rgb}{1.000000,1.000000,1.000000}%
\pgfsetstrokecolor{textcolor}%
\pgfsetfillcolor{textcolor}%
\pgftext[x=1.467302in,y=0.838317in,,]{\color{textcolor}\sffamily\fontsize{10.000000}{12.000000}\selectfont 9}%
\end{pgfscope}%
\begin{pgfscope}%
\definecolor{textcolor}{rgb}{1.000000,1.000000,1.000000}%
\pgfsetstrokecolor{textcolor}%
\pgfsetfillcolor{textcolor}%
\pgftext[x=1.768242in,y=0.426552in,,]{\color{textcolor}\sffamily\fontsize{10.000000}{12.000000}\selectfont 11}%
\end{pgfscope}%
\begin{pgfscope}%
\definecolor{textcolor}{rgb}{1.000000,1.000000,1.000000}%
\pgfsetstrokecolor{textcolor}%
\pgfsetfillcolor{textcolor}%
\pgftext[x=2.570749in,y=3.720670in,,]{\color{textcolor}\sffamily\fontsize{10.000000}{12.000000}\selectfont 23}%
\end{pgfscope}%
\begin{pgfscope}%
\definecolor{textcolor}{rgb}{1.000000,1.000000,1.000000}%
\pgfsetstrokecolor{textcolor}%
\pgfsetfillcolor{textcolor}%
\pgftext[x=2.637625in,y=3.308905in,,]{\color{textcolor}\sffamily\fontsize{10.000000}{12.000000}\selectfont 25}%
\end{pgfscope}%
\begin{pgfscope}%
\definecolor{textcolor}{rgb}{1.000000,1.000000,1.000000}%
\pgfsetstrokecolor{textcolor}%
\pgfsetfillcolor{textcolor}%
\pgftext[x=2.119339in,y=2.897141in,,]{\color{textcolor}\sffamily\fontsize{10.000000}{12.000000}\selectfont 14}%
\end{pgfscope}%
\begin{pgfscope}%
\definecolor{textcolor}{rgb}{1.000000,1.000000,1.000000}%
\pgfsetstrokecolor{textcolor}%
\pgfsetfillcolor{textcolor}%
\pgftext[x=3.958418in,y=2.485376in,,]{\color{textcolor}\sffamily\fontsize{10.000000}{12.000000}\selectfont 48}%
\end{pgfscope}%
\begin{pgfscope}%
\definecolor{textcolor}{rgb}{1.000000,1.000000,1.000000}%
\pgfsetstrokecolor{textcolor}%
\pgfsetfillcolor{textcolor}%
\pgftext[x=5.496556in,y=2.073611in,,]{\color{textcolor}\sffamily\fontsize{10.000000}{12.000000}\selectfont 26}%
\end{pgfscope}%
\begin{pgfscope}%
\definecolor{textcolor}{rgb}{1.000000,1.000000,1.000000}%
\pgfsetstrokecolor{textcolor}%
\pgfsetfillcolor{textcolor}%
\pgftext[x=1.835118in,y=1.661846in,,]{\color{textcolor}\sffamily\fontsize{10.000000}{12.000000}\selectfont 13}%
\end{pgfscope}%
\begin{pgfscope}%
\definecolor{textcolor}{rgb}{1.000000,1.000000,1.000000}%
\pgfsetstrokecolor{textcolor}%
\pgfsetfillcolor{textcolor}%
\pgftext[x=6.248907in,y=1.250082in,,]{\color{textcolor}\sffamily\fontsize{10.000000}{12.000000}\selectfont 27}%
\end{pgfscope}%
\begin{pgfscope}%
\definecolor{textcolor}{rgb}{1.000000,1.000000,1.000000}%
\pgfsetstrokecolor{textcolor}%
\pgfsetfillcolor{textcolor}%
\pgftext[x=1.901993in,y=0.838317in,,]{\color{textcolor}\sffamily\fontsize{10.000000}{12.000000}\selectfont 17}%
\end{pgfscope}%
\begin{pgfscope}%
\definecolor{textcolor}{rgb}{1.000000,1.000000,1.000000}%
\pgfsetstrokecolor{textcolor}%
\pgfsetfillcolor{textcolor}%
\pgftext[x=2.286528in,y=0.426552in,,]{\color{textcolor}\sffamily\fontsize{10.000000}{12.000000}\selectfont 20}%
\end{pgfscope}%
\begin{pgfscope}%
\definecolor{textcolor}{rgb}{1.000000,1.000000,1.000000}%
\pgfsetstrokecolor{textcolor}%
\pgfsetfillcolor{textcolor}%
\pgftext[x=3.323100in,y=3.720670in,,]{\color{textcolor}\sffamily\fontsize{10.000000}{12.000000}\selectfont 22}%
\end{pgfscope}%
\begin{pgfscope}%
\definecolor{textcolor}{rgb}{1.000000,1.000000,1.000000}%
\pgfsetstrokecolor{textcolor}%
\pgfsetfillcolor{textcolor}%
\pgftext[x=3.690915in,y=3.308905in,,]{\color{textcolor}\sffamily\fontsize{10.000000}{12.000000}\selectfont 38}%
\end{pgfscope}%
\begin{pgfscope}%
\definecolor{textcolor}{rgb}{1.000000,1.000000,1.000000}%
\pgfsetstrokecolor{textcolor}%
\pgfsetfillcolor{textcolor}%
\pgftext[x=2.554030in,y=2.897141in,,]{\color{textcolor}\sffamily\fontsize{10.000000}{12.000000}\selectfont 12}%
\end{pgfscope}%
\begin{pgfscope}%
\definecolor{textcolor}{rgb}{1.000000,1.000000,1.000000}%
\pgfsetstrokecolor{textcolor}%
\pgfsetfillcolor{textcolor}%
\pgftext[x=5.412962in,y=2.485376in,,]{\color{textcolor}\sffamily\fontsize{10.000000}{12.000000}\selectfont 39}%
\end{pgfscope}%
\begin{pgfscope}%
\definecolor{textcolor}{rgb}{1.000000,1.000000,1.000000}%
\pgfsetstrokecolor{textcolor}%
\pgfsetfillcolor{textcolor}%
\pgftext[x=6.365939in,y=2.073611in,,]{\color{textcolor}\sffamily\fontsize{10.000000}{12.000000}\selectfont 26}%
\end{pgfscope}%
\begin{pgfscope}%
\definecolor{textcolor}{rgb}{1.000000,1.000000,1.000000}%
\pgfsetstrokecolor{textcolor}%
\pgfsetfillcolor{textcolor}%
\pgftext[x=2.386841in,y=1.661846in,,]{\color{textcolor}\sffamily\fontsize{10.000000}{12.000000}\selectfont 20}%
\end{pgfscope}%
\begin{pgfscope}%
\definecolor{textcolor}{rgb}{1.000000,1.000000,1.000000}%
\pgfsetstrokecolor{textcolor}%
\pgfsetfillcolor{textcolor}%
\pgftext[x=6.867506in,y=1.250082in,,]{\color{textcolor}\sffamily\fontsize{10.000000}{12.000000}\selectfont 10}%
\end{pgfscope}%
\begin{pgfscope}%
\definecolor{textcolor}{rgb}{1.000000,1.000000,1.000000}%
\pgfsetstrokecolor{textcolor}%
\pgfsetfillcolor{textcolor}%
\pgftext[x=2.671063in,y=0.838317in,,]{\color{textcolor}\sffamily\fontsize{10.000000}{12.000000}\selectfont 29}%
\end{pgfscope}%
\begin{pgfscope}%
\definecolor{textcolor}{rgb}{1.000000,1.000000,1.000000}%
\pgfsetstrokecolor{textcolor}%
\pgfsetfillcolor{textcolor}%
\pgftext[x=2.905127in,y=0.426552in,,]{\color{textcolor}\sffamily\fontsize{10.000000}{12.000000}\selectfont 17}%
\end{pgfscope}%
\begin{pgfscope}%
\definecolor{textcolor}{rgb}{0.662745,0.662745,0.662745}%
\pgfsetstrokecolor{textcolor}%
\pgfsetfillcolor{textcolor}%
\pgftext[x=4.276077in,y=3.720670in,,]{\color{textcolor}\sffamily\fontsize{10.000000}{12.000000}\selectfont 35}%
\end{pgfscope}%
\begin{pgfscope}%
\definecolor{textcolor}{rgb}{0.662745,0.662745,0.662745}%
\pgfsetstrokecolor{textcolor}%
\pgfsetfillcolor{textcolor}%
\pgftext[x=4.744206in,y=3.308905in,,]{\color{textcolor}\sffamily\fontsize{10.000000}{12.000000}\selectfont 25}%
\end{pgfscope}%
\begin{pgfscope}%
\definecolor{textcolor}{rgb}{0.662745,0.662745,0.662745}%
\pgfsetstrokecolor{textcolor}%
\pgfsetfillcolor{textcolor}%
\pgftext[x=3.072316in,y=2.897141in,,]{\color{textcolor}\sffamily\fontsize{10.000000}{12.000000}\selectfont 19}%
\end{pgfscope}%
\begin{pgfscope}%
\definecolor{textcolor}{rgb}{0.662745,0.662745,0.662745}%
\pgfsetstrokecolor{textcolor}%
\pgfsetfillcolor{textcolor}%
\pgftext[x=6.399377in,y=2.485376in,,]{\color{textcolor}\sffamily\fontsize{10.000000}{12.000000}\selectfont 20}%
\end{pgfscope}%
\begin{pgfscope}%
\definecolor{textcolor}{rgb}{0.662745,0.662745,0.662745}%
\pgfsetstrokecolor{textcolor}%
\pgfsetfillcolor{textcolor}%
\pgftext[x=7.068132in,y=2.073611in,,]{\color{textcolor}\sffamily\fontsize{10.000000}{12.000000}\selectfont 16}%
\end{pgfscope}%
\begin{pgfscope}%
\definecolor{textcolor}{rgb}{0.662745,0.662745,0.662745}%
\pgfsetstrokecolor{textcolor}%
\pgfsetfillcolor{textcolor}%
\pgftext[x=3.139192in,y=1.661846in,,]{\color{textcolor}\sffamily\fontsize{10.000000}{12.000000}\selectfont 25}%
\end{pgfscope}%
\begin{pgfscope}%
\definecolor{textcolor}{rgb}{0.662745,0.662745,0.662745}%
\pgfsetstrokecolor{textcolor}%
\pgfsetfillcolor{textcolor}%
\pgftext[x=7.369073in,y=1.250082in,,]{\color{textcolor}\sffamily\fontsize{10.000000}{12.000000}\selectfont 20}%
\end{pgfscope}%
\begin{pgfscope}%
\definecolor{textcolor}{rgb}{0.662745,0.662745,0.662745}%
\pgfsetstrokecolor{textcolor}%
\pgfsetfillcolor{textcolor}%
\pgftext[x=3.707634in,y=0.838317in,,]{\color{textcolor}\sffamily\fontsize{10.000000}{12.000000}\selectfont 33}%
\end{pgfscope}%
\begin{pgfscope}%
\definecolor{textcolor}{rgb}{0.662745,0.662745,0.662745}%
\pgfsetstrokecolor{textcolor}%
\pgfsetfillcolor{textcolor}%
\pgftext[x=3.523726in,y=0.426552in,,]{\color{textcolor}\sffamily\fontsize{10.000000}{12.000000}\selectfont 20}%
\end{pgfscope}%
\begin{pgfscope}%
\definecolor{textcolor}{rgb}{0.662745,0.662745,0.662745}%
\pgfsetstrokecolor{textcolor}%
\pgfsetfillcolor{textcolor}%
\pgftext[x=5.295929in,y=3.720670in,,]{\color{textcolor}\sffamily\fontsize{10.000000}{12.000000}\selectfont 26}%
\end{pgfscope}%
\begin{pgfscope}%
\definecolor{textcolor}{rgb}{0.662745,0.662745,0.662745}%
\pgfsetstrokecolor{textcolor}%
\pgfsetfillcolor{textcolor}%
\pgftext[x=5.780777in,y=3.308905in,,]{\color{textcolor}\sffamily\fontsize{10.000000}{12.000000}\selectfont 37}%
\end{pgfscope}%
\begin{pgfscope}%
\definecolor{textcolor}{rgb}{0.662745,0.662745,0.662745}%
\pgfsetstrokecolor{textcolor}%
\pgfsetfillcolor{textcolor}%
\pgftext[x=3.640759in,y=2.897141in,,]{\color{textcolor}\sffamily\fontsize{10.000000}{12.000000}\selectfont 15}%
\end{pgfscope}%
\begin{pgfscope}%
\definecolor{textcolor}{rgb}{0.662745,0.662745,0.662745}%
\pgfsetstrokecolor{textcolor}%
\pgfsetfillcolor{textcolor}%
\pgftext[x=7.118289in,y=2.485376in,,]{\color{textcolor}\sffamily\fontsize{10.000000}{12.000000}\selectfont 23}%
\end{pgfscope}%
\begin{pgfscope}%
\definecolor{textcolor}{rgb}{0.662745,0.662745,0.662745}%
\pgfsetstrokecolor{textcolor}%
\pgfsetfillcolor{textcolor}%
\pgftext[x=7.536262in,y=2.073611in,,]{\color{textcolor}\sffamily\fontsize{10.000000}{12.000000}\selectfont 12}%
\end{pgfscope}%
\begin{pgfscope}%
\definecolor{textcolor}{rgb}{0.662745,0.662745,0.662745}%
\pgfsetstrokecolor{textcolor}%
\pgfsetfillcolor{textcolor}%
\pgftext[x=3.991855in,y=1.661846in,,]{\color{textcolor}\sffamily\fontsize{10.000000}{12.000000}\selectfont 26}%
\end{pgfscope}%
\begin{pgfscope}%
\definecolor{textcolor}{rgb}{0.662745,0.662745,0.662745}%
\pgfsetstrokecolor{textcolor}%
\pgfsetfillcolor{textcolor}%
\pgftext[x=7.887358in,y=1.250082in,,]{\color{textcolor}\sffamily\fontsize{10.000000}{12.000000}\selectfont 11}%
\end{pgfscope}%
\begin{pgfscope}%
\definecolor{textcolor}{rgb}{0.662745,0.662745,0.662745}%
\pgfsetstrokecolor{textcolor}%
\pgfsetfillcolor{textcolor}%
\pgftext[x=4.777644in,y=0.838317in,,]{\color{textcolor}\sffamily\fontsize{10.000000}{12.000000}\selectfont 31}%
\end{pgfscope}%
\begin{pgfscope}%
\definecolor{textcolor}{rgb}{0.662745,0.662745,0.662745}%
\pgfsetstrokecolor{textcolor}%
\pgfsetfillcolor{textcolor}%
\pgftext[x=4.393109in,y=0.426552in,,]{\color{textcolor}\sffamily\fontsize{10.000000}{12.000000}\selectfont 32}%
\end{pgfscope}%
\begin{pgfscope}%
\definecolor{textcolor}{rgb}{1.000000,1.000000,1.000000}%
\pgfsetstrokecolor{textcolor}%
\pgfsetfillcolor{textcolor}%
\pgftext[x=6.115155in,y=3.720670in,,]{\color{textcolor}\sffamily\fontsize{10.000000}{12.000000}\selectfont 23}%
\end{pgfscope}%
\begin{pgfscope}%
\definecolor{textcolor}{rgb}{1.000000,1.000000,1.000000}%
\pgfsetstrokecolor{textcolor}%
\pgfsetfillcolor{textcolor}%
\pgftext[x=6.951100in,y=3.308905in,,]{\color{textcolor}\sffamily\fontsize{10.000000}{12.000000}\selectfont 33}%
\end{pgfscope}%
\begin{pgfscope}%
\definecolor{textcolor}{rgb}{1.000000,1.000000,1.000000}%
\pgfsetstrokecolor{textcolor}%
\pgfsetfillcolor{textcolor}%
\pgftext[x=4.593736in,y=2.897141in,,]{\color{textcolor}\sffamily\fontsize{10.000000}{12.000000}\selectfont 42}%
\end{pgfscope}%
\begin{pgfscope}%
\definecolor{textcolor}{rgb}{1.000000,1.000000,1.000000}%
\pgfsetstrokecolor{textcolor}%
\pgfsetfillcolor{textcolor}%
\pgftext[x=7.770326in,y=2.485376in,,]{\color{textcolor}\sffamily\fontsize{10.000000}{12.000000}\selectfont 16}%
\end{pgfscope}%
\begin{pgfscope}%
\definecolor{textcolor}{rgb}{1.000000,1.000000,1.000000}%
\pgfsetstrokecolor{textcolor}%
\pgfsetfillcolor{textcolor}%
\pgftext[x=7.887358in,y=2.073611in,,]{\color{textcolor}\sffamily\fontsize{10.000000}{12.000000}\selectfont 9}%
\end{pgfscope}%
\begin{pgfscope}%
\definecolor{textcolor}{rgb}{1.000000,1.000000,1.000000}%
\pgfsetstrokecolor{textcolor}%
\pgfsetfillcolor{textcolor}%
\pgftext[x=4.961551in,y=1.661846in,,]{\color{textcolor}\sffamily\fontsize{10.000000}{12.000000}\selectfont 32}%
\end{pgfscope}%
\begin{pgfscope}%
\definecolor{textcolor}{rgb}{1.000000,1.000000,1.000000}%
\pgfsetstrokecolor{textcolor}%
\pgfsetfillcolor{textcolor}%
\pgftext[x=8.121423in,y=1.250082in,,]{\color{textcolor}\sffamily\fontsize{10.000000}{12.000000}\selectfont 3}%
\end{pgfscope}%
\begin{pgfscope}%
\definecolor{textcolor}{rgb}{1.000000,1.000000,1.000000}%
\pgfsetstrokecolor{textcolor}%
\pgfsetfillcolor{textcolor}%
\pgftext[x=5.797496in,y=0.838317in,,]{\color{textcolor}\sffamily\fontsize{10.000000}{12.000000}\selectfont 30}%
\end{pgfscope}%
\begin{pgfscope}%
\definecolor{textcolor}{rgb}{1.000000,1.000000,1.000000}%
\pgfsetstrokecolor{textcolor}%
\pgfsetfillcolor{textcolor}%
\pgftext[x=5.346086in,y=0.426552in,,]{\color{textcolor}\sffamily\fontsize{10.000000}{12.000000}\selectfont 25}%
\end{pgfscope}%
\begin{pgfscope}%
\definecolor{textcolor}{rgb}{1.000000,1.000000,1.000000}%
\pgfsetstrokecolor{textcolor}%
\pgfsetfillcolor{textcolor}%
\pgftext[x=6.850787in,y=3.720670in,,]{\color{textcolor}\sffamily\fontsize{10.000000}{12.000000}\selectfont 21}%
\end{pgfscope}%
\begin{pgfscope}%
\definecolor{textcolor}{rgb}{1.000000,1.000000,1.000000}%
\pgfsetstrokecolor{textcolor}%
\pgfsetfillcolor{textcolor}%
\pgftext[x=7.736888in,y=3.308905in,,]{\color{textcolor}\sffamily\fontsize{10.000000}{12.000000}\selectfont 14}%
\end{pgfscope}%
\begin{pgfscope}%
\definecolor{textcolor}{rgb}{1.000000,1.000000,1.000000}%
\pgfsetstrokecolor{textcolor}%
\pgfsetfillcolor{textcolor}%
\pgftext[x=6.081718in,y=2.897141in,,]{\color{textcolor}\sffamily\fontsize{10.000000}{12.000000}\selectfont 47}%
\end{pgfscope}%
\begin{pgfscope}%
\definecolor{textcolor}{rgb}{1.000000,1.000000,1.000000}%
\pgfsetstrokecolor{textcolor}%
\pgfsetfillcolor{textcolor}%
\pgftext[x=8.188299in,y=2.485376in,,]{\color{textcolor}\sffamily\fontsize{10.000000}{12.000000}\selectfont 9}%
\end{pgfscope}%
\begin{pgfscope}%
\definecolor{textcolor}{rgb}{1.000000,1.000000,1.000000}%
\pgfsetstrokecolor{textcolor}%
\pgfsetfillcolor{textcolor}%
\pgftext[x=8.071266in,y=2.073611in,,]{\color{textcolor}\sffamily\fontsize{10.000000}{12.000000}\selectfont 2}%
\end{pgfscope}%
\begin{pgfscope}%
\definecolor{textcolor}{rgb}{1.000000,1.000000,1.000000}%
\pgfsetstrokecolor{textcolor}%
\pgfsetfillcolor{textcolor}%
\pgftext[x=6.265625in,y=1.661846in,,]{\color{textcolor}\sffamily\fontsize{10.000000}{12.000000}\selectfont 46}%
\end{pgfscope}%
\begin{pgfscope}%
\definecolor{textcolor}{rgb}{1.000000,1.000000,1.000000}%
\pgfsetstrokecolor{textcolor}%
\pgfsetfillcolor{textcolor}%
\pgftext[x=8.238455in,y=1.250082in,,]{\color{textcolor}\sffamily\fontsize{10.000000}{12.000000}\selectfont 4}%
\end{pgfscope}%
\begin{pgfscope}%
\definecolor{textcolor}{rgb}{1.000000,1.000000,1.000000}%
\pgfsetstrokecolor{textcolor}%
\pgfsetfillcolor{textcolor}%
\pgftext[x=6.934381in,y=0.838317in,,]{\color{textcolor}\sffamily\fontsize{10.000000}{12.000000}\selectfont 38}%
\end{pgfscope}%
\begin{pgfscope}%
\definecolor{textcolor}{rgb}{1.000000,1.000000,1.000000}%
\pgfsetstrokecolor{textcolor}%
\pgfsetfillcolor{textcolor}%
\pgftext[x=6.299063in,y=0.426552in,,]{\color{textcolor}\sffamily\fontsize{10.000000}{12.000000}\selectfont 32}%
\end{pgfscope}%
\begin{pgfscope}%
\definecolor{textcolor}{rgb}{1.000000,1.000000,1.000000}%
\pgfsetstrokecolor{textcolor}%
\pgfsetfillcolor{textcolor}%
\pgftext[x=7.787045in,y=3.720670in,,]{\color{textcolor}\sffamily\fontsize{10.000000}{12.000000}\selectfont 35}%
\end{pgfscope}%
\begin{pgfscope}%
\definecolor{textcolor}{rgb}{1.000000,1.000000,1.000000}%
\pgfsetstrokecolor{textcolor}%
\pgfsetfillcolor{textcolor}%
\pgftext[x=8.171580in,y=3.308905in,,]{\color{textcolor}\sffamily\fontsize{10.000000}{12.000000}\selectfont 12}%
\end{pgfscope}%
\begin{pgfscope}%
\definecolor{textcolor}{rgb}{1.000000,1.000000,1.000000}%
\pgfsetstrokecolor{textcolor}%
\pgfsetfillcolor{textcolor}%
\pgftext[x=7.619856in,y=2.897141in,,]{\color{textcolor}\sffamily\fontsize{10.000000}{12.000000}\selectfont 45}%
\end{pgfscope}%
\begin{pgfscope}%
\definecolor{textcolor}{rgb}{1.000000,1.000000,1.000000}%
\pgfsetstrokecolor{textcolor}%
\pgfsetfillcolor{textcolor}%
\pgftext[x=8.355487in,y=2.485376in,,]{\color{textcolor}\sffamily\fontsize{10.000000}{12.000000}\selectfont 1}%
\end{pgfscope}%
\begin{pgfscope}%
\definecolor{textcolor}{rgb}{1.000000,1.000000,1.000000}%
\pgfsetstrokecolor{textcolor}%
\pgfsetfillcolor{textcolor}%
\pgftext[x=8.238455in,y=2.073611in,,]{\color{textcolor}\sffamily\fontsize{10.000000}{12.000000}\selectfont 8}%
\end{pgfscope}%
\begin{pgfscope}%
\definecolor{textcolor}{rgb}{1.000000,1.000000,1.000000}%
\pgfsetstrokecolor{textcolor}%
\pgfsetfillcolor{textcolor}%
\pgftext[x=7.703451in,y=1.661846in,,]{\color{textcolor}\sffamily\fontsize{10.000000}{12.000000}\selectfont 40}%
\end{pgfscope}%
\begin{pgfscope}%
\definecolor{textcolor}{rgb}{1.000000,1.000000,1.000000}%
\pgfsetstrokecolor{textcolor}%
\pgfsetfillcolor{textcolor}%
\pgftext[x=8.338769in,y=1.250082in,,]{\color{textcolor}\sffamily\fontsize{10.000000}{12.000000}\selectfont 2}%
\end{pgfscope}%
\begin{pgfscope}%
\definecolor{textcolor}{rgb}{1.000000,1.000000,1.000000}%
\pgfsetstrokecolor{textcolor}%
\pgfsetfillcolor{textcolor}%
\pgftext[x=7.970953in,y=0.838317in,,]{\color{textcolor}\sffamily\fontsize{10.000000}{12.000000}\selectfont 24}%
\end{pgfscope}%
\begin{pgfscope}%
\definecolor{textcolor}{rgb}{1.000000,1.000000,1.000000}%
\pgfsetstrokecolor{textcolor}%
\pgfsetfillcolor{textcolor}%
\pgftext[x=7.603137in,y=0.426552in,,]{\color{textcolor}\sffamily\fontsize{10.000000}{12.000000}\selectfont 46}%
\end{pgfscope}%
\begin{pgfscope}%
\pgfsetbuttcap%
\pgfsetmiterjoin%
\definecolor{currentfill}{rgb}{1.000000,1.000000,1.000000}%
\pgfsetfillcolor{currentfill}%
\pgfsetfillopacity{0.800000}%
\pgfsetlinewidth{1.003750pt}%
\definecolor{currentstroke}{rgb}{0.800000,0.800000,0.800000}%
\pgfsetstrokecolor{currentstroke}%
\pgfsetstrokeopacity{0.800000}%
\pgfsetdash{}{0pt}%
\pgfpathmoveto{\pgfqpoint{1.330943in}{4.056458in}}%
\pgfpathlineto{\pgfqpoint{6.806306in}{4.056458in}}%
\pgfpathquadraticcurveto{\pgfqpoint{6.829445in}{4.056458in}}{\pgfqpoint{6.829445in}{4.079597in}}%
\pgfpathlineto{\pgfqpoint{6.829445in}{4.237841in}}%
\pgfpathquadraticcurveto{\pgfqpoint{6.829445in}{4.260980in}}{\pgfqpoint{6.806306in}{4.260980in}}%
\pgfpathlineto{\pgfqpoint{1.330943in}{4.260980in}}%
\pgfpathquadraticcurveto{\pgfqpoint{1.307804in}{4.260980in}}{\pgfqpoint{1.307804in}{4.237841in}}%
\pgfpathlineto{\pgfqpoint{1.307804in}{4.079597in}}%
\pgfpathquadraticcurveto{\pgfqpoint{1.307804in}{4.056458in}}{\pgfqpoint{1.330943in}{4.056458in}}%
\pgfpathclose%
\pgfusepath{stroke,fill}%
\end{pgfscope}%
\begin{pgfscope}%
\pgfsetbuttcap%
\pgfsetmiterjoin%
\definecolor{currentfill}{rgb}{0.248058,0.667205,0.350250}%
\pgfsetfillcolor{currentfill}%
\pgfsetfillopacity{0.500000}%
\pgfsetlinewidth{0.000000pt}%
\definecolor{currentstroke}{rgb}{0.000000,0.000000,0.000000}%
\pgfsetstrokecolor{currentstroke}%
\pgfsetstrokeopacity{0.500000}%
\pgfsetdash{}{0pt}%
\pgfpathmoveto{\pgfqpoint{1.354081in}{4.126801in}}%
\pgfpathlineto{\pgfqpoint{1.585470in}{4.126801in}}%
\pgfpathlineto{\pgfqpoint{1.585470in}{4.207787in}}%
\pgfpathlineto{\pgfqpoint{1.354081in}{4.207787in}}%
\pgfpathclose%
\pgfusepath{fill}%
\end{pgfscope}%
\begin{pgfscope}%
\definecolor{textcolor}{rgb}{0.000000,0.000000,0.000000}%
\pgfsetstrokecolor{textcolor}%
\pgfsetfillcolor{textcolor}%
\pgftext[x=1.678026in,y=4.126801in,left,base]{\color{textcolor}\sffamily\fontsize{8.330000}{9.996000}\selectfont 1}%
\end{pgfscope}%
\begin{pgfscope}%
\pgfsetbuttcap%
\pgfsetmiterjoin%
\definecolor{currentfill}{rgb}{0.488581,0.779931,0.397924}%
\pgfsetfillcolor{currentfill}%
\pgfsetfillopacity{0.500000}%
\pgfsetlinewidth{0.000000pt}%
\definecolor{currentstroke}{rgb}{0.000000,0.000000,0.000000}%
\pgfsetstrokecolor{currentstroke}%
\pgfsetstrokeopacity{0.500000}%
\pgfsetdash{}{0pt}%
\pgfpathmoveto{\pgfqpoint{1.983023in}{4.126801in}}%
\pgfpathlineto{\pgfqpoint{2.214412in}{4.126801in}}%
\pgfpathlineto{\pgfqpoint{2.214412in}{4.207787in}}%
\pgfpathlineto{\pgfqpoint{1.983023in}{4.207787in}}%
\pgfpathclose%
\pgfusepath{fill}%
\end{pgfscope}%
\begin{pgfscope}%
\definecolor{textcolor}{rgb}{0.000000,0.000000,0.000000}%
\pgfsetstrokecolor{textcolor}%
\pgfsetfillcolor{textcolor}%
\pgftext[x=2.306968in,y=4.126801in,left,base]{\color{textcolor}\sffamily\fontsize{8.330000}{9.996000}\selectfont 2}%
\end{pgfscope}%
\begin{pgfscope}%
\pgfsetbuttcap%
\pgfsetmiterjoin%
\definecolor{currentfill}{rgb}{0.701961,0.872972,0.448674}%
\pgfsetfillcolor{currentfill}%
\pgfsetfillopacity{0.500000}%
\pgfsetlinewidth{0.000000pt}%
\definecolor{currentstroke}{rgb}{0.000000,0.000000,0.000000}%
\pgfsetstrokecolor{currentstroke}%
\pgfsetstrokeopacity{0.500000}%
\pgfsetdash{}{0pt}%
\pgfpathmoveto{\pgfqpoint{2.611965in}{4.126801in}}%
\pgfpathlineto{\pgfqpoint{2.843354in}{4.126801in}}%
\pgfpathlineto{\pgfqpoint{2.843354in}{4.207787in}}%
\pgfpathlineto{\pgfqpoint{2.611965in}{4.207787in}}%
\pgfpathclose%
\pgfusepath{fill}%
\end{pgfscope}%
\begin{pgfscope}%
\definecolor{textcolor}{rgb}{0.000000,0.000000,0.000000}%
\pgfsetstrokecolor{textcolor}%
\pgfsetfillcolor{textcolor}%
\pgftext[x=2.935909in,y=4.126801in,left,base]{\color{textcolor}\sffamily\fontsize{8.330000}{9.996000}\selectfont 3}%
\end{pgfscope}%
\begin{pgfscope}%
\pgfsetbuttcap%
\pgfsetmiterjoin%
\definecolor{currentfill}{rgb}{0.868512,0.944637,0.569089}%
\pgfsetfillcolor{currentfill}%
\pgfsetfillopacity{0.500000}%
\pgfsetlinewidth{0.000000pt}%
\definecolor{currentstroke}{rgb}{0.000000,0.000000,0.000000}%
\pgfsetstrokecolor{currentstroke}%
\pgfsetstrokeopacity{0.500000}%
\pgfsetdash{}{0pt}%
\pgfpathmoveto{\pgfqpoint{3.240906in}{4.126801in}}%
\pgfpathlineto{\pgfqpoint{3.472295in}{4.126801in}}%
\pgfpathlineto{\pgfqpoint{3.472295in}{4.207787in}}%
\pgfpathlineto{\pgfqpoint{3.240906in}{4.207787in}}%
\pgfpathclose%
\pgfusepath{fill}%
\end{pgfscope}%
\begin{pgfscope}%
\definecolor{textcolor}{rgb}{0.000000,0.000000,0.000000}%
\pgfsetstrokecolor{textcolor}%
\pgfsetfillcolor{textcolor}%
\pgftext[x=3.564851in,y=4.126801in,left,base]{\color{textcolor}\sffamily\fontsize{8.330000}{9.996000}\selectfont 4}%
\end{pgfscope}%
\begin{pgfscope}%
\pgfsetbuttcap%
\pgfsetmiterjoin%
\definecolor{currentfill}{rgb}{0.997078,0.998770,0.745021}%
\pgfsetfillcolor{currentfill}%
\pgfsetfillopacity{0.500000}%
\pgfsetlinewidth{0.000000pt}%
\definecolor{currentstroke}{rgb}{0.000000,0.000000,0.000000}%
\pgfsetstrokecolor{currentstroke}%
\pgfsetstrokeopacity{0.500000}%
\pgfsetdash{}{0pt}%
\pgfpathmoveto{\pgfqpoint{3.869848in}{4.126801in}}%
\pgfpathlineto{\pgfqpoint{4.101237in}{4.126801in}}%
\pgfpathlineto{\pgfqpoint{4.101237in}{4.207787in}}%
\pgfpathlineto{\pgfqpoint{3.869848in}{4.207787in}}%
\pgfpathclose%
\pgfusepath{fill}%
\end{pgfscope}%
\begin{pgfscope}%
\definecolor{textcolor}{rgb}{0.000000,0.000000,0.000000}%
\pgfsetstrokecolor{textcolor}%
\pgfsetfillcolor{textcolor}%
\pgftext[x=4.193792in,y=4.126801in,left,base]{\color{textcolor}\sffamily\fontsize{8.330000}{9.996000}\selectfont 5}%
\end{pgfscope}%
\begin{pgfscope}%
\pgfsetbuttcap%
\pgfsetmiterjoin%
\definecolor{currentfill}{rgb}{0.996540,0.892734,0.569089}%
\pgfsetfillcolor{currentfill}%
\pgfsetfillopacity{0.500000}%
\pgfsetlinewidth{0.000000pt}%
\definecolor{currentstroke}{rgb}{0.000000,0.000000,0.000000}%
\pgfsetstrokecolor{currentstroke}%
\pgfsetstrokeopacity{0.500000}%
\pgfsetdash{}{0pt}%
\pgfpathmoveto{\pgfqpoint{4.498790in}{4.126801in}}%
\pgfpathlineto{\pgfqpoint{4.730179in}{4.126801in}}%
\pgfpathlineto{\pgfqpoint{4.730179in}{4.207787in}}%
\pgfpathlineto{\pgfqpoint{4.498790in}{4.207787in}}%
\pgfpathclose%
\pgfusepath{fill}%
\end{pgfscope}%
\begin{pgfscope}%
\definecolor{textcolor}{rgb}{0.000000,0.000000,0.000000}%
\pgfsetstrokecolor{textcolor}%
\pgfsetfillcolor{textcolor}%
\pgftext[x=4.822734in,y=4.126801in,left,base]{\color{textcolor}\sffamily\fontsize{8.330000}{9.996000}\selectfont 6}%
\end{pgfscope}%
\begin{pgfscope}%
\pgfsetbuttcap%
\pgfsetmiterjoin%
\definecolor{currentfill}{rgb}{0.993156,0.732334,0.422376}%
\pgfsetfillcolor{currentfill}%
\pgfsetfillopacity{0.500000}%
\pgfsetlinewidth{0.000000pt}%
\definecolor{currentstroke}{rgb}{0.000000,0.000000,0.000000}%
\pgfsetstrokecolor{currentstroke}%
\pgfsetstrokeopacity{0.500000}%
\pgfsetdash{}{0pt}%
\pgfpathmoveto{\pgfqpoint{5.127731in}{4.126801in}}%
\pgfpathlineto{\pgfqpoint{5.359120in}{4.126801in}}%
\pgfpathlineto{\pgfqpoint{5.359120in}{4.207787in}}%
\pgfpathlineto{\pgfqpoint{5.127731in}{4.207787in}}%
\pgfpathclose%
\pgfusepath{fill}%
\end{pgfscope}%
\begin{pgfscope}%
\definecolor{textcolor}{rgb}{0.000000,0.000000,0.000000}%
\pgfsetstrokecolor{textcolor}%
\pgfsetfillcolor{textcolor}%
\pgftext[x=5.451676in,y=4.126801in,left,base]{\color{textcolor}\sffamily\fontsize{8.330000}{9.996000}\selectfont 7}%
\end{pgfscope}%
\begin{pgfscope}%
\pgfsetbuttcap%
\pgfsetmiterjoin%
\definecolor{currentfill}{rgb}{0.969319,0.517416,0.304268}%
\pgfsetfillcolor{currentfill}%
\pgfsetfillopacity{0.500000}%
\pgfsetlinewidth{0.000000pt}%
\definecolor{currentstroke}{rgb}{0.000000,0.000000,0.000000}%
\pgfsetstrokecolor{currentstroke}%
\pgfsetstrokeopacity{0.500000}%
\pgfsetdash{}{0pt}%
\pgfpathmoveto{\pgfqpoint{5.756673in}{4.126801in}}%
\pgfpathlineto{\pgfqpoint{5.988062in}{4.126801in}}%
\pgfpathlineto{\pgfqpoint{5.988062in}{4.207787in}}%
\pgfpathlineto{\pgfqpoint{5.756673in}{4.207787in}}%
\pgfpathclose%
\pgfusepath{fill}%
\end{pgfscope}%
\begin{pgfscope}%
\definecolor{textcolor}{rgb}{0.000000,0.000000,0.000000}%
\pgfsetstrokecolor{textcolor}%
\pgfsetfillcolor{textcolor}%
\pgftext[x=6.080617in,y=4.126801in,left,base]{\color{textcolor}\sffamily\fontsize{8.330000}{9.996000}\selectfont 8}%
\end{pgfscope}%
\begin{pgfscope}%
\pgfsetbuttcap%
\pgfsetmiterjoin%
\definecolor{currentfill}{rgb}{0.898885,0.305498,0.206767}%
\pgfsetfillcolor{currentfill}%
\pgfsetfillopacity{0.500000}%
\pgfsetlinewidth{0.000000pt}%
\definecolor{currentstroke}{rgb}{0.000000,0.000000,0.000000}%
\pgfsetstrokecolor{currentstroke}%
\pgfsetstrokeopacity{0.500000}%
\pgfsetdash{}{0pt}%
\pgfpathmoveto{\pgfqpoint{6.385615in}{4.126801in}}%
\pgfpathlineto{\pgfqpoint{6.617004in}{4.126801in}}%
\pgfpathlineto{\pgfqpoint{6.617004in}{4.207787in}}%
\pgfpathlineto{\pgfqpoint{6.385615in}{4.207787in}}%
\pgfpathclose%
\pgfusepath{fill}%
\end{pgfscope}%
\begin{pgfscope}%
\definecolor{textcolor}{rgb}{0.000000,0.000000,0.000000}%
\pgfsetstrokecolor{textcolor}%
\pgfsetfillcolor{textcolor}%
\pgftext[x=6.709559in,y=4.126801in,left,base]{\color{textcolor}\sffamily\fontsize{8.330000}{9.996000}\selectfont 9}%
\end{pgfscope}%
\end{pgfpicture}%
\makeatother%
\endgroup%
}
    \caption{The figure represents distribution for Section 3 of survey. The participants were asked to rank the images based on photorealism(1 being the best) by comparing images from all 9 datasets.}
    \label{fig:question3}
\end{figure}

\begin{figure}
    \centering
    \resizebox{0.75\textwidth}{!}{%% Creator: Matplotlib, PGF backend
%%
%% To include the figure in your LaTeX document, write
%%   \input{<filename>.pgf}
%%
%% Make sure the required packages are loaded in your preamble
%%   \usepackage{pgf}
%%
%% Figures using additional raster images can only be included by \input if
%% they are in the same directory as the main LaTeX file. For loading figures
%% from other directories you can use the `import` package
%%   \usepackage{import}
%%
%% and then include the figures with
%%   \import{<path to file>}{<filename>.pgf}
%%
%% Matplotlib used the following preamble
%%   \usepackage{fontspec}
%%   \setmainfont{DejaVuSerif.ttf}[Path=\detokenize{/Users/apple/opt/anaconda3/envs/kaolin/lib/python3.7/site-packages/matplotlib/mpl-data/fonts/ttf/}]
%%   \setsansfont{DejaVuSans.ttf}[Path=\detokenize{/Users/apple/opt/anaconda3/envs/kaolin/lib/python3.7/site-packages/matplotlib/mpl-data/fonts/ttf/}]
%%   \setmonofont{DejaVuSansMono.ttf}[Path=\detokenize{/Users/apple/opt/anaconda3/envs/kaolin/lib/python3.7/site-packages/matplotlib/mpl-data/fonts/ttf/}]
%%
\begingroup%
\makeatletter%
\begin{pgfpicture}%
\pgfpathrectangle{\pgfpointorigin}{\pgfqpoint{5.543431in}{5.133701in}}%
\pgfusepath{use as bounding box, clip}%
\begin{pgfscope}%
\pgfsetbuttcap%
\pgfsetmiterjoin%
\definecolor{currentfill}{rgb}{1.000000,1.000000,1.000000}%
\pgfsetfillcolor{currentfill}%
\pgfsetlinewidth{0.000000pt}%
\definecolor{currentstroke}{rgb}{1.000000,1.000000,1.000000}%
\pgfsetstrokecolor{currentstroke}%
\pgfsetdash{}{0pt}%
\pgfpathmoveto{\pgfqpoint{0.000000in}{0.000000in}}%
\pgfpathlineto{\pgfqpoint{5.543431in}{0.000000in}}%
\pgfpathlineto{\pgfqpoint{5.543431in}{5.133701in}}%
\pgfpathlineto{\pgfqpoint{0.000000in}{5.133701in}}%
\pgfpathclose%
\pgfusepath{fill}%
\end{pgfscope}%
\begin{pgfscope}%
\pgfsetbuttcap%
\pgfsetmiterjoin%
\definecolor{currentfill}{rgb}{1.000000,1.000000,1.000000}%
\pgfsetfillcolor{currentfill}%
\pgfsetlinewidth{0.000000pt}%
\definecolor{currentstroke}{rgb}{0.000000,0.000000,0.000000}%
\pgfsetstrokecolor{currentstroke}%
\pgfsetstrokeopacity{0.000000}%
\pgfsetdash{}{0pt}%
\pgfpathmoveto{\pgfqpoint{0.475556in}{1.127740in}}%
\pgfpathlineto{\pgfqpoint{5.435556in}{1.127740in}}%
\pgfpathlineto{\pgfqpoint{5.435556in}{4.823740in}}%
\pgfpathlineto{\pgfqpoint{0.475556in}{4.823740in}}%
\pgfpathclose%
\pgfusepath{fill}%
\end{pgfscope}%
\begin{pgfscope}%
\pgfsetbuttcap%
\pgfsetroundjoin%
\definecolor{currentfill}{rgb}{0.000000,0.000000,0.000000}%
\pgfsetfillcolor{currentfill}%
\pgfsetlinewidth{0.803000pt}%
\definecolor{currentstroke}{rgb}{0.000000,0.000000,0.000000}%
\pgfsetstrokecolor{currentstroke}%
\pgfsetdash{}{0pt}%
\pgfsys@defobject{currentmarker}{\pgfqpoint{0.000000in}{-0.048611in}}{\pgfqpoint{0.000000in}{0.000000in}}{%
\pgfpathmoveto{\pgfqpoint{0.000000in}{0.000000in}}%
\pgfpathlineto{\pgfqpoint{0.000000in}{-0.048611in}}%
\pgfusepath{stroke,fill}%
}%
\begin{pgfscope}%
\pgfsys@transformshift{0.751111in}{1.127740in}%
\pgfsys@useobject{currentmarker}{}%
\end{pgfscope}%
\end{pgfscope}%
\begin{pgfscope}%
\definecolor{textcolor}{rgb}{0.000000,0.000000,0.000000}%
\pgfsetstrokecolor{textcolor}%
\pgfsetfillcolor{textcolor}%
\pgftext[x=0.541410in, y=0.482311in, left, base,rotate=45.000000]{\color{textcolor}\sffamily\fontsize{10.000000}{12.000000}\selectfont 3DFRONT}%
\end{pgfscope}%
\begin{pgfscope}%
\pgfsetbuttcap%
\pgfsetroundjoin%
\definecolor{currentfill}{rgb}{0.000000,0.000000,0.000000}%
\pgfsetfillcolor{currentfill}%
\pgfsetlinewidth{0.803000pt}%
\definecolor{currentstroke}{rgb}{0.000000,0.000000,0.000000}%
\pgfsetstrokecolor{currentstroke}%
\pgfsetdash{}{0pt}%
\pgfsys@defobject{currentmarker}{\pgfqpoint{0.000000in}{-0.048611in}}{\pgfqpoint{0.000000in}{0.000000in}}{%
\pgfpathmoveto{\pgfqpoint{0.000000in}{0.000000in}}%
\pgfpathlineto{\pgfqpoint{0.000000in}{-0.048611in}}%
\pgfusepath{stroke,fill}%
}%
\begin{pgfscope}%
\pgfsys@transformshift{1.302222in}{1.127740in}%
\pgfsys@useobject{currentmarker}{}%
\end{pgfscope}%
\end{pgfscope}%
\begin{pgfscope}%
\definecolor{textcolor}{rgb}{0.000000,0.000000,0.000000}%
\pgfsetstrokecolor{textcolor}%
\pgfsetfillcolor{textcolor}%
\pgftext[x=1.114652in, y=0.526572in, left, base,rotate=45.000000]{\color{textcolor}\sffamily\fontsize{10.000000}{12.000000}\selectfont ai2THOR}%
\end{pgfscope}%
\begin{pgfscope}%
\pgfsetbuttcap%
\pgfsetroundjoin%
\definecolor{currentfill}{rgb}{0.000000,0.000000,0.000000}%
\pgfsetfillcolor{currentfill}%
\pgfsetlinewidth{0.803000pt}%
\definecolor{currentstroke}{rgb}{0.000000,0.000000,0.000000}%
\pgfsetstrokecolor{currentstroke}%
\pgfsetdash{}{0pt}%
\pgfsys@defobject{currentmarker}{\pgfqpoint{0.000000in}{-0.048611in}}{\pgfqpoint{0.000000in}{0.000000in}}{%
\pgfpathmoveto{\pgfqpoint{0.000000in}{0.000000in}}%
\pgfpathlineto{\pgfqpoint{0.000000in}{-0.048611in}}%
\pgfusepath{stroke,fill}%
}%
\begin{pgfscope}%
\pgfsys@transformshift{1.853334in}{1.127740in}%
\pgfsys@useobject{currentmarker}{}%
\end{pgfscope}%
\end{pgfscope}%
\begin{pgfscope}%
\definecolor{textcolor}{rgb}{0.000000,0.000000,0.000000}%
\pgfsetstrokecolor{textcolor}%
\pgfsetfillcolor{textcolor}%
\pgftext[x=1.582875in, y=0.360796in, left, base,rotate=45.000000]{\color{textcolor}\sffamily\fontsize{10.000000}{12.000000}\selectfont Blenderproc}%
\end{pgfscope}%
\begin{pgfscope}%
\pgfsetbuttcap%
\pgfsetroundjoin%
\definecolor{currentfill}{rgb}{0.000000,0.000000,0.000000}%
\pgfsetfillcolor{currentfill}%
\pgfsetlinewidth{0.803000pt}%
\definecolor{currentstroke}{rgb}{0.000000,0.000000,0.000000}%
\pgfsetstrokecolor{currentstroke}%
\pgfsetdash{}{0pt}%
\pgfsys@defobject{currentmarker}{\pgfqpoint{0.000000in}{-0.048611in}}{\pgfqpoint{0.000000in}{0.000000in}}{%
\pgfpathmoveto{\pgfqpoint{0.000000in}{0.000000in}}%
\pgfpathlineto{\pgfqpoint{0.000000in}{-0.048611in}}%
\pgfusepath{stroke,fill}%
}%
\begin{pgfscope}%
\pgfsys@transformshift{2.404445in}{1.127740in}%
\pgfsys@useobject{currentmarker}{}%
\end{pgfscope}%
\end{pgfscope}%
\begin{pgfscope}%
\definecolor{textcolor}{rgb}{0.000000,0.000000,0.000000}%
\pgfsetstrokecolor{textcolor}%
\pgfsetfillcolor{textcolor}%
\pgftext[x=2.196925in, y=0.486674in, left, base,rotate=45.000000]{\color{textcolor}\sffamily\fontsize{10.000000}{12.000000}\selectfont Hyperism}%
\end{pgfscope}%
\begin{pgfscope}%
\pgfsetbuttcap%
\pgfsetroundjoin%
\definecolor{currentfill}{rgb}{0.000000,0.000000,0.000000}%
\pgfsetfillcolor{currentfill}%
\pgfsetlinewidth{0.803000pt}%
\definecolor{currentstroke}{rgb}{0.000000,0.000000,0.000000}%
\pgfsetstrokecolor{currentstroke}%
\pgfsetdash{}{0pt}%
\pgfsys@defobject{currentmarker}{\pgfqpoint{0.000000in}{-0.048611in}}{\pgfqpoint{0.000000in}{0.000000in}}{%
\pgfpathmoveto{\pgfqpoint{0.000000in}{0.000000in}}%
\pgfpathlineto{\pgfqpoint{0.000000in}{-0.048611in}}%
\pgfusepath{stroke,fill}%
}%
\begin{pgfscope}%
\pgfsys@transformshift{2.955556in}{1.127740in}%
\pgfsys@useobject{currentmarker}{}%
\end{pgfscope}%
\end{pgfscope}%
\begin{pgfscope}%
\definecolor{textcolor}{rgb}{0.000000,0.000000,0.000000}%
\pgfsetstrokecolor{textcolor}%
\pgfsetfillcolor{textcolor}%
\pgftext[x=2.717322in, y=0.425246in, left, base,rotate=45.000000]{\color{textcolor}\sffamily\fontsize{10.000000}{12.000000}\selectfont InteriorNet}%
\end{pgfscope}%
\begin{pgfscope}%
\pgfsetbuttcap%
\pgfsetroundjoin%
\definecolor{currentfill}{rgb}{0.000000,0.000000,0.000000}%
\pgfsetfillcolor{currentfill}%
\pgfsetlinewidth{0.803000pt}%
\definecolor{currentstroke}{rgb}{0.000000,0.000000,0.000000}%
\pgfsetstrokecolor{currentstroke}%
\pgfsetdash{}{0pt}%
\pgfsys@defobject{currentmarker}{\pgfqpoint{0.000000in}{-0.048611in}}{\pgfqpoint{0.000000in}{0.000000in}}{%
\pgfpathmoveto{\pgfqpoint{0.000000in}{0.000000in}}%
\pgfpathlineto{\pgfqpoint{0.000000in}{-0.048611in}}%
\pgfusepath{stroke,fill}%
}%
\begin{pgfscope}%
\pgfsys@transformshift{3.506667in}{1.127740in}%
\pgfsys@useobject{currentmarker}{}%
\end{pgfscope}%
\end{pgfscope}%
\begin{pgfscope}%
\definecolor{textcolor}{rgb}{0.000000,0.000000,0.000000}%
\pgfsetstrokecolor{textcolor}%
\pgfsetfillcolor{textcolor}%
\pgftext[x=3.237191in, y=0.362762in, left, base,rotate=45.000000]{\color{textcolor}\sffamily\fontsize{10.000000}{12.000000}\selectfont OpenRooms}%
\end{pgfscope}%
\begin{pgfscope}%
\pgfsetbuttcap%
\pgfsetroundjoin%
\definecolor{currentfill}{rgb}{0.000000,0.000000,0.000000}%
\pgfsetfillcolor{currentfill}%
\pgfsetlinewidth{0.803000pt}%
\definecolor{currentstroke}{rgb}{0.000000,0.000000,0.000000}%
\pgfsetstrokecolor{currentstroke}%
\pgfsetdash{}{0pt}%
\pgfsys@defobject{currentmarker}{\pgfqpoint{0.000000in}{-0.048611in}}{\pgfqpoint{0.000000in}{0.000000in}}{%
\pgfpathmoveto{\pgfqpoint{0.000000in}{0.000000in}}%
\pgfpathlineto{\pgfqpoint{0.000000in}{-0.048611in}}%
\pgfusepath{stroke,fill}%
}%
\begin{pgfscope}%
\pgfsys@transformshift{4.057778in}{1.127740in}%
\pgfsys@useobject{currentmarker}{}%
\end{pgfscope}%
\end{pgfscope}%
\begin{pgfscope}%
\definecolor{textcolor}{rgb}{0.000000,0.000000,0.000000}%
\pgfsetstrokecolor{textcolor}%
\pgfsetfillcolor{textcolor}%
\pgftext[x=3.944583in, y=0.675324in, left, base,rotate=45.000000]{\color{textcolor}\sffamily\fontsize{10.000000}{12.000000}\selectfont Pix3D}%
\end{pgfscope}%
\begin{pgfscope}%
\pgfsetbuttcap%
\pgfsetroundjoin%
\definecolor{currentfill}{rgb}{0.000000,0.000000,0.000000}%
\pgfsetfillcolor{currentfill}%
\pgfsetlinewidth{0.803000pt}%
\definecolor{currentstroke}{rgb}{0.000000,0.000000,0.000000}%
\pgfsetstrokecolor{currentstroke}%
\pgfsetdash{}{0pt}%
\pgfsys@defobject{currentmarker}{\pgfqpoint{0.000000in}{-0.048611in}}{\pgfqpoint{0.000000in}{0.000000in}}{%
\pgfpathmoveto{\pgfqpoint{0.000000in}{0.000000in}}%
\pgfpathlineto{\pgfqpoint{0.000000in}{-0.048611in}}%
\pgfusepath{stroke,fill}%
}%
\begin{pgfscope}%
\pgfsys@transformshift{4.608889in}{1.127740in}%
\pgfsys@useobject{currentmarker}{}%
\end{pgfscope}%
\end{pgfscope}%
\begin{pgfscope}%
\definecolor{textcolor}{rgb}{0.000000,0.000000,0.000000}%
\pgfsetstrokecolor{textcolor}%
\pgfsetfillcolor{textcolor}%
\pgftext[x=4.313230in, y=0.310396in, left, base,rotate=45.000000]{\color{textcolor}\sffamily\fontsize{10.000000}{12.000000}\selectfont S2R:3D-FREE}%
\end{pgfscope}%
\begin{pgfscope}%
\pgfsetbuttcap%
\pgfsetroundjoin%
\definecolor{currentfill}{rgb}{0.000000,0.000000,0.000000}%
\pgfsetfillcolor{currentfill}%
\pgfsetlinewidth{0.803000pt}%
\definecolor{currentstroke}{rgb}{0.000000,0.000000,0.000000}%
\pgfsetstrokecolor{currentstroke}%
\pgfsetdash{}{0pt}%
\pgfsys@defobject{currentmarker}{\pgfqpoint{0.000000in}{-0.048611in}}{\pgfqpoint{0.000000in}{0.000000in}}{%
\pgfpathmoveto{\pgfqpoint{0.000000in}{0.000000in}}%
\pgfpathlineto{\pgfqpoint{0.000000in}{-0.048611in}}%
\pgfusepath{stroke,fill}%
}%
\begin{pgfscope}%
\pgfsys@transformshift{5.160000in}{1.127740in}%
\pgfsys@useobject{currentmarker}{}%
\end{pgfscope}%
\end{pgfscope}%
\begin{pgfscope}%
\definecolor{textcolor}{rgb}{0.000000,0.000000,0.000000}%
\pgfsetstrokecolor{textcolor}%
\pgfsetfillcolor{textcolor}%
\pgftext[x=4.951186in, y=0.484085in, left, base,rotate=45.000000]{\color{textcolor}\sffamily\fontsize{10.000000}{12.000000}\selectfont SceneNet}%
\end{pgfscope}%
\begin{pgfscope}%
\definecolor{textcolor}{rgb}{0.000000,0.000000,0.000000}%
\pgfsetstrokecolor{textcolor}%
\pgfsetfillcolor{textcolor}%
\pgftext[x=2.955556in,y=0.234413in,,top]{\color{textcolor}\sffamily\fontsize{10.000000}{12.000000}\selectfont Datasets}%
\end{pgfscope}%
\begin{pgfscope}%
\pgfpathrectangle{\pgfqpoint{0.475556in}{1.127740in}}{\pgfqpoint{4.960000in}{3.696000in}}%
\pgfusepath{clip}%
\pgfsetrectcap%
\pgfsetroundjoin%
\pgfsetlinewidth{0.803000pt}%
\definecolor{currentstroke}{rgb}{0.690196,0.690196,0.690196}%
\pgfsetstrokecolor{currentstroke}%
\pgfsetdash{}{0pt}%
\pgfpathmoveto{\pgfqpoint{0.475556in}{1.295740in}}%
\pgfpathlineto{\pgfqpoint{5.435556in}{1.295740in}}%
\pgfusepath{stroke}%
\end{pgfscope}%
\begin{pgfscope}%
\pgfsetbuttcap%
\pgfsetroundjoin%
\definecolor{currentfill}{rgb}{0.000000,0.000000,0.000000}%
\pgfsetfillcolor{currentfill}%
\pgfsetlinewidth{0.803000pt}%
\definecolor{currentstroke}{rgb}{0.000000,0.000000,0.000000}%
\pgfsetstrokecolor{currentstroke}%
\pgfsetdash{}{0pt}%
\pgfsys@defobject{currentmarker}{\pgfqpoint{-0.048611in}{0.000000in}}{\pgfqpoint{-0.000000in}{0.000000in}}{%
\pgfpathmoveto{\pgfqpoint{-0.000000in}{0.000000in}}%
\pgfpathlineto{\pgfqpoint{-0.048611in}{0.000000in}}%
\pgfusepath{stroke,fill}%
}%
\begin{pgfscope}%
\pgfsys@transformshift{0.475556in}{1.295740in}%
\pgfsys@useobject{currentmarker}{}%
\end{pgfscope}%
\end{pgfscope}%
\begin{pgfscope}%
\definecolor{textcolor}{rgb}{0.000000,0.000000,0.000000}%
\pgfsetstrokecolor{textcolor}%
\pgfsetfillcolor{textcolor}%
\pgftext[x=0.289968in, y=1.242979in, left, base]{\color{textcolor}\sffamily\fontsize{10.000000}{12.000000}\selectfont 1}%
\end{pgfscope}%
\begin{pgfscope}%
\pgfpathrectangle{\pgfqpoint{0.475556in}{1.127740in}}{\pgfqpoint{4.960000in}{3.696000in}}%
\pgfusepath{clip}%
\pgfsetrectcap%
\pgfsetroundjoin%
\pgfsetlinewidth{0.803000pt}%
\definecolor{currentstroke}{rgb}{0.690196,0.690196,0.690196}%
\pgfsetstrokecolor{currentstroke}%
\pgfsetdash{}{0pt}%
\pgfpathmoveto{\pgfqpoint{0.475556in}{1.715740in}}%
\pgfpathlineto{\pgfqpoint{5.435556in}{1.715740in}}%
\pgfusepath{stroke}%
\end{pgfscope}%
\begin{pgfscope}%
\pgfsetbuttcap%
\pgfsetroundjoin%
\definecolor{currentfill}{rgb}{0.000000,0.000000,0.000000}%
\pgfsetfillcolor{currentfill}%
\pgfsetlinewidth{0.803000pt}%
\definecolor{currentstroke}{rgb}{0.000000,0.000000,0.000000}%
\pgfsetstrokecolor{currentstroke}%
\pgfsetdash{}{0pt}%
\pgfsys@defobject{currentmarker}{\pgfqpoint{-0.048611in}{0.000000in}}{\pgfqpoint{-0.000000in}{0.000000in}}{%
\pgfpathmoveto{\pgfqpoint{-0.000000in}{0.000000in}}%
\pgfpathlineto{\pgfqpoint{-0.048611in}{0.000000in}}%
\pgfusepath{stroke,fill}%
}%
\begin{pgfscope}%
\pgfsys@transformshift{0.475556in}{1.715740in}%
\pgfsys@useobject{currentmarker}{}%
\end{pgfscope}%
\end{pgfscope}%
\begin{pgfscope}%
\definecolor{textcolor}{rgb}{0.000000,0.000000,0.000000}%
\pgfsetstrokecolor{textcolor}%
\pgfsetfillcolor{textcolor}%
\pgftext[x=0.289968in, y=1.662979in, left, base]{\color{textcolor}\sffamily\fontsize{10.000000}{12.000000}\selectfont 2}%
\end{pgfscope}%
\begin{pgfscope}%
\pgfpathrectangle{\pgfqpoint{0.475556in}{1.127740in}}{\pgfqpoint{4.960000in}{3.696000in}}%
\pgfusepath{clip}%
\pgfsetrectcap%
\pgfsetroundjoin%
\pgfsetlinewidth{0.803000pt}%
\definecolor{currentstroke}{rgb}{0.690196,0.690196,0.690196}%
\pgfsetstrokecolor{currentstroke}%
\pgfsetdash{}{0pt}%
\pgfpathmoveto{\pgfqpoint{0.475556in}{2.135740in}}%
\pgfpathlineto{\pgfqpoint{5.435556in}{2.135740in}}%
\pgfusepath{stroke}%
\end{pgfscope}%
\begin{pgfscope}%
\pgfsetbuttcap%
\pgfsetroundjoin%
\definecolor{currentfill}{rgb}{0.000000,0.000000,0.000000}%
\pgfsetfillcolor{currentfill}%
\pgfsetlinewidth{0.803000pt}%
\definecolor{currentstroke}{rgb}{0.000000,0.000000,0.000000}%
\pgfsetstrokecolor{currentstroke}%
\pgfsetdash{}{0pt}%
\pgfsys@defobject{currentmarker}{\pgfqpoint{-0.048611in}{0.000000in}}{\pgfqpoint{-0.000000in}{0.000000in}}{%
\pgfpathmoveto{\pgfqpoint{-0.000000in}{0.000000in}}%
\pgfpathlineto{\pgfqpoint{-0.048611in}{0.000000in}}%
\pgfusepath{stroke,fill}%
}%
\begin{pgfscope}%
\pgfsys@transformshift{0.475556in}{2.135740in}%
\pgfsys@useobject{currentmarker}{}%
\end{pgfscope}%
\end{pgfscope}%
\begin{pgfscope}%
\definecolor{textcolor}{rgb}{0.000000,0.000000,0.000000}%
\pgfsetstrokecolor{textcolor}%
\pgfsetfillcolor{textcolor}%
\pgftext[x=0.289968in, y=2.082979in, left, base]{\color{textcolor}\sffamily\fontsize{10.000000}{12.000000}\selectfont 3}%
\end{pgfscope}%
\begin{pgfscope}%
\pgfpathrectangle{\pgfqpoint{0.475556in}{1.127740in}}{\pgfqpoint{4.960000in}{3.696000in}}%
\pgfusepath{clip}%
\pgfsetrectcap%
\pgfsetroundjoin%
\pgfsetlinewidth{0.803000pt}%
\definecolor{currentstroke}{rgb}{0.690196,0.690196,0.690196}%
\pgfsetstrokecolor{currentstroke}%
\pgfsetdash{}{0pt}%
\pgfpathmoveto{\pgfqpoint{0.475556in}{2.555740in}}%
\pgfpathlineto{\pgfqpoint{5.435556in}{2.555740in}}%
\pgfusepath{stroke}%
\end{pgfscope}%
\begin{pgfscope}%
\pgfsetbuttcap%
\pgfsetroundjoin%
\definecolor{currentfill}{rgb}{0.000000,0.000000,0.000000}%
\pgfsetfillcolor{currentfill}%
\pgfsetlinewidth{0.803000pt}%
\definecolor{currentstroke}{rgb}{0.000000,0.000000,0.000000}%
\pgfsetstrokecolor{currentstroke}%
\pgfsetdash{}{0pt}%
\pgfsys@defobject{currentmarker}{\pgfqpoint{-0.048611in}{0.000000in}}{\pgfqpoint{-0.000000in}{0.000000in}}{%
\pgfpathmoveto{\pgfqpoint{-0.000000in}{0.000000in}}%
\pgfpathlineto{\pgfqpoint{-0.048611in}{0.000000in}}%
\pgfusepath{stroke,fill}%
}%
\begin{pgfscope}%
\pgfsys@transformshift{0.475556in}{2.555740in}%
\pgfsys@useobject{currentmarker}{}%
\end{pgfscope}%
\end{pgfscope}%
\begin{pgfscope}%
\definecolor{textcolor}{rgb}{0.000000,0.000000,0.000000}%
\pgfsetstrokecolor{textcolor}%
\pgfsetfillcolor{textcolor}%
\pgftext[x=0.289968in, y=2.502979in, left, base]{\color{textcolor}\sffamily\fontsize{10.000000}{12.000000}\selectfont 4}%
\end{pgfscope}%
\begin{pgfscope}%
\pgfpathrectangle{\pgfqpoint{0.475556in}{1.127740in}}{\pgfqpoint{4.960000in}{3.696000in}}%
\pgfusepath{clip}%
\pgfsetrectcap%
\pgfsetroundjoin%
\pgfsetlinewidth{0.803000pt}%
\definecolor{currentstroke}{rgb}{0.690196,0.690196,0.690196}%
\pgfsetstrokecolor{currentstroke}%
\pgfsetdash{}{0pt}%
\pgfpathmoveto{\pgfqpoint{0.475556in}{2.975740in}}%
\pgfpathlineto{\pgfqpoint{5.435556in}{2.975740in}}%
\pgfusepath{stroke}%
\end{pgfscope}%
\begin{pgfscope}%
\pgfsetbuttcap%
\pgfsetroundjoin%
\definecolor{currentfill}{rgb}{0.000000,0.000000,0.000000}%
\pgfsetfillcolor{currentfill}%
\pgfsetlinewidth{0.803000pt}%
\definecolor{currentstroke}{rgb}{0.000000,0.000000,0.000000}%
\pgfsetstrokecolor{currentstroke}%
\pgfsetdash{}{0pt}%
\pgfsys@defobject{currentmarker}{\pgfqpoint{-0.048611in}{0.000000in}}{\pgfqpoint{-0.000000in}{0.000000in}}{%
\pgfpathmoveto{\pgfqpoint{-0.000000in}{0.000000in}}%
\pgfpathlineto{\pgfqpoint{-0.048611in}{0.000000in}}%
\pgfusepath{stroke,fill}%
}%
\begin{pgfscope}%
\pgfsys@transformshift{0.475556in}{2.975740in}%
\pgfsys@useobject{currentmarker}{}%
\end{pgfscope}%
\end{pgfscope}%
\begin{pgfscope}%
\definecolor{textcolor}{rgb}{0.000000,0.000000,0.000000}%
\pgfsetstrokecolor{textcolor}%
\pgfsetfillcolor{textcolor}%
\pgftext[x=0.289968in, y=2.922979in, left, base]{\color{textcolor}\sffamily\fontsize{10.000000}{12.000000}\selectfont 5}%
\end{pgfscope}%
\begin{pgfscope}%
\pgfpathrectangle{\pgfqpoint{0.475556in}{1.127740in}}{\pgfqpoint{4.960000in}{3.696000in}}%
\pgfusepath{clip}%
\pgfsetrectcap%
\pgfsetroundjoin%
\pgfsetlinewidth{0.803000pt}%
\definecolor{currentstroke}{rgb}{0.690196,0.690196,0.690196}%
\pgfsetstrokecolor{currentstroke}%
\pgfsetdash{}{0pt}%
\pgfpathmoveto{\pgfqpoint{0.475556in}{3.395740in}}%
\pgfpathlineto{\pgfqpoint{5.435556in}{3.395740in}}%
\pgfusepath{stroke}%
\end{pgfscope}%
\begin{pgfscope}%
\pgfsetbuttcap%
\pgfsetroundjoin%
\definecolor{currentfill}{rgb}{0.000000,0.000000,0.000000}%
\pgfsetfillcolor{currentfill}%
\pgfsetlinewidth{0.803000pt}%
\definecolor{currentstroke}{rgb}{0.000000,0.000000,0.000000}%
\pgfsetstrokecolor{currentstroke}%
\pgfsetdash{}{0pt}%
\pgfsys@defobject{currentmarker}{\pgfqpoint{-0.048611in}{0.000000in}}{\pgfqpoint{-0.000000in}{0.000000in}}{%
\pgfpathmoveto{\pgfqpoint{-0.000000in}{0.000000in}}%
\pgfpathlineto{\pgfqpoint{-0.048611in}{0.000000in}}%
\pgfusepath{stroke,fill}%
}%
\begin{pgfscope}%
\pgfsys@transformshift{0.475556in}{3.395740in}%
\pgfsys@useobject{currentmarker}{}%
\end{pgfscope}%
\end{pgfscope}%
\begin{pgfscope}%
\definecolor{textcolor}{rgb}{0.000000,0.000000,0.000000}%
\pgfsetstrokecolor{textcolor}%
\pgfsetfillcolor{textcolor}%
\pgftext[x=0.289968in, y=3.342979in, left, base]{\color{textcolor}\sffamily\fontsize{10.000000}{12.000000}\selectfont 6}%
\end{pgfscope}%
\begin{pgfscope}%
\pgfpathrectangle{\pgfqpoint{0.475556in}{1.127740in}}{\pgfqpoint{4.960000in}{3.696000in}}%
\pgfusepath{clip}%
\pgfsetrectcap%
\pgfsetroundjoin%
\pgfsetlinewidth{0.803000pt}%
\definecolor{currentstroke}{rgb}{0.690196,0.690196,0.690196}%
\pgfsetstrokecolor{currentstroke}%
\pgfsetdash{}{0pt}%
\pgfpathmoveto{\pgfqpoint{0.475556in}{3.815740in}}%
\pgfpathlineto{\pgfqpoint{5.435556in}{3.815740in}}%
\pgfusepath{stroke}%
\end{pgfscope}%
\begin{pgfscope}%
\pgfsetbuttcap%
\pgfsetroundjoin%
\definecolor{currentfill}{rgb}{0.000000,0.000000,0.000000}%
\pgfsetfillcolor{currentfill}%
\pgfsetlinewidth{0.803000pt}%
\definecolor{currentstroke}{rgb}{0.000000,0.000000,0.000000}%
\pgfsetstrokecolor{currentstroke}%
\pgfsetdash{}{0pt}%
\pgfsys@defobject{currentmarker}{\pgfqpoint{-0.048611in}{0.000000in}}{\pgfqpoint{-0.000000in}{0.000000in}}{%
\pgfpathmoveto{\pgfqpoint{-0.000000in}{0.000000in}}%
\pgfpathlineto{\pgfqpoint{-0.048611in}{0.000000in}}%
\pgfusepath{stroke,fill}%
}%
\begin{pgfscope}%
\pgfsys@transformshift{0.475556in}{3.815740in}%
\pgfsys@useobject{currentmarker}{}%
\end{pgfscope}%
\end{pgfscope}%
\begin{pgfscope}%
\definecolor{textcolor}{rgb}{0.000000,0.000000,0.000000}%
\pgfsetstrokecolor{textcolor}%
\pgfsetfillcolor{textcolor}%
\pgftext[x=0.289968in, y=3.762979in, left, base]{\color{textcolor}\sffamily\fontsize{10.000000}{12.000000}\selectfont 7}%
\end{pgfscope}%
\begin{pgfscope}%
\pgfpathrectangle{\pgfqpoint{0.475556in}{1.127740in}}{\pgfqpoint{4.960000in}{3.696000in}}%
\pgfusepath{clip}%
\pgfsetrectcap%
\pgfsetroundjoin%
\pgfsetlinewidth{0.803000pt}%
\definecolor{currentstroke}{rgb}{0.690196,0.690196,0.690196}%
\pgfsetstrokecolor{currentstroke}%
\pgfsetdash{}{0pt}%
\pgfpathmoveto{\pgfqpoint{0.475556in}{4.235740in}}%
\pgfpathlineto{\pgfqpoint{5.435556in}{4.235740in}}%
\pgfusepath{stroke}%
\end{pgfscope}%
\begin{pgfscope}%
\pgfsetbuttcap%
\pgfsetroundjoin%
\definecolor{currentfill}{rgb}{0.000000,0.000000,0.000000}%
\pgfsetfillcolor{currentfill}%
\pgfsetlinewidth{0.803000pt}%
\definecolor{currentstroke}{rgb}{0.000000,0.000000,0.000000}%
\pgfsetstrokecolor{currentstroke}%
\pgfsetdash{}{0pt}%
\pgfsys@defobject{currentmarker}{\pgfqpoint{-0.048611in}{0.000000in}}{\pgfqpoint{-0.000000in}{0.000000in}}{%
\pgfpathmoveto{\pgfqpoint{-0.000000in}{0.000000in}}%
\pgfpathlineto{\pgfqpoint{-0.048611in}{0.000000in}}%
\pgfusepath{stroke,fill}%
}%
\begin{pgfscope}%
\pgfsys@transformshift{0.475556in}{4.235740in}%
\pgfsys@useobject{currentmarker}{}%
\end{pgfscope}%
\end{pgfscope}%
\begin{pgfscope}%
\definecolor{textcolor}{rgb}{0.000000,0.000000,0.000000}%
\pgfsetstrokecolor{textcolor}%
\pgfsetfillcolor{textcolor}%
\pgftext[x=0.289968in, y=4.182979in, left, base]{\color{textcolor}\sffamily\fontsize{10.000000}{12.000000}\selectfont 8}%
\end{pgfscope}%
\begin{pgfscope}%
\pgfpathrectangle{\pgfqpoint{0.475556in}{1.127740in}}{\pgfqpoint{4.960000in}{3.696000in}}%
\pgfusepath{clip}%
\pgfsetrectcap%
\pgfsetroundjoin%
\pgfsetlinewidth{0.803000pt}%
\definecolor{currentstroke}{rgb}{0.690196,0.690196,0.690196}%
\pgfsetstrokecolor{currentstroke}%
\pgfsetdash{}{0pt}%
\pgfpathmoveto{\pgfqpoint{0.475556in}{4.655740in}}%
\pgfpathlineto{\pgfqpoint{5.435556in}{4.655740in}}%
\pgfusepath{stroke}%
\end{pgfscope}%
\begin{pgfscope}%
\pgfsetbuttcap%
\pgfsetroundjoin%
\definecolor{currentfill}{rgb}{0.000000,0.000000,0.000000}%
\pgfsetfillcolor{currentfill}%
\pgfsetlinewidth{0.803000pt}%
\definecolor{currentstroke}{rgb}{0.000000,0.000000,0.000000}%
\pgfsetstrokecolor{currentstroke}%
\pgfsetdash{}{0pt}%
\pgfsys@defobject{currentmarker}{\pgfqpoint{-0.048611in}{0.000000in}}{\pgfqpoint{-0.000000in}{0.000000in}}{%
\pgfpathmoveto{\pgfqpoint{-0.000000in}{0.000000in}}%
\pgfpathlineto{\pgfqpoint{-0.048611in}{0.000000in}}%
\pgfusepath{stroke,fill}%
}%
\begin{pgfscope}%
\pgfsys@transformshift{0.475556in}{4.655740in}%
\pgfsys@useobject{currentmarker}{}%
\end{pgfscope}%
\end{pgfscope}%
\begin{pgfscope}%
\definecolor{textcolor}{rgb}{0.000000,0.000000,0.000000}%
\pgfsetstrokecolor{textcolor}%
\pgfsetfillcolor{textcolor}%
\pgftext[x=0.289968in, y=4.602979in, left, base]{\color{textcolor}\sffamily\fontsize{10.000000}{12.000000}\selectfont 9}%
\end{pgfscope}%
\begin{pgfscope}%
\definecolor{textcolor}{rgb}{0.000000,0.000000,0.000000}%
\pgfsetstrokecolor{textcolor}%
\pgfsetfillcolor{textcolor}%
\pgftext[x=0.234413in,y=2.975740in,,bottom,rotate=90.000000]{\color{textcolor}\sffamily\fontsize{10.000000}{12.000000}\selectfont Ranks}%
\end{pgfscope}%
\begin{pgfscope}%
\pgfpathrectangle{\pgfqpoint{0.475556in}{1.127740in}}{\pgfqpoint{4.960000in}{3.696000in}}%
\pgfusepath{clip}%
\pgfsetrectcap%
\pgfsetroundjoin%
\pgfsetlinewidth{1.003750pt}%
\definecolor{currentstroke}{rgb}{0.000000,0.000000,0.000000}%
\pgfsetstrokecolor{currentstroke}%
\pgfsetdash{}{0pt}%
\pgfpathmoveto{\pgfqpoint{0.751111in}{2.555740in}}%
\pgfpathlineto{\pgfqpoint{0.751111in}{1.295740in}}%
\pgfusepath{stroke}%
\end{pgfscope}%
\begin{pgfscope}%
\pgfpathrectangle{\pgfqpoint{0.475556in}{1.127740in}}{\pgfqpoint{4.960000in}{3.696000in}}%
\pgfusepath{clip}%
\pgfsetrectcap%
\pgfsetroundjoin%
\pgfsetlinewidth{1.003750pt}%
\definecolor{currentstroke}{rgb}{0.000000,0.000000,0.000000}%
\pgfsetstrokecolor{currentstroke}%
\pgfsetdash{}{0pt}%
\pgfpathmoveto{\pgfqpoint{0.751111in}{4.235740in}}%
\pgfpathlineto{\pgfqpoint{0.751111in}{4.655740in}}%
\pgfusepath{stroke}%
\end{pgfscope}%
\begin{pgfscope}%
\pgfpathrectangle{\pgfqpoint{0.475556in}{1.127740in}}{\pgfqpoint{4.960000in}{3.696000in}}%
\pgfusepath{clip}%
\pgfsetrectcap%
\pgfsetroundjoin%
\pgfsetlinewidth{1.003750pt}%
\definecolor{currentstroke}{rgb}{0.000000,0.000000,0.000000}%
\pgfsetstrokecolor{currentstroke}%
\pgfsetdash{}{0pt}%
\pgfpathmoveto{\pgfqpoint{0.682223in}{1.295740in}}%
\pgfpathlineto{\pgfqpoint{0.820000in}{1.295740in}}%
\pgfusepath{stroke}%
\end{pgfscope}%
\begin{pgfscope}%
\pgfpathrectangle{\pgfqpoint{0.475556in}{1.127740in}}{\pgfqpoint{4.960000in}{3.696000in}}%
\pgfusepath{clip}%
\pgfsetrectcap%
\pgfsetroundjoin%
\pgfsetlinewidth{1.003750pt}%
\definecolor{currentstroke}{rgb}{0.000000,0.000000,0.000000}%
\pgfsetstrokecolor{currentstroke}%
\pgfsetdash{}{0pt}%
\pgfpathmoveto{\pgfqpoint{0.682223in}{4.655740in}}%
\pgfpathlineto{\pgfqpoint{0.820000in}{4.655740in}}%
\pgfusepath{stroke}%
\end{pgfscope}%
\begin{pgfscope}%
\pgfpathrectangle{\pgfqpoint{0.475556in}{1.127740in}}{\pgfqpoint{4.960000in}{3.696000in}}%
\pgfusepath{clip}%
\pgfsetrectcap%
\pgfsetroundjoin%
\pgfsetlinewidth{1.003750pt}%
\definecolor{currentstroke}{rgb}{0.000000,0.000000,0.000000}%
\pgfsetstrokecolor{currentstroke}%
\pgfsetdash{}{0pt}%
\pgfpathmoveto{\pgfqpoint{1.302222in}{2.135740in}}%
\pgfpathlineto{\pgfqpoint{1.302222in}{1.295740in}}%
\pgfusepath{stroke}%
\end{pgfscope}%
\begin{pgfscope}%
\pgfpathrectangle{\pgfqpoint{0.475556in}{1.127740in}}{\pgfqpoint{4.960000in}{3.696000in}}%
\pgfusepath{clip}%
\pgfsetrectcap%
\pgfsetroundjoin%
\pgfsetlinewidth{1.003750pt}%
\definecolor{currentstroke}{rgb}{0.000000,0.000000,0.000000}%
\pgfsetstrokecolor{currentstroke}%
\pgfsetdash{}{0pt}%
\pgfpathmoveto{\pgfqpoint{1.302222in}{3.815740in}}%
\pgfpathlineto{\pgfqpoint{1.302222in}{4.655740in}}%
\pgfusepath{stroke}%
\end{pgfscope}%
\begin{pgfscope}%
\pgfpathrectangle{\pgfqpoint{0.475556in}{1.127740in}}{\pgfqpoint{4.960000in}{3.696000in}}%
\pgfusepath{clip}%
\pgfsetrectcap%
\pgfsetroundjoin%
\pgfsetlinewidth{1.003750pt}%
\definecolor{currentstroke}{rgb}{0.000000,0.000000,0.000000}%
\pgfsetstrokecolor{currentstroke}%
\pgfsetdash{}{0pt}%
\pgfpathmoveto{\pgfqpoint{1.233334in}{1.295740in}}%
\pgfpathlineto{\pgfqpoint{1.371111in}{1.295740in}}%
\pgfusepath{stroke}%
\end{pgfscope}%
\begin{pgfscope}%
\pgfpathrectangle{\pgfqpoint{0.475556in}{1.127740in}}{\pgfqpoint{4.960000in}{3.696000in}}%
\pgfusepath{clip}%
\pgfsetrectcap%
\pgfsetroundjoin%
\pgfsetlinewidth{1.003750pt}%
\definecolor{currentstroke}{rgb}{0.000000,0.000000,0.000000}%
\pgfsetstrokecolor{currentstroke}%
\pgfsetdash{}{0pt}%
\pgfpathmoveto{\pgfqpoint{1.233334in}{4.655740in}}%
\pgfpathlineto{\pgfqpoint{1.371111in}{4.655740in}}%
\pgfusepath{stroke}%
\end{pgfscope}%
\begin{pgfscope}%
\pgfpathrectangle{\pgfqpoint{0.475556in}{1.127740in}}{\pgfqpoint{4.960000in}{3.696000in}}%
\pgfusepath{clip}%
\pgfsetrectcap%
\pgfsetroundjoin%
\pgfsetlinewidth{1.003750pt}%
\definecolor{currentstroke}{rgb}{0.000000,0.000000,0.000000}%
\pgfsetstrokecolor{currentstroke}%
\pgfsetdash{}{0pt}%
\pgfpathmoveto{\pgfqpoint{1.853334in}{2.975740in}}%
\pgfpathlineto{\pgfqpoint{1.853334in}{1.295740in}}%
\pgfusepath{stroke}%
\end{pgfscope}%
\begin{pgfscope}%
\pgfpathrectangle{\pgfqpoint{0.475556in}{1.127740in}}{\pgfqpoint{4.960000in}{3.696000in}}%
\pgfusepath{clip}%
\pgfsetrectcap%
\pgfsetroundjoin%
\pgfsetlinewidth{1.003750pt}%
\definecolor{currentstroke}{rgb}{0.000000,0.000000,0.000000}%
\pgfsetstrokecolor{currentstroke}%
\pgfsetdash{}{0pt}%
\pgfpathmoveto{\pgfqpoint{1.853334in}{4.235740in}}%
\pgfpathlineto{\pgfqpoint{1.853334in}{4.655740in}}%
\pgfusepath{stroke}%
\end{pgfscope}%
\begin{pgfscope}%
\pgfpathrectangle{\pgfqpoint{0.475556in}{1.127740in}}{\pgfqpoint{4.960000in}{3.696000in}}%
\pgfusepath{clip}%
\pgfsetrectcap%
\pgfsetroundjoin%
\pgfsetlinewidth{1.003750pt}%
\definecolor{currentstroke}{rgb}{0.000000,0.000000,0.000000}%
\pgfsetstrokecolor{currentstroke}%
\pgfsetdash{}{0pt}%
\pgfpathmoveto{\pgfqpoint{1.784445in}{1.295740in}}%
\pgfpathlineto{\pgfqpoint{1.922222in}{1.295740in}}%
\pgfusepath{stroke}%
\end{pgfscope}%
\begin{pgfscope}%
\pgfpathrectangle{\pgfqpoint{0.475556in}{1.127740in}}{\pgfqpoint{4.960000in}{3.696000in}}%
\pgfusepath{clip}%
\pgfsetrectcap%
\pgfsetroundjoin%
\pgfsetlinewidth{1.003750pt}%
\definecolor{currentstroke}{rgb}{0.000000,0.000000,0.000000}%
\pgfsetstrokecolor{currentstroke}%
\pgfsetdash{}{0pt}%
\pgfpathmoveto{\pgfqpoint{1.784445in}{4.655740in}}%
\pgfpathlineto{\pgfqpoint{1.922222in}{4.655740in}}%
\pgfusepath{stroke}%
\end{pgfscope}%
\begin{pgfscope}%
\pgfpathrectangle{\pgfqpoint{0.475556in}{1.127740in}}{\pgfqpoint{4.960000in}{3.696000in}}%
\pgfusepath{clip}%
\pgfsetrectcap%
\pgfsetroundjoin%
\pgfsetlinewidth{1.003750pt}%
\definecolor{currentstroke}{rgb}{0.000000,0.000000,0.000000}%
\pgfsetstrokecolor{currentstroke}%
\pgfsetdash{}{0pt}%
\pgfpathmoveto{\pgfqpoint{2.404445in}{1.715740in}}%
\pgfpathlineto{\pgfqpoint{2.404445in}{1.295740in}}%
\pgfusepath{stroke}%
\end{pgfscope}%
\begin{pgfscope}%
\pgfpathrectangle{\pgfqpoint{0.475556in}{1.127740in}}{\pgfqpoint{4.960000in}{3.696000in}}%
\pgfusepath{clip}%
\pgfsetrectcap%
\pgfsetroundjoin%
\pgfsetlinewidth{1.003750pt}%
\definecolor{currentstroke}{rgb}{0.000000,0.000000,0.000000}%
\pgfsetstrokecolor{currentstroke}%
\pgfsetdash{}{0pt}%
\pgfpathmoveto{\pgfqpoint{2.404445in}{2.975740in}}%
\pgfpathlineto{\pgfqpoint{2.404445in}{4.655740in}}%
\pgfusepath{stroke}%
\end{pgfscope}%
\begin{pgfscope}%
\pgfpathrectangle{\pgfqpoint{0.475556in}{1.127740in}}{\pgfqpoint{4.960000in}{3.696000in}}%
\pgfusepath{clip}%
\pgfsetrectcap%
\pgfsetroundjoin%
\pgfsetlinewidth{1.003750pt}%
\definecolor{currentstroke}{rgb}{0.000000,0.000000,0.000000}%
\pgfsetstrokecolor{currentstroke}%
\pgfsetdash{}{0pt}%
\pgfpathmoveto{\pgfqpoint{2.335556in}{1.295740in}}%
\pgfpathlineto{\pgfqpoint{2.473334in}{1.295740in}}%
\pgfusepath{stroke}%
\end{pgfscope}%
\begin{pgfscope}%
\pgfpathrectangle{\pgfqpoint{0.475556in}{1.127740in}}{\pgfqpoint{4.960000in}{3.696000in}}%
\pgfusepath{clip}%
\pgfsetrectcap%
\pgfsetroundjoin%
\pgfsetlinewidth{1.003750pt}%
\definecolor{currentstroke}{rgb}{0.000000,0.000000,0.000000}%
\pgfsetstrokecolor{currentstroke}%
\pgfsetdash{}{0pt}%
\pgfpathmoveto{\pgfqpoint{2.335556in}{4.655740in}}%
\pgfpathlineto{\pgfqpoint{2.473334in}{4.655740in}}%
\pgfusepath{stroke}%
\end{pgfscope}%
\begin{pgfscope}%
\pgfpathrectangle{\pgfqpoint{0.475556in}{1.127740in}}{\pgfqpoint{4.960000in}{3.696000in}}%
\pgfusepath{clip}%
\pgfsetrectcap%
\pgfsetroundjoin%
\pgfsetlinewidth{1.003750pt}%
\definecolor{currentstroke}{rgb}{0.000000,0.000000,0.000000}%
\pgfsetstrokecolor{currentstroke}%
\pgfsetdash{}{0pt}%
\pgfpathmoveto{\pgfqpoint{2.955556in}{1.715740in}}%
\pgfpathlineto{\pgfqpoint{2.955556in}{1.295740in}}%
\pgfusepath{stroke}%
\end{pgfscope}%
\begin{pgfscope}%
\pgfpathrectangle{\pgfqpoint{0.475556in}{1.127740in}}{\pgfqpoint{4.960000in}{3.696000in}}%
\pgfusepath{clip}%
\pgfsetrectcap%
\pgfsetroundjoin%
\pgfsetlinewidth{1.003750pt}%
\definecolor{currentstroke}{rgb}{0.000000,0.000000,0.000000}%
\pgfsetstrokecolor{currentstroke}%
\pgfsetdash{}{0pt}%
\pgfpathmoveto{\pgfqpoint{2.955556in}{2.555740in}}%
\pgfpathlineto{\pgfqpoint{2.955556in}{3.815740in}}%
\pgfusepath{stroke}%
\end{pgfscope}%
\begin{pgfscope}%
\pgfpathrectangle{\pgfqpoint{0.475556in}{1.127740in}}{\pgfqpoint{4.960000in}{3.696000in}}%
\pgfusepath{clip}%
\pgfsetrectcap%
\pgfsetroundjoin%
\pgfsetlinewidth{1.003750pt}%
\definecolor{currentstroke}{rgb}{0.000000,0.000000,0.000000}%
\pgfsetstrokecolor{currentstroke}%
\pgfsetdash{}{0pt}%
\pgfpathmoveto{\pgfqpoint{2.886667in}{1.295740in}}%
\pgfpathlineto{\pgfqpoint{3.024445in}{1.295740in}}%
\pgfusepath{stroke}%
\end{pgfscope}%
\begin{pgfscope}%
\pgfpathrectangle{\pgfqpoint{0.475556in}{1.127740in}}{\pgfqpoint{4.960000in}{3.696000in}}%
\pgfusepath{clip}%
\pgfsetrectcap%
\pgfsetroundjoin%
\pgfsetlinewidth{1.003750pt}%
\definecolor{currentstroke}{rgb}{0.000000,0.000000,0.000000}%
\pgfsetstrokecolor{currentstroke}%
\pgfsetdash{}{0pt}%
\pgfpathmoveto{\pgfqpoint{2.886667in}{3.815740in}}%
\pgfpathlineto{\pgfqpoint{3.024445in}{3.815740in}}%
\pgfusepath{stroke}%
\end{pgfscope}%
\begin{pgfscope}%
\pgfpathrectangle{\pgfqpoint{0.475556in}{1.127740in}}{\pgfqpoint{4.960000in}{3.696000in}}%
\pgfusepath{clip}%
\pgfsetbuttcap%
\pgfsetroundjoin%
\definecolor{currentfill}{rgb}{0.000000,0.000000,0.000000}%
\pgfsetfillcolor{currentfill}%
\pgfsetfillopacity{0.000000}%
\pgfsetlinewidth{1.003750pt}%
\definecolor{currentstroke}{rgb}{0.000000,0.000000,0.000000}%
\pgfsetstrokecolor{currentstroke}%
\pgfsetdash{}{0pt}%
\pgfsys@defobject{currentmarker}{\pgfqpoint{-0.041667in}{-0.041667in}}{\pgfqpoint{0.041667in}{0.041667in}}{%
\pgfpathmoveto{\pgfqpoint{0.000000in}{-0.041667in}}%
\pgfpathcurveto{\pgfqpoint{0.011050in}{-0.041667in}}{\pgfqpoint{0.021649in}{-0.037276in}}{\pgfqpoint{0.029463in}{-0.029463in}}%
\pgfpathcurveto{\pgfqpoint{0.037276in}{-0.021649in}}{\pgfqpoint{0.041667in}{-0.011050in}}{\pgfqpoint{0.041667in}{0.000000in}}%
\pgfpathcurveto{\pgfqpoint{0.041667in}{0.011050in}}{\pgfqpoint{0.037276in}{0.021649in}}{\pgfqpoint{0.029463in}{0.029463in}}%
\pgfpathcurveto{\pgfqpoint{0.021649in}{0.037276in}}{\pgfqpoint{0.011050in}{0.041667in}}{\pgfqpoint{0.000000in}{0.041667in}}%
\pgfpathcurveto{\pgfqpoint{-0.011050in}{0.041667in}}{\pgfqpoint{-0.021649in}{0.037276in}}{\pgfqpoint{-0.029463in}{0.029463in}}%
\pgfpathcurveto{\pgfqpoint{-0.037276in}{0.021649in}}{\pgfqpoint{-0.041667in}{0.011050in}}{\pgfqpoint{-0.041667in}{0.000000in}}%
\pgfpathcurveto{\pgfqpoint{-0.041667in}{-0.011050in}}{\pgfqpoint{-0.037276in}{-0.021649in}}{\pgfqpoint{-0.029463in}{-0.029463in}}%
\pgfpathcurveto{\pgfqpoint{-0.021649in}{-0.037276in}}{\pgfqpoint{-0.011050in}{-0.041667in}}{\pgfqpoint{0.000000in}{-0.041667in}}%
\pgfpathclose%
\pgfusepath{stroke,fill}%
}%
\begin{pgfscope}%
\pgfsys@transformshift{2.955556in}{4.235740in}%
\pgfsys@useobject{currentmarker}{}%
\end{pgfscope}%
\begin{pgfscope}%
\pgfsys@transformshift{2.955556in}{4.235740in}%
\pgfsys@useobject{currentmarker}{}%
\end{pgfscope}%
\begin{pgfscope}%
\pgfsys@transformshift{2.955556in}{4.655740in}%
\pgfsys@useobject{currentmarker}{}%
\end{pgfscope}%
\begin{pgfscope}%
\pgfsys@transformshift{2.955556in}{4.655740in}%
\pgfsys@useobject{currentmarker}{}%
\end{pgfscope}%
\begin{pgfscope}%
\pgfsys@transformshift{2.955556in}{4.655740in}%
\pgfsys@useobject{currentmarker}{}%
\end{pgfscope}%
\begin{pgfscope}%
\pgfsys@transformshift{2.955556in}{4.655740in}%
\pgfsys@useobject{currentmarker}{}%
\end{pgfscope}%
\begin{pgfscope}%
\pgfsys@transformshift{2.955556in}{4.655740in}%
\pgfsys@useobject{currentmarker}{}%
\end{pgfscope}%
\begin{pgfscope}%
\pgfsys@transformshift{2.955556in}{4.655740in}%
\pgfsys@useobject{currentmarker}{}%
\end{pgfscope}%
\begin{pgfscope}%
\pgfsys@transformshift{2.955556in}{4.655740in}%
\pgfsys@useobject{currentmarker}{}%
\end{pgfscope}%
\begin{pgfscope}%
\pgfsys@transformshift{2.955556in}{4.655740in}%
\pgfsys@useobject{currentmarker}{}%
\end{pgfscope}%
\end{pgfscope}%
\begin{pgfscope}%
\pgfpathrectangle{\pgfqpoint{0.475556in}{1.127740in}}{\pgfqpoint{4.960000in}{3.696000in}}%
\pgfusepath{clip}%
\pgfsetrectcap%
\pgfsetroundjoin%
\pgfsetlinewidth{1.003750pt}%
\definecolor{currentstroke}{rgb}{0.000000,0.000000,0.000000}%
\pgfsetstrokecolor{currentstroke}%
\pgfsetdash{}{0pt}%
\pgfpathmoveto{\pgfqpoint{3.506667in}{2.975740in}}%
\pgfpathlineto{\pgfqpoint{3.506667in}{1.295740in}}%
\pgfusepath{stroke}%
\end{pgfscope}%
\begin{pgfscope}%
\pgfpathrectangle{\pgfqpoint{0.475556in}{1.127740in}}{\pgfqpoint{4.960000in}{3.696000in}}%
\pgfusepath{clip}%
\pgfsetrectcap%
\pgfsetroundjoin%
\pgfsetlinewidth{1.003750pt}%
\definecolor{currentstroke}{rgb}{0.000000,0.000000,0.000000}%
\pgfsetstrokecolor{currentstroke}%
\pgfsetdash{}{0pt}%
\pgfpathmoveto{\pgfqpoint{3.506667in}{4.235740in}}%
\pgfpathlineto{\pgfqpoint{3.506667in}{4.655740in}}%
\pgfusepath{stroke}%
\end{pgfscope}%
\begin{pgfscope}%
\pgfpathrectangle{\pgfqpoint{0.475556in}{1.127740in}}{\pgfqpoint{4.960000in}{3.696000in}}%
\pgfusepath{clip}%
\pgfsetrectcap%
\pgfsetroundjoin%
\pgfsetlinewidth{1.003750pt}%
\definecolor{currentstroke}{rgb}{0.000000,0.000000,0.000000}%
\pgfsetstrokecolor{currentstroke}%
\pgfsetdash{}{0pt}%
\pgfpathmoveto{\pgfqpoint{3.437778in}{1.295740in}}%
\pgfpathlineto{\pgfqpoint{3.575556in}{1.295740in}}%
\pgfusepath{stroke}%
\end{pgfscope}%
\begin{pgfscope}%
\pgfpathrectangle{\pgfqpoint{0.475556in}{1.127740in}}{\pgfqpoint{4.960000in}{3.696000in}}%
\pgfusepath{clip}%
\pgfsetrectcap%
\pgfsetroundjoin%
\pgfsetlinewidth{1.003750pt}%
\definecolor{currentstroke}{rgb}{0.000000,0.000000,0.000000}%
\pgfsetstrokecolor{currentstroke}%
\pgfsetdash{}{0pt}%
\pgfpathmoveto{\pgfqpoint{3.437778in}{4.655740in}}%
\pgfpathlineto{\pgfqpoint{3.575556in}{4.655740in}}%
\pgfusepath{stroke}%
\end{pgfscope}%
\begin{pgfscope}%
\pgfpathrectangle{\pgfqpoint{0.475556in}{1.127740in}}{\pgfqpoint{4.960000in}{3.696000in}}%
\pgfusepath{clip}%
\pgfsetrectcap%
\pgfsetroundjoin%
\pgfsetlinewidth{1.003750pt}%
\definecolor{currentstroke}{rgb}{0.000000,0.000000,0.000000}%
\pgfsetstrokecolor{currentstroke}%
\pgfsetdash{}{0pt}%
\pgfpathmoveto{\pgfqpoint{4.057778in}{1.295740in}}%
\pgfpathlineto{\pgfqpoint{4.057778in}{1.295740in}}%
\pgfusepath{stroke}%
\end{pgfscope}%
\begin{pgfscope}%
\pgfpathrectangle{\pgfqpoint{0.475556in}{1.127740in}}{\pgfqpoint{4.960000in}{3.696000in}}%
\pgfusepath{clip}%
\pgfsetrectcap%
\pgfsetroundjoin%
\pgfsetlinewidth{1.003750pt}%
\definecolor{currentstroke}{rgb}{0.000000,0.000000,0.000000}%
\pgfsetstrokecolor{currentstroke}%
\pgfsetdash{}{0pt}%
\pgfpathmoveto{\pgfqpoint{4.057778in}{2.135740in}}%
\pgfpathlineto{\pgfqpoint{4.057778in}{3.395740in}}%
\pgfusepath{stroke}%
\end{pgfscope}%
\begin{pgfscope}%
\pgfpathrectangle{\pgfqpoint{0.475556in}{1.127740in}}{\pgfqpoint{4.960000in}{3.696000in}}%
\pgfusepath{clip}%
\pgfsetrectcap%
\pgfsetroundjoin%
\pgfsetlinewidth{1.003750pt}%
\definecolor{currentstroke}{rgb}{0.000000,0.000000,0.000000}%
\pgfsetstrokecolor{currentstroke}%
\pgfsetdash{}{0pt}%
\pgfpathmoveto{\pgfqpoint{3.988889in}{1.295740in}}%
\pgfpathlineto{\pgfqpoint{4.126667in}{1.295740in}}%
\pgfusepath{stroke}%
\end{pgfscope}%
\begin{pgfscope}%
\pgfpathrectangle{\pgfqpoint{0.475556in}{1.127740in}}{\pgfqpoint{4.960000in}{3.696000in}}%
\pgfusepath{clip}%
\pgfsetrectcap%
\pgfsetroundjoin%
\pgfsetlinewidth{1.003750pt}%
\definecolor{currentstroke}{rgb}{0.000000,0.000000,0.000000}%
\pgfsetstrokecolor{currentstroke}%
\pgfsetdash{}{0pt}%
\pgfpathmoveto{\pgfqpoint{3.988889in}{3.395740in}}%
\pgfpathlineto{\pgfqpoint{4.126667in}{3.395740in}}%
\pgfusepath{stroke}%
\end{pgfscope}%
\begin{pgfscope}%
\pgfpathrectangle{\pgfqpoint{0.475556in}{1.127740in}}{\pgfqpoint{4.960000in}{3.696000in}}%
\pgfusepath{clip}%
\pgfsetbuttcap%
\pgfsetroundjoin%
\definecolor{currentfill}{rgb}{0.000000,0.000000,0.000000}%
\pgfsetfillcolor{currentfill}%
\pgfsetfillopacity{0.000000}%
\pgfsetlinewidth{1.003750pt}%
\definecolor{currentstroke}{rgb}{0.000000,0.000000,0.000000}%
\pgfsetstrokecolor{currentstroke}%
\pgfsetdash{}{0pt}%
\pgfsys@defobject{currentmarker}{\pgfqpoint{-0.041667in}{-0.041667in}}{\pgfqpoint{0.041667in}{0.041667in}}{%
\pgfpathmoveto{\pgfqpoint{0.000000in}{-0.041667in}}%
\pgfpathcurveto{\pgfqpoint{0.011050in}{-0.041667in}}{\pgfqpoint{0.021649in}{-0.037276in}}{\pgfqpoint{0.029463in}{-0.029463in}}%
\pgfpathcurveto{\pgfqpoint{0.037276in}{-0.021649in}}{\pgfqpoint{0.041667in}{-0.011050in}}{\pgfqpoint{0.041667in}{0.000000in}}%
\pgfpathcurveto{\pgfqpoint{0.041667in}{0.011050in}}{\pgfqpoint{0.037276in}{0.021649in}}{\pgfqpoint{0.029463in}{0.029463in}}%
\pgfpathcurveto{\pgfqpoint{0.021649in}{0.037276in}}{\pgfqpoint{0.011050in}{0.041667in}}{\pgfqpoint{0.000000in}{0.041667in}}%
\pgfpathcurveto{\pgfqpoint{-0.011050in}{0.041667in}}{\pgfqpoint{-0.021649in}{0.037276in}}{\pgfqpoint{-0.029463in}{0.029463in}}%
\pgfpathcurveto{\pgfqpoint{-0.037276in}{0.021649in}}{\pgfqpoint{-0.041667in}{0.011050in}}{\pgfqpoint{-0.041667in}{0.000000in}}%
\pgfpathcurveto{\pgfqpoint{-0.041667in}{-0.011050in}}{\pgfqpoint{-0.037276in}{-0.021649in}}{\pgfqpoint{-0.029463in}{-0.029463in}}%
\pgfpathcurveto{\pgfqpoint{-0.021649in}{-0.037276in}}{\pgfqpoint{-0.011050in}{-0.041667in}}{\pgfqpoint{0.000000in}{-0.041667in}}%
\pgfpathclose%
\pgfusepath{stroke,fill}%
}%
\begin{pgfscope}%
\pgfsys@transformshift{4.057778in}{3.815740in}%
\pgfsys@useobject{currentmarker}{}%
\end{pgfscope}%
\begin{pgfscope}%
\pgfsys@transformshift{4.057778in}{3.815740in}%
\pgfsys@useobject{currentmarker}{}%
\end{pgfscope}%
\begin{pgfscope}%
\pgfsys@transformshift{4.057778in}{3.815740in}%
\pgfsys@useobject{currentmarker}{}%
\end{pgfscope}%
\begin{pgfscope}%
\pgfsys@transformshift{4.057778in}{4.235740in}%
\pgfsys@useobject{currentmarker}{}%
\end{pgfscope}%
\begin{pgfscope}%
\pgfsys@transformshift{4.057778in}{4.235740in}%
\pgfsys@useobject{currentmarker}{}%
\end{pgfscope}%
\begin{pgfscope}%
\pgfsys@transformshift{4.057778in}{4.235740in}%
\pgfsys@useobject{currentmarker}{}%
\end{pgfscope}%
\begin{pgfscope}%
\pgfsys@transformshift{4.057778in}{4.235740in}%
\pgfsys@useobject{currentmarker}{}%
\end{pgfscope}%
\begin{pgfscope}%
\pgfsys@transformshift{4.057778in}{4.655740in}%
\pgfsys@useobject{currentmarker}{}%
\end{pgfscope}%
\begin{pgfscope}%
\pgfsys@transformshift{4.057778in}{4.655740in}%
\pgfsys@useobject{currentmarker}{}%
\end{pgfscope}%
\end{pgfscope}%
\begin{pgfscope}%
\pgfpathrectangle{\pgfqpoint{0.475556in}{1.127740in}}{\pgfqpoint{4.960000in}{3.696000in}}%
\pgfusepath{clip}%
\pgfsetrectcap%
\pgfsetroundjoin%
\pgfsetlinewidth{1.003750pt}%
\definecolor{currentstroke}{rgb}{0.000000,0.000000,0.000000}%
\pgfsetstrokecolor{currentstroke}%
\pgfsetdash{}{0pt}%
\pgfpathmoveto{\pgfqpoint{4.608889in}{2.555740in}}%
\pgfpathlineto{\pgfqpoint{4.608889in}{1.295740in}}%
\pgfusepath{stroke}%
\end{pgfscope}%
\begin{pgfscope}%
\pgfpathrectangle{\pgfqpoint{0.475556in}{1.127740in}}{\pgfqpoint{4.960000in}{3.696000in}}%
\pgfusepath{clip}%
\pgfsetrectcap%
\pgfsetroundjoin%
\pgfsetlinewidth{1.003750pt}%
\definecolor{currentstroke}{rgb}{0.000000,0.000000,0.000000}%
\pgfsetstrokecolor{currentstroke}%
\pgfsetdash{}{0pt}%
\pgfpathmoveto{\pgfqpoint{4.608889in}{4.235740in}}%
\pgfpathlineto{\pgfqpoint{4.608889in}{4.655740in}}%
\pgfusepath{stroke}%
\end{pgfscope}%
\begin{pgfscope}%
\pgfpathrectangle{\pgfqpoint{0.475556in}{1.127740in}}{\pgfqpoint{4.960000in}{3.696000in}}%
\pgfusepath{clip}%
\pgfsetrectcap%
\pgfsetroundjoin%
\pgfsetlinewidth{1.003750pt}%
\definecolor{currentstroke}{rgb}{0.000000,0.000000,0.000000}%
\pgfsetstrokecolor{currentstroke}%
\pgfsetdash{}{0pt}%
\pgfpathmoveto{\pgfqpoint{4.540000in}{1.295740in}}%
\pgfpathlineto{\pgfqpoint{4.677778in}{1.295740in}}%
\pgfusepath{stroke}%
\end{pgfscope}%
\begin{pgfscope}%
\pgfpathrectangle{\pgfqpoint{0.475556in}{1.127740in}}{\pgfqpoint{4.960000in}{3.696000in}}%
\pgfusepath{clip}%
\pgfsetrectcap%
\pgfsetroundjoin%
\pgfsetlinewidth{1.003750pt}%
\definecolor{currentstroke}{rgb}{0.000000,0.000000,0.000000}%
\pgfsetstrokecolor{currentstroke}%
\pgfsetdash{}{0pt}%
\pgfpathmoveto{\pgfqpoint{4.540000in}{4.655740in}}%
\pgfpathlineto{\pgfqpoint{4.677778in}{4.655740in}}%
\pgfusepath{stroke}%
\end{pgfscope}%
\begin{pgfscope}%
\pgfpathrectangle{\pgfqpoint{0.475556in}{1.127740in}}{\pgfqpoint{4.960000in}{3.696000in}}%
\pgfusepath{clip}%
\pgfsetrectcap%
\pgfsetroundjoin%
\pgfsetlinewidth{1.003750pt}%
\definecolor{currentstroke}{rgb}{0.000000,0.000000,0.000000}%
\pgfsetstrokecolor{currentstroke}%
\pgfsetdash{}{0pt}%
\pgfpathmoveto{\pgfqpoint{5.160000in}{2.555740in}}%
\pgfpathlineto{\pgfqpoint{5.160000in}{1.295740in}}%
\pgfusepath{stroke}%
\end{pgfscope}%
\begin{pgfscope}%
\pgfpathrectangle{\pgfqpoint{0.475556in}{1.127740in}}{\pgfqpoint{4.960000in}{3.696000in}}%
\pgfusepath{clip}%
\pgfsetrectcap%
\pgfsetroundjoin%
\pgfsetlinewidth{1.003750pt}%
\definecolor{currentstroke}{rgb}{0.000000,0.000000,0.000000}%
\pgfsetstrokecolor{currentstroke}%
\pgfsetdash{}{0pt}%
\pgfpathmoveto{\pgfqpoint{5.160000in}{4.235740in}}%
\pgfpathlineto{\pgfqpoint{5.160000in}{4.655740in}}%
\pgfusepath{stroke}%
\end{pgfscope}%
\begin{pgfscope}%
\pgfpathrectangle{\pgfqpoint{0.475556in}{1.127740in}}{\pgfqpoint{4.960000in}{3.696000in}}%
\pgfusepath{clip}%
\pgfsetrectcap%
\pgfsetroundjoin%
\pgfsetlinewidth{1.003750pt}%
\definecolor{currentstroke}{rgb}{0.000000,0.000000,0.000000}%
\pgfsetstrokecolor{currentstroke}%
\pgfsetdash{}{0pt}%
\pgfpathmoveto{\pgfqpoint{5.091111in}{1.295740in}}%
\pgfpathlineto{\pgfqpoint{5.228889in}{1.295740in}}%
\pgfusepath{stroke}%
\end{pgfscope}%
\begin{pgfscope}%
\pgfpathrectangle{\pgfqpoint{0.475556in}{1.127740in}}{\pgfqpoint{4.960000in}{3.696000in}}%
\pgfusepath{clip}%
\pgfsetrectcap%
\pgfsetroundjoin%
\pgfsetlinewidth{1.003750pt}%
\definecolor{currentstroke}{rgb}{0.000000,0.000000,0.000000}%
\pgfsetstrokecolor{currentstroke}%
\pgfsetdash{}{0pt}%
\pgfpathmoveto{\pgfqpoint{5.091111in}{4.655740in}}%
\pgfpathlineto{\pgfqpoint{5.228889in}{4.655740in}}%
\pgfusepath{stroke}%
\end{pgfscope}%
\begin{pgfscope}%
\pgfpathrectangle{\pgfqpoint{0.475556in}{1.127740in}}{\pgfqpoint{4.960000in}{3.696000in}}%
\pgfusepath{clip}%
\pgfsetbuttcap%
\pgfsetmiterjoin%
\definecolor{currentfill}{rgb}{0.831065,0.238447,0.308804}%
\pgfsetfillcolor{currentfill}%
\pgfsetlinewidth{1.003750pt}%
\definecolor{currentstroke}{rgb}{0.000000,0.000000,0.000000}%
\pgfsetstrokecolor{currentstroke}%
\pgfsetdash{}{0pt}%
\pgfpathmoveto{\pgfqpoint{0.613334in}{2.555740in}}%
\pgfpathlineto{\pgfqpoint{0.888889in}{2.555740in}}%
\pgfpathlineto{\pgfqpoint{0.888889in}{4.235740in}}%
\pgfpathlineto{\pgfqpoint{0.613334in}{4.235740in}}%
\pgfpathlineto{\pgfqpoint{0.613334in}{2.555740in}}%
\pgfpathclose%
\pgfusepath{stroke,fill}%
\end{pgfscope}%
\begin{pgfscope}%
\pgfpathrectangle{\pgfqpoint{0.475556in}{1.127740in}}{\pgfqpoint{4.960000in}{3.696000in}}%
\pgfusepath{clip}%
\pgfsetbuttcap%
\pgfsetmiterjoin%
\definecolor{currentfill}{rgb}{0.956863,0.427451,0.262745}%
\pgfsetfillcolor{currentfill}%
\pgfsetlinewidth{1.003750pt}%
\definecolor{currentstroke}{rgb}{0.000000,0.000000,0.000000}%
\pgfsetstrokecolor{currentstroke}%
\pgfsetdash{}{0pt}%
\pgfpathmoveto{\pgfqpoint{1.164445in}{2.135740in}}%
\pgfpathlineto{\pgfqpoint{1.440000in}{2.135740in}}%
\pgfpathlineto{\pgfqpoint{1.440000in}{3.815740in}}%
\pgfpathlineto{\pgfqpoint{1.164445in}{3.815740in}}%
\pgfpathlineto{\pgfqpoint{1.164445in}{2.135740in}}%
\pgfpathclose%
\pgfusepath{stroke,fill}%
\end{pgfscope}%
\begin{pgfscope}%
\pgfpathrectangle{\pgfqpoint{0.475556in}{1.127740in}}{\pgfqpoint{4.960000in}{3.696000in}}%
\pgfusepath{clip}%
\pgfsetbuttcap%
\pgfsetmiterjoin%
\definecolor{currentfill}{rgb}{0.991465,0.677355,0.378085}%
\pgfsetfillcolor{currentfill}%
\pgfsetlinewidth{1.003750pt}%
\definecolor{currentstroke}{rgb}{0.000000,0.000000,0.000000}%
\pgfsetstrokecolor{currentstroke}%
\pgfsetdash{}{0pt}%
\pgfpathmoveto{\pgfqpoint{1.715556in}{2.975740in}}%
\pgfpathlineto{\pgfqpoint{1.991111in}{2.975740in}}%
\pgfpathlineto{\pgfqpoint{1.991111in}{4.235740in}}%
\pgfpathlineto{\pgfqpoint{1.715556in}{4.235740in}}%
\pgfpathlineto{\pgfqpoint{1.715556in}{2.975740in}}%
\pgfpathclose%
\pgfusepath{stroke,fill}%
\end{pgfscope}%
\begin{pgfscope}%
\pgfpathrectangle{\pgfqpoint{0.475556in}{1.127740in}}{\pgfqpoint{4.960000in}{3.696000in}}%
\pgfusepath{clip}%
\pgfsetbuttcap%
\pgfsetmiterjoin%
\definecolor{currentfill}{rgb}{0.996078,0.878431,0.545098}%
\pgfsetfillcolor{currentfill}%
\pgfsetlinewidth{1.003750pt}%
\definecolor{currentstroke}{rgb}{0.000000,0.000000,0.000000}%
\pgfsetstrokecolor{currentstroke}%
\pgfsetdash{}{0pt}%
\pgfpathmoveto{\pgfqpoint{2.266667in}{1.715740in}}%
\pgfpathlineto{\pgfqpoint{2.542222in}{1.715740in}}%
\pgfpathlineto{\pgfqpoint{2.542222in}{2.975740in}}%
\pgfpathlineto{\pgfqpoint{2.266667in}{2.975740in}}%
\pgfpathlineto{\pgfqpoint{2.266667in}{1.715740in}}%
\pgfpathclose%
\pgfusepath{stroke,fill}%
\end{pgfscope}%
\begin{pgfscope}%
\pgfpathrectangle{\pgfqpoint{0.475556in}{1.127740in}}{\pgfqpoint{4.960000in}{3.696000in}}%
\pgfusepath{clip}%
\pgfsetbuttcap%
\pgfsetmiterjoin%
\definecolor{currentfill}{rgb}{0.998078,0.999231,0.746021}%
\pgfsetfillcolor{currentfill}%
\pgfsetlinewidth{1.003750pt}%
\definecolor{currentstroke}{rgb}{0.000000,0.000000,0.000000}%
\pgfsetstrokecolor{currentstroke}%
\pgfsetdash{}{0pt}%
\pgfpathmoveto{\pgfqpoint{2.817778in}{1.715740in}}%
\pgfpathlineto{\pgfqpoint{3.093334in}{1.715740in}}%
\pgfpathlineto{\pgfqpoint{3.093334in}{2.555740in}}%
\pgfpathlineto{\pgfqpoint{2.817778in}{2.555740in}}%
\pgfpathlineto{\pgfqpoint{2.817778in}{1.715740in}}%
\pgfpathclose%
\pgfusepath{stroke,fill}%
\end{pgfscope}%
\begin{pgfscope}%
\pgfpathrectangle{\pgfqpoint{0.475556in}{1.127740in}}{\pgfqpoint{4.960000in}{3.696000in}}%
\pgfusepath{clip}%
\pgfsetbuttcap%
\pgfsetmiterjoin%
\definecolor{currentfill}{rgb}{0.901961,0.960784,0.596078}%
\pgfsetfillcolor{currentfill}%
\pgfsetlinewidth{1.003750pt}%
\definecolor{currentstroke}{rgb}{0.000000,0.000000,0.000000}%
\pgfsetstrokecolor{currentstroke}%
\pgfsetdash{}{0pt}%
\pgfpathmoveto{\pgfqpoint{3.368889in}{2.975740in}}%
\pgfpathlineto{\pgfqpoint{3.644445in}{2.975740in}}%
\pgfpathlineto{\pgfqpoint{3.644445in}{4.235740in}}%
\pgfpathlineto{\pgfqpoint{3.368889in}{4.235740in}}%
\pgfpathlineto{\pgfqpoint{3.368889in}{2.975740in}}%
\pgfpathclose%
\pgfusepath{stroke,fill}%
\end{pgfscope}%
\begin{pgfscope}%
\pgfpathrectangle{\pgfqpoint{0.475556in}{1.127740in}}{\pgfqpoint{4.960000in}{3.696000in}}%
\pgfusepath{clip}%
\pgfsetbuttcap%
\pgfsetmiterjoin%
\definecolor{currentfill}{rgb}{0.665283,0.864591,0.643214}%
\pgfsetfillcolor{currentfill}%
\pgfsetlinewidth{1.003750pt}%
\definecolor{currentstroke}{rgb}{0.000000,0.000000,0.000000}%
\pgfsetstrokecolor{currentstroke}%
\pgfsetdash{}{0pt}%
\pgfpathmoveto{\pgfqpoint{3.920000in}{1.295740in}}%
\pgfpathlineto{\pgfqpoint{4.195556in}{1.295740in}}%
\pgfpathlineto{\pgfqpoint{4.195556in}{2.135740in}}%
\pgfpathlineto{\pgfqpoint{3.920000in}{2.135740in}}%
\pgfpathlineto{\pgfqpoint{3.920000in}{1.295740in}}%
\pgfpathclose%
\pgfusepath{stroke,fill}%
\end{pgfscope}%
\begin{pgfscope}%
\pgfpathrectangle{\pgfqpoint{0.475556in}{1.127740in}}{\pgfqpoint{4.960000in}{3.696000in}}%
\pgfusepath{clip}%
\pgfsetbuttcap%
\pgfsetmiterjoin%
\definecolor{currentfill}{rgb}{0.400000,0.760784,0.647059}%
\pgfsetfillcolor{currentfill}%
\pgfsetlinewidth{1.003750pt}%
\definecolor{currentstroke}{rgb}{0.000000,0.000000,0.000000}%
\pgfsetstrokecolor{currentstroke}%
\pgfsetdash{}{0pt}%
\pgfpathmoveto{\pgfqpoint{4.471111in}{2.555740in}}%
\pgfpathlineto{\pgfqpoint{4.746667in}{2.555740in}}%
\pgfpathlineto{\pgfqpoint{4.746667in}{4.235740in}}%
\pgfpathlineto{\pgfqpoint{4.471111in}{4.235740in}}%
\pgfpathlineto{\pgfqpoint{4.471111in}{2.555740in}}%
\pgfpathclose%
\pgfusepath{stroke,fill}%
\end{pgfscope}%
\begin{pgfscope}%
\pgfpathrectangle{\pgfqpoint{0.475556in}{1.127740in}}{\pgfqpoint{4.960000in}{3.696000in}}%
\pgfusepath{clip}%
\pgfsetbuttcap%
\pgfsetmiterjoin%
\definecolor{currentfill}{rgb}{0.199462,0.528950,0.739100}%
\pgfsetfillcolor{currentfill}%
\pgfsetlinewidth{1.003750pt}%
\definecolor{currentstroke}{rgb}{0.000000,0.000000,0.000000}%
\pgfsetstrokecolor{currentstroke}%
\pgfsetdash{}{0pt}%
\pgfpathmoveto{\pgfqpoint{5.022222in}{2.555740in}}%
\pgfpathlineto{\pgfqpoint{5.297778in}{2.555740in}}%
\pgfpathlineto{\pgfqpoint{5.297778in}{4.235740in}}%
\pgfpathlineto{\pgfqpoint{5.022222in}{4.235740in}}%
\pgfpathlineto{\pgfqpoint{5.022222in}{2.555740in}}%
\pgfpathclose%
\pgfusepath{stroke,fill}%
\end{pgfscope}%
\begin{pgfscope}%
\pgfpathrectangle{\pgfqpoint{0.475556in}{1.127740in}}{\pgfqpoint{4.960000in}{3.696000in}}%
\pgfusepath{clip}%
\pgfsetrectcap%
\pgfsetroundjoin%
\pgfsetlinewidth{1.505625pt}%
\definecolor{currentstroke}{rgb}{0.054902,0.560784,0.027451}%
\pgfsetstrokecolor{currentstroke}%
\pgfsetdash{}{0pt}%
\pgfpathmoveto{\pgfqpoint{0.613334in}{2.975740in}}%
\pgfpathlineto{\pgfqpoint{0.888889in}{2.975740in}}%
\pgfusepath{stroke}%
\end{pgfscope}%
\begin{pgfscope}%
\pgfpathrectangle{\pgfqpoint{0.475556in}{1.127740in}}{\pgfqpoint{4.960000in}{3.696000in}}%
\pgfusepath{clip}%
\pgfsetbuttcap%
\pgfsetroundjoin%
\pgfsetlinewidth{1.505625pt}%
\definecolor{currentstroke}{rgb}{0.019608,0.078431,0.470588}%
\pgfsetstrokecolor{currentstroke}%
\pgfsetdash{{5.550000pt}{2.400000pt}}{0.000000pt}%
\pgfpathmoveto{\pgfqpoint{0.613334in}{3.194613in}}%
\pgfpathlineto{\pgfqpoint{0.888889in}{3.194613in}}%
\pgfusepath{stroke}%
\end{pgfscope}%
\begin{pgfscope}%
\pgfpathrectangle{\pgfqpoint{0.475556in}{1.127740in}}{\pgfqpoint{4.960000in}{3.696000in}}%
\pgfusepath{clip}%
\pgfsetrectcap%
\pgfsetroundjoin%
\pgfsetlinewidth{1.505625pt}%
\definecolor{currentstroke}{rgb}{0.054902,0.560784,0.027451}%
\pgfsetstrokecolor{currentstroke}%
\pgfsetdash{}{0pt}%
\pgfpathmoveto{\pgfqpoint{1.164445in}{2.975740in}}%
\pgfpathlineto{\pgfqpoint{1.440000in}{2.975740in}}%
\pgfusepath{stroke}%
\end{pgfscope}%
\begin{pgfscope}%
\pgfpathrectangle{\pgfqpoint{0.475556in}{1.127740in}}{\pgfqpoint{4.960000in}{3.696000in}}%
\pgfusepath{clip}%
\pgfsetbuttcap%
\pgfsetroundjoin%
\pgfsetlinewidth{1.505625pt}%
\definecolor{currentstroke}{rgb}{0.019608,0.078431,0.470588}%
\pgfsetstrokecolor{currentstroke}%
\pgfsetdash{{5.550000pt}{2.400000pt}}{0.000000pt}%
\pgfpathmoveto{\pgfqpoint{1.164445in}{2.983627in}}%
\pgfpathlineto{\pgfqpoint{1.440000in}{2.983627in}}%
\pgfusepath{stroke}%
\end{pgfscope}%
\begin{pgfscope}%
\pgfpathrectangle{\pgfqpoint{0.475556in}{1.127740in}}{\pgfqpoint{4.960000in}{3.696000in}}%
\pgfusepath{clip}%
\pgfsetrectcap%
\pgfsetroundjoin%
\pgfsetlinewidth{1.505625pt}%
\definecolor{currentstroke}{rgb}{0.054902,0.560784,0.027451}%
\pgfsetstrokecolor{currentstroke}%
\pgfsetdash{}{0pt}%
\pgfpathmoveto{\pgfqpoint{1.715556in}{3.815740in}}%
\pgfpathlineto{\pgfqpoint{1.991111in}{3.815740in}}%
\pgfusepath{stroke}%
\end{pgfscope}%
\begin{pgfscope}%
\pgfpathrectangle{\pgfqpoint{0.475556in}{1.127740in}}{\pgfqpoint{4.960000in}{3.696000in}}%
\pgfusepath{clip}%
\pgfsetbuttcap%
\pgfsetroundjoin%
\pgfsetlinewidth{1.505625pt}%
\definecolor{currentstroke}{rgb}{0.019608,0.078431,0.470588}%
\pgfsetstrokecolor{currentstroke}%
\pgfsetdash{{5.550000pt}{2.400000pt}}{0.000000pt}%
\pgfpathmoveto{\pgfqpoint{1.715556in}{3.596867in}}%
\pgfpathlineto{\pgfqpoint{1.991111in}{3.596867in}}%
\pgfusepath{stroke}%
\end{pgfscope}%
\begin{pgfscope}%
\pgfpathrectangle{\pgfqpoint{0.475556in}{1.127740in}}{\pgfqpoint{4.960000in}{3.696000in}}%
\pgfusepath{clip}%
\pgfsetrectcap%
\pgfsetroundjoin%
\pgfsetlinewidth{1.505625pt}%
\definecolor{currentstroke}{rgb}{0.054902,0.560784,0.027451}%
\pgfsetstrokecolor{currentstroke}%
\pgfsetdash{}{0pt}%
\pgfpathmoveto{\pgfqpoint{2.266667in}{2.555740in}}%
\pgfpathlineto{\pgfqpoint{2.542222in}{2.555740in}}%
\pgfusepath{stroke}%
\end{pgfscope}%
\begin{pgfscope}%
\pgfpathrectangle{\pgfqpoint{0.475556in}{1.127740in}}{\pgfqpoint{4.960000in}{3.696000in}}%
\pgfusepath{clip}%
\pgfsetbuttcap%
\pgfsetroundjoin%
\pgfsetlinewidth{1.505625pt}%
\definecolor{currentstroke}{rgb}{0.019608,0.078431,0.470588}%
\pgfsetstrokecolor{currentstroke}%
\pgfsetdash{{5.550000pt}{2.400000pt}}{0.000000pt}%
\pgfpathmoveto{\pgfqpoint{2.266667in}{2.514332in}}%
\pgfpathlineto{\pgfqpoint{2.542222in}{2.514332in}}%
\pgfusepath{stroke}%
\end{pgfscope}%
\begin{pgfscope}%
\pgfpathrectangle{\pgfqpoint{0.475556in}{1.127740in}}{\pgfqpoint{4.960000in}{3.696000in}}%
\pgfusepath{clip}%
\pgfsetrectcap%
\pgfsetroundjoin%
\pgfsetlinewidth{1.505625pt}%
\definecolor{currentstroke}{rgb}{0.054902,0.560784,0.027451}%
\pgfsetstrokecolor{currentstroke}%
\pgfsetdash{}{0pt}%
\pgfpathmoveto{\pgfqpoint{2.817778in}{1.715740in}}%
\pgfpathlineto{\pgfqpoint{3.093334in}{1.715740in}}%
\pgfusepath{stroke}%
\end{pgfscope}%
\begin{pgfscope}%
\pgfpathrectangle{\pgfqpoint{0.475556in}{1.127740in}}{\pgfqpoint{4.960000in}{3.696000in}}%
\pgfusepath{clip}%
\pgfsetbuttcap%
\pgfsetroundjoin%
\pgfsetlinewidth{1.505625pt}%
\definecolor{currentstroke}{rgb}{0.019608,0.078431,0.470588}%
\pgfsetstrokecolor{currentstroke}%
\pgfsetdash{{5.550000pt}{2.400000pt}}{0.000000pt}%
\pgfpathmoveto{\pgfqpoint{2.817778in}{2.187008in}}%
\pgfpathlineto{\pgfqpoint{3.093334in}{2.187008in}}%
\pgfusepath{stroke}%
\end{pgfscope}%
\begin{pgfscope}%
\pgfpathrectangle{\pgfqpoint{0.475556in}{1.127740in}}{\pgfqpoint{4.960000in}{3.696000in}}%
\pgfusepath{clip}%
\pgfsetrectcap%
\pgfsetroundjoin%
\pgfsetlinewidth{1.505625pt}%
\definecolor{currentstroke}{rgb}{0.054902,0.560784,0.027451}%
\pgfsetstrokecolor{currentstroke}%
\pgfsetdash{}{0pt}%
\pgfpathmoveto{\pgfqpoint{3.368889in}{3.815740in}}%
\pgfpathlineto{\pgfqpoint{3.644445in}{3.815740in}}%
\pgfusepath{stroke}%
\end{pgfscope}%
\begin{pgfscope}%
\pgfpathrectangle{\pgfqpoint{0.475556in}{1.127740in}}{\pgfqpoint{4.960000in}{3.696000in}}%
\pgfusepath{clip}%
\pgfsetbuttcap%
\pgfsetroundjoin%
\pgfsetlinewidth{1.505625pt}%
\definecolor{currentstroke}{rgb}{0.019608,0.078431,0.470588}%
\pgfsetstrokecolor{currentstroke}%
\pgfsetdash{{5.550000pt}{2.400000pt}}{0.000000pt}%
\pgfpathmoveto{\pgfqpoint{3.368889in}{3.575177in}}%
\pgfpathlineto{\pgfqpoint{3.644445in}{3.575177in}}%
\pgfusepath{stroke}%
\end{pgfscope}%
\begin{pgfscope}%
\pgfpathrectangle{\pgfqpoint{0.475556in}{1.127740in}}{\pgfqpoint{4.960000in}{3.696000in}}%
\pgfusepath{clip}%
\pgfsetrectcap%
\pgfsetroundjoin%
\pgfsetlinewidth{1.505625pt}%
\definecolor{currentstroke}{rgb}{0.054902,0.560784,0.027451}%
\pgfsetstrokecolor{currentstroke}%
\pgfsetdash{}{0pt}%
\pgfpathmoveto{\pgfqpoint{3.920000in}{1.715740in}}%
\pgfpathlineto{\pgfqpoint{4.195556in}{1.715740in}}%
\pgfusepath{stroke}%
\end{pgfscope}%
\begin{pgfscope}%
\pgfpathrectangle{\pgfqpoint{0.475556in}{1.127740in}}{\pgfqpoint{4.960000in}{3.696000in}}%
\pgfusepath{clip}%
\pgfsetbuttcap%
\pgfsetroundjoin%
\pgfsetlinewidth{1.505625pt}%
\definecolor{currentstroke}{rgb}{0.019608,0.078431,0.470588}%
\pgfsetstrokecolor{currentstroke}%
\pgfsetdash{{5.550000pt}{2.400000pt}}{0.000000pt}%
\pgfpathmoveto{\pgfqpoint{3.920000in}{1.920811in}}%
\pgfpathlineto{\pgfqpoint{4.195556in}{1.920811in}}%
\pgfusepath{stroke}%
\end{pgfscope}%
\begin{pgfscope}%
\pgfpathrectangle{\pgfqpoint{0.475556in}{1.127740in}}{\pgfqpoint{4.960000in}{3.696000in}}%
\pgfusepath{clip}%
\pgfsetrectcap%
\pgfsetroundjoin%
\pgfsetlinewidth{1.505625pt}%
\definecolor{currentstroke}{rgb}{0.054902,0.560784,0.027451}%
\pgfsetstrokecolor{currentstroke}%
\pgfsetdash{}{0pt}%
\pgfpathmoveto{\pgfqpoint{4.471111in}{3.395740in}}%
\pgfpathlineto{\pgfqpoint{4.746667in}{3.395740in}}%
\pgfusepath{stroke}%
\end{pgfscope}%
\begin{pgfscope}%
\pgfpathrectangle{\pgfqpoint{0.475556in}{1.127740in}}{\pgfqpoint{4.960000in}{3.696000in}}%
\pgfusepath{clip}%
\pgfsetbuttcap%
\pgfsetroundjoin%
\pgfsetlinewidth{1.505625pt}%
\definecolor{currentstroke}{rgb}{0.019608,0.078431,0.470588}%
\pgfsetstrokecolor{currentstroke}%
\pgfsetdash{{5.550000pt}{2.400000pt}}{0.000000pt}%
\pgfpathmoveto{\pgfqpoint{4.471111in}{3.376022in}}%
\pgfpathlineto{\pgfqpoint{4.746667in}{3.376022in}}%
\pgfusepath{stroke}%
\end{pgfscope}%
\begin{pgfscope}%
\pgfpathrectangle{\pgfqpoint{0.475556in}{1.127740in}}{\pgfqpoint{4.960000in}{3.696000in}}%
\pgfusepath{clip}%
\pgfsetrectcap%
\pgfsetroundjoin%
\pgfsetlinewidth{1.505625pt}%
\definecolor{currentstroke}{rgb}{0.054902,0.560784,0.027451}%
\pgfsetstrokecolor{currentstroke}%
\pgfsetdash{}{0pt}%
\pgfpathmoveto{\pgfqpoint{5.022222in}{3.395740in}}%
\pgfpathlineto{\pgfqpoint{5.297778in}{3.395740in}}%
\pgfusepath{stroke}%
\end{pgfscope}%
\begin{pgfscope}%
\pgfpathrectangle{\pgfqpoint{0.475556in}{1.127740in}}{\pgfqpoint{4.960000in}{3.696000in}}%
\pgfusepath{clip}%
\pgfsetbuttcap%
\pgfsetroundjoin%
\pgfsetlinewidth{1.505625pt}%
\definecolor{currentstroke}{rgb}{0.019608,0.078431,0.470588}%
\pgfsetstrokecolor{currentstroke}%
\pgfsetdash{{5.550000pt}{2.400000pt}}{0.000000pt}%
\pgfpathmoveto{\pgfqpoint{5.022222in}{3.433205in}}%
\pgfpathlineto{\pgfqpoint{5.297778in}{3.433205in}}%
\pgfusepath{stroke}%
\end{pgfscope}%
\begin{pgfscope}%
\pgfsetrectcap%
\pgfsetmiterjoin%
\pgfsetlinewidth{0.803000pt}%
\definecolor{currentstroke}{rgb}{0.000000,0.000000,0.000000}%
\pgfsetstrokecolor{currentstroke}%
\pgfsetdash{}{0pt}%
\pgfpathmoveto{\pgfqpoint{0.475556in}{1.127740in}}%
\pgfpathlineto{\pgfqpoint{0.475556in}{4.823740in}}%
\pgfusepath{stroke}%
\end{pgfscope}%
\begin{pgfscope}%
\pgfsetrectcap%
\pgfsetmiterjoin%
\pgfsetlinewidth{0.803000pt}%
\definecolor{currentstroke}{rgb}{0.000000,0.000000,0.000000}%
\pgfsetstrokecolor{currentstroke}%
\pgfsetdash{}{0pt}%
\pgfpathmoveto{\pgfqpoint{5.435556in}{1.127740in}}%
\pgfpathlineto{\pgfqpoint{5.435556in}{4.823740in}}%
\pgfusepath{stroke}%
\end{pgfscope}%
\begin{pgfscope}%
\pgfsetrectcap%
\pgfsetmiterjoin%
\pgfsetlinewidth{0.803000pt}%
\definecolor{currentstroke}{rgb}{0.000000,0.000000,0.000000}%
\pgfsetstrokecolor{currentstroke}%
\pgfsetdash{}{0pt}%
\pgfpathmoveto{\pgfqpoint{0.475556in}{1.127740in}}%
\pgfpathlineto{\pgfqpoint{5.435556in}{1.127740in}}%
\pgfusepath{stroke}%
\end{pgfscope}%
\begin{pgfscope}%
\pgfsetrectcap%
\pgfsetmiterjoin%
\pgfsetlinewidth{0.803000pt}%
\definecolor{currentstroke}{rgb}{0.000000,0.000000,0.000000}%
\pgfsetstrokecolor{currentstroke}%
\pgfsetdash{}{0pt}%
\pgfpathmoveto{\pgfqpoint{0.475556in}{4.823740in}}%
\pgfpathlineto{\pgfqpoint{5.435556in}{4.823740in}}%
\pgfusepath{stroke}%
\end{pgfscope}%
\begin{pgfscope}%
\definecolor{textcolor}{rgb}{0.000000,0.000000,0.000000}%
\pgfsetstrokecolor{textcolor}%
\pgfsetfillcolor{textcolor}%
\pgftext[x=2.955556in,y=4.907073in,,base]{\color{textcolor}\sffamily\fontsize{12.000000}{14.400000}\selectfont Rectangular box plot}%
\end{pgfscope}%
\end{pgfpicture}%
\makeatother%
\endgroup%
}
    \caption{The figure represents box plot for section 3. The Green horizontal line within the box indicates median, the blue dotted line indicates mean.}
    \label{fig:question3_2}
\end{figure}

\subsubsection{Summary for survey}
In the survey we observe that Pix3D was rightly chosen as photorealistic image as they are collected from real world data.
Though the proposed \gls{free} dataset falls behind in comparison to manually designed images from Hyperism, InteriorNet and \gls{front},
it is comparable to automated datasets from Blenderproc, OpenRooms and SceneNet.
It also trumps over AI2THOR which is manually designed using Unity game engines.

\section{Domain gaps}\label{sec:domain-gaps}

In this section, we verify if the synthetic dataset;
\gls{free}, has domain gap with the real dataset(Pix3D).
Along with new synthetic dataset, we will compare the datasets used for survey in~\ref{sec:a-survey-on-photorealism}.
Qualitatively we will visualise dataset embeddings using T-SNE, and quantitatively we will compare the distributions of all the synthetic dataset and real dataset.

\subsection{Qualitative}\label{subsec:qualitative}Visualize domain gap between \gls{free} and real image(pix3d)

For qualitative assessment of the domains for each of the dataset, we utilize T-SNE visualisations of embedding space from \gls{vgg}~\cite{simonyan2015deep}.
We consider a \gls{vgg}16 model pretrained on ImageNet~\cite{Deng2009ImageNetAL} and use it as an encoder to embed the image space of all the images from each of the datasets.
This latent space embedding is then converted to 2-Dimensional representation using T-SNE visualisation.
A model trained on ImageNet can be used for encoding the images since it contains all the furniture categories present in Pix3D\@.
And hence the images will be respectfully embedded and mapped to 2D space.

For each dataset, 30 images were randomly chosen and passed through the encoder at the same time.
The images are less in quantity because not all datasets provide images directly.
Some datasets like Hyperism, SceneNet, Openrooms are built for training SLAM(Simultaneous Localisation and Mapping) models, and thus not all frames contain furnitures.
We had to filter the images containing furniture so that time images appear to be in same embedding space.

In figure ~\ref{fig:photorealistic tsne}, we see the embedding for each dataset used for the survey discussed in~\ref{sec:a-survey-on-photorealism}.
For better visualisation, the points are connected to centroid of the dataset,
if not we found it difficult to comprehend scatterplot of all datasets in single graph.
The proposed \gls{free} and the real dataset;Pix3D are highlighted for better focus.

From the T-SNE visualisation, we see that the real dataset(Pix3D) is spread across the space and lies at the centre of the plot.
\gls{ai2thor}, openrooms, hyperism and \gls{front} have their embedding mapped in the outer region.
BlenderProc and SceneNet seem to be closest to the real dataset, but the latent space is not widespread indicating lesser randomisation.
\gls{free} has a wide spread and is closer to the real dataset.
This can also be seen in figure~\ref{fig:pix3d_s2r3dfree}.

We plot the visualisation for each dataset individually with real dataset in figure~\ref{fig:tsne per dataset}.
From this figure we can clearly observe that the latent space of real dataset and \gls{free} are very close to each other and spread across the space.
We can also see that \gls{ai2thor} has the maximum separation, while Blenderproc is as close as \gls{free}.
Openrooms and SceneNet also have latent space significantly apart from real dataset, while Hyperism and \gls{front} have a widespread space that intersects real data.

\begin{figure}
    \centering
    \resizebox{\textwidth}{!}{%% Creator: Matplotlib, PGF backend
%%
%% To include the figure in your LaTeX document, write
%%   \input{<filename>.pgf}
%%
%% Make sure the required packages are loaded in your preamble
%%   \usepackage{pgf}
%%
%% Figures using additional raster images can only be included by \input if
%% they are in the same directory as the main LaTeX file. For loading figures
%% from other directories you can use the `import` package
%%   \usepackage{import}
%%
%% and then include the figures with
%%   \import{<path to file>}{<filename>.pgf}
%%
%% Matplotlib used the following preamble
%%   \usepackage{fontspec}
%%   \setmainfont{DejaVuSerif.ttf}[Path=\detokenize{/Users/apple/opt/anaconda3/envs/kaolin/lib/python3.7/site-packages/matplotlib/mpl-data/fonts/ttf/}]
%%   \setsansfont{DejaVuSans.ttf}[Path=\detokenize{/Users/apple/opt/anaconda3/envs/kaolin/lib/python3.7/site-packages/matplotlib/mpl-data/fonts/ttf/}]
%%   \setmonofont{DejaVuSansMono.ttf}[Path=\detokenize{/Users/apple/opt/anaconda3/envs/kaolin/lib/python3.7/site-packages/matplotlib/mpl-data/fonts/ttf/}]
%%
\begingroup%
\makeatletter%
\begin{pgfpicture}%
\pgfpathrectangle{\pgfpointorigin}{\pgfqpoint{11.330057in}{8.341596in}}%
\pgfusepath{use as bounding box, clip}%
\begin{pgfscope}%
\pgfsetbuttcap%
\pgfsetmiterjoin%
\definecolor{currentfill}{rgb}{1.000000,1.000000,1.000000}%
\pgfsetfillcolor{currentfill}%
\pgfsetlinewidth{0.000000pt}%
\definecolor{currentstroke}{rgb}{1.000000,1.000000,1.000000}%
\pgfsetstrokecolor{currentstroke}%
\pgfsetdash{}{0pt}%
\pgfpathmoveto{\pgfqpoint{0.000000in}{0.000000in}}%
\pgfpathlineto{\pgfqpoint{11.330057in}{0.000000in}}%
\pgfpathlineto{\pgfqpoint{11.330057in}{8.341596in}}%
\pgfpathlineto{\pgfqpoint{0.000000in}{8.341596in}}%
\pgfpathclose%
\pgfusepath{fill}%
\end{pgfscope}%
\begin{pgfscope}%
\pgfsetbuttcap%
\pgfsetmiterjoin%
\definecolor{currentfill}{rgb}{1.000000,1.000000,1.000000}%
\pgfsetfillcolor{currentfill}%
\pgfsetlinewidth{0.000000pt}%
\definecolor{currentstroke}{rgb}{0.000000,0.000000,0.000000}%
\pgfsetstrokecolor{currentstroke}%
\pgfsetstrokeopacity{0.000000}%
\pgfsetdash{}{0pt}%
\pgfpathmoveto{\pgfqpoint{0.526127in}{0.331635in}}%
\pgfpathlineto{\pgfqpoint{9.826127in}{0.331635in}}%
\pgfpathlineto{\pgfqpoint{9.826127in}{8.031635in}}%
\pgfpathlineto{\pgfqpoint{0.526127in}{8.031635in}}%
\pgfpathclose%
\pgfusepath{fill}%
\end{pgfscope}%
\begin{pgfscope}%
\pgfpathrectangle{\pgfqpoint{0.526127in}{0.331635in}}{\pgfqpoint{9.300000in}{7.700000in}}%
\pgfusepath{clip}%
\pgfsetbuttcap%
\pgfsetroundjoin%
\definecolor{currentfill}{rgb}{0.631373,0.788235,0.956863}%
\pgfsetfillcolor{currentfill}%
\pgfsetlinewidth{0.481800pt}%
\definecolor{currentstroke}{rgb}{1.000000,1.000000,1.000000}%
\pgfsetstrokecolor{currentstroke}%
\pgfsetdash{}{0pt}%
\pgfpathmoveto{\pgfqpoint{8.423233in}{5.749929in}}%
\pgfpathcurveto{\pgfqpoint{8.434283in}{5.749929in}}{\pgfqpoint{8.444882in}{5.754319in}}{\pgfqpoint{8.452696in}{5.762133in}}%
\pgfpathcurveto{\pgfqpoint{8.460509in}{5.769946in}}{\pgfqpoint{8.464899in}{5.780545in}}{\pgfqpoint{8.464899in}{5.791596in}}%
\pgfpathcurveto{\pgfqpoint{8.464899in}{5.802646in}}{\pgfqpoint{8.460509in}{5.813245in}}{\pgfqpoint{8.452696in}{5.821058in}}%
\pgfpathcurveto{\pgfqpoint{8.444882in}{5.828872in}}{\pgfqpoint{8.434283in}{5.833262in}}{\pgfqpoint{8.423233in}{5.833262in}}%
\pgfpathcurveto{\pgfqpoint{8.412183in}{5.833262in}}{\pgfqpoint{8.401584in}{5.828872in}}{\pgfqpoint{8.393770in}{5.821058in}}%
\pgfpathcurveto{\pgfqpoint{8.385956in}{5.813245in}}{\pgfqpoint{8.381566in}{5.802646in}}{\pgfqpoint{8.381566in}{5.791596in}}%
\pgfpathcurveto{\pgfqpoint{8.381566in}{5.780545in}}{\pgfqpoint{8.385956in}{5.769946in}}{\pgfqpoint{8.393770in}{5.762133in}}%
\pgfpathcurveto{\pgfqpoint{8.401584in}{5.754319in}}{\pgfqpoint{8.412183in}{5.749929in}}{\pgfqpoint{8.423233in}{5.749929in}}%
\pgfpathclose%
\pgfusepath{stroke,fill}%
\end{pgfscope}%
\begin{pgfscope}%
\pgfpathrectangle{\pgfqpoint{0.526127in}{0.331635in}}{\pgfqpoint{9.300000in}{7.700000in}}%
\pgfusepath{clip}%
\pgfsetbuttcap%
\pgfsetroundjoin%
\definecolor{currentfill}{rgb}{0.631373,0.788235,0.956863}%
\pgfsetfillcolor{currentfill}%
\pgfsetlinewidth{0.481800pt}%
\definecolor{currentstroke}{rgb}{1.000000,1.000000,1.000000}%
\pgfsetstrokecolor{currentstroke}%
\pgfsetdash{}{0pt}%
\pgfpathmoveto{\pgfqpoint{4.747368in}{3.873443in}}%
\pgfpathcurveto{\pgfqpoint{4.758419in}{3.873443in}}{\pgfqpoint{4.769018in}{3.877833in}}{\pgfqpoint{4.776831in}{3.885647in}}%
\pgfpathcurveto{\pgfqpoint{4.784645in}{3.893461in}}{\pgfqpoint{4.789035in}{3.904060in}}{\pgfqpoint{4.789035in}{3.915110in}}%
\pgfpathcurveto{\pgfqpoint{4.789035in}{3.926160in}}{\pgfqpoint{4.784645in}{3.936759in}}{\pgfqpoint{4.776831in}{3.944573in}}%
\pgfpathcurveto{\pgfqpoint{4.769018in}{3.952386in}}{\pgfqpoint{4.758419in}{3.956777in}}{\pgfqpoint{4.747368in}{3.956777in}}%
\pgfpathcurveto{\pgfqpoint{4.736318in}{3.956777in}}{\pgfqpoint{4.725719in}{3.952386in}}{\pgfqpoint{4.717906in}{3.944573in}}%
\pgfpathcurveto{\pgfqpoint{4.710092in}{3.936759in}}{\pgfqpoint{4.705702in}{3.926160in}}{\pgfqpoint{4.705702in}{3.915110in}}%
\pgfpathcurveto{\pgfqpoint{4.705702in}{3.904060in}}{\pgfqpoint{4.710092in}{3.893461in}}{\pgfqpoint{4.717906in}{3.885647in}}%
\pgfpathcurveto{\pgfqpoint{4.725719in}{3.877833in}}{\pgfqpoint{4.736318in}{3.873443in}}{\pgfqpoint{4.747368in}{3.873443in}}%
\pgfpathclose%
\pgfusepath{stroke,fill}%
\end{pgfscope}%
\begin{pgfscope}%
\pgfpathrectangle{\pgfqpoint{0.526127in}{0.331635in}}{\pgfqpoint{9.300000in}{7.700000in}}%
\pgfusepath{clip}%
\pgfsetbuttcap%
\pgfsetroundjoin%
\definecolor{currentfill}{rgb}{0.631373,0.788235,0.956863}%
\pgfsetfillcolor{currentfill}%
\pgfsetlinewidth{0.481800pt}%
\definecolor{currentstroke}{rgb}{1.000000,1.000000,1.000000}%
\pgfsetstrokecolor{currentstroke}%
\pgfsetdash{}{0pt}%
\pgfpathmoveto{\pgfqpoint{6.866294in}{4.133304in}}%
\pgfpathcurveto{\pgfqpoint{6.877344in}{4.133304in}}{\pgfqpoint{6.887943in}{4.137694in}}{\pgfqpoint{6.895757in}{4.145508in}}%
\pgfpathcurveto{\pgfqpoint{6.903570in}{4.153322in}}{\pgfqpoint{6.907961in}{4.163921in}}{\pgfqpoint{6.907961in}{4.174971in}}%
\pgfpathcurveto{\pgfqpoint{6.907961in}{4.186021in}}{\pgfqpoint{6.903570in}{4.196620in}}{\pgfqpoint{6.895757in}{4.204433in}}%
\pgfpathcurveto{\pgfqpoint{6.887943in}{4.212247in}}{\pgfqpoint{6.877344in}{4.216637in}}{\pgfqpoint{6.866294in}{4.216637in}}%
\pgfpathcurveto{\pgfqpoint{6.855244in}{4.216637in}}{\pgfqpoint{6.844645in}{4.212247in}}{\pgfqpoint{6.836831in}{4.204433in}}%
\pgfpathcurveto{\pgfqpoint{6.829018in}{4.196620in}}{\pgfqpoint{6.824627in}{4.186021in}}{\pgfqpoint{6.824627in}{4.174971in}}%
\pgfpathcurveto{\pgfqpoint{6.824627in}{4.163921in}}{\pgfqpoint{6.829018in}{4.153322in}}{\pgfqpoint{6.836831in}{4.145508in}}%
\pgfpathcurveto{\pgfqpoint{6.844645in}{4.137694in}}{\pgfqpoint{6.855244in}{4.133304in}}{\pgfqpoint{6.866294in}{4.133304in}}%
\pgfpathclose%
\pgfusepath{stroke,fill}%
\end{pgfscope}%
\begin{pgfscope}%
\pgfpathrectangle{\pgfqpoint{0.526127in}{0.331635in}}{\pgfqpoint{9.300000in}{7.700000in}}%
\pgfusepath{clip}%
\pgfsetbuttcap%
\pgfsetroundjoin%
\definecolor{currentfill}{rgb}{0.631373,0.788235,0.956863}%
\pgfsetfillcolor{currentfill}%
\pgfsetlinewidth{0.481800pt}%
\definecolor{currentstroke}{rgb}{1.000000,1.000000,1.000000}%
\pgfsetstrokecolor{currentstroke}%
\pgfsetdash{}{0pt}%
\pgfpathmoveto{\pgfqpoint{8.257844in}{5.197892in}}%
\pgfpathcurveto{\pgfqpoint{8.268894in}{5.197892in}}{\pgfqpoint{8.279493in}{5.202283in}}{\pgfqpoint{8.287306in}{5.210096in}}%
\pgfpathcurveto{\pgfqpoint{8.295120in}{5.217910in}}{\pgfqpoint{8.299510in}{5.228509in}}{\pgfqpoint{8.299510in}{5.239559in}}%
\pgfpathcurveto{\pgfqpoint{8.299510in}{5.250609in}}{\pgfqpoint{8.295120in}{5.261208in}}{\pgfqpoint{8.287306in}{5.269022in}}%
\pgfpathcurveto{\pgfqpoint{8.279493in}{5.276835in}}{\pgfqpoint{8.268894in}{5.281226in}}{\pgfqpoint{8.257844in}{5.281226in}}%
\pgfpathcurveto{\pgfqpoint{8.246794in}{5.281226in}}{\pgfqpoint{8.236195in}{5.276835in}}{\pgfqpoint{8.228381in}{5.269022in}}%
\pgfpathcurveto{\pgfqpoint{8.220567in}{5.261208in}}{\pgfqpoint{8.216177in}{5.250609in}}{\pgfqpoint{8.216177in}{5.239559in}}%
\pgfpathcurveto{\pgfqpoint{8.216177in}{5.228509in}}{\pgfqpoint{8.220567in}{5.217910in}}{\pgfqpoint{8.228381in}{5.210096in}}%
\pgfpathcurveto{\pgfqpoint{8.236195in}{5.202283in}}{\pgfqpoint{8.246794in}{5.197892in}}{\pgfqpoint{8.257844in}{5.197892in}}%
\pgfpathclose%
\pgfusepath{stroke,fill}%
\end{pgfscope}%
\begin{pgfscope}%
\pgfpathrectangle{\pgfqpoint{0.526127in}{0.331635in}}{\pgfqpoint{9.300000in}{7.700000in}}%
\pgfusepath{clip}%
\pgfsetbuttcap%
\pgfsetroundjoin%
\definecolor{currentfill}{rgb}{0.631373,0.788235,0.956863}%
\pgfsetfillcolor{currentfill}%
\pgfsetlinewidth{0.481800pt}%
\definecolor{currentstroke}{rgb}{1.000000,1.000000,1.000000}%
\pgfsetstrokecolor{currentstroke}%
\pgfsetdash{}{0pt}%
\pgfpathmoveto{\pgfqpoint{3.377597in}{2.706317in}}%
\pgfpathcurveto{\pgfqpoint{3.388647in}{2.706317in}}{\pgfqpoint{3.399246in}{2.710708in}}{\pgfqpoint{3.407060in}{2.718521in}}%
\pgfpathcurveto{\pgfqpoint{3.414873in}{2.726335in}}{\pgfqpoint{3.419263in}{2.736934in}}{\pgfqpoint{3.419263in}{2.747984in}}%
\pgfpathcurveto{\pgfqpoint{3.419263in}{2.759034in}}{\pgfqpoint{3.414873in}{2.769633in}}{\pgfqpoint{3.407060in}{2.777447in}}%
\pgfpathcurveto{\pgfqpoint{3.399246in}{2.785260in}}{\pgfqpoint{3.388647in}{2.789651in}}{\pgfqpoint{3.377597in}{2.789651in}}%
\pgfpathcurveto{\pgfqpoint{3.366547in}{2.789651in}}{\pgfqpoint{3.355948in}{2.785260in}}{\pgfqpoint{3.348134in}{2.777447in}}%
\pgfpathcurveto{\pgfqpoint{3.340320in}{2.769633in}}{\pgfqpoint{3.335930in}{2.759034in}}{\pgfqpoint{3.335930in}{2.747984in}}%
\pgfpathcurveto{\pgfqpoint{3.335930in}{2.736934in}}{\pgfqpoint{3.340320in}{2.726335in}}{\pgfqpoint{3.348134in}{2.718521in}}%
\pgfpathcurveto{\pgfqpoint{3.355948in}{2.710708in}}{\pgfqpoint{3.366547in}{2.706317in}}{\pgfqpoint{3.377597in}{2.706317in}}%
\pgfpathclose%
\pgfusepath{stroke,fill}%
\end{pgfscope}%
\begin{pgfscope}%
\pgfpathrectangle{\pgfqpoint{0.526127in}{0.331635in}}{\pgfqpoint{9.300000in}{7.700000in}}%
\pgfusepath{clip}%
\pgfsetbuttcap%
\pgfsetroundjoin%
\definecolor{currentfill}{rgb}{0.631373,0.788235,0.956863}%
\pgfsetfillcolor{currentfill}%
\pgfsetlinewidth{0.481800pt}%
\definecolor{currentstroke}{rgb}{1.000000,1.000000,1.000000}%
\pgfsetstrokecolor{currentstroke}%
\pgfsetdash{}{0pt}%
\pgfpathmoveto{\pgfqpoint{5.374384in}{6.532835in}}%
\pgfpathcurveto{\pgfqpoint{5.385434in}{6.532835in}}{\pgfqpoint{5.396033in}{6.537225in}}{\pgfqpoint{5.403847in}{6.545039in}}%
\pgfpathcurveto{\pgfqpoint{5.411661in}{6.552852in}}{\pgfqpoint{5.416051in}{6.563451in}}{\pgfqpoint{5.416051in}{6.574501in}}%
\pgfpathcurveto{\pgfqpoint{5.416051in}{6.585552in}}{\pgfqpoint{5.411661in}{6.596151in}}{\pgfqpoint{5.403847in}{6.603964in}}%
\pgfpathcurveto{\pgfqpoint{5.396033in}{6.611778in}}{\pgfqpoint{5.385434in}{6.616168in}}{\pgfqpoint{5.374384in}{6.616168in}}%
\pgfpathcurveto{\pgfqpoint{5.363334in}{6.616168in}}{\pgfqpoint{5.352735in}{6.611778in}}{\pgfqpoint{5.344922in}{6.603964in}}%
\pgfpathcurveto{\pgfqpoint{5.337108in}{6.596151in}}{\pgfqpoint{5.332718in}{6.585552in}}{\pgfqpoint{5.332718in}{6.574501in}}%
\pgfpathcurveto{\pgfqpoint{5.332718in}{6.563451in}}{\pgfqpoint{5.337108in}{6.552852in}}{\pgfqpoint{5.344922in}{6.545039in}}%
\pgfpathcurveto{\pgfqpoint{5.352735in}{6.537225in}}{\pgfqpoint{5.363334in}{6.532835in}}{\pgfqpoint{5.374384in}{6.532835in}}%
\pgfpathclose%
\pgfusepath{stroke,fill}%
\end{pgfscope}%
\begin{pgfscope}%
\pgfpathrectangle{\pgfqpoint{0.526127in}{0.331635in}}{\pgfqpoint{9.300000in}{7.700000in}}%
\pgfusepath{clip}%
\pgfsetbuttcap%
\pgfsetroundjoin%
\definecolor{currentfill}{rgb}{0.631373,0.788235,0.956863}%
\pgfsetfillcolor{currentfill}%
\pgfsetlinewidth{0.481800pt}%
\definecolor{currentstroke}{rgb}{1.000000,1.000000,1.000000}%
\pgfsetstrokecolor{currentstroke}%
\pgfsetdash{}{0pt}%
\pgfpathmoveto{\pgfqpoint{3.945884in}{4.821923in}}%
\pgfpathcurveto{\pgfqpoint{3.956934in}{4.821923in}}{\pgfqpoint{3.967533in}{4.826313in}}{\pgfqpoint{3.975347in}{4.834127in}}%
\pgfpathcurveto{\pgfqpoint{3.983161in}{4.841941in}}{\pgfqpoint{3.987551in}{4.852540in}}{\pgfqpoint{3.987551in}{4.863590in}}%
\pgfpathcurveto{\pgfqpoint{3.987551in}{4.874640in}}{\pgfqpoint{3.983161in}{4.885239in}}{\pgfqpoint{3.975347in}{4.893053in}}%
\pgfpathcurveto{\pgfqpoint{3.967533in}{4.900866in}}{\pgfqpoint{3.956934in}{4.905256in}}{\pgfqpoint{3.945884in}{4.905256in}}%
\pgfpathcurveto{\pgfqpoint{3.934834in}{4.905256in}}{\pgfqpoint{3.924235in}{4.900866in}}{\pgfqpoint{3.916421in}{4.893053in}}%
\pgfpathcurveto{\pgfqpoint{3.908608in}{4.885239in}}{\pgfqpoint{3.904218in}{4.874640in}}{\pgfqpoint{3.904218in}{4.863590in}}%
\pgfpathcurveto{\pgfqpoint{3.904218in}{4.852540in}}{\pgfqpoint{3.908608in}{4.841941in}}{\pgfqpoint{3.916421in}{4.834127in}}%
\pgfpathcurveto{\pgfqpoint{3.924235in}{4.826313in}}{\pgfqpoint{3.934834in}{4.821923in}}{\pgfqpoint{3.945884in}{4.821923in}}%
\pgfpathclose%
\pgfusepath{stroke,fill}%
\end{pgfscope}%
\begin{pgfscope}%
\pgfpathrectangle{\pgfqpoint{0.526127in}{0.331635in}}{\pgfqpoint{9.300000in}{7.700000in}}%
\pgfusepath{clip}%
\pgfsetbuttcap%
\pgfsetroundjoin%
\definecolor{currentfill}{rgb}{0.631373,0.788235,0.956863}%
\pgfsetfillcolor{currentfill}%
\pgfsetlinewidth{0.481800pt}%
\definecolor{currentstroke}{rgb}{1.000000,1.000000,1.000000}%
\pgfsetstrokecolor{currentstroke}%
\pgfsetdash{}{0pt}%
\pgfpathmoveto{\pgfqpoint{6.422655in}{6.020429in}}%
\pgfpathcurveto{\pgfqpoint{6.433705in}{6.020429in}}{\pgfqpoint{6.444304in}{6.024819in}}{\pgfqpoint{6.452117in}{6.032633in}}%
\pgfpathcurveto{\pgfqpoint{6.459931in}{6.040447in}}{\pgfqpoint{6.464321in}{6.051046in}}{\pgfqpoint{6.464321in}{6.062096in}}%
\pgfpathcurveto{\pgfqpoint{6.464321in}{6.073146in}}{\pgfqpoint{6.459931in}{6.083745in}}{\pgfqpoint{6.452117in}{6.091559in}}%
\pgfpathcurveto{\pgfqpoint{6.444304in}{6.099372in}}{\pgfqpoint{6.433705in}{6.103762in}}{\pgfqpoint{6.422655in}{6.103762in}}%
\pgfpathcurveto{\pgfqpoint{6.411604in}{6.103762in}}{\pgfqpoint{6.401005in}{6.099372in}}{\pgfqpoint{6.393192in}{6.091559in}}%
\pgfpathcurveto{\pgfqpoint{6.385378in}{6.083745in}}{\pgfqpoint{6.380988in}{6.073146in}}{\pgfqpoint{6.380988in}{6.062096in}}%
\pgfpathcurveto{\pgfqpoint{6.380988in}{6.051046in}}{\pgfqpoint{6.385378in}{6.040447in}}{\pgfqpoint{6.393192in}{6.032633in}}%
\pgfpathcurveto{\pgfqpoint{6.401005in}{6.024819in}}{\pgfqpoint{6.411604in}{6.020429in}}{\pgfqpoint{6.422655in}{6.020429in}}%
\pgfpathclose%
\pgfusepath{stroke,fill}%
\end{pgfscope}%
\begin{pgfscope}%
\pgfpathrectangle{\pgfqpoint{0.526127in}{0.331635in}}{\pgfqpoint{9.300000in}{7.700000in}}%
\pgfusepath{clip}%
\pgfsetbuttcap%
\pgfsetroundjoin%
\definecolor{currentfill}{rgb}{0.631373,0.788235,0.956863}%
\pgfsetfillcolor{currentfill}%
\pgfsetlinewidth{0.481800pt}%
\definecolor{currentstroke}{rgb}{1.000000,1.000000,1.000000}%
\pgfsetstrokecolor{currentstroke}%
\pgfsetdash{}{0pt}%
\pgfpathmoveto{\pgfqpoint{3.784322in}{5.153163in}}%
\pgfpathcurveto{\pgfqpoint{3.795372in}{5.153163in}}{\pgfqpoint{3.805971in}{5.157553in}}{\pgfqpoint{3.813785in}{5.165367in}}%
\pgfpathcurveto{\pgfqpoint{3.821599in}{5.173180in}}{\pgfqpoint{3.825989in}{5.183779in}}{\pgfqpoint{3.825989in}{5.194829in}}%
\pgfpathcurveto{\pgfqpoint{3.825989in}{5.205880in}}{\pgfqpoint{3.821599in}{5.216479in}}{\pgfqpoint{3.813785in}{5.224292in}}%
\pgfpathcurveto{\pgfqpoint{3.805971in}{5.232106in}}{\pgfqpoint{3.795372in}{5.236496in}}{\pgfqpoint{3.784322in}{5.236496in}}%
\pgfpathcurveto{\pgfqpoint{3.773272in}{5.236496in}}{\pgfqpoint{3.762673in}{5.232106in}}{\pgfqpoint{3.754860in}{5.224292in}}%
\pgfpathcurveto{\pgfqpoint{3.747046in}{5.216479in}}{\pgfqpoint{3.742656in}{5.205880in}}{\pgfqpoint{3.742656in}{5.194829in}}%
\pgfpathcurveto{\pgfqpoint{3.742656in}{5.183779in}}{\pgfqpoint{3.747046in}{5.173180in}}{\pgfqpoint{3.754860in}{5.165367in}}%
\pgfpathcurveto{\pgfqpoint{3.762673in}{5.157553in}}{\pgfqpoint{3.773272in}{5.153163in}}{\pgfqpoint{3.784322in}{5.153163in}}%
\pgfpathclose%
\pgfusepath{stroke,fill}%
\end{pgfscope}%
\begin{pgfscope}%
\pgfpathrectangle{\pgfqpoint{0.526127in}{0.331635in}}{\pgfqpoint{9.300000in}{7.700000in}}%
\pgfusepath{clip}%
\pgfsetbuttcap%
\pgfsetroundjoin%
\definecolor{currentfill}{rgb}{0.631373,0.788235,0.956863}%
\pgfsetfillcolor{currentfill}%
\pgfsetlinewidth{0.481800pt}%
\definecolor{currentstroke}{rgb}{1.000000,1.000000,1.000000}%
\pgfsetstrokecolor{currentstroke}%
\pgfsetdash{}{0pt}%
\pgfpathmoveto{\pgfqpoint{4.807076in}{4.359110in}}%
\pgfpathcurveto{\pgfqpoint{4.818126in}{4.359110in}}{\pgfqpoint{4.828725in}{4.363500in}}{\pgfqpoint{4.836539in}{4.371314in}}%
\pgfpathcurveto{\pgfqpoint{4.844352in}{4.379128in}}{\pgfqpoint{4.848743in}{4.389727in}}{\pgfqpoint{4.848743in}{4.400777in}}%
\pgfpathcurveto{\pgfqpoint{4.848743in}{4.411827in}}{\pgfqpoint{4.844352in}{4.422426in}}{\pgfqpoint{4.836539in}{4.430240in}}%
\pgfpathcurveto{\pgfqpoint{4.828725in}{4.438053in}}{\pgfqpoint{4.818126in}{4.442444in}}{\pgfqpoint{4.807076in}{4.442444in}}%
\pgfpathcurveto{\pgfqpoint{4.796026in}{4.442444in}}{\pgfqpoint{4.785427in}{4.438053in}}{\pgfqpoint{4.777613in}{4.430240in}}%
\pgfpathcurveto{\pgfqpoint{4.769800in}{4.422426in}}{\pgfqpoint{4.765409in}{4.411827in}}{\pgfqpoint{4.765409in}{4.400777in}}%
\pgfpathcurveto{\pgfqpoint{4.765409in}{4.389727in}}{\pgfqpoint{4.769800in}{4.379128in}}{\pgfqpoint{4.777613in}{4.371314in}}%
\pgfpathcurveto{\pgfqpoint{4.785427in}{4.363500in}}{\pgfqpoint{4.796026in}{4.359110in}}{\pgfqpoint{4.807076in}{4.359110in}}%
\pgfpathclose%
\pgfusepath{stroke,fill}%
\end{pgfscope}%
\begin{pgfscope}%
\pgfpathrectangle{\pgfqpoint{0.526127in}{0.331635in}}{\pgfqpoint{9.300000in}{7.700000in}}%
\pgfusepath{clip}%
\pgfsetbuttcap%
\pgfsetroundjoin%
\definecolor{currentfill}{rgb}{0.631373,0.788235,0.956863}%
\pgfsetfillcolor{currentfill}%
\pgfsetlinewidth{0.481800pt}%
\definecolor{currentstroke}{rgb}{1.000000,1.000000,1.000000}%
\pgfsetstrokecolor{currentstroke}%
\pgfsetdash{}{0pt}%
\pgfpathmoveto{\pgfqpoint{7.809864in}{6.873138in}}%
\pgfpathcurveto{\pgfqpoint{7.820915in}{6.873138in}}{\pgfqpoint{7.831514in}{6.877528in}}{\pgfqpoint{7.839327in}{6.885342in}}%
\pgfpathcurveto{\pgfqpoint{7.847141in}{6.893155in}}{\pgfqpoint{7.851531in}{6.903754in}}{\pgfqpoint{7.851531in}{6.914805in}}%
\pgfpathcurveto{\pgfqpoint{7.851531in}{6.925855in}}{\pgfqpoint{7.847141in}{6.936454in}}{\pgfqpoint{7.839327in}{6.944267in}}%
\pgfpathcurveto{\pgfqpoint{7.831514in}{6.952081in}}{\pgfqpoint{7.820915in}{6.956471in}}{\pgfqpoint{7.809864in}{6.956471in}}%
\pgfpathcurveto{\pgfqpoint{7.798814in}{6.956471in}}{\pgfqpoint{7.788215in}{6.952081in}}{\pgfqpoint{7.780402in}{6.944267in}}%
\pgfpathcurveto{\pgfqpoint{7.772588in}{6.936454in}}{\pgfqpoint{7.768198in}{6.925855in}}{\pgfqpoint{7.768198in}{6.914805in}}%
\pgfpathcurveto{\pgfqpoint{7.768198in}{6.903754in}}{\pgfqpoint{7.772588in}{6.893155in}}{\pgfqpoint{7.780402in}{6.885342in}}%
\pgfpathcurveto{\pgfqpoint{7.788215in}{6.877528in}}{\pgfqpoint{7.798814in}{6.873138in}}{\pgfqpoint{7.809864in}{6.873138in}}%
\pgfpathclose%
\pgfusepath{stroke,fill}%
\end{pgfscope}%
\begin{pgfscope}%
\pgfpathrectangle{\pgfqpoint{0.526127in}{0.331635in}}{\pgfqpoint{9.300000in}{7.700000in}}%
\pgfusepath{clip}%
\pgfsetbuttcap%
\pgfsetroundjoin%
\definecolor{currentfill}{rgb}{0.631373,0.788235,0.956863}%
\pgfsetfillcolor{currentfill}%
\pgfsetlinewidth{0.481800pt}%
\definecolor{currentstroke}{rgb}{1.000000,1.000000,1.000000}%
\pgfsetstrokecolor{currentstroke}%
\pgfsetdash{}{0pt}%
\pgfpathmoveto{\pgfqpoint{7.744328in}{2.730779in}}%
\pgfpathcurveto{\pgfqpoint{7.755378in}{2.730779in}}{\pgfqpoint{7.765977in}{2.735169in}}{\pgfqpoint{7.773791in}{2.742983in}}%
\pgfpathcurveto{\pgfqpoint{7.781604in}{2.750796in}}{\pgfqpoint{7.785995in}{2.761395in}}{\pgfqpoint{7.785995in}{2.772445in}}%
\pgfpathcurveto{\pgfqpoint{7.785995in}{2.783495in}}{\pgfqpoint{7.781604in}{2.794095in}}{\pgfqpoint{7.773791in}{2.801908in}}%
\pgfpathcurveto{\pgfqpoint{7.765977in}{2.809722in}}{\pgfqpoint{7.755378in}{2.814112in}}{\pgfqpoint{7.744328in}{2.814112in}}%
\pgfpathcurveto{\pgfqpoint{7.733278in}{2.814112in}}{\pgfqpoint{7.722679in}{2.809722in}}{\pgfqpoint{7.714865in}{2.801908in}}%
\pgfpathcurveto{\pgfqpoint{7.707051in}{2.794095in}}{\pgfqpoint{7.702661in}{2.783495in}}{\pgfqpoint{7.702661in}{2.772445in}}%
\pgfpathcurveto{\pgfqpoint{7.702661in}{2.761395in}}{\pgfqpoint{7.707051in}{2.750796in}}{\pgfqpoint{7.714865in}{2.742983in}}%
\pgfpathcurveto{\pgfqpoint{7.722679in}{2.735169in}}{\pgfqpoint{7.733278in}{2.730779in}}{\pgfqpoint{7.744328in}{2.730779in}}%
\pgfpathclose%
\pgfusepath{stroke,fill}%
\end{pgfscope}%
\begin{pgfscope}%
\pgfpathrectangle{\pgfqpoint{0.526127in}{0.331635in}}{\pgfqpoint{9.300000in}{7.700000in}}%
\pgfusepath{clip}%
\pgfsetbuttcap%
\pgfsetroundjoin%
\definecolor{currentfill}{rgb}{0.631373,0.788235,0.956863}%
\pgfsetfillcolor{currentfill}%
\pgfsetlinewidth{0.481800pt}%
\definecolor{currentstroke}{rgb}{1.000000,1.000000,1.000000}%
\pgfsetstrokecolor{currentstroke}%
\pgfsetdash{}{0pt}%
\pgfpathmoveto{\pgfqpoint{6.917985in}{5.041600in}}%
\pgfpathcurveto{\pgfqpoint{6.929036in}{5.041600in}}{\pgfqpoint{6.939635in}{5.045990in}}{\pgfqpoint{6.947448in}{5.053804in}}%
\pgfpathcurveto{\pgfqpoint{6.955262in}{5.061617in}}{\pgfqpoint{6.959652in}{5.072216in}}{\pgfqpoint{6.959652in}{5.083266in}}%
\pgfpathcurveto{\pgfqpoint{6.959652in}{5.094316in}}{\pgfqpoint{6.955262in}{5.104915in}}{\pgfqpoint{6.947448in}{5.112729in}}%
\pgfpathcurveto{\pgfqpoint{6.939635in}{5.120543in}}{\pgfqpoint{6.929036in}{5.124933in}}{\pgfqpoint{6.917985in}{5.124933in}}%
\pgfpathcurveto{\pgfqpoint{6.906935in}{5.124933in}}{\pgfqpoint{6.896336in}{5.120543in}}{\pgfqpoint{6.888523in}{5.112729in}}%
\pgfpathcurveto{\pgfqpoint{6.880709in}{5.104915in}}{\pgfqpoint{6.876319in}{5.094316in}}{\pgfqpoint{6.876319in}{5.083266in}}%
\pgfpathcurveto{\pgfqpoint{6.876319in}{5.072216in}}{\pgfqpoint{6.880709in}{5.061617in}}{\pgfqpoint{6.888523in}{5.053804in}}%
\pgfpathcurveto{\pgfqpoint{6.896336in}{5.045990in}}{\pgfqpoint{6.906935in}{5.041600in}}{\pgfqpoint{6.917985in}{5.041600in}}%
\pgfpathclose%
\pgfusepath{stroke,fill}%
\end{pgfscope}%
\begin{pgfscope}%
\pgfpathrectangle{\pgfqpoint{0.526127in}{0.331635in}}{\pgfqpoint{9.300000in}{7.700000in}}%
\pgfusepath{clip}%
\pgfsetbuttcap%
\pgfsetroundjoin%
\definecolor{currentfill}{rgb}{0.631373,0.788235,0.956863}%
\pgfsetfillcolor{currentfill}%
\pgfsetlinewidth{0.481800pt}%
\definecolor{currentstroke}{rgb}{1.000000,1.000000,1.000000}%
\pgfsetstrokecolor{currentstroke}%
\pgfsetdash{}{0pt}%
\pgfpathmoveto{\pgfqpoint{5.740650in}{6.378204in}}%
\pgfpathcurveto{\pgfqpoint{5.751700in}{6.378204in}}{\pgfqpoint{5.762299in}{6.382595in}}{\pgfqpoint{5.770113in}{6.390408in}}%
\pgfpathcurveto{\pgfqpoint{5.777926in}{6.398222in}}{\pgfqpoint{5.782317in}{6.408821in}}{\pgfqpoint{5.782317in}{6.419871in}}%
\pgfpathcurveto{\pgfqpoint{5.782317in}{6.430921in}}{\pgfqpoint{5.777926in}{6.441520in}}{\pgfqpoint{5.770113in}{6.449334in}}%
\pgfpathcurveto{\pgfqpoint{5.762299in}{6.457148in}}{\pgfqpoint{5.751700in}{6.461538in}}{\pgfqpoint{5.740650in}{6.461538in}}%
\pgfpathcurveto{\pgfqpoint{5.729600in}{6.461538in}}{\pgfqpoint{5.719001in}{6.457148in}}{\pgfqpoint{5.711187in}{6.449334in}}%
\pgfpathcurveto{\pgfqpoint{5.703374in}{6.441520in}}{\pgfqpoint{5.698983in}{6.430921in}}{\pgfqpoint{5.698983in}{6.419871in}}%
\pgfpathcurveto{\pgfqpoint{5.698983in}{6.408821in}}{\pgfqpoint{5.703374in}{6.398222in}}{\pgfqpoint{5.711187in}{6.390408in}}%
\pgfpathcurveto{\pgfqpoint{5.719001in}{6.382595in}}{\pgfqpoint{5.729600in}{6.378204in}}{\pgfqpoint{5.740650in}{6.378204in}}%
\pgfpathclose%
\pgfusepath{stroke,fill}%
\end{pgfscope}%
\begin{pgfscope}%
\pgfpathrectangle{\pgfqpoint{0.526127in}{0.331635in}}{\pgfqpoint{9.300000in}{7.700000in}}%
\pgfusepath{clip}%
\pgfsetbuttcap%
\pgfsetroundjoin%
\definecolor{currentfill}{rgb}{0.631373,0.788235,0.956863}%
\pgfsetfillcolor{currentfill}%
\pgfsetlinewidth{0.481800pt}%
\definecolor{currentstroke}{rgb}{1.000000,1.000000,1.000000}%
\pgfsetstrokecolor{currentstroke}%
\pgfsetdash{}{0pt}%
\pgfpathmoveto{\pgfqpoint{7.681588in}{6.050549in}}%
\pgfpathcurveto{\pgfqpoint{7.692638in}{6.050549in}}{\pgfqpoint{7.703237in}{6.054939in}}{\pgfqpoint{7.711051in}{6.062753in}}%
\pgfpathcurveto{\pgfqpoint{7.718865in}{6.070566in}}{\pgfqpoint{7.723255in}{6.081165in}}{\pgfqpoint{7.723255in}{6.092216in}}%
\pgfpathcurveto{\pgfqpoint{7.723255in}{6.103266in}}{\pgfqpoint{7.718865in}{6.113865in}}{\pgfqpoint{7.711051in}{6.121678in}}%
\pgfpathcurveto{\pgfqpoint{7.703237in}{6.129492in}}{\pgfqpoint{7.692638in}{6.133882in}}{\pgfqpoint{7.681588in}{6.133882in}}%
\pgfpathcurveto{\pgfqpoint{7.670538in}{6.133882in}}{\pgfqpoint{7.659939in}{6.129492in}}{\pgfqpoint{7.652126in}{6.121678in}}%
\pgfpathcurveto{\pgfqpoint{7.644312in}{6.113865in}}{\pgfqpoint{7.639922in}{6.103266in}}{\pgfqpoint{7.639922in}{6.092216in}}%
\pgfpathcurveto{\pgfqpoint{7.639922in}{6.081165in}}{\pgfqpoint{7.644312in}{6.070566in}}{\pgfqpoint{7.652126in}{6.062753in}}%
\pgfpathcurveto{\pgfqpoint{7.659939in}{6.054939in}}{\pgfqpoint{7.670538in}{6.050549in}}{\pgfqpoint{7.681588in}{6.050549in}}%
\pgfpathclose%
\pgfusepath{stroke,fill}%
\end{pgfscope}%
\begin{pgfscope}%
\pgfpathrectangle{\pgfqpoint{0.526127in}{0.331635in}}{\pgfqpoint{9.300000in}{7.700000in}}%
\pgfusepath{clip}%
\pgfsetbuttcap%
\pgfsetroundjoin%
\definecolor{currentfill}{rgb}{0.631373,0.788235,0.956863}%
\pgfsetfillcolor{currentfill}%
\pgfsetlinewidth{0.481800pt}%
\definecolor{currentstroke}{rgb}{1.000000,1.000000,1.000000}%
\pgfsetstrokecolor{currentstroke}%
\pgfsetdash{}{0pt}%
\pgfpathmoveto{\pgfqpoint{6.865837in}{4.587417in}}%
\pgfpathcurveto{\pgfqpoint{6.876887in}{4.587417in}}{\pgfqpoint{6.887486in}{4.591808in}}{\pgfqpoint{6.895300in}{4.599621in}}%
\pgfpathcurveto{\pgfqpoint{6.903113in}{4.607435in}}{\pgfqpoint{6.907504in}{4.618034in}}{\pgfqpoint{6.907504in}{4.629084in}}%
\pgfpathcurveto{\pgfqpoint{6.907504in}{4.640134in}}{\pgfqpoint{6.903113in}{4.650733in}}{\pgfqpoint{6.895300in}{4.658547in}}%
\pgfpathcurveto{\pgfqpoint{6.887486in}{4.666360in}}{\pgfqpoint{6.876887in}{4.670751in}}{\pgfqpoint{6.865837in}{4.670751in}}%
\pgfpathcurveto{\pgfqpoint{6.854787in}{4.670751in}}{\pgfqpoint{6.844188in}{4.666360in}}{\pgfqpoint{6.836374in}{4.658547in}}%
\pgfpathcurveto{\pgfqpoint{6.828561in}{4.650733in}}{\pgfqpoint{6.824170in}{4.640134in}}{\pgfqpoint{6.824170in}{4.629084in}}%
\pgfpathcurveto{\pgfqpoint{6.824170in}{4.618034in}}{\pgfqpoint{6.828561in}{4.607435in}}{\pgfqpoint{6.836374in}{4.599621in}}%
\pgfpathcurveto{\pgfqpoint{6.844188in}{4.591808in}}{\pgfqpoint{6.854787in}{4.587417in}}{\pgfqpoint{6.865837in}{4.587417in}}%
\pgfpathclose%
\pgfusepath{stroke,fill}%
\end{pgfscope}%
\begin{pgfscope}%
\pgfpathrectangle{\pgfqpoint{0.526127in}{0.331635in}}{\pgfqpoint{9.300000in}{7.700000in}}%
\pgfusepath{clip}%
\pgfsetbuttcap%
\pgfsetroundjoin%
\definecolor{currentfill}{rgb}{0.631373,0.788235,0.956863}%
\pgfsetfillcolor{currentfill}%
\pgfsetlinewidth{0.481800pt}%
\definecolor{currentstroke}{rgb}{1.000000,1.000000,1.000000}%
\pgfsetstrokecolor{currentstroke}%
\pgfsetdash{}{0pt}%
\pgfpathmoveto{\pgfqpoint{7.703805in}{3.254938in}}%
\pgfpathcurveto{\pgfqpoint{7.714856in}{3.254938in}}{\pgfqpoint{7.725455in}{3.259328in}}{\pgfqpoint{7.733268in}{3.267142in}}%
\pgfpathcurveto{\pgfqpoint{7.741082in}{3.274956in}}{\pgfqpoint{7.745472in}{3.285555in}}{\pgfqpoint{7.745472in}{3.296605in}}%
\pgfpathcurveto{\pgfqpoint{7.745472in}{3.307655in}}{\pgfqpoint{7.741082in}{3.318254in}}{\pgfqpoint{7.733268in}{3.326068in}}%
\pgfpathcurveto{\pgfqpoint{7.725455in}{3.333881in}}{\pgfqpoint{7.714856in}{3.338271in}}{\pgfqpoint{7.703805in}{3.338271in}}%
\pgfpathcurveto{\pgfqpoint{7.692755in}{3.338271in}}{\pgfqpoint{7.682156in}{3.333881in}}{\pgfqpoint{7.674343in}{3.326068in}}%
\pgfpathcurveto{\pgfqpoint{7.666529in}{3.318254in}}{\pgfqpoint{7.662139in}{3.307655in}}{\pgfqpoint{7.662139in}{3.296605in}}%
\pgfpathcurveto{\pgfqpoint{7.662139in}{3.285555in}}{\pgfqpoint{7.666529in}{3.274956in}}{\pgfqpoint{7.674343in}{3.267142in}}%
\pgfpathcurveto{\pgfqpoint{7.682156in}{3.259328in}}{\pgfqpoint{7.692755in}{3.254938in}}{\pgfqpoint{7.703805in}{3.254938in}}%
\pgfpathclose%
\pgfusepath{stroke,fill}%
\end{pgfscope}%
\begin{pgfscope}%
\pgfpathrectangle{\pgfqpoint{0.526127in}{0.331635in}}{\pgfqpoint{9.300000in}{7.700000in}}%
\pgfusepath{clip}%
\pgfsetbuttcap%
\pgfsetroundjoin%
\definecolor{currentfill}{rgb}{0.631373,0.788235,0.956863}%
\pgfsetfillcolor{currentfill}%
\pgfsetlinewidth{0.481800pt}%
\definecolor{currentstroke}{rgb}{1.000000,1.000000,1.000000}%
\pgfsetstrokecolor{currentstroke}%
\pgfsetdash{}{0pt}%
\pgfpathmoveto{\pgfqpoint{2.458995in}{0.988152in}}%
\pgfpathcurveto{\pgfqpoint{2.470045in}{0.988152in}}{\pgfqpoint{2.480644in}{0.992543in}}{\pgfqpoint{2.488458in}{1.000356in}}%
\pgfpathcurveto{\pgfqpoint{2.496271in}{1.008170in}}{\pgfqpoint{2.500662in}{1.018769in}}{\pgfqpoint{2.500662in}{1.029819in}}%
\pgfpathcurveto{\pgfqpoint{2.500662in}{1.040869in}}{\pgfqpoint{2.496271in}{1.051468in}}{\pgfqpoint{2.488458in}{1.059282in}}%
\pgfpathcurveto{\pgfqpoint{2.480644in}{1.067095in}}{\pgfqpoint{2.470045in}{1.071486in}}{\pgfqpoint{2.458995in}{1.071486in}}%
\pgfpathcurveto{\pgfqpoint{2.447945in}{1.071486in}}{\pgfqpoint{2.437346in}{1.067095in}}{\pgfqpoint{2.429532in}{1.059282in}}%
\pgfpathcurveto{\pgfqpoint{2.421719in}{1.051468in}}{\pgfqpoint{2.417328in}{1.040869in}}{\pgfqpoint{2.417328in}{1.029819in}}%
\pgfpathcurveto{\pgfqpoint{2.417328in}{1.018769in}}{\pgfqpoint{2.421719in}{1.008170in}}{\pgfqpoint{2.429532in}{1.000356in}}%
\pgfpathcurveto{\pgfqpoint{2.437346in}{0.992543in}}{\pgfqpoint{2.447945in}{0.988152in}}{\pgfqpoint{2.458995in}{0.988152in}}%
\pgfpathclose%
\pgfusepath{stroke,fill}%
\end{pgfscope}%
\begin{pgfscope}%
\pgfpathrectangle{\pgfqpoint{0.526127in}{0.331635in}}{\pgfqpoint{9.300000in}{7.700000in}}%
\pgfusepath{clip}%
\pgfsetbuttcap%
\pgfsetroundjoin%
\definecolor{currentfill}{rgb}{0.631373,0.788235,0.956863}%
\pgfsetfillcolor{currentfill}%
\pgfsetlinewidth{0.481800pt}%
\definecolor{currentstroke}{rgb}{1.000000,1.000000,1.000000}%
\pgfsetstrokecolor{currentstroke}%
\pgfsetdash{}{0pt}%
\pgfpathmoveto{\pgfqpoint{4.621174in}{4.016137in}}%
\pgfpathcurveto{\pgfqpoint{4.632224in}{4.016137in}}{\pgfqpoint{4.642823in}{4.020527in}}{\pgfqpoint{4.650636in}{4.028341in}}%
\pgfpathcurveto{\pgfqpoint{4.658450in}{4.036155in}}{\pgfqpoint{4.662840in}{4.046754in}}{\pgfqpoint{4.662840in}{4.057804in}}%
\pgfpathcurveto{\pgfqpoint{4.662840in}{4.068854in}}{\pgfqpoint{4.658450in}{4.079453in}}{\pgfqpoint{4.650636in}{4.087266in}}%
\pgfpathcurveto{\pgfqpoint{4.642823in}{4.095080in}}{\pgfqpoint{4.632224in}{4.099470in}}{\pgfqpoint{4.621174in}{4.099470in}}%
\pgfpathcurveto{\pgfqpoint{4.610123in}{4.099470in}}{\pgfqpoint{4.599524in}{4.095080in}}{\pgfqpoint{4.591711in}{4.087266in}}%
\pgfpathcurveto{\pgfqpoint{4.583897in}{4.079453in}}{\pgfqpoint{4.579507in}{4.068854in}}{\pgfqpoint{4.579507in}{4.057804in}}%
\pgfpathcurveto{\pgfqpoint{4.579507in}{4.046754in}}{\pgfqpoint{4.583897in}{4.036155in}}{\pgfqpoint{4.591711in}{4.028341in}}%
\pgfpathcurveto{\pgfqpoint{4.599524in}{4.020527in}}{\pgfqpoint{4.610123in}{4.016137in}}{\pgfqpoint{4.621174in}{4.016137in}}%
\pgfpathclose%
\pgfusepath{stroke,fill}%
\end{pgfscope}%
\begin{pgfscope}%
\pgfpathrectangle{\pgfqpoint{0.526127in}{0.331635in}}{\pgfqpoint{9.300000in}{7.700000in}}%
\pgfusepath{clip}%
\pgfsetbuttcap%
\pgfsetroundjoin%
\definecolor{currentfill}{rgb}{0.631373,0.788235,0.956863}%
\pgfsetfillcolor{currentfill}%
\pgfsetlinewidth{0.481800pt}%
\definecolor{currentstroke}{rgb}{1.000000,1.000000,1.000000}%
\pgfsetstrokecolor{currentstroke}%
\pgfsetdash{}{0pt}%
\pgfpathmoveto{\pgfqpoint{5.970043in}{6.856085in}}%
\pgfpathcurveto{\pgfqpoint{5.981093in}{6.856085in}}{\pgfqpoint{5.991692in}{6.860475in}}{\pgfqpoint{5.999505in}{6.868289in}}%
\pgfpathcurveto{\pgfqpoint{6.007319in}{6.876102in}}{\pgfqpoint{6.011709in}{6.886701in}}{\pgfqpoint{6.011709in}{6.897752in}}%
\pgfpathcurveto{\pgfqpoint{6.011709in}{6.908802in}}{\pgfqpoint{6.007319in}{6.919401in}}{\pgfqpoint{5.999505in}{6.927214in}}%
\pgfpathcurveto{\pgfqpoint{5.991692in}{6.935028in}}{\pgfqpoint{5.981093in}{6.939418in}}{\pgfqpoint{5.970043in}{6.939418in}}%
\pgfpathcurveto{\pgfqpoint{5.958992in}{6.939418in}}{\pgfqpoint{5.948393in}{6.935028in}}{\pgfqpoint{5.940580in}{6.927214in}}%
\pgfpathcurveto{\pgfqpoint{5.932766in}{6.919401in}}{\pgfqpoint{5.928376in}{6.908802in}}{\pgfqpoint{5.928376in}{6.897752in}}%
\pgfpathcurveto{\pgfqpoint{5.928376in}{6.886701in}}{\pgfqpoint{5.932766in}{6.876102in}}{\pgfqpoint{5.940580in}{6.868289in}}%
\pgfpathcurveto{\pgfqpoint{5.948393in}{6.860475in}}{\pgfqpoint{5.958992in}{6.856085in}}{\pgfqpoint{5.970043in}{6.856085in}}%
\pgfpathclose%
\pgfusepath{stroke,fill}%
\end{pgfscope}%
\begin{pgfscope}%
\pgfpathrectangle{\pgfqpoint{0.526127in}{0.331635in}}{\pgfqpoint{9.300000in}{7.700000in}}%
\pgfusepath{clip}%
\pgfsetbuttcap%
\pgfsetroundjoin%
\definecolor{currentfill}{rgb}{0.631373,0.788235,0.956863}%
\pgfsetfillcolor{currentfill}%
\pgfsetlinewidth{0.481800pt}%
\definecolor{currentstroke}{rgb}{1.000000,1.000000,1.000000}%
\pgfsetstrokecolor{currentstroke}%
\pgfsetdash{}{0pt}%
\pgfpathmoveto{\pgfqpoint{5.368377in}{3.368977in}}%
\pgfpathcurveto{\pgfqpoint{5.379427in}{3.368977in}}{\pgfqpoint{5.390026in}{3.373367in}}{\pgfqpoint{5.397840in}{3.381181in}}%
\pgfpathcurveto{\pgfqpoint{5.405654in}{3.388994in}}{\pgfqpoint{5.410044in}{3.399593in}}{\pgfqpoint{5.410044in}{3.410644in}}%
\pgfpathcurveto{\pgfqpoint{5.410044in}{3.421694in}}{\pgfqpoint{5.405654in}{3.432293in}}{\pgfqpoint{5.397840in}{3.440106in}}%
\pgfpathcurveto{\pgfqpoint{5.390026in}{3.447920in}}{\pgfqpoint{5.379427in}{3.452310in}}{\pgfqpoint{5.368377in}{3.452310in}}%
\pgfpathcurveto{\pgfqpoint{5.357327in}{3.452310in}}{\pgfqpoint{5.346728in}{3.447920in}}{\pgfqpoint{5.338915in}{3.440106in}}%
\pgfpathcurveto{\pgfqpoint{5.331101in}{3.432293in}}{\pgfqpoint{5.326711in}{3.421694in}}{\pgfqpoint{5.326711in}{3.410644in}}%
\pgfpathcurveto{\pgfqpoint{5.326711in}{3.399593in}}{\pgfqpoint{5.331101in}{3.388994in}}{\pgfqpoint{5.338915in}{3.381181in}}%
\pgfpathcurveto{\pgfqpoint{5.346728in}{3.373367in}}{\pgfqpoint{5.357327in}{3.368977in}}{\pgfqpoint{5.368377in}{3.368977in}}%
\pgfpathclose%
\pgfusepath{stroke,fill}%
\end{pgfscope}%
\begin{pgfscope}%
\pgfpathrectangle{\pgfqpoint{0.526127in}{0.331635in}}{\pgfqpoint{9.300000in}{7.700000in}}%
\pgfusepath{clip}%
\pgfsetbuttcap%
\pgfsetroundjoin%
\definecolor{currentfill}{rgb}{0.631373,0.788235,0.956863}%
\pgfsetfillcolor{currentfill}%
\pgfsetlinewidth{0.481800pt}%
\definecolor{currentstroke}{rgb}{1.000000,1.000000,1.000000}%
\pgfsetstrokecolor{currentstroke}%
\pgfsetdash{}{0pt}%
\pgfpathmoveto{\pgfqpoint{6.364240in}{6.562434in}}%
\pgfpathcurveto{\pgfqpoint{6.375290in}{6.562434in}}{\pgfqpoint{6.385889in}{6.566825in}}{\pgfqpoint{6.393703in}{6.574638in}}%
\pgfpathcurveto{\pgfqpoint{6.401517in}{6.582452in}}{\pgfqpoint{6.405907in}{6.593051in}}{\pgfqpoint{6.405907in}{6.604101in}}%
\pgfpathcurveto{\pgfqpoint{6.405907in}{6.615151in}}{\pgfqpoint{6.401517in}{6.625750in}}{\pgfqpoint{6.393703in}{6.633564in}}%
\pgfpathcurveto{\pgfqpoint{6.385889in}{6.641378in}}{\pgfqpoint{6.375290in}{6.645768in}}{\pgfqpoint{6.364240in}{6.645768in}}%
\pgfpathcurveto{\pgfqpoint{6.353190in}{6.645768in}}{\pgfqpoint{6.342591in}{6.641378in}}{\pgfqpoint{6.334778in}{6.633564in}}%
\pgfpathcurveto{\pgfqpoint{6.326964in}{6.625750in}}{\pgfqpoint{6.322574in}{6.615151in}}{\pgfqpoint{6.322574in}{6.604101in}}%
\pgfpathcurveto{\pgfqpoint{6.322574in}{6.593051in}}{\pgfqpoint{6.326964in}{6.582452in}}{\pgfqpoint{6.334778in}{6.574638in}}%
\pgfpathcurveto{\pgfqpoint{6.342591in}{6.566825in}}{\pgfqpoint{6.353190in}{6.562434in}}{\pgfqpoint{6.364240in}{6.562434in}}%
\pgfpathclose%
\pgfusepath{stroke,fill}%
\end{pgfscope}%
\begin{pgfscope}%
\pgfpathrectangle{\pgfqpoint{0.526127in}{0.331635in}}{\pgfqpoint{9.300000in}{7.700000in}}%
\pgfusepath{clip}%
\pgfsetbuttcap%
\pgfsetroundjoin%
\definecolor{currentfill}{rgb}{0.631373,0.788235,0.956863}%
\pgfsetfillcolor{currentfill}%
\pgfsetlinewidth{0.481800pt}%
\definecolor{currentstroke}{rgb}{1.000000,1.000000,1.000000}%
\pgfsetstrokecolor{currentstroke}%
\pgfsetdash{}{0pt}%
\pgfpathmoveto{\pgfqpoint{6.258495in}{6.773269in}}%
\pgfpathcurveto{\pgfqpoint{6.269546in}{6.773269in}}{\pgfqpoint{6.280145in}{6.777659in}}{\pgfqpoint{6.287958in}{6.785472in}}%
\pgfpathcurveto{\pgfqpoint{6.295772in}{6.793286in}}{\pgfqpoint{6.300162in}{6.803885in}}{\pgfqpoint{6.300162in}{6.814935in}}%
\pgfpathcurveto{\pgfqpoint{6.300162in}{6.825985in}}{\pgfqpoint{6.295772in}{6.836584in}}{\pgfqpoint{6.287958in}{6.844398in}}%
\pgfpathcurveto{\pgfqpoint{6.280145in}{6.852212in}}{\pgfqpoint{6.269546in}{6.856602in}}{\pgfqpoint{6.258495in}{6.856602in}}%
\pgfpathcurveto{\pgfqpoint{6.247445in}{6.856602in}}{\pgfqpoint{6.236846in}{6.852212in}}{\pgfqpoint{6.229033in}{6.844398in}}%
\pgfpathcurveto{\pgfqpoint{6.221219in}{6.836584in}}{\pgfqpoint{6.216829in}{6.825985in}}{\pgfqpoint{6.216829in}{6.814935in}}%
\pgfpathcurveto{\pgfqpoint{6.216829in}{6.803885in}}{\pgfqpoint{6.221219in}{6.793286in}}{\pgfqpoint{6.229033in}{6.785472in}}%
\pgfpathcurveto{\pgfqpoint{6.236846in}{6.777659in}}{\pgfqpoint{6.247445in}{6.773269in}}{\pgfqpoint{6.258495in}{6.773269in}}%
\pgfpathclose%
\pgfusepath{stroke,fill}%
\end{pgfscope}%
\begin{pgfscope}%
\pgfpathrectangle{\pgfqpoint{0.526127in}{0.331635in}}{\pgfqpoint{9.300000in}{7.700000in}}%
\pgfusepath{clip}%
\pgfsetbuttcap%
\pgfsetroundjoin%
\definecolor{currentfill}{rgb}{0.631373,0.788235,0.956863}%
\pgfsetfillcolor{currentfill}%
\pgfsetlinewidth{0.481800pt}%
\definecolor{currentstroke}{rgb}{1.000000,1.000000,1.000000}%
\pgfsetstrokecolor{currentstroke}%
\pgfsetdash{}{0pt}%
\pgfpathmoveto{\pgfqpoint{7.197145in}{4.652940in}}%
\pgfpathcurveto{\pgfqpoint{7.208195in}{4.652940in}}{\pgfqpoint{7.218794in}{4.657330in}}{\pgfqpoint{7.226607in}{4.665144in}}%
\pgfpathcurveto{\pgfqpoint{7.234421in}{4.672958in}}{\pgfqpoint{7.238811in}{4.683557in}}{\pgfqpoint{7.238811in}{4.694607in}}%
\pgfpathcurveto{\pgfqpoint{7.238811in}{4.705657in}}{\pgfqpoint{7.234421in}{4.716256in}}{\pgfqpoint{7.226607in}{4.724069in}}%
\pgfpathcurveto{\pgfqpoint{7.218794in}{4.731883in}}{\pgfqpoint{7.208195in}{4.736273in}}{\pgfqpoint{7.197145in}{4.736273in}}%
\pgfpathcurveto{\pgfqpoint{7.186095in}{4.736273in}}{\pgfqpoint{7.175496in}{4.731883in}}{\pgfqpoint{7.167682in}{4.724069in}}%
\pgfpathcurveto{\pgfqpoint{7.159868in}{4.716256in}}{\pgfqpoint{7.155478in}{4.705657in}}{\pgfqpoint{7.155478in}{4.694607in}}%
\pgfpathcurveto{\pgfqpoint{7.155478in}{4.683557in}}{\pgfqpoint{7.159868in}{4.672958in}}{\pgfqpoint{7.167682in}{4.665144in}}%
\pgfpathcurveto{\pgfqpoint{7.175496in}{4.657330in}}{\pgfqpoint{7.186095in}{4.652940in}}{\pgfqpoint{7.197145in}{4.652940in}}%
\pgfpathclose%
\pgfusepath{stroke,fill}%
\end{pgfscope}%
\begin{pgfscope}%
\pgfpathrectangle{\pgfqpoint{0.526127in}{0.331635in}}{\pgfqpoint{9.300000in}{7.700000in}}%
\pgfusepath{clip}%
\pgfsetbuttcap%
\pgfsetroundjoin%
\definecolor{currentfill}{rgb}{0.631373,0.788235,0.956863}%
\pgfsetfillcolor{currentfill}%
\pgfsetlinewidth{0.481800pt}%
\definecolor{currentstroke}{rgb}{1.000000,1.000000,1.000000}%
\pgfsetstrokecolor{currentstroke}%
\pgfsetdash{}{0pt}%
\pgfpathmoveto{\pgfqpoint{7.051687in}{6.205810in}}%
\pgfpathcurveto{\pgfqpoint{7.062738in}{6.205810in}}{\pgfqpoint{7.073337in}{6.210200in}}{\pgfqpoint{7.081150in}{6.218013in}}%
\pgfpathcurveto{\pgfqpoint{7.088964in}{6.225827in}}{\pgfqpoint{7.093354in}{6.236426in}}{\pgfqpoint{7.093354in}{6.247476in}}%
\pgfpathcurveto{\pgfqpoint{7.093354in}{6.258526in}}{\pgfqpoint{7.088964in}{6.269125in}}{\pgfqpoint{7.081150in}{6.276939in}}%
\pgfpathcurveto{\pgfqpoint{7.073337in}{6.284753in}}{\pgfqpoint{7.062738in}{6.289143in}}{\pgfqpoint{7.051687in}{6.289143in}}%
\pgfpathcurveto{\pgfqpoint{7.040637in}{6.289143in}}{\pgfqpoint{7.030038in}{6.284753in}}{\pgfqpoint{7.022225in}{6.276939in}}%
\pgfpathcurveto{\pgfqpoint{7.014411in}{6.269125in}}{\pgfqpoint{7.010021in}{6.258526in}}{\pgfqpoint{7.010021in}{6.247476in}}%
\pgfpathcurveto{\pgfqpoint{7.010021in}{6.236426in}}{\pgfqpoint{7.014411in}{6.225827in}}{\pgfqpoint{7.022225in}{6.218013in}}%
\pgfpathcurveto{\pgfqpoint{7.030038in}{6.210200in}}{\pgfqpoint{7.040637in}{6.205810in}}{\pgfqpoint{7.051687in}{6.205810in}}%
\pgfpathclose%
\pgfusepath{stroke,fill}%
\end{pgfscope}%
\begin{pgfscope}%
\pgfpathrectangle{\pgfqpoint{0.526127in}{0.331635in}}{\pgfqpoint{9.300000in}{7.700000in}}%
\pgfusepath{clip}%
\pgfsetbuttcap%
\pgfsetroundjoin%
\definecolor{currentfill}{rgb}{0.631373,0.788235,0.956863}%
\pgfsetfillcolor{currentfill}%
\pgfsetlinewidth{0.481800pt}%
\definecolor{currentstroke}{rgb}{1.000000,1.000000,1.000000}%
\pgfsetstrokecolor{currentstroke}%
\pgfsetdash{}{0pt}%
\pgfpathmoveto{\pgfqpoint{6.775438in}{7.196897in}}%
\pgfpathcurveto{\pgfqpoint{6.786488in}{7.196897in}}{\pgfqpoint{6.797087in}{7.201287in}}{\pgfqpoint{6.804901in}{7.209101in}}%
\pgfpathcurveto{\pgfqpoint{6.812715in}{7.216914in}}{\pgfqpoint{6.817105in}{7.227513in}}{\pgfqpoint{6.817105in}{7.238564in}}%
\pgfpathcurveto{\pgfqpoint{6.817105in}{7.249614in}}{\pgfqpoint{6.812715in}{7.260213in}}{\pgfqpoint{6.804901in}{7.268026in}}%
\pgfpathcurveto{\pgfqpoint{6.797087in}{7.275840in}}{\pgfqpoint{6.786488in}{7.280230in}}{\pgfqpoint{6.775438in}{7.280230in}}%
\pgfpathcurveto{\pgfqpoint{6.764388in}{7.280230in}}{\pgfqpoint{6.753789in}{7.275840in}}{\pgfqpoint{6.745975in}{7.268026in}}%
\pgfpathcurveto{\pgfqpoint{6.738162in}{7.260213in}}{\pgfqpoint{6.733772in}{7.249614in}}{\pgfqpoint{6.733772in}{7.238564in}}%
\pgfpathcurveto{\pgfqpoint{6.733772in}{7.227513in}}{\pgfqpoint{6.738162in}{7.216914in}}{\pgfqpoint{6.745975in}{7.209101in}}%
\pgfpathcurveto{\pgfqpoint{6.753789in}{7.201287in}}{\pgfqpoint{6.764388in}{7.196897in}}{\pgfqpoint{6.775438in}{7.196897in}}%
\pgfpathclose%
\pgfusepath{stroke,fill}%
\end{pgfscope}%
\begin{pgfscope}%
\pgfpathrectangle{\pgfqpoint{0.526127in}{0.331635in}}{\pgfqpoint{9.300000in}{7.700000in}}%
\pgfusepath{clip}%
\pgfsetbuttcap%
\pgfsetroundjoin%
\definecolor{currentfill}{rgb}{0.631373,0.788235,0.956863}%
\pgfsetfillcolor{currentfill}%
\pgfsetlinewidth{0.481800pt}%
\definecolor{currentstroke}{rgb}{1.000000,1.000000,1.000000}%
\pgfsetstrokecolor{currentstroke}%
\pgfsetdash{}{0pt}%
\pgfpathmoveto{\pgfqpoint{4.179928in}{4.064430in}}%
\pgfpathcurveto{\pgfqpoint{4.190978in}{4.064430in}}{\pgfqpoint{4.201577in}{4.068820in}}{\pgfqpoint{4.209391in}{4.076634in}}%
\pgfpathcurveto{\pgfqpoint{4.217204in}{4.084447in}}{\pgfqpoint{4.221595in}{4.095046in}}{\pgfqpoint{4.221595in}{4.106096in}}%
\pgfpathcurveto{\pgfqpoint{4.221595in}{4.117147in}}{\pgfqpoint{4.217204in}{4.127746in}}{\pgfqpoint{4.209391in}{4.135559in}}%
\pgfpathcurveto{\pgfqpoint{4.201577in}{4.143373in}}{\pgfqpoint{4.190978in}{4.147763in}}{\pgfqpoint{4.179928in}{4.147763in}}%
\pgfpathcurveto{\pgfqpoint{4.168878in}{4.147763in}}{\pgfqpoint{4.158279in}{4.143373in}}{\pgfqpoint{4.150465in}{4.135559in}}%
\pgfpathcurveto{\pgfqpoint{4.142652in}{4.127746in}}{\pgfqpoint{4.138261in}{4.117147in}}{\pgfqpoint{4.138261in}{4.106096in}}%
\pgfpathcurveto{\pgfqpoint{4.138261in}{4.095046in}}{\pgfqpoint{4.142652in}{4.084447in}}{\pgfqpoint{4.150465in}{4.076634in}}%
\pgfpathcurveto{\pgfqpoint{4.158279in}{4.068820in}}{\pgfqpoint{4.168878in}{4.064430in}}{\pgfqpoint{4.179928in}{4.064430in}}%
\pgfpathclose%
\pgfusepath{stroke,fill}%
\end{pgfscope}%
\begin{pgfscope}%
\pgfpathrectangle{\pgfqpoint{0.526127in}{0.331635in}}{\pgfqpoint{9.300000in}{7.700000in}}%
\pgfusepath{clip}%
\pgfsetbuttcap%
\pgfsetroundjoin%
\definecolor{currentfill}{rgb}{0.631373,0.788235,0.956863}%
\pgfsetfillcolor{currentfill}%
\pgfsetlinewidth{0.481800pt}%
\definecolor{currentstroke}{rgb}{1.000000,1.000000,1.000000}%
\pgfsetstrokecolor{currentstroke}%
\pgfsetdash{}{0pt}%
\pgfpathmoveto{\pgfqpoint{8.098361in}{3.028530in}}%
\pgfpathcurveto{\pgfqpoint{8.109412in}{3.028530in}}{\pgfqpoint{8.120011in}{3.032920in}}{\pgfqpoint{8.127824in}{3.040734in}}%
\pgfpathcurveto{\pgfqpoint{8.135638in}{3.048548in}}{\pgfqpoint{8.140028in}{3.059147in}}{\pgfqpoint{8.140028in}{3.070197in}}%
\pgfpathcurveto{\pgfqpoint{8.140028in}{3.081247in}}{\pgfqpoint{8.135638in}{3.091846in}}{\pgfqpoint{8.127824in}{3.099660in}}%
\pgfpathcurveto{\pgfqpoint{8.120011in}{3.107473in}}{\pgfqpoint{8.109412in}{3.111864in}}{\pgfqpoint{8.098361in}{3.111864in}}%
\pgfpathcurveto{\pgfqpoint{8.087311in}{3.111864in}}{\pgfqpoint{8.076712in}{3.107473in}}{\pgfqpoint{8.068899in}{3.099660in}}%
\pgfpathcurveto{\pgfqpoint{8.061085in}{3.091846in}}{\pgfqpoint{8.056695in}{3.081247in}}{\pgfqpoint{8.056695in}{3.070197in}}%
\pgfpathcurveto{\pgfqpoint{8.056695in}{3.059147in}}{\pgfqpoint{8.061085in}{3.048548in}}{\pgfqpoint{8.068899in}{3.040734in}}%
\pgfpathcurveto{\pgfqpoint{8.076712in}{3.032920in}}{\pgfqpoint{8.087311in}{3.028530in}}{\pgfqpoint{8.098361in}{3.028530in}}%
\pgfpathclose%
\pgfusepath{stroke,fill}%
\end{pgfscope}%
\begin{pgfscope}%
\pgfpathrectangle{\pgfqpoint{0.526127in}{0.331635in}}{\pgfqpoint{9.300000in}{7.700000in}}%
\pgfusepath{clip}%
\pgfsetbuttcap%
\pgfsetroundjoin%
\definecolor{currentfill}{rgb}{1.000000,0.705882,0.509804}%
\pgfsetfillcolor{currentfill}%
\pgfsetlinewidth{0.481800pt}%
\definecolor{currentstroke}{rgb}{1.000000,1.000000,1.000000}%
\pgfsetstrokecolor{currentstroke}%
\pgfsetdash{}{0pt}%
\pgfpathmoveto{\pgfqpoint{6.611324in}{4.087918in}}%
\pgfpathcurveto{\pgfqpoint{6.622374in}{4.087918in}}{\pgfqpoint{6.632973in}{4.092308in}}{\pgfqpoint{6.640787in}{4.100122in}}%
\pgfpathcurveto{\pgfqpoint{6.648601in}{4.107935in}}{\pgfqpoint{6.652991in}{4.118534in}}{\pgfqpoint{6.652991in}{4.129584in}}%
\pgfpathcurveto{\pgfqpoint{6.652991in}{4.140634in}}{\pgfqpoint{6.648601in}{4.151234in}}{\pgfqpoint{6.640787in}{4.159047in}}%
\pgfpathcurveto{\pgfqpoint{6.632973in}{4.166861in}}{\pgfqpoint{6.622374in}{4.171251in}}{\pgfqpoint{6.611324in}{4.171251in}}%
\pgfpathcurveto{\pgfqpoint{6.600274in}{4.171251in}}{\pgfqpoint{6.589675in}{4.166861in}}{\pgfqpoint{6.581861in}{4.159047in}}%
\pgfpathcurveto{\pgfqpoint{6.574048in}{4.151234in}}{\pgfqpoint{6.569657in}{4.140634in}}{\pgfqpoint{6.569657in}{4.129584in}}%
\pgfpathcurveto{\pgfqpoint{6.569657in}{4.118534in}}{\pgfqpoint{6.574048in}{4.107935in}}{\pgfqpoint{6.581861in}{4.100122in}}%
\pgfpathcurveto{\pgfqpoint{6.589675in}{4.092308in}}{\pgfqpoint{6.600274in}{4.087918in}}{\pgfqpoint{6.611324in}{4.087918in}}%
\pgfpathclose%
\pgfusepath{stroke,fill}%
\end{pgfscope}%
\begin{pgfscope}%
\pgfpathrectangle{\pgfqpoint{0.526127in}{0.331635in}}{\pgfqpoint{9.300000in}{7.700000in}}%
\pgfusepath{clip}%
\pgfsetbuttcap%
\pgfsetroundjoin%
\definecolor{currentfill}{rgb}{1.000000,0.705882,0.509804}%
\pgfsetfillcolor{currentfill}%
\pgfsetlinewidth{0.481800pt}%
\definecolor{currentstroke}{rgb}{1.000000,1.000000,1.000000}%
\pgfsetstrokecolor{currentstroke}%
\pgfsetdash{}{0pt}%
\pgfpathmoveto{\pgfqpoint{4.265030in}{6.967810in}}%
\pgfpathcurveto{\pgfqpoint{4.276080in}{6.967810in}}{\pgfqpoint{4.286679in}{6.972200in}}{\pgfqpoint{4.294493in}{6.980014in}}%
\pgfpathcurveto{\pgfqpoint{4.302306in}{6.987827in}}{\pgfqpoint{4.306697in}{6.998426in}}{\pgfqpoint{4.306697in}{7.009476in}}%
\pgfpathcurveto{\pgfqpoint{4.306697in}{7.020527in}}{\pgfqpoint{4.302306in}{7.031126in}}{\pgfqpoint{4.294493in}{7.038939in}}%
\pgfpathcurveto{\pgfqpoint{4.286679in}{7.046753in}}{\pgfqpoint{4.276080in}{7.051143in}}{\pgfqpoint{4.265030in}{7.051143in}}%
\pgfpathcurveto{\pgfqpoint{4.253980in}{7.051143in}}{\pgfqpoint{4.243381in}{7.046753in}}{\pgfqpoint{4.235567in}{7.038939in}}%
\pgfpathcurveto{\pgfqpoint{4.227753in}{7.031126in}}{\pgfqpoint{4.223363in}{7.020527in}}{\pgfqpoint{4.223363in}{7.009476in}}%
\pgfpathcurveto{\pgfqpoint{4.223363in}{6.998426in}}{\pgfqpoint{4.227753in}{6.987827in}}{\pgfqpoint{4.235567in}{6.980014in}}%
\pgfpathcurveto{\pgfqpoint{4.243381in}{6.972200in}}{\pgfqpoint{4.253980in}{6.967810in}}{\pgfqpoint{4.265030in}{6.967810in}}%
\pgfpathclose%
\pgfusepath{stroke,fill}%
\end{pgfscope}%
\begin{pgfscope}%
\pgfpathrectangle{\pgfqpoint{0.526127in}{0.331635in}}{\pgfqpoint{9.300000in}{7.700000in}}%
\pgfusepath{clip}%
\pgfsetbuttcap%
\pgfsetroundjoin%
\definecolor{currentfill}{rgb}{1.000000,0.705882,0.509804}%
\pgfsetfillcolor{currentfill}%
\pgfsetlinewidth{0.481800pt}%
\definecolor{currentstroke}{rgb}{1.000000,1.000000,1.000000}%
\pgfsetstrokecolor{currentstroke}%
\pgfsetdash{}{0pt}%
\pgfpathmoveto{\pgfqpoint{5.530805in}{6.526749in}}%
\pgfpathcurveto{\pgfqpoint{5.541855in}{6.526749in}}{\pgfqpoint{5.552454in}{6.531140in}}{\pgfqpoint{5.560268in}{6.538953in}}%
\pgfpathcurveto{\pgfqpoint{5.568082in}{6.546767in}}{\pgfqpoint{5.572472in}{6.557366in}}{\pgfqpoint{5.572472in}{6.568416in}}%
\pgfpathcurveto{\pgfqpoint{5.572472in}{6.579466in}}{\pgfqpoint{5.568082in}{6.590065in}}{\pgfqpoint{5.560268in}{6.597879in}}%
\pgfpathcurveto{\pgfqpoint{5.552454in}{6.605692in}}{\pgfqpoint{5.541855in}{6.610083in}}{\pgfqpoint{5.530805in}{6.610083in}}%
\pgfpathcurveto{\pgfqpoint{5.519755in}{6.610083in}}{\pgfqpoint{5.509156in}{6.605692in}}{\pgfqpoint{5.501342in}{6.597879in}}%
\pgfpathcurveto{\pgfqpoint{5.493529in}{6.590065in}}{\pgfqpoint{5.489139in}{6.579466in}}{\pgfqpoint{5.489139in}{6.568416in}}%
\pgfpathcurveto{\pgfqpoint{5.489139in}{6.557366in}}{\pgfqpoint{5.493529in}{6.546767in}}{\pgfqpoint{5.501342in}{6.538953in}}%
\pgfpathcurveto{\pgfqpoint{5.509156in}{6.531140in}}{\pgfqpoint{5.519755in}{6.526749in}}{\pgfqpoint{5.530805in}{6.526749in}}%
\pgfpathclose%
\pgfusepath{stroke,fill}%
\end{pgfscope}%
\begin{pgfscope}%
\pgfpathrectangle{\pgfqpoint{0.526127in}{0.331635in}}{\pgfqpoint{9.300000in}{7.700000in}}%
\pgfusepath{clip}%
\pgfsetbuttcap%
\pgfsetroundjoin%
\definecolor{currentfill}{rgb}{1.000000,0.705882,0.509804}%
\pgfsetfillcolor{currentfill}%
\pgfsetlinewidth{0.481800pt}%
\definecolor{currentstroke}{rgb}{1.000000,1.000000,1.000000}%
\pgfsetstrokecolor{currentstroke}%
\pgfsetdash{}{0pt}%
\pgfpathmoveto{\pgfqpoint{6.584340in}{5.691002in}}%
\pgfpathcurveto{\pgfqpoint{6.595390in}{5.691002in}}{\pgfqpoint{6.605989in}{5.695392in}}{\pgfqpoint{6.613803in}{5.703206in}}%
\pgfpathcurveto{\pgfqpoint{6.621616in}{5.711019in}}{\pgfqpoint{6.626007in}{5.721618in}}{\pgfqpoint{6.626007in}{5.732668in}}%
\pgfpathcurveto{\pgfqpoint{6.626007in}{5.743719in}}{\pgfqpoint{6.621616in}{5.754318in}}{\pgfqpoint{6.613803in}{5.762131in}}%
\pgfpathcurveto{\pgfqpoint{6.605989in}{5.769945in}}{\pgfqpoint{6.595390in}{5.774335in}}{\pgfqpoint{6.584340in}{5.774335in}}%
\pgfpathcurveto{\pgfqpoint{6.573290in}{5.774335in}}{\pgfqpoint{6.562691in}{5.769945in}}{\pgfqpoint{6.554877in}{5.762131in}}%
\pgfpathcurveto{\pgfqpoint{6.547064in}{5.754318in}}{\pgfqpoint{6.542673in}{5.743719in}}{\pgfqpoint{6.542673in}{5.732668in}}%
\pgfpathcurveto{\pgfqpoint{6.542673in}{5.721618in}}{\pgfqpoint{6.547064in}{5.711019in}}{\pgfqpoint{6.554877in}{5.703206in}}%
\pgfpathcurveto{\pgfqpoint{6.562691in}{5.695392in}}{\pgfqpoint{6.573290in}{5.691002in}}{\pgfqpoint{6.584340in}{5.691002in}}%
\pgfpathclose%
\pgfusepath{stroke,fill}%
\end{pgfscope}%
\begin{pgfscope}%
\pgfpathrectangle{\pgfqpoint{0.526127in}{0.331635in}}{\pgfqpoint{9.300000in}{7.700000in}}%
\pgfusepath{clip}%
\pgfsetbuttcap%
\pgfsetroundjoin%
\definecolor{currentfill}{rgb}{1.000000,0.705882,0.509804}%
\pgfsetfillcolor{currentfill}%
\pgfsetlinewidth{0.481800pt}%
\definecolor{currentstroke}{rgb}{1.000000,1.000000,1.000000}%
\pgfsetstrokecolor{currentstroke}%
\pgfsetdash{}{0pt}%
\pgfpathmoveto{\pgfqpoint{6.306902in}{2.479853in}}%
\pgfpathcurveto{\pgfqpoint{6.317952in}{2.479853in}}{\pgfqpoint{6.328551in}{2.484244in}}{\pgfqpoint{6.336365in}{2.492057in}}%
\pgfpathcurveto{\pgfqpoint{6.344179in}{2.499871in}}{\pgfqpoint{6.348569in}{2.510470in}}{\pgfqpoint{6.348569in}{2.521520in}}%
\pgfpathcurveto{\pgfqpoint{6.348569in}{2.532570in}}{\pgfqpoint{6.344179in}{2.543169in}}{\pgfqpoint{6.336365in}{2.550983in}}%
\pgfpathcurveto{\pgfqpoint{6.328551in}{2.558797in}}{\pgfqpoint{6.317952in}{2.563187in}}{\pgfqpoint{6.306902in}{2.563187in}}%
\pgfpathcurveto{\pgfqpoint{6.295852in}{2.563187in}}{\pgfqpoint{6.285253in}{2.558797in}}{\pgfqpoint{6.277439in}{2.550983in}}%
\pgfpathcurveto{\pgfqpoint{6.269626in}{2.543169in}}{\pgfqpoint{6.265235in}{2.532570in}}{\pgfqpoint{6.265235in}{2.521520in}}%
\pgfpathcurveto{\pgfqpoint{6.265235in}{2.510470in}}{\pgfqpoint{6.269626in}{2.499871in}}{\pgfqpoint{6.277439in}{2.492057in}}%
\pgfpathcurveto{\pgfqpoint{6.285253in}{2.484244in}}{\pgfqpoint{6.295852in}{2.479853in}}{\pgfqpoint{6.306902in}{2.479853in}}%
\pgfpathclose%
\pgfusepath{stroke,fill}%
\end{pgfscope}%
\begin{pgfscope}%
\pgfpathrectangle{\pgfqpoint{0.526127in}{0.331635in}}{\pgfqpoint{9.300000in}{7.700000in}}%
\pgfusepath{clip}%
\pgfsetbuttcap%
\pgfsetroundjoin%
\definecolor{currentfill}{rgb}{1.000000,0.705882,0.509804}%
\pgfsetfillcolor{currentfill}%
\pgfsetlinewidth{0.481800pt}%
\definecolor{currentstroke}{rgb}{1.000000,1.000000,1.000000}%
\pgfsetstrokecolor{currentstroke}%
\pgfsetdash{}{0pt}%
\pgfpathmoveto{\pgfqpoint{7.529402in}{3.679205in}}%
\pgfpathcurveto{\pgfqpoint{7.540452in}{3.679205in}}{\pgfqpoint{7.551051in}{3.683596in}}{\pgfqpoint{7.558865in}{3.691409in}}%
\pgfpathcurveto{\pgfqpoint{7.566678in}{3.699223in}}{\pgfqpoint{7.571069in}{3.709822in}}{\pgfqpoint{7.571069in}{3.720872in}}%
\pgfpathcurveto{\pgfqpoint{7.571069in}{3.731922in}}{\pgfqpoint{7.566678in}{3.742521in}}{\pgfqpoint{7.558865in}{3.750335in}}%
\pgfpathcurveto{\pgfqpoint{7.551051in}{3.758148in}}{\pgfqpoint{7.540452in}{3.762539in}}{\pgfqpoint{7.529402in}{3.762539in}}%
\pgfpathcurveto{\pgfqpoint{7.518352in}{3.762539in}}{\pgfqpoint{7.507753in}{3.758148in}}{\pgfqpoint{7.499939in}{3.750335in}}%
\pgfpathcurveto{\pgfqpoint{7.492126in}{3.742521in}}{\pgfqpoint{7.487735in}{3.731922in}}{\pgfqpoint{7.487735in}{3.720872in}}%
\pgfpathcurveto{\pgfqpoint{7.487735in}{3.709822in}}{\pgfqpoint{7.492126in}{3.699223in}}{\pgfqpoint{7.499939in}{3.691409in}}%
\pgfpathcurveto{\pgfqpoint{7.507753in}{3.683596in}}{\pgfqpoint{7.518352in}{3.679205in}}{\pgfqpoint{7.529402in}{3.679205in}}%
\pgfpathclose%
\pgfusepath{stroke,fill}%
\end{pgfscope}%
\begin{pgfscope}%
\pgfpathrectangle{\pgfqpoint{0.526127in}{0.331635in}}{\pgfqpoint{9.300000in}{7.700000in}}%
\pgfusepath{clip}%
\pgfsetbuttcap%
\pgfsetroundjoin%
\definecolor{currentfill}{rgb}{1.000000,0.705882,0.509804}%
\pgfsetfillcolor{currentfill}%
\pgfsetlinewidth{0.481800pt}%
\definecolor{currentstroke}{rgb}{1.000000,1.000000,1.000000}%
\pgfsetstrokecolor{currentstroke}%
\pgfsetdash{}{0pt}%
\pgfpathmoveto{\pgfqpoint{7.857488in}{3.630724in}}%
\pgfpathcurveto{\pgfqpoint{7.868538in}{3.630724in}}{\pgfqpoint{7.879137in}{3.635114in}}{\pgfqpoint{7.886951in}{3.642928in}}%
\pgfpathcurveto{\pgfqpoint{7.894764in}{3.650741in}}{\pgfqpoint{7.899155in}{3.661340in}}{\pgfqpoint{7.899155in}{3.672391in}}%
\pgfpathcurveto{\pgfqpoint{7.899155in}{3.683441in}}{\pgfqpoint{7.894764in}{3.694040in}}{\pgfqpoint{7.886951in}{3.701853in}}%
\pgfpathcurveto{\pgfqpoint{7.879137in}{3.709667in}}{\pgfqpoint{7.868538in}{3.714057in}}{\pgfqpoint{7.857488in}{3.714057in}}%
\pgfpathcurveto{\pgfqpoint{7.846438in}{3.714057in}}{\pgfqpoint{7.835839in}{3.709667in}}{\pgfqpoint{7.828025in}{3.701853in}}%
\pgfpathcurveto{\pgfqpoint{7.820212in}{3.694040in}}{\pgfqpoint{7.815821in}{3.683441in}}{\pgfqpoint{7.815821in}{3.672391in}}%
\pgfpathcurveto{\pgfqpoint{7.815821in}{3.661340in}}{\pgfqpoint{7.820212in}{3.650741in}}{\pgfqpoint{7.828025in}{3.642928in}}%
\pgfpathcurveto{\pgfqpoint{7.835839in}{3.635114in}}{\pgfqpoint{7.846438in}{3.630724in}}{\pgfqpoint{7.857488in}{3.630724in}}%
\pgfpathclose%
\pgfusepath{stroke,fill}%
\end{pgfscope}%
\begin{pgfscope}%
\pgfpathrectangle{\pgfqpoint{0.526127in}{0.331635in}}{\pgfqpoint{9.300000in}{7.700000in}}%
\pgfusepath{clip}%
\pgfsetbuttcap%
\pgfsetroundjoin%
\definecolor{currentfill}{rgb}{1.000000,0.705882,0.509804}%
\pgfsetfillcolor{currentfill}%
\pgfsetlinewidth{0.481800pt}%
\definecolor{currentstroke}{rgb}{1.000000,1.000000,1.000000}%
\pgfsetstrokecolor{currentstroke}%
\pgfsetdash{}{0pt}%
\pgfpathmoveto{\pgfqpoint{6.572528in}{4.183325in}}%
\pgfpathcurveto{\pgfqpoint{6.583578in}{4.183325in}}{\pgfqpoint{6.594177in}{4.187715in}}{\pgfqpoint{6.601990in}{4.195528in}}%
\pgfpathcurveto{\pgfqpoint{6.609804in}{4.203342in}}{\pgfqpoint{6.614194in}{4.213941in}}{\pgfqpoint{6.614194in}{4.224991in}}%
\pgfpathcurveto{\pgfqpoint{6.614194in}{4.236041in}}{\pgfqpoint{6.609804in}{4.246640in}}{\pgfqpoint{6.601990in}{4.254454in}}%
\pgfpathcurveto{\pgfqpoint{6.594177in}{4.262268in}}{\pgfqpoint{6.583578in}{4.266658in}}{\pgfqpoint{6.572528in}{4.266658in}}%
\pgfpathcurveto{\pgfqpoint{6.561477in}{4.266658in}}{\pgfqpoint{6.550878in}{4.262268in}}{\pgfqpoint{6.543065in}{4.254454in}}%
\pgfpathcurveto{\pgfqpoint{6.535251in}{4.246640in}}{\pgfqpoint{6.530861in}{4.236041in}}{\pgfqpoint{6.530861in}{4.224991in}}%
\pgfpathcurveto{\pgfqpoint{6.530861in}{4.213941in}}{\pgfqpoint{6.535251in}{4.203342in}}{\pgfqpoint{6.543065in}{4.195528in}}%
\pgfpathcurveto{\pgfqpoint{6.550878in}{4.187715in}}{\pgfqpoint{6.561477in}{4.183325in}}{\pgfqpoint{6.572528in}{4.183325in}}%
\pgfpathclose%
\pgfusepath{stroke,fill}%
\end{pgfscope}%
\begin{pgfscope}%
\pgfpathrectangle{\pgfqpoint{0.526127in}{0.331635in}}{\pgfqpoint{9.300000in}{7.700000in}}%
\pgfusepath{clip}%
\pgfsetbuttcap%
\pgfsetroundjoin%
\definecolor{currentfill}{rgb}{1.000000,0.705882,0.509804}%
\pgfsetfillcolor{currentfill}%
\pgfsetlinewidth{0.481800pt}%
\definecolor{currentstroke}{rgb}{1.000000,1.000000,1.000000}%
\pgfsetstrokecolor{currentstroke}%
\pgfsetdash{}{0pt}%
\pgfpathmoveto{\pgfqpoint{5.154620in}{3.724228in}}%
\pgfpathcurveto{\pgfqpoint{5.165670in}{3.724228in}}{\pgfqpoint{5.176269in}{3.728618in}}{\pgfqpoint{5.184083in}{3.736431in}}%
\pgfpathcurveto{\pgfqpoint{5.191896in}{3.744245in}}{\pgfqpoint{5.196287in}{3.754844in}}{\pgfqpoint{5.196287in}{3.765894in}}%
\pgfpathcurveto{\pgfqpoint{5.196287in}{3.776944in}}{\pgfqpoint{5.191896in}{3.787543in}}{\pgfqpoint{5.184083in}{3.795357in}}%
\pgfpathcurveto{\pgfqpoint{5.176269in}{3.803171in}}{\pgfqpoint{5.165670in}{3.807561in}}{\pgfqpoint{5.154620in}{3.807561in}}%
\pgfpathcurveto{\pgfqpoint{5.143570in}{3.807561in}}{\pgfqpoint{5.132971in}{3.803171in}}{\pgfqpoint{5.125157in}{3.795357in}}%
\pgfpathcurveto{\pgfqpoint{5.117344in}{3.787543in}}{\pgfqpoint{5.112953in}{3.776944in}}{\pgfqpoint{5.112953in}{3.765894in}}%
\pgfpathcurveto{\pgfqpoint{5.112953in}{3.754844in}}{\pgfqpoint{5.117344in}{3.744245in}}{\pgfqpoint{5.125157in}{3.736431in}}%
\pgfpathcurveto{\pgfqpoint{5.132971in}{3.728618in}}{\pgfqpoint{5.143570in}{3.724228in}}{\pgfqpoint{5.154620in}{3.724228in}}%
\pgfpathclose%
\pgfusepath{stroke,fill}%
\end{pgfscope}%
\begin{pgfscope}%
\pgfpathrectangle{\pgfqpoint{0.526127in}{0.331635in}}{\pgfqpoint{9.300000in}{7.700000in}}%
\pgfusepath{clip}%
\pgfsetbuttcap%
\pgfsetroundjoin%
\definecolor{currentfill}{rgb}{1.000000,0.705882,0.509804}%
\pgfsetfillcolor{currentfill}%
\pgfsetlinewidth{0.481800pt}%
\definecolor{currentstroke}{rgb}{1.000000,1.000000,1.000000}%
\pgfsetstrokecolor{currentstroke}%
\pgfsetdash{}{0pt}%
\pgfpathmoveto{\pgfqpoint{5.190149in}{3.797647in}}%
\pgfpathcurveto{\pgfqpoint{5.201199in}{3.797647in}}{\pgfqpoint{5.211798in}{3.802038in}}{\pgfqpoint{5.219612in}{3.809851in}}%
\pgfpathcurveto{\pgfqpoint{5.227426in}{3.817665in}}{\pgfqpoint{5.231816in}{3.828264in}}{\pgfqpoint{5.231816in}{3.839314in}}%
\pgfpathcurveto{\pgfqpoint{5.231816in}{3.850364in}}{\pgfqpoint{5.227426in}{3.860963in}}{\pgfqpoint{5.219612in}{3.868777in}}%
\pgfpathcurveto{\pgfqpoint{5.211798in}{3.876591in}}{\pgfqpoint{5.201199in}{3.880981in}}{\pgfqpoint{5.190149in}{3.880981in}}%
\pgfpathcurveto{\pgfqpoint{5.179099in}{3.880981in}}{\pgfqpoint{5.168500in}{3.876591in}}{\pgfqpoint{5.160687in}{3.868777in}}%
\pgfpathcurveto{\pgfqpoint{5.152873in}{3.860963in}}{\pgfqpoint{5.148483in}{3.850364in}}{\pgfqpoint{5.148483in}{3.839314in}}%
\pgfpathcurveto{\pgfqpoint{5.148483in}{3.828264in}}{\pgfqpoint{5.152873in}{3.817665in}}{\pgfqpoint{5.160687in}{3.809851in}}%
\pgfpathcurveto{\pgfqpoint{5.168500in}{3.802038in}}{\pgfqpoint{5.179099in}{3.797647in}}{\pgfqpoint{5.190149in}{3.797647in}}%
\pgfpathclose%
\pgfusepath{stroke,fill}%
\end{pgfscope}%
\begin{pgfscope}%
\pgfpathrectangle{\pgfqpoint{0.526127in}{0.331635in}}{\pgfqpoint{9.300000in}{7.700000in}}%
\pgfusepath{clip}%
\pgfsetbuttcap%
\pgfsetroundjoin%
\definecolor{currentfill}{rgb}{1.000000,0.705882,0.509804}%
\pgfsetfillcolor{currentfill}%
\pgfsetlinewidth{0.481800pt}%
\definecolor{currentstroke}{rgb}{1.000000,1.000000,1.000000}%
\pgfsetstrokecolor{currentstroke}%
\pgfsetdash{}{0pt}%
\pgfpathmoveto{\pgfqpoint{6.190070in}{4.268157in}}%
\pgfpathcurveto{\pgfqpoint{6.201120in}{4.268157in}}{\pgfqpoint{6.211719in}{4.272548in}}{\pgfqpoint{6.219532in}{4.280361in}}%
\pgfpathcurveto{\pgfqpoint{6.227346in}{4.288175in}}{\pgfqpoint{6.231736in}{4.298774in}}{\pgfqpoint{6.231736in}{4.309824in}}%
\pgfpathcurveto{\pgfqpoint{6.231736in}{4.320874in}}{\pgfqpoint{6.227346in}{4.331473in}}{\pgfqpoint{6.219532in}{4.339287in}}%
\pgfpathcurveto{\pgfqpoint{6.211719in}{4.347101in}}{\pgfqpoint{6.201120in}{4.351491in}}{\pgfqpoint{6.190070in}{4.351491in}}%
\pgfpathcurveto{\pgfqpoint{6.179019in}{4.351491in}}{\pgfqpoint{6.168420in}{4.347101in}}{\pgfqpoint{6.160607in}{4.339287in}}%
\pgfpathcurveto{\pgfqpoint{6.152793in}{4.331473in}}{\pgfqpoint{6.148403in}{4.320874in}}{\pgfqpoint{6.148403in}{4.309824in}}%
\pgfpathcurveto{\pgfqpoint{6.148403in}{4.298774in}}{\pgfqpoint{6.152793in}{4.288175in}}{\pgfqpoint{6.160607in}{4.280361in}}%
\pgfpathcurveto{\pgfqpoint{6.168420in}{4.272548in}}{\pgfqpoint{6.179019in}{4.268157in}}{\pgfqpoint{6.190070in}{4.268157in}}%
\pgfpathclose%
\pgfusepath{stroke,fill}%
\end{pgfscope}%
\begin{pgfscope}%
\pgfpathrectangle{\pgfqpoint{0.526127in}{0.331635in}}{\pgfqpoint{9.300000in}{7.700000in}}%
\pgfusepath{clip}%
\pgfsetbuttcap%
\pgfsetroundjoin%
\definecolor{currentfill}{rgb}{1.000000,0.705882,0.509804}%
\pgfsetfillcolor{currentfill}%
\pgfsetlinewidth{0.481800pt}%
\definecolor{currentstroke}{rgb}{1.000000,1.000000,1.000000}%
\pgfsetstrokecolor{currentstroke}%
\pgfsetdash{}{0pt}%
\pgfpathmoveto{\pgfqpoint{6.008150in}{4.950389in}}%
\pgfpathcurveto{\pgfqpoint{6.019200in}{4.950389in}}{\pgfqpoint{6.029799in}{4.954779in}}{\pgfqpoint{6.037612in}{4.962593in}}%
\pgfpathcurveto{\pgfqpoint{6.045426in}{4.970406in}}{\pgfqpoint{6.049816in}{4.981005in}}{\pgfqpoint{6.049816in}{4.992055in}}%
\pgfpathcurveto{\pgfqpoint{6.049816in}{5.003105in}}{\pgfqpoint{6.045426in}{5.013705in}}{\pgfqpoint{6.037612in}{5.021518in}}%
\pgfpathcurveto{\pgfqpoint{6.029799in}{5.029332in}}{\pgfqpoint{6.019200in}{5.033722in}}{\pgfqpoint{6.008150in}{5.033722in}}%
\pgfpathcurveto{\pgfqpoint{5.997099in}{5.033722in}}{\pgfqpoint{5.986500in}{5.029332in}}{\pgfqpoint{5.978687in}{5.021518in}}%
\pgfpathcurveto{\pgfqpoint{5.970873in}{5.013705in}}{\pgfqpoint{5.966483in}{5.003105in}}{\pgfqpoint{5.966483in}{4.992055in}}%
\pgfpathcurveto{\pgfqpoint{5.966483in}{4.981005in}}{\pgfqpoint{5.970873in}{4.970406in}}{\pgfqpoint{5.978687in}{4.962593in}}%
\pgfpathcurveto{\pgfqpoint{5.986500in}{4.954779in}}{\pgfqpoint{5.997099in}{4.950389in}}{\pgfqpoint{6.008150in}{4.950389in}}%
\pgfpathclose%
\pgfusepath{stroke,fill}%
\end{pgfscope}%
\begin{pgfscope}%
\pgfpathrectangle{\pgfqpoint{0.526127in}{0.331635in}}{\pgfqpoint{9.300000in}{7.700000in}}%
\pgfusepath{clip}%
\pgfsetbuttcap%
\pgfsetroundjoin%
\definecolor{currentfill}{rgb}{1.000000,0.705882,0.509804}%
\pgfsetfillcolor{currentfill}%
\pgfsetlinewidth{0.481800pt}%
\definecolor{currentstroke}{rgb}{1.000000,1.000000,1.000000}%
\pgfsetstrokecolor{currentstroke}%
\pgfsetdash{}{0pt}%
\pgfpathmoveto{\pgfqpoint{3.566645in}{0.639968in}}%
\pgfpathcurveto{\pgfqpoint{3.577695in}{0.639968in}}{\pgfqpoint{3.588294in}{0.644359in}}{\pgfqpoint{3.596108in}{0.652172in}}%
\pgfpathcurveto{\pgfqpoint{3.603921in}{0.659986in}}{\pgfqpoint{3.608311in}{0.670585in}}{\pgfqpoint{3.608311in}{0.681635in}}%
\pgfpathcurveto{\pgfqpoint{3.608311in}{0.692685in}}{\pgfqpoint{3.603921in}{0.703284in}}{\pgfqpoint{3.596108in}{0.711098in}}%
\pgfpathcurveto{\pgfqpoint{3.588294in}{0.718911in}}{\pgfqpoint{3.577695in}{0.723302in}}{\pgfqpoint{3.566645in}{0.723302in}}%
\pgfpathcurveto{\pgfqpoint{3.555595in}{0.723302in}}{\pgfqpoint{3.544996in}{0.718911in}}{\pgfqpoint{3.537182in}{0.711098in}}%
\pgfpathcurveto{\pgfqpoint{3.529368in}{0.703284in}}{\pgfqpoint{3.524978in}{0.692685in}}{\pgfqpoint{3.524978in}{0.681635in}}%
\pgfpathcurveto{\pgfqpoint{3.524978in}{0.670585in}}{\pgfqpoint{3.529368in}{0.659986in}}{\pgfqpoint{3.537182in}{0.652172in}}%
\pgfpathcurveto{\pgfqpoint{3.544996in}{0.644359in}}{\pgfqpoint{3.555595in}{0.639968in}}{\pgfqpoint{3.566645in}{0.639968in}}%
\pgfpathclose%
\pgfusepath{stroke,fill}%
\end{pgfscope}%
\begin{pgfscope}%
\pgfpathrectangle{\pgfqpoint{0.526127in}{0.331635in}}{\pgfqpoint{9.300000in}{7.700000in}}%
\pgfusepath{clip}%
\pgfsetbuttcap%
\pgfsetroundjoin%
\definecolor{currentfill}{rgb}{1.000000,0.705882,0.509804}%
\pgfsetfillcolor{currentfill}%
\pgfsetlinewidth{0.481800pt}%
\definecolor{currentstroke}{rgb}{1.000000,1.000000,1.000000}%
\pgfsetstrokecolor{currentstroke}%
\pgfsetdash{}{0pt}%
\pgfpathmoveto{\pgfqpoint{7.924332in}{3.429377in}}%
\pgfpathcurveto{\pgfqpoint{7.935383in}{3.429377in}}{\pgfqpoint{7.945982in}{3.433767in}}{\pgfqpoint{7.953795in}{3.441580in}}%
\pgfpathcurveto{\pgfqpoint{7.961609in}{3.449394in}}{\pgfqpoint{7.965999in}{3.459993in}}{\pgfqpoint{7.965999in}{3.471043in}}%
\pgfpathcurveto{\pgfqpoint{7.965999in}{3.482093in}}{\pgfqpoint{7.961609in}{3.492692in}}{\pgfqpoint{7.953795in}{3.500506in}}%
\pgfpathcurveto{\pgfqpoint{7.945982in}{3.508320in}}{\pgfqpoint{7.935383in}{3.512710in}}{\pgfqpoint{7.924332in}{3.512710in}}%
\pgfpathcurveto{\pgfqpoint{7.913282in}{3.512710in}}{\pgfqpoint{7.902683in}{3.508320in}}{\pgfqpoint{7.894870in}{3.500506in}}%
\pgfpathcurveto{\pgfqpoint{7.887056in}{3.492692in}}{\pgfqpoint{7.882666in}{3.482093in}}{\pgfqpoint{7.882666in}{3.471043in}}%
\pgfpathcurveto{\pgfqpoint{7.882666in}{3.459993in}}{\pgfqpoint{7.887056in}{3.449394in}}{\pgfqpoint{7.894870in}{3.441580in}}%
\pgfpathcurveto{\pgfqpoint{7.902683in}{3.433767in}}{\pgfqpoint{7.913282in}{3.429377in}}{\pgfqpoint{7.924332in}{3.429377in}}%
\pgfpathclose%
\pgfusepath{stroke,fill}%
\end{pgfscope}%
\begin{pgfscope}%
\pgfpathrectangle{\pgfqpoint{0.526127in}{0.331635in}}{\pgfqpoint{9.300000in}{7.700000in}}%
\pgfusepath{clip}%
\pgfsetbuttcap%
\pgfsetroundjoin%
\definecolor{currentfill}{rgb}{1.000000,0.705882,0.509804}%
\pgfsetfillcolor{currentfill}%
\pgfsetlinewidth{0.481800pt}%
\definecolor{currentstroke}{rgb}{1.000000,1.000000,1.000000}%
\pgfsetstrokecolor{currentstroke}%
\pgfsetdash{}{0pt}%
\pgfpathmoveto{\pgfqpoint{6.132760in}{3.833143in}}%
\pgfpathcurveto{\pgfqpoint{6.143811in}{3.833143in}}{\pgfqpoint{6.154410in}{3.837534in}}{\pgfqpoint{6.162223in}{3.845347in}}%
\pgfpathcurveto{\pgfqpoint{6.170037in}{3.853161in}}{\pgfqpoint{6.174427in}{3.863760in}}{\pgfqpoint{6.174427in}{3.874810in}}%
\pgfpathcurveto{\pgfqpoint{6.174427in}{3.885860in}}{\pgfqpoint{6.170037in}{3.896459in}}{\pgfqpoint{6.162223in}{3.904273in}}%
\pgfpathcurveto{\pgfqpoint{6.154410in}{3.912086in}}{\pgfqpoint{6.143811in}{3.916477in}}{\pgfqpoint{6.132760in}{3.916477in}}%
\pgfpathcurveto{\pgfqpoint{6.121710in}{3.916477in}}{\pgfqpoint{6.111111in}{3.912086in}}{\pgfqpoint{6.103298in}{3.904273in}}%
\pgfpathcurveto{\pgfqpoint{6.095484in}{3.896459in}}{\pgfqpoint{6.091094in}{3.885860in}}{\pgfqpoint{6.091094in}{3.874810in}}%
\pgfpathcurveto{\pgfqpoint{6.091094in}{3.863760in}}{\pgfqpoint{6.095484in}{3.853161in}}{\pgfqpoint{6.103298in}{3.845347in}}%
\pgfpathcurveto{\pgfqpoint{6.111111in}{3.837534in}}{\pgfqpoint{6.121710in}{3.833143in}}{\pgfqpoint{6.132760in}{3.833143in}}%
\pgfpathclose%
\pgfusepath{stroke,fill}%
\end{pgfscope}%
\begin{pgfscope}%
\pgfpathrectangle{\pgfqpoint{0.526127in}{0.331635in}}{\pgfqpoint{9.300000in}{7.700000in}}%
\pgfusepath{clip}%
\pgfsetbuttcap%
\pgfsetroundjoin%
\definecolor{currentfill}{rgb}{1.000000,0.705882,0.509804}%
\pgfsetfillcolor{currentfill}%
\pgfsetlinewidth{0.481800pt}%
\definecolor{currentstroke}{rgb}{1.000000,1.000000,1.000000}%
\pgfsetstrokecolor{currentstroke}%
\pgfsetdash{}{0pt}%
\pgfpathmoveto{\pgfqpoint{6.193019in}{4.731699in}}%
\pgfpathcurveto{\pgfqpoint{6.204069in}{4.731699in}}{\pgfqpoint{6.214668in}{4.736090in}}{\pgfqpoint{6.222482in}{4.743903in}}%
\pgfpathcurveto{\pgfqpoint{6.230295in}{4.751717in}}{\pgfqpoint{6.234686in}{4.762316in}}{\pgfqpoint{6.234686in}{4.773366in}}%
\pgfpathcurveto{\pgfqpoint{6.234686in}{4.784416in}}{\pgfqpoint{6.230295in}{4.795015in}}{\pgfqpoint{6.222482in}{4.802829in}}%
\pgfpathcurveto{\pgfqpoint{6.214668in}{4.810642in}}{\pgfqpoint{6.204069in}{4.815033in}}{\pgfqpoint{6.193019in}{4.815033in}}%
\pgfpathcurveto{\pgfqpoint{6.181969in}{4.815033in}}{\pgfqpoint{6.171370in}{4.810642in}}{\pgfqpoint{6.163556in}{4.802829in}}%
\pgfpathcurveto{\pgfqpoint{6.155743in}{4.795015in}}{\pgfqpoint{6.151352in}{4.784416in}}{\pgfqpoint{6.151352in}{4.773366in}}%
\pgfpathcurveto{\pgfqpoint{6.151352in}{4.762316in}}{\pgfqpoint{6.155743in}{4.751717in}}{\pgfqpoint{6.163556in}{4.743903in}}%
\pgfpathcurveto{\pgfqpoint{6.171370in}{4.736090in}}{\pgfqpoint{6.181969in}{4.731699in}}{\pgfqpoint{6.193019in}{4.731699in}}%
\pgfpathclose%
\pgfusepath{stroke,fill}%
\end{pgfscope}%
\begin{pgfscope}%
\pgfpathrectangle{\pgfqpoint{0.526127in}{0.331635in}}{\pgfqpoint{9.300000in}{7.700000in}}%
\pgfusepath{clip}%
\pgfsetbuttcap%
\pgfsetroundjoin%
\definecolor{currentfill}{rgb}{1.000000,0.705882,0.509804}%
\pgfsetfillcolor{currentfill}%
\pgfsetlinewidth{0.481800pt}%
\definecolor{currentstroke}{rgb}{1.000000,1.000000,1.000000}%
\pgfsetstrokecolor{currentstroke}%
\pgfsetdash{}{0pt}%
\pgfpathmoveto{\pgfqpoint{6.027104in}{5.585847in}}%
\pgfpathcurveto{\pgfqpoint{6.038154in}{5.585847in}}{\pgfqpoint{6.048753in}{5.590238in}}{\pgfqpoint{6.056567in}{5.598051in}}%
\pgfpathcurveto{\pgfqpoint{6.064380in}{5.605865in}}{\pgfqpoint{6.068770in}{5.616464in}}{\pgfqpoint{6.068770in}{5.627514in}}%
\pgfpathcurveto{\pgfqpoint{6.068770in}{5.638564in}}{\pgfqpoint{6.064380in}{5.649163in}}{\pgfqpoint{6.056567in}{5.656977in}}%
\pgfpathcurveto{\pgfqpoint{6.048753in}{5.664790in}}{\pgfqpoint{6.038154in}{5.669181in}}{\pgfqpoint{6.027104in}{5.669181in}}%
\pgfpathcurveto{\pgfqpoint{6.016054in}{5.669181in}}{\pgfqpoint{6.005455in}{5.664790in}}{\pgfqpoint{5.997641in}{5.656977in}}%
\pgfpathcurveto{\pgfqpoint{5.989827in}{5.649163in}}{\pgfqpoint{5.985437in}{5.638564in}}{\pgfqpoint{5.985437in}{5.627514in}}%
\pgfpathcurveto{\pgfqpoint{5.985437in}{5.616464in}}{\pgfqpoint{5.989827in}{5.605865in}}{\pgfqpoint{5.997641in}{5.598051in}}%
\pgfpathcurveto{\pgfqpoint{6.005455in}{5.590238in}}{\pgfqpoint{6.016054in}{5.585847in}}{\pgfqpoint{6.027104in}{5.585847in}}%
\pgfpathclose%
\pgfusepath{stroke,fill}%
\end{pgfscope}%
\begin{pgfscope}%
\pgfpathrectangle{\pgfqpoint{0.526127in}{0.331635in}}{\pgfqpoint{9.300000in}{7.700000in}}%
\pgfusepath{clip}%
\pgfsetbuttcap%
\pgfsetroundjoin%
\definecolor{currentfill}{rgb}{1.000000,0.705882,0.509804}%
\pgfsetfillcolor{currentfill}%
\pgfsetlinewidth{0.481800pt}%
\definecolor{currentstroke}{rgb}{1.000000,1.000000,1.000000}%
\pgfsetstrokecolor{currentstroke}%
\pgfsetdash{}{0pt}%
\pgfpathmoveto{\pgfqpoint{6.988150in}{5.526950in}}%
\pgfpathcurveto{\pgfqpoint{6.999201in}{5.526950in}}{\pgfqpoint{7.009800in}{5.531340in}}{\pgfqpoint{7.017613in}{5.539154in}}%
\pgfpathcurveto{\pgfqpoint{7.025427in}{5.546967in}}{\pgfqpoint{7.029817in}{5.557566in}}{\pgfqpoint{7.029817in}{5.568616in}}%
\pgfpathcurveto{\pgfqpoint{7.029817in}{5.579667in}}{\pgfqpoint{7.025427in}{5.590266in}}{\pgfqpoint{7.017613in}{5.598079in}}%
\pgfpathcurveto{\pgfqpoint{7.009800in}{5.605893in}}{\pgfqpoint{6.999201in}{5.610283in}}{\pgfqpoint{6.988150in}{5.610283in}}%
\pgfpathcurveto{\pgfqpoint{6.977100in}{5.610283in}}{\pgfqpoint{6.966501in}{5.605893in}}{\pgfqpoint{6.958688in}{5.598079in}}%
\pgfpathcurveto{\pgfqpoint{6.950874in}{5.590266in}}{\pgfqpoint{6.946484in}{5.579667in}}{\pgfqpoint{6.946484in}{5.568616in}}%
\pgfpathcurveto{\pgfqpoint{6.946484in}{5.557566in}}{\pgfqpoint{6.950874in}{5.546967in}}{\pgfqpoint{6.958688in}{5.539154in}}%
\pgfpathcurveto{\pgfqpoint{6.966501in}{5.531340in}}{\pgfqpoint{6.977100in}{5.526950in}}{\pgfqpoint{6.988150in}{5.526950in}}%
\pgfpathclose%
\pgfusepath{stroke,fill}%
\end{pgfscope}%
\begin{pgfscope}%
\pgfpathrectangle{\pgfqpoint{0.526127in}{0.331635in}}{\pgfqpoint{9.300000in}{7.700000in}}%
\pgfusepath{clip}%
\pgfsetbuttcap%
\pgfsetroundjoin%
\definecolor{currentfill}{rgb}{1.000000,0.705882,0.509804}%
\pgfsetfillcolor{currentfill}%
\pgfsetlinewidth{0.481800pt}%
\definecolor{currentstroke}{rgb}{1.000000,1.000000,1.000000}%
\pgfsetstrokecolor{currentstroke}%
\pgfsetdash{}{0pt}%
\pgfpathmoveto{\pgfqpoint{6.794558in}{3.513100in}}%
\pgfpathcurveto{\pgfqpoint{6.805608in}{3.513100in}}{\pgfqpoint{6.816207in}{3.517491in}}{\pgfqpoint{6.824021in}{3.525304in}}%
\pgfpathcurveto{\pgfqpoint{6.831835in}{3.533118in}}{\pgfqpoint{6.836225in}{3.543717in}}{\pgfqpoint{6.836225in}{3.554767in}}%
\pgfpathcurveto{\pgfqpoint{6.836225in}{3.565817in}}{\pgfqpoint{6.831835in}{3.576416in}}{\pgfqpoint{6.824021in}{3.584230in}}%
\pgfpathcurveto{\pgfqpoint{6.816207in}{3.592043in}}{\pgfqpoint{6.805608in}{3.596434in}}{\pgfqpoint{6.794558in}{3.596434in}}%
\pgfpathcurveto{\pgfqpoint{6.783508in}{3.596434in}}{\pgfqpoint{6.772909in}{3.592043in}}{\pgfqpoint{6.765096in}{3.584230in}}%
\pgfpathcurveto{\pgfqpoint{6.757282in}{3.576416in}}{\pgfqpoint{6.752892in}{3.565817in}}{\pgfqpoint{6.752892in}{3.554767in}}%
\pgfpathcurveto{\pgfqpoint{6.752892in}{3.543717in}}{\pgfqpoint{6.757282in}{3.533118in}}{\pgfqpoint{6.765096in}{3.525304in}}%
\pgfpathcurveto{\pgfqpoint{6.772909in}{3.517491in}}{\pgfqpoint{6.783508in}{3.513100in}}{\pgfqpoint{6.794558in}{3.513100in}}%
\pgfpathclose%
\pgfusepath{stroke,fill}%
\end{pgfscope}%
\begin{pgfscope}%
\pgfpathrectangle{\pgfqpoint{0.526127in}{0.331635in}}{\pgfqpoint{9.300000in}{7.700000in}}%
\pgfusepath{clip}%
\pgfsetbuttcap%
\pgfsetroundjoin%
\definecolor{currentfill}{rgb}{1.000000,0.705882,0.509804}%
\pgfsetfillcolor{currentfill}%
\pgfsetlinewidth{0.481800pt}%
\definecolor{currentstroke}{rgb}{1.000000,1.000000,1.000000}%
\pgfsetstrokecolor{currentstroke}%
\pgfsetdash{}{0pt}%
\pgfpathmoveto{\pgfqpoint{7.563920in}{3.414147in}}%
\pgfpathcurveto{\pgfqpoint{7.574970in}{3.414147in}}{\pgfqpoint{7.585570in}{3.418537in}}{\pgfqpoint{7.593383in}{3.426350in}}%
\pgfpathcurveto{\pgfqpoint{7.601197in}{3.434164in}}{\pgfqpoint{7.605587in}{3.444763in}}{\pgfqpoint{7.605587in}{3.455813in}}%
\pgfpathcurveto{\pgfqpoint{7.605587in}{3.466863in}}{\pgfqpoint{7.601197in}{3.477462in}}{\pgfqpoint{7.593383in}{3.485276in}}%
\pgfpathcurveto{\pgfqpoint{7.585570in}{3.493090in}}{\pgfqpoint{7.574970in}{3.497480in}}{\pgfqpoint{7.563920in}{3.497480in}}%
\pgfpathcurveto{\pgfqpoint{7.552870in}{3.497480in}}{\pgfqpoint{7.542271in}{3.493090in}}{\pgfqpoint{7.534458in}{3.485276in}}%
\pgfpathcurveto{\pgfqpoint{7.526644in}{3.477462in}}{\pgfqpoint{7.522254in}{3.466863in}}{\pgfqpoint{7.522254in}{3.455813in}}%
\pgfpathcurveto{\pgfqpoint{7.522254in}{3.444763in}}{\pgfqpoint{7.526644in}{3.434164in}}{\pgfqpoint{7.534458in}{3.426350in}}%
\pgfpathcurveto{\pgfqpoint{7.542271in}{3.418537in}}{\pgfqpoint{7.552870in}{3.414147in}}{\pgfqpoint{7.563920in}{3.414147in}}%
\pgfpathclose%
\pgfusepath{stroke,fill}%
\end{pgfscope}%
\begin{pgfscope}%
\pgfpathrectangle{\pgfqpoint{0.526127in}{0.331635in}}{\pgfqpoint{9.300000in}{7.700000in}}%
\pgfusepath{clip}%
\pgfsetbuttcap%
\pgfsetroundjoin%
\definecolor{currentfill}{rgb}{1.000000,0.705882,0.509804}%
\pgfsetfillcolor{currentfill}%
\pgfsetlinewidth{0.481800pt}%
\definecolor{currentstroke}{rgb}{1.000000,1.000000,1.000000}%
\pgfsetstrokecolor{currentstroke}%
\pgfsetdash{}{0pt}%
\pgfpathmoveto{\pgfqpoint{5.276797in}{5.708763in}}%
\pgfpathcurveto{\pgfqpoint{5.287847in}{5.708763in}}{\pgfqpoint{5.298446in}{5.713153in}}{\pgfqpoint{5.306260in}{5.720967in}}%
\pgfpathcurveto{\pgfqpoint{5.314073in}{5.728780in}}{\pgfqpoint{5.318464in}{5.739379in}}{\pgfqpoint{5.318464in}{5.750429in}}%
\pgfpathcurveto{\pgfqpoint{5.318464in}{5.761480in}}{\pgfqpoint{5.314073in}{5.772079in}}{\pgfqpoint{5.306260in}{5.779892in}}%
\pgfpathcurveto{\pgfqpoint{5.298446in}{5.787706in}}{\pgfqpoint{5.287847in}{5.792096in}}{\pgfqpoint{5.276797in}{5.792096in}}%
\pgfpathcurveto{\pgfqpoint{5.265747in}{5.792096in}}{\pgfqpoint{5.255148in}{5.787706in}}{\pgfqpoint{5.247334in}{5.779892in}}%
\pgfpathcurveto{\pgfqpoint{5.239521in}{5.772079in}}{\pgfqpoint{5.235130in}{5.761480in}}{\pgfqpoint{5.235130in}{5.750429in}}%
\pgfpathcurveto{\pgfqpoint{5.235130in}{5.739379in}}{\pgfqpoint{5.239521in}{5.728780in}}{\pgfqpoint{5.247334in}{5.720967in}}%
\pgfpathcurveto{\pgfqpoint{5.255148in}{5.713153in}}{\pgfqpoint{5.265747in}{5.708763in}}{\pgfqpoint{5.276797in}{5.708763in}}%
\pgfpathclose%
\pgfusepath{stroke,fill}%
\end{pgfscope}%
\begin{pgfscope}%
\pgfpathrectangle{\pgfqpoint{0.526127in}{0.331635in}}{\pgfqpoint{9.300000in}{7.700000in}}%
\pgfusepath{clip}%
\pgfsetbuttcap%
\pgfsetroundjoin%
\definecolor{currentfill}{rgb}{1.000000,0.705882,0.509804}%
\pgfsetfillcolor{currentfill}%
\pgfsetlinewidth{0.481800pt}%
\definecolor{currentstroke}{rgb}{1.000000,1.000000,1.000000}%
\pgfsetstrokecolor{currentstroke}%
\pgfsetdash{}{0pt}%
\pgfpathmoveto{\pgfqpoint{7.155517in}{3.669209in}}%
\pgfpathcurveto{\pgfqpoint{7.166567in}{3.669209in}}{\pgfqpoint{7.177166in}{3.673599in}}{\pgfqpoint{7.184980in}{3.681413in}}%
\pgfpathcurveto{\pgfqpoint{7.192794in}{3.689227in}}{\pgfqpoint{7.197184in}{3.699826in}}{\pgfqpoint{7.197184in}{3.710876in}}%
\pgfpathcurveto{\pgfqpoint{7.197184in}{3.721926in}}{\pgfqpoint{7.192794in}{3.732525in}}{\pgfqpoint{7.184980in}{3.740339in}}%
\pgfpathcurveto{\pgfqpoint{7.177166in}{3.748152in}}{\pgfqpoint{7.166567in}{3.752543in}}{\pgfqpoint{7.155517in}{3.752543in}}%
\pgfpathcurveto{\pgfqpoint{7.144467in}{3.752543in}}{\pgfqpoint{7.133868in}{3.748152in}}{\pgfqpoint{7.126055in}{3.740339in}}%
\pgfpathcurveto{\pgfqpoint{7.118241in}{3.732525in}}{\pgfqpoint{7.113851in}{3.721926in}}{\pgfqpoint{7.113851in}{3.710876in}}%
\pgfpathcurveto{\pgfqpoint{7.113851in}{3.699826in}}{\pgfqpoint{7.118241in}{3.689227in}}{\pgfqpoint{7.126055in}{3.681413in}}%
\pgfpathcurveto{\pgfqpoint{7.133868in}{3.673599in}}{\pgfqpoint{7.144467in}{3.669209in}}{\pgfqpoint{7.155517in}{3.669209in}}%
\pgfpathclose%
\pgfusepath{stroke,fill}%
\end{pgfscope}%
\begin{pgfscope}%
\pgfpathrectangle{\pgfqpoint{0.526127in}{0.331635in}}{\pgfqpoint{9.300000in}{7.700000in}}%
\pgfusepath{clip}%
\pgfsetbuttcap%
\pgfsetroundjoin%
\definecolor{currentfill}{rgb}{1.000000,0.705882,0.509804}%
\pgfsetfillcolor{currentfill}%
\pgfsetlinewidth{0.481800pt}%
\definecolor{currentstroke}{rgb}{1.000000,1.000000,1.000000}%
\pgfsetstrokecolor{currentstroke}%
\pgfsetdash{}{0pt}%
\pgfpathmoveto{\pgfqpoint{6.008087in}{6.360043in}}%
\pgfpathcurveto{\pgfqpoint{6.019137in}{6.360043in}}{\pgfqpoint{6.029736in}{6.364433in}}{\pgfqpoint{6.037549in}{6.372246in}}%
\pgfpathcurveto{\pgfqpoint{6.045363in}{6.380060in}}{\pgfqpoint{6.049753in}{6.390659in}}{\pgfqpoint{6.049753in}{6.401709in}}%
\pgfpathcurveto{\pgfqpoint{6.049753in}{6.412759in}}{\pgfqpoint{6.045363in}{6.423358in}}{\pgfqpoint{6.037549in}{6.431172in}}%
\pgfpathcurveto{\pgfqpoint{6.029736in}{6.438986in}}{\pgfqpoint{6.019137in}{6.443376in}}{\pgfqpoint{6.008087in}{6.443376in}}%
\pgfpathcurveto{\pgfqpoint{5.997037in}{6.443376in}}{\pgfqpoint{5.986438in}{6.438986in}}{\pgfqpoint{5.978624in}{6.431172in}}%
\pgfpathcurveto{\pgfqpoint{5.970810in}{6.423358in}}{\pgfqpoint{5.966420in}{6.412759in}}{\pgfqpoint{5.966420in}{6.401709in}}%
\pgfpathcurveto{\pgfqpoint{5.966420in}{6.390659in}}{\pgfqpoint{5.970810in}{6.380060in}}{\pgfqpoint{5.978624in}{6.372246in}}%
\pgfpathcurveto{\pgfqpoint{5.986438in}{6.364433in}}{\pgfqpoint{5.997037in}{6.360043in}}{\pgfqpoint{6.008087in}{6.360043in}}%
\pgfpathclose%
\pgfusepath{stroke,fill}%
\end{pgfscope}%
\begin{pgfscope}%
\pgfpathrectangle{\pgfqpoint{0.526127in}{0.331635in}}{\pgfqpoint{9.300000in}{7.700000in}}%
\pgfusepath{clip}%
\pgfsetbuttcap%
\pgfsetroundjoin%
\definecolor{currentfill}{rgb}{1.000000,0.705882,0.509804}%
\pgfsetfillcolor{currentfill}%
\pgfsetlinewidth{0.481800pt}%
\definecolor{currentstroke}{rgb}{1.000000,1.000000,1.000000}%
\pgfsetstrokecolor{currentstroke}%
\pgfsetdash{}{0pt}%
\pgfpathmoveto{\pgfqpoint{7.439477in}{5.247476in}}%
\pgfpathcurveto{\pgfqpoint{7.450527in}{5.247476in}}{\pgfqpoint{7.461126in}{5.251867in}}{\pgfqpoint{7.468940in}{5.259680in}}%
\pgfpathcurveto{\pgfqpoint{7.476754in}{5.267494in}}{\pgfqpoint{7.481144in}{5.278093in}}{\pgfqpoint{7.481144in}{5.289143in}}%
\pgfpathcurveto{\pgfqpoint{7.481144in}{5.300193in}}{\pgfqpoint{7.476754in}{5.310792in}}{\pgfqpoint{7.468940in}{5.318606in}}%
\pgfpathcurveto{\pgfqpoint{7.461126in}{5.326419in}}{\pgfqpoint{7.450527in}{5.330810in}}{\pgfqpoint{7.439477in}{5.330810in}}%
\pgfpathcurveto{\pgfqpoint{7.428427in}{5.330810in}}{\pgfqpoint{7.417828in}{5.326419in}}{\pgfqpoint{7.410014in}{5.318606in}}%
\pgfpathcurveto{\pgfqpoint{7.402201in}{5.310792in}}{\pgfqpoint{7.397811in}{5.300193in}}{\pgfqpoint{7.397811in}{5.289143in}}%
\pgfpathcurveto{\pgfqpoint{7.397811in}{5.278093in}}{\pgfqpoint{7.402201in}{5.267494in}}{\pgfqpoint{7.410014in}{5.259680in}}%
\pgfpathcurveto{\pgfqpoint{7.417828in}{5.251867in}}{\pgfqpoint{7.428427in}{5.247476in}}{\pgfqpoint{7.439477in}{5.247476in}}%
\pgfpathclose%
\pgfusepath{stroke,fill}%
\end{pgfscope}%
\begin{pgfscope}%
\pgfpathrectangle{\pgfqpoint{0.526127in}{0.331635in}}{\pgfqpoint{9.300000in}{7.700000in}}%
\pgfusepath{clip}%
\pgfsetbuttcap%
\pgfsetroundjoin%
\definecolor{currentfill}{rgb}{1.000000,0.705882,0.509804}%
\pgfsetfillcolor{currentfill}%
\pgfsetlinewidth{0.481800pt}%
\definecolor{currentstroke}{rgb}{1.000000,1.000000,1.000000}%
\pgfsetstrokecolor{currentstroke}%
\pgfsetdash{}{0pt}%
\pgfpathmoveto{\pgfqpoint{7.682374in}{4.833782in}}%
\pgfpathcurveto{\pgfqpoint{7.693424in}{4.833782in}}{\pgfqpoint{7.704023in}{4.838173in}}{\pgfqpoint{7.711837in}{4.845986in}}%
\pgfpathcurveto{\pgfqpoint{7.719650in}{4.853800in}}{\pgfqpoint{7.724041in}{4.864399in}}{\pgfqpoint{7.724041in}{4.875449in}}%
\pgfpathcurveto{\pgfqpoint{7.724041in}{4.886499in}}{\pgfqpoint{7.719650in}{4.897098in}}{\pgfqpoint{7.711837in}{4.904912in}}%
\pgfpathcurveto{\pgfqpoint{7.704023in}{4.912725in}}{\pgfqpoint{7.693424in}{4.917116in}}{\pgfqpoint{7.682374in}{4.917116in}}%
\pgfpathcurveto{\pgfqpoint{7.671324in}{4.917116in}}{\pgfqpoint{7.660725in}{4.912725in}}{\pgfqpoint{7.652911in}{4.904912in}}%
\pgfpathcurveto{\pgfqpoint{7.645098in}{4.897098in}}{\pgfqpoint{7.640707in}{4.886499in}}{\pgfqpoint{7.640707in}{4.875449in}}%
\pgfpathcurveto{\pgfqpoint{7.640707in}{4.864399in}}{\pgfqpoint{7.645098in}{4.853800in}}{\pgfqpoint{7.652911in}{4.845986in}}%
\pgfpathcurveto{\pgfqpoint{7.660725in}{4.838173in}}{\pgfqpoint{7.671324in}{4.833782in}}{\pgfqpoint{7.682374in}{4.833782in}}%
\pgfpathclose%
\pgfusepath{stroke,fill}%
\end{pgfscope}%
\begin{pgfscope}%
\pgfpathrectangle{\pgfqpoint{0.526127in}{0.331635in}}{\pgfqpoint{9.300000in}{7.700000in}}%
\pgfusepath{clip}%
\pgfsetbuttcap%
\pgfsetroundjoin%
\definecolor{currentfill}{rgb}{1.000000,0.705882,0.509804}%
\pgfsetfillcolor{currentfill}%
\pgfsetlinewidth{0.481800pt}%
\definecolor{currentstroke}{rgb}{1.000000,1.000000,1.000000}%
\pgfsetstrokecolor{currentstroke}%
\pgfsetdash{}{0pt}%
\pgfpathmoveto{\pgfqpoint{5.929161in}{6.085863in}}%
\pgfpathcurveto{\pgfqpoint{5.940211in}{6.085863in}}{\pgfqpoint{5.950810in}{6.090253in}}{\pgfqpoint{5.958624in}{6.098067in}}%
\pgfpathcurveto{\pgfqpoint{5.966437in}{6.105880in}}{\pgfqpoint{5.970827in}{6.116479in}}{\pgfqpoint{5.970827in}{6.127529in}}%
\pgfpathcurveto{\pgfqpoint{5.970827in}{6.138579in}}{\pgfqpoint{5.966437in}{6.149178in}}{\pgfqpoint{5.958624in}{6.156992in}}%
\pgfpathcurveto{\pgfqpoint{5.950810in}{6.164806in}}{\pgfqpoint{5.940211in}{6.169196in}}{\pgfqpoint{5.929161in}{6.169196in}}%
\pgfpathcurveto{\pgfqpoint{5.918111in}{6.169196in}}{\pgfqpoint{5.907512in}{6.164806in}}{\pgfqpoint{5.899698in}{6.156992in}}%
\pgfpathcurveto{\pgfqpoint{5.891884in}{6.149178in}}{\pgfqpoint{5.887494in}{6.138579in}}{\pgfqpoint{5.887494in}{6.127529in}}%
\pgfpathcurveto{\pgfqpoint{5.887494in}{6.116479in}}{\pgfqpoint{5.891884in}{6.105880in}}{\pgfqpoint{5.899698in}{6.098067in}}%
\pgfpathcurveto{\pgfqpoint{5.907512in}{6.090253in}}{\pgfqpoint{5.918111in}{6.085863in}}{\pgfqpoint{5.929161in}{6.085863in}}%
\pgfpathclose%
\pgfusepath{stroke,fill}%
\end{pgfscope}%
\begin{pgfscope}%
\pgfpathrectangle{\pgfqpoint{0.526127in}{0.331635in}}{\pgfqpoint{9.300000in}{7.700000in}}%
\pgfusepath{clip}%
\pgfsetbuttcap%
\pgfsetroundjoin%
\definecolor{currentfill}{rgb}{1.000000,0.705882,0.509804}%
\pgfsetfillcolor{currentfill}%
\pgfsetlinewidth{0.481800pt}%
\definecolor{currentstroke}{rgb}{1.000000,1.000000,1.000000}%
\pgfsetstrokecolor{currentstroke}%
\pgfsetdash{}{0pt}%
\pgfpathmoveto{\pgfqpoint{8.724020in}{3.258644in}}%
\pgfpathcurveto{\pgfqpoint{8.735071in}{3.258644in}}{\pgfqpoint{8.745670in}{3.263034in}}{\pgfqpoint{8.753483in}{3.270848in}}%
\pgfpathcurveto{\pgfqpoint{8.761297in}{3.278662in}}{\pgfqpoint{8.765687in}{3.289261in}}{\pgfqpoint{8.765687in}{3.300311in}}%
\pgfpathcurveto{\pgfqpoint{8.765687in}{3.311361in}}{\pgfqpoint{8.761297in}{3.321960in}}{\pgfqpoint{8.753483in}{3.329774in}}%
\pgfpathcurveto{\pgfqpoint{8.745670in}{3.337587in}}{\pgfqpoint{8.735071in}{3.341978in}}{\pgfqpoint{8.724020in}{3.341978in}}%
\pgfpathcurveto{\pgfqpoint{8.712970in}{3.341978in}}{\pgfqpoint{8.702371in}{3.337587in}}{\pgfqpoint{8.694558in}{3.329774in}}%
\pgfpathcurveto{\pgfqpoint{8.686744in}{3.321960in}}{\pgfqpoint{8.682354in}{3.311361in}}{\pgfqpoint{8.682354in}{3.300311in}}%
\pgfpathcurveto{\pgfqpoint{8.682354in}{3.289261in}}{\pgfqpoint{8.686744in}{3.278662in}}{\pgfqpoint{8.694558in}{3.270848in}}%
\pgfpathcurveto{\pgfqpoint{8.702371in}{3.263034in}}{\pgfqpoint{8.712970in}{3.258644in}}{\pgfqpoint{8.724020in}{3.258644in}}%
\pgfpathclose%
\pgfusepath{stroke,fill}%
\end{pgfscope}%
\begin{pgfscope}%
\pgfpathrectangle{\pgfqpoint{0.526127in}{0.331635in}}{\pgfqpoint{9.300000in}{7.700000in}}%
\pgfusepath{clip}%
\pgfsetbuttcap%
\pgfsetroundjoin%
\definecolor{currentfill}{rgb}{1.000000,0.705882,0.509804}%
\pgfsetfillcolor{currentfill}%
\pgfsetlinewidth{0.481800pt}%
\definecolor{currentstroke}{rgb}{1.000000,1.000000,1.000000}%
\pgfsetstrokecolor{currentstroke}%
\pgfsetdash{}{0pt}%
\pgfpathmoveto{\pgfqpoint{6.318203in}{3.404579in}}%
\pgfpathcurveto{\pgfqpoint{6.329253in}{3.404579in}}{\pgfqpoint{6.339852in}{3.408969in}}{\pgfqpoint{6.347666in}{3.416783in}}%
\pgfpathcurveto{\pgfqpoint{6.355479in}{3.424596in}}{\pgfqpoint{6.359870in}{3.435196in}}{\pgfqpoint{6.359870in}{3.446246in}}%
\pgfpathcurveto{\pgfqpoint{6.359870in}{3.457296in}}{\pgfqpoint{6.355479in}{3.467895in}}{\pgfqpoint{6.347666in}{3.475708in}}%
\pgfpathcurveto{\pgfqpoint{6.339852in}{3.483522in}}{\pgfqpoint{6.329253in}{3.487912in}}{\pgfqpoint{6.318203in}{3.487912in}}%
\pgfpathcurveto{\pgfqpoint{6.307153in}{3.487912in}}{\pgfqpoint{6.296554in}{3.483522in}}{\pgfqpoint{6.288740in}{3.475708in}}%
\pgfpathcurveto{\pgfqpoint{6.280927in}{3.467895in}}{\pgfqpoint{6.276536in}{3.457296in}}{\pgfqpoint{6.276536in}{3.446246in}}%
\pgfpathcurveto{\pgfqpoint{6.276536in}{3.435196in}}{\pgfqpoint{6.280927in}{3.424596in}}{\pgfqpoint{6.288740in}{3.416783in}}%
\pgfpathcurveto{\pgfqpoint{6.296554in}{3.408969in}}{\pgfqpoint{6.307153in}{3.404579in}}{\pgfqpoint{6.318203in}{3.404579in}}%
\pgfpathclose%
\pgfusepath{stroke,fill}%
\end{pgfscope}%
\begin{pgfscope}%
\pgfpathrectangle{\pgfqpoint{0.526127in}{0.331635in}}{\pgfqpoint{9.300000in}{7.700000in}}%
\pgfusepath{clip}%
\pgfsetbuttcap%
\pgfsetroundjoin%
\definecolor{currentfill}{rgb}{0.552941,0.898039,0.631373}%
\pgfsetfillcolor{currentfill}%
\pgfsetlinewidth{0.481800pt}%
\definecolor{currentstroke}{rgb}{1.000000,1.000000,1.000000}%
\pgfsetstrokecolor{currentstroke}%
\pgfsetdash{}{0pt}%
\pgfpathmoveto{\pgfqpoint{7.719822in}{2.245611in}}%
\pgfpathcurveto{\pgfqpoint{7.730872in}{2.245611in}}{\pgfqpoint{7.741471in}{2.250001in}}{\pgfqpoint{7.749285in}{2.257815in}}%
\pgfpathcurveto{\pgfqpoint{7.757098in}{2.265628in}}{\pgfqpoint{7.761489in}{2.276227in}}{\pgfqpoint{7.761489in}{2.287277in}}%
\pgfpathcurveto{\pgfqpoint{7.761489in}{2.298327in}}{\pgfqpoint{7.757098in}{2.308926in}}{\pgfqpoint{7.749285in}{2.316740in}}%
\pgfpathcurveto{\pgfqpoint{7.741471in}{2.324554in}}{\pgfqpoint{7.730872in}{2.328944in}}{\pgfqpoint{7.719822in}{2.328944in}}%
\pgfpathcurveto{\pgfqpoint{7.708772in}{2.328944in}}{\pgfqpoint{7.698173in}{2.324554in}}{\pgfqpoint{7.690359in}{2.316740in}}%
\pgfpathcurveto{\pgfqpoint{7.682546in}{2.308926in}}{\pgfqpoint{7.678155in}{2.298327in}}{\pgfqpoint{7.678155in}{2.287277in}}%
\pgfpathcurveto{\pgfqpoint{7.678155in}{2.276227in}}{\pgfqpoint{7.682546in}{2.265628in}}{\pgfqpoint{7.690359in}{2.257815in}}%
\pgfpathcurveto{\pgfqpoint{7.698173in}{2.250001in}}{\pgfqpoint{7.708772in}{2.245611in}}{\pgfqpoint{7.719822in}{2.245611in}}%
\pgfpathclose%
\pgfusepath{stroke,fill}%
\end{pgfscope}%
\begin{pgfscope}%
\pgfpathrectangle{\pgfqpoint{0.526127in}{0.331635in}}{\pgfqpoint{9.300000in}{7.700000in}}%
\pgfusepath{clip}%
\pgfsetbuttcap%
\pgfsetroundjoin%
\definecolor{currentfill}{rgb}{0.552941,0.898039,0.631373}%
\pgfsetfillcolor{currentfill}%
\pgfsetlinewidth{0.481800pt}%
\definecolor{currentstroke}{rgb}{1.000000,1.000000,1.000000}%
\pgfsetstrokecolor{currentstroke}%
\pgfsetdash{}{0pt}%
\pgfpathmoveto{\pgfqpoint{5.865422in}{1.569557in}}%
\pgfpathcurveto{\pgfqpoint{5.876472in}{1.569557in}}{\pgfqpoint{5.887071in}{1.573947in}}{\pgfqpoint{5.894885in}{1.581761in}}%
\pgfpathcurveto{\pgfqpoint{5.902698in}{1.589574in}}{\pgfqpoint{5.907089in}{1.600173in}}{\pgfqpoint{5.907089in}{1.611223in}}%
\pgfpathcurveto{\pgfqpoint{5.907089in}{1.622274in}}{\pgfqpoint{5.902698in}{1.632873in}}{\pgfqpoint{5.894885in}{1.640686in}}%
\pgfpathcurveto{\pgfqpoint{5.887071in}{1.648500in}}{\pgfqpoint{5.876472in}{1.652890in}}{\pgfqpoint{5.865422in}{1.652890in}}%
\pgfpathcurveto{\pgfqpoint{5.854372in}{1.652890in}}{\pgfqpoint{5.843773in}{1.648500in}}{\pgfqpoint{5.835959in}{1.640686in}}%
\pgfpathcurveto{\pgfqpoint{5.828145in}{1.632873in}}{\pgfqpoint{5.823755in}{1.622274in}}{\pgfqpoint{5.823755in}{1.611223in}}%
\pgfpathcurveto{\pgfqpoint{5.823755in}{1.600173in}}{\pgfqpoint{5.828145in}{1.589574in}}{\pgfqpoint{5.835959in}{1.581761in}}%
\pgfpathcurveto{\pgfqpoint{5.843773in}{1.573947in}}{\pgfqpoint{5.854372in}{1.569557in}}{\pgfqpoint{5.865422in}{1.569557in}}%
\pgfpathclose%
\pgfusepath{stroke,fill}%
\end{pgfscope}%
\begin{pgfscope}%
\pgfpathrectangle{\pgfqpoint{0.526127in}{0.331635in}}{\pgfqpoint{9.300000in}{7.700000in}}%
\pgfusepath{clip}%
\pgfsetbuttcap%
\pgfsetroundjoin%
\definecolor{currentfill}{rgb}{0.552941,0.898039,0.631373}%
\pgfsetfillcolor{currentfill}%
\pgfsetlinewidth{0.481800pt}%
\definecolor{currentstroke}{rgb}{1.000000,1.000000,1.000000}%
\pgfsetstrokecolor{currentstroke}%
\pgfsetdash{}{0pt}%
\pgfpathmoveto{\pgfqpoint{6.692745in}{2.167260in}}%
\pgfpathcurveto{\pgfqpoint{6.703795in}{2.167260in}}{\pgfqpoint{6.714394in}{2.171651in}}{\pgfqpoint{6.722208in}{2.179464in}}%
\pgfpathcurveto{\pgfqpoint{6.730022in}{2.187278in}}{\pgfqpoint{6.734412in}{2.197877in}}{\pgfqpoint{6.734412in}{2.208927in}}%
\pgfpathcurveto{\pgfqpoint{6.734412in}{2.219977in}}{\pgfqpoint{6.730022in}{2.230576in}}{\pgfqpoint{6.722208in}{2.238390in}}%
\pgfpathcurveto{\pgfqpoint{6.714394in}{2.246203in}}{\pgfqpoint{6.703795in}{2.250594in}}{\pgfqpoint{6.692745in}{2.250594in}}%
\pgfpathcurveto{\pgfqpoint{6.681695in}{2.250594in}}{\pgfqpoint{6.671096in}{2.246203in}}{\pgfqpoint{6.663282in}{2.238390in}}%
\pgfpathcurveto{\pgfqpoint{6.655469in}{2.230576in}}{\pgfqpoint{6.651078in}{2.219977in}}{\pgfqpoint{6.651078in}{2.208927in}}%
\pgfpathcurveto{\pgfqpoint{6.651078in}{2.197877in}}{\pgfqpoint{6.655469in}{2.187278in}}{\pgfqpoint{6.663282in}{2.179464in}}%
\pgfpathcurveto{\pgfqpoint{6.671096in}{2.171651in}}{\pgfqpoint{6.681695in}{2.167260in}}{\pgfqpoint{6.692745in}{2.167260in}}%
\pgfpathclose%
\pgfusepath{stroke,fill}%
\end{pgfscope}%
\begin{pgfscope}%
\pgfpathrectangle{\pgfqpoint{0.526127in}{0.331635in}}{\pgfqpoint{9.300000in}{7.700000in}}%
\pgfusepath{clip}%
\pgfsetbuttcap%
\pgfsetroundjoin%
\definecolor{currentfill}{rgb}{0.552941,0.898039,0.631373}%
\pgfsetfillcolor{currentfill}%
\pgfsetlinewidth{0.481800pt}%
\definecolor{currentstroke}{rgb}{1.000000,1.000000,1.000000}%
\pgfsetstrokecolor{currentstroke}%
\pgfsetdash{}{0pt}%
\pgfpathmoveto{\pgfqpoint{2.147474in}{3.354463in}}%
\pgfpathcurveto{\pgfqpoint{2.158524in}{3.354463in}}{\pgfqpoint{2.169123in}{3.358853in}}{\pgfqpoint{2.176937in}{3.366667in}}%
\pgfpathcurveto{\pgfqpoint{2.184750in}{3.374481in}}{\pgfqpoint{2.189140in}{3.385080in}}{\pgfqpoint{2.189140in}{3.396130in}}%
\pgfpathcurveto{\pgfqpoint{2.189140in}{3.407180in}}{\pgfqpoint{2.184750in}{3.417779in}}{\pgfqpoint{2.176937in}{3.425593in}}%
\pgfpathcurveto{\pgfqpoint{2.169123in}{3.433406in}}{\pgfqpoint{2.158524in}{3.437796in}}{\pgfqpoint{2.147474in}{3.437796in}}%
\pgfpathcurveto{\pgfqpoint{2.136424in}{3.437796in}}{\pgfqpoint{2.125825in}{3.433406in}}{\pgfqpoint{2.118011in}{3.425593in}}%
\pgfpathcurveto{\pgfqpoint{2.110197in}{3.417779in}}{\pgfqpoint{2.105807in}{3.407180in}}{\pgfqpoint{2.105807in}{3.396130in}}%
\pgfpathcurveto{\pgfqpoint{2.105807in}{3.385080in}}{\pgfqpoint{2.110197in}{3.374481in}}{\pgfqpoint{2.118011in}{3.366667in}}%
\pgfpathcurveto{\pgfqpoint{2.125825in}{3.358853in}}{\pgfqpoint{2.136424in}{3.354463in}}{\pgfqpoint{2.147474in}{3.354463in}}%
\pgfpathclose%
\pgfusepath{stroke,fill}%
\end{pgfscope}%
\begin{pgfscope}%
\pgfpathrectangle{\pgfqpoint{0.526127in}{0.331635in}}{\pgfqpoint{9.300000in}{7.700000in}}%
\pgfusepath{clip}%
\pgfsetbuttcap%
\pgfsetroundjoin%
\definecolor{currentfill}{rgb}{0.552941,0.898039,0.631373}%
\pgfsetfillcolor{currentfill}%
\pgfsetlinewidth{0.481800pt}%
\definecolor{currentstroke}{rgb}{1.000000,1.000000,1.000000}%
\pgfsetstrokecolor{currentstroke}%
\pgfsetdash{}{0pt}%
\pgfpathmoveto{\pgfqpoint{7.116859in}{3.099327in}}%
\pgfpathcurveto{\pgfqpoint{7.127909in}{3.099327in}}{\pgfqpoint{7.138508in}{3.103717in}}{\pgfqpoint{7.146322in}{3.111531in}}%
\pgfpathcurveto{\pgfqpoint{7.154135in}{3.119345in}}{\pgfqpoint{7.158526in}{3.129944in}}{\pgfqpoint{7.158526in}{3.140994in}}%
\pgfpathcurveto{\pgfqpoint{7.158526in}{3.152044in}}{\pgfqpoint{7.154135in}{3.162643in}}{\pgfqpoint{7.146322in}{3.170456in}}%
\pgfpathcurveto{\pgfqpoint{7.138508in}{3.178270in}}{\pgfqpoint{7.127909in}{3.182660in}}{\pgfqpoint{7.116859in}{3.182660in}}%
\pgfpathcurveto{\pgfqpoint{7.105809in}{3.182660in}}{\pgfqpoint{7.095210in}{3.178270in}}{\pgfqpoint{7.087396in}{3.170456in}}%
\pgfpathcurveto{\pgfqpoint{7.079583in}{3.162643in}}{\pgfqpoint{7.075192in}{3.152044in}}{\pgfqpoint{7.075192in}{3.140994in}}%
\pgfpathcurveto{\pgfqpoint{7.075192in}{3.129944in}}{\pgfqpoint{7.079583in}{3.119345in}}{\pgfqpoint{7.087396in}{3.111531in}}%
\pgfpathcurveto{\pgfqpoint{7.095210in}{3.103717in}}{\pgfqpoint{7.105809in}{3.099327in}}{\pgfqpoint{7.116859in}{3.099327in}}%
\pgfpathclose%
\pgfusepath{stroke,fill}%
\end{pgfscope}%
\begin{pgfscope}%
\pgfpathrectangle{\pgfqpoint{0.526127in}{0.331635in}}{\pgfqpoint{9.300000in}{7.700000in}}%
\pgfusepath{clip}%
\pgfsetbuttcap%
\pgfsetroundjoin%
\definecolor{currentfill}{rgb}{0.552941,0.898039,0.631373}%
\pgfsetfillcolor{currentfill}%
\pgfsetlinewidth{0.481800pt}%
\definecolor{currentstroke}{rgb}{1.000000,1.000000,1.000000}%
\pgfsetstrokecolor{currentstroke}%
\pgfsetdash{}{0pt}%
\pgfpathmoveto{\pgfqpoint{3.558492in}{4.103567in}}%
\pgfpathcurveto{\pgfqpoint{3.569542in}{4.103567in}}{\pgfqpoint{3.580141in}{4.107957in}}{\pgfqpoint{3.587955in}{4.115771in}}%
\pgfpathcurveto{\pgfqpoint{3.595769in}{4.123584in}}{\pgfqpoint{3.600159in}{4.134183in}}{\pgfqpoint{3.600159in}{4.145234in}}%
\pgfpathcurveto{\pgfqpoint{3.600159in}{4.156284in}}{\pgfqpoint{3.595769in}{4.166883in}}{\pgfqpoint{3.587955in}{4.174696in}}%
\pgfpathcurveto{\pgfqpoint{3.580141in}{4.182510in}}{\pgfqpoint{3.569542in}{4.186900in}}{\pgfqpoint{3.558492in}{4.186900in}}%
\pgfpathcurveto{\pgfqpoint{3.547442in}{4.186900in}}{\pgfqpoint{3.536843in}{4.182510in}}{\pgfqpoint{3.529029in}{4.174696in}}%
\pgfpathcurveto{\pgfqpoint{3.521216in}{4.166883in}}{\pgfqpoint{3.516826in}{4.156284in}}{\pgfqpoint{3.516826in}{4.145234in}}%
\pgfpathcurveto{\pgfqpoint{3.516826in}{4.134183in}}{\pgfqpoint{3.521216in}{4.123584in}}{\pgfqpoint{3.529029in}{4.115771in}}%
\pgfpathcurveto{\pgfqpoint{3.536843in}{4.107957in}}{\pgfqpoint{3.547442in}{4.103567in}}{\pgfqpoint{3.558492in}{4.103567in}}%
\pgfpathclose%
\pgfusepath{stroke,fill}%
\end{pgfscope}%
\begin{pgfscope}%
\pgfpathrectangle{\pgfqpoint{0.526127in}{0.331635in}}{\pgfqpoint{9.300000in}{7.700000in}}%
\pgfusepath{clip}%
\pgfsetbuttcap%
\pgfsetroundjoin%
\definecolor{currentfill}{rgb}{0.552941,0.898039,0.631373}%
\pgfsetfillcolor{currentfill}%
\pgfsetlinewidth{0.481800pt}%
\definecolor{currentstroke}{rgb}{1.000000,1.000000,1.000000}%
\pgfsetstrokecolor{currentstroke}%
\pgfsetdash{}{0pt}%
\pgfpathmoveto{\pgfqpoint{3.383051in}{1.527628in}}%
\pgfpathcurveto{\pgfqpoint{3.394101in}{1.527628in}}{\pgfqpoint{3.404700in}{1.532019in}}{\pgfqpoint{3.412514in}{1.539832in}}%
\pgfpathcurveto{\pgfqpoint{3.420328in}{1.547646in}}{\pgfqpoint{3.424718in}{1.558245in}}{\pgfqpoint{3.424718in}{1.569295in}}%
\pgfpathcurveto{\pgfqpoint{3.424718in}{1.580345in}}{\pgfqpoint{3.420328in}{1.590944in}}{\pgfqpoint{3.412514in}{1.598758in}}%
\pgfpathcurveto{\pgfqpoint{3.404700in}{1.606571in}}{\pgfqpoint{3.394101in}{1.610962in}}{\pgfqpoint{3.383051in}{1.610962in}}%
\pgfpathcurveto{\pgfqpoint{3.372001in}{1.610962in}}{\pgfqpoint{3.361402in}{1.606571in}}{\pgfqpoint{3.353588in}{1.598758in}}%
\pgfpathcurveto{\pgfqpoint{3.345775in}{1.590944in}}{\pgfqpoint{3.341385in}{1.580345in}}{\pgfqpoint{3.341385in}{1.569295in}}%
\pgfpathcurveto{\pgfqpoint{3.341385in}{1.558245in}}{\pgfqpoint{3.345775in}{1.547646in}}{\pgfqpoint{3.353588in}{1.539832in}}%
\pgfpathcurveto{\pgfqpoint{3.361402in}{1.532019in}}{\pgfqpoint{3.372001in}{1.527628in}}{\pgfqpoint{3.383051in}{1.527628in}}%
\pgfpathclose%
\pgfusepath{stroke,fill}%
\end{pgfscope}%
\begin{pgfscope}%
\pgfpathrectangle{\pgfqpoint{0.526127in}{0.331635in}}{\pgfqpoint{9.300000in}{7.700000in}}%
\pgfusepath{clip}%
\pgfsetbuttcap%
\pgfsetroundjoin%
\definecolor{currentfill}{rgb}{0.552941,0.898039,0.631373}%
\pgfsetfillcolor{currentfill}%
\pgfsetlinewidth{0.481800pt}%
\definecolor{currentstroke}{rgb}{1.000000,1.000000,1.000000}%
\pgfsetstrokecolor{currentstroke}%
\pgfsetdash{}{0pt}%
\pgfpathmoveto{\pgfqpoint{6.701250in}{2.564722in}}%
\pgfpathcurveto{\pgfqpoint{6.712300in}{2.564722in}}{\pgfqpoint{6.722899in}{2.569113in}}{\pgfqpoint{6.730713in}{2.576926in}}%
\pgfpathcurveto{\pgfqpoint{6.738526in}{2.584740in}}{\pgfqpoint{6.742917in}{2.595339in}}{\pgfqpoint{6.742917in}{2.606389in}}%
\pgfpathcurveto{\pgfqpoint{6.742917in}{2.617439in}}{\pgfqpoint{6.738526in}{2.628038in}}{\pgfqpoint{6.730713in}{2.635852in}}%
\pgfpathcurveto{\pgfqpoint{6.722899in}{2.643666in}}{\pgfqpoint{6.712300in}{2.648056in}}{\pgfqpoint{6.701250in}{2.648056in}}%
\pgfpathcurveto{\pgfqpoint{6.690200in}{2.648056in}}{\pgfqpoint{6.679601in}{2.643666in}}{\pgfqpoint{6.671787in}{2.635852in}}%
\pgfpathcurveto{\pgfqpoint{6.663973in}{2.628038in}}{\pgfqpoint{6.659583in}{2.617439in}}{\pgfqpoint{6.659583in}{2.606389in}}%
\pgfpathcurveto{\pgfqpoint{6.659583in}{2.595339in}}{\pgfqpoint{6.663973in}{2.584740in}}{\pgfqpoint{6.671787in}{2.576926in}}%
\pgfpathcurveto{\pgfqpoint{6.679601in}{2.569113in}}{\pgfqpoint{6.690200in}{2.564722in}}{\pgfqpoint{6.701250in}{2.564722in}}%
\pgfpathclose%
\pgfusepath{stroke,fill}%
\end{pgfscope}%
\begin{pgfscope}%
\pgfpathrectangle{\pgfqpoint{0.526127in}{0.331635in}}{\pgfqpoint{9.300000in}{7.700000in}}%
\pgfusepath{clip}%
\pgfsetbuttcap%
\pgfsetroundjoin%
\definecolor{currentfill}{rgb}{0.552941,0.898039,0.631373}%
\pgfsetfillcolor{currentfill}%
\pgfsetlinewidth{0.481800pt}%
\definecolor{currentstroke}{rgb}{1.000000,1.000000,1.000000}%
\pgfsetstrokecolor{currentstroke}%
\pgfsetdash{}{0pt}%
\pgfpathmoveto{\pgfqpoint{6.060531in}{3.385094in}}%
\pgfpathcurveto{\pgfqpoint{6.071581in}{3.385094in}}{\pgfqpoint{6.082180in}{3.389484in}}{\pgfqpoint{6.089994in}{3.397298in}}%
\pgfpathcurveto{\pgfqpoint{6.097808in}{3.405111in}}{\pgfqpoint{6.102198in}{3.415711in}}{\pgfqpoint{6.102198in}{3.426761in}}%
\pgfpathcurveto{\pgfqpoint{6.102198in}{3.437811in}}{\pgfqpoint{6.097808in}{3.448410in}}{\pgfqpoint{6.089994in}{3.456223in}}%
\pgfpathcurveto{\pgfqpoint{6.082180in}{3.464037in}}{\pgfqpoint{6.071581in}{3.468427in}}{\pgfqpoint{6.060531in}{3.468427in}}%
\pgfpathcurveto{\pgfqpoint{6.049481in}{3.468427in}}{\pgfqpoint{6.038882in}{3.464037in}}{\pgfqpoint{6.031068in}{3.456223in}}%
\pgfpathcurveto{\pgfqpoint{6.023255in}{3.448410in}}{\pgfqpoint{6.018865in}{3.437811in}}{\pgfqpoint{6.018865in}{3.426761in}}%
\pgfpathcurveto{\pgfqpoint{6.018865in}{3.415711in}}{\pgfqpoint{6.023255in}{3.405111in}}{\pgfqpoint{6.031068in}{3.397298in}}%
\pgfpathcurveto{\pgfqpoint{6.038882in}{3.389484in}}{\pgfqpoint{6.049481in}{3.385094in}}{\pgfqpoint{6.060531in}{3.385094in}}%
\pgfpathclose%
\pgfusepath{stroke,fill}%
\end{pgfscope}%
\begin{pgfscope}%
\pgfpathrectangle{\pgfqpoint{0.526127in}{0.331635in}}{\pgfqpoint{9.300000in}{7.700000in}}%
\pgfusepath{clip}%
\pgfsetbuttcap%
\pgfsetroundjoin%
\definecolor{currentfill}{rgb}{0.552941,0.898039,0.631373}%
\pgfsetfillcolor{currentfill}%
\pgfsetlinewidth{0.481800pt}%
\definecolor{currentstroke}{rgb}{1.000000,1.000000,1.000000}%
\pgfsetstrokecolor{currentstroke}%
\pgfsetdash{}{0pt}%
\pgfpathmoveto{\pgfqpoint{3.545333in}{4.472892in}}%
\pgfpathcurveto{\pgfqpoint{3.556383in}{4.472892in}}{\pgfqpoint{3.566982in}{4.477282in}}{\pgfqpoint{3.574796in}{4.485096in}}%
\pgfpathcurveto{\pgfqpoint{3.582609in}{4.492910in}}{\pgfqpoint{3.586999in}{4.503509in}}{\pgfqpoint{3.586999in}{4.514559in}}%
\pgfpathcurveto{\pgfqpoint{3.586999in}{4.525609in}}{\pgfqpoint{3.582609in}{4.536208in}}{\pgfqpoint{3.574796in}{4.544021in}}%
\pgfpathcurveto{\pgfqpoint{3.566982in}{4.551835in}}{\pgfqpoint{3.556383in}{4.556225in}}{\pgfqpoint{3.545333in}{4.556225in}}%
\pgfpathcurveto{\pgfqpoint{3.534283in}{4.556225in}}{\pgfqpoint{3.523684in}{4.551835in}}{\pgfqpoint{3.515870in}{4.544021in}}%
\pgfpathcurveto{\pgfqpoint{3.508056in}{4.536208in}}{\pgfqpoint{3.503666in}{4.525609in}}{\pgfqpoint{3.503666in}{4.514559in}}%
\pgfpathcurveto{\pgfqpoint{3.503666in}{4.503509in}}{\pgfqpoint{3.508056in}{4.492910in}}{\pgfqpoint{3.515870in}{4.485096in}}%
\pgfpathcurveto{\pgfqpoint{3.523684in}{4.477282in}}{\pgfqpoint{3.534283in}{4.472892in}}{\pgfqpoint{3.545333in}{4.472892in}}%
\pgfpathclose%
\pgfusepath{stroke,fill}%
\end{pgfscope}%
\begin{pgfscope}%
\pgfpathrectangle{\pgfqpoint{0.526127in}{0.331635in}}{\pgfqpoint{9.300000in}{7.700000in}}%
\pgfusepath{clip}%
\pgfsetbuttcap%
\pgfsetroundjoin%
\definecolor{currentfill}{rgb}{0.552941,0.898039,0.631373}%
\pgfsetfillcolor{currentfill}%
\pgfsetlinewidth{0.481800pt}%
\definecolor{currentstroke}{rgb}{1.000000,1.000000,1.000000}%
\pgfsetstrokecolor{currentstroke}%
\pgfsetdash{}{0pt}%
\pgfpathmoveto{\pgfqpoint{2.060713in}{2.880644in}}%
\pgfpathcurveto{\pgfqpoint{2.071764in}{2.880644in}}{\pgfqpoint{2.082363in}{2.885034in}}{\pgfqpoint{2.090176in}{2.892848in}}%
\pgfpathcurveto{\pgfqpoint{2.097990in}{2.900661in}}{\pgfqpoint{2.102380in}{2.911261in}}{\pgfqpoint{2.102380in}{2.922311in}}%
\pgfpathcurveto{\pgfqpoint{2.102380in}{2.933361in}}{\pgfqpoint{2.097990in}{2.943960in}}{\pgfqpoint{2.090176in}{2.951773in}}%
\pgfpathcurveto{\pgfqpoint{2.082363in}{2.959587in}}{\pgfqpoint{2.071764in}{2.963977in}}{\pgfqpoint{2.060713in}{2.963977in}}%
\pgfpathcurveto{\pgfqpoint{2.049663in}{2.963977in}}{\pgfqpoint{2.039064in}{2.959587in}}{\pgfqpoint{2.031251in}{2.951773in}}%
\pgfpathcurveto{\pgfqpoint{2.023437in}{2.943960in}}{\pgfqpoint{2.019047in}{2.933361in}}{\pgfqpoint{2.019047in}{2.922311in}}%
\pgfpathcurveto{\pgfqpoint{2.019047in}{2.911261in}}{\pgfqpoint{2.023437in}{2.900661in}}{\pgfqpoint{2.031251in}{2.892848in}}%
\pgfpathcurveto{\pgfqpoint{2.039064in}{2.885034in}}{\pgfqpoint{2.049663in}{2.880644in}}{\pgfqpoint{2.060713in}{2.880644in}}%
\pgfpathclose%
\pgfusepath{stroke,fill}%
\end{pgfscope}%
\begin{pgfscope}%
\pgfpathrectangle{\pgfqpoint{0.526127in}{0.331635in}}{\pgfqpoint{9.300000in}{7.700000in}}%
\pgfusepath{clip}%
\pgfsetbuttcap%
\pgfsetroundjoin%
\definecolor{currentfill}{rgb}{0.552941,0.898039,0.631373}%
\pgfsetfillcolor{currentfill}%
\pgfsetlinewidth{0.481800pt}%
\definecolor{currentstroke}{rgb}{1.000000,1.000000,1.000000}%
\pgfsetstrokecolor{currentstroke}%
\pgfsetdash{}{0pt}%
\pgfpathmoveto{\pgfqpoint{9.178578in}{4.662248in}}%
\pgfpathcurveto{\pgfqpoint{9.189628in}{4.662248in}}{\pgfqpoint{9.200227in}{4.666638in}}{\pgfqpoint{9.208041in}{4.674452in}}%
\pgfpathcurveto{\pgfqpoint{9.215854in}{4.682265in}}{\pgfqpoint{9.220244in}{4.692865in}}{\pgfqpoint{9.220244in}{4.703915in}}%
\pgfpathcurveto{\pgfqpoint{9.220244in}{4.714965in}}{\pgfqpoint{9.215854in}{4.725564in}}{\pgfqpoint{9.208041in}{4.733377in}}%
\pgfpathcurveto{\pgfqpoint{9.200227in}{4.741191in}}{\pgfqpoint{9.189628in}{4.745581in}}{\pgfqpoint{9.178578in}{4.745581in}}%
\pgfpathcurveto{\pgfqpoint{9.167528in}{4.745581in}}{\pgfqpoint{9.156929in}{4.741191in}}{\pgfqpoint{9.149115in}{4.733377in}}%
\pgfpathcurveto{\pgfqpoint{9.141301in}{4.725564in}}{\pgfqpoint{9.136911in}{4.714965in}}{\pgfqpoint{9.136911in}{4.703915in}}%
\pgfpathcurveto{\pgfqpoint{9.136911in}{4.692865in}}{\pgfqpoint{9.141301in}{4.682265in}}{\pgfqpoint{9.149115in}{4.674452in}}%
\pgfpathcurveto{\pgfqpoint{9.156929in}{4.666638in}}{\pgfqpoint{9.167528in}{4.662248in}}{\pgfqpoint{9.178578in}{4.662248in}}%
\pgfpathclose%
\pgfusepath{stroke,fill}%
\end{pgfscope}%
\begin{pgfscope}%
\pgfpathrectangle{\pgfqpoint{0.526127in}{0.331635in}}{\pgfqpoint{9.300000in}{7.700000in}}%
\pgfusepath{clip}%
\pgfsetbuttcap%
\pgfsetroundjoin%
\definecolor{currentfill}{rgb}{0.552941,0.898039,0.631373}%
\pgfsetfillcolor{currentfill}%
\pgfsetlinewidth{0.481800pt}%
\definecolor{currentstroke}{rgb}{1.000000,1.000000,1.000000}%
\pgfsetstrokecolor{currentstroke}%
\pgfsetdash{}{0pt}%
\pgfpathmoveto{\pgfqpoint{2.528674in}{2.413856in}}%
\pgfpathcurveto{\pgfqpoint{2.539724in}{2.413856in}}{\pgfqpoint{2.550323in}{2.418246in}}{\pgfqpoint{2.558137in}{2.426060in}}%
\pgfpathcurveto{\pgfqpoint{2.565950in}{2.433873in}}{\pgfqpoint{2.570341in}{2.444472in}}{\pgfqpoint{2.570341in}{2.455523in}}%
\pgfpathcurveto{\pgfqpoint{2.570341in}{2.466573in}}{\pgfqpoint{2.565950in}{2.477172in}}{\pgfqpoint{2.558137in}{2.484985in}}%
\pgfpathcurveto{\pgfqpoint{2.550323in}{2.492799in}}{\pgfqpoint{2.539724in}{2.497189in}}{\pgfqpoint{2.528674in}{2.497189in}}%
\pgfpathcurveto{\pgfqpoint{2.517624in}{2.497189in}}{\pgfqpoint{2.507025in}{2.492799in}}{\pgfqpoint{2.499211in}{2.484985in}}%
\pgfpathcurveto{\pgfqpoint{2.491397in}{2.477172in}}{\pgfqpoint{2.487007in}{2.466573in}}{\pgfqpoint{2.487007in}{2.455523in}}%
\pgfpathcurveto{\pgfqpoint{2.487007in}{2.444472in}}{\pgfqpoint{2.491397in}{2.433873in}}{\pgfqpoint{2.499211in}{2.426060in}}%
\pgfpathcurveto{\pgfqpoint{2.507025in}{2.418246in}}{\pgfqpoint{2.517624in}{2.413856in}}{\pgfqpoint{2.528674in}{2.413856in}}%
\pgfpathclose%
\pgfusepath{stroke,fill}%
\end{pgfscope}%
\begin{pgfscope}%
\pgfpathrectangle{\pgfqpoint{0.526127in}{0.331635in}}{\pgfqpoint{9.300000in}{7.700000in}}%
\pgfusepath{clip}%
\pgfsetbuttcap%
\pgfsetroundjoin%
\definecolor{currentfill}{rgb}{0.552941,0.898039,0.631373}%
\pgfsetfillcolor{currentfill}%
\pgfsetlinewidth{0.481800pt}%
\definecolor{currentstroke}{rgb}{1.000000,1.000000,1.000000}%
\pgfsetstrokecolor{currentstroke}%
\pgfsetdash{}{0pt}%
\pgfpathmoveto{\pgfqpoint{2.638599in}{2.924571in}}%
\pgfpathcurveto{\pgfqpoint{2.649649in}{2.924571in}}{\pgfqpoint{2.660248in}{2.928962in}}{\pgfqpoint{2.668062in}{2.936775in}}%
\pgfpathcurveto{\pgfqpoint{2.675876in}{2.944589in}}{\pgfqpoint{2.680266in}{2.955188in}}{\pgfqpoint{2.680266in}{2.966238in}}%
\pgfpathcurveto{\pgfqpoint{2.680266in}{2.977288in}}{\pgfqpoint{2.675876in}{2.987887in}}{\pgfqpoint{2.668062in}{2.995701in}}%
\pgfpathcurveto{\pgfqpoint{2.660248in}{3.003515in}}{\pgfqpoint{2.649649in}{3.007905in}}{\pgfqpoint{2.638599in}{3.007905in}}%
\pgfpathcurveto{\pgfqpoint{2.627549in}{3.007905in}}{\pgfqpoint{2.616950in}{3.003515in}}{\pgfqpoint{2.609137in}{2.995701in}}%
\pgfpathcurveto{\pgfqpoint{2.601323in}{2.987887in}}{\pgfqpoint{2.596933in}{2.977288in}}{\pgfqpoint{2.596933in}{2.966238in}}%
\pgfpathcurveto{\pgfqpoint{2.596933in}{2.955188in}}{\pgfqpoint{2.601323in}{2.944589in}}{\pgfqpoint{2.609137in}{2.936775in}}%
\pgfpathcurveto{\pgfqpoint{2.616950in}{2.928962in}}{\pgfqpoint{2.627549in}{2.924571in}}{\pgfqpoint{2.638599in}{2.924571in}}%
\pgfpathclose%
\pgfusepath{stroke,fill}%
\end{pgfscope}%
\begin{pgfscope}%
\pgfpathrectangle{\pgfqpoint{0.526127in}{0.331635in}}{\pgfqpoint{9.300000in}{7.700000in}}%
\pgfusepath{clip}%
\pgfsetbuttcap%
\pgfsetroundjoin%
\definecolor{currentfill}{rgb}{0.552941,0.898039,0.631373}%
\pgfsetfillcolor{currentfill}%
\pgfsetlinewidth{0.481800pt}%
\definecolor{currentstroke}{rgb}{1.000000,1.000000,1.000000}%
\pgfsetstrokecolor{currentstroke}%
\pgfsetdash{}{0pt}%
\pgfpathmoveto{\pgfqpoint{2.718231in}{2.230056in}}%
\pgfpathcurveto{\pgfqpoint{2.729281in}{2.230056in}}{\pgfqpoint{2.739880in}{2.234446in}}{\pgfqpoint{2.747694in}{2.242259in}}%
\pgfpathcurveto{\pgfqpoint{2.755508in}{2.250073in}}{\pgfqpoint{2.759898in}{2.260672in}}{\pgfqpoint{2.759898in}{2.271722in}}%
\pgfpathcurveto{\pgfqpoint{2.759898in}{2.282772in}}{\pgfqpoint{2.755508in}{2.293371in}}{\pgfqpoint{2.747694in}{2.301185in}}%
\pgfpathcurveto{\pgfqpoint{2.739880in}{2.308999in}}{\pgfqpoint{2.729281in}{2.313389in}}{\pgfqpoint{2.718231in}{2.313389in}}%
\pgfpathcurveto{\pgfqpoint{2.707181in}{2.313389in}}{\pgfqpoint{2.696582in}{2.308999in}}{\pgfqpoint{2.688768in}{2.301185in}}%
\pgfpathcurveto{\pgfqpoint{2.680955in}{2.293371in}}{\pgfqpoint{2.676564in}{2.282772in}}{\pgfqpoint{2.676564in}{2.271722in}}%
\pgfpathcurveto{\pgfqpoint{2.676564in}{2.260672in}}{\pgfqpoint{2.680955in}{2.250073in}}{\pgfqpoint{2.688768in}{2.242259in}}%
\pgfpathcurveto{\pgfqpoint{2.696582in}{2.234446in}}{\pgfqpoint{2.707181in}{2.230056in}}{\pgfqpoint{2.718231in}{2.230056in}}%
\pgfpathclose%
\pgfusepath{stroke,fill}%
\end{pgfscope}%
\begin{pgfscope}%
\pgfpathrectangle{\pgfqpoint{0.526127in}{0.331635in}}{\pgfqpoint{9.300000in}{7.700000in}}%
\pgfusepath{clip}%
\pgfsetbuttcap%
\pgfsetroundjoin%
\definecolor{currentfill}{rgb}{0.552941,0.898039,0.631373}%
\pgfsetfillcolor{currentfill}%
\pgfsetlinewidth{0.481800pt}%
\definecolor{currentstroke}{rgb}{1.000000,1.000000,1.000000}%
\pgfsetstrokecolor{currentstroke}%
\pgfsetdash{}{0pt}%
\pgfpathmoveto{\pgfqpoint{1.976182in}{2.480597in}}%
\pgfpathcurveto{\pgfqpoint{1.987232in}{2.480597in}}{\pgfqpoint{1.997831in}{2.484987in}}{\pgfqpoint{2.005645in}{2.492801in}}%
\pgfpathcurveto{\pgfqpoint{2.013459in}{2.500614in}}{\pgfqpoint{2.017849in}{2.511213in}}{\pgfqpoint{2.017849in}{2.522263in}}%
\pgfpathcurveto{\pgfqpoint{2.017849in}{2.533314in}}{\pgfqpoint{2.013459in}{2.543913in}}{\pgfqpoint{2.005645in}{2.551726in}}%
\pgfpathcurveto{\pgfqpoint{1.997831in}{2.559540in}}{\pgfqpoint{1.987232in}{2.563930in}}{\pgfqpoint{1.976182in}{2.563930in}}%
\pgfpathcurveto{\pgfqpoint{1.965132in}{2.563930in}}{\pgfqpoint{1.954533in}{2.559540in}}{\pgfqpoint{1.946719in}{2.551726in}}%
\pgfpathcurveto{\pgfqpoint{1.938906in}{2.543913in}}{\pgfqpoint{1.934515in}{2.533314in}}{\pgfqpoint{1.934515in}{2.522263in}}%
\pgfpathcurveto{\pgfqpoint{1.934515in}{2.511213in}}{\pgfqpoint{1.938906in}{2.500614in}}{\pgfqpoint{1.946719in}{2.492801in}}%
\pgfpathcurveto{\pgfqpoint{1.954533in}{2.484987in}}{\pgfqpoint{1.965132in}{2.480597in}}{\pgfqpoint{1.976182in}{2.480597in}}%
\pgfpathclose%
\pgfusepath{stroke,fill}%
\end{pgfscope}%
\begin{pgfscope}%
\pgfpathrectangle{\pgfqpoint{0.526127in}{0.331635in}}{\pgfqpoint{9.300000in}{7.700000in}}%
\pgfusepath{clip}%
\pgfsetbuttcap%
\pgfsetroundjoin%
\definecolor{currentfill}{rgb}{0.552941,0.898039,0.631373}%
\pgfsetfillcolor{currentfill}%
\pgfsetlinewidth{0.481800pt}%
\definecolor{currentstroke}{rgb}{1.000000,1.000000,1.000000}%
\pgfsetstrokecolor{currentstroke}%
\pgfsetdash{}{0pt}%
\pgfpathmoveto{\pgfqpoint{2.309460in}{2.873355in}}%
\pgfpathcurveto{\pgfqpoint{2.320510in}{2.873355in}}{\pgfqpoint{2.331109in}{2.877745in}}{\pgfqpoint{2.338923in}{2.885559in}}%
\pgfpathcurveto{\pgfqpoint{2.346736in}{2.893372in}}{\pgfqpoint{2.351127in}{2.903971in}}{\pgfqpoint{2.351127in}{2.915022in}}%
\pgfpathcurveto{\pgfqpoint{2.351127in}{2.926072in}}{\pgfqpoint{2.346736in}{2.936671in}}{\pgfqpoint{2.338923in}{2.944484in}}%
\pgfpathcurveto{\pgfqpoint{2.331109in}{2.952298in}}{\pgfqpoint{2.320510in}{2.956688in}}{\pgfqpoint{2.309460in}{2.956688in}}%
\pgfpathcurveto{\pgfqpoint{2.298410in}{2.956688in}}{\pgfqpoint{2.287811in}{2.952298in}}{\pgfqpoint{2.279997in}{2.944484in}}%
\pgfpathcurveto{\pgfqpoint{2.272184in}{2.936671in}}{\pgfqpoint{2.267793in}{2.926072in}}{\pgfqpoint{2.267793in}{2.915022in}}%
\pgfpathcurveto{\pgfqpoint{2.267793in}{2.903971in}}{\pgfqpoint{2.272184in}{2.893372in}}{\pgfqpoint{2.279997in}{2.885559in}}%
\pgfpathcurveto{\pgfqpoint{2.287811in}{2.877745in}}{\pgfqpoint{2.298410in}{2.873355in}}{\pgfqpoint{2.309460in}{2.873355in}}%
\pgfpathclose%
\pgfusepath{stroke,fill}%
\end{pgfscope}%
\begin{pgfscope}%
\pgfpathrectangle{\pgfqpoint{0.526127in}{0.331635in}}{\pgfqpoint{9.300000in}{7.700000in}}%
\pgfusepath{clip}%
\pgfsetbuttcap%
\pgfsetroundjoin%
\definecolor{currentfill}{rgb}{0.552941,0.898039,0.631373}%
\pgfsetfillcolor{currentfill}%
\pgfsetlinewidth{0.481800pt}%
\definecolor{currentstroke}{rgb}{1.000000,1.000000,1.000000}%
\pgfsetstrokecolor{currentstroke}%
\pgfsetdash{}{0pt}%
\pgfpathmoveto{\pgfqpoint{1.801343in}{3.465174in}}%
\pgfpathcurveto{\pgfqpoint{1.812393in}{3.465174in}}{\pgfqpoint{1.822992in}{3.469565in}}{\pgfqpoint{1.830806in}{3.477378in}}%
\pgfpathcurveto{\pgfqpoint{1.838619in}{3.485192in}}{\pgfqpoint{1.843009in}{3.495791in}}{\pgfqpoint{1.843009in}{3.506841in}}%
\pgfpathcurveto{\pgfqpoint{1.843009in}{3.517891in}}{\pgfqpoint{1.838619in}{3.528490in}}{\pgfqpoint{1.830806in}{3.536304in}}%
\pgfpathcurveto{\pgfqpoint{1.822992in}{3.544117in}}{\pgfqpoint{1.812393in}{3.548508in}}{\pgfqpoint{1.801343in}{3.548508in}}%
\pgfpathcurveto{\pgfqpoint{1.790293in}{3.548508in}}{\pgfqpoint{1.779694in}{3.544117in}}{\pgfqpoint{1.771880in}{3.536304in}}%
\pgfpathcurveto{\pgfqpoint{1.764066in}{3.528490in}}{\pgfqpoint{1.759676in}{3.517891in}}{\pgfqpoint{1.759676in}{3.506841in}}%
\pgfpathcurveto{\pgfqpoint{1.759676in}{3.495791in}}{\pgfqpoint{1.764066in}{3.485192in}}{\pgfqpoint{1.771880in}{3.477378in}}%
\pgfpathcurveto{\pgfqpoint{1.779694in}{3.469565in}}{\pgfqpoint{1.790293in}{3.465174in}}{\pgfqpoint{1.801343in}{3.465174in}}%
\pgfpathclose%
\pgfusepath{stroke,fill}%
\end{pgfscope}%
\begin{pgfscope}%
\pgfpathrectangle{\pgfqpoint{0.526127in}{0.331635in}}{\pgfqpoint{9.300000in}{7.700000in}}%
\pgfusepath{clip}%
\pgfsetbuttcap%
\pgfsetroundjoin%
\definecolor{currentfill}{rgb}{0.552941,0.898039,0.631373}%
\pgfsetfillcolor{currentfill}%
\pgfsetlinewidth{0.481800pt}%
\definecolor{currentstroke}{rgb}{1.000000,1.000000,1.000000}%
\pgfsetstrokecolor{currentstroke}%
\pgfsetdash{}{0pt}%
\pgfpathmoveto{\pgfqpoint{2.821384in}{2.771711in}}%
\pgfpathcurveto{\pgfqpoint{2.832434in}{2.771711in}}{\pgfqpoint{2.843033in}{2.776101in}}{\pgfqpoint{2.850847in}{2.783914in}}%
\pgfpathcurveto{\pgfqpoint{2.858660in}{2.791728in}}{\pgfqpoint{2.863051in}{2.802327in}}{\pgfqpoint{2.863051in}{2.813377in}}%
\pgfpathcurveto{\pgfqpoint{2.863051in}{2.824427in}}{\pgfqpoint{2.858660in}{2.835026in}}{\pgfqpoint{2.850847in}{2.842840in}}%
\pgfpathcurveto{\pgfqpoint{2.843033in}{2.850654in}}{\pgfqpoint{2.832434in}{2.855044in}}{\pgfqpoint{2.821384in}{2.855044in}}%
\pgfpathcurveto{\pgfqpoint{2.810334in}{2.855044in}}{\pgfqpoint{2.799735in}{2.850654in}}{\pgfqpoint{2.791921in}{2.842840in}}%
\pgfpathcurveto{\pgfqpoint{2.784107in}{2.835026in}}{\pgfqpoint{2.779717in}{2.824427in}}{\pgfqpoint{2.779717in}{2.813377in}}%
\pgfpathcurveto{\pgfqpoint{2.779717in}{2.802327in}}{\pgfqpoint{2.784107in}{2.791728in}}{\pgfqpoint{2.791921in}{2.783914in}}%
\pgfpathcurveto{\pgfqpoint{2.799735in}{2.776101in}}{\pgfqpoint{2.810334in}{2.771711in}}{\pgfqpoint{2.821384in}{2.771711in}}%
\pgfpathclose%
\pgfusepath{stroke,fill}%
\end{pgfscope}%
\begin{pgfscope}%
\pgfpathrectangle{\pgfqpoint{0.526127in}{0.331635in}}{\pgfqpoint{9.300000in}{7.700000in}}%
\pgfusepath{clip}%
\pgfsetbuttcap%
\pgfsetroundjoin%
\definecolor{currentfill}{rgb}{0.552941,0.898039,0.631373}%
\pgfsetfillcolor{currentfill}%
\pgfsetlinewidth{0.481800pt}%
\definecolor{currentstroke}{rgb}{1.000000,1.000000,1.000000}%
\pgfsetstrokecolor{currentstroke}%
\pgfsetdash{}{0pt}%
\pgfpathmoveto{\pgfqpoint{7.035760in}{1.010655in}}%
\pgfpathcurveto{\pgfqpoint{7.046810in}{1.010655in}}{\pgfqpoint{7.057409in}{1.015045in}}{\pgfqpoint{7.065223in}{1.022858in}}%
\pgfpathcurveto{\pgfqpoint{7.073037in}{1.030672in}}{\pgfqpoint{7.077427in}{1.041271in}}{\pgfqpoint{7.077427in}{1.052321in}}%
\pgfpathcurveto{\pgfqpoint{7.077427in}{1.063371in}}{\pgfqpoint{7.073037in}{1.073970in}}{\pgfqpoint{7.065223in}{1.081784in}}%
\pgfpathcurveto{\pgfqpoint{7.057409in}{1.089598in}}{\pgfqpoint{7.046810in}{1.093988in}}{\pgfqpoint{7.035760in}{1.093988in}}%
\pgfpathcurveto{\pgfqpoint{7.024710in}{1.093988in}}{\pgfqpoint{7.014111in}{1.089598in}}{\pgfqpoint{7.006297in}{1.081784in}}%
\pgfpathcurveto{\pgfqpoint{6.998484in}{1.073970in}}{\pgfqpoint{6.994094in}{1.063371in}}{\pgfqpoint{6.994094in}{1.052321in}}%
\pgfpathcurveto{\pgfqpoint{6.994094in}{1.041271in}}{\pgfqpoint{6.998484in}{1.030672in}}{\pgfqpoint{7.006297in}{1.022858in}}%
\pgfpathcurveto{\pgfqpoint{7.014111in}{1.015045in}}{\pgfqpoint{7.024710in}{1.010655in}}{\pgfqpoint{7.035760in}{1.010655in}}%
\pgfpathclose%
\pgfusepath{stroke,fill}%
\end{pgfscope}%
\begin{pgfscope}%
\pgfpathrectangle{\pgfqpoint{0.526127in}{0.331635in}}{\pgfqpoint{9.300000in}{7.700000in}}%
\pgfusepath{clip}%
\pgfsetbuttcap%
\pgfsetroundjoin%
\definecolor{currentfill}{rgb}{0.552941,0.898039,0.631373}%
\pgfsetfillcolor{currentfill}%
\pgfsetlinewidth{0.481800pt}%
\definecolor{currentstroke}{rgb}{1.000000,1.000000,1.000000}%
\pgfsetstrokecolor{currentstroke}%
\pgfsetdash{}{0pt}%
\pgfpathmoveto{\pgfqpoint{6.494262in}{1.134274in}}%
\pgfpathcurveto{\pgfqpoint{6.505313in}{1.134274in}}{\pgfqpoint{6.515912in}{1.138664in}}{\pgfqpoint{6.523725in}{1.146477in}}%
\pgfpathcurveto{\pgfqpoint{6.531539in}{1.154291in}}{\pgfqpoint{6.535929in}{1.164890in}}{\pgfqpoint{6.535929in}{1.175940in}}%
\pgfpathcurveto{\pgfqpoint{6.535929in}{1.186990in}}{\pgfqpoint{6.531539in}{1.197589in}}{\pgfqpoint{6.523725in}{1.205403in}}%
\pgfpathcurveto{\pgfqpoint{6.515912in}{1.213217in}}{\pgfqpoint{6.505313in}{1.217607in}}{\pgfqpoint{6.494262in}{1.217607in}}%
\pgfpathcurveto{\pgfqpoint{6.483212in}{1.217607in}}{\pgfqpoint{6.472613in}{1.213217in}}{\pgfqpoint{6.464800in}{1.205403in}}%
\pgfpathcurveto{\pgfqpoint{6.456986in}{1.197589in}}{\pgfqpoint{6.452596in}{1.186990in}}{\pgfqpoint{6.452596in}{1.175940in}}%
\pgfpathcurveto{\pgfqpoint{6.452596in}{1.164890in}}{\pgfqpoint{6.456986in}{1.154291in}}{\pgfqpoint{6.464800in}{1.146477in}}%
\pgfpathcurveto{\pgfqpoint{6.472613in}{1.138664in}}{\pgfqpoint{6.483212in}{1.134274in}}{\pgfqpoint{6.494262in}{1.134274in}}%
\pgfpathclose%
\pgfusepath{stroke,fill}%
\end{pgfscope}%
\begin{pgfscope}%
\pgfpathrectangle{\pgfqpoint{0.526127in}{0.331635in}}{\pgfqpoint{9.300000in}{7.700000in}}%
\pgfusepath{clip}%
\pgfsetbuttcap%
\pgfsetroundjoin%
\definecolor{currentfill}{rgb}{0.552941,0.898039,0.631373}%
\pgfsetfillcolor{currentfill}%
\pgfsetlinewidth{0.481800pt}%
\definecolor{currentstroke}{rgb}{1.000000,1.000000,1.000000}%
\pgfsetstrokecolor{currentstroke}%
\pgfsetdash{}{0pt}%
\pgfpathmoveto{\pgfqpoint{5.499012in}{2.290551in}}%
\pgfpathcurveto{\pgfqpoint{5.510062in}{2.290551in}}{\pgfqpoint{5.520661in}{2.294941in}}{\pgfqpoint{5.528474in}{2.302755in}}%
\pgfpathcurveto{\pgfqpoint{5.536288in}{2.310568in}}{\pgfqpoint{5.540678in}{2.321167in}}{\pgfqpoint{5.540678in}{2.332218in}}%
\pgfpathcurveto{\pgfqpoint{5.540678in}{2.343268in}}{\pgfqpoint{5.536288in}{2.353867in}}{\pgfqpoint{5.528474in}{2.361680in}}%
\pgfpathcurveto{\pgfqpoint{5.520661in}{2.369494in}}{\pgfqpoint{5.510062in}{2.373884in}}{\pgfqpoint{5.499012in}{2.373884in}}%
\pgfpathcurveto{\pgfqpoint{5.487961in}{2.373884in}}{\pgfqpoint{5.477362in}{2.369494in}}{\pgfqpoint{5.469549in}{2.361680in}}%
\pgfpathcurveto{\pgfqpoint{5.461735in}{2.353867in}}{\pgfqpoint{5.457345in}{2.343268in}}{\pgfqpoint{5.457345in}{2.332218in}}%
\pgfpathcurveto{\pgfqpoint{5.457345in}{2.321167in}}{\pgfqpoint{5.461735in}{2.310568in}}{\pgfqpoint{5.469549in}{2.302755in}}%
\pgfpathcurveto{\pgfqpoint{5.477362in}{2.294941in}}{\pgfqpoint{5.487961in}{2.290551in}}{\pgfqpoint{5.499012in}{2.290551in}}%
\pgfpathclose%
\pgfusepath{stroke,fill}%
\end{pgfscope}%
\begin{pgfscope}%
\pgfpathrectangle{\pgfqpoint{0.526127in}{0.331635in}}{\pgfqpoint{9.300000in}{7.700000in}}%
\pgfusepath{clip}%
\pgfsetbuttcap%
\pgfsetroundjoin%
\definecolor{currentfill}{rgb}{0.552941,0.898039,0.631373}%
\pgfsetfillcolor{currentfill}%
\pgfsetlinewidth{0.481800pt}%
\definecolor{currentstroke}{rgb}{1.000000,1.000000,1.000000}%
\pgfsetstrokecolor{currentstroke}%
\pgfsetdash{}{0pt}%
\pgfpathmoveto{\pgfqpoint{6.082548in}{1.279265in}}%
\pgfpathcurveto{\pgfqpoint{6.093598in}{1.279265in}}{\pgfqpoint{6.104197in}{1.283655in}}{\pgfqpoint{6.112011in}{1.291469in}}%
\pgfpathcurveto{\pgfqpoint{6.119825in}{1.299282in}}{\pgfqpoint{6.124215in}{1.309881in}}{\pgfqpoint{6.124215in}{1.320932in}}%
\pgfpathcurveto{\pgfqpoint{6.124215in}{1.331982in}}{\pgfqpoint{6.119825in}{1.342581in}}{\pgfqpoint{6.112011in}{1.350394in}}%
\pgfpathcurveto{\pgfqpoint{6.104197in}{1.358208in}}{\pgfqpoint{6.093598in}{1.362598in}}{\pgfqpoint{6.082548in}{1.362598in}}%
\pgfpathcurveto{\pgfqpoint{6.071498in}{1.362598in}}{\pgfqpoint{6.060899in}{1.358208in}}{\pgfqpoint{6.053086in}{1.350394in}}%
\pgfpathcurveto{\pgfqpoint{6.045272in}{1.342581in}}{\pgfqpoint{6.040882in}{1.331982in}}{\pgfqpoint{6.040882in}{1.320932in}}%
\pgfpathcurveto{\pgfqpoint{6.040882in}{1.309881in}}{\pgfqpoint{6.045272in}{1.299282in}}{\pgfqpoint{6.053086in}{1.291469in}}%
\pgfpathcurveto{\pgfqpoint{6.060899in}{1.283655in}}{\pgfqpoint{6.071498in}{1.279265in}}{\pgfqpoint{6.082548in}{1.279265in}}%
\pgfpathclose%
\pgfusepath{stroke,fill}%
\end{pgfscope}%
\begin{pgfscope}%
\pgfpathrectangle{\pgfqpoint{0.526127in}{0.331635in}}{\pgfqpoint{9.300000in}{7.700000in}}%
\pgfusepath{clip}%
\pgfsetbuttcap%
\pgfsetroundjoin%
\definecolor{currentfill}{rgb}{0.552941,0.898039,0.631373}%
\pgfsetfillcolor{currentfill}%
\pgfsetlinewidth{0.481800pt}%
\definecolor{currentstroke}{rgb}{1.000000,1.000000,1.000000}%
\pgfsetstrokecolor{currentstroke}%
\pgfsetdash{}{0pt}%
\pgfpathmoveto{\pgfqpoint{1.873484in}{3.865802in}}%
\pgfpathcurveto{\pgfqpoint{1.884534in}{3.865802in}}{\pgfqpoint{1.895133in}{3.870192in}}{\pgfqpoint{1.902947in}{3.878006in}}%
\pgfpathcurveto{\pgfqpoint{1.910760in}{3.885819in}}{\pgfqpoint{1.915151in}{3.896419in}}{\pgfqpoint{1.915151in}{3.907469in}}%
\pgfpathcurveto{\pgfqpoint{1.915151in}{3.918519in}}{\pgfqpoint{1.910760in}{3.929118in}}{\pgfqpoint{1.902947in}{3.936931in}}%
\pgfpathcurveto{\pgfqpoint{1.895133in}{3.944745in}}{\pgfqpoint{1.884534in}{3.949135in}}{\pgfqpoint{1.873484in}{3.949135in}}%
\pgfpathcurveto{\pgfqpoint{1.862434in}{3.949135in}}{\pgfqpoint{1.851835in}{3.944745in}}{\pgfqpoint{1.844021in}{3.936931in}}%
\pgfpathcurveto{\pgfqpoint{1.836207in}{3.929118in}}{\pgfqpoint{1.831817in}{3.918519in}}{\pgfqpoint{1.831817in}{3.907469in}}%
\pgfpathcurveto{\pgfqpoint{1.831817in}{3.896419in}}{\pgfqpoint{1.836207in}{3.885819in}}{\pgfqpoint{1.844021in}{3.878006in}}%
\pgfpathcurveto{\pgfqpoint{1.851835in}{3.870192in}}{\pgfqpoint{1.862434in}{3.865802in}}{\pgfqpoint{1.873484in}{3.865802in}}%
\pgfpathclose%
\pgfusepath{stroke,fill}%
\end{pgfscope}%
\begin{pgfscope}%
\pgfpathrectangle{\pgfqpoint{0.526127in}{0.331635in}}{\pgfqpoint{9.300000in}{7.700000in}}%
\pgfusepath{clip}%
\pgfsetbuttcap%
\pgfsetroundjoin%
\definecolor{currentfill}{rgb}{0.552941,0.898039,0.631373}%
\pgfsetfillcolor{currentfill}%
\pgfsetlinewidth{0.481800pt}%
\definecolor{currentstroke}{rgb}{1.000000,1.000000,1.000000}%
\pgfsetstrokecolor{currentstroke}%
\pgfsetdash{}{0pt}%
\pgfpathmoveto{\pgfqpoint{5.027322in}{2.986482in}}%
\pgfpathcurveto{\pgfqpoint{5.038372in}{2.986482in}}{\pgfqpoint{5.048971in}{2.990872in}}{\pgfqpoint{5.056784in}{2.998686in}}%
\pgfpathcurveto{\pgfqpoint{5.064598in}{3.006499in}}{\pgfqpoint{5.068988in}{3.017098in}}{\pgfqpoint{5.068988in}{3.028149in}}%
\pgfpathcurveto{\pgfqpoint{5.068988in}{3.039199in}}{\pgfqpoint{5.064598in}{3.049798in}}{\pgfqpoint{5.056784in}{3.057611in}}%
\pgfpathcurveto{\pgfqpoint{5.048971in}{3.065425in}}{\pgfqpoint{5.038372in}{3.069815in}}{\pgfqpoint{5.027322in}{3.069815in}}%
\pgfpathcurveto{\pgfqpoint{5.016272in}{3.069815in}}{\pgfqpoint{5.005673in}{3.065425in}}{\pgfqpoint{4.997859in}{3.057611in}}%
\pgfpathcurveto{\pgfqpoint{4.990045in}{3.049798in}}{\pgfqpoint{4.985655in}{3.039199in}}{\pgfqpoint{4.985655in}{3.028149in}}%
\pgfpathcurveto{\pgfqpoint{4.985655in}{3.017098in}}{\pgfqpoint{4.990045in}{3.006499in}}{\pgfqpoint{4.997859in}{2.998686in}}%
\pgfpathcurveto{\pgfqpoint{5.005673in}{2.990872in}}{\pgfqpoint{5.016272in}{2.986482in}}{\pgfqpoint{5.027322in}{2.986482in}}%
\pgfpathclose%
\pgfusepath{stroke,fill}%
\end{pgfscope}%
\begin{pgfscope}%
\pgfpathrectangle{\pgfqpoint{0.526127in}{0.331635in}}{\pgfqpoint{9.300000in}{7.700000in}}%
\pgfusepath{clip}%
\pgfsetbuttcap%
\pgfsetroundjoin%
\definecolor{currentfill}{rgb}{0.552941,0.898039,0.631373}%
\pgfsetfillcolor{currentfill}%
\pgfsetlinewidth{0.481800pt}%
\definecolor{currentstroke}{rgb}{1.000000,1.000000,1.000000}%
\pgfsetstrokecolor{currentstroke}%
\pgfsetdash{}{0pt}%
\pgfpathmoveto{\pgfqpoint{1.754896in}{3.458290in}}%
\pgfpathcurveto{\pgfqpoint{1.765946in}{3.458290in}}{\pgfqpoint{1.776545in}{3.462680in}}{\pgfqpoint{1.784359in}{3.470494in}}%
\pgfpathcurveto{\pgfqpoint{1.792173in}{3.478307in}}{\pgfqpoint{1.796563in}{3.488906in}}{\pgfqpoint{1.796563in}{3.499956in}}%
\pgfpathcurveto{\pgfqpoint{1.796563in}{3.511007in}}{\pgfqpoint{1.792173in}{3.521606in}}{\pgfqpoint{1.784359in}{3.529419in}}%
\pgfpathcurveto{\pgfqpoint{1.776545in}{3.537233in}}{\pgfqpoint{1.765946in}{3.541623in}}{\pgfqpoint{1.754896in}{3.541623in}}%
\pgfpathcurveto{\pgfqpoint{1.743846in}{3.541623in}}{\pgfqpoint{1.733247in}{3.537233in}}{\pgfqpoint{1.725433in}{3.529419in}}%
\pgfpathcurveto{\pgfqpoint{1.717620in}{3.521606in}}{\pgfqpoint{1.713230in}{3.511007in}}{\pgfqpoint{1.713230in}{3.499956in}}%
\pgfpathcurveto{\pgfqpoint{1.713230in}{3.488906in}}{\pgfqpoint{1.717620in}{3.478307in}}{\pgfqpoint{1.725433in}{3.470494in}}%
\pgfpathcurveto{\pgfqpoint{1.733247in}{3.462680in}}{\pgfqpoint{1.743846in}{3.458290in}}{\pgfqpoint{1.754896in}{3.458290in}}%
\pgfpathclose%
\pgfusepath{stroke,fill}%
\end{pgfscope}%
\begin{pgfscope}%
\pgfpathrectangle{\pgfqpoint{0.526127in}{0.331635in}}{\pgfqpoint{9.300000in}{7.700000in}}%
\pgfusepath{clip}%
\pgfsetbuttcap%
\pgfsetroundjoin%
\definecolor{currentfill}{rgb}{0.552941,0.898039,0.631373}%
\pgfsetfillcolor{currentfill}%
\pgfsetlinewidth{0.481800pt}%
\definecolor{currentstroke}{rgb}{1.000000,1.000000,1.000000}%
\pgfsetstrokecolor{currentstroke}%
\pgfsetdash{}{0pt}%
\pgfpathmoveto{\pgfqpoint{3.811390in}{5.303206in}}%
\pgfpathcurveto{\pgfqpoint{3.822441in}{5.303206in}}{\pgfqpoint{3.833040in}{5.307596in}}{\pgfqpoint{3.840853in}{5.315410in}}%
\pgfpathcurveto{\pgfqpoint{3.848667in}{5.323224in}}{\pgfqpoint{3.853057in}{5.333823in}}{\pgfqpoint{3.853057in}{5.344873in}}%
\pgfpathcurveto{\pgfqpoint{3.853057in}{5.355923in}}{\pgfqpoint{3.848667in}{5.366522in}}{\pgfqpoint{3.840853in}{5.374335in}}%
\pgfpathcurveto{\pgfqpoint{3.833040in}{5.382149in}}{\pgfqpoint{3.822441in}{5.386539in}}{\pgfqpoint{3.811390in}{5.386539in}}%
\pgfpathcurveto{\pgfqpoint{3.800340in}{5.386539in}}{\pgfqpoint{3.789741in}{5.382149in}}{\pgfqpoint{3.781928in}{5.374335in}}%
\pgfpathcurveto{\pgfqpoint{3.774114in}{5.366522in}}{\pgfqpoint{3.769724in}{5.355923in}}{\pgfqpoint{3.769724in}{5.344873in}}%
\pgfpathcurveto{\pgfqpoint{3.769724in}{5.333823in}}{\pgfqpoint{3.774114in}{5.323224in}}{\pgfqpoint{3.781928in}{5.315410in}}%
\pgfpathcurveto{\pgfqpoint{3.789741in}{5.307596in}}{\pgfqpoint{3.800340in}{5.303206in}}{\pgfqpoint{3.811390in}{5.303206in}}%
\pgfpathclose%
\pgfusepath{stroke,fill}%
\end{pgfscope}%
\begin{pgfscope}%
\pgfpathrectangle{\pgfqpoint{0.526127in}{0.331635in}}{\pgfqpoint{9.300000in}{7.700000in}}%
\pgfusepath{clip}%
\pgfsetbuttcap%
\pgfsetroundjoin%
\definecolor{currentfill}{rgb}{0.552941,0.898039,0.631373}%
\pgfsetfillcolor{currentfill}%
\pgfsetlinewidth{0.481800pt}%
\definecolor{currentstroke}{rgb}{1.000000,1.000000,1.000000}%
\pgfsetstrokecolor{currentstroke}%
\pgfsetdash{}{0pt}%
\pgfpathmoveto{\pgfqpoint{4.820098in}{3.381921in}}%
\pgfpathcurveto{\pgfqpoint{4.831148in}{3.381921in}}{\pgfqpoint{4.841747in}{3.386311in}}{\pgfqpoint{4.849561in}{3.394125in}}%
\pgfpathcurveto{\pgfqpoint{4.857375in}{3.401939in}}{\pgfqpoint{4.861765in}{3.412538in}}{\pgfqpoint{4.861765in}{3.423588in}}%
\pgfpathcurveto{\pgfqpoint{4.861765in}{3.434638in}}{\pgfqpoint{4.857375in}{3.445237in}}{\pgfqpoint{4.849561in}{3.453051in}}%
\pgfpathcurveto{\pgfqpoint{4.841747in}{3.460864in}}{\pgfqpoint{4.831148in}{3.465255in}}{\pgfqpoint{4.820098in}{3.465255in}}%
\pgfpathcurveto{\pgfqpoint{4.809048in}{3.465255in}}{\pgfqpoint{4.798449in}{3.460864in}}{\pgfqpoint{4.790635in}{3.453051in}}%
\pgfpathcurveto{\pgfqpoint{4.782822in}{3.445237in}}{\pgfqpoint{4.778432in}{3.434638in}}{\pgfqpoint{4.778432in}{3.423588in}}%
\pgfpathcurveto{\pgfqpoint{4.778432in}{3.412538in}}{\pgfqpoint{4.782822in}{3.401939in}}{\pgfqpoint{4.790635in}{3.394125in}}%
\pgfpathcurveto{\pgfqpoint{4.798449in}{3.386311in}}{\pgfqpoint{4.809048in}{3.381921in}}{\pgfqpoint{4.820098in}{3.381921in}}%
\pgfpathclose%
\pgfusepath{stroke,fill}%
\end{pgfscope}%
\begin{pgfscope}%
\pgfpathrectangle{\pgfqpoint{0.526127in}{0.331635in}}{\pgfqpoint{9.300000in}{7.700000in}}%
\pgfusepath{clip}%
\pgfsetbuttcap%
\pgfsetroundjoin%
\definecolor{currentfill}{rgb}{1.000000,0.623529,0.607843}%
\pgfsetfillcolor{currentfill}%
\pgfsetlinewidth{0.481800pt}%
\definecolor{currentstroke}{rgb}{1.000000,1.000000,1.000000}%
\pgfsetstrokecolor{currentstroke}%
\pgfsetdash{}{0pt}%
\pgfpathmoveto{\pgfqpoint{6.036344in}{7.639968in}}%
\pgfpathcurveto{\pgfqpoint{6.047394in}{7.639968in}}{\pgfqpoint{6.057993in}{7.644359in}}{\pgfqpoint{6.065807in}{7.652172in}}%
\pgfpathcurveto{\pgfqpoint{6.073621in}{7.659986in}}{\pgfqpoint{6.078011in}{7.670585in}}{\pgfqpoint{6.078011in}{7.681635in}}%
\pgfpathcurveto{\pgfqpoint{6.078011in}{7.692685in}}{\pgfqpoint{6.073621in}{7.703284in}}{\pgfqpoint{6.065807in}{7.711098in}}%
\pgfpathcurveto{\pgfqpoint{6.057993in}{7.718911in}}{\pgfqpoint{6.047394in}{7.723302in}}{\pgfqpoint{6.036344in}{7.723302in}}%
\pgfpathcurveto{\pgfqpoint{6.025294in}{7.723302in}}{\pgfqpoint{6.014695in}{7.718911in}}{\pgfqpoint{6.006881in}{7.711098in}}%
\pgfpathcurveto{\pgfqpoint{5.999068in}{7.703284in}}{\pgfqpoint{5.994678in}{7.692685in}}{\pgfqpoint{5.994678in}{7.681635in}}%
\pgfpathcurveto{\pgfqpoint{5.994678in}{7.670585in}}{\pgfqpoint{5.999068in}{7.659986in}}{\pgfqpoint{6.006881in}{7.652172in}}%
\pgfpathcurveto{\pgfqpoint{6.014695in}{7.644359in}}{\pgfqpoint{6.025294in}{7.639968in}}{\pgfqpoint{6.036344in}{7.639968in}}%
\pgfpathclose%
\pgfusepath{stroke,fill}%
\end{pgfscope}%
\begin{pgfscope}%
\pgfpathrectangle{\pgfqpoint{0.526127in}{0.331635in}}{\pgfqpoint{9.300000in}{7.700000in}}%
\pgfusepath{clip}%
\pgfsetbuttcap%
\pgfsetroundjoin%
\definecolor{currentfill}{rgb}{1.000000,0.623529,0.607843}%
\pgfsetfillcolor{currentfill}%
\pgfsetlinewidth{0.481800pt}%
\definecolor{currentstroke}{rgb}{1.000000,1.000000,1.000000}%
\pgfsetstrokecolor{currentstroke}%
\pgfsetdash{}{0pt}%
\pgfpathmoveto{\pgfqpoint{5.059438in}{4.695820in}}%
\pgfpathcurveto{\pgfqpoint{5.070488in}{4.695820in}}{\pgfqpoint{5.081087in}{4.700210in}}{\pgfqpoint{5.088900in}{4.708024in}}%
\pgfpathcurveto{\pgfqpoint{5.096714in}{4.715837in}}{\pgfqpoint{5.101104in}{4.726436in}}{\pgfqpoint{5.101104in}{4.737486in}}%
\pgfpathcurveto{\pgfqpoint{5.101104in}{4.748537in}}{\pgfqpoint{5.096714in}{4.759136in}}{\pgfqpoint{5.088900in}{4.766949in}}%
\pgfpathcurveto{\pgfqpoint{5.081087in}{4.774763in}}{\pgfqpoint{5.070488in}{4.779153in}}{\pgfqpoint{5.059438in}{4.779153in}}%
\pgfpathcurveto{\pgfqpoint{5.048387in}{4.779153in}}{\pgfqpoint{5.037788in}{4.774763in}}{\pgfqpoint{5.029975in}{4.766949in}}%
\pgfpathcurveto{\pgfqpoint{5.022161in}{4.759136in}}{\pgfqpoint{5.017771in}{4.748537in}}{\pgfqpoint{5.017771in}{4.737486in}}%
\pgfpathcurveto{\pgfqpoint{5.017771in}{4.726436in}}{\pgfqpoint{5.022161in}{4.715837in}}{\pgfqpoint{5.029975in}{4.708024in}}%
\pgfpathcurveto{\pgfqpoint{5.037788in}{4.700210in}}{\pgfqpoint{5.048387in}{4.695820in}}{\pgfqpoint{5.059438in}{4.695820in}}%
\pgfpathclose%
\pgfusepath{stroke,fill}%
\end{pgfscope}%
\begin{pgfscope}%
\pgfpathrectangle{\pgfqpoint{0.526127in}{0.331635in}}{\pgfqpoint{9.300000in}{7.700000in}}%
\pgfusepath{clip}%
\pgfsetbuttcap%
\pgfsetroundjoin%
\definecolor{currentfill}{rgb}{1.000000,0.623529,0.607843}%
\pgfsetfillcolor{currentfill}%
\pgfsetlinewidth{0.481800pt}%
\definecolor{currentstroke}{rgb}{1.000000,1.000000,1.000000}%
\pgfsetstrokecolor{currentstroke}%
\pgfsetdash{}{0pt}%
\pgfpathmoveto{\pgfqpoint{3.537682in}{4.723912in}}%
\pgfpathcurveto{\pgfqpoint{3.548733in}{4.723912in}}{\pgfqpoint{3.559332in}{4.728303in}}{\pgfqpoint{3.567145in}{4.736116in}}%
\pgfpathcurveto{\pgfqpoint{3.574959in}{4.743930in}}{\pgfqpoint{3.579349in}{4.754529in}}{\pgfqpoint{3.579349in}{4.765579in}}%
\pgfpathcurveto{\pgfqpoint{3.579349in}{4.776629in}}{\pgfqpoint{3.574959in}{4.787228in}}{\pgfqpoint{3.567145in}{4.795042in}}%
\pgfpathcurveto{\pgfqpoint{3.559332in}{4.802856in}}{\pgfqpoint{3.548733in}{4.807246in}}{\pgfqpoint{3.537682in}{4.807246in}}%
\pgfpathcurveto{\pgfqpoint{3.526632in}{4.807246in}}{\pgfqpoint{3.516033in}{4.802856in}}{\pgfqpoint{3.508220in}{4.795042in}}%
\pgfpathcurveto{\pgfqpoint{3.500406in}{4.787228in}}{\pgfqpoint{3.496016in}{4.776629in}}{\pgfqpoint{3.496016in}{4.765579in}}%
\pgfpathcurveto{\pgfqpoint{3.496016in}{4.754529in}}{\pgfqpoint{3.500406in}{4.743930in}}{\pgfqpoint{3.508220in}{4.736116in}}%
\pgfpathcurveto{\pgfqpoint{3.516033in}{4.728303in}}{\pgfqpoint{3.526632in}{4.723912in}}{\pgfqpoint{3.537682in}{4.723912in}}%
\pgfpathclose%
\pgfusepath{stroke,fill}%
\end{pgfscope}%
\begin{pgfscope}%
\pgfpathrectangle{\pgfqpoint{0.526127in}{0.331635in}}{\pgfqpoint{9.300000in}{7.700000in}}%
\pgfusepath{clip}%
\pgfsetbuttcap%
\pgfsetroundjoin%
\definecolor{currentfill}{rgb}{1.000000,0.623529,0.607843}%
\pgfsetfillcolor{currentfill}%
\pgfsetlinewidth{0.481800pt}%
\definecolor{currentstroke}{rgb}{1.000000,1.000000,1.000000}%
\pgfsetstrokecolor{currentstroke}%
\pgfsetdash{}{0pt}%
\pgfpathmoveto{\pgfqpoint{3.454692in}{5.724756in}}%
\pgfpathcurveto{\pgfqpoint{3.465742in}{5.724756in}}{\pgfqpoint{3.476341in}{5.729147in}}{\pgfqpoint{3.484155in}{5.736960in}}%
\pgfpathcurveto{\pgfqpoint{3.491968in}{5.744774in}}{\pgfqpoint{3.496359in}{5.755373in}}{\pgfqpoint{3.496359in}{5.766423in}}%
\pgfpathcurveto{\pgfqpoint{3.496359in}{5.777473in}}{\pgfqpoint{3.491968in}{5.788072in}}{\pgfqpoint{3.484155in}{5.795886in}}%
\pgfpathcurveto{\pgfqpoint{3.476341in}{5.803699in}}{\pgfqpoint{3.465742in}{5.808090in}}{\pgfqpoint{3.454692in}{5.808090in}}%
\pgfpathcurveto{\pgfqpoint{3.443642in}{5.808090in}}{\pgfqpoint{3.433043in}{5.803699in}}{\pgfqpoint{3.425229in}{5.795886in}}%
\pgfpathcurveto{\pgfqpoint{3.417416in}{5.788072in}}{\pgfqpoint{3.413025in}{5.777473in}}{\pgfqpoint{3.413025in}{5.766423in}}%
\pgfpathcurveto{\pgfqpoint{3.413025in}{5.755373in}}{\pgfqpoint{3.417416in}{5.744774in}}{\pgfqpoint{3.425229in}{5.736960in}}%
\pgfpathcurveto{\pgfqpoint{3.433043in}{5.729147in}}{\pgfqpoint{3.443642in}{5.724756in}}{\pgfqpoint{3.454692in}{5.724756in}}%
\pgfpathclose%
\pgfusepath{stroke,fill}%
\end{pgfscope}%
\begin{pgfscope}%
\pgfpathrectangle{\pgfqpoint{0.526127in}{0.331635in}}{\pgfqpoint{9.300000in}{7.700000in}}%
\pgfusepath{clip}%
\pgfsetbuttcap%
\pgfsetroundjoin%
\definecolor{currentfill}{rgb}{1.000000,0.623529,0.607843}%
\pgfsetfillcolor{currentfill}%
\pgfsetlinewidth{0.481800pt}%
\definecolor{currentstroke}{rgb}{1.000000,1.000000,1.000000}%
\pgfsetstrokecolor{currentstroke}%
\pgfsetdash{}{0pt}%
\pgfpathmoveto{\pgfqpoint{5.176399in}{7.364906in}}%
\pgfpathcurveto{\pgfqpoint{5.187449in}{7.364906in}}{\pgfqpoint{5.198048in}{7.369297in}}{\pgfqpoint{5.205862in}{7.377110in}}%
\pgfpathcurveto{\pgfqpoint{5.213675in}{7.384924in}}{\pgfqpoint{5.218066in}{7.395523in}}{\pgfqpoint{5.218066in}{7.406573in}}%
\pgfpathcurveto{\pgfqpoint{5.218066in}{7.417623in}}{\pgfqpoint{5.213675in}{7.428222in}}{\pgfqpoint{5.205862in}{7.436036in}}%
\pgfpathcurveto{\pgfqpoint{5.198048in}{7.443849in}}{\pgfqpoint{5.187449in}{7.448240in}}{\pgfqpoint{5.176399in}{7.448240in}}%
\pgfpathcurveto{\pgfqpoint{5.165349in}{7.448240in}}{\pgfqpoint{5.154750in}{7.443849in}}{\pgfqpoint{5.146936in}{7.436036in}}%
\pgfpathcurveto{\pgfqpoint{5.139123in}{7.428222in}}{\pgfqpoint{5.134732in}{7.417623in}}{\pgfqpoint{5.134732in}{7.406573in}}%
\pgfpathcurveto{\pgfqpoint{5.134732in}{7.395523in}}{\pgfqpoint{5.139123in}{7.384924in}}{\pgfqpoint{5.146936in}{7.377110in}}%
\pgfpathcurveto{\pgfqpoint{5.154750in}{7.369297in}}{\pgfqpoint{5.165349in}{7.364906in}}{\pgfqpoint{5.176399in}{7.364906in}}%
\pgfpathclose%
\pgfusepath{stroke,fill}%
\end{pgfscope}%
\begin{pgfscope}%
\pgfpathrectangle{\pgfqpoint{0.526127in}{0.331635in}}{\pgfqpoint{9.300000in}{7.700000in}}%
\pgfusepath{clip}%
\pgfsetbuttcap%
\pgfsetroundjoin%
\definecolor{currentfill}{rgb}{1.000000,0.623529,0.607843}%
\pgfsetfillcolor{currentfill}%
\pgfsetlinewidth{0.481800pt}%
\definecolor{currentstroke}{rgb}{1.000000,1.000000,1.000000}%
\pgfsetstrokecolor{currentstroke}%
\pgfsetdash{}{0pt}%
\pgfpathmoveto{\pgfqpoint{5.896285in}{3.450847in}}%
\pgfpathcurveto{\pgfqpoint{5.907335in}{3.450847in}}{\pgfqpoint{5.917934in}{3.455238in}}{\pgfqpoint{5.925748in}{3.463051in}}%
\pgfpathcurveto{\pgfqpoint{5.933562in}{3.470865in}}{\pgfqpoint{5.937952in}{3.481464in}}{\pgfqpoint{5.937952in}{3.492514in}}%
\pgfpathcurveto{\pgfqpoint{5.937952in}{3.503564in}}{\pgfqpoint{5.933562in}{3.514163in}}{\pgfqpoint{5.925748in}{3.521977in}}%
\pgfpathcurveto{\pgfqpoint{5.917934in}{3.529790in}}{\pgfqpoint{5.907335in}{3.534181in}}{\pgfqpoint{5.896285in}{3.534181in}}%
\pgfpathcurveto{\pgfqpoint{5.885235in}{3.534181in}}{\pgfqpoint{5.874636in}{3.529790in}}{\pgfqpoint{5.866822in}{3.521977in}}%
\pgfpathcurveto{\pgfqpoint{5.859009in}{3.514163in}}{\pgfqpoint{5.854618in}{3.503564in}}{\pgfqpoint{5.854618in}{3.492514in}}%
\pgfpathcurveto{\pgfqpoint{5.854618in}{3.481464in}}{\pgfqpoint{5.859009in}{3.470865in}}{\pgfqpoint{5.866822in}{3.463051in}}%
\pgfpathcurveto{\pgfqpoint{5.874636in}{3.455238in}}{\pgfqpoint{5.885235in}{3.450847in}}{\pgfqpoint{5.896285in}{3.450847in}}%
\pgfpathclose%
\pgfusepath{stroke,fill}%
\end{pgfscope}%
\begin{pgfscope}%
\pgfpathrectangle{\pgfqpoint{0.526127in}{0.331635in}}{\pgfqpoint{9.300000in}{7.700000in}}%
\pgfusepath{clip}%
\pgfsetbuttcap%
\pgfsetroundjoin%
\definecolor{currentfill}{rgb}{1.000000,0.623529,0.607843}%
\pgfsetfillcolor{currentfill}%
\pgfsetlinewidth{0.481800pt}%
\definecolor{currentstroke}{rgb}{1.000000,1.000000,1.000000}%
\pgfsetstrokecolor{currentstroke}%
\pgfsetdash{}{0pt}%
\pgfpathmoveto{\pgfqpoint{5.765801in}{5.086687in}}%
\pgfpathcurveto{\pgfqpoint{5.776851in}{5.086687in}}{\pgfqpoint{5.787450in}{5.091077in}}{\pgfqpoint{5.795264in}{5.098891in}}%
\pgfpathcurveto{\pgfqpoint{5.803077in}{5.106705in}}{\pgfqpoint{5.807467in}{5.117304in}}{\pgfqpoint{5.807467in}{5.128354in}}%
\pgfpathcurveto{\pgfqpoint{5.807467in}{5.139404in}}{\pgfqpoint{5.803077in}{5.150003in}}{\pgfqpoint{5.795264in}{5.157817in}}%
\pgfpathcurveto{\pgfqpoint{5.787450in}{5.165630in}}{\pgfqpoint{5.776851in}{5.170020in}}{\pgfqpoint{5.765801in}{5.170020in}}%
\pgfpathcurveto{\pgfqpoint{5.754751in}{5.170020in}}{\pgfqpoint{5.744152in}{5.165630in}}{\pgfqpoint{5.736338in}{5.157817in}}%
\pgfpathcurveto{\pgfqpoint{5.728524in}{5.150003in}}{\pgfqpoint{5.724134in}{5.139404in}}{\pgfqpoint{5.724134in}{5.128354in}}%
\pgfpathcurveto{\pgfqpoint{5.724134in}{5.117304in}}{\pgfqpoint{5.728524in}{5.106705in}}{\pgfqpoint{5.736338in}{5.098891in}}%
\pgfpathcurveto{\pgfqpoint{5.744152in}{5.091077in}}{\pgfqpoint{5.754751in}{5.086687in}}{\pgfqpoint{5.765801in}{5.086687in}}%
\pgfpathclose%
\pgfusepath{stroke,fill}%
\end{pgfscope}%
\begin{pgfscope}%
\pgfpathrectangle{\pgfqpoint{0.526127in}{0.331635in}}{\pgfqpoint{9.300000in}{7.700000in}}%
\pgfusepath{clip}%
\pgfsetbuttcap%
\pgfsetroundjoin%
\definecolor{currentfill}{rgb}{1.000000,0.623529,0.607843}%
\pgfsetfillcolor{currentfill}%
\pgfsetlinewidth{0.481800pt}%
\definecolor{currentstroke}{rgb}{1.000000,1.000000,1.000000}%
\pgfsetstrokecolor{currentstroke}%
\pgfsetdash{}{0pt}%
\pgfpathmoveto{\pgfqpoint{3.919124in}{4.234069in}}%
\pgfpathcurveto{\pgfqpoint{3.930174in}{4.234069in}}{\pgfqpoint{3.940773in}{4.238460in}}{\pgfqpoint{3.948587in}{4.246273in}}%
\pgfpathcurveto{\pgfqpoint{3.956400in}{4.254087in}}{\pgfqpoint{3.960791in}{4.264686in}}{\pgfqpoint{3.960791in}{4.275736in}}%
\pgfpathcurveto{\pgfqpoint{3.960791in}{4.286786in}}{\pgfqpoint{3.956400in}{4.297385in}}{\pgfqpoint{3.948587in}{4.305199in}}%
\pgfpathcurveto{\pgfqpoint{3.940773in}{4.313013in}}{\pgfqpoint{3.930174in}{4.317403in}}{\pgfqpoint{3.919124in}{4.317403in}}%
\pgfpathcurveto{\pgfqpoint{3.908074in}{4.317403in}}{\pgfqpoint{3.897475in}{4.313013in}}{\pgfqpoint{3.889661in}{4.305199in}}%
\pgfpathcurveto{\pgfqpoint{3.881848in}{4.297385in}}{\pgfqpoint{3.877457in}{4.286786in}}{\pgfqpoint{3.877457in}{4.275736in}}%
\pgfpathcurveto{\pgfqpoint{3.877457in}{4.264686in}}{\pgfqpoint{3.881848in}{4.254087in}}{\pgfqpoint{3.889661in}{4.246273in}}%
\pgfpathcurveto{\pgfqpoint{3.897475in}{4.238460in}}{\pgfqpoint{3.908074in}{4.234069in}}{\pgfqpoint{3.919124in}{4.234069in}}%
\pgfpathclose%
\pgfusepath{stroke,fill}%
\end{pgfscope}%
\begin{pgfscope}%
\pgfpathrectangle{\pgfqpoint{0.526127in}{0.331635in}}{\pgfqpoint{9.300000in}{7.700000in}}%
\pgfusepath{clip}%
\pgfsetbuttcap%
\pgfsetroundjoin%
\definecolor{currentfill}{rgb}{1.000000,0.623529,0.607843}%
\pgfsetfillcolor{currentfill}%
\pgfsetlinewidth{0.481800pt}%
\definecolor{currentstroke}{rgb}{1.000000,1.000000,1.000000}%
\pgfsetstrokecolor{currentstroke}%
\pgfsetdash{}{0pt}%
\pgfpathmoveto{\pgfqpoint{7.219680in}{3.360479in}}%
\pgfpathcurveto{\pgfqpoint{7.230730in}{3.360479in}}{\pgfqpoint{7.241329in}{3.364869in}}{\pgfqpoint{7.249143in}{3.372683in}}%
\pgfpathcurveto{\pgfqpoint{7.256956in}{3.380497in}}{\pgfqpoint{7.261346in}{3.391096in}}{\pgfqpoint{7.261346in}{3.402146in}}%
\pgfpathcurveto{\pgfqpoint{7.261346in}{3.413196in}}{\pgfqpoint{7.256956in}{3.423795in}}{\pgfqpoint{7.249143in}{3.431609in}}%
\pgfpathcurveto{\pgfqpoint{7.241329in}{3.439422in}}{\pgfqpoint{7.230730in}{3.443812in}}{\pgfqpoint{7.219680in}{3.443812in}}%
\pgfpathcurveto{\pgfqpoint{7.208630in}{3.443812in}}{\pgfqpoint{7.198031in}{3.439422in}}{\pgfqpoint{7.190217in}{3.431609in}}%
\pgfpathcurveto{\pgfqpoint{7.182403in}{3.423795in}}{\pgfqpoint{7.178013in}{3.413196in}}{\pgfqpoint{7.178013in}{3.402146in}}%
\pgfpathcurveto{\pgfqpoint{7.178013in}{3.391096in}}{\pgfqpoint{7.182403in}{3.380497in}}{\pgfqpoint{7.190217in}{3.372683in}}%
\pgfpathcurveto{\pgfqpoint{7.198031in}{3.364869in}}{\pgfqpoint{7.208630in}{3.360479in}}{\pgfqpoint{7.219680in}{3.360479in}}%
\pgfpathclose%
\pgfusepath{stroke,fill}%
\end{pgfscope}%
\begin{pgfscope}%
\pgfpathrectangle{\pgfqpoint{0.526127in}{0.331635in}}{\pgfqpoint{9.300000in}{7.700000in}}%
\pgfusepath{clip}%
\pgfsetbuttcap%
\pgfsetroundjoin%
\definecolor{currentfill}{rgb}{1.000000,0.623529,0.607843}%
\pgfsetfillcolor{currentfill}%
\pgfsetlinewidth{0.481800pt}%
\definecolor{currentstroke}{rgb}{1.000000,1.000000,1.000000}%
\pgfsetstrokecolor{currentstroke}%
\pgfsetdash{}{0pt}%
\pgfpathmoveto{\pgfqpoint{4.459787in}{4.742341in}}%
\pgfpathcurveto{\pgfqpoint{4.470837in}{4.742341in}}{\pgfqpoint{4.481437in}{4.746732in}}{\pgfqpoint{4.489250in}{4.754545in}}%
\pgfpathcurveto{\pgfqpoint{4.497064in}{4.762359in}}{\pgfqpoint{4.501454in}{4.772958in}}{\pgfqpoint{4.501454in}{4.784008in}}%
\pgfpathcurveto{\pgfqpoint{4.501454in}{4.795058in}}{\pgfqpoint{4.497064in}{4.805657in}}{\pgfqpoint{4.489250in}{4.813471in}}%
\pgfpathcurveto{\pgfqpoint{4.481437in}{4.821284in}}{\pgfqpoint{4.470837in}{4.825675in}}{\pgfqpoint{4.459787in}{4.825675in}}%
\pgfpathcurveto{\pgfqpoint{4.448737in}{4.825675in}}{\pgfqpoint{4.438138in}{4.821284in}}{\pgfqpoint{4.430325in}{4.813471in}}%
\pgfpathcurveto{\pgfqpoint{4.422511in}{4.805657in}}{\pgfqpoint{4.418121in}{4.795058in}}{\pgfqpoint{4.418121in}{4.784008in}}%
\pgfpathcurveto{\pgfqpoint{4.418121in}{4.772958in}}{\pgfqpoint{4.422511in}{4.762359in}}{\pgfqpoint{4.430325in}{4.754545in}}%
\pgfpathcurveto{\pgfqpoint{4.438138in}{4.746732in}}{\pgfqpoint{4.448737in}{4.742341in}}{\pgfqpoint{4.459787in}{4.742341in}}%
\pgfpathclose%
\pgfusepath{stroke,fill}%
\end{pgfscope}%
\begin{pgfscope}%
\pgfpathrectangle{\pgfqpoint{0.526127in}{0.331635in}}{\pgfqpoint{9.300000in}{7.700000in}}%
\pgfusepath{clip}%
\pgfsetbuttcap%
\pgfsetroundjoin%
\definecolor{currentfill}{rgb}{1.000000,0.623529,0.607843}%
\pgfsetfillcolor{currentfill}%
\pgfsetlinewidth{0.481800pt}%
\definecolor{currentstroke}{rgb}{1.000000,1.000000,1.000000}%
\pgfsetstrokecolor{currentstroke}%
\pgfsetdash{}{0pt}%
\pgfpathmoveto{\pgfqpoint{7.438369in}{3.950682in}}%
\pgfpathcurveto{\pgfqpoint{7.449419in}{3.950682in}}{\pgfqpoint{7.460018in}{3.955072in}}{\pgfqpoint{7.467832in}{3.962886in}}%
\pgfpathcurveto{\pgfqpoint{7.475645in}{3.970700in}}{\pgfqpoint{7.480035in}{3.981299in}}{\pgfqpoint{7.480035in}{3.992349in}}%
\pgfpathcurveto{\pgfqpoint{7.480035in}{4.003399in}}{\pgfqpoint{7.475645in}{4.013998in}}{\pgfqpoint{7.467832in}{4.021812in}}%
\pgfpathcurveto{\pgfqpoint{7.460018in}{4.029625in}}{\pgfqpoint{7.449419in}{4.034016in}}{\pgfqpoint{7.438369in}{4.034016in}}%
\pgfpathcurveto{\pgfqpoint{7.427319in}{4.034016in}}{\pgfqpoint{7.416720in}{4.029625in}}{\pgfqpoint{7.408906in}{4.021812in}}%
\pgfpathcurveto{\pgfqpoint{7.401092in}{4.013998in}}{\pgfqpoint{7.396702in}{4.003399in}}{\pgfqpoint{7.396702in}{3.992349in}}%
\pgfpathcurveto{\pgfqpoint{7.396702in}{3.981299in}}{\pgfqpoint{7.401092in}{3.970700in}}{\pgfqpoint{7.408906in}{3.962886in}}%
\pgfpathcurveto{\pgfqpoint{7.416720in}{3.955072in}}{\pgfqpoint{7.427319in}{3.950682in}}{\pgfqpoint{7.438369in}{3.950682in}}%
\pgfpathclose%
\pgfusepath{stroke,fill}%
\end{pgfscope}%
\begin{pgfscope}%
\pgfpathrectangle{\pgfqpoint{0.526127in}{0.331635in}}{\pgfqpoint{9.300000in}{7.700000in}}%
\pgfusepath{clip}%
\pgfsetbuttcap%
\pgfsetroundjoin%
\definecolor{currentfill}{rgb}{1.000000,0.623529,0.607843}%
\pgfsetfillcolor{currentfill}%
\pgfsetlinewidth{0.481800pt}%
\definecolor{currentstroke}{rgb}{1.000000,1.000000,1.000000}%
\pgfsetstrokecolor{currentstroke}%
\pgfsetdash{}{0pt}%
\pgfpathmoveto{\pgfqpoint{4.906391in}{4.922709in}}%
\pgfpathcurveto{\pgfqpoint{4.917441in}{4.922709in}}{\pgfqpoint{4.928040in}{4.927099in}}{\pgfqpoint{4.935854in}{4.934913in}}%
\pgfpathcurveto{\pgfqpoint{4.943667in}{4.942726in}}{\pgfqpoint{4.948058in}{4.953326in}}{\pgfqpoint{4.948058in}{4.964376in}}%
\pgfpathcurveto{\pgfqpoint{4.948058in}{4.975426in}}{\pgfqpoint{4.943667in}{4.986025in}}{\pgfqpoint{4.935854in}{4.993838in}}%
\pgfpathcurveto{\pgfqpoint{4.928040in}{5.001652in}}{\pgfqpoint{4.917441in}{5.006042in}}{\pgfqpoint{4.906391in}{5.006042in}}%
\pgfpathcurveto{\pgfqpoint{4.895341in}{5.006042in}}{\pgfqpoint{4.884742in}{5.001652in}}{\pgfqpoint{4.876928in}{4.993838in}}%
\pgfpathcurveto{\pgfqpoint{4.869115in}{4.986025in}}{\pgfqpoint{4.864724in}{4.975426in}}{\pgfqpoint{4.864724in}{4.964376in}}%
\pgfpathcurveto{\pgfqpoint{4.864724in}{4.953326in}}{\pgfqpoint{4.869115in}{4.942726in}}{\pgfqpoint{4.876928in}{4.934913in}}%
\pgfpathcurveto{\pgfqpoint{4.884742in}{4.927099in}}{\pgfqpoint{4.895341in}{4.922709in}}{\pgfqpoint{4.906391in}{4.922709in}}%
\pgfpathclose%
\pgfusepath{stroke,fill}%
\end{pgfscope}%
\begin{pgfscope}%
\pgfpathrectangle{\pgfqpoint{0.526127in}{0.331635in}}{\pgfqpoint{9.300000in}{7.700000in}}%
\pgfusepath{clip}%
\pgfsetbuttcap%
\pgfsetroundjoin%
\definecolor{currentfill}{rgb}{1.000000,0.623529,0.607843}%
\pgfsetfillcolor{currentfill}%
\pgfsetlinewidth{0.481800pt}%
\definecolor{currentstroke}{rgb}{1.000000,1.000000,1.000000}%
\pgfsetstrokecolor{currentstroke}%
\pgfsetdash{}{0pt}%
\pgfpathmoveto{\pgfqpoint{5.942000in}{5.870150in}}%
\pgfpathcurveto{\pgfqpoint{5.953050in}{5.870150in}}{\pgfqpoint{5.963649in}{5.874540in}}{\pgfqpoint{5.971463in}{5.882354in}}%
\pgfpathcurveto{\pgfqpoint{5.979276in}{5.890167in}}{\pgfqpoint{5.983666in}{5.900766in}}{\pgfqpoint{5.983666in}{5.911816in}}%
\pgfpathcurveto{\pgfqpoint{5.983666in}{5.922867in}}{\pgfqpoint{5.979276in}{5.933466in}}{\pgfqpoint{5.971463in}{5.941279in}}%
\pgfpathcurveto{\pgfqpoint{5.963649in}{5.949093in}}{\pgfqpoint{5.953050in}{5.953483in}}{\pgfqpoint{5.942000in}{5.953483in}}%
\pgfpathcurveto{\pgfqpoint{5.930950in}{5.953483in}}{\pgfqpoint{5.920351in}{5.949093in}}{\pgfqpoint{5.912537in}{5.941279in}}%
\pgfpathcurveto{\pgfqpoint{5.904723in}{5.933466in}}{\pgfqpoint{5.900333in}{5.922867in}}{\pgfqpoint{5.900333in}{5.911816in}}%
\pgfpathcurveto{\pgfqpoint{5.900333in}{5.900766in}}{\pgfqpoint{5.904723in}{5.890167in}}{\pgfqpoint{5.912537in}{5.882354in}}%
\pgfpathcurveto{\pgfqpoint{5.920351in}{5.874540in}}{\pgfqpoint{5.930950in}{5.870150in}}{\pgfqpoint{5.942000in}{5.870150in}}%
\pgfpathclose%
\pgfusepath{stroke,fill}%
\end{pgfscope}%
\begin{pgfscope}%
\pgfpathrectangle{\pgfqpoint{0.526127in}{0.331635in}}{\pgfqpoint{9.300000in}{7.700000in}}%
\pgfusepath{clip}%
\pgfsetbuttcap%
\pgfsetroundjoin%
\definecolor{currentfill}{rgb}{1.000000,0.623529,0.607843}%
\pgfsetfillcolor{currentfill}%
\pgfsetlinewidth{0.481800pt}%
\definecolor{currentstroke}{rgb}{1.000000,1.000000,1.000000}%
\pgfsetstrokecolor{currentstroke}%
\pgfsetdash{}{0pt}%
\pgfpathmoveto{\pgfqpoint{5.122725in}{5.229954in}}%
\pgfpathcurveto{\pgfqpoint{5.133775in}{5.229954in}}{\pgfqpoint{5.144374in}{5.234344in}}{\pgfqpoint{5.152188in}{5.242157in}}%
\pgfpathcurveto{\pgfqpoint{5.160002in}{5.249971in}}{\pgfqpoint{5.164392in}{5.260570in}}{\pgfqpoint{5.164392in}{5.271620in}}%
\pgfpathcurveto{\pgfqpoint{5.164392in}{5.282670in}}{\pgfqpoint{5.160002in}{5.293269in}}{\pgfqpoint{5.152188in}{5.301083in}}%
\pgfpathcurveto{\pgfqpoint{5.144374in}{5.308897in}}{\pgfqpoint{5.133775in}{5.313287in}}{\pgfqpoint{5.122725in}{5.313287in}}%
\pgfpathcurveto{\pgfqpoint{5.111675in}{5.313287in}}{\pgfqpoint{5.101076in}{5.308897in}}{\pgfqpoint{5.093263in}{5.301083in}}%
\pgfpathcurveto{\pgfqpoint{5.085449in}{5.293269in}}{\pgfqpoint{5.081059in}{5.282670in}}{\pgfqpoint{5.081059in}{5.271620in}}%
\pgfpathcurveto{\pgfqpoint{5.081059in}{5.260570in}}{\pgfqpoint{5.085449in}{5.249971in}}{\pgfqpoint{5.093263in}{5.242157in}}%
\pgfpathcurveto{\pgfqpoint{5.101076in}{5.234344in}}{\pgfqpoint{5.111675in}{5.229954in}}{\pgfqpoint{5.122725in}{5.229954in}}%
\pgfpathclose%
\pgfusepath{stroke,fill}%
\end{pgfscope}%
\begin{pgfscope}%
\pgfpathrectangle{\pgfqpoint{0.526127in}{0.331635in}}{\pgfqpoint{9.300000in}{7.700000in}}%
\pgfusepath{clip}%
\pgfsetbuttcap%
\pgfsetroundjoin%
\definecolor{currentfill}{rgb}{1.000000,0.623529,0.607843}%
\pgfsetfillcolor{currentfill}%
\pgfsetlinewidth{0.481800pt}%
\definecolor{currentstroke}{rgb}{1.000000,1.000000,1.000000}%
\pgfsetstrokecolor{currentstroke}%
\pgfsetdash{}{0pt}%
\pgfpathmoveto{\pgfqpoint{6.430193in}{5.068925in}}%
\pgfpathcurveto{\pgfqpoint{6.441243in}{5.068925in}}{\pgfqpoint{6.451842in}{5.073315in}}{\pgfqpoint{6.459656in}{5.081129in}}%
\pgfpathcurveto{\pgfqpoint{6.467469in}{5.088942in}}{\pgfqpoint{6.471860in}{5.099541in}}{\pgfqpoint{6.471860in}{5.110591in}}%
\pgfpathcurveto{\pgfqpoint{6.471860in}{5.121641in}}{\pgfqpoint{6.467469in}{5.132240in}}{\pgfqpoint{6.459656in}{5.140054in}}%
\pgfpathcurveto{\pgfqpoint{6.451842in}{5.147868in}}{\pgfqpoint{6.441243in}{5.152258in}}{\pgfqpoint{6.430193in}{5.152258in}}%
\pgfpathcurveto{\pgfqpoint{6.419143in}{5.152258in}}{\pgfqpoint{6.408544in}{5.147868in}}{\pgfqpoint{6.400730in}{5.140054in}}%
\pgfpathcurveto{\pgfqpoint{6.392917in}{5.132240in}}{\pgfqpoint{6.388526in}{5.121641in}}{\pgfqpoint{6.388526in}{5.110591in}}%
\pgfpathcurveto{\pgfqpoint{6.388526in}{5.099541in}}{\pgfqpoint{6.392917in}{5.088942in}}{\pgfqpoint{6.400730in}{5.081129in}}%
\pgfpathcurveto{\pgfqpoint{6.408544in}{5.073315in}}{\pgfqpoint{6.419143in}{5.068925in}}{\pgfqpoint{6.430193in}{5.068925in}}%
\pgfpathclose%
\pgfusepath{stroke,fill}%
\end{pgfscope}%
\begin{pgfscope}%
\pgfpathrectangle{\pgfqpoint{0.526127in}{0.331635in}}{\pgfqpoint{9.300000in}{7.700000in}}%
\pgfusepath{clip}%
\pgfsetbuttcap%
\pgfsetroundjoin%
\definecolor{currentfill}{rgb}{1.000000,0.623529,0.607843}%
\pgfsetfillcolor{currentfill}%
\pgfsetlinewidth{0.481800pt}%
\definecolor{currentstroke}{rgb}{1.000000,1.000000,1.000000}%
\pgfsetstrokecolor{currentstroke}%
\pgfsetdash{}{0pt}%
\pgfpathmoveto{\pgfqpoint{5.393295in}{4.009107in}}%
\pgfpathcurveto{\pgfqpoint{5.404345in}{4.009107in}}{\pgfqpoint{5.414944in}{4.013497in}}{\pgfqpoint{5.422758in}{4.021311in}}%
\pgfpathcurveto{\pgfqpoint{5.430571in}{4.029125in}}{\pgfqpoint{5.434962in}{4.039724in}}{\pgfqpoint{5.434962in}{4.050774in}}%
\pgfpathcurveto{\pgfqpoint{5.434962in}{4.061824in}}{\pgfqpoint{5.430571in}{4.072423in}}{\pgfqpoint{5.422758in}{4.080237in}}%
\pgfpathcurveto{\pgfqpoint{5.414944in}{4.088050in}}{\pgfqpoint{5.404345in}{4.092441in}}{\pgfqpoint{5.393295in}{4.092441in}}%
\pgfpathcurveto{\pgfqpoint{5.382245in}{4.092441in}}{\pgfqpoint{5.371646in}{4.088050in}}{\pgfqpoint{5.363832in}{4.080237in}}%
\pgfpathcurveto{\pgfqpoint{5.356019in}{4.072423in}}{\pgfqpoint{5.351628in}{4.061824in}}{\pgfqpoint{5.351628in}{4.050774in}}%
\pgfpathcurveto{\pgfqpoint{5.351628in}{4.039724in}}{\pgfqpoint{5.356019in}{4.029125in}}{\pgfqpoint{5.363832in}{4.021311in}}%
\pgfpathcurveto{\pgfqpoint{5.371646in}{4.013497in}}{\pgfqpoint{5.382245in}{4.009107in}}{\pgfqpoint{5.393295in}{4.009107in}}%
\pgfpathclose%
\pgfusepath{stroke,fill}%
\end{pgfscope}%
\begin{pgfscope}%
\pgfpathrectangle{\pgfqpoint{0.526127in}{0.331635in}}{\pgfqpoint{9.300000in}{7.700000in}}%
\pgfusepath{clip}%
\pgfsetbuttcap%
\pgfsetroundjoin%
\definecolor{currentfill}{rgb}{1.000000,0.623529,0.607843}%
\pgfsetfillcolor{currentfill}%
\pgfsetlinewidth{0.481800pt}%
\definecolor{currentstroke}{rgb}{1.000000,1.000000,1.000000}%
\pgfsetstrokecolor{currentstroke}%
\pgfsetdash{}{0pt}%
\pgfpathmoveto{\pgfqpoint{4.106444in}{4.492949in}}%
\pgfpathcurveto{\pgfqpoint{4.117494in}{4.492949in}}{\pgfqpoint{4.128093in}{4.497339in}}{\pgfqpoint{4.135907in}{4.505153in}}%
\pgfpathcurveto{\pgfqpoint{4.143720in}{4.512967in}}{\pgfqpoint{4.148111in}{4.523566in}}{\pgfqpoint{4.148111in}{4.534616in}}%
\pgfpathcurveto{\pgfqpoint{4.148111in}{4.545666in}}{\pgfqpoint{4.143720in}{4.556265in}}{\pgfqpoint{4.135907in}{4.564079in}}%
\pgfpathcurveto{\pgfqpoint{4.128093in}{4.571892in}}{\pgfqpoint{4.117494in}{4.576283in}}{\pgfqpoint{4.106444in}{4.576283in}}%
\pgfpathcurveto{\pgfqpoint{4.095394in}{4.576283in}}{\pgfqpoint{4.084795in}{4.571892in}}{\pgfqpoint{4.076981in}{4.564079in}}%
\pgfpathcurveto{\pgfqpoint{4.069168in}{4.556265in}}{\pgfqpoint{4.064777in}{4.545666in}}{\pgfqpoint{4.064777in}{4.534616in}}%
\pgfpathcurveto{\pgfqpoint{4.064777in}{4.523566in}}{\pgfqpoint{4.069168in}{4.512967in}}{\pgfqpoint{4.076981in}{4.505153in}}%
\pgfpathcurveto{\pgfqpoint{4.084795in}{4.497339in}}{\pgfqpoint{4.095394in}{4.492949in}}{\pgfqpoint{4.106444in}{4.492949in}}%
\pgfpathclose%
\pgfusepath{stroke,fill}%
\end{pgfscope}%
\begin{pgfscope}%
\pgfpathrectangle{\pgfqpoint{0.526127in}{0.331635in}}{\pgfqpoint{9.300000in}{7.700000in}}%
\pgfusepath{clip}%
\pgfsetbuttcap%
\pgfsetroundjoin%
\definecolor{currentfill}{rgb}{1.000000,0.623529,0.607843}%
\pgfsetfillcolor{currentfill}%
\pgfsetlinewidth{0.481800pt}%
\definecolor{currentstroke}{rgb}{1.000000,1.000000,1.000000}%
\pgfsetstrokecolor{currentstroke}%
\pgfsetdash{}{0pt}%
\pgfpathmoveto{\pgfqpoint{3.340315in}{5.074842in}}%
\pgfpathcurveto{\pgfqpoint{3.351365in}{5.074842in}}{\pgfqpoint{3.361964in}{5.079233in}}{\pgfqpoint{3.369777in}{5.087046in}}%
\pgfpathcurveto{\pgfqpoint{3.377591in}{5.094860in}}{\pgfqpoint{3.381981in}{5.105459in}}{\pgfqpoint{3.381981in}{5.116509in}}%
\pgfpathcurveto{\pgfqpoint{3.381981in}{5.127559in}}{\pgfqpoint{3.377591in}{5.138158in}}{\pgfqpoint{3.369777in}{5.145972in}}%
\pgfpathcurveto{\pgfqpoint{3.361964in}{5.153785in}}{\pgfqpoint{3.351365in}{5.158176in}}{\pgfqpoint{3.340315in}{5.158176in}}%
\pgfpathcurveto{\pgfqpoint{3.329264in}{5.158176in}}{\pgfqpoint{3.318665in}{5.153785in}}{\pgfqpoint{3.310852in}{5.145972in}}%
\pgfpathcurveto{\pgfqpoint{3.303038in}{5.138158in}}{\pgfqpoint{3.298648in}{5.127559in}}{\pgfqpoint{3.298648in}{5.116509in}}%
\pgfpathcurveto{\pgfqpoint{3.298648in}{5.105459in}}{\pgfqpoint{3.303038in}{5.094860in}}{\pgfqpoint{3.310852in}{5.087046in}}%
\pgfpathcurveto{\pgfqpoint{3.318665in}{5.079233in}}{\pgfqpoint{3.329264in}{5.074842in}}{\pgfqpoint{3.340315in}{5.074842in}}%
\pgfpathclose%
\pgfusepath{stroke,fill}%
\end{pgfscope}%
\begin{pgfscope}%
\pgfpathrectangle{\pgfqpoint{0.526127in}{0.331635in}}{\pgfqpoint{9.300000in}{7.700000in}}%
\pgfusepath{clip}%
\pgfsetbuttcap%
\pgfsetroundjoin%
\definecolor{currentfill}{rgb}{1.000000,0.623529,0.607843}%
\pgfsetfillcolor{currentfill}%
\pgfsetlinewidth{0.481800pt}%
\definecolor{currentstroke}{rgb}{1.000000,1.000000,1.000000}%
\pgfsetstrokecolor{currentstroke}%
\pgfsetdash{}{0pt}%
\pgfpathmoveto{\pgfqpoint{5.209743in}{4.817984in}}%
\pgfpathcurveto{\pgfqpoint{5.220793in}{4.817984in}}{\pgfqpoint{5.231392in}{4.822374in}}{\pgfqpoint{5.239206in}{4.830188in}}%
\pgfpathcurveto{\pgfqpoint{5.247020in}{4.838001in}}{\pgfqpoint{5.251410in}{4.848600in}}{\pgfqpoint{5.251410in}{4.859650in}}%
\pgfpathcurveto{\pgfqpoint{5.251410in}{4.870701in}}{\pgfqpoint{5.247020in}{4.881300in}}{\pgfqpoint{5.239206in}{4.889113in}}%
\pgfpathcurveto{\pgfqpoint{5.231392in}{4.896927in}}{\pgfqpoint{5.220793in}{4.901317in}}{\pgfqpoint{5.209743in}{4.901317in}}%
\pgfpathcurveto{\pgfqpoint{5.198693in}{4.901317in}}{\pgfqpoint{5.188094in}{4.896927in}}{\pgfqpoint{5.180280in}{4.889113in}}%
\pgfpathcurveto{\pgfqpoint{5.172467in}{4.881300in}}{\pgfqpoint{5.168077in}{4.870701in}}{\pgfqpoint{5.168077in}{4.859650in}}%
\pgfpathcurveto{\pgfqpoint{5.168077in}{4.848600in}}{\pgfqpoint{5.172467in}{4.838001in}}{\pgfqpoint{5.180280in}{4.830188in}}%
\pgfpathcurveto{\pgfqpoint{5.188094in}{4.822374in}}{\pgfqpoint{5.198693in}{4.817984in}}{\pgfqpoint{5.209743in}{4.817984in}}%
\pgfpathclose%
\pgfusepath{stroke,fill}%
\end{pgfscope}%
\begin{pgfscope}%
\pgfpathrectangle{\pgfqpoint{0.526127in}{0.331635in}}{\pgfqpoint{9.300000in}{7.700000in}}%
\pgfusepath{clip}%
\pgfsetbuttcap%
\pgfsetroundjoin%
\definecolor{currentfill}{rgb}{1.000000,0.623529,0.607843}%
\pgfsetfillcolor{currentfill}%
\pgfsetlinewidth{0.481800pt}%
\definecolor{currentstroke}{rgb}{1.000000,1.000000,1.000000}%
\pgfsetstrokecolor{currentstroke}%
\pgfsetdash{}{0pt}%
\pgfpathmoveto{\pgfqpoint{4.735213in}{4.833487in}}%
\pgfpathcurveto{\pgfqpoint{4.746263in}{4.833487in}}{\pgfqpoint{4.756862in}{4.837877in}}{\pgfqpoint{4.764676in}{4.845691in}}%
\pgfpathcurveto{\pgfqpoint{4.772489in}{4.853505in}}{\pgfqpoint{4.776879in}{4.864104in}}{\pgfqpoint{4.776879in}{4.875154in}}%
\pgfpathcurveto{\pgfqpoint{4.776879in}{4.886204in}}{\pgfqpoint{4.772489in}{4.896803in}}{\pgfqpoint{4.764676in}{4.904616in}}%
\pgfpathcurveto{\pgfqpoint{4.756862in}{4.912430in}}{\pgfqpoint{4.746263in}{4.916820in}}{\pgfqpoint{4.735213in}{4.916820in}}%
\pgfpathcurveto{\pgfqpoint{4.724163in}{4.916820in}}{\pgfqpoint{4.713564in}{4.912430in}}{\pgfqpoint{4.705750in}{4.904616in}}%
\pgfpathcurveto{\pgfqpoint{4.697936in}{4.896803in}}{\pgfqpoint{4.693546in}{4.886204in}}{\pgfqpoint{4.693546in}{4.875154in}}%
\pgfpathcurveto{\pgfqpoint{4.693546in}{4.864104in}}{\pgfqpoint{4.697936in}{4.853505in}}{\pgfqpoint{4.705750in}{4.845691in}}%
\pgfpathcurveto{\pgfqpoint{4.713564in}{4.837877in}}{\pgfqpoint{4.724163in}{4.833487in}}{\pgfqpoint{4.735213in}{4.833487in}}%
\pgfpathclose%
\pgfusepath{stroke,fill}%
\end{pgfscope}%
\begin{pgfscope}%
\pgfpathrectangle{\pgfqpoint{0.526127in}{0.331635in}}{\pgfqpoint{9.300000in}{7.700000in}}%
\pgfusepath{clip}%
\pgfsetbuttcap%
\pgfsetroundjoin%
\definecolor{currentfill}{rgb}{1.000000,0.623529,0.607843}%
\pgfsetfillcolor{currentfill}%
\pgfsetlinewidth{0.481800pt}%
\definecolor{currentstroke}{rgb}{1.000000,1.000000,1.000000}%
\pgfsetstrokecolor{currentstroke}%
\pgfsetdash{}{0pt}%
\pgfpathmoveto{\pgfqpoint{3.352726in}{5.641625in}}%
\pgfpathcurveto{\pgfqpoint{3.363777in}{5.641625in}}{\pgfqpoint{3.374376in}{5.646015in}}{\pgfqpoint{3.382189in}{5.653829in}}%
\pgfpathcurveto{\pgfqpoint{3.390003in}{5.661642in}}{\pgfqpoint{3.394393in}{5.672241in}}{\pgfqpoint{3.394393in}{5.683291in}}%
\pgfpathcurveto{\pgfqpoint{3.394393in}{5.694342in}}{\pgfqpoint{3.390003in}{5.704941in}}{\pgfqpoint{3.382189in}{5.712754in}}%
\pgfpathcurveto{\pgfqpoint{3.374376in}{5.720568in}}{\pgfqpoint{3.363777in}{5.724958in}}{\pgfqpoint{3.352726in}{5.724958in}}%
\pgfpathcurveto{\pgfqpoint{3.341676in}{5.724958in}}{\pgfqpoint{3.331077in}{5.720568in}}{\pgfqpoint{3.323264in}{5.712754in}}%
\pgfpathcurveto{\pgfqpoint{3.315450in}{5.704941in}}{\pgfqpoint{3.311060in}{5.694342in}}{\pgfqpoint{3.311060in}{5.683291in}}%
\pgfpathcurveto{\pgfqpoint{3.311060in}{5.672241in}}{\pgfqpoint{3.315450in}{5.661642in}}{\pgfqpoint{3.323264in}{5.653829in}}%
\pgfpathcurveto{\pgfqpoint{3.331077in}{5.646015in}}{\pgfqpoint{3.341676in}{5.641625in}}{\pgfqpoint{3.352726in}{5.641625in}}%
\pgfpathclose%
\pgfusepath{stroke,fill}%
\end{pgfscope}%
\begin{pgfscope}%
\pgfpathrectangle{\pgfqpoint{0.526127in}{0.331635in}}{\pgfqpoint{9.300000in}{7.700000in}}%
\pgfusepath{clip}%
\pgfsetbuttcap%
\pgfsetroundjoin%
\definecolor{currentfill}{rgb}{1.000000,0.623529,0.607843}%
\pgfsetfillcolor{currentfill}%
\pgfsetlinewidth{0.481800pt}%
\definecolor{currentstroke}{rgb}{1.000000,1.000000,1.000000}%
\pgfsetstrokecolor{currentstroke}%
\pgfsetdash{}{0pt}%
\pgfpathmoveto{\pgfqpoint{4.527282in}{5.262823in}}%
\pgfpathcurveto{\pgfqpoint{4.538332in}{5.262823in}}{\pgfqpoint{4.548931in}{5.267214in}}{\pgfqpoint{4.556744in}{5.275027in}}%
\pgfpathcurveto{\pgfqpoint{4.564558in}{5.282841in}}{\pgfqpoint{4.568948in}{5.293440in}}{\pgfqpoint{4.568948in}{5.304490in}}%
\pgfpathcurveto{\pgfqpoint{4.568948in}{5.315540in}}{\pgfqpoint{4.564558in}{5.326139in}}{\pgfqpoint{4.556744in}{5.333953in}}%
\pgfpathcurveto{\pgfqpoint{4.548931in}{5.341766in}}{\pgfqpoint{4.538332in}{5.346157in}}{\pgfqpoint{4.527282in}{5.346157in}}%
\pgfpathcurveto{\pgfqpoint{4.516232in}{5.346157in}}{\pgfqpoint{4.505633in}{5.341766in}}{\pgfqpoint{4.497819in}{5.333953in}}%
\pgfpathcurveto{\pgfqpoint{4.490005in}{5.326139in}}{\pgfqpoint{4.485615in}{5.315540in}}{\pgfqpoint{4.485615in}{5.304490in}}%
\pgfpathcurveto{\pgfqpoint{4.485615in}{5.293440in}}{\pgfqpoint{4.490005in}{5.282841in}}{\pgfqpoint{4.497819in}{5.275027in}}%
\pgfpathcurveto{\pgfqpoint{4.505633in}{5.267214in}}{\pgfqpoint{4.516232in}{5.262823in}}{\pgfqpoint{4.527282in}{5.262823in}}%
\pgfpathclose%
\pgfusepath{stroke,fill}%
\end{pgfscope}%
\begin{pgfscope}%
\pgfpathrectangle{\pgfqpoint{0.526127in}{0.331635in}}{\pgfqpoint{9.300000in}{7.700000in}}%
\pgfusepath{clip}%
\pgfsetbuttcap%
\pgfsetroundjoin%
\definecolor{currentfill}{rgb}{1.000000,0.623529,0.607843}%
\pgfsetfillcolor{currentfill}%
\pgfsetlinewidth{0.481800pt}%
\definecolor{currentstroke}{rgb}{1.000000,1.000000,1.000000}%
\pgfsetstrokecolor{currentstroke}%
\pgfsetdash{}{0pt}%
\pgfpathmoveto{\pgfqpoint{4.542592in}{3.093998in}}%
\pgfpathcurveto{\pgfqpoint{4.553642in}{3.093998in}}{\pgfqpoint{4.564241in}{3.098388in}}{\pgfqpoint{4.572055in}{3.106201in}}%
\pgfpathcurveto{\pgfqpoint{4.579868in}{3.114015in}}{\pgfqpoint{4.584259in}{3.124614in}}{\pgfqpoint{4.584259in}{3.135664in}}%
\pgfpathcurveto{\pgfqpoint{4.584259in}{3.146714in}}{\pgfqpoint{4.579868in}{3.157313in}}{\pgfqpoint{4.572055in}{3.165127in}}%
\pgfpathcurveto{\pgfqpoint{4.564241in}{3.172941in}}{\pgfqpoint{4.553642in}{3.177331in}}{\pgfqpoint{4.542592in}{3.177331in}}%
\pgfpathcurveto{\pgfqpoint{4.531542in}{3.177331in}}{\pgfqpoint{4.520943in}{3.172941in}}{\pgfqpoint{4.513129in}{3.165127in}}%
\pgfpathcurveto{\pgfqpoint{4.505316in}{3.157313in}}{\pgfqpoint{4.500925in}{3.146714in}}{\pgfqpoint{4.500925in}{3.135664in}}%
\pgfpathcurveto{\pgfqpoint{4.500925in}{3.124614in}}{\pgfqpoint{4.505316in}{3.114015in}}{\pgfqpoint{4.513129in}{3.106201in}}%
\pgfpathcurveto{\pgfqpoint{4.520943in}{3.098388in}}{\pgfqpoint{4.531542in}{3.093998in}}{\pgfqpoint{4.542592in}{3.093998in}}%
\pgfpathclose%
\pgfusepath{stroke,fill}%
\end{pgfscope}%
\begin{pgfscope}%
\pgfpathrectangle{\pgfqpoint{0.526127in}{0.331635in}}{\pgfqpoint{9.300000in}{7.700000in}}%
\pgfusepath{clip}%
\pgfsetbuttcap%
\pgfsetroundjoin%
\definecolor{currentfill}{rgb}{1.000000,0.623529,0.607843}%
\pgfsetfillcolor{currentfill}%
\pgfsetlinewidth{0.481800pt}%
\definecolor{currentstroke}{rgb}{1.000000,1.000000,1.000000}%
\pgfsetstrokecolor{currentstroke}%
\pgfsetdash{}{0pt}%
\pgfpathmoveto{\pgfqpoint{6.055654in}{5.581923in}}%
\pgfpathcurveto{\pgfqpoint{6.066704in}{5.581923in}}{\pgfqpoint{6.077303in}{5.586314in}}{\pgfqpoint{6.085117in}{5.594127in}}%
\pgfpathcurveto{\pgfqpoint{6.092930in}{5.601941in}}{\pgfqpoint{6.097320in}{5.612540in}}{\pgfqpoint{6.097320in}{5.623590in}}%
\pgfpathcurveto{\pgfqpoint{6.097320in}{5.634640in}}{\pgfqpoint{6.092930in}{5.645239in}}{\pgfqpoint{6.085117in}{5.653053in}}%
\pgfpathcurveto{\pgfqpoint{6.077303in}{5.660866in}}{\pgfqpoint{6.066704in}{5.665257in}}{\pgfqpoint{6.055654in}{5.665257in}}%
\pgfpathcurveto{\pgfqpoint{6.044604in}{5.665257in}}{\pgfqpoint{6.034005in}{5.660866in}}{\pgfqpoint{6.026191in}{5.653053in}}%
\pgfpathcurveto{\pgfqpoint{6.018377in}{5.645239in}}{\pgfqpoint{6.013987in}{5.634640in}}{\pgfqpoint{6.013987in}{5.623590in}}%
\pgfpathcurveto{\pgfqpoint{6.013987in}{5.612540in}}{\pgfqpoint{6.018377in}{5.601941in}}{\pgfqpoint{6.026191in}{5.594127in}}%
\pgfpathcurveto{\pgfqpoint{6.034005in}{5.586314in}}{\pgfqpoint{6.044604in}{5.581923in}}{\pgfqpoint{6.055654in}{5.581923in}}%
\pgfpathclose%
\pgfusepath{stroke,fill}%
\end{pgfscope}%
\begin{pgfscope}%
\pgfpathrectangle{\pgfqpoint{0.526127in}{0.331635in}}{\pgfqpoint{9.300000in}{7.700000in}}%
\pgfusepath{clip}%
\pgfsetbuttcap%
\pgfsetroundjoin%
\definecolor{currentfill}{rgb}{1.000000,0.623529,0.607843}%
\pgfsetfillcolor{currentfill}%
\pgfsetlinewidth{0.481800pt}%
\definecolor{currentstroke}{rgb}{1.000000,1.000000,1.000000}%
\pgfsetstrokecolor{currentstroke}%
\pgfsetdash{}{0pt}%
\pgfpathmoveto{\pgfqpoint{6.518607in}{5.384306in}}%
\pgfpathcurveto{\pgfqpoint{6.529657in}{5.384306in}}{\pgfqpoint{6.540256in}{5.388696in}}{\pgfqpoint{6.548070in}{5.396510in}}%
\pgfpathcurveto{\pgfqpoint{6.555884in}{5.404324in}}{\pgfqpoint{6.560274in}{5.414923in}}{\pgfqpoint{6.560274in}{5.425973in}}%
\pgfpathcurveto{\pgfqpoint{6.560274in}{5.437023in}}{\pgfqpoint{6.555884in}{5.447622in}}{\pgfqpoint{6.548070in}{5.455435in}}%
\pgfpathcurveto{\pgfqpoint{6.540256in}{5.463249in}}{\pgfqpoint{6.529657in}{5.467639in}}{\pgfqpoint{6.518607in}{5.467639in}}%
\pgfpathcurveto{\pgfqpoint{6.507557in}{5.467639in}}{\pgfqpoint{6.496958in}{5.463249in}}{\pgfqpoint{6.489144in}{5.455435in}}%
\pgfpathcurveto{\pgfqpoint{6.481331in}{5.447622in}}{\pgfqpoint{6.476941in}{5.437023in}}{\pgfqpoint{6.476941in}{5.425973in}}%
\pgfpathcurveto{\pgfqpoint{6.476941in}{5.414923in}}{\pgfqpoint{6.481331in}{5.404324in}}{\pgfqpoint{6.489144in}{5.396510in}}%
\pgfpathcurveto{\pgfqpoint{6.496958in}{5.388696in}}{\pgfqpoint{6.507557in}{5.384306in}}{\pgfqpoint{6.518607in}{5.384306in}}%
\pgfpathclose%
\pgfusepath{stroke,fill}%
\end{pgfscope}%
\begin{pgfscope}%
\pgfpathrectangle{\pgfqpoint{0.526127in}{0.331635in}}{\pgfqpoint{9.300000in}{7.700000in}}%
\pgfusepath{clip}%
\pgfsetbuttcap%
\pgfsetroundjoin%
\definecolor{currentfill}{rgb}{1.000000,0.623529,0.607843}%
\pgfsetfillcolor{currentfill}%
\pgfsetlinewidth{0.481800pt}%
\definecolor{currentstroke}{rgb}{1.000000,1.000000,1.000000}%
\pgfsetstrokecolor{currentstroke}%
\pgfsetdash{}{0pt}%
\pgfpathmoveto{\pgfqpoint{4.977203in}{5.447654in}}%
\pgfpathcurveto{\pgfqpoint{4.988253in}{5.447654in}}{\pgfqpoint{4.998852in}{5.452044in}}{\pgfqpoint{5.006666in}{5.459857in}}%
\pgfpathcurveto{\pgfqpoint{5.014480in}{5.467671in}}{\pgfqpoint{5.018870in}{5.478270in}}{\pgfqpoint{5.018870in}{5.489320in}}%
\pgfpathcurveto{\pgfqpoint{5.018870in}{5.500370in}}{\pgfqpoint{5.014480in}{5.510969in}}{\pgfqpoint{5.006666in}{5.518783in}}%
\pgfpathcurveto{\pgfqpoint{4.998852in}{5.526597in}}{\pgfqpoint{4.988253in}{5.530987in}}{\pgfqpoint{4.977203in}{5.530987in}}%
\pgfpathcurveto{\pgfqpoint{4.966153in}{5.530987in}}{\pgfqpoint{4.955554in}{5.526597in}}{\pgfqpoint{4.947740in}{5.518783in}}%
\pgfpathcurveto{\pgfqpoint{4.939927in}{5.510969in}}{\pgfqpoint{4.935537in}{5.500370in}}{\pgfqpoint{4.935537in}{5.489320in}}%
\pgfpathcurveto{\pgfqpoint{4.935537in}{5.478270in}}{\pgfqpoint{4.939927in}{5.467671in}}{\pgfqpoint{4.947740in}{5.459857in}}%
\pgfpathcurveto{\pgfqpoint{4.955554in}{5.452044in}}{\pgfqpoint{4.966153in}{5.447654in}}{\pgfqpoint{4.977203in}{5.447654in}}%
\pgfpathclose%
\pgfusepath{stroke,fill}%
\end{pgfscope}%
\begin{pgfscope}%
\pgfpathrectangle{\pgfqpoint{0.526127in}{0.331635in}}{\pgfqpoint{9.300000in}{7.700000in}}%
\pgfusepath{clip}%
\pgfsetbuttcap%
\pgfsetroundjoin%
\definecolor{currentfill}{rgb}{1.000000,0.623529,0.607843}%
\pgfsetfillcolor{currentfill}%
\pgfsetlinewidth{0.481800pt}%
\definecolor{currentstroke}{rgb}{1.000000,1.000000,1.000000}%
\pgfsetstrokecolor{currentstroke}%
\pgfsetdash{}{0pt}%
\pgfpathmoveto{\pgfqpoint{4.655568in}{5.295323in}}%
\pgfpathcurveto{\pgfqpoint{4.666618in}{5.295323in}}{\pgfqpoint{4.677217in}{5.299713in}}{\pgfqpoint{4.685031in}{5.307527in}}%
\pgfpathcurveto{\pgfqpoint{4.692845in}{5.315340in}}{\pgfqpoint{4.697235in}{5.325939in}}{\pgfqpoint{4.697235in}{5.336989in}}%
\pgfpathcurveto{\pgfqpoint{4.697235in}{5.348040in}}{\pgfqpoint{4.692845in}{5.358639in}}{\pgfqpoint{4.685031in}{5.366452in}}%
\pgfpathcurveto{\pgfqpoint{4.677217in}{5.374266in}}{\pgfqpoint{4.666618in}{5.378656in}}{\pgfqpoint{4.655568in}{5.378656in}}%
\pgfpathcurveto{\pgfqpoint{4.644518in}{5.378656in}}{\pgfqpoint{4.633919in}{5.374266in}}{\pgfqpoint{4.626105in}{5.366452in}}%
\pgfpathcurveto{\pgfqpoint{4.618292in}{5.358639in}}{\pgfqpoint{4.613902in}{5.348040in}}{\pgfqpoint{4.613902in}{5.336989in}}%
\pgfpathcurveto{\pgfqpoint{4.613902in}{5.325939in}}{\pgfqpoint{4.618292in}{5.315340in}}{\pgfqpoint{4.626105in}{5.307527in}}%
\pgfpathcurveto{\pgfqpoint{4.633919in}{5.299713in}}{\pgfqpoint{4.644518in}{5.295323in}}{\pgfqpoint{4.655568in}{5.295323in}}%
\pgfpathclose%
\pgfusepath{stroke,fill}%
\end{pgfscope}%
\begin{pgfscope}%
\pgfpathrectangle{\pgfqpoint{0.526127in}{0.331635in}}{\pgfqpoint{9.300000in}{7.700000in}}%
\pgfusepath{clip}%
\pgfsetbuttcap%
\pgfsetroundjoin%
\definecolor{currentfill}{rgb}{1.000000,0.623529,0.607843}%
\pgfsetfillcolor{currentfill}%
\pgfsetlinewidth{0.481800pt}%
\definecolor{currentstroke}{rgb}{1.000000,1.000000,1.000000}%
\pgfsetstrokecolor{currentstroke}%
\pgfsetdash{}{0pt}%
\pgfpathmoveto{\pgfqpoint{5.314689in}{4.304703in}}%
\pgfpathcurveto{\pgfqpoint{5.325739in}{4.304703in}}{\pgfqpoint{5.336338in}{4.309093in}}{\pgfqpoint{5.344152in}{4.316907in}}%
\pgfpathcurveto{\pgfqpoint{5.351965in}{4.324720in}}{\pgfqpoint{5.356356in}{4.335319in}}{\pgfqpoint{5.356356in}{4.346369in}}%
\pgfpathcurveto{\pgfqpoint{5.356356in}{4.357420in}}{\pgfqpoint{5.351965in}{4.368019in}}{\pgfqpoint{5.344152in}{4.375832in}}%
\pgfpathcurveto{\pgfqpoint{5.336338in}{4.383646in}}{\pgfqpoint{5.325739in}{4.388036in}}{\pgfqpoint{5.314689in}{4.388036in}}%
\pgfpathcurveto{\pgfqpoint{5.303639in}{4.388036in}}{\pgfqpoint{5.293040in}{4.383646in}}{\pgfqpoint{5.285226in}{4.375832in}}%
\pgfpathcurveto{\pgfqpoint{5.277413in}{4.368019in}}{\pgfqpoint{5.273022in}{4.357420in}}{\pgfqpoint{5.273022in}{4.346369in}}%
\pgfpathcurveto{\pgfqpoint{5.273022in}{4.335319in}}{\pgfqpoint{5.277413in}{4.324720in}}{\pgfqpoint{5.285226in}{4.316907in}}%
\pgfpathcurveto{\pgfqpoint{5.293040in}{4.309093in}}{\pgfqpoint{5.303639in}{4.304703in}}{\pgfqpoint{5.314689in}{4.304703in}}%
\pgfpathclose%
\pgfusepath{stroke,fill}%
\end{pgfscope}%
\begin{pgfscope}%
\pgfpathrectangle{\pgfqpoint{0.526127in}{0.331635in}}{\pgfqpoint{9.300000in}{7.700000in}}%
\pgfusepath{clip}%
\pgfsetbuttcap%
\pgfsetroundjoin%
\definecolor{currentfill}{rgb}{0.815686,0.733333,1.000000}%
\pgfsetfillcolor{currentfill}%
\pgfsetlinewidth{0.481800pt}%
\definecolor{currentstroke}{rgb}{1.000000,1.000000,1.000000}%
\pgfsetstrokecolor{currentstroke}%
\pgfsetdash{}{0pt}%
\pgfpathmoveto{\pgfqpoint{6.917849in}{2.661457in}}%
\pgfpathcurveto{\pgfqpoint{6.928899in}{2.661457in}}{\pgfqpoint{6.939498in}{2.665847in}}{\pgfqpoint{6.947312in}{2.673661in}}%
\pgfpathcurveto{\pgfqpoint{6.955125in}{2.681475in}}{\pgfqpoint{6.959516in}{2.692074in}}{\pgfqpoint{6.959516in}{2.703124in}}%
\pgfpathcurveto{\pgfqpoint{6.959516in}{2.714174in}}{\pgfqpoint{6.955125in}{2.724773in}}{\pgfqpoint{6.947312in}{2.732587in}}%
\pgfpathcurveto{\pgfqpoint{6.939498in}{2.740400in}}{\pgfqpoint{6.928899in}{2.744790in}}{\pgfqpoint{6.917849in}{2.744790in}}%
\pgfpathcurveto{\pgfqpoint{6.906799in}{2.744790in}}{\pgfqpoint{6.896200in}{2.740400in}}{\pgfqpoint{6.888386in}{2.732587in}}%
\pgfpathcurveto{\pgfqpoint{6.880572in}{2.724773in}}{\pgfqpoint{6.876182in}{2.714174in}}{\pgfqpoint{6.876182in}{2.703124in}}%
\pgfpathcurveto{\pgfqpoint{6.876182in}{2.692074in}}{\pgfqpoint{6.880572in}{2.681475in}}{\pgfqpoint{6.888386in}{2.673661in}}%
\pgfpathcurveto{\pgfqpoint{6.896200in}{2.665847in}}{\pgfqpoint{6.906799in}{2.661457in}}{\pgfqpoint{6.917849in}{2.661457in}}%
\pgfpathclose%
\pgfusepath{stroke,fill}%
\end{pgfscope}%
\begin{pgfscope}%
\pgfpathrectangle{\pgfqpoint{0.526127in}{0.331635in}}{\pgfqpoint{9.300000in}{7.700000in}}%
\pgfusepath{clip}%
\pgfsetbuttcap%
\pgfsetroundjoin%
\definecolor{currentfill}{rgb}{0.815686,0.733333,1.000000}%
\pgfsetfillcolor{currentfill}%
\pgfsetlinewidth{0.481800pt}%
\definecolor{currentstroke}{rgb}{1.000000,1.000000,1.000000}%
\pgfsetstrokecolor{currentstroke}%
\pgfsetdash{}{0pt}%
\pgfpathmoveto{\pgfqpoint{6.254342in}{2.041288in}}%
\pgfpathcurveto{\pgfqpoint{6.265392in}{2.041288in}}{\pgfqpoint{6.275991in}{2.045678in}}{\pgfqpoint{6.283804in}{2.053492in}}%
\pgfpathcurveto{\pgfqpoint{6.291618in}{2.061305in}}{\pgfqpoint{6.296008in}{2.071904in}}{\pgfqpoint{6.296008in}{2.082954in}}%
\pgfpathcurveto{\pgfqpoint{6.296008in}{2.094005in}}{\pgfqpoint{6.291618in}{2.104604in}}{\pgfqpoint{6.283804in}{2.112417in}}%
\pgfpathcurveto{\pgfqpoint{6.275991in}{2.120231in}}{\pgfqpoint{6.265392in}{2.124621in}}{\pgfqpoint{6.254342in}{2.124621in}}%
\pgfpathcurveto{\pgfqpoint{6.243291in}{2.124621in}}{\pgfqpoint{6.232692in}{2.120231in}}{\pgfqpoint{6.224879in}{2.112417in}}%
\pgfpathcurveto{\pgfqpoint{6.217065in}{2.104604in}}{\pgfqpoint{6.212675in}{2.094005in}}{\pgfqpoint{6.212675in}{2.082954in}}%
\pgfpathcurveto{\pgfqpoint{6.212675in}{2.071904in}}{\pgfqpoint{6.217065in}{2.061305in}}{\pgfqpoint{6.224879in}{2.053492in}}%
\pgfpathcurveto{\pgfqpoint{6.232692in}{2.045678in}}{\pgfqpoint{6.243291in}{2.041288in}}{\pgfqpoint{6.254342in}{2.041288in}}%
\pgfpathclose%
\pgfusepath{stroke,fill}%
\end{pgfscope}%
\begin{pgfscope}%
\pgfpathrectangle{\pgfqpoint{0.526127in}{0.331635in}}{\pgfqpoint{9.300000in}{7.700000in}}%
\pgfusepath{clip}%
\pgfsetbuttcap%
\pgfsetroundjoin%
\definecolor{currentfill}{rgb}{0.815686,0.733333,1.000000}%
\pgfsetfillcolor{currentfill}%
\pgfsetlinewidth{0.481800pt}%
\definecolor{currentstroke}{rgb}{1.000000,1.000000,1.000000}%
\pgfsetstrokecolor{currentstroke}%
\pgfsetdash{}{0pt}%
\pgfpathmoveto{\pgfqpoint{4.500103in}{1.819196in}}%
\pgfpathcurveto{\pgfqpoint{4.511153in}{1.819196in}}{\pgfqpoint{4.521752in}{1.823586in}}{\pgfqpoint{4.529566in}{1.831400in}}%
\pgfpathcurveto{\pgfqpoint{4.537379in}{1.839214in}}{\pgfqpoint{4.541770in}{1.849813in}}{\pgfqpoint{4.541770in}{1.860863in}}%
\pgfpathcurveto{\pgfqpoint{4.541770in}{1.871913in}}{\pgfqpoint{4.537379in}{1.882512in}}{\pgfqpoint{4.529566in}{1.890326in}}%
\pgfpathcurveto{\pgfqpoint{4.521752in}{1.898139in}}{\pgfqpoint{4.511153in}{1.902529in}}{\pgfqpoint{4.500103in}{1.902529in}}%
\pgfpathcurveto{\pgfqpoint{4.489053in}{1.902529in}}{\pgfqpoint{4.478454in}{1.898139in}}{\pgfqpoint{4.470640in}{1.890326in}}%
\pgfpathcurveto{\pgfqpoint{4.462826in}{1.882512in}}{\pgfqpoint{4.458436in}{1.871913in}}{\pgfqpoint{4.458436in}{1.860863in}}%
\pgfpathcurveto{\pgfqpoint{4.458436in}{1.849813in}}{\pgfqpoint{4.462826in}{1.839214in}}{\pgfqpoint{4.470640in}{1.831400in}}%
\pgfpathcurveto{\pgfqpoint{4.478454in}{1.823586in}}{\pgfqpoint{4.489053in}{1.819196in}}{\pgfqpoint{4.500103in}{1.819196in}}%
\pgfpathclose%
\pgfusepath{stroke,fill}%
\end{pgfscope}%
\begin{pgfscope}%
\pgfpathrectangle{\pgfqpoint{0.526127in}{0.331635in}}{\pgfqpoint{9.300000in}{7.700000in}}%
\pgfusepath{clip}%
\pgfsetbuttcap%
\pgfsetroundjoin%
\definecolor{currentfill}{rgb}{0.815686,0.733333,1.000000}%
\pgfsetfillcolor{currentfill}%
\pgfsetlinewidth{0.481800pt}%
\definecolor{currentstroke}{rgb}{1.000000,1.000000,1.000000}%
\pgfsetstrokecolor{currentstroke}%
\pgfsetdash{}{0pt}%
\pgfpathmoveto{\pgfqpoint{3.205256in}{1.791054in}}%
\pgfpathcurveto{\pgfqpoint{3.216306in}{1.791054in}}{\pgfqpoint{3.226905in}{1.795444in}}{\pgfqpoint{3.234719in}{1.803258in}}%
\pgfpathcurveto{\pgfqpoint{3.242532in}{1.811072in}}{\pgfqpoint{3.246923in}{1.821671in}}{\pgfqpoint{3.246923in}{1.832721in}}%
\pgfpathcurveto{\pgfqpoint{3.246923in}{1.843771in}}{\pgfqpoint{3.242532in}{1.854370in}}{\pgfqpoint{3.234719in}{1.862184in}}%
\pgfpathcurveto{\pgfqpoint{3.226905in}{1.869997in}}{\pgfqpoint{3.216306in}{1.874387in}}{\pgfqpoint{3.205256in}{1.874387in}}%
\pgfpathcurveto{\pgfqpoint{3.194206in}{1.874387in}}{\pgfqpoint{3.183607in}{1.869997in}}{\pgfqpoint{3.175793in}{1.862184in}}%
\pgfpathcurveto{\pgfqpoint{3.167980in}{1.854370in}}{\pgfqpoint{3.163589in}{1.843771in}}{\pgfqpoint{3.163589in}{1.832721in}}%
\pgfpathcurveto{\pgfqpoint{3.163589in}{1.821671in}}{\pgfqpoint{3.167980in}{1.811072in}}{\pgfqpoint{3.175793in}{1.803258in}}%
\pgfpathcurveto{\pgfqpoint{3.183607in}{1.795444in}}{\pgfqpoint{3.194206in}{1.791054in}}{\pgfqpoint{3.205256in}{1.791054in}}%
\pgfpathclose%
\pgfusepath{stroke,fill}%
\end{pgfscope}%
\begin{pgfscope}%
\pgfpathrectangle{\pgfqpoint{0.526127in}{0.331635in}}{\pgfqpoint{9.300000in}{7.700000in}}%
\pgfusepath{clip}%
\pgfsetbuttcap%
\pgfsetroundjoin%
\definecolor{currentfill}{rgb}{0.815686,0.733333,1.000000}%
\pgfsetfillcolor{currentfill}%
\pgfsetlinewidth{0.481800pt}%
\definecolor{currentstroke}{rgb}{1.000000,1.000000,1.000000}%
\pgfsetstrokecolor{currentstroke}%
\pgfsetdash{}{0pt}%
\pgfpathmoveto{\pgfqpoint{6.199623in}{2.176425in}}%
\pgfpathcurveto{\pgfqpoint{6.210674in}{2.176425in}}{\pgfqpoint{6.221273in}{2.180815in}}{\pgfqpoint{6.229086in}{2.188629in}}%
\pgfpathcurveto{\pgfqpoint{6.236900in}{2.196442in}}{\pgfqpoint{6.241290in}{2.207041in}}{\pgfqpoint{6.241290in}{2.218091in}}%
\pgfpathcurveto{\pgfqpoint{6.241290in}{2.229142in}}{\pgfqpoint{6.236900in}{2.239741in}}{\pgfqpoint{6.229086in}{2.247554in}}%
\pgfpathcurveto{\pgfqpoint{6.221273in}{2.255368in}}{\pgfqpoint{6.210674in}{2.259758in}}{\pgfqpoint{6.199623in}{2.259758in}}%
\pgfpathcurveto{\pgfqpoint{6.188573in}{2.259758in}}{\pgfqpoint{6.177974in}{2.255368in}}{\pgfqpoint{6.170161in}{2.247554in}}%
\pgfpathcurveto{\pgfqpoint{6.162347in}{2.239741in}}{\pgfqpoint{6.157957in}{2.229142in}}{\pgfqpoint{6.157957in}{2.218091in}}%
\pgfpathcurveto{\pgfqpoint{6.157957in}{2.207041in}}{\pgfqpoint{6.162347in}{2.196442in}}{\pgfqpoint{6.170161in}{2.188629in}}%
\pgfpathcurveto{\pgfqpoint{6.177974in}{2.180815in}}{\pgfqpoint{6.188573in}{2.176425in}}{\pgfqpoint{6.199623in}{2.176425in}}%
\pgfpathclose%
\pgfusepath{stroke,fill}%
\end{pgfscope}%
\begin{pgfscope}%
\pgfpathrectangle{\pgfqpoint{0.526127in}{0.331635in}}{\pgfqpoint{9.300000in}{7.700000in}}%
\pgfusepath{clip}%
\pgfsetbuttcap%
\pgfsetroundjoin%
\definecolor{currentfill}{rgb}{0.815686,0.733333,1.000000}%
\pgfsetfillcolor{currentfill}%
\pgfsetlinewidth{0.481800pt}%
\definecolor{currentstroke}{rgb}{1.000000,1.000000,1.000000}%
\pgfsetstrokecolor{currentstroke}%
\pgfsetdash{}{0pt}%
\pgfpathmoveto{\pgfqpoint{7.384542in}{2.688185in}}%
\pgfpathcurveto{\pgfqpoint{7.395592in}{2.688185in}}{\pgfqpoint{7.406191in}{2.692576in}}{\pgfqpoint{7.414005in}{2.700389in}}%
\pgfpathcurveto{\pgfqpoint{7.421818in}{2.708203in}}{\pgfqpoint{7.426208in}{2.718802in}}{\pgfqpoint{7.426208in}{2.729852in}}%
\pgfpathcurveto{\pgfqpoint{7.426208in}{2.740902in}}{\pgfqpoint{7.421818in}{2.751501in}}{\pgfqpoint{7.414005in}{2.759315in}}%
\pgfpathcurveto{\pgfqpoint{7.406191in}{2.767128in}}{\pgfqpoint{7.395592in}{2.771519in}}{\pgfqpoint{7.384542in}{2.771519in}}%
\pgfpathcurveto{\pgfqpoint{7.373492in}{2.771519in}}{\pgfqpoint{7.362893in}{2.767128in}}{\pgfqpoint{7.355079in}{2.759315in}}%
\pgfpathcurveto{\pgfqpoint{7.347265in}{2.751501in}}{\pgfqpoint{7.342875in}{2.740902in}}{\pgfqpoint{7.342875in}{2.729852in}}%
\pgfpathcurveto{\pgfqpoint{7.342875in}{2.718802in}}{\pgfqpoint{7.347265in}{2.708203in}}{\pgfqpoint{7.355079in}{2.700389in}}%
\pgfpathcurveto{\pgfqpoint{7.362893in}{2.692576in}}{\pgfqpoint{7.373492in}{2.688185in}}{\pgfqpoint{7.384542in}{2.688185in}}%
\pgfpathclose%
\pgfusepath{stroke,fill}%
\end{pgfscope}%
\begin{pgfscope}%
\pgfpathrectangle{\pgfqpoint{0.526127in}{0.331635in}}{\pgfqpoint{9.300000in}{7.700000in}}%
\pgfusepath{clip}%
\pgfsetbuttcap%
\pgfsetroundjoin%
\definecolor{currentfill}{rgb}{0.815686,0.733333,1.000000}%
\pgfsetfillcolor{currentfill}%
\pgfsetlinewidth{0.481800pt}%
\definecolor{currentstroke}{rgb}{1.000000,1.000000,1.000000}%
\pgfsetstrokecolor{currentstroke}%
\pgfsetdash{}{0pt}%
\pgfpathmoveto{\pgfqpoint{2.373781in}{3.541707in}}%
\pgfpathcurveto{\pgfqpoint{2.384831in}{3.541707in}}{\pgfqpoint{2.395430in}{3.546098in}}{\pgfqpoint{2.403244in}{3.553911in}}%
\pgfpathcurveto{\pgfqpoint{2.411058in}{3.561725in}}{\pgfqpoint{2.415448in}{3.572324in}}{\pgfqpoint{2.415448in}{3.583374in}}%
\pgfpathcurveto{\pgfqpoint{2.415448in}{3.594424in}}{\pgfqpoint{2.411058in}{3.605023in}}{\pgfqpoint{2.403244in}{3.612837in}}%
\pgfpathcurveto{\pgfqpoint{2.395430in}{3.620650in}}{\pgfqpoint{2.384831in}{3.625041in}}{\pgfqpoint{2.373781in}{3.625041in}}%
\pgfpathcurveto{\pgfqpoint{2.362731in}{3.625041in}}{\pgfqpoint{2.352132in}{3.620650in}}{\pgfqpoint{2.344318in}{3.612837in}}%
\pgfpathcurveto{\pgfqpoint{2.336505in}{3.605023in}}{\pgfqpoint{2.332115in}{3.594424in}}{\pgfqpoint{2.332115in}{3.583374in}}%
\pgfpathcurveto{\pgfqpoint{2.332115in}{3.572324in}}{\pgfqpoint{2.336505in}{3.561725in}}{\pgfqpoint{2.344318in}{3.553911in}}%
\pgfpathcurveto{\pgfqpoint{2.352132in}{3.546098in}}{\pgfqpoint{2.362731in}{3.541707in}}{\pgfqpoint{2.373781in}{3.541707in}}%
\pgfpathclose%
\pgfusepath{stroke,fill}%
\end{pgfscope}%
\begin{pgfscope}%
\pgfpathrectangle{\pgfqpoint{0.526127in}{0.331635in}}{\pgfqpoint{9.300000in}{7.700000in}}%
\pgfusepath{clip}%
\pgfsetbuttcap%
\pgfsetroundjoin%
\definecolor{currentfill}{rgb}{0.815686,0.733333,1.000000}%
\pgfsetfillcolor{currentfill}%
\pgfsetlinewidth{0.481800pt}%
\definecolor{currentstroke}{rgb}{1.000000,1.000000,1.000000}%
\pgfsetstrokecolor{currentstroke}%
\pgfsetdash{}{0pt}%
\pgfpathmoveto{\pgfqpoint{7.393947in}{1.637329in}}%
\pgfpathcurveto{\pgfqpoint{7.404997in}{1.637329in}}{\pgfqpoint{7.415596in}{1.641719in}}{\pgfqpoint{7.423410in}{1.649532in}}%
\pgfpathcurveto{\pgfqpoint{7.431224in}{1.657346in}}{\pgfqpoint{7.435614in}{1.667945in}}{\pgfqpoint{7.435614in}{1.678995in}}%
\pgfpathcurveto{\pgfqpoint{7.435614in}{1.690045in}}{\pgfqpoint{7.431224in}{1.700644in}}{\pgfqpoint{7.423410in}{1.708458in}}%
\pgfpathcurveto{\pgfqpoint{7.415596in}{1.716272in}}{\pgfqpoint{7.404997in}{1.720662in}}{\pgfqpoint{7.393947in}{1.720662in}}%
\pgfpathcurveto{\pgfqpoint{7.382897in}{1.720662in}}{\pgfqpoint{7.372298in}{1.716272in}}{\pgfqpoint{7.364485in}{1.708458in}}%
\pgfpathcurveto{\pgfqpoint{7.356671in}{1.700644in}}{\pgfqpoint{7.352281in}{1.690045in}}{\pgfqpoint{7.352281in}{1.678995in}}%
\pgfpathcurveto{\pgfqpoint{7.352281in}{1.667945in}}{\pgfqpoint{7.356671in}{1.657346in}}{\pgfqpoint{7.364485in}{1.649532in}}%
\pgfpathcurveto{\pgfqpoint{7.372298in}{1.641719in}}{\pgfqpoint{7.382897in}{1.637329in}}{\pgfqpoint{7.393947in}{1.637329in}}%
\pgfpathclose%
\pgfusepath{stroke,fill}%
\end{pgfscope}%
\begin{pgfscope}%
\pgfpathrectangle{\pgfqpoint{0.526127in}{0.331635in}}{\pgfqpoint{9.300000in}{7.700000in}}%
\pgfusepath{clip}%
\pgfsetbuttcap%
\pgfsetroundjoin%
\definecolor{currentfill}{rgb}{0.815686,0.733333,1.000000}%
\pgfsetfillcolor{currentfill}%
\pgfsetlinewidth{0.481800pt}%
\definecolor{currentstroke}{rgb}{1.000000,1.000000,1.000000}%
\pgfsetstrokecolor{currentstroke}%
\pgfsetdash{}{0pt}%
\pgfpathmoveto{\pgfqpoint{7.162597in}{2.816526in}}%
\pgfpathcurveto{\pgfqpoint{7.173647in}{2.816526in}}{\pgfqpoint{7.184247in}{2.820916in}}{\pgfqpoint{7.192060in}{2.828730in}}%
\pgfpathcurveto{\pgfqpoint{7.199874in}{2.836543in}}{\pgfqpoint{7.204264in}{2.847142in}}{\pgfqpoint{7.204264in}{2.858193in}}%
\pgfpathcurveto{\pgfqpoint{7.204264in}{2.869243in}}{\pgfqpoint{7.199874in}{2.879842in}}{\pgfqpoint{7.192060in}{2.887655in}}%
\pgfpathcurveto{\pgfqpoint{7.184247in}{2.895469in}}{\pgfqpoint{7.173647in}{2.899859in}}{\pgfqpoint{7.162597in}{2.899859in}}%
\pgfpathcurveto{\pgfqpoint{7.151547in}{2.899859in}}{\pgfqpoint{7.140948in}{2.895469in}}{\pgfqpoint{7.133135in}{2.887655in}}%
\pgfpathcurveto{\pgfqpoint{7.125321in}{2.879842in}}{\pgfqpoint{7.120931in}{2.869243in}}{\pgfqpoint{7.120931in}{2.858193in}}%
\pgfpathcurveto{\pgfqpoint{7.120931in}{2.847142in}}{\pgfqpoint{7.125321in}{2.836543in}}{\pgfqpoint{7.133135in}{2.828730in}}%
\pgfpathcurveto{\pgfqpoint{7.140948in}{2.820916in}}{\pgfqpoint{7.151547in}{2.816526in}}{\pgfqpoint{7.162597in}{2.816526in}}%
\pgfpathclose%
\pgfusepath{stroke,fill}%
\end{pgfscope}%
\begin{pgfscope}%
\pgfpathrectangle{\pgfqpoint{0.526127in}{0.331635in}}{\pgfqpoint{9.300000in}{7.700000in}}%
\pgfusepath{clip}%
\pgfsetbuttcap%
\pgfsetroundjoin%
\definecolor{currentfill}{rgb}{0.815686,0.733333,1.000000}%
\pgfsetfillcolor{currentfill}%
\pgfsetlinewidth{0.481800pt}%
\definecolor{currentstroke}{rgb}{1.000000,1.000000,1.000000}%
\pgfsetstrokecolor{currentstroke}%
\pgfsetdash{}{0pt}%
\pgfpathmoveto{\pgfqpoint{6.664367in}{4.855590in}}%
\pgfpathcurveto{\pgfqpoint{6.675418in}{4.855590in}}{\pgfqpoint{6.686017in}{4.859980in}}{\pgfqpoint{6.693830in}{4.867794in}}%
\pgfpathcurveto{\pgfqpoint{6.701644in}{4.875607in}}{\pgfqpoint{6.706034in}{4.886206in}}{\pgfqpoint{6.706034in}{4.897256in}}%
\pgfpathcurveto{\pgfqpoint{6.706034in}{4.908306in}}{\pgfqpoint{6.701644in}{4.918906in}}{\pgfqpoint{6.693830in}{4.926719in}}%
\pgfpathcurveto{\pgfqpoint{6.686017in}{4.934533in}}{\pgfqpoint{6.675418in}{4.938923in}}{\pgfqpoint{6.664367in}{4.938923in}}%
\pgfpathcurveto{\pgfqpoint{6.653317in}{4.938923in}}{\pgfqpoint{6.642718in}{4.934533in}}{\pgfqpoint{6.634905in}{4.926719in}}%
\pgfpathcurveto{\pgfqpoint{6.627091in}{4.918906in}}{\pgfqpoint{6.622701in}{4.908306in}}{\pgfqpoint{6.622701in}{4.897256in}}%
\pgfpathcurveto{\pgfqpoint{6.622701in}{4.886206in}}{\pgfqpoint{6.627091in}{4.875607in}}{\pgfqpoint{6.634905in}{4.867794in}}%
\pgfpathcurveto{\pgfqpoint{6.642718in}{4.859980in}}{\pgfqpoint{6.653317in}{4.855590in}}{\pgfqpoint{6.664367in}{4.855590in}}%
\pgfpathclose%
\pgfusepath{stroke,fill}%
\end{pgfscope}%
\begin{pgfscope}%
\pgfpathrectangle{\pgfqpoint{0.526127in}{0.331635in}}{\pgfqpoint{9.300000in}{7.700000in}}%
\pgfusepath{clip}%
\pgfsetbuttcap%
\pgfsetroundjoin%
\definecolor{currentfill}{rgb}{0.815686,0.733333,1.000000}%
\pgfsetfillcolor{currentfill}%
\pgfsetlinewidth{0.481800pt}%
\definecolor{currentstroke}{rgb}{1.000000,1.000000,1.000000}%
\pgfsetstrokecolor{currentstroke}%
\pgfsetdash{}{0pt}%
\pgfpathmoveto{\pgfqpoint{5.711097in}{2.461358in}}%
\pgfpathcurveto{\pgfqpoint{5.722148in}{2.461358in}}{\pgfqpoint{5.732747in}{2.465749in}}{\pgfqpoint{5.740560in}{2.473562in}}%
\pgfpathcurveto{\pgfqpoint{5.748374in}{2.481376in}}{\pgfqpoint{5.752764in}{2.491975in}}{\pgfqpoint{5.752764in}{2.503025in}}%
\pgfpathcurveto{\pgfqpoint{5.752764in}{2.514075in}}{\pgfqpoint{5.748374in}{2.524674in}}{\pgfqpoint{5.740560in}{2.532488in}}%
\pgfpathcurveto{\pgfqpoint{5.732747in}{2.540301in}}{\pgfqpoint{5.722148in}{2.544692in}}{\pgfqpoint{5.711097in}{2.544692in}}%
\pgfpathcurveto{\pgfqpoint{5.700047in}{2.544692in}}{\pgfqpoint{5.689448in}{2.540301in}}{\pgfqpoint{5.681635in}{2.532488in}}%
\pgfpathcurveto{\pgfqpoint{5.673821in}{2.524674in}}{\pgfqpoint{5.669431in}{2.514075in}}{\pgfqpoint{5.669431in}{2.503025in}}%
\pgfpathcurveto{\pgfqpoint{5.669431in}{2.491975in}}{\pgfqpoint{5.673821in}{2.481376in}}{\pgfqpoint{5.681635in}{2.473562in}}%
\pgfpathcurveto{\pgfqpoint{5.689448in}{2.465749in}}{\pgfqpoint{5.700047in}{2.461358in}}{\pgfqpoint{5.711097in}{2.461358in}}%
\pgfpathclose%
\pgfusepath{stroke,fill}%
\end{pgfscope}%
\begin{pgfscope}%
\pgfpathrectangle{\pgfqpoint{0.526127in}{0.331635in}}{\pgfqpoint{9.300000in}{7.700000in}}%
\pgfusepath{clip}%
\pgfsetbuttcap%
\pgfsetroundjoin%
\definecolor{currentfill}{rgb}{0.815686,0.733333,1.000000}%
\pgfsetfillcolor{currentfill}%
\pgfsetlinewidth{0.481800pt}%
\definecolor{currentstroke}{rgb}{1.000000,1.000000,1.000000}%
\pgfsetstrokecolor{currentstroke}%
\pgfsetdash{}{0pt}%
\pgfpathmoveto{\pgfqpoint{6.805509in}{2.175050in}}%
\pgfpathcurveto{\pgfqpoint{6.816559in}{2.175050in}}{\pgfqpoint{6.827158in}{2.179440in}}{\pgfqpoint{6.834972in}{2.187254in}}%
\pgfpathcurveto{\pgfqpoint{6.842785in}{2.195068in}}{\pgfqpoint{6.847175in}{2.205667in}}{\pgfqpoint{6.847175in}{2.216717in}}%
\pgfpathcurveto{\pgfqpoint{6.847175in}{2.227767in}}{\pgfqpoint{6.842785in}{2.238366in}}{\pgfqpoint{6.834972in}{2.246180in}}%
\pgfpathcurveto{\pgfqpoint{6.827158in}{2.253993in}}{\pgfqpoint{6.816559in}{2.258383in}}{\pgfqpoint{6.805509in}{2.258383in}}%
\pgfpathcurveto{\pgfqpoint{6.794459in}{2.258383in}}{\pgfqpoint{6.783860in}{2.253993in}}{\pgfqpoint{6.776046in}{2.246180in}}%
\pgfpathcurveto{\pgfqpoint{6.768232in}{2.238366in}}{\pgfqpoint{6.763842in}{2.227767in}}{\pgfqpoint{6.763842in}{2.216717in}}%
\pgfpathcurveto{\pgfqpoint{6.763842in}{2.205667in}}{\pgfqpoint{6.768232in}{2.195068in}}{\pgfqpoint{6.776046in}{2.187254in}}%
\pgfpathcurveto{\pgfqpoint{6.783860in}{2.179440in}}{\pgfqpoint{6.794459in}{2.175050in}}{\pgfqpoint{6.805509in}{2.175050in}}%
\pgfpathclose%
\pgfusepath{stroke,fill}%
\end{pgfscope}%
\begin{pgfscope}%
\pgfpathrectangle{\pgfqpoint{0.526127in}{0.331635in}}{\pgfqpoint{9.300000in}{7.700000in}}%
\pgfusepath{clip}%
\pgfsetbuttcap%
\pgfsetroundjoin%
\definecolor{currentfill}{rgb}{0.815686,0.733333,1.000000}%
\pgfsetfillcolor{currentfill}%
\pgfsetlinewidth{0.481800pt}%
\definecolor{currentstroke}{rgb}{1.000000,1.000000,1.000000}%
\pgfsetstrokecolor{currentstroke}%
\pgfsetdash{}{0pt}%
\pgfpathmoveto{\pgfqpoint{3.875538in}{2.264873in}}%
\pgfpathcurveto{\pgfqpoint{3.886588in}{2.264873in}}{\pgfqpoint{3.897187in}{2.269263in}}{\pgfqpoint{3.905000in}{2.277077in}}%
\pgfpathcurveto{\pgfqpoint{3.912814in}{2.284890in}}{\pgfqpoint{3.917204in}{2.295490in}}{\pgfqpoint{3.917204in}{2.306540in}}%
\pgfpathcurveto{\pgfqpoint{3.917204in}{2.317590in}}{\pgfqpoint{3.912814in}{2.328189in}}{\pgfqpoint{3.905000in}{2.336002in}}%
\pgfpathcurveto{\pgfqpoint{3.897187in}{2.343816in}}{\pgfqpoint{3.886588in}{2.348206in}}{\pgfqpoint{3.875538in}{2.348206in}}%
\pgfpathcurveto{\pgfqpoint{3.864487in}{2.348206in}}{\pgfqpoint{3.853888in}{2.343816in}}{\pgfqpoint{3.846075in}{2.336002in}}%
\pgfpathcurveto{\pgfqpoint{3.838261in}{2.328189in}}{\pgfqpoint{3.833871in}{2.317590in}}{\pgfqpoint{3.833871in}{2.306540in}}%
\pgfpathcurveto{\pgfqpoint{3.833871in}{2.295490in}}{\pgfqpoint{3.838261in}{2.284890in}}{\pgfqpoint{3.846075in}{2.277077in}}%
\pgfpathcurveto{\pgfqpoint{3.853888in}{2.269263in}}{\pgfqpoint{3.864487in}{2.264873in}}{\pgfqpoint{3.875538in}{2.264873in}}%
\pgfpathclose%
\pgfusepath{stroke,fill}%
\end{pgfscope}%
\begin{pgfscope}%
\pgfpathrectangle{\pgfqpoint{0.526127in}{0.331635in}}{\pgfqpoint{9.300000in}{7.700000in}}%
\pgfusepath{clip}%
\pgfsetbuttcap%
\pgfsetroundjoin%
\definecolor{currentfill}{rgb}{0.815686,0.733333,1.000000}%
\pgfsetfillcolor{currentfill}%
\pgfsetlinewidth{0.481800pt}%
\definecolor{currentstroke}{rgb}{1.000000,1.000000,1.000000}%
\pgfsetstrokecolor{currentstroke}%
\pgfsetdash{}{0pt}%
\pgfpathmoveto{\pgfqpoint{1.970636in}{4.841715in}}%
\pgfpathcurveto{\pgfqpoint{1.981686in}{4.841715in}}{\pgfqpoint{1.992285in}{4.846105in}}{\pgfqpoint{2.000099in}{4.853919in}}%
\pgfpathcurveto{\pgfqpoint{2.007912in}{4.861733in}}{\pgfqpoint{2.012303in}{4.872332in}}{\pgfqpoint{2.012303in}{4.883382in}}%
\pgfpathcurveto{\pgfqpoint{2.012303in}{4.894432in}}{\pgfqpoint{2.007912in}{4.905031in}}{\pgfqpoint{2.000099in}{4.912845in}}%
\pgfpathcurveto{\pgfqpoint{1.992285in}{4.920658in}}{\pgfqpoint{1.981686in}{4.925048in}}{\pgfqpoint{1.970636in}{4.925048in}}%
\pgfpathcurveto{\pgfqpoint{1.959586in}{4.925048in}}{\pgfqpoint{1.948987in}{4.920658in}}{\pgfqpoint{1.941173in}{4.912845in}}%
\pgfpathcurveto{\pgfqpoint{1.933359in}{4.905031in}}{\pgfqpoint{1.928969in}{4.894432in}}{\pgfqpoint{1.928969in}{4.883382in}}%
\pgfpathcurveto{\pgfqpoint{1.928969in}{4.872332in}}{\pgfqpoint{1.933359in}{4.861733in}}{\pgfqpoint{1.941173in}{4.853919in}}%
\pgfpathcurveto{\pgfqpoint{1.948987in}{4.846105in}}{\pgfqpoint{1.959586in}{4.841715in}}{\pgfqpoint{1.970636in}{4.841715in}}%
\pgfpathclose%
\pgfusepath{stroke,fill}%
\end{pgfscope}%
\begin{pgfscope}%
\pgfpathrectangle{\pgfqpoint{0.526127in}{0.331635in}}{\pgfqpoint{9.300000in}{7.700000in}}%
\pgfusepath{clip}%
\pgfsetbuttcap%
\pgfsetroundjoin%
\definecolor{currentfill}{rgb}{0.815686,0.733333,1.000000}%
\pgfsetfillcolor{currentfill}%
\pgfsetlinewidth{0.481800pt}%
\definecolor{currentstroke}{rgb}{1.000000,1.000000,1.000000}%
\pgfsetstrokecolor{currentstroke}%
\pgfsetdash{}{0pt}%
\pgfpathmoveto{\pgfqpoint{3.343614in}{2.625512in}}%
\pgfpathcurveto{\pgfqpoint{3.354664in}{2.625512in}}{\pgfqpoint{3.365263in}{2.629902in}}{\pgfqpoint{3.373077in}{2.637716in}}%
\pgfpathcurveto{\pgfqpoint{3.380890in}{2.645530in}}{\pgfqpoint{3.385281in}{2.656129in}}{\pgfqpoint{3.385281in}{2.667179in}}%
\pgfpathcurveto{\pgfqpoint{3.385281in}{2.678229in}}{\pgfqpoint{3.380890in}{2.688828in}}{\pgfqpoint{3.373077in}{2.696641in}}%
\pgfpathcurveto{\pgfqpoint{3.365263in}{2.704455in}}{\pgfqpoint{3.354664in}{2.708845in}}{\pgfqpoint{3.343614in}{2.708845in}}%
\pgfpathcurveto{\pgfqpoint{3.332564in}{2.708845in}}{\pgfqpoint{3.321965in}{2.704455in}}{\pgfqpoint{3.314151in}{2.696641in}}%
\pgfpathcurveto{\pgfqpoint{3.306337in}{2.688828in}}{\pgfqpoint{3.301947in}{2.678229in}}{\pgfqpoint{3.301947in}{2.667179in}}%
\pgfpathcurveto{\pgfqpoint{3.301947in}{2.656129in}}{\pgfqpoint{3.306337in}{2.645530in}}{\pgfqpoint{3.314151in}{2.637716in}}%
\pgfpathcurveto{\pgfqpoint{3.321965in}{2.629902in}}{\pgfqpoint{3.332564in}{2.625512in}}{\pgfqpoint{3.343614in}{2.625512in}}%
\pgfpathclose%
\pgfusepath{stroke,fill}%
\end{pgfscope}%
\begin{pgfscope}%
\pgfpathrectangle{\pgfqpoint{0.526127in}{0.331635in}}{\pgfqpoint{9.300000in}{7.700000in}}%
\pgfusepath{clip}%
\pgfsetbuttcap%
\pgfsetroundjoin%
\definecolor{currentfill}{rgb}{0.815686,0.733333,1.000000}%
\pgfsetfillcolor{currentfill}%
\pgfsetlinewidth{0.481800pt}%
\definecolor{currentstroke}{rgb}{1.000000,1.000000,1.000000}%
\pgfsetstrokecolor{currentstroke}%
\pgfsetdash{}{0pt}%
\pgfpathmoveto{\pgfqpoint{6.735659in}{3.815807in}}%
\pgfpathcurveto{\pgfqpoint{6.746709in}{3.815807in}}{\pgfqpoint{6.757308in}{3.820198in}}{\pgfqpoint{6.765122in}{3.828011in}}%
\pgfpathcurveto{\pgfqpoint{6.772935in}{3.835825in}}{\pgfqpoint{6.777326in}{3.846424in}}{\pgfqpoint{6.777326in}{3.857474in}}%
\pgfpathcurveto{\pgfqpoint{6.777326in}{3.868524in}}{\pgfqpoint{6.772935in}{3.879123in}}{\pgfqpoint{6.765122in}{3.886937in}}%
\pgfpathcurveto{\pgfqpoint{6.757308in}{3.894751in}}{\pgfqpoint{6.746709in}{3.899141in}}{\pgfqpoint{6.735659in}{3.899141in}}%
\pgfpathcurveto{\pgfqpoint{6.724609in}{3.899141in}}{\pgfqpoint{6.714010in}{3.894751in}}{\pgfqpoint{6.706196in}{3.886937in}}%
\pgfpathcurveto{\pgfqpoint{6.698383in}{3.879123in}}{\pgfqpoint{6.693992in}{3.868524in}}{\pgfqpoint{6.693992in}{3.857474in}}%
\pgfpathcurveto{\pgfqpoint{6.693992in}{3.846424in}}{\pgfqpoint{6.698383in}{3.835825in}}{\pgfqpoint{6.706196in}{3.828011in}}%
\pgfpathcurveto{\pgfqpoint{6.714010in}{3.820198in}}{\pgfqpoint{6.724609in}{3.815807in}}{\pgfqpoint{6.735659in}{3.815807in}}%
\pgfpathclose%
\pgfusepath{stroke,fill}%
\end{pgfscope}%
\begin{pgfscope}%
\pgfpathrectangle{\pgfqpoint{0.526127in}{0.331635in}}{\pgfqpoint{9.300000in}{7.700000in}}%
\pgfusepath{clip}%
\pgfsetbuttcap%
\pgfsetroundjoin%
\definecolor{currentfill}{rgb}{0.815686,0.733333,1.000000}%
\pgfsetfillcolor{currentfill}%
\pgfsetlinewidth{0.481800pt}%
\definecolor{currentstroke}{rgb}{1.000000,1.000000,1.000000}%
\pgfsetstrokecolor{currentstroke}%
\pgfsetdash{}{0pt}%
\pgfpathmoveto{\pgfqpoint{2.507937in}{5.074246in}}%
\pgfpathcurveto{\pgfqpoint{2.518987in}{5.074246in}}{\pgfqpoint{2.529586in}{5.078636in}}{\pgfqpoint{2.537400in}{5.086450in}}%
\pgfpathcurveto{\pgfqpoint{2.545214in}{5.094263in}}{\pgfqpoint{2.549604in}{5.104862in}}{\pgfqpoint{2.549604in}{5.115912in}}%
\pgfpathcurveto{\pgfqpoint{2.549604in}{5.126963in}}{\pgfqpoint{2.545214in}{5.137562in}}{\pgfqpoint{2.537400in}{5.145375in}}%
\pgfpathcurveto{\pgfqpoint{2.529586in}{5.153189in}}{\pgfqpoint{2.518987in}{5.157579in}}{\pgfqpoint{2.507937in}{5.157579in}}%
\pgfpathcurveto{\pgfqpoint{2.496887in}{5.157579in}}{\pgfqpoint{2.486288in}{5.153189in}}{\pgfqpoint{2.478475in}{5.145375in}}%
\pgfpathcurveto{\pgfqpoint{2.470661in}{5.137562in}}{\pgfqpoint{2.466271in}{5.126963in}}{\pgfqpoint{2.466271in}{5.115912in}}%
\pgfpathcurveto{\pgfqpoint{2.466271in}{5.104862in}}{\pgfqpoint{2.470661in}{5.094263in}}{\pgfqpoint{2.478475in}{5.086450in}}%
\pgfpathcurveto{\pgfqpoint{2.486288in}{5.078636in}}{\pgfqpoint{2.496887in}{5.074246in}}{\pgfqpoint{2.507937in}{5.074246in}}%
\pgfpathclose%
\pgfusepath{stroke,fill}%
\end{pgfscope}%
\begin{pgfscope}%
\pgfpathrectangle{\pgfqpoint{0.526127in}{0.331635in}}{\pgfqpoint{9.300000in}{7.700000in}}%
\pgfusepath{clip}%
\pgfsetbuttcap%
\pgfsetroundjoin%
\definecolor{currentfill}{rgb}{0.815686,0.733333,1.000000}%
\pgfsetfillcolor{currentfill}%
\pgfsetlinewidth{0.481800pt}%
\definecolor{currentstroke}{rgb}{1.000000,1.000000,1.000000}%
\pgfsetstrokecolor{currentstroke}%
\pgfsetdash{}{0pt}%
\pgfpathmoveto{\pgfqpoint{8.582505in}{3.915961in}}%
\pgfpathcurveto{\pgfqpoint{8.593555in}{3.915961in}}{\pgfqpoint{8.604154in}{3.920351in}}{\pgfqpoint{8.611968in}{3.928165in}}%
\pgfpathcurveto{\pgfqpoint{8.619781in}{3.935978in}}{\pgfqpoint{8.624171in}{3.946577in}}{\pgfqpoint{8.624171in}{3.957628in}}%
\pgfpathcurveto{\pgfqpoint{8.624171in}{3.968678in}}{\pgfqpoint{8.619781in}{3.979277in}}{\pgfqpoint{8.611968in}{3.987090in}}%
\pgfpathcurveto{\pgfqpoint{8.604154in}{3.994904in}}{\pgfqpoint{8.593555in}{3.999294in}}{\pgfqpoint{8.582505in}{3.999294in}}%
\pgfpathcurveto{\pgfqpoint{8.571455in}{3.999294in}}{\pgfqpoint{8.560856in}{3.994904in}}{\pgfqpoint{8.553042in}{3.987090in}}%
\pgfpathcurveto{\pgfqpoint{8.545228in}{3.979277in}}{\pgfqpoint{8.540838in}{3.968678in}}{\pgfqpoint{8.540838in}{3.957628in}}%
\pgfpathcurveto{\pgfqpoint{8.540838in}{3.946577in}}{\pgfqpoint{8.545228in}{3.935978in}}{\pgfqpoint{8.553042in}{3.928165in}}%
\pgfpathcurveto{\pgfqpoint{8.560856in}{3.920351in}}{\pgfqpoint{8.571455in}{3.915961in}}{\pgfqpoint{8.582505in}{3.915961in}}%
\pgfpathclose%
\pgfusepath{stroke,fill}%
\end{pgfscope}%
\begin{pgfscope}%
\pgfpathrectangle{\pgfqpoint{0.526127in}{0.331635in}}{\pgfqpoint{9.300000in}{7.700000in}}%
\pgfusepath{clip}%
\pgfsetbuttcap%
\pgfsetroundjoin%
\definecolor{currentfill}{rgb}{0.815686,0.733333,1.000000}%
\pgfsetfillcolor{currentfill}%
\pgfsetlinewidth{0.481800pt}%
\definecolor{currentstroke}{rgb}{1.000000,1.000000,1.000000}%
\pgfsetstrokecolor{currentstroke}%
\pgfsetdash{}{0pt}%
\pgfpathmoveto{\pgfqpoint{5.435254in}{1.741114in}}%
\pgfpathcurveto{\pgfqpoint{5.446304in}{1.741114in}}{\pgfqpoint{5.456903in}{1.745505in}}{\pgfqpoint{5.464717in}{1.753318in}}%
\pgfpathcurveto{\pgfqpoint{5.472530in}{1.761132in}}{\pgfqpoint{5.476921in}{1.771731in}}{\pgfqpoint{5.476921in}{1.782781in}}%
\pgfpathcurveto{\pgfqpoint{5.476921in}{1.793831in}}{\pgfqpoint{5.472530in}{1.804430in}}{\pgfqpoint{5.464717in}{1.812244in}}%
\pgfpathcurveto{\pgfqpoint{5.456903in}{1.820057in}}{\pgfqpoint{5.446304in}{1.824448in}}{\pgfqpoint{5.435254in}{1.824448in}}%
\pgfpathcurveto{\pgfqpoint{5.424204in}{1.824448in}}{\pgfqpoint{5.413605in}{1.820057in}}{\pgfqpoint{5.405791in}{1.812244in}}%
\pgfpathcurveto{\pgfqpoint{5.397978in}{1.804430in}}{\pgfqpoint{5.393587in}{1.793831in}}{\pgfqpoint{5.393587in}{1.782781in}}%
\pgfpathcurveto{\pgfqpoint{5.393587in}{1.771731in}}{\pgfqpoint{5.397978in}{1.761132in}}{\pgfqpoint{5.405791in}{1.753318in}}%
\pgfpathcurveto{\pgfqpoint{5.413605in}{1.745505in}}{\pgfqpoint{5.424204in}{1.741114in}}{\pgfqpoint{5.435254in}{1.741114in}}%
\pgfpathclose%
\pgfusepath{stroke,fill}%
\end{pgfscope}%
\begin{pgfscope}%
\pgfpathrectangle{\pgfqpoint{0.526127in}{0.331635in}}{\pgfqpoint{9.300000in}{7.700000in}}%
\pgfusepath{clip}%
\pgfsetbuttcap%
\pgfsetroundjoin%
\definecolor{currentfill}{rgb}{0.815686,0.733333,1.000000}%
\pgfsetfillcolor{currentfill}%
\pgfsetlinewidth{0.481800pt}%
\definecolor{currentstroke}{rgb}{1.000000,1.000000,1.000000}%
\pgfsetstrokecolor{currentstroke}%
\pgfsetdash{}{0pt}%
\pgfpathmoveto{\pgfqpoint{5.661250in}{4.627366in}}%
\pgfpathcurveto{\pgfqpoint{5.672300in}{4.627366in}}{\pgfqpoint{5.682899in}{4.631756in}}{\pgfqpoint{5.690713in}{4.639570in}}%
\pgfpathcurveto{\pgfqpoint{5.698527in}{4.647383in}}{\pgfqpoint{5.702917in}{4.657982in}}{\pgfqpoint{5.702917in}{4.669033in}}%
\pgfpathcurveto{\pgfqpoint{5.702917in}{4.680083in}}{\pgfqpoint{5.698527in}{4.690682in}}{\pgfqpoint{5.690713in}{4.698495in}}%
\pgfpathcurveto{\pgfqpoint{5.682899in}{4.706309in}}{\pgfqpoint{5.672300in}{4.710699in}}{\pgfqpoint{5.661250in}{4.710699in}}%
\pgfpathcurveto{\pgfqpoint{5.650200in}{4.710699in}}{\pgfqpoint{5.639601in}{4.706309in}}{\pgfqpoint{5.631787in}{4.698495in}}%
\pgfpathcurveto{\pgfqpoint{5.623974in}{4.690682in}}{\pgfqpoint{5.619583in}{4.680083in}}{\pgfqpoint{5.619583in}{4.669033in}}%
\pgfpathcurveto{\pgfqpoint{5.619583in}{4.657982in}}{\pgfqpoint{5.623974in}{4.647383in}}{\pgfqpoint{5.631787in}{4.639570in}}%
\pgfpathcurveto{\pgfqpoint{5.639601in}{4.631756in}}{\pgfqpoint{5.650200in}{4.627366in}}{\pgfqpoint{5.661250in}{4.627366in}}%
\pgfpathclose%
\pgfusepath{stroke,fill}%
\end{pgfscope}%
\begin{pgfscope}%
\pgfpathrectangle{\pgfqpoint{0.526127in}{0.331635in}}{\pgfqpoint{9.300000in}{7.700000in}}%
\pgfusepath{clip}%
\pgfsetbuttcap%
\pgfsetroundjoin%
\definecolor{currentfill}{rgb}{0.815686,0.733333,1.000000}%
\pgfsetfillcolor{currentfill}%
\pgfsetlinewidth{0.481800pt}%
\definecolor{currentstroke}{rgb}{1.000000,1.000000,1.000000}%
\pgfsetstrokecolor{currentstroke}%
\pgfsetdash{}{0pt}%
\pgfpathmoveto{\pgfqpoint{2.380624in}{2.662548in}}%
\pgfpathcurveto{\pgfqpoint{2.391674in}{2.662548in}}{\pgfqpoint{2.402273in}{2.666939in}}{\pgfqpoint{2.410087in}{2.674752in}}%
\pgfpathcurveto{\pgfqpoint{2.417900in}{2.682566in}}{\pgfqpoint{2.422290in}{2.693165in}}{\pgfqpoint{2.422290in}{2.704215in}}%
\pgfpathcurveto{\pgfqpoint{2.422290in}{2.715265in}}{\pgfqpoint{2.417900in}{2.725864in}}{\pgfqpoint{2.410087in}{2.733678in}}%
\pgfpathcurveto{\pgfqpoint{2.402273in}{2.741491in}}{\pgfqpoint{2.391674in}{2.745882in}}{\pgfqpoint{2.380624in}{2.745882in}}%
\pgfpathcurveto{\pgfqpoint{2.369574in}{2.745882in}}{\pgfqpoint{2.358975in}{2.741491in}}{\pgfqpoint{2.351161in}{2.733678in}}%
\pgfpathcurveto{\pgfqpoint{2.343347in}{2.725864in}}{\pgfqpoint{2.338957in}{2.715265in}}{\pgfqpoint{2.338957in}{2.704215in}}%
\pgfpathcurveto{\pgfqpoint{2.338957in}{2.693165in}}{\pgfqpoint{2.343347in}{2.682566in}}{\pgfqpoint{2.351161in}{2.674752in}}%
\pgfpathcurveto{\pgfqpoint{2.358975in}{2.666939in}}{\pgfqpoint{2.369574in}{2.662548in}}{\pgfqpoint{2.380624in}{2.662548in}}%
\pgfpathclose%
\pgfusepath{stroke,fill}%
\end{pgfscope}%
\begin{pgfscope}%
\pgfpathrectangle{\pgfqpoint{0.526127in}{0.331635in}}{\pgfqpoint{9.300000in}{7.700000in}}%
\pgfusepath{clip}%
\pgfsetbuttcap%
\pgfsetroundjoin%
\definecolor{currentfill}{rgb}{0.815686,0.733333,1.000000}%
\pgfsetfillcolor{currentfill}%
\pgfsetlinewidth{0.481800pt}%
\definecolor{currentstroke}{rgb}{1.000000,1.000000,1.000000}%
\pgfsetstrokecolor{currentstroke}%
\pgfsetdash{}{0pt}%
\pgfpathmoveto{\pgfqpoint{3.490954in}{3.314820in}}%
\pgfpathcurveto{\pgfqpoint{3.502004in}{3.314820in}}{\pgfqpoint{3.512603in}{3.319210in}}{\pgfqpoint{3.520417in}{3.327024in}}%
\pgfpathcurveto{\pgfqpoint{3.528231in}{3.334838in}}{\pgfqpoint{3.532621in}{3.345437in}}{\pgfqpoint{3.532621in}{3.356487in}}%
\pgfpathcurveto{\pgfqpoint{3.532621in}{3.367537in}}{\pgfqpoint{3.528231in}{3.378136in}}{\pgfqpoint{3.520417in}{3.385950in}}%
\pgfpathcurveto{\pgfqpoint{3.512603in}{3.393763in}}{\pgfqpoint{3.502004in}{3.398154in}}{\pgfqpoint{3.490954in}{3.398154in}}%
\pgfpathcurveto{\pgfqpoint{3.479904in}{3.398154in}}{\pgfqpoint{3.469305in}{3.393763in}}{\pgfqpoint{3.461491in}{3.385950in}}%
\pgfpathcurveto{\pgfqpoint{3.453678in}{3.378136in}}{\pgfqpoint{3.449287in}{3.367537in}}{\pgfqpoint{3.449287in}{3.356487in}}%
\pgfpathcurveto{\pgfqpoint{3.449287in}{3.345437in}}{\pgfqpoint{3.453678in}{3.334838in}}{\pgfqpoint{3.461491in}{3.327024in}}%
\pgfpathcurveto{\pgfqpoint{3.469305in}{3.319210in}}{\pgfqpoint{3.479904in}{3.314820in}}{\pgfqpoint{3.490954in}{3.314820in}}%
\pgfpathclose%
\pgfusepath{stroke,fill}%
\end{pgfscope}%
\begin{pgfscope}%
\pgfpathrectangle{\pgfqpoint{0.526127in}{0.331635in}}{\pgfqpoint{9.300000in}{7.700000in}}%
\pgfusepath{clip}%
\pgfsetbuttcap%
\pgfsetroundjoin%
\definecolor{currentfill}{rgb}{0.815686,0.733333,1.000000}%
\pgfsetfillcolor{currentfill}%
\pgfsetlinewidth{0.481800pt}%
\definecolor{currentstroke}{rgb}{1.000000,1.000000,1.000000}%
\pgfsetstrokecolor{currentstroke}%
\pgfsetdash{}{0pt}%
\pgfpathmoveto{\pgfqpoint{4.801136in}{0.732417in}}%
\pgfpathcurveto{\pgfqpoint{4.812186in}{0.732417in}}{\pgfqpoint{4.822785in}{0.736808in}}{\pgfqpoint{4.830599in}{0.744621in}}%
\pgfpathcurveto{\pgfqpoint{4.838412in}{0.752435in}}{\pgfqpoint{4.842803in}{0.763034in}}{\pgfqpoint{4.842803in}{0.774084in}}%
\pgfpathcurveto{\pgfqpoint{4.842803in}{0.785134in}}{\pgfqpoint{4.838412in}{0.795733in}}{\pgfqpoint{4.830599in}{0.803547in}}%
\pgfpathcurveto{\pgfqpoint{4.822785in}{0.811360in}}{\pgfqpoint{4.812186in}{0.815751in}}{\pgfqpoint{4.801136in}{0.815751in}}%
\pgfpathcurveto{\pgfqpoint{4.790086in}{0.815751in}}{\pgfqpoint{4.779487in}{0.811360in}}{\pgfqpoint{4.771673in}{0.803547in}}%
\pgfpathcurveto{\pgfqpoint{4.763860in}{0.795733in}}{\pgfqpoint{4.759469in}{0.785134in}}{\pgfqpoint{4.759469in}{0.774084in}}%
\pgfpathcurveto{\pgfqpoint{4.759469in}{0.763034in}}{\pgfqpoint{4.763860in}{0.752435in}}{\pgfqpoint{4.771673in}{0.744621in}}%
\pgfpathcurveto{\pgfqpoint{4.779487in}{0.736808in}}{\pgfqpoint{4.790086in}{0.732417in}}{\pgfqpoint{4.801136in}{0.732417in}}%
\pgfpathclose%
\pgfusepath{stroke,fill}%
\end{pgfscope}%
\begin{pgfscope}%
\pgfpathrectangle{\pgfqpoint{0.526127in}{0.331635in}}{\pgfqpoint{9.300000in}{7.700000in}}%
\pgfusepath{clip}%
\pgfsetbuttcap%
\pgfsetroundjoin%
\definecolor{currentfill}{rgb}{0.815686,0.733333,1.000000}%
\pgfsetfillcolor{currentfill}%
\pgfsetlinewidth{0.481800pt}%
\definecolor{currentstroke}{rgb}{1.000000,1.000000,1.000000}%
\pgfsetstrokecolor{currentstroke}%
\pgfsetdash{}{0pt}%
\pgfpathmoveto{\pgfqpoint{7.283490in}{1.678495in}}%
\pgfpathcurveto{\pgfqpoint{7.294540in}{1.678495in}}{\pgfqpoint{7.305139in}{1.682885in}}{\pgfqpoint{7.312953in}{1.690699in}}%
\pgfpathcurveto{\pgfqpoint{7.320766in}{1.698512in}}{\pgfqpoint{7.325157in}{1.709111in}}{\pgfqpoint{7.325157in}{1.720161in}}%
\pgfpathcurveto{\pgfqpoint{7.325157in}{1.731212in}}{\pgfqpoint{7.320766in}{1.741811in}}{\pgfqpoint{7.312953in}{1.749624in}}%
\pgfpathcurveto{\pgfqpoint{7.305139in}{1.757438in}}{\pgfqpoint{7.294540in}{1.761828in}}{\pgfqpoint{7.283490in}{1.761828in}}%
\pgfpathcurveto{\pgfqpoint{7.272440in}{1.761828in}}{\pgfqpoint{7.261841in}{1.757438in}}{\pgfqpoint{7.254027in}{1.749624in}}%
\pgfpathcurveto{\pgfqpoint{7.246214in}{1.741811in}}{\pgfqpoint{7.241823in}{1.731212in}}{\pgfqpoint{7.241823in}{1.720161in}}%
\pgfpathcurveto{\pgfqpoint{7.241823in}{1.709111in}}{\pgfqpoint{7.246214in}{1.698512in}}{\pgfqpoint{7.254027in}{1.690699in}}%
\pgfpathcurveto{\pgfqpoint{7.261841in}{1.682885in}}{\pgfqpoint{7.272440in}{1.678495in}}{\pgfqpoint{7.283490in}{1.678495in}}%
\pgfpathclose%
\pgfusepath{stroke,fill}%
\end{pgfscope}%
\begin{pgfscope}%
\pgfpathrectangle{\pgfqpoint{0.526127in}{0.331635in}}{\pgfqpoint{9.300000in}{7.700000in}}%
\pgfusepath{clip}%
\pgfsetbuttcap%
\pgfsetroundjoin%
\definecolor{currentfill}{rgb}{0.815686,0.733333,1.000000}%
\pgfsetfillcolor{currentfill}%
\pgfsetlinewidth{0.481800pt}%
\definecolor{currentstroke}{rgb}{1.000000,1.000000,1.000000}%
\pgfsetstrokecolor{currentstroke}%
\pgfsetdash{}{0pt}%
\pgfpathmoveto{\pgfqpoint{4.932772in}{1.200226in}}%
\pgfpathcurveto{\pgfqpoint{4.943822in}{1.200226in}}{\pgfqpoint{4.954421in}{1.204617in}}{\pgfqpoint{4.962235in}{1.212430in}}%
\pgfpathcurveto{\pgfqpoint{4.970048in}{1.220244in}}{\pgfqpoint{4.974438in}{1.230843in}}{\pgfqpoint{4.974438in}{1.241893in}}%
\pgfpathcurveto{\pgfqpoint{4.974438in}{1.252943in}}{\pgfqpoint{4.970048in}{1.263542in}}{\pgfqpoint{4.962235in}{1.271356in}}%
\pgfpathcurveto{\pgfqpoint{4.954421in}{1.279169in}}{\pgfqpoint{4.943822in}{1.283560in}}{\pgfqpoint{4.932772in}{1.283560in}}%
\pgfpathcurveto{\pgfqpoint{4.921722in}{1.283560in}}{\pgfqpoint{4.911123in}{1.279169in}}{\pgfqpoint{4.903309in}{1.271356in}}%
\pgfpathcurveto{\pgfqpoint{4.895495in}{1.263542in}}{\pgfqpoint{4.891105in}{1.252943in}}{\pgfqpoint{4.891105in}{1.241893in}}%
\pgfpathcurveto{\pgfqpoint{4.891105in}{1.230843in}}{\pgfqpoint{4.895495in}{1.220244in}}{\pgfqpoint{4.903309in}{1.212430in}}%
\pgfpathcurveto{\pgfqpoint{4.911123in}{1.204617in}}{\pgfqpoint{4.921722in}{1.200226in}}{\pgfqpoint{4.932772in}{1.200226in}}%
\pgfpathclose%
\pgfusepath{stroke,fill}%
\end{pgfscope}%
\begin{pgfscope}%
\pgfpathrectangle{\pgfqpoint{0.526127in}{0.331635in}}{\pgfqpoint{9.300000in}{7.700000in}}%
\pgfusepath{clip}%
\pgfsetbuttcap%
\pgfsetroundjoin%
\definecolor{currentfill}{rgb}{0.815686,0.733333,1.000000}%
\pgfsetfillcolor{currentfill}%
\pgfsetlinewidth{0.481800pt}%
\definecolor{currentstroke}{rgb}{1.000000,1.000000,1.000000}%
\pgfsetstrokecolor{currentstroke}%
\pgfsetdash{}{0pt}%
\pgfpathmoveto{\pgfqpoint{7.494242in}{1.650056in}}%
\pgfpathcurveto{\pgfqpoint{7.505292in}{1.650056in}}{\pgfqpoint{7.515891in}{1.654446in}}{\pgfqpoint{7.523705in}{1.662260in}}%
\pgfpathcurveto{\pgfqpoint{7.531519in}{1.670074in}}{\pgfqpoint{7.535909in}{1.680673in}}{\pgfqpoint{7.535909in}{1.691723in}}%
\pgfpathcurveto{\pgfqpoint{7.535909in}{1.702773in}}{\pgfqpoint{7.531519in}{1.713372in}}{\pgfqpoint{7.523705in}{1.721186in}}%
\pgfpathcurveto{\pgfqpoint{7.515891in}{1.728999in}}{\pgfqpoint{7.505292in}{1.733390in}}{\pgfqpoint{7.494242in}{1.733390in}}%
\pgfpathcurveto{\pgfqpoint{7.483192in}{1.733390in}}{\pgfqpoint{7.472593in}{1.728999in}}{\pgfqpoint{7.464779in}{1.721186in}}%
\pgfpathcurveto{\pgfqpoint{7.456966in}{1.713372in}}{\pgfqpoint{7.452576in}{1.702773in}}{\pgfqpoint{7.452576in}{1.691723in}}%
\pgfpathcurveto{\pgfqpoint{7.452576in}{1.680673in}}{\pgfqpoint{7.456966in}{1.670074in}}{\pgfqpoint{7.464779in}{1.662260in}}%
\pgfpathcurveto{\pgfqpoint{7.472593in}{1.654446in}}{\pgfqpoint{7.483192in}{1.650056in}}{\pgfqpoint{7.494242in}{1.650056in}}%
\pgfpathclose%
\pgfusepath{stroke,fill}%
\end{pgfscope}%
\begin{pgfscope}%
\pgfpathrectangle{\pgfqpoint{0.526127in}{0.331635in}}{\pgfqpoint{9.300000in}{7.700000in}}%
\pgfusepath{clip}%
\pgfsetbuttcap%
\pgfsetroundjoin%
\definecolor{currentfill}{rgb}{0.815686,0.733333,1.000000}%
\pgfsetfillcolor{currentfill}%
\pgfsetlinewidth{0.481800pt}%
\definecolor{currentstroke}{rgb}{1.000000,1.000000,1.000000}%
\pgfsetstrokecolor{currentstroke}%
\pgfsetdash{}{0pt}%
\pgfpathmoveto{\pgfqpoint{8.328870in}{2.285356in}}%
\pgfpathcurveto{\pgfqpoint{8.339920in}{2.285356in}}{\pgfqpoint{8.350519in}{2.289746in}}{\pgfqpoint{8.358333in}{2.297560in}}%
\pgfpathcurveto{\pgfqpoint{8.366146in}{2.305374in}}{\pgfqpoint{8.370537in}{2.315973in}}{\pgfqpoint{8.370537in}{2.327023in}}%
\pgfpathcurveto{\pgfqpoint{8.370537in}{2.338073in}}{\pgfqpoint{8.366146in}{2.348672in}}{\pgfqpoint{8.358333in}{2.356486in}}%
\pgfpathcurveto{\pgfqpoint{8.350519in}{2.364299in}}{\pgfqpoint{8.339920in}{2.368689in}}{\pgfqpoint{8.328870in}{2.368689in}}%
\pgfpathcurveto{\pgfqpoint{8.317820in}{2.368689in}}{\pgfqpoint{8.307221in}{2.364299in}}{\pgfqpoint{8.299407in}{2.356486in}}%
\pgfpathcurveto{\pgfqpoint{8.291593in}{2.348672in}}{\pgfqpoint{8.287203in}{2.338073in}}{\pgfqpoint{8.287203in}{2.327023in}}%
\pgfpathcurveto{\pgfqpoint{8.287203in}{2.315973in}}{\pgfqpoint{8.291593in}{2.305374in}}{\pgfqpoint{8.299407in}{2.297560in}}%
\pgfpathcurveto{\pgfqpoint{8.307221in}{2.289746in}}{\pgfqpoint{8.317820in}{2.285356in}}{\pgfqpoint{8.328870in}{2.285356in}}%
\pgfpathclose%
\pgfusepath{stroke,fill}%
\end{pgfscope}%
\begin{pgfscope}%
\pgfpathrectangle{\pgfqpoint{0.526127in}{0.331635in}}{\pgfqpoint{9.300000in}{7.700000in}}%
\pgfusepath{clip}%
\pgfsetbuttcap%
\pgfsetroundjoin%
\definecolor{currentfill}{rgb}{0.815686,0.733333,1.000000}%
\pgfsetfillcolor{currentfill}%
\pgfsetlinewidth{0.481800pt}%
\definecolor{currentstroke}{rgb}{1.000000,1.000000,1.000000}%
\pgfsetstrokecolor{currentstroke}%
\pgfsetdash{}{0pt}%
\pgfpathmoveto{\pgfqpoint{2.658761in}{4.375525in}}%
\pgfpathcurveto{\pgfqpoint{2.669811in}{4.375525in}}{\pgfqpoint{2.680410in}{4.379916in}}{\pgfqpoint{2.688224in}{4.387729in}}%
\pgfpathcurveto{\pgfqpoint{2.696038in}{4.395543in}}{\pgfqpoint{2.700428in}{4.406142in}}{\pgfqpoint{2.700428in}{4.417192in}}%
\pgfpathcurveto{\pgfqpoint{2.700428in}{4.428242in}}{\pgfqpoint{2.696038in}{4.438841in}}{\pgfqpoint{2.688224in}{4.446655in}}%
\pgfpathcurveto{\pgfqpoint{2.680410in}{4.454468in}}{\pgfqpoint{2.669811in}{4.458859in}}{\pgfqpoint{2.658761in}{4.458859in}}%
\pgfpathcurveto{\pgfqpoint{2.647711in}{4.458859in}}{\pgfqpoint{2.637112in}{4.454468in}}{\pgfqpoint{2.629298in}{4.446655in}}%
\pgfpathcurveto{\pgfqpoint{2.621485in}{4.438841in}}{\pgfqpoint{2.617095in}{4.428242in}}{\pgfqpoint{2.617095in}{4.417192in}}%
\pgfpathcurveto{\pgfqpoint{2.617095in}{4.406142in}}{\pgfqpoint{2.621485in}{4.395543in}}{\pgfqpoint{2.629298in}{4.387729in}}%
\pgfpathcurveto{\pgfqpoint{2.637112in}{4.379916in}}{\pgfqpoint{2.647711in}{4.375525in}}{\pgfqpoint{2.658761in}{4.375525in}}%
\pgfpathclose%
\pgfusepath{stroke,fill}%
\end{pgfscope}%
\begin{pgfscope}%
\pgfpathrectangle{\pgfqpoint{0.526127in}{0.331635in}}{\pgfqpoint{9.300000in}{7.700000in}}%
\pgfusepath{clip}%
\pgfsetbuttcap%
\pgfsetroundjoin%
\definecolor{currentfill}{rgb}{0.870588,0.733333,0.607843}%
\pgfsetfillcolor{currentfill}%
\pgfsetlinewidth{0.481800pt}%
\definecolor{currentstroke}{rgb}{1.000000,1.000000,1.000000}%
\pgfsetstrokecolor{currentstroke}%
\pgfsetdash{}{0pt}%
\pgfpathmoveto{\pgfqpoint{3.293241in}{4.272945in}}%
\pgfpathcurveto{\pgfqpoint{3.304291in}{4.272945in}}{\pgfqpoint{3.314890in}{4.277336in}}{\pgfqpoint{3.322704in}{4.285149in}}%
\pgfpathcurveto{\pgfqpoint{3.330518in}{4.292963in}}{\pgfqpoint{3.334908in}{4.303562in}}{\pgfqpoint{3.334908in}{4.314612in}}%
\pgfpathcurveto{\pgfqpoint{3.334908in}{4.325662in}}{\pgfqpoint{3.330518in}{4.336261in}}{\pgfqpoint{3.322704in}{4.344075in}}%
\pgfpathcurveto{\pgfqpoint{3.314890in}{4.351889in}}{\pgfqpoint{3.304291in}{4.356279in}}{\pgfqpoint{3.293241in}{4.356279in}}%
\pgfpathcurveto{\pgfqpoint{3.282191in}{4.356279in}}{\pgfqpoint{3.271592in}{4.351889in}}{\pgfqpoint{3.263778in}{4.344075in}}%
\pgfpathcurveto{\pgfqpoint{3.255965in}{4.336261in}}{\pgfqpoint{3.251575in}{4.325662in}}{\pgfqpoint{3.251575in}{4.314612in}}%
\pgfpathcurveto{\pgfqpoint{3.251575in}{4.303562in}}{\pgfqpoint{3.255965in}{4.292963in}}{\pgfqpoint{3.263778in}{4.285149in}}%
\pgfpathcurveto{\pgfqpoint{3.271592in}{4.277336in}}{\pgfqpoint{3.282191in}{4.272945in}}{\pgfqpoint{3.293241in}{4.272945in}}%
\pgfpathclose%
\pgfusepath{stroke,fill}%
\end{pgfscope}%
\begin{pgfscope}%
\pgfpathrectangle{\pgfqpoint{0.526127in}{0.331635in}}{\pgfqpoint{9.300000in}{7.700000in}}%
\pgfusepath{clip}%
\pgfsetbuttcap%
\pgfsetroundjoin%
\definecolor{currentfill}{rgb}{0.870588,0.733333,0.607843}%
\pgfsetfillcolor{currentfill}%
\pgfsetlinewidth{0.481800pt}%
\definecolor{currentstroke}{rgb}{1.000000,1.000000,1.000000}%
\pgfsetstrokecolor{currentstroke}%
\pgfsetdash{}{0pt}%
\pgfpathmoveto{\pgfqpoint{4.449865in}{2.592672in}}%
\pgfpathcurveto{\pgfqpoint{4.460915in}{2.592672in}}{\pgfqpoint{4.471514in}{2.597063in}}{\pgfqpoint{4.479327in}{2.604876in}}%
\pgfpathcurveto{\pgfqpoint{4.487141in}{2.612690in}}{\pgfqpoint{4.491531in}{2.623289in}}{\pgfqpoint{4.491531in}{2.634339in}}%
\pgfpathcurveto{\pgfqpoint{4.491531in}{2.645389in}}{\pgfqpoint{4.487141in}{2.655988in}}{\pgfqpoint{4.479327in}{2.663802in}}%
\pgfpathcurveto{\pgfqpoint{4.471514in}{2.671615in}}{\pgfqpoint{4.460915in}{2.676006in}}{\pgfqpoint{4.449865in}{2.676006in}}%
\pgfpathcurveto{\pgfqpoint{4.438814in}{2.676006in}}{\pgfqpoint{4.428215in}{2.671615in}}{\pgfqpoint{4.420402in}{2.663802in}}%
\pgfpathcurveto{\pgfqpoint{4.412588in}{2.655988in}}{\pgfqpoint{4.408198in}{2.645389in}}{\pgfqpoint{4.408198in}{2.634339in}}%
\pgfpathcurveto{\pgfqpoint{4.408198in}{2.623289in}}{\pgfqpoint{4.412588in}{2.612690in}}{\pgfqpoint{4.420402in}{2.604876in}}%
\pgfpathcurveto{\pgfqpoint{4.428215in}{2.597063in}}{\pgfqpoint{4.438814in}{2.592672in}}{\pgfqpoint{4.449865in}{2.592672in}}%
\pgfpathclose%
\pgfusepath{stroke,fill}%
\end{pgfscope}%
\begin{pgfscope}%
\pgfpathrectangle{\pgfqpoint{0.526127in}{0.331635in}}{\pgfqpoint{9.300000in}{7.700000in}}%
\pgfusepath{clip}%
\pgfsetbuttcap%
\pgfsetroundjoin%
\definecolor{currentfill}{rgb}{0.870588,0.733333,0.607843}%
\pgfsetfillcolor{currentfill}%
\pgfsetlinewidth{0.481800pt}%
\definecolor{currentstroke}{rgb}{1.000000,1.000000,1.000000}%
\pgfsetstrokecolor{currentstroke}%
\pgfsetdash{}{0pt}%
\pgfpathmoveto{\pgfqpoint{5.505397in}{0.933762in}}%
\pgfpathcurveto{\pgfqpoint{5.516447in}{0.933762in}}{\pgfqpoint{5.527046in}{0.938152in}}{\pgfqpoint{5.534860in}{0.945966in}}%
\pgfpathcurveto{\pgfqpoint{5.542673in}{0.953780in}}{\pgfqpoint{5.547064in}{0.964379in}}{\pgfqpoint{5.547064in}{0.975429in}}%
\pgfpathcurveto{\pgfqpoint{5.547064in}{0.986479in}}{\pgfqpoint{5.542673in}{0.997078in}}{\pgfqpoint{5.534860in}{1.004892in}}%
\pgfpathcurveto{\pgfqpoint{5.527046in}{1.012705in}}{\pgfqpoint{5.516447in}{1.017096in}}{\pgfqpoint{5.505397in}{1.017096in}}%
\pgfpathcurveto{\pgfqpoint{5.494347in}{1.017096in}}{\pgfqpoint{5.483748in}{1.012705in}}{\pgfqpoint{5.475934in}{1.004892in}}%
\pgfpathcurveto{\pgfqpoint{5.468121in}{0.997078in}}{\pgfqpoint{5.463730in}{0.986479in}}{\pgfqpoint{5.463730in}{0.975429in}}%
\pgfpathcurveto{\pgfqpoint{5.463730in}{0.964379in}}{\pgfqpoint{5.468121in}{0.953780in}}{\pgfqpoint{5.475934in}{0.945966in}}%
\pgfpathcurveto{\pgfqpoint{5.483748in}{0.938152in}}{\pgfqpoint{5.494347in}{0.933762in}}{\pgfqpoint{5.505397in}{0.933762in}}%
\pgfpathclose%
\pgfusepath{stroke,fill}%
\end{pgfscope}%
\begin{pgfscope}%
\pgfpathrectangle{\pgfqpoint{0.526127in}{0.331635in}}{\pgfqpoint{9.300000in}{7.700000in}}%
\pgfusepath{clip}%
\pgfsetbuttcap%
\pgfsetroundjoin%
\definecolor{currentfill}{rgb}{0.870588,0.733333,0.607843}%
\pgfsetfillcolor{currentfill}%
\pgfsetlinewidth{0.481800pt}%
\definecolor{currentstroke}{rgb}{1.000000,1.000000,1.000000}%
\pgfsetstrokecolor{currentstroke}%
\pgfsetdash{}{0pt}%
\pgfpathmoveto{\pgfqpoint{3.975782in}{2.759819in}}%
\pgfpathcurveto{\pgfqpoint{3.986832in}{2.759819in}}{\pgfqpoint{3.997431in}{2.764209in}}{\pgfqpoint{4.005245in}{2.772023in}}%
\pgfpathcurveto{\pgfqpoint{4.013059in}{2.779837in}}{\pgfqpoint{4.017449in}{2.790436in}}{\pgfqpoint{4.017449in}{2.801486in}}%
\pgfpathcurveto{\pgfqpoint{4.017449in}{2.812536in}}{\pgfqpoint{4.013059in}{2.823135in}}{\pgfqpoint{4.005245in}{2.830949in}}%
\pgfpathcurveto{\pgfqpoint{3.997431in}{2.838762in}}{\pgfqpoint{3.986832in}{2.843153in}}{\pgfqpoint{3.975782in}{2.843153in}}%
\pgfpathcurveto{\pgfqpoint{3.964732in}{2.843153in}}{\pgfqpoint{3.954133in}{2.838762in}}{\pgfqpoint{3.946319in}{2.830949in}}%
\pgfpathcurveto{\pgfqpoint{3.938506in}{2.823135in}}{\pgfqpoint{3.934116in}{2.812536in}}{\pgfqpoint{3.934116in}{2.801486in}}%
\pgfpathcurveto{\pgfqpoint{3.934116in}{2.790436in}}{\pgfqpoint{3.938506in}{2.779837in}}{\pgfqpoint{3.946319in}{2.772023in}}%
\pgfpathcurveto{\pgfqpoint{3.954133in}{2.764209in}}{\pgfqpoint{3.964732in}{2.759819in}}{\pgfqpoint{3.975782in}{2.759819in}}%
\pgfpathclose%
\pgfusepath{stroke,fill}%
\end{pgfscope}%
\begin{pgfscope}%
\pgfpathrectangle{\pgfqpoint{0.526127in}{0.331635in}}{\pgfqpoint{9.300000in}{7.700000in}}%
\pgfusepath{clip}%
\pgfsetbuttcap%
\pgfsetroundjoin%
\definecolor{currentfill}{rgb}{0.870588,0.733333,0.607843}%
\pgfsetfillcolor{currentfill}%
\pgfsetlinewidth{0.481800pt}%
\definecolor{currentstroke}{rgb}{1.000000,1.000000,1.000000}%
\pgfsetstrokecolor{currentstroke}%
\pgfsetdash{}{0pt}%
\pgfpathmoveto{\pgfqpoint{5.150340in}{3.107207in}}%
\pgfpathcurveto{\pgfqpoint{5.161390in}{3.107207in}}{\pgfqpoint{5.171989in}{3.111597in}}{\pgfqpoint{5.179802in}{3.119411in}}%
\pgfpathcurveto{\pgfqpoint{5.187616in}{3.127224in}}{\pgfqpoint{5.192006in}{3.137823in}}{\pgfqpoint{5.192006in}{3.148874in}}%
\pgfpathcurveto{\pgfqpoint{5.192006in}{3.159924in}}{\pgfqpoint{5.187616in}{3.170523in}}{\pgfqpoint{5.179802in}{3.178336in}}%
\pgfpathcurveto{\pgfqpoint{5.171989in}{3.186150in}}{\pgfqpoint{5.161390in}{3.190540in}}{\pgfqpoint{5.150340in}{3.190540in}}%
\pgfpathcurveto{\pgfqpoint{5.139289in}{3.190540in}}{\pgfqpoint{5.128690in}{3.186150in}}{\pgfqpoint{5.120877in}{3.178336in}}%
\pgfpathcurveto{\pgfqpoint{5.113063in}{3.170523in}}{\pgfqpoint{5.108673in}{3.159924in}}{\pgfqpoint{5.108673in}{3.148874in}}%
\pgfpathcurveto{\pgfqpoint{5.108673in}{3.137823in}}{\pgfqpoint{5.113063in}{3.127224in}}{\pgfqpoint{5.120877in}{3.119411in}}%
\pgfpathcurveto{\pgfqpoint{5.128690in}{3.111597in}}{\pgfqpoint{5.139289in}{3.107207in}}{\pgfqpoint{5.150340in}{3.107207in}}%
\pgfpathclose%
\pgfusepath{stroke,fill}%
\end{pgfscope}%
\begin{pgfscope}%
\pgfpathrectangle{\pgfqpoint{0.526127in}{0.331635in}}{\pgfqpoint{9.300000in}{7.700000in}}%
\pgfusepath{clip}%
\pgfsetbuttcap%
\pgfsetroundjoin%
\definecolor{currentfill}{rgb}{0.870588,0.733333,0.607843}%
\pgfsetfillcolor{currentfill}%
\pgfsetlinewidth{0.481800pt}%
\definecolor{currentstroke}{rgb}{1.000000,1.000000,1.000000}%
\pgfsetstrokecolor{currentstroke}%
\pgfsetdash{}{0pt}%
\pgfpathmoveto{\pgfqpoint{6.303025in}{4.215839in}}%
\pgfpathcurveto{\pgfqpoint{6.314076in}{4.215839in}}{\pgfqpoint{6.324675in}{4.220230in}}{\pgfqpoint{6.332488in}{4.228043in}}%
\pgfpathcurveto{\pgfqpoint{6.340302in}{4.235857in}}{\pgfqpoint{6.344692in}{4.246456in}}{\pgfqpoint{6.344692in}{4.257506in}}%
\pgfpathcurveto{\pgfqpoint{6.344692in}{4.268556in}}{\pgfqpoint{6.340302in}{4.279155in}}{\pgfqpoint{6.332488in}{4.286969in}}%
\pgfpathcurveto{\pgfqpoint{6.324675in}{4.294782in}}{\pgfqpoint{6.314076in}{4.299173in}}{\pgfqpoint{6.303025in}{4.299173in}}%
\pgfpathcurveto{\pgfqpoint{6.291975in}{4.299173in}}{\pgfqpoint{6.281376in}{4.294782in}}{\pgfqpoint{6.273563in}{4.286969in}}%
\pgfpathcurveto{\pgfqpoint{6.265749in}{4.279155in}}{\pgfqpoint{6.261359in}{4.268556in}}{\pgfqpoint{6.261359in}{4.257506in}}%
\pgfpathcurveto{\pgfqpoint{6.261359in}{4.246456in}}{\pgfqpoint{6.265749in}{4.235857in}}{\pgfqpoint{6.273563in}{4.228043in}}%
\pgfpathcurveto{\pgfqpoint{6.281376in}{4.220230in}}{\pgfqpoint{6.291975in}{4.215839in}}{\pgfqpoint{6.303025in}{4.215839in}}%
\pgfpathclose%
\pgfusepath{stroke,fill}%
\end{pgfscope}%
\begin{pgfscope}%
\pgfpathrectangle{\pgfqpoint{0.526127in}{0.331635in}}{\pgfqpoint{9.300000in}{7.700000in}}%
\pgfusepath{clip}%
\pgfsetbuttcap%
\pgfsetroundjoin%
\definecolor{currentfill}{rgb}{0.870588,0.733333,0.607843}%
\pgfsetfillcolor{currentfill}%
\pgfsetlinewidth{0.481800pt}%
\definecolor{currentstroke}{rgb}{1.000000,1.000000,1.000000}%
\pgfsetstrokecolor{currentstroke}%
\pgfsetdash{}{0pt}%
\pgfpathmoveto{\pgfqpoint{3.101691in}{7.304554in}}%
\pgfpathcurveto{\pgfqpoint{3.112741in}{7.304554in}}{\pgfqpoint{3.123340in}{7.308944in}}{\pgfqpoint{3.131154in}{7.316758in}}%
\pgfpathcurveto{\pgfqpoint{3.138968in}{7.324571in}}{\pgfqpoint{3.143358in}{7.335170in}}{\pgfqpoint{3.143358in}{7.346220in}}%
\pgfpathcurveto{\pgfqpoint{3.143358in}{7.357270in}}{\pgfqpoint{3.138968in}{7.367869in}}{\pgfqpoint{3.131154in}{7.375683in}}%
\pgfpathcurveto{\pgfqpoint{3.123340in}{7.383497in}}{\pgfqpoint{3.112741in}{7.387887in}}{\pgfqpoint{3.101691in}{7.387887in}}%
\pgfpathcurveto{\pgfqpoint{3.090641in}{7.387887in}}{\pgfqpoint{3.080042in}{7.383497in}}{\pgfqpoint{3.072229in}{7.375683in}}%
\pgfpathcurveto{\pgfqpoint{3.064415in}{7.367869in}}{\pgfqpoint{3.060025in}{7.357270in}}{\pgfqpoint{3.060025in}{7.346220in}}%
\pgfpathcurveto{\pgfqpoint{3.060025in}{7.335170in}}{\pgfqpoint{3.064415in}{7.324571in}}{\pgfqpoint{3.072229in}{7.316758in}}%
\pgfpathcurveto{\pgfqpoint{3.080042in}{7.308944in}}{\pgfqpoint{3.090641in}{7.304554in}}{\pgfqpoint{3.101691in}{7.304554in}}%
\pgfpathclose%
\pgfusepath{stroke,fill}%
\end{pgfscope}%
\begin{pgfscope}%
\pgfpathrectangle{\pgfqpoint{0.526127in}{0.331635in}}{\pgfqpoint{9.300000in}{7.700000in}}%
\pgfusepath{clip}%
\pgfsetbuttcap%
\pgfsetroundjoin%
\definecolor{currentfill}{rgb}{0.870588,0.733333,0.607843}%
\pgfsetfillcolor{currentfill}%
\pgfsetlinewidth{0.481800pt}%
\definecolor{currentstroke}{rgb}{1.000000,1.000000,1.000000}%
\pgfsetstrokecolor{currentstroke}%
\pgfsetdash{}{0pt}%
\pgfpathmoveto{\pgfqpoint{6.025455in}{1.506152in}}%
\pgfpathcurveto{\pgfqpoint{6.036505in}{1.506152in}}{\pgfqpoint{6.047104in}{1.510542in}}{\pgfqpoint{6.054918in}{1.518356in}}%
\pgfpathcurveto{\pgfqpoint{6.062731in}{1.526170in}}{\pgfqpoint{6.067122in}{1.536769in}}{\pgfqpoint{6.067122in}{1.547819in}}%
\pgfpathcurveto{\pgfqpoint{6.067122in}{1.558869in}}{\pgfqpoint{6.062731in}{1.569468in}}{\pgfqpoint{6.054918in}{1.577282in}}%
\pgfpathcurveto{\pgfqpoint{6.047104in}{1.585095in}}{\pgfqpoint{6.036505in}{1.589485in}}{\pgfqpoint{6.025455in}{1.589485in}}%
\pgfpathcurveto{\pgfqpoint{6.014405in}{1.589485in}}{\pgfqpoint{6.003806in}{1.585095in}}{\pgfqpoint{5.995992in}{1.577282in}}%
\pgfpathcurveto{\pgfqpoint{5.988179in}{1.569468in}}{\pgfqpoint{5.983788in}{1.558869in}}{\pgfqpoint{5.983788in}{1.547819in}}%
\pgfpathcurveto{\pgfqpoint{5.983788in}{1.536769in}}{\pgfqpoint{5.988179in}{1.526170in}}{\pgfqpoint{5.995992in}{1.518356in}}%
\pgfpathcurveto{\pgfqpoint{6.003806in}{1.510542in}}{\pgfqpoint{6.014405in}{1.506152in}}{\pgfqpoint{6.025455in}{1.506152in}}%
\pgfpathclose%
\pgfusepath{stroke,fill}%
\end{pgfscope}%
\begin{pgfscope}%
\pgfpathrectangle{\pgfqpoint{0.526127in}{0.331635in}}{\pgfqpoint{9.300000in}{7.700000in}}%
\pgfusepath{clip}%
\pgfsetbuttcap%
\pgfsetroundjoin%
\definecolor{currentfill}{rgb}{0.870588,0.733333,0.607843}%
\pgfsetfillcolor{currentfill}%
\pgfsetlinewidth{0.481800pt}%
\definecolor{currentstroke}{rgb}{1.000000,1.000000,1.000000}%
\pgfsetstrokecolor{currentstroke}%
\pgfsetdash{}{0pt}%
\pgfpathmoveto{\pgfqpoint{2.005207in}{1.954811in}}%
\pgfpathcurveto{\pgfqpoint{2.016257in}{1.954811in}}{\pgfqpoint{2.026856in}{1.959201in}}{\pgfqpoint{2.034670in}{1.967015in}}%
\pgfpathcurveto{\pgfqpoint{2.042483in}{1.974828in}}{\pgfqpoint{2.046874in}{1.985427in}}{\pgfqpoint{2.046874in}{1.996477in}}%
\pgfpathcurveto{\pgfqpoint{2.046874in}{2.007528in}}{\pgfqpoint{2.042483in}{2.018127in}}{\pgfqpoint{2.034670in}{2.025940in}}%
\pgfpathcurveto{\pgfqpoint{2.026856in}{2.033754in}}{\pgfqpoint{2.016257in}{2.038144in}}{\pgfqpoint{2.005207in}{2.038144in}}%
\pgfpathcurveto{\pgfqpoint{1.994157in}{2.038144in}}{\pgfqpoint{1.983558in}{2.033754in}}{\pgfqpoint{1.975744in}{2.025940in}}%
\pgfpathcurveto{\pgfqpoint{1.967930in}{2.018127in}}{\pgfqpoint{1.963540in}{2.007528in}}{\pgfqpoint{1.963540in}{1.996477in}}%
\pgfpathcurveto{\pgfqpoint{1.963540in}{1.985427in}}{\pgfqpoint{1.967930in}{1.974828in}}{\pgfqpoint{1.975744in}{1.967015in}}%
\pgfpathcurveto{\pgfqpoint{1.983558in}{1.959201in}}{\pgfqpoint{1.994157in}{1.954811in}}{\pgfqpoint{2.005207in}{1.954811in}}%
\pgfpathclose%
\pgfusepath{stroke,fill}%
\end{pgfscope}%
\begin{pgfscope}%
\pgfpathrectangle{\pgfqpoint{0.526127in}{0.331635in}}{\pgfqpoint{9.300000in}{7.700000in}}%
\pgfusepath{clip}%
\pgfsetbuttcap%
\pgfsetroundjoin%
\definecolor{currentfill}{rgb}{0.870588,0.733333,0.607843}%
\pgfsetfillcolor{currentfill}%
\pgfsetlinewidth{0.481800pt}%
\definecolor{currentstroke}{rgb}{1.000000,1.000000,1.000000}%
\pgfsetstrokecolor{currentstroke}%
\pgfsetdash{}{0pt}%
\pgfpathmoveto{\pgfqpoint{0.948854in}{3.258509in}}%
\pgfpathcurveto{\pgfqpoint{0.959904in}{3.258509in}}{\pgfqpoint{0.970503in}{3.262899in}}{\pgfqpoint{0.978317in}{3.270713in}}%
\pgfpathcurveto{\pgfqpoint{0.986130in}{3.278527in}}{\pgfqpoint{0.990521in}{3.289126in}}{\pgfqpoint{0.990521in}{3.300176in}}%
\pgfpathcurveto{\pgfqpoint{0.990521in}{3.311226in}}{\pgfqpoint{0.986130in}{3.321825in}}{\pgfqpoint{0.978317in}{3.329639in}}%
\pgfpathcurveto{\pgfqpoint{0.970503in}{3.337452in}}{\pgfqpoint{0.959904in}{3.341842in}}{\pgfqpoint{0.948854in}{3.341842in}}%
\pgfpathcurveto{\pgfqpoint{0.937804in}{3.341842in}}{\pgfqpoint{0.927205in}{3.337452in}}{\pgfqpoint{0.919391in}{3.329639in}}%
\pgfpathcurveto{\pgfqpoint{0.911578in}{3.321825in}}{\pgfqpoint{0.907187in}{3.311226in}}{\pgfqpoint{0.907187in}{3.300176in}}%
\pgfpathcurveto{\pgfqpoint{0.907187in}{3.289126in}}{\pgfqpoint{0.911578in}{3.278527in}}{\pgfqpoint{0.919391in}{3.270713in}}%
\pgfpathcurveto{\pgfqpoint{0.927205in}{3.262899in}}{\pgfqpoint{0.937804in}{3.258509in}}{\pgfqpoint{0.948854in}{3.258509in}}%
\pgfpathclose%
\pgfusepath{stroke,fill}%
\end{pgfscope}%
\begin{pgfscope}%
\pgfpathrectangle{\pgfqpoint{0.526127in}{0.331635in}}{\pgfqpoint{9.300000in}{7.700000in}}%
\pgfusepath{clip}%
\pgfsetbuttcap%
\pgfsetroundjoin%
\definecolor{currentfill}{rgb}{0.870588,0.733333,0.607843}%
\pgfsetfillcolor{currentfill}%
\pgfsetlinewidth{0.481800pt}%
\definecolor{currentstroke}{rgb}{1.000000,1.000000,1.000000}%
\pgfsetstrokecolor{currentstroke}%
\pgfsetdash{}{0pt}%
\pgfpathmoveto{\pgfqpoint{2.231357in}{2.117700in}}%
\pgfpathcurveto{\pgfqpoint{2.242407in}{2.117700in}}{\pgfqpoint{2.253006in}{2.122090in}}{\pgfqpoint{2.260820in}{2.129904in}}%
\pgfpathcurveto{\pgfqpoint{2.268634in}{2.137718in}}{\pgfqpoint{2.273024in}{2.148317in}}{\pgfqpoint{2.273024in}{2.159367in}}%
\pgfpathcurveto{\pgfqpoint{2.273024in}{2.170417in}}{\pgfqpoint{2.268634in}{2.181016in}}{\pgfqpoint{2.260820in}{2.188830in}}%
\pgfpathcurveto{\pgfqpoint{2.253006in}{2.196643in}}{\pgfqpoint{2.242407in}{2.201034in}}{\pgfqpoint{2.231357in}{2.201034in}}%
\pgfpathcurveto{\pgfqpoint{2.220307in}{2.201034in}}{\pgfqpoint{2.209708in}{2.196643in}}{\pgfqpoint{2.201894in}{2.188830in}}%
\pgfpathcurveto{\pgfqpoint{2.194081in}{2.181016in}}{\pgfqpoint{2.189691in}{2.170417in}}{\pgfqpoint{2.189691in}{2.159367in}}%
\pgfpathcurveto{\pgfqpoint{2.189691in}{2.148317in}}{\pgfqpoint{2.194081in}{2.137718in}}{\pgfqpoint{2.201894in}{2.129904in}}%
\pgfpathcurveto{\pgfqpoint{2.209708in}{2.122090in}}{\pgfqpoint{2.220307in}{2.117700in}}{\pgfqpoint{2.231357in}{2.117700in}}%
\pgfpathclose%
\pgfusepath{stroke,fill}%
\end{pgfscope}%
\begin{pgfscope}%
\pgfpathrectangle{\pgfqpoint{0.526127in}{0.331635in}}{\pgfqpoint{9.300000in}{7.700000in}}%
\pgfusepath{clip}%
\pgfsetbuttcap%
\pgfsetroundjoin%
\definecolor{currentfill}{rgb}{0.870588,0.733333,0.607843}%
\pgfsetfillcolor{currentfill}%
\pgfsetlinewidth{0.481800pt}%
\definecolor{currentstroke}{rgb}{1.000000,1.000000,1.000000}%
\pgfsetstrokecolor{currentstroke}%
\pgfsetdash{}{0pt}%
\pgfpathmoveto{\pgfqpoint{5.273237in}{2.683641in}}%
\pgfpathcurveto{\pgfqpoint{5.284287in}{2.683641in}}{\pgfqpoint{5.294886in}{2.688031in}}{\pgfqpoint{5.302700in}{2.695845in}}%
\pgfpathcurveto{\pgfqpoint{5.310513in}{2.703658in}}{\pgfqpoint{5.314903in}{2.714257in}}{\pgfqpoint{5.314903in}{2.725307in}}%
\pgfpathcurveto{\pgfqpoint{5.314903in}{2.736358in}}{\pgfqpoint{5.310513in}{2.746957in}}{\pgfqpoint{5.302700in}{2.754770in}}%
\pgfpathcurveto{\pgfqpoint{5.294886in}{2.762584in}}{\pgfqpoint{5.284287in}{2.766974in}}{\pgfqpoint{5.273237in}{2.766974in}}%
\pgfpathcurveto{\pgfqpoint{5.262187in}{2.766974in}}{\pgfqpoint{5.251588in}{2.762584in}}{\pgfqpoint{5.243774in}{2.754770in}}%
\pgfpathcurveto{\pgfqpoint{5.235960in}{2.746957in}}{\pgfqpoint{5.231570in}{2.736358in}}{\pgfqpoint{5.231570in}{2.725307in}}%
\pgfpathcurveto{\pgfqpoint{5.231570in}{2.714257in}}{\pgfqpoint{5.235960in}{2.703658in}}{\pgfqpoint{5.243774in}{2.695845in}}%
\pgfpathcurveto{\pgfqpoint{5.251588in}{2.688031in}}{\pgfqpoint{5.262187in}{2.683641in}}{\pgfqpoint{5.273237in}{2.683641in}}%
\pgfpathclose%
\pgfusepath{stroke,fill}%
\end{pgfscope}%
\begin{pgfscope}%
\pgfpathrectangle{\pgfqpoint{0.526127in}{0.331635in}}{\pgfqpoint{9.300000in}{7.700000in}}%
\pgfusepath{clip}%
\pgfsetbuttcap%
\pgfsetroundjoin%
\definecolor{currentfill}{rgb}{0.870588,0.733333,0.607843}%
\pgfsetfillcolor{currentfill}%
\pgfsetlinewidth{0.481800pt}%
\definecolor{currentstroke}{rgb}{1.000000,1.000000,1.000000}%
\pgfsetstrokecolor{currentstroke}%
\pgfsetdash{}{0pt}%
\pgfpathmoveto{\pgfqpoint{4.721664in}{2.278825in}}%
\pgfpathcurveto{\pgfqpoint{4.732714in}{2.278825in}}{\pgfqpoint{4.743313in}{2.283215in}}{\pgfqpoint{4.751127in}{2.291028in}}%
\pgfpathcurveto{\pgfqpoint{4.758940in}{2.298842in}}{\pgfqpoint{4.763331in}{2.309441in}}{\pgfqpoint{4.763331in}{2.320491in}}%
\pgfpathcurveto{\pgfqpoint{4.763331in}{2.331541in}}{\pgfqpoint{4.758940in}{2.342140in}}{\pgfqpoint{4.751127in}{2.349954in}}%
\pgfpathcurveto{\pgfqpoint{4.743313in}{2.357768in}}{\pgfqpoint{4.732714in}{2.362158in}}{\pgfqpoint{4.721664in}{2.362158in}}%
\pgfpathcurveto{\pgfqpoint{4.710614in}{2.362158in}}{\pgfqpoint{4.700015in}{2.357768in}}{\pgfqpoint{4.692201in}{2.349954in}}%
\pgfpathcurveto{\pgfqpoint{4.684388in}{2.342140in}}{\pgfqpoint{4.679997in}{2.331541in}}{\pgfqpoint{4.679997in}{2.320491in}}%
\pgfpathcurveto{\pgfqpoint{4.679997in}{2.309441in}}{\pgfqpoint{4.684388in}{2.298842in}}{\pgfqpoint{4.692201in}{2.291028in}}%
\pgfpathcurveto{\pgfqpoint{4.700015in}{2.283215in}}{\pgfqpoint{4.710614in}{2.278825in}}{\pgfqpoint{4.721664in}{2.278825in}}%
\pgfpathclose%
\pgfusepath{stroke,fill}%
\end{pgfscope}%
\begin{pgfscope}%
\pgfpathrectangle{\pgfqpoint{0.526127in}{0.331635in}}{\pgfqpoint{9.300000in}{7.700000in}}%
\pgfusepath{clip}%
\pgfsetbuttcap%
\pgfsetroundjoin%
\definecolor{currentfill}{rgb}{0.870588,0.733333,0.607843}%
\pgfsetfillcolor{currentfill}%
\pgfsetlinewidth{0.481800pt}%
\definecolor{currentstroke}{rgb}{1.000000,1.000000,1.000000}%
\pgfsetstrokecolor{currentstroke}%
\pgfsetdash{}{0pt}%
\pgfpathmoveto{\pgfqpoint{5.294260in}{1.656223in}}%
\pgfpathcurveto{\pgfqpoint{5.305310in}{1.656223in}}{\pgfqpoint{5.315909in}{1.660613in}}{\pgfqpoint{5.323723in}{1.668427in}}%
\pgfpathcurveto{\pgfqpoint{5.331537in}{1.676240in}}{\pgfqpoint{5.335927in}{1.686839in}}{\pgfqpoint{5.335927in}{1.697890in}}%
\pgfpathcurveto{\pgfqpoint{5.335927in}{1.708940in}}{\pgfqpoint{5.331537in}{1.719539in}}{\pgfqpoint{5.323723in}{1.727352in}}%
\pgfpathcurveto{\pgfqpoint{5.315909in}{1.735166in}}{\pgfqpoint{5.305310in}{1.739556in}}{\pgfqpoint{5.294260in}{1.739556in}}%
\pgfpathcurveto{\pgfqpoint{5.283210in}{1.739556in}}{\pgfqpoint{5.272611in}{1.735166in}}{\pgfqpoint{5.264797in}{1.727352in}}%
\pgfpathcurveto{\pgfqpoint{5.256984in}{1.719539in}}{\pgfqpoint{5.252594in}{1.708940in}}{\pgfqpoint{5.252594in}{1.697890in}}%
\pgfpathcurveto{\pgfqpoint{5.252594in}{1.686839in}}{\pgfqpoint{5.256984in}{1.676240in}}{\pgfqpoint{5.264797in}{1.668427in}}%
\pgfpathcurveto{\pgfqpoint{5.272611in}{1.660613in}}{\pgfqpoint{5.283210in}{1.656223in}}{\pgfqpoint{5.294260in}{1.656223in}}%
\pgfpathclose%
\pgfusepath{stroke,fill}%
\end{pgfscope}%
\begin{pgfscope}%
\pgfpathrectangle{\pgfqpoint{0.526127in}{0.331635in}}{\pgfqpoint{9.300000in}{7.700000in}}%
\pgfusepath{clip}%
\pgfsetbuttcap%
\pgfsetroundjoin%
\definecolor{currentfill}{rgb}{0.870588,0.733333,0.607843}%
\pgfsetfillcolor{currentfill}%
\pgfsetlinewidth{0.481800pt}%
\definecolor{currentstroke}{rgb}{1.000000,1.000000,1.000000}%
\pgfsetstrokecolor{currentstroke}%
\pgfsetdash{}{0pt}%
\pgfpathmoveto{\pgfqpoint{3.016007in}{6.448638in}}%
\pgfpathcurveto{\pgfqpoint{3.027057in}{6.448638in}}{\pgfqpoint{3.037657in}{6.453028in}}{\pgfqpoint{3.045470in}{6.460842in}}%
\pgfpathcurveto{\pgfqpoint{3.053284in}{6.468655in}}{\pgfqpoint{3.057674in}{6.479254in}}{\pgfqpoint{3.057674in}{6.490305in}}%
\pgfpathcurveto{\pgfqpoint{3.057674in}{6.501355in}}{\pgfqpoint{3.053284in}{6.511954in}}{\pgfqpoint{3.045470in}{6.519767in}}%
\pgfpathcurveto{\pgfqpoint{3.037657in}{6.527581in}}{\pgfqpoint{3.027057in}{6.531971in}}{\pgfqpoint{3.016007in}{6.531971in}}%
\pgfpathcurveto{\pgfqpoint{3.004957in}{6.531971in}}{\pgfqpoint{2.994358in}{6.527581in}}{\pgfqpoint{2.986545in}{6.519767in}}%
\pgfpathcurveto{\pgfqpoint{2.978731in}{6.511954in}}{\pgfqpoint{2.974341in}{6.501355in}}{\pgfqpoint{2.974341in}{6.490305in}}%
\pgfpathcurveto{\pgfqpoint{2.974341in}{6.479254in}}{\pgfqpoint{2.978731in}{6.468655in}}{\pgfqpoint{2.986545in}{6.460842in}}%
\pgfpathcurveto{\pgfqpoint{2.994358in}{6.453028in}}{\pgfqpoint{3.004957in}{6.448638in}}{\pgfqpoint{3.016007in}{6.448638in}}%
\pgfpathclose%
\pgfusepath{stroke,fill}%
\end{pgfscope}%
\begin{pgfscope}%
\pgfpathrectangle{\pgfqpoint{0.526127in}{0.331635in}}{\pgfqpoint{9.300000in}{7.700000in}}%
\pgfusepath{clip}%
\pgfsetbuttcap%
\pgfsetroundjoin%
\definecolor{currentfill}{rgb}{0.870588,0.733333,0.607843}%
\pgfsetfillcolor{currentfill}%
\pgfsetlinewidth{0.481800pt}%
\definecolor{currentstroke}{rgb}{1.000000,1.000000,1.000000}%
\pgfsetstrokecolor{currentstroke}%
\pgfsetdash{}{0pt}%
\pgfpathmoveto{\pgfqpoint{0.975724in}{3.271591in}}%
\pgfpathcurveto{\pgfqpoint{0.986774in}{3.271591in}}{\pgfqpoint{0.997373in}{3.275982in}}{\pgfqpoint{1.005187in}{3.283795in}}%
\pgfpathcurveto{\pgfqpoint{1.013000in}{3.291609in}}{\pgfqpoint{1.017391in}{3.302208in}}{\pgfqpoint{1.017391in}{3.313258in}}%
\pgfpathcurveto{\pgfqpoint{1.017391in}{3.324308in}}{\pgfqpoint{1.013000in}{3.334907in}}{\pgfqpoint{1.005187in}{3.342721in}}%
\pgfpathcurveto{\pgfqpoint{0.997373in}{3.350534in}}{\pgfqpoint{0.986774in}{3.354925in}}{\pgfqpoint{0.975724in}{3.354925in}}%
\pgfpathcurveto{\pgfqpoint{0.964674in}{3.354925in}}{\pgfqpoint{0.954075in}{3.350534in}}{\pgfqpoint{0.946261in}{3.342721in}}%
\pgfpathcurveto{\pgfqpoint{0.938448in}{3.334907in}}{\pgfqpoint{0.934057in}{3.324308in}}{\pgfqpoint{0.934057in}{3.313258in}}%
\pgfpathcurveto{\pgfqpoint{0.934057in}{3.302208in}}{\pgfqpoint{0.938448in}{3.291609in}}{\pgfqpoint{0.946261in}{3.283795in}}%
\pgfpathcurveto{\pgfqpoint{0.954075in}{3.275982in}}{\pgfqpoint{0.964674in}{3.271591in}}{\pgfqpoint{0.975724in}{3.271591in}}%
\pgfpathclose%
\pgfusepath{stroke,fill}%
\end{pgfscope}%
\begin{pgfscope}%
\pgfpathrectangle{\pgfqpoint{0.526127in}{0.331635in}}{\pgfqpoint{9.300000in}{7.700000in}}%
\pgfusepath{clip}%
\pgfsetbuttcap%
\pgfsetroundjoin%
\definecolor{currentfill}{rgb}{0.870588,0.733333,0.607843}%
\pgfsetfillcolor{currentfill}%
\pgfsetlinewidth{0.481800pt}%
\definecolor{currentstroke}{rgb}{1.000000,1.000000,1.000000}%
\pgfsetstrokecolor{currentstroke}%
\pgfsetdash{}{0pt}%
\pgfpathmoveto{\pgfqpoint{6.443064in}{3.164056in}}%
\pgfpathcurveto{\pgfqpoint{6.454114in}{3.164056in}}{\pgfqpoint{6.464713in}{3.168446in}}{\pgfqpoint{6.472527in}{3.176260in}}%
\pgfpathcurveto{\pgfqpoint{6.480341in}{3.184073in}}{\pgfqpoint{6.484731in}{3.194672in}}{\pgfqpoint{6.484731in}{3.205723in}}%
\pgfpathcurveto{\pgfqpoint{6.484731in}{3.216773in}}{\pgfqpoint{6.480341in}{3.227372in}}{\pgfqpoint{6.472527in}{3.235185in}}%
\pgfpathcurveto{\pgfqpoint{6.464713in}{3.242999in}}{\pgfqpoint{6.454114in}{3.247389in}}{\pgfqpoint{6.443064in}{3.247389in}}%
\pgfpathcurveto{\pgfqpoint{6.432014in}{3.247389in}}{\pgfqpoint{6.421415in}{3.242999in}}{\pgfqpoint{6.413601in}{3.235185in}}%
\pgfpathcurveto{\pgfqpoint{6.405788in}{3.227372in}}{\pgfqpoint{6.401397in}{3.216773in}}{\pgfqpoint{6.401397in}{3.205723in}}%
\pgfpathcurveto{\pgfqpoint{6.401397in}{3.194672in}}{\pgfqpoint{6.405788in}{3.184073in}}{\pgfqpoint{6.413601in}{3.176260in}}%
\pgfpathcurveto{\pgfqpoint{6.421415in}{3.168446in}}{\pgfqpoint{6.432014in}{3.164056in}}{\pgfqpoint{6.443064in}{3.164056in}}%
\pgfpathclose%
\pgfusepath{stroke,fill}%
\end{pgfscope}%
\begin{pgfscope}%
\pgfpathrectangle{\pgfqpoint{0.526127in}{0.331635in}}{\pgfqpoint{9.300000in}{7.700000in}}%
\pgfusepath{clip}%
\pgfsetbuttcap%
\pgfsetroundjoin%
\definecolor{currentfill}{rgb}{0.870588,0.733333,0.607843}%
\pgfsetfillcolor{currentfill}%
\pgfsetlinewidth{0.481800pt}%
\definecolor{currentstroke}{rgb}{1.000000,1.000000,1.000000}%
\pgfsetstrokecolor{currentstroke}%
\pgfsetdash{}{0pt}%
\pgfpathmoveto{\pgfqpoint{5.818974in}{2.915205in}}%
\pgfpathcurveto{\pgfqpoint{5.830024in}{2.915205in}}{\pgfqpoint{5.840623in}{2.919595in}}{\pgfqpoint{5.848437in}{2.927408in}}%
\pgfpathcurveto{\pgfqpoint{5.856250in}{2.935222in}}{\pgfqpoint{5.860641in}{2.945821in}}{\pgfqpoint{5.860641in}{2.956871in}}%
\pgfpathcurveto{\pgfqpoint{5.860641in}{2.967921in}}{\pgfqpoint{5.856250in}{2.978520in}}{\pgfqpoint{5.848437in}{2.986334in}}%
\pgfpathcurveto{\pgfqpoint{5.840623in}{2.994148in}}{\pgfqpoint{5.830024in}{2.998538in}}{\pgfqpoint{5.818974in}{2.998538in}}%
\pgfpathcurveto{\pgfqpoint{5.807924in}{2.998538in}}{\pgfqpoint{5.797325in}{2.994148in}}{\pgfqpoint{5.789511in}{2.986334in}}%
\pgfpathcurveto{\pgfqpoint{5.781698in}{2.978520in}}{\pgfqpoint{5.777307in}{2.967921in}}{\pgfqpoint{5.777307in}{2.956871in}}%
\pgfpathcurveto{\pgfqpoint{5.777307in}{2.945821in}}{\pgfqpoint{5.781698in}{2.935222in}}{\pgfqpoint{5.789511in}{2.927408in}}%
\pgfpathcurveto{\pgfqpoint{5.797325in}{2.919595in}}{\pgfqpoint{5.807924in}{2.915205in}}{\pgfqpoint{5.818974in}{2.915205in}}%
\pgfpathclose%
\pgfusepath{stroke,fill}%
\end{pgfscope}%
\begin{pgfscope}%
\pgfpathrectangle{\pgfqpoint{0.526127in}{0.331635in}}{\pgfqpoint{9.300000in}{7.700000in}}%
\pgfusepath{clip}%
\pgfsetbuttcap%
\pgfsetroundjoin%
\definecolor{currentfill}{rgb}{0.870588,0.733333,0.607843}%
\pgfsetfillcolor{currentfill}%
\pgfsetlinewidth{0.481800pt}%
\definecolor{currentstroke}{rgb}{1.000000,1.000000,1.000000}%
\pgfsetstrokecolor{currentstroke}%
\pgfsetdash{}{0pt}%
\pgfpathmoveto{\pgfqpoint{3.186993in}{2.546735in}}%
\pgfpathcurveto{\pgfqpoint{3.198043in}{2.546735in}}{\pgfqpoint{3.208642in}{2.551125in}}{\pgfqpoint{3.216456in}{2.558938in}}%
\pgfpathcurveto{\pgfqpoint{3.224269in}{2.566752in}}{\pgfqpoint{3.228659in}{2.577351in}}{\pgfqpoint{3.228659in}{2.588401in}}%
\pgfpathcurveto{\pgfqpoint{3.228659in}{2.599451in}}{\pgfqpoint{3.224269in}{2.610050in}}{\pgfqpoint{3.216456in}{2.617864in}}%
\pgfpathcurveto{\pgfqpoint{3.208642in}{2.625678in}}{\pgfqpoint{3.198043in}{2.630068in}}{\pgfqpoint{3.186993in}{2.630068in}}%
\pgfpathcurveto{\pgfqpoint{3.175943in}{2.630068in}}{\pgfqpoint{3.165344in}{2.625678in}}{\pgfqpoint{3.157530in}{2.617864in}}%
\pgfpathcurveto{\pgfqpoint{3.149716in}{2.610050in}}{\pgfqpoint{3.145326in}{2.599451in}}{\pgfqpoint{3.145326in}{2.588401in}}%
\pgfpathcurveto{\pgfqpoint{3.145326in}{2.577351in}}{\pgfqpoint{3.149716in}{2.566752in}}{\pgfqpoint{3.157530in}{2.558938in}}%
\pgfpathcurveto{\pgfqpoint{3.165344in}{2.551125in}}{\pgfqpoint{3.175943in}{2.546735in}}{\pgfqpoint{3.186993in}{2.546735in}}%
\pgfpathclose%
\pgfusepath{stroke,fill}%
\end{pgfscope}%
\begin{pgfscope}%
\pgfpathrectangle{\pgfqpoint{0.526127in}{0.331635in}}{\pgfqpoint{9.300000in}{7.700000in}}%
\pgfusepath{clip}%
\pgfsetbuttcap%
\pgfsetroundjoin%
\definecolor{currentfill}{rgb}{0.870588,0.733333,0.607843}%
\pgfsetfillcolor{currentfill}%
\pgfsetlinewidth{0.481800pt}%
\definecolor{currentstroke}{rgb}{1.000000,1.000000,1.000000}%
\pgfsetstrokecolor{currentstroke}%
\pgfsetdash{}{0pt}%
\pgfpathmoveto{\pgfqpoint{5.009171in}{2.382315in}}%
\pgfpathcurveto{\pgfqpoint{5.020222in}{2.382315in}}{\pgfqpoint{5.030821in}{2.386706in}}{\pgfqpoint{5.038634in}{2.394519in}}%
\pgfpathcurveto{\pgfqpoint{5.046448in}{2.402333in}}{\pgfqpoint{5.050838in}{2.412932in}}{\pgfqpoint{5.050838in}{2.423982in}}%
\pgfpathcurveto{\pgfqpoint{5.050838in}{2.435032in}}{\pgfqpoint{5.046448in}{2.445631in}}{\pgfqpoint{5.038634in}{2.453445in}}%
\pgfpathcurveto{\pgfqpoint{5.030821in}{2.461259in}}{\pgfqpoint{5.020222in}{2.465649in}}{\pgfqpoint{5.009171in}{2.465649in}}%
\pgfpathcurveto{\pgfqpoint{4.998121in}{2.465649in}}{\pgfqpoint{4.987522in}{2.461259in}}{\pgfqpoint{4.979709in}{2.453445in}}%
\pgfpathcurveto{\pgfqpoint{4.971895in}{2.445631in}}{\pgfqpoint{4.967505in}{2.435032in}}{\pgfqpoint{4.967505in}{2.423982in}}%
\pgfpathcurveto{\pgfqpoint{4.967505in}{2.412932in}}{\pgfqpoint{4.971895in}{2.402333in}}{\pgfqpoint{4.979709in}{2.394519in}}%
\pgfpathcurveto{\pgfqpoint{4.987522in}{2.386706in}}{\pgfqpoint{4.998121in}{2.382315in}}{\pgfqpoint{5.009171in}{2.382315in}}%
\pgfpathclose%
\pgfusepath{stroke,fill}%
\end{pgfscope}%
\begin{pgfscope}%
\pgfpathrectangle{\pgfqpoint{0.526127in}{0.331635in}}{\pgfqpoint{9.300000in}{7.700000in}}%
\pgfusepath{clip}%
\pgfsetbuttcap%
\pgfsetroundjoin%
\definecolor{currentfill}{rgb}{0.870588,0.733333,0.607843}%
\pgfsetfillcolor{currentfill}%
\pgfsetlinewidth{0.481800pt}%
\definecolor{currentstroke}{rgb}{1.000000,1.000000,1.000000}%
\pgfsetstrokecolor{currentstroke}%
\pgfsetdash{}{0pt}%
\pgfpathmoveto{\pgfqpoint{5.648571in}{1.105803in}}%
\pgfpathcurveto{\pgfqpoint{5.659622in}{1.105803in}}{\pgfqpoint{5.670221in}{1.110194in}}{\pgfqpoint{5.678034in}{1.118007in}}%
\pgfpathcurveto{\pgfqpoint{5.685848in}{1.125821in}}{\pgfqpoint{5.690238in}{1.136420in}}{\pgfqpoint{5.690238in}{1.147470in}}%
\pgfpathcurveto{\pgfqpoint{5.690238in}{1.158520in}}{\pgfqpoint{5.685848in}{1.169119in}}{\pgfqpoint{5.678034in}{1.176933in}}%
\pgfpathcurveto{\pgfqpoint{5.670221in}{1.184747in}}{\pgfqpoint{5.659622in}{1.189137in}}{\pgfqpoint{5.648571in}{1.189137in}}%
\pgfpathcurveto{\pgfqpoint{5.637521in}{1.189137in}}{\pgfqpoint{5.626922in}{1.184747in}}{\pgfqpoint{5.619109in}{1.176933in}}%
\pgfpathcurveto{\pgfqpoint{5.611295in}{1.169119in}}{\pgfqpoint{5.606905in}{1.158520in}}{\pgfqpoint{5.606905in}{1.147470in}}%
\pgfpathcurveto{\pgfqpoint{5.606905in}{1.136420in}}{\pgfqpoint{5.611295in}{1.125821in}}{\pgfqpoint{5.619109in}{1.118007in}}%
\pgfpathcurveto{\pgfqpoint{5.626922in}{1.110194in}}{\pgfqpoint{5.637521in}{1.105803in}}{\pgfqpoint{5.648571in}{1.105803in}}%
\pgfpathclose%
\pgfusepath{stroke,fill}%
\end{pgfscope}%
\begin{pgfscope}%
\pgfpathrectangle{\pgfqpoint{0.526127in}{0.331635in}}{\pgfqpoint{9.300000in}{7.700000in}}%
\pgfusepath{clip}%
\pgfsetbuttcap%
\pgfsetroundjoin%
\definecolor{currentfill}{rgb}{0.870588,0.733333,0.607843}%
\pgfsetfillcolor{currentfill}%
\pgfsetlinewidth{0.481800pt}%
\definecolor{currentstroke}{rgb}{1.000000,1.000000,1.000000}%
\pgfsetstrokecolor{currentstroke}%
\pgfsetdash{}{0pt}%
\pgfpathmoveto{\pgfqpoint{4.947462in}{1.651842in}}%
\pgfpathcurveto{\pgfqpoint{4.958512in}{1.651842in}}{\pgfqpoint{4.969111in}{1.656233in}}{\pgfqpoint{4.976925in}{1.664046in}}%
\pgfpathcurveto{\pgfqpoint{4.984739in}{1.671860in}}{\pgfqpoint{4.989129in}{1.682459in}}{\pgfqpoint{4.989129in}{1.693509in}}%
\pgfpathcurveto{\pgfqpoint{4.989129in}{1.704559in}}{\pgfqpoint{4.984739in}{1.715158in}}{\pgfqpoint{4.976925in}{1.722972in}}%
\pgfpathcurveto{\pgfqpoint{4.969111in}{1.730785in}}{\pgfqpoint{4.958512in}{1.735176in}}{\pgfqpoint{4.947462in}{1.735176in}}%
\pgfpathcurveto{\pgfqpoint{4.936412in}{1.735176in}}{\pgfqpoint{4.925813in}{1.730785in}}{\pgfqpoint{4.917999in}{1.722972in}}%
\pgfpathcurveto{\pgfqpoint{4.910186in}{1.715158in}}{\pgfqpoint{4.905795in}{1.704559in}}{\pgfqpoint{4.905795in}{1.693509in}}%
\pgfpathcurveto{\pgfqpoint{4.905795in}{1.682459in}}{\pgfqpoint{4.910186in}{1.671860in}}{\pgfqpoint{4.917999in}{1.664046in}}%
\pgfpathcurveto{\pgfqpoint{4.925813in}{1.656233in}}{\pgfqpoint{4.936412in}{1.651842in}}{\pgfqpoint{4.947462in}{1.651842in}}%
\pgfpathclose%
\pgfusepath{stroke,fill}%
\end{pgfscope}%
\begin{pgfscope}%
\pgfpathrectangle{\pgfqpoint{0.526127in}{0.331635in}}{\pgfqpoint{9.300000in}{7.700000in}}%
\pgfusepath{clip}%
\pgfsetbuttcap%
\pgfsetroundjoin%
\definecolor{currentfill}{rgb}{0.870588,0.733333,0.607843}%
\pgfsetfillcolor{currentfill}%
\pgfsetlinewidth{0.481800pt}%
\definecolor{currentstroke}{rgb}{1.000000,1.000000,1.000000}%
\pgfsetstrokecolor{currentstroke}%
\pgfsetdash{}{0pt}%
\pgfpathmoveto{\pgfqpoint{5.178715in}{1.632599in}}%
\pgfpathcurveto{\pgfqpoint{5.189765in}{1.632599in}}{\pgfqpoint{5.200364in}{1.636990in}}{\pgfqpoint{5.208177in}{1.644803in}}%
\pgfpathcurveto{\pgfqpoint{5.215991in}{1.652617in}}{\pgfqpoint{5.220381in}{1.663216in}}{\pgfqpoint{5.220381in}{1.674266in}}%
\pgfpathcurveto{\pgfqpoint{5.220381in}{1.685316in}}{\pgfqpoint{5.215991in}{1.695915in}}{\pgfqpoint{5.208177in}{1.703729in}}%
\pgfpathcurveto{\pgfqpoint{5.200364in}{1.711543in}}{\pgfqpoint{5.189765in}{1.715933in}}{\pgfqpoint{5.178715in}{1.715933in}}%
\pgfpathcurveto{\pgfqpoint{5.167664in}{1.715933in}}{\pgfqpoint{5.157065in}{1.711543in}}{\pgfqpoint{5.149252in}{1.703729in}}%
\pgfpathcurveto{\pgfqpoint{5.141438in}{1.695915in}}{\pgfqpoint{5.137048in}{1.685316in}}{\pgfqpoint{5.137048in}{1.674266in}}%
\pgfpathcurveto{\pgfqpoint{5.137048in}{1.663216in}}{\pgfqpoint{5.141438in}{1.652617in}}{\pgfqpoint{5.149252in}{1.644803in}}%
\pgfpathcurveto{\pgfqpoint{5.157065in}{1.636990in}}{\pgfqpoint{5.167664in}{1.632599in}}{\pgfqpoint{5.178715in}{1.632599in}}%
\pgfpathclose%
\pgfusepath{stroke,fill}%
\end{pgfscope}%
\begin{pgfscope}%
\pgfpathrectangle{\pgfqpoint{0.526127in}{0.331635in}}{\pgfqpoint{9.300000in}{7.700000in}}%
\pgfusepath{clip}%
\pgfsetbuttcap%
\pgfsetroundjoin%
\definecolor{currentfill}{rgb}{0.870588,0.733333,0.607843}%
\pgfsetfillcolor{currentfill}%
\pgfsetlinewidth{0.481800pt}%
\definecolor{currentstroke}{rgb}{1.000000,1.000000,1.000000}%
\pgfsetstrokecolor{currentstroke}%
\pgfsetdash{}{0pt}%
\pgfpathmoveto{\pgfqpoint{6.316003in}{1.896725in}}%
\pgfpathcurveto{\pgfqpoint{6.327054in}{1.896725in}}{\pgfqpoint{6.337653in}{1.901115in}}{\pgfqpoint{6.345466in}{1.908929in}}%
\pgfpathcurveto{\pgfqpoint{6.353280in}{1.916742in}}{\pgfqpoint{6.357670in}{1.927341in}}{\pgfqpoint{6.357670in}{1.938392in}}%
\pgfpathcurveto{\pgfqpoint{6.357670in}{1.949442in}}{\pgfqpoint{6.353280in}{1.960041in}}{\pgfqpoint{6.345466in}{1.967854in}}%
\pgfpathcurveto{\pgfqpoint{6.337653in}{1.975668in}}{\pgfqpoint{6.327054in}{1.980058in}}{\pgfqpoint{6.316003in}{1.980058in}}%
\pgfpathcurveto{\pgfqpoint{6.304953in}{1.980058in}}{\pgfqpoint{6.294354in}{1.975668in}}{\pgfqpoint{6.286541in}{1.967854in}}%
\pgfpathcurveto{\pgfqpoint{6.278727in}{1.960041in}}{\pgfqpoint{6.274337in}{1.949442in}}{\pgfqpoint{6.274337in}{1.938392in}}%
\pgfpathcurveto{\pgfqpoint{6.274337in}{1.927341in}}{\pgfqpoint{6.278727in}{1.916742in}}{\pgfqpoint{6.286541in}{1.908929in}}%
\pgfpathcurveto{\pgfqpoint{6.294354in}{1.901115in}}{\pgfqpoint{6.304953in}{1.896725in}}{\pgfqpoint{6.316003in}{1.896725in}}%
\pgfpathclose%
\pgfusepath{stroke,fill}%
\end{pgfscope}%
\begin{pgfscope}%
\pgfpathrectangle{\pgfqpoint{0.526127in}{0.331635in}}{\pgfqpoint{9.300000in}{7.700000in}}%
\pgfusepath{clip}%
\pgfsetbuttcap%
\pgfsetroundjoin%
\definecolor{currentfill}{rgb}{0.870588,0.733333,0.607843}%
\pgfsetfillcolor{currentfill}%
\pgfsetlinewidth{0.481800pt}%
\definecolor{currentstroke}{rgb}{1.000000,1.000000,1.000000}%
\pgfsetstrokecolor{currentstroke}%
\pgfsetdash{}{0pt}%
\pgfpathmoveto{\pgfqpoint{4.799485in}{2.332859in}}%
\pgfpathcurveto{\pgfqpoint{4.810535in}{2.332859in}}{\pgfqpoint{4.821134in}{2.337250in}}{\pgfqpoint{4.828947in}{2.345063in}}%
\pgfpathcurveto{\pgfqpoint{4.836761in}{2.352877in}}{\pgfqpoint{4.841151in}{2.363476in}}{\pgfqpoint{4.841151in}{2.374526in}}%
\pgfpathcurveto{\pgfqpoint{4.841151in}{2.385576in}}{\pgfqpoint{4.836761in}{2.396175in}}{\pgfqpoint{4.828947in}{2.403989in}}%
\pgfpathcurveto{\pgfqpoint{4.821134in}{2.411802in}}{\pgfqpoint{4.810535in}{2.416193in}}{\pgfqpoint{4.799485in}{2.416193in}}%
\pgfpathcurveto{\pgfqpoint{4.788434in}{2.416193in}}{\pgfqpoint{4.777835in}{2.411802in}}{\pgfqpoint{4.770022in}{2.403989in}}%
\pgfpathcurveto{\pgfqpoint{4.762208in}{2.396175in}}{\pgfqpoint{4.757818in}{2.385576in}}{\pgfqpoint{4.757818in}{2.374526in}}%
\pgfpathcurveto{\pgfqpoint{4.757818in}{2.363476in}}{\pgfqpoint{4.762208in}{2.352877in}}{\pgfqpoint{4.770022in}{2.345063in}}%
\pgfpathcurveto{\pgfqpoint{4.777835in}{2.337250in}}{\pgfqpoint{4.788434in}{2.332859in}}{\pgfqpoint{4.799485in}{2.332859in}}%
\pgfpathclose%
\pgfusepath{stroke,fill}%
\end{pgfscope}%
\begin{pgfscope}%
\pgfpathrectangle{\pgfqpoint{0.526127in}{0.331635in}}{\pgfqpoint{9.300000in}{7.700000in}}%
\pgfusepath{clip}%
\pgfsetbuttcap%
\pgfsetroundjoin%
\definecolor{currentfill}{rgb}{0.870588,0.733333,0.607843}%
\pgfsetfillcolor{currentfill}%
\pgfsetlinewidth{0.481800pt}%
\definecolor{currentstroke}{rgb}{1.000000,1.000000,1.000000}%
\pgfsetstrokecolor{currentstroke}%
\pgfsetdash{}{0pt}%
\pgfpathmoveto{\pgfqpoint{3.925973in}{2.826021in}}%
\pgfpathcurveto{\pgfqpoint{3.937023in}{2.826021in}}{\pgfqpoint{3.947622in}{2.830411in}}{\pgfqpoint{3.955436in}{2.838225in}}%
\pgfpathcurveto{\pgfqpoint{3.963249in}{2.846038in}}{\pgfqpoint{3.967639in}{2.856637in}}{\pgfqpoint{3.967639in}{2.867687in}}%
\pgfpathcurveto{\pgfqpoint{3.967639in}{2.878738in}}{\pgfqpoint{3.963249in}{2.889337in}}{\pgfqpoint{3.955436in}{2.897150in}}%
\pgfpathcurveto{\pgfqpoint{3.947622in}{2.904964in}}{\pgfqpoint{3.937023in}{2.909354in}}{\pgfqpoint{3.925973in}{2.909354in}}%
\pgfpathcurveto{\pgfqpoint{3.914923in}{2.909354in}}{\pgfqpoint{3.904324in}{2.904964in}}{\pgfqpoint{3.896510in}{2.897150in}}%
\pgfpathcurveto{\pgfqpoint{3.888696in}{2.889337in}}{\pgfqpoint{3.884306in}{2.878738in}}{\pgfqpoint{3.884306in}{2.867687in}}%
\pgfpathcurveto{\pgfqpoint{3.884306in}{2.856637in}}{\pgfqpoint{3.888696in}{2.846038in}}{\pgfqpoint{3.896510in}{2.838225in}}%
\pgfpathcurveto{\pgfqpoint{3.904324in}{2.830411in}}{\pgfqpoint{3.914923in}{2.826021in}}{\pgfqpoint{3.925973in}{2.826021in}}%
\pgfpathclose%
\pgfusepath{stroke,fill}%
\end{pgfscope}%
\begin{pgfscope}%
\pgfpathrectangle{\pgfqpoint{0.526127in}{0.331635in}}{\pgfqpoint{9.300000in}{7.700000in}}%
\pgfusepath{clip}%
\pgfsetbuttcap%
\pgfsetroundjoin%
\definecolor{currentfill}{rgb}{0.870588,0.733333,0.607843}%
\pgfsetfillcolor{currentfill}%
\pgfsetlinewidth{0.481800pt}%
\definecolor{currentstroke}{rgb}{1.000000,1.000000,1.000000}%
\pgfsetstrokecolor{currentstroke}%
\pgfsetdash{}{0pt}%
\pgfpathmoveto{\pgfqpoint{2.545411in}{1.753691in}}%
\pgfpathcurveto{\pgfqpoint{2.556462in}{1.753691in}}{\pgfqpoint{2.567061in}{1.758081in}}{\pgfqpoint{2.574874in}{1.765895in}}%
\pgfpathcurveto{\pgfqpoint{2.582688in}{1.773708in}}{\pgfqpoint{2.587078in}{1.784307in}}{\pgfqpoint{2.587078in}{1.795358in}}%
\pgfpathcurveto{\pgfqpoint{2.587078in}{1.806408in}}{\pgfqpoint{2.582688in}{1.817007in}}{\pgfqpoint{2.574874in}{1.824820in}}%
\pgfpathcurveto{\pgfqpoint{2.567061in}{1.832634in}}{\pgfqpoint{2.556462in}{1.837024in}}{\pgfqpoint{2.545411in}{1.837024in}}%
\pgfpathcurveto{\pgfqpoint{2.534361in}{1.837024in}}{\pgfqpoint{2.523762in}{1.832634in}}{\pgfqpoint{2.515949in}{1.824820in}}%
\pgfpathcurveto{\pgfqpoint{2.508135in}{1.817007in}}{\pgfqpoint{2.503745in}{1.806408in}}{\pgfqpoint{2.503745in}{1.795358in}}%
\pgfpathcurveto{\pgfqpoint{2.503745in}{1.784307in}}{\pgfqpoint{2.508135in}{1.773708in}}{\pgfqpoint{2.515949in}{1.765895in}}%
\pgfpathcurveto{\pgfqpoint{2.523762in}{1.758081in}}{\pgfqpoint{2.534361in}{1.753691in}}{\pgfqpoint{2.545411in}{1.753691in}}%
\pgfpathclose%
\pgfusepath{stroke,fill}%
\end{pgfscope}%
\begin{pgfscope}%
\pgfpathrectangle{\pgfqpoint{0.526127in}{0.331635in}}{\pgfqpoint{9.300000in}{7.700000in}}%
\pgfusepath{clip}%
\pgfsetbuttcap%
\pgfsetroundjoin%
\definecolor{currentfill}{rgb}{0.870588,0.733333,0.607843}%
\pgfsetfillcolor{currentfill}%
\pgfsetlinewidth{0.481800pt}%
\definecolor{currentstroke}{rgb}{1.000000,1.000000,1.000000}%
\pgfsetstrokecolor{currentstroke}%
\pgfsetdash{}{0pt}%
\pgfpathmoveto{\pgfqpoint{4.257376in}{1.879369in}}%
\pgfpathcurveto{\pgfqpoint{4.268426in}{1.879369in}}{\pgfqpoint{4.279025in}{1.883759in}}{\pgfqpoint{4.286839in}{1.891572in}}%
\pgfpathcurveto{\pgfqpoint{4.294653in}{1.899386in}}{\pgfqpoint{4.299043in}{1.909985in}}{\pgfqpoint{4.299043in}{1.921035in}}%
\pgfpathcurveto{\pgfqpoint{4.299043in}{1.932085in}}{\pgfqpoint{4.294653in}{1.942684in}}{\pgfqpoint{4.286839in}{1.950498in}}%
\pgfpathcurveto{\pgfqpoint{4.279025in}{1.958312in}}{\pgfqpoint{4.268426in}{1.962702in}}{\pgfqpoint{4.257376in}{1.962702in}}%
\pgfpathcurveto{\pgfqpoint{4.246326in}{1.962702in}}{\pgfqpoint{4.235727in}{1.958312in}}{\pgfqpoint{4.227913in}{1.950498in}}%
\pgfpathcurveto{\pgfqpoint{4.220100in}{1.942684in}}{\pgfqpoint{4.215709in}{1.932085in}}{\pgfqpoint{4.215709in}{1.921035in}}%
\pgfpathcurveto{\pgfqpoint{4.215709in}{1.909985in}}{\pgfqpoint{4.220100in}{1.899386in}}{\pgfqpoint{4.227913in}{1.891572in}}%
\pgfpathcurveto{\pgfqpoint{4.235727in}{1.883759in}}{\pgfqpoint{4.246326in}{1.879369in}}{\pgfqpoint{4.257376in}{1.879369in}}%
\pgfpathclose%
\pgfusepath{stroke,fill}%
\end{pgfscope}%
\begin{pgfscope}%
\pgfpathrectangle{\pgfqpoint{0.526127in}{0.331635in}}{\pgfqpoint{9.300000in}{7.700000in}}%
\pgfusepath{clip}%
\pgfsetbuttcap%
\pgfsetroundjoin%
\definecolor{currentfill}{rgb}{0.980392,0.690196,0.894118}%
\pgfsetfillcolor{currentfill}%
\pgfsetlinewidth{0.481800pt}%
\definecolor{currentstroke}{rgb}{1.000000,1.000000,1.000000}%
\pgfsetstrokecolor{currentstroke}%
\pgfsetdash{}{0pt}%
\pgfpathmoveto{\pgfqpoint{8.127617in}{4.787796in}}%
\pgfpathcurveto{\pgfqpoint{8.138667in}{4.787796in}}{\pgfqpoint{8.149266in}{4.792186in}}{\pgfqpoint{8.157079in}{4.800000in}}%
\pgfpathcurveto{\pgfqpoint{8.164893in}{4.807814in}}{\pgfqpoint{8.169283in}{4.818413in}}{\pgfqpoint{8.169283in}{4.829463in}}%
\pgfpathcurveto{\pgfqpoint{8.169283in}{4.840513in}}{\pgfqpoint{8.164893in}{4.851112in}}{\pgfqpoint{8.157079in}{4.858926in}}%
\pgfpathcurveto{\pgfqpoint{8.149266in}{4.866739in}}{\pgfqpoint{8.138667in}{4.871129in}}{\pgfqpoint{8.127617in}{4.871129in}}%
\pgfpathcurveto{\pgfqpoint{8.116566in}{4.871129in}}{\pgfqpoint{8.105967in}{4.866739in}}{\pgfqpoint{8.098154in}{4.858926in}}%
\pgfpathcurveto{\pgfqpoint{8.090340in}{4.851112in}}{\pgfqpoint{8.085950in}{4.840513in}}{\pgfqpoint{8.085950in}{4.829463in}}%
\pgfpathcurveto{\pgfqpoint{8.085950in}{4.818413in}}{\pgfqpoint{8.090340in}{4.807814in}}{\pgfqpoint{8.098154in}{4.800000in}}%
\pgfpathcurveto{\pgfqpoint{8.105967in}{4.792186in}}{\pgfqpoint{8.116566in}{4.787796in}}{\pgfqpoint{8.127617in}{4.787796in}}%
\pgfpathclose%
\pgfusepath{stroke,fill}%
\end{pgfscope}%
\begin{pgfscope}%
\pgfpathrectangle{\pgfqpoint{0.526127in}{0.331635in}}{\pgfqpoint{9.300000in}{7.700000in}}%
\pgfusepath{clip}%
\pgfsetbuttcap%
\pgfsetroundjoin%
\definecolor{currentfill}{rgb}{0.980392,0.690196,0.894118}%
\pgfsetfillcolor{currentfill}%
\pgfsetlinewidth{0.481800pt}%
\definecolor{currentstroke}{rgb}{1.000000,1.000000,1.000000}%
\pgfsetstrokecolor{currentstroke}%
\pgfsetdash{}{0pt}%
\pgfpathmoveto{\pgfqpoint{5.250982in}{7.293098in}}%
\pgfpathcurveto{\pgfqpoint{5.262032in}{7.293098in}}{\pgfqpoint{5.272631in}{7.297488in}}{\pgfqpoint{5.280445in}{7.305302in}}%
\pgfpathcurveto{\pgfqpoint{5.288258in}{7.313115in}}{\pgfqpoint{5.292648in}{7.323714in}}{\pgfqpoint{5.292648in}{7.334764in}}%
\pgfpathcurveto{\pgfqpoint{5.292648in}{7.345814in}}{\pgfqpoint{5.288258in}{7.356414in}}{\pgfqpoint{5.280445in}{7.364227in}}%
\pgfpathcurveto{\pgfqpoint{5.272631in}{7.372041in}}{\pgfqpoint{5.262032in}{7.376431in}}{\pgfqpoint{5.250982in}{7.376431in}}%
\pgfpathcurveto{\pgfqpoint{5.239932in}{7.376431in}}{\pgfqpoint{5.229333in}{7.372041in}}{\pgfqpoint{5.221519in}{7.364227in}}%
\pgfpathcurveto{\pgfqpoint{5.213705in}{7.356414in}}{\pgfqpoint{5.209315in}{7.345814in}}{\pgfqpoint{5.209315in}{7.334764in}}%
\pgfpathcurveto{\pgfqpoint{5.209315in}{7.323714in}}{\pgfqpoint{5.213705in}{7.313115in}}{\pgfqpoint{5.221519in}{7.305302in}}%
\pgfpathcurveto{\pgfqpoint{5.229333in}{7.297488in}}{\pgfqpoint{5.239932in}{7.293098in}}{\pgfqpoint{5.250982in}{7.293098in}}%
\pgfpathclose%
\pgfusepath{stroke,fill}%
\end{pgfscope}%
\begin{pgfscope}%
\pgfpathrectangle{\pgfqpoint{0.526127in}{0.331635in}}{\pgfqpoint{9.300000in}{7.700000in}}%
\pgfusepath{clip}%
\pgfsetbuttcap%
\pgfsetroundjoin%
\definecolor{currentfill}{rgb}{0.980392,0.690196,0.894118}%
\pgfsetfillcolor{currentfill}%
\pgfsetlinewidth{0.481800pt}%
\definecolor{currentstroke}{rgb}{1.000000,1.000000,1.000000}%
\pgfsetstrokecolor{currentstroke}%
\pgfsetdash{}{0pt}%
\pgfpathmoveto{\pgfqpoint{4.648550in}{2.894499in}}%
\pgfpathcurveto{\pgfqpoint{4.659600in}{2.894499in}}{\pgfqpoint{4.670199in}{2.898889in}}{\pgfqpoint{4.678012in}{2.906703in}}%
\pgfpathcurveto{\pgfqpoint{4.685826in}{2.914516in}}{\pgfqpoint{4.690216in}{2.925115in}}{\pgfqpoint{4.690216in}{2.936165in}}%
\pgfpathcurveto{\pgfqpoint{4.690216in}{2.947216in}}{\pgfqpoint{4.685826in}{2.957815in}}{\pgfqpoint{4.678012in}{2.965628in}}%
\pgfpathcurveto{\pgfqpoint{4.670199in}{2.973442in}}{\pgfqpoint{4.659600in}{2.977832in}}{\pgfqpoint{4.648550in}{2.977832in}}%
\pgfpathcurveto{\pgfqpoint{4.637499in}{2.977832in}}{\pgfqpoint{4.626900in}{2.973442in}}{\pgfqpoint{4.619087in}{2.965628in}}%
\pgfpathcurveto{\pgfqpoint{4.611273in}{2.957815in}}{\pgfqpoint{4.606883in}{2.947216in}}{\pgfqpoint{4.606883in}{2.936165in}}%
\pgfpathcurveto{\pgfqpoint{4.606883in}{2.925115in}}{\pgfqpoint{4.611273in}{2.914516in}}{\pgfqpoint{4.619087in}{2.906703in}}%
\pgfpathcurveto{\pgfqpoint{4.626900in}{2.898889in}}{\pgfqpoint{4.637499in}{2.894499in}}{\pgfqpoint{4.648550in}{2.894499in}}%
\pgfpathclose%
\pgfusepath{stroke,fill}%
\end{pgfscope}%
\begin{pgfscope}%
\pgfpathrectangle{\pgfqpoint{0.526127in}{0.331635in}}{\pgfqpoint{9.300000in}{7.700000in}}%
\pgfusepath{clip}%
\pgfsetbuttcap%
\pgfsetroundjoin%
\definecolor{currentfill}{rgb}{0.980392,0.690196,0.894118}%
\pgfsetfillcolor{currentfill}%
\pgfsetlinewidth{0.481800pt}%
\definecolor{currentstroke}{rgb}{1.000000,1.000000,1.000000}%
\pgfsetstrokecolor{currentstroke}%
\pgfsetdash{}{0pt}%
\pgfpathmoveto{\pgfqpoint{6.518469in}{6.081978in}}%
\pgfpathcurveto{\pgfqpoint{6.529519in}{6.081978in}}{\pgfqpoint{6.540118in}{6.086368in}}{\pgfqpoint{6.547932in}{6.094182in}}%
\pgfpathcurveto{\pgfqpoint{6.555745in}{6.101995in}}{\pgfqpoint{6.560136in}{6.112594in}}{\pgfqpoint{6.560136in}{6.123645in}}%
\pgfpathcurveto{\pgfqpoint{6.560136in}{6.134695in}}{\pgfqpoint{6.555745in}{6.145294in}}{\pgfqpoint{6.547932in}{6.153107in}}%
\pgfpathcurveto{\pgfqpoint{6.540118in}{6.160921in}}{\pgfqpoint{6.529519in}{6.165311in}}{\pgfqpoint{6.518469in}{6.165311in}}%
\pgfpathcurveto{\pgfqpoint{6.507419in}{6.165311in}}{\pgfqpoint{6.496820in}{6.160921in}}{\pgfqpoint{6.489006in}{6.153107in}}%
\pgfpathcurveto{\pgfqpoint{6.481193in}{6.145294in}}{\pgfqpoint{6.476802in}{6.134695in}}{\pgfqpoint{6.476802in}{6.123645in}}%
\pgfpathcurveto{\pgfqpoint{6.476802in}{6.112594in}}{\pgfqpoint{6.481193in}{6.101995in}}{\pgfqpoint{6.489006in}{6.094182in}}%
\pgfpathcurveto{\pgfqpoint{6.496820in}{6.086368in}}{\pgfqpoint{6.507419in}{6.081978in}}{\pgfqpoint{6.518469in}{6.081978in}}%
\pgfpathclose%
\pgfusepath{stroke,fill}%
\end{pgfscope}%
\begin{pgfscope}%
\pgfpathrectangle{\pgfqpoint{0.526127in}{0.331635in}}{\pgfqpoint{9.300000in}{7.700000in}}%
\pgfusepath{clip}%
\pgfsetbuttcap%
\pgfsetroundjoin%
\definecolor{currentfill}{rgb}{0.980392,0.690196,0.894118}%
\pgfsetfillcolor{currentfill}%
\pgfsetlinewidth{0.481800pt}%
\definecolor{currentstroke}{rgb}{1.000000,1.000000,1.000000}%
\pgfsetstrokecolor{currentstroke}%
\pgfsetdash{}{0pt}%
\pgfpathmoveto{\pgfqpoint{8.083074in}{4.699371in}}%
\pgfpathcurveto{\pgfqpoint{8.094124in}{4.699371in}}{\pgfqpoint{8.104723in}{4.703761in}}{\pgfqpoint{8.112537in}{4.711575in}}%
\pgfpathcurveto{\pgfqpoint{8.120350in}{4.719388in}}{\pgfqpoint{8.124741in}{4.729987in}}{\pgfqpoint{8.124741in}{4.741037in}}%
\pgfpathcurveto{\pgfqpoint{8.124741in}{4.752088in}}{\pgfqpoint{8.120350in}{4.762687in}}{\pgfqpoint{8.112537in}{4.770500in}}%
\pgfpathcurveto{\pgfqpoint{8.104723in}{4.778314in}}{\pgfqpoint{8.094124in}{4.782704in}}{\pgfqpoint{8.083074in}{4.782704in}}%
\pgfpathcurveto{\pgfqpoint{8.072024in}{4.782704in}}{\pgfqpoint{8.061425in}{4.778314in}}{\pgfqpoint{8.053611in}{4.770500in}}%
\pgfpathcurveto{\pgfqpoint{8.045798in}{4.762687in}}{\pgfqpoint{8.041407in}{4.752088in}}{\pgfqpoint{8.041407in}{4.741037in}}%
\pgfpathcurveto{\pgfqpoint{8.041407in}{4.729987in}}{\pgfqpoint{8.045798in}{4.719388in}}{\pgfqpoint{8.053611in}{4.711575in}}%
\pgfpathcurveto{\pgfqpoint{8.061425in}{4.703761in}}{\pgfqpoint{8.072024in}{4.699371in}}{\pgfqpoint{8.083074in}{4.699371in}}%
\pgfpathclose%
\pgfusepath{stroke,fill}%
\end{pgfscope}%
\begin{pgfscope}%
\pgfpathrectangle{\pgfqpoint{0.526127in}{0.331635in}}{\pgfqpoint{9.300000in}{7.700000in}}%
\pgfusepath{clip}%
\pgfsetbuttcap%
\pgfsetroundjoin%
\definecolor{currentfill}{rgb}{0.980392,0.690196,0.894118}%
\pgfsetfillcolor{currentfill}%
\pgfsetlinewidth{0.481800pt}%
\definecolor{currentstroke}{rgb}{1.000000,1.000000,1.000000}%
\pgfsetstrokecolor{currentstroke}%
\pgfsetdash{}{0pt}%
\pgfpathmoveto{\pgfqpoint{7.058761in}{4.090315in}}%
\pgfpathcurveto{\pgfqpoint{7.069811in}{4.090315in}}{\pgfqpoint{7.080410in}{4.094705in}}{\pgfqpoint{7.088224in}{4.102518in}}%
\pgfpathcurveto{\pgfqpoint{7.096038in}{4.110332in}}{\pgfqpoint{7.100428in}{4.120931in}}{\pgfqpoint{7.100428in}{4.131981in}}%
\pgfpathcurveto{\pgfqpoint{7.100428in}{4.143031in}}{\pgfqpoint{7.096038in}{4.153630in}}{\pgfqpoint{7.088224in}{4.161444in}}%
\pgfpathcurveto{\pgfqpoint{7.080410in}{4.169258in}}{\pgfqpoint{7.069811in}{4.173648in}}{\pgfqpoint{7.058761in}{4.173648in}}%
\pgfpathcurveto{\pgfqpoint{7.047711in}{4.173648in}}{\pgfqpoint{7.037112in}{4.169258in}}{\pgfqpoint{7.029298in}{4.161444in}}%
\pgfpathcurveto{\pgfqpoint{7.021485in}{4.153630in}}{\pgfqpoint{7.017094in}{4.143031in}}{\pgfqpoint{7.017094in}{4.131981in}}%
\pgfpathcurveto{\pgfqpoint{7.017094in}{4.120931in}}{\pgfqpoint{7.021485in}{4.110332in}}{\pgfqpoint{7.029298in}{4.102518in}}%
\pgfpathcurveto{\pgfqpoint{7.037112in}{4.094705in}}{\pgfqpoint{7.047711in}{4.090315in}}{\pgfqpoint{7.058761in}{4.090315in}}%
\pgfpathclose%
\pgfusepath{stroke,fill}%
\end{pgfscope}%
\begin{pgfscope}%
\pgfpathrectangle{\pgfqpoint{0.526127in}{0.331635in}}{\pgfqpoint{9.300000in}{7.700000in}}%
\pgfusepath{clip}%
\pgfsetbuttcap%
\pgfsetroundjoin%
\definecolor{currentfill}{rgb}{0.980392,0.690196,0.894118}%
\pgfsetfillcolor{currentfill}%
\pgfsetlinewidth{0.481800pt}%
\definecolor{currentstroke}{rgb}{1.000000,1.000000,1.000000}%
\pgfsetstrokecolor{currentstroke}%
\pgfsetdash{}{0pt}%
\pgfpathmoveto{\pgfqpoint{8.288664in}{4.840301in}}%
\pgfpathcurveto{\pgfqpoint{8.299714in}{4.840301in}}{\pgfqpoint{8.310313in}{4.844691in}}{\pgfqpoint{8.318127in}{4.852504in}}%
\pgfpathcurveto{\pgfqpoint{8.325940in}{4.860318in}}{\pgfqpoint{8.330331in}{4.870917in}}{\pgfqpoint{8.330331in}{4.881967in}}%
\pgfpathcurveto{\pgfqpoint{8.330331in}{4.893017in}}{\pgfqpoint{8.325940in}{4.903616in}}{\pgfqpoint{8.318127in}{4.911430in}}%
\pgfpathcurveto{\pgfqpoint{8.310313in}{4.919244in}}{\pgfqpoint{8.299714in}{4.923634in}}{\pgfqpoint{8.288664in}{4.923634in}}%
\pgfpathcurveto{\pgfqpoint{8.277614in}{4.923634in}}{\pgfqpoint{8.267015in}{4.919244in}}{\pgfqpoint{8.259201in}{4.911430in}}%
\pgfpathcurveto{\pgfqpoint{8.251388in}{4.903616in}}{\pgfqpoint{8.246997in}{4.893017in}}{\pgfqpoint{8.246997in}{4.881967in}}%
\pgfpathcurveto{\pgfqpoint{8.246997in}{4.870917in}}{\pgfqpoint{8.251388in}{4.860318in}}{\pgfqpoint{8.259201in}{4.852504in}}%
\pgfpathcurveto{\pgfqpoint{8.267015in}{4.844691in}}{\pgfqpoint{8.277614in}{4.840301in}}{\pgfqpoint{8.288664in}{4.840301in}}%
\pgfpathclose%
\pgfusepath{stroke,fill}%
\end{pgfscope}%
\begin{pgfscope}%
\pgfpathrectangle{\pgfqpoint{0.526127in}{0.331635in}}{\pgfqpoint{9.300000in}{7.700000in}}%
\pgfusepath{clip}%
\pgfsetbuttcap%
\pgfsetroundjoin%
\definecolor{currentfill}{rgb}{0.980392,0.690196,0.894118}%
\pgfsetfillcolor{currentfill}%
\pgfsetlinewidth{0.481800pt}%
\definecolor{currentstroke}{rgb}{1.000000,1.000000,1.000000}%
\pgfsetstrokecolor{currentstroke}%
\pgfsetdash{}{0pt}%
\pgfpathmoveto{\pgfqpoint{8.365658in}{4.173936in}}%
\pgfpathcurveto{\pgfqpoint{8.376708in}{4.173936in}}{\pgfqpoint{8.387307in}{4.178326in}}{\pgfqpoint{8.395121in}{4.186139in}}%
\pgfpathcurveto{\pgfqpoint{8.402935in}{4.193953in}}{\pgfqpoint{8.407325in}{4.204552in}}{\pgfqpoint{8.407325in}{4.215602in}}%
\pgfpathcurveto{\pgfqpoint{8.407325in}{4.226652in}}{\pgfqpoint{8.402935in}{4.237251in}}{\pgfqpoint{8.395121in}{4.245065in}}%
\pgfpathcurveto{\pgfqpoint{8.387307in}{4.252879in}}{\pgfqpoint{8.376708in}{4.257269in}}{\pgfqpoint{8.365658in}{4.257269in}}%
\pgfpathcurveto{\pgfqpoint{8.354608in}{4.257269in}}{\pgfqpoint{8.344009in}{4.252879in}}{\pgfqpoint{8.336195in}{4.245065in}}%
\pgfpathcurveto{\pgfqpoint{8.328382in}{4.237251in}}{\pgfqpoint{8.323991in}{4.226652in}}{\pgfqpoint{8.323991in}{4.215602in}}%
\pgfpathcurveto{\pgfqpoint{8.323991in}{4.204552in}}{\pgfqpoint{8.328382in}{4.193953in}}{\pgfqpoint{8.336195in}{4.186139in}}%
\pgfpathcurveto{\pgfqpoint{8.344009in}{4.178326in}}{\pgfqpoint{8.354608in}{4.173936in}}{\pgfqpoint{8.365658in}{4.173936in}}%
\pgfpathclose%
\pgfusepath{stroke,fill}%
\end{pgfscope}%
\begin{pgfscope}%
\pgfpathrectangle{\pgfqpoint{0.526127in}{0.331635in}}{\pgfqpoint{9.300000in}{7.700000in}}%
\pgfusepath{clip}%
\pgfsetbuttcap%
\pgfsetroundjoin%
\definecolor{currentfill}{rgb}{0.980392,0.690196,0.894118}%
\pgfsetfillcolor{currentfill}%
\pgfsetlinewidth{0.481800pt}%
\definecolor{currentstroke}{rgb}{1.000000,1.000000,1.000000}%
\pgfsetstrokecolor{currentstroke}%
\pgfsetdash{}{0pt}%
\pgfpathmoveto{\pgfqpoint{5.586667in}{5.980161in}}%
\pgfpathcurveto{\pgfqpoint{5.597717in}{5.980161in}}{\pgfqpoint{5.608316in}{5.984552in}}{\pgfqpoint{5.616130in}{5.992365in}}%
\pgfpathcurveto{\pgfqpoint{5.623944in}{6.000179in}}{\pgfqpoint{5.628334in}{6.010778in}}{\pgfqpoint{5.628334in}{6.021828in}}%
\pgfpathcurveto{\pgfqpoint{5.628334in}{6.032878in}}{\pgfqpoint{5.623944in}{6.043477in}}{\pgfqpoint{5.616130in}{6.051291in}}%
\pgfpathcurveto{\pgfqpoint{5.608316in}{6.059104in}}{\pgfqpoint{5.597717in}{6.063495in}}{\pgfqpoint{5.586667in}{6.063495in}}%
\pgfpathcurveto{\pgfqpoint{5.575617in}{6.063495in}}{\pgfqpoint{5.565018in}{6.059104in}}{\pgfqpoint{5.557205in}{6.051291in}}%
\pgfpathcurveto{\pgfqpoint{5.549391in}{6.043477in}}{\pgfqpoint{5.545001in}{6.032878in}}{\pgfqpoint{5.545001in}{6.021828in}}%
\pgfpathcurveto{\pgfqpoint{5.545001in}{6.010778in}}{\pgfqpoint{5.549391in}{6.000179in}}{\pgfqpoint{5.557205in}{5.992365in}}%
\pgfpathcurveto{\pgfqpoint{5.565018in}{5.984552in}}{\pgfqpoint{5.575617in}{5.980161in}}{\pgfqpoint{5.586667in}{5.980161in}}%
\pgfpathclose%
\pgfusepath{stroke,fill}%
\end{pgfscope}%
\begin{pgfscope}%
\pgfpathrectangle{\pgfqpoint{0.526127in}{0.331635in}}{\pgfqpoint{9.300000in}{7.700000in}}%
\pgfusepath{clip}%
\pgfsetbuttcap%
\pgfsetroundjoin%
\definecolor{currentfill}{rgb}{0.980392,0.690196,0.894118}%
\pgfsetfillcolor{currentfill}%
\pgfsetlinewidth{0.481800pt}%
\definecolor{currentstroke}{rgb}{1.000000,1.000000,1.000000}%
\pgfsetstrokecolor{currentstroke}%
\pgfsetdash{}{0pt}%
\pgfpathmoveto{\pgfqpoint{7.411729in}{4.222944in}}%
\pgfpathcurveto{\pgfqpoint{7.422780in}{4.222944in}}{\pgfqpoint{7.433379in}{4.227334in}}{\pgfqpoint{7.441192in}{4.235148in}}%
\pgfpathcurveto{\pgfqpoint{7.449006in}{4.242961in}}{\pgfqpoint{7.453396in}{4.253560in}}{\pgfqpoint{7.453396in}{4.264611in}}%
\pgfpathcurveto{\pgfqpoint{7.453396in}{4.275661in}}{\pgfqpoint{7.449006in}{4.286260in}}{\pgfqpoint{7.441192in}{4.294073in}}%
\pgfpathcurveto{\pgfqpoint{7.433379in}{4.301887in}}{\pgfqpoint{7.422780in}{4.306277in}}{\pgfqpoint{7.411729in}{4.306277in}}%
\pgfpathcurveto{\pgfqpoint{7.400679in}{4.306277in}}{\pgfqpoint{7.390080in}{4.301887in}}{\pgfqpoint{7.382267in}{4.294073in}}%
\pgfpathcurveto{\pgfqpoint{7.374453in}{4.286260in}}{\pgfqpoint{7.370063in}{4.275661in}}{\pgfqpoint{7.370063in}{4.264611in}}%
\pgfpathcurveto{\pgfqpoint{7.370063in}{4.253560in}}{\pgfqpoint{7.374453in}{4.242961in}}{\pgfqpoint{7.382267in}{4.235148in}}%
\pgfpathcurveto{\pgfqpoint{7.390080in}{4.227334in}}{\pgfqpoint{7.400679in}{4.222944in}}{\pgfqpoint{7.411729in}{4.222944in}}%
\pgfpathclose%
\pgfusepath{stroke,fill}%
\end{pgfscope}%
\begin{pgfscope}%
\pgfpathrectangle{\pgfqpoint{0.526127in}{0.331635in}}{\pgfqpoint{9.300000in}{7.700000in}}%
\pgfusepath{clip}%
\pgfsetbuttcap%
\pgfsetroundjoin%
\definecolor{currentfill}{rgb}{0.980392,0.690196,0.894118}%
\pgfsetfillcolor{currentfill}%
\pgfsetlinewidth{0.481800pt}%
\definecolor{currentstroke}{rgb}{1.000000,1.000000,1.000000}%
\pgfsetstrokecolor{currentstroke}%
\pgfsetdash{}{0pt}%
\pgfpathmoveto{\pgfqpoint{5.552197in}{5.456007in}}%
\pgfpathcurveto{\pgfqpoint{5.563247in}{5.456007in}}{\pgfqpoint{5.573846in}{5.460398in}}{\pgfqpoint{5.581660in}{5.468211in}}%
\pgfpathcurveto{\pgfqpoint{5.589473in}{5.476025in}}{\pgfqpoint{5.593864in}{5.486624in}}{\pgfqpoint{5.593864in}{5.497674in}}%
\pgfpathcurveto{\pgfqpoint{5.593864in}{5.508724in}}{\pgfqpoint{5.589473in}{5.519323in}}{\pgfqpoint{5.581660in}{5.527137in}}%
\pgfpathcurveto{\pgfqpoint{5.573846in}{5.534951in}}{\pgfqpoint{5.563247in}{5.539341in}}{\pgfqpoint{5.552197in}{5.539341in}}%
\pgfpathcurveto{\pgfqpoint{5.541147in}{5.539341in}}{\pgfqpoint{5.530548in}{5.534951in}}{\pgfqpoint{5.522734in}{5.527137in}}%
\pgfpathcurveto{\pgfqpoint{5.514921in}{5.519323in}}{\pgfqpoint{5.510530in}{5.508724in}}{\pgfqpoint{5.510530in}{5.497674in}}%
\pgfpathcurveto{\pgfqpoint{5.510530in}{5.486624in}}{\pgfqpoint{5.514921in}{5.476025in}}{\pgfqpoint{5.522734in}{5.468211in}}%
\pgfpathcurveto{\pgfqpoint{5.530548in}{5.460398in}}{\pgfqpoint{5.541147in}{5.456007in}}{\pgfqpoint{5.552197in}{5.456007in}}%
\pgfpathclose%
\pgfusepath{stroke,fill}%
\end{pgfscope}%
\begin{pgfscope}%
\pgfpathrectangle{\pgfqpoint{0.526127in}{0.331635in}}{\pgfqpoint{9.300000in}{7.700000in}}%
\pgfusepath{clip}%
\pgfsetbuttcap%
\pgfsetroundjoin%
\definecolor{currentfill}{rgb}{0.980392,0.690196,0.894118}%
\pgfsetfillcolor{currentfill}%
\pgfsetlinewidth{0.481800pt}%
\definecolor{currentstroke}{rgb}{1.000000,1.000000,1.000000}%
\pgfsetstrokecolor{currentstroke}%
\pgfsetdash{}{0pt}%
\pgfpathmoveto{\pgfqpoint{5.112613in}{5.981865in}}%
\pgfpathcurveto{\pgfqpoint{5.123663in}{5.981865in}}{\pgfqpoint{5.134262in}{5.986255in}}{\pgfqpoint{5.142076in}{5.994069in}}%
\pgfpathcurveto{\pgfqpoint{5.149889in}{6.001882in}}{\pgfqpoint{5.154280in}{6.012481in}}{\pgfqpoint{5.154280in}{6.023531in}}%
\pgfpathcurveto{\pgfqpoint{5.154280in}{6.034582in}}{\pgfqpoint{5.149889in}{6.045181in}}{\pgfqpoint{5.142076in}{6.052994in}}%
\pgfpathcurveto{\pgfqpoint{5.134262in}{6.060808in}}{\pgfqpoint{5.123663in}{6.065198in}}{\pgfqpoint{5.112613in}{6.065198in}}%
\pgfpathcurveto{\pgfqpoint{5.101563in}{6.065198in}}{\pgfqpoint{5.090964in}{6.060808in}}{\pgfqpoint{5.083150in}{6.052994in}}%
\pgfpathcurveto{\pgfqpoint{5.075337in}{6.045181in}}{\pgfqpoint{5.070946in}{6.034582in}}{\pgfqpoint{5.070946in}{6.023531in}}%
\pgfpathcurveto{\pgfqpoint{5.070946in}{6.012481in}}{\pgfqpoint{5.075337in}{6.001882in}}{\pgfqpoint{5.083150in}{5.994069in}}%
\pgfpathcurveto{\pgfqpoint{5.090964in}{5.986255in}}{\pgfqpoint{5.101563in}{5.981865in}}{\pgfqpoint{5.112613in}{5.981865in}}%
\pgfpathclose%
\pgfusepath{stroke,fill}%
\end{pgfscope}%
\begin{pgfscope}%
\pgfpathrectangle{\pgfqpoint{0.526127in}{0.331635in}}{\pgfqpoint{9.300000in}{7.700000in}}%
\pgfusepath{clip}%
\pgfsetbuttcap%
\pgfsetroundjoin%
\definecolor{currentfill}{rgb}{0.980392,0.690196,0.894118}%
\pgfsetfillcolor{currentfill}%
\pgfsetlinewidth{0.481800pt}%
\definecolor{currentstroke}{rgb}{1.000000,1.000000,1.000000}%
\pgfsetstrokecolor{currentstroke}%
\pgfsetdash{}{0pt}%
\pgfpathmoveto{\pgfqpoint{3.785574in}{6.619565in}}%
\pgfpathcurveto{\pgfqpoint{3.796624in}{6.619565in}}{\pgfqpoint{3.807223in}{6.623955in}}{\pgfqpoint{3.815037in}{6.631768in}}%
\pgfpathcurveto{\pgfqpoint{3.822850in}{6.639582in}}{\pgfqpoint{3.827241in}{6.650181in}}{\pgfqpoint{3.827241in}{6.661231in}}%
\pgfpathcurveto{\pgfqpoint{3.827241in}{6.672281in}}{\pgfqpoint{3.822850in}{6.682880in}}{\pgfqpoint{3.815037in}{6.690694in}}%
\pgfpathcurveto{\pgfqpoint{3.807223in}{6.698508in}}{\pgfqpoint{3.796624in}{6.702898in}}{\pgfqpoint{3.785574in}{6.702898in}}%
\pgfpathcurveto{\pgfqpoint{3.774524in}{6.702898in}}{\pgfqpoint{3.763925in}{6.698508in}}{\pgfqpoint{3.756111in}{6.690694in}}%
\pgfpathcurveto{\pgfqpoint{3.748298in}{6.682880in}}{\pgfqpoint{3.743907in}{6.672281in}}{\pgfqpoint{3.743907in}{6.661231in}}%
\pgfpathcurveto{\pgfqpoint{3.743907in}{6.650181in}}{\pgfqpoint{3.748298in}{6.639582in}}{\pgfqpoint{3.756111in}{6.631768in}}%
\pgfpathcurveto{\pgfqpoint{3.763925in}{6.623955in}}{\pgfqpoint{3.774524in}{6.619565in}}{\pgfqpoint{3.785574in}{6.619565in}}%
\pgfpathclose%
\pgfusepath{stroke,fill}%
\end{pgfscope}%
\begin{pgfscope}%
\pgfpathrectangle{\pgfqpoint{0.526127in}{0.331635in}}{\pgfqpoint{9.300000in}{7.700000in}}%
\pgfusepath{clip}%
\pgfsetbuttcap%
\pgfsetroundjoin%
\definecolor{currentfill}{rgb}{0.980392,0.690196,0.894118}%
\pgfsetfillcolor{currentfill}%
\pgfsetlinewidth{0.481800pt}%
\definecolor{currentstroke}{rgb}{1.000000,1.000000,1.000000}%
\pgfsetstrokecolor{currentstroke}%
\pgfsetdash{}{0pt}%
\pgfpathmoveto{\pgfqpoint{3.136695in}{6.197030in}}%
\pgfpathcurveto{\pgfqpoint{3.147746in}{6.197030in}}{\pgfqpoint{3.158345in}{6.201420in}}{\pgfqpoint{3.166158in}{6.209234in}}%
\pgfpathcurveto{\pgfqpoint{3.173972in}{6.217047in}}{\pgfqpoint{3.178362in}{6.227646in}}{\pgfqpoint{3.178362in}{6.238697in}}%
\pgfpathcurveto{\pgfqpoint{3.178362in}{6.249747in}}{\pgfqpoint{3.173972in}{6.260346in}}{\pgfqpoint{3.166158in}{6.268159in}}%
\pgfpathcurveto{\pgfqpoint{3.158345in}{6.275973in}}{\pgfqpoint{3.147746in}{6.280363in}}{\pgfqpoint{3.136695in}{6.280363in}}%
\pgfpathcurveto{\pgfqpoint{3.125645in}{6.280363in}}{\pgfqpoint{3.115046in}{6.275973in}}{\pgfqpoint{3.107233in}{6.268159in}}%
\pgfpathcurveto{\pgfqpoint{3.099419in}{6.260346in}}{\pgfqpoint{3.095029in}{6.249747in}}{\pgfqpoint{3.095029in}{6.238697in}}%
\pgfpathcurveto{\pgfqpoint{3.095029in}{6.227646in}}{\pgfqpoint{3.099419in}{6.217047in}}{\pgfqpoint{3.107233in}{6.209234in}}%
\pgfpathcurveto{\pgfqpoint{3.115046in}{6.201420in}}{\pgfqpoint{3.125645in}{6.197030in}}{\pgfqpoint{3.136695in}{6.197030in}}%
\pgfpathclose%
\pgfusepath{stroke,fill}%
\end{pgfscope}%
\begin{pgfscope}%
\pgfpathrectangle{\pgfqpoint{0.526127in}{0.331635in}}{\pgfqpoint{9.300000in}{7.700000in}}%
\pgfusepath{clip}%
\pgfsetbuttcap%
\pgfsetroundjoin%
\definecolor{currentfill}{rgb}{0.980392,0.690196,0.894118}%
\pgfsetfillcolor{currentfill}%
\pgfsetlinewidth{0.481800pt}%
\definecolor{currentstroke}{rgb}{1.000000,1.000000,1.000000}%
\pgfsetstrokecolor{currentstroke}%
\pgfsetdash{}{0pt}%
\pgfpathmoveto{\pgfqpoint{8.338475in}{3.480006in}}%
\pgfpathcurveto{\pgfqpoint{8.349526in}{3.480006in}}{\pgfqpoint{8.360125in}{3.484396in}}{\pgfqpoint{8.367938in}{3.492210in}}%
\pgfpathcurveto{\pgfqpoint{8.375752in}{3.500024in}}{\pgfqpoint{8.380142in}{3.510623in}}{\pgfqpoint{8.380142in}{3.521673in}}%
\pgfpathcurveto{\pgfqpoint{8.380142in}{3.532723in}}{\pgfqpoint{8.375752in}{3.543322in}}{\pgfqpoint{8.367938in}{3.551136in}}%
\pgfpathcurveto{\pgfqpoint{8.360125in}{3.558949in}}{\pgfqpoint{8.349526in}{3.563340in}}{\pgfqpoint{8.338475in}{3.563340in}}%
\pgfpathcurveto{\pgfqpoint{8.327425in}{3.563340in}}{\pgfqpoint{8.316826in}{3.558949in}}{\pgfqpoint{8.309013in}{3.551136in}}%
\pgfpathcurveto{\pgfqpoint{8.301199in}{3.543322in}}{\pgfqpoint{8.296809in}{3.532723in}}{\pgfqpoint{8.296809in}{3.521673in}}%
\pgfpathcurveto{\pgfqpoint{8.296809in}{3.510623in}}{\pgfqpoint{8.301199in}{3.500024in}}{\pgfqpoint{8.309013in}{3.492210in}}%
\pgfpathcurveto{\pgfqpoint{8.316826in}{3.484396in}}{\pgfqpoint{8.327425in}{3.480006in}}{\pgfqpoint{8.338475in}{3.480006in}}%
\pgfpathclose%
\pgfusepath{stroke,fill}%
\end{pgfscope}%
\begin{pgfscope}%
\pgfpathrectangle{\pgfqpoint{0.526127in}{0.331635in}}{\pgfqpoint{9.300000in}{7.700000in}}%
\pgfusepath{clip}%
\pgfsetbuttcap%
\pgfsetroundjoin%
\definecolor{currentfill}{rgb}{0.980392,0.690196,0.894118}%
\pgfsetfillcolor{currentfill}%
\pgfsetlinewidth{0.481800pt}%
\definecolor{currentstroke}{rgb}{1.000000,1.000000,1.000000}%
\pgfsetstrokecolor{currentstroke}%
\pgfsetdash{}{0pt}%
\pgfpathmoveto{\pgfqpoint{2.725282in}{4.706332in}}%
\pgfpathcurveto{\pgfqpoint{2.736332in}{4.706332in}}{\pgfqpoint{2.746931in}{4.710723in}}{\pgfqpoint{2.754745in}{4.718536in}}%
\pgfpathcurveto{\pgfqpoint{2.762559in}{4.726350in}}{\pgfqpoint{2.766949in}{4.736949in}}{\pgfqpoint{2.766949in}{4.747999in}}%
\pgfpathcurveto{\pgfqpoint{2.766949in}{4.759049in}}{\pgfqpoint{2.762559in}{4.769648in}}{\pgfqpoint{2.754745in}{4.777462in}}%
\pgfpathcurveto{\pgfqpoint{2.746931in}{4.785275in}}{\pgfqpoint{2.736332in}{4.789666in}}{\pgfqpoint{2.725282in}{4.789666in}}%
\pgfpathcurveto{\pgfqpoint{2.714232in}{4.789666in}}{\pgfqpoint{2.703633in}{4.785275in}}{\pgfqpoint{2.695819in}{4.777462in}}%
\pgfpathcurveto{\pgfqpoint{2.688006in}{4.769648in}}{\pgfqpoint{2.683616in}{4.759049in}}{\pgfqpoint{2.683616in}{4.747999in}}%
\pgfpathcurveto{\pgfqpoint{2.683616in}{4.736949in}}{\pgfqpoint{2.688006in}{4.726350in}}{\pgfqpoint{2.695819in}{4.718536in}}%
\pgfpathcurveto{\pgfqpoint{2.703633in}{4.710723in}}{\pgfqpoint{2.714232in}{4.706332in}}{\pgfqpoint{2.725282in}{4.706332in}}%
\pgfpathclose%
\pgfusepath{stroke,fill}%
\end{pgfscope}%
\begin{pgfscope}%
\pgfpathrectangle{\pgfqpoint{0.526127in}{0.331635in}}{\pgfqpoint{9.300000in}{7.700000in}}%
\pgfusepath{clip}%
\pgfsetbuttcap%
\pgfsetroundjoin%
\definecolor{currentfill}{rgb}{0.980392,0.690196,0.894118}%
\pgfsetfillcolor{currentfill}%
\pgfsetlinewidth{0.481800pt}%
\definecolor{currentstroke}{rgb}{1.000000,1.000000,1.000000}%
\pgfsetstrokecolor{currentstroke}%
\pgfsetdash{}{0pt}%
\pgfpathmoveto{\pgfqpoint{5.804992in}{2.150742in}}%
\pgfpathcurveto{\pgfqpoint{5.816042in}{2.150742in}}{\pgfqpoint{5.826641in}{2.155133in}}{\pgfqpoint{5.834454in}{2.162946in}}%
\pgfpathcurveto{\pgfqpoint{5.842268in}{2.170760in}}{\pgfqpoint{5.846658in}{2.181359in}}{\pgfqpoint{5.846658in}{2.192409in}}%
\pgfpathcurveto{\pgfqpoint{5.846658in}{2.203459in}}{\pgfqpoint{5.842268in}{2.214058in}}{\pgfqpoint{5.834454in}{2.221872in}}%
\pgfpathcurveto{\pgfqpoint{5.826641in}{2.229686in}}{\pgfqpoint{5.816042in}{2.234076in}}{\pgfqpoint{5.804992in}{2.234076in}}%
\pgfpathcurveto{\pgfqpoint{5.793942in}{2.234076in}}{\pgfqpoint{5.783342in}{2.229686in}}{\pgfqpoint{5.775529in}{2.221872in}}%
\pgfpathcurveto{\pgfqpoint{5.767715in}{2.214058in}}{\pgfqpoint{5.763325in}{2.203459in}}{\pgfqpoint{5.763325in}{2.192409in}}%
\pgfpathcurveto{\pgfqpoint{5.763325in}{2.181359in}}{\pgfqpoint{5.767715in}{2.170760in}}{\pgfqpoint{5.775529in}{2.162946in}}%
\pgfpathcurveto{\pgfqpoint{5.783342in}{2.155133in}}{\pgfqpoint{5.793942in}{2.150742in}}{\pgfqpoint{5.804992in}{2.150742in}}%
\pgfpathclose%
\pgfusepath{stroke,fill}%
\end{pgfscope}%
\begin{pgfscope}%
\pgfpathrectangle{\pgfqpoint{0.526127in}{0.331635in}}{\pgfqpoint{9.300000in}{7.700000in}}%
\pgfusepath{clip}%
\pgfsetbuttcap%
\pgfsetroundjoin%
\definecolor{currentfill}{rgb}{0.980392,0.690196,0.894118}%
\pgfsetfillcolor{currentfill}%
\pgfsetlinewidth{0.481800pt}%
\definecolor{currentstroke}{rgb}{1.000000,1.000000,1.000000}%
\pgfsetstrokecolor{currentstroke}%
\pgfsetdash{}{0pt}%
\pgfpathmoveto{\pgfqpoint{4.387621in}{6.270253in}}%
\pgfpathcurveto{\pgfqpoint{4.398672in}{6.270253in}}{\pgfqpoint{4.409271in}{6.274643in}}{\pgfqpoint{4.417084in}{6.282457in}}%
\pgfpathcurveto{\pgfqpoint{4.424898in}{6.290270in}}{\pgfqpoint{4.429288in}{6.300869in}}{\pgfqpoint{4.429288in}{6.311919in}}%
\pgfpathcurveto{\pgfqpoint{4.429288in}{6.322970in}}{\pgfqpoint{4.424898in}{6.333569in}}{\pgfqpoint{4.417084in}{6.341382in}}%
\pgfpathcurveto{\pgfqpoint{4.409271in}{6.349196in}}{\pgfqpoint{4.398672in}{6.353586in}}{\pgfqpoint{4.387621in}{6.353586in}}%
\pgfpathcurveto{\pgfqpoint{4.376571in}{6.353586in}}{\pgfqpoint{4.365972in}{6.349196in}}{\pgfqpoint{4.358159in}{6.341382in}}%
\pgfpathcurveto{\pgfqpoint{4.350345in}{6.333569in}}{\pgfqpoint{4.345955in}{6.322970in}}{\pgfqpoint{4.345955in}{6.311919in}}%
\pgfpathcurveto{\pgfqpoint{4.345955in}{6.300869in}}{\pgfqpoint{4.350345in}{6.290270in}}{\pgfqpoint{4.358159in}{6.282457in}}%
\pgfpathcurveto{\pgfqpoint{4.365972in}{6.274643in}}{\pgfqpoint{4.376571in}{6.270253in}}{\pgfqpoint{4.387621in}{6.270253in}}%
\pgfpathclose%
\pgfusepath{stroke,fill}%
\end{pgfscope}%
\begin{pgfscope}%
\pgfpathrectangle{\pgfqpoint{0.526127in}{0.331635in}}{\pgfqpoint{9.300000in}{7.700000in}}%
\pgfusepath{clip}%
\pgfsetbuttcap%
\pgfsetroundjoin%
\definecolor{currentfill}{rgb}{0.980392,0.690196,0.894118}%
\pgfsetfillcolor{currentfill}%
\pgfsetlinewidth{0.481800pt}%
\definecolor{currentstroke}{rgb}{1.000000,1.000000,1.000000}%
\pgfsetstrokecolor{currentstroke}%
\pgfsetdash{}{0pt}%
\pgfpathmoveto{\pgfqpoint{4.353160in}{5.759122in}}%
\pgfpathcurveto{\pgfqpoint{4.364210in}{5.759122in}}{\pgfqpoint{4.374810in}{5.763512in}}{\pgfqpoint{4.382623in}{5.771326in}}%
\pgfpathcurveto{\pgfqpoint{4.390437in}{5.779140in}}{\pgfqpoint{4.394827in}{5.789739in}}{\pgfqpoint{4.394827in}{5.800789in}}%
\pgfpathcurveto{\pgfqpoint{4.394827in}{5.811839in}}{\pgfqpoint{4.390437in}{5.822438in}}{\pgfqpoint{4.382623in}{5.830252in}}%
\pgfpathcurveto{\pgfqpoint{4.374810in}{5.838065in}}{\pgfqpoint{4.364210in}{5.842456in}}{\pgfqpoint{4.353160in}{5.842456in}}%
\pgfpathcurveto{\pgfqpoint{4.342110in}{5.842456in}}{\pgfqpoint{4.331511in}{5.838065in}}{\pgfqpoint{4.323698in}{5.830252in}}%
\pgfpathcurveto{\pgfqpoint{4.315884in}{5.822438in}}{\pgfqpoint{4.311494in}{5.811839in}}{\pgfqpoint{4.311494in}{5.800789in}}%
\pgfpathcurveto{\pgfqpoint{4.311494in}{5.789739in}}{\pgfqpoint{4.315884in}{5.779140in}}{\pgfqpoint{4.323698in}{5.771326in}}%
\pgfpathcurveto{\pgfqpoint{4.331511in}{5.763512in}}{\pgfqpoint{4.342110in}{5.759122in}}{\pgfqpoint{4.353160in}{5.759122in}}%
\pgfpathclose%
\pgfusepath{stroke,fill}%
\end{pgfscope}%
\begin{pgfscope}%
\pgfpathrectangle{\pgfqpoint{0.526127in}{0.331635in}}{\pgfqpoint{9.300000in}{7.700000in}}%
\pgfusepath{clip}%
\pgfsetbuttcap%
\pgfsetroundjoin%
\definecolor{currentfill}{rgb}{0.980392,0.690196,0.894118}%
\pgfsetfillcolor{currentfill}%
\pgfsetlinewidth{0.481800pt}%
\definecolor{currentstroke}{rgb}{1.000000,1.000000,1.000000}%
\pgfsetstrokecolor{currentstroke}%
\pgfsetdash{}{0pt}%
\pgfpathmoveto{\pgfqpoint{4.693280in}{5.787436in}}%
\pgfpathcurveto{\pgfqpoint{4.704330in}{5.787436in}}{\pgfqpoint{4.714929in}{5.791826in}}{\pgfqpoint{4.722743in}{5.799640in}}%
\pgfpathcurveto{\pgfqpoint{4.730556in}{5.807453in}}{\pgfqpoint{4.734947in}{5.818052in}}{\pgfqpoint{4.734947in}{5.829103in}}%
\pgfpathcurveto{\pgfqpoint{4.734947in}{5.840153in}}{\pgfqpoint{4.730556in}{5.850752in}}{\pgfqpoint{4.722743in}{5.858565in}}%
\pgfpathcurveto{\pgfqpoint{4.714929in}{5.866379in}}{\pgfqpoint{4.704330in}{5.870769in}}{\pgfqpoint{4.693280in}{5.870769in}}%
\pgfpathcurveto{\pgfqpoint{4.682230in}{5.870769in}}{\pgfqpoint{4.671631in}{5.866379in}}{\pgfqpoint{4.663817in}{5.858565in}}%
\pgfpathcurveto{\pgfqpoint{4.656003in}{5.850752in}}{\pgfqpoint{4.651613in}{5.840153in}}{\pgfqpoint{4.651613in}{5.829103in}}%
\pgfpathcurveto{\pgfqpoint{4.651613in}{5.818052in}}{\pgfqpoint{4.656003in}{5.807453in}}{\pgfqpoint{4.663817in}{5.799640in}}%
\pgfpathcurveto{\pgfqpoint{4.671631in}{5.791826in}}{\pgfqpoint{4.682230in}{5.787436in}}{\pgfqpoint{4.693280in}{5.787436in}}%
\pgfpathclose%
\pgfusepath{stroke,fill}%
\end{pgfscope}%
\begin{pgfscope}%
\pgfpathrectangle{\pgfqpoint{0.526127in}{0.331635in}}{\pgfqpoint{9.300000in}{7.700000in}}%
\pgfusepath{clip}%
\pgfsetbuttcap%
\pgfsetroundjoin%
\definecolor{currentfill}{rgb}{0.980392,0.690196,0.894118}%
\pgfsetfillcolor{currentfill}%
\pgfsetlinewidth{0.481800pt}%
\definecolor{currentstroke}{rgb}{1.000000,1.000000,1.000000}%
\pgfsetstrokecolor{currentstroke}%
\pgfsetdash{}{0pt}%
\pgfpathmoveto{\pgfqpoint{9.403399in}{5.614452in}}%
\pgfpathcurveto{\pgfqpoint{9.414450in}{5.614452in}}{\pgfqpoint{9.425049in}{5.618842in}}{\pgfqpoint{9.432862in}{5.626656in}}%
\pgfpathcurveto{\pgfqpoint{9.440676in}{5.634469in}}{\pgfqpoint{9.445066in}{5.645068in}}{\pgfqpoint{9.445066in}{5.656119in}}%
\pgfpathcurveto{\pgfqpoint{9.445066in}{5.667169in}}{\pgfqpoint{9.440676in}{5.677768in}}{\pgfqpoint{9.432862in}{5.685581in}}%
\pgfpathcurveto{\pgfqpoint{9.425049in}{5.693395in}}{\pgfqpoint{9.414450in}{5.697785in}}{\pgfqpoint{9.403399in}{5.697785in}}%
\pgfpathcurveto{\pgfqpoint{9.392349in}{5.697785in}}{\pgfqpoint{9.381750in}{5.693395in}}{\pgfqpoint{9.373937in}{5.685581in}}%
\pgfpathcurveto{\pgfqpoint{9.366123in}{5.677768in}}{\pgfqpoint{9.361733in}{5.667169in}}{\pgfqpoint{9.361733in}{5.656119in}}%
\pgfpathcurveto{\pgfqpoint{9.361733in}{5.645068in}}{\pgfqpoint{9.366123in}{5.634469in}}{\pgfqpoint{9.373937in}{5.626656in}}%
\pgfpathcurveto{\pgfqpoint{9.381750in}{5.618842in}}{\pgfqpoint{9.392349in}{5.614452in}}{\pgfqpoint{9.403399in}{5.614452in}}%
\pgfpathclose%
\pgfusepath{stroke,fill}%
\end{pgfscope}%
\begin{pgfscope}%
\pgfpathrectangle{\pgfqpoint{0.526127in}{0.331635in}}{\pgfqpoint{9.300000in}{7.700000in}}%
\pgfusepath{clip}%
\pgfsetbuttcap%
\pgfsetroundjoin%
\definecolor{currentfill}{rgb}{0.980392,0.690196,0.894118}%
\pgfsetfillcolor{currentfill}%
\pgfsetlinewidth{0.481800pt}%
\definecolor{currentstroke}{rgb}{1.000000,1.000000,1.000000}%
\pgfsetstrokecolor{currentstroke}%
\pgfsetdash{}{0pt}%
\pgfpathmoveto{\pgfqpoint{4.320810in}{3.309729in}}%
\pgfpathcurveto{\pgfqpoint{4.331861in}{3.309729in}}{\pgfqpoint{4.342460in}{3.314120in}}{\pgfqpoint{4.350273in}{3.321933in}}%
\pgfpathcurveto{\pgfqpoint{4.358087in}{3.329747in}}{\pgfqpoint{4.362477in}{3.340346in}}{\pgfqpoint{4.362477in}{3.351396in}}%
\pgfpathcurveto{\pgfqpoint{4.362477in}{3.362446in}}{\pgfqpoint{4.358087in}{3.373045in}}{\pgfqpoint{4.350273in}{3.380859in}}%
\pgfpathcurveto{\pgfqpoint{4.342460in}{3.388672in}}{\pgfqpoint{4.331861in}{3.393063in}}{\pgfqpoint{4.320810in}{3.393063in}}%
\pgfpathcurveto{\pgfqpoint{4.309760in}{3.393063in}}{\pgfqpoint{4.299161in}{3.388672in}}{\pgfqpoint{4.291348in}{3.380859in}}%
\pgfpathcurveto{\pgfqpoint{4.283534in}{3.373045in}}{\pgfqpoint{4.279144in}{3.362446in}}{\pgfqpoint{4.279144in}{3.351396in}}%
\pgfpathcurveto{\pgfqpoint{4.279144in}{3.340346in}}{\pgfqpoint{4.283534in}{3.329747in}}{\pgfqpoint{4.291348in}{3.321933in}}%
\pgfpathcurveto{\pgfqpoint{4.299161in}{3.314120in}}{\pgfqpoint{4.309760in}{3.309729in}}{\pgfqpoint{4.320810in}{3.309729in}}%
\pgfpathclose%
\pgfusepath{stroke,fill}%
\end{pgfscope}%
\begin{pgfscope}%
\pgfpathrectangle{\pgfqpoint{0.526127in}{0.331635in}}{\pgfqpoint{9.300000in}{7.700000in}}%
\pgfusepath{clip}%
\pgfsetbuttcap%
\pgfsetroundjoin%
\definecolor{currentfill}{rgb}{0.980392,0.690196,0.894118}%
\pgfsetfillcolor{currentfill}%
\pgfsetlinewidth{0.481800pt}%
\definecolor{currentstroke}{rgb}{1.000000,1.000000,1.000000}%
\pgfsetstrokecolor{currentstroke}%
\pgfsetdash{}{0pt}%
\pgfpathmoveto{\pgfqpoint{1.703182in}{5.215231in}}%
\pgfpathcurveto{\pgfqpoint{1.714232in}{5.215231in}}{\pgfqpoint{1.724831in}{5.219621in}}{\pgfqpoint{1.732644in}{5.227435in}}%
\pgfpathcurveto{\pgfqpoint{1.740458in}{5.235248in}}{\pgfqpoint{1.744848in}{5.245847in}}{\pgfqpoint{1.744848in}{5.256898in}}%
\pgfpathcurveto{\pgfqpoint{1.744848in}{5.267948in}}{\pgfqpoint{1.740458in}{5.278547in}}{\pgfqpoint{1.732644in}{5.286360in}}%
\pgfpathcurveto{\pgfqpoint{1.724831in}{5.294174in}}{\pgfqpoint{1.714232in}{5.298564in}}{\pgfqpoint{1.703182in}{5.298564in}}%
\pgfpathcurveto{\pgfqpoint{1.692131in}{5.298564in}}{\pgfqpoint{1.681532in}{5.294174in}}{\pgfqpoint{1.673719in}{5.286360in}}%
\pgfpathcurveto{\pgfqpoint{1.665905in}{5.278547in}}{\pgfqpoint{1.661515in}{5.267948in}}{\pgfqpoint{1.661515in}{5.256898in}}%
\pgfpathcurveto{\pgfqpoint{1.661515in}{5.245847in}}{\pgfqpoint{1.665905in}{5.235248in}}{\pgfqpoint{1.673719in}{5.227435in}}%
\pgfpathcurveto{\pgfqpoint{1.681532in}{5.219621in}}{\pgfqpoint{1.692131in}{5.215231in}}{\pgfqpoint{1.703182in}{5.215231in}}%
\pgfpathclose%
\pgfusepath{stroke,fill}%
\end{pgfscope}%
\begin{pgfscope}%
\pgfpathrectangle{\pgfqpoint{0.526127in}{0.331635in}}{\pgfqpoint{9.300000in}{7.700000in}}%
\pgfusepath{clip}%
\pgfsetbuttcap%
\pgfsetroundjoin%
\definecolor{currentfill}{rgb}{0.980392,0.690196,0.894118}%
\pgfsetfillcolor{currentfill}%
\pgfsetlinewidth{0.481800pt}%
\definecolor{currentstroke}{rgb}{1.000000,1.000000,1.000000}%
\pgfsetstrokecolor{currentstroke}%
\pgfsetdash{}{0pt}%
\pgfpathmoveto{\pgfqpoint{4.829760in}{6.437133in}}%
\pgfpathcurveto{\pgfqpoint{4.840811in}{6.437133in}}{\pgfqpoint{4.851410in}{6.441524in}}{\pgfqpoint{4.859223in}{6.449337in}}%
\pgfpathcurveto{\pgfqpoint{4.867037in}{6.457151in}}{\pgfqpoint{4.871427in}{6.467750in}}{\pgfqpoint{4.871427in}{6.478800in}}%
\pgfpathcurveto{\pgfqpoint{4.871427in}{6.489850in}}{\pgfqpoint{4.867037in}{6.500449in}}{\pgfqpoint{4.859223in}{6.508263in}}%
\pgfpathcurveto{\pgfqpoint{4.851410in}{6.516076in}}{\pgfqpoint{4.840811in}{6.520467in}}{\pgfqpoint{4.829760in}{6.520467in}}%
\pgfpathcurveto{\pgfqpoint{4.818710in}{6.520467in}}{\pgfqpoint{4.808111in}{6.516076in}}{\pgfqpoint{4.800298in}{6.508263in}}%
\pgfpathcurveto{\pgfqpoint{4.792484in}{6.500449in}}{\pgfqpoint{4.788094in}{6.489850in}}{\pgfqpoint{4.788094in}{6.478800in}}%
\pgfpathcurveto{\pgfqpoint{4.788094in}{6.467750in}}{\pgfqpoint{4.792484in}{6.457151in}}{\pgfqpoint{4.800298in}{6.449337in}}%
\pgfpathcurveto{\pgfqpoint{4.808111in}{6.441524in}}{\pgfqpoint{4.818710in}{6.437133in}}{\pgfqpoint{4.829760in}{6.437133in}}%
\pgfpathclose%
\pgfusepath{stroke,fill}%
\end{pgfscope}%
\begin{pgfscope}%
\pgfpathrectangle{\pgfqpoint{0.526127in}{0.331635in}}{\pgfqpoint{9.300000in}{7.700000in}}%
\pgfusepath{clip}%
\pgfsetbuttcap%
\pgfsetroundjoin%
\definecolor{currentfill}{rgb}{0.980392,0.690196,0.894118}%
\pgfsetfillcolor{currentfill}%
\pgfsetlinewidth{0.481800pt}%
\definecolor{currentstroke}{rgb}{1.000000,1.000000,1.000000}%
\pgfsetstrokecolor{currentstroke}%
\pgfsetdash{}{0pt}%
\pgfpathmoveto{\pgfqpoint{5.312702in}{4.293977in}}%
\pgfpathcurveto{\pgfqpoint{5.323753in}{4.293977in}}{\pgfqpoint{5.334352in}{4.298367in}}{\pgfqpoint{5.342165in}{4.306181in}}%
\pgfpathcurveto{\pgfqpoint{5.349979in}{4.313995in}}{\pgfqpoint{5.354369in}{4.324594in}}{\pgfqpoint{5.354369in}{4.335644in}}%
\pgfpathcurveto{\pgfqpoint{5.354369in}{4.346694in}}{\pgfqpoint{5.349979in}{4.357293in}}{\pgfqpoint{5.342165in}{4.365107in}}%
\pgfpathcurveto{\pgfqpoint{5.334352in}{4.372920in}}{\pgfqpoint{5.323753in}{4.377310in}}{\pgfqpoint{5.312702in}{4.377310in}}%
\pgfpathcurveto{\pgfqpoint{5.301652in}{4.377310in}}{\pgfqpoint{5.291053in}{4.372920in}}{\pgfqpoint{5.283240in}{4.365107in}}%
\pgfpathcurveto{\pgfqpoint{5.275426in}{4.357293in}}{\pgfqpoint{5.271036in}{4.346694in}}{\pgfqpoint{5.271036in}{4.335644in}}%
\pgfpathcurveto{\pgfqpoint{5.271036in}{4.324594in}}{\pgfqpoint{5.275426in}{4.313995in}}{\pgfqpoint{5.283240in}{4.306181in}}%
\pgfpathcurveto{\pgfqpoint{5.291053in}{4.298367in}}{\pgfqpoint{5.301652in}{4.293977in}}{\pgfqpoint{5.312702in}{4.293977in}}%
\pgfpathclose%
\pgfusepath{stroke,fill}%
\end{pgfscope}%
\begin{pgfscope}%
\pgfpathrectangle{\pgfqpoint{0.526127in}{0.331635in}}{\pgfqpoint{9.300000in}{7.700000in}}%
\pgfusepath{clip}%
\pgfsetbuttcap%
\pgfsetroundjoin%
\definecolor{currentfill}{rgb}{0.980392,0.690196,0.894118}%
\pgfsetfillcolor{currentfill}%
\pgfsetlinewidth{0.481800pt}%
\definecolor{currentstroke}{rgb}{1.000000,1.000000,1.000000}%
\pgfsetstrokecolor{currentstroke}%
\pgfsetdash{}{0pt}%
\pgfpathmoveto{\pgfqpoint{8.614970in}{4.727871in}}%
\pgfpathcurveto{\pgfqpoint{8.626020in}{4.727871in}}{\pgfqpoint{8.636619in}{4.732261in}}{\pgfqpoint{8.644433in}{4.740075in}}%
\pgfpathcurveto{\pgfqpoint{8.652247in}{4.747889in}}{\pgfqpoint{8.656637in}{4.758488in}}{\pgfqpoint{8.656637in}{4.769538in}}%
\pgfpathcurveto{\pgfqpoint{8.656637in}{4.780588in}}{\pgfqpoint{8.652247in}{4.791187in}}{\pgfqpoint{8.644433in}{4.799001in}}%
\pgfpathcurveto{\pgfqpoint{8.636619in}{4.806814in}}{\pgfqpoint{8.626020in}{4.811204in}}{\pgfqpoint{8.614970in}{4.811204in}}%
\pgfpathcurveto{\pgfqpoint{8.603920in}{4.811204in}}{\pgfqpoint{8.593321in}{4.806814in}}{\pgfqpoint{8.585508in}{4.799001in}}%
\pgfpathcurveto{\pgfqpoint{8.577694in}{4.791187in}}{\pgfqpoint{8.573304in}{4.780588in}}{\pgfqpoint{8.573304in}{4.769538in}}%
\pgfpathcurveto{\pgfqpoint{8.573304in}{4.758488in}}{\pgfqpoint{8.577694in}{4.747889in}}{\pgfqpoint{8.585508in}{4.740075in}}%
\pgfpathcurveto{\pgfqpoint{8.593321in}{4.732261in}}{\pgfqpoint{8.603920in}{4.727871in}}{\pgfqpoint{8.614970in}{4.727871in}}%
\pgfpathclose%
\pgfusepath{stroke,fill}%
\end{pgfscope}%
\begin{pgfscope}%
\pgfpathrectangle{\pgfqpoint{0.526127in}{0.331635in}}{\pgfqpoint{9.300000in}{7.700000in}}%
\pgfusepath{clip}%
\pgfsetbuttcap%
\pgfsetroundjoin%
\definecolor{currentfill}{rgb}{0.980392,0.690196,0.894118}%
\pgfsetfillcolor{currentfill}%
\pgfsetlinewidth{0.481800pt}%
\definecolor{currentstroke}{rgb}{1.000000,1.000000,1.000000}%
\pgfsetstrokecolor{currentstroke}%
\pgfsetdash{}{0pt}%
\pgfpathmoveto{\pgfqpoint{8.457929in}{5.790002in}}%
\pgfpathcurveto{\pgfqpoint{8.468979in}{5.790002in}}{\pgfqpoint{8.479578in}{5.794393in}}{\pgfqpoint{8.487392in}{5.802206in}}%
\pgfpathcurveto{\pgfqpoint{8.495205in}{5.810020in}}{\pgfqpoint{8.499596in}{5.820619in}}{\pgfqpoint{8.499596in}{5.831669in}}%
\pgfpathcurveto{\pgfqpoint{8.499596in}{5.842719in}}{\pgfqpoint{8.495205in}{5.853318in}}{\pgfqpoint{8.487392in}{5.861132in}}%
\pgfpathcurveto{\pgfqpoint{8.479578in}{5.868945in}}{\pgfqpoint{8.468979in}{5.873336in}}{\pgfqpoint{8.457929in}{5.873336in}}%
\pgfpathcurveto{\pgfqpoint{8.446879in}{5.873336in}}{\pgfqpoint{8.436280in}{5.868945in}}{\pgfqpoint{8.428466in}{5.861132in}}%
\pgfpathcurveto{\pgfqpoint{8.420653in}{5.853318in}}{\pgfqpoint{8.416262in}{5.842719in}}{\pgfqpoint{8.416262in}{5.831669in}}%
\pgfpathcurveto{\pgfqpoint{8.416262in}{5.820619in}}{\pgfqpoint{8.420653in}{5.810020in}}{\pgfqpoint{8.428466in}{5.802206in}}%
\pgfpathcurveto{\pgfqpoint{8.436280in}{5.794393in}}{\pgfqpoint{8.446879in}{5.790002in}}{\pgfqpoint{8.457929in}{5.790002in}}%
\pgfpathclose%
\pgfusepath{stroke,fill}%
\end{pgfscope}%
\begin{pgfscope}%
\pgfpathrectangle{\pgfqpoint{0.526127in}{0.331635in}}{\pgfqpoint{9.300000in}{7.700000in}}%
\pgfusepath{clip}%
\pgfsetbuttcap%
\pgfsetroundjoin%
\definecolor{currentfill}{rgb}{0.980392,0.690196,0.894118}%
\pgfsetfillcolor{currentfill}%
\pgfsetlinewidth{0.481800pt}%
\definecolor{currentstroke}{rgb}{1.000000,1.000000,1.000000}%
\pgfsetstrokecolor{currentstroke}%
\pgfsetdash{}{0pt}%
\pgfpathmoveto{\pgfqpoint{7.835370in}{4.136992in}}%
\pgfpathcurveto{\pgfqpoint{7.846421in}{4.136992in}}{\pgfqpoint{7.857020in}{4.141382in}}{\pgfqpoint{7.864833in}{4.149196in}}%
\pgfpathcurveto{\pgfqpoint{7.872647in}{4.157010in}}{\pgfqpoint{7.877037in}{4.167609in}}{\pgfqpoint{7.877037in}{4.178659in}}%
\pgfpathcurveto{\pgfqpoint{7.877037in}{4.189709in}}{\pgfqpoint{7.872647in}{4.200308in}}{\pgfqpoint{7.864833in}{4.208122in}}%
\pgfpathcurveto{\pgfqpoint{7.857020in}{4.215935in}}{\pgfqpoint{7.846421in}{4.220326in}}{\pgfqpoint{7.835370in}{4.220326in}}%
\pgfpathcurveto{\pgfqpoint{7.824320in}{4.220326in}}{\pgfqpoint{7.813721in}{4.215935in}}{\pgfqpoint{7.805908in}{4.208122in}}%
\pgfpathcurveto{\pgfqpoint{7.798094in}{4.200308in}}{\pgfqpoint{7.793704in}{4.189709in}}{\pgfqpoint{7.793704in}{4.178659in}}%
\pgfpathcurveto{\pgfqpoint{7.793704in}{4.167609in}}{\pgfqpoint{7.798094in}{4.157010in}}{\pgfqpoint{7.805908in}{4.149196in}}%
\pgfpathcurveto{\pgfqpoint{7.813721in}{4.141382in}}{\pgfqpoint{7.824320in}{4.136992in}}{\pgfqpoint{7.835370in}{4.136992in}}%
\pgfpathclose%
\pgfusepath{stroke,fill}%
\end{pgfscope}%
\begin{pgfscope}%
\pgfpathrectangle{\pgfqpoint{0.526127in}{0.331635in}}{\pgfqpoint{9.300000in}{7.700000in}}%
\pgfusepath{clip}%
\pgfsetbuttcap%
\pgfsetroundjoin%
\definecolor{currentfill}{rgb}{0.721569,0.521569,0.039216}%
\pgfsetfillcolor{currentfill}%
\pgfsetlinewidth{0.481800pt}%
\definecolor{currentstroke}{rgb}{1.000000,1.000000,1.000000}%
\pgfsetstrokecolor{currentstroke}%
\pgfsetdash{}{0pt}%
\pgfpathmoveto{\pgfqpoint{5.776420in}{4.037481in}}%
\pgfpathcurveto{\pgfqpoint{5.787470in}{4.037481in}}{\pgfqpoint{5.798069in}{4.041871in}}{\pgfqpoint{5.805883in}{4.049684in}}%
\pgfpathcurveto{\pgfqpoint{5.813696in}{4.057498in}}{\pgfqpoint{5.818087in}{4.068097in}}{\pgfqpoint{5.818087in}{4.079147in}}%
\pgfpathcurveto{\pgfqpoint{5.818087in}{4.090197in}}{\pgfqpoint{5.813696in}{4.100796in}}{\pgfqpoint{5.805883in}{4.108610in}}%
\pgfpathcurveto{\pgfqpoint{5.798069in}{4.116424in}}{\pgfqpoint{5.787470in}{4.120814in}}{\pgfqpoint{5.776420in}{4.120814in}}%
\pgfpathcurveto{\pgfqpoint{5.765370in}{4.120814in}}{\pgfqpoint{5.754771in}{4.116424in}}{\pgfqpoint{5.746957in}{4.108610in}}%
\pgfpathcurveto{\pgfqpoint{5.739144in}{4.100796in}}{\pgfqpoint{5.734753in}{4.090197in}}{\pgfqpoint{5.734753in}{4.079147in}}%
\pgfpathcurveto{\pgfqpoint{5.734753in}{4.068097in}}{\pgfqpoint{5.739144in}{4.057498in}}{\pgfqpoint{5.746957in}{4.049684in}}%
\pgfpathcurveto{\pgfqpoint{5.754771in}{4.041871in}}{\pgfqpoint{5.765370in}{4.037481in}}{\pgfqpoint{5.776420in}{4.037481in}}%
\pgfpathclose%
\pgfusepath{stroke,fill}%
\end{pgfscope}%
\begin{pgfscope}%
\pgfpathrectangle{\pgfqpoint{0.526127in}{0.331635in}}{\pgfqpoint{9.300000in}{7.700000in}}%
\pgfusepath{clip}%
\pgfsetbuttcap%
\pgfsetroundjoin%
\definecolor{currentfill}{rgb}{0.721569,0.521569,0.039216}%
\pgfsetfillcolor{currentfill}%
\pgfsetlinewidth{0.481800pt}%
\definecolor{currentstroke}{rgb}{1.000000,1.000000,1.000000}%
\pgfsetstrokecolor{currentstroke}%
\pgfsetdash{}{0pt}%
\pgfpathmoveto{\pgfqpoint{8.729922in}{5.014306in}}%
\pgfpathcurveto{\pgfqpoint{8.740972in}{5.014306in}}{\pgfqpoint{8.751571in}{5.018696in}}{\pgfqpoint{8.759385in}{5.026510in}}%
\pgfpathcurveto{\pgfqpoint{8.767198in}{5.034323in}}{\pgfqpoint{8.771589in}{5.044922in}}{\pgfqpoint{8.771589in}{5.055972in}}%
\pgfpathcurveto{\pgfqpoint{8.771589in}{5.067022in}}{\pgfqpoint{8.767198in}{5.077622in}}{\pgfqpoint{8.759385in}{5.085435in}}%
\pgfpathcurveto{\pgfqpoint{8.751571in}{5.093249in}}{\pgfqpoint{8.740972in}{5.097639in}}{\pgfqpoint{8.729922in}{5.097639in}}%
\pgfpathcurveto{\pgfqpoint{8.718872in}{5.097639in}}{\pgfqpoint{8.708273in}{5.093249in}}{\pgfqpoint{8.700459in}{5.085435in}}%
\pgfpathcurveto{\pgfqpoint{8.692646in}{5.077622in}}{\pgfqpoint{8.688255in}{5.067022in}}{\pgfqpoint{8.688255in}{5.055972in}}%
\pgfpathcurveto{\pgfqpoint{8.688255in}{5.044922in}}{\pgfqpoint{8.692646in}{5.034323in}}{\pgfqpoint{8.700459in}{5.026510in}}%
\pgfpathcurveto{\pgfqpoint{8.708273in}{5.018696in}}{\pgfqpoint{8.718872in}{5.014306in}}{\pgfqpoint{8.729922in}{5.014306in}}%
\pgfpathclose%
\pgfusepath{stroke,fill}%
\end{pgfscope}%
\begin{pgfscope}%
\pgfpathrectangle{\pgfqpoint{0.526127in}{0.331635in}}{\pgfqpoint{9.300000in}{7.700000in}}%
\pgfusepath{clip}%
\pgfsetbuttcap%
\pgfsetroundjoin%
\definecolor{currentfill}{rgb}{0.721569,0.521569,0.039216}%
\pgfsetfillcolor{currentfill}%
\pgfsetlinewidth{0.481800pt}%
\definecolor{currentstroke}{rgb}{1.000000,1.000000,1.000000}%
\pgfsetstrokecolor{currentstroke}%
\pgfsetdash{}{0pt}%
\pgfpathmoveto{\pgfqpoint{2.913096in}{4.616803in}}%
\pgfpathcurveto{\pgfqpoint{2.924146in}{4.616803in}}{\pgfqpoint{2.934745in}{4.621194in}}{\pgfqpoint{2.942558in}{4.629007in}}%
\pgfpathcurveto{\pgfqpoint{2.950372in}{4.636821in}}{\pgfqpoint{2.954762in}{4.647420in}}{\pgfqpoint{2.954762in}{4.658470in}}%
\pgfpathcurveto{\pgfqpoint{2.954762in}{4.669520in}}{\pgfqpoint{2.950372in}{4.680119in}}{\pgfqpoint{2.942558in}{4.687933in}}%
\pgfpathcurveto{\pgfqpoint{2.934745in}{4.695746in}}{\pgfqpoint{2.924146in}{4.700137in}}{\pgfqpoint{2.913096in}{4.700137in}}%
\pgfpathcurveto{\pgfqpoint{2.902045in}{4.700137in}}{\pgfqpoint{2.891446in}{4.695746in}}{\pgfqpoint{2.883633in}{4.687933in}}%
\pgfpathcurveto{\pgfqpoint{2.875819in}{4.680119in}}{\pgfqpoint{2.871429in}{4.669520in}}{\pgfqpoint{2.871429in}{4.658470in}}%
\pgfpathcurveto{\pgfqpoint{2.871429in}{4.647420in}}{\pgfqpoint{2.875819in}{4.636821in}}{\pgfqpoint{2.883633in}{4.629007in}}%
\pgfpathcurveto{\pgfqpoint{2.891446in}{4.621194in}}{\pgfqpoint{2.902045in}{4.616803in}}{\pgfqpoint{2.913096in}{4.616803in}}%
\pgfpathclose%
\pgfusepath{stroke,fill}%
\end{pgfscope}%
\begin{pgfscope}%
\pgfpathrectangle{\pgfqpoint{0.526127in}{0.331635in}}{\pgfqpoint{9.300000in}{7.700000in}}%
\pgfusepath{clip}%
\pgfsetbuttcap%
\pgfsetroundjoin%
\definecolor{currentfill}{rgb}{0.721569,0.521569,0.039216}%
\pgfsetfillcolor{currentfill}%
\pgfsetlinewidth{0.481800pt}%
\definecolor{currentstroke}{rgb}{1.000000,1.000000,1.000000}%
\pgfsetstrokecolor{currentstroke}%
\pgfsetdash{}{0pt}%
\pgfpathmoveto{\pgfqpoint{3.918807in}{5.738183in}}%
\pgfpathcurveto{\pgfqpoint{3.929858in}{5.738183in}}{\pgfqpoint{3.940457in}{5.742573in}}{\pgfqpoint{3.948270in}{5.750387in}}%
\pgfpathcurveto{\pgfqpoint{3.956084in}{5.758200in}}{\pgfqpoint{3.960474in}{5.768799in}}{\pgfqpoint{3.960474in}{5.779850in}}%
\pgfpathcurveto{\pgfqpoint{3.960474in}{5.790900in}}{\pgfqpoint{3.956084in}{5.801499in}}{\pgfqpoint{3.948270in}{5.809312in}}%
\pgfpathcurveto{\pgfqpoint{3.940457in}{5.817126in}}{\pgfqpoint{3.929858in}{5.821516in}}{\pgfqpoint{3.918807in}{5.821516in}}%
\pgfpathcurveto{\pgfqpoint{3.907757in}{5.821516in}}{\pgfqpoint{3.897158in}{5.817126in}}{\pgfqpoint{3.889345in}{5.809312in}}%
\pgfpathcurveto{\pgfqpoint{3.881531in}{5.801499in}}{\pgfqpoint{3.877141in}{5.790900in}}{\pgfqpoint{3.877141in}{5.779850in}}%
\pgfpathcurveto{\pgfqpoint{3.877141in}{5.768799in}}{\pgfqpoint{3.881531in}{5.758200in}}{\pgfqpoint{3.889345in}{5.750387in}}%
\pgfpathcurveto{\pgfqpoint{3.897158in}{5.742573in}}{\pgfqpoint{3.907757in}{5.738183in}}{\pgfqpoint{3.918807in}{5.738183in}}%
\pgfpathclose%
\pgfusepath{stroke,fill}%
\end{pgfscope}%
\begin{pgfscope}%
\pgfpathrectangle{\pgfqpoint{0.526127in}{0.331635in}}{\pgfqpoint{9.300000in}{7.700000in}}%
\pgfusepath{clip}%
\pgfsetbuttcap%
\pgfsetroundjoin%
\definecolor{currentfill}{rgb}{0.721569,0.521569,0.039216}%
\pgfsetfillcolor{currentfill}%
\pgfsetlinewidth{0.481800pt}%
\definecolor{currentstroke}{rgb}{1.000000,1.000000,1.000000}%
\pgfsetstrokecolor{currentstroke}%
\pgfsetdash{}{0pt}%
\pgfpathmoveto{\pgfqpoint{2.663268in}{3.834370in}}%
\pgfpathcurveto{\pgfqpoint{2.674319in}{3.834370in}}{\pgfqpoint{2.684918in}{3.838760in}}{\pgfqpoint{2.692731in}{3.846574in}}%
\pgfpathcurveto{\pgfqpoint{2.700545in}{3.854387in}}{\pgfqpoint{2.704935in}{3.864986in}}{\pgfqpoint{2.704935in}{3.876036in}}%
\pgfpathcurveto{\pgfqpoint{2.704935in}{3.887087in}}{\pgfqpoint{2.700545in}{3.897686in}}{\pgfqpoint{2.692731in}{3.905499in}}%
\pgfpathcurveto{\pgfqpoint{2.684918in}{3.913313in}}{\pgfqpoint{2.674319in}{3.917703in}}{\pgfqpoint{2.663268in}{3.917703in}}%
\pgfpathcurveto{\pgfqpoint{2.652218in}{3.917703in}}{\pgfqpoint{2.641619in}{3.913313in}}{\pgfqpoint{2.633806in}{3.905499in}}%
\pgfpathcurveto{\pgfqpoint{2.625992in}{3.897686in}}{\pgfqpoint{2.621602in}{3.887087in}}{\pgfqpoint{2.621602in}{3.876036in}}%
\pgfpathcurveto{\pgfqpoint{2.621602in}{3.864986in}}{\pgfqpoint{2.625992in}{3.854387in}}{\pgfqpoint{2.633806in}{3.846574in}}%
\pgfpathcurveto{\pgfqpoint{2.641619in}{3.838760in}}{\pgfqpoint{2.652218in}{3.834370in}}{\pgfqpoint{2.663268in}{3.834370in}}%
\pgfpathclose%
\pgfusepath{stroke,fill}%
\end{pgfscope}%
\begin{pgfscope}%
\pgfpathrectangle{\pgfqpoint{0.526127in}{0.331635in}}{\pgfqpoint{9.300000in}{7.700000in}}%
\pgfusepath{clip}%
\pgfsetbuttcap%
\pgfsetroundjoin%
\definecolor{currentfill}{rgb}{0.721569,0.521569,0.039216}%
\pgfsetfillcolor{currentfill}%
\pgfsetlinewidth{0.481800pt}%
\definecolor{currentstroke}{rgb}{1.000000,1.000000,1.000000}%
\pgfsetstrokecolor{currentstroke}%
\pgfsetdash{}{0pt}%
\pgfpathmoveto{\pgfqpoint{6.282605in}{4.913350in}}%
\pgfpathcurveto{\pgfqpoint{6.293655in}{4.913350in}}{\pgfqpoint{6.304254in}{4.917741in}}{\pgfqpoint{6.312068in}{4.925554in}}%
\pgfpathcurveto{\pgfqpoint{6.319881in}{4.933368in}}{\pgfqpoint{6.324272in}{4.943967in}}{\pgfqpoint{6.324272in}{4.955017in}}%
\pgfpathcurveto{\pgfqpoint{6.324272in}{4.966067in}}{\pgfqpoint{6.319881in}{4.976666in}}{\pgfqpoint{6.312068in}{4.984480in}}%
\pgfpathcurveto{\pgfqpoint{6.304254in}{4.992293in}}{\pgfqpoint{6.293655in}{4.996684in}}{\pgfqpoint{6.282605in}{4.996684in}}%
\pgfpathcurveto{\pgfqpoint{6.271555in}{4.996684in}}{\pgfqpoint{6.260956in}{4.992293in}}{\pgfqpoint{6.253142in}{4.984480in}}%
\pgfpathcurveto{\pgfqpoint{6.245328in}{4.976666in}}{\pgfqpoint{6.240938in}{4.966067in}}{\pgfqpoint{6.240938in}{4.955017in}}%
\pgfpathcurveto{\pgfqpoint{6.240938in}{4.943967in}}{\pgfqpoint{6.245328in}{4.933368in}}{\pgfqpoint{6.253142in}{4.925554in}}%
\pgfpathcurveto{\pgfqpoint{6.260956in}{4.917741in}}{\pgfqpoint{6.271555in}{4.913350in}}{\pgfqpoint{6.282605in}{4.913350in}}%
\pgfpathclose%
\pgfusepath{stroke,fill}%
\end{pgfscope}%
\begin{pgfscope}%
\pgfpathrectangle{\pgfqpoint{0.526127in}{0.331635in}}{\pgfqpoint{9.300000in}{7.700000in}}%
\pgfusepath{clip}%
\pgfsetbuttcap%
\pgfsetroundjoin%
\definecolor{currentfill}{rgb}{0.721569,0.521569,0.039216}%
\pgfsetfillcolor{currentfill}%
\pgfsetlinewidth{0.481800pt}%
\definecolor{currentstroke}{rgb}{1.000000,1.000000,1.000000}%
\pgfsetstrokecolor{currentstroke}%
\pgfsetdash{}{0pt}%
\pgfpathmoveto{\pgfqpoint{3.179398in}{3.411320in}}%
\pgfpathcurveto{\pgfqpoint{3.190448in}{3.411320in}}{\pgfqpoint{3.201047in}{3.415710in}}{\pgfqpoint{3.208861in}{3.423524in}}%
\pgfpathcurveto{\pgfqpoint{3.216674in}{3.431338in}}{\pgfqpoint{3.221065in}{3.441937in}}{\pgfqpoint{3.221065in}{3.452987in}}%
\pgfpathcurveto{\pgfqpoint{3.221065in}{3.464037in}}{\pgfqpoint{3.216674in}{3.474636in}}{\pgfqpoint{3.208861in}{3.482450in}}%
\pgfpathcurveto{\pgfqpoint{3.201047in}{3.490263in}}{\pgfqpoint{3.190448in}{3.494653in}}{\pgfqpoint{3.179398in}{3.494653in}}%
\pgfpathcurveto{\pgfqpoint{3.168348in}{3.494653in}}{\pgfqpoint{3.157749in}{3.490263in}}{\pgfqpoint{3.149935in}{3.482450in}}%
\pgfpathcurveto{\pgfqpoint{3.142122in}{3.474636in}}{\pgfqpoint{3.137731in}{3.464037in}}{\pgfqpoint{3.137731in}{3.452987in}}%
\pgfpathcurveto{\pgfqpoint{3.137731in}{3.441937in}}{\pgfqpoint{3.142122in}{3.431338in}}{\pgfqpoint{3.149935in}{3.423524in}}%
\pgfpathcurveto{\pgfqpoint{3.157749in}{3.415710in}}{\pgfqpoint{3.168348in}{3.411320in}}{\pgfqpoint{3.179398in}{3.411320in}}%
\pgfpathclose%
\pgfusepath{stroke,fill}%
\end{pgfscope}%
\begin{pgfscope}%
\pgfpathrectangle{\pgfqpoint{0.526127in}{0.331635in}}{\pgfqpoint{9.300000in}{7.700000in}}%
\pgfusepath{clip}%
\pgfsetbuttcap%
\pgfsetroundjoin%
\definecolor{currentfill}{rgb}{0.721569,0.521569,0.039216}%
\pgfsetfillcolor{currentfill}%
\pgfsetlinewidth{0.481800pt}%
\definecolor{currentstroke}{rgb}{1.000000,1.000000,1.000000}%
\pgfsetstrokecolor{currentstroke}%
\pgfsetdash{}{0pt}%
\pgfpathmoveto{\pgfqpoint{4.240309in}{7.078791in}}%
\pgfpathcurveto{\pgfqpoint{4.251359in}{7.078791in}}{\pgfqpoint{4.261958in}{7.083182in}}{\pgfqpoint{4.269772in}{7.090995in}}%
\pgfpathcurveto{\pgfqpoint{4.277585in}{7.098809in}}{\pgfqpoint{4.281976in}{7.109408in}}{\pgfqpoint{4.281976in}{7.120458in}}%
\pgfpathcurveto{\pgfqpoint{4.281976in}{7.131508in}}{\pgfqpoint{4.277585in}{7.142107in}}{\pgfqpoint{4.269772in}{7.149921in}}%
\pgfpathcurveto{\pgfqpoint{4.261958in}{7.157734in}}{\pgfqpoint{4.251359in}{7.162125in}}{\pgfqpoint{4.240309in}{7.162125in}}%
\pgfpathcurveto{\pgfqpoint{4.229259in}{7.162125in}}{\pgfqpoint{4.218660in}{7.157734in}}{\pgfqpoint{4.210846in}{7.149921in}}%
\pgfpathcurveto{\pgfqpoint{4.203033in}{7.142107in}}{\pgfqpoint{4.198642in}{7.131508in}}{\pgfqpoint{4.198642in}{7.120458in}}%
\pgfpathcurveto{\pgfqpoint{4.198642in}{7.109408in}}{\pgfqpoint{4.203033in}{7.098809in}}{\pgfqpoint{4.210846in}{7.090995in}}%
\pgfpathcurveto{\pgfqpoint{4.218660in}{7.083182in}}{\pgfqpoint{4.229259in}{7.078791in}}{\pgfqpoint{4.240309in}{7.078791in}}%
\pgfpathclose%
\pgfusepath{stroke,fill}%
\end{pgfscope}%
\begin{pgfscope}%
\pgfpathrectangle{\pgfqpoint{0.526127in}{0.331635in}}{\pgfqpoint{9.300000in}{7.700000in}}%
\pgfusepath{clip}%
\pgfsetbuttcap%
\pgfsetroundjoin%
\definecolor{currentfill}{rgb}{0.721569,0.521569,0.039216}%
\pgfsetfillcolor{currentfill}%
\pgfsetlinewidth{0.481800pt}%
\definecolor{currentstroke}{rgb}{1.000000,1.000000,1.000000}%
\pgfsetstrokecolor{currentstroke}%
\pgfsetdash{}{0pt}%
\pgfpathmoveto{\pgfqpoint{5.395356in}{4.851976in}}%
\pgfpathcurveto{\pgfqpoint{5.406406in}{4.851976in}}{\pgfqpoint{5.417005in}{4.856367in}}{\pgfqpoint{5.424819in}{4.864180in}}%
\pgfpathcurveto{\pgfqpoint{5.432632in}{4.871994in}}{\pgfqpoint{5.437023in}{4.882593in}}{\pgfqpoint{5.437023in}{4.893643in}}%
\pgfpathcurveto{\pgfqpoint{5.437023in}{4.904693in}}{\pgfqpoint{5.432632in}{4.915292in}}{\pgfqpoint{5.424819in}{4.923106in}}%
\pgfpathcurveto{\pgfqpoint{5.417005in}{4.930919in}}{\pgfqpoint{5.406406in}{4.935310in}}{\pgfqpoint{5.395356in}{4.935310in}}%
\pgfpathcurveto{\pgfqpoint{5.384306in}{4.935310in}}{\pgfqpoint{5.373707in}{4.930919in}}{\pgfqpoint{5.365893in}{4.923106in}}%
\pgfpathcurveto{\pgfqpoint{5.358080in}{4.915292in}}{\pgfqpoint{5.353689in}{4.904693in}}{\pgfqpoint{5.353689in}{4.893643in}}%
\pgfpathcurveto{\pgfqpoint{5.353689in}{4.882593in}}{\pgfqpoint{5.358080in}{4.871994in}}{\pgfqpoint{5.365893in}{4.864180in}}%
\pgfpathcurveto{\pgfqpoint{5.373707in}{4.856367in}}{\pgfqpoint{5.384306in}{4.851976in}}{\pgfqpoint{5.395356in}{4.851976in}}%
\pgfpathclose%
\pgfusepath{stroke,fill}%
\end{pgfscope}%
\begin{pgfscope}%
\pgfpathrectangle{\pgfqpoint{0.526127in}{0.331635in}}{\pgfqpoint{9.300000in}{7.700000in}}%
\pgfusepath{clip}%
\pgfsetbuttcap%
\pgfsetroundjoin%
\definecolor{currentfill}{rgb}{0.721569,0.521569,0.039216}%
\pgfsetfillcolor{currentfill}%
\pgfsetlinewidth{0.481800pt}%
\definecolor{currentstroke}{rgb}{1.000000,1.000000,1.000000}%
\pgfsetstrokecolor{currentstroke}%
\pgfsetdash{}{0pt}%
\pgfpathmoveto{\pgfqpoint{4.154359in}{5.088748in}}%
\pgfpathcurveto{\pgfqpoint{4.165409in}{5.088748in}}{\pgfqpoint{4.176008in}{5.093138in}}{\pgfqpoint{4.183822in}{5.100952in}}%
\pgfpathcurveto{\pgfqpoint{4.191636in}{5.108766in}}{\pgfqpoint{4.196026in}{5.119365in}}{\pgfqpoint{4.196026in}{5.130415in}}%
\pgfpathcurveto{\pgfqpoint{4.196026in}{5.141465in}}{\pgfqpoint{4.191636in}{5.152064in}}{\pgfqpoint{4.183822in}{5.159878in}}%
\pgfpathcurveto{\pgfqpoint{4.176008in}{5.167691in}}{\pgfqpoint{4.165409in}{5.172081in}}{\pgfqpoint{4.154359in}{5.172081in}}%
\pgfpathcurveto{\pgfqpoint{4.143309in}{5.172081in}}{\pgfqpoint{4.132710in}{5.167691in}}{\pgfqpoint{4.124896in}{5.159878in}}%
\pgfpathcurveto{\pgfqpoint{4.117083in}{5.152064in}}{\pgfqpoint{4.112693in}{5.141465in}}{\pgfqpoint{4.112693in}{5.130415in}}%
\pgfpathcurveto{\pgfqpoint{4.112693in}{5.119365in}}{\pgfqpoint{4.117083in}{5.108766in}}{\pgfqpoint{4.124896in}{5.100952in}}%
\pgfpathcurveto{\pgfqpoint{4.132710in}{5.093138in}}{\pgfqpoint{4.143309in}{5.088748in}}{\pgfqpoint{4.154359in}{5.088748in}}%
\pgfpathclose%
\pgfusepath{stroke,fill}%
\end{pgfscope}%
\begin{pgfscope}%
\pgfpathrectangle{\pgfqpoint{0.526127in}{0.331635in}}{\pgfqpoint{9.300000in}{7.700000in}}%
\pgfusepath{clip}%
\pgfsetbuttcap%
\pgfsetroundjoin%
\definecolor{currentfill}{rgb}{0.721569,0.521569,0.039216}%
\pgfsetfillcolor{currentfill}%
\pgfsetlinewidth{0.481800pt}%
\definecolor{currentstroke}{rgb}{1.000000,1.000000,1.000000}%
\pgfsetstrokecolor{currentstroke}%
\pgfsetdash{}{0pt}%
\pgfpathmoveto{\pgfqpoint{3.012801in}{4.435592in}}%
\pgfpathcurveto{\pgfqpoint{3.023851in}{4.435592in}}{\pgfqpoint{3.034450in}{4.439982in}}{\pgfqpoint{3.042264in}{4.447795in}}%
\pgfpathcurveto{\pgfqpoint{3.050078in}{4.455609in}}{\pgfqpoint{3.054468in}{4.466208in}}{\pgfqpoint{3.054468in}{4.477258in}}%
\pgfpathcurveto{\pgfqpoint{3.054468in}{4.488308in}}{\pgfqpoint{3.050078in}{4.498907in}}{\pgfqpoint{3.042264in}{4.506721in}}%
\pgfpathcurveto{\pgfqpoint{3.034450in}{4.514535in}}{\pgfqpoint{3.023851in}{4.518925in}}{\pgfqpoint{3.012801in}{4.518925in}}%
\pgfpathcurveto{\pgfqpoint{3.001751in}{4.518925in}}{\pgfqpoint{2.991152in}{4.514535in}}{\pgfqpoint{2.983338in}{4.506721in}}%
\pgfpathcurveto{\pgfqpoint{2.975525in}{4.498907in}}{\pgfqpoint{2.971134in}{4.488308in}}{\pgfqpoint{2.971134in}{4.477258in}}%
\pgfpathcurveto{\pgfqpoint{2.971134in}{4.466208in}}{\pgfqpoint{2.975525in}{4.455609in}}{\pgfqpoint{2.983338in}{4.447795in}}%
\pgfpathcurveto{\pgfqpoint{2.991152in}{4.439982in}}{\pgfqpoint{3.001751in}{4.435592in}}{\pgfqpoint{3.012801in}{4.435592in}}%
\pgfpathclose%
\pgfusepath{stroke,fill}%
\end{pgfscope}%
\begin{pgfscope}%
\pgfpathrectangle{\pgfqpoint{0.526127in}{0.331635in}}{\pgfqpoint{9.300000in}{7.700000in}}%
\pgfusepath{clip}%
\pgfsetbuttcap%
\pgfsetroundjoin%
\definecolor{currentfill}{rgb}{0.721569,0.521569,0.039216}%
\pgfsetfillcolor{currentfill}%
\pgfsetlinewidth{0.481800pt}%
\definecolor{currentstroke}{rgb}{1.000000,1.000000,1.000000}%
\pgfsetstrokecolor{currentstroke}%
\pgfsetdash{}{0pt}%
\pgfpathmoveto{\pgfqpoint{6.132477in}{0.970105in}}%
\pgfpathcurveto{\pgfqpoint{6.143527in}{0.970105in}}{\pgfqpoint{6.154126in}{0.974495in}}{\pgfqpoint{6.161940in}{0.982309in}}%
\pgfpathcurveto{\pgfqpoint{6.169754in}{0.990123in}}{\pgfqpoint{6.174144in}{1.000722in}}{\pgfqpoint{6.174144in}{1.011772in}}%
\pgfpathcurveto{\pgfqpoint{6.174144in}{1.022822in}}{\pgfqpoint{6.169754in}{1.033421in}}{\pgfqpoint{6.161940in}{1.041235in}}%
\pgfpathcurveto{\pgfqpoint{6.154126in}{1.049048in}}{\pgfqpoint{6.143527in}{1.053438in}}{\pgfqpoint{6.132477in}{1.053438in}}%
\pgfpathcurveto{\pgfqpoint{6.121427in}{1.053438in}}{\pgfqpoint{6.110828in}{1.049048in}}{\pgfqpoint{6.103015in}{1.041235in}}%
\pgfpathcurveto{\pgfqpoint{6.095201in}{1.033421in}}{\pgfqpoint{6.090811in}{1.022822in}}{\pgfqpoint{6.090811in}{1.011772in}}%
\pgfpathcurveto{\pgfqpoint{6.090811in}{1.000722in}}{\pgfqpoint{6.095201in}{0.990123in}}{\pgfqpoint{6.103015in}{0.982309in}}%
\pgfpathcurveto{\pgfqpoint{6.110828in}{0.974495in}}{\pgfqpoint{6.121427in}{0.970105in}}{\pgfqpoint{6.132477in}{0.970105in}}%
\pgfpathclose%
\pgfusepath{stroke,fill}%
\end{pgfscope}%
\begin{pgfscope}%
\pgfpathrectangle{\pgfqpoint{0.526127in}{0.331635in}}{\pgfqpoint{9.300000in}{7.700000in}}%
\pgfusepath{clip}%
\pgfsetbuttcap%
\pgfsetroundjoin%
\definecolor{currentfill}{rgb}{0.721569,0.521569,0.039216}%
\pgfsetfillcolor{currentfill}%
\pgfsetlinewidth{0.481800pt}%
\definecolor{currentstroke}{rgb}{1.000000,1.000000,1.000000}%
\pgfsetstrokecolor{currentstroke}%
\pgfsetdash{}{0pt}%
\pgfpathmoveto{\pgfqpoint{8.967963in}{2.858023in}}%
\pgfpathcurveto{\pgfqpoint{8.979013in}{2.858023in}}{\pgfqpoint{8.989612in}{2.862414in}}{\pgfqpoint{8.997426in}{2.870227in}}%
\pgfpathcurveto{\pgfqpoint{9.005239in}{2.878041in}}{\pgfqpoint{9.009629in}{2.888640in}}{\pgfqpoint{9.009629in}{2.899690in}}%
\pgfpathcurveto{\pgfqpoint{9.009629in}{2.910740in}}{\pgfqpoint{9.005239in}{2.921339in}}{\pgfqpoint{8.997426in}{2.929153in}}%
\pgfpathcurveto{\pgfqpoint{8.989612in}{2.936966in}}{\pgfqpoint{8.979013in}{2.941357in}}{\pgfqpoint{8.967963in}{2.941357in}}%
\pgfpathcurveto{\pgfqpoint{8.956913in}{2.941357in}}{\pgfqpoint{8.946314in}{2.936966in}}{\pgfqpoint{8.938500in}{2.929153in}}%
\pgfpathcurveto{\pgfqpoint{8.930686in}{2.921339in}}{\pgfqpoint{8.926296in}{2.910740in}}{\pgfqpoint{8.926296in}{2.899690in}}%
\pgfpathcurveto{\pgfqpoint{8.926296in}{2.888640in}}{\pgfqpoint{8.930686in}{2.878041in}}{\pgfqpoint{8.938500in}{2.870227in}}%
\pgfpathcurveto{\pgfqpoint{8.946314in}{2.862414in}}{\pgfqpoint{8.956913in}{2.858023in}}{\pgfqpoint{8.967963in}{2.858023in}}%
\pgfpathclose%
\pgfusepath{stroke,fill}%
\end{pgfscope}%
\begin{pgfscope}%
\pgfpathrectangle{\pgfqpoint{0.526127in}{0.331635in}}{\pgfqpoint{9.300000in}{7.700000in}}%
\pgfusepath{clip}%
\pgfsetbuttcap%
\pgfsetroundjoin%
\definecolor{currentfill}{rgb}{0.721569,0.521569,0.039216}%
\pgfsetfillcolor{currentfill}%
\pgfsetlinewidth{0.481800pt}%
\definecolor{currentstroke}{rgb}{1.000000,1.000000,1.000000}%
\pgfsetstrokecolor{currentstroke}%
\pgfsetdash{}{0pt}%
\pgfpathmoveto{\pgfqpoint{2.684948in}{3.897660in}}%
\pgfpathcurveto{\pgfqpoint{2.695998in}{3.897660in}}{\pgfqpoint{2.706597in}{3.902050in}}{\pgfqpoint{2.714411in}{3.909864in}}%
\pgfpathcurveto{\pgfqpoint{2.722225in}{3.917678in}}{\pgfqpoint{2.726615in}{3.928277in}}{\pgfqpoint{2.726615in}{3.939327in}}%
\pgfpathcurveto{\pgfqpoint{2.726615in}{3.950377in}}{\pgfqpoint{2.722225in}{3.960976in}}{\pgfqpoint{2.714411in}{3.968790in}}%
\pgfpathcurveto{\pgfqpoint{2.706597in}{3.976603in}}{\pgfqpoint{2.695998in}{3.980993in}}{\pgfqpoint{2.684948in}{3.980993in}}%
\pgfpathcurveto{\pgfqpoint{2.673898in}{3.980993in}}{\pgfqpoint{2.663299in}{3.976603in}}{\pgfqpoint{2.655485in}{3.968790in}}%
\pgfpathcurveto{\pgfqpoint{2.647672in}{3.960976in}}{\pgfqpoint{2.643281in}{3.950377in}}{\pgfqpoint{2.643281in}{3.939327in}}%
\pgfpathcurveto{\pgfqpoint{2.643281in}{3.928277in}}{\pgfqpoint{2.647672in}{3.917678in}}{\pgfqpoint{2.655485in}{3.909864in}}%
\pgfpathcurveto{\pgfqpoint{2.663299in}{3.902050in}}{\pgfqpoint{2.673898in}{3.897660in}}{\pgfqpoint{2.684948in}{3.897660in}}%
\pgfpathclose%
\pgfusepath{stroke,fill}%
\end{pgfscope}%
\begin{pgfscope}%
\pgfpathrectangle{\pgfqpoint{0.526127in}{0.331635in}}{\pgfqpoint{9.300000in}{7.700000in}}%
\pgfusepath{clip}%
\pgfsetbuttcap%
\pgfsetroundjoin%
\definecolor{currentfill}{rgb}{0.721569,0.521569,0.039216}%
\pgfsetfillcolor{currentfill}%
\pgfsetlinewidth{0.481800pt}%
\definecolor{currentstroke}{rgb}{1.000000,1.000000,1.000000}%
\pgfsetstrokecolor{currentstroke}%
\pgfsetdash{}{0pt}%
\pgfpathmoveto{\pgfqpoint{7.144222in}{6.603868in}}%
\pgfpathcurveto{\pgfqpoint{7.155272in}{6.603868in}}{\pgfqpoint{7.165871in}{6.608258in}}{\pgfqpoint{7.173685in}{6.616072in}}%
\pgfpathcurveto{\pgfqpoint{7.181499in}{6.623885in}}{\pgfqpoint{7.185889in}{6.634484in}}{\pgfqpoint{7.185889in}{6.645535in}}%
\pgfpathcurveto{\pgfqpoint{7.185889in}{6.656585in}}{\pgfqpoint{7.181499in}{6.667184in}}{\pgfqpoint{7.173685in}{6.674997in}}%
\pgfpathcurveto{\pgfqpoint{7.165871in}{6.682811in}}{\pgfqpoint{7.155272in}{6.687201in}}{\pgfqpoint{7.144222in}{6.687201in}}%
\pgfpathcurveto{\pgfqpoint{7.133172in}{6.687201in}}{\pgfqpoint{7.122573in}{6.682811in}}{\pgfqpoint{7.114759in}{6.674997in}}%
\pgfpathcurveto{\pgfqpoint{7.106946in}{6.667184in}}{\pgfqpoint{7.102556in}{6.656585in}}{\pgfqpoint{7.102556in}{6.645535in}}%
\pgfpathcurveto{\pgfqpoint{7.102556in}{6.634484in}}{\pgfqpoint{7.106946in}{6.623885in}}{\pgfqpoint{7.114759in}{6.616072in}}%
\pgfpathcurveto{\pgfqpoint{7.122573in}{6.608258in}}{\pgfqpoint{7.133172in}{6.603868in}}{\pgfqpoint{7.144222in}{6.603868in}}%
\pgfpathclose%
\pgfusepath{stroke,fill}%
\end{pgfscope}%
\begin{pgfscope}%
\pgfpathrectangle{\pgfqpoint{0.526127in}{0.331635in}}{\pgfqpoint{9.300000in}{7.700000in}}%
\pgfusepath{clip}%
\pgfsetbuttcap%
\pgfsetroundjoin%
\definecolor{currentfill}{rgb}{0.721569,0.521569,0.039216}%
\pgfsetfillcolor{currentfill}%
\pgfsetlinewidth{0.481800pt}%
\definecolor{currentstroke}{rgb}{1.000000,1.000000,1.000000}%
\pgfsetstrokecolor{currentstroke}%
\pgfsetdash{}{0pt}%
\pgfpathmoveto{\pgfqpoint{5.755207in}{1.704388in}}%
\pgfpathcurveto{\pgfqpoint{5.766257in}{1.704388in}}{\pgfqpoint{5.776856in}{1.708778in}}{\pgfqpoint{5.784670in}{1.716591in}}%
\pgfpathcurveto{\pgfqpoint{5.792484in}{1.724405in}}{\pgfqpoint{5.796874in}{1.735004in}}{\pgfqpoint{5.796874in}{1.746054in}}%
\pgfpathcurveto{\pgfqpoint{5.796874in}{1.757104in}}{\pgfqpoint{5.792484in}{1.767703in}}{\pgfqpoint{5.784670in}{1.775517in}}%
\pgfpathcurveto{\pgfqpoint{5.776856in}{1.783331in}}{\pgfqpoint{5.766257in}{1.787721in}}{\pgfqpoint{5.755207in}{1.787721in}}%
\pgfpathcurveto{\pgfqpoint{5.744157in}{1.787721in}}{\pgfqpoint{5.733558in}{1.783331in}}{\pgfqpoint{5.725744in}{1.775517in}}%
\pgfpathcurveto{\pgfqpoint{5.717931in}{1.767703in}}{\pgfqpoint{5.713541in}{1.757104in}}{\pgfqpoint{5.713541in}{1.746054in}}%
\pgfpathcurveto{\pgfqpoint{5.713541in}{1.735004in}}{\pgfqpoint{5.717931in}{1.724405in}}{\pgfqpoint{5.725744in}{1.716591in}}%
\pgfpathcurveto{\pgfqpoint{5.733558in}{1.708778in}}{\pgfqpoint{5.744157in}{1.704388in}}{\pgfqpoint{5.755207in}{1.704388in}}%
\pgfpathclose%
\pgfusepath{stroke,fill}%
\end{pgfscope}%
\begin{pgfscope}%
\pgfpathrectangle{\pgfqpoint{0.526127in}{0.331635in}}{\pgfqpoint{9.300000in}{7.700000in}}%
\pgfusepath{clip}%
\pgfsetbuttcap%
\pgfsetroundjoin%
\definecolor{currentfill}{rgb}{0.721569,0.521569,0.039216}%
\pgfsetfillcolor{currentfill}%
\pgfsetlinewidth{0.481800pt}%
\definecolor{currentstroke}{rgb}{1.000000,1.000000,1.000000}%
\pgfsetstrokecolor{currentstroke}%
\pgfsetdash{}{0pt}%
\pgfpathmoveto{\pgfqpoint{9.079707in}{4.375025in}}%
\pgfpathcurveto{\pgfqpoint{9.090757in}{4.375025in}}{\pgfqpoint{9.101356in}{4.379415in}}{\pgfqpoint{9.109170in}{4.387229in}}%
\pgfpathcurveto{\pgfqpoint{9.116984in}{4.395042in}}{\pgfqpoint{9.121374in}{4.405641in}}{\pgfqpoint{9.121374in}{4.416691in}}%
\pgfpathcurveto{\pgfqpoint{9.121374in}{4.427741in}}{\pgfqpoint{9.116984in}{4.438341in}}{\pgfqpoint{9.109170in}{4.446154in}}%
\pgfpathcurveto{\pgfqpoint{9.101356in}{4.453968in}}{\pgfqpoint{9.090757in}{4.458358in}}{\pgfqpoint{9.079707in}{4.458358in}}%
\pgfpathcurveto{\pgfqpoint{9.068657in}{4.458358in}}{\pgfqpoint{9.058058in}{4.453968in}}{\pgfqpoint{9.050244in}{4.446154in}}%
\pgfpathcurveto{\pgfqpoint{9.042431in}{4.438341in}}{\pgfqpoint{9.038040in}{4.427741in}}{\pgfqpoint{9.038040in}{4.416691in}}%
\pgfpathcurveto{\pgfqpoint{9.038040in}{4.405641in}}{\pgfqpoint{9.042431in}{4.395042in}}{\pgfqpoint{9.050244in}{4.387229in}}%
\pgfpathcurveto{\pgfqpoint{9.058058in}{4.379415in}}{\pgfqpoint{9.068657in}{4.375025in}}{\pgfqpoint{9.079707in}{4.375025in}}%
\pgfpathclose%
\pgfusepath{stroke,fill}%
\end{pgfscope}%
\begin{pgfscope}%
\pgfpathrectangle{\pgfqpoint{0.526127in}{0.331635in}}{\pgfqpoint{9.300000in}{7.700000in}}%
\pgfusepath{clip}%
\pgfsetbuttcap%
\pgfsetroundjoin%
\definecolor{currentfill}{rgb}{0.721569,0.521569,0.039216}%
\pgfsetfillcolor{currentfill}%
\pgfsetlinewidth{0.481800pt}%
\definecolor{currentstroke}{rgb}{1.000000,1.000000,1.000000}%
\pgfsetstrokecolor{currentstroke}%
\pgfsetdash{}{0pt}%
\pgfpathmoveto{\pgfqpoint{4.104581in}{1.271214in}}%
\pgfpathcurveto{\pgfqpoint{4.115632in}{1.271214in}}{\pgfqpoint{4.126231in}{1.275604in}}{\pgfqpoint{4.134044in}{1.283418in}}%
\pgfpathcurveto{\pgfqpoint{4.141858in}{1.291231in}}{\pgfqpoint{4.146248in}{1.301831in}}{\pgfqpoint{4.146248in}{1.312881in}}%
\pgfpathcurveto{\pgfqpoint{4.146248in}{1.323931in}}{\pgfqpoint{4.141858in}{1.334530in}}{\pgfqpoint{4.134044in}{1.342343in}}%
\pgfpathcurveto{\pgfqpoint{4.126231in}{1.350157in}}{\pgfqpoint{4.115632in}{1.354547in}}{\pgfqpoint{4.104581in}{1.354547in}}%
\pgfpathcurveto{\pgfqpoint{4.093531in}{1.354547in}}{\pgfqpoint{4.082932in}{1.350157in}}{\pgfqpoint{4.075119in}{1.342343in}}%
\pgfpathcurveto{\pgfqpoint{4.067305in}{1.334530in}}{\pgfqpoint{4.062915in}{1.323931in}}{\pgfqpoint{4.062915in}{1.312881in}}%
\pgfpathcurveto{\pgfqpoint{4.062915in}{1.301831in}}{\pgfqpoint{4.067305in}{1.291231in}}{\pgfqpoint{4.075119in}{1.283418in}}%
\pgfpathcurveto{\pgfqpoint{4.082932in}{1.275604in}}{\pgfqpoint{4.093531in}{1.271214in}}{\pgfqpoint{4.104581in}{1.271214in}}%
\pgfpathclose%
\pgfusepath{stroke,fill}%
\end{pgfscope}%
\begin{pgfscope}%
\pgfpathrectangle{\pgfqpoint{0.526127in}{0.331635in}}{\pgfqpoint{9.300000in}{7.700000in}}%
\pgfusepath{clip}%
\pgfsetbuttcap%
\pgfsetroundjoin%
\definecolor{currentfill}{rgb}{0.721569,0.521569,0.039216}%
\pgfsetfillcolor{currentfill}%
\pgfsetlinewidth{0.481800pt}%
\definecolor{currentstroke}{rgb}{1.000000,1.000000,1.000000}%
\pgfsetstrokecolor{currentstroke}%
\pgfsetdash{}{0pt}%
\pgfpathmoveto{\pgfqpoint{1.641796in}{4.114808in}}%
\pgfpathcurveto{\pgfqpoint{1.652846in}{4.114808in}}{\pgfqpoint{1.663445in}{4.119198in}}{\pgfqpoint{1.671258in}{4.127012in}}%
\pgfpathcurveto{\pgfqpoint{1.679072in}{4.134825in}}{\pgfqpoint{1.683462in}{4.145424in}}{\pgfqpoint{1.683462in}{4.156474in}}%
\pgfpathcurveto{\pgfqpoint{1.683462in}{4.167525in}}{\pgfqpoint{1.679072in}{4.178124in}}{\pgfqpoint{1.671258in}{4.185937in}}%
\pgfpathcurveto{\pgfqpoint{1.663445in}{4.193751in}}{\pgfqpoint{1.652846in}{4.198141in}}{\pgfqpoint{1.641796in}{4.198141in}}%
\pgfpathcurveto{\pgfqpoint{1.630745in}{4.198141in}}{\pgfqpoint{1.620146in}{4.193751in}}{\pgfqpoint{1.612333in}{4.185937in}}%
\pgfpathcurveto{\pgfqpoint{1.604519in}{4.178124in}}{\pgfqpoint{1.600129in}{4.167525in}}{\pgfqpoint{1.600129in}{4.156474in}}%
\pgfpathcurveto{\pgfqpoint{1.600129in}{4.145424in}}{\pgfqpoint{1.604519in}{4.134825in}}{\pgfqpoint{1.612333in}{4.127012in}}%
\pgfpathcurveto{\pgfqpoint{1.620146in}{4.119198in}}{\pgfqpoint{1.630745in}{4.114808in}}{\pgfqpoint{1.641796in}{4.114808in}}%
\pgfpathclose%
\pgfusepath{stroke,fill}%
\end{pgfscope}%
\begin{pgfscope}%
\pgfpathrectangle{\pgfqpoint{0.526127in}{0.331635in}}{\pgfqpoint{9.300000in}{7.700000in}}%
\pgfusepath{clip}%
\pgfsetbuttcap%
\pgfsetroundjoin%
\definecolor{currentfill}{rgb}{0.721569,0.521569,0.039216}%
\pgfsetfillcolor{currentfill}%
\pgfsetlinewidth{0.481800pt}%
\definecolor{currentstroke}{rgb}{1.000000,1.000000,1.000000}%
\pgfsetstrokecolor{currentstroke}%
\pgfsetdash{}{0pt}%
\pgfpathmoveto{\pgfqpoint{2.312627in}{5.957100in}}%
\pgfpathcurveto{\pgfqpoint{2.323677in}{5.957100in}}{\pgfqpoint{2.334276in}{5.961490in}}{\pgfqpoint{2.342089in}{5.969304in}}%
\pgfpathcurveto{\pgfqpoint{2.349903in}{5.977117in}}{\pgfqpoint{2.354293in}{5.987717in}}{\pgfqpoint{2.354293in}{5.998767in}}%
\pgfpathcurveto{\pgfqpoint{2.354293in}{6.009817in}}{\pgfqpoint{2.349903in}{6.020416in}}{\pgfqpoint{2.342089in}{6.028229in}}%
\pgfpathcurveto{\pgfqpoint{2.334276in}{6.036043in}}{\pgfqpoint{2.323677in}{6.040433in}}{\pgfqpoint{2.312627in}{6.040433in}}%
\pgfpathcurveto{\pgfqpoint{2.301576in}{6.040433in}}{\pgfqpoint{2.290977in}{6.036043in}}{\pgfqpoint{2.283164in}{6.028229in}}%
\pgfpathcurveto{\pgfqpoint{2.275350in}{6.020416in}}{\pgfqpoint{2.270960in}{6.009817in}}{\pgfqpoint{2.270960in}{5.998767in}}%
\pgfpathcurveto{\pgfqpoint{2.270960in}{5.987717in}}{\pgfqpoint{2.275350in}{5.977117in}}{\pgfqpoint{2.283164in}{5.969304in}}%
\pgfpathcurveto{\pgfqpoint{2.290977in}{5.961490in}}{\pgfqpoint{2.301576in}{5.957100in}}{\pgfqpoint{2.312627in}{5.957100in}}%
\pgfpathclose%
\pgfusepath{stroke,fill}%
\end{pgfscope}%
\begin{pgfscope}%
\pgfpathrectangle{\pgfqpoint{0.526127in}{0.331635in}}{\pgfqpoint{9.300000in}{7.700000in}}%
\pgfusepath{clip}%
\pgfsetbuttcap%
\pgfsetroundjoin%
\definecolor{currentfill}{rgb}{0.721569,0.521569,0.039216}%
\pgfsetfillcolor{currentfill}%
\pgfsetlinewidth{0.481800pt}%
\definecolor{currentstroke}{rgb}{1.000000,1.000000,1.000000}%
\pgfsetstrokecolor{currentstroke}%
\pgfsetdash{}{0pt}%
\pgfpathmoveto{\pgfqpoint{7.299025in}{5.711229in}}%
\pgfpathcurveto{\pgfqpoint{7.310075in}{5.711229in}}{\pgfqpoint{7.320674in}{5.715620in}}{\pgfqpoint{7.328487in}{5.723433in}}%
\pgfpathcurveto{\pgfqpoint{7.336301in}{5.731247in}}{\pgfqpoint{7.340691in}{5.741846in}}{\pgfqpoint{7.340691in}{5.752896in}}%
\pgfpathcurveto{\pgfqpoint{7.340691in}{5.763946in}}{\pgfqpoint{7.336301in}{5.774545in}}{\pgfqpoint{7.328487in}{5.782359in}}%
\pgfpathcurveto{\pgfqpoint{7.320674in}{5.790172in}}{\pgfqpoint{7.310075in}{5.794563in}}{\pgfqpoint{7.299025in}{5.794563in}}%
\pgfpathcurveto{\pgfqpoint{7.287975in}{5.794563in}}{\pgfqpoint{7.277375in}{5.790172in}}{\pgfqpoint{7.269562in}{5.782359in}}%
\pgfpathcurveto{\pgfqpoint{7.261748in}{5.774545in}}{\pgfqpoint{7.257358in}{5.763946in}}{\pgfqpoint{7.257358in}{5.752896in}}%
\pgfpathcurveto{\pgfqpoint{7.257358in}{5.741846in}}{\pgfqpoint{7.261748in}{5.731247in}}{\pgfqpoint{7.269562in}{5.723433in}}%
\pgfpathcurveto{\pgfqpoint{7.277375in}{5.715620in}}{\pgfqpoint{7.287975in}{5.711229in}}{\pgfqpoint{7.299025in}{5.711229in}}%
\pgfpathclose%
\pgfusepath{stroke,fill}%
\end{pgfscope}%
\begin{pgfscope}%
\pgfpathrectangle{\pgfqpoint{0.526127in}{0.331635in}}{\pgfqpoint{9.300000in}{7.700000in}}%
\pgfusepath{clip}%
\pgfsetbuttcap%
\pgfsetroundjoin%
\definecolor{currentfill}{rgb}{0.721569,0.521569,0.039216}%
\pgfsetfillcolor{currentfill}%
\pgfsetlinewidth{0.481800pt}%
\definecolor{currentstroke}{rgb}{1.000000,1.000000,1.000000}%
\pgfsetstrokecolor{currentstroke}%
\pgfsetdash{}{0pt}%
\pgfpathmoveto{\pgfqpoint{1.663441in}{4.064401in}}%
\pgfpathcurveto{\pgfqpoint{1.674491in}{4.064401in}}{\pgfqpoint{1.685090in}{4.068791in}}{\pgfqpoint{1.692904in}{4.076605in}}%
\pgfpathcurveto{\pgfqpoint{1.700717in}{4.084419in}}{\pgfqpoint{1.705108in}{4.095018in}}{\pgfqpoint{1.705108in}{4.106068in}}%
\pgfpathcurveto{\pgfqpoint{1.705108in}{4.117118in}}{\pgfqpoint{1.700717in}{4.127717in}}{\pgfqpoint{1.692904in}{4.135531in}}%
\pgfpathcurveto{\pgfqpoint{1.685090in}{4.143344in}}{\pgfqpoint{1.674491in}{4.147734in}}{\pgfqpoint{1.663441in}{4.147734in}}%
\pgfpathcurveto{\pgfqpoint{1.652391in}{4.147734in}}{\pgfqpoint{1.641792in}{4.143344in}}{\pgfqpoint{1.633978in}{4.135531in}}%
\pgfpathcurveto{\pgfqpoint{1.626165in}{4.127717in}}{\pgfqpoint{1.621774in}{4.117118in}}{\pgfqpoint{1.621774in}{4.106068in}}%
\pgfpathcurveto{\pgfqpoint{1.621774in}{4.095018in}}{\pgfqpoint{1.626165in}{4.084419in}}{\pgfqpoint{1.633978in}{4.076605in}}%
\pgfpathcurveto{\pgfqpoint{1.641792in}{4.068791in}}{\pgfqpoint{1.652391in}{4.064401in}}{\pgfqpoint{1.663441in}{4.064401in}}%
\pgfpathclose%
\pgfusepath{stroke,fill}%
\end{pgfscope}%
\begin{pgfscope}%
\pgfpathrectangle{\pgfqpoint{0.526127in}{0.331635in}}{\pgfqpoint{9.300000in}{7.700000in}}%
\pgfusepath{clip}%
\pgfsetbuttcap%
\pgfsetroundjoin%
\definecolor{currentfill}{rgb}{0.721569,0.521569,0.039216}%
\pgfsetfillcolor{currentfill}%
\pgfsetlinewidth{0.481800pt}%
\definecolor{currentstroke}{rgb}{1.000000,1.000000,1.000000}%
\pgfsetstrokecolor{currentstroke}%
\pgfsetdash{}{0pt}%
\pgfpathmoveto{\pgfqpoint{9.274981in}{5.517114in}}%
\pgfpathcurveto{\pgfqpoint{9.286031in}{5.517114in}}{\pgfqpoint{9.296630in}{5.521504in}}{\pgfqpoint{9.304444in}{5.529318in}}%
\pgfpathcurveto{\pgfqpoint{9.312257in}{5.537132in}}{\pgfqpoint{9.316647in}{5.547731in}}{\pgfqpoint{9.316647in}{5.558781in}}%
\pgfpathcurveto{\pgfqpoint{9.316647in}{5.569831in}}{\pgfqpoint{9.312257in}{5.580430in}}{\pgfqpoint{9.304444in}{5.588244in}}%
\pgfpathcurveto{\pgfqpoint{9.296630in}{5.596057in}}{\pgfqpoint{9.286031in}{5.600448in}}{\pgfqpoint{9.274981in}{5.600448in}}%
\pgfpathcurveto{\pgfqpoint{9.263931in}{5.600448in}}{\pgfqpoint{9.253332in}{5.596057in}}{\pgfqpoint{9.245518in}{5.588244in}}%
\pgfpathcurveto{\pgfqpoint{9.237704in}{5.580430in}}{\pgfqpoint{9.233314in}{5.569831in}}{\pgfqpoint{9.233314in}{5.558781in}}%
\pgfpathcurveto{\pgfqpoint{9.233314in}{5.547731in}}{\pgfqpoint{9.237704in}{5.537132in}}{\pgfqpoint{9.245518in}{5.529318in}}%
\pgfpathcurveto{\pgfqpoint{9.253332in}{5.521504in}}{\pgfqpoint{9.263931in}{5.517114in}}{\pgfqpoint{9.274981in}{5.517114in}}%
\pgfpathclose%
\pgfusepath{stroke,fill}%
\end{pgfscope}%
\begin{pgfscope}%
\pgfpathrectangle{\pgfqpoint{0.526127in}{0.331635in}}{\pgfqpoint{9.300000in}{7.700000in}}%
\pgfusepath{clip}%
\pgfsetbuttcap%
\pgfsetroundjoin%
\definecolor{currentfill}{rgb}{0.721569,0.521569,0.039216}%
\pgfsetfillcolor{currentfill}%
\pgfsetlinewidth{0.481800pt}%
\definecolor{currentstroke}{rgb}{1.000000,1.000000,1.000000}%
\pgfsetstrokecolor{currentstroke}%
\pgfsetdash{}{0pt}%
\pgfpathmoveto{\pgfqpoint{9.099248in}{4.648975in}}%
\pgfpathcurveto{\pgfqpoint{9.110298in}{4.648975in}}{\pgfqpoint{9.120897in}{4.653365in}}{\pgfqpoint{9.128711in}{4.661179in}}%
\pgfpathcurveto{\pgfqpoint{9.136524in}{4.668992in}}{\pgfqpoint{9.140915in}{4.679591in}}{\pgfqpoint{9.140915in}{4.690642in}}%
\pgfpathcurveto{\pgfqpoint{9.140915in}{4.701692in}}{\pgfqpoint{9.136524in}{4.712291in}}{\pgfqpoint{9.128711in}{4.720104in}}%
\pgfpathcurveto{\pgfqpoint{9.120897in}{4.727918in}}{\pgfqpoint{9.110298in}{4.732308in}}{\pgfqpoint{9.099248in}{4.732308in}}%
\pgfpathcurveto{\pgfqpoint{9.088198in}{4.732308in}}{\pgfqpoint{9.077599in}{4.727918in}}{\pgfqpoint{9.069785in}{4.720104in}}%
\pgfpathcurveto{\pgfqpoint{9.061972in}{4.712291in}}{\pgfqpoint{9.057581in}{4.701692in}}{\pgfqpoint{9.057581in}{4.690642in}}%
\pgfpathcurveto{\pgfqpoint{9.057581in}{4.679591in}}{\pgfqpoint{9.061972in}{4.668992in}}{\pgfqpoint{9.069785in}{4.661179in}}%
\pgfpathcurveto{\pgfqpoint{9.077599in}{4.653365in}}{\pgfqpoint{9.088198in}{4.648975in}}{\pgfqpoint{9.099248in}{4.648975in}}%
\pgfpathclose%
\pgfusepath{stroke,fill}%
\end{pgfscope}%
\begin{pgfscope}%
\pgfpathrectangle{\pgfqpoint{0.526127in}{0.331635in}}{\pgfqpoint{9.300000in}{7.700000in}}%
\pgfusepath{clip}%
\pgfsetbuttcap%
\pgfsetroundjoin%
\definecolor{currentfill}{rgb}{0.721569,0.521569,0.039216}%
\pgfsetfillcolor{currentfill}%
\pgfsetlinewidth{0.481800pt}%
\definecolor{currentstroke}{rgb}{1.000000,1.000000,1.000000}%
\pgfsetstrokecolor{currentstroke}%
\pgfsetdash{}{0pt}%
\pgfpathmoveto{\pgfqpoint{4.144255in}{5.844477in}}%
\pgfpathcurveto{\pgfqpoint{4.155306in}{5.844477in}}{\pgfqpoint{4.165905in}{5.848867in}}{\pgfqpoint{4.173718in}{5.856681in}}%
\pgfpathcurveto{\pgfqpoint{4.181532in}{5.864494in}}{\pgfqpoint{4.185922in}{5.875093in}}{\pgfqpoint{4.185922in}{5.886144in}}%
\pgfpathcurveto{\pgfqpoint{4.185922in}{5.897194in}}{\pgfqpoint{4.181532in}{5.907793in}}{\pgfqpoint{4.173718in}{5.915606in}}%
\pgfpathcurveto{\pgfqpoint{4.165905in}{5.923420in}}{\pgfqpoint{4.155306in}{5.927810in}}{\pgfqpoint{4.144255in}{5.927810in}}%
\pgfpathcurveto{\pgfqpoint{4.133205in}{5.927810in}}{\pgfqpoint{4.122606in}{5.923420in}}{\pgfqpoint{4.114793in}{5.915606in}}%
\pgfpathcurveto{\pgfqpoint{4.106979in}{5.907793in}}{\pgfqpoint{4.102589in}{5.897194in}}{\pgfqpoint{4.102589in}{5.886144in}}%
\pgfpathcurveto{\pgfqpoint{4.102589in}{5.875093in}}{\pgfqpoint{4.106979in}{5.864494in}}{\pgfqpoint{4.114793in}{5.856681in}}%
\pgfpathcurveto{\pgfqpoint{4.122606in}{5.848867in}}{\pgfqpoint{4.133205in}{5.844477in}}{\pgfqpoint{4.144255in}{5.844477in}}%
\pgfpathclose%
\pgfusepath{stroke,fill}%
\end{pgfscope}%
\begin{pgfscope}%
\pgfpathrectangle{\pgfqpoint{0.526127in}{0.331635in}}{\pgfqpoint{9.300000in}{7.700000in}}%
\pgfusepath{clip}%
\pgfsetbuttcap%
\pgfsetroundjoin%
\definecolor{currentfill}{rgb}{0.721569,0.521569,0.039216}%
\pgfsetfillcolor{currentfill}%
\pgfsetlinewidth{0.481800pt}%
\definecolor{currentstroke}{rgb}{1.000000,1.000000,1.000000}%
\pgfsetstrokecolor{currentstroke}%
\pgfsetdash{}{0pt}%
\pgfpathmoveto{\pgfqpoint{6.389607in}{4.472796in}}%
\pgfpathcurveto{\pgfqpoint{6.400657in}{4.472796in}}{\pgfqpoint{6.411256in}{4.477187in}}{\pgfqpoint{6.419069in}{4.485000in}}%
\pgfpathcurveto{\pgfqpoint{6.426883in}{4.492814in}}{\pgfqpoint{6.431273in}{4.503413in}}{\pgfqpoint{6.431273in}{4.514463in}}%
\pgfpathcurveto{\pgfqpoint{6.431273in}{4.525513in}}{\pgfqpoint{6.426883in}{4.536112in}}{\pgfqpoint{6.419069in}{4.543926in}}%
\pgfpathcurveto{\pgfqpoint{6.411256in}{4.551739in}}{\pgfqpoint{6.400657in}{4.556130in}}{\pgfqpoint{6.389607in}{4.556130in}}%
\pgfpathcurveto{\pgfqpoint{6.378556in}{4.556130in}}{\pgfqpoint{6.367957in}{4.551739in}}{\pgfqpoint{6.360144in}{4.543926in}}%
\pgfpathcurveto{\pgfqpoint{6.352330in}{4.536112in}}{\pgfqpoint{6.347940in}{4.525513in}}{\pgfqpoint{6.347940in}{4.514463in}}%
\pgfpathcurveto{\pgfqpoint{6.347940in}{4.503413in}}{\pgfqpoint{6.352330in}{4.492814in}}{\pgfqpoint{6.360144in}{4.485000in}}%
\pgfpathcurveto{\pgfqpoint{6.367957in}{4.477187in}}{\pgfqpoint{6.378556in}{4.472796in}}{\pgfqpoint{6.389607in}{4.472796in}}%
\pgfpathclose%
\pgfusepath{stroke,fill}%
\end{pgfscope}%
\begin{pgfscope}%
\pgfpathrectangle{\pgfqpoint{0.526127in}{0.331635in}}{\pgfqpoint{9.300000in}{7.700000in}}%
\pgfusepath{clip}%
\pgfsetbuttcap%
\pgfsetroundjoin%
\definecolor{currentfill}{rgb}{0.721569,0.521569,0.039216}%
\pgfsetfillcolor{currentfill}%
\pgfsetlinewidth{0.481800pt}%
\definecolor{currentstroke}{rgb}{1.000000,1.000000,1.000000}%
\pgfsetstrokecolor{currentstroke}%
\pgfsetdash{}{0pt}%
\pgfpathmoveto{\pgfqpoint{3.057300in}{5.078843in}}%
\pgfpathcurveto{\pgfqpoint{3.068351in}{5.078843in}}{\pgfqpoint{3.078950in}{5.083233in}}{\pgfqpoint{3.086763in}{5.091046in}}%
\pgfpathcurveto{\pgfqpoint{3.094577in}{5.098860in}}{\pgfqpoint{3.098967in}{5.109459in}}{\pgfqpoint{3.098967in}{5.120509in}}%
\pgfpathcurveto{\pgfqpoint{3.098967in}{5.131559in}}{\pgfqpoint{3.094577in}{5.142158in}}{\pgfqpoint{3.086763in}{5.149972in}}%
\pgfpathcurveto{\pgfqpoint{3.078950in}{5.157786in}}{\pgfqpoint{3.068351in}{5.162176in}}{\pgfqpoint{3.057300in}{5.162176in}}%
\pgfpathcurveto{\pgfqpoint{3.046250in}{5.162176in}}{\pgfqpoint{3.035651in}{5.157786in}}{\pgfqpoint{3.027838in}{5.149972in}}%
\pgfpathcurveto{\pgfqpoint{3.020024in}{5.142158in}}{\pgfqpoint{3.015634in}{5.131559in}}{\pgfqpoint{3.015634in}{5.120509in}}%
\pgfpathcurveto{\pgfqpoint{3.015634in}{5.109459in}}{\pgfqpoint{3.020024in}{5.098860in}}{\pgfqpoint{3.027838in}{5.091046in}}%
\pgfpathcurveto{\pgfqpoint{3.035651in}{5.083233in}}{\pgfqpoint{3.046250in}{5.078843in}}{\pgfqpoint{3.057300in}{5.078843in}}%
\pgfpathclose%
\pgfusepath{stroke,fill}%
\end{pgfscope}%
\begin{pgfscope}%
\pgfpathrectangle{\pgfqpoint{0.526127in}{0.331635in}}{\pgfqpoint{9.300000in}{7.700000in}}%
\pgfusepath{clip}%
\pgfsetbuttcap%
\pgfsetroundjoin%
\definecolor{currentfill}{rgb}{0.721569,0.521569,0.039216}%
\pgfsetfillcolor{currentfill}%
\pgfsetlinewidth{0.481800pt}%
\definecolor{currentstroke}{rgb}{1.000000,1.000000,1.000000}%
\pgfsetstrokecolor{currentstroke}%
\pgfsetdash{}{0pt}%
\pgfpathmoveto{\pgfqpoint{1.903215in}{4.884775in}}%
\pgfpathcurveto{\pgfqpoint{1.914265in}{4.884775in}}{\pgfqpoint{1.924864in}{4.889165in}}{\pgfqpoint{1.932678in}{4.896979in}}%
\pgfpathcurveto{\pgfqpoint{1.940492in}{4.904793in}}{\pgfqpoint{1.944882in}{4.915392in}}{\pgfqpoint{1.944882in}{4.926442in}}%
\pgfpathcurveto{\pgfqpoint{1.944882in}{4.937492in}}{\pgfqpoint{1.940492in}{4.948091in}}{\pgfqpoint{1.932678in}{4.955904in}}%
\pgfpathcurveto{\pgfqpoint{1.924864in}{4.963718in}}{\pgfqpoint{1.914265in}{4.968108in}}{\pgfqpoint{1.903215in}{4.968108in}}%
\pgfpathcurveto{\pgfqpoint{1.892165in}{4.968108in}}{\pgfqpoint{1.881566in}{4.963718in}}{\pgfqpoint{1.873752in}{4.955904in}}%
\pgfpathcurveto{\pgfqpoint{1.865939in}{4.948091in}}{\pgfqpoint{1.861549in}{4.937492in}}{\pgfqpoint{1.861549in}{4.926442in}}%
\pgfpathcurveto{\pgfqpoint{1.861549in}{4.915392in}}{\pgfqpoint{1.865939in}{4.904793in}}{\pgfqpoint{1.873752in}{4.896979in}}%
\pgfpathcurveto{\pgfqpoint{1.881566in}{4.889165in}}{\pgfqpoint{1.892165in}{4.884775in}}{\pgfqpoint{1.903215in}{4.884775in}}%
\pgfpathclose%
\pgfusepath{stroke,fill}%
\end{pgfscope}%
\begin{pgfscope}%
\pgfpathrectangle{\pgfqpoint{0.526127in}{0.331635in}}{\pgfqpoint{9.300000in}{7.700000in}}%
\pgfusepath{clip}%
\pgfsetbuttcap%
\pgfsetroundjoin%
\definecolor{currentfill}{rgb}{0.631373,0.788235,0.956863}%
\pgfsetfillcolor{currentfill}%
\pgfsetlinewidth{1.003750pt}%
\definecolor{currentstroke}{rgb}{0.631373,0.788235,0.956863}%
\pgfsetstrokecolor{currentstroke}%
\pgfsetdash{}{0pt}%
\pgfsys@defobject{currentmarker}{\pgfqpoint{-0.041667in}{-0.041667in}}{\pgfqpoint{0.041667in}{0.041667in}}{%
\pgfpathmoveto{\pgfqpoint{0.000000in}{-0.041667in}}%
\pgfpathcurveto{\pgfqpoint{0.011050in}{-0.041667in}}{\pgfqpoint{0.021649in}{-0.037276in}}{\pgfqpoint{0.029463in}{-0.029463in}}%
\pgfpathcurveto{\pgfqpoint{0.037276in}{-0.021649in}}{\pgfqpoint{0.041667in}{-0.011050in}}{\pgfqpoint{0.041667in}{0.000000in}}%
\pgfpathcurveto{\pgfqpoint{0.041667in}{0.011050in}}{\pgfqpoint{0.037276in}{0.021649in}}{\pgfqpoint{0.029463in}{0.029463in}}%
\pgfpathcurveto{\pgfqpoint{0.021649in}{0.037276in}}{\pgfqpoint{0.011050in}{0.041667in}}{\pgfqpoint{0.000000in}{0.041667in}}%
\pgfpathcurveto{\pgfqpoint{-0.011050in}{0.041667in}}{\pgfqpoint{-0.021649in}{0.037276in}}{\pgfqpoint{-0.029463in}{0.029463in}}%
\pgfpathcurveto{\pgfqpoint{-0.037276in}{0.021649in}}{\pgfqpoint{-0.041667in}{0.011050in}}{\pgfqpoint{-0.041667in}{0.000000in}}%
\pgfpathcurveto{\pgfqpoint{-0.041667in}{-0.011050in}}{\pgfqpoint{-0.037276in}{-0.021649in}}{\pgfqpoint{-0.029463in}{-0.029463in}}%
\pgfpathcurveto{\pgfqpoint{-0.021649in}{-0.037276in}}{\pgfqpoint{-0.011050in}{-0.041667in}}{\pgfqpoint{0.000000in}{-0.041667in}}%
\pgfpathclose%
\pgfusepath{stroke,fill}%
}%
\end{pgfscope}%
\begin{pgfscope}%
\pgfpathrectangle{\pgfqpoint{0.526127in}{0.331635in}}{\pgfqpoint{9.300000in}{7.700000in}}%
\pgfusepath{clip}%
\pgfsetbuttcap%
\pgfsetroundjoin%
\definecolor{currentfill}{rgb}{1.000000,0.705882,0.509804}%
\pgfsetfillcolor{currentfill}%
\pgfsetlinewidth{1.003750pt}%
\definecolor{currentstroke}{rgb}{1.000000,0.705882,0.509804}%
\pgfsetstrokecolor{currentstroke}%
\pgfsetdash{}{0pt}%
\pgfsys@defobject{currentmarker}{\pgfqpoint{-0.041667in}{-0.041667in}}{\pgfqpoint{0.041667in}{0.041667in}}{%
\pgfpathmoveto{\pgfqpoint{0.000000in}{-0.041667in}}%
\pgfpathcurveto{\pgfqpoint{0.011050in}{-0.041667in}}{\pgfqpoint{0.021649in}{-0.037276in}}{\pgfqpoint{0.029463in}{-0.029463in}}%
\pgfpathcurveto{\pgfqpoint{0.037276in}{-0.021649in}}{\pgfqpoint{0.041667in}{-0.011050in}}{\pgfqpoint{0.041667in}{0.000000in}}%
\pgfpathcurveto{\pgfqpoint{0.041667in}{0.011050in}}{\pgfqpoint{0.037276in}{0.021649in}}{\pgfqpoint{0.029463in}{0.029463in}}%
\pgfpathcurveto{\pgfqpoint{0.021649in}{0.037276in}}{\pgfqpoint{0.011050in}{0.041667in}}{\pgfqpoint{0.000000in}{0.041667in}}%
\pgfpathcurveto{\pgfqpoint{-0.011050in}{0.041667in}}{\pgfqpoint{-0.021649in}{0.037276in}}{\pgfqpoint{-0.029463in}{0.029463in}}%
\pgfpathcurveto{\pgfqpoint{-0.037276in}{0.021649in}}{\pgfqpoint{-0.041667in}{0.011050in}}{\pgfqpoint{-0.041667in}{0.000000in}}%
\pgfpathcurveto{\pgfqpoint{-0.041667in}{-0.011050in}}{\pgfqpoint{-0.037276in}{-0.021649in}}{\pgfqpoint{-0.029463in}{-0.029463in}}%
\pgfpathcurveto{\pgfqpoint{-0.021649in}{-0.037276in}}{\pgfqpoint{-0.011050in}{-0.041667in}}{\pgfqpoint{0.000000in}{-0.041667in}}%
\pgfpathclose%
\pgfusepath{stroke,fill}%
}%
\end{pgfscope}%
\begin{pgfscope}%
\pgfpathrectangle{\pgfqpoint{0.526127in}{0.331635in}}{\pgfqpoint{9.300000in}{7.700000in}}%
\pgfusepath{clip}%
\pgfsetbuttcap%
\pgfsetroundjoin%
\definecolor{currentfill}{rgb}{0.552941,0.898039,0.631373}%
\pgfsetfillcolor{currentfill}%
\pgfsetlinewidth{1.003750pt}%
\definecolor{currentstroke}{rgb}{0.552941,0.898039,0.631373}%
\pgfsetstrokecolor{currentstroke}%
\pgfsetdash{}{0pt}%
\pgfsys@defobject{currentmarker}{\pgfqpoint{-0.041667in}{-0.041667in}}{\pgfqpoint{0.041667in}{0.041667in}}{%
\pgfpathmoveto{\pgfqpoint{0.000000in}{-0.041667in}}%
\pgfpathcurveto{\pgfqpoint{0.011050in}{-0.041667in}}{\pgfqpoint{0.021649in}{-0.037276in}}{\pgfqpoint{0.029463in}{-0.029463in}}%
\pgfpathcurveto{\pgfqpoint{0.037276in}{-0.021649in}}{\pgfqpoint{0.041667in}{-0.011050in}}{\pgfqpoint{0.041667in}{0.000000in}}%
\pgfpathcurveto{\pgfqpoint{0.041667in}{0.011050in}}{\pgfqpoint{0.037276in}{0.021649in}}{\pgfqpoint{0.029463in}{0.029463in}}%
\pgfpathcurveto{\pgfqpoint{0.021649in}{0.037276in}}{\pgfqpoint{0.011050in}{0.041667in}}{\pgfqpoint{0.000000in}{0.041667in}}%
\pgfpathcurveto{\pgfqpoint{-0.011050in}{0.041667in}}{\pgfqpoint{-0.021649in}{0.037276in}}{\pgfqpoint{-0.029463in}{0.029463in}}%
\pgfpathcurveto{\pgfqpoint{-0.037276in}{0.021649in}}{\pgfqpoint{-0.041667in}{0.011050in}}{\pgfqpoint{-0.041667in}{0.000000in}}%
\pgfpathcurveto{\pgfqpoint{-0.041667in}{-0.011050in}}{\pgfqpoint{-0.037276in}{-0.021649in}}{\pgfqpoint{-0.029463in}{-0.029463in}}%
\pgfpathcurveto{\pgfqpoint{-0.021649in}{-0.037276in}}{\pgfqpoint{-0.011050in}{-0.041667in}}{\pgfqpoint{0.000000in}{-0.041667in}}%
\pgfpathclose%
\pgfusepath{stroke,fill}%
}%
\end{pgfscope}%
\begin{pgfscope}%
\pgfpathrectangle{\pgfqpoint{0.526127in}{0.331635in}}{\pgfqpoint{9.300000in}{7.700000in}}%
\pgfusepath{clip}%
\pgfsetbuttcap%
\pgfsetroundjoin%
\definecolor{currentfill}{rgb}{1.000000,0.623529,0.607843}%
\pgfsetfillcolor{currentfill}%
\pgfsetlinewidth{1.003750pt}%
\definecolor{currentstroke}{rgb}{1.000000,0.623529,0.607843}%
\pgfsetstrokecolor{currentstroke}%
\pgfsetdash{}{0pt}%
\pgfsys@defobject{currentmarker}{\pgfqpoint{-0.041667in}{-0.041667in}}{\pgfqpoint{0.041667in}{0.041667in}}{%
\pgfpathmoveto{\pgfqpoint{0.000000in}{-0.041667in}}%
\pgfpathcurveto{\pgfqpoint{0.011050in}{-0.041667in}}{\pgfqpoint{0.021649in}{-0.037276in}}{\pgfqpoint{0.029463in}{-0.029463in}}%
\pgfpathcurveto{\pgfqpoint{0.037276in}{-0.021649in}}{\pgfqpoint{0.041667in}{-0.011050in}}{\pgfqpoint{0.041667in}{0.000000in}}%
\pgfpathcurveto{\pgfqpoint{0.041667in}{0.011050in}}{\pgfqpoint{0.037276in}{0.021649in}}{\pgfqpoint{0.029463in}{0.029463in}}%
\pgfpathcurveto{\pgfqpoint{0.021649in}{0.037276in}}{\pgfqpoint{0.011050in}{0.041667in}}{\pgfqpoint{0.000000in}{0.041667in}}%
\pgfpathcurveto{\pgfqpoint{-0.011050in}{0.041667in}}{\pgfqpoint{-0.021649in}{0.037276in}}{\pgfqpoint{-0.029463in}{0.029463in}}%
\pgfpathcurveto{\pgfqpoint{-0.037276in}{0.021649in}}{\pgfqpoint{-0.041667in}{0.011050in}}{\pgfqpoint{-0.041667in}{0.000000in}}%
\pgfpathcurveto{\pgfqpoint{-0.041667in}{-0.011050in}}{\pgfqpoint{-0.037276in}{-0.021649in}}{\pgfqpoint{-0.029463in}{-0.029463in}}%
\pgfpathcurveto{\pgfqpoint{-0.021649in}{-0.037276in}}{\pgfqpoint{-0.011050in}{-0.041667in}}{\pgfqpoint{0.000000in}{-0.041667in}}%
\pgfpathclose%
\pgfusepath{stroke,fill}%
}%
\end{pgfscope}%
\begin{pgfscope}%
\pgfpathrectangle{\pgfqpoint{0.526127in}{0.331635in}}{\pgfqpoint{9.300000in}{7.700000in}}%
\pgfusepath{clip}%
\pgfsetbuttcap%
\pgfsetroundjoin%
\definecolor{currentfill}{rgb}{0.815686,0.733333,1.000000}%
\pgfsetfillcolor{currentfill}%
\pgfsetlinewidth{1.003750pt}%
\definecolor{currentstroke}{rgb}{0.815686,0.733333,1.000000}%
\pgfsetstrokecolor{currentstroke}%
\pgfsetdash{}{0pt}%
\pgfsys@defobject{currentmarker}{\pgfqpoint{-0.041667in}{-0.041667in}}{\pgfqpoint{0.041667in}{0.041667in}}{%
\pgfpathmoveto{\pgfqpoint{0.000000in}{-0.041667in}}%
\pgfpathcurveto{\pgfqpoint{0.011050in}{-0.041667in}}{\pgfqpoint{0.021649in}{-0.037276in}}{\pgfqpoint{0.029463in}{-0.029463in}}%
\pgfpathcurveto{\pgfqpoint{0.037276in}{-0.021649in}}{\pgfqpoint{0.041667in}{-0.011050in}}{\pgfqpoint{0.041667in}{0.000000in}}%
\pgfpathcurveto{\pgfqpoint{0.041667in}{0.011050in}}{\pgfqpoint{0.037276in}{0.021649in}}{\pgfqpoint{0.029463in}{0.029463in}}%
\pgfpathcurveto{\pgfqpoint{0.021649in}{0.037276in}}{\pgfqpoint{0.011050in}{0.041667in}}{\pgfqpoint{0.000000in}{0.041667in}}%
\pgfpathcurveto{\pgfqpoint{-0.011050in}{0.041667in}}{\pgfqpoint{-0.021649in}{0.037276in}}{\pgfqpoint{-0.029463in}{0.029463in}}%
\pgfpathcurveto{\pgfqpoint{-0.037276in}{0.021649in}}{\pgfqpoint{-0.041667in}{0.011050in}}{\pgfqpoint{-0.041667in}{0.000000in}}%
\pgfpathcurveto{\pgfqpoint{-0.041667in}{-0.011050in}}{\pgfqpoint{-0.037276in}{-0.021649in}}{\pgfqpoint{-0.029463in}{-0.029463in}}%
\pgfpathcurveto{\pgfqpoint{-0.021649in}{-0.037276in}}{\pgfqpoint{-0.011050in}{-0.041667in}}{\pgfqpoint{0.000000in}{-0.041667in}}%
\pgfpathclose%
\pgfusepath{stroke,fill}%
}%
\end{pgfscope}%
\begin{pgfscope}%
\pgfpathrectangle{\pgfqpoint{0.526127in}{0.331635in}}{\pgfqpoint{9.300000in}{7.700000in}}%
\pgfusepath{clip}%
\pgfsetbuttcap%
\pgfsetroundjoin%
\definecolor{currentfill}{rgb}{0.870588,0.733333,0.607843}%
\pgfsetfillcolor{currentfill}%
\pgfsetlinewidth{1.003750pt}%
\definecolor{currentstroke}{rgb}{0.870588,0.733333,0.607843}%
\pgfsetstrokecolor{currentstroke}%
\pgfsetdash{}{0pt}%
\pgfsys@defobject{currentmarker}{\pgfqpoint{-0.041667in}{-0.041667in}}{\pgfqpoint{0.041667in}{0.041667in}}{%
\pgfpathmoveto{\pgfqpoint{0.000000in}{-0.041667in}}%
\pgfpathcurveto{\pgfqpoint{0.011050in}{-0.041667in}}{\pgfqpoint{0.021649in}{-0.037276in}}{\pgfqpoint{0.029463in}{-0.029463in}}%
\pgfpathcurveto{\pgfqpoint{0.037276in}{-0.021649in}}{\pgfqpoint{0.041667in}{-0.011050in}}{\pgfqpoint{0.041667in}{0.000000in}}%
\pgfpathcurveto{\pgfqpoint{0.041667in}{0.011050in}}{\pgfqpoint{0.037276in}{0.021649in}}{\pgfqpoint{0.029463in}{0.029463in}}%
\pgfpathcurveto{\pgfqpoint{0.021649in}{0.037276in}}{\pgfqpoint{0.011050in}{0.041667in}}{\pgfqpoint{0.000000in}{0.041667in}}%
\pgfpathcurveto{\pgfqpoint{-0.011050in}{0.041667in}}{\pgfqpoint{-0.021649in}{0.037276in}}{\pgfqpoint{-0.029463in}{0.029463in}}%
\pgfpathcurveto{\pgfqpoint{-0.037276in}{0.021649in}}{\pgfqpoint{-0.041667in}{0.011050in}}{\pgfqpoint{-0.041667in}{0.000000in}}%
\pgfpathcurveto{\pgfqpoint{-0.041667in}{-0.011050in}}{\pgfqpoint{-0.037276in}{-0.021649in}}{\pgfqpoint{-0.029463in}{-0.029463in}}%
\pgfpathcurveto{\pgfqpoint{-0.021649in}{-0.037276in}}{\pgfqpoint{-0.011050in}{-0.041667in}}{\pgfqpoint{0.000000in}{-0.041667in}}%
\pgfpathclose%
\pgfusepath{stroke,fill}%
}%
\end{pgfscope}%
\begin{pgfscope}%
\pgfpathrectangle{\pgfqpoint{0.526127in}{0.331635in}}{\pgfqpoint{9.300000in}{7.700000in}}%
\pgfusepath{clip}%
\pgfsetbuttcap%
\pgfsetroundjoin%
\definecolor{currentfill}{rgb}{0.980392,0.690196,0.894118}%
\pgfsetfillcolor{currentfill}%
\pgfsetlinewidth{1.003750pt}%
\definecolor{currentstroke}{rgb}{0.980392,0.690196,0.894118}%
\pgfsetstrokecolor{currentstroke}%
\pgfsetdash{}{0pt}%
\pgfsys@defobject{currentmarker}{\pgfqpoint{-0.041667in}{-0.041667in}}{\pgfqpoint{0.041667in}{0.041667in}}{%
\pgfpathmoveto{\pgfqpoint{0.000000in}{-0.041667in}}%
\pgfpathcurveto{\pgfqpoint{0.011050in}{-0.041667in}}{\pgfqpoint{0.021649in}{-0.037276in}}{\pgfqpoint{0.029463in}{-0.029463in}}%
\pgfpathcurveto{\pgfqpoint{0.037276in}{-0.021649in}}{\pgfqpoint{0.041667in}{-0.011050in}}{\pgfqpoint{0.041667in}{0.000000in}}%
\pgfpathcurveto{\pgfqpoint{0.041667in}{0.011050in}}{\pgfqpoint{0.037276in}{0.021649in}}{\pgfqpoint{0.029463in}{0.029463in}}%
\pgfpathcurveto{\pgfqpoint{0.021649in}{0.037276in}}{\pgfqpoint{0.011050in}{0.041667in}}{\pgfqpoint{0.000000in}{0.041667in}}%
\pgfpathcurveto{\pgfqpoint{-0.011050in}{0.041667in}}{\pgfqpoint{-0.021649in}{0.037276in}}{\pgfqpoint{-0.029463in}{0.029463in}}%
\pgfpathcurveto{\pgfqpoint{-0.037276in}{0.021649in}}{\pgfqpoint{-0.041667in}{0.011050in}}{\pgfqpoint{-0.041667in}{0.000000in}}%
\pgfpathcurveto{\pgfqpoint{-0.041667in}{-0.011050in}}{\pgfqpoint{-0.037276in}{-0.021649in}}{\pgfqpoint{-0.029463in}{-0.029463in}}%
\pgfpathcurveto{\pgfqpoint{-0.021649in}{-0.037276in}}{\pgfqpoint{-0.011050in}{-0.041667in}}{\pgfqpoint{0.000000in}{-0.041667in}}%
\pgfpathclose%
\pgfusepath{stroke,fill}%
}%
\end{pgfscope}%
\begin{pgfscope}%
\pgfpathrectangle{\pgfqpoint{0.526127in}{0.331635in}}{\pgfqpoint{9.300000in}{7.700000in}}%
\pgfusepath{clip}%
\pgfsetbuttcap%
\pgfsetroundjoin%
\definecolor{currentfill}{rgb}{0.721569,0.521569,0.039216}%
\pgfsetfillcolor{currentfill}%
\pgfsetlinewidth{1.003750pt}%
\definecolor{currentstroke}{rgb}{0.721569,0.521569,0.039216}%
\pgfsetstrokecolor{currentstroke}%
\pgfsetdash{}{0pt}%
\pgfsys@defobject{currentmarker}{\pgfqpoint{-0.041667in}{-0.041667in}}{\pgfqpoint{0.041667in}{0.041667in}}{%
\pgfpathmoveto{\pgfqpoint{0.000000in}{-0.041667in}}%
\pgfpathcurveto{\pgfqpoint{0.011050in}{-0.041667in}}{\pgfqpoint{0.021649in}{-0.037276in}}{\pgfqpoint{0.029463in}{-0.029463in}}%
\pgfpathcurveto{\pgfqpoint{0.037276in}{-0.021649in}}{\pgfqpoint{0.041667in}{-0.011050in}}{\pgfqpoint{0.041667in}{0.000000in}}%
\pgfpathcurveto{\pgfqpoint{0.041667in}{0.011050in}}{\pgfqpoint{0.037276in}{0.021649in}}{\pgfqpoint{0.029463in}{0.029463in}}%
\pgfpathcurveto{\pgfqpoint{0.021649in}{0.037276in}}{\pgfqpoint{0.011050in}{0.041667in}}{\pgfqpoint{0.000000in}{0.041667in}}%
\pgfpathcurveto{\pgfqpoint{-0.011050in}{0.041667in}}{\pgfqpoint{-0.021649in}{0.037276in}}{\pgfqpoint{-0.029463in}{0.029463in}}%
\pgfpathcurveto{\pgfqpoint{-0.037276in}{0.021649in}}{\pgfqpoint{-0.041667in}{0.011050in}}{\pgfqpoint{-0.041667in}{0.000000in}}%
\pgfpathcurveto{\pgfqpoint{-0.041667in}{-0.011050in}}{\pgfqpoint{-0.037276in}{-0.021649in}}{\pgfqpoint{-0.029463in}{-0.029463in}}%
\pgfpathcurveto{\pgfqpoint{-0.021649in}{-0.037276in}}{\pgfqpoint{-0.011050in}{-0.041667in}}{\pgfqpoint{0.000000in}{-0.041667in}}%
\pgfpathclose%
\pgfusepath{stroke,fill}%
}%
\end{pgfscope}%
\begin{pgfscope}%
\pgfsetbuttcap%
\pgfsetroundjoin%
\definecolor{currentfill}{rgb}{0.000000,0.000000,0.000000}%
\pgfsetfillcolor{currentfill}%
\pgfsetlinewidth{0.803000pt}%
\definecolor{currentstroke}{rgb}{0.000000,0.000000,0.000000}%
\pgfsetstrokecolor{currentstroke}%
\pgfsetdash{}{0pt}%
\pgfsys@defobject{currentmarker}{\pgfqpoint{0.000000in}{-0.048611in}}{\pgfqpoint{0.000000in}{0.000000in}}{%
\pgfpathmoveto{\pgfqpoint{0.000000in}{0.000000in}}%
\pgfpathlineto{\pgfqpoint{0.000000in}{-0.048611in}}%
\pgfusepath{stroke,fill}%
}%
\begin{pgfscope}%
\pgfsys@transformshift{1.954606in}{0.331635in}%
\pgfsys@useobject{currentmarker}{}%
\end{pgfscope}%
\end{pgfscope}%
\begin{pgfscope}%
\definecolor{textcolor}{rgb}{0.000000,0.000000,0.000000}%
\pgfsetstrokecolor{textcolor}%
\pgfsetfillcolor{textcolor}%
\pgftext[x=1.954606in,y=0.234413in,,top]{\color{textcolor}\sffamily\fontsize{10.000000}{12.000000}\selectfont \ensuremath{-}10}%
\end{pgfscope}%
\begin{pgfscope}%
\pgfsetbuttcap%
\pgfsetroundjoin%
\definecolor{currentfill}{rgb}{0.000000,0.000000,0.000000}%
\pgfsetfillcolor{currentfill}%
\pgfsetlinewidth{0.803000pt}%
\definecolor{currentstroke}{rgb}{0.000000,0.000000,0.000000}%
\pgfsetstrokecolor{currentstroke}%
\pgfsetdash{}{0pt}%
\pgfsys@defobject{currentmarker}{\pgfqpoint{0.000000in}{-0.048611in}}{\pgfqpoint{0.000000in}{0.000000in}}{%
\pgfpathmoveto{\pgfqpoint{0.000000in}{0.000000in}}%
\pgfpathlineto{\pgfqpoint{0.000000in}{-0.048611in}}%
\pgfusepath{stroke,fill}%
}%
\begin{pgfscope}%
\pgfsys@transformshift{3.838891in}{0.331635in}%
\pgfsys@useobject{currentmarker}{}%
\end{pgfscope}%
\end{pgfscope}%
\begin{pgfscope}%
\definecolor{textcolor}{rgb}{0.000000,0.000000,0.000000}%
\pgfsetstrokecolor{textcolor}%
\pgfsetfillcolor{textcolor}%
\pgftext[x=3.838891in,y=0.234413in,,top]{\color{textcolor}\sffamily\fontsize{10.000000}{12.000000}\selectfont \ensuremath{-}5}%
\end{pgfscope}%
\begin{pgfscope}%
\pgfsetbuttcap%
\pgfsetroundjoin%
\definecolor{currentfill}{rgb}{0.000000,0.000000,0.000000}%
\pgfsetfillcolor{currentfill}%
\pgfsetlinewidth{0.803000pt}%
\definecolor{currentstroke}{rgb}{0.000000,0.000000,0.000000}%
\pgfsetstrokecolor{currentstroke}%
\pgfsetdash{}{0pt}%
\pgfsys@defobject{currentmarker}{\pgfqpoint{0.000000in}{-0.048611in}}{\pgfqpoint{0.000000in}{0.000000in}}{%
\pgfpathmoveto{\pgfqpoint{0.000000in}{0.000000in}}%
\pgfpathlineto{\pgfqpoint{0.000000in}{-0.048611in}}%
\pgfusepath{stroke,fill}%
}%
\begin{pgfscope}%
\pgfsys@transformshift{5.723177in}{0.331635in}%
\pgfsys@useobject{currentmarker}{}%
\end{pgfscope}%
\end{pgfscope}%
\begin{pgfscope}%
\definecolor{textcolor}{rgb}{0.000000,0.000000,0.000000}%
\pgfsetstrokecolor{textcolor}%
\pgfsetfillcolor{textcolor}%
\pgftext[x=5.723177in,y=0.234413in,,top]{\color{textcolor}\sffamily\fontsize{10.000000}{12.000000}\selectfont 0}%
\end{pgfscope}%
\begin{pgfscope}%
\pgfsetbuttcap%
\pgfsetroundjoin%
\definecolor{currentfill}{rgb}{0.000000,0.000000,0.000000}%
\pgfsetfillcolor{currentfill}%
\pgfsetlinewidth{0.803000pt}%
\definecolor{currentstroke}{rgb}{0.000000,0.000000,0.000000}%
\pgfsetstrokecolor{currentstroke}%
\pgfsetdash{}{0pt}%
\pgfsys@defobject{currentmarker}{\pgfqpoint{0.000000in}{-0.048611in}}{\pgfqpoint{0.000000in}{0.000000in}}{%
\pgfpathmoveto{\pgfqpoint{0.000000in}{0.000000in}}%
\pgfpathlineto{\pgfqpoint{0.000000in}{-0.048611in}}%
\pgfusepath{stroke,fill}%
}%
\begin{pgfscope}%
\pgfsys@transformshift{7.607462in}{0.331635in}%
\pgfsys@useobject{currentmarker}{}%
\end{pgfscope}%
\end{pgfscope}%
\begin{pgfscope}%
\definecolor{textcolor}{rgb}{0.000000,0.000000,0.000000}%
\pgfsetstrokecolor{textcolor}%
\pgfsetfillcolor{textcolor}%
\pgftext[x=7.607462in,y=0.234413in,,top]{\color{textcolor}\sffamily\fontsize{10.000000}{12.000000}\selectfont 5}%
\end{pgfscope}%
\begin{pgfscope}%
\pgfsetbuttcap%
\pgfsetroundjoin%
\definecolor{currentfill}{rgb}{0.000000,0.000000,0.000000}%
\pgfsetfillcolor{currentfill}%
\pgfsetlinewidth{0.803000pt}%
\definecolor{currentstroke}{rgb}{0.000000,0.000000,0.000000}%
\pgfsetstrokecolor{currentstroke}%
\pgfsetdash{}{0pt}%
\pgfsys@defobject{currentmarker}{\pgfqpoint{0.000000in}{-0.048611in}}{\pgfqpoint{0.000000in}{0.000000in}}{%
\pgfpathmoveto{\pgfqpoint{0.000000in}{0.000000in}}%
\pgfpathlineto{\pgfqpoint{0.000000in}{-0.048611in}}%
\pgfusepath{stroke,fill}%
}%
\begin{pgfscope}%
\pgfsys@transformshift{9.491748in}{0.331635in}%
\pgfsys@useobject{currentmarker}{}%
\end{pgfscope}%
\end{pgfscope}%
\begin{pgfscope}%
\definecolor{textcolor}{rgb}{0.000000,0.000000,0.000000}%
\pgfsetstrokecolor{textcolor}%
\pgfsetfillcolor{textcolor}%
\pgftext[x=9.491748in,y=0.234413in,,top]{\color{textcolor}\sffamily\fontsize{10.000000}{12.000000}\selectfont 10}%
\end{pgfscope}%
\begin{pgfscope}%
\pgfsetbuttcap%
\pgfsetroundjoin%
\definecolor{currentfill}{rgb}{0.000000,0.000000,0.000000}%
\pgfsetfillcolor{currentfill}%
\pgfsetlinewidth{0.803000pt}%
\definecolor{currentstroke}{rgb}{0.000000,0.000000,0.000000}%
\pgfsetstrokecolor{currentstroke}%
\pgfsetdash{}{0pt}%
\pgfsys@defobject{currentmarker}{\pgfqpoint{-0.048611in}{0.000000in}}{\pgfqpoint{-0.000000in}{0.000000in}}{%
\pgfpathmoveto{\pgfqpoint{-0.000000in}{0.000000in}}%
\pgfpathlineto{\pgfqpoint{-0.048611in}{0.000000in}}%
\pgfusepath{stroke,fill}%
}%
\begin{pgfscope}%
\pgfsys@transformshift{0.526127in}{1.120852in}%
\pgfsys@useobject{currentmarker}{}%
\end{pgfscope}%
\end{pgfscope}%
\begin{pgfscope}%
\definecolor{textcolor}{rgb}{0.000000,0.000000,0.000000}%
\pgfsetstrokecolor{textcolor}%
\pgfsetfillcolor{textcolor}%
\pgftext[x=0.100000in, y=1.068091in, left, base]{\color{textcolor}\sffamily\fontsize{10.000000}{12.000000}\selectfont \ensuremath{-}7.5}%
\end{pgfscope}%
\begin{pgfscope}%
\pgfsetbuttcap%
\pgfsetroundjoin%
\definecolor{currentfill}{rgb}{0.000000,0.000000,0.000000}%
\pgfsetfillcolor{currentfill}%
\pgfsetlinewidth{0.803000pt}%
\definecolor{currentstroke}{rgb}{0.000000,0.000000,0.000000}%
\pgfsetstrokecolor{currentstroke}%
\pgfsetdash{}{0pt}%
\pgfsys@defobject{currentmarker}{\pgfqpoint{-0.048611in}{0.000000in}}{\pgfqpoint{-0.000000in}{0.000000in}}{%
\pgfpathmoveto{\pgfqpoint{-0.000000in}{0.000000in}}%
\pgfpathlineto{\pgfqpoint{-0.048611in}{0.000000in}}%
\pgfusepath{stroke,fill}%
}%
\begin{pgfscope}%
\pgfsys@transformshift{0.526127in}{2.007930in}%
\pgfsys@useobject{currentmarker}{}%
\end{pgfscope}%
\end{pgfscope}%
\begin{pgfscope}%
\definecolor{textcolor}{rgb}{0.000000,0.000000,0.000000}%
\pgfsetstrokecolor{textcolor}%
\pgfsetfillcolor{textcolor}%
\pgftext[x=0.100000in, y=1.955168in, left, base]{\color{textcolor}\sffamily\fontsize{10.000000}{12.000000}\selectfont \ensuremath{-}5.0}%
\end{pgfscope}%
\begin{pgfscope}%
\pgfsetbuttcap%
\pgfsetroundjoin%
\definecolor{currentfill}{rgb}{0.000000,0.000000,0.000000}%
\pgfsetfillcolor{currentfill}%
\pgfsetlinewidth{0.803000pt}%
\definecolor{currentstroke}{rgb}{0.000000,0.000000,0.000000}%
\pgfsetstrokecolor{currentstroke}%
\pgfsetdash{}{0pt}%
\pgfsys@defobject{currentmarker}{\pgfqpoint{-0.048611in}{0.000000in}}{\pgfqpoint{-0.000000in}{0.000000in}}{%
\pgfpathmoveto{\pgfqpoint{-0.000000in}{0.000000in}}%
\pgfpathlineto{\pgfqpoint{-0.048611in}{0.000000in}}%
\pgfusepath{stroke,fill}%
}%
\begin{pgfscope}%
\pgfsys@transformshift{0.526127in}{2.895007in}%
\pgfsys@useobject{currentmarker}{}%
\end{pgfscope}%
\end{pgfscope}%
\begin{pgfscope}%
\definecolor{textcolor}{rgb}{0.000000,0.000000,0.000000}%
\pgfsetstrokecolor{textcolor}%
\pgfsetfillcolor{textcolor}%
\pgftext[x=0.100000in, y=2.842246in, left, base]{\color{textcolor}\sffamily\fontsize{10.000000}{12.000000}\selectfont \ensuremath{-}2.5}%
\end{pgfscope}%
\begin{pgfscope}%
\pgfsetbuttcap%
\pgfsetroundjoin%
\definecolor{currentfill}{rgb}{0.000000,0.000000,0.000000}%
\pgfsetfillcolor{currentfill}%
\pgfsetlinewidth{0.803000pt}%
\definecolor{currentstroke}{rgb}{0.000000,0.000000,0.000000}%
\pgfsetstrokecolor{currentstroke}%
\pgfsetdash{}{0pt}%
\pgfsys@defobject{currentmarker}{\pgfqpoint{-0.048611in}{0.000000in}}{\pgfqpoint{-0.000000in}{0.000000in}}{%
\pgfpathmoveto{\pgfqpoint{-0.000000in}{0.000000in}}%
\pgfpathlineto{\pgfqpoint{-0.048611in}{0.000000in}}%
\pgfusepath{stroke,fill}%
}%
\begin{pgfscope}%
\pgfsys@transformshift{0.526127in}{3.782085in}%
\pgfsys@useobject{currentmarker}{}%
\end{pgfscope}%
\end{pgfscope}%
\begin{pgfscope}%
\definecolor{textcolor}{rgb}{0.000000,0.000000,0.000000}%
\pgfsetstrokecolor{textcolor}%
\pgfsetfillcolor{textcolor}%
\pgftext[x=0.208025in, y=3.729323in, left, base]{\color{textcolor}\sffamily\fontsize{10.000000}{12.000000}\selectfont 0.0}%
\end{pgfscope}%
\begin{pgfscope}%
\pgfsetbuttcap%
\pgfsetroundjoin%
\definecolor{currentfill}{rgb}{0.000000,0.000000,0.000000}%
\pgfsetfillcolor{currentfill}%
\pgfsetlinewidth{0.803000pt}%
\definecolor{currentstroke}{rgb}{0.000000,0.000000,0.000000}%
\pgfsetstrokecolor{currentstroke}%
\pgfsetdash{}{0pt}%
\pgfsys@defobject{currentmarker}{\pgfqpoint{-0.048611in}{0.000000in}}{\pgfqpoint{-0.000000in}{0.000000in}}{%
\pgfpathmoveto{\pgfqpoint{-0.000000in}{0.000000in}}%
\pgfpathlineto{\pgfqpoint{-0.048611in}{0.000000in}}%
\pgfusepath{stroke,fill}%
}%
\begin{pgfscope}%
\pgfsys@transformshift{0.526127in}{4.669162in}%
\pgfsys@useobject{currentmarker}{}%
\end{pgfscope}%
\end{pgfscope}%
\begin{pgfscope}%
\definecolor{textcolor}{rgb}{0.000000,0.000000,0.000000}%
\pgfsetstrokecolor{textcolor}%
\pgfsetfillcolor{textcolor}%
\pgftext[x=0.208025in, y=4.616401in, left, base]{\color{textcolor}\sffamily\fontsize{10.000000}{12.000000}\selectfont 2.5}%
\end{pgfscope}%
\begin{pgfscope}%
\pgfsetbuttcap%
\pgfsetroundjoin%
\definecolor{currentfill}{rgb}{0.000000,0.000000,0.000000}%
\pgfsetfillcolor{currentfill}%
\pgfsetlinewidth{0.803000pt}%
\definecolor{currentstroke}{rgb}{0.000000,0.000000,0.000000}%
\pgfsetstrokecolor{currentstroke}%
\pgfsetdash{}{0pt}%
\pgfsys@defobject{currentmarker}{\pgfqpoint{-0.048611in}{0.000000in}}{\pgfqpoint{-0.000000in}{0.000000in}}{%
\pgfpathmoveto{\pgfqpoint{-0.000000in}{0.000000in}}%
\pgfpathlineto{\pgfqpoint{-0.048611in}{0.000000in}}%
\pgfusepath{stroke,fill}%
}%
\begin{pgfscope}%
\pgfsys@transformshift{0.526127in}{5.556240in}%
\pgfsys@useobject{currentmarker}{}%
\end{pgfscope}%
\end{pgfscope}%
\begin{pgfscope}%
\definecolor{textcolor}{rgb}{0.000000,0.000000,0.000000}%
\pgfsetstrokecolor{textcolor}%
\pgfsetfillcolor{textcolor}%
\pgftext[x=0.208025in, y=5.503478in, left, base]{\color{textcolor}\sffamily\fontsize{10.000000}{12.000000}\selectfont 5.0}%
\end{pgfscope}%
\begin{pgfscope}%
\pgfsetbuttcap%
\pgfsetroundjoin%
\definecolor{currentfill}{rgb}{0.000000,0.000000,0.000000}%
\pgfsetfillcolor{currentfill}%
\pgfsetlinewidth{0.803000pt}%
\definecolor{currentstroke}{rgb}{0.000000,0.000000,0.000000}%
\pgfsetstrokecolor{currentstroke}%
\pgfsetdash{}{0pt}%
\pgfsys@defobject{currentmarker}{\pgfqpoint{-0.048611in}{0.000000in}}{\pgfqpoint{-0.000000in}{0.000000in}}{%
\pgfpathmoveto{\pgfqpoint{-0.000000in}{0.000000in}}%
\pgfpathlineto{\pgfqpoint{-0.048611in}{0.000000in}}%
\pgfusepath{stroke,fill}%
}%
\begin{pgfscope}%
\pgfsys@transformshift{0.526127in}{6.443317in}%
\pgfsys@useobject{currentmarker}{}%
\end{pgfscope}%
\end{pgfscope}%
\begin{pgfscope}%
\definecolor{textcolor}{rgb}{0.000000,0.000000,0.000000}%
\pgfsetstrokecolor{textcolor}%
\pgfsetfillcolor{textcolor}%
\pgftext[x=0.208025in, y=6.390556in, left, base]{\color{textcolor}\sffamily\fontsize{10.000000}{12.000000}\selectfont 7.5}%
\end{pgfscope}%
\begin{pgfscope}%
\pgfsetbuttcap%
\pgfsetroundjoin%
\definecolor{currentfill}{rgb}{0.000000,0.000000,0.000000}%
\pgfsetfillcolor{currentfill}%
\pgfsetlinewidth{0.803000pt}%
\definecolor{currentstroke}{rgb}{0.000000,0.000000,0.000000}%
\pgfsetstrokecolor{currentstroke}%
\pgfsetdash{}{0pt}%
\pgfsys@defobject{currentmarker}{\pgfqpoint{-0.048611in}{0.000000in}}{\pgfqpoint{-0.000000in}{0.000000in}}{%
\pgfpathmoveto{\pgfqpoint{-0.000000in}{0.000000in}}%
\pgfpathlineto{\pgfqpoint{-0.048611in}{0.000000in}}%
\pgfusepath{stroke,fill}%
}%
\begin{pgfscope}%
\pgfsys@transformshift{0.526127in}{7.330395in}%
\pgfsys@useobject{currentmarker}{}%
\end{pgfscope}%
\end{pgfscope}%
\begin{pgfscope}%
\definecolor{textcolor}{rgb}{0.000000,0.000000,0.000000}%
\pgfsetstrokecolor{textcolor}%
\pgfsetfillcolor{textcolor}%
\pgftext[x=0.119660in, y=7.277633in, left, base]{\color{textcolor}\sffamily\fontsize{10.000000}{12.000000}\selectfont 10.0}%
\end{pgfscope}%
\begin{pgfscope}%
\pgfpathrectangle{\pgfqpoint{0.526127in}{0.331635in}}{\pgfqpoint{9.300000in}{7.700000in}}%
\pgfusepath{clip}%
\pgfsetrectcap%
\pgfsetroundjoin%
\pgfsetlinewidth{1.505625pt}%
\definecolor{currentstroke}{rgb}{0.631373,0.788235,0.956863}%
\pgfsetstrokecolor{currentstroke}%
\pgfsetstrokeopacity{0.200000}%
\pgfsetdash{}{0pt}%
\pgfpathmoveto{\pgfqpoint{8.423233in}{5.791596in}}%
\pgfpathlineto{\pgfqpoint{6.100521in}{4.940903in}}%
\pgfusepath{stroke}%
\end{pgfscope}%
\begin{pgfscope}%
\pgfpathrectangle{\pgfqpoint{0.526127in}{0.331635in}}{\pgfqpoint{9.300000in}{7.700000in}}%
\pgfusepath{clip}%
\pgfsetrectcap%
\pgfsetroundjoin%
\pgfsetlinewidth{1.505625pt}%
\definecolor{currentstroke}{rgb}{0.631373,0.788235,0.956863}%
\pgfsetstrokecolor{currentstroke}%
\pgfsetstrokeopacity{0.200000}%
\pgfsetdash{}{0pt}%
\pgfpathmoveto{\pgfqpoint{4.747368in}{3.915110in}}%
\pgfpathlineto{\pgfqpoint{6.100521in}{4.940903in}}%
\pgfusepath{stroke}%
\end{pgfscope}%
\begin{pgfscope}%
\pgfpathrectangle{\pgfqpoint{0.526127in}{0.331635in}}{\pgfqpoint{9.300000in}{7.700000in}}%
\pgfusepath{clip}%
\pgfsetrectcap%
\pgfsetroundjoin%
\pgfsetlinewidth{1.505625pt}%
\definecolor{currentstroke}{rgb}{0.631373,0.788235,0.956863}%
\pgfsetstrokecolor{currentstroke}%
\pgfsetstrokeopacity{0.200000}%
\pgfsetdash{}{0pt}%
\pgfpathmoveto{\pgfqpoint{6.866294in}{4.174971in}}%
\pgfpathlineto{\pgfqpoint{6.100521in}{4.940903in}}%
\pgfusepath{stroke}%
\end{pgfscope}%
\begin{pgfscope}%
\pgfpathrectangle{\pgfqpoint{0.526127in}{0.331635in}}{\pgfqpoint{9.300000in}{7.700000in}}%
\pgfusepath{clip}%
\pgfsetrectcap%
\pgfsetroundjoin%
\pgfsetlinewidth{1.505625pt}%
\definecolor{currentstroke}{rgb}{0.631373,0.788235,0.956863}%
\pgfsetstrokecolor{currentstroke}%
\pgfsetstrokeopacity{0.200000}%
\pgfsetdash{}{0pt}%
\pgfpathmoveto{\pgfqpoint{8.257844in}{5.239559in}}%
\pgfpathlineto{\pgfqpoint{6.100521in}{4.940903in}}%
\pgfusepath{stroke}%
\end{pgfscope}%
\begin{pgfscope}%
\pgfpathrectangle{\pgfqpoint{0.526127in}{0.331635in}}{\pgfqpoint{9.300000in}{7.700000in}}%
\pgfusepath{clip}%
\pgfsetrectcap%
\pgfsetroundjoin%
\pgfsetlinewidth{1.505625pt}%
\definecolor{currentstroke}{rgb}{0.631373,0.788235,0.956863}%
\pgfsetstrokecolor{currentstroke}%
\pgfsetstrokeopacity{0.200000}%
\pgfsetdash{}{0pt}%
\pgfpathmoveto{\pgfqpoint{3.377597in}{2.747984in}}%
\pgfpathlineto{\pgfqpoint{6.100521in}{4.940903in}}%
\pgfusepath{stroke}%
\end{pgfscope}%
\begin{pgfscope}%
\pgfpathrectangle{\pgfqpoint{0.526127in}{0.331635in}}{\pgfqpoint{9.300000in}{7.700000in}}%
\pgfusepath{clip}%
\pgfsetrectcap%
\pgfsetroundjoin%
\pgfsetlinewidth{1.505625pt}%
\definecolor{currentstroke}{rgb}{0.631373,0.788235,0.956863}%
\pgfsetstrokecolor{currentstroke}%
\pgfsetstrokeopacity{0.200000}%
\pgfsetdash{}{0pt}%
\pgfpathmoveto{\pgfqpoint{5.374384in}{6.574501in}}%
\pgfpathlineto{\pgfqpoint{6.100521in}{4.940903in}}%
\pgfusepath{stroke}%
\end{pgfscope}%
\begin{pgfscope}%
\pgfpathrectangle{\pgfqpoint{0.526127in}{0.331635in}}{\pgfqpoint{9.300000in}{7.700000in}}%
\pgfusepath{clip}%
\pgfsetrectcap%
\pgfsetroundjoin%
\pgfsetlinewidth{1.505625pt}%
\definecolor{currentstroke}{rgb}{0.631373,0.788235,0.956863}%
\pgfsetstrokecolor{currentstroke}%
\pgfsetstrokeopacity{0.200000}%
\pgfsetdash{}{0pt}%
\pgfpathmoveto{\pgfqpoint{3.945884in}{4.863590in}}%
\pgfpathlineto{\pgfqpoint{6.100521in}{4.940903in}}%
\pgfusepath{stroke}%
\end{pgfscope}%
\begin{pgfscope}%
\pgfpathrectangle{\pgfqpoint{0.526127in}{0.331635in}}{\pgfqpoint{9.300000in}{7.700000in}}%
\pgfusepath{clip}%
\pgfsetrectcap%
\pgfsetroundjoin%
\pgfsetlinewidth{1.505625pt}%
\definecolor{currentstroke}{rgb}{0.631373,0.788235,0.956863}%
\pgfsetstrokecolor{currentstroke}%
\pgfsetstrokeopacity{0.200000}%
\pgfsetdash{}{0pt}%
\pgfpathmoveto{\pgfqpoint{6.422655in}{6.062096in}}%
\pgfpathlineto{\pgfqpoint{6.100521in}{4.940903in}}%
\pgfusepath{stroke}%
\end{pgfscope}%
\begin{pgfscope}%
\pgfpathrectangle{\pgfqpoint{0.526127in}{0.331635in}}{\pgfqpoint{9.300000in}{7.700000in}}%
\pgfusepath{clip}%
\pgfsetrectcap%
\pgfsetroundjoin%
\pgfsetlinewidth{1.505625pt}%
\definecolor{currentstroke}{rgb}{0.631373,0.788235,0.956863}%
\pgfsetstrokecolor{currentstroke}%
\pgfsetstrokeopacity{0.200000}%
\pgfsetdash{}{0pt}%
\pgfpathmoveto{\pgfqpoint{3.784322in}{5.194829in}}%
\pgfpathlineto{\pgfqpoint{6.100521in}{4.940903in}}%
\pgfusepath{stroke}%
\end{pgfscope}%
\begin{pgfscope}%
\pgfpathrectangle{\pgfqpoint{0.526127in}{0.331635in}}{\pgfqpoint{9.300000in}{7.700000in}}%
\pgfusepath{clip}%
\pgfsetrectcap%
\pgfsetroundjoin%
\pgfsetlinewidth{1.505625pt}%
\definecolor{currentstroke}{rgb}{0.631373,0.788235,0.956863}%
\pgfsetstrokecolor{currentstroke}%
\pgfsetstrokeopacity{0.200000}%
\pgfsetdash{}{0pt}%
\pgfpathmoveto{\pgfqpoint{4.807076in}{4.400777in}}%
\pgfpathlineto{\pgfqpoint{6.100521in}{4.940903in}}%
\pgfusepath{stroke}%
\end{pgfscope}%
\begin{pgfscope}%
\pgfpathrectangle{\pgfqpoint{0.526127in}{0.331635in}}{\pgfqpoint{9.300000in}{7.700000in}}%
\pgfusepath{clip}%
\pgfsetrectcap%
\pgfsetroundjoin%
\pgfsetlinewidth{1.505625pt}%
\definecolor{currentstroke}{rgb}{0.631373,0.788235,0.956863}%
\pgfsetstrokecolor{currentstroke}%
\pgfsetstrokeopacity{0.200000}%
\pgfsetdash{}{0pt}%
\pgfpathmoveto{\pgfqpoint{7.809864in}{6.914805in}}%
\pgfpathlineto{\pgfqpoint{6.100521in}{4.940903in}}%
\pgfusepath{stroke}%
\end{pgfscope}%
\begin{pgfscope}%
\pgfpathrectangle{\pgfqpoint{0.526127in}{0.331635in}}{\pgfqpoint{9.300000in}{7.700000in}}%
\pgfusepath{clip}%
\pgfsetrectcap%
\pgfsetroundjoin%
\pgfsetlinewidth{1.505625pt}%
\definecolor{currentstroke}{rgb}{0.631373,0.788235,0.956863}%
\pgfsetstrokecolor{currentstroke}%
\pgfsetstrokeopacity{0.200000}%
\pgfsetdash{}{0pt}%
\pgfpathmoveto{\pgfqpoint{7.744328in}{2.772445in}}%
\pgfpathlineto{\pgfqpoint{6.100521in}{4.940903in}}%
\pgfusepath{stroke}%
\end{pgfscope}%
\begin{pgfscope}%
\pgfpathrectangle{\pgfqpoint{0.526127in}{0.331635in}}{\pgfqpoint{9.300000in}{7.700000in}}%
\pgfusepath{clip}%
\pgfsetrectcap%
\pgfsetroundjoin%
\pgfsetlinewidth{1.505625pt}%
\definecolor{currentstroke}{rgb}{0.631373,0.788235,0.956863}%
\pgfsetstrokecolor{currentstroke}%
\pgfsetstrokeopacity{0.200000}%
\pgfsetdash{}{0pt}%
\pgfpathmoveto{\pgfqpoint{6.917985in}{5.083266in}}%
\pgfpathlineto{\pgfqpoint{6.100521in}{4.940903in}}%
\pgfusepath{stroke}%
\end{pgfscope}%
\begin{pgfscope}%
\pgfpathrectangle{\pgfqpoint{0.526127in}{0.331635in}}{\pgfqpoint{9.300000in}{7.700000in}}%
\pgfusepath{clip}%
\pgfsetrectcap%
\pgfsetroundjoin%
\pgfsetlinewidth{1.505625pt}%
\definecolor{currentstroke}{rgb}{0.631373,0.788235,0.956863}%
\pgfsetstrokecolor{currentstroke}%
\pgfsetstrokeopacity{0.200000}%
\pgfsetdash{}{0pt}%
\pgfpathmoveto{\pgfqpoint{5.740650in}{6.419871in}}%
\pgfpathlineto{\pgfqpoint{6.100521in}{4.940903in}}%
\pgfusepath{stroke}%
\end{pgfscope}%
\begin{pgfscope}%
\pgfpathrectangle{\pgfqpoint{0.526127in}{0.331635in}}{\pgfqpoint{9.300000in}{7.700000in}}%
\pgfusepath{clip}%
\pgfsetrectcap%
\pgfsetroundjoin%
\pgfsetlinewidth{1.505625pt}%
\definecolor{currentstroke}{rgb}{0.631373,0.788235,0.956863}%
\pgfsetstrokecolor{currentstroke}%
\pgfsetstrokeopacity{0.200000}%
\pgfsetdash{}{0pt}%
\pgfpathmoveto{\pgfqpoint{7.681588in}{6.092216in}}%
\pgfpathlineto{\pgfqpoint{6.100521in}{4.940903in}}%
\pgfusepath{stroke}%
\end{pgfscope}%
\begin{pgfscope}%
\pgfpathrectangle{\pgfqpoint{0.526127in}{0.331635in}}{\pgfqpoint{9.300000in}{7.700000in}}%
\pgfusepath{clip}%
\pgfsetrectcap%
\pgfsetroundjoin%
\pgfsetlinewidth{1.505625pt}%
\definecolor{currentstroke}{rgb}{0.631373,0.788235,0.956863}%
\pgfsetstrokecolor{currentstroke}%
\pgfsetstrokeopacity{0.200000}%
\pgfsetdash{}{0pt}%
\pgfpathmoveto{\pgfqpoint{6.865837in}{4.629084in}}%
\pgfpathlineto{\pgfqpoint{6.100521in}{4.940903in}}%
\pgfusepath{stroke}%
\end{pgfscope}%
\begin{pgfscope}%
\pgfpathrectangle{\pgfqpoint{0.526127in}{0.331635in}}{\pgfqpoint{9.300000in}{7.700000in}}%
\pgfusepath{clip}%
\pgfsetrectcap%
\pgfsetroundjoin%
\pgfsetlinewidth{1.505625pt}%
\definecolor{currentstroke}{rgb}{0.631373,0.788235,0.956863}%
\pgfsetstrokecolor{currentstroke}%
\pgfsetstrokeopacity{0.200000}%
\pgfsetdash{}{0pt}%
\pgfpathmoveto{\pgfqpoint{7.703805in}{3.296605in}}%
\pgfpathlineto{\pgfqpoint{6.100521in}{4.940903in}}%
\pgfusepath{stroke}%
\end{pgfscope}%
\begin{pgfscope}%
\pgfpathrectangle{\pgfqpoint{0.526127in}{0.331635in}}{\pgfqpoint{9.300000in}{7.700000in}}%
\pgfusepath{clip}%
\pgfsetrectcap%
\pgfsetroundjoin%
\pgfsetlinewidth{1.505625pt}%
\definecolor{currentstroke}{rgb}{0.631373,0.788235,0.956863}%
\pgfsetstrokecolor{currentstroke}%
\pgfsetstrokeopacity{0.200000}%
\pgfsetdash{}{0pt}%
\pgfpathmoveto{\pgfqpoint{2.458995in}{1.029819in}}%
\pgfpathlineto{\pgfqpoint{6.100521in}{4.940903in}}%
\pgfusepath{stroke}%
\end{pgfscope}%
\begin{pgfscope}%
\pgfpathrectangle{\pgfqpoint{0.526127in}{0.331635in}}{\pgfqpoint{9.300000in}{7.700000in}}%
\pgfusepath{clip}%
\pgfsetrectcap%
\pgfsetroundjoin%
\pgfsetlinewidth{1.505625pt}%
\definecolor{currentstroke}{rgb}{0.631373,0.788235,0.956863}%
\pgfsetstrokecolor{currentstroke}%
\pgfsetstrokeopacity{0.200000}%
\pgfsetdash{}{0pt}%
\pgfpathmoveto{\pgfqpoint{4.621174in}{4.057804in}}%
\pgfpathlineto{\pgfqpoint{6.100521in}{4.940903in}}%
\pgfusepath{stroke}%
\end{pgfscope}%
\begin{pgfscope}%
\pgfpathrectangle{\pgfqpoint{0.526127in}{0.331635in}}{\pgfqpoint{9.300000in}{7.700000in}}%
\pgfusepath{clip}%
\pgfsetrectcap%
\pgfsetroundjoin%
\pgfsetlinewidth{1.505625pt}%
\definecolor{currentstroke}{rgb}{0.631373,0.788235,0.956863}%
\pgfsetstrokecolor{currentstroke}%
\pgfsetstrokeopacity{0.200000}%
\pgfsetdash{}{0pt}%
\pgfpathmoveto{\pgfqpoint{5.970043in}{6.897752in}}%
\pgfpathlineto{\pgfqpoint{6.100521in}{4.940903in}}%
\pgfusepath{stroke}%
\end{pgfscope}%
\begin{pgfscope}%
\pgfpathrectangle{\pgfqpoint{0.526127in}{0.331635in}}{\pgfqpoint{9.300000in}{7.700000in}}%
\pgfusepath{clip}%
\pgfsetrectcap%
\pgfsetroundjoin%
\pgfsetlinewidth{1.505625pt}%
\definecolor{currentstroke}{rgb}{0.631373,0.788235,0.956863}%
\pgfsetstrokecolor{currentstroke}%
\pgfsetstrokeopacity{0.200000}%
\pgfsetdash{}{0pt}%
\pgfpathmoveto{\pgfqpoint{5.368377in}{3.410644in}}%
\pgfpathlineto{\pgfqpoint{6.100521in}{4.940903in}}%
\pgfusepath{stroke}%
\end{pgfscope}%
\begin{pgfscope}%
\pgfpathrectangle{\pgfqpoint{0.526127in}{0.331635in}}{\pgfqpoint{9.300000in}{7.700000in}}%
\pgfusepath{clip}%
\pgfsetrectcap%
\pgfsetroundjoin%
\pgfsetlinewidth{1.505625pt}%
\definecolor{currentstroke}{rgb}{0.631373,0.788235,0.956863}%
\pgfsetstrokecolor{currentstroke}%
\pgfsetstrokeopacity{0.200000}%
\pgfsetdash{}{0pt}%
\pgfpathmoveto{\pgfqpoint{6.364240in}{6.604101in}}%
\pgfpathlineto{\pgfqpoint{6.100521in}{4.940903in}}%
\pgfusepath{stroke}%
\end{pgfscope}%
\begin{pgfscope}%
\pgfpathrectangle{\pgfqpoint{0.526127in}{0.331635in}}{\pgfqpoint{9.300000in}{7.700000in}}%
\pgfusepath{clip}%
\pgfsetrectcap%
\pgfsetroundjoin%
\pgfsetlinewidth{1.505625pt}%
\definecolor{currentstroke}{rgb}{0.631373,0.788235,0.956863}%
\pgfsetstrokecolor{currentstroke}%
\pgfsetstrokeopacity{0.200000}%
\pgfsetdash{}{0pt}%
\pgfpathmoveto{\pgfqpoint{6.258495in}{6.814935in}}%
\pgfpathlineto{\pgfqpoint{6.100521in}{4.940903in}}%
\pgfusepath{stroke}%
\end{pgfscope}%
\begin{pgfscope}%
\pgfpathrectangle{\pgfqpoint{0.526127in}{0.331635in}}{\pgfqpoint{9.300000in}{7.700000in}}%
\pgfusepath{clip}%
\pgfsetrectcap%
\pgfsetroundjoin%
\pgfsetlinewidth{1.505625pt}%
\definecolor{currentstroke}{rgb}{0.631373,0.788235,0.956863}%
\pgfsetstrokecolor{currentstroke}%
\pgfsetstrokeopacity{0.200000}%
\pgfsetdash{}{0pt}%
\pgfpathmoveto{\pgfqpoint{7.197145in}{4.694607in}}%
\pgfpathlineto{\pgfqpoint{6.100521in}{4.940903in}}%
\pgfusepath{stroke}%
\end{pgfscope}%
\begin{pgfscope}%
\pgfpathrectangle{\pgfqpoint{0.526127in}{0.331635in}}{\pgfqpoint{9.300000in}{7.700000in}}%
\pgfusepath{clip}%
\pgfsetrectcap%
\pgfsetroundjoin%
\pgfsetlinewidth{1.505625pt}%
\definecolor{currentstroke}{rgb}{0.631373,0.788235,0.956863}%
\pgfsetstrokecolor{currentstroke}%
\pgfsetstrokeopacity{0.200000}%
\pgfsetdash{}{0pt}%
\pgfpathmoveto{\pgfqpoint{7.051687in}{6.247476in}}%
\pgfpathlineto{\pgfqpoint{6.100521in}{4.940903in}}%
\pgfusepath{stroke}%
\end{pgfscope}%
\begin{pgfscope}%
\pgfpathrectangle{\pgfqpoint{0.526127in}{0.331635in}}{\pgfqpoint{9.300000in}{7.700000in}}%
\pgfusepath{clip}%
\pgfsetrectcap%
\pgfsetroundjoin%
\pgfsetlinewidth{1.505625pt}%
\definecolor{currentstroke}{rgb}{0.631373,0.788235,0.956863}%
\pgfsetstrokecolor{currentstroke}%
\pgfsetstrokeopacity{0.200000}%
\pgfsetdash{}{0pt}%
\pgfpathmoveto{\pgfqpoint{6.775438in}{7.238564in}}%
\pgfpathlineto{\pgfqpoint{6.100521in}{4.940903in}}%
\pgfusepath{stroke}%
\end{pgfscope}%
\begin{pgfscope}%
\pgfpathrectangle{\pgfqpoint{0.526127in}{0.331635in}}{\pgfqpoint{9.300000in}{7.700000in}}%
\pgfusepath{clip}%
\pgfsetrectcap%
\pgfsetroundjoin%
\pgfsetlinewidth{1.505625pt}%
\definecolor{currentstroke}{rgb}{0.631373,0.788235,0.956863}%
\pgfsetstrokecolor{currentstroke}%
\pgfsetstrokeopacity{0.200000}%
\pgfsetdash{}{0pt}%
\pgfpathmoveto{\pgfqpoint{4.179928in}{4.106096in}}%
\pgfpathlineto{\pgfqpoint{6.100521in}{4.940903in}}%
\pgfusepath{stroke}%
\end{pgfscope}%
\begin{pgfscope}%
\pgfpathrectangle{\pgfqpoint{0.526127in}{0.331635in}}{\pgfqpoint{9.300000in}{7.700000in}}%
\pgfusepath{clip}%
\pgfsetrectcap%
\pgfsetroundjoin%
\pgfsetlinewidth{1.505625pt}%
\definecolor{currentstroke}{rgb}{0.631373,0.788235,0.956863}%
\pgfsetstrokecolor{currentstroke}%
\pgfsetstrokeopacity{0.200000}%
\pgfsetdash{}{0pt}%
\pgfpathmoveto{\pgfqpoint{8.098361in}{3.070197in}}%
\pgfpathlineto{\pgfqpoint{6.100521in}{4.940903in}}%
\pgfusepath{stroke}%
\end{pgfscope}%
\begin{pgfscope}%
\pgfpathrectangle{\pgfqpoint{0.526127in}{0.331635in}}{\pgfqpoint{9.300000in}{7.700000in}}%
\pgfusepath{clip}%
\pgfsetrectcap%
\pgfsetroundjoin%
\pgfsetlinewidth{1.505625pt}%
\definecolor{currentstroke}{rgb}{1.000000,0.705882,0.509804}%
\pgfsetstrokecolor{currentstroke}%
\pgfsetstrokeopacity{0.200000}%
\pgfsetdash{}{0pt}%
\pgfpathmoveto{\pgfqpoint{6.611324in}{4.129584in}}%
\pgfpathlineto{\pgfqpoint{6.411605in}{4.442724in}}%
\pgfusepath{stroke}%
\end{pgfscope}%
\begin{pgfscope}%
\pgfpathrectangle{\pgfqpoint{0.526127in}{0.331635in}}{\pgfqpoint{9.300000in}{7.700000in}}%
\pgfusepath{clip}%
\pgfsetrectcap%
\pgfsetroundjoin%
\pgfsetlinewidth{1.505625pt}%
\definecolor{currentstroke}{rgb}{1.000000,0.705882,0.509804}%
\pgfsetstrokecolor{currentstroke}%
\pgfsetstrokeopacity{0.200000}%
\pgfsetdash{}{0pt}%
\pgfpathmoveto{\pgfqpoint{4.265030in}{7.009476in}}%
\pgfpathlineto{\pgfqpoint{6.411605in}{4.442724in}}%
\pgfusepath{stroke}%
\end{pgfscope}%
\begin{pgfscope}%
\pgfpathrectangle{\pgfqpoint{0.526127in}{0.331635in}}{\pgfqpoint{9.300000in}{7.700000in}}%
\pgfusepath{clip}%
\pgfsetrectcap%
\pgfsetroundjoin%
\pgfsetlinewidth{1.505625pt}%
\definecolor{currentstroke}{rgb}{1.000000,0.705882,0.509804}%
\pgfsetstrokecolor{currentstroke}%
\pgfsetstrokeopacity{0.200000}%
\pgfsetdash{}{0pt}%
\pgfpathmoveto{\pgfqpoint{5.530805in}{6.568416in}}%
\pgfpathlineto{\pgfqpoint{6.411605in}{4.442724in}}%
\pgfusepath{stroke}%
\end{pgfscope}%
\begin{pgfscope}%
\pgfpathrectangle{\pgfqpoint{0.526127in}{0.331635in}}{\pgfqpoint{9.300000in}{7.700000in}}%
\pgfusepath{clip}%
\pgfsetrectcap%
\pgfsetroundjoin%
\pgfsetlinewidth{1.505625pt}%
\definecolor{currentstroke}{rgb}{1.000000,0.705882,0.509804}%
\pgfsetstrokecolor{currentstroke}%
\pgfsetstrokeopacity{0.200000}%
\pgfsetdash{}{0pt}%
\pgfpathmoveto{\pgfqpoint{6.584340in}{5.732668in}}%
\pgfpathlineto{\pgfqpoint{6.411605in}{4.442724in}}%
\pgfusepath{stroke}%
\end{pgfscope}%
\begin{pgfscope}%
\pgfpathrectangle{\pgfqpoint{0.526127in}{0.331635in}}{\pgfqpoint{9.300000in}{7.700000in}}%
\pgfusepath{clip}%
\pgfsetrectcap%
\pgfsetroundjoin%
\pgfsetlinewidth{1.505625pt}%
\definecolor{currentstroke}{rgb}{1.000000,0.705882,0.509804}%
\pgfsetstrokecolor{currentstroke}%
\pgfsetstrokeopacity{0.200000}%
\pgfsetdash{}{0pt}%
\pgfpathmoveto{\pgfqpoint{6.306902in}{2.521520in}}%
\pgfpathlineto{\pgfqpoint{6.411605in}{4.442724in}}%
\pgfusepath{stroke}%
\end{pgfscope}%
\begin{pgfscope}%
\pgfpathrectangle{\pgfqpoint{0.526127in}{0.331635in}}{\pgfqpoint{9.300000in}{7.700000in}}%
\pgfusepath{clip}%
\pgfsetrectcap%
\pgfsetroundjoin%
\pgfsetlinewidth{1.505625pt}%
\definecolor{currentstroke}{rgb}{1.000000,0.705882,0.509804}%
\pgfsetstrokecolor{currentstroke}%
\pgfsetstrokeopacity{0.200000}%
\pgfsetdash{}{0pt}%
\pgfpathmoveto{\pgfqpoint{7.529402in}{3.720872in}}%
\pgfpathlineto{\pgfqpoint{6.411605in}{4.442724in}}%
\pgfusepath{stroke}%
\end{pgfscope}%
\begin{pgfscope}%
\pgfpathrectangle{\pgfqpoint{0.526127in}{0.331635in}}{\pgfqpoint{9.300000in}{7.700000in}}%
\pgfusepath{clip}%
\pgfsetrectcap%
\pgfsetroundjoin%
\pgfsetlinewidth{1.505625pt}%
\definecolor{currentstroke}{rgb}{1.000000,0.705882,0.509804}%
\pgfsetstrokecolor{currentstroke}%
\pgfsetstrokeopacity{0.200000}%
\pgfsetdash{}{0pt}%
\pgfpathmoveto{\pgfqpoint{7.857488in}{3.672391in}}%
\pgfpathlineto{\pgfqpoint{6.411605in}{4.442724in}}%
\pgfusepath{stroke}%
\end{pgfscope}%
\begin{pgfscope}%
\pgfpathrectangle{\pgfqpoint{0.526127in}{0.331635in}}{\pgfqpoint{9.300000in}{7.700000in}}%
\pgfusepath{clip}%
\pgfsetrectcap%
\pgfsetroundjoin%
\pgfsetlinewidth{1.505625pt}%
\definecolor{currentstroke}{rgb}{1.000000,0.705882,0.509804}%
\pgfsetstrokecolor{currentstroke}%
\pgfsetstrokeopacity{0.200000}%
\pgfsetdash{}{0pt}%
\pgfpathmoveto{\pgfqpoint{6.572528in}{4.224991in}}%
\pgfpathlineto{\pgfqpoint{6.411605in}{4.442724in}}%
\pgfusepath{stroke}%
\end{pgfscope}%
\begin{pgfscope}%
\pgfpathrectangle{\pgfqpoint{0.526127in}{0.331635in}}{\pgfqpoint{9.300000in}{7.700000in}}%
\pgfusepath{clip}%
\pgfsetrectcap%
\pgfsetroundjoin%
\pgfsetlinewidth{1.505625pt}%
\definecolor{currentstroke}{rgb}{1.000000,0.705882,0.509804}%
\pgfsetstrokecolor{currentstroke}%
\pgfsetstrokeopacity{0.200000}%
\pgfsetdash{}{0pt}%
\pgfpathmoveto{\pgfqpoint{5.154620in}{3.765894in}}%
\pgfpathlineto{\pgfqpoint{6.411605in}{4.442724in}}%
\pgfusepath{stroke}%
\end{pgfscope}%
\begin{pgfscope}%
\pgfpathrectangle{\pgfqpoint{0.526127in}{0.331635in}}{\pgfqpoint{9.300000in}{7.700000in}}%
\pgfusepath{clip}%
\pgfsetrectcap%
\pgfsetroundjoin%
\pgfsetlinewidth{1.505625pt}%
\definecolor{currentstroke}{rgb}{1.000000,0.705882,0.509804}%
\pgfsetstrokecolor{currentstroke}%
\pgfsetstrokeopacity{0.200000}%
\pgfsetdash{}{0pt}%
\pgfpathmoveto{\pgfqpoint{5.190149in}{3.839314in}}%
\pgfpathlineto{\pgfqpoint{6.411605in}{4.442724in}}%
\pgfusepath{stroke}%
\end{pgfscope}%
\begin{pgfscope}%
\pgfpathrectangle{\pgfqpoint{0.526127in}{0.331635in}}{\pgfqpoint{9.300000in}{7.700000in}}%
\pgfusepath{clip}%
\pgfsetrectcap%
\pgfsetroundjoin%
\pgfsetlinewidth{1.505625pt}%
\definecolor{currentstroke}{rgb}{1.000000,0.705882,0.509804}%
\pgfsetstrokecolor{currentstroke}%
\pgfsetstrokeopacity{0.200000}%
\pgfsetdash{}{0pt}%
\pgfpathmoveto{\pgfqpoint{6.190070in}{4.309824in}}%
\pgfpathlineto{\pgfqpoint{6.411605in}{4.442724in}}%
\pgfusepath{stroke}%
\end{pgfscope}%
\begin{pgfscope}%
\pgfpathrectangle{\pgfqpoint{0.526127in}{0.331635in}}{\pgfqpoint{9.300000in}{7.700000in}}%
\pgfusepath{clip}%
\pgfsetrectcap%
\pgfsetroundjoin%
\pgfsetlinewidth{1.505625pt}%
\definecolor{currentstroke}{rgb}{1.000000,0.705882,0.509804}%
\pgfsetstrokecolor{currentstroke}%
\pgfsetstrokeopacity{0.200000}%
\pgfsetdash{}{0pt}%
\pgfpathmoveto{\pgfqpoint{6.008150in}{4.992055in}}%
\pgfpathlineto{\pgfqpoint{6.411605in}{4.442724in}}%
\pgfusepath{stroke}%
\end{pgfscope}%
\begin{pgfscope}%
\pgfpathrectangle{\pgfqpoint{0.526127in}{0.331635in}}{\pgfqpoint{9.300000in}{7.700000in}}%
\pgfusepath{clip}%
\pgfsetrectcap%
\pgfsetroundjoin%
\pgfsetlinewidth{1.505625pt}%
\definecolor{currentstroke}{rgb}{1.000000,0.705882,0.509804}%
\pgfsetstrokecolor{currentstroke}%
\pgfsetstrokeopacity{0.200000}%
\pgfsetdash{}{0pt}%
\pgfpathmoveto{\pgfqpoint{3.566645in}{0.681635in}}%
\pgfpathlineto{\pgfqpoint{6.411605in}{4.442724in}}%
\pgfusepath{stroke}%
\end{pgfscope}%
\begin{pgfscope}%
\pgfpathrectangle{\pgfqpoint{0.526127in}{0.331635in}}{\pgfqpoint{9.300000in}{7.700000in}}%
\pgfusepath{clip}%
\pgfsetrectcap%
\pgfsetroundjoin%
\pgfsetlinewidth{1.505625pt}%
\definecolor{currentstroke}{rgb}{1.000000,0.705882,0.509804}%
\pgfsetstrokecolor{currentstroke}%
\pgfsetstrokeopacity{0.200000}%
\pgfsetdash{}{0pt}%
\pgfpathmoveto{\pgfqpoint{7.924332in}{3.471043in}}%
\pgfpathlineto{\pgfqpoint{6.411605in}{4.442724in}}%
\pgfusepath{stroke}%
\end{pgfscope}%
\begin{pgfscope}%
\pgfpathrectangle{\pgfqpoint{0.526127in}{0.331635in}}{\pgfqpoint{9.300000in}{7.700000in}}%
\pgfusepath{clip}%
\pgfsetrectcap%
\pgfsetroundjoin%
\pgfsetlinewidth{1.505625pt}%
\definecolor{currentstroke}{rgb}{1.000000,0.705882,0.509804}%
\pgfsetstrokecolor{currentstroke}%
\pgfsetstrokeopacity{0.200000}%
\pgfsetdash{}{0pt}%
\pgfpathmoveto{\pgfqpoint{6.132760in}{3.874810in}}%
\pgfpathlineto{\pgfqpoint{6.411605in}{4.442724in}}%
\pgfusepath{stroke}%
\end{pgfscope}%
\begin{pgfscope}%
\pgfpathrectangle{\pgfqpoint{0.526127in}{0.331635in}}{\pgfqpoint{9.300000in}{7.700000in}}%
\pgfusepath{clip}%
\pgfsetrectcap%
\pgfsetroundjoin%
\pgfsetlinewidth{1.505625pt}%
\definecolor{currentstroke}{rgb}{1.000000,0.705882,0.509804}%
\pgfsetstrokecolor{currentstroke}%
\pgfsetstrokeopacity{0.200000}%
\pgfsetdash{}{0pt}%
\pgfpathmoveto{\pgfqpoint{6.193019in}{4.773366in}}%
\pgfpathlineto{\pgfqpoint{6.411605in}{4.442724in}}%
\pgfusepath{stroke}%
\end{pgfscope}%
\begin{pgfscope}%
\pgfpathrectangle{\pgfqpoint{0.526127in}{0.331635in}}{\pgfqpoint{9.300000in}{7.700000in}}%
\pgfusepath{clip}%
\pgfsetrectcap%
\pgfsetroundjoin%
\pgfsetlinewidth{1.505625pt}%
\definecolor{currentstroke}{rgb}{1.000000,0.705882,0.509804}%
\pgfsetstrokecolor{currentstroke}%
\pgfsetstrokeopacity{0.200000}%
\pgfsetdash{}{0pt}%
\pgfpathmoveto{\pgfqpoint{6.027104in}{5.627514in}}%
\pgfpathlineto{\pgfqpoint{6.411605in}{4.442724in}}%
\pgfusepath{stroke}%
\end{pgfscope}%
\begin{pgfscope}%
\pgfpathrectangle{\pgfqpoint{0.526127in}{0.331635in}}{\pgfqpoint{9.300000in}{7.700000in}}%
\pgfusepath{clip}%
\pgfsetrectcap%
\pgfsetroundjoin%
\pgfsetlinewidth{1.505625pt}%
\definecolor{currentstroke}{rgb}{1.000000,0.705882,0.509804}%
\pgfsetstrokecolor{currentstroke}%
\pgfsetstrokeopacity{0.200000}%
\pgfsetdash{}{0pt}%
\pgfpathmoveto{\pgfqpoint{6.988150in}{5.568616in}}%
\pgfpathlineto{\pgfqpoint{6.411605in}{4.442724in}}%
\pgfusepath{stroke}%
\end{pgfscope}%
\begin{pgfscope}%
\pgfpathrectangle{\pgfqpoint{0.526127in}{0.331635in}}{\pgfqpoint{9.300000in}{7.700000in}}%
\pgfusepath{clip}%
\pgfsetrectcap%
\pgfsetroundjoin%
\pgfsetlinewidth{1.505625pt}%
\definecolor{currentstroke}{rgb}{1.000000,0.705882,0.509804}%
\pgfsetstrokecolor{currentstroke}%
\pgfsetstrokeopacity{0.200000}%
\pgfsetdash{}{0pt}%
\pgfpathmoveto{\pgfqpoint{6.794558in}{3.554767in}}%
\pgfpathlineto{\pgfqpoint{6.411605in}{4.442724in}}%
\pgfusepath{stroke}%
\end{pgfscope}%
\begin{pgfscope}%
\pgfpathrectangle{\pgfqpoint{0.526127in}{0.331635in}}{\pgfqpoint{9.300000in}{7.700000in}}%
\pgfusepath{clip}%
\pgfsetrectcap%
\pgfsetroundjoin%
\pgfsetlinewidth{1.505625pt}%
\definecolor{currentstroke}{rgb}{1.000000,0.705882,0.509804}%
\pgfsetstrokecolor{currentstroke}%
\pgfsetstrokeopacity{0.200000}%
\pgfsetdash{}{0pt}%
\pgfpathmoveto{\pgfqpoint{7.563920in}{3.455813in}}%
\pgfpathlineto{\pgfqpoint{6.411605in}{4.442724in}}%
\pgfusepath{stroke}%
\end{pgfscope}%
\begin{pgfscope}%
\pgfpathrectangle{\pgfqpoint{0.526127in}{0.331635in}}{\pgfqpoint{9.300000in}{7.700000in}}%
\pgfusepath{clip}%
\pgfsetrectcap%
\pgfsetroundjoin%
\pgfsetlinewidth{1.505625pt}%
\definecolor{currentstroke}{rgb}{1.000000,0.705882,0.509804}%
\pgfsetstrokecolor{currentstroke}%
\pgfsetstrokeopacity{0.200000}%
\pgfsetdash{}{0pt}%
\pgfpathmoveto{\pgfqpoint{5.276797in}{5.750429in}}%
\pgfpathlineto{\pgfqpoint{6.411605in}{4.442724in}}%
\pgfusepath{stroke}%
\end{pgfscope}%
\begin{pgfscope}%
\pgfpathrectangle{\pgfqpoint{0.526127in}{0.331635in}}{\pgfqpoint{9.300000in}{7.700000in}}%
\pgfusepath{clip}%
\pgfsetrectcap%
\pgfsetroundjoin%
\pgfsetlinewidth{1.505625pt}%
\definecolor{currentstroke}{rgb}{1.000000,0.705882,0.509804}%
\pgfsetstrokecolor{currentstroke}%
\pgfsetstrokeopacity{0.200000}%
\pgfsetdash{}{0pt}%
\pgfpathmoveto{\pgfqpoint{7.155517in}{3.710876in}}%
\pgfpathlineto{\pgfqpoint{6.411605in}{4.442724in}}%
\pgfusepath{stroke}%
\end{pgfscope}%
\begin{pgfscope}%
\pgfpathrectangle{\pgfqpoint{0.526127in}{0.331635in}}{\pgfqpoint{9.300000in}{7.700000in}}%
\pgfusepath{clip}%
\pgfsetrectcap%
\pgfsetroundjoin%
\pgfsetlinewidth{1.505625pt}%
\definecolor{currentstroke}{rgb}{1.000000,0.705882,0.509804}%
\pgfsetstrokecolor{currentstroke}%
\pgfsetstrokeopacity{0.200000}%
\pgfsetdash{}{0pt}%
\pgfpathmoveto{\pgfqpoint{6.008087in}{6.401709in}}%
\pgfpathlineto{\pgfqpoint{6.411605in}{4.442724in}}%
\pgfusepath{stroke}%
\end{pgfscope}%
\begin{pgfscope}%
\pgfpathrectangle{\pgfqpoint{0.526127in}{0.331635in}}{\pgfqpoint{9.300000in}{7.700000in}}%
\pgfusepath{clip}%
\pgfsetrectcap%
\pgfsetroundjoin%
\pgfsetlinewidth{1.505625pt}%
\definecolor{currentstroke}{rgb}{1.000000,0.705882,0.509804}%
\pgfsetstrokecolor{currentstroke}%
\pgfsetstrokeopacity{0.200000}%
\pgfsetdash{}{0pt}%
\pgfpathmoveto{\pgfqpoint{7.439477in}{5.289143in}}%
\pgfpathlineto{\pgfqpoint{6.411605in}{4.442724in}}%
\pgfusepath{stroke}%
\end{pgfscope}%
\begin{pgfscope}%
\pgfpathrectangle{\pgfqpoint{0.526127in}{0.331635in}}{\pgfqpoint{9.300000in}{7.700000in}}%
\pgfusepath{clip}%
\pgfsetrectcap%
\pgfsetroundjoin%
\pgfsetlinewidth{1.505625pt}%
\definecolor{currentstroke}{rgb}{1.000000,0.705882,0.509804}%
\pgfsetstrokecolor{currentstroke}%
\pgfsetstrokeopacity{0.200000}%
\pgfsetdash{}{0pt}%
\pgfpathmoveto{\pgfqpoint{7.682374in}{4.875449in}}%
\pgfpathlineto{\pgfqpoint{6.411605in}{4.442724in}}%
\pgfusepath{stroke}%
\end{pgfscope}%
\begin{pgfscope}%
\pgfpathrectangle{\pgfqpoint{0.526127in}{0.331635in}}{\pgfqpoint{9.300000in}{7.700000in}}%
\pgfusepath{clip}%
\pgfsetrectcap%
\pgfsetroundjoin%
\pgfsetlinewidth{1.505625pt}%
\definecolor{currentstroke}{rgb}{1.000000,0.705882,0.509804}%
\pgfsetstrokecolor{currentstroke}%
\pgfsetstrokeopacity{0.200000}%
\pgfsetdash{}{0pt}%
\pgfpathmoveto{\pgfqpoint{5.929161in}{6.127529in}}%
\pgfpathlineto{\pgfqpoint{6.411605in}{4.442724in}}%
\pgfusepath{stroke}%
\end{pgfscope}%
\begin{pgfscope}%
\pgfpathrectangle{\pgfqpoint{0.526127in}{0.331635in}}{\pgfqpoint{9.300000in}{7.700000in}}%
\pgfusepath{clip}%
\pgfsetrectcap%
\pgfsetroundjoin%
\pgfsetlinewidth{1.505625pt}%
\definecolor{currentstroke}{rgb}{1.000000,0.705882,0.509804}%
\pgfsetstrokecolor{currentstroke}%
\pgfsetstrokeopacity{0.200000}%
\pgfsetdash{}{0pt}%
\pgfpathmoveto{\pgfqpoint{8.724020in}{3.300311in}}%
\pgfpathlineto{\pgfqpoint{6.411605in}{4.442724in}}%
\pgfusepath{stroke}%
\end{pgfscope}%
\begin{pgfscope}%
\pgfpathrectangle{\pgfqpoint{0.526127in}{0.331635in}}{\pgfqpoint{9.300000in}{7.700000in}}%
\pgfusepath{clip}%
\pgfsetrectcap%
\pgfsetroundjoin%
\pgfsetlinewidth{1.505625pt}%
\definecolor{currentstroke}{rgb}{1.000000,0.705882,0.509804}%
\pgfsetstrokecolor{currentstroke}%
\pgfsetstrokeopacity{0.200000}%
\pgfsetdash{}{0pt}%
\pgfpathmoveto{\pgfqpoint{6.318203in}{3.446246in}}%
\pgfpathlineto{\pgfqpoint{6.411605in}{4.442724in}}%
\pgfusepath{stroke}%
\end{pgfscope}%
\begin{pgfscope}%
\pgfpathrectangle{\pgfqpoint{0.526127in}{0.331635in}}{\pgfqpoint{9.300000in}{7.700000in}}%
\pgfusepath{clip}%
\pgfsetrectcap%
\pgfsetroundjoin%
\pgfsetlinewidth{1.505625pt}%
\definecolor{currentstroke}{rgb}{0.552941,0.898039,0.631373}%
\pgfsetstrokecolor{currentstroke}%
\pgfsetstrokeopacity{0.200000}%
\pgfsetdash{}{0pt}%
\pgfpathmoveto{\pgfqpoint{7.719822in}{2.287277in}}%
\pgfpathlineto{\pgfqpoint{4.400818in}{2.895337in}}%
\pgfusepath{stroke}%
\end{pgfscope}%
\begin{pgfscope}%
\pgfpathrectangle{\pgfqpoint{0.526127in}{0.331635in}}{\pgfqpoint{9.300000in}{7.700000in}}%
\pgfusepath{clip}%
\pgfsetrectcap%
\pgfsetroundjoin%
\pgfsetlinewidth{1.505625pt}%
\definecolor{currentstroke}{rgb}{0.552941,0.898039,0.631373}%
\pgfsetstrokecolor{currentstroke}%
\pgfsetstrokeopacity{0.200000}%
\pgfsetdash{}{0pt}%
\pgfpathmoveto{\pgfqpoint{5.865422in}{1.611223in}}%
\pgfpathlineto{\pgfqpoint{4.400818in}{2.895337in}}%
\pgfusepath{stroke}%
\end{pgfscope}%
\begin{pgfscope}%
\pgfpathrectangle{\pgfqpoint{0.526127in}{0.331635in}}{\pgfqpoint{9.300000in}{7.700000in}}%
\pgfusepath{clip}%
\pgfsetrectcap%
\pgfsetroundjoin%
\pgfsetlinewidth{1.505625pt}%
\definecolor{currentstroke}{rgb}{0.552941,0.898039,0.631373}%
\pgfsetstrokecolor{currentstroke}%
\pgfsetstrokeopacity{0.200000}%
\pgfsetdash{}{0pt}%
\pgfpathmoveto{\pgfqpoint{6.692745in}{2.208927in}}%
\pgfpathlineto{\pgfqpoint{4.400818in}{2.895337in}}%
\pgfusepath{stroke}%
\end{pgfscope}%
\begin{pgfscope}%
\pgfpathrectangle{\pgfqpoint{0.526127in}{0.331635in}}{\pgfqpoint{9.300000in}{7.700000in}}%
\pgfusepath{clip}%
\pgfsetrectcap%
\pgfsetroundjoin%
\pgfsetlinewidth{1.505625pt}%
\definecolor{currentstroke}{rgb}{0.552941,0.898039,0.631373}%
\pgfsetstrokecolor{currentstroke}%
\pgfsetstrokeopacity{0.200000}%
\pgfsetdash{}{0pt}%
\pgfpathmoveto{\pgfqpoint{2.147474in}{3.396130in}}%
\pgfpathlineto{\pgfqpoint{4.400818in}{2.895337in}}%
\pgfusepath{stroke}%
\end{pgfscope}%
\begin{pgfscope}%
\pgfpathrectangle{\pgfqpoint{0.526127in}{0.331635in}}{\pgfqpoint{9.300000in}{7.700000in}}%
\pgfusepath{clip}%
\pgfsetrectcap%
\pgfsetroundjoin%
\pgfsetlinewidth{1.505625pt}%
\definecolor{currentstroke}{rgb}{0.552941,0.898039,0.631373}%
\pgfsetstrokecolor{currentstroke}%
\pgfsetstrokeopacity{0.200000}%
\pgfsetdash{}{0pt}%
\pgfpathmoveto{\pgfqpoint{7.116859in}{3.140994in}}%
\pgfpathlineto{\pgfqpoint{4.400818in}{2.895337in}}%
\pgfusepath{stroke}%
\end{pgfscope}%
\begin{pgfscope}%
\pgfpathrectangle{\pgfqpoint{0.526127in}{0.331635in}}{\pgfqpoint{9.300000in}{7.700000in}}%
\pgfusepath{clip}%
\pgfsetrectcap%
\pgfsetroundjoin%
\pgfsetlinewidth{1.505625pt}%
\definecolor{currentstroke}{rgb}{0.552941,0.898039,0.631373}%
\pgfsetstrokecolor{currentstroke}%
\pgfsetstrokeopacity{0.200000}%
\pgfsetdash{}{0pt}%
\pgfpathmoveto{\pgfqpoint{3.558492in}{4.145234in}}%
\pgfpathlineto{\pgfqpoint{4.400818in}{2.895337in}}%
\pgfusepath{stroke}%
\end{pgfscope}%
\begin{pgfscope}%
\pgfpathrectangle{\pgfqpoint{0.526127in}{0.331635in}}{\pgfqpoint{9.300000in}{7.700000in}}%
\pgfusepath{clip}%
\pgfsetrectcap%
\pgfsetroundjoin%
\pgfsetlinewidth{1.505625pt}%
\definecolor{currentstroke}{rgb}{0.552941,0.898039,0.631373}%
\pgfsetstrokecolor{currentstroke}%
\pgfsetstrokeopacity{0.200000}%
\pgfsetdash{}{0pt}%
\pgfpathmoveto{\pgfqpoint{3.383051in}{1.569295in}}%
\pgfpathlineto{\pgfqpoint{4.400818in}{2.895337in}}%
\pgfusepath{stroke}%
\end{pgfscope}%
\begin{pgfscope}%
\pgfpathrectangle{\pgfqpoint{0.526127in}{0.331635in}}{\pgfqpoint{9.300000in}{7.700000in}}%
\pgfusepath{clip}%
\pgfsetrectcap%
\pgfsetroundjoin%
\pgfsetlinewidth{1.505625pt}%
\definecolor{currentstroke}{rgb}{0.552941,0.898039,0.631373}%
\pgfsetstrokecolor{currentstroke}%
\pgfsetstrokeopacity{0.200000}%
\pgfsetdash{}{0pt}%
\pgfpathmoveto{\pgfqpoint{6.701250in}{2.606389in}}%
\pgfpathlineto{\pgfqpoint{4.400818in}{2.895337in}}%
\pgfusepath{stroke}%
\end{pgfscope}%
\begin{pgfscope}%
\pgfpathrectangle{\pgfqpoint{0.526127in}{0.331635in}}{\pgfqpoint{9.300000in}{7.700000in}}%
\pgfusepath{clip}%
\pgfsetrectcap%
\pgfsetroundjoin%
\pgfsetlinewidth{1.505625pt}%
\definecolor{currentstroke}{rgb}{0.552941,0.898039,0.631373}%
\pgfsetstrokecolor{currentstroke}%
\pgfsetstrokeopacity{0.200000}%
\pgfsetdash{}{0pt}%
\pgfpathmoveto{\pgfqpoint{6.060531in}{3.426761in}}%
\pgfpathlineto{\pgfqpoint{4.400818in}{2.895337in}}%
\pgfusepath{stroke}%
\end{pgfscope}%
\begin{pgfscope}%
\pgfpathrectangle{\pgfqpoint{0.526127in}{0.331635in}}{\pgfqpoint{9.300000in}{7.700000in}}%
\pgfusepath{clip}%
\pgfsetrectcap%
\pgfsetroundjoin%
\pgfsetlinewidth{1.505625pt}%
\definecolor{currentstroke}{rgb}{0.552941,0.898039,0.631373}%
\pgfsetstrokecolor{currentstroke}%
\pgfsetstrokeopacity{0.200000}%
\pgfsetdash{}{0pt}%
\pgfpathmoveto{\pgfqpoint{3.545333in}{4.514559in}}%
\pgfpathlineto{\pgfqpoint{4.400818in}{2.895337in}}%
\pgfusepath{stroke}%
\end{pgfscope}%
\begin{pgfscope}%
\pgfpathrectangle{\pgfqpoint{0.526127in}{0.331635in}}{\pgfqpoint{9.300000in}{7.700000in}}%
\pgfusepath{clip}%
\pgfsetrectcap%
\pgfsetroundjoin%
\pgfsetlinewidth{1.505625pt}%
\definecolor{currentstroke}{rgb}{0.552941,0.898039,0.631373}%
\pgfsetstrokecolor{currentstroke}%
\pgfsetstrokeopacity{0.200000}%
\pgfsetdash{}{0pt}%
\pgfpathmoveto{\pgfqpoint{2.060713in}{2.922311in}}%
\pgfpathlineto{\pgfqpoint{4.400818in}{2.895337in}}%
\pgfusepath{stroke}%
\end{pgfscope}%
\begin{pgfscope}%
\pgfpathrectangle{\pgfqpoint{0.526127in}{0.331635in}}{\pgfqpoint{9.300000in}{7.700000in}}%
\pgfusepath{clip}%
\pgfsetrectcap%
\pgfsetroundjoin%
\pgfsetlinewidth{1.505625pt}%
\definecolor{currentstroke}{rgb}{0.552941,0.898039,0.631373}%
\pgfsetstrokecolor{currentstroke}%
\pgfsetstrokeopacity{0.200000}%
\pgfsetdash{}{0pt}%
\pgfpathmoveto{\pgfqpoint{9.178578in}{4.703915in}}%
\pgfpathlineto{\pgfqpoint{4.400818in}{2.895337in}}%
\pgfusepath{stroke}%
\end{pgfscope}%
\begin{pgfscope}%
\pgfpathrectangle{\pgfqpoint{0.526127in}{0.331635in}}{\pgfqpoint{9.300000in}{7.700000in}}%
\pgfusepath{clip}%
\pgfsetrectcap%
\pgfsetroundjoin%
\pgfsetlinewidth{1.505625pt}%
\definecolor{currentstroke}{rgb}{0.552941,0.898039,0.631373}%
\pgfsetstrokecolor{currentstroke}%
\pgfsetstrokeopacity{0.200000}%
\pgfsetdash{}{0pt}%
\pgfpathmoveto{\pgfqpoint{2.528674in}{2.455523in}}%
\pgfpathlineto{\pgfqpoint{4.400818in}{2.895337in}}%
\pgfusepath{stroke}%
\end{pgfscope}%
\begin{pgfscope}%
\pgfpathrectangle{\pgfqpoint{0.526127in}{0.331635in}}{\pgfqpoint{9.300000in}{7.700000in}}%
\pgfusepath{clip}%
\pgfsetrectcap%
\pgfsetroundjoin%
\pgfsetlinewidth{1.505625pt}%
\definecolor{currentstroke}{rgb}{0.552941,0.898039,0.631373}%
\pgfsetstrokecolor{currentstroke}%
\pgfsetstrokeopacity{0.200000}%
\pgfsetdash{}{0pt}%
\pgfpathmoveto{\pgfqpoint{2.638599in}{2.966238in}}%
\pgfpathlineto{\pgfqpoint{4.400818in}{2.895337in}}%
\pgfusepath{stroke}%
\end{pgfscope}%
\begin{pgfscope}%
\pgfpathrectangle{\pgfqpoint{0.526127in}{0.331635in}}{\pgfqpoint{9.300000in}{7.700000in}}%
\pgfusepath{clip}%
\pgfsetrectcap%
\pgfsetroundjoin%
\pgfsetlinewidth{1.505625pt}%
\definecolor{currentstroke}{rgb}{0.552941,0.898039,0.631373}%
\pgfsetstrokecolor{currentstroke}%
\pgfsetstrokeopacity{0.200000}%
\pgfsetdash{}{0pt}%
\pgfpathmoveto{\pgfqpoint{2.718231in}{2.271722in}}%
\pgfpathlineto{\pgfqpoint{4.400818in}{2.895337in}}%
\pgfusepath{stroke}%
\end{pgfscope}%
\begin{pgfscope}%
\pgfpathrectangle{\pgfqpoint{0.526127in}{0.331635in}}{\pgfqpoint{9.300000in}{7.700000in}}%
\pgfusepath{clip}%
\pgfsetrectcap%
\pgfsetroundjoin%
\pgfsetlinewidth{1.505625pt}%
\definecolor{currentstroke}{rgb}{0.552941,0.898039,0.631373}%
\pgfsetstrokecolor{currentstroke}%
\pgfsetstrokeopacity{0.200000}%
\pgfsetdash{}{0pt}%
\pgfpathmoveto{\pgfqpoint{1.976182in}{2.522263in}}%
\pgfpathlineto{\pgfqpoint{4.400818in}{2.895337in}}%
\pgfusepath{stroke}%
\end{pgfscope}%
\begin{pgfscope}%
\pgfpathrectangle{\pgfqpoint{0.526127in}{0.331635in}}{\pgfqpoint{9.300000in}{7.700000in}}%
\pgfusepath{clip}%
\pgfsetrectcap%
\pgfsetroundjoin%
\pgfsetlinewidth{1.505625pt}%
\definecolor{currentstroke}{rgb}{0.552941,0.898039,0.631373}%
\pgfsetstrokecolor{currentstroke}%
\pgfsetstrokeopacity{0.200000}%
\pgfsetdash{}{0pt}%
\pgfpathmoveto{\pgfqpoint{2.309460in}{2.915022in}}%
\pgfpathlineto{\pgfqpoint{4.400818in}{2.895337in}}%
\pgfusepath{stroke}%
\end{pgfscope}%
\begin{pgfscope}%
\pgfpathrectangle{\pgfqpoint{0.526127in}{0.331635in}}{\pgfqpoint{9.300000in}{7.700000in}}%
\pgfusepath{clip}%
\pgfsetrectcap%
\pgfsetroundjoin%
\pgfsetlinewidth{1.505625pt}%
\definecolor{currentstroke}{rgb}{0.552941,0.898039,0.631373}%
\pgfsetstrokecolor{currentstroke}%
\pgfsetstrokeopacity{0.200000}%
\pgfsetdash{}{0pt}%
\pgfpathmoveto{\pgfqpoint{1.801343in}{3.506841in}}%
\pgfpathlineto{\pgfqpoint{4.400818in}{2.895337in}}%
\pgfusepath{stroke}%
\end{pgfscope}%
\begin{pgfscope}%
\pgfpathrectangle{\pgfqpoint{0.526127in}{0.331635in}}{\pgfqpoint{9.300000in}{7.700000in}}%
\pgfusepath{clip}%
\pgfsetrectcap%
\pgfsetroundjoin%
\pgfsetlinewidth{1.505625pt}%
\definecolor{currentstroke}{rgb}{0.552941,0.898039,0.631373}%
\pgfsetstrokecolor{currentstroke}%
\pgfsetstrokeopacity{0.200000}%
\pgfsetdash{}{0pt}%
\pgfpathmoveto{\pgfqpoint{2.821384in}{2.813377in}}%
\pgfpathlineto{\pgfqpoint{4.400818in}{2.895337in}}%
\pgfusepath{stroke}%
\end{pgfscope}%
\begin{pgfscope}%
\pgfpathrectangle{\pgfqpoint{0.526127in}{0.331635in}}{\pgfqpoint{9.300000in}{7.700000in}}%
\pgfusepath{clip}%
\pgfsetrectcap%
\pgfsetroundjoin%
\pgfsetlinewidth{1.505625pt}%
\definecolor{currentstroke}{rgb}{0.552941,0.898039,0.631373}%
\pgfsetstrokecolor{currentstroke}%
\pgfsetstrokeopacity{0.200000}%
\pgfsetdash{}{0pt}%
\pgfpathmoveto{\pgfqpoint{7.035760in}{1.052321in}}%
\pgfpathlineto{\pgfqpoint{4.400818in}{2.895337in}}%
\pgfusepath{stroke}%
\end{pgfscope}%
\begin{pgfscope}%
\pgfpathrectangle{\pgfqpoint{0.526127in}{0.331635in}}{\pgfqpoint{9.300000in}{7.700000in}}%
\pgfusepath{clip}%
\pgfsetrectcap%
\pgfsetroundjoin%
\pgfsetlinewidth{1.505625pt}%
\definecolor{currentstroke}{rgb}{0.552941,0.898039,0.631373}%
\pgfsetstrokecolor{currentstroke}%
\pgfsetstrokeopacity{0.200000}%
\pgfsetdash{}{0pt}%
\pgfpathmoveto{\pgfqpoint{6.494262in}{1.175940in}}%
\pgfpathlineto{\pgfqpoint{4.400818in}{2.895337in}}%
\pgfusepath{stroke}%
\end{pgfscope}%
\begin{pgfscope}%
\pgfpathrectangle{\pgfqpoint{0.526127in}{0.331635in}}{\pgfqpoint{9.300000in}{7.700000in}}%
\pgfusepath{clip}%
\pgfsetrectcap%
\pgfsetroundjoin%
\pgfsetlinewidth{1.505625pt}%
\definecolor{currentstroke}{rgb}{0.552941,0.898039,0.631373}%
\pgfsetstrokecolor{currentstroke}%
\pgfsetstrokeopacity{0.200000}%
\pgfsetdash{}{0pt}%
\pgfpathmoveto{\pgfqpoint{5.499012in}{2.332218in}}%
\pgfpathlineto{\pgfqpoint{4.400818in}{2.895337in}}%
\pgfusepath{stroke}%
\end{pgfscope}%
\begin{pgfscope}%
\pgfpathrectangle{\pgfqpoint{0.526127in}{0.331635in}}{\pgfqpoint{9.300000in}{7.700000in}}%
\pgfusepath{clip}%
\pgfsetrectcap%
\pgfsetroundjoin%
\pgfsetlinewidth{1.505625pt}%
\definecolor{currentstroke}{rgb}{0.552941,0.898039,0.631373}%
\pgfsetstrokecolor{currentstroke}%
\pgfsetstrokeopacity{0.200000}%
\pgfsetdash{}{0pt}%
\pgfpathmoveto{\pgfqpoint{6.082548in}{1.320932in}}%
\pgfpathlineto{\pgfqpoint{4.400818in}{2.895337in}}%
\pgfusepath{stroke}%
\end{pgfscope}%
\begin{pgfscope}%
\pgfpathrectangle{\pgfqpoint{0.526127in}{0.331635in}}{\pgfqpoint{9.300000in}{7.700000in}}%
\pgfusepath{clip}%
\pgfsetrectcap%
\pgfsetroundjoin%
\pgfsetlinewidth{1.505625pt}%
\definecolor{currentstroke}{rgb}{0.552941,0.898039,0.631373}%
\pgfsetstrokecolor{currentstroke}%
\pgfsetstrokeopacity{0.200000}%
\pgfsetdash{}{0pt}%
\pgfpathmoveto{\pgfqpoint{1.873484in}{3.907469in}}%
\pgfpathlineto{\pgfqpoint{4.400818in}{2.895337in}}%
\pgfusepath{stroke}%
\end{pgfscope}%
\begin{pgfscope}%
\pgfpathrectangle{\pgfqpoint{0.526127in}{0.331635in}}{\pgfqpoint{9.300000in}{7.700000in}}%
\pgfusepath{clip}%
\pgfsetrectcap%
\pgfsetroundjoin%
\pgfsetlinewidth{1.505625pt}%
\definecolor{currentstroke}{rgb}{0.552941,0.898039,0.631373}%
\pgfsetstrokecolor{currentstroke}%
\pgfsetstrokeopacity{0.200000}%
\pgfsetdash{}{0pt}%
\pgfpathmoveto{\pgfqpoint{5.027322in}{3.028149in}}%
\pgfpathlineto{\pgfqpoint{4.400818in}{2.895337in}}%
\pgfusepath{stroke}%
\end{pgfscope}%
\begin{pgfscope}%
\pgfpathrectangle{\pgfqpoint{0.526127in}{0.331635in}}{\pgfqpoint{9.300000in}{7.700000in}}%
\pgfusepath{clip}%
\pgfsetrectcap%
\pgfsetroundjoin%
\pgfsetlinewidth{1.505625pt}%
\definecolor{currentstroke}{rgb}{0.552941,0.898039,0.631373}%
\pgfsetstrokecolor{currentstroke}%
\pgfsetstrokeopacity{0.200000}%
\pgfsetdash{}{0pt}%
\pgfpathmoveto{\pgfqpoint{1.754896in}{3.499956in}}%
\pgfpathlineto{\pgfqpoint{4.400818in}{2.895337in}}%
\pgfusepath{stroke}%
\end{pgfscope}%
\begin{pgfscope}%
\pgfpathrectangle{\pgfqpoint{0.526127in}{0.331635in}}{\pgfqpoint{9.300000in}{7.700000in}}%
\pgfusepath{clip}%
\pgfsetrectcap%
\pgfsetroundjoin%
\pgfsetlinewidth{1.505625pt}%
\definecolor{currentstroke}{rgb}{0.552941,0.898039,0.631373}%
\pgfsetstrokecolor{currentstroke}%
\pgfsetstrokeopacity{0.200000}%
\pgfsetdash{}{0pt}%
\pgfpathmoveto{\pgfqpoint{3.811390in}{5.344873in}}%
\pgfpathlineto{\pgfqpoint{4.400818in}{2.895337in}}%
\pgfusepath{stroke}%
\end{pgfscope}%
\begin{pgfscope}%
\pgfpathrectangle{\pgfqpoint{0.526127in}{0.331635in}}{\pgfqpoint{9.300000in}{7.700000in}}%
\pgfusepath{clip}%
\pgfsetrectcap%
\pgfsetroundjoin%
\pgfsetlinewidth{1.505625pt}%
\definecolor{currentstroke}{rgb}{0.552941,0.898039,0.631373}%
\pgfsetstrokecolor{currentstroke}%
\pgfsetstrokeopacity{0.200000}%
\pgfsetdash{}{0pt}%
\pgfpathmoveto{\pgfqpoint{4.820098in}{3.423588in}}%
\pgfpathlineto{\pgfqpoint{4.400818in}{2.895337in}}%
\pgfusepath{stroke}%
\end{pgfscope}%
\begin{pgfscope}%
\pgfpathrectangle{\pgfqpoint{0.526127in}{0.331635in}}{\pgfqpoint{9.300000in}{7.700000in}}%
\pgfusepath{clip}%
\pgfsetrectcap%
\pgfsetroundjoin%
\pgfsetlinewidth{1.505625pt}%
\definecolor{currentstroke}{rgb}{1.000000,0.623529,0.607843}%
\pgfsetstrokecolor{currentstroke}%
\pgfsetstrokeopacity{0.200000}%
\pgfsetdash{}{0pt}%
\pgfpathmoveto{\pgfqpoint{6.036344in}{7.681635in}}%
\pgfpathlineto{\pgfqpoint{5.110509in}{5.016914in}}%
\pgfusepath{stroke}%
\end{pgfscope}%
\begin{pgfscope}%
\pgfpathrectangle{\pgfqpoint{0.526127in}{0.331635in}}{\pgfqpoint{9.300000in}{7.700000in}}%
\pgfusepath{clip}%
\pgfsetrectcap%
\pgfsetroundjoin%
\pgfsetlinewidth{1.505625pt}%
\definecolor{currentstroke}{rgb}{1.000000,0.623529,0.607843}%
\pgfsetstrokecolor{currentstroke}%
\pgfsetstrokeopacity{0.200000}%
\pgfsetdash{}{0pt}%
\pgfpathmoveto{\pgfqpoint{5.059438in}{4.737486in}}%
\pgfpathlineto{\pgfqpoint{5.110509in}{5.016914in}}%
\pgfusepath{stroke}%
\end{pgfscope}%
\begin{pgfscope}%
\pgfpathrectangle{\pgfqpoint{0.526127in}{0.331635in}}{\pgfqpoint{9.300000in}{7.700000in}}%
\pgfusepath{clip}%
\pgfsetrectcap%
\pgfsetroundjoin%
\pgfsetlinewidth{1.505625pt}%
\definecolor{currentstroke}{rgb}{1.000000,0.623529,0.607843}%
\pgfsetstrokecolor{currentstroke}%
\pgfsetstrokeopacity{0.200000}%
\pgfsetdash{}{0pt}%
\pgfpathmoveto{\pgfqpoint{3.537682in}{4.765579in}}%
\pgfpathlineto{\pgfqpoint{5.110509in}{5.016914in}}%
\pgfusepath{stroke}%
\end{pgfscope}%
\begin{pgfscope}%
\pgfpathrectangle{\pgfqpoint{0.526127in}{0.331635in}}{\pgfqpoint{9.300000in}{7.700000in}}%
\pgfusepath{clip}%
\pgfsetrectcap%
\pgfsetroundjoin%
\pgfsetlinewidth{1.505625pt}%
\definecolor{currentstroke}{rgb}{1.000000,0.623529,0.607843}%
\pgfsetstrokecolor{currentstroke}%
\pgfsetstrokeopacity{0.200000}%
\pgfsetdash{}{0pt}%
\pgfpathmoveto{\pgfqpoint{3.454692in}{5.766423in}}%
\pgfpathlineto{\pgfqpoint{5.110509in}{5.016914in}}%
\pgfusepath{stroke}%
\end{pgfscope}%
\begin{pgfscope}%
\pgfpathrectangle{\pgfqpoint{0.526127in}{0.331635in}}{\pgfqpoint{9.300000in}{7.700000in}}%
\pgfusepath{clip}%
\pgfsetrectcap%
\pgfsetroundjoin%
\pgfsetlinewidth{1.505625pt}%
\definecolor{currentstroke}{rgb}{1.000000,0.623529,0.607843}%
\pgfsetstrokecolor{currentstroke}%
\pgfsetstrokeopacity{0.200000}%
\pgfsetdash{}{0pt}%
\pgfpathmoveto{\pgfqpoint{5.176399in}{7.406573in}}%
\pgfpathlineto{\pgfqpoint{5.110509in}{5.016914in}}%
\pgfusepath{stroke}%
\end{pgfscope}%
\begin{pgfscope}%
\pgfpathrectangle{\pgfqpoint{0.526127in}{0.331635in}}{\pgfqpoint{9.300000in}{7.700000in}}%
\pgfusepath{clip}%
\pgfsetrectcap%
\pgfsetroundjoin%
\pgfsetlinewidth{1.505625pt}%
\definecolor{currentstroke}{rgb}{1.000000,0.623529,0.607843}%
\pgfsetstrokecolor{currentstroke}%
\pgfsetstrokeopacity{0.200000}%
\pgfsetdash{}{0pt}%
\pgfpathmoveto{\pgfqpoint{5.896285in}{3.492514in}}%
\pgfpathlineto{\pgfqpoint{5.110509in}{5.016914in}}%
\pgfusepath{stroke}%
\end{pgfscope}%
\begin{pgfscope}%
\pgfpathrectangle{\pgfqpoint{0.526127in}{0.331635in}}{\pgfqpoint{9.300000in}{7.700000in}}%
\pgfusepath{clip}%
\pgfsetrectcap%
\pgfsetroundjoin%
\pgfsetlinewidth{1.505625pt}%
\definecolor{currentstroke}{rgb}{1.000000,0.623529,0.607843}%
\pgfsetstrokecolor{currentstroke}%
\pgfsetstrokeopacity{0.200000}%
\pgfsetdash{}{0pt}%
\pgfpathmoveto{\pgfqpoint{5.765801in}{5.128354in}}%
\pgfpathlineto{\pgfqpoint{5.110509in}{5.016914in}}%
\pgfusepath{stroke}%
\end{pgfscope}%
\begin{pgfscope}%
\pgfpathrectangle{\pgfqpoint{0.526127in}{0.331635in}}{\pgfqpoint{9.300000in}{7.700000in}}%
\pgfusepath{clip}%
\pgfsetrectcap%
\pgfsetroundjoin%
\pgfsetlinewidth{1.505625pt}%
\definecolor{currentstroke}{rgb}{1.000000,0.623529,0.607843}%
\pgfsetstrokecolor{currentstroke}%
\pgfsetstrokeopacity{0.200000}%
\pgfsetdash{}{0pt}%
\pgfpathmoveto{\pgfqpoint{3.919124in}{4.275736in}}%
\pgfpathlineto{\pgfqpoint{5.110509in}{5.016914in}}%
\pgfusepath{stroke}%
\end{pgfscope}%
\begin{pgfscope}%
\pgfpathrectangle{\pgfqpoint{0.526127in}{0.331635in}}{\pgfqpoint{9.300000in}{7.700000in}}%
\pgfusepath{clip}%
\pgfsetrectcap%
\pgfsetroundjoin%
\pgfsetlinewidth{1.505625pt}%
\definecolor{currentstroke}{rgb}{1.000000,0.623529,0.607843}%
\pgfsetstrokecolor{currentstroke}%
\pgfsetstrokeopacity{0.200000}%
\pgfsetdash{}{0pt}%
\pgfpathmoveto{\pgfqpoint{7.219680in}{3.402146in}}%
\pgfpathlineto{\pgfqpoint{5.110509in}{5.016914in}}%
\pgfusepath{stroke}%
\end{pgfscope}%
\begin{pgfscope}%
\pgfpathrectangle{\pgfqpoint{0.526127in}{0.331635in}}{\pgfqpoint{9.300000in}{7.700000in}}%
\pgfusepath{clip}%
\pgfsetrectcap%
\pgfsetroundjoin%
\pgfsetlinewidth{1.505625pt}%
\definecolor{currentstroke}{rgb}{1.000000,0.623529,0.607843}%
\pgfsetstrokecolor{currentstroke}%
\pgfsetstrokeopacity{0.200000}%
\pgfsetdash{}{0pt}%
\pgfpathmoveto{\pgfqpoint{4.459787in}{4.784008in}}%
\pgfpathlineto{\pgfqpoint{5.110509in}{5.016914in}}%
\pgfusepath{stroke}%
\end{pgfscope}%
\begin{pgfscope}%
\pgfpathrectangle{\pgfqpoint{0.526127in}{0.331635in}}{\pgfqpoint{9.300000in}{7.700000in}}%
\pgfusepath{clip}%
\pgfsetrectcap%
\pgfsetroundjoin%
\pgfsetlinewidth{1.505625pt}%
\definecolor{currentstroke}{rgb}{1.000000,0.623529,0.607843}%
\pgfsetstrokecolor{currentstroke}%
\pgfsetstrokeopacity{0.200000}%
\pgfsetdash{}{0pt}%
\pgfpathmoveto{\pgfqpoint{7.438369in}{3.992349in}}%
\pgfpathlineto{\pgfqpoint{5.110509in}{5.016914in}}%
\pgfusepath{stroke}%
\end{pgfscope}%
\begin{pgfscope}%
\pgfpathrectangle{\pgfqpoint{0.526127in}{0.331635in}}{\pgfqpoint{9.300000in}{7.700000in}}%
\pgfusepath{clip}%
\pgfsetrectcap%
\pgfsetroundjoin%
\pgfsetlinewidth{1.505625pt}%
\definecolor{currentstroke}{rgb}{1.000000,0.623529,0.607843}%
\pgfsetstrokecolor{currentstroke}%
\pgfsetstrokeopacity{0.200000}%
\pgfsetdash{}{0pt}%
\pgfpathmoveto{\pgfqpoint{4.906391in}{4.964376in}}%
\pgfpathlineto{\pgfqpoint{5.110509in}{5.016914in}}%
\pgfusepath{stroke}%
\end{pgfscope}%
\begin{pgfscope}%
\pgfpathrectangle{\pgfqpoint{0.526127in}{0.331635in}}{\pgfqpoint{9.300000in}{7.700000in}}%
\pgfusepath{clip}%
\pgfsetrectcap%
\pgfsetroundjoin%
\pgfsetlinewidth{1.505625pt}%
\definecolor{currentstroke}{rgb}{1.000000,0.623529,0.607843}%
\pgfsetstrokecolor{currentstroke}%
\pgfsetstrokeopacity{0.200000}%
\pgfsetdash{}{0pt}%
\pgfpathmoveto{\pgfqpoint{5.942000in}{5.911816in}}%
\pgfpathlineto{\pgfqpoint{5.110509in}{5.016914in}}%
\pgfusepath{stroke}%
\end{pgfscope}%
\begin{pgfscope}%
\pgfpathrectangle{\pgfqpoint{0.526127in}{0.331635in}}{\pgfqpoint{9.300000in}{7.700000in}}%
\pgfusepath{clip}%
\pgfsetrectcap%
\pgfsetroundjoin%
\pgfsetlinewidth{1.505625pt}%
\definecolor{currentstroke}{rgb}{1.000000,0.623529,0.607843}%
\pgfsetstrokecolor{currentstroke}%
\pgfsetstrokeopacity{0.200000}%
\pgfsetdash{}{0pt}%
\pgfpathmoveto{\pgfqpoint{5.122725in}{5.271620in}}%
\pgfpathlineto{\pgfqpoint{5.110509in}{5.016914in}}%
\pgfusepath{stroke}%
\end{pgfscope}%
\begin{pgfscope}%
\pgfpathrectangle{\pgfqpoint{0.526127in}{0.331635in}}{\pgfqpoint{9.300000in}{7.700000in}}%
\pgfusepath{clip}%
\pgfsetrectcap%
\pgfsetroundjoin%
\pgfsetlinewidth{1.505625pt}%
\definecolor{currentstroke}{rgb}{1.000000,0.623529,0.607843}%
\pgfsetstrokecolor{currentstroke}%
\pgfsetstrokeopacity{0.200000}%
\pgfsetdash{}{0pt}%
\pgfpathmoveto{\pgfqpoint{6.430193in}{5.110591in}}%
\pgfpathlineto{\pgfqpoint{5.110509in}{5.016914in}}%
\pgfusepath{stroke}%
\end{pgfscope}%
\begin{pgfscope}%
\pgfpathrectangle{\pgfqpoint{0.526127in}{0.331635in}}{\pgfqpoint{9.300000in}{7.700000in}}%
\pgfusepath{clip}%
\pgfsetrectcap%
\pgfsetroundjoin%
\pgfsetlinewidth{1.505625pt}%
\definecolor{currentstroke}{rgb}{1.000000,0.623529,0.607843}%
\pgfsetstrokecolor{currentstroke}%
\pgfsetstrokeopacity{0.200000}%
\pgfsetdash{}{0pt}%
\pgfpathmoveto{\pgfqpoint{5.393295in}{4.050774in}}%
\pgfpathlineto{\pgfqpoint{5.110509in}{5.016914in}}%
\pgfusepath{stroke}%
\end{pgfscope}%
\begin{pgfscope}%
\pgfpathrectangle{\pgfqpoint{0.526127in}{0.331635in}}{\pgfqpoint{9.300000in}{7.700000in}}%
\pgfusepath{clip}%
\pgfsetrectcap%
\pgfsetroundjoin%
\pgfsetlinewidth{1.505625pt}%
\definecolor{currentstroke}{rgb}{1.000000,0.623529,0.607843}%
\pgfsetstrokecolor{currentstroke}%
\pgfsetstrokeopacity{0.200000}%
\pgfsetdash{}{0pt}%
\pgfpathmoveto{\pgfqpoint{4.106444in}{4.534616in}}%
\pgfpathlineto{\pgfqpoint{5.110509in}{5.016914in}}%
\pgfusepath{stroke}%
\end{pgfscope}%
\begin{pgfscope}%
\pgfpathrectangle{\pgfqpoint{0.526127in}{0.331635in}}{\pgfqpoint{9.300000in}{7.700000in}}%
\pgfusepath{clip}%
\pgfsetrectcap%
\pgfsetroundjoin%
\pgfsetlinewidth{1.505625pt}%
\definecolor{currentstroke}{rgb}{1.000000,0.623529,0.607843}%
\pgfsetstrokecolor{currentstroke}%
\pgfsetstrokeopacity{0.200000}%
\pgfsetdash{}{0pt}%
\pgfpathmoveto{\pgfqpoint{3.340315in}{5.116509in}}%
\pgfpathlineto{\pgfqpoint{5.110509in}{5.016914in}}%
\pgfusepath{stroke}%
\end{pgfscope}%
\begin{pgfscope}%
\pgfpathrectangle{\pgfqpoint{0.526127in}{0.331635in}}{\pgfqpoint{9.300000in}{7.700000in}}%
\pgfusepath{clip}%
\pgfsetrectcap%
\pgfsetroundjoin%
\pgfsetlinewidth{1.505625pt}%
\definecolor{currentstroke}{rgb}{1.000000,0.623529,0.607843}%
\pgfsetstrokecolor{currentstroke}%
\pgfsetstrokeopacity{0.200000}%
\pgfsetdash{}{0pt}%
\pgfpathmoveto{\pgfqpoint{5.209743in}{4.859650in}}%
\pgfpathlineto{\pgfqpoint{5.110509in}{5.016914in}}%
\pgfusepath{stroke}%
\end{pgfscope}%
\begin{pgfscope}%
\pgfpathrectangle{\pgfqpoint{0.526127in}{0.331635in}}{\pgfqpoint{9.300000in}{7.700000in}}%
\pgfusepath{clip}%
\pgfsetrectcap%
\pgfsetroundjoin%
\pgfsetlinewidth{1.505625pt}%
\definecolor{currentstroke}{rgb}{1.000000,0.623529,0.607843}%
\pgfsetstrokecolor{currentstroke}%
\pgfsetstrokeopacity{0.200000}%
\pgfsetdash{}{0pt}%
\pgfpathmoveto{\pgfqpoint{4.735213in}{4.875154in}}%
\pgfpathlineto{\pgfqpoint{5.110509in}{5.016914in}}%
\pgfusepath{stroke}%
\end{pgfscope}%
\begin{pgfscope}%
\pgfpathrectangle{\pgfqpoint{0.526127in}{0.331635in}}{\pgfqpoint{9.300000in}{7.700000in}}%
\pgfusepath{clip}%
\pgfsetrectcap%
\pgfsetroundjoin%
\pgfsetlinewidth{1.505625pt}%
\definecolor{currentstroke}{rgb}{1.000000,0.623529,0.607843}%
\pgfsetstrokecolor{currentstroke}%
\pgfsetstrokeopacity{0.200000}%
\pgfsetdash{}{0pt}%
\pgfpathmoveto{\pgfqpoint{3.352726in}{5.683291in}}%
\pgfpathlineto{\pgfqpoint{5.110509in}{5.016914in}}%
\pgfusepath{stroke}%
\end{pgfscope}%
\begin{pgfscope}%
\pgfpathrectangle{\pgfqpoint{0.526127in}{0.331635in}}{\pgfqpoint{9.300000in}{7.700000in}}%
\pgfusepath{clip}%
\pgfsetrectcap%
\pgfsetroundjoin%
\pgfsetlinewidth{1.505625pt}%
\definecolor{currentstroke}{rgb}{1.000000,0.623529,0.607843}%
\pgfsetstrokecolor{currentstroke}%
\pgfsetstrokeopacity{0.200000}%
\pgfsetdash{}{0pt}%
\pgfpathmoveto{\pgfqpoint{4.527282in}{5.304490in}}%
\pgfpathlineto{\pgfqpoint{5.110509in}{5.016914in}}%
\pgfusepath{stroke}%
\end{pgfscope}%
\begin{pgfscope}%
\pgfpathrectangle{\pgfqpoint{0.526127in}{0.331635in}}{\pgfqpoint{9.300000in}{7.700000in}}%
\pgfusepath{clip}%
\pgfsetrectcap%
\pgfsetroundjoin%
\pgfsetlinewidth{1.505625pt}%
\definecolor{currentstroke}{rgb}{1.000000,0.623529,0.607843}%
\pgfsetstrokecolor{currentstroke}%
\pgfsetstrokeopacity{0.200000}%
\pgfsetdash{}{0pt}%
\pgfpathmoveto{\pgfqpoint{4.542592in}{3.135664in}}%
\pgfpathlineto{\pgfqpoint{5.110509in}{5.016914in}}%
\pgfusepath{stroke}%
\end{pgfscope}%
\begin{pgfscope}%
\pgfpathrectangle{\pgfqpoint{0.526127in}{0.331635in}}{\pgfqpoint{9.300000in}{7.700000in}}%
\pgfusepath{clip}%
\pgfsetrectcap%
\pgfsetroundjoin%
\pgfsetlinewidth{1.505625pt}%
\definecolor{currentstroke}{rgb}{1.000000,0.623529,0.607843}%
\pgfsetstrokecolor{currentstroke}%
\pgfsetstrokeopacity{0.200000}%
\pgfsetdash{}{0pt}%
\pgfpathmoveto{\pgfqpoint{6.055654in}{5.623590in}}%
\pgfpathlineto{\pgfqpoint{5.110509in}{5.016914in}}%
\pgfusepath{stroke}%
\end{pgfscope}%
\begin{pgfscope}%
\pgfpathrectangle{\pgfqpoint{0.526127in}{0.331635in}}{\pgfqpoint{9.300000in}{7.700000in}}%
\pgfusepath{clip}%
\pgfsetrectcap%
\pgfsetroundjoin%
\pgfsetlinewidth{1.505625pt}%
\definecolor{currentstroke}{rgb}{1.000000,0.623529,0.607843}%
\pgfsetstrokecolor{currentstroke}%
\pgfsetstrokeopacity{0.200000}%
\pgfsetdash{}{0pt}%
\pgfpathmoveto{\pgfqpoint{6.518607in}{5.425973in}}%
\pgfpathlineto{\pgfqpoint{5.110509in}{5.016914in}}%
\pgfusepath{stroke}%
\end{pgfscope}%
\begin{pgfscope}%
\pgfpathrectangle{\pgfqpoint{0.526127in}{0.331635in}}{\pgfqpoint{9.300000in}{7.700000in}}%
\pgfusepath{clip}%
\pgfsetrectcap%
\pgfsetroundjoin%
\pgfsetlinewidth{1.505625pt}%
\definecolor{currentstroke}{rgb}{1.000000,0.623529,0.607843}%
\pgfsetstrokecolor{currentstroke}%
\pgfsetstrokeopacity{0.200000}%
\pgfsetdash{}{0pt}%
\pgfpathmoveto{\pgfqpoint{4.977203in}{5.489320in}}%
\pgfpathlineto{\pgfqpoint{5.110509in}{5.016914in}}%
\pgfusepath{stroke}%
\end{pgfscope}%
\begin{pgfscope}%
\pgfpathrectangle{\pgfqpoint{0.526127in}{0.331635in}}{\pgfqpoint{9.300000in}{7.700000in}}%
\pgfusepath{clip}%
\pgfsetrectcap%
\pgfsetroundjoin%
\pgfsetlinewidth{1.505625pt}%
\definecolor{currentstroke}{rgb}{1.000000,0.623529,0.607843}%
\pgfsetstrokecolor{currentstroke}%
\pgfsetstrokeopacity{0.200000}%
\pgfsetdash{}{0pt}%
\pgfpathmoveto{\pgfqpoint{4.655568in}{5.336989in}}%
\pgfpathlineto{\pgfqpoint{5.110509in}{5.016914in}}%
\pgfusepath{stroke}%
\end{pgfscope}%
\begin{pgfscope}%
\pgfpathrectangle{\pgfqpoint{0.526127in}{0.331635in}}{\pgfqpoint{9.300000in}{7.700000in}}%
\pgfusepath{clip}%
\pgfsetrectcap%
\pgfsetroundjoin%
\pgfsetlinewidth{1.505625pt}%
\definecolor{currentstroke}{rgb}{1.000000,0.623529,0.607843}%
\pgfsetstrokecolor{currentstroke}%
\pgfsetstrokeopacity{0.200000}%
\pgfsetdash{}{0pt}%
\pgfpathmoveto{\pgfqpoint{5.314689in}{4.346369in}}%
\pgfpathlineto{\pgfqpoint{5.110509in}{5.016914in}}%
\pgfusepath{stroke}%
\end{pgfscope}%
\begin{pgfscope}%
\pgfpathrectangle{\pgfqpoint{0.526127in}{0.331635in}}{\pgfqpoint{9.300000in}{7.700000in}}%
\pgfusepath{clip}%
\pgfsetrectcap%
\pgfsetroundjoin%
\pgfsetlinewidth{1.505625pt}%
\definecolor{currentstroke}{rgb}{0.815686,0.733333,1.000000}%
\pgfsetstrokecolor{currentstroke}%
\pgfsetstrokeopacity{0.200000}%
\pgfsetdash{}{0pt}%
\pgfpathmoveto{\pgfqpoint{6.917849in}{2.703124in}}%
\pgfpathlineto{\pgfqpoint{5.359152in}{2.808495in}}%
\pgfusepath{stroke}%
\end{pgfscope}%
\begin{pgfscope}%
\pgfpathrectangle{\pgfqpoint{0.526127in}{0.331635in}}{\pgfqpoint{9.300000in}{7.700000in}}%
\pgfusepath{clip}%
\pgfsetrectcap%
\pgfsetroundjoin%
\pgfsetlinewidth{1.505625pt}%
\definecolor{currentstroke}{rgb}{0.815686,0.733333,1.000000}%
\pgfsetstrokecolor{currentstroke}%
\pgfsetstrokeopacity{0.200000}%
\pgfsetdash{}{0pt}%
\pgfpathmoveto{\pgfqpoint{6.254342in}{2.082954in}}%
\pgfpathlineto{\pgfqpoint{5.359152in}{2.808495in}}%
\pgfusepath{stroke}%
\end{pgfscope}%
\begin{pgfscope}%
\pgfpathrectangle{\pgfqpoint{0.526127in}{0.331635in}}{\pgfqpoint{9.300000in}{7.700000in}}%
\pgfusepath{clip}%
\pgfsetrectcap%
\pgfsetroundjoin%
\pgfsetlinewidth{1.505625pt}%
\definecolor{currentstroke}{rgb}{0.815686,0.733333,1.000000}%
\pgfsetstrokecolor{currentstroke}%
\pgfsetstrokeopacity{0.200000}%
\pgfsetdash{}{0pt}%
\pgfpathmoveto{\pgfqpoint{4.500103in}{1.860863in}}%
\pgfpathlineto{\pgfqpoint{5.359152in}{2.808495in}}%
\pgfusepath{stroke}%
\end{pgfscope}%
\begin{pgfscope}%
\pgfpathrectangle{\pgfqpoint{0.526127in}{0.331635in}}{\pgfqpoint{9.300000in}{7.700000in}}%
\pgfusepath{clip}%
\pgfsetrectcap%
\pgfsetroundjoin%
\pgfsetlinewidth{1.505625pt}%
\definecolor{currentstroke}{rgb}{0.815686,0.733333,1.000000}%
\pgfsetstrokecolor{currentstroke}%
\pgfsetstrokeopacity{0.200000}%
\pgfsetdash{}{0pt}%
\pgfpathmoveto{\pgfqpoint{3.205256in}{1.832721in}}%
\pgfpathlineto{\pgfqpoint{5.359152in}{2.808495in}}%
\pgfusepath{stroke}%
\end{pgfscope}%
\begin{pgfscope}%
\pgfpathrectangle{\pgfqpoint{0.526127in}{0.331635in}}{\pgfqpoint{9.300000in}{7.700000in}}%
\pgfusepath{clip}%
\pgfsetrectcap%
\pgfsetroundjoin%
\pgfsetlinewidth{1.505625pt}%
\definecolor{currentstroke}{rgb}{0.815686,0.733333,1.000000}%
\pgfsetstrokecolor{currentstroke}%
\pgfsetstrokeopacity{0.200000}%
\pgfsetdash{}{0pt}%
\pgfpathmoveto{\pgfqpoint{6.199623in}{2.218091in}}%
\pgfpathlineto{\pgfqpoint{5.359152in}{2.808495in}}%
\pgfusepath{stroke}%
\end{pgfscope}%
\begin{pgfscope}%
\pgfpathrectangle{\pgfqpoint{0.526127in}{0.331635in}}{\pgfqpoint{9.300000in}{7.700000in}}%
\pgfusepath{clip}%
\pgfsetrectcap%
\pgfsetroundjoin%
\pgfsetlinewidth{1.505625pt}%
\definecolor{currentstroke}{rgb}{0.815686,0.733333,1.000000}%
\pgfsetstrokecolor{currentstroke}%
\pgfsetstrokeopacity{0.200000}%
\pgfsetdash{}{0pt}%
\pgfpathmoveto{\pgfqpoint{7.384542in}{2.729852in}}%
\pgfpathlineto{\pgfqpoint{5.359152in}{2.808495in}}%
\pgfusepath{stroke}%
\end{pgfscope}%
\begin{pgfscope}%
\pgfpathrectangle{\pgfqpoint{0.526127in}{0.331635in}}{\pgfqpoint{9.300000in}{7.700000in}}%
\pgfusepath{clip}%
\pgfsetrectcap%
\pgfsetroundjoin%
\pgfsetlinewidth{1.505625pt}%
\definecolor{currentstroke}{rgb}{0.815686,0.733333,1.000000}%
\pgfsetstrokecolor{currentstroke}%
\pgfsetstrokeopacity{0.200000}%
\pgfsetdash{}{0pt}%
\pgfpathmoveto{\pgfqpoint{2.373781in}{3.583374in}}%
\pgfpathlineto{\pgfqpoint{5.359152in}{2.808495in}}%
\pgfusepath{stroke}%
\end{pgfscope}%
\begin{pgfscope}%
\pgfpathrectangle{\pgfqpoint{0.526127in}{0.331635in}}{\pgfqpoint{9.300000in}{7.700000in}}%
\pgfusepath{clip}%
\pgfsetrectcap%
\pgfsetroundjoin%
\pgfsetlinewidth{1.505625pt}%
\definecolor{currentstroke}{rgb}{0.815686,0.733333,1.000000}%
\pgfsetstrokecolor{currentstroke}%
\pgfsetstrokeopacity{0.200000}%
\pgfsetdash{}{0pt}%
\pgfpathmoveto{\pgfqpoint{7.393947in}{1.678995in}}%
\pgfpathlineto{\pgfqpoint{5.359152in}{2.808495in}}%
\pgfusepath{stroke}%
\end{pgfscope}%
\begin{pgfscope}%
\pgfpathrectangle{\pgfqpoint{0.526127in}{0.331635in}}{\pgfqpoint{9.300000in}{7.700000in}}%
\pgfusepath{clip}%
\pgfsetrectcap%
\pgfsetroundjoin%
\pgfsetlinewidth{1.505625pt}%
\definecolor{currentstroke}{rgb}{0.815686,0.733333,1.000000}%
\pgfsetstrokecolor{currentstroke}%
\pgfsetstrokeopacity{0.200000}%
\pgfsetdash{}{0pt}%
\pgfpathmoveto{\pgfqpoint{7.162597in}{2.858193in}}%
\pgfpathlineto{\pgfqpoint{5.359152in}{2.808495in}}%
\pgfusepath{stroke}%
\end{pgfscope}%
\begin{pgfscope}%
\pgfpathrectangle{\pgfqpoint{0.526127in}{0.331635in}}{\pgfqpoint{9.300000in}{7.700000in}}%
\pgfusepath{clip}%
\pgfsetrectcap%
\pgfsetroundjoin%
\pgfsetlinewidth{1.505625pt}%
\definecolor{currentstroke}{rgb}{0.815686,0.733333,1.000000}%
\pgfsetstrokecolor{currentstroke}%
\pgfsetstrokeopacity{0.200000}%
\pgfsetdash{}{0pt}%
\pgfpathmoveto{\pgfqpoint{6.664367in}{4.897256in}}%
\pgfpathlineto{\pgfqpoint{5.359152in}{2.808495in}}%
\pgfusepath{stroke}%
\end{pgfscope}%
\begin{pgfscope}%
\pgfpathrectangle{\pgfqpoint{0.526127in}{0.331635in}}{\pgfqpoint{9.300000in}{7.700000in}}%
\pgfusepath{clip}%
\pgfsetrectcap%
\pgfsetroundjoin%
\pgfsetlinewidth{1.505625pt}%
\definecolor{currentstroke}{rgb}{0.815686,0.733333,1.000000}%
\pgfsetstrokecolor{currentstroke}%
\pgfsetstrokeopacity{0.200000}%
\pgfsetdash{}{0pt}%
\pgfpathmoveto{\pgfqpoint{5.711097in}{2.503025in}}%
\pgfpathlineto{\pgfqpoint{5.359152in}{2.808495in}}%
\pgfusepath{stroke}%
\end{pgfscope}%
\begin{pgfscope}%
\pgfpathrectangle{\pgfqpoint{0.526127in}{0.331635in}}{\pgfqpoint{9.300000in}{7.700000in}}%
\pgfusepath{clip}%
\pgfsetrectcap%
\pgfsetroundjoin%
\pgfsetlinewidth{1.505625pt}%
\definecolor{currentstroke}{rgb}{0.815686,0.733333,1.000000}%
\pgfsetstrokecolor{currentstroke}%
\pgfsetstrokeopacity{0.200000}%
\pgfsetdash{}{0pt}%
\pgfpathmoveto{\pgfqpoint{6.805509in}{2.216717in}}%
\pgfpathlineto{\pgfqpoint{5.359152in}{2.808495in}}%
\pgfusepath{stroke}%
\end{pgfscope}%
\begin{pgfscope}%
\pgfpathrectangle{\pgfqpoint{0.526127in}{0.331635in}}{\pgfqpoint{9.300000in}{7.700000in}}%
\pgfusepath{clip}%
\pgfsetrectcap%
\pgfsetroundjoin%
\pgfsetlinewidth{1.505625pt}%
\definecolor{currentstroke}{rgb}{0.815686,0.733333,1.000000}%
\pgfsetstrokecolor{currentstroke}%
\pgfsetstrokeopacity{0.200000}%
\pgfsetdash{}{0pt}%
\pgfpathmoveto{\pgfqpoint{3.875538in}{2.306540in}}%
\pgfpathlineto{\pgfqpoint{5.359152in}{2.808495in}}%
\pgfusepath{stroke}%
\end{pgfscope}%
\begin{pgfscope}%
\pgfpathrectangle{\pgfqpoint{0.526127in}{0.331635in}}{\pgfqpoint{9.300000in}{7.700000in}}%
\pgfusepath{clip}%
\pgfsetrectcap%
\pgfsetroundjoin%
\pgfsetlinewidth{1.505625pt}%
\definecolor{currentstroke}{rgb}{0.815686,0.733333,1.000000}%
\pgfsetstrokecolor{currentstroke}%
\pgfsetstrokeopacity{0.200000}%
\pgfsetdash{}{0pt}%
\pgfpathmoveto{\pgfqpoint{1.970636in}{4.883382in}}%
\pgfpathlineto{\pgfqpoint{5.359152in}{2.808495in}}%
\pgfusepath{stroke}%
\end{pgfscope}%
\begin{pgfscope}%
\pgfpathrectangle{\pgfqpoint{0.526127in}{0.331635in}}{\pgfqpoint{9.300000in}{7.700000in}}%
\pgfusepath{clip}%
\pgfsetrectcap%
\pgfsetroundjoin%
\pgfsetlinewidth{1.505625pt}%
\definecolor{currentstroke}{rgb}{0.815686,0.733333,1.000000}%
\pgfsetstrokecolor{currentstroke}%
\pgfsetstrokeopacity{0.200000}%
\pgfsetdash{}{0pt}%
\pgfpathmoveto{\pgfqpoint{3.343614in}{2.667179in}}%
\pgfpathlineto{\pgfqpoint{5.359152in}{2.808495in}}%
\pgfusepath{stroke}%
\end{pgfscope}%
\begin{pgfscope}%
\pgfpathrectangle{\pgfqpoint{0.526127in}{0.331635in}}{\pgfqpoint{9.300000in}{7.700000in}}%
\pgfusepath{clip}%
\pgfsetrectcap%
\pgfsetroundjoin%
\pgfsetlinewidth{1.505625pt}%
\definecolor{currentstroke}{rgb}{0.815686,0.733333,1.000000}%
\pgfsetstrokecolor{currentstroke}%
\pgfsetstrokeopacity{0.200000}%
\pgfsetdash{}{0pt}%
\pgfpathmoveto{\pgfqpoint{6.735659in}{3.857474in}}%
\pgfpathlineto{\pgfqpoint{5.359152in}{2.808495in}}%
\pgfusepath{stroke}%
\end{pgfscope}%
\begin{pgfscope}%
\pgfpathrectangle{\pgfqpoint{0.526127in}{0.331635in}}{\pgfqpoint{9.300000in}{7.700000in}}%
\pgfusepath{clip}%
\pgfsetrectcap%
\pgfsetroundjoin%
\pgfsetlinewidth{1.505625pt}%
\definecolor{currentstroke}{rgb}{0.815686,0.733333,1.000000}%
\pgfsetstrokecolor{currentstroke}%
\pgfsetstrokeopacity{0.200000}%
\pgfsetdash{}{0pt}%
\pgfpathmoveto{\pgfqpoint{2.507937in}{5.115912in}}%
\pgfpathlineto{\pgfqpoint{5.359152in}{2.808495in}}%
\pgfusepath{stroke}%
\end{pgfscope}%
\begin{pgfscope}%
\pgfpathrectangle{\pgfqpoint{0.526127in}{0.331635in}}{\pgfqpoint{9.300000in}{7.700000in}}%
\pgfusepath{clip}%
\pgfsetrectcap%
\pgfsetroundjoin%
\pgfsetlinewidth{1.505625pt}%
\definecolor{currentstroke}{rgb}{0.815686,0.733333,1.000000}%
\pgfsetstrokecolor{currentstroke}%
\pgfsetstrokeopacity{0.200000}%
\pgfsetdash{}{0pt}%
\pgfpathmoveto{\pgfqpoint{8.582505in}{3.957628in}}%
\pgfpathlineto{\pgfqpoint{5.359152in}{2.808495in}}%
\pgfusepath{stroke}%
\end{pgfscope}%
\begin{pgfscope}%
\pgfpathrectangle{\pgfqpoint{0.526127in}{0.331635in}}{\pgfqpoint{9.300000in}{7.700000in}}%
\pgfusepath{clip}%
\pgfsetrectcap%
\pgfsetroundjoin%
\pgfsetlinewidth{1.505625pt}%
\definecolor{currentstroke}{rgb}{0.815686,0.733333,1.000000}%
\pgfsetstrokecolor{currentstroke}%
\pgfsetstrokeopacity{0.200000}%
\pgfsetdash{}{0pt}%
\pgfpathmoveto{\pgfqpoint{5.435254in}{1.782781in}}%
\pgfpathlineto{\pgfqpoint{5.359152in}{2.808495in}}%
\pgfusepath{stroke}%
\end{pgfscope}%
\begin{pgfscope}%
\pgfpathrectangle{\pgfqpoint{0.526127in}{0.331635in}}{\pgfqpoint{9.300000in}{7.700000in}}%
\pgfusepath{clip}%
\pgfsetrectcap%
\pgfsetroundjoin%
\pgfsetlinewidth{1.505625pt}%
\definecolor{currentstroke}{rgb}{0.815686,0.733333,1.000000}%
\pgfsetstrokecolor{currentstroke}%
\pgfsetstrokeopacity{0.200000}%
\pgfsetdash{}{0pt}%
\pgfpathmoveto{\pgfqpoint{5.661250in}{4.669033in}}%
\pgfpathlineto{\pgfqpoint{5.359152in}{2.808495in}}%
\pgfusepath{stroke}%
\end{pgfscope}%
\begin{pgfscope}%
\pgfpathrectangle{\pgfqpoint{0.526127in}{0.331635in}}{\pgfqpoint{9.300000in}{7.700000in}}%
\pgfusepath{clip}%
\pgfsetrectcap%
\pgfsetroundjoin%
\pgfsetlinewidth{1.505625pt}%
\definecolor{currentstroke}{rgb}{0.815686,0.733333,1.000000}%
\pgfsetstrokecolor{currentstroke}%
\pgfsetstrokeopacity{0.200000}%
\pgfsetdash{}{0pt}%
\pgfpathmoveto{\pgfqpoint{2.380624in}{2.704215in}}%
\pgfpathlineto{\pgfqpoint{5.359152in}{2.808495in}}%
\pgfusepath{stroke}%
\end{pgfscope}%
\begin{pgfscope}%
\pgfpathrectangle{\pgfqpoint{0.526127in}{0.331635in}}{\pgfqpoint{9.300000in}{7.700000in}}%
\pgfusepath{clip}%
\pgfsetrectcap%
\pgfsetroundjoin%
\pgfsetlinewidth{1.505625pt}%
\definecolor{currentstroke}{rgb}{0.815686,0.733333,1.000000}%
\pgfsetstrokecolor{currentstroke}%
\pgfsetstrokeopacity{0.200000}%
\pgfsetdash{}{0pt}%
\pgfpathmoveto{\pgfqpoint{3.490954in}{3.356487in}}%
\pgfpathlineto{\pgfqpoint{5.359152in}{2.808495in}}%
\pgfusepath{stroke}%
\end{pgfscope}%
\begin{pgfscope}%
\pgfpathrectangle{\pgfqpoint{0.526127in}{0.331635in}}{\pgfqpoint{9.300000in}{7.700000in}}%
\pgfusepath{clip}%
\pgfsetrectcap%
\pgfsetroundjoin%
\pgfsetlinewidth{1.505625pt}%
\definecolor{currentstroke}{rgb}{0.815686,0.733333,1.000000}%
\pgfsetstrokecolor{currentstroke}%
\pgfsetstrokeopacity{0.200000}%
\pgfsetdash{}{0pt}%
\pgfpathmoveto{\pgfqpoint{4.801136in}{0.774084in}}%
\pgfpathlineto{\pgfqpoint{5.359152in}{2.808495in}}%
\pgfusepath{stroke}%
\end{pgfscope}%
\begin{pgfscope}%
\pgfpathrectangle{\pgfqpoint{0.526127in}{0.331635in}}{\pgfqpoint{9.300000in}{7.700000in}}%
\pgfusepath{clip}%
\pgfsetrectcap%
\pgfsetroundjoin%
\pgfsetlinewidth{1.505625pt}%
\definecolor{currentstroke}{rgb}{0.815686,0.733333,1.000000}%
\pgfsetstrokecolor{currentstroke}%
\pgfsetstrokeopacity{0.200000}%
\pgfsetdash{}{0pt}%
\pgfpathmoveto{\pgfqpoint{7.283490in}{1.720161in}}%
\pgfpathlineto{\pgfqpoint{5.359152in}{2.808495in}}%
\pgfusepath{stroke}%
\end{pgfscope}%
\begin{pgfscope}%
\pgfpathrectangle{\pgfqpoint{0.526127in}{0.331635in}}{\pgfqpoint{9.300000in}{7.700000in}}%
\pgfusepath{clip}%
\pgfsetrectcap%
\pgfsetroundjoin%
\pgfsetlinewidth{1.505625pt}%
\definecolor{currentstroke}{rgb}{0.815686,0.733333,1.000000}%
\pgfsetstrokecolor{currentstroke}%
\pgfsetstrokeopacity{0.200000}%
\pgfsetdash{}{0pt}%
\pgfpathmoveto{\pgfqpoint{4.932772in}{1.241893in}}%
\pgfpathlineto{\pgfqpoint{5.359152in}{2.808495in}}%
\pgfusepath{stroke}%
\end{pgfscope}%
\begin{pgfscope}%
\pgfpathrectangle{\pgfqpoint{0.526127in}{0.331635in}}{\pgfqpoint{9.300000in}{7.700000in}}%
\pgfusepath{clip}%
\pgfsetrectcap%
\pgfsetroundjoin%
\pgfsetlinewidth{1.505625pt}%
\definecolor{currentstroke}{rgb}{0.815686,0.733333,1.000000}%
\pgfsetstrokecolor{currentstroke}%
\pgfsetstrokeopacity{0.200000}%
\pgfsetdash{}{0pt}%
\pgfpathmoveto{\pgfqpoint{7.494242in}{1.691723in}}%
\pgfpathlineto{\pgfqpoint{5.359152in}{2.808495in}}%
\pgfusepath{stroke}%
\end{pgfscope}%
\begin{pgfscope}%
\pgfpathrectangle{\pgfqpoint{0.526127in}{0.331635in}}{\pgfqpoint{9.300000in}{7.700000in}}%
\pgfusepath{clip}%
\pgfsetrectcap%
\pgfsetroundjoin%
\pgfsetlinewidth{1.505625pt}%
\definecolor{currentstroke}{rgb}{0.815686,0.733333,1.000000}%
\pgfsetstrokecolor{currentstroke}%
\pgfsetstrokeopacity{0.200000}%
\pgfsetdash{}{0pt}%
\pgfpathmoveto{\pgfqpoint{8.328870in}{2.327023in}}%
\pgfpathlineto{\pgfqpoint{5.359152in}{2.808495in}}%
\pgfusepath{stroke}%
\end{pgfscope}%
\begin{pgfscope}%
\pgfpathrectangle{\pgfqpoint{0.526127in}{0.331635in}}{\pgfqpoint{9.300000in}{7.700000in}}%
\pgfusepath{clip}%
\pgfsetrectcap%
\pgfsetroundjoin%
\pgfsetlinewidth{1.505625pt}%
\definecolor{currentstroke}{rgb}{0.815686,0.733333,1.000000}%
\pgfsetstrokecolor{currentstroke}%
\pgfsetstrokeopacity{0.200000}%
\pgfsetdash{}{0pt}%
\pgfpathmoveto{\pgfqpoint{2.658761in}{4.417192in}}%
\pgfpathlineto{\pgfqpoint{5.359152in}{2.808495in}}%
\pgfusepath{stroke}%
\end{pgfscope}%
\begin{pgfscope}%
\pgfpathrectangle{\pgfqpoint{0.526127in}{0.331635in}}{\pgfqpoint{9.300000in}{7.700000in}}%
\pgfusepath{clip}%
\pgfsetrectcap%
\pgfsetroundjoin%
\pgfsetlinewidth{1.505625pt}%
\definecolor{currentstroke}{rgb}{0.870588,0.733333,0.607843}%
\pgfsetstrokecolor{currentstroke}%
\pgfsetstrokeopacity{0.200000}%
\pgfsetdash{}{0pt}%
\pgfpathmoveto{\pgfqpoint{3.293241in}{4.314612in}}%
\pgfpathlineto{\pgfqpoint{4.298154in}{2.772028in}}%
\pgfusepath{stroke}%
\end{pgfscope}%
\begin{pgfscope}%
\pgfpathrectangle{\pgfqpoint{0.526127in}{0.331635in}}{\pgfqpoint{9.300000in}{7.700000in}}%
\pgfusepath{clip}%
\pgfsetrectcap%
\pgfsetroundjoin%
\pgfsetlinewidth{1.505625pt}%
\definecolor{currentstroke}{rgb}{0.870588,0.733333,0.607843}%
\pgfsetstrokecolor{currentstroke}%
\pgfsetstrokeopacity{0.200000}%
\pgfsetdash{}{0pt}%
\pgfpathmoveto{\pgfqpoint{4.449865in}{2.634339in}}%
\pgfpathlineto{\pgfqpoint{4.298154in}{2.772028in}}%
\pgfusepath{stroke}%
\end{pgfscope}%
\begin{pgfscope}%
\pgfpathrectangle{\pgfqpoint{0.526127in}{0.331635in}}{\pgfqpoint{9.300000in}{7.700000in}}%
\pgfusepath{clip}%
\pgfsetrectcap%
\pgfsetroundjoin%
\pgfsetlinewidth{1.505625pt}%
\definecolor{currentstroke}{rgb}{0.870588,0.733333,0.607843}%
\pgfsetstrokecolor{currentstroke}%
\pgfsetstrokeopacity{0.200000}%
\pgfsetdash{}{0pt}%
\pgfpathmoveto{\pgfqpoint{5.505397in}{0.975429in}}%
\pgfpathlineto{\pgfqpoint{4.298154in}{2.772028in}}%
\pgfusepath{stroke}%
\end{pgfscope}%
\begin{pgfscope}%
\pgfpathrectangle{\pgfqpoint{0.526127in}{0.331635in}}{\pgfqpoint{9.300000in}{7.700000in}}%
\pgfusepath{clip}%
\pgfsetrectcap%
\pgfsetroundjoin%
\pgfsetlinewidth{1.505625pt}%
\definecolor{currentstroke}{rgb}{0.870588,0.733333,0.607843}%
\pgfsetstrokecolor{currentstroke}%
\pgfsetstrokeopacity{0.200000}%
\pgfsetdash{}{0pt}%
\pgfpathmoveto{\pgfqpoint{3.975782in}{2.801486in}}%
\pgfpathlineto{\pgfqpoint{4.298154in}{2.772028in}}%
\pgfusepath{stroke}%
\end{pgfscope}%
\begin{pgfscope}%
\pgfpathrectangle{\pgfqpoint{0.526127in}{0.331635in}}{\pgfqpoint{9.300000in}{7.700000in}}%
\pgfusepath{clip}%
\pgfsetrectcap%
\pgfsetroundjoin%
\pgfsetlinewidth{1.505625pt}%
\definecolor{currentstroke}{rgb}{0.870588,0.733333,0.607843}%
\pgfsetstrokecolor{currentstroke}%
\pgfsetstrokeopacity{0.200000}%
\pgfsetdash{}{0pt}%
\pgfpathmoveto{\pgfqpoint{5.150340in}{3.148874in}}%
\pgfpathlineto{\pgfqpoint{4.298154in}{2.772028in}}%
\pgfusepath{stroke}%
\end{pgfscope}%
\begin{pgfscope}%
\pgfpathrectangle{\pgfqpoint{0.526127in}{0.331635in}}{\pgfqpoint{9.300000in}{7.700000in}}%
\pgfusepath{clip}%
\pgfsetrectcap%
\pgfsetroundjoin%
\pgfsetlinewidth{1.505625pt}%
\definecolor{currentstroke}{rgb}{0.870588,0.733333,0.607843}%
\pgfsetstrokecolor{currentstroke}%
\pgfsetstrokeopacity{0.200000}%
\pgfsetdash{}{0pt}%
\pgfpathmoveto{\pgfqpoint{6.303025in}{4.257506in}}%
\pgfpathlineto{\pgfqpoint{4.298154in}{2.772028in}}%
\pgfusepath{stroke}%
\end{pgfscope}%
\begin{pgfscope}%
\pgfpathrectangle{\pgfqpoint{0.526127in}{0.331635in}}{\pgfqpoint{9.300000in}{7.700000in}}%
\pgfusepath{clip}%
\pgfsetrectcap%
\pgfsetroundjoin%
\pgfsetlinewidth{1.505625pt}%
\definecolor{currentstroke}{rgb}{0.870588,0.733333,0.607843}%
\pgfsetstrokecolor{currentstroke}%
\pgfsetstrokeopacity{0.200000}%
\pgfsetdash{}{0pt}%
\pgfpathmoveto{\pgfqpoint{3.101691in}{7.346220in}}%
\pgfpathlineto{\pgfqpoint{4.298154in}{2.772028in}}%
\pgfusepath{stroke}%
\end{pgfscope}%
\begin{pgfscope}%
\pgfpathrectangle{\pgfqpoint{0.526127in}{0.331635in}}{\pgfqpoint{9.300000in}{7.700000in}}%
\pgfusepath{clip}%
\pgfsetrectcap%
\pgfsetroundjoin%
\pgfsetlinewidth{1.505625pt}%
\definecolor{currentstroke}{rgb}{0.870588,0.733333,0.607843}%
\pgfsetstrokecolor{currentstroke}%
\pgfsetstrokeopacity{0.200000}%
\pgfsetdash{}{0pt}%
\pgfpathmoveto{\pgfqpoint{6.025455in}{1.547819in}}%
\pgfpathlineto{\pgfqpoint{4.298154in}{2.772028in}}%
\pgfusepath{stroke}%
\end{pgfscope}%
\begin{pgfscope}%
\pgfpathrectangle{\pgfqpoint{0.526127in}{0.331635in}}{\pgfqpoint{9.300000in}{7.700000in}}%
\pgfusepath{clip}%
\pgfsetrectcap%
\pgfsetroundjoin%
\pgfsetlinewidth{1.505625pt}%
\definecolor{currentstroke}{rgb}{0.870588,0.733333,0.607843}%
\pgfsetstrokecolor{currentstroke}%
\pgfsetstrokeopacity{0.200000}%
\pgfsetdash{}{0pt}%
\pgfpathmoveto{\pgfqpoint{2.005207in}{1.996477in}}%
\pgfpathlineto{\pgfqpoint{4.298154in}{2.772028in}}%
\pgfusepath{stroke}%
\end{pgfscope}%
\begin{pgfscope}%
\pgfpathrectangle{\pgfqpoint{0.526127in}{0.331635in}}{\pgfqpoint{9.300000in}{7.700000in}}%
\pgfusepath{clip}%
\pgfsetrectcap%
\pgfsetroundjoin%
\pgfsetlinewidth{1.505625pt}%
\definecolor{currentstroke}{rgb}{0.870588,0.733333,0.607843}%
\pgfsetstrokecolor{currentstroke}%
\pgfsetstrokeopacity{0.200000}%
\pgfsetdash{}{0pt}%
\pgfpathmoveto{\pgfqpoint{0.948854in}{3.300176in}}%
\pgfpathlineto{\pgfqpoint{4.298154in}{2.772028in}}%
\pgfusepath{stroke}%
\end{pgfscope}%
\begin{pgfscope}%
\pgfpathrectangle{\pgfqpoint{0.526127in}{0.331635in}}{\pgfqpoint{9.300000in}{7.700000in}}%
\pgfusepath{clip}%
\pgfsetrectcap%
\pgfsetroundjoin%
\pgfsetlinewidth{1.505625pt}%
\definecolor{currentstroke}{rgb}{0.870588,0.733333,0.607843}%
\pgfsetstrokecolor{currentstroke}%
\pgfsetstrokeopacity{0.200000}%
\pgfsetdash{}{0pt}%
\pgfpathmoveto{\pgfqpoint{2.231357in}{2.159367in}}%
\pgfpathlineto{\pgfqpoint{4.298154in}{2.772028in}}%
\pgfusepath{stroke}%
\end{pgfscope}%
\begin{pgfscope}%
\pgfpathrectangle{\pgfqpoint{0.526127in}{0.331635in}}{\pgfqpoint{9.300000in}{7.700000in}}%
\pgfusepath{clip}%
\pgfsetrectcap%
\pgfsetroundjoin%
\pgfsetlinewidth{1.505625pt}%
\definecolor{currentstroke}{rgb}{0.870588,0.733333,0.607843}%
\pgfsetstrokecolor{currentstroke}%
\pgfsetstrokeopacity{0.200000}%
\pgfsetdash{}{0pt}%
\pgfpathmoveto{\pgfqpoint{5.273237in}{2.725307in}}%
\pgfpathlineto{\pgfqpoint{4.298154in}{2.772028in}}%
\pgfusepath{stroke}%
\end{pgfscope}%
\begin{pgfscope}%
\pgfpathrectangle{\pgfqpoint{0.526127in}{0.331635in}}{\pgfqpoint{9.300000in}{7.700000in}}%
\pgfusepath{clip}%
\pgfsetrectcap%
\pgfsetroundjoin%
\pgfsetlinewidth{1.505625pt}%
\definecolor{currentstroke}{rgb}{0.870588,0.733333,0.607843}%
\pgfsetstrokecolor{currentstroke}%
\pgfsetstrokeopacity{0.200000}%
\pgfsetdash{}{0pt}%
\pgfpathmoveto{\pgfqpoint{4.721664in}{2.320491in}}%
\pgfpathlineto{\pgfqpoint{4.298154in}{2.772028in}}%
\pgfusepath{stroke}%
\end{pgfscope}%
\begin{pgfscope}%
\pgfpathrectangle{\pgfqpoint{0.526127in}{0.331635in}}{\pgfqpoint{9.300000in}{7.700000in}}%
\pgfusepath{clip}%
\pgfsetrectcap%
\pgfsetroundjoin%
\pgfsetlinewidth{1.505625pt}%
\definecolor{currentstroke}{rgb}{0.870588,0.733333,0.607843}%
\pgfsetstrokecolor{currentstroke}%
\pgfsetstrokeopacity{0.200000}%
\pgfsetdash{}{0pt}%
\pgfpathmoveto{\pgfqpoint{5.294260in}{1.697890in}}%
\pgfpathlineto{\pgfqpoint{4.298154in}{2.772028in}}%
\pgfusepath{stroke}%
\end{pgfscope}%
\begin{pgfscope}%
\pgfpathrectangle{\pgfqpoint{0.526127in}{0.331635in}}{\pgfqpoint{9.300000in}{7.700000in}}%
\pgfusepath{clip}%
\pgfsetrectcap%
\pgfsetroundjoin%
\pgfsetlinewidth{1.505625pt}%
\definecolor{currentstroke}{rgb}{0.870588,0.733333,0.607843}%
\pgfsetstrokecolor{currentstroke}%
\pgfsetstrokeopacity{0.200000}%
\pgfsetdash{}{0pt}%
\pgfpathmoveto{\pgfqpoint{3.016007in}{6.490305in}}%
\pgfpathlineto{\pgfqpoint{4.298154in}{2.772028in}}%
\pgfusepath{stroke}%
\end{pgfscope}%
\begin{pgfscope}%
\pgfpathrectangle{\pgfqpoint{0.526127in}{0.331635in}}{\pgfqpoint{9.300000in}{7.700000in}}%
\pgfusepath{clip}%
\pgfsetrectcap%
\pgfsetroundjoin%
\pgfsetlinewidth{1.505625pt}%
\definecolor{currentstroke}{rgb}{0.870588,0.733333,0.607843}%
\pgfsetstrokecolor{currentstroke}%
\pgfsetstrokeopacity{0.200000}%
\pgfsetdash{}{0pt}%
\pgfpathmoveto{\pgfqpoint{0.975724in}{3.313258in}}%
\pgfpathlineto{\pgfqpoint{4.298154in}{2.772028in}}%
\pgfusepath{stroke}%
\end{pgfscope}%
\begin{pgfscope}%
\pgfpathrectangle{\pgfqpoint{0.526127in}{0.331635in}}{\pgfqpoint{9.300000in}{7.700000in}}%
\pgfusepath{clip}%
\pgfsetrectcap%
\pgfsetroundjoin%
\pgfsetlinewidth{1.505625pt}%
\definecolor{currentstroke}{rgb}{0.870588,0.733333,0.607843}%
\pgfsetstrokecolor{currentstroke}%
\pgfsetstrokeopacity{0.200000}%
\pgfsetdash{}{0pt}%
\pgfpathmoveto{\pgfqpoint{6.443064in}{3.205723in}}%
\pgfpathlineto{\pgfqpoint{4.298154in}{2.772028in}}%
\pgfusepath{stroke}%
\end{pgfscope}%
\begin{pgfscope}%
\pgfpathrectangle{\pgfqpoint{0.526127in}{0.331635in}}{\pgfqpoint{9.300000in}{7.700000in}}%
\pgfusepath{clip}%
\pgfsetrectcap%
\pgfsetroundjoin%
\pgfsetlinewidth{1.505625pt}%
\definecolor{currentstroke}{rgb}{0.870588,0.733333,0.607843}%
\pgfsetstrokecolor{currentstroke}%
\pgfsetstrokeopacity{0.200000}%
\pgfsetdash{}{0pt}%
\pgfpathmoveto{\pgfqpoint{5.818974in}{2.956871in}}%
\pgfpathlineto{\pgfqpoint{4.298154in}{2.772028in}}%
\pgfusepath{stroke}%
\end{pgfscope}%
\begin{pgfscope}%
\pgfpathrectangle{\pgfqpoint{0.526127in}{0.331635in}}{\pgfqpoint{9.300000in}{7.700000in}}%
\pgfusepath{clip}%
\pgfsetrectcap%
\pgfsetroundjoin%
\pgfsetlinewidth{1.505625pt}%
\definecolor{currentstroke}{rgb}{0.870588,0.733333,0.607843}%
\pgfsetstrokecolor{currentstroke}%
\pgfsetstrokeopacity{0.200000}%
\pgfsetdash{}{0pt}%
\pgfpathmoveto{\pgfqpoint{3.186993in}{2.588401in}}%
\pgfpathlineto{\pgfqpoint{4.298154in}{2.772028in}}%
\pgfusepath{stroke}%
\end{pgfscope}%
\begin{pgfscope}%
\pgfpathrectangle{\pgfqpoint{0.526127in}{0.331635in}}{\pgfqpoint{9.300000in}{7.700000in}}%
\pgfusepath{clip}%
\pgfsetrectcap%
\pgfsetroundjoin%
\pgfsetlinewidth{1.505625pt}%
\definecolor{currentstroke}{rgb}{0.870588,0.733333,0.607843}%
\pgfsetstrokecolor{currentstroke}%
\pgfsetstrokeopacity{0.200000}%
\pgfsetdash{}{0pt}%
\pgfpathmoveto{\pgfqpoint{5.009171in}{2.423982in}}%
\pgfpathlineto{\pgfqpoint{4.298154in}{2.772028in}}%
\pgfusepath{stroke}%
\end{pgfscope}%
\begin{pgfscope}%
\pgfpathrectangle{\pgfqpoint{0.526127in}{0.331635in}}{\pgfqpoint{9.300000in}{7.700000in}}%
\pgfusepath{clip}%
\pgfsetrectcap%
\pgfsetroundjoin%
\pgfsetlinewidth{1.505625pt}%
\definecolor{currentstroke}{rgb}{0.870588,0.733333,0.607843}%
\pgfsetstrokecolor{currentstroke}%
\pgfsetstrokeopacity{0.200000}%
\pgfsetdash{}{0pt}%
\pgfpathmoveto{\pgfqpoint{5.648571in}{1.147470in}}%
\pgfpathlineto{\pgfqpoint{4.298154in}{2.772028in}}%
\pgfusepath{stroke}%
\end{pgfscope}%
\begin{pgfscope}%
\pgfpathrectangle{\pgfqpoint{0.526127in}{0.331635in}}{\pgfqpoint{9.300000in}{7.700000in}}%
\pgfusepath{clip}%
\pgfsetrectcap%
\pgfsetroundjoin%
\pgfsetlinewidth{1.505625pt}%
\definecolor{currentstroke}{rgb}{0.870588,0.733333,0.607843}%
\pgfsetstrokecolor{currentstroke}%
\pgfsetstrokeopacity{0.200000}%
\pgfsetdash{}{0pt}%
\pgfpathmoveto{\pgfqpoint{4.947462in}{1.693509in}}%
\pgfpathlineto{\pgfqpoint{4.298154in}{2.772028in}}%
\pgfusepath{stroke}%
\end{pgfscope}%
\begin{pgfscope}%
\pgfpathrectangle{\pgfqpoint{0.526127in}{0.331635in}}{\pgfqpoint{9.300000in}{7.700000in}}%
\pgfusepath{clip}%
\pgfsetrectcap%
\pgfsetroundjoin%
\pgfsetlinewidth{1.505625pt}%
\definecolor{currentstroke}{rgb}{0.870588,0.733333,0.607843}%
\pgfsetstrokecolor{currentstroke}%
\pgfsetstrokeopacity{0.200000}%
\pgfsetdash{}{0pt}%
\pgfpathmoveto{\pgfqpoint{5.178715in}{1.674266in}}%
\pgfpathlineto{\pgfqpoint{4.298154in}{2.772028in}}%
\pgfusepath{stroke}%
\end{pgfscope}%
\begin{pgfscope}%
\pgfpathrectangle{\pgfqpoint{0.526127in}{0.331635in}}{\pgfqpoint{9.300000in}{7.700000in}}%
\pgfusepath{clip}%
\pgfsetrectcap%
\pgfsetroundjoin%
\pgfsetlinewidth{1.505625pt}%
\definecolor{currentstroke}{rgb}{0.870588,0.733333,0.607843}%
\pgfsetstrokecolor{currentstroke}%
\pgfsetstrokeopacity{0.200000}%
\pgfsetdash{}{0pt}%
\pgfpathmoveto{\pgfqpoint{6.316003in}{1.938392in}}%
\pgfpathlineto{\pgfqpoint{4.298154in}{2.772028in}}%
\pgfusepath{stroke}%
\end{pgfscope}%
\begin{pgfscope}%
\pgfpathrectangle{\pgfqpoint{0.526127in}{0.331635in}}{\pgfqpoint{9.300000in}{7.700000in}}%
\pgfusepath{clip}%
\pgfsetrectcap%
\pgfsetroundjoin%
\pgfsetlinewidth{1.505625pt}%
\definecolor{currentstroke}{rgb}{0.870588,0.733333,0.607843}%
\pgfsetstrokecolor{currentstroke}%
\pgfsetstrokeopacity{0.200000}%
\pgfsetdash{}{0pt}%
\pgfpathmoveto{\pgfqpoint{4.799485in}{2.374526in}}%
\pgfpathlineto{\pgfqpoint{4.298154in}{2.772028in}}%
\pgfusepath{stroke}%
\end{pgfscope}%
\begin{pgfscope}%
\pgfpathrectangle{\pgfqpoint{0.526127in}{0.331635in}}{\pgfqpoint{9.300000in}{7.700000in}}%
\pgfusepath{clip}%
\pgfsetrectcap%
\pgfsetroundjoin%
\pgfsetlinewidth{1.505625pt}%
\definecolor{currentstroke}{rgb}{0.870588,0.733333,0.607843}%
\pgfsetstrokecolor{currentstroke}%
\pgfsetstrokeopacity{0.200000}%
\pgfsetdash{}{0pt}%
\pgfpathmoveto{\pgfqpoint{3.925973in}{2.867687in}}%
\pgfpathlineto{\pgfqpoint{4.298154in}{2.772028in}}%
\pgfusepath{stroke}%
\end{pgfscope}%
\begin{pgfscope}%
\pgfpathrectangle{\pgfqpoint{0.526127in}{0.331635in}}{\pgfqpoint{9.300000in}{7.700000in}}%
\pgfusepath{clip}%
\pgfsetrectcap%
\pgfsetroundjoin%
\pgfsetlinewidth{1.505625pt}%
\definecolor{currentstroke}{rgb}{0.870588,0.733333,0.607843}%
\pgfsetstrokecolor{currentstroke}%
\pgfsetstrokeopacity{0.200000}%
\pgfsetdash{}{0pt}%
\pgfpathmoveto{\pgfqpoint{2.545411in}{1.795358in}}%
\pgfpathlineto{\pgfqpoint{4.298154in}{2.772028in}}%
\pgfusepath{stroke}%
\end{pgfscope}%
\begin{pgfscope}%
\pgfpathrectangle{\pgfqpoint{0.526127in}{0.331635in}}{\pgfqpoint{9.300000in}{7.700000in}}%
\pgfusepath{clip}%
\pgfsetrectcap%
\pgfsetroundjoin%
\pgfsetlinewidth{1.505625pt}%
\definecolor{currentstroke}{rgb}{0.870588,0.733333,0.607843}%
\pgfsetstrokecolor{currentstroke}%
\pgfsetstrokeopacity{0.200000}%
\pgfsetdash{}{0pt}%
\pgfpathmoveto{\pgfqpoint{4.257376in}{1.921035in}}%
\pgfpathlineto{\pgfqpoint{4.298154in}{2.772028in}}%
\pgfusepath{stroke}%
\end{pgfscope}%
\begin{pgfscope}%
\pgfpathrectangle{\pgfqpoint{0.526127in}{0.331635in}}{\pgfqpoint{9.300000in}{7.700000in}}%
\pgfusepath{clip}%
\pgfsetrectcap%
\pgfsetroundjoin%
\pgfsetlinewidth{1.505625pt}%
\definecolor{currentstroke}{rgb}{0.980392,0.690196,0.894118}%
\pgfsetstrokecolor{currentstroke}%
\pgfsetstrokeopacity{0.800000}%
\pgfsetdash{}{0pt}%
\pgfpathmoveto{\pgfqpoint{8.127617in}{4.829463in}}%
\pgfpathlineto{\pgfqpoint{5.989578in}{5.077315in}}%
\pgfusepath{stroke}%
\end{pgfscope}%
\begin{pgfscope}%
\pgfpathrectangle{\pgfqpoint{0.526127in}{0.331635in}}{\pgfqpoint{9.300000in}{7.700000in}}%
\pgfusepath{clip}%
\pgfsetrectcap%
\pgfsetroundjoin%
\pgfsetlinewidth{1.505625pt}%
\definecolor{currentstroke}{rgb}{0.980392,0.690196,0.894118}%
\pgfsetstrokecolor{currentstroke}%
\pgfsetstrokeopacity{0.800000}%
\pgfsetdash{}{0pt}%
\pgfpathmoveto{\pgfqpoint{5.250982in}{7.334764in}}%
\pgfpathlineto{\pgfqpoint{5.989578in}{5.077315in}}%
\pgfusepath{stroke}%
\end{pgfscope}%
\begin{pgfscope}%
\pgfpathrectangle{\pgfqpoint{0.526127in}{0.331635in}}{\pgfqpoint{9.300000in}{7.700000in}}%
\pgfusepath{clip}%
\pgfsetrectcap%
\pgfsetroundjoin%
\pgfsetlinewidth{1.505625pt}%
\definecolor{currentstroke}{rgb}{0.980392,0.690196,0.894118}%
\pgfsetstrokecolor{currentstroke}%
\pgfsetstrokeopacity{0.800000}%
\pgfsetdash{}{0pt}%
\pgfpathmoveto{\pgfqpoint{4.648550in}{2.936165in}}%
\pgfpathlineto{\pgfqpoint{5.989578in}{5.077315in}}%
\pgfusepath{stroke}%
\end{pgfscope}%
\begin{pgfscope}%
\pgfpathrectangle{\pgfqpoint{0.526127in}{0.331635in}}{\pgfqpoint{9.300000in}{7.700000in}}%
\pgfusepath{clip}%
\pgfsetrectcap%
\pgfsetroundjoin%
\pgfsetlinewidth{1.505625pt}%
\definecolor{currentstroke}{rgb}{0.980392,0.690196,0.894118}%
\pgfsetstrokecolor{currentstroke}%
\pgfsetstrokeopacity{0.800000}%
\pgfsetdash{}{0pt}%
\pgfpathmoveto{\pgfqpoint{6.518469in}{6.123645in}}%
\pgfpathlineto{\pgfqpoint{5.989578in}{5.077315in}}%
\pgfusepath{stroke}%
\end{pgfscope}%
\begin{pgfscope}%
\pgfpathrectangle{\pgfqpoint{0.526127in}{0.331635in}}{\pgfqpoint{9.300000in}{7.700000in}}%
\pgfusepath{clip}%
\pgfsetrectcap%
\pgfsetroundjoin%
\pgfsetlinewidth{1.505625pt}%
\definecolor{currentstroke}{rgb}{0.980392,0.690196,0.894118}%
\pgfsetstrokecolor{currentstroke}%
\pgfsetstrokeopacity{0.800000}%
\pgfsetdash{}{0pt}%
\pgfpathmoveto{\pgfqpoint{8.083074in}{4.741037in}}%
\pgfpathlineto{\pgfqpoint{5.989578in}{5.077315in}}%
\pgfusepath{stroke}%
\end{pgfscope}%
\begin{pgfscope}%
\pgfpathrectangle{\pgfqpoint{0.526127in}{0.331635in}}{\pgfqpoint{9.300000in}{7.700000in}}%
\pgfusepath{clip}%
\pgfsetrectcap%
\pgfsetroundjoin%
\pgfsetlinewidth{1.505625pt}%
\definecolor{currentstroke}{rgb}{0.980392,0.690196,0.894118}%
\pgfsetstrokecolor{currentstroke}%
\pgfsetstrokeopacity{0.800000}%
\pgfsetdash{}{0pt}%
\pgfpathmoveto{\pgfqpoint{7.058761in}{4.131981in}}%
\pgfpathlineto{\pgfqpoint{5.989578in}{5.077315in}}%
\pgfusepath{stroke}%
\end{pgfscope}%
\begin{pgfscope}%
\pgfpathrectangle{\pgfqpoint{0.526127in}{0.331635in}}{\pgfqpoint{9.300000in}{7.700000in}}%
\pgfusepath{clip}%
\pgfsetrectcap%
\pgfsetroundjoin%
\pgfsetlinewidth{1.505625pt}%
\definecolor{currentstroke}{rgb}{0.980392,0.690196,0.894118}%
\pgfsetstrokecolor{currentstroke}%
\pgfsetstrokeopacity{0.800000}%
\pgfsetdash{}{0pt}%
\pgfpathmoveto{\pgfqpoint{8.288664in}{4.881967in}}%
\pgfpathlineto{\pgfqpoint{5.989578in}{5.077315in}}%
\pgfusepath{stroke}%
\end{pgfscope}%
\begin{pgfscope}%
\pgfpathrectangle{\pgfqpoint{0.526127in}{0.331635in}}{\pgfqpoint{9.300000in}{7.700000in}}%
\pgfusepath{clip}%
\pgfsetrectcap%
\pgfsetroundjoin%
\pgfsetlinewidth{1.505625pt}%
\definecolor{currentstroke}{rgb}{0.980392,0.690196,0.894118}%
\pgfsetstrokecolor{currentstroke}%
\pgfsetstrokeopacity{0.800000}%
\pgfsetdash{}{0pt}%
\pgfpathmoveto{\pgfqpoint{8.365658in}{4.215602in}}%
\pgfpathlineto{\pgfqpoint{5.989578in}{5.077315in}}%
\pgfusepath{stroke}%
\end{pgfscope}%
\begin{pgfscope}%
\pgfpathrectangle{\pgfqpoint{0.526127in}{0.331635in}}{\pgfqpoint{9.300000in}{7.700000in}}%
\pgfusepath{clip}%
\pgfsetrectcap%
\pgfsetroundjoin%
\pgfsetlinewidth{1.505625pt}%
\definecolor{currentstroke}{rgb}{0.980392,0.690196,0.894118}%
\pgfsetstrokecolor{currentstroke}%
\pgfsetstrokeopacity{0.800000}%
\pgfsetdash{}{0pt}%
\pgfpathmoveto{\pgfqpoint{5.586667in}{6.021828in}}%
\pgfpathlineto{\pgfqpoint{5.989578in}{5.077315in}}%
\pgfusepath{stroke}%
\end{pgfscope}%
\begin{pgfscope}%
\pgfpathrectangle{\pgfqpoint{0.526127in}{0.331635in}}{\pgfqpoint{9.300000in}{7.700000in}}%
\pgfusepath{clip}%
\pgfsetrectcap%
\pgfsetroundjoin%
\pgfsetlinewidth{1.505625pt}%
\definecolor{currentstroke}{rgb}{0.980392,0.690196,0.894118}%
\pgfsetstrokecolor{currentstroke}%
\pgfsetstrokeopacity{0.800000}%
\pgfsetdash{}{0pt}%
\pgfpathmoveto{\pgfqpoint{7.411729in}{4.264611in}}%
\pgfpathlineto{\pgfqpoint{5.989578in}{5.077315in}}%
\pgfusepath{stroke}%
\end{pgfscope}%
\begin{pgfscope}%
\pgfpathrectangle{\pgfqpoint{0.526127in}{0.331635in}}{\pgfqpoint{9.300000in}{7.700000in}}%
\pgfusepath{clip}%
\pgfsetrectcap%
\pgfsetroundjoin%
\pgfsetlinewidth{1.505625pt}%
\definecolor{currentstroke}{rgb}{0.980392,0.690196,0.894118}%
\pgfsetstrokecolor{currentstroke}%
\pgfsetstrokeopacity{0.800000}%
\pgfsetdash{}{0pt}%
\pgfpathmoveto{\pgfqpoint{5.552197in}{5.497674in}}%
\pgfpathlineto{\pgfqpoint{5.989578in}{5.077315in}}%
\pgfusepath{stroke}%
\end{pgfscope}%
\begin{pgfscope}%
\pgfpathrectangle{\pgfqpoint{0.526127in}{0.331635in}}{\pgfqpoint{9.300000in}{7.700000in}}%
\pgfusepath{clip}%
\pgfsetrectcap%
\pgfsetroundjoin%
\pgfsetlinewidth{1.505625pt}%
\definecolor{currentstroke}{rgb}{0.980392,0.690196,0.894118}%
\pgfsetstrokecolor{currentstroke}%
\pgfsetstrokeopacity{0.800000}%
\pgfsetdash{}{0pt}%
\pgfpathmoveto{\pgfqpoint{5.112613in}{6.023531in}}%
\pgfpathlineto{\pgfqpoint{5.989578in}{5.077315in}}%
\pgfusepath{stroke}%
\end{pgfscope}%
\begin{pgfscope}%
\pgfpathrectangle{\pgfqpoint{0.526127in}{0.331635in}}{\pgfqpoint{9.300000in}{7.700000in}}%
\pgfusepath{clip}%
\pgfsetrectcap%
\pgfsetroundjoin%
\pgfsetlinewidth{1.505625pt}%
\definecolor{currentstroke}{rgb}{0.980392,0.690196,0.894118}%
\pgfsetstrokecolor{currentstroke}%
\pgfsetstrokeopacity{0.800000}%
\pgfsetdash{}{0pt}%
\pgfpathmoveto{\pgfqpoint{3.785574in}{6.661231in}}%
\pgfpathlineto{\pgfqpoint{5.989578in}{5.077315in}}%
\pgfusepath{stroke}%
\end{pgfscope}%
\begin{pgfscope}%
\pgfpathrectangle{\pgfqpoint{0.526127in}{0.331635in}}{\pgfqpoint{9.300000in}{7.700000in}}%
\pgfusepath{clip}%
\pgfsetrectcap%
\pgfsetroundjoin%
\pgfsetlinewidth{1.505625pt}%
\definecolor{currentstroke}{rgb}{0.980392,0.690196,0.894118}%
\pgfsetstrokecolor{currentstroke}%
\pgfsetstrokeopacity{0.800000}%
\pgfsetdash{}{0pt}%
\pgfpathmoveto{\pgfqpoint{3.136695in}{6.238697in}}%
\pgfpathlineto{\pgfqpoint{5.989578in}{5.077315in}}%
\pgfusepath{stroke}%
\end{pgfscope}%
\begin{pgfscope}%
\pgfpathrectangle{\pgfqpoint{0.526127in}{0.331635in}}{\pgfqpoint{9.300000in}{7.700000in}}%
\pgfusepath{clip}%
\pgfsetrectcap%
\pgfsetroundjoin%
\pgfsetlinewidth{1.505625pt}%
\definecolor{currentstroke}{rgb}{0.980392,0.690196,0.894118}%
\pgfsetstrokecolor{currentstroke}%
\pgfsetstrokeopacity{0.800000}%
\pgfsetdash{}{0pt}%
\pgfpathmoveto{\pgfqpoint{8.338475in}{3.521673in}}%
\pgfpathlineto{\pgfqpoint{5.989578in}{5.077315in}}%
\pgfusepath{stroke}%
\end{pgfscope}%
\begin{pgfscope}%
\pgfpathrectangle{\pgfqpoint{0.526127in}{0.331635in}}{\pgfqpoint{9.300000in}{7.700000in}}%
\pgfusepath{clip}%
\pgfsetrectcap%
\pgfsetroundjoin%
\pgfsetlinewidth{1.505625pt}%
\definecolor{currentstroke}{rgb}{0.980392,0.690196,0.894118}%
\pgfsetstrokecolor{currentstroke}%
\pgfsetstrokeopacity{0.800000}%
\pgfsetdash{}{0pt}%
\pgfpathmoveto{\pgfqpoint{2.725282in}{4.747999in}}%
\pgfpathlineto{\pgfqpoint{5.989578in}{5.077315in}}%
\pgfusepath{stroke}%
\end{pgfscope}%
\begin{pgfscope}%
\pgfpathrectangle{\pgfqpoint{0.526127in}{0.331635in}}{\pgfqpoint{9.300000in}{7.700000in}}%
\pgfusepath{clip}%
\pgfsetrectcap%
\pgfsetroundjoin%
\pgfsetlinewidth{1.505625pt}%
\definecolor{currentstroke}{rgb}{0.980392,0.690196,0.894118}%
\pgfsetstrokecolor{currentstroke}%
\pgfsetstrokeopacity{0.800000}%
\pgfsetdash{}{0pt}%
\pgfpathmoveto{\pgfqpoint{5.804992in}{2.192409in}}%
\pgfpathlineto{\pgfqpoint{5.989578in}{5.077315in}}%
\pgfusepath{stroke}%
\end{pgfscope}%
\begin{pgfscope}%
\pgfpathrectangle{\pgfqpoint{0.526127in}{0.331635in}}{\pgfqpoint{9.300000in}{7.700000in}}%
\pgfusepath{clip}%
\pgfsetrectcap%
\pgfsetroundjoin%
\pgfsetlinewidth{1.505625pt}%
\definecolor{currentstroke}{rgb}{0.980392,0.690196,0.894118}%
\pgfsetstrokecolor{currentstroke}%
\pgfsetstrokeopacity{0.800000}%
\pgfsetdash{}{0pt}%
\pgfpathmoveto{\pgfqpoint{4.387621in}{6.311919in}}%
\pgfpathlineto{\pgfqpoint{5.989578in}{5.077315in}}%
\pgfusepath{stroke}%
\end{pgfscope}%
\begin{pgfscope}%
\pgfpathrectangle{\pgfqpoint{0.526127in}{0.331635in}}{\pgfqpoint{9.300000in}{7.700000in}}%
\pgfusepath{clip}%
\pgfsetrectcap%
\pgfsetroundjoin%
\pgfsetlinewidth{1.505625pt}%
\definecolor{currentstroke}{rgb}{0.980392,0.690196,0.894118}%
\pgfsetstrokecolor{currentstroke}%
\pgfsetstrokeopacity{0.800000}%
\pgfsetdash{}{0pt}%
\pgfpathmoveto{\pgfqpoint{4.353160in}{5.800789in}}%
\pgfpathlineto{\pgfqpoint{5.989578in}{5.077315in}}%
\pgfusepath{stroke}%
\end{pgfscope}%
\begin{pgfscope}%
\pgfpathrectangle{\pgfqpoint{0.526127in}{0.331635in}}{\pgfqpoint{9.300000in}{7.700000in}}%
\pgfusepath{clip}%
\pgfsetrectcap%
\pgfsetroundjoin%
\pgfsetlinewidth{1.505625pt}%
\definecolor{currentstroke}{rgb}{0.980392,0.690196,0.894118}%
\pgfsetstrokecolor{currentstroke}%
\pgfsetstrokeopacity{0.800000}%
\pgfsetdash{}{0pt}%
\pgfpathmoveto{\pgfqpoint{4.693280in}{5.829103in}}%
\pgfpathlineto{\pgfqpoint{5.989578in}{5.077315in}}%
\pgfusepath{stroke}%
\end{pgfscope}%
\begin{pgfscope}%
\pgfpathrectangle{\pgfqpoint{0.526127in}{0.331635in}}{\pgfqpoint{9.300000in}{7.700000in}}%
\pgfusepath{clip}%
\pgfsetrectcap%
\pgfsetroundjoin%
\pgfsetlinewidth{1.505625pt}%
\definecolor{currentstroke}{rgb}{0.980392,0.690196,0.894118}%
\pgfsetstrokecolor{currentstroke}%
\pgfsetstrokeopacity{0.800000}%
\pgfsetdash{}{0pt}%
\pgfpathmoveto{\pgfqpoint{9.403399in}{5.656119in}}%
\pgfpathlineto{\pgfqpoint{5.989578in}{5.077315in}}%
\pgfusepath{stroke}%
\end{pgfscope}%
\begin{pgfscope}%
\pgfpathrectangle{\pgfqpoint{0.526127in}{0.331635in}}{\pgfqpoint{9.300000in}{7.700000in}}%
\pgfusepath{clip}%
\pgfsetrectcap%
\pgfsetroundjoin%
\pgfsetlinewidth{1.505625pt}%
\definecolor{currentstroke}{rgb}{0.980392,0.690196,0.894118}%
\pgfsetstrokecolor{currentstroke}%
\pgfsetstrokeopacity{0.800000}%
\pgfsetdash{}{0pt}%
\pgfpathmoveto{\pgfqpoint{4.320810in}{3.351396in}}%
\pgfpathlineto{\pgfqpoint{5.989578in}{5.077315in}}%
\pgfusepath{stroke}%
\end{pgfscope}%
\begin{pgfscope}%
\pgfpathrectangle{\pgfqpoint{0.526127in}{0.331635in}}{\pgfqpoint{9.300000in}{7.700000in}}%
\pgfusepath{clip}%
\pgfsetrectcap%
\pgfsetroundjoin%
\pgfsetlinewidth{1.505625pt}%
\definecolor{currentstroke}{rgb}{0.980392,0.690196,0.894118}%
\pgfsetstrokecolor{currentstroke}%
\pgfsetstrokeopacity{0.800000}%
\pgfsetdash{}{0pt}%
\pgfpathmoveto{\pgfqpoint{1.703182in}{5.256898in}}%
\pgfpathlineto{\pgfqpoint{5.989578in}{5.077315in}}%
\pgfusepath{stroke}%
\end{pgfscope}%
\begin{pgfscope}%
\pgfpathrectangle{\pgfqpoint{0.526127in}{0.331635in}}{\pgfqpoint{9.300000in}{7.700000in}}%
\pgfusepath{clip}%
\pgfsetrectcap%
\pgfsetroundjoin%
\pgfsetlinewidth{1.505625pt}%
\definecolor{currentstroke}{rgb}{0.980392,0.690196,0.894118}%
\pgfsetstrokecolor{currentstroke}%
\pgfsetstrokeopacity{0.800000}%
\pgfsetdash{}{0pt}%
\pgfpathmoveto{\pgfqpoint{4.829760in}{6.478800in}}%
\pgfpathlineto{\pgfqpoint{5.989578in}{5.077315in}}%
\pgfusepath{stroke}%
\end{pgfscope}%
\begin{pgfscope}%
\pgfpathrectangle{\pgfqpoint{0.526127in}{0.331635in}}{\pgfqpoint{9.300000in}{7.700000in}}%
\pgfusepath{clip}%
\pgfsetrectcap%
\pgfsetroundjoin%
\pgfsetlinewidth{1.505625pt}%
\definecolor{currentstroke}{rgb}{0.980392,0.690196,0.894118}%
\pgfsetstrokecolor{currentstroke}%
\pgfsetstrokeopacity{0.800000}%
\pgfsetdash{}{0pt}%
\pgfpathmoveto{\pgfqpoint{5.312702in}{4.335644in}}%
\pgfpathlineto{\pgfqpoint{5.989578in}{5.077315in}}%
\pgfusepath{stroke}%
\end{pgfscope}%
\begin{pgfscope}%
\pgfpathrectangle{\pgfqpoint{0.526127in}{0.331635in}}{\pgfqpoint{9.300000in}{7.700000in}}%
\pgfusepath{clip}%
\pgfsetrectcap%
\pgfsetroundjoin%
\pgfsetlinewidth{1.505625pt}%
\definecolor{currentstroke}{rgb}{0.980392,0.690196,0.894118}%
\pgfsetstrokecolor{currentstroke}%
\pgfsetstrokeopacity{0.800000}%
\pgfsetdash{}{0pt}%
\pgfpathmoveto{\pgfqpoint{8.614970in}{4.769538in}}%
\pgfpathlineto{\pgfqpoint{5.989578in}{5.077315in}}%
\pgfusepath{stroke}%
\end{pgfscope}%
\begin{pgfscope}%
\pgfpathrectangle{\pgfqpoint{0.526127in}{0.331635in}}{\pgfqpoint{9.300000in}{7.700000in}}%
\pgfusepath{clip}%
\pgfsetrectcap%
\pgfsetroundjoin%
\pgfsetlinewidth{1.505625pt}%
\definecolor{currentstroke}{rgb}{0.980392,0.690196,0.894118}%
\pgfsetstrokecolor{currentstroke}%
\pgfsetstrokeopacity{0.800000}%
\pgfsetdash{}{0pt}%
\pgfpathmoveto{\pgfqpoint{8.457929in}{5.831669in}}%
\pgfpathlineto{\pgfqpoint{5.989578in}{5.077315in}}%
\pgfusepath{stroke}%
\end{pgfscope}%
\begin{pgfscope}%
\pgfpathrectangle{\pgfqpoint{0.526127in}{0.331635in}}{\pgfqpoint{9.300000in}{7.700000in}}%
\pgfusepath{clip}%
\pgfsetrectcap%
\pgfsetroundjoin%
\pgfsetlinewidth{1.505625pt}%
\definecolor{currentstroke}{rgb}{0.980392,0.690196,0.894118}%
\pgfsetstrokecolor{currentstroke}%
\pgfsetstrokeopacity{0.800000}%
\pgfsetdash{}{0pt}%
\pgfpathmoveto{\pgfqpoint{7.835370in}{4.178659in}}%
\pgfpathlineto{\pgfqpoint{5.989578in}{5.077315in}}%
\pgfusepath{stroke}%
\end{pgfscope}%
\begin{pgfscope}%
\pgfpathrectangle{\pgfqpoint{0.526127in}{0.331635in}}{\pgfqpoint{9.300000in}{7.700000in}}%
\pgfusepath{clip}%
\pgfsetrectcap%
\pgfsetroundjoin%
\pgfsetlinewidth{1.505625pt}%
\definecolor{currentstroke}{rgb}{0.721569,0.521569,0.039216}%
\pgfsetstrokecolor{currentstroke}%
\pgfsetstrokeopacity{0.800000}%
\pgfsetdash{}{0pt}%
\pgfpathmoveto{\pgfqpoint{5.776420in}{4.079147in}}%
\pgfpathlineto{\pgfqpoint{5.032891in}{4.505800in}}%
\pgfusepath{stroke}%
\end{pgfscope}%
\begin{pgfscope}%
\pgfpathrectangle{\pgfqpoint{0.526127in}{0.331635in}}{\pgfqpoint{9.300000in}{7.700000in}}%
\pgfusepath{clip}%
\pgfsetrectcap%
\pgfsetroundjoin%
\pgfsetlinewidth{1.505625pt}%
\definecolor{currentstroke}{rgb}{0.721569,0.521569,0.039216}%
\pgfsetstrokecolor{currentstroke}%
\pgfsetstrokeopacity{0.800000}%
\pgfsetdash{}{0pt}%
\pgfpathmoveto{\pgfqpoint{8.729922in}{5.055972in}}%
\pgfpathlineto{\pgfqpoint{5.032891in}{4.505800in}}%
\pgfusepath{stroke}%
\end{pgfscope}%
\begin{pgfscope}%
\pgfpathrectangle{\pgfqpoint{0.526127in}{0.331635in}}{\pgfqpoint{9.300000in}{7.700000in}}%
\pgfusepath{clip}%
\pgfsetrectcap%
\pgfsetroundjoin%
\pgfsetlinewidth{1.505625pt}%
\definecolor{currentstroke}{rgb}{0.721569,0.521569,0.039216}%
\pgfsetstrokecolor{currentstroke}%
\pgfsetstrokeopacity{0.800000}%
\pgfsetdash{}{0pt}%
\pgfpathmoveto{\pgfqpoint{2.913096in}{4.658470in}}%
\pgfpathlineto{\pgfqpoint{5.032891in}{4.505800in}}%
\pgfusepath{stroke}%
\end{pgfscope}%
\begin{pgfscope}%
\pgfpathrectangle{\pgfqpoint{0.526127in}{0.331635in}}{\pgfqpoint{9.300000in}{7.700000in}}%
\pgfusepath{clip}%
\pgfsetrectcap%
\pgfsetroundjoin%
\pgfsetlinewidth{1.505625pt}%
\definecolor{currentstroke}{rgb}{0.721569,0.521569,0.039216}%
\pgfsetstrokecolor{currentstroke}%
\pgfsetstrokeopacity{0.800000}%
\pgfsetdash{}{0pt}%
\pgfpathmoveto{\pgfqpoint{3.918807in}{5.779850in}}%
\pgfpathlineto{\pgfqpoint{5.032891in}{4.505800in}}%
\pgfusepath{stroke}%
\end{pgfscope}%
\begin{pgfscope}%
\pgfpathrectangle{\pgfqpoint{0.526127in}{0.331635in}}{\pgfqpoint{9.300000in}{7.700000in}}%
\pgfusepath{clip}%
\pgfsetrectcap%
\pgfsetroundjoin%
\pgfsetlinewidth{1.505625pt}%
\definecolor{currentstroke}{rgb}{0.721569,0.521569,0.039216}%
\pgfsetstrokecolor{currentstroke}%
\pgfsetstrokeopacity{0.800000}%
\pgfsetdash{}{0pt}%
\pgfpathmoveto{\pgfqpoint{2.663268in}{3.876036in}}%
\pgfpathlineto{\pgfqpoint{5.032891in}{4.505800in}}%
\pgfusepath{stroke}%
\end{pgfscope}%
\begin{pgfscope}%
\pgfpathrectangle{\pgfqpoint{0.526127in}{0.331635in}}{\pgfqpoint{9.300000in}{7.700000in}}%
\pgfusepath{clip}%
\pgfsetrectcap%
\pgfsetroundjoin%
\pgfsetlinewidth{1.505625pt}%
\definecolor{currentstroke}{rgb}{0.721569,0.521569,0.039216}%
\pgfsetstrokecolor{currentstroke}%
\pgfsetstrokeopacity{0.800000}%
\pgfsetdash{}{0pt}%
\pgfpathmoveto{\pgfqpoint{6.282605in}{4.955017in}}%
\pgfpathlineto{\pgfqpoint{5.032891in}{4.505800in}}%
\pgfusepath{stroke}%
\end{pgfscope}%
\begin{pgfscope}%
\pgfpathrectangle{\pgfqpoint{0.526127in}{0.331635in}}{\pgfqpoint{9.300000in}{7.700000in}}%
\pgfusepath{clip}%
\pgfsetrectcap%
\pgfsetroundjoin%
\pgfsetlinewidth{1.505625pt}%
\definecolor{currentstroke}{rgb}{0.721569,0.521569,0.039216}%
\pgfsetstrokecolor{currentstroke}%
\pgfsetstrokeopacity{0.800000}%
\pgfsetdash{}{0pt}%
\pgfpathmoveto{\pgfqpoint{3.179398in}{3.452987in}}%
\pgfpathlineto{\pgfqpoint{5.032891in}{4.505800in}}%
\pgfusepath{stroke}%
\end{pgfscope}%
\begin{pgfscope}%
\pgfpathrectangle{\pgfqpoint{0.526127in}{0.331635in}}{\pgfqpoint{9.300000in}{7.700000in}}%
\pgfusepath{clip}%
\pgfsetrectcap%
\pgfsetroundjoin%
\pgfsetlinewidth{1.505625pt}%
\definecolor{currentstroke}{rgb}{0.721569,0.521569,0.039216}%
\pgfsetstrokecolor{currentstroke}%
\pgfsetstrokeopacity{0.800000}%
\pgfsetdash{}{0pt}%
\pgfpathmoveto{\pgfqpoint{4.240309in}{7.120458in}}%
\pgfpathlineto{\pgfqpoint{5.032891in}{4.505800in}}%
\pgfusepath{stroke}%
\end{pgfscope}%
\begin{pgfscope}%
\pgfpathrectangle{\pgfqpoint{0.526127in}{0.331635in}}{\pgfqpoint{9.300000in}{7.700000in}}%
\pgfusepath{clip}%
\pgfsetrectcap%
\pgfsetroundjoin%
\pgfsetlinewidth{1.505625pt}%
\definecolor{currentstroke}{rgb}{0.721569,0.521569,0.039216}%
\pgfsetstrokecolor{currentstroke}%
\pgfsetstrokeopacity{0.800000}%
\pgfsetdash{}{0pt}%
\pgfpathmoveto{\pgfqpoint{5.395356in}{4.893643in}}%
\pgfpathlineto{\pgfqpoint{5.032891in}{4.505800in}}%
\pgfusepath{stroke}%
\end{pgfscope}%
\begin{pgfscope}%
\pgfpathrectangle{\pgfqpoint{0.526127in}{0.331635in}}{\pgfqpoint{9.300000in}{7.700000in}}%
\pgfusepath{clip}%
\pgfsetrectcap%
\pgfsetroundjoin%
\pgfsetlinewidth{1.505625pt}%
\definecolor{currentstroke}{rgb}{0.721569,0.521569,0.039216}%
\pgfsetstrokecolor{currentstroke}%
\pgfsetstrokeopacity{0.800000}%
\pgfsetdash{}{0pt}%
\pgfpathmoveto{\pgfqpoint{4.154359in}{5.130415in}}%
\pgfpathlineto{\pgfqpoint{5.032891in}{4.505800in}}%
\pgfusepath{stroke}%
\end{pgfscope}%
\begin{pgfscope}%
\pgfpathrectangle{\pgfqpoint{0.526127in}{0.331635in}}{\pgfqpoint{9.300000in}{7.700000in}}%
\pgfusepath{clip}%
\pgfsetrectcap%
\pgfsetroundjoin%
\pgfsetlinewidth{1.505625pt}%
\definecolor{currentstroke}{rgb}{0.721569,0.521569,0.039216}%
\pgfsetstrokecolor{currentstroke}%
\pgfsetstrokeopacity{0.800000}%
\pgfsetdash{}{0pt}%
\pgfpathmoveto{\pgfqpoint{3.012801in}{4.477258in}}%
\pgfpathlineto{\pgfqpoint{5.032891in}{4.505800in}}%
\pgfusepath{stroke}%
\end{pgfscope}%
\begin{pgfscope}%
\pgfpathrectangle{\pgfqpoint{0.526127in}{0.331635in}}{\pgfqpoint{9.300000in}{7.700000in}}%
\pgfusepath{clip}%
\pgfsetrectcap%
\pgfsetroundjoin%
\pgfsetlinewidth{1.505625pt}%
\definecolor{currentstroke}{rgb}{0.721569,0.521569,0.039216}%
\pgfsetstrokecolor{currentstroke}%
\pgfsetstrokeopacity{0.800000}%
\pgfsetdash{}{0pt}%
\pgfpathmoveto{\pgfqpoint{6.132477in}{1.011772in}}%
\pgfpathlineto{\pgfqpoint{5.032891in}{4.505800in}}%
\pgfusepath{stroke}%
\end{pgfscope}%
\begin{pgfscope}%
\pgfpathrectangle{\pgfqpoint{0.526127in}{0.331635in}}{\pgfqpoint{9.300000in}{7.700000in}}%
\pgfusepath{clip}%
\pgfsetrectcap%
\pgfsetroundjoin%
\pgfsetlinewidth{1.505625pt}%
\definecolor{currentstroke}{rgb}{0.721569,0.521569,0.039216}%
\pgfsetstrokecolor{currentstroke}%
\pgfsetstrokeopacity{0.800000}%
\pgfsetdash{}{0pt}%
\pgfpathmoveto{\pgfqpoint{8.967963in}{2.899690in}}%
\pgfpathlineto{\pgfqpoint{5.032891in}{4.505800in}}%
\pgfusepath{stroke}%
\end{pgfscope}%
\begin{pgfscope}%
\pgfpathrectangle{\pgfqpoint{0.526127in}{0.331635in}}{\pgfqpoint{9.300000in}{7.700000in}}%
\pgfusepath{clip}%
\pgfsetrectcap%
\pgfsetroundjoin%
\pgfsetlinewidth{1.505625pt}%
\definecolor{currentstroke}{rgb}{0.721569,0.521569,0.039216}%
\pgfsetstrokecolor{currentstroke}%
\pgfsetstrokeopacity{0.800000}%
\pgfsetdash{}{0pt}%
\pgfpathmoveto{\pgfqpoint{2.684948in}{3.939327in}}%
\pgfpathlineto{\pgfqpoint{5.032891in}{4.505800in}}%
\pgfusepath{stroke}%
\end{pgfscope}%
\begin{pgfscope}%
\pgfpathrectangle{\pgfqpoint{0.526127in}{0.331635in}}{\pgfqpoint{9.300000in}{7.700000in}}%
\pgfusepath{clip}%
\pgfsetrectcap%
\pgfsetroundjoin%
\pgfsetlinewidth{1.505625pt}%
\definecolor{currentstroke}{rgb}{0.721569,0.521569,0.039216}%
\pgfsetstrokecolor{currentstroke}%
\pgfsetstrokeopacity{0.800000}%
\pgfsetdash{}{0pt}%
\pgfpathmoveto{\pgfqpoint{7.144222in}{6.645535in}}%
\pgfpathlineto{\pgfqpoint{5.032891in}{4.505800in}}%
\pgfusepath{stroke}%
\end{pgfscope}%
\begin{pgfscope}%
\pgfpathrectangle{\pgfqpoint{0.526127in}{0.331635in}}{\pgfqpoint{9.300000in}{7.700000in}}%
\pgfusepath{clip}%
\pgfsetrectcap%
\pgfsetroundjoin%
\pgfsetlinewidth{1.505625pt}%
\definecolor{currentstroke}{rgb}{0.721569,0.521569,0.039216}%
\pgfsetstrokecolor{currentstroke}%
\pgfsetstrokeopacity{0.800000}%
\pgfsetdash{}{0pt}%
\pgfpathmoveto{\pgfqpoint{5.755207in}{1.746054in}}%
\pgfpathlineto{\pgfqpoint{5.032891in}{4.505800in}}%
\pgfusepath{stroke}%
\end{pgfscope}%
\begin{pgfscope}%
\pgfpathrectangle{\pgfqpoint{0.526127in}{0.331635in}}{\pgfqpoint{9.300000in}{7.700000in}}%
\pgfusepath{clip}%
\pgfsetrectcap%
\pgfsetroundjoin%
\pgfsetlinewidth{1.505625pt}%
\definecolor{currentstroke}{rgb}{0.721569,0.521569,0.039216}%
\pgfsetstrokecolor{currentstroke}%
\pgfsetstrokeopacity{0.800000}%
\pgfsetdash{}{0pt}%
\pgfpathmoveto{\pgfqpoint{9.079707in}{4.416691in}}%
\pgfpathlineto{\pgfqpoint{5.032891in}{4.505800in}}%
\pgfusepath{stroke}%
\end{pgfscope}%
\begin{pgfscope}%
\pgfpathrectangle{\pgfqpoint{0.526127in}{0.331635in}}{\pgfqpoint{9.300000in}{7.700000in}}%
\pgfusepath{clip}%
\pgfsetrectcap%
\pgfsetroundjoin%
\pgfsetlinewidth{1.505625pt}%
\definecolor{currentstroke}{rgb}{0.721569,0.521569,0.039216}%
\pgfsetstrokecolor{currentstroke}%
\pgfsetstrokeopacity{0.800000}%
\pgfsetdash{}{0pt}%
\pgfpathmoveto{\pgfqpoint{4.104581in}{1.312881in}}%
\pgfpathlineto{\pgfqpoint{5.032891in}{4.505800in}}%
\pgfusepath{stroke}%
\end{pgfscope}%
\begin{pgfscope}%
\pgfpathrectangle{\pgfqpoint{0.526127in}{0.331635in}}{\pgfqpoint{9.300000in}{7.700000in}}%
\pgfusepath{clip}%
\pgfsetrectcap%
\pgfsetroundjoin%
\pgfsetlinewidth{1.505625pt}%
\definecolor{currentstroke}{rgb}{0.721569,0.521569,0.039216}%
\pgfsetstrokecolor{currentstroke}%
\pgfsetstrokeopacity{0.800000}%
\pgfsetdash{}{0pt}%
\pgfpathmoveto{\pgfqpoint{1.641796in}{4.156474in}}%
\pgfpathlineto{\pgfqpoint{5.032891in}{4.505800in}}%
\pgfusepath{stroke}%
\end{pgfscope}%
\begin{pgfscope}%
\pgfpathrectangle{\pgfqpoint{0.526127in}{0.331635in}}{\pgfqpoint{9.300000in}{7.700000in}}%
\pgfusepath{clip}%
\pgfsetrectcap%
\pgfsetroundjoin%
\pgfsetlinewidth{1.505625pt}%
\definecolor{currentstroke}{rgb}{0.721569,0.521569,0.039216}%
\pgfsetstrokecolor{currentstroke}%
\pgfsetstrokeopacity{0.800000}%
\pgfsetdash{}{0pt}%
\pgfpathmoveto{\pgfqpoint{2.312627in}{5.998767in}}%
\pgfpathlineto{\pgfqpoint{5.032891in}{4.505800in}}%
\pgfusepath{stroke}%
\end{pgfscope}%
\begin{pgfscope}%
\pgfpathrectangle{\pgfqpoint{0.526127in}{0.331635in}}{\pgfqpoint{9.300000in}{7.700000in}}%
\pgfusepath{clip}%
\pgfsetrectcap%
\pgfsetroundjoin%
\pgfsetlinewidth{1.505625pt}%
\definecolor{currentstroke}{rgb}{0.721569,0.521569,0.039216}%
\pgfsetstrokecolor{currentstroke}%
\pgfsetstrokeopacity{0.800000}%
\pgfsetdash{}{0pt}%
\pgfpathmoveto{\pgfqpoint{7.299025in}{5.752896in}}%
\pgfpathlineto{\pgfqpoint{5.032891in}{4.505800in}}%
\pgfusepath{stroke}%
\end{pgfscope}%
\begin{pgfscope}%
\pgfpathrectangle{\pgfqpoint{0.526127in}{0.331635in}}{\pgfqpoint{9.300000in}{7.700000in}}%
\pgfusepath{clip}%
\pgfsetrectcap%
\pgfsetroundjoin%
\pgfsetlinewidth{1.505625pt}%
\definecolor{currentstroke}{rgb}{0.721569,0.521569,0.039216}%
\pgfsetstrokecolor{currentstroke}%
\pgfsetstrokeopacity{0.800000}%
\pgfsetdash{}{0pt}%
\pgfpathmoveto{\pgfqpoint{1.663441in}{4.106068in}}%
\pgfpathlineto{\pgfqpoint{5.032891in}{4.505800in}}%
\pgfusepath{stroke}%
\end{pgfscope}%
\begin{pgfscope}%
\pgfpathrectangle{\pgfqpoint{0.526127in}{0.331635in}}{\pgfqpoint{9.300000in}{7.700000in}}%
\pgfusepath{clip}%
\pgfsetrectcap%
\pgfsetroundjoin%
\pgfsetlinewidth{1.505625pt}%
\definecolor{currentstroke}{rgb}{0.721569,0.521569,0.039216}%
\pgfsetstrokecolor{currentstroke}%
\pgfsetstrokeopacity{0.800000}%
\pgfsetdash{}{0pt}%
\pgfpathmoveto{\pgfqpoint{9.274981in}{5.558781in}}%
\pgfpathlineto{\pgfqpoint{5.032891in}{4.505800in}}%
\pgfusepath{stroke}%
\end{pgfscope}%
\begin{pgfscope}%
\pgfpathrectangle{\pgfqpoint{0.526127in}{0.331635in}}{\pgfqpoint{9.300000in}{7.700000in}}%
\pgfusepath{clip}%
\pgfsetrectcap%
\pgfsetroundjoin%
\pgfsetlinewidth{1.505625pt}%
\definecolor{currentstroke}{rgb}{0.721569,0.521569,0.039216}%
\pgfsetstrokecolor{currentstroke}%
\pgfsetstrokeopacity{0.800000}%
\pgfsetdash{}{0pt}%
\pgfpathmoveto{\pgfqpoint{9.099248in}{4.690642in}}%
\pgfpathlineto{\pgfqpoint{5.032891in}{4.505800in}}%
\pgfusepath{stroke}%
\end{pgfscope}%
\begin{pgfscope}%
\pgfpathrectangle{\pgfqpoint{0.526127in}{0.331635in}}{\pgfqpoint{9.300000in}{7.700000in}}%
\pgfusepath{clip}%
\pgfsetrectcap%
\pgfsetroundjoin%
\pgfsetlinewidth{1.505625pt}%
\definecolor{currentstroke}{rgb}{0.721569,0.521569,0.039216}%
\pgfsetstrokecolor{currentstroke}%
\pgfsetstrokeopacity{0.800000}%
\pgfsetdash{}{0pt}%
\pgfpathmoveto{\pgfqpoint{4.144255in}{5.886144in}}%
\pgfpathlineto{\pgfqpoint{5.032891in}{4.505800in}}%
\pgfusepath{stroke}%
\end{pgfscope}%
\begin{pgfscope}%
\pgfpathrectangle{\pgfqpoint{0.526127in}{0.331635in}}{\pgfqpoint{9.300000in}{7.700000in}}%
\pgfusepath{clip}%
\pgfsetrectcap%
\pgfsetroundjoin%
\pgfsetlinewidth{1.505625pt}%
\definecolor{currentstroke}{rgb}{0.721569,0.521569,0.039216}%
\pgfsetstrokecolor{currentstroke}%
\pgfsetstrokeopacity{0.800000}%
\pgfsetdash{}{0pt}%
\pgfpathmoveto{\pgfqpoint{6.389607in}{4.514463in}}%
\pgfpathlineto{\pgfqpoint{5.032891in}{4.505800in}}%
\pgfusepath{stroke}%
\end{pgfscope}%
\begin{pgfscope}%
\pgfpathrectangle{\pgfqpoint{0.526127in}{0.331635in}}{\pgfqpoint{9.300000in}{7.700000in}}%
\pgfusepath{clip}%
\pgfsetrectcap%
\pgfsetroundjoin%
\pgfsetlinewidth{1.505625pt}%
\definecolor{currentstroke}{rgb}{0.721569,0.521569,0.039216}%
\pgfsetstrokecolor{currentstroke}%
\pgfsetstrokeopacity{0.800000}%
\pgfsetdash{}{0pt}%
\pgfpathmoveto{\pgfqpoint{3.057300in}{5.120509in}}%
\pgfpathlineto{\pgfqpoint{5.032891in}{4.505800in}}%
\pgfusepath{stroke}%
\end{pgfscope}%
\begin{pgfscope}%
\pgfpathrectangle{\pgfqpoint{0.526127in}{0.331635in}}{\pgfqpoint{9.300000in}{7.700000in}}%
\pgfusepath{clip}%
\pgfsetrectcap%
\pgfsetroundjoin%
\pgfsetlinewidth{1.505625pt}%
\definecolor{currentstroke}{rgb}{0.721569,0.521569,0.039216}%
\pgfsetstrokecolor{currentstroke}%
\pgfsetstrokeopacity{0.800000}%
\pgfsetdash{}{0pt}%
\pgfpathmoveto{\pgfqpoint{1.903215in}{4.926442in}}%
\pgfpathlineto{\pgfqpoint{5.032891in}{4.505800in}}%
\pgfusepath{stroke}%
\end{pgfscope}%
\begin{pgfscope}%
\pgfsetrectcap%
\pgfsetmiterjoin%
\pgfsetlinewidth{0.803000pt}%
\definecolor{currentstroke}{rgb}{0.000000,0.000000,0.000000}%
\pgfsetstrokecolor{currentstroke}%
\pgfsetdash{}{0pt}%
\pgfpathmoveto{\pgfqpoint{0.526127in}{0.331635in}}%
\pgfpathlineto{\pgfqpoint{0.526127in}{8.031635in}}%
\pgfusepath{stroke}%
\end{pgfscope}%
\begin{pgfscope}%
\pgfsetrectcap%
\pgfsetmiterjoin%
\pgfsetlinewidth{0.803000pt}%
\definecolor{currentstroke}{rgb}{0.000000,0.000000,0.000000}%
\pgfsetstrokecolor{currentstroke}%
\pgfsetdash{}{0pt}%
\pgfpathmoveto{\pgfqpoint{9.826127in}{0.331635in}}%
\pgfpathlineto{\pgfqpoint{9.826127in}{8.031635in}}%
\pgfusepath{stroke}%
\end{pgfscope}%
\begin{pgfscope}%
\pgfsetrectcap%
\pgfsetmiterjoin%
\pgfsetlinewidth{0.803000pt}%
\definecolor{currentstroke}{rgb}{0.000000,0.000000,0.000000}%
\pgfsetstrokecolor{currentstroke}%
\pgfsetdash{}{0pt}%
\pgfpathmoveto{\pgfqpoint{0.526127in}{0.331635in}}%
\pgfpathlineto{\pgfqpoint{9.826127in}{0.331635in}}%
\pgfusepath{stroke}%
\end{pgfscope}%
\begin{pgfscope}%
\pgfsetrectcap%
\pgfsetmiterjoin%
\pgfsetlinewidth{0.803000pt}%
\definecolor{currentstroke}{rgb}{0.000000,0.000000,0.000000}%
\pgfsetstrokecolor{currentstroke}%
\pgfsetdash{}{0pt}%
\pgfpathmoveto{\pgfqpoint{0.526127in}{8.031635in}}%
\pgfpathlineto{\pgfqpoint{9.826127in}{8.031635in}}%
\pgfusepath{stroke}%
\end{pgfscope}%
\begin{pgfscope}%
\definecolor{textcolor}{rgb}{0.000000,0.000000,0.000000}%
\pgfsetstrokecolor{textcolor}%
\pgfsetfillcolor{textcolor}%
\pgftext[x=5.176127in,y=8.114968in,,base]{\color{textcolor}\sffamily\fontsize{12.000000}{14.400000}\selectfont Photo-Realistic Images}%
\end{pgfscope}%
\begin{pgfscope}%
\pgfsetbuttcap%
\pgfsetmiterjoin%
\definecolor{currentfill}{rgb}{1.000000,1.000000,1.000000}%
\pgfsetfillcolor{currentfill}%
\pgfsetfillopacity{0.800000}%
\pgfsetlinewidth{1.003750pt}%
\definecolor{currentstroke}{rgb}{0.800000,0.800000,0.800000}%
\pgfsetstrokecolor{currentstroke}%
\pgfsetstrokeopacity{0.800000}%
\pgfsetdash{}{0pt}%
\pgfpathmoveto{\pgfqpoint{9.923349in}{3.345373in}}%
\pgfpathlineto{\pgfqpoint{11.202280in}{3.345373in}}%
\pgfpathquadraticcurveto{\pgfqpoint{11.230057in}{3.345373in}}{\pgfqpoint{11.230057in}{3.373151in}}%
\pgfpathlineto{\pgfqpoint{11.230057in}{4.990119in}}%
\pgfpathquadraticcurveto{\pgfqpoint{11.230057in}{5.017897in}}{\pgfqpoint{11.202280in}{5.017897in}}%
\pgfpathlineto{\pgfqpoint{9.923349in}{5.017897in}}%
\pgfpathquadraticcurveto{\pgfqpoint{9.895571in}{5.017897in}}{\pgfqpoint{9.895571in}{4.990119in}}%
\pgfpathlineto{\pgfqpoint{9.895571in}{3.373151in}}%
\pgfpathquadraticcurveto{\pgfqpoint{9.895571in}{3.345373in}}{\pgfqpoint{9.923349in}{3.345373in}}%
\pgfpathclose%
\pgfusepath{stroke,fill}%
\end{pgfscope}%
\begin{pgfscope}%
\pgfsetbuttcap%
\pgfsetroundjoin%
\definecolor{currentfill}{rgb}{0.631373,0.788235,0.956863}%
\pgfsetfillcolor{currentfill}%
\pgfsetlinewidth{1.003750pt}%
\definecolor{currentstroke}{rgb}{0.631373,0.788235,0.956863}%
\pgfsetstrokecolor{currentstroke}%
\pgfsetdash{}{0pt}%
\pgfsys@defobject{currentmarker}{\pgfqpoint{-0.041667in}{-0.041667in}}{\pgfqpoint{0.041667in}{0.041667in}}{%
\pgfpathmoveto{\pgfqpoint{0.000000in}{-0.041667in}}%
\pgfpathcurveto{\pgfqpoint{0.011050in}{-0.041667in}}{\pgfqpoint{0.021649in}{-0.037276in}}{\pgfqpoint{0.029463in}{-0.029463in}}%
\pgfpathcurveto{\pgfqpoint{0.037276in}{-0.021649in}}{\pgfqpoint{0.041667in}{-0.011050in}}{\pgfqpoint{0.041667in}{0.000000in}}%
\pgfpathcurveto{\pgfqpoint{0.041667in}{0.011050in}}{\pgfqpoint{0.037276in}{0.021649in}}{\pgfqpoint{0.029463in}{0.029463in}}%
\pgfpathcurveto{\pgfqpoint{0.021649in}{0.037276in}}{\pgfqpoint{0.011050in}{0.041667in}}{\pgfqpoint{0.000000in}{0.041667in}}%
\pgfpathcurveto{\pgfqpoint{-0.011050in}{0.041667in}}{\pgfqpoint{-0.021649in}{0.037276in}}{\pgfqpoint{-0.029463in}{0.029463in}}%
\pgfpathcurveto{\pgfqpoint{-0.037276in}{0.021649in}}{\pgfqpoint{-0.041667in}{0.011050in}}{\pgfqpoint{-0.041667in}{0.000000in}}%
\pgfpathcurveto{\pgfqpoint{-0.041667in}{-0.011050in}}{\pgfqpoint{-0.037276in}{-0.021649in}}{\pgfqpoint{-0.029463in}{-0.029463in}}%
\pgfpathcurveto{\pgfqpoint{-0.021649in}{-0.037276in}}{\pgfqpoint{-0.011050in}{-0.041667in}}{\pgfqpoint{0.000000in}{-0.041667in}}%
\pgfpathclose%
\pgfusepath{stroke,fill}%
}%
\begin{pgfscope}%
\pgfsys@transformshift{10.090016in}{4.893277in}%
\pgfsys@useobject{currentmarker}{}%
\end{pgfscope}%
\end{pgfscope}%
\begin{pgfscope}%
\definecolor{textcolor}{rgb}{0.000000,0.000000,0.000000}%
\pgfsetstrokecolor{textcolor}%
\pgfsetfillcolor{textcolor}%
\pgftext[x=10.340016in,y=4.856819in,left,base]{\color{textcolor}\sffamily\fontsize{10.000000}{12.000000}\selectfont openrooms}%
\end{pgfscope}%
\begin{pgfscope}%
\pgfsetbuttcap%
\pgfsetroundjoin%
\definecolor{currentfill}{rgb}{1.000000,0.705882,0.509804}%
\pgfsetfillcolor{currentfill}%
\pgfsetlinewidth{1.003750pt}%
\definecolor{currentstroke}{rgb}{1.000000,0.705882,0.509804}%
\pgfsetstrokecolor{currentstroke}%
\pgfsetdash{}{0pt}%
\pgfsys@defobject{currentmarker}{\pgfqpoint{-0.041667in}{-0.041667in}}{\pgfqpoint{0.041667in}{0.041667in}}{%
\pgfpathmoveto{\pgfqpoint{0.000000in}{-0.041667in}}%
\pgfpathcurveto{\pgfqpoint{0.011050in}{-0.041667in}}{\pgfqpoint{0.021649in}{-0.037276in}}{\pgfqpoint{0.029463in}{-0.029463in}}%
\pgfpathcurveto{\pgfqpoint{0.037276in}{-0.021649in}}{\pgfqpoint{0.041667in}{-0.011050in}}{\pgfqpoint{0.041667in}{0.000000in}}%
\pgfpathcurveto{\pgfqpoint{0.041667in}{0.011050in}}{\pgfqpoint{0.037276in}{0.021649in}}{\pgfqpoint{0.029463in}{0.029463in}}%
\pgfpathcurveto{\pgfqpoint{0.021649in}{0.037276in}}{\pgfqpoint{0.011050in}{0.041667in}}{\pgfqpoint{0.000000in}{0.041667in}}%
\pgfpathcurveto{\pgfqpoint{-0.011050in}{0.041667in}}{\pgfqpoint{-0.021649in}{0.037276in}}{\pgfqpoint{-0.029463in}{0.029463in}}%
\pgfpathcurveto{\pgfqpoint{-0.037276in}{0.021649in}}{\pgfqpoint{-0.041667in}{0.011050in}}{\pgfqpoint{-0.041667in}{0.000000in}}%
\pgfpathcurveto{\pgfqpoint{-0.041667in}{-0.011050in}}{\pgfqpoint{-0.037276in}{-0.021649in}}{\pgfqpoint{-0.029463in}{-0.029463in}}%
\pgfpathcurveto{\pgfqpoint{-0.021649in}{-0.037276in}}{\pgfqpoint{-0.011050in}{-0.041667in}}{\pgfqpoint{0.000000in}{-0.041667in}}%
\pgfpathclose%
\pgfusepath{stroke,fill}%
}%
\begin{pgfscope}%
\pgfsys@transformshift{10.090016in}{4.689420in}%
\pgfsys@useobject{currentmarker}{}%
\end{pgfscope}%
\end{pgfscope}%
\begin{pgfscope}%
\definecolor{textcolor}{rgb}{0.000000,0.000000,0.000000}%
\pgfsetstrokecolor{textcolor}%
\pgfsetfillcolor{textcolor}%
\pgftext[x=10.340016in,y=4.652961in,left,base]{\color{textcolor}\sffamily\fontsize{10.000000}{12.000000}\selectfont scenenet}%
\end{pgfscope}%
\begin{pgfscope}%
\pgfsetbuttcap%
\pgfsetroundjoin%
\definecolor{currentfill}{rgb}{0.552941,0.898039,0.631373}%
\pgfsetfillcolor{currentfill}%
\pgfsetlinewidth{1.003750pt}%
\definecolor{currentstroke}{rgb}{0.552941,0.898039,0.631373}%
\pgfsetstrokecolor{currentstroke}%
\pgfsetdash{}{0pt}%
\pgfsys@defobject{currentmarker}{\pgfqpoint{-0.041667in}{-0.041667in}}{\pgfqpoint{0.041667in}{0.041667in}}{%
\pgfpathmoveto{\pgfqpoint{0.000000in}{-0.041667in}}%
\pgfpathcurveto{\pgfqpoint{0.011050in}{-0.041667in}}{\pgfqpoint{0.021649in}{-0.037276in}}{\pgfqpoint{0.029463in}{-0.029463in}}%
\pgfpathcurveto{\pgfqpoint{0.037276in}{-0.021649in}}{\pgfqpoint{0.041667in}{-0.011050in}}{\pgfqpoint{0.041667in}{0.000000in}}%
\pgfpathcurveto{\pgfqpoint{0.041667in}{0.011050in}}{\pgfqpoint{0.037276in}{0.021649in}}{\pgfqpoint{0.029463in}{0.029463in}}%
\pgfpathcurveto{\pgfqpoint{0.021649in}{0.037276in}}{\pgfqpoint{0.011050in}{0.041667in}}{\pgfqpoint{0.000000in}{0.041667in}}%
\pgfpathcurveto{\pgfqpoint{-0.011050in}{0.041667in}}{\pgfqpoint{-0.021649in}{0.037276in}}{\pgfqpoint{-0.029463in}{0.029463in}}%
\pgfpathcurveto{\pgfqpoint{-0.037276in}{0.021649in}}{\pgfqpoint{-0.041667in}{0.011050in}}{\pgfqpoint{-0.041667in}{0.000000in}}%
\pgfpathcurveto{\pgfqpoint{-0.041667in}{-0.011050in}}{\pgfqpoint{-0.037276in}{-0.021649in}}{\pgfqpoint{-0.029463in}{-0.029463in}}%
\pgfpathcurveto{\pgfqpoint{-0.021649in}{-0.037276in}}{\pgfqpoint{-0.011050in}{-0.041667in}}{\pgfqpoint{0.000000in}{-0.041667in}}%
\pgfpathclose%
\pgfusepath{stroke,fill}%
}%
\begin{pgfscope}%
\pgfsys@transformshift{10.090016in}{4.485562in}%
\pgfsys@useobject{currentmarker}{}%
\end{pgfscope}%
\end{pgfscope}%
\begin{pgfscope}%
\definecolor{textcolor}{rgb}{0.000000,0.000000,0.000000}%
\pgfsetstrokecolor{textcolor}%
\pgfsetfillcolor{textcolor}%
\pgftext[x=10.340016in,y=4.449104in,left,base]{\color{textcolor}\sffamily\fontsize{10.000000}{12.000000}\selectfont ai2thor}%
\end{pgfscope}%
\begin{pgfscope}%
\pgfsetbuttcap%
\pgfsetroundjoin%
\definecolor{currentfill}{rgb}{1.000000,0.623529,0.607843}%
\pgfsetfillcolor{currentfill}%
\pgfsetlinewidth{1.003750pt}%
\definecolor{currentstroke}{rgb}{1.000000,0.623529,0.607843}%
\pgfsetstrokecolor{currentstroke}%
\pgfsetdash{}{0pt}%
\pgfsys@defobject{currentmarker}{\pgfqpoint{-0.041667in}{-0.041667in}}{\pgfqpoint{0.041667in}{0.041667in}}{%
\pgfpathmoveto{\pgfqpoint{0.000000in}{-0.041667in}}%
\pgfpathcurveto{\pgfqpoint{0.011050in}{-0.041667in}}{\pgfqpoint{0.021649in}{-0.037276in}}{\pgfqpoint{0.029463in}{-0.029463in}}%
\pgfpathcurveto{\pgfqpoint{0.037276in}{-0.021649in}}{\pgfqpoint{0.041667in}{-0.011050in}}{\pgfqpoint{0.041667in}{0.000000in}}%
\pgfpathcurveto{\pgfqpoint{0.041667in}{0.011050in}}{\pgfqpoint{0.037276in}{0.021649in}}{\pgfqpoint{0.029463in}{0.029463in}}%
\pgfpathcurveto{\pgfqpoint{0.021649in}{0.037276in}}{\pgfqpoint{0.011050in}{0.041667in}}{\pgfqpoint{0.000000in}{0.041667in}}%
\pgfpathcurveto{\pgfqpoint{-0.011050in}{0.041667in}}{\pgfqpoint{-0.021649in}{0.037276in}}{\pgfqpoint{-0.029463in}{0.029463in}}%
\pgfpathcurveto{\pgfqpoint{-0.037276in}{0.021649in}}{\pgfqpoint{-0.041667in}{0.011050in}}{\pgfqpoint{-0.041667in}{0.000000in}}%
\pgfpathcurveto{\pgfqpoint{-0.041667in}{-0.011050in}}{\pgfqpoint{-0.037276in}{-0.021649in}}{\pgfqpoint{-0.029463in}{-0.029463in}}%
\pgfpathcurveto{\pgfqpoint{-0.021649in}{-0.037276in}}{\pgfqpoint{-0.011050in}{-0.041667in}}{\pgfqpoint{0.000000in}{-0.041667in}}%
\pgfpathclose%
\pgfusepath{stroke,fill}%
}%
\begin{pgfscope}%
\pgfsys@transformshift{10.090016in}{4.281705in}%
\pgfsys@useobject{currentmarker}{}%
\end{pgfscope}%
\end{pgfscope}%
\begin{pgfscope}%
\definecolor{textcolor}{rgb}{0.000000,0.000000,0.000000}%
\pgfsetstrokecolor{textcolor}%
\pgfsetfillcolor{textcolor}%
\pgftext[x=10.340016in,y=4.245247in,left,base]{\color{textcolor}\sffamily\fontsize{10.000000}{12.000000}\selectfont blenderproc}%
\end{pgfscope}%
\begin{pgfscope}%
\pgfsetbuttcap%
\pgfsetroundjoin%
\definecolor{currentfill}{rgb}{0.815686,0.733333,1.000000}%
\pgfsetfillcolor{currentfill}%
\pgfsetlinewidth{1.003750pt}%
\definecolor{currentstroke}{rgb}{0.815686,0.733333,1.000000}%
\pgfsetstrokecolor{currentstroke}%
\pgfsetdash{}{0pt}%
\pgfsys@defobject{currentmarker}{\pgfqpoint{-0.041667in}{-0.041667in}}{\pgfqpoint{0.041667in}{0.041667in}}{%
\pgfpathmoveto{\pgfqpoint{0.000000in}{-0.041667in}}%
\pgfpathcurveto{\pgfqpoint{0.011050in}{-0.041667in}}{\pgfqpoint{0.021649in}{-0.037276in}}{\pgfqpoint{0.029463in}{-0.029463in}}%
\pgfpathcurveto{\pgfqpoint{0.037276in}{-0.021649in}}{\pgfqpoint{0.041667in}{-0.011050in}}{\pgfqpoint{0.041667in}{0.000000in}}%
\pgfpathcurveto{\pgfqpoint{0.041667in}{0.011050in}}{\pgfqpoint{0.037276in}{0.021649in}}{\pgfqpoint{0.029463in}{0.029463in}}%
\pgfpathcurveto{\pgfqpoint{0.021649in}{0.037276in}}{\pgfqpoint{0.011050in}{0.041667in}}{\pgfqpoint{0.000000in}{0.041667in}}%
\pgfpathcurveto{\pgfqpoint{-0.011050in}{0.041667in}}{\pgfqpoint{-0.021649in}{0.037276in}}{\pgfqpoint{-0.029463in}{0.029463in}}%
\pgfpathcurveto{\pgfqpoint{-0.037276in}{0.021649in}}{\pgfqpoint{-0.041667in}{0.011050in}}{\pgfqpoint{-0.041667in}{0.000000in}}%
\pgfpathcurveto{\pgfqpoint{-0.041667in}{-0.011050in}}{\pgfqpoint{-0.037276in}{-0.021649in}}{\pgfqpoint{-0.029463in}{-0.029463in}}%
\pgfpathcurveto{\pgfqpoint{-0.021649in}{-0.037276in}}{\pgfqpoint{-0.011050in}{-0.041667in}}{\pgfqpoint{0.000000in}{-0.041667in}}%
\pgfpathclose%
\pgfusepath{stroke,fill}%
}%
\begin{pgfscope}%
\pgfsys@transformshift{10.090016in}{4.077848in}%
\pgfsys@useobject{currentmarker}{}%
\end{pgfscope}%
\end{pgfscope}%
\begin{pgfscope}%
\definecolor{textcolor}{rgb}{0.000000,0.000000,0.000000}%
\pgfsetstrokecolor{textcolor}%
\pgfsetfillcolor{textcolor}%
\pgftext[x=10.340016in,y=4.041390in,left,base]{\color{textcolor}\sffamily\fontsize{10.000000}{12.000000}\selectfont hypersim}%
\end{pgfscope}%
\begin{pgfscope}%
\pgfsetbuttcap%
\pgfsetroundjoin%
\definecolor{currentfill}{rgb}{0.870588,0.733333,0.607843}%
\pgfsetfillcolor{currentfill}%
\pgfsetlinewidth{1.003750pt}%
\definecolor{currentstroke}{rgb}{0.870588,0.733333,0.607843}%
\pgfsetstrokecolor{currentstroke}%
\pgfsetdash{}{0pt}%
\pgfsys@defobject{currentmarker}{\pgfqpoint{-0.041667in}{-0.041667in}}{\pgfqpoint{0.041667in}{0.041667in}}{%
\pgfpathmoveto{\pgfqpoint{0.000000in}{-0.041667in}}%
\pgfpathcurveto{\pgfqpoint{0.011050in}{-0.041667in}}{\pgfqpoint{0.021649in}{-0.037276in}}{\pgfqpoint{0.029463in}{-0.029463in}}%
\pgfpathcurveto{\pgfqpoint{0.037276in}{-0.021649in}}{\pgfqpoint{0.041667in}{-0.011050in}}{\pgfqpoint{0.041667in}{0.000000in}}%
\pgfpathcurveto{\pgfqpoint{0.041667in}{0.011050in}}{\pgfqpoint{0.037276in}{0.021649in}}{\pgfqpoint{0.029463in}{0.029463in}}%
\pgfpathcurveto{\pgfqpoint{0.021649in}{0.037276in}}{\pgfqpoint{0.011050in}{0.041667in}}{\pgfqpoint{0.000000in}{0.041667in}}%
\pgfpathcurveto{\pgfqpoint{-0.011050in}{0.041667in}}{\pgfqpoint{-0.021649in}{0.037276in}}{\pgfqpoint{-0.029463in}{0.029463in}}%
\pgfpathcurveto{\pgfqpoint{-0.037276in}{0.021649in}}{\pgfqpoint{-0.041667in}{0.011050in}}{\pgfqpoint{-0.041667in}{0.000000in}}%
\pgfpathcurveto{\pgfqpoint{-0.041667in}{-0.011050in}}{\pgfqpoint{-0.037276in}{-0.021649in}}{\pgfqpoint{-0.029463in}{-0.029463in}}%
\pgfpathcurveto{\pgfqpoint{-0.021649in}{-0.037276in}}{\pgfqpoint{-0.011050in}{-0.041667in}}{\pgfqpoint{0.000000in}{-0.041667in}}%
\pgfpathclose%
\pgfusepath{stroke,fill}%
}%
\begin{pgfscope}%
\pgfsys@transformshift{10.090016in}{3.873991in}%
\pgfsys@useobject{currentmarker}{}%
\end{pgfscope}%
\end{pgfscope}%
\begin{pgfscope}%
\definecolor{textcolor}{rgb}{0.000000,0.000000,0.000000}%
\pgfsetstrokecolor{textcolor}%
\pgfsetfillcolor{textcolor}%
\pgftext[x=10.340016in,y=3.837533in,left,base]{\color{textcolor}\sffamily\fontsize{10.000000}{12.000000}\selectfont 3dfront}%
\end{pgfscope}%
\begin{pgfscope}%
\pgfsetbuttcap%
\pgfsetroundjoin%
\definecolor{currentfill}{rgb}{0.980392,0.690196,0.894118}%
\pgfsetfillcolor{currentfill}%
\pgfsetlinewidth{1.003750pt}%
\definecolor{currentstroke}{rgb}{0.980392,0.690196,0.894118}%
\pgfsetstrokecolor{currentstroke}%
\pgfsetdash{}{0pt}%
\pgfsys@defobject{currentmarker}{\pgfqpoint{-0.041667in}{-0.041667in}}{\pgfqpoint{0.041667in}{0.041667in}}{%
\pgfpathmoveto{\pgfqpoint{0.000000in}{-0.041667in}}%
\pgfpathcurveto{\pgfqpoint{0.011050in}{-0.041667in}}{\pgfqpoint{0.021649in}{-0.037276in}}{\pgfqpoint{0.029463in}{-0.029463in}}%
\pgfpathcurveto{\pgfqpoint{0.037276in}{-0.021649in}}{\pgfqpoint{0.041667in}{-0.011050in}}{\pgfqpoint{0.041667in}{0.000000in}}%
\pgfpathcurveto{\pgfqpoint{0.041667in}{0.011050in}}{\pgfqpoint{0.037276in}{0.021649in}}{\pgfqpoint{0.029463in}{0.029463in}}%
\pgfpathcurveto{\pgfqpoint{0.021649in}{0.037276in}}{\pgfqpoint{0.011050in}{0.041667in}}{\pgfqpoint{0.000000in}{0.041667in}}%
\pgfpathcurveto{\pgfqpoint{-0.011050in}{0.041667in}}{\pgfqpoint{-0.021649in}{0.037276in}}{\pgfqpoint{-0.029463in}{0.029463in}}%
\pgfpathcurveto{\pgfqpoint{-0.037276in}{0.021649in}}{\pgfqpoint{-0.041667in}{0.011050in}}{\pgfqpoint{-0.041667in}{0.000000in}}%
\pgfpathcurveto{\pgfqpoint{-0.041667in}{-0.011050in}}{\pgfqpoint{-0.037276in}{-0.021649in}}{\pgfqpoint{-0.029463in}{-0.029463in}}%
\pgfpathcurveto{\pgfqpoint{-0.021649in}{-0.037276in}}{\pgfqpoint{-0.011050in}{-0.041667in}}{\pgfqpoint{0.000000in}{-0.041667in}}%
\pgfpathclose%
\pgfusepath{stroke,fill}%
}%
\begin{pgfscope}%
\pgfsys@transformshift{10.090016in}{3.670134in}%
\pgfsys@useobject{currentmarker}{}%
\end{pgfscope}%
\end{pgfscope}%
\begin{pgfscope}%
\definecolor{textcolor}{rgb}{0.000000,0.000000,0.000000}%
\pgfsetstrokecolor{textcolor}%
\pgfsetfillcolor{textcolor}%
\pgftext[x=10.340016in,y=3.633675in,left,base]{\color{textcolor}\sffamily\fontsize{10.000000}{12.000000}\selectfont s2r-3dfree}%
\end{pgfscope}%
\begin{pgfscope}%
\pgfsetbuttcap%
\pgfsetroundjoin%
\definecolor{currentfill}{rgb}{0.721569,0.521569,0.039216}%
\pgfsetfillcolor{currentfill}%
\pgfsetlinewidth{1.003750pt}%
\definecolor{currentstroke}{rgb}{0.721569,0.521569,0.039216}%
\pgfsetstrokecolor{currentstroke}%
\pgfsetdash{}{0pt}%
\pgfsys@defobject{currentmarker}{\pgfqpoint{-0.041667in}{-0.041667in}}{\pgfqpoint{0.041667in}{0.041667in}}{%
\pgfpathmoveto{\pgfqpoint{0.000000in}{-0.041667in}}%
\pgfpathcurveto{\pgfqpoint{0.011050in}{-0.041667in}}{\pgfqpoint{0.021649in}{-0.037276in}}{\pgfqpoint{0.029463in}{-0.029463in}}%
\pgfpathcurveto{\pgfqpoint{0.037276in}{-0.021649in}}{\pgfqpoint{0.041667in}{-0.011050in}}{\pgfqpoint{0.041667in}{0.000000in}}%
\pgfpathcurveto{\pgfqpoint{0.041667in}{0.011050in}}{\pgfqpoint{0.037276in}{0.021649in}}{\pgfqpoint{0.029463in}{0.029463in}}%
\pgfpathcurveto{\pgfqpoint{0.021649in}{0.037276in}}{\pgfqpoint{0.011050in}{0.041667in}}{\pgfqpoint{0.000000in}{0.041667in}}%
\pgfpathcurveto{\pgfqpoint{-0.011050in}{0.041667in}}{\pgfqpoint{-0.021649in}{0.037276in}}{\pgfqpoint{-0.029463in}{0.029463in}}%
\pgfpathcurveto{\pgfqpoint{-0.037276in}{0.021649in}}{\pgfqpoint{-0.041667in}{0.011050in}}{\pgfqpoint{-0.041667in}{0.000000in}}%
\pgfpathcurveto{\pgfqpoint{-0.041667in}{-0.011050in}}{\pgfqpoint{-0.037276in}{-0.021649in}}{\pgfqpoint{-0.029463in}{-0.029463in}}%
\pgfpathcurveto{\pgfqpoint{-0.021649in}{-0.037276in}}{\pgfqpoint{-0.011050in}{-0.041667in}}{\pgfqpoint{0.000000in}{-0.041667in}}%
\pgfpathclose%
\pgfusepath{stroke,fill}%
}%
\begin{pgfscope}%
\pgfsys@transformshift{10.090016in}{3.466276in}%
\pgfsys@useobject{currentmarker}{}%
\end{pgfscope}%
\end{pgfscope}%
\begin{pgfscope}%
\definecolor{textcolor}{rgb}{0.000000,0.000000,0.000000}%
\pgfsetstrokecolor{textcolor}%
\pgfsetfillcolor{textcolor}%
\pgftext[x=10.340016in,y=3.429818in,left,base]{\color{textcolor}\sffamily\fontsize{10.000000}{12.000000}\selectfont pix3d}%
\end{pgfscope}%
\end{pgfpicture}%
\makeatother%
\endgroup%
}
    \caption{T-SNE visualisation for images from various photo-realistic synthetic dataset. Pix3d and \gls{free} is highlighted with bolder colors.}
    \label{fig:photorealistic tsne}
\end{figure}

\begin{figure}
    \centering
    \resizebox{\textwidth}{!}{%% Creator: Matplotlib, PGF backend
%%
%% To include the figure in your LaTeX document, write
%%   \input{<filename>.pgf}
%%
%% Make sure the required packages are loaded in your preamble
%%   \usepackage{pgf}
%%
%% Figures using additional raster images can only be included by \input if
%% they are in the same directory as the main LaTeX file. For loading figures
%% from other directories you can use the `import` package
%%   \usepackage{import}
%%
%% and then include the figures with
%%   \import{<path to file>}{<filename>.pgf}
%%
%% Matplotlib used the following preamble
%%   \usepackage{fontspec}
%%   \setmainfont{DejaVuSerif.ttf}[Path=\detokenize{/Users/apple/opt/anaconda3/envs/kaolin/lib/python3.7/site-packages/matplotlib/mpl-data/fonts/ttf/}]
%%   \setsansfont{DejaVuSans.ttf}[Path=\detokenize{/Users/apple/opt/anaconda3/envs/kaolin/lib/python3.7/site-packages/matplotlib/mpl-data/fonts/ttf/}]
%%   \setmonofont{DejaVuSansMono.ttf}[Path=\detokenize{/Users/apple/opt/anaconda3/envs/kaolin/lib/python3.7/site-packages/matplotlib/mpl-data/fonts/ttf/}]
%%
\begingroup%
\makeatletter%
\begin{pgfpicture}%
\pgfpathrectangle{\pgfpointorigin}{\pgfqpoint{11.314188in}{8.341596in}}%
\pgfusepath{use as bounding box, clip}%
\begin{pgfscope}%
\pgfsetbuttcap%
\pgfsetmiterjoin%
\definecolor{currentfill}{rgb}{1.000000,1.000000,1.000000}%
\pgfsetfillcolor{currentfill}%
\pgfsetlinewidth{0.000000pt}%
\definecolor{currentstroke}{rgb}{1.000000,1.000000,1.000000}%
\pgfsetstrokecolor{currentstroke}%
\pgfsetdash{}{0pt}%
\pgfpathmoveto{\pgfqpoint{0.000000in}{0.000000in}}%
\pgfpathlineto{\pgfqpoint{11.314188in}{0.000000in}}%
\pgfpathlineto{\pgfqpoint{11.314188in}{8.341596in}}%
\pgfpathlineto{\pgfqpoint{0.000000in}{8.341596in}}%
\pgfpathclose%
\pgfusepath{fill}%
\end{pgfscope}%
\begin{pgfscope}%
\pgfsetbuttcap%
\pgfsetmiterjoin%
\definecolor{currentfill}{rgb}{1.000000,1.000000,1.000000}%
\pgfsetfillcolor{currentfill}%
\pgfsetlinewidth{0.000000pt}%
\definecolor{currentstroke}{rgb}{0.000000,0.000000,0.000000}%
\pgfsetstrokecolor{currentstroke}%
\pgfsetstrokeopacity{0.000000}%
\pgfsetdash{}{0pt}%
\pgfpathmoveto{\pgfqpoint{0.481978in}{0.331635in}}%
\pgfpathlineto{\pgfqpoint{9.781978in}{0.331635in}}%
\pgfpathlineto{\pgfqpoint{9.781978in}{8.031635in}}%
\pgfpathlineto{\pgfqpoint{0.481978in}{8.031635in}}%
\pgfpathclose%
\pgfusepath{fill}%
\end{pgfscope}%
\begin{pgfscope}%
\pgfpathrectangle{\pgfqpoint{0.481978in}{0.331635in}}{\pgfqpoint{9.300000in}{7.700000in}}%
\pgfusepath{clip}%
\pgfsetbuttcap%
\pgfsetroundjoin%
\definecolor{currentfill}{rgb}{0.631373,0.788235,0.956863}%
\pgfsetfillcolor{currentfill}%
\pgfsetlinewidth{0.481800pt}%
\definecolor{currentstroke}{rgb}{1.000000,1.000000,1.000000}%
\pgfsetstrokecolor{currentstroke}%
\pgfsetdash{}{0pt}%
\pgfpathmoveto{\pgfqpoint{5.562292in}{2.004512in}}%
\pgfpathcurveto{\pgfqpoint{5.573342in}{2.004512in}}{\pgfqpoint{5.583941in}{2.008902in}}{\pgfqpoint{5.591755in}{2.016716in}}%
\pgfpathcurveto{\pgfqpoint{5.599569in}{2.024529in}}{\pgfqpoint{5.603959in}{2.035128in}}{\pgfqpoint{5.603959in}{2.046179in}}%
\pgfpathcurveto{\pgfqpoint{5.603959in}{2.057229in}}{\pgfqpoint{5.599569in}{2.067828in}}{\pgfqpoint{5.591755in}{2.075641in}}%
\pgfpathcurveto{\pgfqpoint{5.583941in}{2.083455in}}{\pgfqpoint{5.573342in}{2.087845in}}{\pgfqpoint{5.562292in}{2.087845in}}%
\pgfpathcurveto{\pgfqpoint{5.551242in}{2.087845in}}{\pgfqpoint{5.540643in}{2.083455in}}{\pgfqpoint{5.532829in}{2.075641in}}%
\pgfpathcurveto{\pgfqpoint{5.525016in}{2.067828in}}{\pgfqpoint{5.520625in}{2.057229in}}{\pgfqpoint{5.520625in}{2.046179in}}%
\pgfpathcurveto{\pgfqpoint{5.520625in}{2.035128in}}{\pgfqpoint{5.525016in}{2.024529in}}{\pgfqpoint{5.532829in}{2.016716in}}%
\pgfpathcurveto{\pgfqpoint{5.540643in}{2.008902in}}{\pgfqpoint{5.551242in}{2.004512in}}{\pgfqpoint{5.562292in}{2.004512in}}%
\pgfpathclose%
\pgfusepath{stroke,fill}%
\end{pgfscope}%
\begin{pgfscope}%
\pgfpathrectangle{\pgfqpoint{0.481978in}{0.331635in}}{\pgfqpoint{9.300000in}{7.700000in}}%
\pgfusepath{clip}%
\pgfsetbuttcap%
\pgfsetroundjoin%
\definecolor{currentfill}{rgb}{0.631373,0.788235,0.956863}%
\pgfsetfillcolor{currentfill}%
\pgfsetlinewidth{0.481800pt}%
\definecolor{currentstroke}{rgb}{1.000000,1.000000,1.000000}%
\pgfsetstrokecolor{currentstroke}%
\pgfsetdash{}{0pt}%
\pgfpathmoveto{\pgfqpoint{6.909912in}{4.418435in}}%
\pgfpathcurveto{\pgfqpoint{6.920962in}{4.418435in}}{\pgfqpoint{6.931561in}{4.422826in}}{\pgfqpoint{6.939375in}{4.430639in}}%
\pgfpathcurveto{\pgfqpoint{6.947188in}{4.438453in}}{\pgfqpoint{6.951578in}{4.449052in}}{\pgfqpoint{6.951578in}{4.460102in}}%
\pgfpathcurveto{\pgfqpoint{6.951578in}{4.471152in}}{\pgfqpoint{6.947188in}{4.481751in}}{\pgfqpoint{6.939375in}{4.489565in}}%
\pgfpathcurveto{\pgfqpoint{6.931561in}{4.497378in}}{\pgfqpoint{6.920962in}{4.501769in}}{\pgfqpoint{6.909912in}{4.501769in}}%
\pgfpathcurveto{\pgfqpoint{6.898862in}{4.501769in}}{\pgfqpoint{6.888263in}{4.497378in}}{\pgfqpoint{6.880449in}{4.489565in}}%
\pgfpathcurveto{\pgfqpoint{6.872635in}{4.481751in}}{\pgfqpoint{6.868245in}{4.471152in}}{\pgfqpoint{6.868245in}{4.460102in}}%
\pgfpathcurveto{\pgfqpoint{6.868245in}{4.449052in}}{\pgfqpoint{6.872635in}{4.438453in}}{\pgfqpoint{6.880449in}{4.430639in}}%
\pgfpathcurveto{\pgfqpoint{6.888263in}{4.422826in}}{\pgfqpoint{6.898862in}{4.418435in}}{\pgfqpoint{6.909912in}{4.418435in}}%
\pgfpathclose%
\pgfusepath{stroke,fill}%
\end{pgfscope}%
\begin{pgfscope}%
\pgfpathrectangle{\pgfqpoint{0.481978in}{0.331635in}}{\pgfqpoint{9.300000in}{7.700000in}}%
\pgfusepath{clip}%
\pgfsetbuttcap%
\pgfsetroundjoin%
\definecolor{currentfill}{rgb}{0.631373,0.788235,0.956863}%
\pgfsetfillcolor{currentfill}%
\pgfsetlinewidth{0.481800pt}%
\definecolor{currentstroke}{rgb}{1.000000,1.000000,1.000000}%
\pgfsetstrokecolor{currentstroke}%
\pgfsetdash{}{0pt}%
\pgfpathmoveto{\pgfqpoint{7.565874in}{5.841434in}}%
\pgfpathcurveto{\pgfqpoint{7.576924in}{5.841434in}}{\pgfqpoint{7.587523in}{5.845824in}}{\pgfqpoint{7.595337in}{5.853638in}}%
\pgfpathcurveto{\pgfqpoint{7.603150in}{5.861451in}}{\pgfqpoint{7.607541in}{5.872050in}}{\pgfqpoint{7.607541in}{5.883101in}}%
\pgfpathcurveto{\pgfqpoint{7.607541in}{5.894151in}}{\pgfqpoint{7.603150in}{5.904750in}}{\pgfqpoint{7.595337in}{5.912563in}}%
\pgfpathcurveto{\pgfqpoint{7.587523in}{5.920377in}}{\pgfqpoint{7.576924in}{5.924767in}}{\pgfqpoint{7.565874in}{5.924767in}}%
\pgfpathcurveto{\pgfqpoint{7.554824in}{5.924767in}}{\pgfqpoint{7.544225in}{5.920377in}}{\pgfqpoint{7.536411in}{5.912563in}}%
\pgfpathcurveto{\pgfqpoint{7.528598in}{5.904750in}}{\pgfqpoint{7.524207in}{5.894151in}}{\pgfqpoint{7.524207in}{5.883101in}}%
\pgfpathcurveto{\pgfqpoint{7.524207in}{5.872050in}}{\pgfqpoint{7.528598in}{5.861451in}}{\pgfqpoint{7.536411in}{5.853638in}}%
\pgfpathcurveto{\pgfqpoint{7.544225in}{5.845824in}}{\pgfqpoint{7.554824in}{5.841434in}}{\pgfqpoint{7.565874in}{5.841434in}}%
\pgfpathclose%
\pgfusepath{stroke,fill}%
\end{pgfscope}%
\begin{pgfscope}%
\pgfpathrectangle{\pgfqpoint{0.481978in}{0.331635in}}{\pgfqpoint{9.300000in}{7.700000in}}%
\pgfusepath{clip}%
\pgfsetbuttcap%
\pgfsetroundjoin%
\definecolor{currentfill}{rgb}{0.631373,0.788235,0.956863}%
\pgfsetfillcolor{currentfill}%
\pgfsetlinewidth{0.481800pt}%
\definecolor{currentstroke}{rgb}{1.000000,1.000000,1.000000}%
\pgfsetstrokecolor{currentstroke}%
\pgfsetdash{}{0pt}%
\pgfpathmoveto{\pgfqpoint{8.104384in}{5.147465in}}%
\pgfpathcurveto{\pgfqpoint{8.115434in}{5.147465in}}{\pgfqpoint{8.126033in}{5.151855in}}{\pgfqpoint{8.133846in}{5.159669in}}%
\pgfpathcurveto{\pgfqpoint{8.141660in}{5.167482in}}{\pgfqpoint{8.146050in}{5.178081in}}{\pgfqpoint{8.146050in}{5.189132in}}%
\pgfpathcurveto{\pgfqpoint{8.146050in}{5.200182in}}{\pgfqpoint{8.141660in}{5.210781in}}{\pgfqpoint{8.133846in}{5.218594in}}%
\pgfpathcurveto{\pgfqpoint{8.126033in}{5.226408in}}{\pgfqpoint{8.115434in}{5.230798in}}{\pgfqpoint{8.104384in}{5.230798in}}%
\pgfpathcurveto{\pgfqpoint{8.093334in}{5.230798in}}{\pgfqpoint{8.082735in}{5.226408in}}{\pgfqpoint{8.074921in}{5.218594in}}%
\pgfpathcurveto{\pgfqpoint{8.067107in}{5.210781in}}{\pgfqpoint{8.062717in}{5.200182in}}{\pgfqpoint{8.062717in}{5.189132in}}%
\pgfpathcurveto{\pgfqpoint{8.062717in}{5.178081in}}{\pgfqpoint{8.067107in}{5.167482in}}{\pgfqpoint{8.074921in}{5.159669in}}%
\pgfpathcurveto{\pgfqpoint{8.082735in}{5.151855in}}{\pgfqpoint{8.093334in}{5.147465in}}{\pgfqpoint{8.104384in}{5.147465in}}%
\pgfpathclose%
\pgfusepath{stroke,fill}%
\end{pgfscope}%
\begin{pgfscope}%
\pgfpathrectangle{\pgfqpoint{0.481978in}{0.331635in}}{\pgfqpoint{9.300000in}{7.700000in}}%
\pgfusepath{clip}%
\pgfsetbuttcap%
\pgfsetroundjoin%
\definecolor{currentfill}{rgb}{0.631373,0.788235,0.956863}%
\pgfsetfillcolor{currentfill}%
\pgfsetlinewidth{0.481800pt}%
\definecolor{currentstroke}{rgb}{1.000000,1.000000,1.000000}%
\pgfsetstrokecolor{currentstroke}%
\pgfsetdash{}{0pt}%
\pgfpathmoveto{\pgfqpoint{7.967295in}{4.784121in}}%
\pgfpathcurveto{\pgfqpoint{7.978345in}{4.784121in}}{\pgfqpoint{7.988944in}{4.788511in}}{\pgfqpoint{7.996758in}{4.796325in}}%
\pgfpathcurveto{\pgfqpoint{8.004571in}{4.804138in}}{\pgfqpoint{8.008962in}{4.814737in}}{\pgfqpoint{8.008962in}{4.825787in}}%
\pgfpathcurveto{\pgfqpoint{8.008962in}{4.836837in}}{\pgfqpoint{8.004571in}{4.847437in}}{\pgfqpoint{7.996758in}{4.855250in}}%
\pgfpathcurveto{\pgfqpoint{7.988944in}{4.863064in}}{\pgfqpoint{7.978345in}{4.867454in}}{\pgfqpoint{7.967295in}{4.867454in}}%
\pgfpathcurveto{\pgfqpoint{7.956245in}{4.867454in}}{\pgfqpoint{7.945646in}{4.863064in}}{\pgfqpoint{7.937832in}{4.855250in}}%
\pgfpathcurveto{\pgfqpoint{7.930019in}{4.847437in}}{\pgfqpoint{7.925628in}{4.836837in}}{\pgfqpoint{7.925628in}{4.825787in}}%
\pgfpathcurveto{\pgfqpoint{7.925628in}{4.814737in}}{\pgfqpoint{7.930019in}{4.804138in}}{\pgfqpoint{7.937832in}{4.796325in}}%
\pgfpathcurveto{\pgfqpoint{7.945646in}{4.788511in}}{\pgfqpoint{7.956245in}{4.784121in}}{\pgfqpoint{7.967295in}{4.784121in}}%
\pgfpathclose%
\pgfusepath{stroke,fill}%
\end{pgfscope}%
\begin{pgfscope}%
\pgfpathrectangle{\pgfqpoint{0.481978in}{0.331635in}}{\pgfqpoint{9.300000in}{7.700000in}}%
\pgfusepath{clip}%
\pgfsetbuttcap%
\pgfsetroundjoin%
\definecolor{currentfill}{rgb}{0.631373,0.788235,0.956863}%
\pgfsetfillcolor{currentfill}%
\pgfsetlinewidth{0.481800pt}%
\definecolor{currentstroke}{rgb}{1.000000,1.000000,1.000000}%
\pgfsetstrokecolor{currentstroke}%
\pgfsetdash{}{0pt}%
\pgfpathmoveto{\pgfqpoint{5.184843in}{3.663871in}}%
\pgfpathcurveto{\pgfqpoint{5.195893in}{3.663871in}}{\pgfqpoint{5.206492in}{3.668261in}}{\pgfqpoint{5.214305in}{3.676074in}}%
\pgfpathcurveto{\pgfqpoint{5.222119in}{3.683888in}}{\pgfqpoint{5.226509in}{3.694487in}}{\pgfqpoint{5.226509in}{3.705537in}}%
\pgfpathcurveto{\pgfqpoint{5.226509in}{3.716587in}}{\pgfqpoint{5.222119in}{3.727186in}}{\pgfqpoint{5.214305in}{3.735000in}}%
\pgfpathcurveto{\pgfqpoint{5.206492in}{3.742814in}}{\pgfqpoint{5.195893in}{3.747204in}}{\pgfqpoint{5.184843in}{3.747204in}}%
\pgfpathcurveto{\pgfqpoint{5.173793in}{3.747204in}}{\pgfqpoint{5.163194in}{3.742814in}}{\pgfqpoint{5.155380in}{3.735000in}}%
\pgfpathcurveto{\pgfqpoint{5.147566in}{3.727186in}}{\pgfqpoint{5.143176in}{3.716587in}}{\pgfqpoint{5.143176in}{3.705537in}}%
\pgfpathcurveto{\pgfqpoint{5.143176in}{3.694487in}}{\pgfqpoint{5.147566in}{3.683888in}}{\pgfqpoint{5.155380in}{3.676074in}}%
\pgfpathcurveto{\pgfqpoint{5.163194in}{3.668261in}}{\pgfqpoint{5.173793in}{3.663871in}}{\pgfqpoint{5.184843in}{3.663871in}}%
\pgfpathclose%
\pgfusepath{stroke,fill}%
\end{pgfscope}%
\begin{pgfscope}%
\pgfpathrectangle{\pgfqpoint{0.481978in}{0.331635in}}{\pgfqpoint{9.300000in}{7.700000in}}%
\pgfusepath{clip}%
\pgfsetbuttcap%
\pgfsetroundjoin%
\definecolor{currentfill}{rgb}{0.631373,0.788235,0.956863}%
\pgfsetfillcolor{currentfill}%
\pgfsetlinewidth{0.481800pt}%
\definecolor{currentstroke}{rgb}{1.000000,1.000000,1.000000}%
\pgfsetstrokecolor{currentstroke}%
\pgfsetdash{}{0pt}%
\pgfpathmoveto{\pgfqpoint{2.850031in}{2.106168in}}%
\pgfpathcurveto{\pgfqpoint{2.861081in}{2.106168in}}{\pgfqpoint{2.871680in}{2.110558in}}{\pgfqpoint{2.879493in}{2.118372in}}%
\pgfpathcurveto{\pgfqpoint{2.887307in}{2.126186in}}{\pgfqpoint{2.891697in}{2.136785in}}{\pgfqpoint{2.891697in}{2.147835in}}%
\pgfpathcurveto{\pgfqpoint{2.891697in}{2.158885in}}{\pgfqpoint{2.887307in}{2.169484in}}{\pgfqpoint{2.879493in}{2.177298in}}%
\pgfpathcurveto{\pgfqpoint{2.871680in}{2.185111in}}{\pgfqpoint{2.861081in}{2.189501in}}{\pgfqpoint{2.850031in}{2.189501in}}%
\pgfpathcurveto{\pgfqpoint{2.838980in}{2.189501in}}{\pgfqpoint{2.828381in}{2.185111in}}{\pgfqpoint{2.820568in}{2.177298in}}%
\pgfpathcurveto{\pgfqpoint{2.812754in}{2.169484in}}{\pgfqpoint{2.808364in}{2.158885in}}{\pgfqpoint{2.808364in}{2.147835in}}%
\pgfpathcurveto{\pgfqpoint{2.808364in}{2.136785in}}{\pgfqpoint{2.812754in}{2.126186in}}{\pgfqpoint{2.820568in}{2.118372in}}%
\pgfpathcurveto{\pgfqpoint{2.828381in}{2.110558in}}{\pgfqpoint{2.838980in}{2.106168in}}{\pgfqpoint{2.850031in}{2.106168in}}%
\pgfpathclose%
\pgfusepath{stroke,fill}%
\end{pgfscope}%
\begin{pgfscope}%
\pgfpathrectangle{\pgfqpoint{0.481978in}{0.331635in}}{\pgfqpoint{9.300000in}{7.700000in}}%
\pgfusepath{clip}%
\pgfsetbuttcap%
\pgfsetroundjoin%
\definecolor{currentfill}{rgb}{0.631373,0.788235,0.956863}%
\pgfsetfillcolor{currentfill}%
\pgfsetlinewidth{0.481800pt}%
\definecolor{currentstroke}{rgb}{1.000000,1.000000,1.000000}%
\pgfsetstrokecolor{currentstroke}%
\pgfsetdash{}{0pt}%
\pgfpathmoveto{\pgfqpoint{5.765949in}{2.898526in}}%
\pgfpathcurveto{\pgfqpoint{5.777000in}{2.898526in}}{\pgfqpoint{5.787599in}{2.902916in}}{\pgfqpoint{5.795412in}{2.910730in}}%
\pgfpathcurveto{\pgfqpoint{5.803226in}{2.918544in}}{\pgfqpoint{5.807616in}{2.929143in}}{\pgfqpoint{5.807616in}{2.940193in}}%
\pgfpathcurveto{\pgfqpoint{5.807616in}{2.951243in}}{\pgfqpoint{5.803226in}{2.961842in}}{\pgfqpoint{5.795412in}{2.969656in}}%
\pgfpathcurveto{\pgfqpoint{5.787599in}{2.977469in}}{\pgfqpoint{5.777000in}{2.981860in}}{\pgfqpoint{5.765949in}{2.981860in}}%
\pgfpathcurveto{\pgfqpoint{5.754899in}{2.981860in}}{\pgfqpoint{5.744300in}{2.977469in}}{\pgfqpoint{5.736487in}{2.969656in}}%
\pgfpathcurveto{\pgfqpoint{5.728673in}{2.961842in}}{\pgfqpoint{5.724283in}{2.951243in}}{\pgfqpoint{5.724283in}{2.940193in}}%
\pgfpathcurveto{\pgfqpoint{5.724283in}{2.929143in}}{\pgfqpoint{5.728673in}{2.918544in}}{\pgfqpoint{5.736487in}{2.910730in}}%
\pgfpathcurveto{\pgfqpoint{5.744300in}{2.902916in}}{\pgfqpoint{5.754899in}{2.898526in}}{\pgfqpoint{5.765949in}{2.898526in}}%
\pgfpathclose%
\pgfusepath{stroke,fill}%
\end{pgfscope}%
\begin{pgfscope}%
\pgfpathrectangle{\pgfqpoint{0.481978in}{0.331635in}}{\pgfqpoint{9.300000in}{7.700000in}}%
\pgfusepath{clip}%
\pgfsetbuttcap%
\pgfsetroundjoin%
\definecolor{currentfill}{rgb}{0.631373,0.788235,0.956863}%
\pgfsetfillcolor{currentfill}%
\pgfsetlinewidth{0.481800pt}%
\definecolor{currentstroke}{rgb}{1.000000,1.000000,1.000000}%
\pgfsetstrokecolor{currentstroke}%
\pgfsetdash{}{0pt}%
\pgfpathmoveto{\pgfqpoint{6.395914in}{3.188558in}}%
\pgfpathcurveto{\pgfqpoint{6.406964in}{3.188558in}}{\pgfqpoint{6.417563in}{3.192948in}}{\pgfqpoint{6.425377in}{3.200762in}}%
\pgfpathcurveto{\pgfqpoint{6.433191in}{3.208576in}}{\pgfqpoint{6.437581in}{3.219175in}}{\pgfqpoint{6.437581in}{3.230225in}}%
\pgfpathcurveto{\pgfqpoint{6.437581in}{3.241275in}}{\pgfqpoint{6.433191in}{3.251874in}}{\pgfqpoint{6.425377in}{3.259688in}}%
\pgfpathcurveto{\pgfqpoint{6.417563in}{3.267501in}}{\pgfqpoint{6.406964in}{3.271892in}}{\pgfqpoint{6.395914in}{3.271892in}}%
\pgfpathcurveto{\pgfqpoint{6.384864in}{3.271892in}}{\pgfqpoint{6.374265in}{3.267501in}}{\pgfqpoint{6.366451in}{3.259688in}}%
\pgfpathcurveto{\pgfqpoint{6.358638in}{3.251874in}}{\pgfqpoint{6.354248in}{3.241275in}}{\pgfqpoint{6.354248in}{3.230225in}}%
\pgfpathcurveto{\pgfqpoint{6.354248in}{3.219175in}}{\pgfqpoint{6.358638in}{3.208576in}}{\pgfqpoint{6.366451in}{3.200762in}}%
\pgfpathcurveto{\pgfqpoint{6.374265in}{3.192948in}}{\pgfqpoint{6.384864in}{3.188558in}}{\pgfqpoint{6.395914in}{3.188558in}}%
\pgfpathclose%
\pgfusepath{stroke,fill}%
\end{pgfscope}%
\begin{pgfscope}%
\pgfpathrectangle{\pgfqpoint{0.481978in}{0.331635in}}{\pgfqpoint{9.300000in}{7.700000in}}%
\pgfusepath{clip}%
\pgfsetbuttcap%
\pgfsetroundjoin%
\definecolor{currentfill}{rgb}{0.631373,0.788235,0.956863}%
\pgfsetfillcolor{currentfill}%
\pgfsetlinewidth{0.481800pt}%
\definecolor{currentstroke}{rgb}{1.000000,1.000000,1.000000}%
\pgfsetstrokecolor{currentstroke}%
\pgfsetdash{}{0pt}%
\pgfpathmoveto{\pgfqpoint{4.098452in}{6.229781in}}%
\pgfpathcurveto{\pgfqpoint{4.109502in}{6.229781in}}{\pgfqpoint{4.120101in}{6.234172in}}{\pgfqpoint{4.127915in}{6.241985in}}%
\pgfpathcurveto{\pgfqpoint{4.135729in}{6.249799in}}{\pgfqpoint{4.140119in}{6.260398in}}{\pgfqpoint{4.140119in}{6.271448in}}%
\pgfpathcurveto{\pgfqpoint{4.140119in}{6.282498in}}{\pgfqpoint{4.135729in}{6.293097in}}{\pgfqpoint{4.127915in}{6.300911in}}%
\pgfpathcurveto{\pgfqpoint{4.120101in}{6.308725in}}{\pgfqpoint{4.109502in}{6.313115in}}{\pgfqpoint{4.098452in}{6.313115in}}%
\pgfpathcurveto{\pgfqpoint{4.087402in}{6.313115in}}{\pgfqpoint{4.076803in}{6.308725in}}{\pgfqpoint{4.068989in}{6.300911in}}%
\pgfpathcurveto{\pgfqpoint{4.061176in}{6.293097in}}{\pgfqpoint{4.056785in}{6.282498in}}{\pgfqpoint{4.056785in}{6.271448in}}%
\pgfpathcurveto{\pgfqpoint{4.056785in}{6.260398in}}{\pgfqpoint{4.061176in}{6.249799in}}{\pgfqpoint{4.068989in}{6.241985in}}%
\pgfpathcurveto{\pgfqpoint{4.076803in}{6.234172in}}{\pgfqpoint{4.087402in}{6.229781in}}{\pgfqpoint{4.098452in}{6.229781in}}%
\pgfpathclose%
\pgfusepath{stroke,fill}%
\end{pgfscope}%
\begin{pgfscope}%
\pgfpathrectangle{\pgfqpoint{0.481978in}{0.331635in}}{\pgfqpoint{9.300000in}{7.700000in}}%
\pgfusepath{clip}%
\pgfsetbuttcap%
\pgfsetroundjoin%
\definecolor{currentfill}{rgb}{0.631373,0.788235,0.956863}%
\pgfsetfillcolor{currentfill}%
\pgfsetlinewidth{0.481800pt}%
\definecolor{currentstroke}{rgb}{1.000000,1.000000,1.000000}%
\pgfsetstrokecolor{currentstroke}%
\pgfsetdash{}{0pt}%
\pgfpathmoveto{\pgfqpoint{6.735020in}{4.467203in}}%
\pgfpathcurveto{\pgfqpoint{6.746071in}{4.467203in}}{\pgfqpoint{6.756670in}{4.471594in}}{\pgfqpoint{6.764483in}{4.479407in}}%
\pgfpathcurveto{\pgfqpoint{6.772297in}{4.487221in}}{\pgfqpoint{6.776687in}{4.497820in}}{\pgfqpoint{6.776687in}{4.508870in}}%
\pgfpathcurveto{\pgfqpoint{6.776687in}{4.519920in}}{\pgfqpoint{6.772297in}{4.530519in}}{\pgfqpoint{6.764483in}{4.538333in}}%
\pgfpathcurveto{\pgfqpoint{6.756670in}{4.546147in}}{\pgfqpoint{6.746071in}{4.550537in}}{\pgfqpoint{6.735020in}{4.550537in}}%
\pgfpathcurveto{\pgfqpoint{6.723970in}{4.550537in}}{\pgfqpoint{6.713371in}{4.546147in}}{\pgfqpoint{6.705558in}{4.538333in}}%
\pgfpathcurveto{\pgfqpoint{6.697744in}{4.530519in}}{\pgfqpoint{6.693354in}{4.519920in}}{\pgfqpoint{6.693354in}{4.508870in}}%
\pgfpathcurveto{\pgfqpoint{6.693354in}{4.497820in}}{\pgfqpoint{6.697744in}{4.487221in}}{\pgfqpoint{6.705558in}{4.479407in}}%
\pgfpathcurveto{\pgfqpoint{6.713371in}{4.471594in}}{\pgfqpoint{6.723970in}{4.467203in}}{\pgfqpoint{6.735020in}{4.467203in}}%
\pgfpathclose%
\pgfusepath{stroke,fill}%
\end{pgfscope}%
\begin{pgfscope}%
\pgfpathrectangle{\pgfqpoint{0.481978in}{0.331635in}}{\pgfqpoint{9.300000in}{7.700000in}}%
\pgfusepath{clip}%
\pgfsetbuttcap%
\pgfsetroundjoin%
\definecolor{currentfill}{rgb}{0.631373,0.788235,0.956863}%
\pgfsetfillcolor{currentfill}%
\pgfsetlinewidth{0.481800pt}%
\definecolor{currentstroke}{rgb}{1.000000,1.000000,1.000000}%
\pgfsetstrokecolor{currentstroke}%
\pgfsetdash{}{0pt}%
\pgfpathmoveto{\pgfqpoint{5.876608in}{2.487289in}}%
\pgfpathcurveto{\pgfqpoint{5.887658in}{2.487289in}}{\pgfqpoint{5.898257in}{2.491679in}}{\pgfqpoint{5.906071in}{2.499492in}}%
\pgfpathcurveto{\pgfqpoint{5.913885in}{2.507306in}}{\pgfqpoint{5.918275in}{2.517905in}}{\pgfqpoint{5.918275in}{2.528955in}}%
\pgfpathcurveto{\pgfqpoint{5.918275in}{2.540005in}}{\pgfqpoint{5.913885in}{2.550604in}}{\pgfqpoint{5.906071in}{2.558418in}}%
\pgfpathcurveto{\pgfqpoint{5.898257in}{2.566232in}}{\pgfqpoint{5.887658in}{2.570622in}}{\pgfqpoint{5.876608in}{2.570622in}}%
\pgfpathcurveto{\pgfqpoint{5.865558in}{2.570622in}}{\pgfqpoint{5.854959in}{2.566232in}}{\pgfqpoint{5.847146in}{2.558418in}}%
\pgfpathcurveto{\pgfqpoint{5.839332in}{2.550604in}}{\pgfqpoint{5.834942in}{2.540005in}}{\pgfqpoint{5.834942in}{2.528955in}}%
\pgfpathcurveto{\pgfqpoint{5.834942in}{2.517905in}}{\pgfqpoint{5.839332in}{2.507306in}}{\pgfqpoint{5.847146in}{2.499492in}}%
\pgfpathcurveto{\pgfqpoint{5.854959in}{2.491679in}}{\pgfqpoint{5.865558in}{2.487289in}}{\pgfqpoint{5.876608in}{2.487289in}}%
\pgfpathclose%
\pgfusepath{stroke,fill}%
\end{pgfscope}%
\begin{pgfscope}%
\pgfpathrectangle{\pgfqpoint{0.481978in}{0.331635in}}{\pgfqpoint{9.300000in}{7.700000in}}%
\pgfusepath{clip}%
\pgfsetbuttcap%
\pgfsetroundjoin%
\definecolor{currentfill}{rgb}{0.631373,0.788235,0.956863}%
\pgfsetfillcolor{currentfill}%
\pgfsetlinewidth{0.481800pt}%
\definecolor{currentstroke}{rgb}{1.000000,1.000000,1.000000}%
\pgfsetstrokecolor{currentstroke}%
\pgfsetdash{}{0pt}%
\pgfpathmoveto{\pgfqpoint{4.683187in}{4.383813in}}%
\pgfpathcurveto{\pgfqpoint{4.694237in}{4.383813in}}{\pgfqpoint{4.704837in}{4.388203in}}{\pgfqpoint{4.712650in}{4.396017in}}%
\pgfpathcurveto{\pgfqpoint{4.720464in}{4.403831in}}{\pgfqpoint{4.724854in}{4.414430in}}{\pgfqpoint{4.724854in}{4.425480in}}%
\pgfpathcurveto{\pgfqpoint{4.724854in}{4.436530in}}{\pgfqpoint{4.720464in}{4.447129in}}{\pgfqpoint{4.712650in}{4.454942in}}%
\pgfpathcurveto{\pgfqpoint{4.704837in}{4.462756in}}{\pgfqpoint{4.694237in}{4.467146in}}{\pgfqpoint{4.683187in}{4.467146in}}%
\pgfpathcurveto{\pgfqpoint{4.672137in}{4.467146in}}{\pgfqpoint{4.661538in}{4.462756in}}{\pgfqpoint{4.653725in}{4.454942in}}%
\pgfpathcurveto{\pgfqpoint{4.645911in}{4.447129in}}{\pgfqpoint{4.641521in}{4.436530in}}{\pgfqpoint{4.641521in}{4.425480in}}%
\pgfpathcurveto{\pgfqpoint{4.641521in}{4.414430in}}{\pgfqpoint{4.645911in}{4.403831in}}{\pgfqpoint{4.653725in}{4.396017in}}%
\pgfpathcurveto{\pgfqpoint{4.661538in}{4.388203in}}{\pgfqpoint{4.672137in}{4.383813in}}{\pgfqpoint{4.683187in}{4.383813in}}%
\pgfpathclose%
\pgfusepath{stroke,fill}%
\end{pgfscope}%
\begin{pgfscope}%
\pgfpathrectangle{\pgfqpoint{0.481978in}{0.331635in}}{\pgfqpoint{9.300000in}{7.700000in}}%
\pgfusepath{clip}%
\pgfsetbuttcap%
\pgfsetroundjoin%
\definecolor{currentfill}{rgb}{0.631373,0.788235,0.956863}%
\pgfsetfillcolor{currentfill}%
\pgfsetlinewidth{0.481800pt}%
\definecolor{currentstroke}{rgb}{1.000000,1.000000,1.000000}%
\pgfsetstrokecolor{currentstroke}%
\pgfsetdash{}{0pt}%
\pgfpathmoveto{\pgfqpoint{2.854341in}{1.999996in}}%
\pgfpathcurveto{\pgfqpoint{2.865391in}{1.999996in}}{\pgfqpoint{2.875990in}{2.004386in}}{\pgfqpoint{2.883804in}{2.012200in}}%
\pgfpathcurveto{\pgfqpoint{2.891618in}{2.020013in}}{\pgfqpoint{2.896008in}{2.030612in}}{\pgfqpoint{2.896008in}{2.041662in}}%
\pgfpathcurveto{\pgfqpoint{2.896008in}{2.052713in}}{\pgfqpoint{2.891618in}{2.063312in}}{\pgfqpoint{2.883804in}{2.071125in}}%
\pgfpathcurveto{\pgfqpoint{2.875990in}{2.078939in}}{\pgfqpoint{2.865391in}{2.083329in}}{\pgfqpoint{2.854341in}{2.083329in}}%
\pgfpathcurveto{\pgfqpoint{2.843291in}{2.083329in}}{\pgfqpoint{2.832692in}{2.078939in}}{\pgfqpoint{2.824878in}{2.071125in}}%
\pgfpathcurveto{\pgfqpoint{2.817065in}{2.063312in}}{\pgfqpoint{2.812674in}{2.052713in}}{\pgfqpoint{2.812674in}{2.041662in}}%
\pgfpathcurveto{\pgfqpoint{2.812674in}{2.030612in}}{\pgfqpoint{2.817065in}{2.020013in}}{\pgfqpoint{2.824878in}{2.012200in}}%
\pgfpathcurveto{\pgfqpoint{2.832692in}{2.004386in}}{\pgfqpoint{2.843291in}{1.999996in}}{\pgfqpoint{2.854341in}{1.999996in}}%
\pgfpathclose%
\pgfusepath{stroke,fill}%
\end{pgfscope}%
\begin{pgfscope}%
\pgfpathrectangle{\pgfqpoint{0.481978in}{0.331635in}}{\pgfqpoint{9.300000in}{7.700000in}}%
\pgfusepath{clip}%
\pgfsetbuttcap%
\pgfsetroundjoin%
\definecolor{currentfill}{rgb}{0.631373,0.788235,0.956863}%
\pgfsetfillcolor{currentfill}%
\pgfsetlinewidth{0.481800pt}%
\definecolor{currentstroke}{rgb}{1.000000,1.000000,1.000000}%
\pgfsetstrokecolor{currentstroke}%
\pgfsetdash{}{0pt}%
\pgfpathmoveto{\pgfqpoint{7.738499in}{4.278526in}}%
\pgfpathcurveto{\pgfqpoint{7.749549in}{4.278526in}}{\pgfqpoint{7.760148in}{4.282916in}}{\pgfqpoint{7.767962in}{4.290730in}}%
\pgfpathcurveto{\pgfqpoint{7.775775in}{4.298543in}}{\pgfqpoint{7.780165in}{4.309143in}}{\pgfqpoint{7.780165in}{4.320193in}}%
\pgfpathcurveto{\pgfqpoint{7.780165in}{4.331243in}}{\pgfqpoint{7.775775in}{4.341842in}}{\pgfqpoint{7.767962in}{4.349655in}}%
\pgfpathcurveto{\pgfqpoint{7.760148in}{4.357469in}}{\pgfqpoint{7.749549in}{4.361859in}}{\pgfqpoint{7.738499in}{4.361859in}}%
\pgfpathcurveto{\pgfqpoint{7.727449in}{4.361859in}}{\pgfqpoint{7.716850in}{4.357469in}}{\pgfqpoint{7.709036in}{4.349655in}}%
\pgfpathcurveto{\pgfqpoint{7.701222in}{4.341842in}}{\pgfqpoint{7.696832in}{4.331243in}}{\pgfqpoint{7.696832in}{4.320193in}}%
\pgfpathcurveto{\pgfqpoint{7.696832in}{4.309143in}}{\pgfqpoint{7.701222in}{4.298543in}}{\pgfqpoint{7.709036in}{4.290730in}}%
\pgfpathcurveto{\pgfqpoint{7.716850in}{4.282916in}}{\pgfqpoint{7.727449in}{4.278526in}}{\pgfqpoint{7.738499in}{4.278526in}}%
\pgfpathclose%
\pgfusepath{stroke,fill}%
\end{pgfscope}%
\begin{pgfscope}%
\pgfpathrectangle{\pgfqpoint{0.481978in}{0.331635in}}{\pgfqpoint{9.300000in}{7.700000in}}%
\pgfusepath{clip}%
\pgfsetbuttcap%
\pgfsetroundjoin%
\definecolor{currentfill}{rgb}{0.631373,0.788235,0.956863}%
\pgfsetfillcolor{currentfill}%
\pgfsetlinewidth{0.481800pt}%
\definecolor{currentstroke}{rgb}{1.000000,1.000000,1.000000}%
\pgfsetstrokecolor{currentstroke}%
\pgfsetdash{}{0pt}%
\pgfpathmoveto{\pgfqpoint{4.181272in}{7.628054in}}%
\pgfpathcurveto{\pgfqpoint{4.192323in}{7.628054in}}{\pgfqpoint{4.202922in}{7.632444in}}{\pgfqpoint{4.210735in}{7.640258in}}%
\pgfpathcurveto{\pgfqpoint{4.218549in}{7.648072in}}{\pgfqpoint{4.222939in}{7.658671in}}{\pgfqpoint{4.222939in}{7.669721in}}%
\pgfpathcurveto{\pgfqpoint{4.222939in}{7.680771in}}{\pgfqpoint{4.218549in}{7.691370in}}{\pgfqpoint{4.210735in}{7.699184in}}%
\pgfpathcurveto{\pgfqpoint{4.202922in}{7.706997in}}{\pgfqpoint{4.192323in}{7.711388in}}{\pgfqpoint{4.181272in}{7.711388in}}%
\pgfpathcurveto{\pgfqpoint{4.170222in}{7.711388in}}{\pgfqpoint{4.159623in}{7.706997in}}{\pgfqpoint{4.151810in}{7.699184in}}%
\pgfpathcurveto{\pgfqpoint{4.143996in}{7.691370in}}{\pgfqpoint{4.139606in}{7.680771in}}{\pgfqpoint{4.139606in}{7.669721in}}%
\pgfpathcurveto{\pgfqpoint{4.139606in}{7.658671in}}{\pgfqpoint{4.143996in}{7.648072in}}{\pgfqpoint{4.151810in}{7.640258in}}%
\pgfpathcurveto{\pgfqpoint{4.159623in}{7.632444in}}{\pgfqpoint{4.170222in}{7.628054in}}{\pgfqpoint{4.181272in}{7.628054in}}%
\pgfpathclose%
\pgfusepath{stroke,fill}%
\end{pgfscope}%
\begin{pgfscope}%
\pgfpathrectangle{\pgfqpoint{0.481978in}{0.331635in}}{\pgfqpoint{9.300000in}{7.700000in}}%
\pgfusepath{clip}%
\pgfsetbuttcap%
\pgfsetroundjoin%
\definecolor{currentfill}{rgb}{0.631373,0.788235,0.956863}%
\pgfsetfillcolor{currentfill}%
\pgfsetlinewidth{0.481800pt}%
\definecolor{currentstroke}{rgb}{1.000000,1.000000,1.000000}%
\pgfsetstrokecolor{currentstroke}%
\pgfsetdash{}{0pt}%
\pgfpathmoveto{\pgfqpoint{6.285628in}{4.415552in}}%
\pgfpathcurveto{\pgfqpoint{6.296678in}{4.415552in}}{\pgfqpoint{6.307277in}{4.419943in}}{\pgfqpoint{6.315090in}{4.427756in}}%
\pgfpathcurveto{\pgfqpoint{6.322904in}{4.435570in}}{\pgfqpoint{6.327294in}{4.446169in}}{\pgfqpoint{6.327294in}{4.457219in}}%
\pgfpathcurveto{\pgfqpoint{6.327294in}{4.468269in}}{\pgfqpoint{6.322904in}{4.478868in}}{\pgfqpoint{6.315090in}{4.486682in}}%
\pgfpathcurveto{\pgfqpoint{6.307277in}{4.494495in}}{\pgfqpoint{6.296678in}{4.498886in}}{\pgfqpoint{6.285628in}{4.498886in}}%
\pgfpathcurveto{\pgfqpoint{6.274578in}{4.498886in}}{\pgfqpoint{6.263979in}{4.494495in}}{\pgfqpoint{6.256165in}{4.486682in}}%
\pgfpathcurveto{\pgfqpoint{6.248351in}{4.478868in}}{\pgfqpoint{6.243961in}{4.468269in}}{\pgfqpoint{6.243961in}{4.457219in}}%
\pgfpathcurveto{\pgfqpoint{6.243961in}{4.446169in}}{\pgfqpoint{6.248351in}{4.435570in}}{\pgfqpoint{6.256165in}{4.427756in}}%
\pgfpathcurveto{\pgfqpoint{6.263979in}{4.419943in}}{\pgfqpoint{6.274578in}{4.415552in}}{\pgfqpoint{6.285628in}{4.415552in}}%
\pgfpathclose%
\pgfusepath{stroke,fill}%
\end{pgfscope}%
\begin{pgfscope}%
\pgfpathrectangle{\pgfqpoint{0.481978in}{0.331635in}}{\pgfqpoint{9.300000in}{7.700000in}}%
\pgfusepath{clip}%
\pgfsetbuttcap%
\pgfsetroundjoin%
\definecolor{currentfill}{rgb}{0.631373,0.788235,0.956863}%
\pgfsetfillcolor{currentfill}%
\pgfsetlinewidth{0.481800pt}%
\definecolor{currentstroke}{rgb}{1.000000,1.000000,1.000000}%
\pgfsetstrokecolor{currentstroke}%
\pgfsetdash{}{0pt}%
\pgfpathmoveto{\pgfqpoint{7.326306in}{5.190183in}}%
\pgfpathcurveto{\pgfqpoint{7.337356in}{5.190183in}}{\pgfqpoint{7.347955in}{5.194573in}}{\pgfqpoint{7.355769in}{5.202387in}}%
\pgfpathcurveto{\pgfqpoint{7.363583in}{5.210200in}}{\pgfqpoint{7.367973in}{5.220799in}}{\pgfqpoint{7.367973in}{5.231850in}}%
\pgfpathcurveto{\pgfqpoint{7.367973in}{5.242900in}}{\pgfqpoint{7.363583in}{5.253499in}}{\pgfqpoint{7.355769in}{5.261312in}}%
\pgfpathcurveto{\pgfqpoint{7.347955in}{5.269126in}}{\pgfqpoint{7.337356in}{5.273516in}}{\pgfqpoint{7.326306in}{5.273516in}}%
\pgfpathcurveto{\pgfqpoint{7.315256in}{5.273516in}}{\pgfqpoint{7.304657in}{5.269126in}}{\pgfqpoint{7.296843in}{5.261312in}}%
\pgfpathcurveto{\pgfqpoint{7.289030in}{5.253499in}}{\pgfqpoint{7.284639in}{5.242900in}}{\pgfqpoint{7.284639in}{5.231850in}}%
\pgfpathcurveto{\pgfqpoint{7.284639in}{5.220799in}}{\pgfqpoint{7.289030in}{5.210200in}}{\pgfqpoint{7.296843in}{5.202387in}}%
\pgfpathcurveto{\pgfqpoint{7.304657in}{5.194573in}}{\pgfqpoint{7.315256in}{5.190183in}}{\pgfqpoint{7.326306in}{5.190183in}}%
\pgfpathclose%
\pgfusepath{stroke,fill}%
\end{pgfscope}%
\begin{pgfscope}%
\pgfpathrectangle{\pgfqpoint{0.481978in}{0.331635in}}{\pgfqpoint{9.300000in}{7.700000in}}%
\pgfusepath{clip}%
\pgfsetbuttcap%
\pgfsetroundjoin%
\definecolor{currentfill}{rgb}{0.631373,0.788235,0.956863}%
\pgfsetfillcolor{currentfill}%
\pgfsetlinewidth{0.481800pt}%
\definecolor{currentstroke}{rgb}{1.000000,1.000000,1.000000}%
\pgfsetstrokecolor{currentstroke}%
\pgfsetdash{}{0pt}%
\pgfpathmoveto{\pgfqpoint{5.231346in}{2.233549in}}%
\pgfpathcurveto{\pgfqpoint{5.242397in}{2.233549in}}{\pgfqpoint{5.252996in}{2.237939in}}{\pgfqpoint{5.260809in}{2.245753in}}%
\pgfpathcurveto{\pgfqpoint{5.268623in}{2.253566in}}{\pgfqpoint{5.273013in}{2.264165in}}{\pgfqpoint{5.273013in}{2.275215in}}%
\pgfpathcurveto{\pgfqpoint{5.273013in}{2.286265in}}{\pgfqpoint{5.268623in}{2.296865in}}{\pgfqpoint{5.260809in}{2.304678in}}%
\pgfpathcurveto{\pgfqpoint{5.252996in}{2.312492in}}{\pgfqpoint{5.242397in}{2.316882in}}{\pgfqpoint{5.231346in}{2.316882in}}%
\pgfpathcurveto{\pgfqpoint{5.220296in}{2.316882in}}{\pgfqpoint{5.209697in}{2.312492in}}{\pgfqpoint{5.201884in}{2.304678in}}%
\pgfpathcurveto{\pgfqpoint{5.194070in}{2.296865in}}{\pgfqpoint{5.189680in}{2.286265in}}{\pgfqpoint{5.189680in}{2.275215in}}%
\pgfpathcurveto{\pgfqpoint{5.189680in}{2.264165in}}{\pgfqpoint{5.194070in}{2.253566in}}{\pgfqpoint{5.201884in}{2.245753in}}%
\pgfpathcurveto{\pgfqpoint{5.209697in}{2.237939in}}{\pgfqpoint{5.220296in}{2.233549in}}{\pgfqpoint{5.231346in}{2.233549in}}%
\pgfpathclose%
\pgfusepath{stroke,fill}%
\end{pgfscope}%
\begin{pgfscope}%
\pgfpathrectangle{\pgfqpoint{0.481978in}{0.331635in}}{\pgfqpoint{9.300000in}{7.700000in}}%
\pgfusepath{clip}%
\pgfsetbuttcap%
\pgfsetroundjoin%
\definecolor{currentfill}{rgb}{0.631373,0.788235,0.956863}%
\pgfsetfillcolor{currentfill}%
\pgfsetlinewidth{0.481800pt}%
\definecolor{currentstroke}{rgb}{1.000000,1.000000,1.000000}%
\pgfsetstrokecolor{currentstroke}%
\pgfsetdash{}{0pt}%
\pgfpathmoveto{\pgfqpoint{2.924270in}{1.890349in}}%
\pgfpathcurveto{\pgfqpoint{2.935320in}{1.890349in}}{\pgfqpoint{2.945919in}{1.894739in}}{\pgfqpoint{2.953733in}{1.902553in}}%
\pgfpathcurveto{\pgfqpoint{2.961547in}{1.910367in}}{\pgfqpoint{2.965937in}{1.920966in}}{\pgfqpoint{2.965937in}{1.932016in}}%
\pgfpathcurveto{\pgfqpoint{2.965937in}{1.943066in}}{\pgfqpoint{2.961547in}{1.953665in}}{\pgfqpoint{2.953733in}{1.961478in}}%
\pgfpathcurveto{\pgfqpoint{2.945919in}{1.969292in}}{\pgfqpoint{2.935320in}{1.973682in}}{\pgfqpoint{2.924270in}{1.973682in}}%
\pgfpathcurveto{\pgfqpoint{2.913220in}{1.973682in}}{\pgfqpoint{2.902621in}{1.969292in}}{\pgfqpoint{2.894807in}{1.961478in}}%
\pgfpathcurveto{\pgfqpoint{2.886994in}{1.953665in}}{\pgfqpoint{2.882604in}{1.943066in}}{\pgfqpoint{2.882604in}{1.932016in}}%
\pgfpathcurveto{\pgfqpoint{2.882604in}{1.920966in}}{\pgfqpoint{2.886994in}{1.910367in}}{\pgfqpoint{2.894807in}{1.902553in}}%
\pgfpathcurveto{\pgfqpoint{2.902621in}{1.894739in}}{\pgfqpoint{2.913220in}{1.890349in}}{\pgfqpoint{2.924270in}{1.890349in}}%
\pgfpathclose%
\pgfusepath{stroke,fill}%
\end{pgfscope}%
\begin{pgfscope}%
\pgfpathrectangle{\pgfqpoint{0.481978in}{0.331635in}}{\pgfqpoint{9.300000in}{7.700000in}}%
\pgfusepath{clip}%
\pgfsetbuttcap%
\pgfsetroundjoin%
\definecolor{currentfill}{rgb}{0.631373,0.788235,0.956863}%
\pgfsetfillcolor{currentfill}%
\pgfsetlinewidth{0.481800pt}%
\definecolor{currentstroke}{rgb}{1.000000,1.000000,1.000000}%
\pgfsetstrokecolor{currentstroke}%
\pgfsetdash{}{0pt}%
\pgfpathmoveto{\pgfqpoint{5.426953in}{7.035913in}}%
\pgfpathcurveto{\pgfqpoint{5.438003in}{7.035913in}}{\pgfqpoint{5.448602in}{7.040304in}}{\pgfqpoint{5.456415in}{7.048117in}}%
\pgfpathcurveto{\pgfqpoint{5.464229in}{7.055931in}}{\pgfqpoint{5.468619in}{7.066530in}}{\pgfqpoint{5.468619in}{7.077580in}}%
\pgfpathcurveto{\pgfqpoint{5.468619in}{7.088630in}}{\pgfqpoint{5.464229in}{7.099229in}}{\pgfqpoint{5.456415in}{7.107043in}}%
\pgfpathcurveto{\pgfqpoint{5.448602in}{7.114856in}}{\pgfqpoint{5.438003in}{7.119247in}}{\pgfqpoint{5.426953in}{7.119247in}}%
\pgfpathcurveto{\pgfqpoint{5.415902in}{7.119247in}}{\pgfqpoint{5.405303in}{7.114856in}}{\pgfqpoint{5.397490in}{7.107043in}}%
\pgfpathcurveto{\pgfqpoint{5.389676in}{7.099229in}}{\pgfqpoint{5.385286in}{7.088630in}}{\pgfqpoint{5.385286in}{7.077580in}}%
\pgfpathcurveto{\pgfqpoint{5.385286in}{7.066530in}}{\pgfqpoint{5.389676in}{7.055931in}}{\pgfqpoint{5.397490in}{7.048117in}}%
\pgfpathcurveto{\pgfqpoint{5.405303in}{7.040304in}}{\pgfqpoint{5.415902in}{7.035913in}}{\pgfqpoint{5.426953in}{7.035913in}}%
\pgfpathclose%
\pgfusepath{stroke,fill}%
\end{pgfscope}%
\begin{pgfscope}%
\pgfpathrectangle{\pgfqpoint{0.481978in}{0.331635in}}{\pgfqpoint{9.300000in}{7.700000in}}%
\pgfusepath{clip}%
\pgfsetbuttcap%
\pgfsetroundjoin%
\definecolor{currentfill}{rgb}{0.631373,0.788235,0.956863}%
\pgfsetfillcolor{currentfill}%
\pgfsetlinewidth{0.481800pt}%
\definecolor{currentstroke}{rgb}{1.000000,1.000000,1.000000}%
\pgfsetstrokecolor{currentstroke}%
\pgfsetdash{}{0pt}%
\pgfpathmoveto{\pgfqpoint{8.241979in}{5.227581in}}%
\pgfpathcurveto{\pgfqpoint{8.253029in}{5.227581in}}{\pgfqpoint{8.263628in}{5.231971in}}{\pgfqpoint{8.271442in}{5.239784in}}%
\pgfpathcurveto{\pgfqpoint{8.279256in}{5.247598in}}{\pgfqpoint{8.283646in}{5.258197in}}{\pgfqpoint{8.283646in}{5.269247in}}%
\pgfpathcurveto{\pgfqpoint{8.283646in}{5.280297in}}{\pgfqpoint{8.279256in}{5.290896in}}{\pgfqpoint{8.271442in}{5.298710in}}%
\pgfpathcurveto{\pgfqpoint{8.263628in}{5.306524in}}{\pgfqpoint{8.253029in}{5.310914in}}{\pgfqpoint{8.241979in}{5.310914in}}%
\pgfpathcurveto{\pgfqpoint{8.230929in}{5.310914in}}{\pgfqpoint{8.220330in}{5.306524in}}{\pgfqpoint{8.212516in}{5.298710in}}%
\pgfpathcurveto{\pgfqpoint{8.204703in}{5.290896in}}{\pgfqpoint{8.200313in}{5.280297in}}{\pgfqpoint{8.200313in}{5.269247in}}%
\pgfpathcurveto{\pgfqpoint{8.200313in}{5.258197in}}{\pgfqpoint{8.204703in}{5.247598in}}{\pgfqpoint{8.212516in}{5.239784in}}%
\pgfpathcurveto{\pgfqpoint{8.220330in}{5.231971in}}{\pgfqpoint{8.230929in}{5.227581in}}{\pgfqpoint{8.241979in}{5.227581in}}%
\pgfpathclose%
\pgfusepath{stroke,fill}%
\end{pgfscope}%
\begin{pgfscope}%
\pgfpathrectangle{\pgfqpoint{0.481978in}{0.331635in}}{\pgfqpoint{9.300000in}{7.700000in}}%
\pgfusepath{clip}%
\pgfsetbuttcap%
\pgfsetroundjoin%
\definecolor{currentfill}{rgb}{0.631373,0.788235,0.956863}%
\pgfsetfillcolor{currentfill}%
\pgfsetlinewidth{0.481800pt}%
\definecolor{currentstroke}{rgb}{1.000000,1.000000,1.000000}%
\pgfsetstrokecolor{currentstroke}%
\pgfsetdash{}{0pt}%
\pgfpathmoveto{\pgfqpoint{5.988042in}{2.698357in}}%
\pgfpathcurveto{\pgfqpoint{5.999092in}{2.698357in}}{\pgfqpoint{6.009691in}{2.702747in}}{\pgfqpoint{6.017505in}{2.710561in}}%
\pgfpathcurveto{\pgfqpoint{6.025319in}{2.718375in}}{\pgfqpoint{6.029709in}{2.728974in}}{\pgfqpoint{6.029709in}{2.740024in}}%
\pgfpathcurveto{\pgfqpoint{6.029709in}{2.751074in}}{\pgfqpoint{6.025319in}{2.761673in}}{\pgfqpoint{6.017505in}{2.769487in}}%
\pgfpathcurveto{\pgfqpoint{6.009691in}{2.777300in}}{\pgfqpoint{5.999092in}{2.781690in}}{\pgfqpoint{5.988042in}{2.781690in}}%
\pgfpathcurveto{\pgfqpoint{5.976992in}{2.781690in}}{\pgfqpoint{5.966393in}{2.777300in}}{\pgfqpoint{5.958579in}{2.769487in}}%
\pgfpathcurveto{\pgfqpoint{5.950766in}{2.761673in}}{\pgfqpoint{5.946375in}{2.751074in}}{\pgfqpoint{5.946375in}{2.740024in}}%
\pgfpathcurveto{\pgfqpoint{5.946375in}{2.728974in}}{\pgfqpoint{5.950766in}{2.718375in}}{\pgfqpoint{5.958579in}{2.710561in}}%
\pgfpathcurveto{\pgfqpoint{5.966393in}{2.702747in}}{\pgfqpoint{5.976992in}{2.698357in}}{\pgfqpoint{5.988042in}{2.698357in}}%
\pgfpathclose%
\pgfusepath{stroke,fill}%
\end{pgfscope}%
\begin{pgfscope}%
\pgfpathrectangle{\pgfqpoint{0.481978in}{0.331635in}}{\pgfqpoint{9.300000in}{7.700000in}}%
\pgfusepath{clip}%
\pgfsetbuttcap%
\pgfsetroundjoin%
\definecolor{currentfill}{rgb}{0.631373,0.788235,0.956863}%
\pgfsetfillcolor{currentfill}%
\pgfsetlinewidth{0.481800pt}%
\definecolor{currentstroke}{rgb}{1.000000,1.000000,1.000000}%
\pgfsetstrokecolor{currentstroke}%
\pgfsetdash{}{0pt}%
\pgfpathmoveto{\pgfqpoint{6.404379in}{5.365759in}}%
\pgfpathcurveto{\pgfqpoint{6.415429in}{5.365759in}}{\pgfqpoint{6.426028in}{5.370150in}}{\pgfqpoint{6.433842in}{5.377963in}}%
\pgfpathcurveto{\pgfqpoint{6.441656in}{5.385777in}}{\pgfqpoint{6.446046in}{5.396376in}}{\pgfqpoint{6.446046in}{5.407426in}}%
\pgfpathcurveto{\pgfqpoint{6.446046in}{5.418476in}}{\pgfqpoint{6.441656in}{5.429075in}}{\pgfqpoint{6.433842in}{5.436889in}}%
\pgfpathcurveto{\pgfqpoint{6.426028in}{5.444702in}}{\pgfqpoint{6.415429in}{5.449093in}}{\pgfqpoint{6.404379in}{5.449093in}}%
\pgfpathcurveto{\pgfqpoint{6.393329in}{5.449093in}}{\pgfqpoint{6.382730in}{5.444702in}}{\pgfqpoint{6.374916in}{5.436889in}}%
\pgfpathcurveto{\pgfqpoint{6.367103in}{5.429075in}}{\pgfqpoint{6.362713in}{5.418476in}}{\pgfqpoint{6.362713in}{5.407426in}}%
\pgfpathcurveto{\pgfqpoint{6.362713in}{5.396376in}}{\pgfqpoint{6.367103in}{5.385777in}}{\pgfqpoint{6.374916in}{5.377963in}}%
\pgfpathcurveto{\pgfqpoint{6.382730in}{5.370150in}}{\pgfqpoint{6.393329in}{5.365759in}}{\pgfqpoint{6.404379in}{5.365759in}}%
\pgfpathclose%
\pgfusepath{stroke,fill}%
\end{pgfscope}%
\begin{pgfscope}%
\pgfpathrectangle{\pgfqpoint{0.481978in}{0.331635in}}{\pgfqpoint{9.300000in}{7.700000in}}%
\pgfusepath{clip}%
\pgfsetbuttcap%
\pgfsetroundjoin%
\definecolor{currentfill}{rgb}{0.631373,0.788235,0.956863}%
\pgfsetfillcolor{currentfill}%
\pgfsetlinewidth{0.481800pt}%
\definecolor{currentstroke}{rgb}{1.000000,1.000000,1.000000}%
\pgfsetstrokecolor{currentstroke}%
\pgfsetdash{}{0pt}%
\pgfpathmoveto{\pgfqpoint{7.465449in}{4.115115in}}%
\pgfpathcurveto{\pgfqpoint{7.476499in}{4.115115in}}{\pgfqpoint{7.487098in}{4.119506in}}{\pgfqpoint{7.494912in}{4.127319in}}%
\pgfpathcurveto{\pgfqpoint{7.502725in}{4.135133in}}{\pgfqpoint{7.507115in}{4.145732in}}{\pgfqpoint{7.507115in}{4.156782in}}%
\pgfpathcurveto{\pgfqpoint{7.507115in}{4.167832in}}{\pgfqpoint{7.502725in}{4.178431in}}{\pgfqpoint{7.494912in}{4.186245in}}%
\pgfpathcurveto{\pgfqpoint{7.487098in}{4.194059in}}{\pgfqpoint{7.476499in}{4.198449in}}{\pgfqpoint{7.465449in}{4.198449in}}%
\pgfpathcurveto{\pgfqpoint{7.454399in}{4.198449in}}{\pgfqpoint{7.443800in}{4.194059in}}{\pgfqpoint{7.435986in}{4.186245in}}%
\pgfpathcurveto{\pgfqpoint{7.428172in}{4.178431in}}{\pgfqpoint{7.423782in}{4.167832in}}{\pgfqpoint{7.423782in}{4.156782in}}%
\pgfpathcurveto{\pgfqpoint{7.423782in}{4.145732in}}{\pgfqpoint{7.428172in}{4.135133in}}{\pgfqpoint{7.435986in}{4.127319in}}%
\pgfpathcurveto{\pgfqpoint{7.443800in}{4.119506in}}{\pgfqpoint{7.454399in}{4.115115in}}{\pgfqpoint{7.465449in}{4.115115in}}%
\pgfpathclose%
\pgfusepath{stroke,fill}%
\end{pgfscope}%
\begin{pgfscope}%
\pgfpathrectangle{\pgfqpoint{0.481978in}{0.331635in}}{\pgfqpoint{9.300000in}{7.700000in}}%
\pgfusepath{clip}%
\pgfsetbuttcap%
\pgfsetroundjoin%
\definecolor{currentfill}{rgb}{0.631373,0.788235,0.956863}%
\pgfsetfillcolor{currentfill}%
\pgfsetlinewidth{0.481800pt}%
\definecolor{currentstroke}{rgb}{1.000000,1.000000,1.000000}%
\pgfsetstrokecolor{currentstroke}%
\pgfsetdash{}{0pt}%
\pgfpathmoveto{\pgfqpoint{6.297588in}{4.817365in}}%
\pgfpathcurveto{\pgfqpoint{6.308638in}{4.817365in}}{\pgfqpoint{6.319237in}{4.821756in}}{\pgfqpoint{6.327051in}{4.829569in}}%
\pgfpathcurveto{\pgfqpoint{6.334864in}{4.837383in}}{\pgfqpoint{6.339255in}{4.847982in}}{\pgfqpoint{6.339255in}{4.859032in}}%
\pgfpathcurveto{\pgfqpoint{6.339255in}{4.870082in}}{\pgfqpoint{6.334864in}{4.880681in}}{\pgfqpoint{6.327051in}{4.888495in}}%
\pgfpathcurveto{\pgfqpoint{6.319237in}{4.896308in}}{\pgfqpoint{6.308638in}{4.900699in}}{\pgfqpoint{6.297588in}{4.900699in}}%
\pgfpathcurveto{\pgfqpoint{6.286538in}{4.900699in}}{\pgfqpoint{6.275939in}{4.896308in}}{\pgfqpoint{6.268125in}{4.888495in}}%
\pgfpathcurveto{\pgfqpoint{6.260312in}{4.880681in}}{\pgfqpoint{6.255921in}{4.870082in}}{\pgfqpoint{6.255921in}{4.859032in}}%
\pgfpathcurveto{\pgfqpoint{6.255921in}{4.847982in}}{\pgfqpoint{6.260312in}{4.837383in}}{\pgfqpoint{6.268125in}{4.829569in}}%
\pgfpathcurveto{\pgfqpoint{6.275939in}{4.821756in}}{\pgfqpoint{6.286538in}{4.817365in}}{\pgfqpoint{6.297588in}{4.817365in}}%
\pgfpathclose%
\pgfusepath{stroke,fill}%
\end{pgfscope}%
\begin{pgfscope}%
\pgfpathrectangle{\pgfqpoint{0.481978in}{0.331635in}}{\pgfqpoint{9.300000in}{7.700000in}}%
\pgfusepath{clip}%
\pgfsetbuttcap%
\pgfsetroundjoin%
\definecolor{currentfill}{rgb}{0.631373,0.788235,0.956863}%
\pgfsetfillcolor{currentfill}%
\pgfsetlinewidth{0.481800pt}%
\definecolor{currentstroke}{rgb}{1.000000,1.000000,1.000000}%
\pgfsetstrokecolor{currentstroke}%
\pgfsetdash{}{0pt}%
\pgfpathmoveto{\pgfqpoint{7.598727in}{4.490361in}}%
\pgfpathcurveto{\pgfqpoint{7.609778in}{4.490361in}}{\pgfqpoint{7.620377in}{4.494752in}}{\pgfqpoint{7.628190in}{4.502565in}}%
\pgfpathcurveto{\pgfqpoint{7.636004in}{4.510379in}}{\pgfqpoint{7.640394in}{4.520978in}}{\pgfqpoint{7.640394in}{4.532028in}}%
\pgfpathcurveto{\pgfqpoint{7.640394in}{4.543078in}}{\pgfqpoint{7.636004in}{4.553677in}}{\pgfqpoint{7.628190in}{4.561491in}}%
\pgfpathcurveto{\pgfqpoint{7.620377in}{4.569304in}}{\pgfqpoint{7.609778in}{4.573695in}}{\pgfqpoint{7.598727in}{4.573695in}}%
\pgfpathcurveto{\pgfqpoint{7.587677in}{4.573695in}}{\pgfqpoint{7.577078in}{4.569304in}}{\pgfqpoint{7.569265in}{4.561491in}}%
\pgfpathcurveto{\pgfqpoint{7.561451in}{4.553677in}}{\pgfqpoint{7.557061in}{4.543078in}}{\pgfqpoint{7.557061in}{4.532028in}}%
\pgfpathcurveto{\pgfqpoint{7.557061in}{4.520978in}}{\pgfqpoint{7.561451in}{4.510379in}}{\pgfqpoint{7.569265in}{4.502565in}}%
\pgfpathcurveto{\pgfqpoint{7.577078in}{4.494752in}}{\pgfqpoint{7.587677in}{4.490361in}}{\pgfqpoint{7.598727in}{4.490361in}}%
\pgfpathclose%
\pgfusepath{stroke,fill}%
\end{pgfscope}%
\begin{pgfscope}%
\pgfpathrectangle{\pgfqpoint{0.481978in}{0.331635in}}{\pgfqpoint{9.300000in}{7.700000in}}%
\pgfusepath{clip}%
\pgfsetbuttcap%
\pgfsetroundjoin%
\definecolor{currentfill}{rgb}{0.631373,0.788235,0.956863}%
\pgfsetfillcolor{currentfill}%
\pgfsetlinewidth{0.481800pt}%
\definecolor{currentstroke}{rgb}{1.000000,1.000000,1.000000}%
\pgfsetstrokecolor{currentstroke}%
\pgfsetdash{}{0pt}%
\pgfpathmoveto{\pgfqpoint{4.778349in}{5.109947in}}%
\pgfpathcurveto{\pgfqpoint{4.789399in}{5.109947in}}{\pgfqpoint{4.799998in}{5.114337in}}{\pgfqpoint{4.807812in}{5.122150in}}%
\pgfpathcurveto{\pgfqpoint{4.815625in}{5.129964in}}{\pgfqpoint{4.820015in}{5.140563in}}{\pgfqpoint{4.820015in}{5.151613in}}%
\pgfpathcurveto{\pgfqpoint{4.820015in}{5.162663in}}{\pgfqpoint{4.815625in}{5.173262in}}{\pgfqpoint{4.807812in}{5.181076in}}%
\pgfpathcurveto{\pgfqpoint{4.799998in}{5.188890in}}{\pgfqpoint{4.789399in}{5.193280in}}{\pgfqpoint{4.778349in}{5.193280in}}%
\pgfpathcurveto{\pgfqpoint{4.767299in}{5.193280in}}{\pgfqpoint{4.756700in}{5.188890in}}{\pgfqpoint{4.748886in}{5.181076in}}%
\pgfpathcurveto{\pgfqpoint{4.741072in}{5.173262in}}{\pgfqpoint{4.736682in}{5.162663in}}{\pgfqpoint{4.736682in}{5.151613in}}%
\pgfpathcurveto{\pgfqpoint{4.736682in}{5.140563in}}{\pgfqpoint{4.741072in}{5.129964in}}{\pgfqpoint{4.748886in}{5.122150in}}%
\pgfpathcurveto{\pgfqpoint{4.756700in}{5.114337in}}{\pgfqpoint{4.767299in}{5.109947in}}{\pgfqpoint{4.778349in}{5.109947in}}%
\pgfpathclose%
\pgfusepath{stroke,fill}%
\end{pgfscope}%
\begin{pgfscope}%
\pgfpathrectangle{\pgfqpoint{0.481978in}{0.331635in}}{\pgfqpoint{9.300000in}{7.700000in}}%
\pgfusepath{clip}%
\pgfsetbuttcap%
\pgfsetroundjoin%
\definecolor{currentfill}{rgb}{0.631373,0.788235,0.956863}%
\pgfsetfillcolor{currentfill}%
\pgfsetlinewidth{0.481800pt}%
\definecolor{currentstroke}{rgb}{1.000000,1.000000,1.000000}%
\pgfsetstrokecolor{currentstroke}%
\pgfsetdash{}{0pt}%
\pgfpathmoveto{\pgfqpoint{5.249516in}{5.477553in}}%
\pgfpathcurveto{\pgfqpoint{5.260566in}{5.477553in}}{\pgfqpoint{5.271166in}{5.481944in}}{\pgfqpoint{5.278979in}{5.489757in}}%
\pgfpathcurveto{\pgfqpoint{5.286793in}{5.497571in}}{\pgfqpoint{5.291183in}{5.508170in}}{\pgfqpoint{5.291183in}{5.519220in}}%
\pgfpathcurveto{\pgfqpoint{5.291183in}{5.530270in}}{\pgfqpoint{5.286793in}{5.540869in}}{\pgfqpoint{5.278979in}{5.548683in}}%
\pgfpathcurveto{\pgfqpoint{5.271166in}{5.556496in}}{\pgfqpoint{5.260566in}{5.560887in}}{\pgfqpoint{5.249516in}{5.560887in}}%
\pgfpathcurveto{\pgfqpoint{5.238466in}{5.560887in}}{\pgfqpoint{5.227867in}{5.556496in}}{\pgfqpoint{5.220054in}{5.548683in}}%
\pgfpathcurveto{\pgfqpoint{5.212240in}{5.540869in}}{\pgfqpoint{5.207850in}{5.530270in}}{\pgfqpoint{5.207850in}{5.519220in}}%
\pgfpathcurveto{\pgfqpoint{5.207850in}{5.508170in}}{\pgfqpoint{5.212240in}{5.497571in}}{\pgfqpoint{5.220054in}{5.489757in}}%
\pgfpathcurveto{\pgfqpoint{5.227867in}{5.481944in}}{\pgfqpoint{5.238466in}{5.477553in}}{\pgfqpoint{5.249516in}{5.477553in}}%
\pgfpathclose%
\pgfusepath{stroke,fill}%
\end{pgfscope}%
\begin{pgfscope}%
\pgfpathrectangle{\pgfqpoint{0.481978in}{0.331635in}}{\pgfqpoint{9.300000in}{7.700000in}}%
\pgfusepath{clip}%
\pgfsetbuttcap%
\pgfsetroundjoin%
\definecolor{currentfill}{rgb}{0.631373,0.788235,0.956863}%
\pgfsetfillcolor{currentfill}%
\pgfsetlinewidth{0.481800pt}%
\definecolor{currentstroke}{rgb}{1.000000,1.000000,1.000000}%
\pgfsetstrokecolor{currentstroke}%
\pgfsetdash{}{0pt}%
\pgfpathmoveto{\pgfqpoint{6.692326in}{1.648181in}}%
\pgfpathcurveto{\pgfqpoint{6.703376in}{1.648181in}}{\pgfqpoint{6.713976in}{1.652572in}}{\pgfqpoint{6.721789in}{1.660385in}}%
\pgfpathcurveto{\pgfqpoint{6.729603in}{1.668199in}}{\pgfqpoint{6.733993in}{1.678798in}}{\pgfqpoint{6.733993in}{1.689848in}}%
\pgfpathcurveto{\pgfqpoint{6.733993in}{1.700898in}}{\pgfqpoint{6.729603in}{1.711497in}}{\pgfqpoint{6.721789in}{1.719311in}}%
\pgfpathcurveto{\pgfqpoint{6.713976in}{1.727124in}}{\pgfqpoint{6.703376in}{1.731515in}}{\pgfqpoint{6.692326in}{1.731515in}}%
\pgfpathcurveto{\pgfqpoint{6.681276in}{1.731515in}}{\pgfqpoint{6.670677in}{1.727124in}}{\pgfqpoint{6.662864in}{1.719311in}}%
\pgfpathcurveto{\pgfqpoint{6.655050in}{1.711497in}}{\pgfqpoint{6.650660in}{1.700898in}}{\pgfqpoint{6.650660in}{1.689848in}}%
\pgfpathcurveto{\pgfqpoint{6.650660in}{1.678798in}}{\pgfqpoint{6.655050in}{1.668199in}}{\pgfqpoint{6.662864in}{1.660385in}}%
\pgfpathcurveto{\pgfqpoint{6.670677in}{1.652572in}}{\pgfqpoint{6.681276in}{1.648181in}}{\pgfqpoint{6.692326in}{1.648181in}}%
\pgfpathclose%
\pgfusepath{stroke,fill}%
\end{pgfscope}%
\begin{pgfscope}%
\pgfpathrectangle{\pgfqpoint{0.481978in}{0.331635in}}{\pgfqpoint{9.300000in}{7.700000in}}%
\pgfusepath{clip}%
\pgfsetbuttcap%
\pgfsetroundjoin%
\definecolor{currentfill}{rgb}{0.631373,0.788235,0.956863}%
\pgfsetfillcolor{currentfill}%
\pgfsetlinewidth{0.481800pt}%
\definecolor{currentstroke}{rgb}{1.000000,1.000000,1.000000}%
\pgfsetstrokecolor{currentstroke}%
\pgfsetdash{}{0pt}%
\pgfpathmoveto{\pgfqpoint{4.317631in}{1.907085in}}%
\pgfpathcurveto{\pgfqpoint{4.328682in}{1.907085in}}{\pgfqpoint{4.339281in}{1.911475in}}{\pgfqpoint{4.347094in}{1.919288in}}%
\pgfpathcurveto{\pgfqpoint{4.354908in}{1.927102in}}{\pgfqpoint{4.359298in}{1.937701in}}{\pgfqpoint{4.359298in}{1.948751in}}%
\pgfpathcurveto{\pgfqpoint{4.359298in}{1.959801in}}{\pgfqpoint{4.354908in}{1.970400in}}{\pgfqpoint{4.347094in}{1.978214in}}%
\pgfpathcurveto{\pgfqpoint{4.339281in}{1.986028in}}{\pgfqpoint{4.328682in}{1.990418in}}{\pgfqpoint{4.317631in}{1.990418in}}%
\pgfpathcurveto{\pgfqpoint{4.306581in}{1.990418in}}{\pgfqpoint{4.295982in}{1.986028in}}{\pgfqpoint{4.288169in}{1.978214in}}%
\pgfpathcurveto{\pgfqpoint{4.280355in}{1.970400in}}{\pgfqpoint{4.275965in}{1.959801in}}{\pgfqpoint{4.275965in}{1.948751in}}%
\pgfpathcurveto{\pgfqpoint{4.275965in}{1.937701in}}{\pgfqpoint{4.280355in}{1.927102in}}{\pgfqpoint{4.288169in}{1.919288in}}%
\pgfpathcurveto{\pgfqpoint{4.295982in}{1.911475in}}{\pgfqpoint{4.306581in}{1.907085in}}{\pgfqpoint{4.317631in}{1.907085in}}%
\pgfpathclose%
\pgfusepath{stroke,fill}%
\end{pgfscope}%
\begin{pgfscope}%
\pgfpathrectangle{\pgfqpoint{0.481978in}{0.331635in}}{\pgfqpoint{9.300000in}{7.700000in}}%
\pgfusepath{clip}%
\pgfsetbuttcap%
\pgfsetroundjoin%
\definecolor{currentfill}{rgb}{0.631373,0.788235,0.956863}%
\pgfsetfillcolor{currentfill}%
\pgfsetlinewidth{0.481800pt}%
\definecolor{currentstroke}{rgb}{1.000000,1.000000,1.000000}%
\pgfsetstrokecolor{currentstroke}%
\pgfsetdash{}{0pt}%
\pgfpathmoveto{\pgfqpoint{7.680936in}{6.007917in}}%
\pgfpathcurveto{\pgfqpoint{7.691986in}{6.007917in}}{\pgfqpoint{7.702585in}{6.012307in}}{\pgfqpoint{7.710398in}{6.020120in}}%
\pgfpathcurveto{\pgfqpoint{7.718212in}{6.027934in}}{\pgfqpoint{7.722602in}{6.038533in}}{\pgfqpoint{7.722602in}{6.049583in}}%
\pgfpathcurveto{\pgfqpoint{7.722602in}{6.060633in}}{\pgfqpoint{7.718212in}{6.071232in}}{\pgfqpoint{7.710398in}{6.079046in}}%
\pgfpathcurveto{\pgfqpoint{7.702585in}{6.086860in}}{\pgfqpoint{7.691986in}{6.091250in}}{\pgfqpoint{7.680936in}{6.091250in}}%
\pgfpathcurveto{\pgfqpoint{7.669885in}{6.091250in}}{\pgfqpoint{7.659286in}{6.086860in}}{\pgfqpoint{7.651473in}{6.079046in}}%
\pgfpathcurveto{\pgfqpoint{7.643659in}{6.071232in}}{\pgfqpoint{7.639269in}{6.060633in}}{\pgfqpoint{7.639269in}{6.049583in}}%
\pgfpathcurveto{\pgfqpoint{7.639269in}{6.038533in}}{\pgfqpoint{7.643659in}{6.027934in}}{\pgfqpoint{7.651473in}{6.020120in}}%
\pgfpathcurveto{\pgfqpoint{7.659286in}{6.012307in}}{\pgfqpoint{7.669885in}{6.007917in}}{\pgfqpoint{7.680936in}{6.007917in}}%
\pgfpathclose%
\pgfusepath{stroke,fill}%
\end{pgfscope}%
\begin{pgfscope}%
\pgfpathrectangle{\pgfqpoint{0.481978in}{0.331635in}}{\pgfqpoint{9.300000in}{7.700000in}}%
\pgfusepath{clip}%
\pgfsetbuttcap%
\pgfsetroundjoin%
\definecolor{currentfill}{rgb}{0.631373,0.788235,0.956863}%
\pgfsetfillcolor{currentfill}%
\pgfsetlinewidth{0.481800pt}%
\definecolor{currentstroke}{rgb}{1.000000,1.000000,1.000000}%
\pgfsetstrokecolor{currentstroke}%
\pgfsetdash{}{0pt}%
\pgfpathmoveto{\pgfqpoint{4.088639in}{5.974869in}}%
\pgfpathcurveto{\pgfqpoint{4.099689in}{5.974869in}}{\pgfqpoint{4.110289in}{5.979260in}}{\pgfqpoint{4.118102in}{5.987073in}}%
\pgfpathcurveto{\pgfqpoint{4.125916in}{5.994887in}}{\pgfqpoint{4.130306in}{6.005486in}}{\pgfqpoint{4.130306in}{6.016536in}}%
\pgfpathcurveto{\pgfqpoint{4.130306in}{6.027586in}}{\pgfqpoint{4.125916in}{6.038185in}}{\pgfqpoint{4.118102in}{6.045999in}}%
\pgfpathcurveto{\pgfqpoint{4.110289in}{6.053813in}}{\pgfqpoint{4.099689in}{6.058203in}}{\pgfqpoint{4.088639in}{6.058203in}}%
\pgfpathcurveto{\pgfqpoint{4.077589in}{6.058203in}}{\pgfqpoint{4.066990in}{6.053813in}}{\pgfqpoint{4.059177in}{6.045999in}}%
\pgfpathcurveto{\pgfqpoint{4.051363in}{6.038185in}}{\pgfqpoint{4.046973in}{6.027586in}}{\pgfqpoint{4.046973in}{6.016536in}}%
\pgfpathcurveto{\pgfqpoint{4.046973in}{6.005486in}}{\pgfqpoint{4.051363in}{5.994887in}}{\pgfqpoint{4.059177in}{5.987073in}}%
\pgfpathcurveto{\pgfqpoint{4.066990in}{5.979260in}}{\pgfqpoint{4.077589in}{5.974869in}}{\pgfqpoint{4.088639in}{5.974869in}}%
\pgfpathclose%
\pgfusepath{stroke,fill}%
\end{pgfscope}%
\begin{pgfscope}%
\pgfpathrectangle{\pgfqpoint{0.481978in}{0.331635in}}{\pgfqpoint{9.300000in}{7.700000in}}%
\pgfusepath{clip}%
\pgfsetbuttcap%
\pgfsetroundjoin%
\definecolor{currentfill}{rgb}{0.631373,0.788235,0.956863}%
\pgfsetfillcolor{currentfill}%
\pgfsetlinewidth{0.481800pt}%
\definecolor{currentstroke}{rgb}{1.000000,1.000000,1.000000}%
\pgfsetstrokecolor{currentstroke}%
\pgfsetdash{}{0pt}%
\pgfpathmoveto{\pgfqpoint{2.271824in}{3.772311in}}%
\pgfpathcurveto{\pgfqpoint{2.282874in}{3.772311in}}{\pgfqpoint{2.293473in}{3.776701in}}{\pgfqpoint{2.301287in}{3.784515in}}%
\pgfpathcurveto{\pgfqpoint{2.309101in}{3.792329in}}{\pgfqpoint{2.313491in}{3.802928in}}{\pgfqpoint{2.313491in}{3.813978in}}%
\pgfpathcurveto{\pgfqpoint{2.313491in}{3.825028in}}{\pgfqpoint{2.309101in}{3.835627in}}{\pgfqpoint{2.301287in}{3.843441in}}%
\pgfpathcurveto{\pgfqpoint{2.293473in}{3.851254in}}{\pgfqpoint{2.282874in}{3.855645in}}{\pgfqpoint{2.271824in}{3.855645in}}%
\pgfpathcurveto{\pgfqpoint{2.260774in}{3.855645in}}{\pgfqpoint{2.250175in}{3.851254in}}{\pgfqpoint{2.242361in}{3.843441in}}%
\pgfpathcurveto{\pgfqpoint{2.234548in}{3.835627in}}{\pgfqpoint{2.230158in}{3.825028in}}{\pgfqpoint{2.230158in}{3.813978in}}%
\pgfpathcurveto{\pgfqpoint{2.230158in}{3.802928in}}{\pgfqpoint{2.234548in}{3.792329in}}{\pgfqpoint{2.242361in}{3.784515in}}%
\pgfpathcurveto{\pgfqpoint{2.250175in}{3.776701in}}{\pgfqpoint{2.260774in}{3.772311in}}{\pgfqpoint{2.271824in}{3.772311in}}%
\pgfpathclose%
\pgfusepath{stroke,fill}%
\end{pgfscope}%
\begin{pgfscope}%
\pgfpathrectangle{\pgfqpoint{0.481978in}{0.331635in}}{\pgfqpoint{9.300000in}{7.700000in}}%
\pgfusepath{clip}%
\pgfsetbuttcap%
\pgfsetroundjoin%
\definecolor{currentfill}{rgb}{0.631373,0.788235,0.956863}%
\pgfsetfillcolor{currentfill}%
\pgfsetlinewidth{0.481800pt}%
\definecolor{currentstroke}{rgb}{1.000000,1.000000,1.000000}%
\pgfsetstrokecolor{currentstroke}%
\pgfsetdash{}{0pt}%
\pgfpathmoveto{\pgfqpoint{6.175909in}{2.642661in}}%
\pgfpathcurveto{\pgfqpoint{6.186959in}{2.642661in}}{\pgfqpoint{6.197558in}{2.647052in}}{\pgfqpoint{6.205371in}{2.654865in}}%
\pgfpathcurveto{\pgfqpoint{6.213185in}{2.662679in}}{\pgfqpoint{6.217575in}{2.673278in}}{\pgfqpoint{6.217575in}{2.684328in}}%
\pgfpathcurveto{\pgfqpoint{6.217575in}{2.695378in}}{\pgfqpoint{6.213185in}{2.705977in}}{\pgfqpoint{6.205371in}{2.713791in}}%
\pgfpathcurveto{\pgfqpoint{6.197558in}{2.721604in}}{\pgfqpoint{6.186959in}{2.725995in}}{\pgfqpoint{6.175909in}{2.725995in}}%
\pgfpathcurveto{\pgfqpoint{6.164859in}{2.725995in}}{\pgfqpoint{6.154259in}{2.721604in}}{\pgfqpoint{6.146446in}{2.713791in}}%
\pgfpathcurveto{\pgfqpoint{6.138632in}{2.705977in}}{\pgfqpoint{6.134242in}{2.695378in}}{\pgfqpoint{6.134242in}{2.684328in}}%
\pgfpathcurveto{\pgfqpoint{6.134242in}{2.673278in}}{\pgfqpoint{6.138632in}{2.662679in}}{\pgfqpoint{6.146446in}{2.654865in}}%
\pgfpathcurveto{\pgfqpoint{6.154259in}{2.647052in}}{\pgfqpoint{6.164859in}{2.642661in}}{\pgfqpoint{6.175909in}{2.642661in}}%
\pgfpathclose%
\pgfusepath{stroke,fill}%
\end{pgfscope}%
\begin{pgfscope}%
\pgfpathrectangle{\pgfqpoint{0.481978in}{0.331635in}}{\pgfqpoint{9.300000in}{7.700000in}}%
\pgfusepath{clip}%
\pgfsetbuttcap%
\pgfsetroundjoin%
\definecolor{currentfill}{rgb}{0.631373,0.788235,0.956863}%
\pgfsetfillcolor{currentfill}%
\pgfsetlinewidth{0.481800pt}%
\definecolor{currentstroke}{rgb}{1.000000,1.000000,1.000000}%
\pgfsetstrokecolor{currentstroke}%
\pgfsetdash{}{0pt}%
\pgfpathmoveto{\pgfqpoint{5.076404in}{1.722632in}}%
\pgfpathcurveto{\pgfqpoint{5.087454in}{1.722632in}}{\pgfqpoint{5.098053in}{1.727022in}}{\pgfqpoint{5.105867in}{1.734836in}}%
\pgfpathcurveto{\pgfqpoint{5.113681in}{1.742650in}}{\pgfqpoint{5.118071in}{1.753249in}}{\pgfqpoint{5.118071in}{1.764299in}}%
\pgfpathcurveto{\pgfqpoint{5.118071in}{1.775349in}}{\pgfqpoint{5.113681in}{1.785948in}}{\pgfqpoint{5.105867in}{1.793761in}}%
\pgfpathcurveto{\pgfqpoint{5.098053in}{1.801575in}}{\pgfqpoint{5.087454in}{1.805965in}}{\pgfqpoint{5.076404in}{1.805965in}}%
\pgfpathcurveto{\pgfqpoint{5.065354in}{1.805965in}}{\pgfqpoint{5.054755in}{1.801575in}}{\pgfqpoint{5.046941in}{1.793761in}}%
\pgfpathcurveto{\pgfqpoint{5.039128in}{1.785948in}}{\pgfqpoint{5.034738in}{1.775349in}}{\pgfqpoint{5.034738in}{1.764299in}}%
\pgfpathcurveto{\pgfqpoint{5.034738in}{1.753249in}}{\pgfqpoint{5.039128in}{1.742650in}}{\pgfqpoint{5.046941in}{1.734836in}}%
\pgfpathcurveto{\pgfqpoint{5.054755in}{1.727022in}}{\pgfqpoint{5.065354in}{1.722632in}}{\pgfqpoint{5.076404in}{1.722632in}}%
\pgfpathclose%
\pgfusepath{stroke,fill}%
\end{pgfscope}%
\begin{pgfscope}%
\pgfpathrectangle{\pgfqpoint{0.481978in}{0.331635in}}{\pgfqpoint{9.300000in}{7.700000in}}%
\pgfusepath{clip}%
\pgfsetbuttcap%
\pgfsetroundjoin%
\definecolor{currentfill}{rgb}{0.631373,0.788235,0.956863}%
\pgfsetfillcolor{currentfill}%
\pgfsetlinewidth{0.481800pt}%
\definecolor{currentstroke}{rgb}{1.000000,1.000000,1.000000}%
\pgfsetstrokecolor{currentstroke}%
\pgfsetdash{}{0pt}%
\pgfpathmoveto{\pgfqpoint{8.665313in}{4.817063in}}%
\pgfpathcurveto{\pgfqpoint{8.676363in}{4.817063in}}{\pgfqpoint{8.686962in}{4.821453in}}{\pgfqpoint{8.694776in}{4.829267in}}%
\pgfpathcurveto{\pgfqpoint{8.702589in}{4.837080in}}{\pgfqpoint{8.706980in}{4.847679in}}{\pgfqpoint{8.706980in}{4.858730in}}%
\pgfpathcurveto{\pgfqpoint{8.706980in}{4.869780in}}{\pgfqpoint{8.702589in}{4.880379in}}{\pgfqpoint{8.694776in}{4.888192in}}%
\pgfpathcurveto{\pgfqpoint{8.686962in}{4.896006in}}{\pgfqpoint{8.676363in}{4.900396in}}{\pgfqpoint{8.665313in}{4.900396in}}%
\pgfpathcurveto{\pgfqpoint{8.654263in}{4.900396in}}{\pgfqpoint{8.643664in}{4.896006in}}{\pgfqpoint{8.635850in}{4.888192in}}%
\pgfpathcurveto{\pgfqpoint{8.628037in}{4.880379in}}{\pgfqpoint{8.623646in}{4.869780in}}{\pgfqpoint{8.623646in}{4.858730in}}%
\pgfpathcurveto{\pgfqpoint{8.623646in}{4.847679in}}{\pgfqpoint{8.628037in}{4.837080in}}{\pgfqpoint{8.635850in}{4.829267in}}%
\pgfpathcurveto{\pgfqpoint{8.643664in}{4.821453in}}{\pgfqpoint{8.654263in}{4.817063in}}{\pgfqpoint{8.665313in}{4.817063in}}%
\pgfpathclose%
\pgfusepath{stroke,fill}%
\end{pgfscope}%
\begin{pgfscope}%
\pgfpathrectangle{\pgfqpoint{0.481978in}{0.331635in}}{\pgfqpoint{9.300000in}{7.700000in}}%
\pgfusepath{clip}%
\pgfsetbuttcap%
\pgfsetroundjoin%
\definecolor{currentfill}{rgb}{0.631373,0.788235,0.956863}%
\pgfsetfillcolor{currentfill}%
\pgfsetlinewidth{0.481800pt}%
\definecolor{currentstroke}{rgb}{1.000000,1.000000,1.000000}%
\pgfsetstrokecolor{currentstroke}%
\pgfsetdash{}{0pt}%
\pgfpathmoveto{\pgfqpoint{7.138229in}{2.559103in}}%
\pgfpathcurveto{\pgfqpoint{7.149279in}{2.559103in}}{\pgfqpoint{7.159878in}{2.563493in}}{\pgfqpoint{7.167692in}{2.571306in}}%
\pgfpathcurveto{\pgfqpoint{7.175505in}{2.579120in}}{\pgfqpoint{7.179895in}{2.589719in}}{\pgfqpoint{7.179895in}{2.600769in}}%
\pgfpathcurveto{\pgfqpoint{7.179895in}{2.611819in}}{\pgfqpoint{7.175505in}{2.622418in}}{\pgfqpoint{7.167692in}{2.630232in}}%
\pgfpathcurveto{\pgfqpoint{7.159878in}{2.638046in}}{\pgfqpoint{7.149279in}{2.642436in}}{\pgfqpoint{7.138229in}{2.642436in}}%
\pgfpathcurveto{\pgfqpoint{7.127179in}{2.642436in}}{\pgfqpoint{7.116580in}{2.638046in}}{\pgfqpoint{7.108766in}{2.630232in}}%
\pgfpathcurveto{\pgfqpoint{7.100952in}{2.622418in}}{\pgfqpoint{7.096562in}{2.611819in}}{\pgfqpoint{7.096562in}{2.600769in}}%
\pgfpathcurveto{\pgfqpoint{7.096562in}{2.589719in}}{\pgfqpoint{7.100952in}{2.579120in}}{\pgfqpoint{7.108766in}{2.571306in}}%
\pgfpathcurveto{\pgfqpoint{7.116580in}{2.563493in}}{\pgfqpoint{7.127179in}{2.559103in}}{\pgfqpoint{7.138229in}{2.559103in}}%
\pgfpathclose%
\pgfusepath{stroke,fill}%
\end{pgfscope}%
\begin{pgfscope}%
\pgfpathrectangle{\pgfqpoint{0.481978in}{0.331635in}}{\pgfqpoint{9.300000in}{7.700000in}}%
\pgfusepath{clip}%
\pgfsetbuttcap%
\pgfsetroundjoin%
\definecolor{currentfill}{rgb}{0.631373,0.788235,0.956863}%
\pgfsetfillcolor{currentfill}%
\pgfsetlinewidth{0.481800pt}%
\definecolor{currentstroke}{rgb}{1.000000,1.000000,1.000000}%
\pgfsetstrokecolor{currentstroke}%
\pgfsetdash{}{0pt}%
\pgfpathmoveto{\pgfqpoint{6.963010in}{1.813724in}}%
\pgfpathcurveto{\pgfqpoint{6.974060in}{1.813724in}}{\pgfqpoint{6.984659in}{1.818115in}}{\pgfqpoint{6.992472in}{1.825928in}}%
\pgfpathcurveto{\pgfqpoint{7.000286in}{1.833742in}}{\pgfqpoint{7.004676in}{1.844341in}}{\pgfqpoint{7.004676in}{1.855391in}}%
\pgfpathcurveto{\pgfqpoint{7.004676in}{1.866441in}}{\pgfqpoint{7.000286in}{1.877040in}}{\pgfqpoint{6.992472in}{1.884854in}}%
\pgfpathcurveto{\pgfqpoint{6.984659in}{1.892667in}}{\pgfqpoint{6.974060in}{1.897058in}}{\pgfqpoint{6.963010in}{1.897058in}}%
\pgfpathcurveto{\pgfqpoint{6.951960in}{1.897058in}}{\pgfqpoint{6.941361in}{1.892667in}}{\pgfqpoint{6.933547in}{1.884854in}}%
\pgfpathcurveto{\pgfqpoint{6.925733in}{1.877040in}}{\pgfqpoint{6.921343in}{1.866441in}}{\pgfqpoint{6.921343in}{1.855391in}}%
\pgfpathcurveto{\pgfqpoint{6.921343in}{1.844341in}}{\pgfqpoint{6.925733in}{1.833742in}}{\pgfqpoint{6.933547in}{1.825928in}}%
\pgfpathcurveto{\pgfqpoint{6.941361in}{1.818115in}}{\pgfqpoint{6.951960in}{1.813724in}}{\pgfqpoint{6.963010in}{1.813724in}}%
\pgfpathclose%
\pgfusepath{stroke,fill}%
\end{pgfscope}%
\begin{pgfscope}%
\pgfpathrectangle{\pgfqpoint{0.481978in}{0.331635in}}{\pgfqpoint{9.300000in}{7.700000in}}%
\pgfusepath{clip}%
\pgfsetbuttcap%
\pgfsetroundjoin%
\definecolor{currentfill}{rgb}{0.631373,0.788235,0.956863}%
\pgfsetfillcolor{currentfill}%
\pgfsetlinewidth{0.481800pt}%
\definecolor{currentstroke}{rgb}{1.000000,1.000000,1.000000}%
\pgfsetstrokecolor{currentstroke}%
\pgfsetdash{}{0pt}%
\pgfpathmoveto{\pgfqpoint{8.113494in}{5.267796in}}%
\pgfpathcurveto{\pgfqpoint{8.124544in}{5.267796in}}{\pgfqpoint{8.135143in}{5.272186in}}{\pgfqpoint{8.142957in}{5.280000in}}%
\pgfpathcurveto{\pgfqpoint{8.150771in}{5.287814in}}{\pgfqpoint{8.155161in}{5.298413in}}{\pgfqpoint{8.155161in}{5.309463in}}%
\pgfpathcurveto{\pgfqpoint{8.155161in}{5.320513in}}{\pgfqpoint{8.150771in}{5.331112in}}{\pgfqpoint{8.142957in}{5.338926in}}%
\pgfpathcurveto{\pgfqpoint{8.135143in}{5.346739in}}{\pgfqpoint{8.124544in}{5.351130in}}{\pgfqpoint{8.113494in}{5.351130in}}%
\pgfpathcurveto{\pgfqpoint{8.102444in}{5.351130in}}{\pgfqpoint{8.091845in}{5.346739in}}{\pgfqpoint{8.084031in}{5.338926in}}%
\pgfpathcurveto{\pgfqpoint{8.076218in}{5.331112in}}{\pgfqpoint{8.071827in}{5.320513in}}{\pgfqpoint{8.071827in}{5.309463in}}%
\pgfpathcurveto{\pgfqpoint{8.071827in}{5.298413in}}{\pgfqpoint{8.076218in}{5.287814in}}{\pgfqpoint{8.084031in}{5.280000in}}%
\pgfpathcurveto{\pgfqpoint{8.091845in}{5.272186in}}{\pgfqpoint{8.102444in}{5.267796in}}{\pgfqpoint{8.113494in}{5.267796in}}%
\pgfpathclose%
\pgfusepath{stroke,fill}%
\end{pgfscope}%
\begin{pgfscope}%
\pgfpathrectangle{\pgfqpoint{0.481978in}{0.331635in}}{\pgfqpoint{9.300000in}{7.700000in}}%
\pgfusepath{clip}%
\pgfsetbuttcap%
\pgfsetroundjoin%
\definecolor{currentfill}{rgb}{0.631373,0.788235,0.956863}%
\pgfsetfillcolor{currentfill}%
\pgfsetlinewidth{0.481800pt}%
\definecolor{currentstroke}{rgb}{1.000000,1.000000,1.000000}%
\pgfsetstrokecolor{currentstroke}%
\pgfsetdash{}{0pt}%
\pgfpathmoveto{\pgfqpoint{5.616150in}{1.832494in}}%
\pgfpathcurveto{\pgfqpoint{5.627200in}{1.832494in}}{\pgfqpoint{5.637799in}{1.836884in}}{\pgfqpoint{5.645613in}{1.844698in}}%
\pgfpathcurveto{\pgfqpoint{5.653426in}{1.852511in}}{\pgfqpoint{5.657816in}{1.863110in}}{\pgfqpoint{5.657816in}{1.874161in}}%
\pgfpathcurveto{\pgfqpoint{5.657816in}{1.885211in}}{\pgfqpoint{5.653426in}{1.895810in}}{\pgfqpoint{5.645613in}{1.903623in}}%
\pgfpathcurveto{\pgfqpoint{5.637799in}{1.911437in}}{\pgfqpoint{5.627200in}{1.915827in}}{\pgfqpoint{5.616150in}{1.915827in}}%
\pgfpathcurveto{\pgfqpoint{5.605100in}{1.915827in}}{\pgfqpoint{5.594501in}{1.911437in}}{\pgfqpoint{5.586687in}{1.903623in}}%
\pgfpathcurveto{\pgfqpoint{5.578873in}{1.895810in}}{\pgfqpoint{5.574483in}{1.885211in}}{\pgfqpoint{5.574483in}{1.874161in}}%
\pgfpathcurveto{\pgfqpoint{5.574483in}{1.863110in}}{\pgfqpoint{5.578873in}{1.852511in}}{\pgfqpoint{5.586687in}{1.844698in}}%
\pgfpathcurveto{\pgfqpoint{5.594501in}{1.836884in}}{\pgfqpoint{5.605100in}{1.832494in}}{\pgfqpoint{5.616150in}{1.832494in}}%
\pgfpathclose%
\pgfusepath{stroke,fill}%
\end{pgfscope}%
\begin{pgfscope}%
\pgfpathrectangle{\pgfqpoint{0.481978in}{0.331635in}}{\pgfqpoint{9.300000in}{7.700000in}}%
\pgfusepath{clip}%
\pgfsetbuttcap%
\pgfsetroundjoin%
\definecolor{currentfill}{rgb}{0.631373,0.788235,0.956863}%
\pgfsetfillcolor{currentfill}%
\pgfsetlinewidth{0.481800pt}%
\definecolor{currentstroke}{rgb}{1.000000,1.000000,1.000000}%
\pgfsetstrokecolor{currentstroke}%
\pgfsetdash{}{0pt}%
\pgfpathmoveto{\pgfqpoint{4.785427in}{1.270778in}}%
\pgfpathcurveto{\pgfqpoint{4.796478in}{1.270778in}}{\pgfqpoint{4.807077in}{1.275168in}}{\pgfqpoint{4.814890in}{1.282982in}}%
\pgfpathcurveto{\pgfqpoint{4.822704in}{1.290796in}}{\pgfqpoint{4.827094in}{1.301395in}}{\pgfqpoint{4.827094in}{1.312445in}}%
\pgfpathcurveto{\pgfqpoint{4.827094in}{1.323495in}}{\pgfqpoint{4.822704in}{1.334094in}}{\pgfqpoint{4.814890in}{1.341908in}}%
\pgfpathcurveto{\pgfqpoint{4.807077in}{1.349721in}}{\pgfqpoint{4.796478in}{1.354111in}}{\pgfqpoint{4.785427in}{1.354111in}}%
\pgfpathcurveto{\pgfqpoint{4.774377in}{1.354111in}}{\pgfqpoint{4.763778in}{1.349721in}}{\pgfqpoint{4.755965in}{1.341908in}}%
\pgfpathcurveto{\pgfqpoint{4.748151in}{1.334094in}}{\pgfqpoint{4.743761in}{1.323495in}}{\pgfqpoint{4.743761in}{1.312445in}}%
\pgfpathcurveto{\pgfqpoint{4.743761in}{1.301395in}}{\pgfqpoint{4.748151in}{1.290796in}}{\pgfqpoint{4.755965in}{1.282982in}}%
\pgfpathcurveto{\pgfqpoint{4.763778in}{1.275168in}}{\pgfqpoint{4.774377in}{1.270778in}}{\pgfqpoint{4.785427in}{1.270778in}}%
\pgfpathclose%
\pgfusepath{stroke,fill}%
\end{pgfscope}%
\begin{pgfscope}%
\pgfpathrectangle{\pgfqpoint{0.481978in}{0.331635in}}{\pgfqpoint{9.300000in}{7.700000in}}%
\pgfusepath{clip}%
\pgfsetbuttcap%
\pgfsetroundjoin%
\definecolor{currentfill}{rgb}{0.631373,0.788235,0.956863}%
\pgfsetfillcolor{currentfill}%
\pgfsetlinewidth{0.481800pt}%
\definecolor{currentstroke}{rgb}{1.000000,1.000000,1.000000}%
\pgfsetstrokecolor{currentstroke}%
\pgfsetdash{}{0pt}%
\pgfpathmoveto{\pgfqpoint{8.275701in}{4.967909in}}%
\pgfpathcurveto{\pgfqpoint{8.286751in}{4.967909in}}{\pgfqpoint{8.297350in}{4.972299in}}{\pgfqpoint{8.305164in}{4.980113in}}%
\pgfpathcurveto{\pgfqpoint{8.312978in}{4.987927in}}{\pgfqpoint{8.317368in}{4.998526in}}{\pgfqpoint{8.317368in}{5.009576in}}%
\pgfpathcurveto{\pgfqpoint{8.317368in}{5.020626in}}{\pgfqpoint{8.312978in}{5.031225in}}{\pgfqpoint{8.305164in}{5.039039in}}%
\pgfpathcurveto{\pgfqpoint{8.297350in}{5.046852in}}{\pgfqpoint{8.286751in}{5.051242in}}{\pgfqpoint{8.275701in}{5.051242in}}%
\pgfpathcurveto{\pgfqpoint{8.264651in}{5.051242in}}{\pgfqpoint{8.254052in}{5.046852in}}{\pgfqpoint{8.246238in}{5.039039in}}%
\pgfpathcurveto{\pgfqpoint{8.238425in}{5.031225in}}{\pgfqpoint{8.234035in}{5.020626in}}{\pgfqpoint{8.234035in}{5.009576in}}%
\pgfpathcurveto{\pgfqpoint{8.234035in}{4.998526in}}{\pgfqpoint{8.238425in}{4.987927in}}{\pgfqpoint{8.246238in}{4.980113in}}%
\pgfpathcurveto{\pgfqpoint{8.254052in}{4.972299in}}{\pgfqpoint{8.264651in}{4.967909in}}{\pgfqpoint{8.275701in}{4.967909in}}%
\pgfpathclose%
\pgfusepath{stroke,fill}%
\end{pgfscope}%
\begin{pgfscope}%
\pgfpathrectangle{\pgfqpoint{0.481978in}{0.331635in}}{\pgfqpoint{9.300000in}{7.700000in}}%
\pgfusepath{clip}%
\pgfsetbuttcap%
\pgfsetroundjoin%
\definecolor{currentfill}{rgb}{0.631373,0.788235,0.956863}%
\pgfsetfillcolor{currentfill}%
\pgfsetlinewidth{0.481800pt}%
\definecolor{currentstroke}{rgb}{1.000000,1.000000,1.000000}%
\pgfsetstrokecolor{currentstroke}%
\pgfsetdash{}{0pt}%
\pgfpathmoveto{\pgfqpoint{7.698866in}{5.076520in}}%
\pgfpathcurveto{\pgfqpoint{7.709916in}{5.076520in}}{\pgfqpoint{7.720515in}{5.080910in}}{\pgfqpoint{7.728329in}{5.088724in}}%
\pgfpathcurveto{\pgfqpoint{7.736142in}{5.096538in}}{\pgfqpoint{7.740533in}{5.107137in}}{\pgfqpoint{7.740533in}{5.118187in}}%
\pgfpathcurveto{\pgfqpoint{7.740533in}{5.129237in}}{\pgfqpoint{7.736142in}{5.139836in}}{\pgfqpoint{7.728329in}{5.147650in}}%
\pgfpathcurveto{\pgfqpoint{7.720515in}{5.155463in}}{\pgfqpoint{7.709916in}{5.159853in}}{\pgfqpoint{7.698866in}{5.159853in}}%
\pgfpathcurveto{\pgfqpoint{7.687816in}{5.159853in}}{\pgfqpoint{7.677217in}{5.155463in}}{\pgfqpoint{7.669403in}{5.147650in}}%
\pgfpathcurveto{\pgfqpoint{7.661589in}{5.139836in}}{\pgfqpoint{7.657199in}{5.129237in}}{\pgfqpoint{7.657199in}{5.118187in}}%
\pgfpathcurveto{\pgfqpoint{7.657199in}{5.107137in}}{\pgfqpoint{7.661589in}{5.096538in}}{\pgfqpoint{7.669403in}{5.088724in}}%
\pgfpathcurveto{\pgfqpoint{7.677217in}{5.080910in}}{\pgfqpoint{7.687816in}{5.076520in}}{\pgfqpoint{7.698866in}{5.076520in}}%
\pgfpathclose%
\pgfusepath{stroke,fill}%
\end{pgfscope}%
\begin{pgfscope}%
\pgfpathrectangle{\pgfqpoint{0.481978in}{0.331635in}}{\pgfqpoint{9.300000in}{7.700000in}}%
\pgfusepath{clip}%
\pgfsetbuttcap%
\pgfsetroundjoin%
\definecolor{currentfill}{rgb}{0.631373,0.788235,0.956863}%
\pgfsetfillcolor{currentfill}%
\pgfsetlinewidth{0.481800pt}%
\definecolor{currentstroke}{rgb}{1.000000,1.000000,1.000000}%
\pgfsetstrokecolor{currentstroke}%
\pgfsetdash{}{0pt}%
\pgfpathmoveto{\pgfqpoint{2.744094in}{5.008868in}}%
\pgfpathcurveto{\pgfqpoint{2.755144in}{5.008868in}}{\pgfqpoint{2.765743in}{5.013259in}}{\pgfqpoint{2.773557in}{5.021072in}}%
\pgfpathcurveto{\pgfqpoint{2.781370in}{5.028886in}}{\pgfqpoint{2.785760in}{5.039485in}}{\pgfqpoint{2.785760in}{5.050535in}}%
\pgfpathcurveto{\pgfqpoint{2.785760in}{5.061585in}}{\pgfqpoint{2.781370in}{5.072184in}}{\pgfqpoint{2.773557in}{5.079998in}}%
\pgfpathcurveto{\pgfqpoint{2.765743in}{5.087812in}}{\pgfqpoint{2.755144in}{5.092202in}}{\pgfqpoint{2.744094in}{5.092202in}}%
\pgfpathcurveto{\pgfqpoint{2.733044in}{5.092202in}}{\pgfqpoint{2.722445in}{5.087812in}}{\pgfqpoint{2.714631in}{5.079998in}}%
\pgfpathcurveto{\pgfqpoint{2.706817in}{5.072184in}}{\pgfqpoint{2.702427in}{5.061585in}}{\pgfqpoint{2.702427in}{5.050535in}}%
\pgfpathcurveto{\pgfqpoint{2.702427in}{5.039485in}}{\pgfqpoint{2.706817in}{5.028886in}}{\pgfqpoint{2.714631in}{5.021072in}}%
\pgfpathcurveto{\pgfqpoint{2.722445in}{5.013259in}}{\pgfqpoint{2.733044in}{5.008868in}}{\pgfqpoint{2.744094in}{5.008868in}}%
\pgfpathclose%
\pgfusepath{stroke,fill}%
\end{pgfscope}%
\begin{pgfscope}%
\pgfpathrectangle{\pgfqpoint{0.481978in}{0.331635in}}{\pgfqpoint{9.300000in}{7.700000in}}%
\pgfusepath{clip}%
\pgfsetbuttcap%
\pgfsetroundjoin%
\definecolor{currentfill}{rgb}{0.631373,0.788235,0.956863}%
\pgfsetfillcolor{currentfill}%
\pgfsetlinewidth{0.481800pt}%
\definecolor{currentstroke}{rgb}{1.000000,1.000000,1.000000}%
\pgfsetstrokecolor{currentstroke}%
\pgfsetdash{}{0pt}%
\pgfpathmoveto{\pgfqpoint{7.761828in}{4.277429in}}%
\pgfpathcurveto{\pgfqpoint{7.772878in}{4.277429in}}{\pgfqpoint{7.783477in}{4.281819in}}{\pgfqpoint{7.791291in}{4.289633in}}%
\pgfpathcurveto{\pgfqpoint{7.799104in}{4.297446in}}{\pgfqpoint{7.803495in}{4.308045in}}{\pgfqpoint{7.803495in}{4.319095in}}%
\pgfpathcurveto{\pgfqpoint{7.803495in}{4.330145in}}{\pgfqpoint{7.799104in}{4.340744in}}{\pgfqpoint{7.791291in}{4.348558in}}%
\pgfpathcurveto{\pgfqpoint{7.783477in}{4.356372in}}{\pgfqpoint{7.772878in}{4.360762in}}{\pgfqpoint{7.761828in}{4.360762in}}%
\pgfpathcurveto{\pgfqpoint{7.750778in}{4.360762in}}{\pgfqpoint{7.740179in}{4.356372in}}{\pgfqpoint{7.732365in}{4.348558in}}%
\pgfpathcurveto{\pgfqpoint{7.724551in}{4.340744in}}{\pgfqpoint{7.720161in}{4.330145in}}{\pgfqpoint{7.720161in}{4.319095in}}%
\pgfpathcurveto{\pgfqpoint{7.720161in}{4.308045in}}{\pgfqpoint{7.724551in}{4.297446in}}{\pgfqpoint{7.732365in}{4.289633in}}%
\pgfpathcurveto{\pgfqpoint{7.740179in}{4.281819in}}{\pgfqpoint{7.750778in}{4.277429in}}{\pgfqpoint{7.761828in}{4.277429in}}%
\pgfpathclose%
\pgfusepath{stroke,fill}%
\end{pgfscope}%
\begin{pgfscope}%
\pgfpathrectangle{\pgfqpoint{0.481978in}{0.331635in}}{\pgfqpoint{9.300000in}{7.700000in}}%
\pgfusepath{clip}%
\pgfsetbuttcap%
\pgfsetroundjoin%
\definecolor{currentfill}{rgb}{0.631373,0.788235,0.956863}%
\pgfsetfillcolor{currentfill}%
\pgfsetlinewidth{0.481800pt}%
\definecolor{currentstroke}{rgb}{1.000000,1.000000,1.000000}%
\pgfsetstrokecolor{currentstroke}%
\pgfsetdash{}{0pt}%
\pgfpathmoveto{\pgfqpoint{6.999874in}{2.207732in}}%
\pgfpathcurveto{\pgfqpoint{7.010924in}{2.207732in}}{\pgfqpoint{7.021523in}{2.212123in}}{\pgfqpoint{7.029337in}{2.219936in}}%
\pgfpathcurveto{\pgfqpoint{7.037150in}{2.227750in}}{\pgfqpoint{7.041540in}{2.238349in}}{\pgfqpoint{7.041540in}{2.249399in}}%
\pgfpathcurveto{\pgfqpoint{7.041540in}{2.260449in}}{\pgfqpoint{7.037150in}{2.271048in}}{\pgfqpoint{7.029337in}{2.278862in}}%
\pgfpathcurveto{\pgfqpoint{7.021523in}{2.286675in}}{\pgfqpoint{7.010924in}{2.291066in}}{\pgfqpoint{6.999874in}{2.291066in}}%
\pgfpathcurveto{\pgfqpoint{6.988824in}{2.291066in}}{\pgfqpoint{6.978225in}{2.286675in}}{\pgfqpoint{6.970411in}{2.278862in}}%
\pgfpathcurveto{\pgfqpoint{6.962597in}{2.271048in}}{\pgfqpoint{6.958207in}{2.260449in}}{\pgfqpoint{6.958207in}{2.249399in}}%
\pgfpathcurveto{\pgfqpoint{6.958207in}{2.238349in}}{\pgfqpoint{6.962597in}{2.227750in}}{\pgfqpoint{6.970411in}{2.219936in}}%
\pgfpathcurveto{\pgfqpoint{6.978225in}{2.212123in}}{\pgfqpoint{6.988824in}{2.207732in}}{\pgfqpoint{6.999874in}{2.207732in}}%
\pgfpathclose%
\pgfusepath{stroke,fill}%
\end{pgfscope}%
\begin{pgfscope}%
\pgfpathrectangle{\pgfqpoint{0.481978in}{0.331635in}}{\pgfqpoint{9.300000in}{7.700000in}}%
\pgfusepath{clip}%
\pgfsetbuttcap%
\pgfsetroundjoin%
\definecolor{currentfill}{rgb}{0.631373,0.788235,0.956863}%
\pgfsetfillcolor{currentfill}%
\pgfsetlinewidth{0.481800pt}%
\definecolor{currentstroke}{rgb}{1.000000,1.000000,1.000000}%
\pgfsetstrokecolor{currentstroke}%
\pgfsetdash{}{0pt}%
\pgfpathmoveto{\pgfqpoint{5.604350in}{5.389083in}}%
\pgfpathcurveto{\pgfqpoint{5.615400in}{5.389083in}}{\pgfqpoint{5.625999in}{5.393473in}}{\pgfqpoint{5.633813in}{5.401287in}}%
\pgfpathcurveto{\pgfqpoint{5.641627in}{5.409101in}}{\pgfqpoint{5.646017in}{5.419700in}}{\pgfqpoint{5.646017in}{5.430750in}}%
\pgfpathcurveto{\pgfqpoint{5.646017in}{5.441800in}}{\pgfqpoint{5.641627in}{5.452399in}}{\pgfqpoint{5.633813in}{5.460213in}}%
\pgfpathcurveto{\pgfqpoint{5.625999in}{5.468026in}}{\pgfqpoint{5.615400in}{5.472416in}}{\pgfqpoint{5.604350in}{5.472416in}}%
\pgfpathcurveto{\pgfqpoint{5.593300in}{5.472416in}}{\pgfqpoint{5.582701in}{5.468026in}}{\pgfqpoint{5.574887in}{5.460213in}}%
\pgfpathcurveto{\pgfqpoint{5.567074in}{5.452399in}}{\pgfqpoint{5.562684in}{5.441800in}}{\pgfqpoint{5.562684in}{5.430750in}}%
\pgfpathcurveto{\pgfqpoint{5.562684in}{5.419700in}}{\pgfqpoint{5.567074in}{5.409101in}}{\pgfqpoint{5.574887in}{5.401287in}}%
\pgfpathcurveto{\pgfqpoint{5.582701in}{5.393473in}}{\pgfqpoint{5.593300in}{5.389083in}}{\pgfqpoint{5.604350in}{5.389083in}}%
\pgfpathclose%
\pgfusepath{stroke,fill}%
\end{pgfscope}%
\begin{pgfscope}%
\pgfpathrectangle{\pgfqpoint{0.481978in}{0.331635in}}{\pgfqpoint{9.300000in}{7.700000in}}%
\pgfusepath{clip}%
\pgfsetbuttcap%
\pgfsetroundjoin%
\definecolor{currentfill}{rgb}{0.631373,0.788235,0.956863}%
\pgfsetfillcolor{currentfill}%
\pgfsetlinewidth{0.481800pt}%
\definecolor{currentstroke}{rgb}{1.000000,1.000000,1.000000}%
\pgfsetstrokecolor{currentstroke}%
\pgfsetdash{}{0pt}%
\pgfpathmoveto{\pgfqpoint{2.349323in}{2.233215in}}%
\pgfpathcurveto{\pgfqpoint{2.360373in}{2.233215in}}{\pgfqpoint{2.370972in}{2.237605in}}{\pgfqpoint{2.378786in}{2.245419in}}%
\pgfpathcurveto{\pgfqpoint{2.386599in}{2.253233in}}{\pgfqpoint{2.390990in}{2.263832in}}{\pgfqpoint{2.390990in}{2.274882in}}%
\pgfpathcurveto{\pgfqpoint{2.390990in}{2.285932in}}{\pgfqpoint{2.386599in}{2.296531in}}{\pgfqpoint{2.378786in}{2.304345in}}%
\pgfpathcurveto{\pgfqpoint{2.370972in}{2.312158in}}{\pgfqpoint{2.360373in}{2.316549in}}{\pgfqpoint{2.349323in}{2.316549in}}%
\pgfpathcurveto{\pgfqpoint{2.338273in}{2.316549in}}{\pgfqpoint{2.327674in}{2.312158in}}{\pgfqpoint{2.319860in}{2.304345in}}%
\pgfpathcurveto{\pgfqpoint{2.312046in}{2.296531in}}{\pgfqpoint{2.307656in}{2.285932in}}{\pgfqpoint{2.307656in}{2.274882in}}%
\pgfpathcurveto{\pgfqpoint{2.307656in}{2.263832in}}{\pgfqpoint{2.312046in}{2.253233in}}{\pgfqpoint{2.319860in}{2.245419in}}%
\pgfpathcurveto{\pgfqpoint{2.327674in}{2.237605in}}{\pgfqpoint{2.338273in}{2.233215in}}{\pgfqpoint{2.349323in}{2.233215in}}%
\pgfpathclose%
\pgfusepath{stroke,fill}%
\end{pgfscope}%
\begin{pgfscope}%
\pgfpathrectangle{\pgfqpoint{0.481978in}{0.331635in}}{\pgfqpoint{9.300000in}{7.700000in}}%
\pgfusepath{clip}%
\pgfsetbuttcap%
\pgfsetroundjoin%
\definecolor{currentfill}{rgb}{0.631373,0.788235,0.956863}%
\pgfsetfillcolor{currentfill}%
\pgfsetlinewidth{0.481800pt}%
\definecolor{currentstroke}{rgb}{1.000000,1.000000,1.000000}%
\pgfsetstrokecolor{currentstroke}%
\pgfsetdash{}{0pt}%
\pgfpathmoveto{\pgfqpoint{6.569928in}{3.215334in}}%
\pgfpathcurveto{\pgfqpoint{6.580979in}{3.215334in}}{\pgfqpoint{6.591578in}{3.219724in}}{\pgfqpoint{6.599391in}{3.227538in}}%
\pgfpathcurveto{\pgfqpoint{6.607205in}{3.235351in}}{\pgfqpoint{6.611595in}{3.245950in}}{\pgfqpoint{6.611595in}{3.257001in}}%
\pgfpathcurveto{\pgfqpoint{6.611595in}{3.268051in}}{\pgfqpoint{6.607205in}{3.278650in}}{\pgfqpoint{6.599391in}{3.286463in}}%
\pgfpathcurveto{\pgfqpoint{6.591578in}{3.294277in}}{\pgfqpoint{6.580979in}{3.298667in}}{\pgfqpoint{6.569928in}{3.298667in}}%
\pgfpathcurveto{\pgfqpoint{6.558878in}{3.298667in}}{\pgfqpoint{6.548279in}{3.294277in}}{\pgfqpoint{6.540466in}{3.286463in}}%
\pgfpathcurveto{\pgfqpoint{6.532652in}{3.278650in}}{\pgfqpoint{6.528262in}{3.268051in}}{\pgfqpoint{6.528262in}{3.257001in}}%
\pgfpathcurveto{\pgfqpoint{6.528262in}{3.245950in}}{\pgfqpoint{6.532652in}{3.235351in}}{\pgfqpoint{6.540466in}{3.227538in}}%
\pgfpathcurveto{\pgfqpoint{6.548279in}{3.219724in}}{\pgfqpoint{6.558878in}{3.215334in}}{\pgfqpoint{6.569928in}{3.215334in}}%
\pgfpathclose%
\pgfusepath{stroke,fill}%
\end{pgfscope}%
\begin{pgfscope}%
\pgfpathrectangle{\pgfqpoint{0.481978in}{0.331635in}}{\pgfqpoint{9.300000in}{7.700000in}}%
\pgfusepath{clip}%
\pgfsetbuttcap%
\pgfsetroundjoin%
\definecolor{currentfill}{rgb}{0.631373,0.788235,0.956863}%
\pgfsetfillcolor{currentfill}%
\pgfsetlinewidth{0.481800pt}%
\definecolor{currentstroke}{rgb}{1.000000,1.000000,1.000000}%
\pgfsetstrokecolor{currentstroke}%
\pgfsetdash{}{0pt}%
\pgfpathmoveto{\pgfqpoint{5.147663in}{2.571148in}}%
\pgfpathcurveto{\pgfqpoint{5.158713in}{2.571148in}}{\pgfqpoint{5.169312in}{2.575539in}}{\pgfqpoint{5.177125in}{2.583352in}}%
\pgfpathcurveto{\pgfqpoint{5.184939in}{2.591166in}}{\pgfqpoint{5.189329in}{2.601765in}}{\pgfqpoint{5.189329in}{2.612815in}}%
\pgfpathcurveto{\pgfqpoint{5.189329in}{2.623865in}}{\pgfqpoint{5.184939in}{2.634464in}}{\pgfqpoint{5.177125in}{2.642278in}}%
\pgfpathcurveto{\pgfqpoint{5.169312in}{2.650091in}}{\pgfqpoint{5.158713in}{2.654482in}}{\pgfqpoint{5.147663in}{2.654482in}}%
\pgfpathcurveto{\pgfqpoint{5.136613in}{2.654482in}}{\pgfqpoint{5.126014in}{2.650091in}}{\pgfqpoint{5.118200in}{2.642278in}}%
\pgfpathcurveto{\pgfqpoint{5.110386in}{2.634464in}}{\pgfqpoint{5.105996in}{2.623865in}}{\pgfqpoint{5.105996in}{2.612815in}}%
\pgfpathcurveto{\pgfqpoint{5.105996in}{2.601765in}}{\pgfqpoint{5.110386in}{2.591166in}}{\pgfqpoint{5.118200in}{2.583352in}}%
\pgfpathcurveto{\pgfqpoint{5.126014in}{2.575539in}}{\pgfqpoint{5.136613in}{2.571148in}}{\pgfqpoint{5.147663in}{2.571148in}}%
\pgfpathclose%
\pgfusepath{stroke,fill}%
\end{pgfscope}%
\begin{pgfscope}%
\pgfpathrectangle{\pgfqpoint{0.481978in}{0.331635in}}{\pgfqpoint{9.300000in}{7.700000in}}%
\pgfusepath{clip}%
\pgfsetbuttcap%
\pgfsetroundjoin%
\definecolor{currentfill}{rgb}{0.631373,0.788235,0.956863}%
\pgfsetfillcolor{currentfill}%
\pgfsetlinewidth{0.481800pt}%
\definecolor{currentstroke}{rgb}{1.000000,1.000000,1.000000}%
\pgfsetstrokecolor{currentstroke}%
\pgfsetdash{}{0pt}%
\pgfpathmoveto{\pgfqpoint{6.704806in}{2.191913in}}%
\pgfpathcurveto{\pgfqpoint{6.715856in}{2.191913in}}{\pgfqpoint{6.726455in}{2.196303in}}{\pgfqpoint{6.734269in}{2.204117in}}%
\pgfpathcurveto{\pgfqpoint{6.742083in}{2.211930in}}{\pgfqpoint{6.746473in}{2.222529in}}{\pgfqpoint{6.746473in}{2.233579in}}%
\pgfpathcurveto{\pgfqpoint{6.746473in}{2.244629in}}{\pgfqpoint{6.742083in}{2.255228in}}{\pgfqpoint{6.734269in}{2.263042in}}%
\pgfpathcurveto{\pgfqpoint{6.726455in}{2.270856in}}{\pgfqpoint{6.715856in}{2.275246in}}{\pgfqpoint{6.704806in}{2.275246in}}%
\pgfpathcurveto{\pgfqpoint{6.693756in}{2.275246in}}{\pgfqpoint{6.683157in}{2.270856in}}{\pgfqpoint{6.675343in}{2.263042in}}%
\pgfpathcurveto{\pgfqpoint{6.667530in}{2.255228in}}{\pgfqpoint{6.663140in}{2.244629in}}{\pgfqpoint{6.663140in}{2.233579in}}%
\pgfpathcurveto{\pgfqpoint{6.663140in}{2.222529in}}{\pgfqpoint{6.667530in}{2.211930in}}{\pgfqpoint{6.675343in}{2.204117in}}%
\pgfpathcurveto{\pgfqpoint{6.683157in}{2.196303in}}{\pgfqpoint{6.693756in}{2.191913in}}{\pgfqpoint{6.704806in}{2.191913in}}%
\pgfpathclose%
\pgfusepath{stroke,fill}%
\end{pgfscope}%
\begin{pgfscope}%
\pgfpathrectangle{\pgfqpoint{0.481978in}{0.331635in}}{\pgfqpoint{9.300000in}{7.700000in}}%
\pgfusepath{clip}%
\pgfsetbuttcap%
\pgfsetroundjoin%
\definecolor{currentfill}{rgb}{0.631373,0.788235,0.956863}%
\pgfsetfillcolor{currentfill}%
\pgfsetlinewidth{0.481800pt}%
\definecolor{currentstroke}{rgb}{1.000000,1.000000,1.000000}%
\pgfsetstrokecolor{currentstroke}%
\pgfsetdash{}{0pt}%
\pgfpathmoveto{\pgfqpoint{2.794290in}{5.331159in}}%
\pgfpathcurveto{\pgfqpoint{2.805340in}{5.331159in}}{\pgfqpoint{2.815940in}{5.335549in}}{\pgfqpoint{2.823753in}{5.343363in}}%
\pgfpathcurveto{\pgfqpoint{2.831567in}{5.351176in}}{\pgfqpoint{2.835957in}{5.361775in}}{\pgfqpoint{2.835957in}{5.372826in}}%
\pgfpathcurveto{\pgfqpoint{2.835957in}{5.383876in}}{\pgfqpoint{2.831567in}{5.394475in}}{\pgfqpoint{2.823753in}{5.402288in}}%
\pgfpathcurveto{\pgfqpoint{2.815940in}{5.410102in}}{\pgfqpoint{2.805340in}{5.414492in}}{\pgfqpoint{2.794290in}{5.414492in}}%
\pgfpathcurveto{\pgfqpoint{2.783240in}{5.414492in}}{\pgfqpoint{2.772641in}{5.410102in}}{\pgfqpoint{2.764828in}{5.402288in}}%
\pgfpathcurveto{\pgfqpoint{2.757014in}{5.394475in}}{\pgfqpoint{2.752624in}{5.383876in}}{\pgfqpoint{2.752624in}{5.372826in}}%
\pgfpathcurveto{\pgfqpoint{2.752624in}{5.361775in}}{\pgfqpoint{2.757014in}{5.351176in}}{\pgfqpoint{2.764828in}{5.343363in}}%
\pgfpathcurveto{\pgfqpoint{2.772641in}{5.335549in}}{\pgfqpoint{2.783240in}{5.331159in}}{\pgfqpoint{2.794290in}{5.331159in}}%
\pgfpathclose%
\pgfusepath{stroke,fill}%
\end{pgfscope}%
\begin{pgfscope}%
\pgfpathrectangle{\pgfqpoint{0.481978in}{0.331635in}}{\pgfqpoint{9.300000in}{7.700000in}}%
\pgfusepath{clip}%
\pgfsetbuttcap%
\pgfsetroundjoin%
\definecolor{currentfill}{rgb}{0.631373,0.788235,0.956863}%
\pgfsetfillcolor{currentfill}%
\pgfsetlinewidth{0.481800pt}%
\definecolor{currentstroke}{rgb}{1.000000,1.000000,1.000000}%
\pgfsetstrokecolor{currentstroke}%
\pgfsetdash{}{0pt}%
\pgfpathmoveto{\pgfqpoint{8.083200in}{4.965417in}}%
\pgfpathcurveto{\pgfqpoint{8.094250in}{4.965417in}}{\pgfqpoint{8.104849in}{4.969807in}}{\pgfqpoint{8.112663in}{4.977621in}}%
\pgfpathcurveto{\pgfqpoint{8.120477in}{4.985434in}}{\pgfqpoint{8.124867in}{4.996033in}}{\pgfqpoint{8.124867in}{5.007083in}}%
\pgfpathcurveto{\pgfqpoint{8.124867in}{5.018134in}}{\pgfqpoint{8.120477in}{5.028733in}}{\pgfqpoint{8.112663in}{5.036546in}}%
\pgfpathcurveto{\pgfqpoint{8.104849in}{5.044360in}}{\pgfqpoint{8.094250in}{5.048750in}}{\pgfqpoint{8.083200in}{5.048750in}}%
\pgfpathcurveto{\pgfqpoint{8.072150in}{5.048750in}}{\pgfqpoint{8.061551in}{5.044360in}}{\pgfqpoint{8.053738in}{5.036546in}}%
\pgfpathcurveto{\pgfqpoint{8.045924in}{5.028733in}}{\pgfqpoint{8.041534in}{5.018134in}}{\pgfqpoint{8.041534in}{5.007083in}}%
\pgfpathcurveto{\pgfqpoint{8.041534in}{4.996033in}}{\pgfqpoint{8.045924in}{4.985434in}}{\pgfqpoint{8.053738in}{4.977621in}}%
\pgfpathcurveto{\pgfqpoint{8.061551in}{4.969807in}}{\pgfqpoint{8.072150in}{4.965417in}}{\pgfqpoint{8.083200in}{4.965417in}}%
\pgfpathclose%
\pgfusepath{stroke,fill}%
\end{pgfscope}%
\begin{pgfscope}%
\pgfpathrectangle{\pgfqpoint{0.481978in}{0.331635in}}{\pgfqpoint{9.300000in}{7.700000in}}%
\pgfusepath{clip}%
\pgfsetbuttcap%
\pgfsetroundjoin%
\definecolor{currentfill}{rgb}{0.631373,0.788235,0.956863}%
\pgfsetfillcolor{currentfill}%
\pgfsetlinewidth{0.481800pt}%
\definecolor{currentstroke}{rgb}{1.000000,1.000000,1.000000}%
\pgfsetstrokecolor{currentstroke}%
\pgfsetdash{}{0pt}%
\pgfpathmoveto{\pgfqpoint{4.500321in}{1.281036in}}%
\pgfpathcurveto{\pgfqpoint{4.511371in}{1.281036in}}{\pgfqpoint{4.521970in}{1.285427in}}{\pgfqpoint{4.529783in}{1.293240in}}%
\pgfpathcurveto{\pgfqpoint{4.537597in}{1.301054in}}{\pgfqpoint{4.541987in}{1.311653in}}{\pgfqpoint{4.541987in}{1.322703in}}%
\pgfpathcurveto{\pgfqpoint{4.541987in}{1.333753in}}{\pgfqpoint{4.537597in}{1.344352in}}{\pgfqpoint{4.529783in}{1.352166in}}%
\pgfpathcurveto{\pgfqpoint{4.521970in}{1.359980in}}{\pgfqpoint{4.511371in}{1.364370in}}{\pgfqpoint{4.500321in}{1.364370in}}%
\pgfpathcurveto{\pgfqpoint{4.489270in}{1.364370in}}{\pgfqpoint{4.478671in}{1.359980in}}{\pgfqpoint{4.470858in}{1.352166in}}%
\pgfpathcurveto{\pgfqpoint{4.463044in}{1.344352in}}{\pgfqpoint{4.458654in}{1.333753in}}{\pgfqpoint{4.458654in}{1.322703in}}%
\pgfpathcurveto{\pgfqpoint{4.458654in}{1.311653in}}{\pgfqpoint{4.463044in}{1.301054in}}{\pgfqpoint{4.470858in}{1.293240in}}%
\pgfpathcurveto{\pgfqpoint{4.478671in}{1.285427in}}{\pgfqpoint{4.489270in}{1.281036in}}{\pgfqpoint{4.500321in}{1.281036in}}%
\pgfpathclose%
\pgfusepath{stroke,fill}%
\end{pgfscope}%
\begin{pgfscope}%
\pgfpathrectangle{\pgfqpoint{0.481978in}{0.331635in}}{\pgfqpoint{9.300000in}{7.700000in}}%
\pgfusepath{clip}%
\pgfsetbuttcap%
\pgfsetroundjoin%
\definecolor{currentfill}{rgb}{0.631373,0.788235,0.956863}%
\pgfsetfillcolor{currentfill}%
\pgfsetlinewidth{0.481800pt}%
\definecolor{currentstroke}{rgb}{1.000000,1.000000,1.000000}%
\pgfsetstrokecolor{currentstroke}%
\pgfsetdash{}{0pt}%
\pgfpathmoveto{\pgfqpoint{3.168751in}{1.333831in}}%
\pgfpathcurveto{\pgfqpoint{3.179801in}{1.333831in}}{\pgfqpoint{3.190400in}{1.338221in}}{\pgfqpoint{3.198214in}{1.346035in}}%
\pgfpathcurveto{\pgfqpoint{3.206027in}{1.353849in}}{\pgfqpoint{3.210418in}{1.364448in}}{\pgfqpoint{3.210418in}{1.375498in}}%
\pgfpathcurveto{\pgfqpoint{3.210418in}{1.386548in}}{\pgfqpoint{3.206027in}{1.397147in}}{\pgfqpoint{3.198214in}{1.404961in}}%
\pgfpathcurveto{\pgfqpoint{3.190400in}{1.412774in}}{\pgfqpoint{3.179801in}{1.417164in}}{\pgfqpoint{3.168751in}{1.417164in}}%
\pgfpathcurveto{\pgfqpoint{3.157701in}{1.417164in}}{\pgfqpoint{3.147102in}{1.412774in}}{\pgfqpoint{3.139288in}{1.404961in}}%
\pgfpathcurveto{\pgfqpoint{3.131475in}{1.397147in}}{\pgfqpoint{3.127084in}{1.386548in}}{\pgfqpoint{3.127084in}{1.375498in}}%
\pgfpathcurveto{\pgfqpoint{3.127084in}{1.364448in}}{\pgfqpoint{3.131475in}{1.353849in}}{\pgfqpoint{3.139288in}{1.346035in}}%
\pgfpathcurveto{\pgfqpoint{3.147102in}{1.338221in}}{\pgfqpoint{3.157701in}{1.333831in}}{\pgfqpoint{3.168751in}{1.333831in}}%
\pgfpathclose%
\pgfusepath{stroke,fill}%
\end{pgfscope}%
\begin{pgfscope}%
\pgfpathrectangle{\pgfqpoint{0.481978in}{0.331635in}}{\pgfqpoint{9.300000in}{7.700000in}}%
\pgfusepath{clip}%
\pgfsetbuttcap%
\pgfsetroundjoin%
\definecolor{currentfill}{rgb}{0.631373,0.788235,0.956863}%
\pgfsetfillcolor{currentfill}%
\pgfsetlinewidth{0.481800pt}%
\definecolor{currentstroke}{rgb}{1.000000,1.000000,1.000000}%
\pgfsetstrokecolor{currentstroke}%
\pgfsetdash{}{0pt}%
\pgfpathmoveto{\pgfqpoint{4.177608in}{7.639968in}}%
\pgfpathcurveto{\pgfqpoint{4.188659in}{7.639968in}}{\pgfqpoint{4.199258in}{7.644359in}}{\pgfqpoint{4.207071in}{7.652172in}}%
\pgfpathcurveto{\pgfqpoint{4.214885in}{7.659986in}}{\pgfqpoint{4.219275in}{7.670585in}}{\pgfqpoint{4.219275in}{7.681635in}}%
\pgfpathcurveto{\pgfqpoint{4.219275in}{7.692685in}}{\pgfqpoint{4.214885in}{7.703284in}}{\pgfqpoint{4.207071in}{7.711098in}}%
\pgfpathcurveto{\pgfqpoint{4.199258in}{7.718911in}}{\pgfqpoint{4.188659in}{7.723302in}}{\pgfqpoint{4.177608in}{7.723302in}}%
\pgfpathcurveto{\pgfqpoint{4.166558in}{7.723302in}}{\pgfqpoint{4.155959in}{7.718911in}}{\pgfqpoint{4.148146in}{7.711098in}}%
\pgfpathcurveto{\pgfqpoint{4.140332in}{7.703284in}}{\pgfqpoint{4.135942in}{7.692685in}}{\pgfqpoint{4.135942in}{7.681635in}}%
\pgfpathcurveto{\pgfqpoint{4.135942in}{7.670585in}}{\pgfqpoint{4.140332in}{7.659986in}}{\pgfqpoint{4.148146in}{7.652172in}}%
\pgfpathcurveto{\pgfqpoint{4.155959in}{7.644359in}}{\pgfqpoint{4.166558in}{7.639968in}}{\pgfqpoint{4.177608in}{7.639968in}}%
\pgfpathclose%
\pgfusepath{stroke,fill}%
\end{pgfscope}%
\begin{pgfscope}%
\pgfpathrectangle{\pgfqpoint{0.481978in}{0.331635in}}{\pgfqpoint{9.300000in}{7.700000in}}%
\pgfusepath{clip}%
\pgfsetbuttcap%
\pgfsetroundjoin%
\definecolor{currentfill}{rgb}{0.631373,0.788235,0.956863}%
\pgfsetfillcolor{currentfill}%
\pgfsetlinewidth{0.481800pt}%
\definecolor{currentstroke}{rgb}{1.000000,1.000000,1.000000}%
\pgfsetstrokecolor{currentstroke}%
\pgfsetdash{}{0pt}%
\pgfpathmoveto{\pgfqpoint{6.895459in}{3.133907in}}%
\pgfpathcurveto{\pgfqpoint{6.906509in}{3.133907in}}{\pgfqpoint{6.917108in}{3.138297in}}{\pgfqpoint{6.924921in}{3.146111in}}%
\pgfpathcurveto{\pgfqpoint{6.932735in}{3.153925in}}{\pgfqpoint{6.937125in}{3.164524in}}{\pgfqpoint{6.937125in}{3.175574in}}%
\pgfpathcurveto{\pgfqpoint{6.937125in}{3.186624in}}{\pgfqpoint{6.932735in}{3.197223in}}{\pgfqpoint{6.924921in}{3.205037in}}%
\pgfpathcurveto{\pgfqpoint{6.917108in}{3.212850in}}{\pgfqpoint{6.906509in}{3.217240in}}{\pgfqpoint{6.895459in}{3.217240in}}%
\pgfpathcurveto{\pgfqpoint{6.884409in}{3.217240in}}{\pgfqpoint{6.873810in}{3.212850in}}{\pgfqpoint{6.865996in}{3.205037in}}%
\pgfpathcurveto{\pgfqpoint{6.858182in}{3.197223in}}{\pgfqpoint{6.853792in}{3.186624in}}{\pgfqpoint{6.853792in}{3.175574in}}%
\pgfpathcurveto{\pgfqpoint{6.853792in}{3.164524in}}{\pgfqpoint{6.858182in}{3.153925in}}{\pgfqpoint{6.865996in}{3.146111in}}%
\pgfpathcurveto{\pgfqpoint{6.873810in}{3.138297in}}{\pgfqpoint{6.884409in}{3.133907in}}{\pgfqpoint{6.895459in}{3.133907in}}%
\pgfpathclose%
\pgfusepath{stroke,fill}%
\end{pgfscope}%
\begin{pgfscope}%
\pgfpathrectangle{\pgfqpoint{0.481978in}{0.331635in}}{\pgfqpoint{9.300000in}{7.700000in}}%
\pgfusepath{clip}%
\pgfsetbuttcap%
\pgfsetroundjoin%
\definecolor{currentfill}{rgb}{0.631373,0.788235,0.956863}%
\pgfsetfillcolor{currentfill}%
\pgfsetlinewidth{0.481800pt}%
\definecolor{currentstroke}{rgb}{1.000000,1.000000,1.000000}%
\pgfsetstrokecolor{currentstroke}%
\pgfsetdash{}{0pt}%
\pgfpathmoveto{\pgfqpoint{7.647915in}{1.867122in}}%
\pgfpathcurveto{\pgfqpoint{7.658966in}{1.867122in}}{\pgfqpoint{7.669565in}{1.871512in}}{\pgfqpoint{7.677378in}{1.879325in}}%
\pgfpathcurveto{\pgfqpoint{7.685192in}{1.887139in}}{\pgfqpoint{7.689582in}{1.897738in}}{\pgfqpoint{7.689582in}{1.908788in}}%
\pgfpathcurveto{\pgfqpoint{7.689582in}{1.919838in}}{\pgfqpoint{7.685192in}{1.930437in}}{\pgfqpoint{7.677378in}{1.938251in}}%
\pgfpathcurveto{\pgfqpoint{7.669565in}{1.946065in}}{\pgfqpoint{7.658966in}{1.950455in}}{\pgfqpoint{7.647915in}{1.950455in}}%
\pgfpathcurveto{\pgfqpoint{7.636865in}{1.950455in}}{\pgfqpoint{7.626266in}{1.946065in}}{\pgfqpoint{7.618453in}{1.938251in}}%
\pgfpathcurveto{\pgfqpoint{7.610639in}{1.930437in}}{\pgfqpoint{7.606249in}{1.919838in}}{\pgfqpoint{7.606249in}{1.908788in}}%
\pgfpathcurveto{\pgfqpoint{7.606249in}{1.897738in}}{\pgfqpoint{7.610639in}{1.887139in}}{\pgfqpoint{7.618453in}{1.879325in}}%
\pgfpathcurveto{\pgfqpoint{7.626266in}{1.871512in}}{\pgfqpoint{7.636865in}{1.867122in}}{\pgfqpoint{7.647915in}{1.867122in}}%
\pgfpathclose%
\pgfusepath{stroke,fill}%
\end{pgfscope}%
\begin{pgfscope}%
\pgfpathrectangle{\pgfqpoint{0.481978in}{0.331635in}}{\pgfqpoint{9.300000in}{7.700000in}}%
\pgfusepath{clip}%
\pgfsetbuttcap%
\pgfsetroundjoin%
\definecolor{currentfill}{rgb}{0.631373,0.788235,0.956863}%
\pgfsetfillcolor{currentfill}%
\pgfsetlinewidth{0.481800pt}%
\definecolor{currentstroke}{rgb}{1.000000,1.000000,1.000000}%
\pgfsetstrokecolor{currentstroke}%
\pgfsetdash{}{0pt}%
\pgfpathmoveto{\pgfqpoint{7.913291in}{4.938828in}}%
\pgfpathcurveto{\pgfqpoint{7.924341in}{4.938828in}}{\pgfqpoint{7.934940in}{4.943219in}}{\pgfqpoint{7.942753in}{4.951032in}}%
\pgfpathcurveto{\pgfqpoint{7.950567in}{4.958846in}}{\pgfqpoint{7.954957in}{4.969445in}}{\pgfqpoint{7.954957in}{4.980495in}}%
\pgfpathcurveto{\pgfqpoint{7.954957in}{4.991545in}}{\pgfqpoint{7.950567in}{5.002144in}}{\pgfqpoint{7.942753in}{5.009958in}}%
\pgfpathcurveto{\pgfqpoint{7.934940in}{5.017771in}}{\pgfqpoint{7.924341in}{5.022162in}}{\pgfqpoint{7.913291in}{5.022162in}}%
\pgfpathcurveto{\pgfqpoint{7.902240in}{5.022162in}}{\pgfqpoint{7.891641in}{5.017771in}}{\pgfqpoint{7.883828in}{5.009958in}}%
\pgfpathcurveto{\pgfqpoint{7.876014in}{5.002144in}}{\pgfqpoint{7.871624in}{4.991545in}}{\pgfqpoint{7.871624in}{4.980495in}}%
\pgfpathcurveto{\pgfqpoint{7.871624in}{4.969445in}}{\pgfqpoint{7.876014in}{4.958846in}}{\pgfqpoint{7.883828in}{4.951032in}}%
\pgfpathcurveto{\pgfqpoint{7.891641in}{4.943219in}}{\pgfqpoint{7.902240in}{4.938828in}}{\pgfqpoint{7.913291in}{4.938828in}}%
\pgfpathclose%
\pgfusepath{stroke,fill}%
\end{pgfscope}%
\begin{pgfscope}%
\pgfpathrectangle{\pgfqpoint{0.481978in}{0.331635in}}{\pgfqpoint{9.300000in}{7.700000in}}%
\pgfusepath{clip}%
\pgfsetbuttcap%
\pgfsetroundjoin%
\definecolor{currentfill}{rgb}{0.631373,0.788235,0.956863}%
\pgfsetfillcolor{currentfill}%
\pgfsetlinewidth{0.481800pt}%
\definecolor{currentstroke}{rgb}{1.000000,1.000000,1.000000}%
\pgfsetstrokecolor{currentstroke}%
\pgfsetdash{}{0pt}%
\pgfpathmoveto{\pgfqpoint{5.605585in}{2.822785in}}%
\pgfpathcurveto{\pgfqpoint{5.616635in}{2.822785in}}{\pgfqpoint{5.627234in}{2.827175in}}{\pgfqpoint{5.635048in}{2.834989in}}%
\pgfpathcurveto{\pgfqpoint{5.642861in}{2.842803in}}{\pgfqpoint{5.647252in}{2.853402in}}{\pgfqpoint{5.647252in}{2.864452in}}%
\pgfpathcurveto{\pgfqpoint{5.647252in}{2.875502in}}{\pgfqpoint{5.642861in}{2.886101in}}{\pgfqpoint{5.635048in}{2.893915in}}%
\pgfpathcurveto{\pgfqpoint{5.627234in}{2.901728in}}{\pgfqpoint{5.616635in}{2.906118in}}{\pgfqpoint{5.605585in}{2.906118in}}%
\pgfpathcurveto{\pgfqpoint{5.594535in}{2.906118in}}{\pgfqpoint{5.583936in}{2.901728in}}{\pgfqpoint{5.576122in}{2.893915in}}%
\pgfpathcurveto{\pgfqpoint{5.568309in}{2.886101in}}{\pgfqpoint{5.563918in}{2.875502in}}{\pgfqpoint{5.563918in}{2.864452in}}%
\pgfpathcurveto{\pgfqpoint{5.563918in}{2.853402in}}{\pgfqpoint{5.568309in}{2.842803in}}{\pgfqpoint{5.576122in}{2.834989in}}%
\pgfpathcurveto{\pgfqpoint{5.583936in}{2.827175in}}{\pgfqpoint{5.594535in}{2.822785in}}{\pgfqpoint{5.605585in}{2.822785in}}%
\pgfpathclose%
\pgfusepath{stroke,fill}%
\end{pgfscope}%
\begin{pgfscope}%
\pgfpathrectangle{\pgfqpoint{0.481978in}{0.331635in}}{\pgfqpoint{9.300000in}{7.700000in}}%
\pgfusepath{clip}%
\pgfsetbuttcap%
\pgfsetroundjoin%
\definecolor{currentfill}{rgb}{0.631373,0.788235,0.956863}%
\pgfsetfillcolor{currentfill}%
\pgfsetlinewidth{0.481800pt}%
\definecolor{currentstroke}{rgb}{1.000000,1.000000,1.000000}%
\pgfsetstrokecolor{currentstroke}%
\pgfsetdash{}{0pt}%
\pgfpathmoveto{\pgfqpoint{4.066101in}{5.114608in}}%
\pgfpathcurveto{\pgfqpoint{4.077151in}{5.114608in}}{\pgfqpoint{4.087750in}{5.118998in}}{\pgfqpoint{4.095563in}{5.126812in}}%
\pgfpathcurveto{\pgfqpoint{4.103377in}{5.134625in}}{\pgfqpoint{4.107767in}{5.145224in}}{\pgfqpoint{4.107767in}{5.156275in}}%
\pgfpathcurveto{\pgfqpoint{4.107767in}{5.167325in}}{\pgfqpoint{4.103377in}{5.177924in}}{\pgfqpoint{4.095563in}{5.185737in}}%
\pgfpathcurveto{\pgfqpoint{4.087750in}{5.193551in}}{\pgfqpoint{4.077151in}{5.197941in}}{\pgfqpoint{4.066101in}{5.197941in}}%
\pgfpathcurveto{\pgfqpoint{4.055051in}{5.197941in}}{\pgfqpoint{4.044452in}{5.193551in}}{\pgfqpoint{4.036638in}{5.185737in}}%
\pgfpathcurveto{\pgfqpoint{4.028824in}{5.177924in}}{\pgfqpoint{4.024434in}{5.167325in}}{\pgfqpoint{4.024434in}{5.156275in}}%
\pgfpathcurveto{\pgfqpoint{4.024434in}{5.145224in}}{\pgfqpoint{4.028824in}{5.134625in}}{\pgfqpoint{4.036638in}{5.126812in}}%
\pgfpathcurveto{\pgfqpoint{4.044452in}{5.118998in}}{\pgfqpoint{4.055051in}{5.114608in}}{\pgfqpoint{4.066101in}{5.114608in}}%
\pgfpathclose%
\pgfusepath{stroke,fill}%
\end{pgfscope}%
\begin{pgfscope}%
\pgfpathrectangle{\pgfqpoint{0.481978in}{0.331635in}}{\pgfqpoint{9.300000in}{7.700000in}}%
\pgfusepath{clip}%
\pgfsetbuttcap%
\pgfsetroundjoin%
\definecolor{currentfill}{rgb}{0.631373,0.788235,0.956863}%
\pgfsetfillcolor{currentfill}%
\pgfsetlinewidth{0.481800pt}%
\definecolor{currentstroke}{rgb}{1.000000,1.000000,1.000000}%
\pgfsetstrokecolor{currentstroke}%
\pgfsetdash{}{0pt}%
\pgfpathmoveto{\pgfqpoint{6.162715in}{2.512318in}}%
\pgfpathcurveto{\pgfqpoint{6.173765in}{2.512318in}}{\pgfqpoint{6.184364in}{2.516708in}}{\pgfqpoint{6.192178in}{2.524522in}}%
\pgfpathcurveto{\pgfqpoint{6.199991in}{2.532335in}}{\pgfqpoint{6.204382in}{2.542934in}}{\pgfqpoint{6.204382in}{2.553984in}}%
\pgfpathcurveto{\pgfqpoint{6.204382in}{2.565035in}}{\pgfqpoint{6.199991in}{2.575634in}}{\pgfqpoint{6.192178in}{2.583447in}}%
\pgfpathcurveto{\pgfqpoint{6.184364in}{2.591261in}}{\pgfqpoint{6.173765in}{2.595651in}}{\pgfqpoint{6.162715in}{2.595651in}}%
\pgfpathcurveto{\pgfqpoint{6.151665in}{2.595651in}}{\pgfqpoint{6.141066in}{2.591261in}}{\pgfqpoint{6.133252in}{2.583447in}}%
\pgfpathcurveto{\pgfqpoint{6.125439in}{2.575634in}}{\pgfqpoint{6.121048in}{2.565035in}}{\pgfqpoint{6.121048in}{2.553984in}}%
\pgfpathcurveto{\pgfqpoint{6.121048in}{2.542934in}}{\pgfqpoint{6.125439in}{2.532335in}}{\pgfqpoint{6.133252in}{2.524522in}}%
\pgfpathcurveto{\pgfqpoint{6.141066in}{2.516708in}}{\pgfqpoint{6.151665in}{2.512318in}}{\pgfqpoint{6.162715in}{2.512318in}}%
\pgfpathclose%
\pgfusepath{stroke,fill}%
\end{pgfscope}%
\begin{pgfscope}%
\pgfpathrectangle{\pgfqpoint{0.481978in}{0.331635in}}{\pgfqpoint{9.300000in}{7.700000in}}%
\pgfusepath{clip}%
\pgfsetbuttcap%
\pgfsetroundjoin%
\definecolor{currentfill}{rgb}{0.631373,0.788235,0.956863}%
\pgfsetfillcolor{currentfill}%
\pgfsetlinewidth{0.481800pt}%
\definecolor{currentstroke}{rgb}{1.000000,1.000000,1.000000}%
\pgfsetstrokecolor{currentstroke}%
\pgfsetdash{}{0pt}%
\pgfpathmoveto{\pgfqpoint{6.160576in}{3.090672in}}%
\pgfpathcurveto{\pgfqpoint{6.171626in}{3.090672in}}{\pgfqpoint{6.182225in}{3.095062in}}{\pgfqpoint{6.190039in}{3.102876in}}%
\pgfpathcurveto{\pgfqpoint{6.197852in}{3.110690in}}{\pgfqpoint{6.202242in}{3.121289in}}{\pgfqpoint{6.202242in}{3.132339in}}%
\pgfpathcurveto{\pgfqpoint{6.202242in}{3.143389in}}{\pgfqpoint{6.197852in}{3.153988in}}{\pgfqpoint{6.190039in}{3.161801in}}%
\pgfpathcurveto{\pgfqpoint{6.182225in}{3.169615in}}{\pgfqpoint{6.171626in}{3.174005in}}{\pgfqpoint{6.160576in}{3.174005in}}%
\pgfpathcurveto{\pgfqpoint{6.149526in}{3.174005in}}{\pgfqpoint{6.138927in}{3.169615in}}{\pgfqpoint{6.131113in}{3.161801in}}%
\pgfpathcurveto{\pgfqpoint{6.123299in}{3.153988in}}{\pgfqpoint{6.118909in}{3.143389in}}{\pgfqpoint{6.118909in}{3.132339in}}%
\pgfpathcurveto{\pgfqpoint{6.118909in}{3.121289in}}{\pgfqpoint{6.123299in}{3.110690in}}{\pgfqpoint{6.131113in}{3.102876in}}%
\pgfpathcurveto{\pgfqpoint{6.138927in}{3.095062in}}{\pgfqpoint{6.149526in}{3.090672in}}{\pgfqpoint{6.160576in}{3.090672in}}%
\pgfpathclose%
\pgfusepath{stroke,fill}%
\end{pgfscope}%
\begin{pgfscope}%
\pgfpathrectangle{\pgfqpoint{0.481978in}{0.331635in}}{\pgfqpoint{9.300000in}{7.700000in}}%
\pgfusepath{clip}%
\pgfsetbuttcap%
\pgfsetroundjoin%
\definecolor{currentfill}{rgb}{0.631373,0.788235,0.956863}%
\pgfsetfillcolor{currentfill}%
\pgfsetlinewidth{0.481800pt}%
\definecolor{currentstroke}{rgb}{1.000000,1.000000,1.000000}%
\pgfsetstrokecolor{currentstroke}%
\pgfsetdash{}{0pt}%
\pgfpathmoveto{\pgfqpoint{4.893800in}{6.368383in}}%
\pgfpathcurveto{\pgfqpoint{4.904850in}{6.368383in}}{\pgfqpoint{4.915449in}{6.372773in}}{\pgfqpoint{4.923262in}{6.380587in}}%
\pgfpathcurveto{\pgfqpoint{4.931076in}{6.388400in}}{\pgfqpoint{4.935466in}{6.398999in}}{\pgfqpoint{4.935466in}{6.410049in}}%
\pgfpathcurveto{\pgfqpoint{4.935466in}{6.421100in}}{\pgfqpoint{4.931076in}{6.431699in}}{\pgfqpoint{4.923262in}{6.439512in}}%
\pgfpathcurveto{\pgfqpoint{4.915449in}{6.447326in}}{\pgfqpoint{4.904850in}{6.451716in}}{\pgfqpoint{4.893800in}{6.451716in}}%
\pgfpathcurveto{\pgfqpoint{4.882749in}{6.451716in}}{\pgfqpoint{4.872150in}{6.447326in}}{\pgfqpoint{4.864337in}{6.439512in}}%
\pgfpathcurveto{\pgfqpoint{4.856523in}{6.431699in}}{\pgfqpoint{4.852133in}{6.421100in}}{\pgfqpoint{4.852133in}{6.410049in}}%
\pgfpathcurveto{\pgfqpoint{4.852133in}{6.398999in}}{\pgfqpoint{4.856523in}{6.388400in}}{\pgfqpoint{4.864337in}{6.380587in}}%
\pgfpathcurveto{\pgfqpoint{4.872150in}{6.372773in}}{\pgfqpoint{4.882749in}{6.368383in}}{\pgfqpoint{4.893800in}{6.368383in}}%
\pgfpathclose%
\pgfusepath{stroke,fill}%
\end{pgfscope}%
\begin{pgfscope}%
\pgfpathrectangle{\pgfqpoint{0.481978in}{0.331635in}}{\pgfqpoint{9.300000in}{7.700000in}}%
\pgfusepath{clip}%
\pgfsetbuttcap%
\pgfsetroundjoin%
\definecolor{currentfill}{rgb}{0.631373,0.788235,0.956863}%
\pgfsetfillcolor{currentfill}%
\pgfsetlinewidth{0.481800pt}%
\definecolor{currentstroke}{rgb}{1.000000,1.000000,1.000000}%
\pgfsetstrokecolor{currentstroke}%
\pgfsetdash{}{0pt}%
\pgfpathmoveto{\pgfqpoint{3.428502in}{1.813110in}}%
\pgfpathcurveto{\pgfqpoint{3.439552in}{1.813110in}}{\pgfqpoint{3.450151in}{1.817500in}}{\pgfqpoint{3.457965in}{1.825314in}}%
\pgfpathcurveto{\pgfqpoint{3.465779in}{1.833128in}}{\pgfqpoint{3.470169in}{1.843727in}}{\pgfqpoint{3.470169in}{1.854777in}}%
\pgfpathcurveto{\pgfqpoint{3.470169in}{1.865827in}}{\pgfqpoint{3.465779in}{1.876426in}}{\pgfqpoint{3.457965in}{1.884240in}}%
\pgfpathcurveto{\pgfqpoint{3.450151in}{1.892053in}}{\pgfqpoint{3.439552in}{1.896443in}}{\pgfqpoint{3.428502in}{1.896443in}}%
\pgfpathcurveto{\pgfqpoint{3.417452in}{1.896443in}}{\pgfqpoint{3.406853in}{1.892053in}}{\pgfqpoint{3.399039in}{1.884240in}}%
\pgfpathcurveto{\pgfqpoint{3.391226in}{1.876426in}}{\pgfqpoint{3.386836in}{1.865827in}}{\pgfqpoint{3.386836in}{1.854777in}}%
\pgfpathcurveto{\pgfqpoint{3.386836in}{1.843727in}}{\pgfqpoint{3.391226in}{1.833128in}}{\pgfqpoint{3.399039in}{1.825314in}}%
\pgfpathcurveto{\pgfqpoint{3.406853in}{1.817500in}}{\pgfqpoint{3.417452in}{1.813110in}}{\pgfqpoint{3.428502in}{1.813110in}}%
\pgfpathclose%
\pgfusepath{stroke,fill}%
\end{pgfscope}%
\begin{pgfscope}%
\pgfpathrectangle{\pgfqpoint{0.481978in}{0.331635in}}{\pgfqpoint{9.300000in}{7.700000in}}%
\pgfusepath{clip}%
\pgfsetbuttcap%
\pgfsetroundjoin%
\definecolor{currentfill}{rgb}{0.631373,0.788235,0.956863}%
\pgfsetfillcolor{currentfill}%
\pgfsetlinewidth{0.481800pt}%
\definecolor{currentstroke}{rgb}{1.000000,1.000000,1.000000}%
\pgfsetstrokecolor{currentstroke}%
\pgfsetdash{}{0pt}%
\pgfpathmoveto{\pgfqpoint{3.370561in}{7.113677in}}%
\pgfpathcurveto{\pgfqpoint{3.381612in}{7.113677in}}{\pgfqpoint{3.392211in}{7.118067in}}{\pgfqpoint{3.400024in}{7.125881in}}%
\pgfpathcurveto{\pgfqpoint{3.407838in}{7.133695in}}{\pgfqpoint{3.412228in}{7.144294in}}{\pgfqpoint{3.412228in}{7.155344in}}%
\pgfpathcurveto{\pgfqpoint{3.412228in}{7.166394in}}{\pgfqpoint{3.407838in}{7.176993in}}{\pgfqpoint{3.400024in}{7.184807in}}%
\pgfpathcurveto{\pgfqpoint{3.392211in}{7.192620in}}{\pgfqpoint{3.381612in}{7.197011in}}{\pgfqpoint{3.370561in}{7.197011in}}%
\pgfpathcurveto{\pgfqpoint{3.359511in}{7.197011in}}{\pgfqpoint{3.348912in}{7.192620in}}{\pgfqpoint{3.341099in}{7.184807in}}%
\pgfpathcurveto{\pgfqpoint{3.333285in}{7.176993in}}{\pgfqpoint{3.328895in}{7.166394in}}{\pgfqpoint{3.328895in}{7.155344in}}%
\pgfpathcurveto{\pgfqpoint{3.328895in}{7.144294in}}{\pgfqpoint{3.333285in}{7.133695in}}{\pgfqpoint{3.341099in}{7.125881in}}%
\pgfpathcurveto{\pgfqpoint{3.348912in}{7.118067in}}{\pgfqpoint{3.359511in}{7.113677in}}{\pgfqpoint{3.370561in}{7.113677in}}%
\pgfpathclose%
\pgfusepath{stroke,fill}%
\end{pgfscope}%
\begin{pgfscope}%
\pgfpathrectangle{\pgfqpoint{0.481978in}{0.331635in}}{\pgfqpoint{9.300000in}{7.700000in}}%
\pgfusepath{clip}%
\pgfsetbuttcap%
\pgfsetroundjoin%
\definecolor{currentfill}{rgb}{0.631373,0.788235,0.956863}%
\pgfsetfillcolor{currentfill}%
\pgfsetlinewidth{0.481800pt}%
\definecolor{currentstroke}{rgb}{1.000000,1.000000,1.000000}%
\pgfsetstrokecolor{currentstroke}%
\pgfsetdash{}{0pt}%
\pgfpathmoveto{\pgfqpoint{6.915286in}{4.156789in}}%
\pgfpathcurveto{\pgfqpoint{6.926336in}{4.156789in}}{\pgfqpoint{6.936935in}{4.161179in}}{\pgfqpoint{6.944748in}{4.168993in}}%
\pgfpathcurveto{\pgfqpoint{6.952562in}{4.176806in}}{\pgfqpoint{6.956952in}{4.187405in}}{\pgfqpoint{6.956952in}{4.198455in}}%
\pgfpathcurveto{\pgfqpoint{6.956952in}{4.209506in}}{\pgfqpoint{6.952562in}{4.220105in}}{\pgfqpoint{6.944748in}{4.227918in}}%
\pgfpathcurveto{\pgfqpoint{6.936935in}{4.235732in}}{\pgfqpoint{6.926336in}{4.240122in}}{\pgfqpoint{6.915286in}{4.240122in}}%
\pgfpathcurveto{\pgfqpoint{6.904235in}{4.240122in}}{\pgfqpoint{6.893636in}{4.235732in}}{\pgfqpoint{6.885823in}{4.227918in}}%
\pgfpathcurveto{\pgfqpoint{6.878009in}{4.220105in}}{\pgfqpoint{6.873619in}{4.209506in}}{\pgfqpoint{6.873619in}{4.198455in}}%
\pgfpathcurveto{\pgfqpoint{6.873619in}{4.187405in}}{\pgfqpoint{6.878009in}{4.176806in}}{\pgfqpoint{6.885823in}{4.168993in}}%
\pgfpathcurveto{\pgfqpoint{6.893636in}{4.161179in}}{\pgfqpoint{6.904235in}{4.156789in}}{\pgfqpoint{6.915286in}{4.156789in}}%
\pgfpathclose%
\pgfusepath{stroke,fill}%
\end{pgfscope}%
\begin{pgfscope}%
\pgfpathrectangle{\pgfqpoint{0.481978in}{0.331635in}}{\pgfqpoint{9.300000in}{7.700000in}}%
\pgfusepath{clip}%
\pgfsetbuttcap%
\pgfsetroundjoin%
\definecolor{currentfill}{rgb}{0.631373,0.788235,0.956863}%
\pgfsetfillcolor{currentfill}%
\pgfsetlinewidth{0.481800pt}%
\definecolor{currentstroke}{rgb}{1.000000,1.000000,1.000000}%
\pgfsetstrokecolor{currentstroke}%
\pgfsetdash{}{0pt}%
\pgfpathmoveto{\pgfqpoint{3.604723in}{5.484870in}}%
\pgfpathcurveto{\pgfqpoint{3.615773in}{5.484870in}}{\pgfqpoint{3.626372in}{5.489260in}}{\pgfqpoint{3.634185in}{5.497074in}}%
\pgfpathcurveto{\pgfqpoint{3.641999in}{5.504888in}}{\pgfqpoint{3.646389in}{5.515487in}}{\pgfqpoint{3.646389in}{5.526537in}}%
\pgfpathcurveto{\pgfqpoint{3.646389in}{5.537587in}}{\pgfqpoint{3.641999in}{5.548186in}}{\pgfqpoint{3.634185in}{5.556000in}}%
\pgfpathcurveto{\pgfqpoint{3.626372in}{5.563813in}}{\pgfqpoint{3.615773in}{5.568204in}}{\pgfqpoint{3.604723in}{5.568204in}}%
\pgfpathcurveto{\pgfqpoint{3.593672in}{5.568204in}}{\pgfqpoint{3.583073in}{5.563813in}}{\pgfqpoint{3.575260in}{5.556000in}}%
\pgfpathcurveto{\pgfqpoint{3.567446in}{5.548186in}}{\pgfqpoint{3.563056in}{5.537587in}}{\pgfqpoint{3.563056in}{5.526537in}}%
\pgfpathcurveto{\pgfqpoint{3.563056in}{5.515487in}}{\pgfqpoint{3.567446in}{5.504888in}}{\pgfqpoint{3.575260in}{5.497074in}}%
\pgfpathcurveto{\pgfqpoint{3.583073in}{5.489260in}}{\pgfqpoint{3.593672in}{5.484870in}}{\pgfqpoint{3.604723in}{5.484870in}}%
\pgfpathclose%
\pgfusepath{stroke,fill}%
\end{pgfscope}%
\begin{pgfscope}%
\pgfpathrectangle{\pgfqpoint{0.481978in}{0.331635in}}{\pgfqpoint{9.300000in}{7.700000in}}%
\pgfusepath{clip}%
\pgfsetbuttcap%
\pgfsetroundjoin%
\definecolor{currentfill}{rgb}{0.631373,0.788235,0.956863}%
\pgfsetfillcolor{currentfill}%
\pgfsetlinewidth{0.481800pt}%
\definecolor{currentstroke}{rgb}{1.000000,1.000000,1.000000}%
\pgfsetstrokecolor{currentstroke}%
\pgfsetdash{}{0pt}%
\pgfpathmoveto{\pgfqpoint{4.283816in}{3.497421in}}%
\pgfpathcurveto{\pgfqpoint{4.294866in}{3.497421in}}{\pgfqpoint{4.305465in}{3.501811in}}{\pgfqpoint{4.313279in}{3.509624in}}%
\pgfpathcurveto{\pgfqpoint{4.321093in}{3.517438in}}{\pgfqpoint{4.325483in}{3.528037in}}{\pgfqpoint{4.325483in}{3.539087in}}%
\pgfpathcurveto{\pgfqpoint{4.325483in}{3.550137in}}{\pgfqpoint{4.321093in}{3.560736in}}{\pgfqpoint{4.313279in}{3.568550in}}%
\pgfpathcurveto{\pgfqpoint{4.305465in}{3.576364in}}{\pgfqpoint{4.294866in}{3.580754in}}{\pgfqpoint{4.283816in}{3.580754in}}%
\pgfpathcurveto{\pgfqpoint{4.272766in}{3.580754in}}{\pgfqpoint{4.262167in}{3.576364in}}{\pgfqpoint{4.254353in}{3.568550in}}%
\pgfpathcurveto{\pgfqpoint{4.246540in}{3.560736in}}{\pgfqpoint{4.242150in}{3.550137in}}{\pgfqpoint{4.242150in}{3.539087in}}%
\pgfpathcurveto{\pgfqpoint{4.242150in}{3.528037in}}{\pgfqpoint{4.246540in}{3.517438in}}{\pgfqpoint{4.254353in}{3.509624in}}%
\pgfpathcurveto{\pgfqpoint{4.262167in}{3.501811in}}{\pgfqpoint{4.272766in}{3.497421in}}{\pgfqpoint{4.283816in}{3.497421in}}%
\pgfpathclose%
\pgfusepath{stroke,fill}%
\end{pgfscope}%
\begin{pgfscope}%
\pgfpathrectangle{\pgfqpoint{0.481978in}{0.331635in}}{\pgfqpoint{9.300000in}{7.700000in}}%
\pgfusepath{clip}%
\pgfsetbuttcap%
\pgfsetroundjoin%
\definecolor{currentfill}{rgb}{0.631373,0.788235,0.956863}%
\pgfsetfillcolor{currentfill}%
\pgfsetlinewidth{0.481800pt}%
\definecolor{currentstroke}{rgb}{1.000000,1.000000,1.000000}%
\pgfsetstrokecolor{currentstroke}%
\pgfsetdash{}{0pt}%
\pgfpathmoveto{\pgfqpoint{7.064950in}{1.902700in}}%
\pgfpathcurveto{\pgfqpoint{7.076000in}{1.902700in}}{\pgfqpoint{7.086599in}{1.907091in}}{\pgfqpoint{7.094413in}{1.914904in}}%
\pgfpathcurveto{\pgfqpoint{7.102226in}{1.922718in}}{\pgfqpoint{7.106617in}{1.933317in}}{\pgfqpoint{7.106617in}{1.944367in}}%
\pgfpathcurveto{\pgfqpoint{7.106617in}{1.955417in}}{\pgfqpoint{7.102226in}{1.966016in}}{\pgfqpoint{7.094413in}{1.973830in}}%
\pgfpathcurveto{\pgfqpoint{7.086599in}{1.981643in}}{\pgfqpoint{7.076000in}{1.986034in}}{\pgfqpoint{7.064950in}{1.986034in}}%
\pgfpathcurveto{\pgfqpoint{7.053900in}{1.986034in}}{\pgfqpoint{7.043301in}{1.981643in}}{\pgfqpoint{7.035487in}{1.973830in}}%
\pgfpathcurveto{\pgfqpoint{7.027673in}{1.966016in}}{\pgfqpoint{7.023283in}{1.955417in}}{\pgfqpoint{7.023283in}{1.944367in}}%
\pgfpathcurveto{\pgfqpoint{7.023283in}{1.933317in}}{\pgfqpoint{7.027673in}{1.922718in}}{\pgfqpoint{7.035487in}{1.914904in}}%
\pgfpathcurveto{\pgfqpoint{7.043301in}{1.907091in}}{\pgfqpoint{7.053900in}{1.902700in}}{\pgfqpoint{7.064950in}{1.902700in}}%
\pgfpathclose%
\pgfusepath{stroke,fill}%
\end{pgfscope}%
\begin{pgfscope}%
\pgfpathrectangle{\pgfqpoint{0.481978in}{0.331635in}}{\pgfqpoint{9.300000in}{7.700000in}}%
\pgfusepath{clip}%
\pgfsetbuttcap%
\pgfsetroundjoin%
\definecolor{currentfill}{rgb}{0.631373,0.788235,0.956863}%
\pgfsetfillcolor{currentfill}%
\pgfsetlinewidth{0.481800pt}%
\definecolor{currentstroke}{rgb}{1.000000,1.000000,1.000000}%
\pgfsetstrokecolor{currentstroke}%
\pgfsetdash{}{0pt}%
\pgfpathmoveto{\pgfqpoint{3.497950in}{6.601752in}}%
\pgfpathcurveto{\pgfqpoint{3.509000in}{6.601752in}}{\pgfqpoint{3.519599in}{6.606142in}}{\pgfqpoint{3.527412in}{6.613956in}}%
\pgfpathcurveto{\pgfqpoint{3.535226in}{6.621770in}}{\pgfqpoint{3.539616in}{6.632369in}}{\pgfqpoint{3.539616in}{6.643419in}}%
\pgfpathcurveto{\pgfqpoint{3.539616in}{6.654469in}}{\pgfqpoint{3.535226in}{6.665068in}}{\pgfqpoint{3.527412in}{6.672881in}}%
\pgfpathcurveto{\pgfqpoint{3.519599in}{6.680695in}}{\pgfqpoint{3.509000in}{6.685085in}}{\pgfqpoint{3.497950in}{6.685085in}}%
\pgfpathcurveto{\pgfqpoint{3.486900in}{6.685085in}}{\pgfqpoint{3.476301in}{6.680695in}}{\pgfqpoint{3.468487in}{6.672881in}}%
\pgfpathcurveto{\pgfqpoint{3.460673in}{6.665068in}}{\pgfqpoint{3.456283in}{6.654469in}}{\pgfqpoint{3.456283in}{6.643419in}}%
\pgfpathcurveto{\pgfqpoint{3.456283in}{6.632369in}}{\pgfqpoint{3.460673in}{6.621770in}}{\pgfqpoint{3.468487in}{6.613956in}}%
\pgfpathcurveto{\pgfqpoint{3.476301in}{6.606142in}}{\pgfqpoint{3.486900in}{6.601752in}}{\pgfqpoint{3.497950in}{6.601752in}}%
\pgfpathclose%
\pgfusepath{stroke,fill}%
\end{pgfscope}%
\begin{pgfscope}%
\pgfpathrectangle{\pgfqpoint{0.481978in}{0.331635in}}{\pgfqpoint{9.300000in}{7.700000in}}%
\pgfusepath{clip}%
\pgfsetbuttcap%
\pgfsetroundjoin%
\definecolor{currentfill}{rgb}{0.631373,0.788235,0.956863}%
\pgfsetfillcolor{currentfill}%
\pgfsetlinewidth{0.481800pt}%
\definecolor{currentstroke}{rgb}{1.000000,1.000000,1.000000}%
\pgfsetstrokecolor{currentstroke}%
\pgfsetdash{}{0pt}%
\pgfpathmoveto{\pgfqpoint{5.579377in}{3.166101in}}%
\pgfpathcurveto{\pgfqpoint{5.590427in}{3.166101in}}{\pgfqpoint{5.601026in}{3.170491in}}{\pgfqpoint{5.608840in}{3.178304in}}%
\pgfpathcurveto{\pgfqpoint{5.616653in}{3.186118in}}{\pgfqpoint{5.621043in}{3.196717in}}{\pgfqpoint{5.621043in}{3.207767in}}%
\pgfpathcurveto{\pgfqpoint{5.621043in}{3.218817in}}{\pgfqpoint{5.616653in}{3.229416in}}{\pgfqpoint{5.608840in}{3.237230in}}%
\pgfpathcurveto{\pgfqpoint{5.601026in}{3.245044in}}{\pgfqpoint{5.590427in}{3.249434in}}{\pgfqpoint{5.579377in}{3.249434in}}%
\pgfpathcurveto{\pgfqpoint{5.568327in}{3.249434in}}{\pgfqpoint{5.557728in}{3.245044in}}{\pgfqpoint{5.549914in}{3.237230in}}%
\pgfpathcurveto{\pgfqpoint{5.542100in}{3.229416in}}{\pgfqpoint{5.537710in}{3.218817in}}{\pgfqpoint{5.537710in}{3.207767in}}%
\pgfpathcurveto{\pgfqpoint{5.537710in}{3.196717in}}{\pgfqpoint{5.542100in}{3.186118in}}{\pgfqpoint{5.549914in}{3.178304in}}%
\pgfpathcurveto{\pgfqpoint{5.557728in}{3.170491in}}{\pgfqpoint{5.568327in}{3.166101in}}{\pgfqpoint{5.579377in}{3.166101in}}%
\pgfpathclose%
\pgfusepath{stroke,fill}%
\end{pgfscope}%
\begin{pgfscope}%
\pgfpathrectangle{\pgfqpoint{0.481978in}{0.331635in}}{\pgfqpoint{9.300000in}{7.700000in}}%
\pgfusepath{clip}%
\pgfsetbuttcap%
\pgfsetroundjoin%
\definecolor{currentfill}{rgb}{0.631373,0.788235,0.956863}%
\pgfsetfillcolor{currentfill}%
\pgfsetlinewidth{0.481800pt}%
\definecolor{currentstroke}{rgb}{1.000000,1.000000,1.000000}%
\pgfsetstrokecolor{currentstroke}%
\pgfsetdash{}{0pt}%
\pgfpathmoveto{\pgfqpoint{8.844215in}{5.030878in}}%
\pgfpathcurveto{\pgfqpoint{8.855265in}{5.030878in}}{\pgfqpoint{8.865864in}{5.035269in}}{\pgfqpoint{8.873678in}{5.043082in}}%
\pgfpathcurveto{\pgfqpoint{8.881491in}{5.050896in}}{\pgfqpoint{8.885882in}{5.061495in}}{\pgfqpoint{8.885882in}{5.072545in}}%
\pgfpathcurveto{\pgfqpoint{8.885882in}{5.083595in}}{\pgfqpoint{8.881491in}{5.094194in}}{\pgfqpoint{8.873678in}{5.102008in}}%
\pgfpathcurveto{\pgfqpoint{8.865864in}{5.109821in}}{\pgfqpoint{8.855265in}{5.114212in}}{\pgfqpoint{8.844215in}{5.114212in}}%
\pgfpathcurveto{\pgfqpoint{8.833165in}{5.114212in}}{\pgfqpoint{8.822566in}{5.109821in}}{\pgfqpoint{8.814752in}{5.102008in}}%
\pgfpathcurveto{\pgfqpoint{8.806939in}{5.094194in}}{\pgfqpoint{8.802548in}{5.083595in}}{\pgfqpoint{8.802548in}{5.072545in}}%
\pgfpathcurveto{\pgfqpoint{8.802548in}{5.061495in}}{\pgfqpoint{8.806939in}{5.050896in}}{\pgfqpoint{8.814752in}{5.043082in}}%
\pgfpathcurveto{\pgfqpoint{8.822566in}{5.035269in}}{\pgfqpoint{8.833165in}{5.030878in}}{\pgfqpoint{8.844215in}{5.030878in}}%
\pgfpathclose%
\pgfusepath{stroke,fill}%
\end{pgfscope}%
\begin{pgfscope}%
\pgfpathrectangle{\pgfqpoint{0.481978in}{0.331635in}}{\pgfqpoint{9.300000in}{7.700000in}}%
\pgfusepath{clip}%
\pgfsetbuttcap%
\pgfsetroundjoin%
\definecolor{currentfill}{rgb}{0.631373,0.788235,0.956863}%
\pgfsetfillcolor{currentfill}%
\pgfsetlinewidth{0.481800pt}%
\definecolor{currentstroke}{rgb}{1.000000,1.000000,1.000000}%
\pgfsetstrokecolor{currentstroke}%
\pgfsetdash{}{0pt}%
\pgfpathmoveto{\pgfqpoint{4.858261in}{1.811183in}}%
\pgfpathcurveto{\pgfqpoint{4.869311in}{1.811183in}}{\pgfqpoint{4.879910in}{1.815573in}}{\pgfqpoint{4.887723in}{1.823387in}}%
\pgfpathcurveto{\pgfqpoint{4.895537in}{1.831201in}}{\pgfqpoint{4.899927in}{1.841800in}}{\pgfqpoint{4.899927in}{1.852850in}}%
\pgfpathcurveto{\pgfqpoint{4.899927in}{1.863900in}}{\pgfqpoint{4.895537in}{1.874499in}}{\pgfqpoint{4.887723in}{1.882312in}}%
\pgfpathcurveto{\pgfqpoint{4.879910in}{1.890126in}}{\pgfqpoint{4.869311in}{1.894516in}}{\pgfqpoint{4.858261in}{1.894516in}}%
\pgfpathcurveto{\pgfqpoint{4.847210in}{1.894516in}}{\pgfqpoint{4.836611in}{1.890126in}}{\pgfqpoint{4.828798in}{1.882312in}}%
\pgfpathcurveto{\pgfqpoint{4.820984in}{1.874499in}}{\pgfqpoint{4.816594in}{1.863900in}}{\pgfqpoint{4.816594in}{1.852850in}}%
\pgfpathcurveto{\pgfqpoint{4.816594in}{1.841800in}}{\pgfqpoint{4.820984in}{1.831201in}}{\pgfqpoint{4.828798in}{1.823387in}}%
\pgfpathcurveto{\pgfqpoint{4.836611in}{1.815573in}}{\pgfqpoint{4.847210in}{1.811183in}}{\pgfqpoint{4.858261in}{1.811183in}}%
\pgfpathclose%
\pgfusepath{stroke,fill}%
\end{pgfscope}%
\begin{pgfscope}%
\pgfpathrectangle{\pgfqpoint{0.481978in}{0.331635in}}{\pgfqpoint{9.300000in}{7.700000in}}%
\pgfusepath{clip}%
\pgfsetbuttcap%
\pgfsetroundjoin%
\definecolor{currentfill}{rgb}{0.631373,0.788235,0.956863}%
\pgfsetfillcolor{currentfill}%
\pgfsetlinewidth{0.481800pt}%
\definecolor{currentstroke}{rgb}{1.000000,1.000000,1.000000}%
\pgfsetstrokecolor{currentstroke}%
\pgfsetdash{}{0pt}%
\pgfpathmoveto{\pgfqpoint{7.438148in}{1.928502in}}%
\pgfpathcurveto{\pgfqpoint{7.449198in}{1.928502in}}{\pgfqpoint{7.459797in}{1.932892in}}{\pgfqpoint{7.467611in}{1.940705in}}%
\pgfpathcurveto{\pgfqpoint{7.475425in}{1.948519in}}{\pgfqpoint{7.479815in}{1.959118in}}{\pgfqpoint{7.479815in}{1.970168in}}%
\pgfpathcurveto{\pgfqpoint{7.479815in}{1.981218in}}{\pgfqpoint{7.475425in}{1.991817in}}{\pgfqpoint{7.467611in}{1.999631in}}%
\pgfpathcurveto{\pgfqpoint{7.459797in}{2.007445in}}{\pgfqpoint{7.449198in}{2.011835in}}{\pgfqpoint{7.438148in}{2.011835in}}%
\pgfpathcurveto{\pgfqpoint{7.427098in}{2.011835in}}{\pgfqpoint{7.416499in}{2.007445in}}{\pgfqpoint{7.408685in}{1.999631in}}%
\pgfpathcurveto{\pgfqpoint{7.400872in}{1.991817in}}{\pgfqpoint{7.396481in}{1.981218in}}{\pgfqpoint{7.396481in}{1.970168in}}%
\pgfpathcurveto{\pgfqpoint{7.396481in}{1.959118in}}{\pgfqpoint{7.400872in}{1.948519in}}{\pgfqpoint{7.408685in}{1.940705in}}%
\pgfpathcurveto{\pgfqpoint{7.416499in}{1.932892in}}{\pgfqpoint{7.427098in}{1.928502in}}{\pgfqpoint{7.438148in}{1.928502in}}%
\pgfpathclose%
\pgfusepath{stroke,fill}%
\end{pgfscope}%
\begin{pgfscope}%
\pgfpathrectangle{\pgfqpoint{0.481978in}{0.331635in}}{\pgfqpoint{9.300000in}{7.700000in}}%
\pgfusepath{clip}%
\pgfsetbuttcap%
\pgfsetroundjoin%
\definecolor{currentfill}{rgb}{0.631373,0.788235,0.956863}%
\pgfsetfillcolor{currentfill}%
\pgfsetlinewidth{0.481800pt}%
\definecolor{currentstroke}{rgb}{1.000000,1.000000,1.000000}%
\pgfsetstrokecolor{currentstroke}%
\pgfsetdash{}{0pt}%
\pgfpathmoveto{\pgfqpoint{3.422792in}{0.998594in}}%
\pgfpathcurveto{\pgfqpoint{3.433842in}{0.998594in}}{\pgfqpoint{3.444441in}{1.002985in}}{\pgfqpoint{3.452255in}{1.010798in}}%
\pgfpathcurveto{\pgfqpoint{3.460069in}{1.018612in}}{\pgfqpoint{3.464459in}{1.029211in}}{\pgfqpoint{3.464459in}{1.040261in}}%
\pgfpathcurveto{\pgfqpoint{3.464459in}{1.051311in}}{\pgfqpoint{3.460069in}{1.061910in}}{\pgfqpoint{3.452255in}{1.069724in}}%
\pgfpathcurveto{\pgfqpoint{3.444441in}{1.077537in}}{\pgfqpoint{3.433842in}{1.081928in}}{\pgfqpoint{3.422792in}{1.081928in}}%
\pgfpathcurveto{\pgfqpoint{3.411742in}{1.081928in}}{\pgfqpoint{3.401143in}{1.077537in}}{\pgfqpoint{3.393329in}{1.069724in}}%
\pgfpathcurveto{\pgfqpoint{3.385516in}{1.061910in}}{\pgfqpoint{3.381126in}{1.051311in}}{\pgfqpoint{3.381126in}{1.040261in}}%
\pgfpathcurveto{\pgfqpoint{3.381126in}{1.029211in}}{\pgfqpoint{3.385516in}{1.018612in}}{\pgfqpoint{3.393329in}{1.010798in}}%
\pgfpathcurveto{\pgfqpoint{3.401143in}{1.002985in}}{\pgfqpoint{3.411742in}{0.998594in}}{\pgfqpoint{3.422792in}{0.998594in}}%
\pgfpathclose%
\pgfusepath{stroke,fill}%
\end{pgfscope}%
\begin{pgfscope}%
\pgfpathrectangle{\pgfqpoint{0.481978in}{0.331635in}}{\pgfqpoint{9.300000in}{7.700000in}}%
\pgfusepath{clip}%
\pgfsetbuttcap%
\pgfsetroundjoin%
\definecolor{currentfill}{rgb}{0.631373,0.788235,0.956863}%
\pgfsetfillcolor{currentfill}%
\pgfsetlinewidth{0.481800pt}%
\definecolor{currentstroke}{rgb}{1.000000,1.000000,1.000000}%
\pgfsetstrokecolor{currentstroke}%
\pgfsetdash{}{0pt}%
\pgfpathmoveto{\pgfqpoint{2.139723in}{2.443266in}}%
\pgfpathcurveto{\pgfqpoint{2.150773in}{2.443266in}}{\pgfqpoint{2.161372in}{2.447656in}}{\pgfqpoint{2.169186in}{2.455469in}}%
\pgfpathcurveto{\pgfqpoint{2.176999in}{2.463283in}}{\pgfqpoint{2.181390in}{2.473882in}}{\pgfqpoint{2.181390in}{2.484932in}}%
\pgfpathcurveto{\pgfqpoint{2.181390in}{2.495982in}}{\pgfqpoint{2.176999in}{2.506581in}}{\pgfqpoint{2.169186in}{2.514395in}}%
\pgfpathcurveto{\pgfqpoint{2.161372in}{2.522209in}}{\pgfqpoint{2.150773in}{2.526599in}}{\pgfqpoint{2.139723in}{2.526599in}}%
\pgfpathcurveto{\pgfqpoint{2.128673in}{2.526599in}}{\pgfqpoint{2.118074in}{2.522209in}}{\pgfqpoint{2.110260in}{2.514395in}}%
\pgfpathcurveto{\pgfqpoint{2.102447in}{2.506581in}}{\pgfqpoint{2.098056in}{2.495982in}}{\pgfqpoint{2.098056in}{2.484932in}}%
\pgfpathcurveto{\pgfqpoint{2.098056in}{2.473882in}}{\pgfqpoint{2.102447in}{2.463283in}}{\pgfqpoint{2.110260in}{2.455469in}}%
\pgfpathcurveto{\pgfqpoint{2.118074in}{2.447656in}}{\pgfqpoint{2.128673in}{2.443266in}}{\pgfqpoint{2.139723in}{2.443266in}}%
\pgfpathclose%
\pgfusepath{stroke,fill}%
\end{pgfscope}%
\begin{pgfscope}%
\pgfpathrectangle{\pgfqpoint{0.481978in}{0.331635in}}{\pgfqpoint{9.300000in}{7.700000in}}%
\pgfusepath{clip}%
\pgfsetbuttcap%
\pgfsetroundjoin%
\definecolor{currentfill}{rgb}{0.631373,0.788235,0.956863}%
\pgfsetfillcolor{currentfill}%
\pgfsetlinewidth{0.481800pt}%
\definecolor{currentstroke}{rgb}{1.000000,1.000000,1.000000}%
\pgfsetstrokecolor{currentstroke}%
\pgfsetdash{}{0pt}%
\pgfpathmoveto{\pgfqpoint{8.049554in}{5.424436in}}%
\pgfpathcurveto{\pgfqpoint{8.060604in}{5.424436in}}{\pgfqpoint{8.071203in}{5.428826in}}{\pgfqpoint{8.079017in}{5.436640in}}%
\pgfpathcurveto{\pgfqpoint{8.086831in}{5.444453in}}{\pgfqpoint{8.091221in}{5.455052in}}{\pgfqpoint{8.091221in}{5.466103in}}%
\pgfpathcurveto{\pgfqpoint{8.091221in}{5.477153in}}{\pgfqpoint{8.086831in}{5.487752in}}{\pgfqpoint{8.079017in}{5.495565in}}%
\pgfpathcurveto{\pgfqpoint{8.071203in}{5.503379in}}{\pgfqpoint{8.060604in}{5.507769in}}{\pgfqpoint{8.049554in}{5.507769in}}%
\pgfpathcurveto{\pgfqpoint{8.038504in}{5.507769in}}{\pgfqpoint{8.027905in}{5.503379in}}{\pgfqpoint{8.020091in}{5.495565in}}%
\pgfpathcurveto{\pgfqpoint{8.012278in}{5.487752in}}{\pgfqpoint{8.007887in}{5.477153in}}{\pgfqpoint{8.007887in}{5.466103in}}%
\pgfpathcurveto{\pgfqpoint{8.007887in}{5.455052in}}{\pgfqpoint{8.012278in}{5.444453in}}{\pgfqpoint{8.020091in}{5.436640in}}%
\pgfpathcurveto{\pgfqpoint{8.027905in}{5.428826in}}{\pgfqpoint{8.038504in}{5.424436in}}{\pgfqpoint{8.049554in}{5.424436in}}%
\pgfpathclose%
\pgfusepath{stroke,fill}%
\end{pgfscope}%
\begin{pgfscope}%
\pgfpathrectangle{\pgfqpoint{0.481978in}{0.331635in}}{\pgfqpoint{9.300000in}{7.700000in}}%
\pgfusepath{clip}%
\pgfsetbuttcap%
\pgfsetroundjoin%
\definecolor{currentfill}{rgb}{0.631373,0.788235,0.956863}%
\pgfsetfillcolor{currentfill}%
\pgfsetlinewidth{0.481800pt}%
\definecolor{currentstroke}{rgb}{1.000000,1.000000,1.000000}%
\pgfsetstrokecolor{currentstroke}%
\pgfsetdash{}{0pt}%
\pgfpathmoveto{\pgfqpoint{3.389691in}{1.585792in}}%
\pgfpathcurveto{\pgfqpoint{3.400741in}{1.585792in}}{\pgfqpoint{3.411340in}{1.590182in}}{\pgfqpoint{3.419153in}{1.597995in}}%
\pgfpathcurveto{\pgfqpoint{3.426967in}{1.605809in}}{\pgfqpoint{3.431357in}{1.616408in}}{\pgfqpoint{3.431357in}{1.627458in}}%
\pgfpathcurveto{\pgfqpoint{3.431357in}{1.638508in}}{\pgfqpoint{3.426967in}{1.649107in}}{\pgfqpoint{3.419153in}{1.656921in}}%
\pgfpathcurveto{\pgfqpoint{3.411340in}{1.664735in}}{\pgfqpoint{3.400741in}{1.669125in}}{\pgfqpoint{3.389691in}{1.669125in}}%
\pgfpathcurveto{\pgfqpoint{3.378640in}{1.669125in}}{\pgfqpoint{3.368041in}{1.664735in}}{\pgfqpoint{3.360228in}{1.656921in}}%
\pgfpathcurveto{\pgfqpoint{3.352414in}{1.649107in}}{\pgfqpoint{3.348024in}{1.638508in}}{\pgfqpoint{3.348024in}{1.627458in}}%
\pgfpathcurveto{\pgfqpoint{3.348024in}{1.616408in}}{\pgfqpoint{3.352414in}{1.605809in}}{\pgfqpoint{3.360228in}{1.597995in}}%
\pgfpathcurveto{\pgfqpoint{3.368041in}{1.590182in}}{\pgfqpoint{3.378640in}{1.585792in}}{\pgfqpoint{3.389691in}{1.585792in}}%
\pgfpathclose%
\pgfusepath{stroke,fill}%
\end{pgfscope}%
\begin{pgfscope}%
\pgfpathrectangle{\pgfqpoint{0.481978in}{0.331635in}}{\pgfqpoint{9.300000in}{7.700000in}}%
\pgfusepath{clip}%
\pgfsetbuttcap%
\pgfsetroundjoin%
\definecolor{currentfill}{rgb}{0.631373,0.788235,0.956863}%
\pgfsetfillcolor{currentfill}%
\pgfsetlinewidth{0.481800pt}%
\definecolor{currentstroke}{rgb}{1.000000,1.000000,1.000000}%
\pgfsetstrokecolor{currentstroke}%
\pgfsetdash{}{0pt}%
\pgfpathmoveto{\pgfqpoint{6.131345in}{5.208931in}}%
\pgfpathcurveto{\pgfqpoint{6.142395in}{5.208931in}}{\pgfqpoint{6.152994in}{5.213321in}}{\pgfqpoint{6.160808in}{5.221135in}}%
\pgfpathcurveto{\pgfqpoint{6.168621in}{5.228948in}}{\pgfqpoint{6.173012in}{5.239547in}}{\pgfqpoint{6.173012in}{5.250597in}}%
\pgfpathcurveto{\pgfqpoint{6.173012in}{5.261648in}}{\pgfqpoint{6.168621in}{5.272247in}}{\pgfqpoint{6.160808in}{5.280060in}}%
\pgfpathcurveto{\pgfqpoint{6.152994in}{5.287874in}}{\pgfqpoint{6.142395in}{5.292264in}}{\pgfqpoint{6.131345in}{5.292264in}}%
\pgfpathcurveto{\pgfqpoint{6.120295in}{5.292264in}}{\pgfqpoint{6.109696in}{5.287874in}}{\pgfqpoint{6.101882in}{5.280060in}}%
\pgfpathcurveto{\pgfqpoint{6.094069in}{5.272247in}}{\pgfqpoint{6.089678in}{5.261648in}}{\pgfqpoint{6.089678in}{5.250597in}}%
\pgfpathcurveto{\pgfqpoint{6.089678in}{5.239547in}}{\pgfqpoint{6.094069in}{5.228948in}}{\pgfqpoint{6.101882in}{5.221135in}}%
\pgfpathcurveto{\pgfqpoint{6.109696in}{5.213321in}}{\pgfqpoint{6.120295in}{5.208931in}}{\pgfqpoint{6.131345in}{5.208931in}}%
\pgfpathclose%
\pgfusepath{stroke,fill}%
\end{pgfscope}%
\begin{pgfscope}%
\pgfpathrectangle{\pgfqpoint{0.481978in}{0.331635in}}{\pgfqpoint{9.300000in}{7.700000in}}%
\pgfusepath{clip}%
\pgfsetbuttcap%
\pgfsetroundjoin%
\definecolor{currentfill}{rgb}{0.631373,0.788235,0.956863}%
\pgfsetfillcolor{currentfill}%
\pgfsetlinewidth{0.481800pt}%
\definecolor{currentstroke}{rgb}{1.000000,1.000000,1.000000}%
\pgfsetstrokecolor{currentstroke}%
\pgfsetdash{}{0pt}%
\pgfpathmoveto{\pgfqpoint{4.856079in}{6.373981in}}%
\pgfpathcurveto{\pgfqpoint{4.867130in}{6.373981in}}{\pgfqpoint{4.877729in}{6.378371in}}{\pgfqpoint{4.885542in}{6.386185in}}%
\pgfpathcurveto{\pgfqpoint{4.893356in}{6.393998in}}{\pgfqpoint{4.897746in}{6.404597in}}{\pgfqpoint{4.897746in}{6.415647in}}%
\pgfpathcurveto{\pgfqpoint{4.897746in}{6.426698in}}{\pgfqpoint{4.893356in}{6.437297in}}{\pgfqpoint{4.885542in}{6.445110in}}%
\pgfpathcurveto{\pgfqpoint{4.877729in}{6.452924in}}{\pgfqpoint{4.867130in}{6.457314in}}{\pgfqpoint{4.856079in}{6.457314in}}%
\pgfpathcurveto{\pgfqpoint{4.845029in}{6.457314in}}{\pgfqpoint{4.834430in}{6.452924in}}{\pgfqpoint{4.826617in}{6.445110in}}%
\pgfpathcurveto{\pgfqpoint{4.818803in}{6.437297in}}{\pgfqpoint{4.814413in}{6.426698in}}{\pgfqpoint{4.814413in}{6.415647in}}%
\pgfpathcurveto{\pgfqpoint{4.814413in}{6.404597in}}{\pgfqpoint{4.818803in}{6.393998in}}{\pgfqpoint{4.826617in}{6.386185in}}%
\pgfpathcurveto{\pgfqpoint{4.834430in}{6.378371in}}{\pgfqpoint{4.845029in}{6.373981in}}{\pgfqpoint{4.856079in}{6.373981in}}%
\pgfpathclose%
\pgfusepath{stroke,fill}%
\end{pgfscope}%
\begin{pgfscope}%
\pgfpathrectangle{\pgfqpoint{0.481978in}{0.331635in}}{\pgfqpoint{9.300000in}{7.700000in}}%
\pgfusepath{clip}%
\pgfsetbuttcap%
\pgfsetroundjoin%
\definecolor{currentfill}{rgb}{0.631373,0.788235,0.956863}%
\pgfsetfillcolor{currentfill}%
\pgfsetlinewidth{0.481800pt}%
\definecolor{currentstroke}{rgb}{1.000000,1.000000,1.000000}%
\pgfsetstrokecolor{currentstroke}%
\pgfsetdash{}{0pt}%
\pgfpathmoveto{\pgfqpoint{4.975752in}{4.578501in}}%
\pgfpathcurveto{\pgfqpoint{4.986802in}{4.578501in}}{\pgfqpoint{4.997401in}{4.582891in}}{\pgfqpoint{5.005215in}{4.590705in}}%
\pgfpathcurveto{\pgfqpoint{5.013028in}{4.598518in}}{\pgfqpoint{5.017418in}{4.609118in}}{\pgfqpoint{5.017418in}{4.620168in}}%
\pgfpathcurveto{\pgfqpoint{5.017418in}{4.631218in}}{\pgfqpoint{5.013028in}{4.641817in}}{\pgfqpoint{5.005215in}{4.649630in}}%
\pgfpathcurveto{\pgfqpoint{4.997401in}{4.657444in}}{\pgfqpoint{4.986802in}{4.661834in}}{\pgfqpoint{4.975752in}{4.661834in}}%
\pgfpathcurveto{\pgfqpoint{4.964702in}{4.661834in}}{\pgfqpoint{4.954103in}{4.657444in}}{\pgfqpoint{4.946289in}{4.649630in}}%
\pgfpathcurveto{\pgfqpoint{4.938475in}{4.641817in}}{\pgfqpoint{4.934085in}{4.631218in}}{\pgfqpoint{4.934085in}{4.620168in}}%
\pgfpathcurveto{\pgfqpoint{4.934085in}{4.609118in}}{\pgfqpoint{4.938475in}{4.598518in}}{\pgfqpoint{4.946289in}{4.590705in}}%
\pgfpathcurveto{\pgfqpoint{4.954103in}{4.582891in}}{\pgfqpoint{4.964702in}{4.578501in}}{\pgfqpoint{4.975752in}{4.578501in}}%
\pgfpathclose%
\pgfusepath{stroke,fill}%
\end{pgfscope}%
\begin{pgfscope}%
\pgfpathrectangle{\pgfqpoint{0.481978in}{0.331635in}}{\pgfqpoint{9.300000in}{7.700000in}}%
\pgfusepath{clip}%
\pgfsetbuttcap%
\pgfsetroundjoin%
\definecolor{currentfill}{rgb}{0.631373,0.788235,0.956863}%
\pgfsetfillcolor{currentfill}%
\pgfsetlinewidth{0.481800pt}%
\definecolor{currentstroke}{rgb}{1.000000,1.000000,1.000000}%
\pgfsetstrokecolor{currentstroke}%
\pgfsetdash{}{0pt}%
\pgfpathmoveto{\pgfqpoint{5.729756in}{5.704617in}}%
\pgfpathcurveto{\pgfqpoint{5.740806in}{5.704617in}}{\pgfqpoint{5.751405in}{5.709007in}}{\pgfqpoint{5.759219in}{5.716821in}}%
\pgfpathcurveto{\pgfqpoint{5.767033in}{5.724634in}}{\pgfqpoint{5.771423in}{5.735233in}}{\pgfqpoint{5.771423in}{5.746283in}}%
\pgfpathcurveto{\pgfqpoint{5.771423in}{5.757334in}}{\pgfqpoint{5.767033in}{5.767933in}}{\pgfqpoint{5.759219in}{5.775746in}}%
\pgfpathcurveto{\pgfqpoint{5.751405in}{5.783560in}}{\pgfqpoint{5.740806in}{5.787950in}}{\pgfqpoint{5.729756in}{5.787950in}}%
\pgfpathcurveto{\pgfqpoint{5.718706in}{5.787950in}}{\pgfqpoint{5.708107in}{5.783560in}}{\pgfqpoint{5.700293in}{5.775746in}}%
\pgfpathcurveto{\pgfqpoint{5.692480in}{5.767933in}}{\pgfqpoint{5.688089in}{5.757334in}}{\pgfqpoint{5.688089in}{5.746283in}}%
\pgfpathcurveto{\pgfqpoint{5.688089in}{5.735233in}}{\pgfqpoint{5.692480in}{5.724634in}}{\pgfqpoint{5.700293in}{5.716821in}}%
\pgfpathcurveto{\pgfqpoint{5.708107in}{5.709007in}}{\pgfqpoint{5.718706in}{5.704617in}}{\pgfqpoint{5.729756in}{5.704617in}}%
\pgfpathclose%
\pgfusepath{stroke,fill}%
\end{pgfscope}%
\begin{pgfscope}%
\pgfpathrectangle{\pgfqpoint{0.481978in}{0.331635in}}{\pgfqpoint{9.300000in}{7.700000in}}%
\pgfusepath{clip}%
\pgfsetbuttcap%
\pgfsetroundjoin%
\definecolor{currentfill}{rgb}{0.631373,0.788235,0.956863}%
\pgfsetfillcolor{currentfill}%
\pgfsetlinewidth{0.481800pt}%
\definecolor{currentstroke}{rgb}{1.000000,1.000000,1.000000}%
\pgfsetstrokecolor{currentstroke}%
\pgfsetdash{}{0pt}%
\pgfpathmoveto{\pgfqpoint{5.526353in}{1.621745in}}%
\pgfpathcurveto{\pgfqpoint{5.537403in}{1.621745in}}{\pgfqpoint{5.548002in}{1.626135in}}{\pgfqpoint{5.555816in}{1.633949in}}%
\pgfpathcurveto{\pgfqpoint{5.563629in}{1.641763in}}{\pgfqpoint{5.568019in}{1.652362in}}{\pgfqpoint{5.568019in}{1.663412in}}%
\pgfpathcurveto{\pgfqpoint{5.568019in}{1.674462in}}{\pgfqpoint{5.563629in}{1.685061in}}{\pgfqpoint{5.555816in}{1.692875in}}%
\pgfpathcurveto{\pgfqpoint{5.548002in}{1.700688in}}{\pgfqpoint{5.537403in}{1.705078in}}{\pgfqpoint{5.526353in}{1.705078in}}%
\pgfpathcurveto{\pgfqpoint{5.515303in}{1.705078in}}{\pgfqpoint{5.504704in}{1.700688in}}{\pgfqpoint{5.496890in}{1.692875in}}%
\pgfpathcurveto{\pgfqpoint{5.489076in}{1.685061in}}{\pgfqpoint{5.484686in}{1.674462in}}{\pgfqpoint{5.484686in}{1.663412in}}%
\pgfpathcurveto{\pgfqpoint{5.484686in}{1.652362in}}{\pgfqpoint{5.489076in}{1.641763in}}{\pgfqpoint{5.496890in}{1.633949in}}%
\pgfpathcurveto{\pgfqpoint{5.504704in}{1.626135in}}{\pgfqpoint{5.515303in}{1.621745in}}{\pgfqpoint{5.526353in}{1.621745in}}%
\pgfpathclose%
\pgfusepath{stroke,fill}%
\end{pgfscope}%
\begin{pgfscope}%
\pgfpathrectangle{\pgfqpoint{0.481978in}{0.331635in}}{\pgfqpoint{9.300000in}{7.700000in}}%
\pgfusepath{clip}%
\pgfsetbuttcap%
\pgfsetroundjoin%
\definecolor{currentfill}{rgb}{0.631373,0.788235,0.956863}%
\pgfsetfillcolor{currentfill}%
\pgfsetlinewidth{0.481800pt}%
\definecolor{currentstroke}{rgb}{1.000000,1.000000,1.000000}%
\pgfsetstrokecolor{currentstroke}%
\pgfsetdash{}{0pt}%
\pgfpathmoveto{\pgfqpoint{6.278735in}{2.388750in}}%
\pgfpathcurveto{\pgfqpoint{6.289785in}{2.388750in}}{\pgfqpoint{6.300384in}{2.393141in}}{\pgfqpoint{6.308198in}{2.400954in}}%
\pgfpathcurveto{\pgfqpoint{6.316012in}{2.408768in}}{\pgfqpoint{6.320402in}{2.419367in}}{\pgfqpoint{6.320402in}{2.430417in}}%
\pgfpathcurveto{\pgfqpoint{6.320402in}{2.441467in}}{\pgfqpoint{6.316012in}{2.452066in}}{\pgfqpoint{6.308198in}{2.459880in}}%
\pgfpathcurveto{\pgfqpoint{6.300384in}{2.467693in}}{\pgfqpoint{6.289785in}{2.472084in}}{\pgfqpoint{6.278735in}{2.472084in}}%
\pgfpathcurveto{\pgfqpoint{6.267685in}{2.472084in}}{\pgfqpoint{6.257086in}{2.467693in}}{\pgfqpoint{6.249272in}{2.459880in}}%
\pgfpathcurveto{\pgfqpoint{6.241459in}{2.452066in}}{\pgfqpoint{6.237069in}{2.441467in}}{\pgfqpoint{6.237069in}{2.430417in}}%
\pgfpathcurveto{\pgfqpoint{6.237069in}{2.419367in}}{\pgfqpoint{6.241459in}{2.408768in}}{\pgfqpoint{6.249272in}{2.400954in}}%
\pgfpathcurveto{\pgfqpoint{6.257086in}{2.393141in}}{\pgfqpoint{6.267685in}{2.388750in}}{\pgfqpoint{6.278735in}{2.388750in}}%
\pgfpathclose%
\pgfusepath{stroke,fill}%
\end{pgfscope}%
\begin{pgfscope}%
\pgfpathrectangle{\pgfqpoint{0.481978in}{0.331635in}}{\pgfqpoint{9.300000in}{7.700000in}}%
\pgfusepath{clip}%
\pgfsetbuttcap%
\pgfsetroundjoin%
\definecolor{currentfill}{rgb}{0.631373,0.788235,0.956863}%
\pgfsetfillcolor{currentfill}%
\pgfsetlinewidth{0.481800pt}%
\definecolor{currentstroke}{rgb}{1.000000,1.000000,1.000000}%
\pgfsetstrokecolor{currentstroke}%
\pgfsetdash{}{0pt}%
\pgfpathmoveto{\pgfqpoint{2.008732in}{2.104892in}}%
\pgfpathcurveto{\pgfqpoint{2.019782in}{2.104892in}}{\pgfqpoint{2.030381in}{2.109282in}}{\pgfqpoint{2.038195in}{2.117096in}}%
\pgfpathcurveto{\pgfqpoint{2.046008in}{2.124909in}}{\pgfqpoint{2.050399in}{2.135508in}}{\pgfqpoint{2.050399in}{2.146558in}}%
\pgfpathcurveto{\pgfqpoint{2.050399in}{2.157608in}}{\pgfqpoint{2.046008in}{2.168208in}}{\pgfqpoint{2.038195in}{2.176021in}}%
\pgfpathcurveto{\pgfqpoint{2.030381in}{2.183835in}}{\pgfqpoint{2.019782in}{2.188225in}}{\pgfqpoint{2.008732in}{2.188225in}}%
\pgfpathcurveto{\pgfqpoint{1.997682in}{2.188225in}}{\pgfqpoint{1.987083in}{2.183835in}}{\pgfqpoint{1.979269in}{2.176021in}}%
\pgfpathcurveto{\pgfqpoint{1.971456in}{2.168208in}}{\pgfqpoint{1.967065in}{2.157608in}}{\pgfqpoint{1.967065in}{2.146558in}}%
\pgfpathcurveto{\pgfqpoint{1.967065in}{2.135508in}}{\pgfqpoint{1.971456in}{2.124909in}}{\pgfqpoint{1.979269in}{2.117096in}}%
\pgfpathcurveto{\pgfqpoint{1.987083in}{2.109282in}}{\pgfqpoint{1.997682in}{2.104892in}}{\pgfqpoint{2.008732in}{2.104892in}}%
\pgfpathclose%
\pgfusepath{stroke,fill}%
\end{pgfscope}%
\begin{pgfscope}%
\pgfpathrectangle{\pgfqpoint{0.481978in}{0.331635in}}{\pgfqpoint{9.300000in}{7.700000in}}%
\pgfusepath{clip}%
\pgfsetbuttcap%
\pgfsetroundjoin%
\definecolor{currentfill}{rgb}{0.631373,0.788235,0.956863}%
\pgfsetfillcolor{currentfill}%
\pgfsetlinewidth{0.481800pt}%
\definecolor{currentstroke}{rgb}{1.000000,1.000000,1.000000}%
\pgfsetstrokecolor{currentstroke}%
\pgfsetdash{}{0pt}%
\pgfpathmoveto{\pgfqpoint{5.928515in}{1.734169in}}%
\pgfpathcurveto{\pgfqpoint{5.939565in}{1.734169in}}{\pgfqpoint{5.950164in}{1.738560in}}{\pgfqpoint{5.957978in}{1.746373in}}%
\pgfpathcurveto{\pgfqpoint{5.965791in}{1.754187in}}{\pgfqpoint{5.970182in}{1.764786in}}{\pgfqpoint{5.970182in}{1.775836in}}%
\pgfpathcurveto{\pgfqpoint{5.970182in}{1.786886in}}{\pgfqpoint{5.965791in}{1.797485in}}{\pgfqpoint{5.957978in}{1.805299in}}%
\pgfpathcurveto{\pgfqpoint{5.950164in}{1.813112in}}{\pgfqpoint{5.939565in}{1.817503in}}{\pgfqpoint{5.928515in}{1.817503in}}%
\pgfpathcurveto{\pgfqpoint{5.917465in}{1.817503in}}{\pgfqpoint{5.906866in}{1.813112in}}{\pgfqpoint{5.899052in}{1.805299in}}%
\pgfpathcurveto{\pgfqpoint{5.891238in}{1.797485in}}{\pgfqpoint{5.886848in}{1.786886in}}{\pgfqpoint{5.886848in}{1.775836in}}%
\pgfpathcurveto{\pgfqpoint{5.886848in}{1.764786in}}{\pgfqpoint{5.891238in}{1.754187in}}{\pgfqpoint{5.899052in}{1.746373in}}%
\pgfpathcurveto{\pgfqpoint{5.906866in}{1.738560in}}{\pgfqpoint{5.917465in}{1.734169in}}{\pgfqpoint{5.928515in}{1.734169in}}%
\pgfpathclose%
\pgfusepath{stroke,fill}%
\end{pgfscope}%
\begin{pgfscope}%
\pgfpathrectangle{\pgfqpoint{0.481978in}{0.331635in}}{\pgfqpoint{9.300000in}{7.700000in}}%
\pgfusepath{clip}%
\pgfsetbuttcap%
\pgfsetroundjoin%
\definecolor{currentfill}{rgb}{0.631373,0.788235,0.956863}%
\pgfsetfillcolor{currentfill}%
\pgfsetlinewidth{0.481800pt}%
\definecolor{currentstroke}{rgb}{1.000000,1.000000,1.000000}%
\pgfsetstrokecolor{currentstroke}%
\pgfsetdash{}{0pt}%
\pgfpathmoveto{\pgfqpoint{6.134955in}{2.431958in}}%
\pgfpathcurveto{\pgfqpoint{6.146005in}{2.431958in}}{\pgfqpoint{6.156604in}{2.436348in}}{\pgfqpoint{6.164418in}{2.444161in}}%
\pgfpathcurveto{\pgfqpoint{6.172231in}{2.451975in}}{\pgfqpoint{6.176621in}{2.462574in}}{\pgfqpoint{6.176621in}{2.473624in}}%
\pgfpathcurveto{\pgfqpoint{6.176621in}{2.484674in}}{\pgfqpoint{6.172231in}{2.495273in}}{\pgfqpoint{6.164418in}{2.503087in}}%
\pgfpathcurveto{\pgfqpoint{6.156604in}{2.510901in}}{\pgfqpoint{6.146005in}{2.515291in}}{\pgfqpoint{6.134955in}{2.515291in}}%
\pgfpathcurveto{\pgfqpoint{6.123905in}{2.515291in}}{\pgfqpoint{6.113306in}{2.510901in}}{\pgfqpoint{6.105492in}{2.503087in}}%
\pgfpathcurveto{\pgfqpoint{6.097678in}{2.495273in}}{\pgfqpoint{6.093288in}{2.484674in}}{\pgfqpoint{6.093288in}{2.473624in}}%
\pgfpathcurveto{\pgfqpoint{6.093288in}{2.462574in}}{\pgfqpoint{6.097678in}{2.451975in}}{\pgfqpoint{6.105492in}{2.444161in}}%
\pgfpathcurveto{\pgfqpoint{6.113306in}{2.436348in}}{\pgfqpoint{6.123905in}{2.431958in}}{\pgfqpoint{6.134955in}{2.431958in}}%
\pgfpathclose%
\pgfusepath{stroke,fill}%
\end{pgfscope}%
\begin{pgfscope}%
\pgfpathrectangle{\pgfqpoint{0.481978in}{0.331635in}}{\pgfqpoint{9.300000in}{7.700000in}}%
\pgfusepath{clip}%
\pgfsetbuttcap%
\pgfsetroundjoin%
\definecolor{currentfill}{rgb}{0.631373,0.788235,0.956863}%
\pgfsetfillcolor{currentfill}%
\pgfsetlinewidth{0.481800pt}%
\definecolor{currentstroke}{rgb}{1.000000,1.000000,1.000000}%
\pgfsetstrokecolor{currentstroke}%
\pgfsetdash{}{0pt}%
\pgfpathmoveto{\pgfqpoint{5.647865in}{3.622493in}}%
\pgfpathcurveto{\pgfqpoint{5.658915in}{3.622493in}}{\pgfqpoint{5.669514in}{3.626883in}}{\pgfqpoint{5.677327in}{3.634696in}}%
\pgfpathcurveto{\pgfqpoint{5.685141in}{3.642510in}}{\pgfqpoint{5.689531in}{3.653109in}}{\pgfqpoint{5.689531in}{3.664159in}}%
\pgfpathcurveto{\pgfqpoint{5.689531in}{3.675209in}}{\pgfqpoint{5.685141in}{3.685808in}}{\pgfqpoint{5.677327in}{3.693622in}}%
\pgfpathcurveto{\pgfqpoint{5.669514in}{3.701436in}}{\pgfqpoint{5.658915in}{3.705826in}}{\pgfqpoint{5.647865in}{3.705826in}}%
\pgfpathcurveto{\pgfqpoint{5.636814in}{3.705826in}}{\pgfqpoint{5.626215in}{3.701436in}}{\pgfqpoint{5.618402in}{3.693622in}}%
\pgfpathcurveto{\pgfqpoint{5.610588in}{3.685808in}}{\pgfqpoint{5.606198in}{3.675209in}}{\pgfqpoint{5.606198in}{3.664159in}}%
\pgfpathcurveto{\pgfqpoint{5.606198in}{3.653109in}}{\pgfqpoint{5.610588in}{3.642510in}}{\pgfqpoint{5.618402in}{3.634696in}}%
\pgfpathcurveto{\pgfqpoint{5.626215in}{3.626883in}}{\pgfqpoint{5.636814in}{3.622493in}}{\pgfqpoint{5.647865in}{3.622493in}}%
\pgfpathclose%
\pgfusepath{stroke,fill}%
\end{pgfscope}%
\begin{pgfscope}%
\pgfpathrectangle{\pgfqpoint{0.481978in}{0.331635in}}{\pgfqpoint{9.300000in}{7.700000in}}%
\pgfusepath{clip}%
\pgfsetbuttcap%
\pgfsetroundjoin%
\definecolor{currentfill}{rgb}{0.631373,0.788235,0.956863}%
\pgfsetfillcolor{currentfill}%
\pgfsetlinewidth{0.481800pt}%
\definecolor{currentstroke}{rgb}{1.000000,1.000000,1.000000}%
\pgfsetstrokecolor{currentstroke}%
\pgfsetdash{}{0pt}%
\pgfpathmoveto{\pgfqpoint{6.505006in}{5.405721in}}%
\pgfpathcurveto{\pgfqpoint{6.516056in}{5.405721in}}{\pgfqpoint{6.526655in}{5.410111in}}{\pgfqpoint{6.534469in}{5.417924in}}%
\pgfpathcurveto{\pgfqpoint{6.542282in}{5.425738in}}{\pgfqpoint{6.546672in}{5.436337in}}{\pgfqpoint{6.546672in}{5.447387in}}%
\pgfpathcurveto{\pgfqpoint{6.546672in}{5.458437in}}{\pgfqpoint{6.542282in}{5.469036in}}{\pgfqpoint{6.534469in}{5.476850in}}%
\pgfpathcurveto{\pgfqpoint{6.526655in}{5.484664in}}{\pgfqpoint{6.516056in}{5.489054in}}{\pgfqpoint{6.505006in}{5.489054in}}%
\pgfpathcurveto{\pgfqpoint{6.493956in}{5.489054in}}{\pgfqpoint{6.483357in}{5.484664in}}{\pgfqpoint{6.475543in}{5.476850in}}%
\pgfpathcurveto{\pgfqpoint{6.467729in}{5.469036in}}{\pgfqpoint{6.463339in}{5.458437in}}{\pgfqpoint{6.463339in}{5.447387in}}%
\pgfpathcurveto{\pgfqpoint{6.463339in}{5.436337in}}{\pgfqpoint{6.467729in}{5.425738in}}{\pgfqpoint{6.475543in}{5.417924in}}%
\pgfpathcurveto{\pgfqpoint{6.483357in}{5.410111in}}{\pgfqpoint{6.493956in}{5.405721in}}{\pgfqpoint{6.505006in}{5.405721in}}%
\pgfpathclose%
\pgfusepath{stroke,fill}%
\end{pgfscope}%
\begin{pgfscope}%
\pgfpathrectangle{\pgfqpoint{0.481978in}{0.331635in}}{\pgfqpoint{9.300000in}{7.700000in}}%
\pgfusepath{clip}%
\pgfsetbuttcap%
\pgfsetroundjoin%
\definecolor{currentfill}{rgb}{0.631373,0.788235,0.956863}%
\pgfsetfillcolor{currentfill}%
\pgfsetlinewidth{0.481800pt}%
\definecolor{currentstroke}{rgb}{1.000000,1.000000,1.000000}%
\pgfsetstrokecolor{currentstroke}%
\pgfsetdash{}{0pt}%
\pgfpathmoveto{\pgfqpoint{6.819209in}{3.642925in}}%
\pgfpathcurveto{\pgfqpoint{6.830259in}{3.642925in}}{\pgfqpoint{6.840859in}{3.647315in}}{\pgfqpoint{6.848672in}{3.655128in}}%
\pgfpathcurveto{\pgfqpoint{6.856486in}{3.662942in}}{\pgfqpoint{6.860876in}{3.673541in}}{\pgfqpoint{6.860876in}{3.684591in}}%
\pgfpathcurveto{\pgfqpoint{6.860876in}{3.695641in}}{\pgfqpoint{6.856486in}{3.706240in}}{\pgfqpoint{6.848672in}{3.714054in}}%
\pgfpathcurveto{\pgfqpoint{6.840859in}{3.721868in}}{\pgfqpoint{6.830259in}{3.726258in}}{\pgfqpoint{6.819209in}{3.726258in}}%
\pgfpathcurveto{\pgfqpoint{6.808159in}{3.726258in}}{\pgfqpoint{6.797560in}{3.721868in}}{\pgfqpoint{6.789747in}{3.714054in}}%
\pgfpathcurveto{\pgfqpoint{6.781933in}{3.706240in}}{\pgfqpoint{6.777543in}{3.695641in}}{\pgfqpoint{6.777543in}{3.684591in}}%
\pgfpathcurveto{\pgfqpoint{6.777543in}{3.673541in}}{\pgfqpoint{6.781933in}{3.662942in}}{\pgfqpoint{6.789747in}{3.655128in}}%
\pgfpathcurveto{\pgfqpoint{6.797560in}{3.647315in}}{\pgfqpoint{6.808159in}{3.642925in}}{\pgfqpoint{6.819209in}{3.642925in}}%
\pgfpathclose%
\pgfusepath{stroke,fill}%
\end{pgfscope}%
\begin{pgfscope}%
\pgfpathrectangle{\pgfqpoint{0.481978in}{0.331635in}}{\pgfqpoint{9.300000in}{7.700000in}}%
\pgfusepath{clip}%
\pgfsetbuttcap%
\pgfsetroundjoin%
\definecolor{currentfill}{rgb}{0.631373,0.788235,0.956863}%
\pgfsetfillcolor{currentfill}%
\pgfsetlinewidth{0.481800pt}%
\definecolor{currentstroke}{rgb}{1.000000,1.000000,1.000000}%
\pgfsetstrokecolor{currentstroke}%
\pgfsetdash{}{0pt}%
\pgfpathmoveto{\pgfqpoint{7.024686in}{3.292519in}}%
\pgfpathcurveto{\pgfqpoint{7.035736in}{3.292519in}}{\pgfqpoint{7.046335in}{3.296909in}}{\pgfqpoint{7.054149in}{3.304723in}}%
\pgfpathcurveto{\pgfqpoint{7.061963in}{3.312536in}}{\pgfqpoint{7.066353in}{3.323135in}}{\pgfqpoint{7.066353in}{3.334185in}}%
\pgfpathcurveto{\pgfqpoint{7.066353in}{3.345236in}}{\pgfqpoint{7.061963in}{3.355835in}}{\pgfqpoint{7.054149in}{3.363648in}}%
\pgfpathcurveto{\pgfqpoint{7.046335in}{3.371462in}}{\pgfqpoint{7.035736in}{3.375852in}}{\pgfqpoint{7.024686in}{3.375852in}}%
\pgfpathcurveto{\pgfqpoint{7.013636in}{3.375852in}}{\pgfqpoint{7.003037in}{3.371462in}}{\pgfqpoint{6.995223in}{3.363648in}}%
\pgfpathcurveto{\pgfqpoint{6.987410in}{3.355835in}}{\pgfqpoint{6.983019in}{3.345236in}}{\pgfqpoint{6.983019in}{3.334185in}}%
\pgfpathcurveto{\pgfqpoint{6.983019in}{3.323135in}}{\pgfqpoint{6.987410in}{3.312536in}}{\pgfqpoint{6.995223in}{3.304723in}}%
\pgfpathcurveto{\pgfqpoint{7.003037in}{3.296909in}}{\pgfqpoint{7.013636in}{3.292519in}}{\pgfqpoint{7.024686in}{3.292519in}}%
\pgfpathclose%
\pgfusepath{stroke,fill}%
\end{pgfscope}%
\begin{pgfscope}%
\pgfpathrectangle{\pgfqpoint{0.481978in}{0.331635in}}{\pgfqpoint{9.300000in}{7.700000in}}%
\pgfusepath{clip}%
\pgfsetbuttcap%
\pgfsetroundjoin%
\definecolor{currentfill}{rgb}{0.631373,0.788235,0.956863}%
\pgfsetfillcolor{currentfill}%
\pgfsetlinewidth{0.481800pt}%
\definecolor{currentstroke}{rgb}{1.000000,1.000000,1.000000}%
\pgfsetstrokecolor{currentstroke}%
\pgfsetdash{}{0pt}%
\pgfpathmoveto{\pgfqpoint{2.950033in}{6.667632in}}%
\pgfpathcurveto{\pgfqpoint{2.961083in}{6.667632in}}{\pgfqpoint{2.971682in}{6.672022in}}{\pgfqpoint{2.979495in}{6.679836in}}%
\pgfpathcurveto{\pgfqpoint{2.987309in}{6.687650in}}{\pgfqpoint{2.991699in}{6.698249in}}{\pgfqpoint{2.991699in}{6.709299in}}%
\pgfpathcurveto{\pgfqpoint{2.991699in}{6.720349in}}{\pgfqpoint{2.987309in}{6.730948in}}{\pgfqpoint{2.979495in}{6.738761in}}%
\pgfpathcurveto{\pgfqpoint{2.971682in}{6.746575in}}{\pgfqpoint{2.961083in}{6.750965in}}{\pgfqpoint{2.950033in}{6.750965in}}%
\pgfpathcurveto{\pgfqpoint{2.938983in}{6.750965in}}{\pgfqpoint{2.928383in}{6.746575in}}{\pgfqpoint{2.920570in}{6.738761in}}%
\pgfpathcurveto{\pgfqpoint{2.912756in}{6.730948in}}{\pgfqpoint{2.908366in}{6.720349in}}{\pgfqpoint{2.908366in}{6.709299in}}%
\pgfpathcurveto{\pgfqpoint{2.908366in}{6.698249in}}{\pgfqpoint{2.912756in}{6.687650in}}{\pgfqpoint{2.920570in}{6.679836in}}%
\pgfpathcurveto{\pgfqpoint{2.928383in}{6.672022in}}{\pgfqpoint{2.938983in}{6.667632in}}{\pgfqpoint{2.950033in}{6.667632in}}%
\pgfpathclose%
\pgfusepath{stroke,fill}%
\end{pgfscope}%
\begin{pgfscope}%
\pgfpathrectangle{\pgfqpoint{0.481978in}{0.331635in}}{\pgfqpoint{9.300000in}{7.700000in}}%
\pgfusepath{clip}%
\pgfsetbuttcap%
\pgfsetroundjoin%
\definecolor{currentfill}{rgb}{0.631373,0.788235,0.956863}%
\pgfsetfillcolor{currentfill}%
\pgfsetlinewidth{0.481800pt}%
\definecolor{currentstroke}{rgb}{1.000000,1.000000,1.000000}%
\pgfsetstrokecolor{currentstroke}%
\pgfsetdash{}{0pt}%
\pgfpathmoveto{\pgfqpoint{4.721839in}{7.137694in}}%
\pgfpathcurveto{\pgfqpoint{4.732889in}{7.137694in}}{\pgfqpoint{4.743488in}{7.142084in}}{\pgfqpoint{4.751301in}{7.149898in}}%
\pgfpathcurveto{\pgfqpoint{4.759115in}{7.157712in}}{\pgfqpoint{4.763505in}{7.168311in}}{\pgfqpoint{4.763505in}{7.179361in}}%
\pgfpathcurveto{\pgfqpoint{4.763505in}{7.190411in}}{\pgfqpoint{4.759115in}{7.201010in}}{\pgfqpoint{4.751301in}{7.208824in}}%
\pgfpathcurveto{\pgfqpoint{4.743488in}{7.216637in}}{\pgfqpoint{4.732889in}{7.221027in}}{\pgfqpoint{4.721839in}{7.221027in}}%
\pgfpathcurveto{\pgfqpoint{4.710788in}{7.221027in}}{\pgfqpoint{4.700189in}{7.216637in}}{\pgfqpoint{4.692376in}{7.208824in}}%
\pgfpathcurveto{\pgfqpoint{4.684562in}{7.201010in}}{\pgfqpoint{4.680172in}{7.190411in}}{\pgfqpoint{4.680172in}{7.179361in}}%
\pgfpathcurveto{\pgfqpoint{4.680172in}{7.168311in}}{\pgfqpoint{4.684562in}{7.157712in}}{\pgfqpoint{4.692376in}{7.149898in}}%
\pgfpathcurveto{\pgfqpoint{4.700189in}{7.142084in}}{\pgfqpoint{4.710788in}{7.137694in}}{\pgfqpoint{4.721839in}{7.137694in}}%
\pgfpathclose%
\pgfusepath{stroke,fill}%
\end{pgfscope}%
\begin{pgfscope}%
\pgfpathrectangle{\pgfqpoint{0.481978in}{0.331635in}}{\pgfqpoint{9.300000in}{7.700000in}}%
\pgfusepath{clip}%
\pgfsetbuttcap%
\pgfsetroundjoin%
\definecolor{currentfill}{rgb}{0.631373,0.788235,0.956863}%
\pgfsetfillcolor{currentfill}%
\pgfsetlinewidth{0.481800pt}%
\definecolor{currentstroke}{rgb}{1.000000,1.000000,1.000000}%
\pgfsetstrokecolor{currentstroke}%
\pgfsetdash{}{0pt}%
\pgfpathmoveto{\pgfqpoint{6.257325in}{5.295673in}}%
\pgfpathcurveto{\pgfqpoint{6.268375in}{5.295673in}}{\pgfqpoint{6.278974in}{5.300063in}}{\pgfqpoint{6.286787in}{5.307877in}}%
\pgfpathcurveto{\pgfqpoint{6.294601in}{5.315690in}}{\pgfqpoint{6.298991in}{5.326289in}}{\pgfqpoint{6.298991in}{5.337340in}}%
\pgfpathcurveto{\pgfqpoint{6.298991in}{5.348390in}}{\pgfqpoint{6.294601in}{5.358989in}}{\pgfqpoint{6.286787in}{5.366802in}}%
\pgfpathcurveto{\pgfqpoint{6.278974in}{5.374616in}}{\pgfqpoint{6.268375in}{5.379006in}}{\pgfqpoint{6.257325in}{5.379006in}}%
\pgfpathcurveto{\pgfqpoint{6.246275in}{5.379006in}}{\pgfqpoint{6.235676in}{5.374616in}}{\pgfqpoint{6.227862in}{5.366802in}}%
\pgfpathcurveto{\pgfqpoint{6.220048in}{5.358989in}}{\pgfqpoint{6.215658in}{5.348390in}}{\pgfqpoint{6.215658in}{5.337340in}}%
\pgfpathcurveto{\pgfqpoint{6.215658in}{5.326289in}}{\pgfqpoint{6.220048in}{5.315690in}}{\pgfqpoint{6.227862in}{5.307877in}}%
\pgfpathcurveto{\pgfqpoint{6.235676in}{5.300063in}}{\pgfqpoint{6.246275in}{5.295673in}}{\pgfqpoint{6.257325in}{5.295673in}}%
\pgfpathclose%
\pgfusepath{stroke,fill}%
\end{pgfscope}%
\begin{pgfscope}%
\pgfpathrectangle{\pgfqpoint{0.481978in}{0.331635in}}{\pgfqpoint{9.300000in}{7.700000in}}%
\pgfusepath{clip}%
\pgfsetbuttcap%
\pgfsetroundjoin%
\definecolor{currentfill}{rgb}{0.631373,0.788235,0.956863}%
\pgfsetfillcolor{currentfill}%
\pgfsetlinewidth{0.481800pt}%
\definecolor{currentstroke}{rgb}{1.000000,1.000000,1.000000}%
\pgfsetstrokecolor{currentstroke}%
\pgfsetdash{}{0pt}%
\pgfpathmoveto{\pgfqpoint{7.727200in}{4.813224in}}%
\pgfpathcurveto{\pgfqpoint{7.738250in}{4.813224in}}{\pgfqpoint{7.748849in}{4.817614in}}{\pgfqpoint{7.756662in}{4.825428in}}%
\pgfpathcurveto{\pgfqpoint{7.764476in}{4.833242in}}{\pgfqpoint{7.768866in}{4.843841in}}{\pgfqpoint{7.768866in}{4.854891in}}%
\pgfpathcurveto{\pgfqpoint{7.768866in}{4.865941in}}{\pgfqpoint{7.764476in}{4.876540in}}{\pgfqpoint{7.756662in}{4.884354in}}%
\pgfpathcurveto{\pgfqpoint{7.748849in}{4.892167in}}{\pgfqpoint{7.738250in}{4.896557in}}{\pgfqpoint{7.727200in}{4.896557in}}%
\pgfpathcurveto{\pgfqpoint{7.716150in}{4.896557in}}{\pgfqpoint{7.705550in}{4.892167in}}{\pgfqpoint{7.697737in}{4.884354in}}%
\pgfpathcurveto{\pgfqpoint{7.689923in}{4.876540in}}{\pgfqpoint{7.685533in}{4.865941in}}{\pgfqpoint{7.685533in}{4.854891in}}%
\pgfpathcurveto{\pgfqpoint{7.685533in}{4.843841in}}{\pgfqpoint{7.689923in}{4.833242in}}{\pgfqpoint{7.697737in}{4.825428in}}%
\pgfpathcurveto{\pgfqpoint{7.705550in}{4.817614in}}{\pgfqpoint{7.716150in}{4.813224in}}{\pgfqpoint{7.727200in}{4.813224in}}%
\pgfpathclose%
\pgfusepath{stroke,fill}%
\end{pgfscope}%
\begin{pgfscope}%
\pgfpathrectangle{\pgfqpoint{0.481978in}{0.331635in}}{\pgfqpoint{9.300000in}{7.700000in}}%
\pgfusepath{clip}%
\pgfsetbuttcap%
\pgfsetroundjoin%
\definecolor{currentfill}{rgb}{0.631373,0.788235,0.956863}%
\pgfsetfillcolor{currentfill}%
\pgfsetlinewidth{0.481800pt}%
\definecolor{currentstroke}{rgb}{1.000000,1.000000,1.000000}%
\pgfsetstrokecolor{currentstroke}%
\pgfsetdash{}{0pt}%
\pgfpathmoveto{\pgfqpoint{3.030295in}{1.686604in}}%
\pgfpathcurveto{\pgfqpoint{3.041345in}{1.686604in}}{\pgfqpoint{3.051944in}{1.690994in}}{\pgfqpoint{3.059758in}{1.698808in}}%
\pgfpathcurveto{\pgfqpoint{3.067571in}{1.706622in}}{\pgfqpoint{3.071962in}{1.717221in}}{\pgfqpoint{3.071962in}{1.728271in}}%
\pgfpathcurveto{\pgfqpoint{3.071962in}{1.739321in}}{\pgfqpoint{3.067571in}{1.749920in}}{\pgfqpoint{3.059758in}{1.757734in}}%
\pgfpathcurveto{\pgfqpoint{3.051944in}{1.765547in}}{\pgfqpoint{3.041345in}{1.769937in}}{\pgfqpoint{3.030295in}{1.769937in}}%
\pgfpathcurveto{\pgfqpoint{3.019245in}{1.769937in}}{\pgfqpoint{3.008646in}{1.765547in}}{\pgfqpoint{3.000832in}{1.757734in}}%
\pgfpathcurveto{\pgfqpoint{2.993019in}{1.749920in}}{\pgfqpoint{2.988628in}{1.739321in}}{\pgfqpoint{2.988628in}{1.728271in}}%
\pgfpathcurveto{\pgfqpoint{2.988628in}{1.717221in}}{\pgfqpoint{2.993019in}{1.706622in}}{\pgfqpoint{3.000832in}{1.698808in}}%
\pgfpathcurveto{\pgfqpoint{3.008646in}{1.690994in}}{\pgfqpoint{3.019245in}{1.686604in}}{\pgfqpoint{3.030295in}{1.686604in}}%
\pgfpathclose%
\pgfusepath{stroke,fill}%
\end{pgfscope}%
\begin{pgfscope}%
\pgfpathrectangle{\pgfqpoint{0.481978in}{0.331635in}}{\pgfqpoint{9.300000in}{7.700000in}}%
\pgfusepath{clip}%
\pgfsetbuttcap%
\pgfsetroundjoin%
\definecolor{currentfill}{rgb}{0.631373,0.788235,0.956863}%
\pgfsetfillcolor{currentfill}%
\pgfsetlinewidth{0.481800pt}%
\definecolor{currentstroke}{rgb}{1.000000,1.000000,1.000000}%
\pgfsetstrokecolor{currentstroke}%
\pgfsetdash{}{0pt}%
\pgfpathmoveto{\pgfqpoint{4.665732in}{6.455984in}}%
\pgfpathcurveto{\pgfqpoint{4.676782in}{6.455984in}}{\pgfqpoint{4.687381in}{6.460375in}}{\pgfqpoint{4.695194in}{6.468188in}}%
\pgfpathcurveto{\pgfqpoint{4.703008in}{6.476002in}}{\pgfqpoint{4.707398in}{6.486601in}}{\pgfqpoint{4.707398in}{6.497651in}}%
\pgfpathcurveto{\pgfqpoint{4.707398in}{6.508701in}}{\pgfqpoint{4.703008in}{6.519300in}}{\pgfqpoint{4.695194in}{6.527114in}}%
\pgfpathcurveto{\pgfqpoint{4.687381in}{6.534927in}}{\pgfqpoint{4.676782in}{6.539318in}}{\pgfqpoint{4.665732in}{6.539318in}}%
\pgfpathcurveto{\pgfqpoint{4.654681in}{6.539318in}}{\pgfqpoint{4.644082in}{6.534927in}}{\pgfqpoint{4.636269in}{6.527114in}}%
\pgfpathcurveto{\pgfqpoint{4.628455in}{6.519300in}}{\pgfqpoint{4.624065in}{6.508701in}}{\pgfqpoint{4.624065in}{6.497651in}}%
\pgfpathcurveto{\pgfqpoint{4.624065in}{6.486601in}}{\pgfqpoint{4.628455in}{6.476002in}}{\pgfqpoint{4.636269in}{6.468188in}}%
\pgfpathcurveto{\pgfqpoint{4.644082in}{6.460375in}}{\pgfqpoint{4.654681in}{6.455984in}}{\pgfqpoint{4.665732in}{6.455984in}}%
\pgfpathclose%
\pgfusepath{stroke,fill}%
\end{pgfscope}%
\begin{pgfscope}%
\pgfpathrectangle{\pgfqpoint{0.481978in}{0.331635in}}{\pgfqpoint{9.300000in}{7.700000in}}%
\pgfusepath{clip}%
\pgfsetbuttcap%
\pgfsetroundjoin%
\definecolor{currentfill}{rgb}{0.631373,0.788235,0.956863}%
\pgfsetfillcolor{currentfill}%
\pgfsetlinewidth{0.481800pt}%
\definecolor{currentstroke}{rgb}{1.000000,1.000000,1.000000}%
\pgfsetstrokecolor{currentstroke}%
\pgfsetdash{}{0pt}%
\pgfpathmoveto{\pgfqpoint{6.123240in}{1.999637in}}%
\pgfpathcurveto{\pgfqpoint{6.134290in}{1.999637in}}{\pgfqpoint{6.144890in}{2.004027in}}{\pgfqpoint{6.152703in}{2.011841in}}%
\pgfpathcurveto{\pgfqpoint{6.160517in}{2.019655in}}{\pgfqpoint{6.164907in}{2.030254in}}{\pgfqpoint{6.164907in}{2.041304in}}%
\pgfpathcurveto{\pgfqpoint{6.164907in}{2.052354in}}{\pgfqpoint{6.160517in}{2.062953in}}{\pgfqpoint{6.152703in}{2.070766in}}%
\pgfpathcurveto{\pgfqpoint{6.144890in}{2.078580in}}{\pgfqpoint{6.134290in}{2.082970in}}{\pgfqpoint{6.123240in}{2.082970in}}%
\pgfpathcurveto{\pgfqpoint{6.112190in}{2.082970in}}{\pgfqpoint{6.101591in}{2.078580in}}{\pgfqpoint{6.093778in}{2.070766in}}%
\pgfpathcurveto{\pgfqpoint{6.085964in}{2.062953in}}{\pgfqpoint{6.081574in}{2.052354in}}{\pgfqpoint{6.081574in}{2.041304in}}%
\pgfpathcurveto{\pgfqpoint{6.081574in}{2.030254in}}{\pgfqpoint{6.085964in}{2.019655in}}{\pgfqpoint{6.093778in}{2.011841in}}%
\pgfpathcurveto{\pgfqpoint{6.101591in}{2.004027in}}{\pgfqpoint{6.112190in}{1.999637in}}{\pgfqpoint{6.123240in}{1.999637in}}%
\pgfpathclose%
\pgfusepath{stroke,fill}%
\end{pgfscope}%
\begin{pgfscope}%
\pgfpathrectangle{\pgfqpoint{0.481978in}{0.331635in}}{\pgfqpoint{9.300000in}{7.700000in}}%
\pgfusepath{clip}%
\pgfsetbuttcap%
\pgfsetroundjoin%
\definecolor{currentfill}{rgb}{0.631373,0.788235,0.956863}%
\pgfsetfillcolor{currentfill}%
\pgfsetlinewidth{0.481800pt}%
\definecolor{currentstroke}{rgb}{1.000000,1.000000,1.000000}%
\pgfsetstrokecolor{currentstroke}%
\pgfsetdash{}{0pt}%
\pgfpathmoveto{\pgfqpoint{2.813753in}{6.481376in}}%
\pgfpathcurveto{\pgfqpoint{2.824803in}{6.481376in}}{\pgfqpoint{2.835402in}{6.485766in}}{\pgfqpoint{2.843216in}{6.493580in}}%
\pgfpathcurveto{\pgfqpoint{2.851030in}{6.501394in}}{\pgfqpoint{2.855420in}{6.511993in}}{\pgfqpoint{2.855420in}{6.523043in}}%
\pgfpathcurveto{\pgfqpoint{2.855420in}{6.534093in}}{\pgfqpoint{2.851030in}{6.544692in}}{\pgfqpoint{2.843216in}{6.552505in}}%
\pgfpathcurveto{\pgfqpoint{2.835402in}{6.560319in}}{\pgfqpoint{2.824803in}{6.564709in}}{\pgfqpoint{2.813753in}{6.564709in}}%
\pgfpathcurveto{\pgfqpoint{2.802703in}{6.564709in}}{\pgfqpoint{2.792104in}{6.560319in}}{\pgfqpoint{2.784290in}{6.552505in}}%
\pgfpathcurveto{\pgfqpoint{2.776477in}{6.544692in}}{\pgfqpoint{2.772086in}{6.534093in}}{\pgfqpoint{2.772086in}{6.523043in}}%
\pgfpathcurveto{\pgfqpoint{2.772086in}{6.511993in}}{\pgfqpoint{2.776477in}{6.501394in}}{\pgfqpoint{2.784290in}{6.493580in}}%
\pgfpathcurveto{\pgfqpoint{2.792104in}{6.485766in}}{\pgfqpoint{2.802703in}{6.481376in}}{\pgfqpoint{2.813753in}{6.481376in}}%
\pgfpathclose%
\pgfusepath{stroke,fill}%
\end{pgfscope}%
\begin{pgfscope}%
\pgfpathrectangle{\pgfqpoint{0.481978in}{0.331635in}}{\pgfqpoint{9.300000in}{7.700000in}}%
\pgfusepath{clip}%
\pgfsetbuttcap%
\pgfsetroundjoin%
\definecolor{currentfill}{rgb}{0.631373,0.788235,0.956863}%
\pgfsetfillcolor{currentfill}%
\pgfsetlinewidth{0.481800pt}%
\definecolor{currentstroke}{rgb}{1.000000,1.000000,1.000000}%
\pgfsetstrokecolor{currentstroke}%
\pgfsetdash{}{0pt}%
\pgfpathmoveto{\pgfqpoint{8.071034in}{4.382284in}}%
\pgfpathcurveto{\pgfqpoint{8.082085in}{4.382284in}}{\pgfqpoint{8.092684in}{4.386674in}}{\pgfqpoint{8.100497in}{4.394488in}}%
\pgfpathcurveto{\pgfqpoint{8.108311in}{4.402301in}}{\pgfqpoint{8.112701in}{4.412900in}}{\pgfqpoint{8.112701in}{4.423951in}}%
\pgfpathcurveto{\pgfqpoint{8.112701in}{4.435001in}}{\pgfqpoint{8.108311in}{4.445600in}}{\pgfqpoint{8.100497in}{4.453413in}}%
\pgfpathcurveto{\pgfqpoint{8.092684in}{4.461227in}}{\pgfqpoint{8.082085in}{4.465617in}}{\pgfqpoint{8.071034in}{4.465617in}}%
\pgfpathcurveto{\pgfqpoint{8.059984in}{4.465617in}}{\pgfqpoint{8.049385in}{4.461227in}}{\pgfqpoint{8.041572in}{4.453413in}}%
\pgfpathcurveto{\pgfqpoint{8.033758in}{4.445600in}}{\pgfqpoint{8.029368in}{4.435001in}}{\pgfqpoint{8.029368in}{4.423951in}}%
\pgfpathcurveto{\pgfqpoint{8.029368in}{4.412900in}}{\pgfqpoint{8.033758in}{4.402301in}}{\pgfqpoint{8.041572in}{4.394488in}}%
\pgfpathcurveto{\pgfqpoint{8.049385in}{4.386674in}}{\pgfqpoint{8.059984in}{4.382284in}}{\pgfqpoint{8.071034in}{4.382284in}}%
\pgfpathclose%
\pgfusepath{stroke,fill}%
\end{pgfscope}%
\begin{pgfscope}%
\pgfpathrectangle{\pgfqpoint{0.481978in}{0.331635in}}{\pgfqpoint{9.300000in}{7.700000in}}%
\pgfusepath{clip}%
\pgfsetbuttcap%
\pgfsetroundjoin%
\definecolor{currentfill}{rgb}{0.631373,0.788235,0.956863}%
\pgfsetfillcolor{currentfill}%
\pgfsetlinewidth{0.481800pt}%
\definecolor{currentstroke}{rgb}{1.000000,1.000000,1.000000}%
\pgfsetstrokecolor{currentstroke}%
\pgfsetdash{}{0pt}%
\pgfpathmoveto{\pgfqpoint{6.467051in}{2.323379in}}%
\pgfpathcurveto{\pgfqpoint{6.478101in}{2.323379in}}{\pgfqpoint{6.488700in}{2.327769in}}{\pgfqpoint{6.496514in}{2.335583in}}%
\pgfpathcurveto{\pgfqpoint{6.504328in}{2.343396in}}{\pgfqpoint{6.508718in}{2.353995in}}{\pgfqpoint{6.508718in}{2.365045in}}%
\pgfpathcurveto{\pgfqpoint{6.508718in}{2.376096in}}{\pgfqpoint{6.504328in}{2.386695in}}{\pgfqpoint{6.496514in}{2.394508in}}%
\pgfpathcurveto{\pgfqpoint{6.488700in}{2.402322in}}{\pgfqpoint{6.478101in}{2.406712in}}{\pgfqpoint{6.467051in}{2.406712in}}%
\pgfpathcurveto{\pgfqpoint{6.456001in}{2.406712in}}{\pgfqpoint{6.445402in}{2.402322in}}{\pgfqpoint{6.437588in}{2.394508in}}%
\pgfpathcurveto{\pgfqpoint{6.429775in}{2.386695in}}{\pgfqpoint{6.425384in}{2.376096in}}{\pgfqpoint{6.425384in}{2.365045in}}%
\pgfpathcurveto{\pgfqpoint{6.425384in}{2.353995in}}{\pgfqpoint{6.429775in}{2.343396in}}{\pgfqpoint{6.437588in}{2.335583in}}%
\pgfpathcurveto{\pgfqpoint{6.445402in}{2.327769in}}{\pgfqpoint{6.456001in}{2.323379in}}{\pgfqpoint{6.467051in}{2.323379in}}%
\pgfpathclose%
\pgfusepath{stroke,fill}%
\end{pgfscope}%
\begin{pgfscope}%
\pgfpathrectangle{\pgfqpoint{0.481978in}{0.331635in}}{\pgfqpoint{9.300000in}{7.700000in}}%
\pgfusepath{clip}%
\pgfsetbuttcap%
\pgfsetroundjoin%
\definecolor{currentfill}{rgb}{0.631373,0.788235,0.956863}%
\pgfsetfillcolor{currentfill}%
\pgfsetlinewidth{0.481800pt}%
\definecolor{currentstroke}{rgb}{1.000000,1.000000,1.000000}%
\pgfsetstrokecolor{currentstroke}%
\pgfsetdash{}{0pt}%
\pgfpathmoveto{\pgfqpoint{7.133111in}{4.672385in}}%
\pgfpathcurveto{\pgfqpoint{7.144161in}{4.672385in}}{\pgfqpoint{7.154760in}{4.676775in}}{\pgfqpoint{7.162573in}{4.684589in}}%
\pgfpathcurveto{\pgfqpoint{7.170387in}{4.692402in}}{\pgfqpoint{7.174777in}{4.703001in}}{\pgfqpoint{7.174777in}{4.714051in}}%
\pgfpathcurveto{\pgfqpoint{7.174777in}{4.725102in}}{\pgfqpoint{7.170387in}{4.735701in}}{\pgfqpoint{7.162573in}{4.743514in}}%
\pgfpathcurveto{\pgfqpoint{7.154760in}{4.751328in}}{\pgfqpoint{7.144161in}{4.755718in}}{\pgfqpoint{7.133111in}{4.755718in}}%
\pgfpathcurveto{\pgfqpoint{7.122061in}{4.755718in}}{\pgfqpoint{7.111462in}{4.751328in}}{\pgfqpoint{7.103648in}{4.743514in}}%
\pgfpathcurveto{\pgfqpoint{7.095834in}{4.735701in}}{\pgfqpoint{7.091444in}{4.725102in}}{\pgfqpoint{7.091444in}{4.714051in}}%
\pgfpathcurveto{\pgfqpoint{7.091444in}{4.703001in}}{\pgfqpoint{7.095834in}{4.692402in}}{\pgfqpoint{7.103648in}{4.684589in}}%
\pgfpathcurveto{\pgfqpoint{7.111462in}{4.676775in}}{\pgfqpoint{7.122061in}{4.672385in}}{\pgfqpoint{7.133111in}{4.672385in}}%
\pgfpathclose%
\pgfusepath{stroke,fill}%
\end{pgfscope}%
\begin{pgfscope}%
\pgfpathrectangle{\pgfqpoint{0.481978in}{0.331635in}}{\pgfqpoint{9.300000in}{7.700000in}}%
\pgfusepath{clip}%
\pgfsetbuttcap%
\pgfsetroundjoin%
\definecolor{currentfill}{rgb}{0.631373,0.788235,0.956863}%
\pgfsetfillcolor{currentfill}%
\pgfsetlinewidth{0.481800pt}%
\definecolor{currentstroke}{rgb}{1.000000,1.000000,1.000000}%
\pgfsetstrokecolor{currentstroke}%
\pgfsetdash{}{0pt}%
\pgfpathmoveto{\pgfqpoint{5.228856in}{2.150942in}}%
\pgfpathcurveto{\pgfqpoint{5.239906in}{2.150942in}}{\pgfqpoint{5.250505in}{2.155332in}}{\pgfqpoint{5.258319in}{2.163146in}}%
\pgfpathcurveto{\pgfqpoint{5.266132in}{2.170959in}}{\pgfqpoint{5.270523in}{2.181558in}}{\pgfqpoint{5.270523in}{2.192608in}}%
\pgfpathcurveto{\pgfqpoint{5.270523in}{2.203659in}}{\pgfqpoint{5.266132in}{2.214258in}}{\pgfqpoint{5.258319in}{2.222071in}}%
\pgfpathcurveto{\pgfqpoint{5.250505in}{2.229885in}}{\pgfqpoint{5.239906in}{2.234275in}}{\pgfqpoint{5.228856in}{2.234275in}}%
\pgfpathcurveto{\pgfqpoint{5.217806in}{2.234275in}}{\pgfqpoint{5.207207in}{2.229885in}}{\pgfqpoint{5.199393in}{2.222071in}}%
\pgfpathcurveto{\pgfqpoint{5.191580in}{2.214258in}}{\pgfqpoint{5.187189in}{2.203659in}}{\pgfqpoint{5.187189in}{2.192608in}}%
\pgfpathcurveto{\pgfqpoint{5.187189in}{2.181558in}}{\pgfqpoint{5.191580in}{2.170959in}}{\pgfqpoint{5.199393in}{2.163146in}}%
\pgfpathcurveto{\pgfqpoint{5.207207in}{2.155332in}}{\pgfqpoint{5.217806in}{2.150942in}}{\pgfqpoint{5.228856in}{2.150942in}}%
\pgfpathclose%
\pgfusepath{stroke,fill}%
\end{pgfscope}%
\begin{pgfscope}%
\pgfpathrectangle{\pgfqpoint{0.481978in}{0.331635in}}{\pgfqpoint{9.300000in}{7.700000in}}%
\pgfusepath{clip}%
\pgfsetbuttcap%
\pgfsetroundjoin%
\definecolor{currentfill}{rgb}{0.631373,0.788235,0.956863}%
\pgfsetfillcolor{currentfill}%
\pgfsetlinewidth{0.481800pt}%
\definecolor{currentstroke}{rgb}{1.000000,1.000000,1.000000}%
\pgfsetstrokecolor{currentstroke}%
\pgfsetdash{}{0pt}%
\pgfpathmoveto{\pgfqpoint{6.173996in}{3.277145in}}%
\pgfpathcurveto{\pgfqpoint{6.185046in}{3.277145in}}{\pgfqpoint{6.195645in}{3.281535in}}{\pgfqpoint{6.203459in}{3.289349in}}%
\pgfpathcurveto{\pgfqpoint{6.211273in}{3.297162in}}{\pgfqpoint{6.215663in}{3.307761in}}{\pgfqpoint{6.215663in}{3.318811in}}%
\pgfpathcurveto{\pgfqpoint{6.215663in}{3.329861in}}{\pgfqpoint{6.211273in}{3.340460in}}{\pgfqpoint{6.203459in}{3.348274in}}%
\pgfpathcurveto{\pgfqpoint{6.195645in}{3.356088in}}{\pgfqpoint{6.185046in}{3.360478in}}{\pgfqpoint{6.173996in}{3.360478in}}%
\pgfpathcurveto{\pgfqpoint{6.162946in}{3.360478in}}{\pgfqpoint{6.152347in}{3.356088in}}{\pgfqpoint{6.144533in}{3.348274in}}%
\pgfpathcurveto{\pgfqpoint{6.136720in}{3.340460in}}{\pgfqpoint{6.132330in}{3.329861in}}{\pgfqpoint{6.132330in}{3.318811in}}%
\pgfpathcurveto{\pgfqpoint{6.132330in}{3.307761in}}{\pgfqpoint{6.136720in}{3.297162in}}{\pgfqpoint{6.144533in}{3.289349in}}%
\pgfpathcurveto{\pgfqpoint{6.152347in}{3.281535in}}{\pgfqpoint{6.162946in}{3.277145in}}{\pgfqpoint{6.173996in}{3.277145in}}%
\pgfpathclose%
\pgfusepath{stroke,fill}%
\end{pgfscope}%
\begin{pgfscope}%
\pgfpathrectangle{\pgfqpoint{0.481978in}{0.331635in}}{\pgfqpoint{9.300000in}{7.700000in}}%
\pgfusepath{clip}%
\pgfsetbuttcap%
\pgfsetroundjoin%
\definecolor{currentfill}{rgb}{0.631373,0.788235,0.956863}%
\pgfsetfillcolor{currentfill}%
\pgfsetlinewidth{0.481800pt}%
\definecolor{currentstroke}{rgb}{1.000000,1.000000,1.000000}%
\pgfsetstrokecolor{currentstroke}%
\pgfsetdash{}{0pt}%
\pgfpathmoveto{\pgfqpoint{2.961409in}{1.495653in}}%
\pgfpathcurveto{\pgfqpoint{2.972459in}{1.495653in}}{\pgfqpoint{2.983058in}{1.500043in}}{\pgfqpoint{2.990872in}{1.507857in}}%
\pgfpathcurveto{\pgfqpoint{2.998686in}{1.515670in}}{\pgfqpoint{3.003076in}{1.526269in}}{\pgfqpoint{3.003076in}{1.537320in}}%
\pgfpathcurveto{\pgfqpoint{3.003076in}{1.548370in}}{\pgfqpoint{2.998686in}{1.558969in}}{\pgfqpoint{2.990872in}{1.566782in}}%
\pgfpathcurveto{\pgfqpoint{2.983058in}{1.574596in}}{\pgfqpoint{2.972459in}{1.578986in}}{\pgfqpoint{2.961409in}{1.578986in}}%
\pgfpathcurveto{\pgfqpoint{2.950359in}{1.578986in}}{\pgfqpoint{2.939760in}{1.574596in}}{\pgfqpoint{2.931946in}{1.566782in}}%
\pgfpathcurveto{\pgfqpoint{2.924133in}{1.558969in}}{\pgfqpoint{2.919743in}{1.548370in}}{\pgfqpoint{2.919743in}{1.537320in}}%
\pgfpathcurveto{\pgfqpoint{2.919743in}{1.526269in}}{\pgfqpoint{2.924133in}{1.515670in}}{\pgfqpoint{2.931946in}{1.507857in}}%
\pgfpathcurveto{\pgfqpoint{2.939760in}{1.500043in}}{\pgfqpoint{2.950359in}{1.495653in}}{\pgfqpoint{2.961409in}{1.495653in}}%
\pgfpathclose%
\pgfusepath{stroke,fill}%
\end{pgfscope}%
\begin{pgfscope}%
\pgfpathrectangle{\pgfqpoint{0.481978in}{0.331635in}}{\pgfqpoint{9.300000in}{7.700000in}}%
\pgfusepath{clip}%
\pgfsetbuttcap%
\pgfsetroundjoin%
\definecolor{currentfill}{rgb}{0.631373,0.788235,0.956863}%
\pgfsetfillcolor{currentfill}%
\pgfsetlinewidth{0.481800pt}%
\definecolor{currentstroke}{rgb}{1.000000,1.000000,1.000000}%
\pgfsetstrokecolor{currentstroke}%
\pgfsetdash{}{0pt}%
\pgfpathmoveto{\pgfqpoint{5.678749in}{4.846566in}}%
\pgfpathcurveto{\pgfqpoint{5.689799in}{4.846566in}}{\pgfqpoint{5.700398in}{4.850956in}}{\pgfqpoint{5.708212in}{4.858769in}}%
\pgfpathcurveto{\pgfqpoint{5.716026in}{4.866583in}}{\pgfqpoint{5.720416in}{4.877182in}}{\pgfqpoint{5.720416in}{4.888232in}}%
\pgfpathcurveto{\pgfqpoint{5.720416in}{4.899282in}}{\pgfqpoint{5.716026in}{4.909881in}}{\pgfqpoint{5.708212in}{4.917695in}}%
\pgfpathcurveto{\pgfqpoint{5.700398in}{4.925509in}}{\pgfqpoint{5.689799in}{4.929899in}}{\pgfqpoint{5.678749in}{4.929899in}}%
\pgfpathcurveto{\pgfqpoint{5.667699in}{4.929899in}}{\pgfqpoint{5.657100in}{4.925509in}}{\pgfqpoint{5.649286in}{4.917695in}}%
\pgfpathcurveto{\pgfqpoint{5.641473in}{4.909881in}}{\pgfqpoint{5.637082in}{4.899282in}}{\pgfqpoint{5.637082in}{4.888232in}}%
\pgfpathcurveto{\pgfqpoint{5.637082in}{4.877182in}}{\pgfqpoint{5.641473in}{4.866583in}}{\pgfqpoint{5.649286in}{4.858769in}}%
\pgfpathcurveto{\pgfqpoint{5.657100in}{4.850956in}}{\pgfqpoint{5.667699in}{4.846566in}}{\pgfqpoint{5.678749in}{4.846566in}}%
\pgfpathclose%
\pgfusepath{stroke,fill}%
\end{pgfscope}%
\begin{pgfscope}%
\pgfpathrectangle{\pgfqpoint{0.481978in}{0.331635in}}{\pgfqpoint{9.300000in}{7.700000in}}%
\pgfusepath{clip}%
\pgfsetbuttcap%
\pgfsetroundjoin%
\definecolor{currentfill}{rgb}{0.631373,0.788235,0.956863}%
\pgfsetfillcolor{currentfill}%
\pgfsetlinewidth{0.481800pt}%
\definecolor{currentstroke}{rgb}{1.000000,1.000000,1.000000}%
\pgfsetstrokecolor{currentstroke}%
\pgfsetdash{}{0pt}%
\pgfpathmoveto{\pgfqpoint{4.278462in}{1.270944in}}%
\pgfpathcurveto{\pgfqpoint{4.289512in}{1.270944in}}{\pgfqpoint{4.300111in}{1.275334in}}{\pgfqpoint{4.307924in}{1.283148in}}%
\pgfpathcurveto{\pgfqpoint{4.315738in}{1.290962in}}{\pgfqpoint{4.320128in}{1.301561in}}{\pgfqpoint{4.320128in}{1.312611in}}%
\pgfpathcurveto{\pgfqpoint{4.320128in}{1.323661in}}{\pgfqpoint{4.315738in}{1.334260in}}{\pgfqpoint{4.307924in}{1.342074in}}%
\pgfpathcurveto{\pgfqpoint{4.300111in}{1.349887in}}{\pgfqpoint{4.289512in}{1.354277in}}{\pgfqpoint{4.278462in}{1.354277in}}%
\pgfpathcurveto{\pgfqpoint{4.267411in}{1.354277in}}{\pgfqpoint{4.256812in}{1.349887in}}{\pgfqpoint{4.248999in}{1.342074in}}%
\pgfpathcurveto{\pgfqpoint{4.241185in}{1.334260in}}{\pgfqpoint{4.236795in}{1.323661in}}{\pgfqpoint{4.236795in}{1.312611in}}%
\pgfpathcurveto{\pgfqpoint{4.236795in}{1.301561in}}{\pgfqpoint{4.241185in}{1.290962in}}{\pgfqpoint{4.248999in}{1.283148in}}%
\pgfpathcurveto{\pgfqpoint{4.256812in}{1.275334in}}{\pgfqpoint{4.267411in}{1.270944in}}{\pgfqpoint{4.278462in}{1.270944in}}%
\pgfpathclose%
\pgfusepath{stroke,fill}%
\end{pgfscope}%
\begin{pgfscope}%
\pgfpathrectangle{\pgfqpoint{0.481978in}{0.331635in}}{\pgfqpoint{9.300000in}{7.700000in}}%
\pgfusepath{clip}%
\pgfsetbuttcap%
\pgfsetroundjoin%
\definecolor{currentfill}{rgb}{0.631373,0.788235,0.956863}%
\pgfsetfillcolor{currentfill}%
\pgfsetlinewidth{0.481800pt}%
\definecolor{currentstroke}{rgb}{1.000000,1.000000,1.000000}%
\pgfsetstrokecolor{currentstroke}%
\pgfsetdash{}{0pt}%
\pgfpathmoveto{\pgfqpoint{6.739909in}{2.256314in}}%
\pgfpathcurveto{\pgfqpoint{6.750959in}{2.256314in}}{\pgfqpoint{6.761558in}{2.260704in}}{\pgfqpoint{6.769372in}{2.268518in}}%
\pgfpathcurveto{\pgfqpoint{6.777185in}{2.276331in}}{\pgfqpoint{6.781576in}{2.286930in}}{\pgfqpoint{6.781576in}{2.297981in}}%
\pgfpathcurveto{\pgfqpoint{6.781576in}{2.309031in}}{\pgfqpoint{6.777185in}{2.319630in}}{\pgfqpoint{6.769372in}{2.327443in}}%
\pgfpathcurveto{\pgfqpoint{6.761558in}{2.335257in}}{\pgfqpoint{6.750959in}{2.339647in}}{\pgfqpoint{6.739909in}{2.339647in}}%
\pgfpathcurveto{\pgfqpoint{6.728859in}{2.339647in}}{\pgfqpoint{6.718260in}{2.335257in}}{\pgfqpoint{6.710446in}{2.327443in}}%
\pgfpathcurveto{\pgfqpoint{6.702632in}{2.319630in}}{\pgfqpoint{6.698242in}{2.309031in}}{\pgfqpoint{6.698242in}{2.297981in}}%
\pgfpathcurveto{\pgfqpoint{6.698242in}{2.286930in}}{\pgfqpoint{6.702632in}{2.276331in}}{\pgfqpoint{6.710446in}{2.268518in}}%
\pgfpathcurveto{\pgfqpoint{6.718260in}{2.260704in}}{\pgfqpoint{6.728859in}{2.256314in}}{\pgfqpoint{6.739909in}{2.256314in}}%
\pgfpathclose%
\pgfusepath{stroke,fill}%
\end{pgfscope}%
\begin{pgfscope}%
\pgfpathrectangle{\pgfqpoint{0.481978in}{0.331635in}}{\pgfqpoint{9.300000in}{7.700000in}}%
\pgfusepath{clip}%
\pgfsetbuttcap%
\pgfsetroundjoin%
\definecolor{currentfill}{rgb}{0.631373,0.788235,0.956863}%
\pgfsetfillcolor{currentfill}%
\pgfsetlinewidth{0.481800pt}%
\definecolor{currentstroke}{rgb}{1.000000,1.000000,1.000000}%
\pgfsetstrokecolor{currentstroke}%
\pgfsetdash{}{0pt}%
\pgfpathmoveto{\pgfqpoint{5.837904in}{2.538586in}}%
\pgfpathcurveto{\pgfqpoint{5.848954in}{2.538586in}}{\pgfqpoint{5.859553in}{2.542976in}}{\pgfqpoint{5.867366in}{2.550790in}}%
\pgfpathcurveto{\pgfqpoint{5.875180in}{2.558603in}}{\pgfqpoint{5.879570in}{2.569202in}}{\pgfqpoint{5.879570in}{2.580252in}}%
\pgfpathcurveto{\pgfqpoint{5.879570in}{2.591302in}}{\pgfqpoint{5.875180in}{2.601901in}}{\pgfqpoint{5.867366in}{2.609715in}}%
\pgfpathcurveto{\pgfqpoint{5.859553in}{2.617529in}}{\pgfqpoint{5.848954in}{2.621919in}}{\pgfqpoint{5.837904in}{2.621919in}}%
\pgfpathcurveto{\pgfqpoint{5.826854in}{2.621919in}}{\pgfqpoint{5.816254in}{2.617529in}}{\pgfqpoint{5.808441in}{2.609715in}}%
\pgfpathcurveto{\pgfqpoint{5.800627in}{2.601901in}}{\pgfqpoint{5.796237in}{2.591302in}}{\pgfqpoint{5.796237in}{2.580252in}}%
\pgfpathcurveto{\pgfqpoint{5.796237in}{2.569202in}}{\pgfqpoint{5.800627in}{2.558603in}}{\pgfqpoint{5.808441in}{2.550790in}}%
\pgfpathcurveto{\pgfqpoint{5.816254in}{2.542976in}}{\pgfqpoint{5.826854in}{2.538586in}}{\pgfqpoint{5.837904in}{2.538586in}}%
\pgfpathclose%
\pgfusepath{stroke,fill}%
\end{pgfscope}%
\begin{pgfscope}%
\pgfpathrectangle{\pgfqpoint{0.481978in}{0.331635in}}{\pgfqpoint{9.300000in}{7.700000in}}%
\pgfusepath{clip}%
\pgfsetbuttcap%
\pgfsetroundjoin%
\definecolor{currentfill}{rgb}{0.631373,0.788235,0.956863}%
\pgfsetfillcolor{currentfill}%
\pgfsetlinewidth{0.481800pt}%
\definecolor{currentstroke}{rgb}{1.000000,1.000000,1.000000}%
\pgfsetstrokecolor{currentstroke}%
\pgfsetdash{}{0pt}%
\pgfpathmoveto{\pgfqpoint{7.682296in}{2.269395in}}%
\pgfpathcurveto{\pgfqpoint{7.693346in}{2.269395in}}{\pgfqpoint{7.703945in}{2.273785in}}{\pgfqpoint{7.711758in}{2.281598in}}%
\pgfpathcurveto{\pgfqpoint{7.719572in}{2.289412in}}{\pgfqpoint{7.723962in}{2.300011in}}{\pgfqpoint{7.723962in}{2.311061in}}%
\pgfpathcurveto{\pgfqpoint{7.723962in}{2.322111in}}{\pgfqpoint{7.719572in}{2.332710in}}{\pgfqpoint{7.711758in}{2.340524in}}%
\pgfpathcurveto{\pgfqpoint{7.703945in}{2.348338in}}{\pgfqpoint{7.693346in}{2.352728in}}{\pgfqpoint{7.682296in}{2.352728in}}%
\pgfpathcurveto{\pgfqpoint{7.671245in}{2.352728in}}{\pgfqpoint{7.660646in}{2.348338in}}{\pgfqpoint{7.652833in}{2.340524in}}%
\pgfpathcurveto{\pgfqpoint{7.645019in}{2.332710in}}{\pgfqpoint{7.640629in}{2.322111in}}{\pgfqpoint{7.640629in}{2.311061in}}%
\pgfpathcurveto{\pgfqpoint{7.640629in}{2.300011in}}{\pgfqpoint{7.645019in}{2.289412in}}{\pgfqpoint{7.652833in}{2.281598in}}%
\pgfpathcurveto{\pgfqpoint{7.660646in}{2.273785in}}{\pgfqpoint{7.671245in}{2.269395in}}{\pgfqpoint{7.682296in}{2.269395in}}%
\pgfpathclose%
\pgfusepath{stroke,fill}%
\end{pgfscope}%
\begin{pgfscope}%
\pgfpathrectangle{\pgfqpoint{0.481978in}{0.331635in}}{\pgfqpoint{9.300000in}{7.700000in}}%
\pgfusepath{clip}%
\pgfsetbuttcap%
\pgfsetroundjoin%
\definecolor{currentfill}{rgb}{0.631373,0.788235,0.956863}%
\pgfsetfillcolor{currentfill}%
\pgfsetlinewidth{0.481800pt}%
\definecolor{currentstroke}{rgb}{1.000000,1.000000,1.000000}%
\pgfsetstrokecolor{currentstroke}%
\pgfsetdash{}{0pt}%
\pgfpathmoveto{\pgfqpoint{4.017749in}{1.858138in}}%
\pgfpathcurveto{\pgfqpoint{4.028799in}{1.858138in}}{\pgfqpoint{4.039398in}{1.862528in}}{\pgfqpoint{4.047212in}{1.870342in}}%
\pgfpathcurveto{\pgfqpoint{4.055026in}{1.878155in}}{\pgfqpoint{4.059416in}{1.888754in}}{\pgfqpoint{4.059416in}{1.899805in}}%
\pgfpathcurveto{\pgfqpoint{4.059416in}{1.910855in}}{\pgfqpoint{4.055026in}{1.921454in}}{\pgfqpoint{4.047212in}{1.929267in}}%
\pgfpathcurveto{\pgfqpoint{4.039398in}{1.937081in}}{\pgfqpoint{4.028799in}{1.941471in}}{\pgfqpoint{4.017749in}{1.941471in}}%
\pgfpathcurveto{\pgfqpoint{4.006699in}{1.941471in}}{\pgfqpoint{3.996100in}{1.937081in}}{\pgfqpoint{3.988286in}{1.929267in}}%
\pgfpathcurveto{\pgfqpoint{3.980473in}{1.921454in}}{\pgfqpoint{3.976082in}{1.910855in}}{\pgfqpoint{3.976082in}{1.899805in}}%
\pgfpathcurveto{\pgfqpoint{3.976082in}{1.888754in}}{\pgfqpoint{3.980473in}{1.878155in}}{\pgfqpoint{3.988286in}{1.870342in}}%
\pgfpathcurveto{\pgfqpoint{3.996100in}{1.862528in}}{\pgfqpoint{4.006699in}{1.858138in}}{\pgfqpoint{4.017749in}{1.858138in}}%
\pgfpathclose%
\pgfusepath{stroke,fill}%
\end{pgfscope}%
\begin{pgfscope}%
\pgfpathrectangle{\pgfqpoint{0.481978in}{0.331635in}}{\pgfqpoint{9.300000in}{7.700000in}}%
\pgfusepath{clip}%
\pgfsetbuttcap%
\pgfsetroundjoin%
\definecolor{currentfill}{rgb}{0.631373,0.788235,0.956863}%
\pgfsetfillcolor{currentfill}%
\pgfsetlinewidth{0.481800pt}%
\definecolor{currentstroke}{rgb}{1.000000,1.000000,1.000000}%
\pgfsetstrokecolor{currentstroke}%
\pgfsetdash{}{0pt}%
\pgfpathmoveto{\pgfqpoint{6.127545in}{4.842798in}}%
\pgfpathcurveto{\pgfqpoint{6.138595in}{4.842798in}}{\pgfqpoint{6.149194in}{4.847189in}}{\pgfqpoint{6.157008in}{4.855002in}}%
\pgfpathcurveto{\pgfqpoint{6.164821in}{4.862816in}}{\pgfqpoint{6.169212in}{4.873415in}}{\pgfqpoint{6.169212in}{4.884465in}}%
\pgfpathcurveto{\pgfqpoint{6.169212in}{4.895515in}}{\pgfqpoint{6.164821in}{4.906114in}}{\pgfqpoint{6.157008in}{4.913928in}}%
\pgfpathcurveto{\pgfqpoint{6.149194in}{4.921741in}}{\pgfqpoint{6.138595in}{4.926132in}}{\pgfqpoint{6.127545in}{4.926132in}}%
\pgfpathcurveto{\pgfqpoint{6.116495in}{4.926132in}}{\pgfqpoint{6.105896in}{4.921741in}}{\pgfqpoint{6.098082in}{4.913928in}}%
\pgfpathcurveto{\pgfqpoint{6.090269in}{4.906114in}}{\pgfqpoint{6.085878in}{4.895515in}}{\pgfqpoint{6.085878in}{4.884465in}}%
\pgfpathcurveto{\pgfqpoint{6.085878in}{4.873415in}}{\pgfqpoint{6.090269in}{4.862816in}}{\pgfqpoint{6.098082in}{4.855002in}}%
\pgfpathcurveto{\pgfqpoint{6.105896in}{4.847189in}}{\pgfqpoint{6.116495in}{4.842798in}}{\pgfqpoint{6.127545in}{4.842798in}}%
\pgfpathclose%
\pgfusepath{stroke,fill}%
\end{pgfscope}%
\begin{pgfscope}%
\pgfpathrectangle{\pgfqpoint{0.481978in}{0.331635in}}{\pgfqpoint{9.300000in}{7.700000in}}%
\pgfusepath{clip}%
\pgfsetbuttcap%
\pgfsetroundjoin%
\definecolor{currentfill}{rgb}{0.631373,0.788235,0.956863}%
\pgfsetfillcolor{currentfill}%
\pgfsetlinewidth{0.481800pt}%
\definecolor{currentstroke}{rgb}{1.000000,1.000000,1.000000}%
\pgfsetstrokecolor{currentstroke}%
\pgfsetdash{}{0pt}%
\pgfpathmoveto{\pgfqpoint{7.971575in}{5.247803in}}%
\pgfpathcurveto{\pgfqpoint{7.982626in}{5.247803in}}{\pgfqpoint{7.993225in}{5.252193in}}{\pgfqpoint{8.001038in}{5.260007in}}%
\pgfpathcurveto{\pgfqpoint{8.008852in}{5.267821in}}{\pgfqpoint{8.013242in}{5.278420in}}{\pgfqpoint{8.013242in}{5.289470in}}%
\pgfpathcurveto{\pgfqpoint{8.013242in}{5.300520in}}{\pgfqpoint{8.008852in}{5.311119in}}{\pgfqpoint{8.001038in}{5.318933in}}%
\pgfpathcurveto{\pgfqpoint{7.993225in}{5.326746in}}{\pgfqpoint{7.982626in}{5.331136in}}{\pgfqpoint{7.971575in}{5.331136in}}%
\pgfpathcurveto{\pgfqpoint{7.960525in}{5.331136in}}{\pgfqpoint{7.949926in}{5.326746in}}{\pgfqpoint{7.942113in}{5.318933in}}%
\pgfpathcurveto{\pgfqpoint{7.934299in}{5.311119in}}{\pgfqpoint{7.929909in}{5.300520in}}{\pgfqpoint{7.929909in}{5.289470in}}%
\pgfpathcurveto{\pgfqpoint{7.929909in}{5.278420in}}{\pgfqpoint{7.934299in}{5.267821in}}{\pgfqpoint{7.942113in}{5.260007in}}%
\pgfpathcurveto{\pgfqpoint{7.949926in}{5.252193in}}{\pgfqpoint{7.960525in}{5.247803in}}{\pgfqpoint{7.971575in}{5.247803in}}%
\pgfpathclose%
\pgfusepath{stroke,fill}%
\end{pgfscope}%
\begin{pgfscope}%
\pgfpathrectangle{\pgfqpoint{0.481978in}{0.331635in}}{\pgfqpoint{9.300000in}{7.700000in}}%
\pgfusepath{clip}%
\pgfsetbuttcap%
\pgfsetroundjoin%
\definecolor{currentfill}{rgb}{0.631373,0.788235,0.956863}%
\pgfsetfillcolor{currentfill}%
\pgfsetlinewidth{0.481800pt}%
\definecolor{currentstroke}{rgb}{1.000000,1.000000,1.000000}%
\pgfsetstrokecolor{currentstroke}%
\pgfsetdash{}{0pt}%
\pgfpathmoveto{\pgfqpoint{5.508663in}{2.377917in}}%
\pgfpathcurveto{\pgfqpoint{5.519713in}{2.377917in}}{\pgfqpoint{5.530312in}{2.382307in}}{\pgfqpoint{5.538125in}{2.390121in}}%
\pgfpathcurveto{\pgfqpoint{5.545939in}{2.397934in}}{\pgfqpoint{5.550329in}{2.408533in}}{\pgfqpoint{5.550329in}{2.419584in}}%
\pgfpathcurveto{\pgfqpoint{5.550329in}{2.430634in}}{\pgfqpoint{5.545939in}{2.441233in}}{\pgfqpoint{5.538125in}{2.449046in}}%
\pgfpathcurveto{\pgfqpoint{5.530312in}{2.456860in}}{\pgfqpoint{5.519713in}{2.461250in}}{\pgfqpoint{5.508663in}{2.461250in}}%
\pgfpathcurveto{\pgfqpoint{5.497612in}{2.461250in}}{\pgfqpoint{5.487013in}{2.456860in}}{\pgfqpoint{5.479200in}{2.449046in}}%
\pgfpathcurveto{\pgfqpoint{5.471386in}{2.441233in}}{\pgfqpoint{5.466996in}{2.430634in}}{\pgfqpoint{5.466996in}{2.419584in}}%
\pgfpathcurveto{\pgfqpoint{5.466996in}{2.408533in}}{\pgfqpoint{5.471386in}{2.397934in}}{\pgfqpoint{5.479200in}{2.390121in}}%
\pgfpathcurveto{\pgfqpoint{5.487013in}{2.382307in}}{\pgfqpoint{5.497612in}{2.377917in}}{\pgfqpoint{5.508663in}{2.377917in}}%
\pgfpathclose%
\pgfusepath{stroke,fill}%
\end{pgfscope}%
\begin{pgfscope}%
\pgfpathrectangle{\pgfqpoint{0.481978in}{0.331635in}}{\pgfqpoint{9.300000in}{7.700000in}}%
\pgfusepath{clip}%
\pgfsetbuttcap%
\pgfsetroundjoin%
\definecolor{currentfill}{rgb}{0.631373,0.788235,0.956863}%
\pgfsetfillcolor{currentfill}%
\pgfsetlinewidth{0.481800pt}%
\definecolor{currentstroke}{rgb}{1.000000,1.000000,1.000000}%
\pgfsetstrokecolor{currentstroke}%
\pgfsetdash{}{0pt}%
\pgfpathmoveto{\pgfqpoint{4.730932in}{7.139660in}}%
\pgfpathcurveto{\pgfqpoint{4.741982in}{7.139660in}}{\pgfqpoint{4.752581in}{7.144050in}}{\pgfqpoint{4.760395in}{7.151864in}}%
\pgfpathcurveto{\pgfqpoint{4.768209in}{7.159677in}}{\pgfqpoint{4.772599in}{7.170276in}}{\pgfqpoint{4.772599in}{7.181326in}}%
\pgfpathcurveto{\pgfqpoint{4.772599in}{7.192377in}}{\pgfqpoint{4.768209in}{7.202976in}}{\pgfqpoint{4.760395in}{7.210789in}}%
\pgfpathcurveto{\pgfqpoint{4.752581in}{7.218603in}}{\pgfqpoint{4.741982in}{7.222993in}}{\pgfqpoint{4.730932in}{7.222993in}}%
\pgfpathcurveto{\pgfqpoint{4.719882in}{7.222993in}}{\pgfqpoint{4.709283in}{7.218603in}}{\pgfqpoint{4.701469in}{7.210789in}}%
\pgfpathcurveto{\pgfqpoint{4.693656in}{7.202976in}}{\pgfqpoint{4.689266in}{7.192377in}}{\pgfqpoint{4.689266in}{7.181326in}}%
\pgfpathcurveto{\pgfqpoint{4.689266in}{7.170276in}}{\pgfqpoint{4.693656in}{7.159677in}}{\pgfqpoint{4.701469in}{7.151864in}}%
\pgfpathcurveto{\pgfqpoint{4.709283in}{7.144050in}}{\pgfqpoint{4.719882in}{7.139660in}}{\pgfqpoint{4.730932in}{7.139660in}}%
\pgfpathclose%
\pgfusepath{stroke,fill}%
\end{pgfscope}%
\begin{pgfscope}%
\pgfpathrectangle{\pgfqpoint{0.481978in}{0.331635in}}{\pgfqpoint{9.300000in}{7.700000in}}%
\pgfusepath{clip}%
\pgfsetbuttcap%
\pgfsetroundjoin%
\definecolor{currentfill}{rgb}{0.631373,0.788235,0.956863}%
\pgfsetfillcolor{currentfill}%
\pgfsetlinewidth{0.481800pt}%
\definecolor{currentstroke}{rgb}{1.000000,1.000000,1.000000}%
\pgfsetstrokecolor{currentstroke}%
\pgfsetdash{}{0pt}%
\pgfpathmoveto{\pgfqpoint{5.337518in}{5.909868in}}%
\pgfpathcurveto{\pgfqpoint{5.348568in}{5.909868in}}{\pgfqpoint{5.359167in}{5.914259in}}{\pgfqpoint{5.366981in}{5.922072in}}%
\pgfpathcurveto{\pgfqpoint{5.374795in}{5.929886in}}{\pgfqpoint{5.379185in}{5.940485in}}{\pgfqpoint{5.379185in}{5.951535in}}%
\pgfpathcurveto{\pgfqpoint{5.379185in}{5.962585in}}{\pgfqpoint{5.374795in}{5.973184in}}{\pgfqpoint{5.366981in}{5.980998in}}%
\pgfpathcurveto{\pgfqpoint{5.359167in}{5.988811in}}{\pgfqpoint{5.348568in}{5.993202in}}{\pgfqpoint{5.337518in}{5.993202in}}%
\pgfpathcurveto{\pgfqpoint{5.326468in}{5.993202in}}{\pgfqpoint{5.315869in}{5.988811in}}{\pgfqpoint{5.308055in}{5.980998in}}%
\pgfpathcurveto{\pgfqpoint{5.300242in}{5.973184in}}{\pgfqpoint{5.295851in}{5.962585in}}{\pgfqpoint{5.295851in}{5.951535in}}%
\pgfpathcurveto{\pgfqpoint{5.295851in}{5.940485in}}{\pgfqpoint{5.300242in}{5.929886in}}{\pgfqpoint{5.308055in}{5.922072in}}%
\pgfpathcurveto{\pgfqpoint{5.315869in}{5.914259in}}{\pgfqpoint{5.326468in}{5.909868in}}{\pgfqpoint{5.337518in}{5.909868in}}%
\pgfpathclose%
\pgfusepath{stroke,fill}%
\end{pgfscope}%
\begin{pgfscope}%
\pgfpathrectangle{\pgfqpoint{0.481978in}{0.331635in}}{\pgfqpoint{9.300000in}{7.700000in}}%
\pgfusepath{clip}%
\pgfsetbuttcap%
\pgfsetroundjoin%
\definecolor{currentfill}{rgb}{0.631373,0.788235,0.956863}%
\pgfsetfillcolor{currentfill}%
\pgfsetlinewidth{0.481800pt}%
\definecolor{currentstroke}{rgb}{1.000000,1.000000,1.000000}%
\pgfsetstrokecolor{currentstroke}%
\pgfsetdash{}{0pt}%
\pgfpathmoveto{\pgfqpoint{5.165140in}{3.088557in}}%
\pgfpathcurveto{\pgfqpoint{5.176190in}{3.088557in}}{\pgfqpoint{5.186789in}{3.092947in}}{\pgfqpoint{5.194603in}{3.100761in}}%
\pgfpathcurveto{\pgfqpoint{5.202416in}{3.108574in}}{\pgfqpoint{5.206807in}{3.119173in}}{\pgfqpoint{5.206807in}{3.130223in}}%
\pgfpathcurveto{\pgfqpoint{5.206807in}{3.141273in}}{\pgfqpoint{5.202416in}{3.151872in}}{\pgfqpoint{5.194603in}{3.159686in}}%
\pgfpathcurveto{\pgfqpoint{5.186789in}{3.167500in}}{\pgfqpoint{5.176190in}{3.171890in}}{\pgfqpoint{5.165140in}{3.171890in}}%
\pgfpathcurveto{\pgfqpoint{5.154090in}{3.171890in}}{\pgfqpoint{5.143491in}{3.167500in}}{\pgfqpoint{5.135677in}{3.159686in}}%
\pgfpathcurveto{\pgfqpoint{5.127863in}{3.151872in}}{\pgfqpoint{5.123473in}{3.141273in}}{\pgfqpoint{5.123473in}{3.130223in}}%
\pgfpathcurveto{\pgfqpoint{5.123473in}{3.119173in}}{\pgfqpoint{5.127863in}{3.108574in}}{\pgfqpoint{5.135677in}{3.100761in}}%
\pgfpathcurveto{\pgfqpoint{5.143491in}{3.092947in}}{\pgfqpoint{5.154090in}{3.088557in}}{\pgfqpoint{5.165140in}{3.088557in}}%
\pgfpathclose%
\pgfusepath{stroke,fill}%
\end{pgfscope}%
\begin{pgfscope}%
\pgfpathrectangle{\pgfqpoint{0.481978in}{0.331635in}}{\pgfqpoint{9.300000in}{7.700000in}}%
\pgfusepath{clip}%
\pgfsetbuttcap%
\pgfsetroundjoin%
\definecolor{currentfill}{rgb}{0.631373,0.788235,0.956863}%
\pgfsetfillcolor{currentfill}%
\pgfsetlinewidth{0.481800pt}%
\definecolor{currentstroke}{rgb}{1.000000,1.000000,1.000000}%
\pgfsetstrokecolor{currentstroke}%
\pgfsetdash{}{0pt}%
\pgfpathmoveto{\pgfqpoint{6.635187in}{2.582474in}}%
\pgfpathcurveto{\pgfqpoint{6.646237in}{2.582474in}}{\pgfqpoint{6.656836in}{2.586864in}}{\pgfqpoint{6.664649in}{2.594678in}}%
\pgfpathcurveto{\pgfqpoint{6.672463in}{2.602492in}}{\pgfqpoint{6.676853in}{2.613091in}}{\pgfqpoint{6.676853in}{2.624141in}}%
\pgfpathcurveto{\pgfqpoint{6.676853in}{2.635191in}}{\pgfqpoint{6.672463in}{2.645790in}}{\pgfqpoint{6.664649in}{2.653603in}}%
\pgfpathcurveto{\pgfqpoint{6.656836in}{2.661417in}}{\pgfqpoint{6.646237in}{2.665807in}}{\pgfqpoint{6.635187in}{2.665807in}}%
\pgfpathcurveto{\pgfqpoint{6.624136in}{2.665807in}}{\pgfqpoint{6.613537in}{2.661417in}}{\pgfqpoint{6.605724in}{2.653603in}}%
\pgfpathcurveto{\pgfqpoint{6.597910in}{2.645790in}}{\pgfqpoint{6.593520in}{2.635191in}}{\pgfqpoint{6.593520in}{2.624141in}}%
\pgfpathcurveto{\pgfqpoint{6.593520in}{2.613091in}}{\pgfqpoint{6.597910in}{2.602492in}}{\pgfqpoint{6.605724in}{2.594678in}}%
\pgfpathcurveto{\pgfqpoint{6.613537in}{2.586864in}}{\pgfqpoint{6.624136in}{2.582474in}}{\pgfqpoint{6.635187in}{2.582474in}}%
\pgfpathclose%
\pgfusepath{stroke,fill}%
\end{pgfscope}%
\begin{pgfscope}%
\pgfpathrectangle{\pgfqpoint{0.481978in}{0.331635in}}{\pgfqpoint{9.300000in}{7.700000in}}%
\pgfusepath{clip}%
\pgfsetbuttcap%
\pgfsetroundjoin%
\definecolor{currentfill}{rgb}{0.631373,0.788235,0.956863}%
\pgfsetfillcolor{currentfill}%
\pgfsetlinewidth{0.481800pt}%
\definecolor{currentstroke}{rgb}{1.000000,1.000000,1.000000}%
\pgfsetstrokecolor{currentstroke}%
\pgfsetdash{}{0pt}%
\pgfpathmoveto{\pgfqpoint{2.269929in}{6.433789in}}%
\pgfpathcurveto{\pgfqpoint{2.280979in}{6.433789in}}{\pgfqpoint{2.291578in}{6.438180in}}{\pgfqpoint{2.299392in}{6.445993in}}%
\pgfpathcurveto{\pgfqpoint{2.307206in}{6.453807in}}{\pgfqpoint{2.311596in}{6.464406in}}{\pgfqpoint{2.311596in}{6.475456in}}%
\pgfpathcurveto{\pgfqpoint{2.311596in}{6.486506in}}{\pgfqpoint{2.307206in}{6.497105in}}{\pgfqpoint{2.299392in}{6.504919in}}%
\pgfpathcurveto{\pgfqpoint{2.291578in}{6.512732in}}{\pgfqpoint{2.280979in}{6.517123in}}{\pgfqpoint{2.269929in}{6.517123in}}%
\pgfpathcurveto{\pgfqpoint{2.258879in}{6.517123in}}{\pgfqpoint{2.248280in}{6.512732in}}{\pgfqpoint{2.240467in}{6.504919in}}%
\pgfpathcurveto{\pgfqpoint{2.232653in}{6.497105in}}{\pgfqpoint{2.228263in}{6.486506in}}{\pgfqpoint{2.228263in}{6.475456in}}%
\pgfpathcurveto{\pgfqpoint{2.228263in}{6.464406in}}{\pgfqpoint{2.232653in}{6.453807in}}{\pgfqpoint{2.240467in}{6.445993in}}%
\pgfpathcurveto{\pgfqpoint{2.248280in}{6.438180in}}{\pgfqpoint{2.258879in}{6.433789in}}{\pgfqpoint{2.269929in}{6.433789in}}%
\pgfpathclose%
\pgfusepath{stroke,fill}%
\end{pgfscope}%
\begin{pgfscope}%
\pgfpathrectangle{\pgfqpoint{0.481978in}{0.331635in}}{\pgfqpoint{9.300000in}{7.700000in}}%
\pgfusepath{clip}%
\pgfsetbuttcap%
\pgfsetroundjoin%
\definecolor{currentfill}{rgb}{0.631373,0.788235,0.956863}%
\pgfsetfillcolor{currentfill}%
\pgfsetlinewidth{0.481800pt}%
\definecolor{currentstroke}{rgb}{1.000000,1.000000,1.000000}%
\pgfsetstrokecolor{currentstroke}%
\pgfsetdash{}{0pt}%
\pgfpathmoveto{\pgfqpoint{6.041629in}{4.109351in}}%
\pgfpathcurveto{\pgfqpoint{6.052679in}{4.109351in}}{\pgfqpoint{6.063278in}{4.113742in}}{\pgfqpoint{6.071092in}{4.121555in}}%
\pgfpathcurveto{\pgfqpoint{6.078905in}{4.129369in}}{\pgfqpoint{6.083295in}{4.139968in}}{\pgfqpoint{6.083295in}{4.151018in}}%
\pgfpathcurveto{\pgfqpoint{6.083295in}{4.162068in}}{\pgfqpoint{6.078905in}{4.172667in}}{\pgfqpoint{6.071092in}{4.180481in}}%
\pgfpathcurveto{\pgfqpoint{6.063278in}{4.188294in}}{\pgfqpoint{6.052679in}{4.192685in}}{\pgfqpoint{6.041629in}{4.192685in}}%
\pgfpathcurveto{\pgfqpoint{6.030579in}{4.192685in}}{\pgfqpoint{6.019980in}{4.188294in}}{\pgfqpoint{6.012166in}{4.180481in}}%
\pgfpathcurveto{\pgfqpoint{6.004352in}{4.172667in}}{\pgfqpoint{5.999962in}{4.162068in}}{\pgfqpoint{5.999962in}{4.151018in}}%
\pgfpathcurveto{\pgfqpoint{5.999962in}{4.139968in}}{\pgfqpoint{6.004352in}{4.129369in}}{\pgfqpoint{6.012166in}{4.121555in}}%
\pgfpathcurveto{\pgfqpoint{6.019980in}{4.113742in}}{\pgfqpoint{6.030579in}{4.109351in}}{\pgfqpoint{6.041629in}{4.109351in}}%
\pgfpathclose%
\pgfusepath{stroke,fill}%
\end{pgfscope}%
\begin{pgfscope}%
\pgfpathrectangle{\pgfqpoint{0.481978in}{0.331635in}}{\pgfqpoint{9.300000in}{7.700000in}}%
\pgfusepath{clip}%
\pgfsetbuttcap%
\pgfsetroundjoin%
\definecolor{currentfill}{rgb}{0.631373,0.788235,0.956863}%
\pgfsetfillcolor{currentfill}%
\pgfsetlinewidth{0.481800pt}%
\definecolor{currentstroke}{rgb}{1.000000,1.000000,1.000000}%
\pgfsetstrokecolor{currentstroke}%
\pgfsetdash{}{0pt}%
\pgfpathmoveto{\pgfqpoint{7.965678in}{3.983142in}}%
\pgfpathcurveto{\pgfqpoint{7.976728in}{3.983142in}}{\pgfqpoint{7.987327in}{3.987532in}}{\pgfqpoint{7.995140in}{3.995346in}}%
\pgfpathcurveto{\pgfqpoint{8.002954in}{4.003160in}}{\pgfqpoint{8.007344in}{4.013759in}}{\pgfqpoint{8.007344in}{4.024809in}}%
\pgfpathcurveto{\pgfqpoint{8.007344in}{4.035859in}}{\pgfqpoint{8.002954in}{4.046458in}}{\pgfqpoint{7.995140in}{4.054272in}}%
\pgfpathcurveto{\pgfqpoint{7.987327in}{4.062085in}}{\pgfqpoint{7.976728in}{4.066476in}}{\pgfqpoint{7.965678in}{4.066476in}}%
\pgfpathcurveto{\pgfqpoint{7.954627in}{4.066476in}}{\pgfqpoint{7.944028in}{4.062085in}}{\pgfqpoint{7.936215in}{4.054272in}}%
\pgfpathcurveto{\pgfqpoint{7.928401in}{4.046458in}}{\pgfqpoint{7.924011in}{4.035859in}}{\pgfqpoint{7.924011in}{4.024809in}}%
\pgfpathcurveto{\pgfqpoint{7.924011in}{4.013759in}}{\pgfqpoint{7.928401in}{4.003160in}}{\pgfqpoint{7.936215in}{3.995346in}}%
\pgfpathcurveto{\pgfqpoint{7.944028in}{3.987532in}}{\pgfqpoint{7.954627in}{3.983142in}}{\pgfqpoint{7.965678in}{3.983142in}}%
\pgfpathclose%
\pgfusepath{stroke,fill}%
\end{pgfscope}%
\begin{pgfscope}%
\pgfpathrectangle{\pgfqpoint{0.481978in}{0.331635in}}{\pgfqpoint{9.300000in}{7.700000in}}%
\pgfusepath{clip}%
\pgfsetbuttcap%
\pgfsetroundjoin%
\definecolor{currentfill}{rgb}{0.631373,0.788235,0.956863}%
\pgfsetfillcolor{currentfill}%
\pgfsetlinewidth{0.481800pt}%
\definecolor{currentstroke}{rgb}{1.000000,1.000000,1.000000}%
\pgfsetstrokecolor{currentstroke}%
\pgfsetdash{}{0pt}%
\pgfpathmoveto{\pgfqpoint{6.739115in}{1.447662in}}%
\pgfpathcurveto{\pgfqpoint{6.750166in}{1.447662in}}{\pgfqpoint{6.760765in}{1.452052in}}{\pgfqpoint{6.768578in}{1.459866in}}%
\pgfpathcurveto{\pgfqpoint{6.776392in}{1.467680in}}{\pgfqpoint{6.780782in}{1.478279in}}{\pgfqpoint{6.780782in}{1.489329in}}%
\pgfpathcurveto{\pgfqpoint{6.780782in}{1.500379in}}{\pgfqpoint{6.776392in}{1.510978in}}{\pgfqpoint{6.768578in}{1.518792in}}%
\pgfpathcurveto{\pgfqpoint{6.760765in}{1.526605in}}{\pgfqpoint{6.750166in}{1.530995in}}{\pgfqpoint{6.739115in}{1.530995in}}%
\pgfpathcurveto{\pgfqpoint{6.728065in}{1.530995in}}{\pgfqpoint{6.717466in}{1.526605in}}{\pgfqpoint{6.709653in}{1.518792in}}%
\pgfpathcurveto{\pgfqpoint{6.701839in}{1.510978in}}{\pgfqpoint{6.697449in}{1.500379in}}{\pgfqpoint{6.697449in}{1.489329in}}%
\pgfpathcurveto{\pgfqpoint{6.697449in}{1.478279in}}{\pgfqpoint{6.701839in}{1.467680in}}{\pgfqpoint{6.709653in}{1.459866in}}%
\pgfpathcurveto{\pgfqpoint{6.717466in}{1.452052in}}{\pgfqpoint{6.728065in}{1.447662in}}{\pgfqpoint{6.739115in}{1.447662in}}%
\pgfpathclose%
\pgfusepath{stroke,fill}%
\end{pgfscope}%
\begin{pgfscope}%
\pgfpathrectangle{\pgfqpoint{0.481978in}{0.331635in}}{\pgfqpoint{9.300000in}{7.700000in}}%
\pgfusepath{clip}%
\pgfsetbuttcap%
\pgfsetroundjoin%
\definecolor{currentfill}{rgb}{0.631373,0.788235,0.956863}%
\pgfsetfillcolor{currentfill}%
\pgfsetlinewidth{0.481800pt}%
\definecolor{currentstroke}{rgb}{1.000000,1.000000,1.000000}%
\pgfsetstrokecolor{currentstroke}%
\pgfsetdash{}{0pt}%
\pgfpathmoveto{\pgfqpoint{5.908920in}{1.523605in}}%
\pgfpathcurveto{\pgfqpoint{5.919970in}{1.523605in}}{\pgfqpoint{5.930569in}{1.527995in}}{\pgfqpoint{5.938383in}{1.535809in}}%
\pgfpathcurveto{\pgfqpoint{5.946197in}{1.543622in}}{\pgfqpoint{5.950587in}{1.554221in}}{\pgfqpoint{5.950587in}{1.565271in}}%
\pgfpathcurveto{\pgfqpoint{5.950587in}{1.576322in}}{\pgfqpoint{5.946197in}{1.586921in}}{\pgfqpoint{5.938383in}{1.594734in}}%
\pgfpathcurveto{\pgfqpoint{5.930569in}{1.602548in}}{\pgfqpoint{5.919970in}{1.606938in}}{\pgfqpoint{5.908920in}{1.606938in}}%
\pgfpathcurveto{\pgfqpoint{5.897870in}{1.606938in}}{\pgfqpoint{5.887271in}{1.602548in}}{\pgfqpoint{5.879457in}{1.594734in}}%
\pgfpathcurveto{\pgfqpoint{5.871644in}{1.586921in}}{\pgfqpoint{5.867254in}{1.576322in}}{\pgfqpoint{5.867254in}{1.565271in}}%
\pgfpathcurveto{\pgfqpoint{5.867254in}{1.554221in}}{\pgfqpoint{5.871644in}{1.543622in}}{\pgfqpoint{5.879457in}{1.535809in}}%
\pgfpathcurveto{\pgfqpoint{5.887271in}{1.527995in}}{\pgfqpoint{5.897870in}{1.523605in}}{\pgfqpoint{5.908920in}{1.523605in}}%
\pgfpathclose%
\pgfusepath{stroke,fill}%
\end{pgfscope}%
\begin{pgfscope}%
\pgfpathrectangle{\pgfqpoint{0.481978in}{0.331635in}}{\pgfqpoint{9.300000in}{7.700000in}}%
\pgfusepath{clip}%
\pgfsetbuttcap%
\pgfsetroundjoin%
\definecolor{currentfill}{rgb}{0.631373,0.788235,0.956863}%
\pgfsetfillcolor{currentfill}%
\pgfsetlinewidth{0.481800pt}%
\definecolor{currentstroke}{rgb}{1.000000,1.000000,1.000000}%
\pgfsetstrokecolor{currentstroke}%
\pgfsetdash{}{0pt}%
\pgfpathmoveto{\pgfqpoint{4.782057in}{1.286961in}}%
\pgfpathcurveto{\pgfqpoint{4.793108in}{1.286961in}}{\pgfqpoint{4.803707in}{1.291352in}}{\pgfqpoint{4.811520in}{1.299165in}}%
\pgfpathcurveto{\pgfqpoint{4.819334in}{1.306979in}}{\pgfqpoint{4.823724in}{1.317578in}}{\pgfqpoint{4.823724in}{1.328628in}}%
\pgfpathcurveto{\pgfqpoint{4.823724in}{1.339678in}}{\pgfqpoint{4.819334in}{1.350277in}}{\pgfqpoint{4.811520in}{1.358091in}}%
\pgfpathcurveto{\pgfqpoint{4.803707in}{1.365904in}}{\pgfqpoint{4.793108in}{1.370295in}}{\pgfqpoint{4.782057in}{1.370295in}}%
\pgfpathcurveto{\pgfqpoint{4.771007in}{1.370295in}}{\pgfqpoint{4.760408in}{1.365904in}}{\pgfqpoint{4.752595in}{1.358091in}}%
\pgfpathcurveto{\pgfqpoint{4.744781in}{1.350277in}}{\pgfqpoint{4.740391in}{1.339678in}}{\pgfqpoint{4.740391in}{1.328628in}}%
\pgfpathcurveto{\pgfqpoint{4.740391in}{1.317578in}}{\pgfqpoint{4.744781in}{1.306979in}}{\pgfqpoint{4.752595in}{1.299165in}}%
\pgfpathcurveto{\pgfqpoint{4.760408in}{1.291352in}}{\pgfqpoint{4.771007in}{1.286961in}}{\pgfqpoint{4.782057in}{1.286961in}}%
\pgfpathclose%
\pgfusepath{stroke,fill}%
\end{pgfscope}%
\begin{pgfscope}%
\pgfpathrectangle{\pgfqpoint{0.481978in}{0.331635in}}{\pgfqpoint{9.300000in}{7.700000in}}%
\pgfusepath{clip}%
\pgfsetbuttcap%
\pgfsetroundjoin%
\definecolor{currentfill}{rgb}{0.631373,0.788235,0.956863}%
\pgfsetfillcolor{currentfill}%
\pgfsetlinewidth{0.481800pt}%
\definecolor{currentstroke}{rgb}{1.000000,1.000000,1.000000}%
\pgfsetstrokecolor{currentstroke}%
\pgfsetdash{}{0pt}%
\pgfpathmoveto{\pgfqpoint{2.762676in}{1.397155in}}%
\pgfpathcurveto{\pgfqpoint{2.773726in}{1.397155in}}{\pgfqpoint{2.784325in}{1.401546in}}{\pgfqpoint{2.792138in}{1.409359in}}%
\pgfpathcurveto{\pgfqpoint{2.799952in}{1.417173in}}{\pgfqpoint{2.804342in}{1.427772in}}{\pgfqpoint{2.804342in}{1.438822in}}%
\pgfpathcurveto{\pgfqpoint{2.804342in}{1.449872in}}{\pgfqpoint{2.799952in}{1.460471in}}{\pgfqpoint{2.792138in}{1.468285in}}%
\pgfpathcurveto{\pgfqpoint{2.784325in}{1.476099in}}{\pgfqpoint{2.773726in}{1.480489in}}{\pgfqpoint{2.762676in}{1.480489in}}%
\pgfpathcurveto{\pgfqpoint{2.751625in}{1.480489in}}{\pgfqpoint{2.741026in}{1.476099in}}{\pgfqpoint{2.733213in}{1.468285in}}%
\pgfpathcurveto{\pgfqpoint{2.725399in}{1.460471in}}{\pgfqpoint{2.721009in}{1.449872in}}{\pgfqpoint{2.721009in}{1.438822in}}%
\pgfpathcurveto{\pgfqpoint{2.721009in}{1.427772in}}{\pgfqpoint{2.725399in}{1.417173in}}{\pgfqpoint{2.733213in}{1.409359in}}%
\pgfpathcurveto{\pgfqpoint{2.741026in}{1.401546in}}{\pgfqpoint{2.751625in}{1.397155in}}{\pgfqpoint{2.762676in}{1.397155in}}%
\pgfpathclose%
\pgfusepath{stroke,fill}%
\end{pgfscope}%
\begin{pgfscope}%
\pgfpathrectangle{\pgfqpoint{0.481978in}{0.331635in}}{\pgfqpoint{9.300000in}{7.700000in}}%
\pgfusepath{clip}%
\pgfsetbuttcap%
\pgfsetroundjoin%
\definecolor{currentfill}{rgb}{0.631373,0.788235,0.956863}%
\pgfsetfillcolor{currentfill}%
\pgfsetlinewidth{0.481800pt}%
\definecolor{currentstroke}{rgb}{1.000000,1.000000,1.000000}%
\pgfsetstrokecolor{currentstroke}%
\pgfsetdash{}{0pt}%
\pgfpathmoveto{\pgfqpoint{6.674444in}{1.834291in}}%
\pgfpathcurveto{\pgfqpoint{6.685494in}{1.834291in}}{\pgfqpoint{6.696093in}{1.838681in}}{\pgfqpoint{6.703907in}{1.846495in}}%
\pgfpathcurveto{\pgfqpoint{6.711720in}{1.854308in}}{\pgfqpoint{6.716111in}{1.864908in}}{\pgfqpoint{6.716111in}{1.875958in}}%
\pgfpathcurveto{\pgfqpoint{6.716111in}{1.887008in}}{\pgfqpoint{6.711720in}{1.897607in}}{\pgfqpoint{6.703907in}{1.905420in}}%
\pgfpathcurveto{\pgfqpoint{6.696093in}{1.913234in}}{\pgfqpoint{6.685494in}{1.917624in}}{\pgfqpoint{6.674444in}{1.917624in}}%
\pgfpathcurveto{\pgfqpoint{6.663394in}{1.917624in}}{\pgfqpoint{6.652795in}{1.913234in}}{\pgfqpoint{6.644981in}{1.905420in}}%
\pgfpathcurveto{\pgfqpoint{6.637167in}{1.897607in}}{\pgfqpoint{6.632777in}{1.887008in}}{\pgfqpoint{6.632777in}{1.875958in}}%
\pgfpathcurveto{\pgfqpoint{6.632777in}{1.864908in}}{\pgfqpoint{6.637167in}{1.854308in}}{\pgfqpoint{6.644981in}{1.846495in}}%
\pgfpathcurveto{\pgfqpoint{6.652795in}{1.838681in}}{\pgfqpoint{6.663394in}{1.834291in}}{\pgfqpoint{6.674444in}{1.834291in}}%
\pgfpathclose%
\pgfusepath{stroke,fill}%
\end{pgfscope}%
\begin{pgfscope}%
\pgfpathrectangle{\pgfqpoint{0.481978in}{0.331635in}}{\pgfqpoint{9.300000in}{7.700000in}}%
\pgfusepath{clip}%
\pgfsetbuttcap%
\pgfsetroundjoin%
\definecolor{currentfill}{rgb}{0.631373,0.788235,0.956863}%
\pgfsetfillcolor{currentfill}%
\pgfsetlinewidth{0.481800pt}%
\definecolor{currentstroke}{rgb}{1.000000,1.000000,1.000000}%
\pgfsetstrokecolor{currentstroke}%
\pgfsetdash{}{0pt}%
\pgfpathmoveto{\pgfqpoint{7.190919in}{2.075647in}}%
\pgfpathcurveto{\pgfqpoint{7.201969in}{2.075647in}}{\pgfqpoint{7.212568in}{2.080038in}}{\pgfqpoint{7.220381in}{2.087851in}}%
\pgfpathcurveto{\pgfqpoint{7.228195in}{2.095665in}}{\pgfqpoint{7.232585in}{2.106264in}}{\pgfqpoint{7.232585in}{2.117314in}}%
\pgfpathcurveto{\pgfqpoint{7.232585in}{2.128364in}}{\pgfqpoint{7.228195in}{2.138963in}}{\pgfqpoint{7.220381in}{2.146777in}}%
\pgfpathcurveto{\pgfqpoint{7.212568in}{2.154591in}}{\pgfqpoint{7.201969in}{2.158981in}}{\pgfqpoint{7.190919in}{2.158981in}}%
\pgfpathcurveto{\pgfqpoint{7.179868in}{2.158981in}}{\pgfqpoint{7.169269in}{2.154591in}}{\pgfqpoint{7.161456in}{2.146777in}}%
\pgfpathcurveto{\pgfqpoint{7.153642in}{2.138963in}}{\pgfqpoint{7.149252in}{2.128364in}}{\pgfqpoint{7.149252in}{2.117314in}}%
\pgfpathcurveto{\pgfqpoint{7.149252in}{2.106264in}}{\pgfqpoint{7.153642in}{2.095665in}}{\pgfqpoint{7.161456in}{2.087851in}}%
\pgfpathcurveto{\pgfqpoint{7.169269in}{2.080038in}}{\pgfqpoint{7.179868in}{2.075647in}}{\pgfqpoint{7.190919in}{2.075647in}}%
\pgfpathclose%
\pgfusepath{stroke,fill}%
\end{pgfscope}%
\begin{pgfscope}%
\pgfpathrectangle{\pgfqpoint{0.481978in}{0.331635in}}{\pgfqpoint{9.300000in}{7.700000in}}%
\pgfusepath{clip}%
\pgfsetbuttcap%
\pgfsetroundjoin%
\definecolor{currentfill}{rgb}{0.631373,0.788235,0.956863}%
\pgfsetfillcolor{currentfill}%
\pgfsetlinewidth{0.481800pt}%
\definecolor{currentstroke}{rgb}{1.000000,1.000000,1.000000}%
\pgfsetstrokecolor{currentstroke}%
\pgfsetdash{}{0pt}%
\pgfpathmoveto{\pgfqpoint{5.231440in}{6.850407in}}%
\pgfpathcurveto{\pgfqpoint{5.242490in}{6.850407in}}{\pgfqpoint{5.253089in}{6.854797in}}{\pgfqpoint{5.260903in}{6.862611in}}%
\pgfpathcurveto{\pgfqpoint{5.268716in}{6.870424in}}{\pgfqpoint{5.273107in}{6.881023in}}{\pgfqpoint{5.273107in}{6.892073in}}%
\pgfpathcurveto{\pgfqpoint{5.273107in}{6.903123in}}{\pgfqpoint{5.268716in}{6.913722in}}{\pgfqpoint{5.260903in}{6.921536in}}%
\pgfpathcurveto{\pgfqpoint{5.253089in}{6.929350in}}{\pgfqpoint{5.242490in}{6.933740in}}{\pgfqpoint{5.231440in}{6.933740in}}%
\pgfpathcurveto{\pgfqpoint{5.220390in}{6.933740in}}{\pgfqpoint{5.209791in}{6.929350in}}{\pgfqpoint{5.201977in}{6.921536in}}%
\pgfpathcurveto{\pgfqpoint{5.194164in}{6.913722in}}{\pgfqpoint{5.189773in}{6.903123in}}{\pgfqpoint{5.189773in}{6.892073in}}%
\pgfpathcurveto{\pgfqpoint{5.189773in}{6.881023in}}{\pgfqpoint{5.194164in}{6.870424in}}{\pgfqpoint{5.201977in}{6.862611in}}%
\pgfpathcurveto{\pgfqpoint{5.209791in}{6.854797in}}{\pgfqpoint{5.220390in}{6.850407in}}{\pgfqpoint{5.231440in}{6.850407in}}%
\pgfpathclose%
\pgfusepath{stroke,fill}%
\end{pgfscope}%
\begin{pgfscope}%
\pgfpathrectangle{\pgfqpoint{0.481978in}{0.331635in}}{\pgfqpoint{9.300000in}{7.700000in}}%
\pgfusepath{clip}%
\pgfsetbuttcap%
\pgfsetroundjoin%
\definecolor{currentfill}{rgb}{0.631373,0.788235,0.956863}%
\pgfsetfillcolor{currentfill}%
\pgfsetlinewidth{0.481800pt}%
\definecolor{currentstroke}{rgb}{1.000000,1.000000,1.000000}%
\pgfsetstrokecolor{currentstroke}%
\pgfsetdash{}{0pt}%
\pgfpathmoveto{\pgfqpoint{7.641158in}{4.350034in}}%
\pgfpathcurveto{\pgfqpoint{7.652208in}{4.350034in}}{\pgfqpoint{7.662807in}{4.354424in}}{\pgfqpoint{7.670621in}{4.362237in}}%
\pgfpathcurveto{\pgfqpoint{7.678434in}{4.370051in}}{\pgfqpoint{7.682825in}{4.380650in}}{\pgfqpoint{7.682825in}{4.391700in}}%
\pgfpathcurveto{\pgfqpoint{7.682825in}{4.402750in}}{\pgfqpoint{7.678434in}{4.413349in}}{\pgfqpoint{7.670621in}{4.421163in}}%
\pgfpathcurveto{\pgfqpoint{7.662807in}{4.428977in}}{\pgfqpoint{7.652208in}{4.433367in}}{\pgfqpoint{7.641158in}{4.433367in}}%
\pgfpathcurveto{\pgfqpoint{7.630108in}{4.433367in}}{\pgfqpoint{7.619509in}{4.428977in}}{\pgfqpoint{7.611695in}{4.421163in}}%
\pgfpathcurveto{\pgfqpoint{7.603882in}{4.413349in}}{\pgfqpoint{7.599491in}{4.402750in}}{\pgfqpoint{7.599491in}{4.391700in}}%
\pgfpathcurveto{\pgfqpoint{7.599491in}{4.380650in}}{\pgfqpoint{7.603882in}{4.370051in}}{\pgfqpoint{7.611695in}{4.362237in}}%
\pgfpathcurveto{\pgfqpoint{7.619509in}{4.354424in}}{\pgfqpoint{7.630108in}{4.350034in}}{\pgfqpoint{7.641158in}{4.350034in}}%
\pgfpathclose%
\pgfusepath{stroke,fill}%
\end{pgfscope}%
\begin{pgfscope}%
\pgfpathrectangle{\pgfqpoint{0.481978in}{0.331635in}}{\pgfqpoint{9.300000in}{7.700000in}}%
\pgfusepath{clip}%
\pgfsetbuttcap%
\pgfsetroundjoin%
\definecolor{currentfill}{rgb}{0.631373,0.788235,0.956863}%
\pgfsetfillcolor{currentfill}%
\pgfsetlinewidth{0.481800pt}%
\definecolor{currentstroke}{rgb}{1.000000,1.000000,1.000000}%
\pgfsetstrokecolor{currentstroke}%
\pgfsetdash{}{0pt}%
\pgfpathmoveto{\pgfqpoint{5.917655in}{4.335690in}}%
\pgfpathcurveto{\pgfqpoint{5.928705in}{4.335690in}}{\pgfqpoint{5.939305in}{4.340080in}}{\pgfqpoint{5.947118in}{4.347894in}}%
\pgfpathcurveto{\pgfqpoint{5.954932in}{4.355707in}}{\pgfqpoint{5.959322in}{4.366306in}}{\pgfqpoint{5.959322in}{4.377356in}}%
\pgfpathcurveto{\pgfqpoint{5.959322in}{4.388407in}}{\pgfqpoint{5.954932in}{4.399006in}}{\pgfqpoint{5.947118in}{4.406819in}}%
\pgfpathcurveto{\pgfqpoint{5.939305in}{4.414633in}}{\pgfqpoint{5.928705in}{4.419023in}}{\pgfqpoint{5.917655in}{4.419023in}}%
\pgfpathcurveto{\pgfqpoint{5.906605in}{4.419023in}}{\pgfqpoint{5.896006in}{4.414633in}}{\pgfqpoint{5.888193in}{4.406819in}}%
\pgfpathcurveto{\pgfqpoint{5.880379in}{4.399006in}}{\pgfqpoint{5.875989in}{4.388407in}}{\pgfqpoint{5.875989in}{4.377356in}}%
\pgfpathcurveto{\pgfqpoint{5.875989in}{4.366306in}}{\pgfqpoint{5.880379in}{4.355707in}}{\pgfqpoint{5.888193in}{4.347894in}}%
\pgfpathcurveto{\pgfqpoint{5.896006in}{4.340080in}}{\pgfqpoint{5.906605in}{4.335690in}}{\pgfqpoint{5.917655in}{4.335690in}}%
\pgfpathclose%
\pgfusepath{stroke,fill}%
\end{pgfscope}%
\begin{pgfscope}%
\pgfpathrectangle{\pgfqpoint{0.481978in}{0.331635in}}{\pgfqpoint{9.300000in}{7.700000in}}%
\pgfusepath{clip}%
\pgfsetbuttcap%
\pgfsetroundjoin%
\definecolor{currentfill}{rgb}{0.631373,0.788235,0.956863}%
\pgfsetfillcolor{currentfill}%
\pgfsetlinewidth{0.481800pt}%
\definecolor{currentstroke}{rgb}{1.000000,1.000000,1.000000}%
\pgfsetstrokecolor{currentstroke}%
\pgfsetdash{}{0pt}%
\pgfpathmoveto{\pgfqpoint{6.915293in}{3.926889in}}%
\pgfpathcurveto{\pgfqpoint{6.926343in}{3.926889in}}{\pgfqpoint{6.936942in}{3.931279in}}{\pgfqpoint{6.944756in}{3.939093in}}%
\pgfpathcurveto{\pgfqpoint{6.952570in}{3.946906in}}{\pgfqpoint{6.956960in}{3.957505in}}{\pgfqpoint{6.956960in}{3.968555in}}%
\pgfpathcurveto{\pgfqpoint{6.956960in}{3.979605in}}{\pgfqpoint{6.952570in}{3.990204in}}{\pgfqpoint{6.944756in}{3.998018in}}%
\pgfpathcurveto{\pgfqpoint{6.936942in}{4.005832in}}{\pgfqpoint{6.926343in}{4.010222in}}{\pgfqpoint{6.915293in}{4.010222in}}%
\pgfpathcurveto{\pgfqpoint{6.904243in}{4.010222in}}{\pgfqpoint{6.893644in}{4.005832in}}{\pgfqpoint{6.885830in}{3.998018in}}%
\pgfpathcurveto{\pgfqpoint{6.878017in}{3.990204in}}{\pgfqpoint{6.873627in}{3.979605in}}{\pgfqpoint{6.873627in}{3.968555in}}%
\pgfpathcurveto{\pgfqpoint{6.873627in}{3.957505in}}{\pgfqpoint{6.878017in}{3.946906in}}{\pgfqpoint{6.885830in}{3.939093in}}%
\pgfpathcurveto{\pgfqpoint{6.893644in}{3.931279in}}{\pgfqpoint{6.904243in}{3.926889in}}{\pgfqpoint{6.915293in}{3.926889in}}%
\pgfpathclose%
\pgfusepath{stroke,fill}%
\end{pgfscope}%
\begin{pgfscope}%
\pgfpathrectangle{\pgfqpoint{0.481978in}{0.331635in}}{\pgfqpoint{9.300000in}{7.700000in}}%
\pgfusepath{clip}%
\pgfsetbuttcap%
\pgfsetroundjoin%
\definecolor{currentfill}{rgb}{0.631373,0.788235,0.956863}%
\pgfsetfillcolor{currentfill}%
\pgfsetlinewidth{0.481800pt}%
\definecolor{currentstroke}{rgb}{1.000000,1.000000,1.000000}%
\pgfsetstrokecolor{currentstroke}%
\pgfsetdash{}{0pt}%
\pgfpathmoveto{\pgfqpoint{5.971393in}{2.344772in}}%
\pgfpathcurveto{\pgfqpoint{5.982443in}{2.344772in}}{\pgfqpoint{5.993042in}{2.349163in}}{\pgfqpoint{6.000856in}{2.356976in}}%
\pgfpathcurveto{\pgfqpoint{6.008669in}{2.364790in}}{\pgfqpoint{6.013060in}{2.375389in}}{\pgfqpoint{6.013060in}{2.386439in}}%
\pgfpathcurveto{\pgfqpoint{6.013060in}{2.397489in}}{\pgfqpoint{6.008669in}{2.408088in}}{\pgfqpoint{6.000856in}{2.415902in}}%
\pgfpathcurveto{\pgfqpoint{5.993042in}{2.423715in}}{\pgfqpoint{5.982443in}{2.428106in}}{\pgfqpoint{5.971393in}{2.428106in}}%
\pgfpathcurveto{\pgfqpoint{5.960343in}{2.428106in}}{\pgfqpoint{5.949744in}{2.423715in}}{\pgfqpoint{5.941930in}{2.415902in}}%
\pgfpathcurveto{\pgfqpoint{5.934117in}{2.408088in}}{\pgfqpoint{5.929726in}{2.397489in}}{\pgfqpoint{5.929726in}{2.386439in}}%
\pgfpathcurveto{\pgfqpoint{5.929726in}{2.375389in}}{\pgfqpoint{5.934117in}{2.364790in}}{\pgfqpoint{5.941930in}{2.356976in}}%
\pgfpathcurveto{\pgfqpoint{5.949744in}{2.349163in}}{\pgfqpoint{5.960343in}{2.344772in}}{\pgfqpoint{5.971393in}{2.344772in}}%
\pgfpathclose%
\pgfusepath{stroke,fill}%
\end{pgfscope}%
\begin{pgfscope}%
\pgfpathrectangle{\pgfqpoint{0.481978in}{0.331635in}}{\pgfqpoint{9.300000in}{7.700000in}}%
\pgfusepath{clip}%
\pgfsetbuttcap%
\pgfsetroundjoin%
\definecolor{currentfill}{rgb}{0.631373,0.788235,0.956863}%
\pgfsetfillcolor{currentfill}%
\pgfsetlinewidth{0.481800pt}%
\definecolor{currentstroke}{rgb}{1.000000,1.000000,1.000000}%
\pgfsetstrokecolor{currentstroke}%
\pgfsetdash{}{0pt}%
\pgfpathmoveto{\pgfqpoint{2.928552in}{6.577600in}}%
\pgfpathcurveto{\pgfqpoint{2.939602in}{6.577600in}}{\pgfqpoint{2.950201in}{6.581990in}}{\pgfqpoint{2.958014in}{6.589804in}}%
\pgfpathcurveto{\pgfqpoint{2.965828in}{6.597618in}}{\pgfqpoint{2.970218in}{6.608217in}}{\pgfqpoint{2.970218in}{6.619267in}}%
\pgfpathcurveto{\pgfqpoint{2.970218in}{6.630317in}}{\pgfqpoint{2.965828in}{6.640916in}}{\pgfqpoint{2.958014in}{6.648730in}}%
\pgfpathcurveto{\pgfqpoint{2.950201in}{6.656543in}}{\pgfqpoint{2.939602in}{6.660933in}}{\pgfqpoint{2.928552in}{6.660933in}}%
\pgfpathcurveto{\pgfqpoint{2.917501in}{6.660933in}}{\pgfqpoint{2.906902in}{6.656543in}}{\pgfqpoint{2.899089in}{6.648730in}}%
\pgfpathcurveto{\pgfqpoint{2.891275in}{6.640916in}}{\pgfqpoint{2.886885in}{6.630317in}}{\pgfqpoint{2.886885in}{6.619267in}}%
\pgfpathcurveto{\pgfqpoint{2.886885in}{6.608217in}}{\pgfqpoint{2.891275in}{6.597618in}}{\pgfqpoint{2.899089in}{6.589804in}}%
\pgfpathcurveto{\pgfqpoint{2.906902in}{6.581990in}}{\pgfqpoint{2.917501in}{6.577600in}}{\pgfqpoint{2.928552in}{6.577600in}}%
\pgfpathclose%
\pgfusepath{stroke,fill}%
\end{pgfscope}%
\begin{pgfscope}%
\pgfpathrectangle{\pgfqpoint{0.481978in}{0.331635in}}{\pgfqpoint{9.300000in}{7.700000in}}%
\pgfusepath{clip}%
\pgfsetbuttcap%
\pgfsetroundjoin%
\definecolor{currentfill}{rgb}{0.631373,0.788235,0.956863}%
\pgfsetfillcolor{currentfill}%
\pgfsetlinewidth{0.481800pt}%
\definecolor{currentstroke}{rgb}{1.000000,1.000000,1.000000}%
\pgfsetstrokecolor{currentstroke}%
\pgfsetdash{}{0pt}%
\pgfpathmoveto{\pgfqpoint{5.743990in}{2.572575in}}%
\pgfpathcurveto{\pgfqpoint{5.755041in}{2.572575in}}{\pgfqpoint{5.765640in}{2.576965in}}{\pgfqpoint{5.773453in}{2.584779in}}%
\pgfpathcurveto{\pgfqpoint{5.781267in}{2.592593in}}{\pgfqpoint{5.785657in}{2.603192in}}{\pgfqpoint{5.785657in}{2.614242in}}%
\pgfpathcurveto{\pgfqpoint{5.785657in}{2.625292in}}{\pgfqpoint{5.781267in}{2.635891in}}{\pgfqpoint{5.773453in}{2.643705in}}%
\pgfpathcurveto{\pgfqpoint{5.765640in}{2.651518in}}{\pgfqpoint{5.755041in}{2.655909in}}{\pgfqpoint{5.743990in}{2.655909in}}%
\pgfpathcurveto{\pgfqpoint{5.732940in}{2.655909in}}{\pgfqpoint{5.722341in}{2.651518in}}{\pgfqpoint{5.714528in}{2.643705in}}%
\pgfpathcurveto{\pgfqpoint{5.706714in}{2.635891in}}{\pgfqpoint{5.702324in}{2.625292in}}{\pgfqpoint{5.702324in}{2.614242in}}%
\pgfpathcurveto{\pgfqpoint{5.702324in}{2.603192in}}{\pgfqpoint{5.706714in}{2.592593in}}{\pgfqpoint{5.714528in}{2.584779in}}%
\pgfpathcurveto{\pgfqpoint{5.722341in}{2.576965in}}{\pgfqpoint{5.732940in}{2.572575in}}{\pgfqpoint{5.743990in}{2.572575in}}%
\pgfpathclose%
\pgfusepath{stroke,fill}%
\end{pgfscope}%
\begin{pgfscope}%
\pgfpathrectangle{\pgfqpoint{0.481978in}{0.331635in}}{\pgfqpoint{9.300000in}{7.700000in}}%
\pgfusepath{clip}%
\pgfsetbuttcap%
\pgfsetroundjoin%
\definecolor{currentfill}{rgb}{0.631373,0.788235,0.956863}%
\pgfsetfillcolor{currentfill}%
\pgfsetlinewidth{0.481800pt}%
\definecolor{currentstroke}{rgb}{1.000000,1.000000,1.000000}%
\pgfsetstrokecolor{currentstroke}%
\pgfsetdash{}{0pt}%
\pgfpathmoveto{\pgfqpoint{8.326824in}{4.882127in}}%
\pgfpathcurveto{\pgfqpoint{8.337874in}{4.882127in}}{\pgfqpoint{8.348473in}{4.886517in}}{\pgfqpoint{8.356287in}{4.894331in}}%
\pgfpathcurveto{\pgfqpoint{8.364100in}{4.902144in}}{\pgfqpoint{8.368491in}{4.912743in}}{\pgfqpoint{8.368491in}{4.923794in}}%
\pgfpathcurveto{\pgfqpoint{8.368491in}{4.934844in}}{\pgfqpoint{8.364100in}{4.945443in}}{\pgfqpoint{8.356287in}{4.953256in}}%
\pgfpathcurveto{\pgfqpoint{8.348473in}{4.961070in}}{\pgfqpoint{8.337874in}{4.965460in}}{\pgfqpoint{8.326824in}{4.965460in}}%
\pgfpathcurveto{\pgfqpoint{8.315774in}{4.965460in}}{\pgfqpoint{8.305175in}{4.961070in}}{\pgfqpoint{8.297361in}{4.953256in}}%
\pgfpathcurveto{\pgfqpoint{8.289548in}{4.945443in}}{\pgfqpoint{8.285157in}{4.934844in}}{\pgfqpoint{8.285157in}{4.923794in}}%
\pgfpathcurveto{\pgfqpoint{8.285157in}{4.912743in}}{\pgfqpoint{8.289548in}{4.902144in}}{\pgfqpoint{8.297361in}{4.894331in}}%
\pgfpathcurveto{\pgfqpoint{8.305175in}{4.886517in}}{\pgfqpoint{8.315774in}{4.882127in}}{\pgfqpoint{8.326824in}{4.882127in}}%
\pgfpathclose%
\pgfusepath{stroke,fill}%
\end{pgfscope}%
\begin{pgfscope}%
\pgfpathrectangle{\pgfqpoint{0.481978in}{0.331635in}}{\pgfqpoint{9.300000in}{7.700000in}}%
\pgfusepath{clip}%
\pgfsetbuttcap%
\pgfsetroundjoin%
\definecolor{currentfill}{rgb}{0.631373,0.788235,0.956863}%
\pgfsetfillcolor{currentfill}%
\pgfsetlinewidth{0.481800pt}%
\definecolor{currentstroke}{rgb}{1.000000,1.000000,1.000000}%
\pgfsetstrokecolor{currentstroke}%
\pgfsetdash{}{0pt}%
\pgfpathmoveto{\pgfqpoint{4.333517in}{5.742085in}}%
\pgfpathcurveto{\pgfqpoint{4.344567in}{5.742085in}}{\pgfqpoint{4.355166in}{5.746475in}}{\pgfqpoint{4.362979in}{5.754289in}}%
\pgfpathcurveto{\pgfqpoint{4.370793in}{5.762102in}}{\pgfqpoint{4.375183in}{5.772701in}}{\pgfqpoint{4.375183in}{5.783751in}}%
\pgfpathcurveto{\pgfqpoint{4.375183in}{5.794801in}}{\pgfqpoint{4.370793in}{5.805400in}}{\pgfqpoint{4.362979in}{5.813214in}}%
\pgfpathcurveto{\pgfqpoint{4.355166in}{5.821028in}}{\pgfqpoint{4.344567in}{5.825418in}}{\pgfqpoint{4.333517in}{5.825418in}}%
\pgfpathcurveto{\pgfqpoint{4.322466in}{5.825418in}}{\pgfqpoint{4.311867in}{5.821028in}}{\pgfqpoint{4.304054in}{5.813214in}}%
\pgfpathcurveto{\pgfqpoint{4.296240in}{5.805400in}}{\pgfqpoint{4.291850in}{5.794801in}}{\pgfqpoint{4.291850in}{5.783751in}}%
\pgfpathcurveto{\pgfqpoint{4.291850in}{5.772701in}}{\pgfqpoint{4.296240in}{5.762102in}}{\pgfqpoint{4.304054in}{5.754289in}}%
\pgfpathcurveto{\pgfqpoint{4.311867in}{5.746475in}}{\pgfqpoint{4.322466in}{5.742085in}}{\pgfqpoint{4.333517in}{5.742085in}}%
\pgfpathclose%
\pgfusepath{stroke,fill}%
\end{pgfscope}%
\begin{pgfscope}%
\pgfpathrectangle{\pgfqpoint{0.481978in}{0.331635in}}{\pgfqpoint{9.300000in}{7.700000in}}%
\pgfusepath{clip}%
\pgfsetbuttcap%
\pgfsetroundjoin%
\definecolor{currentfill}{rgb}{0.631373,0.788235,0.956863}%
\pgfsetfillcolor{currentfill}%
\pgfsetlinewidth{0.481800pt}%
\definecolor{currentstroke}{rgb}{1.000000,1.000000,1.000000}%
\pgfsetstrokecolor{currentstroke}%
\pgfsetdash{}{0pt}%
\pgfpathmoveto{\pgfqpoint{5.146855in}{2.018893in}}%
\pgfpathcurveto{\pgfqpoint{5.157905in}{2.018893in}}{\pgfqpoint{5.168504in}{2.023284in}}{\pgfqpoint{5.176317in}{2.031097in}}%
\pgfpathcurveto{\pgfqpoint{5.184131in}{2.038911in}}{\pgfqpoint{5.188521in}{2.049510in}}{\pgfqpoint{5.188521in}{2.060560in}}%
\pgfpathcurveto{\pgfqpoint{5.188521in}{2.071610in}}{\pgfqpoint{5.184131in}{2.082209in}}{\pgfqpoint{5.176317in}{2.090023in}}%
\pgfpathcurveto{\pgfqpoint{5.168504in}{2.097836in}}{\pgfqpoint{5.157905in}{2.102227in}}{\pgfqpoint{5.146855in}{2.102227in}}%
\pgfpathcurveto{\pgfqpoint{5.135805in}{2.102227in}}{\pgfqpoint{5.125206in}{2.097836in}}{\pgfqpoint{5.117392in}{2.090023in}}%
\pgfpathcurveto{\pgfqpoint{5.109578in}{2.082209in}}{\pgfqpoint{5.105188in}{2.071610in}}{\pgfqpoint{5.105188in}{2.060560in}}%
\pgfpathcurveto{\pgfqpoint{5.105188in}{2.049510in}}{\pgfqpoint{5.109578in}{2.038911in}}{\pgfqpoint{5.117392in}{2.031097in}}%
\pgfpathcurveto{\pgfqpoint{5.125206in}{2.023284in}}{\pgfqpoint{5.135805in}{2.018893in}}{\pgfqpoint{5.146855in}{2.018893in}}%
\pgfpathclose%
\pgfusepath{stroke,fill}%
\end{pgfscope}%
\begin{pgfscope}%
\pgfpathrectangle{\pgfqpoint{0.481978in}{0.331635in}}{\pgfqpoint{9.300000in}{7.700000in}}%
\pgfusepath{clip}%
\pgfsetbuttcap%
\pgfsetroundjoin%
\definecolor{currentfill}{rgb}{0.631373,0.788235,0.956863}%
\pgfsetfillcolor{currentfill}%
\pgfsetlinewidth{0.481800pt}%
\definecolor{currentstroke}{rgb}{1.000000,1.000000,1.000000}%
\pgfsetstrokecolor{currentstroke}%
\pgfsetdash{}{0pt}%
\pgfpathmoveto{\pgfqpoint{6.726675in}{5.368676in}}%
\pgfpathcurveto{\pgfqpoint{6.737725in}{5.368676in}}{\pgfqpoint{6.748324in}{5.373066in}}{\pgfqpoint{6.756138in}{5.380880in}}%
\pgfpathcurveto{\pgfqpoint{6.763951in}{5.388694in}}{\pgfqpoint{6.768342in}{5.399293in}}{\pgfqpoint{6.768342in}{5.410343in}}%
\pgfpathcurveto{\pgfqpoint{6.768342in}{5.421393in}}{\pgfqpoint{6.763951in}{5.431992in}}{\pgfqpoint{6.756138in}{5.439806in}}%
\pgfpathcurveto{\pgfqpoint{6.748324in}{5.447619in}}{\pgfqpoint{6.737725in}{5.452009in}}{\pgfqpoint{6.726675in}{5.452009in}}%
\pgfpathcurveto{\pgfqpoint{6.715625in}{5.452009in}}{\pgfqpoint{6.705026in}{5.447619in}}{\pgfqpoint{6.697212in}{5.439806in}}%
\pgfpathcurveto{\pgfqpoint{6.689399in}{5.431992in}}{\pgfqpoint{6.685008in}{5.421393in}}{\pgfqpoint{6.685008in}{5.410343in}}%
\pgfpathcurveto{\pgfqpoint{6.685008in}{5.399293in}}{\pgfqpoint{6.689399in}{5.388694in}}{\pgfqpoint{6.697212in}{5.380880in}}%
\pgfpathcurveto{\pgfqpoint{6.705026in}{5.373066in}}{\pgfqpoint{6.715625in}{5.368676in}}{\pgfqpoint{6.726675in}{5.368676in}}%
\pgfpathclose%
\pgfusepath{stroke,fill}%
\end{pgfscope}%
\begin{pgfscope}%
\pgfpathrectangle{\pgfqpoint{0.481978in}{0.331635in}}{\pgfqpoint{9.300000in}{7.700000in}}%
\pgfusepath{clip}%
\pgfsetbuttcap%
\pgfsetroundjoin%
\definecolor{currentfill}{rgb}{0.631373,0.788235,0.956863}%
\pgfsetfillcolor{currentfill}%
\pgfsetlinewidth{0.481800pt}%
\definecolor{currentstroke}{rgb}{1.000000,1.000000,1.000000}%
\pgfsetstrokecolor{currentstroke}%
\pgfsetdash{}{0pt}%
\pgfpathmoveto{\pgfqpoint{5.444209in}{2.533371in}}%
\pgfpathcurveto{\pgfqpoint{5.455260in}{2.533371in}}{\pgfqpoint{5.465859in}{2.537761in}}{\pgfqpoint{5.473672in}{2.545575in}}%
\pgfpathcurveto{\pgfqpoint{5.481486in}{2.553389in}}{\pgfqpoint{5.485876in}{2.563988in}}{\pgfqpoint{5.485876in}{2.575038in}}%
\pgfpathcurveto{\pgfqpoint{5.485876in}{2.586088in}}{\pgfqpoint{5.481486in}{2.596687in}}{\pgfqpoint{5.473672in}{2.604501in}}%
\pgfpathcurveto{\pgfqpoint{5.465859in}{2.612314in}}{\pgfqpoint{5.455260in}{2.616705in}}{\pgfqpoint{5.444209in}{2.616705in}}%
\pgfpathcurveto{\pgfqpoint{5.433159in}{2.616705in}}{\pgfqpoint{5.422560in}{2.612314in}}{\pgfqpoint{5.414747in}{2.604501in}}%
\pgfpathcurveto{\pgfqpoint{5.406933in}{2.596687in}}{\pgfqpoint{5.402543in}{2.586088in}}{\pgfqpoint{5.402543in}{2.575038in}}%
\pgfpathcurveto{\pgfqpoint{5.402543in}{2.563988in}}{\pgfqpoint{5.406933in}{2.553389in}}{\pgfqpoint{5.414747in}{2.545575in}}%
\pgfpathcurveto{\pgfqpoint{5.422560in}{2.537761in}}{\pgfqpoint{5.433159in}{2.533371in}}{\pgfqpoint{5.444209in}{2.533371in}}%
\pgfpathclose%
\pgfusepath{stroke,fill}%
\end{pgfscope}%
\begin{pgfscope}%
\pgfpathrectangle{\pgfqpoint{0.481978in}{0.331635in}}{\pgfqpoint{9.300000in}{7.700000in}}%
\pgfusepath{clip}%
\pgfsetbuttcap%
\pgfsetroundjoin%
\definecolor{currentfill}{rgb}{0.631373,0.788235,0.956863}%
\pgfsetfillcolor{currentfill}%
\pgfsetlinewidth{0.481800pt}%
\definecolor{currentstroke}{rgb}{1.000000,1.000000,1.000000}%
\pgfsetstrokecolor{currentstroke}%
\pgfsetdash{}{0pt}%
\pgfpathmoveto{\pgfqpoint{6.317681in}{2.742307in}}%
\pgfpathcurveto{\pgfqpoint{6.328731in}{2.742307in}}{\pgfqpoint{6.339330in}{2.746697in}}{\pgfqpoint{6.347143in}{2.754511in}}%
\pgfpathcurveto{\pgfqpoint{6.354957in}{2.762325in}}{\pgfqpoint{6.359347in}{2.772924in}}{\pgfqpoint{6.359347in}{2.783974in}}%
\pgfpathcurveto{\pgfqpoint{6.359347in}{2.795024in}}{\pgfqpoint{6.354957in}{2.805623in}}{\pgfqpoint{6.347143in}{2.813437in}}%
\pgfpathcurveto{\pgfqpoint{6.339330in}{2.821250in}}{\pgfqpoint{6.328731in}{2.825641in}}{\pgfqpoint{6.317681in}{2.825641in}}%
\pgfpathcurveto{\pgfqpoint{6.306631in}{2.825641in}}{\pgfqpoint{6.296032in}{2.821250in}}{\pgfqpoint{6.288218in}{2.813437in}}%
\pgfpathcurveto{\pgfqpoint{6.280404in}{2.805623in}}{\pgfqpoint{6.276014in}{2.795024in}}{\pgfqpoint{6.276014in}{2.783974in}}%
\pgfpathcurveto{\pgfqpoint{6.276014in}{2.772924in}}{\pgfqpoint{6.280404in}{2.762325in}}{\pgfqpoint{6.288218in}{2.754511in}}%
\pgfpathcurveto{\pgfqpoint{6.296032in}{2.746697in}}{\pgfqpoint{6.306631in}{2.742307in}}{\pgfqpoint{6.317681in}{2.742307in}}%
\pgfpathclose%
\pgfusepath{stroke,fill}%
\end{pgfscope}%
\begin{pgfscope}%
\pgfpathrectangle{\pgfqpoint{0.481978in}{0.331635in}}{\pgfqpoint{9.300000in}{7.700000in}}%
\pgfusepath{clip}%
\pgfsetbuttcap%
\pgfsetroundjoin%
\definecolor{currentfill}{rgb}{0.631373,0.788235,0.956863}%
\pgfsetfillcolor{currentfill}%
\pgfsetlinewidth{0.481800pt}%
\definecolor{currentstroke}{rgb}{1.000000,1.000000,1.000000}%
\pgfsetstrokecolor{currentstroke}%
\pgfsetdash{}{0pt}%
\pgfpathmoveto{\pgfqpoint{5.001156in}{3.852883in}}%
\pgfpathcurveto{\pgfqpoint{5.012206in}{3.852883in}}{\pgfqpoint{5.022805in}{3.857273in}}{\pgfqpoint{5.030618in}{3.865087in}}%
\pgfpathcurveto{\pgfqpoint{5.038432in}{3.872900in}}{\pgfqpoint{5.042822in}{3.883499in}}{\pgfqpoint{5.042822in}{3.894549in}}%
\pgfpathcurveto{\pgfqpoint{5.042822in}{3.905600in}}{\pgfqpoint{5.038432in}{3.916199in}}{\pgfqpoint{5.030618in}{3.924012in}}%
\pgfpathcurveto{\pgfqpoint{5.022805in}{3.931826in}}{\pgfqpoint{5.012206in}{3.936216in}}{\pgfqpoint{5.001156in}{3.936216in}}%
\pgfpathcurveto{\pgfqpoint{4.990105in}{3.936216in}}{\pgfqpoint{4.979506in}{3.931826in}}{\pgfqpoint{4.971693in}{3.924012in}}%
\pgfpathcurveto{\pgfqpoint{4.963879in}{3.916199in}}{\pgfqpoint{4.959489in}{3.905600in}}{\pgfqpoint{4.959489in}{3.894549in}}%
\pgfpathcurveto{\pgfqpoint{4.959489in}{3.883499in}}{\pgfqpoint{4.963879in}{3.872900in}}{\pgfqpoint{4.971693in}{3.865087in}}%
\pgfpathcurveto{\pgfqpoint{4.979506in}{3.857273in}}{\pgfqpoint{4.990105in}{3.852883in}}{\pgfqpoint{5.001156in}{3.852883in}}%
\pgfpathclose%
\pgfusepath{stroke,fill}%
\end{pgfscope}%
\begin{pgfscope}%
\pgfpathrectangle{\pgfqpoint{0.481978in}{0.331635in}}{\pgfqpoint{9.300000in}{7.700000in}}%
\pgfusepath{clip}%
\pgfsetbuttcap%
\pgfsetroundjoin%
\definecolor{currentfill}{rgb}{0.631373,0.788235,0.956863}%
\pgfsetfillcolor{currentfill}%
\pgfsetlinewidth{0.481800pt}%
\definecolor{currentstroke}{rgb}{1.000000,1.000000,1.000000}%
\pgfsetstrokecolor{currentstroke}%
\pgfsetdash{}{0pt}%
\pgfpathmoveto{\pgfqpoint{5.223174in}{3.234581in}}%
\pgfpathcurveto{\pgfqpoint{5.234224in}{3.234581in}}{\pgfqpoint{5.244823in}{3.238972in}}{\pgfqpoint{5.252637in}{3.246785in}}%
\pgfpathcurveto{\pgfqpoint{5.260450in}{3.254599in}}{\pgfqpoint{5.264841in}{3.265198in}}{\pgfqpoint{5.264841in}{3.276248in}}%
\pgfpathcurveto{\pgfqpoint{5.264841in}{3.287298in}}{\pgfqpoint{5.260450in}{3.297897in}}{\pgfqpoint{5.252637in}{3.305711in}}%
\pgfpathcurveto{\pgfqpoint{5.244823in}{3.313525in}}{\pgfqpoint{5.234224in}{3.317915in}}{\pgfqpoint{5.223174in}{3.317915in}}%
\pgfpathcurveto{\pgfqpoint{5.212124in}{3.317915in}}{\pgfqpoint{5.201525in}{3.313525in}}{\pgfqpoint{5.193711in}{3.305711in}}%
\pgfpathcurveto{\pgfqpoint{5.185897in}{3.297897in}}{\pgfqpoint{5.181507in}{3.287298in}}{\pgfqpoint{5.181507in}{3.276248in}}%
\pgfpathcurveto{\pgfqpoint{5.181507in}{3.265198in}}{\pgfqpoint{5.185897in}{3.254599in}}{\pgfqpoint{5.193711in}{3.246785in}}%
\pgfpathcurveto{\pgfqpoint{5.201525in}{3.238972in}}{\pgfqpoint{5.212124in}{3.234581in}}{\pgfqpoint{5.223174in}{3.234581in}}%
\pgfpathclose%
\pgfusepath{stroke,fill}%
\end{pgfscope}%
\begin{pgfscope}%
\pgfpathrectangle{\pgfqpoint{0.481978in}{0.331635in}}{\pgfqpoint{9.300000in}{7.700000in}}%
\pgfusepath{clip}%
\pgfsetbuttcap%
\pgfsetroundjoin%
\definecolor{currentfill}{rgb}{0.631373,0.788235,0.956863}%
\pgfsetfillcolor{currentfill}%
\pgfsetlinewidth{0.481800pt}%
\definecolor{currentstroke}{rgb}{1.000000,1.000000,1.000000}%
\pgfsetstrokecolor{currentstroke}%
\pgfsetdash{}{0pt}%
\pgfpathmoveto{\pgfqpoint{5.990771in}{2.510229in}}%
\pgfpathcurveto{\pgfqpoint{6.001822in}{2.510229in}}{\pgfqpoint{6.012421in}{2.514620in}}{\pgfqpoint{6.020234in}{2.522433in}}%
\pgfpathcurveto{\pgfqpoint{6.028048in}{2.530247in}}{\pgfqpoint{6.032438in}{2.540846in}}{\pgfqpoint{6.032438in}{2.551896in}}%
\pgfpathcurveto{\pgfqpoint{6.032438in}{2.562946in}}{\pgfqpoint{6.028048in}{2.573545in}}{\pgfqpoint{6.020234in}{2.581359in}}%
\pgfpathcurveto{\pgfqpoint{6.012421in}{2.589172in}}{\pgfqpoint{6.001822in}{2.593563in}}{\pgfqpoint{5.990771in}{2.593563in}}%
\pgfpathcurveto{\pgfqpoint{5.979721in}{2.593563in}}{\pgfqpoint{5.969122in}{2.589172in}}{\pgfqpoint{5.961309in}{2.581359in}}%
\pgfpathcurveto{\pgfqpoint{5.953495in}{2.573545in}}{\pgfqpoint{5.949105in}{2.562946in}}{\pgfqpoint{5.949105in}{2.551896in}}%
\pgfpathcurveto{\pgfqpoint{5.949105in}{2.540846in}}{\pgfqpoint{5.953495in}{2.530247in}}{\pgfqpoint{5.961309in}{2.522433in}}%
\pgfpathcurveto{\pgfqpoint{5.969122in}{2.514620in}}{\pgfqpoint{5.979721in}{2.510229in}}{\pgfqpoint{5.990771in}{2.510229in}}%
\pgfpathclose%
\pgfusepath{stroke,fill}%
\end{pgfscope}%
\begin{pgfscope}%
\pgfpathrectangle{\pgfqpoint{0.481978in}{0.331635in}}{\pgfqpoint{9.300000in}{7.700000in}}%
\pgfusepath{clip}%
\pgfsetbuttcap%
\pgfsetroundjoin%
\definecolor{currentfill}{rgb}{0.631373,0.788235,0.956863}%
\pgfsetfillcolor{currentfill}%
\pgfsetlinewidth{0.481800pt}%
\definecolor{currentstroke}{rgb}{1.000000,1.000000,1.000000}%
\pgfsetstrokecolor{currentstroke}%
\pgfsetdash{}{0pt}%
\pgfpathmoveto{\pgfqpoint{5.037316in}{4.288289in}}%
\pgfpathcurveto{\pgfqpoint{5.048366in}{4.288289in}}{\pgfqpoint{5.058965in}{4.292679in}}{\pgfqpoint{5.066779in}{4.300493in}}%
\pgfpathcurveto{\pgfqpoint{5.074592in}{4.308306in}}{\pgfqpoint{5.078983in}{4.318905in}}{\pgfqpoint{5.078983in}{4.329956in}}%
\pgfpathcurveto{\pgfqpoint{5.078983in}{4.341006in}}{\pgfqpoint{5.074592in}{4.351605in}}{\pgfqpoint{5.066779in}{4.359418in}}%
\pgfpathcurveto{\pgfqpoint{5.058965in}{4.367232in}}{\pgfqpoint{5.048366in}{4.371622in}}{\pgfqpoint{5.037316in}{4.371622in}}%
\pgfpathcurveto{\pgfqpoint{5.026266in}{4.371622in}}{\pgfqpoint{5.015667in}{4.367232in}}{\pgfqpoint{5.007853in}{4.359418in}}%
\pgfpathcurveto{\pgfqpoint{5.000040in}{4.351605in}}{\pgfqpoint{4.995649in}{4.341006in}}{\pgfqpoint{4.995649in}{4.329956in}}%
\pgfpathcurveto{\pgfqpoint{4.995649in}{4.318905in}}{\pgfqpoint{5.000040in}{4.308306in}}{\pgfqpoint{5.007853in}{4.300493in}}%
\pgfpathcurveto{\pgfqpoint{5.015667in}{4.292679in}}{\pgfqpoint{5.026266in}{4.288289in}}{\pgfqpoint{5.037316in}{4.288289in}}%
\pgfpathclose%
\pgfusepath{stroke,fill}%
\end{pgfscope}%
\begin{pgfscope}%
\pgfpathrectangle{\pgfqpoint{0.481978in}{0.331635in}}{\pgfqpoint{9.300000in}{7.700000in}}%
\pgfusepath{clip}%
\pgfsetbuttcap%
\pgfsetroundjoin%
\definecolor{currentfill}{rgb}{0.631373,0.788235,0.956863}%
\pgfsetfillcolor{currentfill}%
\pgfsetlinewidth{0.481800pt}%
\definecolor{currentstroke}{rgb}{1.000000,1.000000,1.000000}%
\pgfsetstrokecolor{currentstroke}%
\pgfsetdash{}{0pt}%
\pgfpathmoveto{\pgfqpoint{4.961622in}{0.843579in}}%
\pgfpathcurveto{\pgfqpoint{4.972673in}{0.843579in}}{\pgfqpoint{4.983272in}{0.847969in}}{\pgfqpoint{4.991085in}{0.855783in}}%
\pgfpathcurveto{\pgfqpoint{4.998899in}{0.863597in}}{\pgfqpoint{5.003289in}{0.874196in}}{\pgfqpoint{5.003289in}{0.885246in}}%
\pgfpathcurveto{\pgfqpoint{5.003289in}{0.896296in}}{\pgfqpoint{4.998899in}{0.906895in}}{\pgfqpoint{4.991085in}{0.914708in}}%
\pgfpathcurveto{\pgfqpoint{4.983272in}{0.922522in}}{\pgfqpoint{4.972673in}{0.926912in}}{\pgfqpoint{4.961622in}{0.926912in}}%
\pgfpathcurveto{\pgfqpoint{4.950572in}{0.926912in}}{\pgfqpoint{4.939973in}{0.922522in}}{\pgfqpoint{4.932160in}{0.914708in}}%
\pgfpathcurveto{\pgfqpoint{4.924346in}{0.906895in}}{\pgfqpoint{4.919956in}{0.896296in}}{\pgfqpoint{4.919956in}{0.885246in}}%
\pgfpathcurveto{\pgfqpoint{4.919956in}{0.874196in}}{\pgfqpoint{4.924346in}{0.863597in}}{\pgfqpoint{4.932160in}{0.855783in}}%
\pgfpathcurveto{\pgfqpoint{4.939973in}{0.847969in}}{\pgfqpoint{4.950572in}{0.843579in}}{\pgfqpoint{4.961622in}{0.843579in}}%
\pgfpathclose%
\pgfusepath{stroke,fill}%
\end{pgfscope}%
\begin{pgfscope}%
\pgfpathrectangle{\pgfqpoint{0.481978in}{0.331635in}}{\pgfqpoint{9.300000in}{7.700000in}}%
\pgfusepath{clip}%
\pgfsetbuttcap%
\pgfsetroundjoin%
\definecolor{currentfill}{rgb}{0.631373,0.788235,0.956863}%
\pgfsetfillcolor{currentfill}%
\pgfsetlinewidth{0.481800pt}%
\definecolor{currentstroke}{rgb}{1.000000,1.000000,1.000000}%
\pgfsetstrokecolor{currentstroke}%
\pgfsetdash{}{0pt}%
\pgfpathmoveto{\pgfqpoint{5.509130in}{2.049624in}}%
\pgfpathcurveto{\pgfqpoint{5.520180in}{2.049624in}}{\pgfqpoint{5.530779in}{2.054014in}}{\pgfqpoint{5.538593in}{2.061828in}}%
\pgfpathcurveto{\pgfqpoint{5.546406in}{2.069641in}}{\pgfqpoint{5.550797in}{2.080240in}}{\pgfqpoint{5.550797in}{2.091290in}}%
\pgfpathcurveto{\pgfqpoint{5.550797in}{2.102340in}}{\pgfqpoint{5.546406in}{2.112940in}}{\pgfqpoint{5.538593in}{2.120753in}}%
\pgfpathcurveto{\pgfqpoint{5.530779in}{2.128567in}}{\pgfqpoint{5.520180in}{2.132957in}}{\pgfqpoint{5.509130in}{2.132957in}}%
\pgfpathcurveto{\pgfqpoint{5.498080in}{2.132957in}}{\pgfqpoint{5.487481in}{2.128567in}}{\pgfqpoint{5.479667in}{2.120753in}}%
\pgfpathcurveto{\pgfqpoint{5.471854in}{2.112940in}}{\pgfqpoint{5.467463in}{2.102340in}}{\pgfqpoint{5.467463in}{2.091290in}}%
\pgfpathcurveto{\pgfqpoint{5.467463in}{2.080240in}}{\pgfqpoint{5.471854in}{2.069641in}}{\pgfqpoint{5.479667in}{2.061828in}}%
\pgfpathcurveto{\pgfqpoint{5.487481in}{2.054014in}}{\pgfqpoint{5.498080in}{2.049624in}}{\pgfqpoint{5.509130in}{2.049624in}}%
\pgfpathclose%
\pgfusepath{stroke,fill}%
\end{pgfscope}%
\begin{pgfscope}%
\pgfpathrectangle{\pgfqpoint{0.481978in}{0.331635in}}{\pgfqpoint{9.300000in}{7.700000in}}%
\pgfusepath{clip}%
\pgfsetbuttcap%
\pgfsetroundjoin%
\definecolor{currentfill}{rgb}{0.631373,0.788235,0.956863}%
\pgfsetfillcolor{currentfill}%
\pgfsetlinewidth{0.481800pt}%
\definecolor{currentstroke}{rgb}{1.000000,1.000000,1.000000}%
\pgfsetstrokecolor{currentstroke}%
\pgfsetdash{}{0pt}%
\pgfpathmoveto{\pgfqpoint{2.965360in}{6.380512in}}%
\pgfpathcurveto{\pgfqpoint{2.976410in}{6.380512in}}{\pgfqpoint{2.987009in}{6.384902in}}{\pgfqpoint{2.994823in}{6.392716in}}%
\pgfpathcurveto{\pgfqpoint{3.002636in}{6.400529in}}{\pgfqpoint{3.007027in}{6.411129in}}{\pgfqpoint{3.007027in}{6.422179in}}%
\pgfpathcurveto{\pgfqpoint{3.007027in}{6.433229in}}{\pgfqpoint{3.002636in}{6.443828in}}{\pgfqpoint{2.994823in}{6.451641in}}%
\pgfpathcurveto{\pgfqpoint{2.987009in}{6.459455in}}{\pgfqpoint{2.976410in}{6.463845in}}{\pgfqpoint{2.965360in}{6.463845in}}%
\pgfpathcurveto{\pgfqpoint{2.954310in}{6.463845in}}{\pgfqpoint{2.943711in}{6.459455in}}{\pgfqpoint{2.935897in}{6.451641in}}%
\pgfpathcurveto{\pgfqpoint{2.928083in}{6.443828in}}{\pgfqpoint{2.923693in}{6.433229in}}{\pgfqpoint{2.923693in}{6.422179in}}%
\pgfpathcurveto{\pgfqpoint{2.923693in}{6.411129in}}{\pgfqpoint{2.928083in}{6.400529in}}{\pgfqpoint{2.935897in}{6.392716in}}%
\pgfpathcurveto{\pgfqpoint{2.943711in}{6.384902in}}{\pgfqpoint{2.954310in}{6.380512in}}{\pgfqpoint{2.965360in}{6.380512in}}%
\pgfpathclose%
\pgfusepath{stroke,fill}%
\end{pgfscope}%
\begin{pgfscope}%
\pgfpathrectangle{\pgfqpoint{0.481978in}{0.331635in}}{\pgfqpoint{9.300000in}{7.700000in}}%
\pgfusepath{clip}%
\pgfsetbuttcap%
\pgfsetroundjoin%
\definecolor{currentfill}{rgb}{0.631373,0.788235,0.956863}%
\pgfsetfillcolor{currentfill}%
\pgfsetlinewidth{0.481800pt}%
\definecolor{currentstroke}{rgb}{1.000000,1.000000,1.000000}%
\pgfsetstrokecolor{currentstroke}%
\pgfsetdash{}{0pt}%
\pgfpathmoveto{\pgfqpoint{6.394060in}{2.525199in}}%
\pgfpathcurveto{\pgfqpoint{6.405110in}{2.525199in}}{\pgfqpoint{6.415709in}{2.529590in}}{\pgfqpoint{6.423523in}{2.537403in}}%
\pgfpathcurveto{\pgfqpoint{6.431336in}{2.545217in}}{\pgfqpoint{6.435727in}{2.555816in}}{\pgfqpoint{6.435727in}{2.566866in}}%
\pgfpathcurveto{\pgfqpoint{6.435727in}{2.577916in}}{\pgfqpoint{6.431336in}{2.588515in}}{\pgfqpoint{6.423523in}{2.596329in}}%
\pgfpathcurveto{\pgfqpoint{6.415709in}{2.604142in}}{\pgfqpoint{6.405110in}{2.608533in}}{\pgfqpoint{6.394060in}{2.608533in}}%
\pgfpathcurveto{\pgfqpoint{6.383010in}{2.608533in}}{\pgfqpoint{6.372411in}{2.604142in}}{\pgfqpoint{6.364597in}{2.596329in}}%
\pgfpathcurveto{\pgfqpoint{6.356784in}{2.588515in}}{\pgfqpoint{6.352393in}{2.577916in}}{\pgfqpoint{6.352393in}{2.566866in}}%
\pgfpathcurveto{\pgfqpoint{6.352393in}{2.555816in}}{\pgfqpoint{6.356784in}{2.545217in}}{\pgfqpoint{6.364597in}{2.537403in}}%
\pgfpathcurveto{\pgfqpoint{6.372411in}{2.529590in}}{\pgfqpoint{6.383010in}{2.525199in}}{\pgfqpoint{6.394060in}{2.525199in}}%
\pgfpathclose%
\pgfusepath{stroke,fill}%
\end{pgfscope}%
\begin{pgfscope}%
\pgfpathrectangle{\pgfqpoint{0.481978in}{0.331635in}}{\pgfqpoint{9.300000in}{7.700000in}}%
\pgfusepath{clip}%
\pgfsetbuttcap%
\pgfsetroundjoin%
\definecolor{currentfill}{rgb}{0.631373,0.788235,0.956863}%
\pgfsetfillcolor{currentfill}%
\pgfsetlinewidth{0.481800pt}%
\definecolor{currentstroke}{rgb}{1.000000,1.000000,1.000000}%
\pgfsetstrokecolor{currentstroke}%
\pgfsetdash{}{0pt}%
\pgfpathmoveto{\pgfqpoint{2.186468in}{6.771610in}}%
\pgfpathcurveto{\pgfqpoint{2.197518in}{6.771610in}}{\pgfqpoint{2.208117in}{6.776000in}}{\pgfqpoint{2.215931in}{6.783814in}}%
\pgfpathcurveto{\pgfqpoint{2.223745in}{6.791628in}}{\pgfqpoint{2.228135in}{6.802227in}}{\pgfqpoint{2.228135in}{6.813277in}}%
\pgfpathcurveto{\pgfqpoint{2.228135in}{6.824327in}}{\pgfqpoint{2.223745in}{6.834926in}}{\pgfqpoint{2.215931in}{6.842740in}}%
\pgfpathcurveto{\pgfqpoint{2.208117in}{6.850553in}}{\pgfqpoint{2.197518in}{6.854944in}}{\pgfqpoint{2.186468in}{6.854944in}}%
\pgfpathcurveto{\pgfqpoint{2.175418in}{6.854944in}}{\pgfqpoint{2.164819in}{6.850553in}}{\pgfqpoint{2.157005in}{6.842740in}}%
\pgfpathcurveto{\pgfqpoint{2.149192in}{6.834926in}}{\pgfqpoint{2.144802in}{6.824327in}}{\pgfqpoint{2.144802in}{6.813277in}}%
\pgfpathcurveto{\pgfqpoint{2.144802in}{6.802227in}}{\pgfqpoint{2.149192in}{6.791628in}}{\pgfqpoint{2.157005in}{6.783814in}}%
\pgfpathcurveto{\pgfqpoint{2.164819in}{6.776000in}}{\pgfqpoint{2.175418in}{6.771610in}}{\pgfqpoint{2.186468in}{6.771610in}}%
\pgfpathclose%
\pgfusepath{stroke,fill}%
\end{pgfscope}%
\begin{pgfscope}%
\pgfpathrectangle{\pgfqpoint{0.481978in}{0.331635in}}{\pgfqpoint{9.300000in}{7.700000in}}%
\pgfusepath{clip}%
\pgfsetbuttcap%
\pgfsetroundjoin%
\definecolor{currentfill}{rgb}{0.631373,0.788235,0.956863}%
\pgfsetfillcolor{currentfill}%
\pgfsetlinewidth{0.481800pt}%
\definecolor{currentstroke}{rgb}{1.000000,1.000000,1.000000}%
\pgfsetstrokecolor{currentstroke}%
\pgfsetdash{}{0pt}%
\pgfpathmoveto{\pgfqpoint{7.121234in}{2.875912in}}%
\pgfpathcurveto{\pgfqpoint{7.132284in}{2.875912in}}{\pgfqpoint{7.142883in}{2.880302in}}{\pgfqpoint{7.150697in}{2.888116in}}%
\pgfpathcurveto{\pgfqpoint{7.158510in}{2.895930in}}{\pgfqpoint{7.162901in}{2.906529in}}{\pgfqpoint{7.162901in}{2.917579in}}%
\pgfpathcurveto{\pgfqpoint{7.162901in}{2.928629in}}{\pgfqpoint{7.158510in}{2.939228in}}{\pgfqpoint{7.150697in}{2.947042in}}%
\pgfpathcurveto{\pgfqpoint{7.142883in}{2.954855in}}{\pgfqpoint{7.132284in}{2.959245in}}{\pgfqpoint{7.121234in}{2.959245in}}%
\pgfpathcurveto{\pgfqpoint{7.110184in}{2.959245in}}{\pgfqpoint{7.099585in}{2.954855in}}{\pgfqpoint{7.091771in}{2.947042in}}%
\pgfpathcurveto{\pgfqpoint{7.083958in}{2.939228in}}{\pgfqpoint{7.079567in}{2.928629in}}{\pgfqpoint{7.079567in}{2.917579in}}%
\pgfpathcurveto{\pgfqpoint{7.079567in}{2.906529in}}{\pgfqpoint{7.083958in}{2.895930in}}{\pgfqpoint{7.091771in}{2.888116in}}%
\pgfpathcurveto{\pgfqpoint{7.099585in}{2.880302in}}{\pgfqpoint{7.110184in}{2.875912in}}{\pgfqpoint{7.121234in}{2.875912in}}%
\pgfpathclose%
\pgfusepath{stroke,fill}%
\end{pgfscope}%
\begin{pgfscope}%
\pgfpathrectangle{\pgfqpoint{0.481978in}{0.331635in}}{\pgfqpoint{9.300000in}{7.700000in}}%
\pgfusepath{clip}%
\pgfsetbuttcap%
\pgfsetroundjoin%
\definecolor{currentfill}{rgb}{0.631373,0.788235,0.956863}%
\pgfsetfillcolor{currentfill}%
\pgfsetlinewidth{0.481800pt}%
\definecolor{currentstroke}{rgb}{1.000000,1.000000,1.000000}%
\pgfsetstrokecolor{currentstroke}%
\pgfsetdash{}{0pt}%
\pgfpathmoveto{\pgfqpoint{8.167603in}{4.962831in}}%
\pgfpathcurveto{\pgfqpoint{8.178653in}{4.962831in}}{\pgfqpoint{8.189252in}{4.967221in}}{\pgfqpoint{8.197066in}{4.975035in}}%
\pgfpathcurveto{\pgfqpoint{8.204879in}{4.982848in}}{\pgfqpoint{8.209270in}{4.993447in}}{\pgfqpoint{8.209270in}{5.004497in}}%
\pgfpathcurveto{\pgfqpoint{8.209270in}{5.015547in}}{\pgfqpoint{8.204879in}{5.026146in}}{\pgfqpoint{8.197066in}{5.033960in}}%
\pgfpathcurveto{\pgfqpoint{8.189252in}{5.041774in}}{\pgfqpoint{8.178653in}{5.046164in}}{\pgfqpoint{8.167603in}{5.046164in}}%
\pgfpathcurveto{\pgfqpoint{8.156553in}{5.046164in}}{\pgfqpoint{8.145954in}{5.041774in}}{\pgfqpoint{8.138140in}{5.033960in}}%
\pgfpathcurveto{\pgfqpoint{8.130327in}{5.026146in}}{\pgfqpoint{8.125936in}{5.015547in}}{\pgfqpoint{8.125936in}{5.004497in}}%
\pgfpathcurveto{\pgfqpoint{8.125936in}{4.993447in}}{\pgfqpoint{8.130327in}{4.982848in}}{\pgfqpoint{8.138140in}{4.975035in}}%
\pgfpathcurveto{\pgfqpoint{8.145954in}{4.967221in}}{\pgfqpoint{8.156553in}{4.962831in}}{\pgfqpoint{8.167603in}{4.962831in}}%
\pgfpathclose%
\pgfusepath{stroke,fill}%
\end{pgfscope}%
\begin{pgfscope}%
\pgfpathrectangle{\pgfqpoint{0.481978in}{0.331635in}}{\pgfqpoint{9.300000in}{7.700000in}}%
\pgfusepath{clip}%
\pgfsetbuttcap%
\pgfsetroundjoin%
\definecolor{currentfill}{rgb}{0.631373,0.788235,0.956863}%
\pgfsetfillcolor{currentfill}%
\pgfsetlinewidth{0.481800pt}%
\definecolor{currentstroke}{rgb}{1.000000,1.000000,1.000000}%
\pgfsetstrokecolor{currentstroke}%
\pgfsetdash{}{0pt}%
\pgfpathmoveto{\pgfqpoint{6.293261in}{4.405108in}}%
\pgfpathcurveto{\pgfqpoint{6.304311in}{4.405108in}}{\pgfqpoint{6.314910in}{4.409499in}}{\pgfqpoint{6.322724in}{4.417312in}}%
\pgfpathcurveto{\pgfqpoint{6.330538in}{4.425126in}}{\pgfqpoint{6.334928in}{4.435725in}}{\pgfqpoint{6.334928in}{4.446775in}}%
\pgfpathcurveto{\pgfqpoint{6.334928in}{4.457825in}}{\pgfqpoint{6.330538in}{4.468424in}}{\pgfqpoint{6.322724in}{4.476238in}}%
\pgfpathcurveto{\pgfqpoint{6.314910in}{4.484051in}}{\pgfqpoint{6.304311in}{4.488442in}}{\pgfqpoint{6.293261in}{4.488442in}}%
\pgfpathcurveto{\pgfqpoint{6.282211in}{4.488442in}}{\pgfqpoint{6.271612in}{4.484051in}}{\pgfqpoint{6.263799in}{4.476238in}}%
\pgfpathcurveto{\pgfqpoint{6.255985in}{4.468424in}}{\pgfqpoint{6.251595in}{4.457825in}}{\pgfqpoint{6.251595in}{4.446775in}}%
\pgfpathcurveto{\pgfqpoint{6.251595in}{4.435725in}}{\pgfqpoint{6.255985in}{4.425126in}}{\pgfqpoint{6.263799in}{4.417312in}}%
\pgfpathcurveto{\pgfqpoint{6.271612in}{4.409499in}}{\pgfqpoint{6.282211in}{4.405108in}}{\pgfqpoint{6.293261in}{4.405108in}}%
\pgfpathclose%
\pgfusepath{stroke,fill}%
\end{pgfscope}%
\begin{pgfscope}%
\pgfpathrectangle{\pgfqpoint{0.481978in}{0.331635in}}{\pgfqpoint{9.300000in}{7.700000in}}%
\pgfusepath{clip}%
\pgfsetbuttcap%
\pgfsetroundjoin%
\definecolor{currentfill}{rgb}{0.631373,0.788235,0.956863}%
\pgfsetfillcolor{currentfill}%
\pgfsetlinewidth{0.481800pt}%
\definecolor{currentstroke}{rgb}{1.000000,1.000000,1.000000}%
\pgfsetstrokecolor{currentstroke}%
\pgfsetdash{}{0pt}%
\pgfpathmoveto{\pgfqpoint{4.855717in}{7.162437in}}%
\pgfpathcurveto{\pgfqpoint{4.866767in}{7.162437in}}{\pgfqpoint{4.877366in}{7.166827in}}{\pgfqpoint{4.885179in}{7.174640in}}%
\pgfpathcurveto{\pgfqpoint{4.892993in}{7.182454in}}{\pgfqpoint{4.897383in}{7.193053in}}{\pgfqpoint{4.897383in}{7.204103in}}%
\pgfpathcurveto{\pgfqpoint{4.897383in}{7.215153in}}{\pgfqpoint{4.892993in}{7.225752in}}{\pgfqpoint{4.885179in}{7.233566in}}%
\pgfpathcurveto{\pgfqpoint{4.877366in}{7.241380in}}{\pgfqpoint{4.866767in}{7.245770in}}{\pgfqpoint{4.855717in}{7.245770in}}%
\pgfpathcurveto{\pgfqpoint{4.844666in}{7.245770in}}{\pgfqpoint{4.834067in}{7.241380in}}{\pgfqpoint{4.826254in}{7.233566in}}%
\pgfpathcurveto{\pgfqpoint{4.818440in}{7.225752in}}{\pgfqpoint{4.814050in}{7.215153in}}{\pgfqpoint{4.814050in}{7.204103in}}%
\pgfpathcurveto{\pgfqpoint{4.814050in}{7.193053in}}{\pgfqpoint{4.818440in}{7.182454in}}{\pgfqpoint{4.826254in}{7.174640in}}%
\pgfpathcurveto{\pgfqpoint{4.834067in}{7.166827in}}{\pgfqpoint{4.844666in}{7.162437in}}{\pgfqpoint{4.855717in}{7.162437in}}%
\pgfpathclose%
\pgfusepath{stroke,fill}%
\end{pgfscope}%
\begin{pgfscope}%
\pgfpathrectangle{\pgfqpoint{0.481978in}{0.331635in}}{\pgfqpoint{9.300000in}{7.700000in}}%
\pgfusepath{clip}%
\pgfsetbuttcap%
\pgfsetroundjoin%
\definecolor{currentfill}{rgb}{0.631373,0.788235,0.956863}%
\pgfsetfillcolor{currentfill}%
\pgfsetlinewidth{0.481800pt}%
\definecolor{currentstroke}{rgb}{1.000000,1.000000,1.000000}%
\pgfsetstrokecolor{currentstroke}%
\pgfsetdash{}{0pt}%
\pgfpathmoveto{\pgfqpoint{8.228148in}{4.788343in}}%
\pgfpathcurveto{\pgfqpoint{8.239199in}{4.788343in}}{\pgfqpoint{8.249798in}{4.792734in}}{\pgfqpoint{8.257611in}{4.800547in}}%
\pgfpathcurveto{\pgfqpoint{8.265425in}{4.808361in}}{\pgfqpoint{8.269815in}{4.818960in}}{\pgfqpoint{8.269815in}{4.830010in}}%
\pgfpathcurveto{\pgfqpoint{8.269815in}{4.841060in}}{\pgfqpoint{8.265425in}{4.851659in}}{\pgfqpoint{8.257611in}{4.859473in}}%
\pgfpathcurveto{\pgfqpoint{8.249798in}{4.867286in}}{\pgfqpoint{8.239199in}{4.871677in}}{\pgfqpoint{8.228148in}{4.871677in}}%
\pgfpathcurveto{\pgfqpoint{8.217098in}{4.871677in}}{\pgfqpoint{8.206499in}{4.867286in}}{\pgfqpoint{8.198686in}{4.859473in}}%
\pgfpathcurveto{\pgfqpoint{8.190872in}{4.851659in}}{\pgfqpoint{8.186482in}{4.841060in}}{\pgfqpoint{8.186482in}{4.830010in}}%
\pgfpathcurveto{\pgfqpoint{8.186482in}{4.818960in}}{\pgfqpoint{8.190872in}{4.808361in}}{\pgfqpoint{8.198686in}{4.800547in}}%
\pgfpathcurveto{\pgfqpoint{8.206499in}{4.792734in}}{\pgfqpoint{8.217098in}{4.788343in}}{\pgfqpoint{8.228148in}{4.788343in}}%
\pgfpathclose%
\pgfusepath{stroke,fill}%
\end{pgfscope}%
\begin{pgfscope}%
\pgfpathrectangle{\pgfqpoint{0.481978in}{0.331635in}}{\pgfqpoint{9.300000in}{7.700000in}}%
\pgfusepath{clip}%
\pgfsetbuttcap%
\pgfsetroundjoin%
\definecolor{currentfill}{rgb}{0.631373,0.788235,0.956863}%
\pgfsetfillcolor{currentfill}%
\pgfsetlinewidth{0.481800pt}%
\definecolor{currentstroke}{rgb}{1.000000,1.000000,1.000000}%
\pgfsetstrokecolor{currentstroke}%
\pgfsetdash{}{0pt}%
\pgfpathmoveto{\pgfqpoint{7.375231in}{1.910568in}}%
\pgfpathcurveto{\pgfqpoint{7.386281in}{1.910568in}}{\pgfqpoint{7.396880in}{1.914958in}}{\pgfqpoint{7.404694in}{1.922772in}}%
\pgfpathcurveto{\pgfqpoint{7.412507in}{1.930585in}}{\pgfqpoint{7.416898in}{1.941184in}}{\pgfqpoint{7.416898in}{1.952235in}}%
\pgfpathcurveto{\pgfqpoint{7.416898in}{1.963285in}}{\pgfqpoint{7.412507in}{1.973884in}}{\pgfqpoint{7.404694in}{1.981697in}}%
\pgfpathcurveto{\pgfqpoint{7.396880in}{1.989511in}}{\pgfqpoint{7.386281in}{1.993901in}}{\pgfqpoint{7.375231in}{1.993901in}}%
\pgfpathcurveto{\pgfqpoint{7.364181in}{1.993901in}}{\pgfqpoint{7.353582in}{1.989511in}}{\pgfqpoint{7.345768in}{1.981697in}}%
\pgfpathcurveto{\pgfqpoint{7.337954in}{1.973884in}}{\pgfqpoint{7.333564in}{1.963285in}}{\pgfqpoint{7.333564in}{1.952235in}}%
\pgfpathcurveto{\pgfqpoint{7.333564in}{1.941184in}}{\pgfqpoint{7.337954in}{1.930585in}}{\pgfqpoint{7.345768in}{1.922772in}}%
\pgfpathcurveto{\pgfqpoint{7.353582in}{1.914958in}}{\pgfqpoint{7.364181in}{1.910568in}}{\pgfqpoint{7.375231in}{1.910568in}}%
\pgfpathclose%
\pgfusepath{stroke,fill}%
\end{pgfscope}%
\begin{pgfscope}%
\pgfpathrectangle{\pgfqpoint{0.481978in}{0.331635in}}{\pgfqpoint{9.300000in}{7.700000in}}%
\pgfusepath{clip}%
\pgfsetbuttcap%
\pgfsetroundjoin%
\definecolor{currentfill}{rgb}{0.631373,0.788235,0.956863}%
\pgfsetfillcolor{currentfill}%
\pgfsetlinewidth{0.481800pt}%
\definecolor{currentstroke}{rgb}{1.000000,1.000000,1.000000}%
\pgfsetstrokecolor{currentstroke}%
\pgfsetdash{}{0pt}%
\pgfpathmoveto{\pgfqpoint{6.949155in}{1.608307in}}%
\pgfpathcurveto{\pgfqpoint{6.960205in}{1.608307in}}{\pgfqpoint{6.970804in}{1.612697in}}{\pgfqpoint{6.978618in}{1.620511in}}%
\pgfpathcurveto{\pgfqpoint{6.986431in}{1.628324in}}{\pgfqpoint{6.990822in}{1.638924in}}{\pgfqpoint{6.990822in}{1.649974in}}%
\pgfpathcurveto{\pgfqpoint{6.990822in}{1.661024in}}{\pgfqpoint{6.986431in}{1.671623in}}{\pgfqpoint{6.978618in}{1.679436in}}%
\pgfpathcurveto{\pgfqpoint{6.970804in}{1.687250in}}{\pgfqpoint{6.960205in}{1.691640in}}{\pgfqpoint{6.949155in}{1.691640in}}%
\pgfpathcurveto{\pgfqpoint{6.938105in}{1.691640in}}{\pgfqpoint{6.927506in}{1.687250in}}{\pgfqpoint{6.919692in}{1.679436in}}%
\pgfpathcurveto{\pgfqpoint{6.911879in}{1.671623in}}{\pgfqpoint{6.907488in}{1.661024in}}{\pgfqpoint{6.907488in}{1.649974in}}%
\pgfpathcurveto{\pgfqpoint{6.907488in}{1.638924in}}{\pgfqpoint{6.911879in}{1.628324in}}{\pgfqpoint{6.919692in}{1.620511in}}%
\pgfpathcurveto{\pgfqpoint{6.927506in}{1.612697in}}{\pgfqpoint{6.938105in}{1.608307in}}{\pgfqpoint{6.949155in}{1.608307in}}%
\pgfpathclose%
\pgfusepath{stroke,fill}%
\end{pgfscope}%
\begin{pgfscope}%
\pgfpathrectangle{\pgfqpoint{0.481978in}{0.331635in}}{\pgfqpoint{9.300000in}{7.700000in}}%
\pgfusepath{clip}%
\pgfsetbuttcap%
\pgfsetroundjoin%
\definecolor{currentfill}{rgb}{0.631373,0.788235,0.956863}%
\pgfsetfillcolor{currentfill}%
\pgfsetlinewidth{0.481800pt}%
\definecolor{currentstroke}{rgb}{1.000000,1.000000,1.000000}%
\pgfsetstrokecolor{currentstroke}%
\pgfsetdash{}{0pt}%
\pgfpathmoveto{\pgfqpoint{7.251466in}{2.478396in}}%
\pgfpathcurveto{\pgfqpoint{7.262516in}{2.478396in}}{\pgfqpoint{7.273115in}{2.482786in}}{\pgfqpoint{7.280929in}{2.490600in}}%
\pgfpathcurveto{\pgfqpoint{7.288742in}{2.498413in}}{\pgfqpoint{7.293133in}{2.509012in}}{\pgfqpoint{7.293133in}{2.520062in}}%
\pgfpathcurveto{\pgfqpoint{7.293133in}{2.531113in}}{\pgfqpoint{7.288742in}{2.541712in}}{\pgfqpoint{7.280929in}{2.549525in}}%
\pgfpathcurveto{\pgfqpoint{7.273115in}{2.557339in}}{\pgfqpoint{7.262516in}{2.561729in}}{\pgfqpoint{7.251466in}{2.561729in}}%
\pgfpathcurveto{\pgfqpoint{7.240416in}{2.561729in}}{\pgfqpoint{7.229817in}{2.557339in}}{\pgfqpoint{7.222003in}{2.549525in}}%
\pgfpathcurveto{\pgfqpoint{7.214189in}{2.541712in}}{\pgfqpoint{7.209799in}{2.531113in}}{\pgfqpoint{7.209799in}{2.520062in}}%
\pgfpathcurveto{\pgfqpoint{7.209799in}{2.509012in}}{\pgfqpoint{7.214189in}{2.498413in}}{\pgfqpoint{7.222003in}{2.490600in}}%
\pgfpathcurveto{\pgfqpoint{7.229817in}{2.482786in}}{\pgfqpoint{7.240416in}{2.478396in}}{\pgfqpoint{7.251466in}{2.478396in}}%
\pgfpathclose%
\pgfusepath{stroke,fill}%
\end{pgfscope}%
\begin{pgfscope}%
\pgfpathrectangle{\pgfqpoint{0.481978in}{0.331635in}}{\pgfqpoint{9.300000in}{7.700000in}}%
\pgfusepath{clip}%
\pgfsetbuttcap%
\pgfsetroundjoin%
\definecolor{currentfill}{rgb}{0.631373,0.788235,0.956863}%
\pgfsetfillcolor{currentfill}%
\pgfsetlinewidth{0.481800pt}%
\definecolor{currentstroke}{rgb}{1.000000,1.000000,1.000000}%
\pgfsetstrokecolor{currentstroke}%
\pgfsetdash{}{0pt}%
\pgfpathmoveto{\pgfqpoint{5.500453in}{2.132128in}}%
\pgfpathcurveto{\pgfqpoint{5.511503in}{2.132128in}}{\pgfqpoint{5.522102in}{2.136518in}}{\pgfqpoint{5.529915in}{2.144332in}}%
\pgfpathcurveto{\pgfqpoint{5.537729in}{2.152146in}}{\pgfqpoint{5.542119in}{2.162745in}}{\pgfqpoint{5.542119in}{2.173795in}}%
\pgfpathcurveto{\pgfqpoint{5.542119in}{2.184845in}}{\pgfqpoint{5.537729in}{2.195444in}}{\pgfqpoint{5.529915in}{2.203258in}}%
\pgfpathcurveto{\pgfqpoint{5.522102in}{2.211071in}}{\pgfqpoint{5.511503in}{2.215461in}}{\pgfqpoint{5.500453in}{2.215461in}}%
\pgfpathcurveto{\pgfqpoint{5.489402in}{2.215461in}}{\pgfqpoint{5.478803in}{2.211071in}}{\pgfqpoint{5.470990in}{2.203258in}}%
\pgfpathcurveto{\pgfqpoint{5.463176in}{2.195444in}}{\pgfqpoint{5.458786in}{2.184845in}}{\pgfqpoint{5.458786in}{2.173795in}}%
\pgfpathcurveto{\pgfqpoint{5.458786in}{2.162745in}}{\pgfqpoint{5.463176in}{2.152146in}}{\pgfqpoint{5.470990in}{2.144332in}}%
\pgfpathcurveto{\pgfqpoint{5.478803in}{2.136518in}}{\pgfqpoint{5.489402in}{2.132128in}}{\pgfqpoint{5.500453in}{2.132128in}}%
\pgfpathclose%
\pgfusepath{stroke,fill}%
\end{pgfscope}%
\begin{pgfscope}%
\pgfpathrectangle{\pgfqpoint{0.481978in}{0.331635in}}{\pgfqpoint{9.300000in}{7.700000in}}%
\pgfusepath{clip}%
\pgfsetbuttcap%
\pgfsetroundjoin%
\definecolor{currentfill}{rgb}{0.631373,0.788235,0.956863}%
\pgfsetfillcolor{currentfill}%
\pgfsetlinewidth{0.481800pt}%
\definecolor{currentstroke}{rgb}{1.000000,1.000000,1.000000}%
\pgfsetstrokecolor{currentstroke}%
\pgfsetdash{}{0pt}%
\pgfpathmoveto{\pgfqpoint{6.149200in}{2.277817in}}%
\pgfpathcurveto{\pgfqpoint{6.160250in}{2.277817in}}{\pgfqpoint{6.170849in}{2.282207in}}{\pgfqpoint{6.178663in}{2.290021in}}%
\pgfpathcurveto{\pgfqpoint{6.186476in}{2.297835in}}{\pgfqpoint{6.190867in}{2.308434in}}{\pgfqpoint{6.190867in}{2.319484in}}%
\pgfpathcurveto{\pgfqpoint{6.190867in}{2.330534in}}{\pgfqpoint{6.186476in}{2.341133in}}{\pgfqpoint{6.178663in}{2.348947in}}%
\pgfpathcurveto{\pgfqpoint{6.170849in}{2.356760in}}{\pgfqpoint{6.160250in}{2.361150in}}{\pgfqpoint{6.149200in}{2.361150in}}%
\pgfpathcurveto{\pgfqpoint{6.138150in}{2.361150in}}{\pgfqpoint{6.127551in}{2.356760in}}{\pgfqpoint{6.119737in}{2.348947in}}%
\pgfpathcurveto{\pgfqpoint{6.111923in}{2.341133in}}{\pgfqpoint{6.107533in}{2.330534in}}{\pgfqpoint{6.107533in}{2.319484in}}%
\pgfpathcurveto{\pgfqpoint{6.107533in}{2.308434in}}{\pgfqpoint{6.111923in}{2.297835in}}{\pgfqpoint{6.119737in}{2.290021in}}%
\pgfpathcurveto{\pgfqpoint{6.127551in}{2.282207in}}{\pgfqpoint{6.138150in}{2.277817in}}{\pgfqpoint{6.149200in}{2.277817in}}%
\pgfpathclose%
\pgfusepath{stroke,fill}%
\end{pgfscope}%
\begin{pgfscope}%
\pgfpathrectangle{\pgfqpoint{0.481978in}{0.331635in}}{\pgfqpoint{9.300000in}{7.700000in}}%
\pgfusepath{clip}%
\pgfsetbuttcap%
\pgfsetroundjoin%
\definecolor{currentfill}{rgb}{0.631373,0.788235,0.956863}%
\pgfsetfillcolor{currentfill}%
\pgfsetlinewidth{0.481800pt}%
\definecolor{currentstroke}{rgb}{1.000000,1.000000,1.000000}%
\pgfsetstrokecolor{currentstroke}%
\pgfsetdash{}{0pt}%
\pgfpathmoveto{\pgfqpoint{2.951748in}{1.113137in}}%
\pgfpathcurveto{\pgfqpoint{2.962798in}{1.113137in}}{\pgfqpoint{2.973397in}{1.117527in}}{\pgfqpoint{2.981211in}{1.125341in}}%
\pgfpathcurveto{\pgfqpoint{2.989024in}{1.133155in}}{\pgfqpoint{2.993415in}{1.143754in}}{\pgfqpoint{2.993415in}{1.154804in}}%
\pgfpathcurveto{\pgfqpoint{2.993415in}{1.165854in}}{\pgfqpoint{2.989024in}{1.176453in}}{\pgfqpoint{2.981211in}{1.184267in}}%
\pgfpathcurveto{\pgfqpoint{2.973397in}{1.192080in}}{\pgfqpoint{2.962798in}{1.196471in}}{\pgfqpoint{2.951748in}{1.196471in}}%
\pgfpathcurveto{\pgfqpoint{2.940698in}{1.196471in}}{\pgfqpoint{2.930099in}{1.192080in}}{\pgfqpoint{2.922285in}{1.184267in}}%
\pgfpathcurveto{\pgfqpoint{2.914471in}{1.176453in}}{\pgfqpoint{2.910081in}{1.165854in}}{\pgfqpoint{2.910081in}{1.154804in}}%
\pgfpathcurveto{\pgfqpoint{2.910081in}{1.143754in}}{\pgfqpoint{2.914471in}{1.133155in}}{\pgfqpoint{2.922285in}{1.125341in}}%
\pgfpathcurveto{\pgfqpoint{2.930099in}{1.117527in}}{\pgfqpoint{2.940698in}{1.113137in}}{\pgfqpoint{2.951748in}{1.113137in}}%
\pgfpathclose%
\pgfusepath{stroke,fill}%
\end{pgfscope}%
\begin{pgfscope}%
\pgfpathrectangle{\pgfqpoint{0.481978in}{0.331635in}}{\pgfqpoint{9.300000in}{7.700000in}}%
\pgfusepath{clip}%
\pgfsetbuttcap%
\pgfsetroundjoin%
\definecolor{currentfill}{rgb}{0.631373,0.788235,0.956863}%
\pgfsetfillcolor{currentfill}%
\pgfsetlinewidth{0.481800pt}%
\definecolor{currentstroke}{rgb}{1.000000,1.000000,1.000000}%
\pgfsetstrokecolor{currentstroke}%
\pgfsetdash{}{0pt}%
\pgfpathmoveto{\pgfqpoint{6.416375in}{3.193658in}}%
\pgfpathcurveto{\pgfqpoint{6.427425in}{3.193658in}}{\pgfqpoint{6.438024in}{3.198048in}}{\pgfqpoint{6.445838in}{3.205862in}}%
\pgfpathcurveto{\pgfqpoint{6.453651in}{3.213676in}}{\pgfqpoint{6.458041in}{3.224275in}}{\pgfqpoint{6.458041in}{3.235325in}}%
\pgfpathcurveto{\pgfqpoint{6.458041in}{3.246375in}}{\pgfqpoint{6.453651in}{3.256974in}}{\pgfqpoint{6.445838in}{3.264788in}}%
\pgfpathcurveto{\pgfqpoint{6.438024in}{3.272601in}}{\pgfqpoint{6.427425in}{3.276991in}}{\pgfqpoint{6.416375in}{3.276991in}}%
\pgfpathcurveto{\pgfqpoint{6.405325in}{3.276991in}}{\pgfqpoint{6.394726in}{3.272601in}}{\pgfqpoint{6.386912in}{3.264788in}}%
\pgfpathcurveto{\pgfqpoint{6.379098in}{3.256974in}}{\pgfqpoint{6.374708in}{3.246375in}}{\pgfqpoint{6.374708in}{3.235325in}}%
\pgfpathcurveto{\pgfqpoint{6.374708in}{3.224275in}}{\pgfqpoint{6.379098in}{3.213676in}}{\pgfqpoint{6.386912in}{3.205862in}}%
\pgfpathcurveto{\pgfqpoint{6.394726in}{3.198048in}}{\pgfqpoint{6.405325in}{3.193658in}}{\pgfqpoint{6.416375in}{3.193658in}}%
\pgfpathclose%
\pgfusepath{stroke,fill}%
\end{pgfscope}%
\begin{pgfscope}%
\pgfpathrectangle{\pgfqpoint{0.481978in}{0.331635in}}{\pgfqpoint{9.300000in}{7.700000in}}%
\pgfusepath{clip}%
\pgfsetbuttcap%
\pgfsetroundjoin%
\definecolor{currentfill}{rgb}{0.631373,0.788235,0.956863}%
\pgfsetfillcolor{currentfill}%
\pgfsetlinewidth{0.481800pt}%
\definecolor{currentstroke}{rgb}{1.000000,1.000000,1.000000}%
\pgfsetstrokecolor{currentstroke}%
\pgfsetdash{}{0pt}%
\pgfpathmoveto{\pgfqpoint{2.918703in}{1.224111in}}%
\pgfpathcurveto{\pgfqpoint{2.929753in}{1.224111in}}{\pgfqpoint{2.940352in}{1.228501in}}{\pgfqpoint{2.948166in}{1.236315in}}%
\pgfpathcurveto{\pgfqpoint{2.955979in}{1.244128in}}{\pgfqpoint{2.960370in}{1.254727in}}{\pgfqpoint{2.960370in}{1.265777in}}%
\pgfpathcurveto{\pgfqpoint{2.960370in}{1.276827in}}{\pgfqpoint{2.955979in}{1.287427in}}{\pgfqpoint{2.948166in}{1.295240in}}%
\pgfpathcurveto{\pgfqpoint{2.940352in}{1.303054in}}{\pgfqpoint{2.929753in}{1.307444in}}{\pgfqpoint{2.918703in}{1.307444in}}%
\pgfpathcurveto{\pgfqpoint{2.907653in}{1.307444in}}{\pgfqpoint{2.897054in}{1.303054in}}{\pgfqpoint{2.889240in}{1.295240in}}%
\pgfpathcurveto{\pgfqpoint{2.881427in}{1.287427in}}{\pgfqpoint{2.877036in}{1.276827in}}{\pgfqpoint{2.877036in}{1.265777in}}%
\pgfpathcurveto{\pgfqpoint{2.877036in}{1.254727in}}{\pgfqpoint{2.881427in}{1.244128in}}{\pgfqpoint{2.889240in}{1.236315in}}%
\pgfpathcurveto{\pgfqpoint{2.897054in}{1.228501in}}{\pgfqpoint{2.907653in}{1.224111in}}{\pgfqpoint{2.918703in}{1.224111in}}%
\pgfpathclose%
\pgfusepath{stroke,fill}%
\end{pgfscope}%
\begin{pgfscope}%
\pgfpathrectangle{\pgfqpoint{0.481978in}{0.331635in}}{\pgfqpoint{9.300000in}{7.700000in}}%
\pgfusepath{clip}%
\pgfsetbuttcap%
\pgfsetroundjoin%
\definecolor{currentfill}{rgb}{0.631373,0.788235,0.956863}%
\pgfsetfillcolor{currentfill}%
\pgfsetlinewidth{0.481800pt}%
\definecolor{currentstroke}{rgb}{1.000000,1.000000,1.000000}%
\pgfsetstrokecolor{currentstroke}%
\pgfsetdash{}{0pt}%
\pgfpathmoveto{\pgfqpoint{2.176114in}{5.829516in}}%
\pgfpathcurveto{\pgfqpoint{2.187164in}{5.829516in}}{\pgfqpoint{2.197764in}{5.833906in}}{\pgfqpoint{2.205577in}{5.841720in}}%
\pgfpathcurveto{\pgfqpoint{2.213391in}{5.849534in}}{\pgfqpoint{2.217781in}{5.860133in}}{\pgfqpoint{2.217781in}{5.871183in}}%
\pgfpathcurveto{\pgfqpoint{2.217781in}{5.882233in}}{\pgfqpoint{2.213391in}{5.892832in}}{\pgfqpoint{2.205577in}{5.900646in}}%
\pgfpathcurveto{\pgfqpoint{2.197764in}{5.908459in}}{\pgfqpoint{2.187164in}{5.912849in}}{\pgfqpoint{2.176114in}{5.912849in}}%
\pgfpathcurveto{\pgfqpoint{2.165064in}{5.912849in}}{\pgfqpoint{2.154465in}{5.908459in}}{\pgfqpoint{2.146652in}{5.900646in}}%
\pgfpathcurveto{\pgfqpoint{2.138838in}{5.892832in}}{\pgfqpoint{2.134448in}{5.882233in}}{\pgfqpoint{2.134448in}{5.871183in}}%
\pgfpathcurveto{\pgfqpoint{2.134448in}{5.860133in}}{\pgfqpoint{2.138838in}{5.849534in}}{\pgfqpoint{2.146652in}{5.841720in}}%
\pgfpathcurveto{\pgfqpoint{2.154465in}{5.833906in}}{\pgfqpoint{2.165064in}{5.829516in}}{\pgfqpoint{2.176114in}{5.829516in}}%
\pgfpathclose%
\pgfusepath{stroke,fill}%
\end{pgfscope}%
\begin{pgfscope}%
\pgfpathrectangle{\pgfqpoint{0.481978in}{0.331635in}}{\pgfqpoint{9.300000in}{7.700000in}}%
\pgfusepath{clip}%
\pgfsetbuttcap%
\pgfsetroundjoin%
\definecolor{currentfill}{rgb}{0.631373,0.788235,0.956863}%
\pgfsetfillcolor{currentfill}%
\pgfsetlinewidth{0.481800pt}%
\definecolor{currentstroke}{rgb}{1.000000,1.000000,1.000000}%
\pgfsetstrokecolor{currentstroke}%
\pgfsetdash{}{0pt}%
\pgfpathmoveto{\pgfqpoint{6.830655in}{1.534579in}}%
\pgfpathcurveto{\pgfqpoint{6.841705in}{1.534579in}}{\pgfqpoint{6.852304in}{1.538970in}}{\pgfqpoint{6.860118in}{1.546783in}}%
\pgfpathcurveto{\pgfqpoint{6.867931in}{1.554597in}}{\pgfqpoint{6.872321in}{1.565196in}}{\pgfqpoint{6.872321in}{1.576246in}}%
\pgfpathcurveto{\pgfqpoint{6.872321in}{1.587296in}}{\pgfqpoint{6.867931in}{1.597895in}}{\pgfqpoint{6.860118in}{1.605709in}}%
\pgfpathcurveto{\pgfqpoint{6.852304in}{1.613523in}}{\pgfqpoint{6.841705in}{1.617913in}}{\pgfqpoint{6.830655in}{1.617913in}}%
\pgfpathcurveto{\pgfqpoint{6.819605in}{1.617913in}}{\pgfqpoint{6.809006in}{1.613523in}}{\pgfqpoint{6.801192in}{1.605709in}}%
\pgfpathcurveto{\pgfqpoint{6.793378in}{1.597895in}}{\pgfqpoint{6.788988in}{1.587296in}}{\pgfqpoint{6.788988in}{1.576246in}}%
\pgfpathcurveto{\pgfqpoint{6.788988in}{1.565196in}}{\pgfqpoint{6.793378in}{1.554597in}}{\pgfqpoint{6.801192in}{1.546783in}}%
\pgfpathcurveto{\pgfqpoint{6.809006in}{1.538970in}}{\pgfqpoint{6.819605in}{1.534579in}}{\pgfqpoint{6.830655in}{1.534579in}}%
\pgfpathclose%
\pgfusepath{stroke,fill}%
\end{pgfscope}%
\begin{pgfscope}%
\pgfpathrectangle{\pgfqpoint{0.481978in}{0.331635in}}{\pgfqpoint{9.300000in}{7.700000in}}%
\pgfusepath{clip}%
\pgfsetbuttcap%
\pgfsetroundjoin%
\definecolor{currentfill}{rgb}{0.631373,0.788235,0.956863}%
\pgfsetfillcolor{currentfill}%
\pgfsetlinewidth{0.481800pt}%
\definecolor{currentstroke}{rgb}{1.000000,1.000000,1.000000}%
\pgfsetstrokecolor{currentstroke}%
\pgfsetdash{}{0pt}%
\pgfpathmoveto{\pgfqpoint{6.368029in}{1.338243in}}%
\pgfpathcurveto{\pgfqpoint{6.379079in}{1.338243in}}{\pgfqpoint{6.389678in}{1.342633in}}{\pgfqpoint{6.397492in}{1.350447in}}%
\pgfpathcurveto{\pgfqpoint{6.405305in}{1.358260in}}{\pgfqpoint{6.409696in}{1.368859in}}{\pgfqpoint{6.409696in}{1.379909in}}%
\pgfpathcurveto{\pgfqpoint{6.409696in}{1.390959in}}{\pgfqpoint{6.405305in}{1.401558in}}{\pgfqpoint{6.397492in}{1.409372in}}%
\pgfpathcurveto{\pgfqpoint{6.389678in}{1.417186in}}{\pgfqpoint{6.379079in}{1.421576in}}{\pgfqpoint{6.368029in}{1.421576in}}%
\pgfpathcurveto{\pgfqpoint{6.356979in}{1.421576in}}{\pgfqpoint{6.346380in}{1.417186in}}{\pgfqpoint{6.338566in}{1.409372in}}%
\pgfpathcurveto{\pgfqpoint{6.330753in}{1.401558in}}{\pgfqpoint{6.326362in}{1.390959in}}{\pgfqpoint{6.326362in}{1.379909in}}%
\pgfpathcurveto{\pgfqpoint{6.326362in}{1.368859in}}{\pgfqpoint{6.330753in}{1.358260in}}{\pgfqpoint{6.338566in}{1.350447in}}%
\pgfpathcurveto{\pgfqpoint{6.346380in}{1.342633in}}{\pgfqpoint{6.356979in}{1.338243in}}{\pgfqpoint{6.368029in}{1.338243in}}%
\pgfpathclose%
\pgfusepath{stroke,fill}%
\end{pgfscope}%
\begin{pgfscope}%
\pgfpathrectangle{\pgfqpoint{0.481978in}{0.331635in}}{\pgfqpoint{9.300000in}{7.700000in}}%
\pgfusepath{clip}%
\pgfsetbuttcap%
\pgfsetroundjoin%
\definecolor{currentfill}{rgb}{0.631373,0.788235,0.956863}%
\pgfsetfillcolor{currentfill}%
\pgfsetlinewidth{0.481800pt}%
\definecolor{currentstroke}{rgb}{1.000000,1.000000,1.000000}%
\pgfsetstrokecolor{currentstroke}%
\pgfsetdash{}{0pt}%
\pgfpathmoveto{\pgfqpoint{7.095890in}{2.657676in}}%
\pgfpathcurveto{\pgfqpoint{7.106940in}{2.657676in}}{\pgfqpoint{7.117539in}{2.662066in}}{\pgfqpoint{7.125353in}{2.669880in}}%
\pgfpathcurveto{\pgfqpoint{7.133166in}{2.677693in}}{\pgfqpoint{7.137557in}{2.688293in}}{\pgfqpoint{7.137557in}{2.699343in}}%
\pgfpathcurveto{\pgfqpoint{7.137557in}{2.710393in}}{\pgfqpoint{7.133166in}{2.720992in}}{\pgfqpoint{7.125353in}{2.728805in}}%
\pgfpathcurveto{\pgfqpoint{7.117539in}{2.736619in}}{\pgfqpoint{7.106940in}{2.741009in}}{\pgfqpoint{7.095890in}{2.741009in}}%
\pgfpathcurveto{\pgfqpoint{7.084840in}{2.741009in}}{\pgfqpoint{7.074241in}{2.736619in}}{\pgfqpoint{7.066427in}{2.728805in}}%
\pgfpathcurveto{\pgfqpoint{7.058613in}{2.720992in}}{\pgfqpoint{7.054223in}{2.710393in}}{\pgfqpoint{7.054223in}{2.699343in}}%
\pgfpathcurveto{\pgfqpoint{7.054223in}{2.688293in}}{\pgfqpoint{7.058613in}{2.677693in}}{\pgfqpoint{7.066427in}{2.669880in}}%
\pgfpathcurveto{\pgfqpoint{7.074241in}{2.662066in}}{\pgfqpoint{7.084840in}{2.657676in}}{\pgfqpoint{7.095890in}{2.657676in}}%
\pgfpathclose%
\pgfusepath{stroke,fill}%
\end{pgfscope}%
\begin{pgfscope}%
\pgfpathrectangle{\pgfqpoint{0.481978in}{0.331635in}}{\pgfqpoint{9.300000in}{7.700000in}}%
\pgfusepath{clip}%
\pgfsetbuttcap%
\pgfsetroundjoin%
\definecolor{currentfill}{rgb}{0.631373,0.788235,0.956863}%
\pgfsetfillcolor{currentfill}%
\pgfsetlinewidth{0.481800pt}%
\definecolor{currentstroke}{rgb}{1.000000,1.000000,1.000000}%
\pgfsetstrokecolor{currentstroke}%
\pgfsetdash{}{0pt}%
\pgfpathmoveto{\pgfqpoint{5.900690in}{3.631899in}}%
\pgfpathcurveto{\pgfqpoint{5.911740in}{3.631899in}}{\pgfqpoint{5.922339in}{3.636289in}}{\pgfqpoint{5.930153in}{3.644103in}}%
\pgfpathcurveto{\pgfqpoint{5.937966in}{3.651916in}}{\pgfqpoint{5.942356in}{3.662515in}}{\pgfqpoint{5.942356in}{3.673565in}}%
\pgfpathcurveto{\pgfqpoint{5.942356in}{3.684616in}}{\pgfqpoint{5.937966in}{3.695215in}}{\pgfqpoint{5.930153in}{3.703028in}}%
\pgfpathcurveto{\pgfqpoint{5.922339in}{3.710842in}}{\pgfqpoint{5.911740in}{3.715232in}}{\pgfqpoint{5.900690in}{3.715232in}}%
\pgfpathcurveto{\pgfqpoint{5.889640in}{3.715232in}}{\pgfqpoint{5.879041in}{3.710842in}}{\pgfqpoint{5.871227in}{3.703028in}}%
\pgfpathcurveto{\pgfqpoint{5.863413in}{3.695215in}}{\pgfqpoint{5.859023in}{3.684616in}}{\pgfqpoint{5.859023in}{3.673565in}}%
\pgfpathcurveto{\pgfqpoint{5.859023in}{3.662515in}}{\pgfqpoint{5.863413in}{3.651916in}}{\pgfqpoint{5.871227in}{3.644103in}}%
\pgfpathcurveto{\pgfqpoint{5.879041in}{3.636289in}}{\pgfqpoint{5.889640in}{3.631899in}}{\pgfqpoint{5.900690in}{3.631899in}}%
\pgfpathclose%
\pgfusepath{stroke,fill}%
\end{pgfscope}%
\begin{pgfscope}%
\pgfpathrectangle{\pgfqpoint{0.481978in}{0.331635in}}{\pgfqpoint{9.300000in}{7.700000in}}%
\pgfusepath{clip}%
\pgfsetbuttcap%
\pgfsetroundjoin%
\definecolor{currentfill}{rgb}{0.631373,0.788235,0.956863}%
\pgfsetfillcolor{currentfill}%
\pgfsetlinewidth{0.481800pt}%
\definecolor{currentstroke}{rgb}{1.000000,1.000000,1.000000}%
\pgfsetstrokecolor{currentstroke}%
\pgfsetdash{}{0pt}%
\pgfpathmoveto{\pgfqpoint{6.724845in}{4.569221in}}%
\pgfpathcurveto{\pgfqpoint{6.735895in}{4.569221in}}{\pgfqpoint{6.746495in}{4.573611in}}{\pgfqpoint{6.754308in}{4.581424in}}%
\pgfpathcurveto{\pgfqpoint{6.762122in}{4.589238in}}{\pgfqpoint{6.766512in}{4.599837in}}{\pgfqpoint{6.766512in}{4.610887in}}%
\pgfpathcurveto{\pgfqpoint{6.766512in}{4.621937in}}{\pgfqpoint{6.762122in}{4.632536in}}{\pgfqpoint{6.754308in}{4.640350in}}%
\pgfpathcurveto{\pgfqpoint{6.746495in}{4.648164in}}{\pgfqpoint{6.735895in}{4.652554in}}{\pgfqpoint{6.724845in}{4.652554in}}%
\pgfpathcurveto{\pgfqpoint{6.713795in}{4.652554in}}{\pgfqpoint{6.703196in}{4.648164in}}{\pgfqpoint{6.695383in}{4.640350in}}%
\pgfpathcurveto{\pgfqpoint{6.687569in}{4.632536in}}{\pgfqpoint{6.683179in}{4.621937in}}{\pgfqpoint{6.683179in}{4.610887in}}%
\pgfpathcurveto{\pgfqpoint{6.683179in}{4.599837in}}{\pgfqpoint{6.687569in}{4.589238in}}{\pgfqpoint{6.695383in}{4.581424in}}%
\pgfpathcurveto{\pgfqpoint{6.703196in}{4.573611in}}{\pgfqpoint{6.713795in}{4.569221in}}{\pgfqpoint{6.724845in}{4.569221in}}%
\pgfpathclose%
\pgfusepath{stroke,fill}%
\end{pgfscope}%
\begin{pgfscope}%
\pgfpathrectangle{\pgfqpoint{0.481978in}{0.331635in}}{\pgfqpoint{9.300000in}{7.700000in}}%
\pgfusepath{clip}%
\pgfsetbuttcap%
\pgfsetroundjoin%
\definecolor{currentfill}{rgb}{0.631373,0.788235,0.956863}%
\pgfsetfillcolor{currentfill}%
\pgfsetlinewidth{0.481800pt}%
\definecolor{currentstroke}{rgb}{1.000000,1.000000,1.000000}%
\pgfsetstrokecolor{currentstroke}%
\pgfsetdash{}{0pt}%
\pgfpathmoveto{\pgfqpoint{8.573001in}{5.110832in}}%
\pgfpathcurveto{\pgfqpoint{8.584051in}{5.110832in}}{\pgfqpoint{8.594650in}{5.115222in}}{\pgfqpoint{8.602463in}{5.123036in}}%
\pgfpathcurveto{\pgfqpoint{8.610277in}{5.130849in}}{\pgfqpoint{8.614667in}{5.141448in}}{\pgfqpoint{8.614667in}{5.152498in}}%
\pgfpathcurveto{\pgfqpoint{8.614667in}{5.163549in}}{\pgfqpoint{8.610277in}{5.174148in}}{\pgfqpoint{8.602463in}{5.181961in}}%
\pgfpathcurveto{\pgfqpoint{8.594650in}{5.189775in}}{\pgfqpoint{8.584051in}{5.194165in}}{\pgfqpoint{8.573001in}{5.194165in}}%
\pgfpathcurveto{\pgfqpoint{8.561950in}{5.194165in}}{\pgfqpoint{8.551351in}{5.189775in}}{\pgfqpoint{8.543538in}{5.181961in}}%
\pgfpathcurveto{\pgfqpoint{8.535724in}{5.174148in}}{\pgfqpoint{8.531334in}{5.163549in}}{\pgfqpoint{8.531334in}{5.152498in}}%
\pgfpathcurveto{\pgfqpoint{8.531334in}{5.141448in}}{\pgfqpoint{8.535724in}{5.130849in}}{\pgfqpoint{8.543538in}{5.123036in}}%
\pgfpathcurveto{\pgfqpoint{8.551351in}{5.115222in}}{\pgfqpoint{8.561950in}{5.110832in}}{\pgfqpoint{8.573001in}{5.110832in}}%
\pgfpathclose%
\pgfusepath{stroke,fill}%
\end{pgfscope}%
\begin{pgfscope}%
\pgfpathrectangle{\pgfqpoint{0.481978in}{0.331635in}}{\pgfqpoint{9.300000in}{7.700000in}}%
\pgfusepath{clip}%
\pgfsetbuttcap%
\pgfsetroundjoin%
\definecolor{currentfill}{rgb}{0.631373,0.788235,0.956863}%
\pgfsetfillcolor{currentfill}%
\pgfsetlinewidth{0.481800pt}%
\definecolor{currentstroke}{rgb}{1.000000,1.000000,1.000000}%
\pgfsetstrokecolor{currentstroke}%
\pgfsetdash{}{0pt}%
\pgfpathmoveto{\pgfqpoint{5.469587in}{6.991376in}}%
\pgfpathcurveto{\pgfqpoint{5.480637in}{6.991376in}}{\pgfqpoint{5.491236in}{6.995766in}}{\pgfqpoint{5.499050in}{7.003579in}}%
\pgfpathcurveto{\pgfqpoint{5.506863in}{7.011393in}}{\pgfqpoint{5.511253in}{7.021992in}}{\pgfqpoint{5.511253in}{7.033042in}}%
\pgfpathcurveto{\pgfqpoint{5.511253in}{7.044092in}}{\pgfqpoint{5.506863in}{7.054691in}}{\pgfqpoint{5.499050in}{7.062505in}}%
\pgfpathcurveto{\pgfqpoint{5.491236in}{7.070319in}}{\pgfqpoint{5.480637in}{7.074709in}}{\pgfqpoint{5.469587in}{7.074709in}}%
\pgfpathcurveto{\pgfqpoint{5.458537in}{7.074709in}}{\pgfqpoint{5.447938in}{7.070319in}}{\pgfqpoint{5.440124in}{7.062505in}}%
\pgfpathcurveto{\pgfqpoint{5.432310in}{7.054691in}}{\pgfqpoint{5.427920in}{7.044092in}}{\pgfqpoint{5.427920in}{7.033042in}}%
\pgfpathcurveto{\pgfqpoint{5.427920in}{7.021992in}}{\pgfqpoint{5.432310in}{7.011393in}}{\pgfqpoint{5.440124in}{7.003579in}}%
\pgfpathcurveto{\pgfqpoint{5.447938in}{6.995766in}}{\pgfqpoint{5.458537in}{6.991376in}}{\pgfqpoint{5.469587in}{6.991376in}}%
\pgfpathclose%
\pgfusepath{stroke,fill}%
\end{pgfscope}%
\begin{pgfscope}%
\pgfpathrectangle{\pgfqpoint{0.481978in}{0.331635in}}{\pgfqpoint{9.300000in}{7.700000in}}%
\pgfusepath{clip}%
\pgfsetbuttcap%
\pgfsetroundjoin%
\definecolor{currentfill}{rgb}{0.631373,0.788235,0.956863}%
\pgfsetfillcolor{currentfill}%
\pgfsetlinewidth{0.481800pt}%
\definecolor{currentstroke}{rgb}{1.000000,1.000000,1.000000}%
\pgfsetstrokecolor{currentstroke}%
\pgfsetdash{}{0pt}%
\pgfpathmoveto{\pgfqpoint{4.990883in}{4.317080in}}%
\pgfpathcurveto{\pgfqpoint{5.001933in}{4.317080in}}{\pgfqpoint{5.012532in}{4.321470in}}{\pgfqpoint{5.020346in}{4.329284in}}%
\pgfpathcurveto{\pgfqpoint{5.028160in}{4.337097in}}{\pgfqpoint{5.032550in}{4.347696in}}{\pgfqpoint{5.032550in}{4.358746in}}%
\pgfpathcurveto{\pgfqpoint{5.032550in}{4.369796in}}{\pgfqpoint{5.028160in}{4.380395in}}{\pgfqpoint{5.020346in}{4.388209in}}%
\pgfpathcurveto{\pgfqpoint{5.012532in}{4.396023in}}{\pgfqpoint{5.001933in}{4.400413in}}{\pgfqpoint{4.990883in}{4.400413in}}%
\pgfpathcurveto{\pgfqpoint{4.979833in}{4.400413in}}{\pgfqpoint{4.969234in}{4.396023in}}{\pgfqpoint{4.961420in}{4.388209in}}%
\pgfpathcurveto{\pgfqpoint{4.953607in}{4.380395in}}{\pgfqpoint{4.949217in}{4.369796in}}{\pgfqpoint{4.949217in}{4.358746in}}%
\pgfpathcurveto{\pgfqpoint{4.949217in}{4.347696in}}{\pgfqpoint{4.953607in}{4.337097in}}{\pgfqpoint{4.961420in}{4.329284in}}%
\pgfpathcurveto{\pgfqpoint{4.969234in}{4.321470in}}{\pgfqpoint{4.979833in}{4.317080in}}{\pgfqpoint{4.990883in}{4.317080in}}%
\pgfpathclose%
\pgfusepath{stroke,fill}%
\end{pgfscope}%
\begin{pgfscope}%
\pgfpathrectangle{\pgfqpoint{0.481978in}{0.331635in}}{\pgfqpoint{9.300000in}{7.700000in}}%
\pgfusepath{clip}%
\pgfsetbuttcap%
\pgfsetroundjoin%
\definecolor{currentfill}{rgb}{0.631373,0.788235,0.956863}%
\pgfsetfillcolor{currentfill}%
\pgfsetlinewidth{0.481800pt}%
\definecolor{currentstroke}{rgb}{1.000000,1.000000,1.000000}%
\pgfsetstrokecolor{currentstroke}%
\pgfsetdash{}{0pt}%
\pgfpathmoveto{\pgfqpoint{6.589980in}{1.645898in}}%
\pgfpathcurveto{\pgfqpoint{6.601030in}{1.645898in}}{\pgfqpoint{6.611629in}{1.650288in}}{\pgfqpoint{6.619443in}{1.658101in}}%
\pgfpathcurveto{\pgfqpoint{6.627256in}{1.665915in}}{\pgfqpoint{6.631647in}{1.676514in}}{\pgfqpoint{6.631647in}{1.687564in}}%
\pgfpathcurveto{\pgfqpoint{6.631647in}{1.698614in}}{\pgfqpoint{6.627256in}{1.709213in}}{\pgfqpoint{6.619443in}{1.717027in}}%
\pgfpathcurveto{\pgfqpoint{6.611629in}{1.724841in}}{\pgfqpoint{6.601030in}{1.729231in}}{\pgfqpoint{6.589980in}{1.729231in}}%
\pgfpathcurveto{\pgfqpoint{6.578930in}{1.729231in}}{\pgfqpoint{6.568331in}{1.724841in}}{\pgfqpoint{6.560517in}{1.717027in}}%
\pgfpathcurveto{\pgfqpoint{6.552704in}{1.709213in}}{\pgfqpoint{6.548313in}{1.698614in}}{\pgfqpoint{6.548313in}{1.687564in}}%
\pgfpathcurveto{\pgfqpoint{6.548313in}{1.676514in}}{\pgfqpoint{6.552704in}{1.665915in}}{\pgfqpoint{6.560517in}{1.658101in}}%
\pgfpathcurveto{\pgfqpoint{6.568331in}{1.650288in}}{\pgfqpoint{6.578930in}{1.645898in}}{\pgfqpoint{6.589980in}{1.645898in}}%
\pgfpathclose%
\pgfusepath{stroke,fill}%
\end{pgfscope}%
\begin{pgfscope}%
\pgfpathrectangle{\pgfqpoint{0.481978in}{0.331635in}}{\pgfqpoint{9.300000in}{7.700000in}}%
\pgfusepath{clip}%
\pgfsetbuttcap%
\pgfsetroundjoin%
\definecolor{currentfill}{rgb}{0.631373,0.788235,0.956863}%
\pgfsetfillcolor{currentfill}%
\pgfsetlinewidth{0.481800pt}%
\definecolor{currentstroke}{rgb}{1.000000,1.000000,1.000000}%
\pgfsetstrokecolor{currentstroke}%
\pgfsetdash{}{0pt}%
\pgfpathmoveto{\pgfqpoint{5.765884in}{2.285447in}}%
\pgfpathcurveto{\pgfqpoint{5.776934in}{2.285447in}}{\pgfqpoint{5.787533in}{2.289837in}}{\pgfqpoint{5.795347in}{2.297651in}}%
\pgfpathcurveto{\pgfqpoint{5.803161in}{2.305464in}}{\pgfqpoint{5.807551in}{2.316063in}}{\pgfqpoint{5.807551in}{2.327113in}}%
\pgfpathcurveto{\pgfqpoint{5.807551in}{2.338163in}}{\pgfqpoint{5.803161in}{2.348762in}}{\pgfqpoint{5.795347in}{2.356576in}}%
\pgfpathcurveto{\pgfqpoint{5.787533in}{2.364390in}}{\pgfqpoint{5.776934in}{2.368780in}}{\pgfqpoint{5.765884in}{2.368780in}}%
\pgfpathcurveto{\pgfqpoint{5.754834in}{2.368780in}}{\pgfqpoint{5.744235in}{2.364390in}}{\pgfqpoint{5.736421in}{2.356576in}}%
\pgfpathcurveto{\pgfqpoint{5.728608in}{2.348762in}}{\pgfqpoint{5.724218in}{2.338163in}}{\pgfqpoint{5.724218in}{2.327113in}}%
\pgfpathcurveto{\pgfqpoint{5.724218in}{2.316063in}}{\pgfqpoint{5.728608in}{2.305464in}}{\pgfqpoint{5.736421in}{2.297651in}}%
\pgfpathcurveto{\pgfqpoint{5.744235in}{2.289837in}}{\pgfqpoint{5.754834in}{2.285447in}}{\pgfqpoint{5.765884in}{2.285447in}}%
\pgfpathclose%
\pgfusepath{stroke,fill}%
\end{pgfscope}%
\begin{pgfscope}%
\pgfpathrectangle{\pgfqpoint{0.481978in}{0.331635in}}{\pgfqpoint{9.300000in}{7.700000in}}%
\pgfusepath{clip}%
\pgfsetbuttcap%
\pgfsetroundjoin%
\definecolor{currentfill}{rgb}{0.631373,0.788235,0.956863}%
\pgfsetfillcolor{currentfill}%
\pgfsetlinewidth{0.481800pt}%
\definecolor{currentstroke}{rgb}{1.000000,1.000000,1.000000}%
\pgfsetstrokecolor{currentstroke}%
\pgfsetdash{}{0pt}%
\pgfpathmoveto{\pgfqpoint{4.097733in}{6.266669in}}%
\pgfpathcurveto{\pgfqpoint{4.108783in}{6.266669in}}{\pgfqpoint{4.119382in}{6.271059in}}{\pgfqpoint{4.127196in}{6.278873in}}%
\pgfpathcurveto{\pgfqpoint{4.135009in}{6.286687in}}{\pgfqpoint{4.139400in}{6.297286in}}{\pgfqpoint{4.139400in}{6.308336in}}%
\pgfpathcurveto{\pgfqpoint{4.139400in}{6.319386in}}{\pgfqpoint{4.135009in}{6.329985in}}{\pgfqpoint{4.127196in}{6.337799in}}%
\pgfpathcurveto{\pgfqpoint{4.119382in}{6.345612in}}{\pgfqpoint{4.108783in}{6.350003in}}{\pgfqpoint{4.097733in}{6.350003in}}%
\pgfpathcurveto{\pgfqpoint{4.086683in}{6.350003in}}{\pgfqpoint{4.076084in}{6.345612in}}{\pgfqpoint{4.068270in}{6.337799in}}%
\pgfpathcurveto{\pgfqpoint{4.060456in}{6.329985in}}{\pgfqpoint{4.056066in}{6.319386in}}{\pgfqpoint{4.056066in}{6.308336in}}%
\pgfpathcurveto{\pgfqpoint{4.056066in}{6.297286in}}{\pgfqpoint{4.060456in}{6.286687in}}{\pgfqpoint{4.068270in}{6.278873in}}%
\pgfpathcurveto{\pgfqpoint{4.076084in}{6.271059in}}{\pgfqpoint{4.086683in}{6.266669in}}{\pgfqpoint{4.097733in}{6.266669in}}%
\pgfpathclose%
\pgfusepath{stroke,fill}%
\end{pgfscope}%
\begin{pgfscope}%
\pgfpathrectangle{\pgfqpoint{0.481978in}{0.331635in}}{\pgfqpoint{9.300000in}{7.700000in}}%
\pgfusepath{clip}%
\pgfsetbuttcap%
\pgfsetroundjoin%
\definecolor{currentfill}{rgb}{0.631373,0.788235,0.956863}%
\pgfsetfillcolor{currentfill}%
\pgfsetlinewidth{0.481800pt}%
\definecolor{currentstroke}{rgb}{1.000000,1.000000,1.000000}%
\pgfsetstrokecolor{currentstroke}%
\pgfsetdash{}{0pt}%
\pgfpathmoveto{\pgfqpoint{7.842322in}{5.264362in}}%
\pgfpathcurveto{\pgfqpoint{7.853372in}{5.264362in}}{\pgfqpoint{7.863971in}{5.268752in}}{\pgfqpoint{7.871784in}{5.276566in}}%
\pgfpathcurveto{\pgfqpoint{7.879598in}{5.284379in}}{\pgfqpoint{7.883988in}{5.294978in}}{\pgfqpoint{7.883988in}{5.306029in}}%
\pgfpathcurveto{\pgfqpoint{7.883988in}{5.317079in}}{\pgfqpoint{7.879598in}{5.327678in}}{\pgfqpoint{7.871784in}{5.335491in}}%
\pgfpathcurveto{\pgfqpoint{7.863971in}{5.343305in}}{\pgfqpoint{7.853372in}{5.347695in}}{\pgfqpoint{7.842322in}{5.347695in}}%
\pgfpathcurveto{\pgfqpoint{7.831271in}{5.347695in}}{\pgfqpoint{7.820672in}{5.343305in}}{\pgfqpoint{7.812859in}{5.335491in}}%
\pgfpathcurveto{\pgfqpoint{7.805045in}{5.327678in}}{\pgfqpoint{7.800655in}{5.317079in}}{\pgfqpoint{7.800655in}{5.306029in}}%
\pgfpathcurveto{\pgfqpoint{7.800655in}{5.294978in}}{\pgfqpoint{7.805045in}{5.284379in}}{\pgfqpoint{7.812859in}{5.276566in}}%
\pgfpathcurveto{\pgfqpoint{7.820672in}{5.268752in}}{\pgfqpoint{7.831271in}{5.264362in}}{\pgfqpoint{7.842322in}{5.264362in}}%
\pgfpathclose%
\pgfusepath{stroke,fill}%
\end{pgfscope}%
\begin{pgfscope}%
\pgfpathrectangle{\pgfqpoint{0.481978in}{0.331635in}}{\pgfqpoint{9.300000in}{7.700000in}}%
\pgfusepath{clip}%
\pgfsetbuttcap%
\pgfsetroundjoin%
\definecolor{currentfill}{rgb}{0.631373,0.788235,0.956863}%
\pgfsetfillcolor{currentfill}%
\pgfsetlinewidth{0.481800pt}%
\definecolor{currentstroke}{rgb}{1.000000,1.000000,1.000000}%
\pgfsetstrokecolor{currentstroke}%
\pgfsetdash{}{0pt}%
\pgfpathmoveto{\pgfqpoint{3.960820in}{4.581563in}}%
\pgfpathcurveto{\pgfqpoint{3.971870in}{4.581563in}}{\pgfqpoint{3.982469in}{4.585953in}}{\pgfqpoint{3.990283in}{4.593767in}}%
\pgfpathcurveto{\pgfqpoint{3.998097in}{4.601581in}}{\pgfqpoint{4.002487in}{4.612180in}}{\pgfqpoint{4.002487in}{4.623230in}}%
\pgfpathcurveto{\pgfqpoint{4.002487in}{4.634280in}}{\pgfqpoint{3.998097in}{4.644879in}}{\pgfqpoint{3.990283in}{4.652693in}}%
\pgfpathcurveto{\pgfqpoint{3.982469in}{4.660506in}}{\pgfqpoint{3.971870in}{4.664896in}}{\pgfqpoint{3.960820in}{4.664896in}}%
\pgfpathcurveto{\pgfqpoint{3.949770in}{4.664896in}}{\pgfqpoint{3.939171in}{4.660506in}}{\pgfqpoint{3.931357in}{4.652693in}}%
\pgfpathcurveto{\pgfqpoint{3.923544in}{4.644879in}}{\pgfqpoint{3.919153in}{4.634280in}}{\pgfqpoint{3.919153in}{4.623230in}}%
\pgfpathcurveto{\pgfqpoint{3.919153in}{4.612180in}}{\pgfqpoint{3.923544in}{4.601581in}}{\pgfqpoint{3.931357in}{4.593767in}}%
\pgfpathcurveto{\pgfqpoint{3.939171in}{4.585953in}}{\pgfqpoint{3.949770in}{4.581563in}}{\pgfqpoint{3.960820in}{4.581563in}}%
\pgfpathclose%
\pgfusepath{stroke,fill}%
\end{pgfscope}%
\begin{pgfscope}%
\pgfpathrectangle{\pgfqpoint{0.481978in}{0.331635in}}{\pgfqpoint{9.300000in}{7.700000in}}%
\pgfusepath{clip}%
\pgfsetbuttcap%
\pgfsetroundjoin%
\definecolor{currentfill}{rgb}{0.631373,0.788235,0.956863}%
\pgfsetfillcolor{currentfill}%
\pgfsetlinewidth{0.481800pt}%
\definecolor{currentstroke}{rgb}{1.000000,1.000000,1.000000}%
\pgfsetstrokecolor{currentstroke}%
\pgfsetdash{}{0pt}%
\pgfpathmoveto{\pgfqpoint{3.426685in}{4.624289in}}%
\pgfpathcurveto{\pgfqpoint{3.437735in}{4.624289in}}{\pgfqpoint{3.448334in}{4.628679in}}{\pgfqpoint{3.456148in}{4.636493in}}%
\pgfpathcurveto{\pgfqpoint{3.463962in}{4.644307in}}{\pgfqpoint{3.468352in}{4.654906in}}{\pgfqpoint{3.468352in}{4.665956in}}%
\pgfpathcurveto{\pgfqpoint{3.468352in}{4.677006in}}{\pgfqpoint{3.463962in}{4.687605in}}{\pgfqpoint{3.456148in}{4.695418in}}%
\pgfpathcurveto{\pgfqpoint{3.448334in}{4.703232in}}{\pgfqpoint{3.437735in}{4.707622in}}{\pgfqpoint{3.426685in}{4.707622in}}%
\pgfpathcurveto{\pgfqpoint{3.415635in}{4.707622in}}{\pgfqpoint{3.405036in}{4.703232in}}{\pgfqpoint{3.397222in}{4.695418in}}%
\pgfpathcurveto{\pgfqpoint{3.389409in}{4.687605in}}{\pgfqpoint{3.385019in}{4.677006in}}{\pgfqpoint{3.385019in}{4.665956in}}%
\pgfpathcurveto{\pgfqpoint{3.385019in}{4.654906in}}{\pgfqpoint{3.389409in}{4.644307in}}{\pgfqpoint{3.397222in}{4.636493in}}%
\pgfpathcurveto{\pgfqpoint{3.405036in}{4.628679in}}{\pgfqpoint{3.415635in}{4.624289in}}{\pgfqpoint{3.426685in}{4.624289in}}%
\pgfpathclose%
\pgfusepath{stroke,fill}%
\end{pgfscope}%
\begin{pgfscope}%
\pgfpathrectangle{\pgfqpoint{0.481978in}{0.331635in}}{\pgfqpoint{9.300000in}{7.700000in}}%
\pgfusepath{clip}%
\pgfsetbuttcap%
\pgfsetroundjoin%
\definecolor{currentfill}{rgb}{0.631373,0.788235,0.956863}%
\pgfsetfillcolor{currentfill}%
\pgfsetlinewidth{0.481800pt}%
\definecolor{currentstroke}{rgb}{1.000000,1.000000,1.000000}%
\pgfsetstrokecolor{currentstroke}%
\pgfsetdash{}{0pt}%
\pgfpathmoveto{\pgfqpoint{6.445366in}{2.049583in}}%
\pgfpathcurveto{\pgfqpoint{6.456416in}{2.049583in}}{\pgfqpoint{6.467016in}{2.053973in}}{\pgfqpoint{6.474829in}{2.061786in}}%
\pgfpathcurveto{\pgfqpoint{6.482643in}{2.069600in}}{\pgfqpoint{6.487033in}{2.080199in}}{\pgfqpoint{6.487033in}{2.091249in}}%
\pgfpathcurveto{\pgfqpoint{6.487033in}{2.102299in}}{\pgfqpoint{6.482643in}{2.112898in}}{\pgfqpoint{6.474829in}{2.120712in}}%
\pgfpathcurveto{\pgfqpoint{6.467016in}{2.128526in}}{\pgfqpoint{6.456416in}{2.132916in}}{\pgfqpoint{6.445366in}{2.132916in}}%
\pgfpathcurveto{\pgfqpoint{6.434316in}{2.132916in}}{\pgfqpoint{6.423717in}{2.128526in}}{\pgfqpoint{6.415904in}{2.120712in}}%
\pgfpathcurveto{\pgfqpoint{6.408090in}{2.112898in}}{\pgfqpoint{6.403700in}{2.102299in}}{\pgfqpoint{6.403700in}{2.091249in}}%
\pgfpathcurveto{\pgfqpoint{6.403700in}{2.080199in}}{\pgfqpoint{6.408090in}{2.069600in}}{\pgfqpoint{6.415904in}{2.061786in}}%
\pgfpathcurveto{\pgfqpoint{6.423717in}{2.053973in}}{\pgfqpoint{6.434316in}{2.049583in}}{\pgfqpoint{6.445366in}{2.049583in}}%
\pgfpathclose%
\pgfusepath{stroke,fill}%
\end{pgfscope}%
\begin{pgfscope}%
\pgfpathrectangle{\pgfqpoint{0.481978in}{0.331635in}}{\pgfqpoint{9.300000in}{7.700000in}}%
\pgfusepath{clip}%
\pgfsetbuttcap%
\pgfsetroundjoin%
\definecolor{currentfill}{rgb}{0.631373,0.788235,0.956863}%
\pgfsetfillcolor{currentfill}%
\pgfsetlinewidth{0.481800pt}%
\definecolor{currentstroke}{rgb}{1.000000,1.000000,1.000000}%
\pgfsetstrokecolor{currentstroke}%
\pgfsetdash{}{0pt}%
\pgfpathmoveto{\pgfqpoint{8.216289in}{4.696577in}}%
\pgfpathcurveto{\pgfqpoint{8.227339in}{4.696577in}}{\pgfqpoint{8.237938in}{4.700967in}}{\pgfqpoint{8.245752in}{4.708781in}}%
\pgfpathcurveto{\pgfqpoint{8.253565in}{4.716595in}}{\pgfqpoint{8.257956in}{4.727194in}}{\pgfqpoint{8.257956in}{4.738244in}}%
\pgfpathcurveto{\pgfqpoint{8.257956in}{4.749294in}}{\pgfqpoint{8.253565in}{4.759893in}}{\pgfqpoint{8.245752in}{4.767707in}}%
\pgfpathcurveto{\pgfqpoint{8.237938in}{4.775520in}}{\pgfqpoint{8.227339in}{4.779910in}}{\pgfqpoint{8.216289in}{4.779910in}}%
\pgfpathcurveto{\pgfqpoint{8.205239in}{4.779910in}}{\pgfqpoint{8.194640in}{4.775520in}}{\pgfqpoint{8.186826in}{4.767707in}}%
\pgfpathcurveto{\pgfqpoint{8.179013in}{4.759893in}}{\pgfqpoint{8.174622in}{4.749294in}}{\pgfqpoint{8.174622in}{4.738244in}}%
\pgfpathcurveto{\pgfqpoint{8.174622in}{4.727194in}}{\pgfqpoint{8.179013in}{4.716595in}}{\pgfqpoint{8.186826in}{4.708781in}}%
\pgfpathcurveto{\pgfqpoint{8.194640in}{4.700967in}}{\pgfqpoint{8.205239in}{4.696577in}}{\pgfqpoint{8.216289in}{4.696577in}}%
\pgfpathclose%
\pgfusepath{stroke,fill}%
\end{pgfscope}%
\begin{pgfscope}%
\pgfpathrectangle{\pgfqpoint{0.481978in}{0.331635in}}{\pgfqpoint{9.300000in}{7.700000in}}%
\pgfusepath{clip}%
\pgfsetbuttcap%
\pgfsetroundjoin%
\definecolor{currentfill}{rgb}{0.631373,0.788235,0.956863}%
\pgfsetfillcolor{currentfill}%
\pgfsetlinewidth{0.481800pt}%
\definecolor{currentstroke}{rgb}{1.000000,1.000000,1.000000}%
\pgfsetstrokecolor{currentstroke}%
\pgfsetdash{}{0pt}%
\pgfpathmoveto{\pgfqpoint{8.236668in}{4.377435in}}%
\pgfpathcurveto{\pgfqpoint{8.247718in}{4.377435in}}{\pgfqpoint{8.258317in}{4.381826in}}{\pgfqpoint{8.266131in}{4.389639in}}%
\pgfpathcurveto{\pgfqpoint{8.273945in}{4.397453in}}{\pgfqpoint{8.278335in}{4.408052in}}{\pgfqpoint{8.278335in}{4.419102in}}%
\pgfpathcurveto{\pgfqpoint{8.278335in}{4.430152in}}{\pgfqpoint{8.273945in}{4.440751in}}{\pgfqpoint{8.266131in}{4.448565in}}%
\pgfpathcurveto{\pgfqpoint{8.258317in}{4.456379in}}{\pgfqpoint{8.247718in}{4.460769in}}{\pgfqpoint{8.236668in}{4.460769in}}%
\pgfpathcurveto{\pgfqpoint{8.225618in}{4.460769in}}{\pgfqpoint{8.215019in}{4.456379in}}{\pgfqpoint{8.207205in}{4.448565in}}%
\pgfpathcurveto{\pgfqpoint{8.199392in}{4.440751in}}{\pgfqpoint{8.195001in}{4.430152in}}{\pgfqpoint{8.195001in}{4.419102in}}%
\pgfpathcurveto{\pgfqpoint{8.195001in}{4.408052in}}{\pgfqpoint{8.199392in}{4.397453in}}{\pgfqpoint{8.207205in}{4.389639in}}%
\pgfpathcurveto{\pgfqpoint{8.215019in}{4.381826in}}{\pgfqpoint{8.225618in}{4.377435in}}{\pgfqpoint{8.236668in}{4.377435in}}%
\pgfpathclose%
\pgfusepath{stroke,fill}%
\end{pgfscope}%
\begin{pgfscope}%
\pgfpathrectangle{\pgfqpoint{0.481978in}{0.331635in}}{\pgfqpoint{9.300000in}{7.700000in}}%
\pgfusepath{clip}%
\pgfsetbuttcap%
\pgfsetroundjoin%
\definecolor{currentfill}{rgb}{0.631373,0.788235,0.956863}%
\pgfsetfillcolor{currentfill}%
\pgfsetlinewidth{0.481800pt}%
\definecolor{currentstroke}{rgb}{1.000000,1.000000,1.000000}%
\pgfsetstrokecolor{currentstroke}%
\pgfsetdash{}{0pt}%
\pgfpathmoveto{\pgfqpoint{7.008316in}{2.323870in}}%
\pgfpathcurveto{\pgfqpoint{7.019366in}{2.323870in}}{\pgfqpoint{7.029965in}{2.328260in}}{\pgfqpoint{7.037779in}{2.336074in}}%
\pgfpathcurveto{\pgfqpoint{7.045593in}{2.343887in}}{\pgfqpoint{7.049983in}{2.354486in}}{\pgfqpoint{7.049983in}{2.365537in}}%
\pgfpathcurveto{\pgfqpoint{7.049983in}{2.376587in}}{\pgfqpoint{7.045593in}{2.387186in}}{\pgfqpoint{7.037779in}{2.394999in}}%
\pgfpathcurveto{\pgfqpoint{7.029965in}{2.402813in}}{\pgfqpoint{7.019366in}{2.407203in}}{\pgfqpoint{7.008316in}{2.407203in}}%
\pgfpathcurveto{\pgfqpoint{6.997266in}{2.407203in}}{\pgfqpoint{6.986667in}{2.402813in}}{\pgfqpoint{6.978853in}{2.394999in}}%
\pgfpathcurveto{\pgfqpoint{6.971040in}{2.387186in}}{\pgfqpoint{6.966650in}{2.376587in}}{\pgfqpoint{6.966650in}{2.365537in}}%
\pgfpathcurveto{\pgfqpoint{6.966650in}{2.354486in}}{\pgfqpoint{6.971040in}{2.343887in}}{\pgfqpoint{6.978853in}{2.336074in}}%
\pgfpathcurveto{\pgfqpoint{6.986667in}{2.328260in}}{\pgfqpoint{6.997266in}{2.323870in}}{\pgfqpoint{7.008316in}{2.323870in}}%
\pgfpathclose%
\pgfusepath{stroke,fill}%
\end{pgfscope}%
\begin{pgfscope}%
\pgfpathrectangle{\pgfqpoint{0.481978in}{0.331635in}}{\pgfqpoint{9.300000in}{7.700000in}}%
\pgfusepath{clip}%
\pgfsetbuttcap%
\pgfsetroundjoin%
\definecolor{currentfill}{rgb}{0.631373,0.788235,0.956863}%
\pgfsetfillcolor{currentfill}%
\pgfsetlinewidth{0.481800pt}%
\definecolor{currentstroke}{rgb}{1.000000,1.000000,1.000000}%
\pgfsetstrokecolor{currentstroke}%
\pgfsetdash{}{0pt}%
\pgfpathmoveto{\pgfqpoint{8.317641in}{5.122719in}}%
\pgfpathcurveto{\pgfqpoint{8.328691in}{5.122719in}}{\pgfqpoint{8.339290in}{5.127109in}}{\pgfqpoint{8.347104in}{5.134923in}}%
\pgfpathcurveto{\pgfqpoint{8.354917in}{5.142736in}}{\pgfqpoint{8.359307in}{5.153335in}}{\pgfqpoint{8.359307in}{5.164385in}}%
\pgfpathcurveto{\pgfqpoint{8.359307in}{5.175435in}}{\pgfqpoint{8.354917in}{5.186034in}}{\pgfqpoint{8.347104in}{5.193848in}}%
\pgfpathcurveto{\pgfqpoint{8.339290in}{5.201662in}}{\pgfqpoint{8.328691in}{5.206052in}}{\pgfqpoint{8.317641in}{5.206052in}}%
\pgfpathcurveto{\pgfqpoint{8.306591in}{5.206052in}}{\pgfqpoint{8.295992in}{5.201662in}}{\pgfqpoint{8.288178in}{5.193848in}}%
\pgfpathcurveto{\pgfqpoint{8.280364in}{5.186034in}}{\pgfqpoint{8.275974in}{5.175435in}}{\pgfqpoint{8.275974in}{5.164385in}}%
\pgfpathcurveto{\pgfqpoint{8.275974in}{5.153335in}}{\pgfqpoint{8.280364in}{5.142736in}}{\pgfqpoint{8.288178in}{5.134923in}}%
\pgfpathcurveto{\pgfqpoint{8.295992in}{5.127109in}}{\pgfqpoint{8.306591in}{5.122719in}}{\pgfqpoint{8.317641in}{5.122719in}}%
\pgfpathclose%
\pgfusepath{stroke,fill}%
\end{pgfscope}%
\begin{pgfscope}%
\pgfpathrectangle{\pgfqpoint{0.481978in}{0.331635in}}{\pgfqpoint{9.300000in}{7.700000in}}%
\pgfusepath{clip}%
\pgfsetbuttcap%
\pgfsetroundjoin%
\definecolor{currentfill}{rgb}{0.631373,0.788235,0.956863}%
\pgfsetfillcolor{currentfill}%
\pgfsetlinewidth{0.481800pt}%
\definecolor{currentstroke}{rgb}{1.000000,1.000000,1.000000}%
\pgfsetstrokecolor{currentstroke}%
\pgfsetdash{}{0pt}%
\pgfpathmoveto{\pgfqpoint{7.082145in}{1.699061in}}%
\pgfpathcurveto{\pgfqpoint{7.093195in}{1.699061in}}{\pgfqpoint{7.103794in}{1.703451in}}{\pgfqpoint{7.111608in}{1.711265in}}%
\pgfpathcurveto{\pgfqpoint{7.119422in}{1.719078in}}{\pgfqpoint{7.123812in}{1.729677in}}{\pgfqpoint{7.123812in}{1.740727in}}%
\pgfpathcurveto{\pgfqpoint{7.123812in}{1.751777in}}{\pgfqpoint{7.119422in}{1.762376in}}{\pgfqpoint{7.111608in}{1.770190in}}%
\pgfpathcurveto{\pgfqpoint{7.103794in}{1.778004in}}{\pgfqpoint{7.093195in}{1.782394in}}{\pgfqpoint{7.082145in}{1.782394in}}%
\pgfpathcurveto{\pgfqpoint{7.071095in}{1.782394in}}{\pgfqpoint{7.060496in}{1.778004in}}{\pgfqpoint{7.052683in}{1.770190in}}%
\pgfpathcurveto{\pgfqpoint{7.044869in}{1.762376in}}{\pgfqpoint{7.040479in}{1.751777in}}{\pgfqpoint{7.040479in}{1.740727in}}%
\pgfpathcurveto{\pgfqpoint{7.040479in}{1.729677in}}{\pgfqpoint{7.044869in}{1.719078in}}{\pgfqpoint{7.052683in}{1.711265in}}%
\pgfpathcurveto{\pgfqpoint{7.060496in}{1.703451in}}{\pgfqpoint{7.071095in}{1.699061in}}{\pgfqpoint{7.082145in}{1.699061in}}%
\pgfpathclose%
\pgfusepath{stroke,fill}%
\end{pgfscope}%
\begin{pgfscope}%
\pgfpathrectangle{\pgfqpoint{0.481978in}{0.331635in}}{\pgfqpoint{9.300000in}{7.700000in}}%
\pgfusepath{clip}%
\pgfsetbuttcap%
\pgfsetroundjoin%
\definecolor{currentfill}{rgb}{0.631373,0.788235,0.956863}%
\pgfsetfillcolor{currentfill}%
\pgfsetlinewidth{0.481800pt}%
\definecolor{currentstroke}{rgb}{1.000000,1.000000,1.000000}%
\pgfsetstrokecolor{currentstroke}%
\pgfsetdash{}{0pt}%
\pgfpathmoveto{\pgfqpoint{5.724298in}{1.762569in}}%
\pgfpathcurveto{\pgfqpoint{5.735348in}{1.762569in}}{\pgfqpoint{5.745947in}{1.766959in}}{\pgfqpoint{5.753761in}{1.774773in}}%
\pgfpathcurveto{\pgfqpoint{5.761575in}{1.782586in}}{\pgfqpoint{5.765965in}{1.793185in}}{\pgfqpoint{5.765965in}{1.804236in}}%
\pgfpathcurveto{\pgfqpoint{5.765965in}{1.815286in}}{\pgfqpoint{5.761575in}{1.825885in}}{\pgfqpoint{5.753761in}{1.833698in}}%
\pgfpathcurveto{\pgfqpoint{5.745947in}{1.841512in}}{\pgfqpoint{5.735348in}{1.845902in}}{\pgfqpoint{5.724298in}{1.845902in}}%
\pgfpathcurveto{\pgfqpoint{5.713248in}{1.845902in}}{\pgfqpoint{5.702649in}{1.841512in}}{\pgfqpoint{5.694835in}{1.833698in}}%
\pgfpathcurveto{\pgfqpoint{5.687022in}{1.825885in}}{\pgfqpoint{5.682632in}{1.815286in}}{\pgfqpoint{5.682632in}{1.804236in}}%
\pgfpathcurveto{\pgfqpoint{5.682632in}{1.793185in}}{\pgfqpoint{5.687022in}{1.782586in}}{\pgfqpoint{5.694835in}{1.774773in}}%
\pgfpathcurveto{\pgfqpoint{5.702649in}{1.766959in}}{\pgfqpoint{5.713248in}{1.762569in}}{\pgfqpoint{5.724298in}{1.762569in}}%
\pgfpathclose%
\pgfusepath{stroke,fill}%
\end{pgfscope}%
\begin{pgfscope}%
\pgfpathrectangle{\pgfqpoint{0.481978in}{0.331635in}}{\pgfqpoint{9.300000in}{7.700000in}}%
\pgfusepath{clip}%
\pgfsetbuttcap%
\pgfsetroundjoin%
\definecolor{currentfill}{rgb}{0.631373,0.788235,0.956863}%
\pgfsetfillcolor{currentfill}%
\pgfsetlinewidth{0.481800pt}%
\definecolor{currentstroke}{rgb}{1.000000,1.000000,1.000000}%
\pgfsetstrokecolor{currentstroke}%
\pgfsetdash{}{0pt}%
\pgfpathmoveto{\pgfqpoint{2.561461in}{1.982774in}}%
\pgfpathcurveto{\pgfqpoint{2.572512in}{1.982774in}}{\pgfqpoint{2.583111in}{1.987164in}}{\pgfqpoint{2.590924in}{1.994978in}}%
\pgfpathcurveto{\pgfqpoint{2.598738in}{2.002791in}}{\pgfqpoint{2.603128in}{2.013390in}}{\pgfqpoint{2.603128in}{2.024440in}}%
\pgfpathcurveto{\pgfqpoint{2.603128in}{2.035491in}}{\pgfqpoint{2.598738in}{2.046090in}}{\pgfqpoint{2.590924in}{2.053903in}}%
\pgfpathcurveto{\pgfqpoint{2.583111in}{2.061717in}}{\pgfqpoint{2.572512in}{2.066107in}}{\pgfqpoint{2.561461in}{2.066107in}}%
\pgfpathcurveto{\pgfqpoint{2.550411in}{2.066107in}}{\pgfqpoint{2.539812in}{2.061717in}}{\pgfqpoint{2.531999in}{2.053903in}}%
\pgfpathcurveto{\pgfqpoint{2.524185in}{2.046090in}}{\pgfqpoint{2.519795in}{2.035491in}}{\pgfqpoint{2.519795in}{2.024440in}}%
\pgfpathcurveto{\pgfqpoint{2.519795in}{2.013390in}}{\pgfqpoint{2.524185in}{2.002791in}}{\pgfqpoint{2.531999in}{1.994978in}}%
\pgfpathcurveto{\pgfqpoint{2.539812in}{1.987164in}}{\pgfqpoint{2.550411in}{1.982774in}}{\pgfqpoint{2.561461in}{1.982774in}}%
\pgfpathclose%
\pgfusepath{stroke,fill}%
\end{pgfscope}%
\begin{pgfscope}%
\pgfpathrectangle{\pgfqpoint{0.481978in}{0.331635in}}{\pgfqpoint{9.300000in}{7.700000in}}%
\pgfusepath{clip}%
\pgfsetbuttcap%
\pgfsetroundjoin%
\definecolor{currentfill}{rgb}{0.631373,0.788235,0.956863}%
\pgfsetfillcolor{currentfill}%
\pgfsetlinewidth{0.481800pt}%
\definecolor{currentstroke}{rgb}{1.000000,1.000000,1.000000}%
\pgfsetstrokecolor{currentstroke}%
\pgfsetdash{}{0pt}%
\pgfpathmoveto{\pgfqpoint{6.177807in}{1.835767in}}%
\pgfpathcurveto{\pgfqpoint{6.188857in}{1.835767in}}{\pgfqpoint{6.199456in}{1.840157in}}{\pgfqpoint{6.207270in}{1.847971in}}%
\pgfpathcurveto{\pgfqpoint{6.215084in}{1.855785in}}{\pgfqpoint{6.219474in}{1.866384in}}{\pgfqpoint{6.219474in}{1.877434in}}%
\pgfpathcurveto{\pgfqpoint{6.219474in}{1.888484in}}{\pgfqpoint{6.215084in}{1.899083in}}{\pgfqpoint{6.207270in}{1.906897in}}%
\pgfpathcurveto{\pgfqpoint{6.199456in}{1.914710in}}{\pgfqpoint{6.188857in}{1.919100in}}{\pgfqpoint{6.177807in}{1.919100in}}%
\pgfpathcurveto{\pgfqpoint{6.166757in}{1.919100in}}{\pgfqpoint{6.156158in}{1.914710in}}{\pgfqpoint{6.148344in}{1.906897in}}%
\pgfpathcurveto{\pgfqpoint{6.140531in}{1.899083in}}{\pgfqpoint{6.136141in}{1.888484in}}{\pgfqpoint{6.136141in}{1.877434in}}%
\pgfpathcurveto{\pgfqpoint{6.136141in}{1.866384in}}{\pgfqpoint{6.140531in}{1.855785in}}{\pgfqpoint{6.148344in}{1.847971in}}%
\pgfpathcurveto{\pgfqpoint{6.156158in}{1.840157in}}{\pgfqpoint{6.166757in}{1.835767in}}{\pgfqpoint{6.177807in}{1.835767in}}%
\pgfpathclose%
\pgfusepath{stroke,fill}%
\end{pgfscope}%
\begin{pgfscope}%
\pgfpathrectangle{\pgfqpoint{0.481978in}{0.331635in}}{\pgfqpoint{9.300000in}{7.700000in}}%
\pgfusepath{clip}%
\pgfsetbuttcap%
\pgfsetroundjoin%
\definecolor{currentfill}{rgb}{0.631373,0.788235,0.956863}%
\pgfsetfillcolor{currentfill}%
\pgfsetlinewidth{0.481800pt}%
\definecolor{currentstroke}{rgb}{1.000000,1.000000,1.000000}%
\pgfsetstrokecolor{currentstroke}%
\pgfsetdash{}{0pt}%
\pgfpathmoveto{\pgfqpoint{6.336424in}{0.639968in}}%
\pgfpathcurveto{\pgfqpoint{6.347474in}{0.639968in}}{\pgfqpoint{6.358073in}{0.644359in}}{\pgfqpoint{6.365887in}{0.652172in}}%
\pgfpathcurveto{\pgfqpoint{6.373700in}{0.659986in}}{\pgfqpoint{6.378091in}{0.670585in}}{\pgfqpoint{6.378091in}{0.681635in}}%
\pgfpathcurveto{\pgfqpoint{6.378091in}{0.692685in}}{\pgfqpoint{6.373700in}{0.703284in}}{\pgfqpoint{6.365887in}{0.711098in}}%
\pgfpathcurveto{\pgfqpoint{6.358073in}{0.718911in}}{\pgfqpoint{6.347474in}{0.723302in}}{\pgfqpoint{6.336424in}{0.723302in}}%
\pgfpathcurveto{\pgfqpoint{6.325374in}{0.723302in}}{\pgfqpoint{6.314775in}{0.718911in}}{\pgfqpoint{6.306961in}{0.711098in}}%
\pgfpathcurveto{\pgfqpoint{6.299148in}{0.703284in}}{\pgfqpoint{6.294757in}{0.692685in}}{\pgfqpoint{6.294757in}{0.681635in}}%
\pgfpathcurveto{\pgfqpoint{6.294757in}{0.670585in}}{\pgfqpoint{6.299148in}{0.659986in}}{\pgfqpoint{6.306961in}{0.652172in}}%
\pgfpathcurveto{\pgfqpoint{6.314775in}{0.644359in}}{\pgfqpoint{6.325374in}{0.639968in}}{\pgfqpoint{6.336424in}{0.639968in}}%
\pgfpathclose%
\pgfusepath{stroke,fill}%
\end{pgfscope}%
\begin{pgfscope}%
\pgfpathrectangle{\pgfqpoint{0.481978in}{0.331635in}}{\pgfqpoint{9.300000in}{7.700000in}}%
\pgfusepath{clip}%
\pgfsetbuttcap%
\pgfsetroundjoin%
\definecolor{currentfill}{rgb}{0.631373,0.788235,0.956863}%
\pgfsetfillcolor{currentfill}%
\pgfsetlinewidth{0.481800pt}%
\definecolor{currentstroke}{rgb}{1.000000,1.000000,1.000000}%
\pgfsetstrokecolor{currentstroke}%
\pgfsetdash{}{0pt}%
\pgfpathmoveto{\pgfqpoint{7.682062in}{5.911755in}}%
\pgfpathcurveto{\pgfqpoint{7.693112in}{5.911755in}}{\pgfqpoint{7.703711in}{5.916145in}}{\pgfqpoint{7.711524in}{5.923959in}}%
\pgfpathcurveto{\pgfqpoint{7.719338in}{5.931772in}}{\pgfqpoint{7.723728in}{5.942371in}}{\pgfqpoint{7.723728in}{5.953421in}}%
\pgfpathcurveto{\pgfqpoint{7.723728in}{5.964471in}}{\pgfqpoint{7.719338in}{5.975071in}}{\pgfqpoint{7.711524in}{5.982884in}}%
\pgfpathcurveto{\pgfqpoint{7.703711in}{5.990698in}}{\pgfqpoint{7.693112in}{5.995088in}}{\pgfqpoint{7.682062in}{5.995088in}}%
\pgfpathcurveto{\pgfqpoint{7.671011in}{5.995088in}}{\pgfqpoint{7.660412in}{5.990698in}}{\pgfqpoint{7.652599in}{5.982884in}}%
\pgfpathcurveto{\pgfqpoint{7.644785in}{5.975071in}}{\pgfqpoint{7.640395in}{5.964471in}}{\pgfqpoint{7.640395in}{5.953421in}}%
\pgfpathcurveto{\pgfqpoint{7.640395in}{5.942371in}}{\pgfqpoint{7.644785in}{5.931772in}}{\pgfqpoint{7.652599in}{5.923959in}}%
\pgfpathcurveto{\pgfqpoint{7.660412in}{5.916145in}}{\pgfqpoint{7.671011in}{5.911755in}}{\pgfqpoint{7.682062in}{5.911755in}}%
\pgfpathclose%
\pgfusepath{stroke,fill}%
\end{pgfscope}%
\begin{pgfscope}%
\pgfpathrectangle{\pgfqpoint{0.481978in}{0.331635in}}{\pgfqpoint{9.300000in}{7.700000in}}%
\pgfusepath{clip}%
\pgfsetbuttcap%
\pgfsetroundjoin%
\definecolor{currentfill}{rgb}{0.631373,0.788235,0.956863}%
\pgfsetfillcolor{currentfill}%
\pgfsetlinewidth{0.481800pt}%
\definecolor{currentstroke}{rgb}{1.000000,1.000000,1.000000}%
\pgfsetstrokecolor{currentstroke}%
\pgfsetdash{}{0pt}%
\pgfpathmoveto{\pgfqpoint{6.477636in}{3.838879in}}%
\pgfpathcurveto{\pgfqpoint{6.488687in}{3.838879in}}{\pgfqpoint{6.499286in}{3.843269in}}{\pgfqpoint{6.507099in}{3.851083in}}%
\pgfpathcurveto{\pgfqpoint{6.514913in}{3.858896in}}{\pgfqpoint{6.519303in}{3.869495in}}{\pgfqpoint{6.519303in}{3.880545in}}%
\pgfpathcurveto{\pgfqpoint{6.519303in}{3.891596in}}{\pgfqpoint{6.514913in}{3.902195in}}{\pgfqpoint{6.507099in}{3.910008in}}%
\pgfpathcurveto{\pgfqpoint{6.499286in}{3.917822in}}{\pgfqpoint{6.488687in}{3.922212in}}{\pgfqpoint{6.477636in}{3.922212in}}%
\pgfpathcurveto{\pgfqpoint{6.466586in}{3.922212in}}{\pgfqpoint{6.455987in}{3.917822in}}{\pgfqpoint{6.448174in}{3.910008in}}%
\pgfpathcurveto{\pgfqpoint{6.440360in}{3.902195in}}{\pgfqpoint{6.435970in}{3.891596in}}{\pgfqpoint{6.435970in}{3.880545in}}%
\pgfpathcurveto{\pgfqpoint{6.435970in}{3.869495in}}{\pgfqpoint{6.440360in}{3.858896in}}{\pgfqpoint{6.448174in}{3.851083in}}%
\pgfpathcurveto{\pgfqpoint{6.455987in}{3.843269in}}{\pgfqpoint{6.466586in}{3.838879in}}{\pgfqpoint{6.477636in}{3.838879in}}%
\pgfpathclose%
\pgfusepath{stroke,fill}%
\end{pgfscope}%
\begin{pgfscope}%
\pgfpathrectangle{\pgfqpoint{0.481978in}{0.331635in}}{\pgfqpoint{9.300000in}{7.700000in}}%
\pgfusepath{clip}%
\pgfsetbuttcap%
\pgfsetroundjoin%
\definecolor{currentfill}{rgb}{0.631373,0.788235,0.956863}%
\pgfsetfillcolor{currentfill}%
\pgfsetlinewidth{0.481800pt}%
\definecolor{currentstroke}{rgb}{1.000000,1.000000,1.000000}%
\pgfsetstrokecolor{currentstroke}%
\pgfsetdash{}{0pt}%
\pgfpathmoveto{\pgfqpoint{6.604622in}{2.909109in}}%
\pgfpathcurveto{\pgfqpoint{6.615672in}{2.909109in}}{\pgfqpoint{6.626271in}{2.913500in}}{\pgfqpoint{6.634085in}{2.921313in}}%
\pgfpathcurveto{\pgfqpoint{6.641898in}{2.929127in}}{\pgfqpoint{6.646289in}{2.939726in}}{\pgfqpoint{6.646289in}{2.950776in}}%
\pgfpathcurveto{\pgfqpoint{6.646289in}{2.961826in}}{\pgfqpoint{6.641898in}{2.972425in}}{\pgfqpoint{6.634085in}{2.980239in}}%
\pgfpathcurveto{\pgfqpoint{6.626271in}{2.988052in}}{\pgfqpoint{6.615672in}{2.992443in}}{\pgfqpoint{6.604622in}{2.992443in}}%
\pgfpathcurveto{\pgfqpoint{6.593572in}{2.992443in}}{\pgfqpoint{6.582973in}{2.988052in}}{\pgfqpoint{6.575159in}{2.980239in}}%
\pgfpathcurveto{\pgfqpoint{6.567346in}{2.972425in}}{\pgfqpoint{6.562955in}{2.961826in}}{\pgfqpoint{6.562955in}{2.950776in}}%
\pgfpathcurveto{\pgfqpoint{6.562955in}{2.939726in}}{\pgfqpoint{6.567346in}{2.929127in}}{\pgfqpoint{6.575159in}{2.921313in}}%
\pgfpathcurveto{\pgfqpoint{6.582973in}{2.913500in}}{\pgfqpoint{6.593572in}{2.909109in}}{\pgfqpoint{6.604622in}{2.909109in}}%
\pgfpathclose%
\pgfusepath{stroke,fill}%
\end{pgfscope}%
\begin{pgfscope}%
\pgfpathrectangle{\pgfqpoint{0.481978in}{0.331635in}}{\pgfqpoint{9.300000in}{7.700000in}}%
\pgfusepath{clip}%
\pgfsetbuttcap%
\pgfsetroundjoin%
\definecolor{currentfill}{rgb}{0.631373,0.788235,0.956863}%
\pgfsetfillcolor{currentfill}%
\pgfsetlinewidth{0.481800pt}%
\definecolor{currentstroke}{rgb}{1.000000,1.000000,1.000000}%
\pgfsetstrokecolor{currentstroke}%
\pgfsetdash{}{0pt}%
\pgfpathmoveto{\pgfqpoint{7.441282in}{2.667306in}}%
\pgfpathcurveto{\pgfqpoint{7.452332in}{2.667306in}}{\pgfqpoint{7.462931in}{2.671696in}}{\pgfqpoint{7.470745in}{2.679510in}}%
\pgfpathcurveto{\pgfqpoint{7.478558in}{2.687323in}}{\pgfqpoint{7.482948in}{2.697922in}}{\pgfqpoint{7.482948in}{2.708973in}}%
\pgfpathcurveto{\pgfqpoint{7.482948in}{2.720023in}}{\pgfqpoint{7.478558in}{2.730622in}}{\pgfqpoint{7.470745in}{2.738435in}}%
\pgfpathcurveto{\pgfqpoint{7.462931in}{2.746249in}}{\pgfqpoint{7.452332in}{2.750639in}}{\pgfqpoint{7.441282in}{2.750639in}}%
\pgfpathcurveto{\pgfqpoint{7.430232in}{2.750639in}}{\pgfqpoint{7.419633in}{2.746249in}}{\pgfqpoint{7.411819in}{2.738435in}}%
\pgfpathcurveto{\pgfqpoint{7.404005in}{2.730622in}}{\pgfqpoint{7.399615in}{2.720023in}}{\pgfqpoint{7.399615in}{2.708973in}}%
\pgfpathcurveto{\pgfqpoint{7.399615in}{2.697922in}}{\pgfqpoint{7.404005in}{2.687323in}}{\pgfqpoint{7.411819in}{2.679510in}}%
\pgfpathcurveto{\pgfqpoint{7.419633in}{2.671696in}}{\pgfqpoint{7.430232in}{2.667306in}}{\pgfqpoint{7.441282in}{2.667306in}}%
\pgfpathclose%
\pgfusepath{stroke,fill}%
\end{pgfscope}%
\begin{pgfscope}%
\pgfpathrectangle{\pgfqpoint{0.481978in}{0.331635in}}{\pgfqpoint{9.300000in}{7.700000in}}%
\pgfusepath{clip}%
\pgfsetbuttcap%
\pgfsetroundjoin%
\definecolor{currentfill}{rgb}{0.631373,0.788235,0.956863}%
\pgfsetfillcolor{currentfill}%
\pgfsetlinewidth{0.481800pt}%
\definecolor{currentstroke}{rgb}{1.000000,1.000000,1.000000}%
\pgfsetstrokecolor{currentstroke}%
\pgfsetdash{}{0pt}%
\pgfpathmoveto{\pgfqpoint{5.021448in}{5.556371in}}%
\pgfpathcurveto{\pgfqpoint{5.032498in}{5.556371in}}{\pgfqpoint{5.043097in}{5.560761in}}{\pgfqpoint{5.050910in}{5.568575in}}%
\pgfpathcurveto{\pgfqpoint{5.058724in}{5.576389in}}{\pgfqpoint{5.063114in}{5.586988in}}{\pgfqpoint{5.063114in}{5.598038in}}%
\pgfpathcurveto{\pgfqpoint{5.063114in}{5.609088in}}{\pgfqpoint{5.058724in}{5.619687in}}{\pgfqpoint{5.050910in}{5.627501in}}%
\pgfpathcurveto{\pgfqpoint{5.043097in}{5.635314in}}{\pgfqpoint{5.032498in}{5.639704in}}{\pgfqpoint{5.021448in}{5.639704in}}%
\pgfpathcurveto{\pgfqpoint{5.010398in}{5.639704in}}{\pgfqpoint{4.999799in}{5.635314in}}{\pgfqpoint{4.991985in}{5.627501in}}%
\pgfpathcurveto{\pgfqpoint{4.984171in}{5.619687in}}{\pgfqpoint{4.979781in}{5.609088in}}{\pgfqpoint{4.979781in}{5.598038in}}%
\pgfpathcurveto{\pgfqpoint{4.979781in}{5.586988in}}{\pgfqpoint{4.984171in}{5.576389in}}{\pgfqpoint{4.991985in}{5.568575in}}%
\pgfpathcurveto{\pgfqpoint{4.999799in}{5.560761in}}{\pgfqpoint{5.010398in}{5.556371in}}{\pgfqpoint{5.021448in}{5.556371in}}%
\pgfpathclose%
\pgfusepath{stroke,fill}%
\end{pgfscope}%
\begin{pgfscope}%
\pgfpathrectangle{\pgfqpoint{0.481978in}{0.331635in}}{\pgfqpoint{9.300000in}{7.700000in}}%
\pgfusepath{clip}%
\pgfsetbuttcap%
\pgfsetroundjoin%
\definecolor{currentfill}{rgb}{0.631373,0.788235,0.956863}%
\pgfsetfillcolor{currentfill}%
\pgfsetlinewidth{0.481800pt}%
\definecolor{currentstroke}{rgb}{1.000000,1.000000,1.000000}%
\pgfsetstrokecolor{currentstroke}%
\pgfsetdash{}{0pt}%
\pgfpathmoveto{\pgfqpoint{6.840343in}{4.402227in}}%
\pgfpathcurveto{\pgfqpoint{6.851393in}{4.402227in}}{\pgfqpoint{6.861992in}{4.406617in}}{\pgfqpoint{6.869806in}{4.414431in}}%
\pgfpathcurveto{\pgfqpoint{6.877619in}{4.422244in}}{\pgfqpoint{6.882010in}{4.432843in}}{\pgfqpoint{6.882010in}{4.443893in}}%
\pgfpathcurveto{\pgfqpoint{6.882010in}{4.454943in}}{\pgfqpoint{6.877619in}{4.465542in}}{\pgfqpoint{6.869806in}{4.473356in}}%
\pgfpathcurveto{\pgfqpoint{6.861992in}{4.481170in}}{\pgfqpoint{6.851393in}{4.485560in}}{\pgfqpoint{6.840343in}{4.485560in}}%
\pgfpathcurveto{\pgfqpoint{6.829293in}{4.485560in}}{\pgfqpoint{6.818694in}{4.481170in}}{\pgfqpoint{6.810880in}{4.473356in}}%
\pgfpathcurveto{\pgfqpoint{6.803066in}{4.465542in}}{\pgfqpoint{6.798676in}{4.454943in}}{\pgfqpoint{6.798676in}{4.443893in}}%
\pgfpathcurveto{\pgfqpoint{6.798676in}{4.432843in}}{\pgfqpoint{6.803066in}{4.422244in}}{\pgfqpoint{6.810880in}{4.414431in}}%
\pgfpathcurveto{\pgfqpoint{6.818694in}{4.406617in}}{\pgfqpoint{6.829293in}{4.402227in}}{\pgfqpoint{6.840343in}{4.402227in}}%
\pgfpathclose%
\pgfusepath{stroke,fill}%
\end{pgfscope}%
\begin{pgfscope}%
\pgfpathrectangle{\pgfqpoint{0.481978in}{0.331635in}}{\pgfqpoint{9.300000in}{7.700000in}}%
\pgfusepath{clip}%
\pgfsetbuttcap%
\pgfsetroundjoin%
\definecolor{currentfill}{rgb}{0.631373,0.788235,0.956863}%
\pgfsetfillcolor{currentfill}%
\pgfsetlinewidth{0.481800pt}%
\definecolor{currentstroke}{rgb}{1.000000,1.000000,1.000000}%
\pgfsetstrokecolor{currentstroke}%
\pgfsetdash{}{0pt}%
\pgfpathmoveto{\pgfqpoint{6.128623in}{5.960400in}}%
\pgfpathcurveto{\pgfqpoint{6.139673in}{5.960400in}}{\pgfqpoint{6.150272in}{5.964790in}}{\pgfqpoint{6.158086in}{5.972603in}}%
\pgfpathcurveto{\pgfqpoint{6.165900in}{5.980417in}}{\pgfqpoint{6.170290in}{5.991016in}}{\pgfqpoint{6.170290in}{6.002066in}}%
\pgfpathcurveto{\pgfqpoint{6.170290in}{6.013116in}}{\pgfqpoint{6.165900in}{6.023715in}}{\pgfqpoint{6.158086in}{6.031529in}}%
\pgfpathcurveto{\pgfqpoint{6.150272in}{6.039343in}}{\pgfqpoint{6.139673in}{6.043733in}}{\pgfqpoint{6.128623in}{6.043733in}}%
\pgfpathcurveto{\pgfqpoint{6.117573in}{6.043733in}}{\pgfqpoint{6.106974in}{6.039343in}}{\pgfqpoint{6.099160in}{6.031529in}}%
\pgfpathcurveto{\pgfqpoint{6.091347in}{6.023715in}}{\pgfqpoint{6.086957in}{6.013116in}}{\pgfqpoint{6.086957in}{6.002066in}}%
\pgfpathcurveto{\pgfqpoint{6.086957in}{5.991016in}}{\pgfqpoint{6.091347in}{5.980417in}}{\pgfqpoint{6.099160in}{5.972603in}}%
\pgfpathcurveto{\pgfqpoint{6.106974in}{5.964790in}}{\pgfqpoint{6.117573in}{5.960400in}}{\pgfqpoint{6.128623in}{5.960400in}}%
\pgfpathclose%
\pgfusepath{stroke,fill}%
\end{pgfscope}%
\begin{pgfscope}%
\pgfpathrectangle{\pgfqpoint{0.481978in}{0.331635in}}{\pgfqpoint{9.300000in}{7.700000in}}%
\pgfusepath{clip}%
\pgfsetbuttcap%
\pgfsetroundjoin%
\definecolor{currentfill}{rgb}{0.631373,0.788235,0.956863}%
\pgfsetfillcolor{currentfill}%
\pgfsetlinewidth{0.481800pt}%
\definecolor{currentstroke}{rgb}{1.000000,1.000000,1.000000}%
\pgfsetstrokecolor{currentstroke}%
\pgfsetdash{}{0pt}%
\pgfpathmoveto{\pgfqpoint{6.130665in}{5.955353in}}%
\pgfpathcurveto{\pgfqpoint{6.141715in}{5.955353in}}{\pgfqpoint{6.152314in}{5.959744in}}{\pgfqpoint{6.160127in}{5.967557in}}%
\pgfpathcurveto{\pgfqpoint{6.167941in}{5.975371in}}{\pgfqpoint{6.172331in}{5.985970in}}{\pgfqpoint{6.172331in}{5.997020in}}%
\pgfpathcurveto{\pgfqpoint{6.172331in}{6.008070in}}{\pgfqpoint{6.167941in}{6.018669in}}{\pgfqpoint{6.160127in}{6.026483in}}%
\pgfpathcurveto{\pgfqpoint{6.152314in}{6.034296in}}{\pgfqpoint{6.141715in}{6.038687in}}{\pgfqpoint{6.130665in}{6.038687in}}%
\pgfpathcurveto{\pgfqpoint{6.119614in}{6.038687in}}{\pgfqpoint{6.109015in}{6.034296in}}{\pgfqpoint{6.101202in}{6.026483in}}%
\pgfpathcurveto{\pgfqpoint{6.093388in}{6.018669in}}{\pgfqpoint{6.088998in}{6.008070in}}{\pgfqpoint{6.088998in}{5.997020in}}%
\pgfpathcurveto{\pgfqpoint{6.088998in}{5.985970in}}{\pgfqpoint{6.093388in}{5.975371in}}{\pgfqpoint{6.101202in}{5.967557in}}%
\pgfpathcurveto{\pgfqpoint{6.109015in}{5.959744in}}{\pgfqpoint{6.119614in}{5.955353in}}{\pgfqpoint{6.130665in}{5.955353in}}%
\pgfpathclose%
\pgfusepath{stroke,fill}%
\end{pgfscope}%
\begin{pgfscope}%
\pgfpathrectangle{\pgfqpoint{0.481978in}{0.331635in}}{\pgfqpoint{9.300000in}{7.700000in}}%
\pgfusepath{clip}%
\pgfsetbuttcap%
\pgfsetroundjoin%
\definecolor{currentfill}{rgb}{0.631373,0.788235,0.956863}%
\pgfsetfillcolor{currentfill}%
\pgfsetlinewidth{0.481800pt}%
\definecolor{currentstroke}{rgb}{1.000000,1.000000,1.000000}%
\pgfsetstrokecolor{currentstroke}%
\pgfsetdash{}{0pt}%
\pgfpathmoveto{\pgfqpoint{5.250013in}{5.297890in}}%
\pgfpathcurveto{\pgfqpoint{5.261063in}{5.297890in}}{\pgfqpoint{5.271662in}{5.302280in}}{\pgfqpoint{5.279476in}{5.310093in}}%
\pgfpathcurveto{\pgfqpoint{5.287289in}{5.317907in}}{\pgfqpoint{5.291679in}{5.328506in}}{\pgfqpoint{5.291679in}{5.339556in}}%
\pgfpathcurveto{\pgfqpoint{5.291679in}{5.350606in}}{\pgfqpoint{5.287289in}{5.361205in}}{\pgfqpoint{5.279476in}{5.369019in}}%
\pgfpathcurveto{\pgfqpoint{5.271662in}{5.376833in}}{\pgfqpoint{5.261063in}{5.381223in}}{\pgfqpoint{5.250013in}{5.381223in}}%
\pgfpathcurveto{\pgfqpoint{5.238963in}{5.381223in}}{\pgfqpoint{5.228364in}{5.376833in}}{\pgfqpoint{5.220550in}{5.369019in}}%
\pgfpathcurveto{\pgfqpoint{5.212736in}{5.361205in}}{\pgfqpoint{5.208346in}{5.350606in}}{\pgfqpoint{5.208346in}{5.339556in}}%
\pgfpathcurveto{\pgfqpoint{5.208346in}{5.328506in}}{\pgfqpoint{5.212736in}{5.317907in}}{\pgfqpoint{5.220550in}{5.310093in}}%
\pgfpathcurveto{\pgfqpoint{5.228364in}{5.302280in}}{\pgfqpoint{5.238963in}{5.297890in}}{\pgfqpoint{5.250013in}{5.297890in}}%
\pgfpathclose%
\pgfusepath{stroke,fill}%
\end{pgfscope}%
\begin{pgfscope}%
\pgfpathrectangle{\pgfqpoint{0.481978in}{0.331635in}}{\pgfqpoint{9.300000in}{7.700000in}}%
\pgfusepath{clip}%
\pgfsetbuttcap%
\pgfsetroundjoin%
\definecolor{currentfill}{rgb}{0.631373,0.788235,0.956863}%
\pgfsetfillcolor{currentfill}%
\pgfsetlinewidth{0.481800pt}%
\definecolor{currentstroke}{rgb}{1.000000,1.000000,1.000000}%
\pgfsetstrokecolor{currentstroke}%
\pgfsetdash{}{0pt}%
\pgfpathmoveto{\pgfqpoint{7.725554in}{5.784948in}}%
\pgfpathcurveto{\pgfqpoint{7.736604in}{5.784948in}}{\pgfqpoint{7.747203in}{5.789339in}}{\pgfqpoint{7.755017in}{5.797152in}}%
\pgfpathcurveto{\pgfqpoint{7.762831in}{5.804966in}}{\pgfqpoint{7.767221in}{5.815565in}}{\pgfqpoint{7.767221in}{5.826615in}}%
\pgfpathcurveto{\pgfqpoint{7.767221in}{5.837665in}}{\pgfqpoint{7.762831in}{5.848264in}}{\pgfqpoint{7.755017in}{5.856078in}}%
\pgfpathcurveto{\pgfqpoint{7.747203in}{5.863891in}}{\pgfqpoint{7.736604in}{5.868282in}}{\pgfqpoint{7.725554in}{5.868282in}}%
\pgfpathcurveto{\pgfqpoint{7.714504in}{5.868282in}}{\pgfqpoint{7.703905in}{5.863891in}}{\pgfqpoint{7.696091in}{5.856078in}}%
\pgfpathcurveto{\pgfqpoint{7.688278in}{5.848264in}}{\pgfqpoint{7.683888in}{5.837665in}}{\pgfqpoint{7.683888in}{5.826615in}}%
\pgfpathcurveto{\pgfqpoint{7.683888in}{5.815565in}}{\pgfqpoint{7.688278in}{5.804966in}}{\pgfqpoint{7.696091in}{5.797152in}}%
\pgfpathcurveto{\pgfqpoint{7.703905in}{5.789339in}}{\pgfqpoint{7.714504in}{5.784948in}}{\pgfqpoint{7.725554in}{5.784948in}}%
\pgfpathclose%
\pgfusepath{stroke,fill}%
\end{pgfscope}%
\begin{pgfscope}%
\pgfpathrectangle{\pgfqpoint{0.481978in}{0.331635in}}{\pgfqpoint{9.300000in}{7.700000in}}%
\pgfusepath{clip}%
\pgfsetbuttcap%
\pgfsetroundjoin%
\definecolor{currentfill}{rgb}{0.631373,0.788235,0.956863}%
\pgfsetfillcolor{currentfill}%
\pgfsetlinewidth{0.481800pt}%
\definecolor{currentstroke}{rgb}{1.000000,1.000000,1.000000}%
\pgfsetstrokecolor{currentstroke}%
\pgfsetdash{}{0pt}%
\pgfpathmoveto{\pgfqpoint{5.804196in}{2.892483in}}%
\pgfpathcurveto{\pgfqpoint{5.815246in}{2.892483in}}{\pgfqpoint{5.825845in}{2.896874in}}{\pgfqpoint{5.833659in}{2.904687in}}%
\pgfpathcurveto{\pgfqpoint{5.841472in}{2.912501in}}{\pgfqpoint{5.845863in}{2.923100in}}{\pgfqpoint{5.845863in}{2.934150in}}%
\pgfpathcurveto{\pgfqpoint{5.845863in}{2.945200in}}{\pgfqpoint{5.841472in}{2.955799in}}{\pgfqpoint{5.833659in}{2.963613in}}%
\pgfpathcurveto{\pgfqpoint{5.825845in}{2.971426in}}{\pgfqpoint{5.815246in}{2.975817in}}{\pgfqpoint{5.804196in}{2.975817in}}%
\pgfpathcurveto{\pgfqpoint{5.793146in}{2.975817in}}{\pgfqpoint{5.782547in}{2.971426in}}{\pgfqpoint{5.774733in}{2.963613in}}%
\pgfpathcurveto{\pgfqpoint{5.766920in}{2.955799in}}{\pgfqpoint{5.762529in}{2.945200in}}{\pgfqpoint{5.762529in}{2.934150in}}%
\pgfpathcurveto{\pgfqpoint{5.762529in}{2.923100in}}{\pgfqpoint{5.766920in}{2.912501in}}{\pgfqpoint{5.774733in}{2.904687in}}%
\pgfpathcurveto{\pgfqpoint{5.782547in}{2.896874in}}{\pgfqpoint{5.793146in}{2.892483in}}{\pgfqpoint{5.804196in}{2.892483in}}%
\pgfpathclose%
\pgfusepath{stroke,fill}%
\end{pgfscope}%
\begin{pgfscope}%
\pgfpathrectangle{\pgfqpoint{0.481978in}{0.331635in}}{\pgfqpoint{9.300000in}{7.700000in}}%
\pgfusepath{clip}%
\pgfsetbuttcap%
\pgfsetroundjoin%
\definecolor{currentfill}{rgb}{0.631373,0.788235,0.956863}%
\pgfsetfillcolor{currentfill}%
\pgfsetlinewidth{0.481800pt}%
\definecolor{currentstroke}{rgb}{1.000000,1.000000,1.000000}%
\pgfsetstrokecolor{currentstroke}%
\pgfsetdash{}{0pt}%
\pgfpathmoveto{\pgfqpoint{4.361412in}{6.101096in}}%
\pgfpathcurveto{\pgfqpoint{4.372462in}{6.101096in}}{\pgfqpoint{4.383061in}{6.105486in}}{\pgfqpoint{4.390875in}{6.113300in}}%
\pgfpathcurveto{\pgfqpoint{4.398689in}{6.121114in}}{\pgfqpoint{4.403079in}{6.131713in}}{\pgfqpoint{4.403079in}{6.142763in}}%
\pgfpathcurveto{\pgfqpoint{4.403079in}{6.153813in}}{\pgfqpoint{4.398689in}{6.164412in}}{\pgfqpoint{4.390875in}{6.172226in}}%
\pgfpathcurveto{\pgfqpoint{4.383061in}{6.180039in}}{\pgfqpoint{4.372462in}{6.184429in}}{\pgfqpoint{4.361412in}{6.184429in}}%
\pgfpathcurveto{\pgfqpoint{4.350362in}{6.184429in}}{\pgfqpoint{4.339763in}{6.180039in}}{\pgfqpoint{4.331949in}{6.172226in}}%
\pgfpathcurveto{\pgfqpoint{4.324136in}{6.164412in}}{\pgfqpoint{4.319745in}{6.153813in}}{\pgfqpoint{4.319745in}{6.142763in}}%
\pgfpathcurveto{\pgfqpoint{4.319745in}{6.131713in}}{\pgfqpoint{4.324136in}{6.121114in}}{\pgfqpoint{4.331949in}{6.113300in}}%
\pgfpathcurveto{\pgfqpoint{4.339763in}{6.105486in}}{\pgfqpoint{4.350362in}{6.101096in}}{\pgfqpoint{4.361412in}{6.101096in}}%
\pgfpathclose%
\pgfusepath{stroke,fill}%
\end{pgfscope}%
\begin{pgfscope}%
\pgfpathrectangle{\pgfqpoint{0.481978in}{0.331635in}}{\pgfqpoint{9.300000in}{7.700000in}}%
\pgfusepath{clip}%
\pgfsetbuttcap%
\pgfsetroundjoin%
\definecolor{currentfill}{rgb}{0.631373,0.788235,0.956863}%
\pgfsetfillcolor{currentfill}%
\pgfsetlinewidth{0.481800pt}%
\definecolor{currentstroke}{rgb}{1.000000,1.000000,1.000000}%
\pgfsetstrokecolor{currentstroke}%
\pgfsetdash{}{0pt}%
\pgfpathmoveto{\pgfqpoint{7.515857in}{5.108571in}}%
\pgfpathcurveto{\pgfqpoint{7.526908in}{5.108571in}}{\pgfqpoint{7.537507in}{5.112961in}}{\pgfqpoint{7.545320in}{5.120775in}}%
\pgfpathcurveto{\pgfqpoint{7.553134in}{5.128588in}}{\pgfqpoint{7.557524in}{5.139187in}}{\pgfqpoint{7.557524in}{5.150238in}}%
\pgfpathcurveto{\pgfqpoint{7.557524in}{5.161288in}}{\pgfqpoint{7.553134in}{5.171887in}}{\pgfqpoint{7.545320in}{5.179700in}}%
\pgfpathcurveto{\pgfqpoint{7.537507in}{5.187514in}}{\pgfqpoint{7.526908in}{5.191904in}}{\pgfqpoint{7.515857in}{5.191904in}}%
\pgfpathcurveto{\pgfqpoint{7.504807in}{5.191904in}}{\pgfqpoint{7.494208in}{5.187514in}}{\pgfqpoint{7.486395in}{5.179700in}}%
\pgfpathcurveto{\pgfqpoint{7.478581in}{5.171887in}}{\pgfqpoint{7.474191in}{5.161288in}}{\pgfqpoint{7.474191in}{5.150238in}}%
\pgfpathcurveto{\pgfqpoint{7.474191in}{5.139187in}}{\pgfqpoint{7.478581in}{5.128588in}}{\pgfqpoint{7.486395in}{5.120775in}}%
\pgfpathcurveto{\pgfqpoint{7.494208in}{5.112961in}}{\pgfqpoint{7.504807in}{5.108571in}}{\pgfqpoint{7.515857in}{5.108571in}}%
\pgfpathclose%
\pgfusepath{stroke,fill}%
\end{pgfscope}%
\begin{pgfscope}%
\pgfpathrectangle{\pgfqpoint{0.481978in}{0.331635in}}{\pgfqpoint{9.300000in}{7.700000in}}%
\pgfusepath{clip}%
\pgfsetbuttcap%
\pgfsetroundjoin%
\definecolor{currentfill}{rgb}{0.631373,0.788235,0.956863}%
\pgfsetfillcolor{currentfill}%
\pgfsetlinewidth{0.481800pt}%
\definecolor{currentstroke}{rgb}{1.000000,1.000000,1.000000}%
\pgfsetstrokecolor{currentstroke}%
\pgfsetdash{}{0pt}%
\pgfpathmoveto{\pgfqpoint{8.605266in}{4.793680in}}%
\pgfpathcurveto{\pgfqpoint{8.616316in}{4.793680in}}{\pgfqpoint{8.626915in}{4.798070in}}{\pgfqpoint{8.634729in}{4.805884in}}%
\pgfpathcurveto{\pgfqpoint{8.642542in}{4.813698in}}{\pgfqpoint{8.646933in}{4.824297in}}{\pgfqpoint{8.646933in}{4.835347in}}%
\pgfpathcurveto{\pgfqpoint{8.646933in}{4.846397in}}{\pgfqpoint{8.642542in}{4.856996in}}{\pgfqpoint{8.634729in}{4.864810in}}%
\pgfpathcurveto{\pgfqpoint{8.626915in}{4.872623in}}{\pgfqpoint{8.616316in}{4.877014in}}{\pgfqpoint{8.605266in}{4.877014in}}%
\pgfpathcurveto{\pgfqpoint{8.594216in}{4.877014in}}{\pgfqpoint{8.583617in}{4.872623in}}{\pgfqpoint{8.575803in}{4.864810in}}%
\pgfpathcurveto{\pgfqpoint{8.567990in}{4.856996in}}{\pgfqpoint{8.563599in}{4.846397in}}{\pgfqpoint{8.563599in}{4.835347in}}%
\pgfpathcurveto{\pgfqpoint{8.563599in}{4.824297in}}{\pgfqpoint{8.567990in}{4.813698in}}{\pgfqpoint{8.575803in}{4.805884in}}%
\pgfpathcurveto{\pgfqpoint{8.583617in}{4.798070in}}{\pgfqpoint{8.594216in}{4.793680in}}{\pgfqpoint{8.605266in}{4.793680in}}%
\pgfpathclose%
\pgfusepath{stroke,fill}%
\end{pgfscope}%
\begin{pgfscope}%
\pgfpathrectangle{\pgfqpoint{0.481978in}{0.331635in}}{\pgfqpoint{9.300000in}{7.700000in}}%
\pgfusepath{clip}%
\pgfsetbuttcap%
\pgfsetroundjoin%
\definecolor{currentfill}{rgb}{0.631373,0.788235,0.956863}%
\pgfsetfillcolor{currentfill}%
\pgfsetlinewidth{0.481800pt}%
\definecolor{currentstroke}{rgb}{1.000000,1.000000,1.000000}%
\pgfsetstrokecolor{currentstroke}%
\pgfsetdash{}{0pt}%
\pgfpathmoveto{\pgfqpoint{6.196847in}{1.907924in}}%
\pgfpathcurveto{\pgfqpoint{6.207897in}{1.907924in}}{\pgfqpoint{6.218496in}{1.912315in}}{\pgfqpoint{6.226310in}{1.920128in}}%
\pgfpathcurveto{\pgfqpoint{6.234124in}{1.927942in}}{\pgfqpoint{6.238514in}{1.938541in}}{\pgfqpoint{6.238514in}{1.949591in}}%
\pgfpathcurveto{\pgfqpoint{6.238514in}{1.960641in}}{\pgfqpoint{6.234124in}{1.971240in}}{\pgfqpoint{6.226310in}{1.979054in}}%
\pgfpathcurveto{\pgfqpoint{6.218496in}{1.986867in}}{\pgfqpoint{6.207897in}{1.991258in}}{\pgfqpoint{6.196847in}{1.991258in}}%
\pgfpathcurveto{\pgfqpoint{6.185797in}{1.991258in}}{\pgfqpoint{6.175198in}{1.986867in}}{\pgfqpoint{6.167384in}{1.979054in}}%
\pgfpathcurveto{\pgfqpoint{6.159571in}{1.971240in}}{\pgfqpoint{6.155181in}{1.960641in}}{\pgfqpoint{6.155181in}{1.949591in}}%
\pgfpathcurveto{\pgfqpoint{6.155181in}{1.938541in}}{\pgfqpoint{6.159571in}{1.927942in}}{\pgfqpoint{6.167384in}{1.920128in}}%
\pgfpathcurveto{\pgfqpoint{6.175198in}{1.912315in}}{\pgfqpoint{6.185797in}{1.907924in}}{\pgfqpoint{6.196847in}{1.907924in}}%
\pgfpathclose%
\pgfusepath{stroke,fill}%
\end{pgfscope}%
\begin{pgfscope}%
\pgfpathrectangle{\pgfqpoint{0.481978in}{0.331635in}}{\pgfqpoint{9.300000in}{7.700000in}}%
\pgfusepath{clip}%
\pgfsetbuttcap%
\pgfsetroundjoin%
\definecolor{currentfill}{rgb}{0.631373,0.788235,0.956863}%
\pgfsetfillcolor{currentfill}%
\pgfsetlinewidth{0.481800pt}%
\definecolor{currentstroke}{rgb}{1.000000,1.000000,1.000000}%
\pgfsetstrokecolor{currentstroke}%
\pgfsetdash{}{0pt}%
\pgfpathmoveto{\pgfqpoint{4.336260in}{2.021267in}}%
\pgfpathcurveto{\pgfqpoint{4.347310in}{2.021267in}}{\pgfqpoint{4.357909in}{2.025657in}}{\pgfqpoint{4.365723in}{2.033471in}}%
\pgfpathcurveto{\pgfqpoint{4.373537in}{2.041285in}}{\pgfqpoint{4.377927in}{2.051884in}}{\pgfqpoint{4.377927in}{2.062934in}}%
\pgfpathcurveto{\pgfqpoint{4.377927in}{2.073984in}}{\pgfqpoint{4.373537in}{2.084583in}}{\pgfqpoint{4.365723in}{2.092397in}}%
\pgfpathcurveto{\pgfqpoint{4.357909in}{2.100210in}}{\pgfqpoint{4.347310in}{2.104601in}}{\pgfqpoint{4.336260in}{2.104601in}}%
\pgfpathcurveto{\pgfqpoint{4.325210in}{2.104601in}}{\pgfqpoint{4.314611in}{2.100210in}}{\pgfqpoint{4.306797in}{2.092397in}}%
\pgfpathcurveto{\pgfqpoint{4.298984in}{2.084583in}}{\pgfqpoint{4.294593in}{2.073984in}}{\pgfqpoint{4.294593in}{2.062934in}}%
\pgfpathcurveto{\pgfqpoint{4.294593in}{2.051884in}}{\pgfqpoint{4.298984in}{2.041285in}}{\pgfqpoint{4.306797in}{2.033471in}}%
\pgfpathcurveto{\pgfqpoint{4.314611in}{2.025657in}}{\pgfqpoint{4.325210in}{2.021267in}}{\pgfqpoint{4.336260in}{2.021267in}}%
\pgfpathclose%
\pgfusepath{stroke,fill}%
\end{pgfscope}%
\begin{pgfscope}%
\pgfpathrectangle{\pgfqpoint{0.481978in}{0.331635in}}{\pgfqpoint{9.300000in}{7.700000in}}%
\pgfusepath{clip}%
\pgfsetbuttcap%
\pgfsetroundjoin%
\definecolor{currentfill}{rgb}{0.631373,0.788235,0.956863}%
\pgfsetfillcolor{currentfill}%
\pgfsetlinewidth{0.481800pt}%
\definecolor{currentstroke}{rgb}{1.000000,1.000000,1.000000}%
\pgfsetstrokecolor{currentstroke}%
\pgfsetdash{}{0pt}%
\pgfpathmoveto{\pgfqpoint{4.441169in}{5.370092in}}%
\pgfpathcurveto{\pgfqpoint{4.452219in}{5.370092in}}{\pgfqpoint{4.462818in}{5.374482in}}{\pgfqpoint{4.470632in}{5.382296in}}%
\pgfpathcurveto{\pgfqpoint{4.478445in}{5.390110in}}{\pgfqpoint{4.482836in}{5.400709in}}{\pgfqpoint{4.482836in}{5.411759in}}%
\pgfpathcurveto{\pgfqpoint{4.482836in}{5.422809in}}{\pgfqpoint{4.478445in}{5.433408in}}{\pgfqpoint{4.470632in}{5.441222in}}%
\pgfpathcurveto{\pgfqpoint{4.462818in}{5.449035in}}{\pgfqpoint{4.452219in}{5.453426in}}{\pgfqpoint{4.441169in}{5.453426in}}%
\pgfpathcurveto{\pgfqpoint{4.430119in}{5.453426in}}{\pgfqpoint{4.419520in}{5.449035in}}{\pgfqpoint{4.411706in}{5.441222in}}%
\pgfpathcurveto{\pgfqpoint{4.403893in}{5.433408in}}{\pgfqpoint{4.399502in}{5.422809in}}{\pgfqpoint{4.399502in}{5.411759in}}%
\pgfpathcurveto{\pgfqpoint{4.399502in}{5.400709in}}{\pgfqpoint{4.403893in}{5.390110in}}{\pgfqpoint{4.411706in}{5.382296in}}%
\pgfpathcurveto{\pgfqpoint{4.419520in}{5.374482in}}{\pgfqpoint{4.430119in}{5.370092in}}{\pgfqpoint{4.441169in}{5.370092in}}%
\pgfpathclose%
\pgfusepath{stroke,fill}%
\end{pgfscope}%
\begin{pgfscope}%
\pgfpathrectangle{\pgfqpoint{0.481978in}{0.331635in}}{\pgfqpoint{9.300000in}{7.700000in}}%
\pgfusepath{clip}%
\pgfsetbuttcap%
\pgfsetroundjoin%
\definecolor{currentfill}{rgb}{0.631373,0.788235,0.956863}%
\pgfsetfillcolor{currentfill}%
\pgfsetlinewidth{0.481800pt}%
\definecolor{currentstroke}{rgb}{1.000000,1.000000,1.000000}%
\pgfsetstrokecolor{currentstroke}%
\pgfsetdash{}{0pt}%
\pgfpathmoveto{\pgfqpoint{6.227427in}{1.656234in}}%
\pgfpathcurveto{\pgfqpoint{6.238477in}{1.656234in}}{\pgfqpoint{6.249076in}{1.660624in}}{\pgfqpoint{6.256889in}{1.668438in}}%
\pgfpathcurveto{\pgfqpoint{6.264703in}{1.676251in}}{\pgfqpoint{6.269093in}{1.686850in}}{\pgfqpoint{6.269093in}{1.697900in}}%
\pgfpathcurveto{\pgfqpoint{6.269093in}{1.708951in}}{\pgfqpoint{6.264703in}{1.719550in}}{\pgfqpoint{6.256889in}{1.727363in}}%
\pgfpathcurveto{\pgfqpoint{6.249076in}{1.735177in}}{\pgfqpoint{6.238477in}{1.739567in}}{\pgfqpoint{6.227427in}{1.739567in}}%
\pgfpathcurveto{\pgfqpoint{6.216376in}{1.739567in}}{\pgfqpoint{6.205777in}{1.735177in}}{\pgfqpoint{6.197964in}{1.727363in}}%
\pgfpathcurveto{\pgfqpoint{6.190150in}{1.719550in}}{\pgfqpoint{6.185760in}{1.708951in}}{\pgfqpoint{6.185760in}{1.697900in}}%
\pgfpathcurveto{\pgfqpoint{6.185760in}{1.686850in}}{\pgfqpoint{6.190150in}{1.676251in}}{\pgfqpoint{6.197964in}{1.668438in}}%
\pgfpathcurveto{\pgfqpoint{6.205777in}{1.660624in}}{\pgfqpoint{6.216376in}{1.656234in}}{\pgfqpoint{6.227427in}{1.656234in}}%
\pgfpathclose%
\pgfusepath{stroke,fill}%
\end{pgfscope}%
\begin{pgfscope}%
\pgfpathrectangle{\pgfqpoint{0.481978in}{0.331635in}}{\pgfqpoint{9.300000in}{7.700000in}}%
\pgfusepath{clip}%
\pgfsetbuttcap%
\pgfsetroundjoin%
\definecolor{currentfill}{rgb}{0.631373,0.788235,0.956863}%
\pgfsetfillcolor{currentfill}%
\pgfsetlinewidth{0.481800pt}%
\definecolor{currentstroke}{rgb}{1.000000,1.000000,1.000000}%
\pgfsetstrokecolor{currentstroke}%
\pgfsetdash{}{0pt}%
\pgfpathmoveto{\pgfqpoint{7.013858in}{2.218280in}}%
\pgfpathcurveto{\pgfqpoint{7.024908in}{2.218280in}}{\pgfqpoint{7.035507in}{2.222671in}}{\pgfqpoint{7.043321in}{2.230484in}}%
\pgfpathcurveto{\pgfqpoint{7.051135in}{2.238298in}}{\pgfqpoint{7.055525in}{2.248897in}}{\pgfqpoint{7.055525in}{2.259947in}}%
\pgfpathcurveto{\pgfqpoint{7.055525in}{2.270997in}}{\pgfqpoint{7.051135in}{2.281596in}}{\pgfqpoint{7.043321in}{2.289410in}}%
\pgfpathcurveto{\pgfqpoint{7.035507in}{2.297223in}}{\pgfqpoint{7.024908in}{2.301614in}}{\pgfqpoint{7.013858in}{2.301614in}}%
\pgfpathcurveto{\pgfqpoint{7.002808in}{2.301614in}}{\pgfqpoint{6.992209in}{2.297223in}}{\pgfqpoint{6.984395in}{2.289410in}}%
\pgfpathcurveto{\pgfqpoint{6.976582in}{2.281596in}}{\pgfqpoint{6.972191in}{2.270997in}}{\pgfqpoint{6.972191in}{2.259947in}}%
\pgfpathcurveto{\pgfqpoint{6.972191in}{2.248897in}}{\pgfqpoint{6.976582in}{2.238298in}}{\pgfqpoint{6.984395in}{2.230484in}}%
\pgfpathcurveto{\pgfqpoint{6.992209in}{2.222671in}}{\pgfqpoint{7.002808in}{2.218280in}}{\pgfqpoint{7.013858in}{2.218280in}}%
\pgfpathclose%
\pgfusepath{stroke,fill}%
\end{pgfscope}%
\begin{pgfscope}%
\pgfpathrectangle{\pgfqpoint{0.481978in}{0.331635in}}{\pgfqpoint{9.300000in}{7.700000in}}%
\pgfusepath{clip}%
\pgfsetbuttcap%
\pgfsetroundjoin%
\definecolor{currentfill}{rgb}{0.631373,0.788235,0.956863}%
\pgfsetfillcolor{currentfill}%
\pgfsetlinewidth{0.481800pt}%
\definecolor{currentstroke}{rgb}{1.000000,1.000000,1.000000}%
\pgfsetstrokecolor{currentstroke}%
\pgfsetdash{}{0pt}%
\pgfpathmoveto{\pgfqpoint{3.180339in}{6.470557in}}%
\pgfpathcurveto{\pgfqpoint{3.191389in}{6.470557in}}{\pgfqpoint{3.201988in}{6.474947in}}{\pgfqpoint{3.209801in}{6.482761in}}%
\pgfpathcurveto{\pgfqpoint{3.217615in}{6.490574in}}{\pgfqpoint{3.222005in}{6.501173in}}{\pgfqpoint{3.222005in}{6.512223in}}%
\pgfpathcurveto{\pgfqpoint{3.222005in}{6.523274in}}{\pgfqpoint{3.217615in}{6.533873in}}{\pgfqpoint{3.209801in}{6.541686in}}%
\pgfpathcurveto{\pgfqpoint{3.201988in}{6.549500in}}{\pgfqpoint{3.191389in}{6.553890in}}{\pgfqpoint{3.180339in}{6.553890in}}%
\pgfpathcurveto{\pgfqpoint{3.169289in}{6.553890in}}{\pgfqpoint{3.158689in}{6.549500in}}{\pgfqpoint{3.150876in}{6.541686in}}%
\pgfpathcurveto{\pgfqpoint{3.143062in}{6.533873in}}{\pgfqpoint{3.138672in}{6.523274in}}{\pgfqpoint{3.138672in}{6.512223in}}%
\pgfpathcurveto{\pgfqpoint{3.138672in}{6.501173in}}{\pgfqpoint{3.143062in}{6.490574in}}{\pgfqpoint{3.150876in}{6.482761in}}%
\pgfpathcurveto{\pgfqpoint{3.158689in}{6.474947in}}{\pgfqpoint{3.169289in}{6.470557in}}{\pgfqpoint{3.180339in}{6.470557in}}%
\pgfpathclose%
\pgfusepath{stroke,fill}%
\end{pgfscope}%
\begin{pgfscope}%
\pgfpathrectangle{\pgfqpoint{0.481978in}{0.331635in}}{\pgfqpoint{9.300000in}{7.700000in}}%
\pgfusepath{clip}%
\pgfsetbuttcap%
\pgfsetroundjoin%
\definecolor{currentfill}{rgb}{0.631373,0.788235,0.956863}%
\pgfsetfillcolor{currentfill}%
\pgfsetlinewidth{0.481800pt}%
\definecolor{currentstroke}{rgb}{1.000000,1.000000,1.000000}%
\pgfsetstrokecolor{currentstroke}%
\pgfsetdash{}{0pt}%
\pgfpathmoveto{\pgfqpoint{4.845738in}{0.823624in}}%
\pgfpathcurveto{\pgfqpoint{4.856788in}{0.823624in}}{\pgfqpoint{4.867387in}{0.828014in}}{\pgfqpoint{4.875201in}{0.835827in}}%
\pgfpathcurveto{\pgfqpoint{4.883015in}{0.843641in}}{\pgfqpoint{4.887405in}{0.854240in}}{\pgfqpoint{4.887405in}{0.865290in}}%
\pgfpathcurveto{\pgfqpoint{4.887405in}{0.876340in}}{\pgfqpoint{4.883015in}{0.886939in}}{\pgfqpoint{4.875201in}{0.894753in}}%
\pgfpathcurveto{\pgfqpoint{4.867387in}{0.902567in}}{\pgfqpoint{4.856788in}{0.906957in}}{\pgfqpoint{4.845738in}{0.906957in}}%
\pgfpathcurveto{\pgfqpoint{4.834688in}{0.906957in}}{\pgfqpoint{4.824089in}{0.902567in}}{\pgfqpoint{4.816275in}{0.894753in}}%
\pgfpathcurveto{\pgfqpoint{4.808462in}{0.886939in}}{\pgfqpoint{4.804071in}{0.876340in}}{\pgfqpoint{4.804071in}{0.865290in}}%
\pgfpathcurveto{\pgfqpoint{4.804071in}{0.854240in}}{\pgfqpoint{4.808462in}{0.843641in}}{\pgfqpoint{4.816275in}{0.835827in}}%
\pgfpathcurveto{\pgfqpoint{4.824089in}{0.828014in}}{\pgfqpoint{4.834688in}{0.823624in}}{\pgfqpoint{4.845738in}{0.823624in}}%
\pgfpathclose%
\pgfusepath{stroke,fill}%
\end{pgfscope}%
\begin{pgfscope}%
\pgfpathrectangle{\pgfqpoint{0.481978in}{0.331635in}}{\pgfqpoint{9.300000in}{7.700000in}}%
\pgfusepath{clip}%
\pgfsetbuttcap%
\pgfsetroundjoin%
\definecolor{currentfill}{rgb}{0.631373,0.788235,0.956863}%
\pgfsetfillcolor{currentfill}%
\pgfsetlinewidth{0.481800pt}%
\definecolor{currentstroke}{rgb}{1.000000,1.000000,1.000000}%
\pgfsetstrokecolor{currentstroke}%
\pgfsetdash{}{0pt}%
\pgfpathmoveto{\pgfqpoint{4.394203in}{1.807179in}}%
\pgfpathcurveto{\pgfqpoint{4.405253in}{1.807179in}}{\pgfqpoint{4.415852in}{1.811569in}}{\pgfqpoint{4.423666in}{1.819383in}}%
\pgfpathcurveto{\pgfqpoint{4.431480in}{1.827196in}}{\pgfqpoint{4.435870in}{1.837796in}}{\pgfqpoint{4.435870in}{1.848846in}}%
\pgfpathcurveto{\pgfqpoint{4.435870in}{1.859896in}}{\pgfqpoint{4.431480in}{1.870495in}}{\pgfqpoint{4.423666in}{1.878308in}}%
\pgfpathcurveto{\pgfqpoint{4.415852in}{1.886122in}}{\pgfqpoint{4.405253in}{1.890512in}}{\pgfqpoint{4.394203in}{1.890512in}}%
\pgfpathcurveto{\pgfqpoint{4.383153in}{1.890512in}}{\pgfqpoint{4.372554in}{1.886122in}}{\pgfqpoint{4.364740in}{1.878308in}}%
\pgfpathcurveto{\pgfqpoint{4.356927in}{1.870495in}}{\pgfqpoint{4.352537in}{1.859896in}}{\pgfqpoint{4.352537in}{1.848846in}}%
\pgfpathcurveto{\pgfqpoint{4.352537in}{1.837796in}}{\pgfqpoint{4.356927in}{1.827196in}}{\pgfqpoint{4.364740in}{1.819383in}}%
\pgfpathcurveto{\pgfqpoint{4.372554in}{1.811569in}}{\pgfqpoint{4.383153in}{1.807179in}}{\pgfqpoint{4.394203in}{1.807179in}}%
\pgfpathclose%
\pgfusepath{stroke,fill}%
\end{pgfscope}%
\begin{pgfscope}%
\pgfpathrectangle{\pgfqpoint{0.481978in}{0.331635in}}{\pgfqpoint{9.300000in}{7.700000in}}%
\pgfusepath{clip}%
\pgfsetbuttcap%
\pgfsetroundjoin%
\definecolor{currentfill}{rgb}{0.631373,0.788235,0.956863}%
\pgfsetfillcolor{currentfill}%
\pgfsetlinewidth{0.481800pt}%
\definecolor{currentstroke}{rgb}{1.000000,1.000000,1.000000}%
\pgfsetstrokecolor{currentstroke}%
\pgfsetdash{}{0pt}%
\pgfpathmoveto{\pgfqpoint{5.179568in}{5.314143in}}%
\pgfpathcurveto{\pgfqpoint{5.190618in}{5.314143in}}{\pgfqpoint{5.201217in}{5.318534in}}{\pgfqpoint{5.209031in}{5.326347in}}%
\pgfpathcurveto{\pgfqpoint{5.216845in}{5.334161in}}{\pgfqpoint{5.221235in}{5.344760in}}{\pgfqpoint{5.221235in}{5.355810in}}%
\pgfpathcurveto{\pgfqpoint{5.221235in}{5.366860in}}{\pgfqpoint{5.216845in}{5.377459in}}{\pgfqpoint{5.209031in}{5.385273in}}%
\pgfpathcurveto{\pgfqpoint{5.201217in}{5.393086in}}{\pgfqpoint{5.190618in}{5.397477in}}{\pgfqpoint{5.179568in}{5.397477in}}%
\pgfpathcurveto{\pgfqpoint{5.168518in}{5.397477in}}{\pgfqpoint{5.157919in}{5.393086in}}{\pgfqpoint{5.150105in}{5.385273in}}%
\pgfpathcurveto{\pgfqpoint{5.142292in}{5.377459in}}{\pgfqpoint{5.137901in}{5.366860in}}{\pgfqpoint{5.137901in}{5.355810in}}%
\pgfpathcurveto{\pgfqpoint{5.137901in}{5.344760in}}{\pgfqpoint{5.142292in}{5.334161in}}{\pgfqpoint{5.150105in}{5.326347in}}%
\pgfpathcurveto{\pgfqpoint{5.157919in}{5.318534in}}{\pgfqpoint{5.168518in}{5.314143in}}{\pgfqpoint{5.179568in}{5.314143in}}%
\pgfpathclose%
\pgfusepath{stroke,fill}%
\end{pgfscope}%
\begin{pgfscope}%
\pgfpathrectangle{\pgfqpoint{0.481978in}{0.331635in}}{\pgfqpoint{9.300000in}{7.700000in}}%
\pgfusepath{clip}%
\pgfsetbuttcap%
\pgfsetroundjoin%
\definecolor{currentfill}{rgb}{0.631373,0.788235,0.956863}%
\pgfsetfillcolor{currentfill}%
\pgfsetlinewidth{0.481800pt}%
\definecolor{currentstroke}{rgb}{1.000000,1.000000,1.000000}%
\pgfsetstrokecolor{currentstroke}%
\pgfsetdash{}{0pt}%
\pgfpathmoveto{\pgfqpoint{7.941055in}{4.945270in}}%
\pgfpathcurveto{\pgfqpoint{7.952105in}{4.945270in}}{\pgfqpoint{7.962704in}{4.949661in}}{\pgfqpoint{7.970518in}{4.957474in}}%
\pgfpathcurveto{\pgfqpoint{7.978331in}{4.965288in}}{\pgfqpoint{7.982722in}{4.975887in}}{\pgfqpoint{7.982722in}{4.986937in}}%
\pgfpathcurveto{\pgfqpoint{7.982722in}{4.997987in}}{\pgfqpoint{7.978331in}{5.008586in}}{\pgfqpoint{7.970518in}{5.016400in}}%
\pgfpathcurveto{\pgfqpoint{7.962704in}{5.024213in}}{\pgfqpoint{7.952105in}{5.028604in}}{\pgfqpoint{7.941055in}{5.028604in}}%
\pgfpathcurveto{\pgfqpoint{7.930005in}{5.028604in}}{\pgfqpoint{7.919406in}{5.024213in}}{\pgfqpoint{7.911592in}{5.016400in}}%
\pgfpathcurveto{\pgfqpoint{7.903779in}{5.008586in}}{\pgfqpoint{7.899388in}{4.997987in}}{\pgfqpoint{7.899388in}{4.986937in}}%
\pgfpathcurveto{\pgfqpoint{7.899388in}{4.975887in}}{\pgfqpoint{7.903779in}{4.965288in}}{\pgfqpoint{7.911592in}{4.957474in}}%
\pgfpathcurveto{\pgfqpoint{7.919406in}{4.949661in}}{\pgfqpoint{7.930005in}{4.945270in}}{\pgfqpoint{7.941055in}{4.945270in}}%
\pgfpathclose%
\pgfusepath{stroke,fill}%
\end{pgfscope}%
\begin{pgfscope}%
\pgfpathrectangle{\pgfqpoint{0.481978in}{0.331635in}}{\pgfqpoint{9.300000in}{7.700000in}}%
\pgfusepath{clip}%
\pgfsetbuttcap%
\pgfsetroundjoin%
\definecolor{currentfill}{rgb}{0.631373,0.788235,0.956863}%
\pgfsetfillcolor{currentfill}%
\pgfsetlinewidth{0.481800pt}%
\definecolor{currentstroke}{rgb}{1.000000,1.000000,1.000000}%
\pgfsetstrokecolor{currentstroke}%
\pgfsetdash{}{0pt}%
\pgfpathmoveto{\pgfqpoint{3.294139in}{1.535735in}}%
\pgfpathcurveto{\pgfqpoint{3.305189in}{1.535735in}}{\pgfqpoint{3.315788in}{1.540125in}}{\pgfqpoint{3.323601in}{1.547939in}}%
\pgfpathcurveto{\pgfqpoint{3.331415in}{1.555753in}}{\pgfqpoint{3.335805in}{1.566352in}}{\pgfqpoint{3.335805in}{1.577402in}}%
\pgfpathcurveto{\pgfqpoint{3.335805in}{1.588452in}}{\pgfqpoint{3.331415in}{1.599051in}}{\pgfqpoint{3.323601in}{1.606865in}}%
\pgfpathcurveto{\pgfqpoint{3.315788in}{1.614678in}}{\pgfqpoint{3.305189in}{1.619068in}}{\pgfqpoint{3.294139in}{1.619068in}}%
\pgfpathcurveto{\pgfqpoint{3.283088in}{1.619068in}}{\pgfqpoint{3.272489in}{1.614678in}}{\pgfqpoint{3.264676in}{1.606865in}}%
\pgfpathcurveto{\pgfqpoint{3.256862in}{1.599051in}}{\pgfqpoint{3.252472in}{1.588452in}}{\pgfqpoint{3.252472in}{1.577402in}}%
\pgfpathcurveto{\pgfqpoint{3.252472in}{1.566352in}}{\pgfqpoint{3.256862in}{1.555753in}}{\pgfqpoint{3.264676in}{1.547939in}}%
\pgfpathcurveto{\pgfqpoint{3.272489in}{1.540125in}}{\pgfqpoint{3.283088in}{1.535735in}}{\pgfqpoint{3.294139in}{1.535735in}}%
\pgfpathclose%
\pgfusepath{stroke,fill}%
\end{pgfscope}%
\begin{pgfscope}%
\pgfpathrectangle{\pgfqpoint{0.481978in}{0.331635in}}{\pgfqpoint{9.300000in}{7.700000in}}%
\pgfusepath{clip}%
\pgfsetbuttcap%
\pgfsetroundjoin%
\definecolor{currentfill}{rgb}{0.631373,0.788235,0.956863}%
\pgfsetfillcolor{currentfill}%
\pgfsetlinewidth{0.481800pt}%
\definecolor{currentstroke}{rgb}{1.000000,1.000000,1.000000}%
\pgfsetstrokecolor{currentstroke}%
\pgfsetdash{}{0pt}%
\pgfpathmoveto{\pgfqpoint{7.359994in}{3.989846in}}%
\pgfpathcurveto{\pgfqpoint{7.371044in}{3.989846in}}{\pgfqpoint{7.381643in}{3.994236in}}{\pgfqpoint{7.389457in}{4.002049in}}%
\pgfpathcurveto{\pgfqpoint{7.397270in}{4.009863in}}{\pgfqpoint{7.401661in}{4.020462in}}{\pgfqpoint{7.401661in}{4.031512in}}%
\pgfpathcurveto{\pgfqpoint{7.401661in}{4.042562in}}{\pgfqpoint{7.397270in}{4.053161in}}{\pgfqpoint{7.389457in}{4.060975in}}%
\pgfpathcurveto{\pgfqpoint{7.381643in}{4.068789in}}{\pgfqpoint{7.371044in}{4.073179in}}{\pgfqpoint{7.359994in}{4.073179in}}%
\pgfpathcurveto{\pgfqpoint{7.348944in}{4.073179in}}{\pgfqpoint{7.338345in}{4.068789in}}{\pgfqpoint{7.330531in}{4.060975in}}%
\pgfpathcurveto{\pgfqpoint{7.322718in}{4.053161in}}{\pgfqpoint{7.318327in}{4.042562in}}{\pgfqpoint{7.318327in}{4.031512in}}%
\pgfpathcurveto{\pgfqpoint{7.318327in}{4.020462in}}{\pgfqpoint{7.322718in}{4.009863in}}{\pgfqpoint{7.330531in}{4.002049in}}%
\pgfpathcurveto{\pgfqpoint{7.338345in}{3.994236in}}{\pgfqpoint{7.348944in}{3.989846in}}{\pgfqpoint{7.359994in}{3.989846in}}%
\pgfpathclose%
\pgfusepath{stroke,fill}%
\end{pgfscope}%
\begin{pgfscope}%
\pgfpathrectangle{\pgfqpoint{0.481978in}{0.331635in}}{\pgfqpoint{9.300000in}{7.700000in}}%
\pgfusepath{clip}%
\pgfsetbuttcap%
\pgfsetroundjoin%
\definecolor{currentfill}{rgb}{0.631373,0.788235,0.956863}%
\pgfsetfillcolor{currentfill}%
\pgfsetlinewidth{0.481800pt}%
\definecolor{currentstroke}{rgb}{1.000000,1.000000,1.000000}%
\pgfsetstrokecolor{currentstroke}%
\pgfsetdash{}{0pt}%
\pgfpathmoveto{\pgfqpoint{7.314751in}{4.873947in}}%
\pgfpathcurveto{\pgfqpoint{7.325801in}{4.873947in}}{\pgfqpoint{7.336400in}{4.878337in}}{\pgfqpoint{7.344213in}{4.886151in}}%
\pgfpathcurveto{\pgfqpoint{7.352027in}{4.893964in}}{\pgfqpoint{7.356417in}{4.904563in}}{\pgfqpoint{7.356417in}{4.915613in}}%
\pgfpathcurveto{\pgfqpoint{7.356417in}{4.926664in}}{\pgfqpoint{7.352027in}{4.937263in}}{\pgfqpoint{7.344213in}{4.945076in}}%
\pgfpathcurveto{\pgfqpoint{7.336400in}{4.952890in}}{\pgfqpoint{7.325801in}{4.957280in}}{\pgfqpoint{7.314751in}{4.957280in}}%
\pgfpathcurveto{\pgfqpoint{7.303700in}{4.957280in}}{\pgfqpoint{7.293101in}{4.952890in}}{\pgfqpoint{7.285288in}{4.945076in}}%
\pgfpathcurveto{\pgfqpoint{7.277474in}{4.937263in}}{\pgfqpoint{7.273084in}{4.926664in}}{\pgfqpoint{7.273084in}{4.915613in}}%
\pgfpathcurveto{\pgfqpoint{7.273084in}{4.904563in}}{\pgfqpoint{7.277474in}{4.893964in}}{\pgfqpoint{7.285288in}{4.886151in}}%
\pgfpathcurveto{\pgfqpoint{7.293101in}{4.878337in}}{\pgfqpoint{7.303700in}{4.873947in}}{\pgfqpoint{7.314751in}{4.873947in}}%
\pgfpathclose%
\pgfusepath{stroke,fill}%
\end{pgfscope}%
\begin{pgfscope}%
\pgfpathrectangle{\pgfqpoint{0.481978in}{0.331635in}}{\pgfqpoint{9.300000in}{7.700000in}}%
\pgfusepath{clip}%
\pgfsetbuttcap%
\pgfsetroundjoin%
\definecolor{currentfill}{rgb}{0.631373,0.788235,0.956863}%
\pgfsetfillcolor{currentfill}%
\pgfsetlinewidth{0.481800pt}%
\definecolor{currentstroke}{rgb}{1.000000,1.000000,1.000000}%
\pgfsetstrokecolor{currentstroke}%
\pgfsetdash{}{0pt}%
\pgfpathmoveto{\pgfqpoint{3.066979in}{6.640361in}}%
\pgfpathcurveto{\pgfqpoint{3.078029in}{6.640361in}}{\pgfqpoint{3.088628in}{6.644751in}}{\pgfqpoint{3.096442in}{6.652565in}}%
\pgfpathcurveto{\pgfqpoint{3.104255in}{6.660379in}}{\pgfqpoint{3.108646in}{6.670978in}}{\pgfqpoint{3.108646in}{6.682028in}}%
\pgfpathcurveto{\pgfqpoint{3.108646in}{6.693078in}}{\pgfqpoint{3.104255in}{6.703677in}}{\pgfqpoint{3.096442in}{6.711490in}}%
\pgfpathcurveto{\pgfqpoint{3.088628in}{6.719304in}}{\pgfqpoint{3.078029in}{6.723694in}}{\pgfqpoint{3.066979in}{6.723694in}}%
\pgfpathcurveto{\pgfqpoint{3.055929in}{6.723694in}}{\pgfqpoint{3.045330in}{6.719304in}}{\pgfqpoint{3.037516in}{6.711490in}}%
\pgfpathcurveto{\pgfqpoint{3.029702in}{6.703677in}}{\pgfqpoint{3.025312in}{6.693078in}}{\pgfqpoint{3.025312in}{6.682028in}}%
\pgfpathcurveto{\pgfqpoint{3.025312in}{6.670978in}}{\pgfqpoint{3.029702in}{6.660379in}}{\pgfqpoint{3.037516in}{6.652565in}}%
\pgfpathcurveto{\pgfqpoint{3.045330in}{6.644751in}}{\pgfqpoint{3.055929in}{6.640361in}}{\pgfqpoint{3.066979in}{6.640361in}}%
\pgfpathclose%
\pgfusepath{stroke,fill}%
\end{pgfscope}%
\begin{pgfscope}%
\pgfpathrectangle{\pgfqpoint{0.481978in}{0.331635in}}{\pgfqpoint{9.300000in}{7.700000in}}%
\pgfusepath{clip}%
\pgfsetbuttcap%
\pgfsetroundjoin%
\definecolor{currentfill}{rgb}{0.631373,0.788235,0.956863}%
\pgfsetfillcolor{currentfill}%
\pgfsetlinewidth{0.481800pt}%
\definecolor{currentstroke}{rgb}{1.000000,1.000000,1.000000}%
\pgfsetstrokecolor{currentstroke}%
\pgfsetdash{}{0pt}%
\pgfpathmoveto{\pgfqpoint{2.849648in}{6.868474in}}%
\pgfpathcurveto{\pgfqpoint{2.860698in}{6.868474in}}{\pgfqpoint{2.871297in}{6.872865in}}{\pgfqpoint{2.879110in}{6.880678in}}%
\pgfpathcurveto{\pgfqpoint{2.886924in}{6.888492in}}{\pgfqpoint{2.891314in}{6.899091in}}{\pgfqpoint{2.891314in}{6.910141in}}%
\pgfpathcurveto{\pgfqpoint{2.891314in}{6.921191in}}{\pgfqpoint{2.886924in}{6.931790in}}{\pgfqpoint{2.879110in}{6.939604in}}%
\pgfpathcurveto{\pgfqpoint{2.871297in}{6.947417in}}{\pgfqpoint{2.860698in}{6.951808in}}{\pgfqpoint{2.849648in}{6.951808in}}%
\pgfpathcurveto{\pgfqpoint{2.838597in}{6.951808in}}{\pgfqpoint{2.827998in}{6.947417in}}{\pgfqpoint{2.820185in}{6.939604in}}%
\pgfpathcurveto{\pgfqpoint{2.812371in}{6.931790in}}{\pgfqpoint{2.807981in}{6.921191in}}{\pgfqpoint{2.807981in}{6.910141in}}%
\pgfpathcurveto{\pgfqpoint{2.807981in}{6.899091in}}{\pgfqpoint{2.812371in}{6.888492in}}{\pgfqpoint{2.820185in}{6.880678in}}%
\pgfpathcurveto{\pgfqpoint{2.827998in}{6.872865in}}{\pgfqpoint{2.838597in}{6.868474in}}{\pgfqpoint{2.849648in}{6.868474in}}%
\pgfpathclose%
\pgfusepath{stroke,fill}%
\end{pgfscope}%
\begin{pgfscope}%
\pgfpathrectangle{\pgfqpoint{0.481978in}{0.331635in}}{\pgfqpoint{9.300000in}{7.700000in}}%
\pgfusepath{clip}%
\pgfsetbuttcap%
\pgfsetroundjoin%
\definecolor{currentfill}{rgb}{0.631373,0.788235,0.956863}%
\pgfsetfillcolor{currentfill}%
\pgfsetlinewidth{0.481800pt}%
\definecolor{currentstroke}{rgb}{1.000000,1.000000,1.000000}%
\pgfsetstrokecolor{currentstroke}%
\pgfsetdash{}{0pt}%
\pgfpathmoveto{\pgfqpoint{2.771056in}{6.623764in}}%
\pgfpathcurveto{\pgfqpoint{2.782106in}{6.623764in}}{\pgfqpoint{2.792705in}{6.628155in}}{\pgfqpoint{2.800519in}{6.635968in}}%
\pgfpathcurveto{\pgfqpoint{2.808332in}{6.643782in}}{\pgfqpoint{2.812722in}{6.654381in}}{\pgfqpoint{2.812722in}{6.665431in}}%
\pgfpathcurveto{\pgfqpoint{2.812722in}{6.676481in}}{\pgfqpoint{2.808332in}{6.687080in}}{\pgfqpoint{2.800519in}{6.694894in}}%
\pgfpathcurveto{\pgfqpoint{2.792705in}{6.702708in}}{\pgfqpoint{2.782106in}{6.707098in}}{\pgfqpoint{2.771056in}{6.707098in}}%
\pgfpathcurveto{\pgfqpoint{2.760006in}{6.707098in}}{\pgfqpoint{2.749407in}{6.702708in}}{\pgfqpoint{2.741593in}{6.694894in}}%
\pgfpathcurveto{\pgfqpoint{2.733779in}{6.687080in}}{\pgfqpoint{2.729389in}{6.676481in}}{\pgfqpoint{2.729389in}{6.665431in}}%
\pgfpathcurveto{\pgfqpoint{2.729389in}{6.654381in}}{\pgfqpoint{2.733779in}{6.643782in}}{\pgfqpoint{2.741593in}{6.635968in}}%
\pgfpathcurveto{\pgfqpoint{2.749407in}{6.628155in}}{\pgfqpoint{2.760006in}{6.623764in}}{\pgfqpoint{2.771056in}{6.623764in}}%
\pgfpathclose%
\pgfusepath{stroke,fill}%
\end{pgfscope}%
\begin{pgfscope}%
\pgfpathrectangle{\pgfqpoint{0.481978in}{0.331635in}}{\pgfqpoint{9.300000in}{7.700000in}}%
\pgfusepath{clip}%
\pgfsetbuttcap%
\pgfsetroundjoin%
\definecolor{currentfill}{rgb}{0.631373,0.788235,0.956863}%
\pgfsetfillcolor{currentfill}%
\pgfsetlinewidth{0.481800pt}%
\definecolor{currentstroke}{rgb}{1.000000,1.000000,1.000000}%
\pgfsetstrokecolor{currentstroke}%
\pgfsetdash{}{0pt}%
\pgfpathmoveto{\pgfqpoint{4.853047in}{5.880469in}}%
\pgfpathcurveto{\pgfqpoint{4.864097in}{5.880469in}}{\pgfqpoint{4.874696in}{5.884859in}}{\pgfqpoint{4.882510in}{5.892673in}}%
\pgfpathcurveto{\pgfqpoint{4.890324in}{5.900486in}}{\pgfqpoint{4.894714in}{5.911085in}}{\pgfqpoint{4.894714in}{5.922135in}}%
\pgfpathcurveto{\pgfqpoint{4.894714in}{5.933186in}}{\pgfqpoint{4.890324in}{5.943785in}}{\pgfqpoint{4.882510in}{5.951598in}}%
\pgfpathcurveto{\pgfqpoint{4.874696in}{5.959412in}}{\pgfqpoint{4.864097in}{5.963802in}}{\pgfqpoint{4.853047in}{5.963802in}}%
\pgfpathcurveto{\pgfqpoint{4.841997in}{5.963802in}}{\pgfqpoint{4.831398in}{5.959412in}}{\pgfqpoint{4.823584in}{5.951598in}}%
\pgfpathcurveto{\pgfqpoint{4.815771in}{5.943785in}}{\pgfqpoint{4.811381in}{5.933186in}}{\pgfqpoint{4.811381in}{5.922135in}}%
\pgfpathcurveto{\pgfqpoint{4.811381in}{5.911085in}}{\pgfqpoint{4.815771in}{5.900486in}}{\pgfqpoint{4.823584in}{5.892673in}}%
\pgfpathcurveto{\pgfqpoint{4.831398in}{5.884859in}}{\pgfqpoint{4.841997in}{5.880469in}}{\pgfqpoint{4.853047in}{5.880469in}}%
\pgfpathclose%
\pgfusepath{stroke,fill}%
\end{pgfscope}%
\begin{pgfscope}%
\pgfpathrectangle{\pgfqpoint{0.481978in}{0.331635in}}{\pgfqpoint{9.300000in}{7.700000in}}%
\pgfusepath{clip}%
\pgfsetbuttcap%
\pgfsetroundjoin%
\definecolor{currentfill}{rgb}{0.631373,0.788235,0.956863}%
\pgfsetfillcolor{currentfill}%
\pgfsetlinewidth{0.481800pt}%
\definecolor{currentstroke}{rgb}{1.000000,1.000000,1.000000}%
\pgfsetstrokecolor{currentstroke}%
\pgfsetdash{}{0pt}%
\pgfpathmoveto{\pgfqpoint{5.879927in}{0.802391in}}%
\pgfpathcurveto{\pgfqpoint{5.890977in}{0.802391in}}{\pgfqpoint{5.901576in}{0.806781in}}{\pgfqpoint{5.909390in}{0.814595in}}%
\pgfpathcurveto{\pgfqpoint{5.917204in}{0.822409in}}{\pgfqpoint{5.921594in}{0.833008in}}{\pgfqpoint{5.921594in}{0.844058in}}%
\pgfpathcurveto{\pgfqpoint{5.921594in}{0.855108in}}{\pgfqpoint{5.917204in}{0.865707in}}{\pgfqpoint{5.909390in}{0.873521in}}%
\pgfpathcurveto{\pgfqpoint{5.901576in}{0.881334in}}{\pgfqpoint{5.890977in}{0.885724in}}{\pgfqpoint{5.879927in}{0.885724in}}%
\pgfpathcurveto{\pgfqpoint{5.868877in}{0.885724in}}{\pgfqpoint{5.858278in}{0.881334in}}{\pgfqpoint{5.850465in}{0.873521in}}%
\pgfpathcurveto{\pgfqpoint{5.842651in}{0.865707in}}{\pgfqpoint{5.838261in}{0.855108in}}{\pgfqpoint{5.838261in}{0.844058in}}%
\pgfpathcurveto{\pgfqpoint{5.838261in}{0.833008in}}{\pgfqpoint{5.842651in}{0.822409in}}{\pgfqpoint{5.850465in}{0.814595in}}%
\pgfpathcurveto{\pgfqpoint{5.858278in}{0.806781in}}{\pgfqpoint{5.868877in}{0.802391in}}{\pgfqpoint{5.879927in}{0.802391in}}%
\pgfpathclose%
\pgfusepath{stroke,fill}%
\end{pgfscope}%
\begin{pgfscope}%
\pgfpathrectangle{\pgfqpoint{0.481978in}{0.331635in}}{\pgfqpoint{9.300000in}{7.700000in}}%
\pgfusepath{clip}%
\pgfsetbuttcap%
\pgfsetroundjoin%
\definecolor{currentfill}{rgb}{0.631373,0.788235,0.956863}%
\pgfsetfillcolor{currentfill}%
\pgfsetlinewidth{0.481800pt}%
\definecolor{currentstroke}{rgb}{1.000000,1.000000,1.000000}%
\pgfsetstrokecolor{currentstroke}%
\pgfsetdash{}{0pt}%
\pgfpathmoveto{\pgfqpoint{6.834691in}{2.041724in}}%
\pgfpathcurveto{\pgfqpoint{6.845741in}{2.041724in}}{\pgfqpoint{6.856340in}{2.046114in}}{\pgfqpoint{6.864153in}{2.053928in}}%
\pgfpathcurveto{\pgfqpoint{6.871967in}{2.061742in}}{\pgfqpoint{6.876357in}{2.072341in}}{\pgfqpoint{6.876357in}{2.083391in}}%
\pgfpathcurveto{\pgfqpoint{6.876357in}{2.094441in}}{\pgfqpoint{6.871967in}{2.105040in}}{\pgfqpoint{6.864153in}{2.112854in}}%
\pgfpathcurveto{\pgfqpoint{6.856340in}{2.120667in}}{\pgfqpoint{6.845741in}{2.125058in}}{\pgfqpoint{6.834691in}{2.125058in}}%
\pgfpathcurveto{\pgfqpoint{6.823641in}{2.125058in}}{\pgfqpoint{6.813041in}{2.120667in}}{\pgfqpoint{6.805228in}{2.112854in}}%
\pgfpathcurveto{\pgfqpoint{6.797414in}{2.105040in}}{\pgfqpoint{6.793024in}{2.094441in}}{\pgfqpoint{6.793024in}{2.083391in}}%
\pgfpathcurveto{\pgfqpoint{6.793024in}{2.072341in}}{\pgfqpoint{6.797414in}{2.061742in}}{\pgfqpoint{6.805228in}{2.053928in}}%
\pgfpathcurveto{\pgfqpoint{6.813041in}{2.046114in}}{\pgfqpoint{6.823641in}{2.041724in}}{\pgfqpoint{6.834691in}{2.041724in}}%
\pgfpathclose%
\pgfusepath{stroke,fill}%
\end{pgfscope}%
\begin{pgfscope}%
\pgfpathrectangle{\pgfqpoint{0.481978in}{0.331635in}}{\pgfqpoint{9.300000in}{7.700000in}}%
\pgfusepath{clip}%
\pgfsetbuttcap%
\pgfsetroundjoin%
\definecolor{currentfill}{rgb}{0.631373,0.788235,0.956863}%
\pgfsetfillcolor{currentfill}%
\pgfsetlinewidth{0.481800pt}%
\definecolor{currentstroke}{rgb}{1.000000,1.000000,1.000000}%
\pgfsetstrokecolor{currentstroke}%
\pgfsetdash{}{0pt}%
\pgfpathmoveto{\pgfqpoint{5.297141in}{2.737467in}}%
\pgfpathcurveto{\pgfqpoint{5.308191in}{2.737467in}}{\pgfqpoint{5.318790in}{2.741857in}}{\pgfqpoint{5.326604in}{2.749671in}}%
\pgfpathcurveto{\pgfqpoint{5.334417in}{2.757484in}}{\pgfqpoint{5.338808in}{2.768083in}}{\pgfqpoint{5.338808in}{2.779134in}}%
\pgfpathcurveto{\pgfqpoint{5.338808in}{2.790184in}}{\pgfqpoint{5.334417in}{2.800783in}}{\pgfqpoint{5.326604in}{2.808596in}}%
\pgfpathcurveto{\pgfqpoint{5.318790in}{2.816410in}}{\pgfqpoint{5.308191in}{2.820800in}}{\pgfqpoint{5.297141in}{2.820800in}}%
\pgfpathcurveto{\pgfqpoint{5.286091in}{2.820800in}}{\pgfqpoint{5.275492in}{2.816410in}}{\pgfqpoint{5.267678in}{2.808596in}}%
\pgfpathcurveto{\pgfqpoint{5.259864in}{2.800783in}}{\pgfqpoint{5.255474in}{2.790184in}}{\pgfqpoint{5.255474in}{2.779134in}}%
\pgfpathcurveto{\pgfqpoint{5.255474in}{2.768083in}}{\pgfqpoint{5.259864in}{2.757484in}}{\pgfqpoint{5.267678in}{2.749671in}}%
\pgfpathcurveto{\pgfqpoint{5.275492in}{2.741857in}}{\pgfqpoint{5.286091in}{2.737467in}}{\pgfqpoint{5.297141in}{2.737467in}}%
\pgfpathclose%
\pgfusepath{stroke,fill}%
\end{pgfscope}%
\begin{pgfscope}%
\pgfpathrectangle{\pgfqpoint{0.481978in}{0.331635in}}{\pgfqpoint{9.300000in}{7.700000in}}%
\pgfusepath{clip}%
\pgfsetbuttcap%
\pgfsetroundjoin%
\definecolor{currentfill}{rgb}{0.631373,0.788235,0.956863}%
\pgfsetfillcolor{currentfill}%
\pgfsetlinewidth{0.481800pt}%
\definecolor{currentstroke}{rgb}{1.000000,1.000000,1.000000}%
\pgfsetstrokecolor{currentstroke}%
\pgfsetdash{}{0pt}%
\pgfpathmoveto{\pgfqpoint{6.713307in}{5.141056in}}%
\pgfpathcurveto{\pgfqpoint{6.724358in}{5.141056in}}{\pgfqpoint{6.734957in}{5.145446in}}{\pgfqpoint{6.742770in}{5.153260in}}%
\pgfpathcurveto{\pgfqpoint{6.750584in}{5.161073in}}{\pgfqpoint{6.754974in}{5.171673in}}{\pgfqpoint{6.754974in}{5.182723in}}%
\pgfpathcurveto{\pgfqpoint{6.754974in}{5.193773in}}{\pgfqpoint{6.750584in}{5.204372in}}{\pgfqpoint{6.742770in}{5.212185in}}%
\pgfpathcurveto{\pgfqpoint{6.734957in}{5.219999in}}{\pgfqpoint{6.724358in}{5.224389in}}{\pgfqpoint{6.713307in}{5.224389in}}%
\pgfpathcurveto{\pgfqpoint{6.702257in}{5.224389in}}{\pgfqpoint{6.691658in}{5.219999in}}{\pgfqpoint{6.683845in}{5.212185in}}%
\pgfpathcurveto{\pgfqpoint{6.676031in}{5.204372in}}{\pgfqpoint{6.671641in}{5.193773in}}{\pgfqpoint{6.671641in}{5.182723in}}%
\pgfpathcurveto{\pgfqpoint{6.671641in}{5.171673in}}{\pgfqpoint{6.676031in}{5.161073in}}{\pgfqpoint{6.683845in}{5.153260in}}%
\pgfpathcurveto{\pgfqpoint{6.691658in}{5.145446in}}{\pgfqpoint{6.702257in}{5.141056in}}{\pgfqpoint{6.713307in}{5.141056in}}%
\pgfpathclose%
\pgfusepath{stroke,fill}%
\end{pgfscope}%
\begin{pgfscope}%
\pgfpathrectangle{\pgfqpoint{0.481978in}{0.331635in}}{\pgfqpoint{9.300000in}{7.700000in}}%
\pgfusepath{clip}%
\pgfsetbuttcap%
\pgfsetroundjoin%
\definecolor{currentfill}{rgb}{0.631373,0.788235,0.956863}%
\pgfsetfillcolor{currentfill}%
\pgfsetlinewidth{0.481800pt}%
\definecolor{currentstroke}{rgb}{1.000000,1.000000,1.000000}%
\pgfsetstrokecolor{currentstroke}%
\pgfsetdash{}{0pt}%
\pgfpathmoveto{\pgfqpoint{5.117952in}{2.310213in}}%
\pgfpathcurveto{\pgfqpoint{5.129002in}{2.310213in}}{\pgfqpoint{5.139601in}{2.314603in}}{\pgfqpoint{5.147415in}{2.322417in}}%
\pgfpathcurveto{\pgfqpoint{5.155228in}{2.330230in}}{\pgfqpoint{5.159618in}{2.340829in}}{\pgfqpoint{5.159618in}{2.351880in}}%
\pgfpathcurveto{\pgfqpoint{5.159618in}{2.362930in}}{\pgfqpoint{5.155228in}{2.373529in}}{\pgfqpoint{5.147415in}{2.381342in}}%
\pgfpathcurveto{\pgfqpoint{5.139601in}{2.389156in}}{\pgfqpoint{5.129002in}{2.393546in}}{\pgfqpoint{5.117952in}{2.393546in}}%
\pgfpathcurveto{\pgfqpoint{5.106902in}{2.393546in}}{\pgfqpoint{5.096303in}{2.389156in}}{\pgfqpoint{5.088489in}{2.381342in}}%
\pgfpathcurveto{\pgfqpoint{5.080675in}{2.373529in}}{\pgfqpoint{5.076285in}{2.362930in}}{\pgfqpoint{5.076285in}{2.351880in}}%
\pgfpathcurveto{\pgfqpoint{5.076285in}{2.340829in}}{\pgfqpoint{5.080675in}{2.330230in}}{\pgfqpoint{5.088489in}{2.322417in}}%
\pgfpathcurveto{\pgfqpoint{5.096303in}{2.314603in}}{\pgfqpoint{5.106902in}{2.310213in}}{\pgfqpoint{5.117952in}{2.310213in}}%
\pgfpathclose%
\pgfusepath{stroke,fill}%
\end{pgfscope}%
\begin{pgfscope}%
\pgfpathrectangle{\pgfqpoint{0.481978in}{0.331635in}}{\pgfqpoint{9.300000in}{7.700000in}}%
\pgfusepath{clip}%
\pgfsetbuttcap%
\pgfsetroundjoin%
\definecolor{currentfill}{rgb}{0.631373,0.788235,0.956863}%
\pgfsetfillcolor{currentfill}%
\pgfsetlinewidth{0.481800pt}%
\definecolor{currentstroke}{rgb}{1.000000,1.000000,1.000000}%
\pgfsetstrokecolor{currentstroke}%
\pgfsetdash{}{0pt}%
\pgfpathmoveto{\pgfqpoint{7.076892in}{3.052698in}}%
\pgfpathcurveto{\pgfqpoint{7.087942in}{3.052698in}}{\pgfqpoint{7.098541in}{3.057088in}}{\pgfqpoint{7.106355in}{3.064902in}}%
\pgfpathcurveto{\pgfqpoint{7.114168in}{3.072716in}}{\pgfqpoint{7.118559in}{3.083315in}}{\pgfqpoint{7.118559in}{3.094365in}}%
\pgfpathcurveto{\pgfqpoint{7.118559in}{3.105415in}}{\pgfqpoint{7.114168in}{3.116014in}}{\pgfqpoint{7.106355in}{3.123828in}}%
\pgfpathcurveto{\pgfqpoint{7.098541in}{3.131641in}}{\pgfqpoint{7.087942in}{3.136031in}}{\pgfqpoint{7.076892in}{3.136031in}}%
\pgfpathcurveto{\pgfqpoint{7.065842in}{3.136031in}}{\pgfqpoint{7.055243in}{3.131641in}}{\pgfqpoint{7.047429in}{3.123828in}}%
\pgfpathcurveto{\pgfqpoint{7.039616in}{3.116014in}}{\pgfqpoint{7.035225in}{3.105415in}}{\pgfqpoint{7.035225in}{3.094365in}}%
\pgfpathcurveto{\pgfqpoint{7.035225in}{3.083315in}}{\pgfqpoint{7.039616in}{3.072716in}}{\pgfqpoint{7.047429in}{3.064902in}}%
\pgfpathcurveto{\pgfqpoint{7.055243in}{3.057088in}}{\pgfqpoint{7.065842in}{3.052698in}}{\pgfqpoint{7.076892in}{3.052698in}}%
\pgfpathclose%
\pgfusepath{stroke,fill}%
\end{pgfscope}%
\begin{pgfscope}%
\pgfpathrectangle{\pgfqpoint{0.481978in}{0.331635in}}{\pgfqpoint{9.300000in}{7.700000in}}%
\pgfusepath{clip}%
\pgfsetbuttcap%
\pgfsetroundjoin%
\definecolor{currentfill}{rgb}{0.631373,0.788235,0.956863}%
\pgfsetfillcolor{currentfill}%
\pgfsetlinewidth{0.481800pt}%
\definecolor{currentstroke}{rgb}{1.000000,1.000000,1.000000}%
\pgfsetstrokecolor{currentstroke}%
\pgfsetdash{}{0pt}%
\pgfpathmoveto{\pgfqpoint{7.507895in}{1.911864in}}%
\pgfpathcurveto{\pgfqpoint{7.518945in}{1.911864in}}{\pgfqpoint{7.529544in}{1.916254in}}{\pgfqpoint{7.537358in}{1.924068in}}%
\pgfpathcurveto{\pgfqpoint{7.545171in}{1.931881in}}{\pgfqpoint{7.549562in}{1.942480in}}{\pgfqpoint{7.549562in}{1.953530in}}%
\pgfpathcurveto{\pgfqpoint{7.549562in}{1.964581in}}{\pgfqpoint{7.545171in}{1.975180in}}{\pgfqpoint{7.537358in}{1.982993in}}%
\pgfpathcurveto{\pgfqpoint{7.529544in}{1.990807in}}{\pgfqpoint{7.518945in}{1.995197in}}{\pgfqpoint{7.507895in}{1.995197in}}%
\pgfpathcurveto{\pgfqpoint{7.496845in}{1.995197in}}{\pgfqpoint{7.486246in}{1.990807in}}{\pgfqpoint{7.478432in}{1.982993in}}%
\pgfpathcurveto{\pgfqpoint{7.470619in}{1.975180in}}{\pgfqpoint{7.466228in}{1.964581in}}{\pgfqpoint{7.466228in}{1.953530in}}%
\pgfpathcurveto{\pgfqpoint{7.466228in}{1.942480in}}{\pgfqpoint{7.470619in}{1.931881in}}{\pgfqpoint{7.478432in}{1.924068in}}%
\pgfpathcurveto{\pgfqpoint{7.486246in}{1.916254in}}{\pgfqpoint{7.496845in}{1.911864in}}{\pgfqpoint{7.507895in}{1.911864in}}%
\pgfpathclose%
\pgfusepath{stroke,fill}%
\end{pgfscope}%
\begin{pgfscope}%
\pgfpathrectangle{\pgfqpoint{0.481978in}{0.331635in}}{\pgfqpoint{9.300000in}{7.700000in}}%
\pgfusepath{clip}%
\pgfsetbuttcap%
\pgfsetroundjoin%
\definecolor{currentfill}{rgb}{0.631373,0.788235,0.956863}%
\pgfsetfillcolor{currentfill}%
\pgfsetlinewidth{0.481800pt}%
\definecolor{currentstroke}{rgb}{1.000000,1.000000,1.000000}%
\pgfsetstrokecolor{currentstroke}%
\pgfsetdash{}{0pt}%
\pgfpathmoveto{\pgfqpoint{4.102323in}{5.788720in}}%
\pgfpathcurveto{\pgfqpoint{4.113373in}{5.788720in}}{\pgfqpoint{4.123972in}{5.793111in}}{\pgfqpoint{4.131786in}{5.800924in}}%
\pgfpathcurveto{\pgfqpoint{4.139599in}{5.808738in}}{\pgfqpoint{4.143989in}{5.819337in}}{\pgfqpoint{4.143989in}{5.830387in}}%
\pgfpathcurveto{\pgfqpoint{4.143989in}{5.841437in}}{\pgfqpoint{4.139599in}{5.852036in}}{\pgfqpoint{4.131786in}{5.859850in}}%
\pgfpathcurveto{\pgfqpoint{4.123972in}{5.867663in}}{\pgfqpoint{4.113373in}{5.872054in}}{\pgfqpoint{4.102323in}{5.872054in}}%
\pgfpathcurveto{\pgfqpoint{4.091273in}{5.872054in}}{\pgfqpoint{4.080674in}{5.867663in}}{\pgfqpoint{4.072860in}{5.859850in}}%
\pgfpathcurveto{\pgfqpoint{4.065046in}{5.852036in}}{\pgfqpoint{4.060656in}{5.841437in}}{\pgfqpoint{4.060656in}{5.830387in}}%
\pgfpathcurveto{\pgfqpoint{4.060656in}{5.819337in}}{\pgfqpoint{4.065046in}{5.808738in}}{\pgfqpoint{4.072860in}{5.800924in}}%
\pgfpathcurveto{\pgfqpoint{4.080674in}{5.793111in}}{\pgfqpoint{4.091273in}{5.788720in}}{\pgfqpoint{4.102323in}{5.788720in}}%
\pgfpathclose%
\pgfusepath{stroke,fill}%
\end{pgfscope}%
\begin{pgfscope}%
\pgfpathrectangle{\pgfqpoint{0.481978in}{0.331635in}}{\pgfqpoint{9.300000in}{7.700000in}}%
\pgfusepath{clip}%
\pgfsetbuttcap%
\pgfsetroundjoin%
\definecolor{currentfill}{rgb}{0.631373,0.788235,0.956863}%
\pgfsetfillcolor{currentfill}%
\pgfsetlinewidth{0.481800pt}%
\definecolor{currentstroke}{rgb}{1.000000,1.000000,1.000000}%
\pgfsetstrokecolor{currentstroke}%
\pgfsetdash{}{0pt}%
\pgfpathmoveto{\pgfqpoint{6.020143in}{5.274735in}}%
\pgfpathcurveto{\pgfqpoint{6.031193in}{5.274735in}}{\pgfqpoint{6.041792in}{5.279126in}}{\pgfqpoint{6.049606in}{5.286939in}}%
\pgfpathcurveto{\pgfqpoint{6.057419in}{5.294753in}}{\pgfqpoint{6.061810in}{5.305352in}}{\pgfqpoint{6.061810in}{5.316402in}}%
\pgfpathcurveto{\pgfqpoint{6.061810in}{5.327452in}}{\pgfqpoint{6.057419in}{5.338051in}}{\pgfqpoint{6.049606in}{5.345865in}}%
\pgfpathcurveto{\pgfqpoint{6.041792in}{5.353678in}}{\pgfqpoint{6.031193in}{5.358069in}}{\pgfqpoint{6.020143in}{5.358069in}}%
\pgfpathcurveto{\pgfqpoint{6.009093in}{5.358069in}}{\pgfqpoint{5.998494in}{5.353678in}}{\pgfqpoint{5.990680in}{5.345865in}}%
\pgfpathcurveto{\pgfqpoint{5.982867in}{5.338051in}}{\pgfqpoint{5.978476in}{5.327452in}}{\pgfqpoint{5.978476in}{5.316402in}}%
\pgfpathcurveto{\pgfqpoint{5.978476in}{5.305352in}}{\pgfqpoint{5.982867in}{5.294753in}}{\pgfqpoint{5.990680in}{5.286939in}}%
\pgfpathcurveto{\pgfqpoint{5.998494in}{5.279126in}}{\pgfqpoint{6.009093in}{5.274735in}}{\pgfqpoint{6.020143in}{5.274735in}}%
\pgfpathclose%
\pgfusepath{stroke,fill}%
\end{pgfscope}%
\begin{pgfscope}%
\pgfpathrectangle{\pgfqpoint{0.481978in}{0.331635in}}{\pgfqpoint{9.300000in}{7.700000in}}%
\pgfusepath{clip}%
\pgfsetbuttcap%
\pgfsetroundjoin%
\definecolor{currentfill}{rgb}{0.631373,0.788235,0.956863}%
\pgfsetfillcolor{currentfill}%
\pgfsetlinewidth{0.481800pt}%
\definecolor{currentstroke}{rgb}{1.000000,1.000000,1.000000}%
\pgfsetstrokecolor{currentstroke}%
\pgfsetdash{}{0pt}%
\pgfpathmoveto{\pgfqpoint{7.865979in}{5.764555in}}%
\pgfpathcurveto{\pgfqpoint{7.877029in}{5.764555in}}{\pgfqpoint{7.887628in}{5.768945in}}{\pgfqpoint{7.895442in}{5.776759in}}%
\pgfpathcurveto{\pgfqpoint{7.903255in}{5.784573in}}{\pgfqpoint{7.907645in}{5.795172in}}{\pgfqpoint{7.907645in}{5.806222in}}%
\pgfpathcurveto{\pgfqpoint{7.907645in}{5.817272in}}{\pgfqpoint{7.903255in}{5.827871in}}{\pgfqpoint{7.895442in}{5.835685in}}%
\pgfpathcurveto{\pgfqpoint{7.887628in}{5.843498in}}{\pgfqpoint{7.877029in}{5.847888in}}{\pgfqpoint{7.865979in}{5.847888in}}%
\pgfpathcurveto{\pgfqpoint{7.854929in}{5.847888in}}{\pgfqpoint{7.844330in}{5.843498in}}{\pgfqpoint{7.836516in}{5.835685in}}%
\pgfpathcurveto{\pgfqpoint{7.828702in}{5.827871in}}{\pgfqpoint{7.824312in}{5.817272in}}{\pgfqpoint{7.824312in}{5.806222in}}%
\pgfpathcurveto{\pgfqpoint{7.824312in}{5.795172in}}{\pgfqpoint{7.828702in}{5.784573in}}{\pgfqpoint{7.836516in}{5.776759in}}%
\pgfpathcurveto{\pgfqpoint{7.844330in}{5.768945in}}{\pgfqpoint{7.854929in}{5.764555in}}{\pgfqpoint{7.865979in}{5.764555in}}%
\pgfpathclose%
\pgfusepath{stroke,fill}%
\end{pgfscope}%
\begin{pgfscope}%
\pgfpathrectangle{\pgfqpoint{0.481978in}{0.331635in}}{\pgfqpoint{9.300000in}{7.700000in}}%
\pgfusepath{clip}%
\pgfsetbuttcap%
\pgfsetroundjoin%
\definecolor{currentfill}{rgb}{0.631373,0.788235,0.956863}%
\pgfsetfillcolor{currentfill}%
\pgfsetlinewidth{0.481800pt}%
\definecolor{currentstroke}{rgb}{1.000000,1.000000,1.000000}%
\pgfsetstrokecolor{currentstroke}%
\pgfsetdash{}{0pt}%
\pgfpathmoveto{\pgfqpoint{5.392504in}{5.371877in}}%
\pgfpathcurveto{\pgfqpoint{5.403554in}{5.371877in}}{\pgfqpoint{5.414153in}{5.376268in}}{\pgfqpoint{5.421967in}{5.384081in}}%
\pgfpathcurveto{\pgfqpoint{5.429780in}{5.391895in}}{\pgfqpoint{5.434171in}{5.402494in}}{\pgfqpoint{5.434171in}{5.413544in}}%
\pgfpathcurveto{\pgfqpoint{5.434171in}{5.424594in}}{\pgfqpoint{5.429780in}{5.435193in}}{\pgfqpoint{5.421967in}{5.443007in}}%
\pgfpathcurveto{\pgfqpoint{5.414153in}{5.450820in}}{\pgfqpoint{5.403554in}{5.455211in}}{\pgfqpoint{5.392504in}{5.455211in}}%
\pgfpathcurveto{\pgfqpoint{5.381454in}{5.455211in}}{\pgfqpoint{5.370855in}{5.450820in}}{\pgfqpoint{5.363041in}{5.443007in}}%
\pgfpathcurveto{\pgfqpoint{5.355228in}{5.435193in}}{\pgfqpoint{5.350837in}{5.424594in}}{\pgfqpoint{5.350837in}{5.413544in}}%
\pgfpathcurveto{\pgfqpoint{5.350837in}{5.402494in}}{\pgfqpoint{5.355228in}{5.391895in}}{\pgfqpoint{5.363041in}{5.384081in}}%
\pgfpathcurveto{\pgfqpoint{5.370855in}{5.376268in}}{\pgfqpoint{5.381454in}{5.371877in}}{\pgfqpoint{5.392504in}{5.371877in}}%
\pgfpathclose%
\pgfusepath{stroke,fill}%
\end{pgfscope}%
\begin{pgfscope}%
\pgfpathrectangle{\pgfqpoint{0.481978in}{0.331635in}}{\pgfqpoint{9.300000in}{7.700000in}}%
\pgfusepath{clip}%
\pgfsetbuttcap%
\pgfsetroundjoin%
\definecolor{currentfill}{rgb}{0.631373,0.788235,0.956863}%
\pgfsetfillcolor{currentfill}%
\pgfsetlinewidth{0.481800pt}%
\definecolor{currentstroke}{rgb}{1.000000,1.000000,1.000000}%
\pgfsetstrokecolor{currentstroke}%
\pgfsetdash{}{0pt}%
\pgfpathmoveto{\pgfqpoint{2.932058in}{5.382422in}}%
\pgfpathcurveto{\pgfqpoint{2.943108in}{5.382422in}}{\pgfqpoint{2.953707in}{5.386813in}}{\pgfqpoint{2.961521in}{5.394626in}}%
\pgfpathcurveto{\pgfqpoint{2.969335in}{5.402440in}}{\pgfqpoint{2.973725in}{5.413039in}}{\pgfqpoint{2.973725in}{5.424089in}}%
\pgfpathcurveto{\pgfqpoint{2.973725in}{5.435139in}}{\pgfqpoint{2.969335in}{5.445738in}}{\pgfqpoint{2.961521in}{5.453552in}}%
\pgfpathcurveto{\pgfqpoint{2.953707in}{5.461366in}}{\pgfqpoint{2.943108in}{5.465756in}}{\pgfqpoint{2.932058in}{5.465756in}}%
\pgfpathcurveto{\pgfqpoint{2.921008in}{5.465756in}}{\pgfqpoint{2.910409in}{5.461366in}}{\pgfqpoint{2.902595in}{5.453552in}}%
\pgfpathcurveto{\pgfqpoint{2.894782in}{5.445738in}}{\pgfqpoint{2.890391in}{5.435139in}}{\pgfqpoint{2.890391in}{5.424089in}}%
\pgfpathcurveto{\pgfqpoint{2.890391in}{5.413039in}}{\pgfqpoint{2.894782in}{5.402440in}}{\pgfqpoint{2.902595in}{5.394626in}}%
\pgfpathcurveto{\pgfqpoint{2.910409in}{5.386813in}}{\pgfqpoint{2.921008in}{5.382422in}}{\pgfqpoint{2.932058in}{5.382422in}}%
\pgfpathclose%
\pgfusepath{stroke,fill}%
\end{pgfscope}%
\begin{pgfscope}%
\pgfpathrectangle{\pgfqpoint{0.481978in}{0.331635in}}{\pgfqpoint{9.300000in}{7.700000in}}%
\pgfusepath{clip}%
\pgfsetbuttcap%
\pgfsetroundjoin%
\definecolor{currentfill}{rgb}{0.631373,0.788235,0.956863}%
\pgfsetfillcolor{currentfill}%
\pgfsetlinewidth{0.481800pt}%
\definecolor{currentstroke}{rgb}{1.000000,1.000000,1.000000}%
\pgfsetstrokecolor{currentstroke}%
\pgfsetdash{}{0pt}%
\pgfpathmoveto{\pgfqpoint{7.779256in}{5.038032in}}%
\pgfpathcurveto{\pgfqpoint{7.790306in}{5.038032in}}{\pgfqpoint{7.800905in}{5.042422in}}{\pgfqpoint{7.808718in}{5.050236in}}%
\pgfpathcurveto{\pgfqpoint{7.816532in}{5.058050in}}{\pgfqpoint{7.820922in}{5.068649in}}{\pgfqpoint{7.820922in}{5.079699in}}%
\pgfpathcurveto{\pgfqpoint{7.820922in}{5.090749in}}{\pgfqpoint{7.816532in}{5.101348in}}{\pgfqpoint{7.808718in}{5.109162in}}%
\pgfpathcurveto{\pgfqpoint{7.800905in}{5.116975in}}{\pgfqpoint{7.790306in}{5.121365in}}{\pgfqpoint{7.779256in}{5.121365in}}%
\pgfpathcurveto{\pgfqpoint{7.768206in}{5.121365in}}{\pgfqpoint{7.757607in}{5.116975in}}{\pgfqpoint{7.749793in}{5.109162in}}%
\pgfpathcurveto{\pgfqpoint{7.741979in}{5.101348in}}{\pgfqpoint{7.737589in}{5.090749in}}{\pgfqpoint{7.737589in}{5.079699in}}%
\pgfpathcurveto{\pgfqpoint{7.737589in}{5.068649in}}{\pgfqpoint{7.741979in}{5.058050in}}{\pgfqpoint{7.749793in}{5.050236in}}%
\pgfpathcurveto{\pgfqpoint{7.757607in}{5.042422in}}{\pgfqpoint{7.768206in}{5.038032in}}{\pgfqpoint{7.779256in}{5.038032in}}%
\pgfpathclose%
\pgfusepath{stroke,fill}%
\end{pgfscope}%
\begin{pgfscope}%
\pgfpathrectangle{\pgfqpoint{0.481978in}{0.331635in}}{\pgfqpoint{9.300000in}{7.700000in}}%
\pgfusepath{clip}%
\pgfsetbuttcap%
\pgfsetroundjoin%
\definecolor{currentfill}{rgb}{0.631373,0.788235,0.956863}%
\pgfsetfillcolor{currentfill}%
\pgfsetlinewidth{0.481800pt}%
\definecolor{currentstroke}{rgb}{1.000000,1.000000,1.000000}%
\pgfsetstrokecolor{currentstroke}%
\pgfsetdash{}{0pt}%
\pgfpathmoveto{\pgfqpoint{7.196974in}{1.589439in}}%
\pgfpathcurveto{\pgfqpoint{7.208024in}{1.589439in}}{\pgfqpoint{7.218623in}{1.593829in}}{\pgfqpoint{7.226437in}{1.601643in}}%
\pgfpathcurveto{\pgfqpoint{7.234251in}{1.609456in}}{\pgfqpoint{7.238641in}{1.620055in}}{\pgfqpoint{7.238641in}{1.631105in}}%
\pgfpathcurveto{\pgfqpoint{7.238641in}{1.642155in}}{\pgfqpoint{7.234251in}{1.652755in}}{\pgfqpoint{7.226437in}{1.660568in}}%
\pgfpathcurveto{\pgfqpoint{7.218623in}{1.668382in}}{\pgfqpoint{7.208024in}{1.672772in}}{\pgfqpoint{7.196974in}{1.672772in}}%
\pgfpathcurveto{\pgfqpoint{7.185924in}{1.672772in}}{\pgfqpoint{7.175325in}{1.668382in}}{\pgfqpoint{7.167512in}{1.660568in}}%
\pgfpathcurveto{\pgfqpoint{7.159698in}{1.652755in}}{\pgfqpoint{7.155308in}{1.642155in}}{\pgfqpoint{7.155308in}{1.631105in}}%
\pgfpathcurveto{\pgfqpoint{7.155308in}{1.620055in}}{\pgfqpoint{7.159698in}{1.609456in}}{\pgfqpoint{7.167512in}{1.601643in}}%
\pgfpathcurveto{\pgfqpoint{7.175325in}{1.593829in}}{\pgfqpoint{7.185924in}{1.589439in}}{\pgfqpoint{7.196974in}{1.589439in}}%
\pgfpathclose%
\pgfusepath{stroke,fill}%
\end{pgfscope}%
\begin{pgfscope}%
\pgfpathrectangle{\pgfqpoint{0.481978in}{0.331635in}}{\pgfqpoint{9.300000in}{7.700000in}}%
\pgfusepath{clip}%
\pgfsetbuttcap%
\pgfsetroundjoin%
\definecolor{currentfill}{rgb}{0.631373,0.788235,0.956863}%
\pgfsetfillcolor{currentfill}%
\pgfsetlinewidth{0.481800pt}%
\definecolor{currentstroke}{rgb}{1.000000,1.000000,1.000000}%
\pgfsetstrokecolor{currentstroke}%
\pgfsetdash{}{0pt}%
\pgfpathmoveto{\pgfqpoint{3.418933in}{1.305418in}}%
\pgfpathcurveto{\pgfqpoint{3.429983in}{1.305418in}}{\pgfqpoint{3.440582in}{1.309808in}}{\pgfqpoint{3.448396in}{1.317621in}}%
\pgfpathcurveto{\pgfqpoint{3.456210in}{1.325435in}}{\pgfqpoint{3.460600in}{1.336034in}}{\pgfqpoint{3.460600in}{1.347084in}}%
\pgfpathcurveto{\pgfqpoint{3.460600in}{1.358134in}}{\pgfqpoint{3.456210in}{1.368733in}}{\pgfqpoint{3.448396in}{1.376547in}}%
\pgfpathcurveto{\pgfqpoint{3.440582in}{1.384361in}}{\pgfqpoint{3.429983in}{1.388751in}}{\pgfqpoint{3.418933in}{1.388751in}}%
\pgfpathcurveto{\pgfqpoint{3.407883in}{1.388751in}}{\pgfqpoint{3.397284in}{1.384361in}}{\pgfqpoint{3.389471in}{1.376547in}}%
\pgfpathcurveto{\pgfqpoint{3.381657in}{1.368733in}}{\pgfqpoint{3.377267in}{1.358134in}}{\pgfqpoint{3.377267in}{1.347084in}}%
\pgfpathcurveto{\pgfqpoint{3.377267in}{1.336034in}}{\pgfqpoint{3.381657in}{1.325435in}}{\pgfqpoint{3.389471in}{1.317621in}}%
\pgfpathcurveto{\pgfqpoint{3.397284in}{1.309808in}}{\pgfqpoint{3.407883in}{1.305418in}}{\pgfqpoint{3.418933in}{1.305418in}}%
\pgfpathclose%
\pgfusepath{stroke,fill}%
\end{pgfscope}%
\begin{pgfscope}%
\pgfpathrectangle{\pgfqpoint{0.481978in}{0.331635in}}{\pgfqpoint{9.300000in}{7.700000in}}%
\pgfusepath{clip}%
\pgfsetbuttcap%
\pgfsetroundjoin%
\definecolor{currentfill}{rgb}{0.631373,0.788235,0.956863}%
\pgfsetfillcolor{currentfill}%
\pgfsetlinewidth{0.481800pt}%
\definecolor{currentstroke}{rgb}{1.000000,1.000000,1.000000}%
\pgfsetstrokecolor{currentstroke}%
\pgfsetdash{}{0pt}%
\pgfpathmoveto{\pgfqpoint{5.333628in}{3.453590in}}%
\pgfpathcurveto{\pgfqpoint{5.344678in}{3.453590in}}{\pgfqpoint{5.355277in}{3.457980in}}{\pgfqpoint{5.363090in}{3.465794in}}%
\pgfpathcurveto{\pgfqpoint{5.370904in}{3.473608in}}{\pgfqpoint{5.375294in}{3.484207in}}{\pgfqpoint{5.375294in}{3.495257in}}%
\pgfpathcurveto{\pgfqpoint{5.375294in}{3.506307in}}{\pgfqpoint{5.370904in}{3.516906in}}{\pgfqpoint{5.363090in}{3.524719in}}%
\pgfpathcurveto{\pgfqpoint{5.355277in}{3.532533in}}{\pgfqpoint{5.344678in}{3.536923in}}{\pgfqpoint{5.333628in}{3.536923in}}%
\pgfpathcurveto{\pgfqpoint{5.322578in}{3.536923in}}{\pgfqpoint{5.311979in}{3.532533in}}{\pgfqpoint{5.304165in}{3.524719in}}%
\pgfpathcurveto{\pgfqpoint{5.296351in}{3.516906in}}{\pgfqpoint{5.291961in}{3.506307in}}{\pgfqpoint{5.291961in}{3.495257in}}%
\pgfpathcurveto{\pgfqpoint{5.291961in}{3.484207in}}{\pgfqpoint{5.296351in}{3.473608in}}{\pgfqpoint{5.304165in}{3.465794in}}%
\pgfpathcurveto{\pgfqpoint{5.311979in}{3.457980in}}{\pgfqpoint{5.322578in}{3.453590in}}{\pgfqpoint{5.333628in}{3.453590in}}%
\pgfpathclose%
\pgfusepath{stroke,fill}%
\end{pgfscope}%
\begin{pgfscope}%
\pgfpathrectangle{\pgfqpoint{0.481978in}{0.331635in}}{\pgfqpoint{9.300000in}{7.700000in}}%
\pgfusepath{clip}%
\pgfsetbuttcap%
\pgfsetroundjoin%
\definecolor{currentfill}{rgb}{0.631373,0.788235,0.956863}%
\pgfsetfillcolor{currentfill}%
\pgfsetlinewidth{0.481800pt}%
\definecolor{currentstroke}{rgb}{1.000000,1.000000,1.000000}%
\pgfsetstrokecolor{currentstroke}%
\pgfsetdash{}{0pt}%
\pgfpathmoveto{\pgfqpoint{6.946844in}{2.793604in}}%
\pgfpathcurveto{\pgfqpoint{6.957894in}{2.793604in}}{\pgfqpoint{6.968493in}{2.797994in}}{\pgfqpoint{6.976307in}{2.805808in}}%
\pgfpathcurveto{\pgfqpoint{6.984120in}{2.813622in}}{\pgfqpoint{6.988511in}{2.824221in}}{\pgfqpoint{6.988511in}{2.835271in}}%
\pgfpathcurveto{\pgfqpoint{6.988511in}{2.846321in}}{\pgfqpoint{6.984120in}{2.856920in}}{\pgfqpoint{6.976307in}{2.864734in}}%
\pgfpathcurveto{\pgfqpoint{6.968493in}{2.872547in}}{\pgfqpoint{6.957894in}{2.876937in}}{\pgfqpoint{6.946844in}{2.876937in}}%
\pgfpathcurveto{\pgfqpoint{6.935794in}{2.876937in}}{\pgfqpoint{6.925195in}{2.872547in}}{\pgfqpoint{6.917381in}{2.864734in}}%
\pgfpathcurveto{\pgfqpoint{6.909568in}{2.856920in}}{\pgfqpoint{6.905177in}{2.846321in}}{\pgfqpoint{6.905177in}{2.835271in}}%
\pgfpathcurveto{\pgfqpoint{6.905177in}{2.824221in}}{\pgfqpoint{6.909568in}{2.813622in}}{\pgfqpoint{6.917381in}{2.805808in}}%
\pgfpathcurveto{\pgfqpoint{6.925195in}{2.797994in}}{\pgfqpoint{6.935794in}{2.793604in}}{\pgfqpoint{6.946844in}{2.793604in}}%
\pgfpathclose%
\pgfusepath{stroke,fill}%
\end{pgfscope}%
\begin{pgfscope}%
\pgfpathrectangle{\pgfqpoint{0.481978in}{0.331635in}}{\pgfqpoint{9.300000in}{7.700000in}}%
\pgfusepath{clip}%
\pgfsetbuttcap%
\pgfsetroundjoin%
\definecolor{currentfill}{rgb}{0.631373,0.788235,0.956863}%
\pgfsetfillcolor{currentfill}%
\pgfsetlinewidth{0.481800pt}%
\definecolor{currentstroke}{rgb}{1.000000,1.000000,1.000000}%
\pgfsetstrokecolor{currentstroke}%
\pgfsetdash{}{0pt}%
\pgfpathmoveto{\pgfqpoint{8.414129in}{4.979055in}}%
\pgfpathcurveto{\pgfqpoint{8.425179in}{4.979055in}}{\pgfqpoint{8.435778in}{4.983445in}}{\pgfqpoint{8.443592in}{4.991258in}}%
\pgfpathcurveto{\pgfqpoint{8.451406in}{4.999072in}}{\pgfqpoint{8.455796in}{5.009671in}}{\pgfqpoint{8.455796in}{5.020721in}}%
\pgfpathcurveto{\pgfqpoint{8.455796in}{5.031771in}}{\pgfqpoint{8.451406in}{5.042370in}}{\pgfqpoint{8.443592in}{5.050184in}}%
\pgfpathcurveto{\pgfqpoint{8.435778in}{5.057998in}}{\pgfqpoint{8.425179in}{5.062388in}}{\pgfqpoint{8.414129in}{5.062388in}}%
\pgfpathcurveto{\pgfqpoint{8.403079in}{5.062388in}}{\pgfqpoint{8.392480in}{5.057998in}}{\pgfqpoint{8.384666in}{5.050184in}}%
\pgfpathcurveto{\pgfqpoint{8.376853in}{5.042370in}}{\pgfqpoint{8.372463in}{5.031771in}}{\pgfqpoint{8.372463in}{5.020721in}}%
\pgfpathcurveto{\pgfqpoint{8.372463in}{5.009671in}}{\pgfqpoint{8.376853in}{4.999072in}}{\pgfqpoint{8.384666in}{4.991258in}}%
\pgfpathcurveto{\pgfqpoint{8.392480in}{4.983445in}}{\pgfqpoint{8.403079in}{4.979055in}}{\pgfqpoint{8.414129in}{4.979055in}}%
\pgfpathclose%
\pgfusepath{stroke,fill}%
\end{pgfscope}%
\begin{pgfscope}%
\pgfpathrectangle{\pgfqpoint{0.481978in}{0.331635in}}{\pgfqpoint{9.300000in}{7.700000in}}%
\pgfusepath{clip}%
\pgfsetbuttcap%
\pgfsetroundjoin%
\definecolor{currentfill}{rgb}{0.631373,0.788235,0.956863}%
\pgfsetfillcolor{currentfill}%
\pgfsetlinewidth{0.481800pt}%
\definecolor{currentstroke}{rgb}{1.000000,1.000000,1.000000}%
\pgfsetstrokecolor{currentstroke}%
\pgfsetdash{}{0pt}%
\pgfpathmoveto{\pgfqpoint{4.957931in}{2.096496in}}%
\pgfpathcurveto{\pgfqpoint{4.968982in}{2.096496in}}{\pgfqpoint{4.979581in}{2.100886in}}{\pgfqpoint{4.987394in}{2.108699in}}%
\pgfpathcurveto{\pgfqpoint{4.995208in}{2.116513in}}{\pgfqpoint{4.999598in}{2.127112in}}{\pgfqpoint{4.999598in}{2.138162in}}%
\pgfpathcurveto{\pgfqpoint{4.999598in}{2.149212in}}{\pgfqpoint{4.995208in}{2.159811in}}{\pgfqpoint{4.987394in}{2.167625in}}%
\pgfpathcurveto{\pgfqpoint{4.979581in}{2.175439in}}{\pgfqpoint{4.968982in}{2.179829in}}{\pgfqpoint{4.957931in}{2.179829in}}%
\pgfpathcurveto{\pgfqpoint{4.946881in}{2.179829in}}{\pgfqpoint{4.936282in}{2.175439in}}{\pgfqpoint{4.928469in}{2.167625in}}%
\pgfpathcurveto{\pgfqpoint{4.920655in}{2.159811in}}{\pgfqpoint{4.916265in}{2.149212in}}{\pgfqpoint{4.916265in}{2.138162in}}%
\pgfpathcurveto{\pgfqpoint{4.916265in}{2.127112in}}{\pgfqpoint{4.920655in}{2.116513in}}{\pgfqpoint{4.928469in}{2.108699in}}%
\pgfpathcurveto{\pgfqpoint{4.936282in}{2.100886in}}{\pgfqpoint{4.946881in}{2.096496in}}{\pgfqpoint{4.957931in}{2.096496in}}%
\pgfpathclose%
\pgfusepath{stroke,fill}%
\end{pgfscope}%
\begin{pgfscope}%
\pgfpathrectangle{\pgfqpoint{0.481978in}{0.331635in}}{\pgfqpoint{9.300000in}{7.700000in}}%
\pgfusepath{clip}%
\pgfsetbuttcap%
\pgfsetroundjoin%
\definecolor{currentfill}{rgb}{0.631373,0.788235,0.956863}%
\pgfsetfillcolor{currentfill}%
\pgfsetlinewidth{0.481800pt}%
\definecolor{currentstroke}{rgb}{1.000000,1.000000,1.000000}%
\pgfsetstrokecolor{currentstroke}%
\pgfsetdash{}{0pt}%
\pgfpathmoveto{\pgfqpoint{7.947630in}{5.352040in}}%
\pgfpathcurveto{\pgfqpoint{7.958680in}{5.352040in}}{\pgfqpoint{7.969279in}{5.356431in}}{\pgfqpoint{7.977093in}{5.364244in}}%
\pgfpathcurveto{\pgfqpoint{7.984906in}{5.372058in}}{\pgfqpoint{7.989297in}{5.382657in}}{\pgfqpoint{7.989297in}{5.393707in}}%
\pgfpathcurveto{\pgfqpoint{7.989297in}{5.404757in}}{\pgfqpoint{7.984906in}{5.415356in}}{\pgfqpoint{7.977093in}{5.423170in}}%
\pgfpathcurveto{\pgfqpoint{7.969279in}{5.430983in}}{\pgfqpoint{7.958680in}{5.435374in}}{\pgfqpoint{7.947630in}{5.435374in}}%
\pgfpathcurveto{\pgfqpoint{7.936580in}{5.435374in}}{\pgfqpoint{7.925981in}{5.430983in}}{\pgfqpoint{7.918167in}{5.423170in}}%
\pgfpathcurveto{\pgfqpoint{7.910354in}{5.415356in}}{\pgfqpoint{7.905963in}{5.404757in}}{\pgfqpoint{7.905963in}{5.393707in}}%
\pgfpathcurveto{\pgfqpoint{7.905963in}{5.382657in}}{\pgfqpoint{7.910354in}{5.372058in}}{\pgfqpoint{7.918167in}{5.364244in}}%
\pgfpathcurveto{\pgfqpoint{7.925981in}{5.356431in}}{\pgfqpoint{7.936580in}{5.352040in}}{\pgfqpoint{7.947630in}{5.352040in}}%
\pgfpathclose%
\pgfusepath{stroke,fill}%
\end{pgfscope}%
\begin{pgfscope}%
\pgfpathrectangle{\pgfqpoint{0.481978in}{0.331635in}}{\pgfqpoint{9.300000in}{7.700000in}}%
\pgfusepath{clip}%
\pgfsetbuttcap%
\pgfsetroundjoin%
\definecolor{currentfill}{rgb}{0.631373,0.788235,0.956863}%
\pgfsetfillcolor{currentfill}%
\pgfsetlinewidth{0.481800pt}%
\definecolor{currentstroke}{rgb}{1.000000,1.000000,1.000000}%
\pgfsetstrokecolor{currentstroke}%
\pgfsetdash{}{0pt}%
\pgfpathmoveto{\pgfqpoint{3.205045in}{1.529755in}}%
\pgfpathcurveto{\pgfqpoint{3.216095in}{1.529755in}}{\pgfqpoint{3.226694in}{1.534145in}}{\pgfqpoint{3.234508in}{1.541959in}}%
\pgfpathcurveto{\pgfqpoint{3.242321in}{1.549773in}}{\pgfqpoint{3.246711in}{1.560372in}}{\pgfqpoint{3.246711in}{1.571422in}}%
\pgfpathcurveto{\pgfqpoint{3.246711in}{1.582472in}}{\pgfqpoint{3.242321in}{1.593071in}}{\pgfqpoint{3.234508in}{1.600885in}}%
\pgfpathcurveto{\pgfqpoint{3.226694in}{1.608698in}}{\pgfqpoint{3.216095in}{1.613088in}}{\pgfqpoint{3.205045in}{1.613088in}}%
\pgfpathcurveto{\pgfqpoint{3.193995in}{1.613088in}}{\pgfqpoint{3.183396in}{1.608698in}}{\pgfqpoint{3.175582in}{1.600885in}}%
\pgfpathcurveto{\pgfqpoint{3.167768in}{1.593071in}}{\pgfqpoint{3.163378in}{1.582472in}}{\pgfqpoint{3.163378in}{1.571422in}}%
\pgfpathcurveto{\pgfqpoint{3.163378in}{1.560372in}}{\pgfqpoint{3.167768in}{1.549773in}}{\pgfqpoint{3.175582in}{1.541959in}}%
\pgfpathcurveto{\pgfqpoint{3.183396in}{1.534145in}}{\pgfqpoint{3.193995in}{1.529755in}}{\pgfqpoint{3.205045in}{1.529755in}}%
\pgfpathclose%
\pgfusepath{stroke,fill}%
\end{pgfscope}%
\begin{pgfscope}%
\pgfpathrectangle{\pgfqpoint{0.481978in}{0.331635in}}{\pgfqpoint{9.300000in}{7.700000in}}%
\pgfusepath{clip}%
\pgfsetbuttcap%
\pgfsetroundjoin%
\definecolor{currentfill}{rgb}{0.631373,0.788235,0.956863}%
\pgfsetfillcolor{currentfill}%
\pgfsetlinewidth{0.481800pt}%
\definecolor{currentstroke}{rgb}{1.000000,1.000000,1.000000}%
\pgfsetstrokecolor{currentstroke}%
\pgfsetdash{}{0pt}%
\pgfpathmoveto{\pgfqpoint{7.723488in}{5.483998in}}%
\pgfpathcurveto{\pgfqpoint{7.734538in}{5.483998in}}{\pgfqpoint{7.745137in}{5.488389in}}{\pgfqpoint{7.752951in}{5.496202in}}%
\pgfpathcurveto{\pgfqpoint{7.760765in}{5.504016in}}{\pgfqpoint{7.765155in}{5.514615in}}{\pgfqpoint{7.765155in}{5.525665in}}%
\pgfpathcurveto{\pgfqpoint{7.765155in}{5.536715in}}{\pgfqpoint{7.760765in}{5.547314in}}{\pgfqpoint{7.752951in}{5.555128in}}%
\pgfpathcurveto{\pgfqpoint{7.745137in}{5.562942in}}{\pgfqpoint{7.734538in}{5.567332in}}{\pgfqpoint{7.723488in}{5.567332in}}%
\pgfpathcurveto{\pgfqpoint{7.712438in}{5.567332in}}{\pgfqpoint{7.701839in}{5.562942in}}{\pgfqpoint{7.694025in}{5.555128in}}%
\pgfpathcurveto{\pgfqpoint{7.686212in}{5.547314in}}{\pgfqpoint{7.681822in}{5.536715in}}{\pgfqpoint{7.681822in}{5.525665in}}%
\pgfpathcurveto{\pgfqpoint{7.681822in}{5.514615in}}{\pgfqpoint{7.686212in}{5.504016in}}{\pgfqpoint{7.694025in}{5.496202in}}%
\pgfpathcurveto{\pgfqpoint{7.701839in}{5.488389in}}{\pgfqpoint{7.712438in}{5.483998in}}{\pgfqpoint{7.723488in}{5.483998in}}%
\pgfpathclose%
\pgfusepath{stroke,fill}%
\end{pgfscope}%
\begin{pgfscope}%
\pgfpathrectangle{\pgfqpoint{0.481978in}{0.331635in}}{\pgfqpoint{9.300000in}{7.700000in}}%
\pgfusepath{clip}%
\pgfsetbuttcap%
\pgfsetroundjoin%
\definecolor{currentfill}{rgb}{0.631373,0.788235,0.956863}%
\pgfsetfillcolor{currentfill}%
\pgfsetlinewidth{0.481800pt}%
\definecolor{currentstroke}{rgb}{1.000000,1.000000,1.000000}%
\pgfsetstrokecolor{currentstroke}%
\pgfsetdash{}{0pt}%
\pgfpathmoveto{\pgfqpoint{4.705109in}{1.044958in}}%
\pgfpathcurveto{\pgfqpoint{4.716159in}{1.044958in}}{\pgfqpoint{4.726758in}{1.049348in}}{\pgfqpoint{4.734572in}{1.057162in}}%
\pgfpathcurveto{\pgfqpoint{4.742386in}{1.064976in}}{\pgfqpoint{4.746776in}{1.075575in}}{\pgfqpoint{4.746776in}{1.086625in}}%
\pgfpathcurveto{\pgfqpoint{4.746776in}{1.097675in}}{\pgfqpoint{4.742386in}{1.108274in}}{\pgfqpoint{4.734572in}{1.116088in}}%
\pgfpathcurveto{\pgfqpoint{4.726758in}{1.123901in}}{\pgfqpoint{4.716159in}{1.128291in}}{\pgfqpoint{4.705109in}{1.128291in}}%
\pgfpathcurveto{\pgfqpoint{4.694059in}{1.128291in}}{\pgfqpoint{4.683460in}{1.123901in}}{\pgfqpoint{4.675646in}{1.116088in}}%
\pgfpathcurveto{\pgfqpoint{4.667833in}{1.108274in}}{\pgfqpoint{4.663442in}{1.097675in}}{\pgfqpoint{4.663442in}{1.086625in}}%
\pgfpathcurveto{\pgfqpoint{4.663442in}{1.075575in}}{\pgfqpoint{4.667833in}{1.064976in}}{\pgfqpoint{4.675646in}{1.057162in}}%
\pgfpathcurveto{\pgfqpoint{4.683460in}{1.049348in}}{\pgfqpoint{4.694059in}{1.044958in}}{\pgfqpoint{4.705109in}{1.044958in}}%
\pgfpathclose%
\pgfusepath{stroke,fill}%
\end{pgfscope}%
\begin{pgfscope}%
\pgfpathrectangle{\pgfqpoint{0.481978in}{0.331635in}}{\pgfqpoint{9.300000in}{7.700000in}}%
\pgfusepath{clip}%
\pgfsetbuttcap%
\pgfsetroundjoin%
\definecolor{currentfill}{rgb}{0.631373,0.788235,0.956863}%
\pgfsetfillcolor{currentfill}%
\pgfsetlinewidth{0.481800pt}%
\definecolor{currentstroke}{rgb}{1.000000,1.000000,1.000000}%
\pgfsetstrokecolor{currentstroke}%
\pgfsetdash{}{0pt}%
\pgfpathmoveto{\pgfqpoint{3.932688in}{3.871886in}}%
\pgfpathcurveto{\pgfqpoint{3.943738in}{3.871886in}}{\pgfqpoint{3.954337in}{3.876276in}}{\pgfqpoint{3.962150in}{3.884090in}}%
\pgfpathcurveto{\pgfqpoint{3.969964in}{3.891903in}}{\pgfqpoint{3.974354in}{3.902502in}}{\pgfqpoint{3.974354in}{3.913553in}}%
\pgfpathcurveto{\pgfqpoint{3.974354in}{3.924603in}}{\pgfqpoint{3.969964in}{3.935202in}}{\pgfqpoint{3.962150in}{3.943015in}}%
\pgfpathcurveto{\pgfqpoint{3.954337in}{3.950829in}}{\pgfqpoint{3.943738in}{3.955219in}}{\pgfqpoint{3.932688in}{3.955219in}}%
\pgfpathcurveto{\pgfqpoint{3.921637in}{3.955219in}}{\pgfqpoint{3.911038in}{3.950829in}}{\pgfqpoint{3.903225in}{3.943015in}}%
\pgfpathcurveto{\pgfqpoint{3.895411in}{3.935202in}}{\pgfqpoint{3.891021in}{3.924603in}}{\pgfqpoint{3.891021in}{3.913553in}}%
\pgfpathcurveto{\pgfqpoint{3.891021in}{3.902502in}}{\pgfqpoint{3.895411in}{3.891903in}}{\pgfqpoint{3.903225in}{3.884090in}}%
\pgfpathcurveto{\pgfqpoint{3.911038in}{3.876276in}}{\pgfqpoint{3.921637in}{3.871886in}}{\pgfqpoint{3.932688in}{3.871886in}}%
\pgfpathclose%
\pgfusepath{stroke,fill}%
\end{pgfscope}%
\begin{pgfscope}%
\pgfpathrectangle{\pgfqpoint{0.481978in}{0.331635in}}{\pgfqpoint{9.300000in}{7.700000in}}%
\pgfusepath{clip}%
\pgfsetbuttcap%
\pgfsetroundjoin%
\definecolor{currentfill}{rgb}{0.631373,0.788235,0.956863}%
\pgfsetfillcolor{currentfill}%
\pgfsetlinewidth{0.481800pt}%
\definecolor{currentstroke}{rgb}{1.000000,1.000000,1.000000}%
\pgfsetstrokecolor{currentstroke}%
\pgfsetdash{}{0pt}%
\pgfpathmoveto{\pgfqpoint{7.599765in}{5.924449in}}%
\pgfpathcurveto{\pgfqpoint{7.610815in}{5.924449in}}{\pgfqpoint{7.621414in}{5.928840in}}{\pgfqpoint{7.629228in}{5.936653in}}%
\pgfpathcurveto{\pgfqpoint{7.637042in}{5.944467in}}{\pgfqpoint{7.641432in}{5.955066in}}{\pgfqpoint{7.641432in}{5.966116in}}%
\pgfpathcurveto{\pgfqpoint{7.641432in}{5.977166in}}{\pgfqpoint{7.637042in}{5.987765in}}{\pgfqpoint{7.629228in}{5.995579in}}%
\pgfpathcurveto{\pgfqpoint{7.621414in}{6.003392in}}{\pgfqpoint{7.610815in}{6.007783in}}{\pgfqpoint{7.599765in}{6.007783in}}%
\pgfpathcurveto{\pgfqpoint{7.588715in}{6.007783in}}{\pgfqpoint{7.578116in}{6.003392in}}{\pgfqpoint{7.570302in}{5.995579in}}%
\pgfpathcurveto{\pgfqpoint{7.562489in}{5.987765in}}{\pgfqpoint{7.558098in}{5.977166in}}{\pgfqpoint{7.558098in}{5.966116in}}%
\pgfpathcurveto{\pgfqpoint{7.558098in}{5.955066in}}{\pgfqpoint{7.562489in}{5.944467in}}{\pgfqpoint{7.570302in}{5.936653in}}%
\pgfpathcurveto{\pgfqpoint{7.578116in}{5.928840in}}{\pgfqpoint{7.588715in}{5.924449in}}{\pgfqpoint{7.599765in}{5.924449in}}%
\pgfpathclose%
\pgfusepath{stroke,fill}%
\end{pgfscope}%
\begin{pgfscope}%
\pgfpathrectangle{\pgfqpoint{0.481978in}{0.331635in}}{\pgfqpoint{9.300000in}{7.700000in}}%
\pgfusepath{clip}%
\pgfsetbuttcap%
\pgfsetroundjoin%
\definecolor{currentfill}{rgb}{0.631373,0.788235,0.956863}%
\pgfsetfillcolor{currentfill}%
\pgfsetlinewidth{0.481800pt}%
\definecolor{currentstroke}{rgb}{1.000000,1.000000,1.000000}%
\pgfsetstrokecolor{currentstroke}%
\pgfsetdash{}{0pt}%
\pgfpathmoveto{\pgfqpoint{6.870443in}{1.661336in}}%
\pgfpathcurveto{\pgfqpoint{6.881493in}{1.661336in}}{\pgfqpoint{6.892092in}{1.665727in}}{\pgfqpoint{6.899905in}{1.673540in}}%
\pgfpathcurveto{\pgfqpoint{6.907719in}{1.681354in}}{\pgfqpoint{6.912109in}{1.691953in}}{\pgfqpoint{6.912109in}{1.703003in}}%
\pgfpathcurveto{\pgfqpoint{6.912109in}{1.714053in}}{\pgfqpoint{6.907719in}{1.724652in}}{\pgfqpoint{6.899905in}{1.732466in}}%
\pgfpathcurveto{\pgfqpoint{6.892092in}{1.740279in}}{\pgfqpoint{6.881493in}{1.744670in}}{\pgfqpoint{6.870443in}{1.744670in}}%
\pgfpathcurveto{\pgfqpoint{6.859393in}{1.744670in}}{\pgfqpoint{6.848794in}{1.740279in}}{\pgfqpoint{6.840980in}{1.732466in}}%
\pgfpathcurveto{\pgfqpoint{6.833166in}{1.724652in}}{\pgfqpoint{6.828776in}{1.714053in}}{\pgfqpoint{6.828776in}{1.703003in}}%
\pgfpathcurveto{\pgfqpoint{6.828776in}{1.691953in}}{\pgfqpoint{6.833166in}{1.681354in}}{\pgfqpoint{6.840980in}{1.673540in}}%
\pgfpathcurveto{\pgfqpoint{6.848794in}{1.665727in}}{\pgfqpoint{6.859393in}{1.661336in}}{\pgfqpoint{6.870443in}{1.661336in}}%
\pgfpathclose%
\pgfusepath{stroke,fill}%
\end{pgfscope}%
\begin{pgfscope}%
\pgfpathrectangle{\pgfqpoint{0.481978in}{0.331635in}}{\pgfqpoint{9.300000in}{7.700000in}}%
\pgfusepath{clip}%
\pgfsetbuttcap%
\pgfsetroundjoin%
\definecolor{currentfill}{rgb}{0.631373,0.788235,0.956863}%
\pgfsetfillcolor{currentfill}%
\pgfsetlinewidth{0.481800pt}%
\definecolor{currentstroke}{rgb}{1.000000,1.000000,1.000000}%
\pgfsetstrokecolor{currentstroke}%
\pgfsetdash{}{0pt}%
\pgfpathmoveto{\pgfqpoint{6.112529in}{2.110735in}}%
\pgfpathcurveto{\pgfqpoint{6.123579in}{2.110735in}}{\pgfqpoint{6.134178in}{2.115125in}}{\pgfqpoint{6.141991in}{2.122939in}}%
\pgfpathcurveto{\pgfqpoint{6.149805in}{2.130752in}}{\pgfqpoint{6.154195in}{2.141352in}}{\pgfqpoint{6.154195in}{2.152402in}}%
\pgfpathcurveto{\pgfqpoint{6.154195in}{2.163452in}}{\pgfqpoint{6.149805in}{2.174051in}}{\pgfqpoint{6.141991in}{2.181864in}}%
\pgfpathcurveto{\pgfqpoint{6.134178in}{2.189678in}}{\pgfqpoint{6.123579in}{2.194068in}}{\pgfqpoint{6.112529in}{2.194068in}}%
\pgfpathcurveto{\pgfqpoint{6.101478in}{2.194068in}}{\pgfqpoint{6.090879in}{2.189678in}}{\pgfqpoint{6.083066in}{2.181864in}}%
\pgfpathcurveto{\pgfqpoint{6.075252in}{2.174051in}}{\pgfqpoint{6.070862in}{2.163452in}}{\pgfqpoint{6.070862in}{2.152402in}}%
\pgfpathcurveto{\pgfqpoint{6.070862in}{2.141352in}}{\pgfqpoint{6.075252in}{2.130752in}}{\pgfqpoint{6.083066in}{2.122939in}}%
\pgfpathcurveto{\pgfqpoint{6.090879in}{2.115125in}}{\pgfqpoint{6.101478in}{2.110735in}}{\pgfqpoint{6.112529in}{2.110735in}}%
\pgfpathclose%
\pgfusepath{stroke,fill}%
\end{pgfscope}%
\begin{pgfscope}%
\pgfpathrectangle{\pgfqpoint{0.481978in}{0.331635in}}{\pgfqpoint{9.300000in}{7.700000in}}%
\pgfusepath{clip}%
\pgfsetbuttcap%
\pgfsetroundjoin%
\definecolor{currentfill}{rgb}{0.631373,0.788235,0.956863}%
\pgfsetfillcolor{currentfill}%
\pgfsetlinewidth{0.481800pt}%
\definecolor{currentstroke}{rgb}{1.000000,1.000000,1.000000}%
\pgfsetstrokecolor{currentstroke}%
\pgfsetdash{}{0pt}%
\pgfpathmoveto{\pgfqpoint{7.807922in}{5.754210in}}%
\pgfpathcurveto{\pgfqpoint{7.818972in}{5.754210in}}{\pgfqpoint{7.829571in}{5.758600in}}{\pgfqpoint{7.837385in}{5.766413in}}%
\pgfpathcurveto{\pgfqpoint{7.845199in}{5.774227in}}{\pgfqpoint{7.849589in}{5.784826in}}{\pgfqpoint{7.849589in}{5.795876in}}%
\pgfpathcurveto{\pgfqpoint{7.849589in}{5.806926in}}{\pgfqpoint{7.845199in}{5.817525in}}{\pgfqpoint{7.837385in}{5.825339in}}%
\pgfpathcurveto{\pgfqpoint{7.829571in}{5.833153in}}{\pgfqpoint{7.818972in}{5.837543in}}{\pgfqpoint{7.807922in}{5.837543in}}%
\pgfpathcurveto{\pgfqpoint{7.796872in}{5.837543in}}{\pgfqpoint{7.786273in}{5.833153in}}{\pgfqpoint{7.778460in}{5.825339in}}%
\pgfpathcurveto{\pgfqpoint{7.770646in}{5.817525in}}{\pgfqpoint{7.766256in}{5.806926in}}{\pgfqpoint{7.766256in}{5.795876in}}%
\pgfpathcurveto{\pgfqpoint{7.766256in}{5.784826in}}{\pgfqpoint{7.770646in}{5.774227in}}{\pgfqpoint{7.778460in}{5.766413in}}%
\pgfpathcurveto{\pgfqpoint{7.786273in}{5.758600in}}{\pgfqpoint{7.796872in}{5.754210in}}{\pgfqpoint{7.807922in}{5.754210in}}%
\pgfpathclose%
\pgfusepath{stroke,fill}%
\end{pgfscope}%
\begin{pgfscope}%
\pgfpathrectangle{\pgfqpoint{0.481978in}{0.331635in}}{\pgfqpoint{9.300000in}{7.700000in}}%
\pgfusepath{clip}%
\pgfsetbuttcap%
\pgfsetroundjoin%
\definecolor{currentfill}{rgb}{0.631373,0.788235,0.956863}%
\pgfsetfillcolor{currentfill}%
\pgfsetlinewidth{0.481800pt}%
\definecolor{currentstroke}{rgb}{1.000000,1.000000,1.000000}%
\pgfsetstrokecolor{currentstroke}%
\pgfsetdash{}{0pt}%
\pgfpathmoveto{\pgfqpoint{3.326485in}{2.016983in}}%
\pgfpathcurveto{\pgfqpoint{3.337535in}{2.016983in}}{\pgfqpoint{3.348134in}{2.021373in}}{\pgfqpoint{3.355947in}{2.029187in}}%
\pgfpathcurveto{\pgfqpoint{3.363761in}{2.037001in}}{\pgfqpoint{3.368151in}{2.047600in}}{\pgfqpoint{3.368151in}{2.058650in}}%
\pgfpathcurveto{\pgfqpoint{3.368151in}{2.069700in}}{\pgfqpoint{3.363761in}{2.080299in}}{\pgfqpoint{3.355947in}{2.088113in}}%
\pgfpathcurveto{\pgfqpoint{3.348134in}{2.095926in}}{\pgfqpoint{3.337535in}{2.100317in}}{\pgfqpoint{3.326485in}{2.100317in}}%
\pgfpathcurveto{\pgfqpoint{3.315435in}{2.100317in}}{\pgfqpoint{3.304836in}{2.095926in}}{\pgfqpoint{3.297022in}{2.088113in}}%
\pgfpathcurveto{\pgfqpoint{3.289208in}{2.080299in}}{\pgfqpoint{3.284818in}{2.069700in}}{\pgfqpoint{3.284818in}{2.058650in}}%
\pgfpathcurveto{\pgfqpoint{3.284818in}{2.047600in}}{\pgfqpoint{3.289208in}{2.037001in}}{\pgfqpoint{3.297022in}{2.029187in}}%
\pgfpathcurveto{\pgfqpoint{3.304836in}{2.021373in}}{\pgfqpoint{3.315435in}{2.016983in}}{\pgfqpoint{3.326485in}{2.016983in}}%
\pgfpathclose%
\pgfusepath{stroke,fill}%
\end{pgfscope}%
\begin{pgfscope}%
\pgfpathrectangle{\pgfqpoint{0.481978in}{0.331635in}}{\pgfqpoint{9.300000in}{7.700000in}}%
\pgfusepath{clip}%
\pgfsetbuttcap%
\pgfsetroundjoin%
\definecolor{currentfill}{rgb}{0.631373,0.788235,0.956863}%
\pgfsetfillcolor{currentfill}%
\pgfsetlinewidth{0.481800pt}%
\definecolor{currentstroke}{rgb}{1.000000,1.000000,1.000000}%
\pgfsetstrokecolor{currentstroke}%
\pgfsetdash{}{0pt}%
\pgfpathmoveto{\pgfqpoint{1.925994in}{5.466001in}}%
\pgfpathcurveto{\pgfqpoint{1.937044in}{5.466001in}}{\pgfqpoint{1.947643in}{5.470391in}}{\pgfqpoint{1.955457in}{5.478205in}}%
\pgfpathcurveto{\pgfqpoint{1.963270in}{5.486018in}}{\pgfqpoint{1.967660in}{5.496617in}}{\pgfqpoint{1.967660in}{5.507668in}}%
\pgfpathcurveto{\pgfqpoint{1.967660in}{5.518718in}}{\pgfqpoint{1.963270in}{5.529317in}}{\pgfqpoint{1.955457in}{5.537130in}}%
\pgfpathcurveto{\pgfqpoint{1.947643in}{5.544944in}}{\pgfqpoint{1.937044in}{5.549334in}}{\pgfqpoint{1.925994in}{5.549334in}}%
\pgfpathcurveto{\pgfqpoint{1.914944in}{5.549334in}}{\pgfqpoint{1.904345in}{5.544944in}}{\pgfqpoint{1.896531in}{5.537130in}}%
\pgfpathcurveto{\pgfqpoint{1.888717in}{5.529317in}}{\pgfqpoint{1.884327in}{5.518718in}}{\pgfqpoint{1.884327in}{5.507668in}}%
\pgfpathcurveto{\pgfqpoint{1.884327in}{5.496617in}}{\pgfqpoint{1.888717in}{5.486018in}}{\pgfqpoint{1.896531in}{5.478205in}}%
\pgfpathcurveto{\pgfqpoint{1.904345in}{5.470391in}}{\pgfqpoint{1.914944in}{5.466001in}}{\pgfqpoint{1.925994in}{5.466001in}}%
\pgfpathclose%
\pgfusepath{stroke,fill}%
\end{pgfscope}%
\begin{pgfscope}%
\pgfpathrectangle{\pgfqpoint{0.481978in}{0.331635in}}{\pgfqpoint{9.300000in}{7.700000in}}%
\pgfusepath{clip}%
\pgfsetbuttcap%
\pgfsetroundjoin%
\definecolor{currentfill}{rgb}{0.631373,0.788235,0.956863}%
\pgfsetfillcolor{currentfill}%
\pgfsetlinewidth{0.481800pt}%
\definecolor{currentstroke}{rgb}{1.000000,1.000000,1.000000}%
\pgfsetstrokecolor{currentstroke}%
\pgfsetdash{}{0pt}%
\pgfpathmoveto{\pgfqpoint{5.878648in}{0.807236in}}%
\pgfpathcurveto{\pgfqpoint{5.889698in}{0.807236in}}{\pgfqpoint{5.900297in}{0.811627in}}{\pgfqpoint{5.908111in}{0.819440in}}%
\pgfpathcurveto{\pgfqpoint{5.915925in}{0.827254in}}{\pgfqpoint{5.920315in}{0.837853in}}{\pgfqpoint{5.920315in}{0.848903in}}%
\pgfpathcurveto{\pgfqpoint{5.920315in}{0.859953in}}{\pgfqpoint{5.915925in}{0.870552in}}{\pgfqpoint{5.908111in}{0.878366in}}%
\pgfpathcurveto{\pgfqpoint{5.900297in}{0.886179in}}{\pgfqpoint{5.889698in}{0.890570in}}{\pgfqpoint{5.878648in}{0.890570in}}%
\pgfpathcurveto{\pgfqpoint{5.867598in}{0.890570in}}{\pgfqpoint{5.856999in}{0.886179in}}{\pgfqpoint{5.849185in}{0.878366in}}%
\pgfpathcurveto{\pgfqpoint{5.841372in}{0.870552in}}{\pgfqpoint{5.836981in}{0.859953in}}{\pgfqpoint{5.836981in}{0.848903in}}%
\pgfpathcurveto{\pgfqpoint{5.836981in}{0.837853in}}{\pgfqpoint{5.841372in}{0.827254in}}{\pgfqpoint{5.849185in}{0.819440in}}%
\pgfpathcurveto{\pgfqpoint{5.856999in}{0.811627in}}{\pgfqpoint{5.867598in}{0.807236in}}{\pgfqpoint{5.878648in}{0.807236in}}%
\pgfpathclose%
\pgfusepath{stroke,fill}%
\end{pgfscope}%
\begin{pgfscope}%
\pgfpathrectangle{\pgfqpoint{0.481978in}{0.331635in}}{\pgfqpoint{9.300000in}{7.700000in}}%
\pgfusepath{clip}%
\pgfsetbuttcap%
\pgfsetroundjoin%
\definecolor{currentfill}{rgb}{0.631373,0.788235,0.956863}%
\pgfsetfillcolor{currentfill}%
\pgfsetlinewidth{0.481800pt}%
\definecolor{currentstroke}{rgb}{1.000000,1.000000,1.000000}%
\pgfsetstrokecolor{currentstroke}%
\pgfsetdash{}{0pt}%
\pgfpathmoveto{\pgfqpoint{8.178842in}{4.564124in}}%
\pgfpathcurveto{\pgfqpoint{8.189892in}{4.564124in}}{\pgfqpoint{8.200491in}{4.568514in}}{\pgfqpoint{8.208305in}{4.576327in}}%
\pgfpathcurveto{\pgfqpoint{8.216119in}{4.584141in}}{\pgfqpoint{8.220509in}{4.594740in}}{\pgfqpoint{8.220509in}{4.605790in}}%
\pgfpathcurveto{\pgfqpoint{8.220509in}{4.616840in}}{\pgfqpoint{8.216119in}{4.627439in}}{\pgfqpoint{8.208305in}{4.635253in}}%
\pgfpathcurveto{\pgfqpoint{8.200491in}{4.643067in}}{\pgfqpoint{8.189892in}{4.647457in}}{\pgfqpoint{8.178842in}{4.647457in}}%
\pgfpathcurveto{\pgfqpoint{8.167792in}{4.647457in}}{\pgfqpoint{8.157193in}{4.643067in}}{\pgfqpoint{8.149379in}{4.635253in}}%
\pgfpathcurveto{\pgfqpoint{8.141566in}{4.627439in}}{\pgfqpoint{8.137176in}{4.616840in}}{\pgfqpoint{8.137176in}{4.605790in}}%
\pgfpathcurveto{\pgfqpoint{8.137176in}{4.594740in}}{\pgfqpoint{8.141566in}{4.584141in}}{\pgfqpoint{8.149379in}{4.576327in}}%
\pgfpathcurveto{\pgfqpoint{8.157193in}{4.568514in}}{\pgfqpoint{8.167792in}{4.564124in}}{\pgfqpoint{8.178842in}{4.564124in}}%
\pgfpathclose%
\pgfusepath{stroke,fill}%
\end{pgfscope}%
\begin{pgfscope}%
\pgfpathrectangle{\pgfqpoint{0.481978in}{0.331635in}}{\pgfqpoint{9.300000in}{7.700000in}}%
\pgfusepath{clip}%
\pgfsetbuttcap%
\pgfsetroundjoin%
\definecolor{currentfill}{rgb}{0.631373,0.788235,0.956863}%
\pgfsetfillcolor{currentfill}%
\pgfsetlinewidth{0.481800pt}%
\definecolor{currentstroke}{rgb}{1.000000,1.000000,1.000000}%
\pgfsetstrokecolor{currentstroke}%
\pgfsetdash{}{0pt}%
\pgfpathmoveto{\pgfqpoint{6.339836in}{2.841214in}}%
\pgfpathcurveto{\pgfqpoint{6.350886in}{2.841214in}}{\pgfqpoint{6.361485in}{2.845605in}}{\pgfqpoint{6.369299in}{2.853418in}}%
\pgfpathcurveto{\pgfqpoint{6.377112in}{2.861232in}}{\pgfqpoint{6.381503in}{2.871831in}}{\pgfqpoint{6.381503in}{2.882881in}}%
\pgfpathcurveto{\pgfqpoint{6.381503in}{2.893931in}}{\pgfqpoint{6.377112in}{2.904530in}}{\pgfqpoint{6.369299in}{2.912344in}}%
\pgfpathcurveto{\pgfqpoint{6.361485in}{2.920157in}}{\pgfqpoint{6.350886in}{2.924548in}}{\pgfqpoint{6.339836in}{2.924548in}}%
\pgfpathcurveto{\pgfqpoint{6.328786in}{2.924548in}}{\pgfqpoint{6.318187in}{2.920157in}}{\pgfqpoint{6.310373in}{2.912344in}}%
\pgfpathcurveto{\pgfqpoint{6.302560in}{2.904530in}}{\pgfqpoint{6.298169in}{2.893931in}}{\pgfqpoint{6.298169in}{2.882881in}}%
\pgfpathcurveto{\pgfqpoint{6.298169in}{2.871831in}}{\pgfqpoint{6.302560in}{2.861232in}}{\pgfqpoint{6.310373in}{2.853418in}}%
\pgfpathcurveto{\pgfqpoint{6.318187in}{2.845605in}}{\pgfqpoint{6.328786in}{2.841214in}}{\pgfqpoint{6.339836in}{2.841214in}}%
\pgfpathclose%
\pgfusepath{stroke,fill}%
\end{pgfscope}%
\begin{pgfscope}%
\pgfpathrectangle{\pgfqpoint{0.481978in}{0.331635in}}{\pgfqpoint{9.300000in}{7.700000in}}%
\pgfusepath{clip}%
\pgfsetbuttcap%
\pgfsetroundjoin%
\definecolor{currentfill}{rgb}{0.631373,0.788235,0.956863}%
\pgfsetfillcolor{currentfill}%
\pgfsetlinewidth{0.481800pt}%
\definecolor{currentstroke}{rgb}{1.000000,1.000000,1.000000}%
\pgfsetstrokecolor{currentstroke}%
\pgfsetdash{}{0pt}%
\pgfpathmoveto{\pgfqpoint{5.971524in}{2.035399in}}%
\pgfpathcurveto{\pgfqpoint{5.982574in}{2.035399in}}{\pgfqpoint{5.993173in}{2.039789in}}{\pgfqpoint{6.000987in}{2.047603in}}%
\pgfpathcurveto{\pgfqpoint{6.008801in}{2.055416in}}{\pgfqpoint{6.013191in}{2.066015in}}{\pgfqpoint{6.013191in}{2.077066in}}%
\pgfpathcurveto{\pgfqpoint{6.013191in}{2.088116in}}{\pgfqpoint{6.008801in}{2.098715in}}{\pgfqpoint{6.000987in}{2.106528in}}%
\pgfpathcurveto{\pgfqpoint{5.993173in}{2.114342in}}{\pgfqpoint{5.982574in}{2.118732in}}{\pgfqpoint{5.971524in}{2.118732in}}%
\pgfpathcurveto{\pgfqpoint{5.960474in}{2.118732in}}{\pgfqpoint{5.949875in}{2.114342in}}{\pgfqpoint{5.942062in}{2.106528in}}%
\pgfpathcurveto{\pgfqpoint{5.934248in}{2.098715in}}{\pgfqpoint{5.929858in}{2.088116in}}{\pgfqpoint{5.929858in}{2.077066in}}%
\pgfpathcurveto{\pgfqpoint{5.929858in}{2.066015in}}{\pgfqpoint{5.934248in}{2.055416in}}{\pgfqpoint{5.942062in}{2.047603in}}%
\pgfpathcurveto{\pgfqpoint{5.949875in}{2.039789in}}{\pgfqpoint{5.960474in}{2.035399in}}{\pgfqpoint{5.971524in}{2.035399in}}%
\pgfpathclose%
\pgfusepath{stroke,fill}%
\end{pgfscope}%
\begin{pgfscope}%
\pgfpathrectangle{\pgfqpoint{0.481978in}{0.331635in}}{\pgfqpoint{9.300000in}{7.700000in}}%
\pgfusepath{clip}%
\pgfsetbuttcap%
\pgfsetroundjoin%
\definecolor{currentfill}{rgb}{0.631373,0.788235,0.956863}%
\pgfsetfillcolor{currentfill}%
\pgfsetlinewidth{0.481800pt}%
\definecolor{currentstroke}{rgb}{1.000000,1.000000,1.000000}%
\pgfsetstrokecolor{currentstroke}%
\pgfsetdash{}{0pt}%
\pgfpathmoveto{\pgfqpoint{3.002811in}{6.449802in}}%
\pgfpathcurveto{\pgfqpoint{3.013861in}{6.449802in}}{\pgfqpoint{3.024460in}{6.454193in}}{\pgfqpoint{3.032274in}{6.462006in}}%
\pgfpathcurveto{\pgfqpoint{3.040087in}{6.469820in}}{\pgfqpoint{3.044478in}{6.480419in}}{\pgfqpoint{3.044478in}{6.491469in}}%
\pgfpathcurveto{\pgfqpoint{3.044478in}{6.502519in}}{\pgfqpoint{3.040087in}{6.513118in}}{\pgfqpoint{3.032274in}{6.520932in}}%
\pgfpathcurveto{\pgfqpoint{3.024460in}{6.528746in}}{\pgfqpoint{3.013861in}{6.533136in}}{\pgfqpoint{3.002811in}{6.533136in}}%
\pgfpathcurveto{\pgfqpoint{2.991761in}{6.533136in}}{\pgfqpoint{2.981162in}{6.528746in}}{\pgfqpoint{2.973348in}{6.520932in}}%
\pgfpathcurveto{\pgfqpoint{2.965534in}{6.513118in}}{\pgfqpoint{2.961144in}{6.502519in}}{\pgfqpoint{2.961144in}{6.491469in}}%
\pgfpathcurveto{\pgfqpoint{2.961144in}{6.480419in}}{\pgfqpoint{2.965534in}{6.469820in}}{\pgfqpoint{2.973348in}{6.462006in}}%
\pgfpathcurveto{\pgfqpoint{2.981162in}{6.454193in}}{\pgfqpoint{2.991761in}{6.449802in}}{\pgfqpoint{3.002811in}{6.449802in}}%
\pgfpathclose%
\pgfusepath{stroke,fill}%
\end{pgfscope}%
\begin{pgfscope}%
\pgfpathrectangle{\pgfqpoint{0.481978in}{0.331635in}}{\pgfqpoint{9.300000in}{7.700000in}}%
\pgfusepath{clip}%
\pgfsetbuttcap%
\pgfsetroundjoin%
\definecolor{currentfill}{rgb}{0.631373,0.788235,0.956863}%
\pgfsetfillcolor{currentfill}%
\pgfsetlinewidth{0.481800pt}%
\definecolor{currentstroke}{rgb}{1.000000,1.000000,1.000000}%
\pgfsetstrokecolor{currentstroke}%
\pgfsetdash{}{0pt}%
\pgfpathmoveto{\pgfqpoint{5.207999in}{6.849862in}}%
\pgfpathcurveto{\pgfqpoint{5.219049in}{6.849862in}}{\pgfqpoint{5.229649in}{6.854253in}}{\pgfqpoint{5.237462in}{6.862066in}}%
\pgfpathcurveto{\pgfqpoint{5.245276in}{6.869880in}}{\pgfqpoint{5.249666in}{6.880479in}}{\pgfqpoint{5.249666in}{6.891529in}}%
\pgfpathcurveto{\pgfqpoint{5.249666in}{6.902579in}}{\pgfqpoint{5.245276in}{6.913178in}}{\pgfqpoint{5.237462in}{6.920992in}}%
\pgfpathcurveto{\pgfqpoint{5.229649in}{6.928805in}}{\pgfqpoint{5.219049in}{6.933196in}}{\pgfqpoint{5.207999in}{6.933196in}}%
\pgfpathcurveto{\pgfqpoint{5.196949in}{6.933196in}}{\pgfqpoint{5.186350in}{6.928805in}}{\pgfqpoint{5.178537in}{6.920992in}}%
\pgfpathcurveto{\pgfqpoint{5.170723in}{6.913178in}}{\pgfqpoint{5.166333in}{6.902579in}}{\pgfqpoint{5.166333in}{6.891529in}}%
\pgfpathcurveto{\pgfqpoint{5.166333in}{6.880479in}}{\pgfqpoint{5.170723in}{6.869880in}}{\pgfqpoint{5.178537in}{6.862066in}}%
\pgfpathcurveto{\pgfqpoint{5.186350in}{6.854253in}}{\pgfqpoint{5.196949in}{6.849862in}}{\pgfqpoint{5.207999in}{6.849862in}}%
\pgfpathclose%
\pgfusepath{stroke,fill}%
\end{pgfscope}%
\begin{pgfscope}%
\pgfpathrectangle{\pgfqpoint{0.481978in}{0.331635in}}{\pgfqpoint{9.300000in}{7.700000in}}%
\pgfusepath{clip}%
\pgfsetbuttcap%
\pgfsetroundjoin%
\definecolor{currentfill}{rgb}{0.631373,0.788235,0.956863}%
\pgfsetfillcolor{currentfill}%
\pgfsetlinewidth{0.481800pt}%
\definecolor{currentstroke}{rgb}{1.000000,1.000000,1.000000}%
\pgfsetstrokecolor{currentstroke}%
\pgfsetdash{}{0pt}%
\pgfpathmoveto{\pgfqpoint{4.854525in}{3.573118in}}%
\pgfpathcurveto{\pgfqpoint{4.865575in}{3.573118in}}{\pgfqpoint{4.876174in}{3.577508in}}{\pgfqpoint{4.883988in}{3.585322in}}%
\pgfpathcurveto{\pgfqpoint{4.891801in}{3.593135in}}{\pgfqpoint{4.896192in}{3.603734in}}{\pgfqpoint{4.896192in}{3.614784in}}%
\pgfpathcurveto{\pgfqpoint{4.896192in}{3.625834in}}{\pgfqpoint{4.891801in}{3.636433in}}{\pgfqpoint{4.883988in}{3.644247in}}%
\pgfpathcurveto{\pgfqpoint{4.876174in}{3.652061in}}{\pgfqpoint{4.865575in}{3.656451in}}{\pgfqpoint{4.854525in}{3.656451in}}%
\pgfpathcurveto{\pgfqpoint{4.843475in}{3.656451in}}{\pgfqpoint{4.832876in}{3.652061in}}{\pgfqpoint{4.825062in}{3.644247in}}%
\pgfpathcurveto{\pgfqpoint{4.817249in}{3.636433in}}{\pgfqpoint{4.812858in}{3.625834in}}{\pgfqpoint{4.812858in}{3.614784in}}%
\pgfpathcurveto{\pgfqpoint{4.812858in}{3.603734in}}{\pgfqpoint{4.817249in}{3.593135in}}{\pgfqpoint{4.825062in}{3.585322in}}%
\pgfpathcurveto{\pgfqpoint{4.832876in}{3.577508in}}{\pgfqpoint{4.843475in}{3.573118in}}{\pgfqpoint{4.854525in}{3.573118in}}%
\pgfpathclose%
\pgfusepath{stroke,fill}%
\end{pgfscope}%
\begin{pgfscope}%
\pgfpathrectangle{\pgfqpoint{0.481978in}{0.331635in}}{\pgfqpoint{9.300000in}{7.700000in}}%
\pgfusepath{clip}%
\pgfsetbuttcap%
\pgfsetroundjoin%
\definecolor{currentfill}{rgb}{0.631373,0.788235,0.956863}%
\pgfsetfillcolor{currentfill}%
\pgfsetlinewidth{0.481800pt}%
\definecolor{currentstroke}{rgb}{1.000000,1.000000,1.000000}%
\pgfsetstrokecolor{currentstroke}%
\pgfsetdash{}{0pt}%
\pgfpathmoveto{\pgfqpoint{3.734095in}{1.462941in}}%
\pgfpathcurveto{\pgfqpoint{3.745145in}{1.462941in}}{\pgfqpoint{3.755744in}{1.467331in}}{\pgfqpoint{3.763558in}{1.475145in}}%
\pgfpathcurveto{\pgfqpoint{3.771371in}{1.482959in}}{\pgfqpoint{3.775761in}{1.493558in}}{\pgfqpoint{3.775761in}{1.504608in}}%
\pgfpathcurveto{\pgfqpoint{3.775761in}{1.515658in}}{\pgfqpoint{3.771371in}{1.526257in}}{\pgfqpoint{3.763558in}{1.534071in}}%
\pgfpathcurveto{\pgfqpoint{3.755744in}{1.541884in}}{\pgfqpoint{3.745145in}{1.546274in}}{\pgfqpoint{3.734095in}{1.546274in}}%
\pgfpathcurveto{\pgfqpoint{3.723045in}{1.546274in}}{\pgfqpoint{3.712446in}{1.541884in}}{\pgfqpoint{3.704632in}{1.534071in}}%
\pgfpathcurveto{\pgfqpoint{3.696818in}{1.526257in}}{\pgfqpoint{3.692428in}{1.515658in}}{\pgfqpoint{3.692428in}{1.504608in}}%
\pgfpathcurveto{\pgfqpoint{3.692428in}{1.493558in}}{\pgfqpoint{3.696818in}{1.482959in}}{\pgfqpoint{3.704632in}{1.475145in}}%
\pgfpathcurveto{\pgfqpoint{3.712446in}{1.467331in}}{\pgfqpoint{3.723045in}{1.462941in}}{\pgfqpoint{3.734095in}{1.462941in}}%
\pgfpathclose%
\pgfusepath{stroke,fill}%
\end{pgfscope}%
\begin{pgfscope}%
\pgfpathrectangle{\pgfqpoint{0.481978in}{0.331635in}}{\pgfqpoint{9.300000in}{7.700000in}}%
\pgfusepath{clip}%
\pgfsetbuttcap%
\pgfsetroundjoin%
\definecolor{currentfill}{rgb}{0.631373,0.788235,0.956863}%
\pgfsetfillcolor{currentfill}%
\pgfsetlinewidth{0.481800pt}%
\definecolor{currentstroke}{rgb}{1.000000,1.000000,1.000000}%
\pgfsetstrokecolor{currentstroke}%
\pgfsetdash{}{0pt}%
\pgfpathmoveto{\pgfqpoint{3.327053in}{1.390346in}}%
\pgfpathcurveto{\pgfqpoint{3.338103in}{1.390346in}}{\pgfqpoint{3.348702in}{1.394736in}}{\pgfqpoint{3.356516in}{1.402550in}}%
\pgfpathcurveto{\pgfqpoint{3.364329in}{1.410363in}}{\pgfqpoint{3.368720in}{1.420962in}}{\pgfqpoint{3.368720in}{1.432012in}}%
\pgfpathcurveto{\pgfqpoint{3.368720in}{1.443062in}}{\pgfqpoint{3.364329in}{1.453661in}}{\pgfqpoint{3.356516in}{1.461475in}}%
\pgfpathcurveto{\pgfqpoint{3.348702in}{1.469289in}}{\pgfqpoint{3.338103in}{1.473679in}}{\pgfqpoint{3.327053in}{1.473679in}}%
\pgfpathcurveto{\pgfqpoint{3.316003in}{1.473679in}}{\pgfqpoint{3.305404in}{1.469289in}}{\pgfqpoint{3.297590in}{1.461475in}}%
\pgfpathcurveto{\pgfqpoint{3.289777in}{1.453661in}}{\pgfqpoint{3.285386in}{1.443062in}}{\pgfqpoint{3.285386in}{1.432012in}}%
\pgfpathcurveto{\pgfqpoint{3.285386in}{1.420962in}}{\pgfqpoint{3.289777in}{1.410363in}}{\pgfqpoint{3.297590in}{1.402550in}}%
\pgfpathcurveto{\pgfqpoint{3.305404in}{1.394736in}}{\pgfqpoint{3.316003in}{1.390346in}}{\pgfqpoint{3.327053in}{1.390346in}}%
\pgfpathclose%
\pgfusepath{stroke,fill}%
\end{pgfscope}%
\begin{pgfscope}%
\pgfpathrectangle{\pgfqpoint{0.481978in}{0.331635in}}{\pgfqpoint{9.300000in}{7.700000in}}%
\pgfusepath{clip}%
\pgfsetbuttcap%
\pgfsetroundjoin%
\definecolor{currentfill}{rgb}{0.631373,0.788235,0.956863}%
\pgfsetfillcolor{currentfill}%
\pgfsetlinewidth{0.481800pt}%
\definecolor{currentstroke}{rgb}{1.000000,1.000000,1.000000}%
\pgfsetstrokecolor{currentstroke}%
\pgfsetdash{}{0pt}%
\pgfpathmoveto{\pgfqpoint{5.992810in}{4.018090in}}%
\pgfpathcurveto{\pgfqpoint{6.003860in}{4.018090in}}{\pgfqpoint{6.014459in}{4.022480in}}{\pgfqpoint{6.022273in}{4.030294in}}%
\pgfpathcurveto{\pgfqpoint{6.030087in}{4.038108in}}{\pgfqpoint{6.034477in}{4.048707in}}{\pgfqpoint{6.034477in}{4.059757in}}%
\pgfpathcurveto{\pgfqpoint{6.034477in}{4.070807in}}{\pgfqpoint{6.030087in}{4.081406in}}{\pgfqpoint{6.022273in}{4.089220in}}%
\pgfpathcurveto{\pgfqpoint{6.014459in}{4.097033in}}{\pgfqpoint{6.003860in}{4.101423in}}{\pgfqpoint{5.992810in}{4.101423in}}%
\pgfpathcurveto{\pgfqpoint{5.981760in}{4.101423in}}{\pgfqpoint{5.971161in}{4.097033in}}{\pgfqpoint{5.963347in}{4.089220in}}%
\pgfpathcurveto{\pgfqpoint{5.955534in}{4.081406in}}{\pgfqpoint{5.951144in}{4.070807in}}{\pgfqpoint{5.951144in}{4.059757in}}%
\pgfpathcurveto{\pgfqpoint{5.951144in}{4.048707in}}{\pgfqpoint{5.955534in}{4.038108in}}{\pgfqpoint{5.963347in}{4.030294in}}%
\pgfpathcurveto{\pgfqpoint{5.971161in}{4.022480in}}{\pgfqpoint{5.981760in}{4.018090in}}{\pgfqpoint{5.992810in}{4.018090in}}%
\pgfpathclose%
\pgfusepath{stroke,fill}%
\end{pgfscope}%
\begin{pgfscope}%
\pgfpathrectangle{\pgfqpoint{0.481978in}{0.331635in}}{\pgfqpoint{9.300000in}{7.700000in}}%
\pgfusepath{clip}%
\pgfsetbuttcap%
\pgfsetroundjoin%
\definecolor{currentfill}{rgb}{0.631373,0.788235,0.956863}%
\pgfsetfillcolor{currentfill}%
\pgfsetlinewidth{0.481800pt}%
\definecolor{currentstroke}{rgb}{1.000000,1.000000,1.000000}%
\pgfsetstrokecolor{currentstroke}%
\pgfsetdash{}{0pt}%
\pgfpathmoveto{\pgfqpoint{5.648838in}{3.512473in}}%
\pgfpathcurveto{\pgfqpoint{5.659888in}{3.512473in}}{\pgfqpoint{5.670487in}{3.516864in}}{\pgfqpoint{5.678301in}{3.524677in}}%
\pgfpathcurveto{\pgfqpoint{5.686115in}{3.532491in}}{\pgfqpoint{5.690505in}{3.543090in}}{\pgfqpoint{5.690505in}{3.554140in}}%
\pgfpathcurveto{\pgfqpoint{5.690505in}{3.565190in}}{\pgfqpoint{5.686115in}{3.575789in}}{\pgfqpoint{5.678301in}{3.583603in}}%
\pgfpathcurveto{\pgfqpoint{5.670487in}{3.591416in}}{\pgfqpoint{5.659888in}{3.595807in}}{\pgfqpoint{5.648838in}{3.595807in}}%
\pgfpathcurveto{\pgfqpoint{5.637788in}{3.595807in}}{\pgfqpoint{5.627189in}{3.591416in}}{\pgfqpoint{5.619376in}{3.583603in}}%
\pgfpathcurveto{\pgfqpoint{5.611562in}{3.575789in}}{\pgfqpoint{5.607172in}{3.565190in}}{\pgfqpoint{5.607172in}{3.554140in}}%
\pgfpathcurveto{\pgfqpoint{5.607172in}{3.543090in}}{\pgfqpoint{5.611562in}{3.532491in}}{\pgfqpoint{5.619376in}{3.524677in}}%
\pgfpathcurveto{\pgfqpoint{5.627189in}{3.516864in}}{\pgfqpoint{5.637788in}{3.512473in}}{\pgfqpoint{5.648838in}{3.512473in}}%
\pgfpathclose%
\pgfusepath{stroke,fill}%
\end{pgfscope}%
\begin{pgfscope}%
\pgfpathrectangle{\pgfqpoint{0.481978in}{0.331635in}}{\pgfqpoint{9.300000in}{7.700000in}}%
\pgfusepath{clip}%
\pgfsetbuttcap%
\pgfsetroundjoin%
\definecolor{currentfill}{rgb}{0.631373,0.788235,0.956863}%
\pgfsetfillcolor{currentfill}%
\pgfsetlinewidth{0.481800pt}%
\definecolor{currentstroke}{rgb}{1.000000,1.000000,1.000000}%
\pgfsetstrokecolor{currentstroke}%
\pgfsetdash{}{0pt}%
\pgfpathmoveto{\pgfqpoint{6.635439in}{5.352176in}}%
\pgfpathcurveto{\pgfqpoint{6.646489in}{5.352176in}}{\pgfqpoint{6.657088in}{5.356566in}}{\pgfqpoint{6.664902in}{5.364380in}}%
\pgfpathcurveto{\pgfqpoint{6.672715in}{5.372193in}}{\pgfqpoint{6.677105in}{5.382792in}}{\pgfqpoint{6.677105in}{5.393842in}}%
\pgfpathcurveto{\pgfqpoint{6.677105in}{5.404892in}}{\pgfqpoint{6.672715in}{5.415491in}}{\pgfqpoint{6.664902in}{5.423305in}}%
\pgfpathcurveto{\pgfqpoint{6.657088in}{5.431119in}}{\pgfqpoint{6.646489in}{5.435509in}}{\pgfqpoint{6.635439in}{5.435509in}}%
\pgfpathcurveto{\pgfqpoint{6.624389in}{5.435509in}}{\pgfqpoint{6.613790in}{5.431119in}}{\pgfqpoint{6.605976in}{5.423305in}}%
\pgfpathcurveto{\pgfqpoint{6.598162in}{5.415491in}}{\pgfqpoint{6.593772in}{5.404892in}}{\pgfqpoint{6.593772in}{5.393842in}}%
\pgfpathcurveto{\pgfqpoint{6.593772in}{5.382792in}}{\pgfqpoint{6.598162in}{5.372193in}}{\pgfqpoint{6.605976in}{5.364380in}}%
\pgfpathcurveto{\pgfqpoint{6.613790in}{5.356566in}}{\pgfqpoint{6.624389in}{5.352176in}}{\pgfqpoint{6.635439in}{5.352176in}}%
\pgfpathclose%
\pgfusepath{stroke,fill}%
\end{pgfscope}%
\begin{pgfscope}%
\pgfpathrectangle{\pgfqpoint{0.481978in}{0.331635in}}{\pgfqpoint{9.300000in}{7.700000in}}%
\pgfusepath{clip}%
\pgfsetbuttcap%
\pgfsetroundjoin%
\definecolor{currentfill}{rgb}{0.631373,0.788235,0.956863}%
\pgfsetfillcolor{currentfill}%
\pgfsetlinewidth{0.481800pt}%
\definecolor{currentstroke}{rgb}{1.000000,1.000000,1.000000}%
\pgfsetstrokecolor{currentstroke}%
\pgfsetdash{}{0pt}%
\pgfpathmoveto{\pgfqpoint{6.399047in}{2.029599in}}%
\pgfpathcurveto{\pgfqpoint{6.410097in}{2.029599in}}{\pgfqpoint{6.420696in}{2.033990in}}{\pgfqpoint{6.428510in}{2.041803in}}%
\pgfpathcurveto{\pgfqpoint{6.436323in}{2.049617in}}{\pgfqpoint{6.440713in}{2.060216in}}{\pgfqpoint{6.440713in}{2.071266in}}%
\pgfpathcurveto{\pgfqpoint{6.440713in}{2.082316in}}{\pgfqpoint{6.436323in}{2.092915in}}{\pgfqpoint{6.428510in}{2.100729in}}%
\pgfpathcurveto{\pgfqpoint{6.420696in}{2.108542in}}{\pgfqpoint{6.410097in}{2.112933in}}{\pgfqpoint{6.399047in}{2.112933in}}%
\pgfpathcurveto{\pgfqpoint{6.387997in}{2.112933in}}{\pgfqpoint{6.377398in}{2.108542in}}{\pgfqpoint{6.369584in}{2.100729in}}%
\pgfpathcurveto{\pgfqpoint{6.361770in}{2.092915in}}{\pgfqpoint{6.357380in}{2.082316in}}{\pgfqpoint{6.357380in}{2.071266in}}%
\pgfpathcurveto{\pgfqpoint{6.357380in}{2.060216in}}{\pgfqpoint{6.361770in}{2.049617in}}{\pgfqpoint{6.369584in}{2.041803in}}%
\pgfpathcurveto{\pgfqpoint{6.377398in}{2.033990in}}{\pgfqpoint{6.387997in}{2.029599in}}{\pgfqpoint{6.399047in}{2.029599in}}%
\pgfpathclose%
\pgfusepath{stroke,fill}%
\end{pgfscope}%
\begin{pgfscope}%
\pgfpathrectangle{\pgfqpoint{0.481978in}{0.331635in}}{\pgfqpoint{9.300000in}{7.700000in}}%
\pgfusepath{clip}%
\pgfsetbuttcap%
\pgfsetroundjoin%
\definecolor{currentfill}{rgb}{0.631373,0.788235,0.956863}%
\pgfsetfillcolor{currentfill}%
\pgfsetlinewidth{0.481800pt}%
\definecolor{currentstroke}{rgb}{1.000000,1.000000,1.000000}%
\pgfsetstrokecolor{currentstroke}%
\pgfsetdash{}{0pt}%
\pgfpathmoveto{\pgfqpoint{2.913660in}{6.822845in}}%
\pgfpathcurveto{\pgfqpoint{2.924710in}{6.822845in}}{\pgfqpoint{2.935309in}{6.827235in}}{\pgfqpoint{2.943123in}{6.835049in}}%
\pgfpathcurveto{\pgfqpoint{2.950937in}{6.842863in}}{\pgfqpoint{2.955327in}{6.853462in}}{\pgfqpoint{2.955327in}{6.864512in}}%
\pgfpathcurveto{\pgfqpoint{2.955327in}{6.875562in}}{\pgfqpoint{2.950937in}{6.886161in}}{\pgfqpoint{2.943123in}{6.893975in}}%
\pgfpathcurveto{\pgfqpoint{2.935309in}{6.901788in}}{\pgfqpoint{2.924710in}{6.906179in}}{\pgfqpoint{2.913660in}{6.906179in}}%
\pgfpathcurveto{\pgfqpoint{2.902610in}{6.906179in}}{\pgfqpoint{2.892011in}{6.901788in}}{\pgfqpoint{2.884197in}{6.893975in}}%
\pgfpathcurveto{\pgfqpoint{2.876384in}{6.886161in}}{\pgfqpoint{2.871994in}{6.875562in}}{\pgfqpoint{2.871994in}{6.864512in}}%
\pgfpathcurveto{\pgfqpoint{2.871994in}{6.853462in}}{\pgfqpoint{2.876384in}{6.842863in}}{\pgfqpoint{2.884197in}{6.835049in}}%
\pgfpathcurveto{\pgfqpoint{2.892011in}{6.827235in}}{\pgfqpoint{2.902610in}{6.822845in}}{\pgfqpoint{2.913660in}{6.822845in}}%
\pgfpathclose%
\pgfusepath{stroke,fill}%
\end{pgfscope}%
\begin{pgfscope}%
\pgfpathrectangle{\pgfqpoint{0.481978in}{0.331635in}}{\pgfqpoint{9.300000in}{7.700000in}}%
\pgfusepath{clip}%
\pgfsetbuttcap%
\pgfsetroundjoin%
\definecolor{currentfill}{rgb}{0.631373,0.788235,0.956863}%
\pgfsetfillcolor{currentfill}%
\pgfsetlinewidth{0.481800pt}%
\definecolor{currentstroke}{rgb}{1.000000,1.000000,1.000000}%
\pgfsetstrokecolor{currentstroke}%
\pgfsetdash{}{0pt}%
\pgfpathmoveto{\pgfqpoint{4.618112in}{4.334558in}}%
\pgfpathcurveto{\pgfqpoint{4.629162in}{4.334558in}}{\pgfqpoint{4.639761in}{4.338948in}}{\pgfqpoint{4.647575in}{4.346762in}}%
\pgfpathcurveto{\pgfqpoint{4.655388in}{4.354575in}}{\pgfqpoint{4.659778in}{4.365174in}}{\pgfqpoint{4.659778in}{4.376225in}}%
\pgfpathcurveto{\pgfqpoint{4.659778in}{4.387275in}}{\pgfqpoint{4.655388in}{4.397874in}}{\pgfqpoint{4.647575in}{4.405687in}}%
\pgfpathcurveto{\pgfqpoint{4.639761in}{4.413501in}}{\pgfqpoint{4.629162in}{4.417891in}}{\pgfqpoint{4.618112in}{4.417891in}}%
\pgfpathcurveto{\pgfqpoint{4.607062in}{4.417891in}}{\pgfqpoint{4.596463in}{4.413501in}}{\pgfqpoint{4.588649in}{4.405687in}}%
\pgfpathcurveto{\pgfqpoint{4.580835in}{4.397874in}}{\pgfqpoint{4.576445in}{4.387275in}}{\pgfqpoint{4.576445in}{4.376225in}}%
\pgfpathcurveto{\pgfqpoint{4.576445in}{4.365174in}}{\pgfqpoint{4.580835in}{4.354575in}}{\pgfqpoint{4.588649in}{4.346762in}}%
\pgfpathcurveto{\pgfqpoint{4.596463in}{4.338948in}}{\pgfqpoint{4.607062in}{4.334558in}}{\pgfqpoint{4.618112in}{4.334558in}}%
\pgfpathclose%
\pgfusepath{stroke,fill}%
\end{pgfscope}%
\begin{pgfscope}%
\pgfpathrectangle{\pgfqpoint{0.481978in}{0.331635in}}{\pgfqpoint{9.300000in}{7.700000in}}%
\pgfusepath{clip}%
\pgfsetbuttcap%
\pgfsetroundjoin%
\definecolor{currentfill}{rgb}{0.631373,0.788235,0.956863}%
\pgfsetfillcolor{currentfill}%
\pgfsetlinewidth{0.481800pt}%
\definecolor{currentstroke}{rgb}{1.000000,1.000000,1.000000}%
\pgfsetstrokecolor{currentstroke}%
\pgfsetdash{}{0pt}%
\pgfpathmoveto{\pgfqpoint{7.244029in}{2.429882in}}%
\pgfpathcurveto{\pgfqpoint{7.255079in}{2.429882in}}{\pgfqpoint{7.265678in}{2.434272in}}{\pgfqpoint{7.273491in}{2.442085in}}%
\pgfpathcurveto{\pgfqpoint{7.281305in}{2.449899in}}{\pgfqpoint{7.285695in}{2.460498in}}{\pgfqpoint{7.285695in}{2.471548in}}%
\pgfpathcurveto{\pgfqpoint{7.285695in}{2.482598in}}{\pgfqpoint{7.281305in}{2.493197in}}{\pgfqpoint{7.273491in}{2.501011in}}%
\pgfpathcurveto{\pgfqpoint{7.265678in}{2.508825in}}{\pgfqpoint{7.255079in}{2.513215in}}{\pgfqpoint{7.244029in}{2.513215in}}%
\pgfpathcurveto{\pgfqpoint{7.232978in}{2.513215in}}{\pgfqpoint{7.222379in}{2.508825in}}{\pgfqpoint{7.214566in}{2.501011in}}%
\pgfpathcurveto{\pgfqpoint{7.206752in}{2.493197in}}{\pgfqpoint{7.202362in}{2.482598in}}{\pgfqpoint{7.202362in}{2.471548in}}%
\pgfpathcurveto{\pgfqpoint{7.202362in}{2.460498in}}{\pgfqpoint{7.206752in}{2.449899in}}{\pgfqpoint{7.214566in}{2.442085in}}%
\pgfpathcurveto{\pgfqpoint{7.222379in}{2.434272in}}{\pgfqpoint{7.232978in}{2.429882in}}{\pgfqpoint{7.244029in}{2.429882in}}%
\pgfpathclose%
\pgfusepath{stroke,fill}%
\end{pgfscope}%
\begin{pgfscope}%
\pgfpathrectangle{\pgfqpoint{0.481978in}{0.331635in}}{\pgfqpoint{9.300000in}{7.700000in}}%
\pgfusepath{clip}%
\pgfsetbuttcap%
\pgfsetroundjoin%
\definecolor{currentfill}{rgb}{0.631373,0.788235,0.956863}%
\pgfsetfillcolor{currentfill}%
\pgfsetlinewidth{0.481800pt}%
\definecolor{currentstroke}{rgb}{1.000000,1.000000,1.000000}%
\pgfsetstrokecolor{currentstroke}%
\pgfsetdash{}{0pt}%
\pgfpathmoveto{\pgfqpoint{6.714656in}{1.525447in}}%
\pgfpathcurveto{\pgfqpoint{6.725706in}{1.525447in}}{\pgfqpoint{6.736305in}{1.529837in}}{\pgfqpoint{6.744119in}{1.537651in}}%
\pgfpathcurveto{\pgfqpoint{6.751932in}{1.545465in}}{\pgfqpoint{6.756322in}{1.556064in}}{\pgfqpoint{6.756322in}{1.567114in}}%
\pgfpathcurveto{\pgfqpoint{6.756322in}{1.578164in}}{\pgfqpoint{6.751932in}{1.588763in}}{\pgfqpoint{6.744119in}{1.596577in}}%
\pgfpathcurveto{\pgfqpoint{6.736305in}{1.604390in}}{\pgfqpoint{6.725706in}{1.608780in}}{\pgfqpoint{6.714656in}{1.608780in}}%
\pgfpathcurveto{\pgfqpoint{6.703606in}{1.608780in}}{\pgfqpoint{6.693007in}{1.604390in}}{\pgfqpoint{6.685193in}{1.596577in}}%
\pgfpathcurveto{\pgfqpoint{6.677379in}{1.588763in}}{\pgfqpoint{6.672989in}{1.578164in}}{\pgfqpoint{6.672989in}{1.567114in}}%
\pgfpathcurveto{\pgfqpoint{6.672989in}{1.556064in}}{\pgfqpoint{6.677379in}{1.545465in}}{\pgfqpoint{6.685193in}{1.537651in}}%
\pgfpathcurveto{\pgfqpoint{6.693007in}{1.529837in}}{\pgfqpoint{6.703606in}{1.525447in}}{\pgfqpoint{6.714656in}{1.525447in}}%
\pgfpathclose%
\pgfusepath{stroke,fill}%
\end{pgfscope}%
\begin{pgfscope}%
\pgfpathrectangle{\pgfqpoint{0.481978in}{0.331635in}}{\pgfqpoint{9.300000in}{7.700000in}}%
\pgfusepath{clip}%
\pgfsetbuttcap%
\pgfsetroundjoin%
\definecolor{currentfill}{rgb}{0.631373,0.788235,0.956863}%
\pgfsetfillcolor{currentfill}%
\pgfsetlinewidth{0.481800pt}%
\definecolor{currentstroke}{rgb}{1.000000,1.000000,1.000000}%
\pgfsetstrokecolor{currentstroke}%
\pgfsetdash{}{0pt}%
\pgfpathmoveto{\pgfqpoint{3.591337in}{1.267658in}}%
\pgfpathcurveto{\pgfqpoint{3.602388in}{1.267658in}}{\pgfqpoint{3.612987in}{1.272048in}}{\pgfqpoint{3.620800in}{1.279862in}}%
\pgfpathcurveto{\pgfqpoint{3.628614in}{1.287675in}}{\pgfqpoint{3.633004in}{1.298274in}}{\pgfqpoint{3.633004in}{1.309324in}}%
\pgfpathcurveto{\pgfqpoint{3.633004in}{1.320375in}}{\pgfqpoint{3.628614in}{1.330974in}}{\pgfqpoint{3.620800in}{1.338787in}}%
\pgfpathcurveto{\pgfqpoint{3.612987in}{1.346601in}}{\pgfqpoint{3.602388in}{1.350991in}}{\pgfqpoint{3.591337in}{1.350991in}}%
\pgfpathcurveto{\pgfqpoint{3.580287in}{1.350991in}}{\pgfqpoint{3.569688in}{1.346601in}}{\pgfqpoint{3.561875in}{1.338787in}}%
\pgfpathcurveto{\pgfqpoint{3.554061in}{1.330974in}}{\pgfqpoint{3.549671in}{1.320375in}}{\pgfqpoint{3.549671in}{1.309324in}}%
\pgfpathcurveto{\pgfqpoint{3.549671in}{1.298274in}}{\pgfqpoint{3.554061in}{1.287675in}}{\pgfqpoint{3.561875in}{1.279862in}}%
\pgfpathcurveto{\pgfqpoint{3.569688in}{1.272048in}}{\pgfqpoint{3.580287in}{1.267658in}}{\pgfqpoint{3.591337in}{1.267658in}}%
\pgfpathclose%
\pgfusepath{stroke,fill}%
\end{pgfscope}%
\begin{pgfscope}%
\pgfpathrectangle{\pgfqpoint{0.481978in}{0.331635in}}{\pgfqpoint{9.300000in}{7.700000in}}%
\pgfusepath{clip}%
\pgfsetbuttcap%
\pgfsetroundjoin%
\definecolor{currentfill}{rgb}{0.631373,0.788235,0.956863}%
\pgfsetfillcolor{currentfill}%
\pgfsetlinewidth{0.481800pt}%
\definecolor{currentstroke}{rgb}{1.000000,1.000000,1.000000}%
\pgfsetstrokecolor{currentstroke}%
\pgfsetdash{}{0pt}%
\pgfpathmoveto{\pgfqpoint{8.476836in}{5.231722in}}%
\pgfpathcurveto{\pgfqpoint{8.487886in}{5.231722in}}{\pgfqpoint{8.498485in}{5.236112in}}{\pgfqpoint{8.506299in}{5.243926in}}%
\pgfpathcurveto{\pgfqpoint{8.514112in}{5.251740in}}{\pgfqpoint{8.518503in}{5.262339in}}{\pgfqpoint{8.518503in}{5.273389in}}%
\pgfpathcurveto{\pgfqpoint{8.518503in}{5.284439in}}{\pgfqpoint{8.514112in}{5.295038in}}{\pgfqpoint{8.506299in}{5.302852in}}%
\pgfpathcurveto{\pgfqpoint{8.498485in}{5.310665in}}{\pgfqpoint{8.487886in}{5.315055in}}{\pgfqpoint{8.476836in}{5.315055in}}%
\pgfpathcurveto{\pgfqpoint{8.465786in}{5.315055in}}{\pgfqpoint{8.455187in}{5.310665in}}{\pgfqpoint{8.447373in}{5.302852in}}%
\pgfpathcurveto{\pgfqpoint{8.439559in}{5.295038in}}{\pgfqpoint{8.435169in}{5.284439in}}{\pgfqpoint{8.435169in}{5.273389in}}%
\pgfpathcurveto{\pgfqpoint{8.435169in}{5.262339in}}{\pgfqpoint{8.439559in}{5.251740in}}{\pgfqpoint{8.447373in}{5.243926in}}%
\pgfpathcurveto{\pgfqpoint{8.455187in}{5.236112in}}{\pgfqpoint{8.465786in}{5.231722in}}{\pgfqpoint{8.476836in}{5.231722in}}%
\pgfpathclose%
\pgfusepath{stroke,fill}%
\end{pgfscope}%
\begin{pgfscope}%
\pgfpathrectangle{\pgfqpoint{0.481978in}{0.331635in}}{\pgfqpoint{9.300000in}{7.700000in}}%
\pgfusepath{clip}%
\pgfsetbuttcap%
\pgfsetroundjoin%
\definecolor{currentfill}{rgb}{0.631373,0.788235,0.956863}%
\pgfsetfillcolor{currentfill}%
\pgfsetlinewidth{0.481800pt}%
\definecolor{currentstroke}{rgb}{1.000000,1.000000,1.000000}%
\pgfsetstrokecolor{currentstroke}%
\pgfsetdash{}{0pt}%
\pgfpathmoveto{\pgfqpoint{6.886735in}{1.411937in}}%
\pgfpathcurveto{\pgfqpoint{6.897785in}{1.411937in}}{\pgfqpoint{6.908385in}{1.416327in}}{\pgfqpoint{6.916198in}{1.424141in}}%
\pgfpathcurveto{\pgfqpoint{6.924012in}{1.431955in}}{\pgfqpoint{6.928402in}{1.442554in}}{\pgfqpoint{6.928402in}{1.453604in}}%
\pgfpathcurveto{\pgfqpoint{6.928402in}{1.464654in}}{\pgfqpoint{6.924012in}{1.475253in}}{\pgfqpoint{6.916198in}{1.483067in}}%
\pgfpathcurveto{\pgfqpoint{6.908385in}{1.490880in}}{\pgfqpoint{6.897785in}{1.495270in}}{\pgfqpoint{6.886735in}{1.495270in}}%
\pgfpathcurveto{\pgfqpoint{6.875685in}{1.495270in}}{\pgfqpoint{6.865086in}{1.490880in}}{\pgfqpoint{6.857273in}{1.483067in}}%
\pgfpathcurveto{\pgfqpoint{6.849459in}{1.475253in}}{\pgfqpoint{6.845069in}{1.464654in}}{\pgfqpoint{6.845069in}{1.453604in}}%
\pgfpathcurveto{\pgfqpoint{6.845069in}{1.442554in}}{\pgfqpoint{6.849459in}{1.431955in}}{\pgfqpoint{6.857273in}{1.424141in}}%
\pgfpathcurveto{\pgfqpoint{6.865086in}{1.416327in}}{\pgfqpoint{6.875685in}{1.411937in}}{\pgfqpoint{6.886735in}{1.411937in}}%
\pgfpathclose%
\pgfusepath{stroke,fill}%
\end{pgfscope}%
\begin{pgfscope}%
\pgfpathrectangle{\pgfqpoint{0.481978in}{0.331635in}}{\pgfqpoint{9.300000in}{7.700000in}}%
\pgfusepath{clip}%
\pgfsetbuttcap%
\pgfsetroundjoin%
\definecolor{currentfill}{rgb}{0.631373,0.788235,0.956863}%
\pgfsetfillcolor{currentfill}%
\pgfsetlinewidth{0.481800pt}%
\definecolor{currentstroke}{rgb}{1.000000,1.000000,1.000000}%
\pgfsetstrokecolor{currentstroke}%
\pgfsetdash{}{0pt}%
\pgfpathmoveto{\pgfqpoint{5.148908in}{1.630570in}}%
\pgfpathcurveto{\pgfqpoint{5.159959in}{1.630570in}}{\pgfqpoint{5.170558in}{1.634961in}}{\pgfqpoint{5.178371in}{1.642774in}}%
\pgfpathcurveto{\pgfqpoint{5.186185in}{1.650588in}}{\pgfqpoint{5.190575in}{1.661187in}}{\pgfqpoint{5.190575in}{1.672237in}}%
\pgfpathcurveto{\pgfqpoint{5.190575in}{1.683287in}}{\pgfqpoint{5.186185in}{1.693886in}}{\pgfqpoint{5.178371in}{1.701700in}}%
\pgfpathcurveto{\pgfqpoint{5.170558in}{1.709514in}}{\pgfqpoint{5.159959in}{1.713904in}}{\pgfqpoint{5.148908in}{1.713904in}}%
\pgfpathcurveto{\pgfqpoint{5.137858in}{1.713904in}}{\pgfqpoint{5.127259in}{1.709514in}}{\pgfqpoint{5.119446in}{1.701700in}}%
\pgfpathcurveto{\pgfqpoint{5.111632in}{1.693886in}}{\pgfqpoint{5.107242in}{1.683287in}}{\pgfqpoint{5.107242in}{1.672237in}}%
\pgfpathcurveto{\pgfqpoint{5.107242in}{1.661187in}}{\pgfqpoint{5.111632in}{1.650588in}}{\pgfqpoint{5.119446in}{1.642774in}}%
\pgfpathcurveto{\pgfqpoint{5.127259in}{1.634961in}}{\pgfqpoint{5.137858in}{1.630570in}}{\pgfqpoint{5.148908in}{1.630570in}}%
\pgfpathclose%
\pgfusepath{stroke,fill}%
\end{pgfscope}%
\begin{pgfscope}%
\pgfpathrectangle{\pgfqpoint{0.481978in}{0.331635in}}{\pgfqpoint{9.300000in}{7.700000in}}%
\pgfusepath{clip}%
\pgfsetbuttcap%
\pgfsetroundjoin%
\definecolor{currentfill}{rgb}{0.631373,0.788235,0.956863}%
\pgfsetfillcolor{currentfill}%
\pgfsetlinewidth{0.481800pt}%
\definecolor{currentstroke}{rgb}{1.000000,1.000000,1.000000}%
\pgfsetstrokecolor{currentstroke}%
\pgfsetdash{}{0pt}%
\pgfpathmoveto{\pgfqpoint{5.442394in}{6.812671in}}%
\pgfpathcurveto{\pgfqpoint{5.453444in}{6.812671in}}{\pgfqpoint{5.464043in}{6.817061in}}{\pgfqpoint{5.471856in}{6.824874in}}%
\pgfpathcurveto{\pgfqpoint{5.479670in}{6.832688in}}{\pgfqpoint{5.484060in}{6.843287in}}{\pgfqpoint{5.484060in}{6.854337in}}%
\pgfpathcurveto{\pgfqpoint{5.484060in}{6.865387in}}{\pgfqpoint{5.479670in}{6.875986in}}{\pgfqpoint{5.471856in}{6.883800in}}%
\pgfpathcurveto{\pgfqpoint{5.464043in}{6.891614in}}{\pgfqpoint{5.453444in}{6.896004in}}{\pgfqpoint{5.442394in}{6.896004in}}%
\pgfpathcurveto{\pgfqpoint{5.431343in}{6.896004in}}{\pgfqpoint{5.420744in}{6.891614in}}{\pgfqpoint{5.412931in}{6.883800in}}%
\pgfpathcurveto{\pgfqpoint{5.405117in}{6.875986in}}{\pgfqpoint{5.400727in}{6.865387in}}{\pgfqpoint{5.400727in}{6.854337in}}%
\pgfpathcurveto{\pgfqpoint{5.400727in}{6.843287in}}{\pgfqpoint{5.405117in}{6.832688in}}{\pgfqpoint{5.412931in}{6.824874in}}%
\pgfpathcurveto{\pgfqpoint{5.420744in}{6.817061in}}{\pgfqpoint{5.431343in}{6.812671in}}{\pgfqpoint{5.442394in}{6.812671in}}%
\pgfpathclose%
\pgfusepath{stroke,fill}%
\end{pgfscope}%
\begin{pgfscope}%
\pgfpathrectangle{\pgfqpoint{0.481978in}{0.331635in}}{\pgfqpoint{9.300000in}{7.700000in}}%
\pgfusepath{clip}%
\pgfsetbuttcap%
\pgfsetroundjoin%
\definecolor{currentfill}{rgb}{0.631373,0.788235,0.956863}%
\pgfsetfillcolor{currentfill}%
\pgfsetlinewidth{0.481800pt}%
\definecolor{currentstroke}{rgb}{1.000000,1.000000,1.000000}%
\pgfsetstrokecolor{currentstroke}%
\pgfsetdash{}{0pt}%
\pgfpathmoveto{\pgfqpoint{6.133252in}{3.334993in}}%
\pgfpathcurveto{\pgfqpoint{6.144302in}{3.334993in}}{\pgfqpoint{6.154901in}{3.339383in}}{\pgfqpoint{6.162714in}{3.347196in}}%
\pgfpathcurveto{\pgfqpoint{6.170528in}{3.355010in}}{\pgfqpoint{6.174918in}{3.365609in}}{\pgfqpoint{6.174918in}{3.376659in}}%
\pgfpathcurveto{\pgfqpoint{6.174918in}{3.387709in}}{\pgfqpoint{6.170528in}{3.398308in}}{\pgfqpoint{6.162714in}{3.406122in}}%
\pgfpathcurveto{\pgfqpoint{6.154901in}{3.413936in}}{\pgfqpoint{6.144302in}{3.418326in}}{\pgfqpoint{6.133252in}{3.418326in}}%
\pgfpathcurveto{\pgfqpoint{6.122201in}{3.418326in}}{\pgfqpoint{6.111602in}{3.413936in}}{\pgfqpoint{6.103789in}{3.406122in}}%
\pgfpathcurveto{\pgfqpoint{6.095975in}{3.398308in}}{\pgfqpoint{6.091585in}{3.387709in}}{\pgfqpoint{6.091585in}{3.376659in}}%
\pgfpathcurveto{\pgfqpoint{6.091585in}{3.365609in}}{\pgfqpoint{6.095975in}{3.355010in}}{\pgfqpoint{6.103789in}{3.347196in}}%
\pgfpathcurveto{\pgfqpoint{6.111602in}{3.339383in}}{\pgfqpoint{6.122201in}{3.334993in}}{\pgfqpoint{6.133252in}{3.334993in}}%
\pgfpathclose%
\pgfusepath{stroke,fill}%
\end{pgfscope}%
\begin{pgfscope}%
\pgfpathrectangle{\pgfqpoint{0.481978in}{0.331635in}}{\pgfqpoint{9.300000in}{7.700000in}}%
\pgfusepath{clip}%
\pgfsetbuttcap%
\pgfsetroundjoin%
\definecolor{currentfill}{rgb}{0.631373,0.788235,0.956863}%
\pgfsetfillcolor{currentfill}%
\pgfsetlinewidth{0.481800pt}%
\definecolor{currentstroke}{rgb}{1.000000,1.000000,1.000000}%
\pgfsetstrokecolor{currentstroke}%
\pgfsetdash{}{0pt}%
\pgfpathmoveto{\pgfqpoint{6.565040in}{5.025815in}}%
\pgfpathcurveto{\pgfqpoint{6.576090in}{5.025815in}}{\pgfqpoint{6.586689in}{5.030205in}}{\pgfqpoint{6.594503in}{5.038019in}}%
\pgfpathcurveto{\pgfqpoint{6.602316in}{5.045832in}}{\pgfqpoint{6.606707in}{5.056431in}}{\pgfqpoint{6.606707in}{5.067481in}}%
\pgfpathcurveto{\pgfqpoint{6.606707in}{5.078532in}}{\pgfqpoint{6.602316in}{5.089131in}}{\pgfqpoint{6.594503in}{5.096944in}}%
\pgfpathcurveto{\pgfqpoint{6.586689in}{5.104758in}}{\pgfqpoint{6.576090in}{5.109148in}}{\pgfqpoint{6.565040in}{5.109148in}}%
\pgfpathcurveto{\pgfqpoint{6.553990in}{5.109148in}}{\pgfqpoint{6.543391in}{5.104758in}}{\pgfqpoint{6.535577in}{5.096944in}}%
\pgfpathcurveto{\pgfqpoint{6.527763in}{5.089131in}}{\pgfqpoint{6.523373in}{5.078532in}}{\pgfqpoint{6.523373in}{5.067481in}}%
\pgfpathcurveto{\pgfqpoint{6.523373in}{5.056431in}}{\pgfqpoint{6.527763in}{5.045832in}}{\pgfqpoint{6.535577in}{5.038019in}}%
\pgfpathcurveto{\pgfqpoint{6.543391in}{5.030205in}}{\pgfqpoint{6.553990in}{5.025815in}}{\pgfqpoint{6.565040in}{5.025815in}}%
\pgfpathclose%
\pgfusepath{stroke,fill}%
\end{pgfscope}%
\begin{pgfscope}%
\pgfpathrectangle{\pgfqpoint{0.481978in}{0.331635in}}{\pgfqpoint{9.300000in}{7.700000in}}%
\pgfusepath{clip}%
\pgfsetbuttcap%
\pgfsetroundjoin%
\definecolor{currentfill}{rgb}{0.631373,0.788235,0.956863}%
\pgfsetfillcolor{currentfill}%
\pgfsetlinewidth{0.481800pt}%
\definecolor{currentstroke}{rgb}{1.000000,1.000000,1.000000}%
\pgfsetstrokecolor{currentstroke}%
\pgfsetdash{}{0pt}%
\pgfpathmoveto{\pgfqpoint{6.488247in}{4.397656in}}%
\pgfpathcurveto{\pgfqpoint{6.499297in}{4.397656in}}{\pgfqpoint{6.509896in}{4.402046in}}{\pgfqpoint{6.517710in}{4.409860in}}%
\pgfpathcurveto{\pgfqpoint{6.525523in}{4.417673in}}{\pgfqpoint{6.529914in}{4.428272in}}{\pgfqpoint{6.529914in}{4.439323in}}%
\pgfpathcurveto{\pgfqpoint{6.529914in}{4.450373in}}{\pgfqpoint{6.525523in}{4.460972in}}{\pgfqpoint{6.517710in}{4.468785in}}%
\pgfpathcurveto{\pgfqpoint{6.509896in}{4.476599in}}{\pgfqpoint{6.499297in}{4.480989in}}{\pgfqpoint{6.488247in}{4.480989in}}%
\pgfpathcurveto{\pgfqpoint{6.477197in}{4.480989in}}{\pgfqpoint{6.466598in}{4.476599in}}{\pgfqpoint{6.458784in}{4.468785in}}%
\pgfpathcurveto{\pgfqpoint{6.450970in}{4.460972in}}{\pgfqpoint{6.446580in}{4.450373in}}{\pgfqpoint{6.446580in}{4.439323in}}%
\pgfpathcurveto{\pgfqpoint{6.446580in}{4.428272in}}{\pgfqpoint{6.450970in}{4.417673in}}{\pgfqpoint{6.458784in}{4.409860in}}%
\pgfpathcurveto{\pgfqpoint{6.466598in}{4.402046in}}{\pgfqpoint{6.477197in}{4.397656in}}{\pgfqpoint{6.488247in}{4.397656in}}%
\pgfpathclose%
\pgfusepath{stroke,fill}%
\end{pgfscope}%
\begin{pgfscope}%
\pgfpathrectangle{\pgfqpoint{0.481978in}{0.331635in}}{\pgfqpoint{9.300000in}{7.700000in}}%
\pgfusepath{clip}%
\pgfsetbuttcap%
\pgfsetroundjoin%
\definecolor{currentfill}{rgb}{0.631373,0.788235,0.956863}%
\pgfsetfillcolor{currentfill}%
\pgfsetlinewidth{0.481800pt}%
\definecolor{currentstroke}{rgb}{1.000000,1.000000,1.000000}%
\pgfsetstrokecolor{currentstroke}%
\pgfsetdash{}{0pt}%
\pgfpathmoveto{\pgfqpoint{6.070621in}{5.004506in}}%
\pgfpathcurveto{\pgfqpoint{6.081671in}{5.004506in}}{\pgfqpoint{6.092270in}{5.008896in}}{\pgfqpoint{6.100084in}{5.016710in}}%
\pgfpathcurveto{\pgfqpoint{6.107897in}{5.024524in}}{\pgfqpoint{6.112288in}{5.035123in}}{\pgfqpoint{6.112288in}{5.046173in}}%
\pgfpathcurveto{\pgfqpoint{6.112288in}{5.057223in}}{\pgfqpoint{6.107897in}{5.067822in}}{\pgfqpoint{6.100084in}{5.075636in}}%
\pgfpathcurveto{\pgfqpoint{6.092270in}{5.083449in}}{\pgfqpoint{6.081671in}{5.087839in}}{\pgfqpoint{6.070621in}{5.087839in}}%
\pgfpathcurveto{\pgfqpoint{6.059571in}{5.087839in}}{\pgfqpoint{6.048972in}{5.083449in}}{\pgfqpoint{6.041158in}{5.075636in}}%
\pgfpathcurveto{\pgfqpoint{6.033344in}{5.067822in}}{\pgfqpoint{6.028954in}{5.057223in}}{\pgfqpoint{6.028954in}{5.046173in}}%
\pgfpathcurveto{\pgfqpoint{6.028954in}{5.035123in}}{\pgfqpoint{6.033344in}{5.024524in}}{\pgfqpoint{6.041158in}{5.016710in}}%
\pgfpathcurveto{\pgfqpoint{6.048972in}{5.008896in}}{\pgfqpoint{6.059571in}{5.004506in}}{\pgfqpoint{6.070621in}{5.004506in}}%
\pgfpathclose%
\pgfusepath{stroke,fill}%
\end{pgfscope}%
\begin{pgfscope}%
\pgfpathrectangle{\pgfqpoint{0.481978in}{0.331635in}}{\pgfqpoint{9.300000in}{7.700000in}}%
\pgfusepath{clip}%
\pgfsetbuttcap%
\pgfsetroundjoin%
\definecolor{currentfill}{rgb}{0.631373,0.788235,0.956863}%
\pgfsetfillcolor{currentfill}%
\pgfsetlinewidth{0.481800pt}%
\definecolor{currentstroke}{rgb}{1.000000,1.000000,1.000000}%
\pgfsetstrokecolor{currentstroke}%
\pgfsetdash{}{0pt}%
\pgfpathmoveto{\pgfqpoint{4.466261in}{1.283836in}}%
\pgfpathcurveto{\pgfqpoint{4.477311in}{1.283836in}}{\pgfqpoint{4.487910in}{1.288226in}}{\pgfqpoint{4.495724in}{1.296040in}}%
\pgfpathcurveto{\pgfqpoint{4.503537in}{1.303854in}}{\pgfqpoint{4.507928in}{1.314453in}}{\pgfqpoint{4.507928in}{1.325503in}}%
\pgfpathcurveto{\pgfqpoint{4.507928in}{1.336553in}}{\pgfqpoint{4.503537in}{1.347152in}}{\pgfqpoint{4.495724in}{1.354966in}}%
\pgfpathcurveto{\pgfqpoint{4.487910in}{1.362779in}}{\pgfqpoint{4.477311in}{1.367169in}}{\pgfqpoint{4.466261in}{1.367169in}}%
\pgfpathcurveto{\pgfqpoint{4.455211in}{1.367169in}}{\pgfqpoint{4.444612in}{1.362779in}}{\pgfqpoint{4.436798in}{1.354966in}}%
\pgfpathcurveto{\pgfqpoint{4.428984in}{1.347152in}}{\pgfqpoint{4.424594in}{1.336553in}}{\pgfqpoint{4.424594in}{1.325503in}}%
\pgfpathcurveto{\pgfqpoint{4.424594in}{1.314453in}}{\pgfqpoint{4.428984in}{1.303854in}}{\pgfqpoint{4.436798in}{1.296040in}}%
\pgfpathcurveto{\pgfqpoint{4.444612in}{1.288226in}}{\pgfqpoint{4.455211in}{1.283836in}}{\pgfqpoint{4.466261in}{1.283836in}}%
\pgfpathclose%
\pgfusepath{stroke,fill}%
\end{pgfscope}%
\begin{pgfscope}%
\pgfpathrectangle{\pgfqpoint{0.481978in}{0.331635in}}{\pgfqpoint{9.300000in}{7.700000in}}%
\pgfusepath{clip}%
\pgfsetbuttcap%
\pgfsetroundjoin%
\definecolor{currentfill}{rgb}{0.631373,0.788235,0.956863}%
\pgfsetfillcolor{currentfill}%
\pgfsetlinewidth{0.481800pt}%
\definecolor{currentstroke}{rgb}{1.000000,1.000000,1.000000}%
\pgfsetstrokecolor{currentstroke}%
\pgfsetdash{}{0pt}%
\pgfpathmoveto{\pgfqpoint{3.012102in}{4.611721in}}%
\pgfpathcurveto{\pgfqpoint{3.023152in}{4.611721in}}{\pgfqpoint{3.033751in}{4.616111in}}{\pgfqpoint{3.041565in}{4.623925in}}%
\pgfpathcurveto{\pgfqpoint{3.049379in}{4.631739in}}{\pgfqpoint{3.053769in}{4.642338in}}{\pgfqpoint{3.053769in}{4.653388in}}%
\pgfpathcurveto{\pgfqpoint{3.053769in}{4.664438in}}{\pgfqpoint{3.049379in}{4.675037in}}{\pgfqpoint{3.041565in}{4.682850in}}%
\pgfpathcurveto{\pgfqpoint{3.033751in}{4.690664in}}{\pgfqpoint{3.023152in}{4.695054in}}{\pgfqpoint{3.012102in}{4.695054in}}%
\pgfpathcurveto{\pgfqpoint{3.001052in}{4.695054in}}{\pgfqpoint{2.990453in}{4.690664in}}{\pgfqpoint{2.982639in}{4.682850in}}%
\pgfpathcurveto{\pgfqpoint{2.974826in}{4.675037in}}{\pgfqpoint{2.970436in}{4.664438in}}{\pgfqpoint{2.970436in}{4.653388in}}%
\pgfpathcurveto{\pgfqpoint{2.970436in}{4.642338in}}{\pgfqpoint{2.974826in}{4.631739in}}{\pgfqpoint{2.982639in}{4.623925in}}%
\pgfpathcurveto{\pgfqpoint{2.990453in}{4.616111in}}{\pgfqpoint{3.001052in}{4.611721in}}{\pgfqpoint{3.012102in}{4.611721in}}%
\pgfpathclose%
\pgfusepath{stroke,fill}%
\end{pgfscope}%
\begin{pgfscope}%
\pgfpathrectangle{\pgfqpoint{0.481978in}{0.331635in}}{\pgfqpoint{9.300000in}{7.700000in}}%
\pgfusepath{clip}%
\pgfsetbuttcap%
\pgfsetroundjoin%
\definecolor{currentfill}{rgb}{0.631373,0.788235,0.956863}%
\pgfsetfillcolor{currentfill}%
\pgfsetlinewidth{0.481800pt}%
\definecolor{currentstroke}{rgb}{1.000000,1.000000,1.000000}%
\pgfsetstrokecolor{currentstroke}%
\pgfsetdash{}{0pt}%
\pgfpathmoveto{\pgfqpoint{6.293539in}{3.584833in}}%
\pgfpathcurveto{\pgfqpoint{6.304589in}{3.584833in}}{\pgfqpoint{6.315188in}{3.589224in}}{\pgfqpoint{6.323001in}{3.597037in}}%
\pgfpathcurveto{\pgfqpoint{6.330815in}{3.604851in}}{\pgfqpoint{6.335205in}{3.615450in}}{\pgfqpoint{6.335205in}{3.626500in}}%
\pgfpathcurveto{\pgfqpoint{6.335205in}{3.637550in}}{\pgfqpoint{6.330815in}{3.648149in}}{\pgfqpoint{6.323001in}{3.655963in}}%
\pgfpathcurveto{\pgfqpoint{6.315188in}{3.663777in}}{\pgfqpoint{6.304589in}{3.668167in}}{\pgfqpoint{6.293539in}{3.668167in}}%
\pgfpathcurveto{\pgfqpoint{6.282489in}{3.668167in}}{\pgfqpoint{6.271890in}{3.663777in}}{\pgfqpoint{6.264076in}{3.655963in}}%
\pgfpathcurveto{\pgfqpoint{6.256262in}{3.648149in}}{\pgfqpoint{6.251872in}{3.637550in}}{\pgfqpoint{6.251872in}{3.626500in}}%
\pgfpathcurveto{\pgfqpoint{6.251872in}{3.615450in}}{\pgfqpoint{6.256262in}{3.604851in}}{\pgfqpoint{6.264076in}{3.597037in}}%
\pgfpathcurveto{\pgfqpoint{6.271890in}{3.589224in}}{\pgfqpoint{6.282489in}{3.584833in}}{\pgfqpoint{6.293539in}{3.584833in}}%
\pgfpathclose%
\pgfusepath{stroke,fill}%
\end{pgfscope}%
\begin{pgfscope}%
\pgfpathrectangle{\pgfqpoint{0.481978in}{0.331635in}}{\pgfqpoint{9.300000in}{7.700000in}}%
\pgfusepath{clip}%
\pgfsetbuttcap%
\pgfsetroundjoin%
\definecolor{currentfill}{rgb}{0.631373,0.788235,0.956863}%
\pgfsetfillcolor{currentfill}%
\pgfsetlinewidth{0.481800pt}%
\definecolor{currentstroke}{rgb}{1.000000,1.000000,1.000000}%
\pgfsetstrokecolor{currentstroke}%
\pgfsetdash{}{0pt}%
\pgfpathmoveto{\pgfqpoint{2.026106in}{1.607439in}}%
\pgfpathcurveto{\pgfqpoint{2.037156in}{1.607439in}}{\pgfqpoint{2.047755in}{1.611829in}}{\pgfqpoint{2.055569in}{1.619643in}}%
\pgfpathcurveto{\pgfqpoint{2.063382in}{1.627456in}}{\pgfqpoint{2.067772in}{1.638056in}}{\pgfqpoint{2.067772in}{1.649106in}}%
\pgfpathcurveto{\pgfqpoint{2.067772in}{1.660156in}}{\pgfqpoint{2.063382in}{1.670755in}}{\pgfqpoint{2.055569in}{1.678568in}}%
\pgfpathcurveto{\pgfqpoint{2.047755in}{1.686382in}}{\pgfqpoint{2.037156in}{1.690772in}}{\pgfqpoint{2.026106in}{1.690772in}}%
\pgfpathcurveto{\pgfqpoint{2.015056in}{1.690772in}}{\pgfqpoint{2.004457in}{1.686382in}}{\pgfqpoint{1.996643in}{1.678568in}}%
\pgfpathcurveto{\pgfqpoint{1.988829in}{1.670755in}}{\pgfqpoint{1.984439in}{1.660156in}}{\pgfqpoint{1.984439in}{1.649106in}}%
\pgfpathcurveto{\pgfqpoint{1.984439in}{1.638056in}}{\pgfqpoint{1.988829in}{1.627456in}}{\pgfqpoint{1.996643in}{1.619643in}}%
\pgfpathcurveto{\pgfqpoint{2.004457in}{1.611829in}}{\pgfqpoint{2.015056in}{1.607439in}}{\pgfqpoint{2.026106in}{1.607439in}}%
\pgfpathclose%
\pgfusepath{stroke,fill}%
\end{pgfscope}%
\begin{pgfscope}%
\pgfpathrectangle{\pgfqpoint{0.481978in}{0.331635in}}{\pgfqpoint{9.300000in}{7.700000in}}%
\pgfusepath{clip}%
\pgfsetbuttcap%
\pgfsetroundjoin%
\definecolor{currentfill}{rgb}{1.000000,0.705882,0.509804}%
\pgfsetfillcolor{currentfill}%
\pgfsetlinewidth{0.481800pt}%
\definecolor{currentstroke}{rgb}{1.000000,1.000000,1.000000}%
\pgfsetstrokecolor{currentstroke}%
\pgfsetdash{}{0pt}%
\pgfpathmoveto{\pgfqpoint{4.850856in}{2.813025in}}%
\pgfpathcurveto{\pgfqpoint{4.861906in}{2.813025in}}{\pgfqpoint{4.872505in}{2.817415in}}{\pgfqpoint{4.880319in}{2.825229in}}%
\pgfpathcurveto{\pgfqpoint{4.888132in}{2.833043in}}{\pgfqpoint{4.892522in}{2.843642in}}{\pgfqpoint{4.892522in}{2.854692in}}%
\pgfpathcurveto{\pgfqpoint{4.892522in}{2.865742in}}{\pgfqpoint{4.888132in}{2.876341in}}{\pgfqpoint{4.880319in}{2.884155in}}%
\pgfpathcurveto{\pgfqpoint{4.872505in}{2.891968in}}{\pgfqpoint{4.861906in}{2.896358in}}{\pgfqpoint{4.850856in}{2.896358in}}%
\pgfpathcurveto{\pgfqpoint{4.839806in}{2.896358in}}{\pgfqpoint{4.829207in}{2.891968in}}{\pgfqpoint{4.821393in}{2.884155in}}%
\pgfpathcurveto{\pgfqpoint{4.813579in}{2.876341in}}{\pgfqpoint{4.809189in}{2.865742in}}{\pgfqpoint{4.809189in}{2.854692in}}%
\pgfpathcurveto{\pgfqpoint{4.809189in}{2.843642in}}{\pgfqpoint{4.813579in}{2.833043in}}{\pgfqpoint{4.821393in}{2.825229in}}%
\pgfpathcurveto{\pgfqpoint{4.829207in}{2.817415in}}{\pgfqpoint{4.839806in}{2.813025in}}{\pgfqpoint{4.850856in}{2.813025in}}%
\pgfpathclose%
\pgfusepath{stroke,fill}%
\end{pgfscope}%
\begin{pgfscope}%
\pgfpathrectangle{\pgfqpoint{0.481978in}{0.331635in}}{\pgfqpoint{9.300000in}{7.700000in}}%
\pgfusepath{clip}%
\pgfsetbuttcap%
\pgfsetroundjoin%
\definecolor{currentfill}{rgb}{1.000000,0.705882,0.509804}%
\pgfsetfillcolor{currentfill}%
\pgfsetlinewidth{0.481800pt}%
\definecolor{currentstroke}{rgb}{1.000000,1.000000,1.000000}%
\pgfsetstrokecolor{currentstroke}%
\pgfsetdash{}{0pt}%
\pgfpathmoveto{\pgfqpoint{2.916721in}{3.078678in}}%
\pgfpathcurveto{\pgfqpoint{2.927771in}{3.078678in}}{\pgfqpoint{2.938370in}{3.083069in}}{\pgfqpoint{2.946183in}{3.090882in}}%
\pgfpathcurveto{\pgfqpoint{2.953997in}{3.098696in}}{\pgfqpoint{2.958387in}{3.109295in}}{\pgfqpoint{2.958387in}{3.120345in}}%
\pgfpathcurveto{\pgfqpoint{2.958387in}{3.131395in}}{\pgfqpoint{2.953997in}{3.141994in}}{\pgfqpoint{2.946183in}{3.149808in}}%
\pgfpathcurveto{\pgfqpoint{2.938370in}{3.157621in}}{\pgfqpoint{2.927771in}{3.162012in}}{\pgfqpoint{2.916721in}{3.162012in}}%
\pgfpathcurveto{\pgfqpoint{2.905671in}{3.162012in}}{\pgfqpoint{2.895072in}{3.157621in}}{\pgfqpoint{2.887258in}{3.149808in}}%
\pgfpathcurveto{\pgfqpoint{2.879444in}{3.141994in}}{\pgfqpoint{2.875054in}{3.131395in}}{\pgfqpoint{2.875054in}{3.120345in}}%
\pgfpathcurveto{\pgfqpoint{2.875054in}{3.109295in}}{\pgfqpoint{2.879444in}{3.098696in}}{\pgfqpoint{2.887258in}{3.090882in}}%
\pgfpathcurveto{\pgfqpoint{2.895072in}{3.083069in}}{\pgfqpoint{2.905671in}{3.078678in}}{\pgfqpoint{2.916721in}{3.078678in}}%
\pgfpathclose%
\pgfusepath{stroke,fill}%
\end{pgfscope}%
\begin{pgfscope}%
\pgfpathrectangle{\pgfqpoint{0.481978in}{0.331635in}}{\pgfqpoint{9.300000in}{7.700000in}}%
\pgfusepath{clip}%
\pgfsetbuttcap%
\pgfsetroundjoin%
\definecolor{currentfill}{rgb}{1.000000,0.705882,0.509804}%
\pgfsetfillcolor{currentfill}%
\pgfsetlinewidth{0.481800pt}%
\definecolor{currentstroke}{rgb}{1.000000,1.000000,1.000000}%
\pgfsetstrokecolor{currentstroke}%
\pgfsetdash{}{0pt}%
\pgfpathmoveto{\pgfqpoint{3.594827in}{5.661176in}}%
\pgfpathcurveto{\pgfqpoint{3.605878in}{5.661176in}}{\pgfqpoint{3.616477in}{5.665566in}}{\pgfqpoint{3.624290in}{5.673380in}}%
\pgfpathcurveto{\pgfqpoint{3.632104in}{5.681193in}}{\pgfqpoint{3.636494in}{5.691792in}}{\pgfqpoint{3.636494in}{5.702842in}}%
\pgfpathcurveto{\pgfqpoint{3.636494in}{5.713892in}}{\pgfqpoint{3.632104in}{5.724491in}}{\pgfqpoint{3.624290in}{5.732305in}}%
\pgfpathcurveto{\pgfqpoint{3.616477in}{5.740119in}}{\pgfqpoint{3.605878in}{5.744509in}}{\pgfqpoint{3.594827in}{5.744509in}}%
\pgfpathcurveto{\pgfqpoint{3.583777in}{5.744509in}}{\pgfqpoint{3.573178in}{5.740119in}}{\pgfqpoint{3.565365in}{5.732305in}}%
\pgfpathcurveto{\pgfqpoint{3.557551in}{5.724491in}}{\pgfqpoint{3.553161in}{5.713892in}}{\pgfqpoint{3.553161in}{5.702842in}}%
\pgfpathcurveto{\pgfqpoint{3.553161in}{5.691792in}}{\pgfqpoint{3.557551in}{5.681193in}}{\pgfqpoint{3.565365in}{5.673380in}}%
\pgfpathcurveto{\pgfqpoint{3.573178in}{5.665566in}}{\pgfqpoint{3.583777in}{5.661176in}}{\pgfqpoint{3.594827in}{5.661176in}}%
\pgfpathclose%
\pgfusepath{stroke,fill}%
\end{pgfscope}%
\begin{pgfscope}%
\pgfpathrectangle{\pgfqpoint{0.481978in}{0.331635in}}{\pgfqpoint{9.300000in}{7.700000in}}%
\pgfusepath{clip}%
\pgfsetbuttcap%
\pgfsetroundjoin%
\definecolor{currentfill}{rgb}{1.000000,0.705882,0.509804}%
\pgfsetfillcolor{currentfill}%
\pgfsetlinewidth{0.481800pt}%
\definecolor{currentstroke}{rgb}{1.000000,1.000000,1.000000}%
\pgfsetstrokecolor{currentstroke}%
\pgfsetdash{}{0pt}%
\pgfpathmoveto{\pgfqpoint{3.147191in}{3.768081in}}%
\pgfpathcurveto{\pgfqpoint{3.158241in}{3.768081in}}{\pgfqpoint{3.168840in}{3.772471in}}{\pgfqpoint{3.176654in}{3.780285in}}%
\pgfpathcurveto{\pgfqpoint{3.184467in}{3.788099in}}{\pgfqpoint{3.188857in}{3.798698in}}{\pgfqpoint{3.188857in}{3.809748in}}%
\pgfpathcurveto{\pgfqpoint{3.188857in}{3.820798in}}{\pgfqpoint{3.184467in}{3.831397in}}{\pgfqpoint{3.176654in}{3.839211in}}%
\pgfpathcurveto{\pgfqpoint{3.168840in}{3.847024in}}{\pgfqpoint{3.158241in}{3.851415in}}{\pgfqpoint{3.147191in}{3.851415in}}%
\pgfpathcurveto{\pgfqpoint{3.136141in}{3.851415in}}{\pgfqpoint{3.125542in}{3.847024in}}{\pgfqpoint{3.117728in}{3.839211in}}%
\pgfpathcurveto{\pgfqpoint{3.109914in}{3.831397in}}{\pgfqpoint{3.105524in}{3.820798in}}{\pgfqpoint{3.105524in}{3.809748in}}%
\pgfpathcurveto{\pgfqpoint{3.105524in}{3.798698in}}{\pgfqpoint{3.109914in}{3.788099in}}{\pgfqpoint{3.117728in}{3.780285in}}%
\pgfpathcurveto{\pgfqpoint{3.125542in}{3.772471in}}{\pgfqpoint{3.136141in}{3.768081in}}{\pgfqpoint{3.147191in}{3.768081in}}%
\pgfpathclose%
\pgfusepath{stroke,fill}%
\end{pgfscope}%
\begin{pgfscope}%
\pgfpathrectangle{\pgfqpoint{0.481978in}{0.331635in}}{\pgfqpoint{9.300000in}{7.700000in}}%
\pgfusepath{clip}%
\pgfsetbuttcap%
\pgfsetroundjoin%
\definecolor{currentfill}{rgb}{1.000000,0.705882,0.509804}%
\pgfsetfillcolor{currentfill}%
\pgfsetlinewidth{0.481800pt}%
\definecolor{currentstroke}{rgb}{1.000000,1.000000,1.000000}%
\pgfsetstrokecolor{currentstroke}%
\pgfsetdash{}{0pt}%
\pgfpathmoveto{\pgfqpoint{3.148403in}{2.273646in}}%
\pgfpathcurveto{\pgfqpoint{3.159453in}{2.273646in}}{\pgfqpoint{3.170052in}{2.278036in}}{\pgfqpoint{3.177866in}{2.285850in}}%
\pgfpathcurveto{\pgfqpoint{3.185679in}{2.293663in}}{\pgfqpoint{3.190070in}{2.304262in}}{\pgfqpoint{3.190070in}{2.315312in}}%
\pgfpathcurveto{\pgfqpoint{3.190070in}{2.326363in}}{\pgfqpoint{3.185679in}{2.336962in}}{\pgfqpoint{3.177866in}{2.344775in}}%
\pgfpathcurveto{\pgfqpoint{3.170052in}{2.352589in}}{\pgfqpoint{3.159453in}{2.356979in}}{\pgfqpoint{3.148403in}{2.356979in}}%
\pgfpathcurveto{\pgfqpoint{3.137353in}{2.356979in}}{\pgfqpoint{3.126754in}{2.352589in}}{\pgfqpoint{3.118940in}{2.344775in}}%
\pgfpathcurveto{\pgfqpoint{3.111127in}{2.336962in}}{\pgfqpoint{3.106736in}{2.326363in}}{\pgfqpoint{3.106736in}{2.315312in}}%
\pgfpathcurveto{\pgfqpoint{3.106736in}{2.304262in}}{\pgfqpoint{3.111127in}{2.293663in}}{\pgfqpoint{3.118940in}{2.285850in}}%
\pgfpathcurveto{\pgfqpoint{3.126754in}{2.278036in}}{\pgfqpoint{3.137353in}{2.273646in}}{\pgfqpoint{3.148403in}{2.273646in}}%
\pgfpathclose%
\pgfusepath{stroke,fill}%
\end{pgfscope}%
\begin{pgfscope}%
\pgfpathrectangle{\pgfqpoint{0.481978in}{0.331635in}}{\pgfqpoint{9.300000in}{7.700000in}}%
\pgfusepath{clip}%
\pgfsetbuttcap%
\pgfsetroundjoin%
\definecolor{currentfill}{rgb}{1.000000,0.705882,0.509804}%
\pgfsetfillcolor{currentfill}%
\pgfsetlinewidth{0.481800pt}%
\definecolor{currentstroke}{rgb}{1.000000,1.000000,1.000000}%
\pgfsetstrokecolor{currentstroke}%
\pgfsetdash{}{0pt}%
\pgfpathmoveto{\pgfqpoint{3.951193in}{0.913087in}}%
\pgfpathcurveto{\pgfqpoint{3.962243in}{0.913087in}}{\pgfqpoint{3.972842in}{0.917477in}}{\pgfqpoint{3.980656in}{0.925291in}}%
\pgfpathcurveto{\pgfqpoint{3.988470in}{0.933105in}}{\pgfqpoint{3.992860in}{0.943704in}}{\pgfqpoint{3.992860in}{0.954754in}}%
\pgfpathcurveto{\pgfqpoint{3.992860in}{0.965804in}}{\pgfqpoint{3.988470in}{0.976403in}}{\pgfqpoint{3.980656in}{0.984217in}}%
\pgfpathcurveto{\pgfqpoint{3.972842in}{0.992030in}}{\pgfqpoint{3.962243in}{0.996421in}}{\pgfqpoint{3.951193in}{0.996421in}}%
\pgfpathcurveto{\pgfqpoint{3.940143in}{0.996421in}}{\pgfqpoint{3.929544in}{0.992030in}}{\pgfqpoint{3.921731in}{0.984217in}}%
\pgfpathcurveto{\pgfqpoint{3.913917in}{0.976403in}}{\pgfqpoint{3.909527in}{0.965804in}}{\pgfqpoint{3.909527in}{0.954754in}}%
\pgfpathcurveto{\pgfqpoint{3.909527in}{0.943704in}}{\pgfqpoint{3.913917in}{0.933105in}}{\pgfqpoint{3.921731in}{0.925291in}}%
\pgfpathcurveto{\pgfqpoint{3.929544in}{0.917477in}}{\pgfqpoint{3.940143in}{0.913087in}}{\pgfqpoint{3.951193in}{0.913087in}}%
\pgfpathclose%
\pgfusepath{stroke,fill}%
\end{pgfscope}%
\begin{pgfscope}%
\pgfpathrectangle{\pgfqpoint{0.481978in}{0.331635in}}{\pgfqpoint{9.300000in}{7.700000in}}%
\pgfusepath{clip}%
\pgfsetbuttcap%
\pgfsetroundjoin%
\definecolor{currentfill}{rgb}{1.000000,0.705882,0.509804}%
\pgfsetfillcolor{currentfill}%
\pgfsetlinewidth{0.481800pt}%
\definecolor{currentstroke}{rgb}{1.000000,1.000000,1.000000}%
\pgfsetstrokecolor{currentstroke}%
\pgfsetdash{}{0pt}%
\pgfpathmoveto{\pgfqpoint{5.646419in}{4.278341in}}%
\pgfpathcurveto{\pgfqpoint{5.657469in}{4.278341in}}{\pgfqpoint{5.668068in}{4.282731in}}{\pgfqpoint{5.675881in}{4.290545in}}%
\pgfpathcurveto{\pgfqpoint{5.683695in}{4.298359in}}{\pgfqpoint{5.688085in}{4.308958in}}{\pgfqpoint{5.688085in}{4.320008in}}%
\pgfpathcurveto{\pgfqpoint{5.688085in}{4.331058in}}{\pgfqpoint{5.683695in}{4.341657in}}{\pgfqpoint{5.675881in}{4.349471in}}%
\pgfpathcurveto{\pgfqpoint{5.668068in}{4.357284in}}{\pgfqpoint{5.657469in}{4.361675in}}{\pgfqpoint{5.646419in}{4.361675in}}%
\pgfpathcurveto{\pgfqpoint{5.635369in}{4.361675in}}{\pgfqpoint{5.624769in}{4.357284in}}{\pgfqpoint{5.616956in}{4.349471in}}%
\pgfpathcurveto{\pgfqpoint{5.609142in}{4.341657in}}{\pgfqpoint{5.604752in}{4.331058in}}{\pgfqpoint{5.604752in}{4.320008in}}%
\pgfpathcurveto{\pgfqpoint{5.604752in}{4.308958in}}{\pgfqpoint{5.609142in}{4.298359in}}{\pgfqpoint{5.616956in}{4.290545in}}%
\pgfpathcurveto{\pgfqpoint{5.624769in}{4.282731in}}{\pgfqpoint{5.635369in}{4.278341in}}{\pgfqpoint{5.646419in}{4.278341in}}%
\pgfpathclose%
\pgfusepath{stroke,fill}%
\end{pgfscope}%
\begin{pgfscope}%
\pgfpathrectangle{\pgfqpoint{0.481978in}{0.331635in}}{\pgfqpoint{9.300000in}{7.700000in}}%
\pgfusepath{clip}%
\pgfsetbuttcap%
\pgfsetroundjoin%
\definecolor{currentfill}{rgb}{1.000000,0.705882,0.509804}%
\pgfsetfillcolor{currentfill}%
\pgfsetlinewidth{0.481800pt}%
\definecolor{currentstroke}{rgb}{1.000000,1.000000,1.000000}%
\pgfsetstrokecolor{currentstroke}%
\pgfsetdash{}{0pt}%
\pgfpathmoveto{\pgfqpoint{2.847205in}{4.307014in}}%
\pgfpathcurveto{\pgfqpoint{2.858255in}{4.307014in}}{\pgfqpoint{2.868854in}{4.311405in}}{\pgfqpoint{2.876668in}{4.319218in}}%
\pgfpathcurveto{\pgfqpoint{2.884481in}{4.327032in}}{\pgfqpoint{2.888872in}{4.337631in}}{\pgfqpoint{2.888872in}{4.348681in}}%
\pgfpathcurveto{\pgfqpoint{2.888872in}{4.359731in}}{\pgfqpoint{2.884481in}{4.370330in}}{\pgfqpoint{2.876668in}{4.378144in}}%
\pgfpathcurveto{\pgfqpoint{2.868854in}{4.385957in}}{\pgfqpoint{2.858255in}{4.390348in}}{\pgfqpoint{2.847205in}{4.390348in}}%
\pgfpathcurveto{\pgfqpoint{2.836155in}{4.390348in}}{\pgfqpoint{2.825556in}{4.385957in}}{\pgfqpoint{2.817742in}{4.378144in}}%
\pgfpathcurveto{\pgfqpoint{2.809928in}{4.370330in}}{\pgfqpoint{2.805538in}{4.359731in}}{\pgfqpoint{2.805538in}{4.348681in}}%
\pgfpathcurveto{\pgfqpoint{2.805538in}{4.337631in}}{\pgfqpoint{2.809928in}{4.327032in}}{\pgfqpoint{2.817742in}{4.319218in}}%
\pgfpathcurveto{\pgfqpoint{2.825556in}{4.311405in}}{\pgfqpoint{2.836155in}{4.307014in}}{\pgfqpoint{2.847205in}{4.307014in}}%
\pgfpathclose%
\pgfusepath{stroke,fill}%
\end{pgfscope}%
\begin{pgfscope}%
\pgfpathrectangle{\pgfqpoint{0.481978in}{0.331635in}}{\pgfqpoint{9.300000in}{7.700000in}}%
\pgfusepath{clip}%
\pgfsetbuttcap%
\pgfsetroundjoin%
\definecolor{currentfill}{rgb}{1.000000,0.705882,0.509804}%
\pgfsetfillcolor{currentfill}%
\pgfsetlinewidth{0.481800pt}%
\definecolor{currentstroke}{rgb}{1.000000,1.000000,1.000000}%
\pgfsetstrokecolor{currentstroke}%
\pgfsetdash{}{0pt}%
\pgfpathmoveto{\pgfqpoint{3.926575in}{2.335876in}}%
\pgfpathcurveto{\pgfqpoint{3.937625in}{2.335876in}}{\pgfqpoint{3.948224in}{2.340266in}}{\pgfqpoint{3.956038in}{2.348080in}}%
\pgfpathcurveto{\pgfqpoint{3.963852in}{2.355894in}}{\pgfqpoint{3.968242in}{2.366493in}}{\pgfqpoint{3.968242in}{2.377543in}}%
\pgfpathcurveto{\pgfqpoint{3.968242in}{2.388593in}}{\pgfqpoint{3.963852in}{2.399192in}}{\pgfqpoint{3.956038in}{2.407006in}}%
\pgfpathcurveto{\pgfqpoint{3.948224in}{2.414819in}}{\pgfqpoint{3.937625in}{2.419209in}}{\pgfqpoint{3.926575in}{2.419209in}}%
\pgfpathcurveto{\pgfqpoint{3.915525in}{2.419209in}}{\pgfqpoint{3.904926in}{2.414819in}}{\pgfqpoint{3.897112in}{2.407006in}}%
\pgfpathcurveto{\pgfqpoint{3.889299in}{2.399192in}}{\pgfqpoint{3.884908in}{2.388593in}}{\pgfqpoint{3.884908in}{2.377543in}}%
\pgfpathcurveto{\pgfqpoint{3.884908in}{2.366493in}}{\pgfqpoint{3.889299in}{2.355894in}}{\pgfqpoint{3.897112in}{2.348080in}}%
\pgfpathcurveto{\pgfqpoint{3.904926in}{2.340266in}}{\pgfqpoint{3.915525in}{2.335876in}}{\pgfqpoint{3.926575in}{2.335876in}}%
\pgfpathclose%
\pgfusepath{stroke,fill}%
\end{pgfscope}%
\begin{pgfscope}%
\pgfpathrectangle{\pgfqpoint{0.481978in}{0.331635in}}{\pgfqpoint{9.300000in}{7.700000in}}%
\pgfusepath{clip}%
\pgfsetbuttcap%
\pgfsetroundjoin%
\definecolor{currentfill}{rgb}{1.000000,0.705882,0.509804}%
\pgfsetfillcolor{currentfill}%
\pgfsetlinewidth{0.481800pt}%
\definecolor{currentstroke}{rgb}{1.000000,1.000000,1.000000}%
\pgfsetstrokecolor{currentstroke}%
\pgfsetdash{}{0pt}%
\pgfpathmoveto{\pgfqpoint{2.406478in}{5.862103in}}%
\pgfpathcurveto{\pgfqpoint{2.417529in}{5.862103in}}{\pgfqpoint{2.428128in}{5.866493in}}{\pgfqpoint{2.435941in}{5.874307in}}%
\pgfpathcurveto{\pgfqpoint{2.443755in}{5.882121in}}{\pgfqpoint{2.448145in}{5.892720in}}{\pgfqpoint{2.448145in}{5.903770in}}%
\pgfpathcurveto{\pgfqpoint{2.448145in}{5.914820in}}{\pgfqpoint{2.443755in}{5.925419in}}{\pgfqpoint{2.435941in}{5.933233in}}%
\pgfpathcurveto{\pgfqpoint{2.428128in}{5.941046in}}{\pgfqpoint{2.417529in}{5.945436in}}{\pgfqpoint{2.406478in}{5.945436in}}%
\pgfpathcurveto{\pgfqpoint{2.395428in}{5.945436in}}{\pgfqpoint{2.384829in}{5.941046in}}{\pgfqpoint{2.377016in}{5.933233in}}%
\pgfpathcurveto{\pgfqpoint{2.369202in}{5.925419in}}{\pgfqpoint{2.364812in}{5.914820in}}{\pgfqpoint{2.364812in}{5.903770in}}%
\pgfpathcurveto{\pgfqpoint{2.364812in}{5.892720in}}{\pgfqpoint{2.369202in}{5.882121in}}{\pgfqpoint{2.377016in}{5.874307in}}%
\pgfpathcurveto{\pgfqpoint{2.384829in}{5.866493in}}{\pgfqpoint{2.395428in}{5.862103in}}{\pgfqpoint{2.406478in}{5.862103in}}%
\pgfpathclose%
\pgfusepath{stroke,fill}%
\end{pgfscope}%
\begin{pgfscope}%
\pgfpathrectangle{\pgfqpoint{0.481978in}{0.331635in}}{\pgfqpoint{9.300000in}{7.700000in}}%
\pgfusepath{clip}%
\pgfsetbuttcap%
\pgfsetroundjoin%
\definecolor{currentfill}{rgb}{1.000000,0.705882,0.509804}%
\pgfsetfillcolor{currentfill}%
\pgfsetlinewidth{0.481800pt}%
\definecolor{currentstroke}{rgb}{1.000000,1.000000,1.000000}%
\pgfsetstrokecolor{currentstroke}%
\pgfsetdash{}{0pt}%
\pgfpathmoveto{\pgfqpoint{4.065253in}{2.330807in}}%
\pgfpathcurveto{\pgfqpoint{4.076304in}{2.330807in}}{\pgfqpoint{4.086903in}{2.335197in}}{\pgfqpoint{4.094716in}{2.343011in}}%
\pgfpathcurveto{\pgfqpoint{4.102530in}{2.350824in}}{\pgfqpoint{4.106920in}{2.361423in}}{\pgfqpoint{4.106920in}{2.372474in}}%
\pgfpathcurveto{\pgfqpoint{4.106920in}{2.383524in}}{\pgfqpoint{4.102530in}{2.394123in}}{\pgfqpoint{4.094716in}{2.401936in}}%
\pgfpathcurveto{\pgfqpoint{4.086903in}{2.409750in}}{\pgfqpoint{4.076304in}{2.414140in}}{\pgfqpoint{4.065253in}{2.414140in}}%
\pgfpathcurveto{\pgfqpoint{4.054203in}{2.414140in}}{\pgfqpoint{4.043604in}{2.409750in}}{\pgfqpoint{4.035791in}{2.401936in}}%
\pgfpathcurveto{\pgfqpoint{4.027977in}{2.394123in}}{\pgfqpoint{4.023587in}{2.383524in}}{\pgfqpoint{4.023587in}{2.372474in}}%
\pgfpathcurveto{\pgfqpoint{4.023587in}{2.361423in}}{\pgfqpoint{4.027977in}{2.350824in}}{\pgfqpoint{4.035791in}{2.343011in}}%
\pgfpathcurveto{\pgfqpoint{4.043604in}{2.335197in}}{\pgfqpoint{4.054203in}{2.330807in}}{\pgfqpoint{4.065253in}{2.330807in}}%
\pgfpathclose%
\pgfusepath{stroke,fill}%
\end{pgfscope}%
\begin{pgfscope}%
\pgfpathrectangle{\pgfqpoint{0.481978in}{0.331635in}}{\pgfqpoint{9.300000in}{7.700000in}}%
\pgfusepath{clip}%
\pgfsetbuttcap%
\pgfsetroundjoin%
\definecolor{currentfill}{rgb}{1.000000,0.705882,0.509804}%
\pgfsetfillcolor{currentfill}%
\pgfsetlinewidth{0.481800pt}%
\definecolor{currentstroke}{rgb}{1.000000,1.000000,1.000000}%
\pgfsetstrokecolor{currentstroke}%
\pgfsetdash{}{0pt}%
\pgfpathmoveto{\pgfqpoint{4.470303in}{5.164449in}}%
\pgfpathcurveto{\pgfqpoint{4.481353in}{5.164449in}}{\pgfqpoint{4.491952in}{5.168840in}}{\pgfqpoint{4.499765in}{5.176653in}}%
\pgfpathcurveto{\pgfqpoint{4.507579in}{5.184467in}}{\pgfqpoint{4.511969in}{5.195066in}}{\pgfqpoint{4.511969in}{5.206116in}}%
\pgfpathcurveto{\pgfqpoint{4.511969in}{5.217166in}}{\pgfqpoint{4.507579in}{5.227765in}}{\pgfqpoint{4.499765in}{5.235579in}}%
\pgfpathcurveto{\pgfqpoint{4.491952in}{5.243392in}}{\pgfqpoint{4.481353in}{5.247783in}}{\pgfqpoint{4.470303in}{5.247783in}}%
\pgfpathcurveto{\pgfqpoint{4.459253in}{5.247783in}}{\pgfqpoint{4.448653in}{5.243392in}}{\pgfqpoint{4.440840in}{5.235579in}}%
\pgfpathcurveto{\pgfqpoint{4.433026in}{5.227765in}}{\pgfqpoint{4.428636in}{5.217166in}}{\pgfqpoint{4.428636in}{5.206116in}}%
\pgfpathcurveto{\pgfqpoint{4.428636in}{5.195066in}}{\pgfqpoint{4.433026in}{5.184467in}}{\pgfqpoint{4.440840in}{5.176653in}}%
\pgfpathcurveto{\pgfqpoint{4.448653in}{5.168840in}}{\pgfqpoint{4.459253in}{5.164449in}}{\pgfqpoint{4.470303in}{5.164449in}}%
\pgfpathclose%
\pgfusepath{stroke,fill}%
\end{pgfscope}%
\begin{pgfscope}%
\pgfpathrectangle{\pgfqpoint{0.481978in}{0.331635in}}{\pgfqpoint{9.300000in}{7.700000in}}%
\pgfusepath{clip}%
\pgfsetbuttcap%
\pgfsetroundjoin%
\definecolor{currentfill}{rgb}{1.000000,0.705882,0.509804}%
\pgfsetfillcolor{currentfill}%
\pgfsetlinewidth{0.481800pt}%
\definecolor{currentstroke}{rgb}{1.000000,1.000000,1.000000}%
\pgfsetstrokecolor{currentstroke}%
\pgfsetdash{}{0pt}%
\pgfpathmoveto{\pgfqpoint{2.649493in}{4.061155in}}%
\pgfpathcurveto{\pgfqpoint{2.660543in}{4.061155in}}{\pgfqpoint{2.671142in}{4.065545in}}{\pgfqpoint{2.678956in}{4.073359in}}%
\pgfpathcurveto{\pgfqpoint{2.686769in}{4.081172in}}{\pgfqpoint{2.691160in}{4.091771in}}{\pgfqpoint{2.691160in}{4.102821in}}%
\pgfpathcurveto{\pgfqpoint{2.691160in}{4.113871in}}{\pgfqpoint{2.686769in}{4.124470in}}{\pgfqpoint{2.678956in}{4.132284in}}%
\pgfpathcurveto{\pgfqpoint{2.671142in}{4.140098in}}{\pgfqpoint{2.660543in}{4.144488in}}{\pgfqpoint{2.649493in}{4.144488in}}%
\pgfpathcurveto{\pgfqpoint{2.638443in}{4.144488in}}{\pgfqpoint{2.627844in}{4.140098in}}{\pgfqpoint{2.620030in}{4.132284in}}%
\pgfpathcurveto{\pgfqpoint{2.612216in}{4.124470in}}{\pgfqpoint{2.607826in}{4.113871in}}{\pgfqpoint{2.607826in}{4.102821in}}%
\pgfpathcurveto{\pgfqpoint{2.607826in}{4.091771in}}{\pgfqpoint{2.612216in}{4.081172in}}{\pgfqpoint{2.620030in}{4.073359in}}%
\pgfpathcurveto{\pgfqpoint{2.627844in}{4.065545in}}{\pgfqpoint{2.638443in}{4.061155in}}{\pgfqpoint{2.649493in}{4.061155in}}%
\pgfpathclose%
\pgfusepath{stroke,fill}%
\end{pgfscope}%
\begin{pgfscope}%
\pgfpathrectangle{\pgfqpoint{0.481978in}{0.331635in}}{\pgfqpoint{9.300000in}{7.700000in}}%
\pgfusepath{clip}%
\pgfsetbuttcap%
\pgfsetroundjoin%
\definecolor{currentfill}{rgb}{1.000000,0.705882,0.509804}%
\pgfsetfillcolor{currentfill}%
\pgfsetlinewidth{0.481800pt}%
\definecolor{currentstroke}{rgb}{1.000000,1.000000,1.000000}%
\pgfsetstrokecolor{currentstroke}%
\pgfsetdash{}{0pt}%
\pgfpathmoveto{\pgfqpoint{2.136818in}{3.273180in}}%
\pgfpathcurveto{\pgfqpoint{2.147868in}{3.273180in}}{\pgfqpoint{2.158467in}{3.277570in}}{\pgfqpoint{2.166281in}{3.285383in}}%
\pgfpathcurveto{\pgfqpoint{2.174094in}{3.293197in}}{\pgfqpoint{2.178485in}{3.303796in}}{\pgfqpoint{2.178485in}{3.314846in}}%
\pgfpathcurveto{\pgfqpoint{2.178485in}{3.325896in}}{\pgfqpoint{2.174094in}{3.336495in}}{\pgfqpoint{2.166281in}{3.344309in}}%
\pgfpathcurveto{\pgfqpoint{2.158467in}{3.352123in}}{\pgfqpoint{2.147868in}{3.356513in}}{\pgfqpoint{2.136818in}{3.356513in}}%
\pgfpathcurveto{\pgfqpoint{2.125768in}{3.356513in}}{\pgfqpoint{2.115169in}{3.352123in}}{\pgfqpoint{2.107355in}{3.344309in}}%
\pgfpathcurveto{\pgfqpoint{2.099542in}{3.336495in}}{\pgfqpoint{2.095151in}{3.325896in}}{\pgfqpoint{2.095151in}{3.314846in}}%
\pgfpathcurveto{\pgfqpoint{2.095151in}{3.303796in}}{\pgfqpoint{2.099542in}{3.293197in}}{\pgfqpoint{2.107355in}{3.285383in}}%
\pgfpathcurveto{\pgfqpoint{2.115169in}{3.277570in}}{\pgfqpoint{2.125768in}{3.273180in}}{\pgfqpoint{2.136818in}{3.273180in}}%
\pgfpathclose%
\pgfusepath{stroke,fill}%
\end{pgfscope}%
\begin{pgfscope}%
\pgfpathrectangle{\pgfqpoint{0.481978in}{0.331635in}}{\pgfqpoint{9.300000in}{7.700000in}}%
\pgfusepath{clip}%
\pgfsetbuttcap%
\pgfsetroundjoin%
\definecolor{currentfill}{rgb}{1.000000,0.705882,0.509804}%
\pgfsetfillcolor{currentfill}%
\pgfsetlinewidth{0.481800pt}%
\definecolor{currentstroke}{rgb}{1.000000,1.000000,1.000000}%
\pgfsetstrokecolor{currentstroke}%
\pgfsetdash{}{0pt}%
\pgfpathmoveto{\pgfqpoint{3.304744in}{3.404313in}}%
\pgfpathcurveto{\pgfqpoint{3.315794in}{3.404313in}}{\pgfqpoint{3.326393in}{3.408703in}}{\pgfqpoint{3.334207in}{3.416517in}}%
\pgfpathcurveto{\pgfqpoint{3.342021in}{3.424331in}}{\pgfqpoint{3.346411in}{3.434930in}}{\pgfqpoint{3.346411in}{3.445980in}}%
\pgfpathcurveto{\pgfqpoint{3.346411in}{3.457030in}}{\pgfqpoint{3.342021in}{3.467629in}}{\pgfqpoint{3.334207in}{3.475443in}}%
\pgfpathcurveto{\pgfqpoint{3.326393in}{3.483256in}}{\pgfqpoint{3.315794in}{3.487646in}}{\pgfqpoint{3.304744in}{3.487646in}}%
\pgfpathcurveto{\pgfqpoint{3.293694in}{3.487646in}}{\pgfqpoint{3.283095in}{3.483256in}}{\pgfqpoint{3.275282in}{3.475443in}}%
\pgfpathcurveto{\pgfqpoint{3.267468in}{3.467629in}}{\pgfqpoint{3.263078in}{3.457030in}}{\pgfqpoint{3.263078in}{3.445980in}}%
\pgfpathcurveto{\pgfqpoint{3.263078in}{3.434930in}}{\pgfqpoint{3.267468in}{3.424331in}}{\pgfqpoint{3.275282in}{3.416517in}}%
\pgfpathcurveto{\pgfqpoint{3.283095in}{3.408703in}}{\pgfqpoint{3.293694in}{3.404313in}}{\pgfqpoint{3.304744in}{3.404313in}}%
\pgfpathclose%
\pgfusepath{stroke,fill}%
\end{pgfscope}%
\begin{pgfscope}%
\pgfpathrectangle{\pgfqpoint{0.481978in}{0.331635in}}{\pgfqpoint{9.300000in}{7.700000in}}%
\pgfusepath{clip}%
\pgfsetbuttcap%
\pgfsetroundjoin%
\definecolor{currentfill}{rgb}{1.000000,0.705882,0.509804}%
\pgfsetfillcolor{currentfill}%
\pgfsetlinewidth{0.481800pt}%
\definecolor{currentstroke}{rgb}{1.000000,1.000000,1.000000}%
\pgfsetstrokecolor{currentstroke}%
\pgfsetdash{}{0pt}%
\pgfpathmoveto{\pgfqpoint{1.284269in}{4.704795in}}%
\pgfpathcurveto{\pgfqpoint{1.295319in}{4.704795in}}{\pgfqpoint{1.305918in}{4.709185in}}{\pgfqpoint{1.313731in}{4.716999in}}%
\pgfpathcurveto{\pgfqpoint{1.321545in}{4.724812in}}{\pgfqpoint{1.325935in}{4.735411in}}{\pgfqpoint{1.325935in}{4.746461in}}%
\pgfpathcurveto{\pgfqpoint{1.325935in}{4.757511in}}{\pgfqpoint{1.321545in}{4.768110in}}{\pgfqpoint{1.313731in}{4.775924in}}%
\pgfpathcurveto{\pgfqpoint{1.305918in}{4.783738in}}{\pgfqpoint{1.295319in}{4.788128in}}{\pgfqpoint{1.284269in}{4.788128in}}%
\pgfpathcurveto{\pgfqpoint{1.273219in}{4.788128in}}{\pgfqpoint{1.262620in}{4.783738in}}{\pgfqpoint{1.254806in}{4.775924in}}%
\pgfpathcurveto{\pgfqpoint{1.246992in}{4.768110in}}{\pgfqpoint{1.242602in}{4.757511in}}{\pgfqpoint{1.242602in}{4.746461in}}%
\pgfpathcurveto{\pgfqpoint{1.242602in}{4.735411in}}{\pgfqpoint{1.246992in}{4.724812in}}{\pgfqpoint{1.254806in}{4.716999in}}%
\pgfpathcurveto{\pgfqpoint{1.262620in}{4.709185in}}{\pgfqpoint{1.273219in}{4.704795in}}{\pgfqpoint{1.284269in}{4.704795in}}%
\pgfpathclose%
\pgfusepath{stroke,fill}%
\end{pgfscope}%
\begin{pgfscope}%
\pgfpathrectangle{\pgfqpoint{0.481978in}{0.331635in}}{\pgfqpoint{9.300000in}{7.700000in}}%
\pgfusepath{clip}%
\pgfsetbuttcap%
\pgfsetroundjoin%
\definecolor{currentfill}{rgb}{1.000000,0.705882,0.509804}%
\pgfsetfillcolor{currentfill}%
\pgfsetlinewidth{0.481800pt}%
\definecolor{currentstroke}{rgb}{1.000000,1.000000,1.000000}%
\pgfsetstrokecolor{currentstroke}%
\pgfsetdash{}{0pt}%
\pgfpathmoveto{\pgfqpoint{3.921640in}{6.496134in}}%
\pgfpathcurveto{\pgfqpoint{3.932691in}{6.496134in}}{\pgfqpoint{3.943290in}{6.500525in}}{\pgfqpoint{3.951103in}{6.508338in}}%
\pgfpathcurveto{\pgfqpoint{3.958917in}{6.516152in}}{\pgfqpoint{3.963307in}{6.526751in}}{\pgfqpoint{3.963307in}{6.537801in}}%
\pgfpathcurveto{\pgfqpoint{3.963307in}{6.548851in}}{\pgfqpoint{3.958917in}{6.559450in}}{\pgfqpoint{3.951103in}{6.567264in}}%
\pgfpathcurveto{\pgfqpoint{3.943290in}{6.575077in}}{\pgfqpoint{3.932691in}{6.579468in}}{\pgfqpoint{3.921640in}{6.579468in}}%
\pgfpathcurveto{\pgfqpoint{3.910590in}{6.579468in}}{\pgfqpoint{3.899991in}{6.575077in}}{\pgfqpoint{3.892178in}{6.567264in}}%
\pgfpathcurveto{\pgfqpoint{3.884364in}{6.559450in}}{\pgfqpoint{3.879974in}{6.548851in}}{\pgfqpoint{3.879974in}{6.537801in}}%
\pgfpathcurveto{\pgfqpoint{3.879974in}{6.526751in}}{\pgfqpoint{3.884364in}{6.516152in}}{\pgfqpoint{3.892178in}{6.508338in}}%
\pgfpathcurveto{\pgfqpoint{3.899991in}{6.500525in}}{\pgfqpoint{3.910590in}{6.496134in}}{\pgfqpoint{3.921640in}{6.496134in}}%
\pgfpathclose%
\pgfusepath{stroke,fill}%
\end{pgfscope}%
\begin{pgfscope}%
\pgfpathrectangle{\pgfqpoint{0.481978in}{0.331635in}}{\pgfqpoint{9.300000in}{7.700000in}}%
\pgfusepath{clip}%
\pgfsetbuttcap%
\pgfsetroundjoin%
\definecolor{currentfill}{rgb}{1.000000,0.705882,0.509804}%
\pgfsetfillcolor{currentfill}%
\pgfsetlinewidth{0.481800pt}%
\definecolor{currentstroke}{rgb}{1.000000,1.000000,1.000000}%
\pgfsetstrokecolor{currentstroke}%
\pgfsetdash{}{0pt}%
\pgfpathmoveto{\pgfqpoint{4.551669in}{3.847471in}}%
\pgfpathcurveto{\pgfqpoint{4.562719in}{3.847471in}}{\pgfqpoint{4.573318in}{3.851862in}}{\pgfqpoint{4.581131in}{3.859675in}}%
\pgfpathcurveto{\pgfqpoint{4.588945in}{3.867489in}}{\pgfqpoint{4.593335in}{3.878088in}}{\pgfqpoint{4.593335in}{3.889138in}}%
\pgfpathcurveto{\pgfqpoint{4.593335in}{3.900188in}}{\pgfqpoint{4.588945in}{3.910787in}}{\pgfqpoint{4.581131in}{3.918601in}}%
\pgfpathcurveto{\pgfqpoint{4.573318in}{3.926414in}}{\pgfqpoint{4.562719in}{3.930805in}}{\pgfqpoint{4.551669in}{3.930805in}}%
\pgfpathcurveto{\pgfqpoint{4.540618in}{3.930805in}}{\pgfqpoint{4.530019in}{3.926414in}}{\pgfqpoint{4.522206in}{3.918601in}}%
\pgfpathcurveto{\pgfqpoint{4.514392in}{3.910787in}}{\pgfqpoint{4.510002in}{3.900188in}}{\pgfqpoint{4.510002in}{3.889138in}}%
\pgfpathcurveto{\pgfqpoint{4.510002in}{3.878088in}}{\pgfqpoint{4.514392in}{3.867489in}}{\pgfqpoint{4.522206in}{3.859675in}}%
\pgfpathcurveto{\pgfqpoint{4.530019in}{3.851862in}}{\pgfqpoint{4.540618in}{3.847471in}}{\pgfqpoint{4.551669in}{3.847471in}}%
\pgfpathclose%
\pgfusepath{stroke,fill}%
\end{pgfscope}%
\begin{pgfscope}%
\pgfpathrectangle{\pgfqpoint{0.481978in}{0.331635in}}{\pgfqpoint{9.300000in}{7.700000in}}%
\pgfusepath{clip}%
\pgfsetbuttcap%
\pgfsetroundjoin%
\definecolor{currentfill}{rgb}{1.000000,0.705882,0.509804}%
\pgfsetfillcolor{currentfill}%
\pgfsetlinewidth{0.481800pt}%
\definecolor{currentstroke}{rgb}{1.000000,1.000000,1.000000}%
\pgfsetstrokecolor{currentstroke}%
\pgfsetdash{}{0pt}%
\pgfpathmoveto{\pgfqpoint{2.003717in}{3.772999in}}%
\pgfpathcurveto{\pgfqpoint{2.014767in}{3.772999in}}{\pgfqpoint{2.025367in}{3.777389in}}{\pgfqpoint{2.033180in}{3.785202in}}%
\pgfpathcurveto{\pgfqpoint{2.040994in}{3.793016in}}{\pgfqpoint{2.045384in}{3.803615in}}{\pgfqpoint{2.045384in}{3.814665in}}%
\pgfpathcurveto{\pgfqpoint{2.045384in}{3.825715in}}{\pgfqpoint{2.040994in}{3.836314in}}{\pgfqpoint{2.033180in}{3.844128in}}%
\pgfpathcurveto{\pgfqpoint{2.025367in}{3.851942in}}{\pgfqpoint{2.014767in}{3.856332in}}{\pgfqpoint{2.003717in}{3.856332in}}%
\pgfpathcurveto{\pgfqpoint{1.992667in}{3.856332in}}{\pgfqpoint{1.982068in}{3.851942in}}{\pgfqpoint{1.974255in}{3.844128in}}%
\pgfpathcurveto{\pgfqpoint{1.966441in}{3.836314in}}{\pgfqpoint{1.962051in}{3.825715in}}{\pgfqpoint{1.962051in}{3.814665in}}%
\pgfpathcurveto{\pgfqpoint{1.962051in}{3.803615in}}{\pgfqpoint{1.966441in}{3.793016in}}{\pgfqpoint{1.974255in}{3.785202in}}%
\pgfpathcurveto{\pgfqpoint{1.982068in}{3.777389in}}{\pgfqpoint{1.992667in}{3.772999in}}{\pgfqpoint{2.003717in}{3.772999in}}%
\pgfpathclose%
\pgfusepath{stroke,fill}%
\end{pgfscope}%
\begin{pgfscope}%
\pgfpathrectangle{\pgfqpoint{0.481978in}{0.331635in}}{\pgfqpoint{9.300000in}{7.700000in}}%
\pgfusepath{clip}%
\pgfsetbuttcap%
\pgfsetroundjoin%
\definecolor{currentfill}{rgb}{1.000000,0.705882,0.509804}%
\pgfsetfillcolor{currentfill}%
\pgfsetlinewidth{0.481800pt}%
\definecolor{currentstroke}{rgb}{1.000000,1.000000,1.000000}%
\pgfsetstrokecolor{currentstroke}%
\pgfsetdash{}{0pt}%
\pgfpathmoveto{\pgfqpoint{2.821317in}{3.686823in}}%
\pgfpathcurveto{\pgfqpoint{2.832367in}{3.686823in}}{\pgfqpoint{2.842966in}{3.691213in}}{\pgfqpoint{2.850780in}{3.699026in}}%
\pgfpathcurveto{\pgfqpoint{2.858593in}{3.706840in}}{\pgfqpoint{2.862983in}{3.717439in}}{\pgfqpoint{2.862983in}{3.728489in}}%
\pgfpathcurveto{\pgfqpoint{2.862983in}{3.739539in}}{\pgfqpoint{2.858593in}{3.750138in}}{\pgfqpoint{2.850780in}{3.757952in}}%
\pgfpathcurveto{\pgfqpoint{2.842966in}{3.765766in}}{\pgfqpoint{2.832367in}{3.770156in}}{\pgfqpoint{2.821317in}{3.770156in}}%
\pgfpathcurveto{\pgfqpoint{2.810267in}{3.770156in}}{\pgfqpoint{2.799668in}{3.765766in}}{\pgfqpoint{2.791854in}{3.757952in}}%
\pgfpathcurveto{\pgfqpoint{2.784040in}{3.750138in}}{\pgfqpoint{2.779650in}{3.739539in}}{\pgfqpoint{2.779650in}{3.728489in}}%
\pgfpathcurveto{\pgfqpoint{2.779650in}{3.717439in}}{\pgfqpoint{2.784040in}{3.706840in}}{\pgfqpoint{2.791854in}{3.699026in}}%
\pgfpathcurveto{\pgfqpoint{2.799668in}{3.691213in}}{\pgfqpoint{2.810267in}{3.686823in}}{\pgfqpoint{2.821317in}{3.686823in}}%
\pgfpathclose%
\pgfusepath{stroke,fill}%
\end{pgfscope}%
\begin{pgfscope}%
\pgfpathrectangle{\pgfqpoint{0.481978in}{0.331635in}}{\pgfqpoint{9.300000in}{7.700000in}}%
\pgfusepath{clip}%
\pgfsetbuttcap%
\pgfsetroundjoin%
\definecolor{currentfill}{rgb}{1.000000,0.705882,0.509804}%
\pgfsetfillcolor{currentfill}%
\pgfsetlinewidth{0.481800pt}%
\definecolor{currentstroke}{rgb}{1.000000,1.000000,1.000000}%
\pgfsetstrokecolor{currentstroke}%
\pgfsetdash{}{0pt}%
\pgfpathmoveto{\pgfqpoint{2.335285in}{3.839174in}}%
\pgfpathcurveto{\pgfqpoint{2.346335in}{3.839174in}}{\pgfqpoint{2.356934in}{3.843564in}}{\pgfqpoint{2.364748in}{3.851378in}}%
\pgfpathcurveto{\pgfqpoint{2.372561in}{3.859192in}}{\pgfqpoint{2.376952in}{3.869791in}}{\pgfqpoint{2.376952in}{3.880841in}}%
\pgfpathcurveto{\pgfqpoint{2.376952in}{3.891891in}}{\pgfqpoint{2.372561in}{3.902490in}}{\pgfqpoint{2.364748in}{3.910304in}}%
\pgfpathcurveto{\pgfqpoint{2.356934in}{3.918117in}}{\pgfqpoint{2.346335in}{3.922507in}}{\pgfqpoint{2.335285in}{3.922507in}}%
\pgfpathcurveto{\pgfqpoint{2.324235in}{3.922507in}}{\pgfqpoint{2.313636in}{3.918117in}}{\pgfqpoint{2.305822in}{3.910304in}}%
\pgfpathcurveto{\pgfqpoint{2.298009in}{3.902490in}}{\pgfqpoint{2.293618in}{3.891891in}}{\pgfqpoint{2.293618in}{3.880841in}}%
\pgfpathcurveto{\pgfqpoint{2.293618in}{3.869791in}}{\pgfqpoint{2.298009in}{3.859192in}}{\pgfqpoint{2.305822in}{3.851378in}}%
\pgfpathcurveto{\pgfqpoint{2.313636in}{3.843564in}}{\pgfqpoint{2.324235in}{3.839174in}}{\pgfqpoint{2.335285in}{3.839174in}}%
\pgfpathclose%
\pgfusepath{stroke,fill}%
\end{pgfscope}%
\begin{pgfscope}%
\pgfpathrectangle{\pgfqpoint{0.481978in}{0.331635in}}{\pgfqpoint{9.300000in}{7.700000in}}%
\pgfusepath{clip}%
\pgfsetbuttcap%
\pgfsetroundjoin%
\definecolor{currentfill}{rgb}{1.000000,0.705882,0.509804}%
\pgfsetfillcolor{currentfill}%
\pgfsetlinewidth{0.481800pt}%
\definecolor{currentstroke}{rgb}{1.000000,1.000000,1.000000}%
\pgfsetstrokecolor{currentstroke}%
\pgfsetdash{}{0pt}%
\pgfpathmoveto{\pgfqpoint{4.485108in}{3.824317in}}%
\pgfpathcurveto{\pgfqpoint{4.496158in}{3.824317in}}{\pgfqpoint{4.506757in}{3.828707in}}{\pgfqpoint{4.514571in}{3.836521in}}%
\pgfpathcurveto{\pgfqpoint{4.522385in}{3.844334in}}{\pgfqpoint{4.526775in}{3.854933in}}{\pgfqpoint{4.526775in}{3.865983in}}%
\pgfpathcurveto{\pgfqpoint{4.526775in}{3.877033in}}{\pgfqpoint{4.522385in}{3.887632in}}{\pgfqpoint{4.514571in}{3.895446in}}%
\pgfpathcurveto{\pgfqpoint{4.506757in}{3.903260in}}{\pgfqpoint{4.496158in}{3.907650in}}{\pgfqpoint{4.485108in}{3.907650in}}%
\pgfpathcurveto{\pgfqpoint{4.474058in}{3.907650in}}{\pgfqpoint{4.463459in}{3.903260in}}{\pgfqpoint{4.455645in}{3.895446in}}%
\pgfpathcurveto{\pgfqpoint{4.447832in}{3.887632in}}{\pgfqpoint{4.443442in}{3.877033in}}{\pgfqpoint{4.443442in}{3.865983in}}%
\pgfpathcurveto{\pgfqpoint{4.443442in}{3.854933in}}{\pgfqpoint{4.447832in}{3.844334in}}{\pgfqpoint{4.455645in}{3.836521in}}%
\pgfpathcurveto{\pgfqpoint{4.463459in}{3.828707in}}{\pgfqpoint{4.474058in}{3.824317in}}{\pgfqpoint{4.485108in}{3.824317in}}%
\pgfpathclose%
\pgfusepath{stroke,fill}%
\end{pgfscope}%
\begin{pgfscope}%
\pgfpathrectangle{\pgfqpoint{0.481978in}{0.331635in}}{\pgfqpoint{9.300000in}{7.700000in}}%
\pgfusepath{clip}%
\pgfsetbuttcap%
\pgfsetroundjoin%
\definecolor{currentfill}{rgb}{1.000000,0.705882,0.509804}%
\pgfsetfillcolor{currentfill}%
\pgfsetlinewidth{0.481800pt}%
\definecolor{currentstroke}{rgb}{1.000000,1.000000,1.000000}%
\pgfsetstrokecolor{currentstroke}%
\pgfsetdash{}{0pt}%
\pgfpathmoveto{\pgfqpoint{1.703134in}{4.248095in}}%
\pgfpathcurveto{\pgfqpoint{1.714184in}{4.248095in}}{\pgfqpoint{1.724783in}{4.252485in}}{\pgfqpoint{1.732597in}{4.260299in}}%
\pgfpathcurveto{\pgfqpoint{1.740411in}{4.268112in}}{\pgfqpoint{1.744801in}{4.278711in}}{\pgfqpoint{1.744801in}{4.289761in}}%
\pgfpathcurveto{\pgfqpoint{1.744801in}{4.300812in}}{\pgfqpoint{1.740411in}{4.311411in}}{\pgfqpoint{1.732597in}{4.319224in}}%
\pgfpathcurveto{\pgfqpoint{1.724783in}{4.327038in}}{\pgfqpoint{1.714184in}{4.331428in}}{\pgfqpoint{1.703134in}{4.331428in}}%
\pgfpathcurveto{\pgfqpoint{1.692084in}{4.331428in}}{\pgfqpoint{1.681485in}{4.327038in}}{\pgfqpoint{1.673671in}{4.319224in}}%
\pgfpathcurveto{\pgfqpoint{1.665858in}{4.311411in}}{\pgfqpoint{1.661467in}{4.300812in}}{\pgfqpoint{1.661467in}{4.289761in}}%
\pgfpathcurveto{\pgfqpoint{1.661467in}{4.278711in}}{\pgfqpoint{1.665858in}{4.268112in}}{\pgfqpoint{1.673671in}{4.260299in}}%
\pgfpathcurveto{\pgfqpoint{1.681485in}{4.252485in}}{\pgfqpoint{1.692084in}{4.248095in}}{\pgfqpoint{1.703134in}{4.248095in}}%
\pgfpathclose%
\pgfusepath{stroke,fill}%
\end{pgfscope}%
\begin{pgfscope}%
\pgfpathrectangle{\pgfqpoint{0.481978in}{0.331635in}}{\pgfqpoint{9.300000in}{7.700000in}}%
\pgfusepath{clip}%
\pgfsetbuttcap%
\pgfsetroundjoin%
\definecolor{currentfill}{rgb}{1.000000,0.705882,0.509804}%
\pgfsetfillcolor{currentfill}%
\pgfsetlinewidth{0.481800pt}%
\definecolor{currentstroke}{rgb}{1.000000,1.000000,1.000000}%
\pgfsetstrokecolor{currentstroke}%
\pgfsetdash{}{0pt}%
\pgfpathmoveto{\pgfqpoint{4.082596in}{4.738956in}}%
\pgfpathcurveto{\pgfqpoint{4.093646in}{4.738956in}}{\pgfqpoint{4.104245in}{4.743347in}}{\pgfqpoint{4.112059in}{4.751160in}}%
\pgfpathcurveto{\pgfqpoint{4.119873in}{4.758974in}}{\pgfqpoint{4.124263in}{4.769573in}}{\pgfqpoint{4.124263in}{4.780623in}}%
\pgfpathcurveto{\pgfqpoint{4.124263in}{4.791673in}}{\pgfqpoint{4.119873in}{4.802272in}}{\pgfqpoint{4.112059in}{4.810086in}}%
\pgfpathcurveto{\pgfqpoint{4.104245in}{4.817899in}}{\pgfqpoint{4.093646in}{4.822290in}}{\pgfqpoint{4.082596in}{4.822290in}}%
\pgfpathcurveto{\pgfqpoint{4.071546in}{4.822290in}}{\pgfqpoint{4.060947in}{4.817899in}}{\pgfqpoint{4.053133in}{4.810086in}}%
\pgfpathcurveto{\pgfqpoint{4.045320in}{4.802272in}}{\pgfqpoint{4.040929in}{4.791673in}}{\pgfqpoint{4.040929in}{4.780623in}}%
\pgfpathcurveto{\pgfqpoint{4.040929in}{4.769573in}}{\pgfqpoint{4.045320in}{4.758974in}}{\pgfqpoint{4.053133in}{4.751160in}}%
\pgfpathcurveto{\pgfqpoint{4.060947in}{4.743347in}}{\pgfqpoint{4.071546in}{4.738956in}}{\pgfqpoint{4.082596in}{4.738956in}}%
\pgfpathclose%
\pgfusepath{stroke,fill}%
\end{pgfscope}%
\begin{pgfscope}%
\pgfpathrectangle{\pgfqpoint{0.481978in}{0.331635in}}{\pgfqpoint{9.300000in}{7.700000in}}%
\pgfusepath{clip}%
\pgfsetbuttcap%
\pgfsetroundjoin%
\definecolor{currentfill}{rgb}{1.000000,0.705882,0.509804}%
\pgfsetfillcolor{currentfill}%
\pgfsetlinewidth{0.481800pt}%
\definecolor{currentstroke}{rgb}{1.000000,1.000000,1.000000}%
\pgfsetstrokecolor{currentstroke}%
\pgfsetdash{}{0pt}%
\pgfpathmoveto{\pgfqpoint{3.376694in}{3.788825in}}%
\pgfpathcurveto{\pgfqpoint{3.387744in}{3.788825in}}{\pgfqpoint{3.398343in}{3.793215in}}{\pgfqpoint{3.406156in}{3.801029in}}%
\pgfpathcurveto{\pgfqpoint{3.413970in}{3.808843in}}{\pgfqpoint{3.418360in}{3.819442in}}{\pgfqpoint{3.418360in}{3.830492in}}%
\pgfpathcurveto{\pgfqpoint{3.418360in}{3.841542in}}{\pgfqpoint{3.413970in}{3.852141in}}{\pgfqpoint{3.406156in}{3.859955in}}%
\pgfpathcurveto{\pgfqpoint{3.398343in}{3.867768in}}{\pgfqpoint{3.387744in}{3.872158in}}{\pgfqpoint{3.376694in}{3.872158in}}%
\pgfpathcurveto{\pgfqpoint{3.365644in}{3.872158in}}{\pgfqpoint{3.355044in}{3.867768in}}{\pgfqpoint{3.347231in}{3.859955in}}%
\pgfpathcurveto{\pgfqpoint{3.339417in}{3.852141in}}{\pgfqpoint{3.335027in}{3.841542in}}{\pgfqpoint{3.335027in}{3.830492in}}%
\pgfpathcurveto{\pgfqpoint{3.335027in}{3.819442in}}{\pgfqpoint{3.339417in}{3.808843in}}{\pgfqpoint{3.347231in}{3.801029in}}%
\pgfpathcurveto{\pgfqpoint{3.355044in}{3.793215in}}{\pgfqpoint{3.365644in}{3.788825in}}{\pgfqpoint{3.376694in}{3.788825in}}%
\pgfpathclose%
\pgfusepath{stroke,fill}%
\end{pgfscope}%
\begin{pgfscope}%
\pgfpathrectangle{\pgfqpoint{0.481978in}{0.331635in}}{\pgfqpoint{9.300000in}{7.700000in}}%
\pgfusepath{clip}%
\pgfsetbuttcap%
\pgfsetroundjoin%
\definecolor{currentfill}{rgb}{1.000000,0.705882,0.509804}%
\pgfsetfillcolor{currentfill}%
\pgfsetlinewidth{0.481800pt}%
\definecolor{currentstroke}{rgb}{1.000000,1.000000,1.000000}%
\pgfsetstrokecolor{currentstroke}%
\pgfsetdash{}{0pt}%
\pgfpathmoveto{\pgfqpoint{4.094855in}{5.240576in}}%
\pgfpathcurveto{\pgfqpoint{4.105906in}{5.240576in}}{\pgfqpoint{4.116505in}{5.244966in}}{\pgfqpoint{4.124318in}{5.252780in}}%
\pgfpathcurveto{\pgfqpoint{4.132132in}{5.260594in}}{\pgfqpoint{4.136522in}{5.271193in}}{\pgfqpoint{4.136522in}{5.282243in}}%
\pgfpathcurveto{\pgfqpoint{4.136522in}{5.293293in}}{\pgfqpoint{4.132132in}{5.303892in}}{\pgfqpoint{4.124318in}{5.311705in}}%
\pgfpathcurveto{\pgfqpoint{4.116505in}{5.319519in}}{\pgfqpoint{4.105906in}{5.323909in}}{\pgfqpoint{4.094855in}{5.323909in}}%
\pgfpathcurveto{\pgfqpoint{4.083805in}{5.323909in}}{\pgfqpoint{4.073206in}{5.319519in}}{\pgfqpoint{4.065393in}{5.311705in}}%
\pgfpathcurveto{\pgfqpoint{4.057579in}{5.303892in}}{\pgfqpoint{4.053189in}{5.293293in}}{\pgfqpoint{4.053189in}{5.282243in}}%
\pgfpathcurveto{\pgfqpoint{4.053189in}{5.271193in}}{\pgfqpoint{4.057579in}{5.260594in}}{\pgfqpoint{4.065393in}{5.252780in}}%
\pgfpathcurveto{\pgfqpoint{4.073206in}{5.244966in}}{\pgfqpoint{4.083805in}{5.240576in}}{\pgfqpoint{4.094855in}{5.240576in}}%
\pgfpathclose%
\pgfusepath{stroke,fill}%
\end{pgfscope}%
\begin{pgfscope}%
\pgfpathrectangle{\pgfqpoint{0.481978in}{0.331635in}}{\pgfqpoint{9.300000in}{7.700000in}}%
\pgfusepath{clip}%
\pgfsetbuttcap%
\pgfsetroundjoin%
\definecolor{currentfill}{rgb}{1.000000,0.705882,0.509804}%
\pgfsetfillcolor{currentfill}%
\pgfsetlinewidth{0.481800pt}%
\definecolor{currentstroke}{rgb}{1.000000,1.000000,1.000000}%
\pgfsetstrokecolor{currentstroke}%
\pgfsetdash{}{0pt}%
\pgfpathmoveto{\pgfqpoint{4.055791in}{5.307232in}}%
\pgfpathcurveto{\pgfqpoint{4.066841in}{5.307232in}}{\pgfqpoint{4.077440in}{5.311623in}}{\pgfqpoint{4.085254in}{5.319436in}}%
\pgfpathcurveto{\pgfqpoint{4.093068in}{5.327250in}}{\pgfqpoint{4.097458in}{5.337849in}}{\pgfqpoint{4.097458in}{5.348899in}}%
\pgfpathcurveto{\pgfqpoint{4.097458in}{5.359949in}}{\pgfqpoint{4.093068in}{5.370548in}}{\pgfqpoint{4.085254in}{5.378362in}}%
\pgfpathcurveto{\pgfqpoint{4.077440in}{5.386175in}}{\pgfqpoint{4.066841in}{5.390566in}}{\pgfqpoint{4.055791in}{5.390566in}}%
\pgfpathcurveto{\pgfqpoint{4.044741in}{5.390566in}}{\pgfqpoint{4.034142in}{5.386175in}}{\pgfqpoint{4.026328in}{5.378362in}}%
\pgfpathcurveto{\pgfqpoint{4.018515in}{5.370548in}}{\pgfqpoint{4.014125in}{5.359949in}}{\pgfqpoint{4.014125in}{5.348899in}}%
\pgfpathcurveto{\pgfqpoint{4.014125in}{5.337849in}}{\pgfqpoint{4.018515in}{5.327250in}}{\pgfqpoint{4.026328in}{5.319436in}}%
\pgfpathcurveto{\pgfqpoint{4.034142in}{5.311623in}}{\pgfqpoint{4.044741in}{5.307232in}}{\pgfqpoint{4.055791in}{5.307232in}}%
\pgfpathclose%
\pgfusepath{stroke,fill}%
\end{pgfscope}%
\begin{pgfscope}%
\pgfpathrectangle{\pgfqpoint{0.481978in}{0.331635in}}{\pgfqpoint{9.300000in}{7.700000in}}%
\pgfusepath{clip}%
\pgfsetbuttcap%
\pgfsetroundjoin%
\definecolor{currentfill}{rgb}{1.000000,0.705882,0.509804}%
\pgfsetfillcolor{currentfill}%
\pgfsetlinewidth{0.481800pt}%
\definecolor{currentstroke}{rgb}{1.000000,1.000000,1.000000}%
\pgfsetstrokecolor{currentstroke}%
\pgfsetdash{}{0pt}%
\pgfpathmoveto{\pgfqpoint{4.975557in}{3.343781in}}%
\pgfpathcurveto{\pgfqpoint{4.986607in}{3.343781in}}{\pgfqpoint{4.997206in}{3.348171in}}{\pgfqpoint{5.005020in}{3.355985in}}%
\pgfpathcurveto{\pgfqpoint{5.012833in}{3.363799in}}{\pgfqpoint{5.017223in}{3.374398in}}{\pgfqpoint{5.017223in}{3.385448in}}%
\pgfpathcurveto{\pgfqpoint{5.017223in}{3.396498in}}{\pgfqpoint{5.012833in}{3.407097in}}{\pgfqpoint{5.005020in}{3.414910in}}%
\pgfpathcurveto{\pgfqpoint{4.997206in}{3.422724in}}{\pgfqpoint{4.986607in}{3.427114in}}{\pgfqpoint{4.975557in}{3.427114in}}%
\pgfpathcurveto{\pgfqpoint{4.964507in}{3.427114in}}{\pgfqpoint{4.953908in}{3.422724in}}{\pgfqpoint{4.946094in}{3.414910in}}%
\pgfpathcurveto{\pgfqpoint{4.938280in}{3.407097in}}{\pgfqpoint{4.933890in}{3.396498in}}{\pgfqpoint{4.933890in}{3.385448in}}%
\pgfpathcurveto{\pgfqpoint{4.933890in}{3.374398in}}{\pgfqpoint{4.938280in}{3.363799in}}{\pgfqpoint{4.946094in}{3.355985in}}%
\pgfpathcurveto{\pgfqpoint{4.953908in}{3.348171in}}{\pgfqpoint{4.964507in}{3.343781in}}{\pgfqpoint{4.975557in}{3.343781in}}%
\pgfpathclose%
\pgfusepath{stroke,fill}%
\end{pgfscope}%
\begin{pgfscope}%
\pgfpathrectangle{\pgfqpoint{0.481978in}{0.331635in}}{\pgfqpoint{9.300000in}{7.700000in}}%
\pgfusepath{clip}%
\pgfsetbuttcap%
\pgfsetroundjoin%
\definecolor{currentfill}{rgb}{1.000000,0.705882,0.509804}%
\pgfsetfillcolor{currentfill}%
\pgfsetlinewidth{0.481800pt}%
\definecolor{currentstroke}{rgb}{1.000000,1.000000,1.000000}%
\pgfsetstrokecolor{currentstroke}%
\pgfsetdash{}{0pt}%
\pgfpathmoveto{\pgfqpoint{2.986906in}{3.677358in}}%
\pgfpathcurveto{\pgfqpoint{2.997956in}{3.677358in}}{\pgfqpoint{3.008555in}{3.681748in}}{\pgfqpoint{3.016369in}{3.689562in}}%
\pgfpathcurveto{\pgfqpoint{3.024183in}{3.697376in}}{\pgfqpoint{3.028573in}{3.707975in}}{\pgfqpoint{3.028573in}{3.719025in}}%
\pgfpathcurveto{\pgfqpoint{3.028573in}{3.730075in}}{\pgfqpoint{3.024183in}{3.740674in}}{\pgfqpoint{3.016369in}{3.748488in}}%
\pgfpathcurveto{\pgfqpoint{3.008555in}{3.756301in}}{\pgfqpoint{2.997956in}{3.760692in}}{\pgfqpoint{2.986906in}{3.760692in}}%
\pgfpathcurveto{\pgfqpoint{2.975856in}{3.760692in}}{\pgfqpoint{2.965257in}{3.756301in}}{\pgfqpoint{2.957444in}{3.748488in}}%
\pgfpathcurveto{\pgfqpoint{2.949630in}{3.740674in}}{\pgfqpoint{2.945240in}{3.730075in}}{\pgfqpoint{2.945240in}{3.719025in}}%
\pgfpathcurveto{\pgfqpoint{2.945240in}{3.707975in}}{\pgfqpoint{2.949630in}{3.697376in}}{\pgfqpoint{2.957444in}{3.689562in}}%
\pgfpathcurveto{\pgfqpoint{2.965257in}{3.681748in}}{\pgfqpoint{2.975856in}{3.677358in}}{\pgfqpoint{2.986906in}{3.677358in}}%
\pgfpathclose%
\pgfusepath{stroke,fill}%
\end{pgfscope}%
\begin{pgfscope}%
\pgfpathrectangle{\pgfqpoint{0.481978in}{0.331635in}}{\pgfqpoint{9.300000in}{7.700000in}}%
\pgfusepath{clip}%
\pgfsetbuttcap%
\pgfsetroundjoin%
\definecolor{currentfill}{rgb}{1.000000,0.705882,0.509804}%
\pgfsetfillcolor{currentfill}%
\pgfsetlinewidth{0.481800pt}%
\definecolor{currentstroke}{rgb}{1.000000,1.000000,1.000000}%
\pgfsetstrokecolor{currentstroke}%
\pgfsetdash{}{0pt}%
\pgfpathmoveto{\pgfqpoint{3.977136in}{2.674234in}}%
\pgfpathcurveto{\pgfqpoint{3.988186in}{2.674234in}}{\pgfqpoint{3.998785in}{2.678625in}}{\pgfqpoint{4.006599in}{2.686438in}}%
\pgfpathcurveto{\pgfqpoint{4.014412in}{2.694252in}}{\pgfqpoint{4.018802in}{2.704851in}}{\pgfqpoint{4.018802in}{2.715901in}}%
\pgfpathcurveto{\pgfqpoint{4.018802in}{2.726951in}}{\pgfqpoint{4.014412in}{2.737550in}}{\pgfqpoint{4.006599in}{2.745364in}}%
\pgfpathcurveto{\pgfqpoint{3.998785in}{2.753178in}}{\pgfqpoint{3.988186in}{2.757568in}}{\pgfqpoint{3.977136in}{2.757568in}}%
\pgfpathcurveto{\pgfqpoint{3.966086in}{2.757568in}}{\pgfqpoint{3.955487in}{2.753178in}}{\pgfqpoint{3.947673in}{2.745364in}}%
\pgfpathcurveto{\pgfqpoint{3.939859in}{2.737550in}}{\pgfqpoint{3.935469in}{2.726951in}}{\pgfqpoint{3.935469in}{2.715901in}}%
\pgfpathcurveto{\pgfqpoint{3.935469in}{2.704851in}}{\pgfqpoint{3.939859in}{2.694252in}}{\pgfqpoint{3.947673in}{2.686438in}}%
\pgfpathcurveto{\pgfqpoint{3.955487in}{2.678625in}}{\pgfqpoint{3.966086in}{2.674234in}}{\pgfqpoint{3.977136in}{2.674234in}}%
\pgfpathclose%
\pgfusepath{stroke,fill}%
\end{pgfscope}%
\begin{pgfscope}%
\pgfpathrectangle{\pgfqpoint{0.481978in}{0.331635in}}{\pgfqpoint{9.300000in}{7.700000in}}%
\pgfusepath{clip}%
\pgfsetbuttcap%
\pgfsetroundjoin%
\definecolor{currentfill}{rgb}{1.000000,0.705882,0.509804}%
\pgfsetfillcolor{currentfill}%
\pgfsetlinewidth{0.481800pt}%
\definecolor{currentstroke}{rgb}{1.000000,1.000000,1.000000}%
\pgfsetstrokecolor{currentstroke}%
\pgfsetdash{}{0pt}%
\pgfpathmoveto{\pgfqpoint{3.838833in}{6.919703in}}%
\pgfpathcurveto{\pgfqpoint{3.849883in}{6.919703in}}{\pgfqpoint{3.860482in}{6.924093in}}{\pgfqpoint{3.868296in}{6.931906in}}%
\pgfpathcurveto{\pgfqpoint{3.876109in}{6.939720in}}{\pgfqpoint{3.880500in}{6.950319in}}{\pgfqpoint{3.880500in}{6.961369in}}%
\pgfpathcurveto{\pgfqpoint{3.880500in}{6.972419in}}{\pgfqpoint{3.876109in}{6.983018in}}{\pgfqpoint{3.868296in}{6.990832in}}%
\pgfpathcurveto{\pgfqpoint{3.860482in}{6.998646in}}{\pgfqpoint{3.849883in}{7.003036in}}{\pgfqpoint{3.838833in}{7.003036in}}%
\pgfpathcurveto{\pgfqpoint{3.827783in}{7.003036in}}{\pgfqpoint{3.817184in}{6.998646in}}{\pgfqpoint{3.809370in}{6.990832in}}%
\pgfpathcurveto{\pgfqpoint{3.801557in}{6.983018in}}{\pgfqpoint{3.797166in}{6.972419in}}{\pgfqpoint{3.797166in}{6.961369in}}%
\pgfpathcurveto{\pgfqpoint{3.797166in}{6.950319in}}{\pgfqpoint{3.801557in}{6.939720in}}{\pgfqpoint{3.809370in}{6.931906in}}%
\pgfpathcurveto{\pgfqpoint{3.817184in}{6.924093in}}{\pgfqpoint{3.827783in}{6.919703in}}{\pgfqpoint{3.838833in}{6.919703in}}%
\pgfpathclose%
\pgfusepath{stroke,fill}%
\end{pgfscope}%
\begin{pgfscope}%
\pgfpathrectangle{\pgfqpoint{0.481978in}{0.331635in}}{\pgfqpoint{9.300000in}{7.700000in}}%
\pgfusepath{clip}%
\pgfsetbuttcap%
\pgfsetroundjoin%
\definecolor{currentfill}{rgb}{1.000000,0.705882,0.509804}%
\pgfsetfillcolor{currentfill}%
\pgfsetlinewidth{0.481800pt}%
\definecolor{currentstroke}{rgb}{1.000000,1.000000,1.000000}%
\pgfsetstrokecolor{currentstroke}%
\pgfsetdash{}{0pt}%
\pgfpathmoveto{\pgfqpoint{2.868567in}{3.772096in}}%
\pgfpathcurveto{\pgfqpoint{2.879617in}{3.772096in}}{\pgfqpoint{2.890216in}{3.776487in}}{\pgfqpoint{2.898030in}{3.784300in}}%
\pgfpathcurveto{\pgfqpoint{2.905843in}{3.792114in}}{\pgfqpoint{2.910234in}{3.802713in}}{\pgfqpoint{2.910234in}{3.813763in}}%
\pgfpathcurveto{\pgfqpoint{2.910234in}{3.824813in}}{\pgfqpoint{2.905843in}{3.835412in}}{\pgfqpoint{2.898030in}{3.843226in}}%
\pgfpathcurveto{\pgfqpoint{2.890216in}{3.851040in}}{\pgfqpoint{2.879617in}{3.855430in}}{\pgfqpoint{2.868567in}{3.855430in}}%
\pgfpathcurveto{\pgfqpoint{2.857517in}{3.855430in}}{\pgfqpoint{2.846918in}{3.851040in}}{\pgfqpoint{2.839104in}{3.843226in}}%
\pgfpathcurveto{\pgfqpoint{2.831290in}{3.835412in}}{\pgfqpoint{2.826900in}{3.824813in}}{\pgfqpoint{2.826900in}{3.813763in}}%
\pgfpathcurveto{\pgfqpoint{2.826900in}{3.802713in}}{\pgfqpoint{2.831290in}{3.792114in}}{\pgfqpoint{2.839104in}{3.784300in}}%
\pgfpathcurveto{\pgfqpoint{2.846918in}{3.776487in}}{\pgfqpoint{2.857517in}{3.772096in}}{\pgfqpoint{2.868567in}{3.772096in}}%
\pgfpathclose%
\pgfusepath{stroke,fill}%
\end{pgfscope}%
\begin{pgfscope}%
\pgfpathrectangle{\pgfqpoint{0.481978in}{0.331635in}}{\pgfqpoint{9.300000in}{7.700000in}}%
\pgfusepath{clip}%
\pgfsetbuttcap%
\pgfsetroundjoin%
\definecolor{currentfill}{rgb}{1.000000,0.705882,0.509804}%
\pgfsetfillcolor{currentfill}%
\pgfsetlinewidth{0.481800pt}%
\definecolor{currentstroke}{rgb}{1.000000,1.000000,1.000000}%
\pgfsetstrokecolor{currentstroke}%
\pgfsetdash{}{0pt}%
\pgfpathmoveto{\pgfqpoint{5.264259in}{4.891797in}}%
\pgfpathcurveto{\pgfqpoint{5.275309in}{4.891797in}}{\pgfqpoint{5.285908in}{4.896188in}}{\pgfqpoint{5.293722in}{4.904001in}}%
\pgfpathcurveto{\pgfqpoint{5.301536in}{4.911815in}}{\pgfqpoint{5.305926in}{4.922414in}}{\pgfqpoint{5.305926in}{4.933464in}}%
\pgfpathcurveto{\pgfqpoint{5.305926in}{4.944514in}}{\pgfqpoint{5.301536in}{4.955113in}}{\pgfqpoint{5.293722in}{4.962927in}}%
\pgfpathcurveto{\pgfqpoint{5.285908in}{4.970740in}}{\pgfqpoint{5.275309in}{4.975131in}}{\pgfqpoint{5.264259in}{4.975131in}}%
\pgfpathcurveto{\pgfqpoint{5.253209in}{4.975131in}}{\pgfqpoint{5.242610in}{4.970740in}}{\pgfqpoint{5.234796in}{4.962927in}}%
\pgfpathcurveto{\pgfqpoint{5.226983in}{4.955113in}}{\pgfqpoint{5.222593in}{4.944514in}}{\pgfqpoint{5.222593in}{4.933464in}}%
\pgfpathcurveto{\pgfqpoint{5.222593in}{4.922414in}}{\pgfqpoint{5.226983in}{4.911815in}}{\pgfqpoint{5.234796in}{4.904001in}}%
\pgfpathcurveto{\pgfqpoint{5.242610in}{4.896188in}}{\pgfqpoint{5.253209in}{4.891797in}}{\pgfqpoint{5.264259in}{4.891797in}}%
\pgfpathclose%
\pgfusepath{stroke,fill}%
\end{pgfscope}%
\begin{pgfscope}%
\pgfpathrectangle{\pgfqpoint{0.481978in}{0.331635in}}{\pgfqpoint{9.300000in}{7.700000in}}%
\pgfusepath{clip}%
\pgfsetbuttcap%
\pgfsetroundjoin%
\definecolor{currentfill}{rgb}{1.000000,0.705882,0.509804}%
\pgfsetfillcolor{currentfill}%
\pgfsetlinewidth{0.481800pt}%
\definecolor{currentstroke}{rgb}{1.000000,1.000000,1.000000}%
\pgfsetstrokecolor{currentstroke}%
\pgfsetdash{}{0pt}%
\pgfpathmoveto{\pgfqpoint{3.655520in}{5.720597in}}%
\pgfpathcurveto{\pgfqpoint{3.666571in}{5.720597in}}{\pgfqpoint{3.677170in}{5.724987in}}{\pgfqpoint{3.684983in}{5.732801in}}%
\pgfpathcurveto{\pgfqpoint{3.692797in}{5.740614in}}{\pgfqpoint{3.697187in}{5.751213in}}{\pgfqpoint{3.697187in}{5.762264in}}%
\pgfpathcurveto{\pgfqpoint{3.697187in}{5.773314in}}{\pgfqpoint{3.692797in}{5.783913in}}{\pgfqpoint{3.684983in}{5.791726in}}%
\pgfpathcurveto{\pgfqpoint{3.677170in}{5.799540in}}{\pgfqpoint{3.666571in}{5.803930in}}{\pgfqpoint{3.655520in}{5.803930in}}%
\pgfpathcurveto{\pgfqpoint{3.644470in}{5.803930in}}{\pgfqpoint{3.633871in}{5.799540in}}{\pgfqpoint{3.626058in}{5.791726in}}%
\pgfpathcurveto{\pgfqpoint{3.618244in}{5.783913in}}{\pgfqpoint{3.613854in}{5.773314in}}{\pgfqpoint{3.613854in}{5.762264in}}%
\pgfpathcurveto{\pgfqpoint{3.613854in}{5.751213in}}{\pgfqpoint{3.618244in}{5.740614in}}{\pgfqpoint{3.626058in}{5.732801in}}%
\pgfpathcurveto{\pgfqpoint{3.633871in}{5.724987in}}{\pgfqpoint{3.644470in}{5.720597in}}{\pgfqpoint{3.655520in}{5.720597in}}%
\pgfpathclose%
\pgfusepath{stroke,fill}%
\end{pgfscope}%
\begin{pgfscope}%
\pgfpathrectangle{\pgfqpoint{0.481978in}{0.331635in}}{\pgfqpoint{9.300000in}{7.700000in}}%
\pgfusepath{clip}%
\pgfsetbuttcap%
\pgfsetroundjoin%
\definecolor{currentfill}{rgb}{1.000000,0.705882,0.509804}%
\pgfsetfillcolor{currentfill}%
\pgfsetlinewidth{0.481800pt}%
\definecolor{currentstroke}{rgb}{1.000000,1.000000,1.000000}%
\pgfsetstrokecolor{currentstroke}%
\pgfsetdash{}{0pt}%
\pgfpathmoveto{\pgfqpoint{1.251672in}{4.022336in}}%
\pgfpathcurveto{\pgfqpoint{1.262722in}{4.022336in}}{\pgfqpoint{1.273321in}{4.026727in}}{\pgfqpoint{1.281134in}{4.034540in}}%
\pgfpathcurveto{\pgfqpoint{1.288948in}{4.042354in}}{\pgfqpoint{1.293338in}{4.052953in}}{\pgfqpoint{1.293338in}{4.064003in}}%
\pgfpathcurveto{\pgfqpoint{1.293338in}{4.075053in}}{\pgfqpoint{1.288948in}{4.085652in}}{\pgfqpoint{1.281134in}{4.093466in}}%
\pgfpathcurveto{\pgfqpoint{1.273321in}{4.101279in}}{\pgfqpoint{1.262722in}{4.105670in}}{\pgfqpoint{1.251672in}{4.105670in}}%
\pgfpathcurveto{\pgfqpoint{1.240621in}{4.105670in}}{\pgfqpoint{1.230022in}{4.101279in}}{\pgfqpoint{1.222209in}{4.093466in}}%
\pgfpathcurveto{\pgfqpoint{1.214395in}{4.085652in}}{\pgfqpoint{1.210005in}{4.075053in}}{\pgfqpoint{1.210005in}{4.064003in}}%
\pgfpathcurveto{\pgfqpoint{1.210005in}{4.052953in}}{\pgfqpoint{1.214395in}{4.042354in}}{\pgfqpoint{1.222209in}{4.034540in}}%
\pgfpathcurveto{\pgfqpoint{1.230022in}{4.026727in}}{\pgfqpoint{1.240621in}{4.022336in}}{\pgfqpoint{1.251672in}{4.022336in}}%
\pgfpathclose%
\pgfusepath{stroke,fill}%
\end{pgfscope}%
\begin{pgfscope}%
\pgfpathrectangle{\pgfqpoint{0.481978in}{0.331635in}}{\pgfqpoint{9.300000in}{7.700000in}}%
\pgfusepath{clip}%
\pgfsetbuttcap%
\pgfsetroundjoin%
\definecolor{currentfill}{rgb}{1.000000,0.705882,0.509804}%
\pgfsetfillcolor{currentfill}%
\pgfsetlinewidth{0.481800pt}%
\definecolor{currentstroke}{rgb}{1.000000,1.000000,1.000000}%
\pgfsetstrokecolor{currentstroke}%
\pgfsetdash{}{0pt}%
\pgfpathmoveto{\pgfqpoint{1.875951in}{4.156394in}}%
\pgfpathcurveto{\pgfqpoint{1.887001in}{4.156394in}}{\pgfqpoint{1.897601in}{4.160785in}}{\pgfqpoint{1.905414in}{4.168598in}}%
\pgfpathcurveto{\pgfqpoint{1.913228in}{4.176412in}}{\pgfqpoint{1.917618in}{4.187011in}}{\pgfqpoint{1.917618in}{4.198061in}}%
\pgfpathcurveto{\pgfqpoint{1.917618in}{4.209111in}}{\pgfqpoint{1.913228in}{4.219710in}}{\pgfqpoint{1.905414in}{4.227524in}}%
\pgfpathcurveto{\pgfqpoint{1.897601in}{4.235337in}}{\pgfqpoint{1.887001in}{4.239728in}}{\pgfqpoint{1.875951in}{4.239728in}}%
\pgfpathcurveto{\pgfqpoint{1.864901in}{4.239728in}}{\pgfqpoint{1.854302in}{4.235337in}}{\pgfqpoint{1.846489in}{4.227524in}}%
\pgfpathcurveto{\pgfqpoint{1.838675in}{4.219710in}}{\pgfqpoint{1.834285in}{4.209111in}}{\pgfqpoint{1.834285in}{4.198061in}}%
\pgfpathcurveto{\pgfqpoint{1.834285in}{4.187011in}}{\pgfqpoint{1.838675in}{4.176412in}}{\pgfqpoint{1.846489in}{4.168598in}}%
\pgfpathcurveto{\pgfqpoint{1.854302in}{4.160785in}}{\pgfqpoint{1.864901in}{4.156394in}}{\pgfqpoint{1.875951in}{4.156394in}}%
\pgfpathclose%
\pgfusepath{stroke,fill}%
\end{pgfscope}%
\begin{pgfscope}%
\pgfpathrectangle{\pgfqpoint{0.481978in}{0.331635in}}{\pgfqpoint{9.300000in}{7.700000in}}%
\pgfusepath{clip}%
\pgfsetbuttcap%
\pgfsetroundjoin%
\definecolor{currentfill}{rgb}{1.000000,0.705882,0.509804}%
\pgfsetfillcolor{currentfill}%
\pgfsetlinewidth{0.481800pt}%
\definecolor{currentstroke}{rgb}{1.000000,1.000000,1.000000}%
\pgfsetstrokecolor{currentstroke}%
\pgfsetdash{}{0pt}%
\pgfpathmoveto{\pgfqpoint{2.496147in}{3.949319in}}%
\pgfpathcurveto{\pgfqpoint{2.507197in}{3.949319in}}{\pgfqpoint{2.517796in}{3.953709in}}{\pgfqpoint{2.525609in}{3.961522in}}%
\pgfpathcurveto{\pgfqpoint{2.533423in}{3.969336in}}{\pgfqpoint{2.537813in}{3.979935in}}{\pgfqpoint{2.537813in}{3.990985in}}%
\pgfpathcurveto{\pgfqpoint{2.537813in}{4.002035in}}{\pgfqpoint{2.533423in}{4.012634in}}{\pgfqpoint{2.525609in}{4.020448in}}%
\pgfpathcurveto{\pgfqpoint{2.517796in}{4.028262in}}{\pgfqpoint{2.507197in}{4.032652in}}{\pgfqpoint{2.496147in}{4.032652in}}%
\pgfpathcurveto{\pgfqpoint{2.485097in}{4.032652in}}{\pgfqpoint{2.474498in}{4.028262in}}{\pgfqpoint{2.466684in}{4.020448in}}%
\pgfpathcurveto{\pgfqpoint{2.458870in}{4.012634in}}{\pgfqpoint{2.454480in}{4.002035in}}{\pgfqpoint{2.454480in}{3.990985in}}%
\pgfpathcurveto{\pgfqpoint{2.454480in}{3.979935in}}{\pgfqpoint{2.458870in}{3.969336in}}{\pgfqpoint{2.466684in}{3.961522in}}%
\pgfpathcurveto{\pgfqpoint{2.474498in}{3.953709in}}{\pgfqpoint{2.485097in}{3.949319in}}{\pgfqpoint{2.496147in}{3.949319in}}%
\pgfpathclose%
\pgfusepath{stroke,fill}%
\end{pgfscope}%
\begin{pgfscope}%
\pgfpathrectangle{\pgfqpoint{0.481978in}{0.331635in}}{\pgfqpoint{9.300000in}{7.700000in}}%
\pgfusepath{clip}%
\pgfsetbuttcap%
\pgfsetroundjoin%
\definecolor{currentfill}{rgb}{1.000000,0.705882,0.509804}%
\pgfsetfillcolor{currentfill}%
\pgfsetlinewidth{0.481800pt}%
\definecolor{currentstroke}{rgb}{1.000000,1.000000,1.000000}%
\pgfsetstrokecolor{currentstroke}%
\pgfsetdash{}{0pt}%
\pgfpathmoveto{\pgfqpoint{3.452404in}{6.239137in}}%
\pgfpathcurveto{\pgfqpoint{3.463454in}{6.239137in}}{\pgfqpoint{3.474053in}{6.243527in}}{\pgfqpoint{3.481867in}{6.251341in}}%
\pgfpathcurveto{\pgfqpoint{3.489680in}{6.259155in}}{\pgfqpoint{3.494071in}{6.269754in}}{\pgfqpoint{3.494071in}{6.280804in}}%
\pgfpathcurveto{\pgfqpoint{3.494071in}{6.291854in}}{\pgfqpoint{3.489680in}{6.302453in}}{\pgfqpoint{3.481867in}{6.310267in}}%
\pgfpathcurveto{\pgfqpoint{3.474053in}{6.318080in}}{\pgfqpoint{3.463454in}{6.322470in}}{\pgfqpoint{3.452404in}{6.322470in}}%
\pgfpathcurveto{\pgfqpoint{3.441354in}{6.322470in}}{\pgfqpoint{3.430755in}{6.318080in}}{\pgfqpoint{3.422941in}{6.310267in}}%
\pgfpathcurveto{\pgfqpoint{3.415128in}{6.302453in}}{\pgfqpoint{3.410737in}{6.291854in}}{\pgfqpoint{3.410737in}{6.280804in}}%
\pgfpathcurveto{\pgfqpoint{3.410737in}{6.269754in}}{\pgfqpoint{3.415128in}{6.259155in}}{\pgfqpoint{3.422941in}{6.251341in}}%
\pgfpathcurveto{\pgfqpoint{3.430755in}{6.243527in}}{\pgfqpoint{3.441354in}{6.239137in}}{\pgfqpoint{3.452404in}{6.239137in}}%
\pgfpathclose%
\pgfusepath{stroke,fill}%
\end{pgfscope}%
\begin{pgfscope}%
\pgfpathrectangle{\pgfqpoint{0.481978in}{0.331635in}}{\pgfqpoint{9.300000in}{7.700000in}}%
\pgfusepath{clip}%
\pgfsetbuttcap%
\pgfsetroundjoin%
\definecolor{currentfill}{rgb}{1.000000,0.705882,0.509804}%
\pgfsetfillcolor{currentfill}%
\pgfsetlinewidth{0.481800pt}%
\definecolor{currentstroke}{rgb}{1.000000,1.000000,1.000000}%
\pgfsetstrokecolor{currentstroke}%
\pgfsetdash{}{0pt}%
\pgfpathmoveto{\pgfqpoint{1.475053in}{4.543874in}}%
\pgfpathcurveto{\pgfqpoint{1.486103in}{4.543874in}}{\pgfqpoint{1.496702in}{4.548264in}}{\pgfqpoint{1.504515in}{4.556078in}}%
\pgfpathcurveto{\pgfqpoint{1.512329in}{4.563891in}}{\pgfqpoint{1.516719in}{4.574490in}}{\pgfqpoint{1.516719in}{4.585541in}}%
\pgfpathcurveto{\pgfqpoint{1.516719in}{4.596591in}}{\pgfqpoint{1.512329in}{4.607190in}}{\pgfqpoint{1.504515in}{4.615003in}}%
\pgfpathcurveto{\pgfqpoint{1.496702in}{4.622817in}}{\pgfqpoint{1.486103in}{4.627207in}}{\pgfqpoint{1.475053in}{4.627207in}}%
\pgfpathcurveto{\pgfqpoint{1.464002in}{4.627207in}}{\pgfqpoint{1.453403in}{4.622817in}}{\pgfqpoint{1.445590in}{4.615003in}}%
\pgfpathcurveto{\pgfqpoint{1.437776in}{4.607190in}}{\pgfqpoint{1.433386in}{4.596591in}}{\pgfqpoint{1.433386in}{4.585541in}}%
\pgfpathcurveto{\pgfqpoint{1.433386in}{4.574490in}}{\pgfqpoint{1.437776in}{4.563891in}}{\pgfqpoint{1.445590in}{4.556078in}}%
\pgfpathcurveto{\pgfqpoint{1.453403in}{4.548264in}}{\pgfqpoint{1.464002in}{4.543874in}}{\pgfqpoint{1.475053in}{4.543874in}}%
\pgfpathclose%
\pgfusepath{stroke,fill}%
\end{pgfscope}%
\begin{pgfscope}%
\pgfpathrectangle{\pgfqpoint{0.481978in}{0.331635in}}{\pgfqpoint{9.300000in}{7.700000in}}%
\pgfusepath{clip}%
\pgfsetbuttcap%
\pgfsetroundjoin%
\definecolor{currentfill}{rgb}{1.000000,0.705882,0.509804}%
\pgfsetfillcolor{currentfill}%
\pgfsetlinewidth{0.481800pt}%
\definecolor{currentstroke}{rgb}{1.000000,1.000000,1.000000}%
\pgfsetstrokecolor{currentstroke}%
\pgfsetdash{}{0pt}%
\pgfpathmoveto{\pgfqpoint{2.980194in}{4.461586in}}%
\pgfpathcurveto{\pgfqpoint{2.991244in}{4.461586in}}{\pgfqpoint{3.001843in}{4.465976in}}{\pgfqpoint{3.009657in}{4.473789in}}%
\pgfpathcurveto{\pgfqpoint{3.017470in}{4.481603in}}{\pgfqpoint{3.021861in}{4.492202in}}{\pgfqpoint{3.021861in}{4.503252in}}%
\pgfpathcurveto{\pgfqpoint{3.021861in}{4.514302in}}{\pgfqpoint{3.017470in}{4.524901in}}{\pgfqpoint{3.009657in}{4.532715in}}%
\pgfpathcurveto{\pgfqpoint{3.001843in}{4.540529in}}{\pgfqpoint{2.991244in}{4.544919in}}{\pgfqpoint{2.980194in}{4.544919in}}%
\pgfpathcurveto{\pgfqpoint{2.969144in}{4.544919in}}{\pgfqpoint{2.958545in}{4.540529in}}{\pgfqpoint{2.950731in}{4.532715in}}%
\pgfpathcurveto{\pgfqpoint{2.942917in}{4.524901in}}{\pgfqpoint{2.938527in}{4.514302in}}{\pgfqpoint{2.938527in}{4.503252in}}%
\pgfpathcurveto{\pgfqpoint{2.938527in}{4.492202in}}{\pgfqpoint{2.942917in}{4.481603in}}{\pgfqpoint{2.950731in}{4.473789in}}%
\pgfpathcurveto{\pgfqpoint{2.958545in}{4.465976in}}{\pgfqpoint{2.969144in}{4.461586in}}{\pgfqpoint{2.980194in}{4.461586in}}%
\pgfpathclose%
\pgfusepath{stroke,fill}%
\end{pgfscope}%
\begin{pgfscope}%
\pgfpathrectangle{\pgfqpoint{0.481978in}{0.331635in}}{\pgfqpoint{9.300000in}{7.700000in}}%
\pgfusepath{clip}%
\pgfsetbuttcap%
\pgfsetroundjoin%
\definecolor{currentfill}{rgb}{1.000000,0.705882,0.509804}%
\pgfsetfillcolor{currentfill}%
\pgfsetlinewidth{0.481800pt}%
\definecolor{currentstroke}{rgb}{1.000000,1.000000,1.000000}%
\pgfsetstrokecolor{currentstroke}%
\pgfsetdash{}{0pt}%
\pgfpathmoveto{\pgfqpoint{4.618269in}{4.659204in}}%
\pgfpathcurveto{\pgfqpoint{4.629319in}{4.659204in}}{\pgfqpoint{4.639918in}{4.663594in}}{\pgfqpoint{4.647732in}{4.671408in}}%
\pgfpathcurveto{\pgfqpoint{4.655545in}{4.679222in}}{\pgfqpoint{4.659935in}{4.689821in}}{\pgfqpoint{4.659935in}{4.700871in}}%
\pgfpathcurveto{\pgfqpoint{4.659935in}{4.711921in}}{\pgfqpoint{4.655545in}{4.722520in}}{\pgfqpoint{4.647732in}{4.730334in}}%
\pgfpathcurveto{\pgfqpoint{4.639918in}{4.738147in}}{\pgfqpoint{4.629319in}{4.742537in}}{\pgfqpoint{4.618269in}{4.742537in}}%
\pgfpathcurveto{\pgfqpoint{4.607219in}{4.742537in}}{\pgfqpoint{4.596620in}{4.738147in}}{\pgfqpoint{4.588806in}{4.730334in}}%
\pgfpathcurveto{\pgfqpoint{4.580992in}{4.722520in}}{\pgfqpoint{4.576602in}{4.711921in}}{\pgfqpoint{4.576602in}{4.700871in}}%
\pgfpathcurveto{\pgfqpoint{4.576602in}{4.689821in}}{\pgfqpoint{4.580992in}{4.679222in}}{\pgfqpoint{4.588806in}{4.671408in}}%
\pgfpathcurveto{\pgfqpoint{4.596620in}{4.663594in}}{\pgfqpoint{4.607219in}{4.659204in}}{\pgfqpoint{4.618269in}{4.659204in}}%
\pgfpathclose%
\pgfusepath{stroke,fill}%
\end{pgfscope}%
\begin{pgfscope}%
\pgfpathrectangle{\pgfqpoint{0.481978in}{0.331635in}}{\pgfqpoint{9.300000in}{7.700000in}}%
\pgfusepath{clip}%
\pgfsetbuttcap%
\pgfsetroundjoin%
\definecolor{currentfill}{rgb}{1.000000,0.705882,0.509804}%
\pgfsetfillcolor{currentfill}%
\pgfsetlinewidth{0.481800pt}%
\definecolor{currentstroke}{rgb}{1.000000,1.000000,1.000000}%
\pgfsetstrokecolor{currentstroke}%
\pgfsetdash{}{0pt}%
\pgfpathmoveto{\pgfqpoint{3.693817in}{4.042801in}}%
\pgfpathcurveto{\pgfqpoint{3.704867in}{4.042801in}}{\pgfqpoint{3.715466in}{4.047191in}}{\pgfqpoint{3.723279in}{4.055005in}}%
\pgfpathcurveto{\pgfqpoint{3.731093in}{4.062819in}}{\pgfqpoint{3.735483in}{4.073418in}}{\pgfqpoint{3.735483in}{4.084468in}}%
\pgfpathcurveto{\pgfqpoint{3.735483in}{4.095518in}}{\pgfqpoint{3.731093in}{4.106117in}}{\pgfqpoint{3.723279in}{4.113930in}}%
\pgfpathcurveto{\pgfqpoint{3.715466in}{4.121744in}}{\pgfqpoint{3.704867in}{4.126134in}}{\pgfqpoint{3.693817in}{4.126134in}}%
\pgfpathcurveto{\pgfqpoint{3.682766in}{4.126134in}}{\pgfqpoint{3.672167in}{4.121744in}}{\pgfqpoint{3.664354in}{4.113930in}}%
\pgfpathcurveto{\pgfqpoint{3.656540in}{4.106117in}}{\pgfqpoint{3.652150in}{4.095518in}}{\pgfqpoint{3.652150in}{4.084468in}}%
\pgfpathcurveto{\pgfqpoint{3.652150in}{4.073418in}}{\pgfqpoint{3.656540in}{4.062819in}}{\pgfqpoint{3.664354in}{4.055005in}}%
\pgfpathcurveto{\pgfqpoint{3.672167in}{4.047191in}}{\pgfqpoint{3.682766in}{4.042801in}}{\pgfqpoint{3.693817in}{4.042801in}}%
\pgfpathclose%
\pgfusepath{stroke,fill}%
\end{pgfscope}%
\begin{pgfscope}%
\pgfpathrectangle{\pgfqpoint{0.481978in}{0.331635in}}{\pgfqpoint{9.300000in}{7.700000in}}%
\pgfusepath{clip}%
\pgfsetbuttcap%
\pgfsetroundjoin%
\definecolor{currentfill}{rgb}{1.000000,0.705882,0.509804}%
\pgfsetfillcolor{currentfill}%
\pgfsetlinewidth{0.481800pt}%
\definecolor{currentstroke}{rgb}{1.000000,1.000000,1.000000}%
\pgfsetstrokecolor{currentstroke}%
\pgfsetdash{}{0pt}%
\pgfpathmoveto{\pgfqpoint{4.823817in}{3.092458in}}%
\pgfpathcurveto{\pgfqpoint{4.834868in}{3.092458in}}{\pgfqpoint{4.845467in}{3.096848in}}{\pgfqpoint{4.853280in}{3.104662in}}%
\pgfpathcurveto{\pgfqpoint{4.861094in}{3.112475in}}{\pgfqpoint{4.865484in}{3.123074in}}{\pgfqpoint{4.865484in}{3.134124in}}%
\pgfpathcurveto{\pgfqpoint{4.865484in}{3.145175in}}{\pgfqpoint{4.861094in}{3.155774in}}{\pgfqpoint{4.853280in}{3.163587in}}%
\pgfpathcurveto{\pgfqpoint{4.845467in}{3.171401in}}{\pgfqpoint{4.834868in}{3.175791in}}{\pgfqpoint{4.823817in}{3.175791in}}%
\pgfpathcurveto{\pgfqpoint{4.812767in}{3.175791in}}{\pgfqpoint{4.802168in}{3.171401in}}{\pgfqpoint{4.794355in}{3.163587in}}%
\pgfpathcurveto{\pgfqpoint{4.786541in}{3.155774in}}{\pgfqpoint{4.782151in}{3.145175in}}{\pgfqpoint{4.782151in}{3.134124in}}%
\pgfpathcurveto{\pgfqpoint{4.782151in}{3.123074in}}{\pgfqpoint{4.786541in}{3.112475in}}{\pgfqpoint{4.794355in}{3.104662in}}%
\pgfpathcurveto{\pgfqpoint{4.802168in}{3.096848in}}{\pgfqpoint{4.812767in}{3.092458in}}{\pgfqpoint{4.823817in}{3.092458in}}%
\pgfpathclose%
\pgfusepath{stroke,fill}%
\end{pgfscope}%
\begin{pgfscope}%
\pgfpathrectangle{\pgfqpoint{0.481978in}{0.331635in}}{\pgfqpoint{9.300000in}{7.700000in}}%
\pgfusepath{clip}%
\pgfsetbuttcap%
\pgfsetroundjoin%
\definecolor{currentfill}{rgb}{1.000000,0.705882,0.509804}%
\pgfsetfillcolor{currentfill}%
\pgfsetlinewidth{0.481800pt}%
\definecolor{currentstroke}{rgb}{1.000000,1.000000,1.000000}%
\pgfsetstrokecolor{currentstroke}%
\pgfsetdash{}{0pt}%
\pgfpathmoveto{\pgfqpoint{2.627728in}{3.873639in}}%
\pgfpathcurveto{\pgfqpoint{2.638778in}{3.873639in}}{\pgfqpoint{2.649377in}{3.878029in}}{\pgfqpoint{2.657191in}{3.885843in}}%
\pgfpathcurveto{\pgfqpoint{2.665004in}{3.893656in}}{\pgfqpoint{2.669395in}{3.904255in}}{\pgfqpoint{2.669395in}{3.915305in}}%
\pgfpathcurveto{\pgfqpoint{2.669395in}{3.926356in}}{\pgfqpoint{2.665004in}{3.936955in}}{\pgfqpoint{2.657191in}{3.944768in}}%
\pgfpathcurveto{\pgfqpoint{2.649377in}{3.952582in}}{\pgfqpoint{2.638778in}{3.956972in}}{\pgfqpoint{2.627728in}{3.956972in}}%
\pgfpathcurveto{\pgfqpoint{2.616678in}{3.956972in}}{\pgfqpoint{2.606079in}{3.952582in}}{\pgfqpoint{2.598265in}{3.944768in}}%
\pgfpathcurveto{\pgfqpoint{2.590452in}{3.936955in}}{\pgfqpoint{2.586061in}{3.926356in}}{\pgfqpoint{2.586061in}{3.915305in}}%
\pgfpathcurveto{\pgfqpoint{2.586061in}{3.904255in}}{\pgfqpoint{2.590452in}{3.893656in}}{\pgfqpoint{2.598265in}{3.885843in}}%
\pgfpathcurveto{\pgfqpoint{2.606079in}{3.878029in}}{\pgfqpoint{2.616678in}{3.873639in}}{\pgfqpoint{2.627728in}{3.873639in}}%
\pgfpathclose%
\pgfusepath{stroke,fill}%
\end{pgfscope}%
\begin{pgfscope}%
\pgfpathrectangle{\pgfqpoint{0.481978in}{0.331635in}}{\pgfqpoint{9.300000in}{7.700000in}}%
\pgfusepath{clip}%
\pgfsetbuttcap%
\pgfsetroundjoin%
\definecolor{currentfill}{rgb}{1.000000,0.705882,0.509804}%
\pgfsetfillcolor{currentfill}%
\pgfsetlinewidth{0.481800pt}%
\definecolor{currentstroke}{rgb}{1.000000,1.000000,1.000000}%
\pgfsetstrokecolor{currentstroke}%
\pgfsetdash{}{0pt}%
\pgfpathmoveto{\pgfqpoint{2.782878in}{3.328820in}}%
\pgfpathcurveto{\pgfqpoint{2.793928in}{3.328820in}}{\pgfqpoint{2.804527in}{3.333210in}}{\pgfqpoint{2.812341in}{3.341024in}}%
\pgfpathcurveto{\pgfqpoint{2.820155in}{3.348837in}}{\pgfqpoint{2.824545in}{3.359436in}}{\pgfqpoint{2.824545in}{3.370487in}}%
\pgfpathcurveto{\pgfqpoint{2.824545in}{3.381537in}}{\pgfqpoint{2.820155in}{3.392136in}}{\pgfqpoint{2.812341in}{3.399949in}}%
\pgfpathcurveto{\pgfqpoint{2.804527in}{3.407763in}}{\pgfqpoint{2.793928in}{3.412153in}}{\pgfqpoint{2.782878in}{3.412153in}}%
\pgfpathcurveto{\pgfqpoint{2.771828in}{3.412153in}}{\pgfqpoint{2.761229in}{3.407763in}}{\pgfqpoint{2.753415in}{3.399949in}}%
\pgfpathcurveto{\pgfqpoint{2.745602in}{3.392136in}}{\pgfqpoint{2.741212in}{3.381537in}}{\pgfqpoint{2.741212in}{3.370487in}}%
\pgfpathcurveto{\pgfqpoint{2.741212in}{3.359436in}}{\pgfqpoint{2.745602in}{3.348837in}}{\pgfqpoint{2.753415in}{3.341024in}}%
\pgfpathcurveto{\pgfqpoint{2.761229in}{3.333210in}}{\pgfqpoint{2.771828in}{3.328820in}}{\pgfqpoint{2.782878in}{3.328820in}}%
\pgfpathclose%
\pgfusepath{stroke,fill}%
\end{pgfscope}%
\begin{pgfscope}%
\pgfpathrectangle{\pgfqpoint{0.481978in}{0.331635in}}{\pgfqpoint{9.300000in}{7.700000in}}%
\pgfusepath{clip}%
\pgfsetbuttcap%
\pgfsetroundjoin%
\definecolor{currentfill}{rgb}{1.000000,0.705882,0.509804}%
\pgfsetfillcolor{currentfill}%
\pgfsetlinewidth{0.481800pt}%
\definecolor{currentstroke}{rgb}{1.000000,1.000000,1.000000}%
\pgfsetstrokecolor{currentstroke}%
\pgfsetdash{}{0pt}%
\pgfpathmoveto{\pgfqpoint{3.755167in}{4.840302in}}%
\pgfpathcurveto{\pgfqpoint{3.766217in}{4.840302in}}{\pgfqpoint{3.776816in}{4.844693in}}{\pgfqpoint{3.784630in}{4.852506in}}%
\pgfpathcurveto{\pgfqpoint{3.792443in}{4.860320in}}{\pgfqpoint{3.796834in}{4.870919in}}{\pgfqpoint{3.796834in}{4.881969in}}%
\pgfpathcurveto{\pgfqpoint{3.796834in}{4.893019in}}{\pgfqpoint{3.792443in}{4.903618in}}{\pgfqpoint{3.784630in}{4.911432in}}%
\pgfpathcurveto{\pgfqpoint{3.776816in}{4.919245in}}{\pgfqpoint{3.766217in}{4.923636in}}{\pgfqpoint{3.755167in}{4.923636in}}%
\pgfpathcurveto{\pgfqpoint{3.744117in}{4.923636in}}{\pgfqpoint{3.733518in}{4.919245in}}{\pgfqpoint{3.725704in}{4.911432in}}%
\pgfpathcurveto{\pgfqpoint{3.717891in}{4.903618in}}{\pgfqpoint{3.713500in}{4.893019in}}{\pgfqpoint{3.713500in}{4.881969in}}%
\pgfpathcurveto{\pgfqpoint{3.713500in}{4.870919in}}{\pgfqpoint{3.717891in}{4.860320in}}{\pgfqpoint{3.725704in}{4.852506in}}%
\pgfpathcurveto{\pgfqpoint{3.733518in}{4.844693in}}{\pgfqpoint{3.744117in}{4.840302in}}{\pgfqpoint{3.755167in}{4.840302in}}%
\pgfpathclose%
\pgfusepath{stroke,fill}%
\end{pgfscope}%
\begin{pgfscope}%
\pgfpathrectangle{\pgfqpoint{0.481978in}{0.331635in}}{\pgfqpoint{9.300000in}{7.700000in}}%
\pgfusepath{clip}%
\pgfsetbuttcap%
\pgfsetroundjoin%
\definecolor{currentfill}{rgb}{1.000000,0.705882,0.509804}%
\pgfsetfillcolor{currentfill}%
\pgfsetlinewidth{0.481800pt}%
\definecolor{currentstroke}{rgb}{1.000000,1.000000,1.000000}%
\pgfsetstrokecolor{currentstroke}%
\pgfsetdash{}{0pt}%
\pgfpathmoveto{\pgfqpoint{4.616665in}{2.856244in}}%
\pgfpathcurveto{\pgfqpoint{4.627715in}{2.856244in}}{\pgfqpoint{4.638314in}{2.860634in}}{\pgfqpoint{4.646128in}{2.868448in}}%
\pgfpathcurveto{\pgfqpoint{4.653941in}{2.876261in}}{\pgfqpoint{4.658332in}{2.886860in}}{\pgfqpoint{4.658332in}{2.897911in}}%
\pgfpathcurveto{\pgfqpoint{4.658332in}{2.908961in}}{\pgfqpoint{4.653941in}{2.919560in}}{\pgfqpoint{4.646128in}{2.927373in}}%
\pgfpathcurveto{\pgfqpoint{4.638314in}{2.935187in}}{\pgfqpoint{4.627715in}{2.939577in}}{\pgfqpoint{4.616665in}{2.939577in}}%
\pgfpathcurveto{\pgfqpoint{4.605615in}{2.939577in}}{\pgfqpoint{4.595016in}{2.935187in}}{\pgfqpoint{4.587202in}{2.927373in}}%
\pgfpathcurveto{\pgfqpoint{4.579389in}{2.919560in}}{\pgfqpoint{4.574998in}{2.908961in}}{\pgfqpoint{4.574998in}{2.897911in}}%
\pgfpathcurveto{\pgfqpoint{4.574998in}{2.886860in}}{\pgfqpoint{4.579389in}{2.876261in}}{\pgfqpoint{4.587202in}{2.868448in}}%
\pgfpathcurveto{\pgfqpoint{4.595016in}{2.860634in}}{\pgfqpoint{4.605615in}{2.856244in}}{\pgfqpoint{4.616665in}{2.856244in}}%
\pgfpathclose%
\pgfusepath{stroke,fill}%
\end{pgfscope}%
\begin{pgfscope}%
\pgfpathrectangle{\pgfqpoint{0.481978in}{0.331635in}}{\pgfqpoint{9.300000in}{7.700000in}}%
\pgfusepath{clip}%
\pgfsetbuttcap%
\pgfsetroundjoin%
\definecolor{currentfill}{rgb}{1.000000,0.705882,0.509804}%
\pgfsetfillcolor{currentfill}%
\pgfsetlinewidth{0.481800pt}%
\definecolor{currentstroke}{rgb}{1.000000,1.000000,1.000000}%
\pgfsetstrokecolor{currentstroke}%
\pgfsetdash{}{0pt}%
\pgfpathmoveto{\pgfqpoint{2.176301in}{2.880759in}}%
\pgfpathcurveto{\pgfqpoint{2.187351in}{2.880759in}}{\pgfqpoint{2.197950in}{2.885149in}}{\pgfqpoint{2.205763in}{2.892963in}}%
\pgfpathcurveto{\pgfqpoint{2.213577in}{2.900777in}}{\pgfqpoint{2.217967in}{2.911376in}}{\pgfqpoint{2.217967in}{2.922426in}}%
\pgfpathcurveto{\pgfqpoint{2.217967in}{2.933476in}}{\pgfqpoint{2.213577in}{2.944075in}}{\pgfqpoint{2.205763in}{2.951889in}}%
\pgfpathcurveto{\pgfqpoint{2.197950in}{2.959702in}}{\pgfqpoint{2.187351in}{2.964092in}}{\pgfqpoint{2.176301in}{2.964092in}}%
\pgfpathcurveto{\pgfqpoint{2.165250in}{2.964092in}}{\pgfqpoint{2.154651in}{2.959702in}}{\pgfqpoint{2.146838in}{2.951889in}}%
\pgfpathcurveto{\pgfqpoint{2.139024in}{2.944075in}}{\pgfqpoint{2.134634in}{2.933476in}}{\pgfqpoint{2.134634in}{2.922426in}}%
\pgfpathcurveto{\pgfqpoint{2.134634in}{2.911376in}}{\pgfqpoint{2.139024in}{2.900777in}}{\pgfqpoint{2.146838in}{2.892963in}}%
\pgfpathcurveto{\pgfqpoint{2.154651in}{2.885149in}}{\pgfqpoint{2.165250in}{2.880759in}}{\pgfqpoint{2.176301in}{2.880759in}}%
\pgfpathclose%
\pgfusepath{stroke,fill}%
\end{pgfscope}%
\begin{pgfscope}%
\pgfpathrectangle{\pgfqpoint{0.481978in}{0.331635in}}{\pgfqpoint{9.300000in}{7.700000in}}%
\pgfusepath{clip}%
\pgfsetbuttcap%
\pgfsetroundjoin%
\definecolor{currentfill}{rgb}{1.000000,0.705882,0.509804}%
\pgfsetfillcolor{currentfill}%
\pgfsetlinewidth{0.481800pt}%
\definecolor{currentstroke}{rgb}{1.000000,1.000000,1.000000}%
\pgfsetstrokecolor{currentstroke}%
\pgfsetdash{}{0pt}%
\pgfpathmoveto{\pgfqpoint{4.459267in}{4.545856in}}%
\pgfpathcurveto{\pgfqpoint{4.470317in}{4.545856in}}{\pgfqpoint{4.480916in}{4.550246in}}{\pgfqpoint{4.488730in}{4.558060in}}%
\pgfpathcurveto{\pgfqpoint{4.496543in}{4.565874in}}{\pgfqpoint{4.500934in}{4.576473in}}{\pgfqpoint{4.500934in}{4.587523in}}%
\pgfpathcurveto{\pgfqpoint{4.500934in}{4.598573in}}{\pgfqpoint{4.496543in}{4.609172in}}{\pgfqpoint{4.488730in}{4.616986in}}%
\pgfpathcurveto{\pgfqpoint{4.480916in}{4.624799in}}{\pgfqpoint{4.470317in}{4.629190in}}{\pgfqpoint{4.459267in}{4.629190in}}%
\pgfpathcurveto{\pgfqpoint{4.448217in}{4.629190in}}{\pgfqpoint{4.437618in}{4.624799in}}{\pgfqpoint{4.429804in}{4.616986in}}%
\pgfpathcurveto{\pgfqpoint{4.421991in}{4.609172in}}{\pgfqpoint{4.417600in}{4.598573in}}{\pgfqpoint{4.417600in}{4.587523in}}%
\pgfpathcurveto{\pgfqpoint{4.417600in}{4.576473in}}{\pgfqpoint{4.421991in}{4.565874in}}{\pgfqpoint{4.429804in}{4.558060in}}%
\pgfpathcurveto{\pgfqpoint{4.437618in}{4.550246in}}{\pgfqpoint{4.448217in}{4.545856in}}{\pgfqpoint{4.459267in}{4.545856in}}%
\pgfpathclose%
\pgfusepath{stroke,fill}%
\end{pgfscope}%
\begin{pgfscope}%
\pgfpathrectangle{\pgfqpoint{0.481978in}{0.331635in}}{\pgfqpoint{9.300000in}{7.700000in}}%
\pgfusepath{clip}%
\pgfsetbuttcap%
\pgfsetroundjoin%
\definecolor{currentfill}{rgb}{1.000000,0.705882,0.509804}%
\pgfsetfillcolor{currentfill}%
\pgfsetlinewidth{0.481800pt}%
\definecolor{currentstroke}{rgb}{1.000000,1.000000,1.000000}%
\pgfsetstrokecolor{currentstroke}%
\pgfsetdash{}{0pt}%
\pgfpathmoveto{\pgfqpoint{4.335059in}{2.978217in}}%
\pgfpathcurveto{\pgfqpoint{4.346109in}{2.978217in}}{\pgfqpoint{4.356708in}{2.982607in}}{\pgfqpoint{4.364522in}{2.990420in}}%
\pgfpathcurveto{\pgfqpoint{4.372336in}{2.998234in}}{\pgfqpoint{4.376726in}{3.008833in}}{\pgfqpoint{4.376726in}{3.019883in}}%
\pgfpathcurveto{\pgfqpoint{4.376726in}{3.030933in}}{\pgfqpoint{4.372336in}{3.041532in}}{\pgfqpoint{4.364522in}{3.049346in}}%
\pgfpathcurveto{\pgfqpoint{4.356708in}{3.057160in}}{\pgfqpoint{4.346109in}{3.061550in}}{\pgfqpoint{4.335059in}{3.061550in}}%
\pgfpathcurveto{\pgfqpoint{4.324009in}{3.061550in}}{\pgfqpoint{4.313410in}{3.057160in}}{\pgfqpoint{4.305597in}{3.049346in}}%
\pgfpathcurveto{\pgfqpoint{4.297783in}{3.041532in}}{\pgfqpoint{4.293393in}{3.030933in}}{\pgfqpoint{4.293393in}{3.019883in}}%
\pgfpathcurveto{\pgfqpoint{4.293393in}{3.008833in}}{\pgfqpoint{4.297783in}{2.998234in}}{\pgfqpoint{4.305597in}{2.990420in}}%
\pgfpathcurveto{\pgfqpoint{4.313410in}{2.982607in}}{\pgfqpoint{4.324009in}{2.978217in}}{\pgfqpoint{4.335059in}{2.978217in}}%
\pgfpathclose%
\pgfusepath{stroke,fill}%
\end{pgfscope}%
\begin{pgfscope}%
\pgfpathrectangle{\pgfqpoint{0.481978in}{0.331635in}}{\pgfqpoint{9.300000in}{7.700000in}}%
\pgfusepath{clip}%
\pgfsetbuttcap%
\pgfsetroundjoin%
\definecolor{currentfill}{rgb}{1.000000,0.705882,0.509804}%
\pgfsetfillcolor{currentfill}%
\pgfsetlinewidth{0.481800pt}%
\definecolor{currentstroke}{rgb}{1.000000,1.000000,1.000000}%
\pgfsetstrokecolor{currentstroke}%
\pgfsetdash{}{0pt}%
\pgfpathmoveto{\pgfqpoint{4.103001in}{2.947748in}}%
\pgfpathcurveto{\pgfqpoint{4.114051in}{2.947748in}}{\pgfqpoint{4.124650in}{2.952138in}}{\pgfqpoint{4.132463in}{2.959952in}}%
\pgfpathcurveto{\pgfqpoint{4.140277in}{2.967766in}}{\pgfqpoint{4.144667in}{2.978365in}}{\pgfqpoint{4.144667in}{2.989415in}}%
\pgfpathcurveto{\pgfqpoint{4.144667in}{3.000465in}}{\pgfqpoint{4.140277in}{3.011064in}}{\pgfqpoint{4.132463in}{3.018878in}}%
\pgfpathcurveto{\pgfqpoint{4.124650in}{3.026691in}}{\pgfqpoint{4.114051in}{3.031082in}}{\pgfqpoint{4.103001in}{3.031082in}}%
\pgfpathcurveto{\pgfqpoint{4.091950in}{3.031082in}}{\pgfqpoint{4.081351in}{3.026691in}}{\pgfqpoint{4.073538in}{3.018878in}}%
\pgfpathcurveto{\pgfqpoint{4.065724in}{3.011064in}}{\pgfqpoint{4.061334in}{3.000465in}}{\pgfqpoint{4.061334in}{2.989415in}}%
\pgfpathcurveto{\pgfqpoint{4.061334in}{2.978365in}}{\pgfqpoint{4.065724in}{2.967766in}}{\pgfqpoint{4.073538in}{2.959952in}}%
\pgfpathcurveto{\pgfqpoint{4.081351in}{2.952138in}}{\pgfqpoint{4.091950in}{2.947748in}}{\pgfqpoint{4.103001in}{2.947748in}}%
\pgfpathclose%
\pgfusepath{stroke,fill}%
\end{pgfscope}%
\begin{pgfscope}%
\pgfpathrectangle{\pgfqpoint{0.481978in}{0.331635in}}{\pgfqpoint{9.300000in}{7.700000in}}%
\pgfusepath{clip}%
\pgfsetbuttcap%
\pgfsetroundjoin%
\definecolor{currentfill}{rgb}{1.000000,0.705882,0.509804}%
\pgfsetfillcolor{currentfill}%
\pgfsetlinewidth{0.481800pt}%
\definecolor{currentstroke}{rgb}{1.000000,1.000000,1.000000}%
\pgfsetstrokecolor{currentstroke}%
\pgfsetdash{}{0pt}%
\pgfpathmoveto{\pgfqpoint{1.240028in}{3.747940in}}%
\pgfpathcurveto{\pgfqpoint{1.251078in}{3.747940in}}{\pgfqpoint{1.261677in}{3.752330in}}{\pgfqpoint{1.269490in}{3.760144in}}%
\pgfpathcurveto{\pgfqpoint{1.277304in}{3.767957in}}{\pgfqpoint{1.281694in}{3.778556in}}{\pgfqpoint{1.281694in}{3.789606in}}%
\pgfpathcurveto{\pgfqpoint{1.281694in}{3.800657in}}{\pgfqpoint{1.277304in}{3.811256in}}{\pgfqpoint{1.269490in}{3.819069in}}%
\pgfpathcurveto{\pgfqpoint{1.261677in}{3.826883in}}{\pgfqpoint{1.251078in}{3.831273in}}{\pgfqpoint{1.240028in}{3.831273in}}%
\pgfpathcurveto{\pgfqpoint{1.228977in}{3.831273in}}{\pgfqpoint{1.218378in}{3.826883in}}{\pgfqpoint{1.210565in}{3.819069in}}%
\pgfpathcurveto{\pgfqpoint{1.202751in}{3.811256in}}{\pgfqpoint{1.198361in}{3.800657in}}{\pgfqpoint{1.198361in}{3.789606in}}%
\pgfpathcurveto{\pgfqpoint{1.198361in}{3.778556in}}{\pgfqpoint{1.202751in}{3.767957in}}{\pgfqpoint{1.210565in}{3.760144in}}%
\pgfpathcurveto{\pgfqpoint{1.218378in}{3.752330in}}{\pgfqpoint{1.228977in}{3.747940in}}{\pgfqpoint{1.240028in}{3.747940in}}%
\pgfpathclose%
\pgfusepath{stroke,fill}%
\end{pgfscope}%
\begin{pgfscope}%
\pgfpathrectangle{\pgfqpoint{0.481978in}{0.331635in}}{\pgfqpoint{9.300000in}{7.700000in}}%
\pgfusepath{clip}%
\pgfsetbuttcap%
\pgfsetroundjoin%
\definecolor{currentfill}{rgb}{1.000000,0.705882,0.509804}%
\pgfsetfillcolor{currentfill}%
\pgfsetlinewidth{0.481800pt}%
\definecolor{currentstroke}{rgb}{1.000000,1.000000,1.000000}%
\pgfsetstrokecolor{currentstroke}%
\pgfsetdash{}{0pt}%
\pgfpathmoveto{\pgfqpoint{3.665627in}{3.670075in}}%
\pgfpathcurveto{\pgfqpoint{3.676677in}{3.670075in}}{\pgfqpoint{3.687276in}{3.674466in}}{\pgfqpoint{3.695090in}{3.682279in}}%
\pgfpathcurveto{\pgfqpoint{3.702904in}{3.690093in}}{\pgfqpoint{3.707294in}{3.700692in}}{\pgfqpoint{3.707294in}{3.711742in}}%
\pgfpathcurveto{\pgfqpoint{3.707294in}{3.722792in}}{\pgfqpoint{3.702904in}{3.733391in}}{\pgfqpoint{3.695090in}{3.741205in}}%
\pgfpathcurveto{\pgfqpoint{3.687276in}{3.749018in}}{\pgfqpoint{3.676677in}{3.753409in}}{\pgfqpoint{3.665627in}{3.753409in}}%
\pgfpathcurveto{\pgfqpoint{3.654577in}{3.753409in}}{\pgfqpoint{3.643978in}{3.749018in}}{\pgfqpoint{3.636164in}{3.741205in}}%
\pgfpathcurveto{\pgfqpoint{3.628351in}{3.733391in}}{\pgfqpoint{3.623961in}{3.722792in}}{\pgfqpoint{3.623961in}{3.711742in}}%
\pgfpathcurveto{\pgfqpoint{3.623961in}{3.700692in}}{\pgfqpoint{3.628351in}{3.690093in}}{\pgfqpoint{3.636164in}{3.682279in}}%
\pgfpathcurveto{\pgfqpoint{3.643978in}{3.674466in}}{\pgfqpoint{3.654577in}{3.670075in}}{\pgfqpoint{3.665627in}{3.670075in}}%
\pgfpathclose%
\pgfusepath{stroke,fill}%
\end{pgfscope}%
\begin{pgfscope}%
\pgfpathrectangle{\pgfqpoint{0.481978in}{0.331635in}}{\pgfqpoint{9.300000in}{7.700000in}}%
\pgfusepath{clip}%
\pgfsetbuttcap%
\pgfsetroundjoin%
\definecolor{currentfill}{rgb}{1.000000,0.705882,0.509804}%
\pgfsetfillcolor{currentfill}%
\pgfsetlinewidth{0.481800pt}%
\definecolor{currentstroke}{rgb}{1.000000,1.000000,1.000000}%
\pgfsetstrokecolor{currentstroke}%
\pgfsetdash{}{0pt}%
\pgfpathmoveto{\pgfqpoint{3.637232in}{4.380124in}}%
\pgfpathcurveto{\pgfqpoint{3.648282in}{4.380124in}}{\pgfqpoint{3.658881in}{4.384514in}}{\pgfqpoint{3.666694in}{4.392328in}}%
\pgfpathcurveto{\pgfqpoint{3.674508in}{4.400142in}}{\pgfqpoint{3.678898in}{4.410741in}}{\pgfqpoint{3.678898in}{4.421791in}}%
\pgfpathcurveto{\pgfqpoint{3.678898in}{4.432841in}}{\pgfqpoint{3.674508in}{4.443440in}}{\pgfqpoint{3.666694in}{4.451253in}}%
\pgfpathcurveto{\pgfqpoint{3.658881in}{4.459067in}}{\pgfqpoint{3.648282in}{4.463457in}}{\pgfqpoint{3.637232in}{4.463457in}}%
\pgfpathcurveto{\pgfqpoint{3.626181in}{4.463457in}}{\pgfqpoint{3.615582in}{4.459067in}}{\pgfqpoint{3.607769in}{4.451253in}}%
\pgfpathcurveto{\pgfqpoint{3.599955in}{4.443440in}}{\pgfqpoint{3.595565in}{4.432841in}}{\pgfqpoint{3.595565in}{4.421791in}}%
\pgfpathcurveto{\pgfqpoint{3.595565in}{4.410741in}}{\pgfqpoint{3.599955in}{4.400142in}}{\pgfqpoint{3.607769in}{4.392328in}}%
\pgfpathcurveto{\pgfqpoint{3.615582in}{4.384514in}}{\pgfqpoint{3.626181in}{4.380124in}}{\pgfqpoint{3.637232in}{4.380124in}}%
\pgfpathclose%
\pgfusepath{stroke,fill}%
\end{pgfscope}%
\begin{pgfscope}%
\pgfpathrectangle{\pgfqpoint{0.481978in}{0.331635in}}{\pgfqpoint{9.300000in}{7.700000in}}%
\pgfusepath{clip}%
\pgfsetbuttcap%
\pgfsetroundjoin%
\definecolor{currentfill}{rgb}{1.000000,0.705882,0.509804}%
\pgfsetfillcolor{currentfill}%
\pgfsetlinewidth{0.481800pt}%
\definecolor{currentstroke}{rgb}{1.000000,1.000000,1.000000}%
\pgfsetstrokecolor{currentstroke}%
\pgfsetdash{}{0pt}%
\pgfpathmoveto{\pgfqpoint{1.338259in}{3.701359in}}%
\pgfpathcurveto{\pgfqpoint{1.349309in}{3.701359in}}{\pgfqpoint{1.359908in}{3.705749in}}{\pgfqpoint{1.367722in}{3.713562in}}%
\pgfpathcurveto{\pgfqpoint{1.375535in}{3.721376in}}{\pgfqpoint{1.379926in}{3.731975in}}{\pgfqpoint{1.379926in}{3.743025in}}%
\pgfpathcurveto{\pgfqpoint{1.379926in}{3.754075in}}{\pgfqpoint{1.375535in}{3.764674in}}{\pgfqpoint{1.367722in}{3.772488in}}%
\pgfpathcurveto{\pgfqpoint{1.359908in}{3.780302in}}{\pgfqpoint{1.349309in}{3.784692in}}{\pgfqpoint{1.338259in}{3.784692in}}%
\pgfpathcurveto{\pgfqpoint{1.327209in}{3.784692in}}{\pgfqpoint{1.316610in}{3.780302in}}{\pgfqpoint{1.308796in}{3.772488in}}%
\pgfpathcurveto{\pgfqpoint{1.300983in}{3.764674in}}{\pgfqpoint{1.296592in}{3.754075in}}{\pgfqpoint{1.296592in}{3.743025in}}%
\pgfpathcurveto{\pgfqpoint{1.296592in}{3.731975in}}{\pgfqpoint{1.300983in}{3.721376in}}{\pgfqpoint{1.308796in}{3.713562in}}%
\pgfpathcurveto{\pgfqpoint{1.316610in}{3.705749in}}{\pgfqpoint{1.327209in}{3.701359in}}{\pgfqpoint{1.338259in}{3.701359in}}%
\pgfpathclose%
\pgfusepath{stroke,fill}%
\end{pgfscope}%
\begin{pgfscope}%
\pgfpathrectangle{\pgfqpoint{0.481978in}{0.331635in}}{\pgfqpoint{9.300000in}{7.700000in}}%
\pgfusepath{clip}%
\pgfsetbuttcap%
\pgfsetroundjoin%
\definecolor{currentfill}{rgb}{1.000000,0.705882,0.509804}%
\pgfsetfillcolor{currentfill}%
\pgfsetlinewidth{0.481800pt}%
\definecolor{currentstroke}{rgb}{1.000000,1.000000,1.000000}%
\pgfsetstrokecolor{currentstroke}%
\pgfsetdash{}{0pt}%
\pgfpathmoveto{\pgfqpoint{1.995054in}{2.990321in}}%
\pgfpathcurveto{\pgfqpoint{2.006104in}{2.990321in}}{\pgfqpoint{2.016703in}{2.994711in}}{\pgfqpoint{2.024516in}{3.002525in}}%
\pgfpathcurveto{\pgfqpoint{2.032330in}{3.010338in}}{\pgfqpoint{2.036720in}{3.020937in}}{\pgfqpoint{2.036720in}{3.031987in}}%
\pgfpathcurveto{\pgfqpoint{2.036720in}{3.043038in}}{\pgfqpoint{2.032330in}{3.053637in}}{\pgfqpoint{2.024516in}{3.061450in}}%
\pgfpathcurveto{\pgfqpoint{2.016703in}{3.069264in}}{\pgfqpoint{2.006104in}{3.073654in}}{\pgfqpoint{1.995054in}{3.073654in}}%
\pgfpathcurveto{\pgfqpoint{1.984003in}{3.073654in}}{\pgfqpoint{1.973404in}{3.069264in}}{\pgfqpoint{1.965591in}{3.061450in}}%
\pgfpathcurveto{\pgfqpoint{1.957777in}{3.053637in}}{\pgfqpoint{1.953387in}{3.043038in}}{\pgfqpoint{1.953387in}{3.031987in}}%
\pgfpathcurveto{\pgfqpoint{1.953387in}{3.020937in}}{\pgfqpoint{1.957777in}{3.010338in}}{\pgfqpoint{1.965591in}{3.002525in}}%
\pgfpathcurveto{\pgfqpoint{1.973404in}{2.994711in}}{\pgfqpoint{1.984003in}{2.990321in}}{\pgfqpoint{1.995054in}{2.990321in}}%
\pgfpathclose%
\pgfusepath{stroke,fill}%
\end{pgfscope}%
\begin{pgfscope}%
\pgfpathrectangle{\pgfqpoint{0.481978in}{0.331635in}}{\pgfqpoint{9.300000in}{7.700000in}}%
\pgfusepath{clip}%
\pgfsetbuttcap%
\pgfsetroundjoin%
\definecolor{currentfill}{rgb}{1.000000,0.705882,0.509804}%
\pgfsetfillcolor{currentfill}%
\pgfsetlinewidth{0.481800pt}%
\definecolor{currentstroke}{rgb}{1.000000,1.000000,1.000000}%
\pgfsetstrokecolor{currentstroke}%
\pgfsetdash{}{0pt}%
\pgfpathmoveto{\pgfqpoint{1.350306in}{3.961584in}}%
\pgfpathcurveto{\pgfqpoint{1.361356in}{3.961584in}}{\pgfqpoint{1.371955in}{3.965974in}}{\pgfqpoint{1.379769in}{3.973788in}}%
\pgfpathcurveto{\pgfqpoint{1.387582in}{3.981601in}}{\pgfqpoint{1.391973in}{3.992200in}}{\pgfqpoint{1.391973in}{4.003250in}}%
\pgfpathcurveto{\pgfqpoint{1.391973in}{4.014301in}}{\pgfqpoint{1.387582in}{4.024900in}}{\pgfqpoint{1.379769in}{4.032713in}}%
\pgfpathcurveto{\pgfqpoint{1.371955in}{4.040527in}}{\pgfqpoint{1.361356in}{4.044917in}}{\pgfqpoint{1.350306in}{4.044917in}}%
\pgfpathcurveto{\pgfqpoint{1.339256in}{4.044917in}}{\pgfqpoint{1.328657in}{4.040527in}}{\pgfqpoint{1.320843in}{4.032713in}}%
\pgfpathcurveto{\pgfqpoint{1.313029in}{4.024900in}}{\pgfqpoint{1.308639in}{4.014301in}}{\pgfqpoint{1.308639in}{4.003250in}}%
\pgfpathcurveto{\pgfqpoint{1.308639in}{3.992200in}}{\pgfqpoint{1.313029in}{3.981601in}}{\pgfqpoint{1.320843in}{3.973788in}}%
\pgfpathcurveto{\pgfqpoint{1.328657in}{3.965974in}}{\pgfqpoint{1.339256in}{3.961584in}}{\pgfqpoint{1.350306in}{3.961584in}}%
\pgfpathclose%
\pgfusepath{stroke,fill}%
\end{pgfscope}%
\begin{pgfscope}%
\pgfpathrectangle{\pgfqpoint{0.481978in}{0.331635in}}{\pgfqpoint{9.300000in}{7.700000in}}%
\pgfusepath{clip}%
\pgfsetbuttcap%
\pgfsetroundjoin%
\definecolor{currentfill}{rgb}{1.000000,0.705882,0.509804}%
\pgfsetfillcolor{currentfill}%
\pgfsetlinewidth{0.481800pt}%
\definecolor{currentstroke}{rgb}{1.000000,1.000000,1.000000}%
\pgfsetstrokecolor{currentstroke}%
\pgfsetdash{}{0pt}%
\pgfpathmoveto{\pgfqpoint{1.701822in}{4.024914in}}%
\pgfpathcurveto{\pgfqpoint{1.712872in}{4.024914in}}{\pgfqpoint{1.723471in}{4.029304in}}{\pgfqpoint{1.731284in}{4.037117in}}%
\pgfpathcurveto{\pgfqpoint{1.739098in}{4.044931in}}{\pgfqpoint{1.743488in}{4.055530in}}{\pgfqpoint{1.743488in}{4.066580in}}%
\pgfpathcurveto{\pgfqpoint{1.743488in}{4.077630in}}{\pgfqpoint{1.739098in}{4.088229in}}{\pgfqpoint{1.731284in}{4.096043in}}%
\pgfpathcurveto{\pgfqpoint{1.723471in}{4.103857in}}{\pgfqpoint{1.712872in}{4.108247in}}{\pgfqpoint{1.701822in}{4.108247in}}%
\pgfpathcurveto{\pgfqpoint{1.690771in}{4.108247in}}{\pgfqpoint{1.680172in}{4.103857in}}{\pgfqpoint{1.672359in}{4.096043in}}%
\pgfpathcurveto{\pgfqpoint{1.664545in}{4.088229in}}{\pgfqpoint{1.660155in}{4.077630in}}{\pgfqpoint{1.660155in}{4.066580in}}%
\pgfpathcurveto{\pgfqpoint{1.660155in}{4.055530in}}{\pgfqpoint{1.664545in}{4.044931in}}{\pgfqpoint{1.672359in}{4.037117in}}%
\pgfpathcurveto{\pgfqpoint{1.680172in}{4.029304in}}{\pgfqpoint{1.690771in}{4.024914in}}{\pgfqpoint{1.701822in}{4.024914in}}%
\pgfpathclose%
\pgfusepath{stroke,fill}%
\end{pgfscope}%
\begin{pgfscope}%
\pgfpathrectangle{\pgfqpoint{0.481978in}{0.331635in}}{\pgfqpoint{9.300000in}{7.700000in}}%
\pgfusepath{clip}%
\pgfsetbuttcap%
\pgfsetroundjoin%
\definecolor{currentfill}{rgb}{1.000000,0.705882,0.509804}%
\pgfsetfillcolor{currentfill}%
\pgfsetlinewidth{0.481800pt}%
\definecolor{currentstroke}{rgb}{1.000000,1.000000,1.000000}%
\pgfsetstrokecolor{currentstroke}%
\pgfsetdash{}{0pt}%
\pgfpathmoveto{\pgfqpoint{3.815204in}{3.964944in}}%
\pgfpathcurveto{\pgfqpoint{3.826254in}{3.964944in}}{\pgfqpoint{3.836853in}{3.969335in}}{\pgfqpoint{3.844667in}{3.977148in}}%
\pgfpathcurveto{\pgfqpoint{3.852481in}{3.984962in}}{\pgfqpoint{3.856871in}{3.995561in}}{\pgfqpoint{3.856871in}{4.006611in}}%
\pgfpathcurveto{\pgfqpoint{3.856871in}{4.017661in}}{\pgfqpoint{3.852481in}{4.028260in}}{\pgfqpoint{3.844667in}{4.036074in}}%
\pgfpathcurveto{\pgfqpoint{3.836853in}{4.043888in}}{\pgfqpoint{3.826254in}{4.048278in}}{\pgfqpoint{3.815204in}{4.048278in}}%
\pgfpathcurveto{\pgfqpoint{3.804154in}{4.048278in}}{\pgfqpoint{3.793555in}{4.043888in}}{\pgfqpoint{3.785742in}{4.036074in}}%
\pgfpathcurveto{\pgfqpoint{3.777928in}{4.028260in}}{\pgfqpoint{3.773538in}{4.017661in}}{\pgfqpoint{3.773538in}{4.006611in}}%
\pgfpathcurveto{\pgfqpoint{3.773538in}{3.995561in}}{\pgfqpoint{3.777928in}{3.984962in}}{\pgfqpoint{3.785742in}{3.977148in}}%
\pgfpathcurveto{\pgfqpoint{3.793555in}{3.969335in}}{\pgfqpoint{3.804154in}{3.964944in}}{\pgfqpoint{3.815204in}{3.964944in}}%
\pgfpathclose%
\pgfusepath{stroke,fill}%
\end{pgfscope}%
\begin{pgfscope}%
\pgfpathrectangle{\pgfqpoint{0.481978in}{0.331635in}}{\pgfqpoint{9.300000in}{7.700000in}}%
\pgfusepath{clip}%
\pgfsetbuttcap%
\pgfsetroundjoin%
\definecolor{currentfill}{rgb}{1.000000,0.705882,0.509804}%
\pgfsetfillcolor{currentfill}%
\pgfsetlinewidth{0.481800pt}%
\definecolor{currentstroke}{rgb}{1.000000,1.000000,1.000000}%
\pgfsetstrokecolor{currentstroke}%
\pgfsetdash{}{0pt}%
\pgfpathmoveto{\pgfqpoint{3.592234in}{4.931619in}}%
\pgfpathcurveto{\pgfqpoint{3.603285in}{4.931619in}}{\pgfqpoint{3.613884in}{4.936009in}}{\pgfqpoint{3.621697in}{4.943823in}}%
\pgfpathcurveto{\pgfqpoint{3.629511in}{4.951636in}}{\pgfqpoint{3.633901in}{4.962235in}}{\pgfqpoint{3.633901in}{4.973285in}}%
\pgfpathcurveto{\pgfqpoint{3.633901in}{4.984335in}}{\pgfqpoint{3.629511in}{4.994934in}}{\pgfqpoint{3.621697in}{5.002748in}}%
\pgfpathcurveto{\pgfqpoint{3.613884in}{5.010562in}}{\pgfqpoint{3.603285in}{5.014952in}}{\pgfqpoint{3.592234in}{5.014952in}}%
\pgfpathcurveto{\pgfqpoint{3.581184in}{5.014952in}}{\pgfqpoint{3.570585in}{5.010562in}}{\pgfqpoint{3.562772in}{5.002748in}}%
\pgfpathcurveto{\pgfqpoint{3.554958in}{4.994934in}}{\pgfqpoint{3.550568in}{4.984335in}}{\pgfqpoint{3.550568in}{4.973285in}}%
\pgfpathcurveto{\pgfqpoint{3.550568in}{4.962235in}}{\pgfqpoint{3.554958in}{4.951636in}}{\pgfqpoint{3.562772in}{4.943823in}}%
\pgfpathcurveto{\pgfqpoint{3.570585in}{4.936009in}}{\pgfqpoint{3.581184in}{4.931619in}}{\pgfqpoint{3.592234in}{4.931619in}}%
\pgfpathclose%
\pgfusepath{stroke,fill}%
\end{pgfscope}%
\begin{pgfscope}%
\pgfpathrectangle{\pgfqpoint{0.481978in}{0.331635in}}{\pgfqpoint{9.300000in}{7.700000in}}%
\pgfusepath{clip}%
\pgfsetbuttcap%
\pgfsetroundjoin%
\definecolor{currentfill}{rgb}{1.000000,0.705882,0.509804}%
\pgfsetfillcolor{currentfill}%
\pgfsetlinewidth{0.481800pt}%
\definecolor{currentstroke}{rgb}{1.000000,1.000000,1.000000}%
\pgfsetstrokecolor{currentstroke}%
\pgfsetdash{}{0pt}%
\pgfpathmoveto{\pgfqpoint{4.400900in}{2.356598in}}%
\pgfpathcurveto{\pgfqpoint{4.411950in}{2.356598in}}{\pgfqpoint{4.422549in}{2.360988in}}{\pgfqpoint{4.430363in}{2.368802in}}%
\pgfpathcurveto{\pgfqpoint{4.438176in}{2.376615in}}{\pgfqpoint{4.442567in}{2.387214in}}{\pgfqpoint{4.442567in}{2.398264in}}%
\pgfpathcurveto{\pgfqpoint{4.442567in}{2.409315in}}{\pgfqpoint{4.438176in}{2.419914in}}{\pgfqpoint{4.430363in}{2.427727in}}%
\pgfpathcurveto{\pgfqpoint{4.422549in}{2.435541in}}{\pgfqpoint{4.411950in}{2.439931in}}{\pgfqpoint{4.400900in}{2.439931in}}%
\pgfpathcurveto{\pgfqpoint{4.389850in}{2.439931in}}{\pgfqpoint{4.379251in}{2.435541in}}{\pgfqpoint{4.371437in}{2.427727in}}%
\pgfpathcurveto{\pgfqpoint{4.363624in}{2.419914in}}{\pgfqpoint{4.359233in}{2.409315in}}{\pgfqpoint{4.359233in}{2.398264in}}%
\pgfpathcurveto{\pgfqpoint{4.359233in}{2.387214in}}{\pgfqpoint{4.363624in}{2.376615in}}{\pgfqpoint{4.371437in}{2.368802in}}%
\pgfpathcurveto{\pgfqpoint{4.379251in}{2.360988in}}{\pgfqpoint{4.389850in}{2.356598in}}{\pgfqpoint{4.400900in}{2.356598in}}%
\pgfpathclose%
\pgfusepath{stroke,fill}%
\end{pgfscope}%
\begin{pgfscope}%
\pgfpathrectangle{\pgfqpoint{0.481978in}{0.331635in}}{\pgfqpoint{9.300000in}{7.700000in}}%
\pgfusepath{clip}%
\pgfsetbuttcap%
\pgfsetroundjoin%
\definecolor{currentfill}{rgb}{1.000000,0.705882,0.509804}%
\pgfsetfillcolor{currentfill}%
\pgfsetlinewidth{0.481800pt}%
\definecolor{currentstroke}{rgb}{1.000000,1.000000,1.000000}%
\pgfsetstrokecolor{currentstroke}%
\pgfsetdash{}{0pt}%
\pgfpathmoveto{\pgfqpoint{4.677099in}{3.453777in}}%
\pgfpathcurveto{\pgfqpoint{4.688149in}{3.453777in}}{\pgfqpoint{4.698748in}{3.458167in}}{\pgfqpoint{4.706562in}{3.465981in}}%
\pgfpathcurveto{\pgfqpoint{4.714375in}{3.473794in}}{\pgfqpoint{4.718766in}{3.484393in}}{\pgfqpoint{4.718766in}{3.495443in}}%
\pgfpathcurveto{\pgfqpoint{4.718766in}{3.506493in}}{\pgfqpoint{4.714375in}{3.517093in}}{\pgfqpoint{4.706562in}{3.524906in}}%
\pgfpathcurveto{\pgfqpoint{4.698748in}{3.532720in}}{\pgfqpoint{4.688149in}{3.537110in}}{\pgfqpoint{4.677099in}{3.537110in}}%
\pgfpathcurveto{\pgfqpoint{4.666049in}{3.537110in}}{\pgfqpoint{4.655450in}{3.532720in}}{\pgfqpoint{4.647636in}{3.524906in}}%
\pgfpathcurveto{\pgfqpoint{4.639823in}{3.517093in}}{\pgfqpoint{4.635432in}{3.506493in}}{\pgfqpoint{4.635432in}{3.495443in}}%
\pgfpathcurveto{\pgfqpoint{4.635432in}{3.484393in}}{\pgfqpoint{4.639823in}{3.473794in}}{\pgfqpoint{4.647636in}{3.465981in}}%
\pgfpathcurveto{\pgfqpoint{4.655450in}{3.458167in}}{\pgfqpoint{4.666049in}{3.453777in}}{\pgfqpoint{4.677099in}{3.453777in}}%
\pgfpathclose%
\pgfusepath{stroke,fill}%
\end{pgfscope}%
\begin{pgfscope}%
\pgfpathrectangle{\pgfqpoint{0.481978in}{0.331635in}}{\pgfqpoint{9.300000in}{7.700000in}}%
\pgfusepath{clip}%
\pgfsetbuttcap%
\pgfsetroundjoin%
\definecolor{currentfill}{rgb}{1.000000,0.705882,0.509804}%
\pgfsetfillcolor{currentfill}%
\pgfsetlinewidth{0.481800pt}%
\definecolor{currentstroke}{rgb}{1.000000,1.000000,1.000000}%
\pgfsetstrokecolor{currentstroke}%
\pgfsetdash{}{0pt}%
\pgfpathmoveto{\pgfqpoint{2.595817in}{5.461427in}}%
\pgfpathcurveto{\pgfqpoint{2.606867in}{5.461427in}}{\pgfqpoint{2.617466in}{5.465817in}}{\pgfqpoint{2.625280in}{5.473631in}}%
\pgfpathcurveto{\pgfqpoint{2.633094in}{5.481444in}}{\pgfqpoint{2.637484in}{5.492043in}}{\pgfqpoint{2.637484in}{5.503093in}}%
\pgfpathcurveto{\pgfqpoint{2.637484in}{5.514144in}}{\pgfqpoint{2.633094in}{5.524743in}}{\pgfqpoint{2.625280in}{5.532556in}}%
\pgfpathcurveto{\pgfqpoint{2.617466in}{5.540370in}}{\pgfqpoint{2.606867in}{5.544760in}}{\pgfqpoint{2.595817in}{5.544760in}}%
\pgfpathcurveto{\pgfqpoint{2.584767in}{5.544760in}}{\pgfqpoint{2.574168in}{5.540370in}}{\pgfqpoint{2.566354in}{5.532556in}}%
\pgfpathcurveto{\pgfqpoint{2.558541in}{5.524743in}}{\pgfqpoint{2.554150in}{5.514144in}}{\pgfqpoint{2.554150in}{5.503093in}}%
\pgfpathcurveto{\pgfqpoint{2.554150in}{5.492043in}}{\pgfqpoint{2.558541in}{5.481444in}}{\pgfqpoint{2.566354in}{5.473631in}}%
\pgfpathcurveto{\pgfqpoint{2.574168in}{5.465817in}}{\pgfqpoint{2.584767in}{5.461427in}}{\pgfqpoint{2.595817in}{5.461427in}}%
\pgfpathclose%
\pgfusepath{stroke,fill}%
\end{pgfscope}%
\begin{pgfscope}%
\pgfpathrectangle{\pgfqpoint{0.481978in}{0.331635in}}{\pgfqpoint{9.300000in}{7.700000in}}%
\pgfusepath{clip}%
\pgfsetbuttcap%
\pgfsetroundjoin%
\definecolor{currentfill}{rgb}{1.000000,0.705882,0.509804}%
\pgfsetfillcolor{currentfill}%
\pgfsetlinewidth{0.481800pt}%
\definecolor{currentstroke}{rgb}{1.000000,1.000000,1.000000}%
\pgfsetstrokecolor{currentstroke}%
\pgfsetdash{}{0pt}%
\pgfpathmoveto{\pgfqpoint{3.380362in}{4.087253in}}%
\pgfpathcurveto{\pgfqpoint{3.391412in}{4.087253in}}{\pgfqpoint{3.402011in}{4.091643in}}{\pgfqpoint{3.409824in}{4.099457in}}%
\pgfpathcurveto{\pgfqpoint{3.417638in}{4.107270in}}{\pgfqpoint{3.422028in}{4.117869in}}{\pgfqpoint{3.422028in}{4.128920in}}%
\pgfpathcurveto{\pgfqpoint{3.422028in}{4.139970in}}{\pgfqpoint{3.417638in}{4.150569in}}{\pgfqpoint{3.409824in}{4.158382in}}%
\pgfpathcurveto{\pgfqpoint{3.402011in}{4.166196in}}{\pgfqpoint{3.391412in}{4.170586in}}{\pgfqpoint{3.380362in}{4.170586in}}%
\pgfpathcurveto{\pgfqpoint{3.369311in}{4.170586in}}{\pgfqpoint{3.358712in}{4.166196in}}{\pgfqpoint{3.350899in}{4.158382in}}%
\pgfpathcurveto{\pgfqpoint{3.343085in}{4.150569in}}{\pgfqpoint{3.338695in}{4.139970in}}{\pgfqpoint{3.338695in}{4.128920in}}%
\pgfpathcurveto{\pgfqpoint{3.338695in}{4.117869in}}{\pgfqpoint{3.343085in}{4.107270in}}{\pgfqpoint{3.350899in}{4.099457in}}%
\pgfpathcurveto{\pgfqpoint{3.358712in}{4.091643in}}{\pgfqpoint{3.369311in}{4.087253in}}{\pgfqpoint{3.380362in}{4.087253in}}%
\pgfpathclose%
\pgfusepath{stroke,fill}%
\end{pgfscope}%
\begin{pgfscope}%
\pgfpathrectangle{\pgfqpoint{0.481978in}{0.331635in}}{\pgfqpoint{9.300000in}{7.700000in}}%
\pgfusepath{clip}%
\pgfsetbuttcap%
\pgfsetroundjoin%
\definecolor{currentfill}{rgb}{1.000000,0.705882,0.509804}%
\pgfsetfillcolor{currentfill}%
\pgfsetlinewidth{0.481800pt}%
\definecolor{currentstroke}{rgb}{1.000000,1.000000,1.000000}%
\pgfsetstrokecolor{currentstroke}%
\pgfsetdash{}{0pt}%
\pgfpathmoveto{\pgfqpoint{8.094040in}{3.873702in}}%
\pgfpathcurveto{\pgfqpoint{8.105090in}{3.873702in}}{\pgfqpoint{8.115689in}{3.878093in}}{\pgfqpoint{8.123503in}{3.885906in}}%
\pgfpathcurveto{\pgfqpoint{8.131317in}{3.893720in}}{\pgfqpoint{8.135707in}{3.904319in}}{\pgfqpoint{8.135707in}{3.915369in}}%
\pgfpathcurveto{\pgfqpoint{8.135707in}{3.926419in}}{\pgfqpoint{8.131317in}{3.937018in}}{\pgfqpoint{8.123503in}{3.944832in}}%
\pgfpathcurveto{\pgfqpoint{8.115689in}{3.952645in}}{\pgfqpoint{8.105090in}{3.957036in}}{\pgfqpoint{8.094040in}{3.957036in}}%
\pgfpathcurveto{\pgfqpoint{8.082990in}{3.957036in}}{\pgfqpoint{8.072391in}{3.952645in}}{\pgfqpoint{8.064577in}{3.944832in}}%
\pgfpathcurveto{\pgfqpoint{8.056764in}{3.937018in}}{\pgfqpoint{8.052374in}{3.926419in}}{\pgfqpoint{8.052374in}{3.915369in}}%
\pgfpathcurveto{\pgfqpoint{8.052374in}{3.904319in}}{\pgfqpoint{8.056764in}{3.893720in}}{\pgfqpoint{8.064577in}{3.885906in}}%
\pgfpathcurveto{\pgfqpoint{8.072391in}{3.878093in}}{\pgfqpoint{8.082990in}{3.873702in}}{\pgfqpoint{8.094040in}{3.873702in}}%
\pgfpathclose%
\pgfusepath{stroke,fill}%
\end{pgfscope}%
\begin{pgfscope}%
\pgfpathrectangle{\pgfqpoint{0.481978in}{0.331635in}}{\pgfqpoint{9.300000in}{7.700000in}}%
\pgfusepath{clip}%
\pgfsetbuttcap%
\pgfsetroundjoin%
\definecolor{currentfill}{rgb}{1.000000,0.705882,0.509804}%
\pgfsetfillcolor{currentfill}%
\pgfsetlinewidth{0.481800pt}%
\definecolor{currentstroke}{rgb}{1.000000,1.000000,1.000000}%
\pgfsetstrokecolor{currentstroke}%
\pgfsetdash{}{0pt}%
\pgfpathmoveto{\pgfqpoint{3.449664in}{4.246640in}}%
\pgfpathcurveto{\pgfqpoint{3.460714in}{4.246640in}}{\pgfqpoint{3.471313in}{4.251030in}}{\pgfqpoint{3.479127in}{4.258844in}}%
\pgfpathcurveto{\pgfqpoint{3.486941in}{4.266658in}}{\pgfqpoint{3.491331in}{4.277257in}}{\pgfqpoint{3.491331in}{4.288307in}}%
\pgfpathcurveto{\pgfqpoint{3.491331in}{4.299357in}}{\pgfqpoint{3.486941in}{4.309956in}}{\pgfqpoint{3.479127in}{4.317770in}}%
\pgfpathcurveto{\pgfqpoint{3.471313in}{4.325583in}}{\pgfqpoint{3.460714in}{4.329974in}}{\pgfqpoint{3.449664in}{4.329974in}}%
\pgfpathcurveto{\pgfqpoint{3.438614in}{4.329974in}}{\pgfqpoint{3.428015in}{4.325583in}}{\pgfqpoint{3.420202in}{4.317770in}}%
\pgfpathcurveto{\pgfqpoint{3.412388in}{4.309956in}}{\pgfqpoint{3.407998in}{4.299357in}}{\pgfqpoint{3.407998in}{4.288307in}}%
\pgfpathcurveto{\pgfqpoint{3.407998in}{4.277257in}}{\pgfqpoint{3.412388in}{4.266658in}}{\pgfqpoint{3.420202in}{4.258844in}}%
\pgfpathcurveto{\pgfqpoint{3.428015in}{4.251030in}}{\pgfqpoint{3.438614in}{4.246640in}}{\pgfqpoint{3.449664in}{4.246640in}}%
\pgfpathclose%
\pgfusepath{stroke,fill}%
\end{pgfscope}%
\begin{pgfscope}%
\pgfpathrectangle{\pgfqpoint{0.481978in}{0.331635in}}{\pgfqpoint{9.300000in}{7.700000in}}%
\pgfusepath{clip}%
\pgfsetbuttcap%
\pgfsetroundjoin%
\definecolor{currentfill}{rgb}{1.000000,0.705882,0.509804}%
\pgfsetfillcolor{currentfill}%
\pgfsetlinewidth{0.481800pt}%
\definecolor{currentstroke}{rgb}{1.000000,1.000000,1.000000}%
\pgfsetstrokecolor{currentstroke}%
\pgfsetdash{}{0pt}%
\pgfpathmoveto{\pgfqpoint{3.549059in}{7.204392in}}%
\pgfpathcurveto{\pgfqpoint{3.560110in}{7.204392in}}{\pgfqpoint{3.570709in}{7.208782in}}{\pgfqpoint{3.578522in}{7.216596in}}%
\pgfpathcurveto{\pgfqpoint{3.586336in}{7.224409in}}{\pgfqpoint{3.590726in}{7.235008in}}{\pgfqpoint{3.590726in}{7.246059in}}%
\pgfpathcurveto{\pgfqpoint{3.590726in}{7.257109in}}{\pgfqpoint{3.586336in}{7.267708in}}{\pgfqpoint{3.578522in}{7.275521in}}%
\pgfpathcurveto{\pgfqpoint{3.570709in}{7.283335in}}{\pgfqpoint{3.560110in}{7.287725in}}{\pgfqpoint{3.549059in}{7.287725in}}%
\pgfpathcurveto{\pgfqpoint{3.538009in}{7.287725in}}{\pgfqpoint{3.527410in}{7.283335in}}{\pgfqpoint{3.519597in}{7.275521in}}%
\pgfpathcurveto{\pgfqpoint{3.511783in}{7.267708in}}{\pgfqpoint{3.507393in}{7.257109in}}{\pgfqpoint{3.507393in}{7.246059in}}%
\pgfpathcurveto{\pgfqpoint{3.507393in}{7.235008in}}{\pgfqpoint{3.511783in}{7.224409in}}{\pgfqpoint{3.519597in}{7.216596in}}%
\pgfpathcurveto{\pgfqpoint{3.527410in}{7.208782in}}{\pgfqpoint{3.538009in}{7.204392in}}{\pgfqpoint{3.549059in}{7.204392in}}%
\pgfpathclose%
\pgfusepath{stroke,fill}%
\end{pgfscope}%
\begin{pgfscope}%
\pgfpathrectangle{\pgfqpoint{0.481978in}{0.331635in}}{\pgfqpoint{9.300000in}{7.700000in}}%
\pgfusepath{clip}%
\pgfsetbuttcap%
\pgfsetroundjoin%
\definecolor{currentfill}{rgb}{1.000000,0.705882,0.509804}%
\pgfsetfillcolor{currentfill}%
\pgfsetlinewidth{0.481800pt}%
\definecolor{currentstroke}{rgb}{1.000000,1.000000,1.000000}%
\pgfsetstrokecolor{currentstroke}%
\pgfsetdash{}{0pt}%
\pgfpathmoveto{\pgfqpoint{1.737569in}{3.281104in}}%
\pgfpathcurveto{\pgfqpoint{1.748619in}{3.281104in}}{\pgfqpoint{1.759218in}{3.285494in}}{\pgfqpoint{1.767031in}{3.293308in}}%
\pgfpathcurveto{\pgfqpoint{1.774845in}{3.301121in}}{\pgfqpoint{1.779235in}{3.311720in}}{\pgfqpoint{1.779235in}{3.322770in}}%
\pgfpathcurveto{\pgfqpoint{1.779235in}{3.333821in}}{\pgfqpoint{1.774845in}{3.344420in}}{\pgfqpoint{1.767031in}{3.352233in}}%
\pgfpathcurveto{\pgfqpoint{1.759218in}{3.360047in}}{\pgfqpoint{1.748619in}{3.364437in}}{\pgfqpoint{1.737569in}{3.364437in}}%
\pgfpathcurveto{\pgfqpoint{1.726518in}{3.364437in}}{\pgfqpoint{1.715919in}{3.360047in}}{\pgfqpoint{1.708106in}{3.352233in}}%
\pgfpathcurveto{\pgfqpoint{1.700292in}{3.344420in}}{\pgfqpoint{1.695902in}{3.333821in}}{\pgfqpoint{1.695902in}{3.322770in}}%
\pgfpathcurveto{\pgfqpoint{1.695902in}{3.311720in}}{\pgfqpoint{1.700292in}{3.301121in}}{\pgfqpoint{1.708106in}{3.293308in}}%
\pgfpathcurveto{\pgfqpoint{1.715919in}{3.285494in}}{\pgfqpoint{1.726518in}{3.281104in}}{\pgfqpoint{1.737569in}{3.281104in}}%
\pgfpathclose%
\pgfusepath{stroke,fill}%
\end{pgfscope}%
\begin{pgfscope}%
\pgfpathrectangle{\pgfqpoint{0.481978in}{0.331635in}}{\pgfqpoint{9.300000in}{7.700000in}}%
\pgfusepath{clip}%
\pgfsetbuttcap%
\pgfsetroundjoin%
\definecolor{currentfill}{rgb}{1.000000,0.705882,0.509804}%
\pgfsetfillcolor{currentfill}%
\pgfsetlinewidth{0.481800pt}%
\definecolor{currentstroke}{rgb}{1.000000,1.000000,1.000000}%
\pgfsetstrokecolor{currentstroke}%
\pgfsetdash{}{0pt}%
\pgfpathmoveto{\pgfqpoint{3.522445in}{3.104341in}}%
\pgfpathcurveto{\pgfqpoint{3.533495in}{3.104341in}}{\pgfqpoint{3.544094in}{3.108731in}}{\pgfqpoint{3.551907in}{3.116545in}}%
\pgfpathcurveto{\pgfqpoint{3.559721in}{3.124358in}}{\pgfqpoint{3.564111in}{3.134957in}}{\pgfqpoint{3.564111in}{3.146007in}}%
\pgfpathcurveto{\pgfqpoint{3.564111in}{3.157057in}}{\pgfqpoint{3.559721in}{3.167657in}}{\pgfqpoint{3.551907in}{3.175470in}}%
\pgfpathcurveto{\pgfqpoint{3.544094in}{3.183284in}}{\pgfqpoint{3.533495in}{3.187674in}}{\pgfqpoint{3.522445in}{3.187674in}}%
\pgfpathcurveto{\pgfqpoint{3.511395in}{3.187674in}}{\pgfqpoint{3.500795in}{3.183284in}}{\pgfqpoint{3.492982in}{3.175470in}}%
\pgfpathcurveto{\pgfqpoint{3.485168in}{3.167657in}}{\pgfqpoint{3.480778in}{3.157057in}}{\pgfqpoint{3.480778in}{3.146007in}}%
\pgfpathcurveto{\pgfqpoint{3.480778in}{3.134957in}}{\pgfqpoint{3.485168in}{3.124358in}}{\pgfqpoint{3.492982in}{3.116545in}}%
\pgfpathcurveto{\pgfqpoint{3.500795in}{3.108731in}}{\pgfqpoint{3.511395in}{3.104341in}}{\pgfqpoint{3.522445in}{3.104341in}}%
\pgfpathclose%
\pgfusepath{stroke,fill}%
\end{pgfscope}%
\begin{pgfscope}%
\pgfpathrectangle{\pgfqpoint{0.481978in}{0.331635in}}{\pgfqpoint{9.300000in}{7.700000in}}%
\pgfusepath{clip}%
\pgfsetbuttcap%
\pgfsetroundjoin%
\definecolor{currentfill}{rgb}{1.000000,0.705882,0.509804}%
\pgfsetfillcolor{currentfill}%
\pgfsetlinewidth{0.481800pt}%
\definecolor{currentstroke}{rgb}{1.000000,1.000000,1.000000}%
\pgfsetstrokecolor{currentstroke}%
\pgfsetdash{}{0pt}%
\pgfpathmoveto{\pgfqpoint{5.442526in}{4.550181in}}%
\pgfpathcurveto{\pgfqpoint{5.453577in}{4.550181in}}{\pgfqpoint{5.464176in}{4.554571in}}{\pgfqpoint{5.471989in}{4.562385in}}%
\pgfpathcurveto{\pgfqpoint{5.479803in}{4.570198in}}{\pgfqpoint{5.484193in}{4.580797in}}{\pgfqpoint{5.484193in}{4.591847in}}%
\pgfpathcurveto{\pgfqpoint{5.484193in}{4.602898in}}{\pgfqpoint{5.479803in}{4.613497in}}{\pgfqpoint{5.471989in}{4.621310in}}%
\pgfpathcurveto{\pgfqpoint{5.464176in}{4.629124in}}{\pgfqpoint{5.453577in}{4.633514in}}{\pgfqpoint{5.442526in}{4.633514in}}%
\pgfpathcurveto{\pgfqpoint{5.431476in}{4.633514in}}{\pgfqpoint{5.420877in}{4.629124in}}{\pgfqpoint{5.413064in}{4.621310in}}%
\pgfpathcurveto{\pgfqpoint{5.405250in}{4.613497in}}{\pgfqpoint{5.400860in}{4.602898in}}{\pgfqpoint{5.400860in}{4.591847in}}%
\pgfpathcurveto{\pgfqpoint{5.400860in}{4.580797in}}{\pgfqpoint{5.405250in}{4.570198in}}{\pgfqpoint{5.413064in}{4.562385in}}%
\pgfpathcurveto{\pgfqpoint{5.420877in}{4.554571in}}{\pgfqpoint{5.431476in}{4.550181in}}{\pgfqpoint{5.442526in}{4.550181in}}%
\pgfpathclose%
\pgfusepath{stroke,fill}%
\end{pgfscope}%
\begin{pgfscope}%
\pgfpathrectangle{\pgfqpoint{0.481978in}{0.331635in}}{\pgfqpoint{9.300000in}{7.700000in}}%
\pgfusepath{clip}%
\pgfsetbuttcap%
\pgfsetroundjoin%
\definecolor{currentfill}{rgb}{1.000000,0.705882,0.509804}%
\pgfsetfillcolor{currentfill}%
\pgfsetlinewidth{0.481800pt}%
\definecolor{currentstroke}{rgb}{1.000000,1.000000,1.000000}%
\pgfsetstrokecolor{currentstroke}%
\pgfsetdash{}{0pt}%
\pgfpathmoveto{\pgfqpoint{3.182729in}{4.161824in}}%
\pgfpathcurveto{\pgfqpoint{3.193779in}{4.161824in}}{\pgfqpoint{3.204378in}{4.166214in}}{\pgfqpoint{3.212192in}{4.174028in}}%
\pgfpathcurveto{\pgfqpoint{3.220005in}{4.181842in}}{\pgfqpoint{3.224396in}{4.192441in}}{\pgfqpoint{3.224396in}{4.203491in}}%
\pgfpathcurveto{\pgfqpoint{3.224396in}{4.214541in}}{\pgfqpoint{3.220005in}{4.225140in}}{\pgfqpoint{3.212192in}{4.232953in}}%
\pgfpathcurveto{\pgfqpoint{3.204378in}{4.240767in}}{\pgfqpoint{3.193779in}{4.245157in}}{\pgfqpoint{3.182729in}{4.245157in}}%
\pgfpathcurveto{\pgfqpoint{3.171679in}{4.245157in}}{\pgfqpoint{3.161080in}{4.240767in}}{\pgfqpoint{3.153266in}{4.232953in}}%
\pgfpathcurveto{\pgfqpoint{3.145453in}{4.225140in}}{\pgfqpoint{3.141062in}{4.214541in}}{\pgfqpoint{3.141062in}{4.203491in}}%
\pgfpathcurveto{\pgfqpoint{3.141062in}{4.192441in}}{\pgfqpoint{3.145453in}{4.181842in}}{\pgfqpoint{3.153266in}{4.174028in}}%
\pgfpathcurveto{\pgfqpoint{3.161080in}{4.166214in}}{\pgfqpoint{3.171679in}{4.161824in}}{\pgfqpoint{3.182729in}{4.161824in}}%
\pgfpathclose%
\pgfusepath{stroke,fill}%
\end{pgfscope}%
\begin{pgfscope}%
\pgfpathrectangle{\pgfqpoint{0.481978in}{0.331635in}}{\pgfqpoint{9.300000in}{7.700000in}}%
\pgfusepath{clip}%
\pgfsetbuttcap%
\pgfsetroundjoin%
\definecolor{currentfill}{rgb}{1.000000,0.705882,0.509804}%
\pgfsetfillcolor{currentfill}%
\pgfsetlinewidth{0.481800pt}%
\definecolor{currentstroke}{rgb}{1.000000,1.000000,1.000000}%
\pgfsetstrokecolor{currentstroke}%
\pgfsetdash{}{0pt}%
\pgfpathmoveto{\pgfqpoint{5.580105in}{6.284102in}}%
\pgfpathcurveto{\pgfqpoint{5.591155in}{6.284102in}}{\pgfqpoint{5.601754in}{6.288492in}}{\pgfqpoint{5.609568in}{6.296305in}}%
\pgfpathcurveto{\pgfqpoint{5.617382in}{6.304119in}}{\pgfqpoint{5.621772in}{6.314718in}}{\pgfqpoint{5.621772in}{6.325768in}}%
\pgfpathcurveto{\pgfqpoint{5.621772in}{6.336818in}}{\pgfqpoint{5.617382in}{6.347417in}}{\pgfqpoint{5.609568in}{6.355231in}}%
\pgfpathcurveto{\pgfqpoint{5.601754in}{6.363045in}}{\pgfqpoint{5.591155in}{6.367435in}}{\pgfqpoint{5.580105in}{6.367435in}}%
\pgfpathcurveto{\pgfqpoint{5.569055in}{6.367435in}}{\pgfqpoint{5.558456in}{6.363045in}}{\pgfqpoint{5.550642in}{6.355231in}}%
\pgfpathcurveto{\pgfqpoint{5.542829in}{6.347417in}}{\pgfqpoint{5.538438in}{6.336818in}}{\pgfqpoint{5.538438in}{6.325768in}}%
\pgfpathcurveto{\pgfqpoint{5.538438in}{6.314718in}}{\pgfqpoint{5.542829in}{6.304119in}}{\pgfqpoint{5.550642in}{6.296305in}}%
\pgfpathcurveto{\pgfqpoint{5.558456in}{6.288492in}}{\pgfqpoint{5.569055in}{6.284102in}}{\pgfqpoint{5.580105in}{6.284102in}}%
\pgfpathclose%
\pgfusepath{stroke,fill}%
\end{pgfscope}%
\begin{pgfscope}%
\pgfpathrectangle{\pgfqpoint{0.481978in}{0.331635in}}{\pgfqpoint{9.300000in}{7.700000in}}%
\pgfusepath{clip}%
\pgfsetbuttcap%
\pgfsetroundjoin%
\definecolor{currentfill}{rgb}{1.000000,0.705882,0.509804}%
\pgfsetfillcolor{currentfill}%
\pgfsetlinewidth{0.481800pt}%
\definecolor{currentstroke}{rgb}{1.000000,1.000000,1.000000}%
\pgfsetstrokecolor{currentstroke}%
\pgfsetdash{}{0pt}%
\pgfpathmoveto{\pgfqpoint{3.383682in}{4.885031in}}%
\pgfpathcurveto{\pgfqpoint{3.394732in}{4.885031in}}{\pgfqpoint{3.405331in}{4.889422in}}{\pgfqpoint{3.413144in}{4.897235in}}%
\pgfpathcurveto{\pgfqpoint{3.420958in}{4.905049in}}{\pgfqpoint{3.425348in}{4.915648in}}{\pgfqpoint{3.425348in}{4.926698in}}%
\pgfpathcurveto{\pgfqpoint{3.425348in}{4.937748in}}{\pgfqpoint{3.420958in}{4.948347in}}{\pgfqpoint{3.413144in}{4.956161in}}%
\pgfpathcurveto{\pgfqpoint{3.405331in}{4.963975in}}{\pgfqpoint{3.394732in}{4.968365in}}{\pgfqpoint{3.383682in}{4.968365in}}%
\pgfpathcurveto{\pgfqpoint{3.372631in}{4.968365in}}{\pgfqpoint{3.362032in}{4.963975in}}{\pgfqpoint{3.354219in}{4.956161in}}%
\pgfpathcurveto{\pgfqpoint{3.346405in}{4.948347in}}{\pgfqpoint{3.342015in}{4.937748in}}{\pgfqpoint{3.342015in}{4.926698in}}%
\pgfpathcurveto{\pgfqpoint{3.342015in}{4.915648in}}{\pgfqpoint{3.346405in}{4.905049in}}{\pgfqpoint{3.354219in}{4.897235in}}%
\pgfpathcurveto{\pgfqpoint{3.362032in}{4.889422in}}{\pgfqpoint{3.372631in}{4.885031in}}{\pgfqpoint{3.383682in}{4.885031in}}%
\pgfpathclose%
\pgfusepath{stroke,fill}%
\end{pgfscope}%
\begin{pgfscope}%
\pgfpathrectangle{\pgfqpoint{0.481978in}{0.331635in}}{\pgfqpoint{9.300000in}{7.700000in}}%
\pgfusepath{clip}%
\pgfsetbuttcap%
\pgfsetroundjoin%
\definecolor{currentfill}{rgb}{1.000000,0.705882,0.509804}%
\pgfsetfillcolor{currentfill}%
\pgfsetlinewidth{0.481800pt}%
\definecolor{currentstroke}{rgb}{1.000000,1.000000,1.000000}%
\pgfsetstrokecolor{currentstroke}%
\pgfsetdash{}{0pt}%
\pgfpathmoveto{\pgfqpoint{3.010244in}{5.125365in}}%
\pgfpathcurveto{\pgfqpoint{3.021294in}{5.125365in}}{\pgfqpoint{3.031893in}{5.129755in}}{\pgfqpoint{3.039707in}{5.137569in}}%
\pgfpathcurveto{\pgfqpoint{3.047521in}{5.145382in}}{\pgfqpoint{3.051911in}{5.155981in}}{\pgfqpoint{3.051911in}{5.167031in}}%
\pgfpathcurveto{\pgfqpoint{3.051911in}{5.178081in}}{\pgfqpoint{3.047521in}{5.188680in}}{\pgfqpoint{3.039707in}{5.196494in}}%
\pgfpathcurveto{\pgfqpoint{3.031893in}{5.204308in}}{\pgfqpoint{3.021294in}{5.208698in}}{\pgfqpoint{3.010244in}{5.208698in}}%
\pgfpathcurveto{\pgfqpoint{2.999194in}{5.208698in}}{\pgfqpoint{2.988595in}{5.204308in}}{\pgfqpoint{2.980781in}{5.196494in}}%
\pgfpathcurveto{\pgfqpoint{2.972968in}{5.188680in}}{\pgfqpoint{2.968578in}{5.178081in}}{\pgfqpoint{2.968578in}{5.167031in}}%
\pgfpathcurveto{\pgfqpoint{2.968578in}{5.155981in}}{\pgfqpoint{2.972968in}{5.145382in}}{\pgfqpoint{2.980781in}{5.137569in}}%
\pgfpathcurveto{\pgfqpoint{2.988595in}{5.129755in}}{\pgfqpoint{2.999194in}{5.125365in}}{\pgfqpoint{3.010244in}{5.125365in}}%
\pgfpathclose%
\pgfusepath{stroke,fill}%
\end{pgfscope}%
\begin{pgfscope}%
\pgfpathrectangle{\pgfqpoint{0.481978in}{0.331635in}}{\pgfqpoint{9.300000in}{7.700000in}}%
\pgfusepath{clip}%
\pgfsetbuttcap%
\pgfsetroundjoin%
\definecolor{currentfill}{rgb}{1.000000,0.705882,0.509804}%
\pgfsetfillcolor{currentfill}%
\pgfsetlinewidth{0.481800pt}%
\definecolor{currentstroke}{rgb}{1.000000,1.000000,1.000000}%
\pgfsetstrokecolor{currentstroke}%
\pgfsetdash{}{0pt}%
\pgfpathmoveto{\pgfqpoint{4.553603in}{3.997453in}}%
\pgfpathcurveto{\pgfqpoint{4.564653in}{3.997453in}}{\pgfqpoint{4.575252in}{4.001843in}}{\pgfqpoint{4.583066in}{4.009657in}}%
\pgfpathcurveto{\pgfqpoint{4.590880in}{4.017470in}}{\pgfqpoint{4.595270in}{4.028069in}}{\pgfqpoint{4.595270in}{4.039119in}}%
\pgfpathcurveto{\pgfqpoint{4.595270in}{4.050170in}}{\pgfqpoint{4.590880in}{4.060769in}}{\pgfqpoint{4.583066in}{4.068582in}}%
\pgfpathcurveto{\pgfqpoint{4.575252in}{4.076396in}}{\pgfqpoint{4.564653in}{4.080786in}}{\pgfqpoint{4.553603in}{4.080786in}}%
\pgfpathcurveto{\pgfqpoint{4.542553in}{4.080786in}}{\pgfqpoint{4.531954in}{4.076396in}}{\pgfqpoint{4.524140in}{4.068582in}}%
\pgfpathcurveto{\pgfqpoint{4.516327in}{4.060769in}}{\pgfqpoint{4.511936in}{4.050170in}}{\pgfqpoint{4.511936in}{4.039119in}}%
\pgfpathcurveto{\pgfqpoint{4.511936in}{4.028069in}}{\pgfqpoint{4.516327in}{4.017470in}}{\pgfqpoint{4.524140in}{4.009657in}}%
\pgfpathcurveto{\pgfqpoint{4.531954in}{4.001843in}}{\pgfqpoint{4.542553in}{3.997453in}}{\pgfqpoint{4.553603in}{3.997453in}}%
\pgfpathclose%
\pgfusepath{stroke,fill}%
\end{pgfscope}%
\begin{pgfscope}%
\pgfpathrectangle{\pgfqpoint{0.481978in}{0.331635in}}{\pgfqpoint{9.300000in}{7.700000in}}%
\pgfusepath{clip}%
\pgfsetbuttcap%
\pgfsetroundjoin%
\definecolor{currentfill}{rgb}{1.000000,0.705882,0.509804}%
\pgfsetfillcolor{currentfill}%
\pgfsetlinewidth{0.481800pt}%
\definecolor{currentstroke}{rgb}{1.000000,1.000000,1.000000}%
\pgfsetstrokecolor{currentstroke}%
\pgfsetdash{}{0pt}%
\pgfpathmoveto{\pgfqpoint{9.359202in}{1.323697in}}%
\pgfpathcurveto{\pgfqpoint{9.370252in}{1.323697in}}{\pgfqpoint{9.380851in}{1.328088in}}{\pgfqpoint{9.388665in}{1.335901in}}%
\pgfpathcurveto{\pgfqpoint{9.396478in}{1.343715in}}{\pgfqpoint{9.400868in}{1.354314in}}{\pgfqpoint{9.400868in}{1.365364in}}%
\pgfpathcurveto{\pgfqpoint{9.400868in}{1.376414in}}{\pgfqpoint{9.396478in}{1.387013in}}{\pgfqpoint{9.388665in}{1.394827in}}%
\pgfpathcurveto{\pgfqpoint{9.380851in}{1.402640in}}{\pgfqpoint{9.370252in}{1.407031in}}{\pgfqpoint{9.359202in}{1.407031in}}%
\pgfpathcurveto{\pgfqpoint{9.348152in}{1.407031in}}{\pgfqpoint{9.337553in}{1.402640in}}{\pgfqpoint{9.329739in}{1.394827in}}%
\pgfpathcurveto{\pgfqpoint{9.321925in}{1.387013in}}{\pgfqpoint{9.317535in}{1.376414in}}{\pgfqpoint{9.317535in}{1.365364in}}%
\pgfpathcurveto{\pgfqpoint{9.317535in}{1.354314in}}{\pgfqpoint{9.321925in}{1.343715in}}{\pgfqpoint{9.329739in}{1.335901in}}%
\pgfpathcurveto{\pgfqpoint{9.337553in}{1.328088in}}{\pgfqpoint{9.348152in}{1.323697in}}{\pgfqpoint{9.359202in}{1.323697in}}%
\pgfpathclose%
\pgfusepath{stroke,fill}%
\end{pgfscope}%
\begin{pgfscope}%
\pgfpathrectangle{\pgfqpoint{0.481978in}{0.331635in}}{\pgfqpoint{9.300000in}{7.700000in}}%
\pgfusepath{clip}%
\pgfsetbuttcap%
\pgfsetroundjoin%
\definecolor{currentfill}{rgb}{1.000000,0.705882,0.509804}%
\pgfsetfillcolor{currentfill}%
\pgfsetlinewidth{0.481800pt}%
\definecolor{currentstroke}{rgb}{1.000000,1.000000,1.000000}%
\pgfsetstrokecolor{currentstroke}%
\pgfsetdash{}{0pt}%
\pgfpathmoveto{\pgfqpoint{3.953659in}{2.869946in}}%
\pgfpathcurveto{\pgfqpoint{3.964709in}{2.869946in}}{\pgfqpoint{3.975308in}{2.874336in}}{\pgfqpoint{3.983121in}{2.882150in}}%
\pgfpathcurveto{\pgfqpoint{3.990935in}{2.889963in}}{\pgfqpoint{3.995325in}{2.900563in}}{\pgfqpoint{3.995325in}{2.911613in}}%
\pgfpathcurveto{\pgfqpoint{3.995325in}{2.922663in}}{\pgfqpoint{3.990935in}{2.933262in}}{\pgfqpoint{3.983121in}{2.941075in}}%
\pgfpathcurveto{\pgfqpoint{3.975308in}{2.948889in}}{\pgfqpoint{3.964709in}{2.953279in}}{\pgfqpoint{3.953659in}{2.953279in}}%
\pgfpathcurveto{\pgfqpoint{3.942609in}{2.953279in}}{\pgfqpoint{3.932009in}{2.948889in}}{\pgfqpoint{3.924196in}{2.941075in}}%
\pgfpathcurveto{\pgfqpoint{3.916382in}{2.933262in}}{\pgfqpoint{3.911992in}{2.922663in}}{\pgfqpoint{3.911992in}{2.911613in}}%
\pgfpathcurveto{\pgfqpoint{3.911992in}{2.900563in}}{\pgfqpoint{3.916382in}{2.889963in}}{\pgfqpoint{3.924196in}{2.882150in}}%
\pgfpathcurveto{\pgfqpoint{3.932009in}{2.874336in}}{\pgfqpoint{3.942609in}{2.869946in}}{\pgfqpoint{3.953659in}{2.869946in}}%
\pgfpathclose%
\pgfusepath{stroke,fill}%
\end{pgfscope}%
\begin{pgfscope}%
\pgfpathrectangle{\pgfqpoint{0.481978in}{0.331635in}}{\pgfqpoint{9.300000in}{7.700000in}}%
\pgfusepath{clip}%
\pgfsetbuttcap%
\pgfsetroundjoin%
\definecolor{currentfill}{rgb}{1.000000,0.705882,0.509804}%
\pgfsetfillcolor{currentfill}%
\pgfsetlinewidth{0.481800pt}%
\definecolor{currentstroke}{rgb}{1.000000,1.000000,1.000000}%
\pgfsetstrokecolor{currentstroke}%
\pgfsetdash{}{0pt}%
\pgfpathmoveto{\pgfqpoint{4.408411in}{2.598323in}}%
\pgfpathcurveto{\pgfqpoint{4.419461in}{2.598323in}}{\pgfqpoint{4.430060in}{2.602713in}}{\pgfqpoint{4.437874in}{2.610527in}}%
\pgfpathcurveto{\pgfqpoint{4.445688in}{2.618341in}}{\pgfqpoint{4.450078in}{2.628940in}}{\pgfqpoint{4.450078in}{2.639990in}}%
\pgfpathcurveto{\pgfqpoint{4.450078in}{2.651040in}}{\pgfqpoint{4.445688in}{2.661639in}}{\pgfqpoint{4.437874in}{2.669453in}}%
\pgfpathcurveto{\pgfqpoint{4.430060in}{2.677266in}}{\pgfqpoint{4.419461in}{2.681656in}}{\pgfqpoint{4.408411in}{2.681656in}}%
\pgfpathcurveto{\pgfqpoint{4.397361in}{2.681656in}}{\pgfqpoint{4.386762in}{2.677266in}}{\pgfqpoint{4.378948in}{2.669453in}}%
\pgfpathcurveto{\pgfqpoint{4.371135in}{2.661639in}}{\pgfqpoint{4.366745in}{2.651040in}}{\pgfqpoint{4.366745in}{2.639990in}}%
\pgfpathcurveto{\pgfqpoint{4.366745in}{2.628940in}}{\pgfqpoint{4.371135in}{2.618341in}}{\pgfqpoint{4.378948in}{2.610527in}}%
\pgfpathcurveto{\pgfqpoint{4.386762in}{2.602713in}}{\pgfqpoint{4.397361in}{2.598323in}}{\pgfqpoint{4.408411in}{2.598323in}}%
\pgfpathclose%
\pgfusepath{stroke,fill}%
\end{pgfscope}%
\begin{pgfscope}%
\pgfpathrectangle{\pgfqpoint{0.481978in}{0.331635in}}{\pgfqpoint{9.300000in}{7.700000in}}%
\pgfusepath{clip}%
\pgfsetbuttcap%
\pgfsetroundjoin%
\definecolor{currentfill}{rgb}{1.000000,0.705882,0.509804}%
\pgfsetfillcolor{currentfill}%
\pgfsetlinewidth{0.481800pt}%
\definecolor{currentstroke}{rgb}{1.000000,1.000000,1.000000}%
\pgfsetstrokecolor{currentstroke}%
\pgfsetdash{}{0pt}%
\pgfpathmoveto{\pgfqpoint{1.290181in}{5.534021in}}%
\pgfpathcurveto{\pgfqpoint{1.301231in}{5.534021in}}{\pgfqpoint{1.311830in}{5.538412in}}{\pgfqpoint{1.319644in}{5.546225in}}%
\pgfpathcurveto{\pgfqpoint{1.327458in}{5.554039in}}{\pgfqpoint{1.331848in}{5.564638in}}{\pgfqpoint{1.331848in}{5.575688in}}%
\pgfpathcurveto{\pgfqpoint{1.331848in}{5.586738in}}{\pgfqpoint{1.327458in}{5.597337in}}{\pgfqpoint{1.319644in}{5.605151in}}%
\pgfpathcurveto{\pgfqpoint{1.311830in}{5.612964in}}{\pgfqpoint{1.301231in}{5.617355in}}{\pgfqpoint{1.290181in}{5.617355in}}%
\pgfpathcurveto{\pgfqpoint{1.279131in}{5.617355in}}{\pgfqpoint{1.268532in}{5.612964in}}{\pgfqpoint{1.260718in}{5.605151in}}%
\pgfpathcurveto{\pgfqpoint{1.252905in}{5.597337in}}{\pgfqpoint{1.248514in}{5.586738in}}{\pgfqpoint{1.248514in}{5.575688in}}%
\pgfpathcurveto{\pgfqpoint{1.248514in}{5.564638in}}{\pgfqpoint{1.252905in}{5.554039in}}{\pgfqpoint{1.260718in}{5.546225in}}%
\pgfpathcurveto{\pgfqpoint{1.268532in}{5.538412in}}{\pgfqpoint{1.279131in}{5.534021in}}{\pgfqpoint{1.290181in}{5.534021in}}%
\pgfpathclose%
\pgfusepath{stroke,fill}%
\end{pgfscope}%
\begin{pgfscope}%
\pgfpathrectangle{\pgfqpoint{0.481978in}{0.331635in}}{\pgfqpoint{9.300000in}{7.700000in}}%
\pgfusepath{clip}%
\pgfsetbuttcap%
\pgfsetroundjoin%
\definecolor{currentfill}{rgb}{1.000000,0.705882,0.509804}%
\pgfsetfillcolor{currentfill}%
\pgfsetlinewidth{0.481800pt}%
\definecolor{currentstroke}{rgb}{1.000000,1.000000,1.000000}%
\pgfsetstrokecolor{currentstroke}%
\pgfsetdash{}{0pt}%
\pgfpathmoveto{\pgfqpoint{2.624493in}{3.220628in}}%
\pgfpathcurveto{\pgfqpoint{2.635543in}{3.220628in}}{\pgfqpoint{2.646142in}{3.225018in}}{\pgfqpoint{2.653956in}{3.232832in}}%
\pgfpathcurveto{\pgfqpoint{2.661769in}{3.240645in}}{\pgfqpoint{2.666160in}{3.251244in}}{\pgfqpoint{2.666160in}{3.262294in}}%
\pgfpathcurveto{\pgfqpoint{2.666160in}{3.273345in}}{\pgfqpoint{2.661769in}{3.283944in}}{\pgfqpoint{2.653956in}{3.291757in}}%
\pgfpathcurveto{\pgfqpoint{2.646142in}{3.299571in}}{\pgfqpoint{2.635543in}{3.303961in}}{\pgfqpoint{2.624493in}{3.303961in}}%
\pgfpathcurveto{\pgfqpoint{2.613443in}{3.303961in}}{\pgfqpoint{2.602844in}{3.299571in}}{\pgfqpoint{2.595030in}{3.291757in}}%
\pgfpathcurveto{\pgfqpoint{2.587216in}{3.283944in}}{\pgfqpoint{2.582826in}{3.273345in}}{\pgfqpoint{2.582826in}{3.262294in}}%
\pgfpathcurveto{\pgfqpoint{2.582826in}{3.251244in}}{\pgfqpoint{2.587216in}{3.240645in}}{\pgfqpoint{2.595030in}{3.232832in}}%
\pgfpathcurveto{\pgfqpoint{2.602844in}{3.225018in}}{\pgfqpoint{2.613443in}{3.220628in}}{\pgfqpoint{2.624493in}{3.220628in}}%
\pgfpathclose%
\pgfusepath{stroke,fill}%
\end{pgfscope}%
\begin{pgfscope}%
\pgfpathrectangle{\pgfqpoint{0.481978in}{0.331635in}}{\pgfqpoint{9.300000in}{7.700000in}}%
\pgfusepath{clip}%
\pgfsetbuttcap%
\pgfsetroundjoin%
\definecolor{currentfill}{rgb}{1.000000,0.705882,0.509804}%
\pgfsetfillcolor{currentfill}%
\pgfsetlinewidth{0.481800pt}%
\definecolor{currentstroke}{rgb}{1.000000,1.000000,1.000000}%
\pgfsetstrokecolor{currentstroke}%
\pgfsetdash{}{0pt}%
\pgfpathmoveto{\pgfqpoint{5.315161in}{4.120952in}}%
\pgfpathcurveto{\pgfqpoint{5.326211in}{4.120952in}}{\pgfqpoint{5.336810in}{4.125342in}}{\pgfqpoint{5.344623in}{4.133156in}}%
\pgfpathcurveto{\pgfqpoint{5.352437in}{4.140970in}}{\pgfqpoint{5.356827in}{4.151569in}}{\pgfqpoint{5.356827in}{4.162619in}}%
\pgfpathcurveto{\pgfqpoint{5.356827in}{4.173669in}}{\pgfqpoint{5.352437in}{4.184268in}}{\pgfqpoint{5.344623in}{4.192082in}}%
\pgfpathcurveto{\pgfqpoint{5.336810in}{4.199895in}}{\pgfqpoint{5.326211in}{4.204285in}}{\pgfqpoint{5.315161in}{4.204285in}}%
\pgfpathcurveto{\pgfqpoint{5.304111in}{4.204285in}}{\pgfqpoint{5.293512in}{4.199895in}}{\pgfqpoint{5.285698in}{4.192082in}}%
\pgfpathcurveto{\pgfqpoint{5.277884in}{4.184268in}}{\pgfqpoint{5.273494in}{4.173669in}}{\pgfqpoint{5.273494in}{4.162619in}}%
\pgfpathcurveto{\pgfqpoint{5.273494in}{4.151569in}}{\pgfqpoint{5.277884in}{4.140970in}}{\pgfqpoint{5.285698in}{4.133156in}}%
\pgfpathcurveto{\pgfqpoint{5.293512in}{4.125342in}}{\pgfqpoint{5.304111in}{4.120952in}}{\pgfqpoint{5.315161in}{4.120952in}}%
\pgfpathclose%
\pgfusepath{stroke,fill}%
\end{pgfscope}%
\begin{pgfscope}%
\pgfpathrectangle{\pgfqpoint{0.481978in}{0.331635in}}{\pgfqpoint{9.300000in}{7.700000in}}%
\pgfusepath{clip}%
\pgfsetbuttcap%
\pgfsetroundjoin%
\definecolor{currentfill}{rgb}{1.000000,0.705882,0.509804}%
\pgfsetfillcolor{currentfill}%
\pgfsetlinewidth{0.481800pt}%
\definecolor{currentstroke}{rgb}{1.000000,1.000000,1.000000}%
\pgfsetstrokecolor{currentstroke}%
\pgfsetdash{}{0pt}%
\pgfpathmoveto{\pgfqpoint{4.062917in}{3.218584in}}%
\pgfpathcurveto{\pgfqpoint{4.073967in}{3.218584in}}{\pgfqpoint{4.084566in}{3.222974in}}{\pgfqpoint{4.092380in}{3.230788in}}%
\pgfpathcurveto{\pgfqpoint{4.100193in}{3.238601in}}{\pgfqpoint{4.104583in}{3.249200in}}{\pgfqpoint{4.104583in}{3.260250in}}%
\pgfpathcurveto{\pgfqpoint{4.104583in}{3.271300in}}{\pgfqpoint{4.100193in}{3.281899in}}{\pgfqpoint{4.092380in}{3.289713in}}%
\pgfpathcurveto{\pgfqpoint{4.084566in}{3.297527in}}{\pgfqpoint{4.073967in}{3.301917in}}{\pgfqpoint{4.062917in}{3.301917in}}%
\pgfpathcurveto{\pgfqpoint{4.051867in}{3.301917in}}{\pgfqpoint{4.041268in}{3.297527in}}{\pgfqpoint{4.033454in}{3.289713in}}%
\pgfpathcurveto{\pgfqpoint{4.025640in}{3.281899in}}{\pgfqpoint{4.021250in}{3.271300in}}{\pgfqpoint{4.021250in}{3.260250in}}%
\pgfpathcurveto{\pgfqpoint{4.021250in}{3.249200in}}{\pgfqpoint{4.025640in}{3.238601in}}{\pgfqpoint{4.033454in}{3.230788in}}%
\pgfpathcurveto{\pgfqpoint{4.041268in}{3.222974in}}{\pgfqpoint{4.051867in}{3.218584in}}{\pgfqpoint{4.062917in}{3.218584in}}%
\pgfpathclose%
\pgfusepath{stroke,fill}%
\end{pgfscope}%
\begin{pgfscope}%
\pgfpathrectangle{\pgfqpoint{0.481978in}{0.331635in}}{\pgfqpoint{9.300000in}{7.700000in}}%
\pgfusepath{clip}%
\pgfsetbuttcap%
\pgfsetroundjoin%
\definecolor{currentfill}{rgb}{1.000000,0.705882,0.509804}%
\pgfsetfillcolor{currentfill}%
\pgfsetlinewidth{0.481800pt}%
\definecolor{currentstroke}{rgb}{1.000000,1.000000,1.000000}%
\pgfsetstrokecolor{currentstroke}%
\pgfsetdash{}{0pt}%
\pgfpathmoveto{\pgfqpoint{4.828561in}{4.028463in}}%
\pgfpathcurveto{\pgfqpoint{4.839611in}{4.028463in}}{\pgfqpoint{4.850211in}{4.032853in}}{\pgfqpoint{4.858024in}{4.040667in}}%
\pgfpathcurveto{\pgfqpoint{4.865838in}{4.048480in}}{\pgfqpoint{4.870228in}{4.059079in}}{\pgfqpoint{4.870228in}{4.070130in}}%
\pgfpathcurveto{\pgfqpoint{4.870228in}{4.081180in}}{\pgfqpoint{4.865838in}{4.091779in}}{\pgfqpoint{4.858024in}{4.099592in}}%
\pgfpathcurveto{\pgfqpoint{4.850211in}{4.107406in}}{\pgfqpoint{4.839611in}{4.111796in}}{\pgfqpoint{4.828561in}{4.111796in}}%
\pgfpathcurveto{\pgfqpoint{4.817511in}{4.111796in}}{\pgfqpoint{4.806912in}{4.107406in}}{\pgfqpoint{4.799099in}{4.099592in}}%
\pgfpathcurveto{\pgfqpoint{4.791285in}{4.091779in}}{\pgfqpoint{4.786895in}{4.081180in}}{\pgfqpoint{4.786895in}{4.070130in}}%
\pgfpathcurveto{\pgfqpoint{4.786895in}{4.059079in}}{\pgfqpoint{4.791285in}{4.048480in}}{\pgfqpoint{4.799099in}{4.040667in}}%
\pgfpathcurveto{\pgfqpoint{4.806912in}{4.032853in}}{\pgfqpoint{4.817511in}{4.028463in}}{\pgfqpoint{4.828561in}{4.028463in}}%
\pgfpathclose%
\pgfusepath{stroke,fill}%
\end{pgfscope}%
\begin{pgfscope}%
\pgfpathrectangle{\pgfqpoint{0.481978in}{0.331635in}}{\pgfqpoint{9.300000in}{7.700000in}}%
\pgfusepath{clip}%
\pgfsetbuttcap%
\pgfsetroundjoin%
\definecolor{currentfill}{rgb}{1.000000,0.705882,0.509804}%
\pgfsetfillcolor{currentfill}%
\pgfsetlinewidth{0.481800pt}%
\definecolor{currentstroke}{rgb}{1.000000,1.000000,1.000000}%
\pgfsetstrokecolor{currentstroke}%
\pgfsetdash{}{0pt}%
\pgfpathmoveto{\pgfqpoint{2.387650in}{3.484085in}}%
\pgfpathcurveto{\pgfqpoint{2.398701in}{3.484085in}}{\pgfqpoint{2.409300in}{3.488475in}}{\pgfqpoint{2.417113in}{3.496289in}}%
\pgfpathcurveto{\pgfqpoint{2.424927in}{3.504102in}}{\pgfqpoint{2.429317in}{3.514701in}}{\pgfqpoint{2.429317in}{3.525751in}}%
\pgfpathcurveto{\pgfqpoint{2.429317in}{3.536801in}}{\pgfqpoint{2.424927in}{3.547400in}}{\pgfqpoint{2.417113in}{3.555214in}}%
\pgfpathcurveto{\pgfqpoint{2.409300in}{3.563028in}}{\pgfqpoint{2.398701in}{3.567418in}}{\pgfqpoint{2.387650in}{3.567418in}}%
\pgfpathcurveto{\pgfqpoint{2.376600in}{3.567418in}}{\pgfqpoint{2.366001in}{3.563028in}}{\pgfqpoint{2.358188in}{3.555214in}}%
\pgfpathcurveto{\pgfqpoint{2.350374in}{3.547400in}}{\pgfqpoint{2.345984in}{3.536801in}}{\pgfqpoint{2.345984in}{3.525751in}}%
\pgfpathcurveto{\pgfqpoint{2.345984in}{3.514701in}}{\pgfqpoint{2.350374in}{3.504102in}}{\pgfqpoint{2.358188in}{3.496289in}}%
\pgfpathcurveto{\pgfqpoint{2.366001in}{3.488475in}}{\pgfqpoint{2.376600in}{3.484085in}}{\pgfqpoint{2.387650in}{3.484085in}}%
\pgfpathclose%
\pgfusepath{stroke,fill}%
\end{pgfscope}%
\begin{pgfscope}%
\pgfpathrectangle{\pgfqpoint{0.481978in}{0.331635in}}{\pgfqpoint{9.300000in}{7.700000in}}%
\pgfusepath{clip}%
\pgfsetbuttcap%
\pgfsetroundjoin%
\definecolor{currentfill}{rgb}{1.000000,0.705882,0.509804}%
\pgfsetfillcolor{currentfill}%
\pgfsetlinewidth{0.481800pt}%
\definecolor{currentstroke}{rgb}{1.000000,1.000000,1.000000}%
\pgfsetstrokecolor{currentstroke}%
\pgfsetdash{}{0pt}%
\pgfpathmoveto{\pgfqpoint{3.134591in}{4.885871in}}%
\pgfpathcurveto{\pgfqpoint{3.145641in}{4.885871in}}{\pgfqpoint{3.156240in}{4.890261in}}{\pgfqpoint{3.164054in}{4.898075in}}%
\pgfpathcurveto{\pgfqpoint{3.171867in}{4.905888in}}{\pgfqpoint{3.176258in}{4.916487in}}{\pgfqpoint{3.176258in}{4.927538in}}%
\pgfpathcurveto{\pgfqpoint{3.176258in}{4.938588in}}{\pgfqpoint{3.171867in}{4.949187in}}{\pgfqpoint{3.164054in}{4.957000in}}%
\pgfpathcurveto{\pgfqpoint{3.156240in}{4.964814in}}{\pgfqpoint{3.145641in}{4.969204in}}{\pgfqpoint{3.134591in}{4.969204in}}%
\pgfpathcurveto{\pgfqpoint{3.123541in}{4.969204in}}{\pgfqpoint{3.112942in}{4.964814in}}{\pgfqpoint{3.105128in}{4.957000in}}%
\pgfpathcurveto{\pgfqpoint{3.097315in}{4.949187in}}{\pgfqpoint{3.092924in}{4.938588in}}{\pgfqpoint{3.092924in}{4.927538in}}%
\pgfpathcurveto{\pgfqpoint{3.092924in}{4.916487in}}{\pgfqpoint{3.097315in}{4.905888in}}{\pgfqpoint{3.105128in}{4.898075in}}%
\pgfpathcurveto{\pgfqpoint{3.112942in}{4.890261in}}{\pgfqpoint{3.123541in}{4.885871in}}{\pgfqpoint{3.134591in}{4.885871in}}%
\pgfpathclose%
\pgfusepath{stroke,fill}%
\end{pgfscope}%
\begin{pgfscope}%
\pgfpathrectangle{\pgfqpoint{0.481978in}{0.331635in}}{\pgfqpoint{9.300000in}{7.700000in}}%
\pgfusepath{clip}%
\pgfsetbuttcap%
\pgfsetroundjoin%
\definecolor{currentfill}{rgb}{1.000000,0.705882,0.509804}%
\pgfsetfillcolor{currentfill}%
\pgfsetlinewidth{0.481800pt}%
\definecolor{currentstroke}{rgb}{1.000000,1.000000,1.000000}%
\pgfsetstrokecolor{currentstroke}%
\pgfsetdash{}{0pt}%
\pgfpathmoveto{\pgfqpoint{1.638268in}{4.660633in}}%
\pgfpathcurveto{\pgfqpoint{1.649318in}{4.660633in}}{\pgfqpoint{1.659917in}{4.665024in}}{\pgfqpoint{1.667731in}{4.672837in}}%
\pgfpathcurveto{\pgfqpoint{1.675544in}{4.680651in}}{\pgfqpoint{1.679935in}{4.691250in}}{\pgfqpoint{1.679935in}{4.702300in}}%
\pgfpathcurveto{\pgfqpoint{1.679935in}{4.713350in}}{\pgfqpoint{1.675544in}{4.723949in}}{\pgfqpoint{1.667731in}{4.731763in}}%
\pgfpathcurveto{\pgfqpoint{1.659917in}{4.739576in}}{\pgfqpoint{1.649318in}{4.743967in}}{\pgfqpoint{1.638268in}{4.743967in}}%
\pgfpathcurveto{\pgfqpoint{1.627218in}{4.743967in}}{\pgfqpoint{1.616619in}{4.739576in}}{\pgfqpoint{1.608805in}{4.731763in}}%
\pgfpathcurveto{\pgfqpoint{1.600992in}{4.723949in}}{\pgfqpoint{1.596601in}{4.713350in}}{\pgfqpoint{1.596601in}{4.702300in}}%
\pgfpathcurveto{\pgfqpoint{1.596601in}{4.691250in}}{\pgfqpoint{1.600992in}{4.680651in}}{\pgfqpoint{1.608805in}{4.672837in}}%
\pgfpathcurveto{\pgfqpoint{1.616619in}{4.665024in}}{\pgfqpoint{1.627218in}{4.660633in}}{\pgfqpoint{1.638268in}{4.660633in}}%
\pgfpathclose%
\pgfusepath{stroke,fill}%
\end{pgfscope}%
\begin{pgfscope}%
\pgfpathrectangle{\pgfqpoint{0.481978in}{0.331635in}}{\pgfqpoint{9.300000in}{7.700000in}}%
\pgfusepath{clip}%
\pgfsetbuttcap%
\pgfsetroundjoin%
\definecolor{currentfill}{rgb}{1.000000,0.705882,0.509804}%
\pgfsetfillcolor{currentfill}%
\pgfsetlinewidth{0.481800pt}%
\definecolor{currentstroke}{rgb}{1.000000,1.000000,1.000000}%
\pgfsetstrokecolor{currentstroke}%
\pgfsetdash{}{0pt}%
\pgfpathmoveto{\pgfqpoint{4.067183in}{6.795571in}}%
\pgfpathcurveto{\pgfqpoint{4.078233in}{6.795571in}}{\pgfqpoint{4.088832in}{6.799962in}}{\pgfqpoint{4.096645in}{6.807775in}}%
\pgfpathcurveto{\pgfqpoint{4.104459in}{6.815589in}}{\pgfqpoint{4.108849in}{6.826188in}}{\pgfqpoint{4.108849in}{6.837238in}}%
\pgfpathcurveto{\pgfqpoint{4.108849in}{6.848288in}}{\pgfqpoint{4.104459in}{6.858887in}}{\pgfqpoint{4.096645in}{6.866701in}}%
\pgfpathcurveto{\pgfqpoint{4.088832in}{6.874514in}}{\pgfqpoint{4.078233in}{6.878905in}}{\pgfqpoint{4.067183in}{6.878905in}}%
\pgfpathcurveto{\pgfqpoint{4.056132in}{6.878905in}}{\pgfqpoint{4.045533in}{6.874514in}}{\pgfqpoint{4.037720in}{6.866701in}}%
\pgfpathcurveto{\pgfqpoint{4.029906in}{6.858887in}}{\pgfqpoint{4.025516in}{6.848288in}}{\pgfqpoint{4.025516in}{6.837238in}}%
\pgfpathcurveto{\pgfqpoint{4.025516in}{6.826188in}}{\pgfqpoint{4.029906in}{6.815589in}}{\pgfqpoint{4.037720in}{6.807775in}}%
\pgfpathcurveto{\pgfqpoint{4.045533in}{6.799962in}}{\pgfqpoint{4.056132in}{6.795571in}}{\pgfqpoint{4.067183in}{6.795571in}}%
\pgfpathclose%
\pgfusepath{stroke,fill}%
\end{pgfscope}%
\begin{pgfscope}%
\pgfpathrectangle{\pgfqpoint{0.481978in}{0.331635in}}{\pgfqpoint{9.300000in}{7.700000in}}%
\pgfusepath{clip}%
\pgfsetbuttcap%
\pgfsetroundjoin%
\definecolor{currentfill}{rgb}{1.000000,0.705882,0.509804}%
\pgfsetfillcolor{currentfill}%
\pgfsetlinewidth{0.481800pt}%
\definecolor{currentstroke}{rgb}{1.000000,1.000000,1.000000}%
\pgfsetstrokecolor{currentstroke}%
\pgfsetdash{}{0pt}%
\pgfpathmoveto{\pgfqpoint{3.079878in}{3.877328in}}%
\pgfpathcurveto{\pgfqpoint{3.090928in}{3.877328in}}{\pgfqpoint{3.101527in}{3.881718in}}{\pgfqpoint{3.109341in}{3.889532in}}%
\pgfpathcurveto{\pgfqpoint{3.117155in}{3.897345in}}{\pgfqpoint{3.121545in}{3.907944in}}{\pgfqpoint{3.121545in}{3.918994in}}%
\pgfpathcurveto{\pgfqpoint{3.121545in}{3.930045in}}{\pgfqpoint{3.117155in}{3.940644in}}{\pgfqpoint{3.109341in}{3.948457in}}%
\pgfpathcurveto{\pgfqpoint{3.101527in}{3.956271in}}{\pgfqpoint{3.090928in}{3.960661in}}{\pgfqpoint{3.079878in}{3.960661in}}%
\pgfpathcurveto{\pgfqpoint{3.068828in}{3.960661in}}{\pgfqpoint{3.058229in}{3.956271in}}{\pgfqpoint{3.050416in}{3.948457in}}%
\pgfpathcurveto{\pgfqpoint{3.042602in}{3.940644in}}{\pgfqpoint{3.038212in}{3.930045in}}{\pgfqpoint{3.038212in}{3.918994in}}%
\pgfpathcurveto{\pgfqpoint{3.038212in}{3.907944in}}{\pgfqpoint{3.042602in}{3.897345in}}{\pgfqpoint{3.050416in}{3.889532in}}%
\pgfpathcurveto{\pgfqpoint{3.058229in}{3.881718in}}{\pgfqpoint{3.068828in}{3.877328in}}{\pgfqpoint{3.079878in}{3.877328in}}%
\pgfpathclose%
\pgfusepath{stroke,fill}%
\end{pgfscope}%
\begin{pgfscope}%
\pgfpathrectangle{\pgfqpoint{0.481978in}{0.331635in}}{\pgfqpoint{9.300000in}{7.700000in}}%
\pgfusepath{clip}%
\pgfsetbuttcap%
\pgfsetroundjoin%
\definecolor{currentfill}{rgb}{1.000000,0.705882,0.509804}%
\pgfsetfillcolor{currentfill}%
\pgfsetlinewidth{0.481800pt}%
\definecolor{currentstroke}{rgb}{1.000000,1.000000,1.000000}%
\pgfsetstrokecolor{currentstroke}%
\pgfsetdash{}{0pt}%
\pgfpathmoveto{\pgfqpoint{3.762277in}{2.592935in}}%
\pgfpathcurveto{\pgfqpoint{3.773328in}{2.592935in}}{\pgfqpoint{3.783927in}{2.597326in}}{\pgfqpoint{3.791740in}{2.605139in}}%
\pgfpathcurveto{\pgfqpoint{3.799554in}{2.612953in}}{\pgfqpoint{3.803944in}{2.623552in}}{\pgfqpoint{3.803944in}{2.634602in}}%
\pgfpathcurveto{\pgfqpoint{3.803944in}{2.645652in}}{\pgfqpoint{3.799554in}{2.656251in}}{\pgfqpoint{3.791740in}{2.664065in}}%
\pgfpathcurveto{\pgfqpoint{3.783927in}{2.671878in}}{\pgfqpoint{3.773328in}{2.676269in}}{\pgfqpoint{3.762277in}{2.676269in}}%
\pgfpathcurveto{\pgfqpoint{3.751227in}{2.676269in}}{\pgfqpoint{3.740628in}{2.671878in}}{\pgfqpoint{3.732815in}{2.664065in}}%
\pgfpathcurveto{\pgfqpoint{3.725001in}{2.656251in}}{\pgfqpoint{3.720611in}{2.645652in}}{\pgfqpoint{3.720611in}{2.634602in}}%
\pgfpathcurveto{\pgfqpoint{3.720611in}{2.623552in}}{\pgfqpoint{3.725001in}{2.612953in}}{\pgfqpoint{3.732815in}{2.605139in}}%
\pgfpathcurveto{\pgfqpoint{3.740628in}{2.597326in}}{\pgfqpoint{3.751227in}{2.592935in}}{\pgfqpoint{3.762277in}{2.592935in}}%
\pgfpathclose%
\pgfusepath{stroke,fill}%
\end{pgfscope}%
\begin{pgfscope}%
\pgfpathrectangle{\pgfqpoint{0.481978in}{0.331635in}}{\pgfqpoint{9.300000in}{7.700000in}}%
\pgfusepath{clip}%
\pgfsetbuttcap%
\pgfsetroundjoin%
\definecolor{currentfill}{rgb}{1.000000,0.705882,0.509804}%
\pgfsetfillcolor{currentfill}%
\pgfsetlinewidth{0.481800pt}%
\definecolor{currentstroke}{rgb}{1.000000,1.000000,1.000000}%
\pgfsetstrokecolor{currentstroke}%
\pgfsetdash{}{0pt}%
\pgfpathmoveto{\pgfqpoint{3.580814in}{3.862560in}}%
\pgfpathcurveto{\pgfqpoint{3.591864in}{3.862560in}}{\pgfqpoint{3.602463in}{3.866950in}}{\pgfqpoint{3.610277in}{3.874764in}}%
\pgfpathcurveto{\pgfqpoint{3.618091in}{3.882578in}}{\pgfqpoint{3.622481in}{3.893177in}}{\pgfqpoint{3.622481in}{3.904227in}}%
\pgfpathcurveto{\pgfqpoint{3.622481in}{3.915277in}}{\pgfqpoint{3.618091in}{3.925876in}}{\pgfqpoint{3.610277in}{3.933689in}}%
\pgfpathcurveto{\pgfqpoint{3.602463in}{3.941503in}}{\pgfqpoint{3.591864in}{3.945893in}}{\pgfqpoint{3.580814in}{3.945893in}}%
\pgfpathcurveto{\pgfqpoint{3.569764in}{3.945893in}}{\pgfqpoint{3.559165in}{3.941503in}}{\pgfqpoint{3.551352in}{3.933689in}}%
\pgfpathcurveto{\pgfqpoint{3.543538in}{3.925876in}}{\pgfqpoint{3.539148in}{3.915277in}}{\pgfqpoint{3.539148in}{3.904227in}}%
\pgfpathcurveto{\pgfqpoint{3.539148in}{3.893177in}}{\pgfqpoint{3.543538in}{3.882578in}}{\pgfqpoint{3.551352in}{3.874764in}}%
\pgfpathcurveto{\pgfqpoint{3.559165in}{3.866950in}}{\pgfqpoint{3.569764in}{3.862560in}}{\pgfqpoint{3.580814in}{3.862560in}}%
\pgfpathclose%
\pgfusepath{stroke,fill}%
\end{pgfscope}%
\begin{pgfscope}%
\pgfpathrectangle{\pgfqpoint{0.481978in}{0.331635in}}{\pgfqpoint{9.300000in}{7.700000in}}%
\pgfusepath{clip}%
\pgfsetbuttcap%
\pgfsetroundjoin%
\definecolor{currentfill}{rgb}{1.000000,0.705882,0.509804}%
\pgfsetfillcolor{currentfill}%
\pgfsetlinewidth{0.481800pt}%
\definecolor{currentstroke}{rgb}{1.000000,1.000000,1.000000}%
\pgfsetstrokecolor{currentstroke}%
\pgfsetdash{}{0pt}%
\pgfpathmoveto{\pgfqpoint{4.361455in}{2.789942in}}%
\pgfpathcurveto{\pgfqpoint{4.372505in}{2.789942in}}{\pgfqpoint{4.383104in}{2.794332in}}{\pgfqpoint{4.390918in}{2.802146in}}%
\pgfpathcurveto{\pgfqpoint{4.398732in}{2.809960in}}{\pgfqpoint{4.403122in}{2.820559in}}{\pgfqpoint{4.403122in}{2.831609in}}%
\pgfpathcurveto{\pgfqpoint{4.403122in}{2.842659in}}{\pgfqpoint{4.398732in}{2.853258in}}{\pgfqpoint{4.390918in}{2.861072in}}%
\pgfpathcurveto{\pgfqpoint{4.383104in}{2.868885in}}{\pgfqpoint{4.372505in}{2.873276in}}{\pgfqpoint{4.361455in}{2.873276in}}%
\pgfpathcurveto{\pgfqpoint{4.350405in}{2.873276in}}{\pgfqpoint{4.339806in}{2.868885in}}{\pgfqpoint{4.331992in}{2.861072in}}%
\pgfpathcurveto{\pgfqpoint{4.324179in}{2.853258in}}{\pgfqpoint{4.319789in}{2.842659in}}{\pgfqpoint{4.319789in}{2.831609in}}%
\pgfpathcurveto{\pgfqpoint{4.319789in}{2.820559in}}{\pgfqpoint{4.324179in}{2.809960in}}{\pgfqpoint{4.331992in}{2.802146in}}%
\pgfpathcurveto{\pgfqpoint{4.339806in}{2.794332in}}{\pgfqpoint{4.350405in}{2.789942in}}{\pgfqpoint{4.361455in}{2.789942in}}%
\pgfpathclose%
\pgfusepath{stroke,fill}%
\end{pgfscope}%
\begin{pgfscope}%
\pgfpathrectangle{\pgfqpoint{0.481978in}{0.331635in}}{\pgfqpoint{9.300000in}{7.700000in}}%
\pgfusepath{clip}%
\pgfsetbuttcap%
\pgfsetroundjoin%
\definecolor{currentfill}{rgb}{1.000000,0.705882,0.509804}%
\pgfsetfillcolor{currentfill}%
\pgfsetlinewidth{0.481800pt}%
\definecolor{currentstroke}{rgb}{1.000000,1.000000,1.000000}%
\pgfsetstrokecolor{currentstroke}%
\pgfsetdash{}{0pt}%
\pgfpathmoveto{\pgfqpoint{3.821252in}{6.879688in}}%
\pgfpathcurveto{\pgfqpoint{3.832302in}{6.879688in}}{\pgfqpoint{3.842901in}{6.884078in}}{\pgfqpoint{3.850714in}{6.891892in}}%
\pgfpathcurveto{\pgfqpoint{3.858528in}{6.899706in}}{\pgfqpoint{3.862918in}{6.910305in}}{\pgfqpoint{3.862918in}{6.921355in}}%
\pgfpathcurveto{\pgfqpoint{3.862918in}{6.932405in}}{\pgfqpoint{3.858528in}{6.943004in}}{\pgfqpoint{3.850714in}{6.950817in}}%
\pgfpathcurveto{\pgfqpoint{3.842901in}{6.958631in}}{\pgfqpoint{3.832302in}{6.963021in}}{\pgfqpoint{3.821252in}{6.963021in}}%
\pgfpathcurveto{\pgfqpoint{3.810202in}{6.963021in}}{\pgfqpoint{3.799602in}{6.958631in}}{\pgfqpoint{3.791789in}{6.950817in}}%
\pgfpathcurveto{\pgfqpoint{3.783975in}{6.943004in}}{\pgfqpoint{3.779585in}{6.932405in}}{\pgfqpoint{3.779585in}{6.921355in}}%
\pgfpathcurveto{\pgfqpoint{3.779585in}{6.910305in}}{\pgfqpoint{3.783975in}{6.899706in}}{\pgfqpoint{3.791789in}{6.891892in}}%
\pgfpathcurveto{\pgfqpoint{3.799602in}{6.884078in}}{\pgfqpoint{3.810202in}{6.879688in}}{\pgfqpoint{3.821252in}{6.879688in}}%
\pgfpathclose%
\pgfusepath{stroke,fill}%
\end{pgfscope}%
\begin{pgfscope}%
\pgfpathrectangle{\pgfqpoint{0.481978in}{0.331635in}}{\pgfqpoint{9.300000in}{7.700000in}}%
\pgfusepath{clip}%
\pgfsetbuttcap%
\pgfsetroundjoin%
\definecolor{currentfill}{rgb}{1.000000,0.705882,0.509804}%
\pgfsetfillcolor{currentfill}%
\pgfsetlinewidth{0.481800pt}%
\definecolor{currentstroke}{rgb}{1.000000,1.000000,1.000000}%
\pgfsetstrokecolor{currentstroke}%
\pgfsetdash{}{0pt}%
\pgfpathmoveto{\pgfqpoint{2.933293in}{5.599270in}}%
\pgfpathcurveto{\pgfqpoint{2.944344in}{5.599270in}}{\pgfqpoint{2.954943in}{5.603660in}}{\pgfqpoint{2.962756in}{5.611473in}}%
\pgfpathcurveto{\pgfqpoint{2.970570in}{5.619287in}}{\pgfqpoint{2.974960in}{5.629886in}}{\pgfqpoint{2.974960in}{5.640936in}}%
\pgfpathcurveto{\pgfqpoint{2.974960in}{5.651986in}}{\pgfqpoint{2.970570in}{5.662585in}}{\pgfqpoint{2.962756in}{5.670399in}}%
\pgfpathcurveto{\pgfqpoint{2.954943in}{5.678213in}}{\pgfqpoint{2.944344in}{5.682603in}}{\pgfqpoint{2.933293in}{5.682603in}}%
\pgfpathcurveto{\pgfqpoint{2.922243in}{5.682603in}}{\pgfqpoint{2.911644in}{5.678213in}}{\pgfqpoint{2.903831in}{5.670399in}}%
\pgfpathcurveto{\pgfqpoint{2.896017in}{5.662585in}}{\pgfqpoint{2.891627in}{5.651986in}}{\pgfqpoint{2.891627in}{5.640936in}}%
\pgfpathcurveto{\pgfqpoint{2.891627in}{5.629886in}}{\pgfqpoint{2.896017in}{5.619287in}}{\pgfqpoint{2.903831in}{5.611473in}}%
\pgfpathcurveto{\pgfqpoint{2.911644in}{5.603660in}}{\pgfqpoint{2.922243in}{5.599270in}}{\pgfqpoint{2.933293in}{5.599270in}}%
\pgfpathclose%
\pgfusepath{stroke,fill}%
\end{pgfscope}%
\begin{pgfscope}%
\pgfpathrectangle{\pgfqpoint{0.481978in}{0.331635in}}{\pgfqpoint{9.300000in}{7.700000in}}%
\pgfusepath{clip}%
\pgfsetbuttcap%
\pgfsetroundjoin%
\definecolor{currentfill}{rgb}{1.000000,0.705882,0.509804}%
\pgfsetfillcolor{currentfill}%
\pgfsetlinewidth{0.481800pt}%
\definecolor{currentstroke}{rgb}{1.000000,1.000000,1.000000}%
\pgfsetstrokecolor{currentstroke}%
\pgfsetdash{}{0pt}%
\pgfpathmoveto{\pgfqpoint{4.420071in}{4.726582in}}%
\pgfpathcurveto{\pgfqpoint{4.431121in}{4.726582in}}{\pgfqpoint{4.441720in}{4.730973in}}{\pgfqpoint{4.449534in}{4.738786in}}%
\pgfpathcurveto{\pgfqpoint{4.457348in}{4.746600in}}{\pgfqpoint{4.461738in}{4.757199in}}{\pgfqpoint{4.461738in}{4.768249in}}%
\pgfpathcurveto{\pgfqpoint{4.461738in}{4.779299in}}{\pgfqpoint{4.457348in}{4.789898in}}{\pgfqpoint{4.449534in}{4.797712in}}%
\pgfpathcurveto{\pgfqpoint{4.441720in}{4.805525in}}{\pgfqpoint{4.431121in}{4.809916in}}{\pgfqpoint{4.420071in}{4.809916in}}%
\pgfpathcurveto{\pgfqpoint{4.409021in}{4.809916in}}{\pgfqpoint{4.398422in}{4.805525in}}{\pgfqpoint{4.390609in}{4.797712in}}%
\pgfpathcurveto{\pgfqpoint{4.382795in}{4.789898in}}{\pgfqpoint{4.378405in}{4.779299in}}{\pgfqpoint{4.378405in}{4.768249in}}%
\pgfpathcurveto{\pgfqpoint{4.378405in}{4.757199in}}{\pgfqpoint{4.382795in}{4.746600in}}{\pgfqpoint{4.390609in}{4.738786in}}%
\pgfpathcurveto{\pgfqpoint{4.398422in}{4.730973in}}{\pgfqpoint{4.409021in}{4.726582in}}{\pgfqpoint{4.420071in}{4.726582in}}%
\pgfpathclose%
\pgfusepath{stroke,fill}%
\end{pgfscope}%
\begin{pgfscope}%
\pgfpathrectangle{\pgfqpoint{0.481978in}{0.331635in}}{\pgfqpoint{9.300000in}{7.700000in}}%
\pgfusepath{clip}%
\pgfsetbuttcap%
\pgfsetroundjoin%
\definecolor{currentfill}{rgb}{1.000000,0.705882,0.509804}%
\pgfsetfillcolor{currentfill}%
\pgfsetlinewidth{0.481800pt}%
\definecolor{currentstroke}{rgb}{1.000000,1.000000,1.000000}%
\pgfsetstrokecolor{currentstroke}%
\pgfsetdash{}{0pt}%
\pgfpathmoveto{\pgfqpoint{2.246703in}{4.966839in}}%
\pgfpathcurveto{\pgfqpoint{2.257753in}{4.966839in}}{\pgfqpoint{2.268352in}{4.971230in}}{\pgfqpoint{2.276166in}{4.979043in}}%
\pgfpathcurveto{\pgfqpoint{2.283979in}{4.986857in}}{\pgfqpoint{2.288370in}{4.997456in}}{\pgfqpoint{2.288370in}{5.008506in}}%
\pgfpathcurveto{\pgfqpoint{2.288370in}{5.019556in}}{\pgfqpoint{2.283979in}{5.030155in}}{\pgfqpoint{2.276166in}{5.037969in}}%
\pgfpathcurveto{\pgfqpoint{2.268352in}{5.045782in}}{\pgfqpoint{2.257753in}{5.050173in}}{\pgfqpoint{2.246703in}{5.050173in}}%
\pgfpathcurveto{\pgfqpoint{2.235653in}{5.050173in}}{\pgfqpoint{2.225054in}{5.045782in}}{\pgfqpoint{2.217240in}{5.037969in}}%
\pgfpathcurveto{\pgfqpoint{2.209427in}{5.030155in}}{\pgfqpoint{2.205036in}{5.019556in}}{\pgfqpoint{2.205036in}{5.008506in}}%
\pgfpathcurveto{\pgfqpoint{2.205036in}{4.997456in}}{\pgfqpoint{2.209427in}{4.986857in}}{\pgfqpoint{2.217240in}{4.979043in}}%
\pgfpathcurveto{\pgfqpoint{2.225054in}{4.971230in}}{\pgfqpoint{2.235653in}{4.966839in}}{\pgfqpoint{2.246703in}{4.966839in}}%
\pgfpathclose%
\pgfusepath{stroke,fill}%
\end{pgfscope}%
\begin{pgfscope}%
\pgfpathrectangle{\pgfqpoint{0.481978in}{0.331635in}}{\pgfqpoint{9.300000in}{7.700000in}}%
\pgfusepath{clip}%
\pgfsetbuttcap%
\pgfsetroundjoin%
\definecolor{currentfill}{rgb}{1.000000,0.705882,0.509804}%
\pgfsetfillcolor{currentfill}%
\pgfsetlinewidth{0.481800pt}%
\definecolor{currentstroke}{rgb}{1.000000,1.000000,1.000000}%
\pgfsetstrokecolor{currentstroke}%
\pgfsetdash{}{0pt}%
\pgfpathmoveto{\pgfqpoint{4.205087in}{5.595225in}}%
\pgfpathcurveto{\pgfqpoint{4.216137in}{5.595225in}}{\pgfqpoint{4.226736in}{5.599616in}}{\pgfqpoint{4.234550in}{5.607429in}}%
\pgfpathcurveto{\pgfqpoint{4.242363in}{5.615243in}}{\pgfqpoint{4.246753in}{5.625842in}}{\pgfqpoint{4.246753in}{5.636892in}}%
\pgfpathcurveto{\pgfqpoint{4.246753in}{5.647942in}}{\pgfqpoint{4.242363in}{5.658541in}}{\pgfqpoint{4.234550in}{5.666355in}}%
\pgfpathcurveto{\pgfqpoint{4.226736in}{5.674168in}}{\pgfqpoint{4.216137in}{5.678559in}}{\pgfqpoint{4.205087in}{5.678559in}}%
\pgfpathcurveto{\pgfqpoint{4.194037in}{5.678559in}}{\pgfqpoint{4.183438in}{5.674168in}}{\pgfqpoint{4.175624in}{5.666355in}}%
\pgfpathcurveto{\pgfqpoint{4.167810in}{5.658541in}}{\pgfqpoint{4.163420in}{5.647942in}}{\pgfqpoint{4.163420in}{5.636892in}}%
\pgfpathcurveto{\pgfqpoint{4.163420in}{5.625842in}}{\pgfqpoint{4.167810in}{5.615243in}}{\pgfqpoint{4.175624in}{5.607429in}}%
\pgfpathcurveto{\pgfqpoint{4.183438in}{5.599616in}}{\pgfqpoint{4.194037in}{5.595225in}}{\pgfqpoint{4.205087in}{5.595225in}}%
\pgfpathclose%
\pgfusepath{stroke,fill}%
\end{pgfscope}%
\begin{pgfscope}%
\pgfpathrectangle{\pgfqpoint{0.481978in}{0.331635in}}{\pgfqpoint{9.300000in}{7.700000in}}%
\pgfusepath{clip}%
\pgfsetbuttcap%
\pgfsetroundjoin%
\definecolor{currentfill}{rgb}{1.000000,0.705882,0.509804}%
\pgfsetfillcolor{currentfill}%
\pgfsetlinewidth{0.481800pt}%
\definecolor{currentstroke}{rgb}{1.000000,1.000000,1.000000}%
\pgfsetstrokecolor{currentstroke}%
\pgfsetdash{}{0pt}%
\pgfpathmoveto{\pgfqpoint{3.838690in}{4.184095in}}%
\pgfpathcurveto{\pgfqpoint{3.849740in}{4.184095in}}{\pgfqpoint{3.860339in}{4.188485in}}{\pgfqpoint{3.868152in}{4.196298in}}%
\pgfpathcurveto{\pgfqpoint{3.875966in}{4.204112in}}{\pgfqpoint{3.880356in}{4.214711in}}{\pgfqpoint{3.880356in}{4.225761in}}%
\pgfpathcurveto{\pgfqpoint{3.880356in}{4.236811in}}{\pgfqpoint{3.875966in}{4.247410in}}{\pgfqpoint{3.868152in}{4.255224in}}%
\pgfpathcurveto{\pgfqpoint{3.860339in}{4.263038in}}{\pgfqpoint{3.849740in}{4.267428in}}{\pgfqpoint{3.838690in}{4.267428in}}%
\pgfpathcurveto{\pgfqpoint{3.827639in}{4.267428in}}{\pgfqpoint{3.817040in}{4.263038in}}{\pgfqpoint{3.809227in}{4.255224in}}%
\pgfpathcurveto{\pgfqpoint{3.801413in}{4.247410in}}{\pgfqpoint{3.797023in}{4.236811in}}{\pgfqpoint{3.797023in}{4.225761in}}%
\pgfpathcurveto{\pgfqpoint{3.797023in}{4.214711in}}{\pgfqpoint{3.801413in}{4.204112in}}{\pgfqpoint{3.809227in}{4.196298in}}%
\pgfpathcurveto{\pgfqpoint{3.817040in}{4.188485in}}{\pgfqpoint{3.827639in}{4.184095in}}{\pgfqpoint{3.838690in}{4.184095in}}%
\pgfpathclose%
\pgfusepath{stroke,fill}%
\end{pgfscope}%
\begin{pgfscope}%
\pgfpathrectangle{\pgfqpoint{0.481978in}{0.331635in}}{\pgfqpoint{9.300000in}{7.700000in}}%
\pgfusepath{clip}%
\pgfsetbuttcap%
\pgfsetroundjoin%
\definecolor{currentfill}{rgb}{1.000000,0.705882,0.509804}%
\pgfsetfillcolor{currentfill}%
\pgfsetlinewidth{0.481800pt}%
\definecolor{currentstroke}{rgb}{1.000000,1.000000,1.000000}%
\pgfsetstrokecolor{currentstroke}%
\pgfsetdash{}{0pt}%
\pgfpathmoveto{\pgfqpoint{1.293480in}{5.530196in}}%
\pgfpathcurveto{\pgfqpoint{1.304530in}{5.530196in}}{\pgfqpoint{1.315129in}{5.534586in}}{\pgfqpoint{1.322943in}{5.542400in}}%
\pgfpathcurveto{\pgfqpoint{1.330756in}{5.550213in}}{\pgfqpoint{1.335147in}{5.560812in}}{\pgfqpoint{1.335147in}{5.571862in}}%
\pgfpathcurveto{\pgfqpoint{1.335147in}{5.582913in}}{\pgfqpoint{1.330756in}{5.593512in}}{\pgfqpoint{1.322943in}{5.601325in}}%
\pgfpathcurveto{\pgfqpoint{1.315129in}{5.609139in}}{\pgfqpoint{1.304530in}{5.613529in}}{\pgfqpoint{1.293480in}{5.613529in}}%
\pgfpathcurveto{\pgfqpoint{1.282430in}{5.613529in}}{\pgfqpoint{1.271831in}{5.609139in}}{\pgfqpoint{1.264017in}{5.601325in}}%
\pgfpathcurveto{\pgfqpoint{1.256204in}{5.593512in}}{\pgfqpoint{1.251813in}{5.582913in}}{\pgfqpoint{1.251813in}{5.571862in}}%
\pgfpathcurveto{\pgfqpoint{1.251813in}{5.560812in}}{\pgfqpoint{1.256204in}{5.550213in}}{\pgfqpoint{1.264017in}{5.542400in}}%
\pgfpathcurveto{\pgfqpoint{1.271831in}{5.534586in}}{\pgfqpoint{1.282430in}{5.530196in}}{\pgfqpoint{1.293480in}{5.530196in}}%
\pgfpathclose%
\pgfusepath{stroke,fill}%
\end{pgfscope}%
\begin{pgfscope}%
\pgfpathrectangle{\pgfqpoint{0.481978in}{0.331635in}}{\pgfqpoint{9.300000in}{7.700000in}}%
\pgfusepath{clip}%
\pgfsetbuttcap%
\pgfsetroundjoin%
\definecolor{currentfill}{rgb}{1.000000,0.705882,0.509804}%
\pgfsetfillcolor{currentfill}%
\pgfsetlinewidth{0.481800pt}%
\definecolor{currentstroke}{rgb}{1.000000,1.000000,1.000000}%
\pgfsetstrokecolor{currentstroke}%
\pgfsetdash{}{0pt}%
\pgfpathmoveto{\pgfqpoint{4.735330in}{2.499100in}}%
\pgfpathcurveto{\pgfqpoint{4.746380in}{2.499100in}}{\pgfqpoint{4.756979in}{2.503490in}}{\pgfqpoint{4.764793in}{2.511303in}}%
\pgfpathcurveto{\pgfqpoint{4.772606in}{2.519117in}}{\pgfqpoint{4.776997in}{2.529716in}}{\pgfqpoint{4.776997in}{2.540766in}}%
\pgfpathcurveto{\pgfqpoint{4.776997in}{2.551816in}}{\pgfqpoint{4.772606in}{2.562415in}}{\pgfqpoint{4.764793in}{2.570229in}}%
\pgfpathcurveto{\pgfqpoint{4.756979in}{2.578043in}}{\pgfqpoint{4.746380in}{2.582433in}}{\pgfqpoint{4.735330in}{2.582433in}}%
\pgfpathcurveto{\pgfqpoint{4.724280in}{2.582433in}}{\pgfqpoint{4.713681in}{2.578043in}}{\pgfqpoint{4.705867in}{2.570229in}}%
\pgfpathcurveto{\pgfqpoint{4.698054in}{2.562415in}}{\pgfqpoint{4.693663in}{2.551816in}}{\pgfqpoint{4.693663in}{2.540766in}}%
\pgfpathcurveto{\pgfqpoint{4.693663in}{2.529716in}}{\pgfqpoint{4.698054in}{2.519117in}}{\pgfqpoint{4.705867in}{2.511303in}}%
\pgfpathcurveto{\pgfqpoint{4.713681in}{2.503490in}}{\pgfqpoint{4.724280in}{2.499100in}}{\pgfqpoint{4.735330in}{2.499100in}}%
\pgfpathclose%
\pgfusepath{stroke,fill}%
\end{pgfscope}%
\begin{pgfscope}%
\pgfpathrectangle{\pgfqpoint{0.481978in}{0.331635in}}{\pgfqpoint{9.300000in}{7.700000in}}%
\pgfusepath{clip}%
\pgfsetbuttcap%
\pgfsetroundjoin%
\definecolor{currentfill}{rgb}{1.000000,0.705882,0.509804}%
\pgfsetfillcolor{currentfill}%
\pgfsetlinewidth{0.481800pt}%
\definecolor{currentstroke}{rgb}{1.000000,1.000000,1.000000}%
\pgfsetstrokecolor{currentstroke}%
\pgfsetdash{}{0pt}%
\pgfpathmoveto{\pgfqpoint{4.464515in}{4.972136in}}%
\pgfpathcurveto{\pgfqpoint{4.475565in}{4.972136in}}{\pgfqpoint{4.486164in}{4.976526in}}{\pgfqpoint{4.493978in}{4.984340in}}%
\pgfpathcurveto{\pgfqpoint{4.501792in}{4.992153in}}{\pgfqpoint{4.506182in}{5.002752in}}{\pgfqpoint{4.506182in}{5.013803in}}%
\pgfpathcurveto{\pgfqpoint{4.506182in}{5.024853in}}{\pgfqpoint{4.501792in}{5.035452in}}{\pgfqpoint{4.493978in}{5.043265in}}%
\pgfpathcurveto{\pgfqpoint{4.486164in}{5.051079in}}{\pgfqpoint{4.475565in}{5.055469in}}{\pgfqpoint{4.464515in}{5.055469in}}%
\pgfpathcurveto{\pgfqpoint{4.453465in}{5.055469in}}{\pgfqpoint{4.442866in}{5.051079in}}{\pgfqpoint{4.435052in}{5.043265in}}%
\pgfpathcurveto{\pgfqpoint{4.427239in}{5.035452in}}{\pgfqpoint{4.422849in}{5.024853in}}{\pgfqpoint{4.422849in}{5.013803in}}%
\pgfpathcurveto{\pgfqpoint{4.422849in}{5.002752in}}{\pgfqpoint{4.427239in}{4.992153in}}{\pgfqpoint{4.435052in}{4.984340in}}%
\pgfpathcurveto{\pgfqpoint{4.442866in}{4.976526in}}{\pgfqpoint{4.453465in}{4.972136in}}{\pgfqpoint{4.464515in}{4.972136in}}%
\pgfpathclose%
\pgfusepath{stroke,fill}%
\end{pgfscope}%
\begin{pgfscope}%
\pgfpathrectangle{\pgfqpoint{0.481978in}{0.331635in}}{\pgfqpoint{9.300000in}{7.700000in}}%
\pgfusepath{clip}%
\pgfsetbuttcap%
\pgfsetroundjoin%
\definecolor{currentfill}{rgb}{1.000000,0.705882,0.509804}%
\pgfsetfillcolor{currentfill}%
\pgfsetlinewidth{0.481800pt}%
\definecolor{currentstroke}{rgb}{1.000000,1.000000,1.000000}%
\pgfsetstrokecolor{currentstroke}%
\pgfsetdash{}{0pt}%
\pgfpathmoveto{\pgfqpoint{3.760896in}{2.867556in}}%
\pgfpathcurveto{\pgfqpoint{3.771946in}{2.867556in}}{\pgfqpoint{3.782545in}{2.871947in}}{\pgfqpoint{3.790359in}{2.879760in}}%
\pgfpathcurveto{\pgfqpoint{3.798172in}{2.887574in}}{\pgfqpoint{3.802563in}{2.898173in}}{\pgfqpoint{3.802563in}{2.909223in}}%
\pgfpathcurveto{\pgfqpoint{3.802563in}{2.920273in}}{\pgfqpoint{3.798172in}{2.930872in}}{\pgfqpoint{3.790359in}{2.938686in}}%
\pgfpathcurveto{\pgfqpoint{3.782545in}{2.946499in}}{\pgfqpoint{3.771946in}{2.950890in}}{\pgfqpoint{3.760896in}{2.950890in}}%
\pgfpathcurveto{\pgfqpoint{3.749846in}{2.950890in}}{\pgfqpoint{3.739247in}{2.946499in}}{\pgfqpoint{3.731433in}{2.938686in}}%
\pgfpathcurveto{\pgfqpoint{3.723620in}{2.930872in}}{\pgfqpoint{3.719229in}{2.920273in}}{\pgfqpoint{3.719229in}{2.909223in}}%
\pgfpathcurveto{\pgfqpoint{3.719229in}{2.898173in}}{\pgfqpoint{3.723620in}{2.887574in}}{\pgfqpoint{3.731433in}{2.879760in}}%
\pgfpathcurveto{\pgfqpoint{3.739247in}{2.871947in}}{\pgfqpoint{3.749846in}{2.867556in}}{\pgfqpoint{3.760896in}{2.867556in}}%
\pgfpathclose%
\pgfusepath{stroke,fill}%
\end{pgfscope}%
\begin{pgfscope}%
\pgfpathrectangle{\pgfqpoint{0.481978in}{0.331635in}}{\pgfqpoint{9.300000in}{7.700000in}}%
\pgfusepath{clip}%
\pgfsetbuttcap%
\pgfsetroundjoin%
\definecolor{currentfill}{rgb}{1.000000,0.705882,0.509804}%
\pgfsetfillcolor{currentfill}%
\pgfsetlinewidth{0.481800pt}%
\definecolor{currentstroke}{rgb}{1.000000,1.000000,1.000000}%
\pgfsetstrokecolor{currentstroke}%
\pgfsetdash{}{0pt}%
\pgfpathmoveto{\pgfqpoint{2.336586in}{3.476510in}}%
\pgfpathcurveto{\pgfqpoint{2.347636in}{3.476510in}}{\pgfqpoint{2.358235in}{3.480900in}}{\pgfqpoint{2.366049in}{3.488714in}}%
\pgfpathcurveto{\pgfqpoint{2.373863in}{3.496528in}}{\pgfqpoint{2.378253in}{3.507127in}}{\pgfqpoint{2.378253in}{3.518177in}}%
\pgfpathcurveto{\pgfqpoint{2.378253in}{3.529227in}}{\pgfqpoint{2.373863in}{3.539826in}}{\pgfqpoint{2.366049in}{3.547639in}}%
\pgfpathcurveto{\pgfqpoint{2.358235in}{3.555453in}}{\pgfqpoint{2.347636in}{3.559843in}}{\pgfqpoint{2.336586in}{3.559843in}}%
\pgfpathcurveto{\pgfqpoint{2.325536in}{3.559843in}}{\pgfqpoint{2.314937in}{3.555453in}}{\pgfqpoint{2.307123in}{3.547639in}}%
\pgfpathcurveto{\pgfqpoint{2.299310in}{3.539826in}}{\pgfqpoint{2.294919in}{3.529227in}}{\pgfqpoint{2.294919in}{3.518177in}}%
\pgfpathcurveto{\pgfqpoint{2.294919in}{3.507127in}}{\pgfqpoint{2.299310in}{3.496528in}}{\pgfqpoint{2.307123in}{3.488714in}}%
\pgfpathcurveto{\pgfqpoint{2.314937in}{3.480900in}}{\pgfqpoint{2.325536in}{3.476510in}}{\pgfqpoint{2.336586in}{3.476510in}}%
\pgfpathclose%
\pgfusepath{stroke,fill}%
\end{pgfscope}%
\begin{pgfscope}%
\pgfpathrectangle{\pgfqpoint{0.481978in}{0.331635in}}{\pgfqpoint{9.300000in}{7.700000in}}%
\pgfusepath{clip}%
\pgfsetbuttcap%
\pgfsetroundjoin%
\definecolor{currentfill}{rgb}{1.000000,0.705882,0.509804}%
\pgfsetfillcolor{currentfill}%
\pgfsetlinewidth{0.481800pt}%
\definecolor{currentstroke}{rgb}{1.000000,1.000000,1.000000}%
\pgfsetstrokecolor{currentstroke}%
\pgfsetdash{}{0pt}%
\pgfpathmoveto{\pgfqpoint{3.224347in}{6.004169in}}%
\pgfpathcurveto{\pgfqpoint{3.235397in}{6.004169in}}{\pgfqpoint{3.245996in}{6.008560in}}{\pgfqpoint{3.253810in}{6.016373in}}%
\pgfpathcurveto{\pgfqpoint{3.261624in}{6.024187in}}{\pgfqpoint{3.266014in}{6.034786in}}{\pgfqpoint{3.266014in}{6.045836in}}%
\pgfpathcurveto{\pgfqpoint{3.266014in}{6.056886in}}{\pgfqpoint{3.261624in}{6.067485in}}{\pgfqpoint{3.253810in}{6.075299in}}%
\pgfpathcurveto{\pgfqpoint{3.245996in}{6.083112in}}{\pgfqpoint{3.235397in}{6.087503in}}{\pgfqpoint{3.224347in}{6.087503in}}%
\pgfpathcurveto{\pgfqpoint{3.213297in}{6.087503in}}{\pgfqpoint{3.202698in}{6.083112in}}{\pgfqpoint{3.194884in}{6.075299in}}%
\pgfpathcurveto{\pgfqpoint{3.187071in}{6.067485in}}{\pgfqpoint{3.182681in}{6.056886in}}{\pgfqpoint{3.182681in}{6.045836in}}%
\pgfpathcurveto{\pgfqpoint{3.182681in}{6.034786in}}{\pgfqpoint{3.187071in}{6.024187in}}{\pgfqpoint{3.194884in}{6.016373in}}%
\pgfpathcurveto{\pgfqpoint{3.202698in}{6.008560in}}{\pgfqpoint{3.213297in}{6.004169in}}{\pgfqpoint{3.224347in}{6.004169in}}%
\pgfpathclose%
\pgfusepath{stroke,fill}%
\end{pgfscope}%
\begin{pgfscope}%
\pgfpathrectangle{\pgfqpoint{0.481978in}{0.331635in}}{\pgfqpoint{9.300000in}{7.700000in}}%
\pgfusepath{clip}%
\pgfsetbuttcap%
\pgfsetroundjoin%
\definecolor{currentfill}{rgb}{1.000000,0.705882,0.509804}%
\pgfsetfillcolor{currentfill}%
\pgfsetlinewidth{0.481800pt}%
\definecolor{currentstroke}{rgb}{1.000000,1.000000,1.000000}%
\pgfsetstrokecolor{currentstroke}%
\pgfsetdash{}{0pt}%
\pgfpathmoveto{\pgfqpoint{3.710549in}{6.521137in}}%
\pgfpathcurveto{\pgfqpoint{3.721600in}{6.521137in}}{\pgfqpoint{3.732199in}{6.525527in}}{\pgfqpoint{3.740012in}{6.533341in}}%
\pgfpathcurveto{\pgfqpoint{3.747826in}{6.541155in}}{\pgfqpoint{3.752216in}{6.551754in}}{\pgfqpoint{3.752216in}{6.562804in}}%
\pgfpathcurveto{\pgfqpoint{3.752216in}{6.573854in}}{\pgfqpoint{3.747826in}{6.584453in}}{\pgfqpoint{3.740012in}{6.592267in}}%
\pgfpathcurveto{\pgfqpoint{3.732199in}{6.600080in}}{\pgfqpoint{3.721600in}{6.604470in}}{\pgfqpoint{3.710549in}{6.604470in}}%
\pgfpathcurveto{\pgfqpoint{3.699499in}{6.604470in}}{\pgfqpoint{3.688900in}{6.600080in}}{\pgfqpoint{3.681087in}{6.592267in}}%
\pgfpathcurveto{\pgfqpoint{3.673273in}{6.584453in}}{\pgfqpoint{3.668883in}{6.573854in}}{\pgfqpoint{3.668883in}{6.562804in}}%
\pgfpathcurveto{\pgfqpoint{3.668883in}{6.551754in}}{\pgfqpoint{3.673273in}{6.541155in}}{\pgfqpoint{3.681087in}{6.533341in}}%
\pgfpathcurveto{\pgfqpoint{3.688900in}{6.525527in}}{\pgfqpoint{3.699499in}{6.521137in}}{\pgfqpoint{3.710549in}{6.521137in}}%
\pgfpathclose%
\pgfusepath{stroke,fill}%
\end{pgfscope}%
\begin{pgfscope}%
\pgfpathrectangle{\pgfqpoint{0.481978in}{0.331635in}}{\pgfqpoint{9.300000in}{7.700000in}}%
\pgfusepath{clip}%
\pgfsetbuttcap%
\pgfsetroundjoin%
\definecolor{currentfill}{rgb}{1.000000,0.705882,0.509804}%
\pgfsetfillcolor{currentfill}%
\pgfsetlinewidth{0.481800pt}%
\definecolor{currentstroke}{rgb}{1.000000,1.000000,1.000000}%
\pgfsetstrokecolor{currentstroke}%
\pgfsetdash{}{0pt}%
\pgfpathmoveto{\pgfqpoint{3.320504in}{5.247157in}}%
\pgfpathcurveto{\pgfqpoint{3.331554in}{5.247157in}}{\pgfqpoint{3.342153in}{5.251548in}}{\pgfqpoint{3.349966in}{5.259361in}}%
\pgfpathcurveto{\pgfqpoint{3.357780in}{5.267175in}}{\pgfqpoint{3.362170in}{5.277774in}}{\pgfqpoint{3.362170in}{5.288824in}}%
\pgfpathcurveto{\pgfqpoint{3.362170in}{5.299874in}}{\pgfqpoint{3.357780in}{5.310473in}}{\pgfqpoint{3.349966in}{5.318287in}}%
\pgfpathcurveto{\pgfqpoint{3.342153in}{5.326100in}}{\pgfqpoint{3.331554in}{5.330491in}}{\pgfqpoint{3.320504in}{5.330491in}}%
\pgfpathcurveto{\pgfqpoint{3.309454in}{5.330491in}}{\pgfqpoint{3.298855in}{5.326100in}}{\pgfqpoint{3.291041in}{5.318287in}}%
\pgfpathcurveto{\pgfqpoint{3.283227in}{5.310473in}}{\pgfqpoint{3.278837in}{5.299874in}}{\pgfqpoint{3.278837in}{5.288824in}}%
\pgfpathcurveto{\pgfqpoint{3.278837in}{5.277774in}}{\pgfqpoint{3.283227in}{5.267175in}}{\pgfqpoint{3.291041in}{5.259361in}}%
\pgfpathcurveto{\pgfqpoint{3.298855in}{5.251548in}}{\pgfqpoint{3.309454in}{5.247157in}}{\pgfqpoint{3.320504in}{5.247157in}}%
\pgfpathclose%
\pgfusepath{stroke,fill}%
\end{pgfscope}%
\begin{pgfscope}%
\pgfpathrectangle{\pgfqpoint{0.481978in}{0.331635in}}{\pgfqpoint{9.300000in}{7.700000in}}%
\pgfusepath{clip}%
\pgfsetbuttcap%
\pgfsetroundjoin%
\definecolor{currentfill}{rgb}{1.000000,0.705882,0.509804}%
\pgfsetfillcolor{currentfill}%
\pgfsetlinewidth{0.481800pt}%
\definecolor{currentstroke}{rgb}{1.000000,1.000000,1.000000}%
\pgfsetstrokecolor{currentstroke}%
\pgfsetdash{}{0pt}%
\pgfpathmoveto{\pgfqpoint{5.389021in}{1.238745in}}%
\pgfpathcurveto{\pgfqpoint{5.400072in}{1.238745in}}{\pgfqpoint{5.410671in}{1.243135in}}{\pgfqpoint{5.418484in}{1.250948in}}%
\pgfpathcurveto{\pgfqpoint{5.426298in}{1.258762in}}{\pgfqpoint{5.430688in}{1.269361in}}{\pgfqpoint{5.430688in}{1.280411in}}%
\pgfpathcurveto{\pgfqpoint{5.430688in}{1.291461in}}{\pgfqpoint{5.426298in}{1.302060in}}{\pgfqpoint{5.418484in}{1.309874in}}%
\pgfpathcurveto{\pgfqpoint{5.410671in}{1.317688in}}{\pgfqpoint{5.400072in}{1.322078in}}{\pgfqpoint{5.389021in}{1.322078in}}%
\pgfpathcurveto{\pgfqpoint{5.377971in}{1.322078in}}{\pgfqpoint{5.367372in}{1.317688in}}{\pgfqpoint{5.359559in}{1.309874in}}%
\pgfpathcurveto{\pgfqpoint{5.351745in}{1.302060in}}{\pgfqpoint{5.347355in}{1.291461in}}{\pgfqpoint{5.347355in}{1.280411in}}%
\pgfpathcurveto{\pgfqpoint{5.347355in}{1.269361in}}{\pgfqpoint{5.351745in}{1.258762in}}{\pgfqpoint{5.359559in}{1.250948in}}%
\pgfpathcurveto{\pgfqpoint{5.367372in}{1.243135in}}{\pgfqpoint{5.377971in}{1.238745in}}{\pgfqpoint{5.389021in}{1.238745in}}%
\pgfpathclose%
\pgfusepath{stroke,fill}%
\end{pgfscope}%
\begin{pgfscope}%
\pgfpathrectangle{\pgfqpoint{0.481978in}{0.331635in}}{\pgfqpoint{9.300000in}{7.700000in}}%
\pgfusepath{clip}%
\pgfsetbuttcap%
\pgfsetroundjoin%
\definecolor{currentfill}{rgb}{1.000000,0.705882,0.509804}%
\pgfsetfillcolor{currentfill}%
\pgfsetlinewidth{0.481800pt}%
\definecolor{currentstroke}{rgb}{1.000000,1.000000,1.000000}%
\pgfsetstrokecolor{currentstroke}%
\pgfsetdash{}{0pt}%
\pgfpathmoveto{\pgfqpoint{1.225270in}{3.155365in}}%
\pgfpathcurveto{\pgfqpoint{1.236320in}{3.155365in}}{\pgfqpoint{1.246919in}{3.159755in}}{\pgfqpoint{1.254733in}{3.167569in}}%
\pgfpathcurveto{\pgfqpoint{1.262547in}{3.175383in}}{\pgfqpoint{1.266937in}{3.185982in}}{\pgfqpoint{1.266937in}{3.197032in}}%
\pgfpathcurveto{\pgfqpoint{1.266937in}{3.208082in}}{\pgfqpoint{1.262547in}{3.218681in}}{\pgfqpoint{1.254733in}{3.226495in}}%
\pgfpathcurveto{\pgfqpoint{1.246919in}{3.234308in}}{\pgfqpoint{1.236320in}{3.238698in}}{\pgfqpoint{1.225270in}{3.238698in}}%
\pgfpathcurveto{\pgfqpoint{1.214220in}{3.238698in}}{\pgfqpoint{1.203621in}{3.234308in}}{\pgfqpoint{1.195807in}{3.226495in}}%
\pgfpathcurveto{\pgfqpoint{1.187994in}{3.218681in}}{\pgfqpoint{1.183603in}{3.208082in}}{\pgfqpoint{1.183603in}{3.197032in}}%
\pgfpathcurveto{\pgfqpoint{1.183603in}{3.185982in}}{\pgfqpoint{1.187994in}{3.175383in}}{\pgfqpoint{1.195807in}{3.167569in}}%
\pgfpathcurveto{\pgfqpoint{1.203621in}{3.159755in}}{\pgfqpoint{1.214220in}{3.155365in}}{\pgfqpoint{1.225270in}{3.155365in}}%
\pgfpathclose%
\pgfusepath{stroke,fill}%
\end{pgfscope}%
\begin{pgfscope}%
\pgfpathrectangle{\pgfqpoint{0.481978in}{0.331635in}}{\pgfqpoint{9.300000in}{7.700000in}}%
\pgfusepath{clip}%
\pgfsetbuttcap%
\pgfsetroundjoin%
\definecolor{currentfill}{rgb}{1.000000,0.705882,0.509804}%
\pgfsetfillcolor{currentfill}%
\pgfsetlinewidth{0.481800pt}%
\definecolor{currentstroke}{rgb}{1.000000,1.000000,1.000000}%
\pgfsetstrokecolor{currentstroke}%
\pgfsetdash{}{0pt}%
\pgfpathmoveto{\pgfqpoint{4.019367in}{2.930474in}}%
\pgfpathcurveto{\pgfqpoint{4.030418in}{2.930474in}}{\pgfqpoint{4.041017in}{2.934864in}}{\pgfqpoint{4.048830in}{2.942677in}}%
\pgfpathcurveto{\pgfqpoint{4.056644in}{2.950491in}}{\pgfqpoint{4.061034in}{2.961090in}}{\pgfqpoint{4.061034in}{2.972140in}}%
\pgfpathcurveto{\pgfqpoint{4.061034in}{2.983190in}}{\pgfqpoint{4.056644in}{2.993789in}}{\pgfqpoint{4.048830in}{3.001603in}}%
\pgfpathcurveto{\pgfqpoint{4.041017in}{3.009417in}}{\pgfqpoint{4.030418in}{3.013807in}}{\pgfqpoint{4.019367in}{3.013807in}}%
\pgfpathcurveto{\pgfqpoint{4.008317in}{3.013807in}}{\pgfqpoint{3.997718in}{3.009417in}}{\pgfqpoint{3.989905in}{3.001603in}}%
\pgfpathcurveto{\pgfqpoint{3.982091in}{2.993789in}}{\pgfqpoint{3.977701in}{2.983190in}}{\pgfqpoint{3.977701in}{2.972140in}}%
\pgfpathcurveto{\pgfqpoint{3.977701in}{2.961090in}}{\pgfqpoint{3.982091in}{2.950491in}}{\pgfqpoint{3.989905in}{2.942677in}}%
\pgfpathcurveto{\pgfqpoint{3.997718in}{2.934864in}}{\pgfqpoint{4.008317in}{2.930474in}}{\pgfqpoint{4.019367in}{2.930474in}}%
\pgfpathclose%
\pgfusepath{stroke,fill}%
\end{pgfscope}%
\begin{pgfscope}%
\pgfpathrectangle{\pgfqpoint{0.481978in}{0.331635in}}{\pgfqpoint{9.300000in}{7.700000in}}%
\pgfusepath{clip}%
\pgfsetbuttcap%
\pgfsetroundjoin%
\definecolor{currentfill}{rgb}{1.000000,0.705882,0.509804}%
\pgfsetfillcolor{currentfill}%
\pgfsetlinewidth{0.481800pt}%
\definecolor{currentstroke}{rgb}{1.000000,1.000000,1.000000}%
\pgfsetstrokecolor{currentstroke}%
\pgfsetdash{}{0pt}%
\pgfpathmoveto{\pgfqpoint{3.098846in}{4.048792in}}%
\pgfpathcurveto{\pgfqpoint{3.109896in}{4.048792in}}{\pgfqpoint{3.120495in}{4.053182in}}{\pgfqpoint{3.128309in}{4.060996in}}%
\pgfpathcurveto{\pgfqpoint{3.136122in}{4.068809in}}{\pgfqpoint{3.140512in}{4.079408in}}{\pgfqpoint{3.140512in}{4.090458in}}%
\pgfpathcurveto{\pgfqpoint{3.140512in}{4.101509in}}{\pgfqpoint{3.136122in}{4.112108in}}{\pgfqpoint{3.128309in}{4.119921in}}%
\pgfpathcurveto{\pgfqpoint{3.120495in}{4.127735in}}{\pgfqpoint{3.109896in}{4.132125in}}{\pgfqpoint{3.098846in}{4.132125in}}%
\pgfpathcurveto{\pgfqpoint{3.087796in}{4.132125in}}{\pgfqpoint{3.077197in}{4.127735in}}{\pgfqpoint{3.069383in}{4.119921in}}%
\pgfpathcurveto{\pgfqpoint{3.061569in}{4.112108in}}{\pgfqpoint{3.057179in}{4.101509in}}{\pgfqpoint{3.057179in}{4.090458in}}%
\pgfpathcurveto{\pgfqpoint{3.057179in}{4.079408in}}{\pgfqpoint{3.061569in}{4.068809in}}{\pgfqpoint{3.069383in}{4.060996in}}%
\pgfpathcurveto{\pgfqpoint{3.077197in}{4.053182in}}{\pgfqpoint{3.087796in}{4.048792in}}{\pgfqpoint{3.098846in}{4.048792in}}%
\pgfpathclose%
\pgfusepath{stroke,fill}%
\end{pgfscope}%
\begin{pgfscope}%
\pgfpathrectangle{\pgfqpoint{0.481978in}{0.331635in}}{\pgfqpoint{9.300000in}{7.700000in}}%
\pgfusepath{clip}%
\pgfsetbuttcap%
\pgfsetroundjoin%
\definecolor{currentfill}{rgb}{1.000000,0.705882,0.509804}%
\pgfsetfillcolor{currentfill}%
\pgfsetlinewidth{0.481800pt}%
\definecolor{currentstroke}{rgb}{1.000000,1.000000,1.000000}%
\pgfsetstrokecolor{currentstroke}%
\pgfsetdash{}{0pt}%
\pgfpathmoveto{\pgfqpoint{2.491460in}{4.500035in}}%
\pgfpathcurveto{\pgfqpoint{2.502510in}{4.500035in}}{\pgfqpoint{2.513109in}{4.504426in}}{\pgfqpoint{2.520923in}{4.512239in}}%
\pgfpathcurveto{\pgfqpoint{2.528736in}{4.520053in}}{\pgfqpoint{2.533126in}{4.530652in}}{\pgfqpoint{2.533126in}{4.541702in}}%
\pgfpathcurveto{\pgfqpoint{2.533126in}{4.552752in}}{\pgfqpoint{2.528736in}{4.563351in}}{\pgfqpoint{2.520923in}{4.571165in}}%
\pgfpathcurveto{\pgfqpoint{2.513109in}{4.578978in}}{\pgfqpoint{2.502510in}{4.583369in}}{\pgfqpoint{2.491460in}{4.583369in}}%
\pgfpathcurveto{\pgfqpoint{2.480410in}{4.583369in}}{\pgfqpoint{2.469811in}{4.578978in}}{\pgfqpoint{2.461997in}{4.571165in}}%
\pgfpathcurveto{\pgfqpoint{2.454183in}{4.563351in}}{\pgfqpoint{2.449793in}{4.552752in}}{\pgfqpoint{2.449793in}{4.541702in}}%
\pgfpathcurveto{\pgfqpoint{2.449793in}{4.530652in}}{\pgfqpoint{2.454183in}{4.520053in}}{\pgfqpoint{2.461997in}{4.512239in}}%
\pgfpathcurveto{\pgfqpoint{2.469811in}{4.504426in}}{\pgfqpoint{2.480410in}{4.500035in}}{\pgfqpoint{2.491460in}{4.500035in}}%
\pgfpathclose%
\pgfusepath{stroke,fill}%
\end{pgfscope}%
\begin{pgfscope}%
\pgfpathrectangle{\pgfqpoint{0.481978in}{0.331635in}}{\pgfqpoint{9.300000in}{7.700000in}}%
\pgfusepath{clip}%
\pgfsetbuttcap%
\pgfsetroundjoin%
\definecolor{currentfill}{rgb}{1.000000,0.705882,0.509804}%
\pgfsetfillcolor{currentfill}%
\pgfsetlinewidth{0.481800pt}%
\definecolor{currentstroke}{rgb}{1.000000,1.000000,1.000000}%
\pgfsetstrokecolor{currentstroke}%
\pgfsetdash{}{0pt}%
\pgfpathmoveto{\pgfqpoint{4.722330in}{3.350611in}}%
\pgfpathcurveto{\pgfqpoint{4.733380in}{3.350611in}}{\pgfqpoint{4.743979in}{3.355001in}}{\pgfqpoint{4.751792in}{3.362814in}}%
\pgfpathcurveto{\pgfqpoint{4.759606in}{3.370628in}}{\pgfqpoint{4.763996in}{3.381227in}}{\pgfqpoint{4.763996in}{3.392277in}}%
\pgfpathcurveto{\pgfqpoint{4.763996in}{3.403327in}}{\pgfqpoint{4.759606in}{3.413926in}}{\pgfqpoint{4.751792in}{3.421740in}}%
\pgfpathcurveto{\pgfqpoint{4.743979in}{3.429554in}}{\pgfqpoint{4.733380in}{3.433944in}}{\pgfqpoint{4.722330in}{3.433944in}}%
\pgfpathcurveto{\pgfqpoint{4.711279in}{3.433944in}}{\pgfqpoint{4.700680in}{3.429554in}}{\pgfqpoint{4.692867in}{3.421740in}}%
\pgfpathcurveto{\pgfqpoint{4.685053in}{3.413926in}}{\pgfqpoint{4.680663in}{3.403327in}}{\pgfqpoint{4.680663in}{3.392277in}}%
\pgfpathcurveto{\pgfqpoint{4.680663in}{3.381227in}}{\pgfqpoint{4.685053in}{3.370628in}}{\pgfqpoint{4.692867in}{3.362814in}}%
\pgfpathcurveto{\pgfqpoint{4.700680in}{3.355001in}}{\pgfqpoint{4.711279in}{3.350611in}}{\pgfqpoint{4.722330in}{3.350611in}}%
\pgfpathclose%
\pgfusepath{stroke,fill}%
\end{pgfscope}%
\begin{pgfscope}%
\pgfpathrectangle{\pgfqpoint{0.481978in}{0.331635in}}{\pgfqpoint{9.300000in}{7.700000in}}%
\pgfusepath{clip}%
\pgfsetbuttcap%
\pgfsetroundjoin%
\definecolor{currentfill}{rgb}{1.000000,0.705882,0.509804}%
\pgfsetfillcolor{currentfill}%
\pgfsetlinewidth{0.481800pt}%
\definecolor{currentstroke}{rgb}{1.000000,1.000000,1.000000}%
\pgfsetstrokecolor{currentstroke}%
\pgfsetdash{}{0pt}%
\pgfpathmoveto{\pgfqpoint{4.962053in}{4.722304in}}%
\pgfpathcurveto{\pgfqpoint{4.973103in}{4.722304in}}{\pgfqpoint{4.983702in}{4.726694in}}{\pgfqpoint{4.991516in}{4.734507in}}%
\pgfpathcurveto{\pgfqpoint{4.999330in}{4.742321in}}{\pgfqpoint{5.003720in}{4.752920in}}{\pgfqpoint{5.003720in}{4.763970in}}%
\pgfpathcurveto{\pgfqpoint{5.003720in}{4.775020in}}{\pgfqpoint{4.999330in}{4.785619in}}{\pgfqpoint{4.991516in}{4.793433in}}%
\pgfpathcurveto{\pgfqpoint{4.983702in}{4.801247in}}{\pgfqpoint{4.973103in}{4.805637in}}{\pgfqpoint{4.962053in}{4.805637in}}%
\pgfpathcurveto{\pgfqpoint{4.951003in}{4.805637in}}{\pgfqpoint{4.940404in}{4.801247in}}{\pgfqpoint{4.932590in}{4.793433in}}%
\pgfpathcurveto{\pgfqpoint{4.924777in}{4.785619in}}{\pgfqpoint{4.920387in}{4.775020in}}{\pgfqpoint{4.920387in}{4.763970in}}%
\pgfpathcurveto{\pgfqpoint{4.920387in}{4.752920in}}{\pgfqpoint{4.924777in}{4.742321in}}{\pgfqpoint{4.932590in}{4.734507in}}%
\pgfpathcurveto{\pgfqpoint{4.940404in}{4.726694in}}{\pgfqpoint{4.951003in}{4.722304in}}{\pgfqpoint{4.962053in}{4.722304in}}%
\pgfpathclose%
\pgfusepath{stroke,fill}%
\end{pgfscope}%
\begin{pgfscope}%
\pgfpathrectangle{\pgfqpoint{0.481978in}{0.331635in}}{\pgfqpoint{9.300000in}{7.700000in}}%
\pgfusepath{clip}%
\pgfsetbuttcap%
\pgfsetroundjoin%
\definecolor{currentfill}{rgb}{1.000000,0.705882,0.509804}%
\pgfsetfillcolor{currentfill}%
\pgfsetlinewidth{0.481800pt}%
\definecolor{currentstroke}{rgb}{1.000000,1.000000,1.000000}%
\pgfsetstrokecolor{currentstroke}%
\pgfsetdash{}{0pt}%
\pgfpathmoveto{\pgfqpoint{4.756086in}{2.480692in}}%
\pgfpathcurveto{\pgfqpoint{4.767136in}{2.480692in}}{\pgfqpoint{4.777735in}{2.485082in}}{\pgfqpoint{4.785549in}{2.492896in}}%
\pgfpathcurveto{\pgfqpoint{4.793363in}{2.500709in}}{\pgfqpoint{4.797753in}{2.511308in}}{\pgfqpoint{4.797753in}{2.522358in}}%
\pgfpathcurveto{\pgfqpoint{4.797753in}{2.533408in}}{\pgfqpoint{4.793363in}{2.544008in}}{\pgfqpoint{4.785549in}{2.551821in}}%
\pgfpathcurveto{\pgfqpoint{4.777735in}{2.559635in}}{\pgfqpoint{4.767136in}{2.564025in}}{\pgfqpoint{4.756086in}{2.564025in}}%
\pgfpathcurveto{\pgfqpoint{4.745036in}{2.564025in}}{\pgfqpoint{4.734437in}{2.559635in}}{\pgfqpoint{4.726623in}{2.551821in}}%
\pgfpathcurveto{\pgfqpoint{4.718810in}{2.544008in}}{\pgfqpoint{4.714420in}{2.533408in}}{\pgfqpoint{4.714420in}{2.522358in}}%
\pgfpathcurveto{\pgfqpoint{4.714420in}{2.511308in}}{\pgfqpoint{4.718810in}{2.500709in}}{\pgfqpoint{4.726623in}{2.492896in}}%
\pgfpathcurveto{\pgfqpoint{4.734437in}{2.485082in}}{\pgfqpoint{4.745036in}{2.480692in}}{\pgfqpoint{4.756086in}{2.480692in}}%
\pgfpathclose%
\pgfusepath{stroke,fill}%
\end{pgfscope}%
\begin{pgfscope}%
\pgfpathrectangle{\pgfqpoint{0.481978in}{0.331635in}}{\pgfqpoint{9.300000in}{7.700000in}}%
\pgfusepath{clip}%
\pgfsetbuttcap%
\pgfsetroundjoin%
\definecolor{currentfill}{rgb}{1.000000,0.705882,0.509804}%
\pgfsetfillcolor{currentfill}%
\pgfsetlinewidth{0.481800pt}%
\definecolor{currentstroke}{rgb}{1.000000,1.000000,1.000000}%
\pgfsetstrokecolor{currentstroke}%
\pgfsetdash{}{0pt}%
\pgfpathmoveto{\pgfqpoint{2.604559in}{2.316516in}}%
\pgfpathcurveto{\pgfqpoint{2.615609in}{2.316516in}}{\pgfqpoint{2.626208in}{2.320906in}}{\pgfqpoint{2.634022in}{2.328720in}}%
\pgfpathcurveto{\pgfqpoint{2.641836in}{2.336534in}}{\pgfqpoint{2.646226in}{2.347133in}}{\pgfqpoint{2.646226in}{2.358183in}}%
\pgfpathcurveto{\pgfqpoint{2.646226in}{2.369233in}}{\pgfqpoint{2.641836in}{2.379832in}}{\pgfqpoint{2.634022in}{2.387645in}}%
\pgfpathcurveto{\pgfqpoint{2.626208in}{2.395459in}}{\pgfqpoint{2.615609in}{2.399849in}}{\pgfqpoint{2.604559in}{2.399849in}}%
\pgfpathcurveto{\pgfqpoint{2.593509in}{2.399849in}}{\pgfqpoint{2.582910in}{2.395459in}}{\pgfqpoint{2.575096in}{2.387645in}}%
\pgfpathcurveto{\pgfqpoint{2.567283in}{2.379832in}}{\pgfqpoint{2.562892in}{2.369233in}}{\pgfqpoint{2.562892in}{2.358183in}}%
\pgfpathcurveto{\pgfqpoint{2.562892in}{2.347133in}}{\pgfqpoint{2.567283in}{2.336534in}}{\pgfqpoint{2.575096in}{2.328720in}}%
\pgfpathcurveto{\pgfqpoint{2.582910in}{2.320906in}}{\pgfqpoint{2.593509in}{2.316516in}}{\pgfqpoint{2.604559in}{2.316516in}}%
\pgfpathclose%
\pgfusepath{stroke,fill}%
\end{pgfscope}%
\begin{pgfscope}%
\pgfpathrectangle{\pgfqpoint{0.481978in}{0.331635in}}{\pgfqpoint{9.300000in}{7.700000in}}%
\pgfusepath{clip}%
\pgfsetbuttcap%
\pgfsetroundjoin%
\definecolor{currentfill}{rgb}{1.000000,0.705882,0.509804}%
\pgfsetfillcolor{currentfill}%
\pgfsetlinewidth{0.481800pt}%
\definecolor{currentstroke}{rgb}{1.000000,1.000000,1.000000}%
\pgfsetstrokecolor{currentstroke}%
\pgfsetdash{}{0pt}%
\pgfpathmoveto{\pgfqpoint{3.564142in}{5.365515in}}%
\pgfpathcurveto{\pgfqpoint{3.575192in}{5.365515in}}{\pgfqpoint{3.585791in}{5.369905in}}{\pgfqpoint{3.593604in}{5.377719in}}%
\pgfpathcurveto{\pgfqpoint{3.601418in}{5.385533in}}{\pgfqpoint{3.605808in}{5.396132in}}{\pgfqpoint{3.605808in}{5.407182in}}%
\pgfpathcurveto{\pgfqpoint{3.605808in}{5.418232in}}{\pgfqpoint{3.601418in}{5.428831in}}{\pgfqpoint{3.593604in}{5.436645in}}%
\pgfpathcurveto{\pgfqpoint{3.585791in}{5.444458in}}{\pgfqpoint{3.575192in}{5.448849in}}{\pgfqpoint{3.564142in}{5.448849in}}%
\pgfpathcurveto{\pgfqpoint{3.553091in}{5.448849in}}{\pgfqpoint{3.542492in}{5.444458in}}{\pgfqpoint{3.534679in}{5.436645in}}%
\pgfpathcurveto{\pgfqpoint{3.526865in}{5.428831in}}{\pgfqpoint{3.522475in}{5.418232in}}{\pgfqpoint{3.522475in}{5.407182in}}%
\pgfpathcurveto{\pgfqpoint{3.522475in}{5.396132in}}{\pgfqpoint{3.526865in}{5.385533in}}{\pgfqpoint{3.534679in}{5.377719in}}%
\pgfpathcurveto{\pgfqpoint{3.542492in}{5.369905in}}{\pgfqpoint{3.553091in}{5.365515in}}{\pgfqpoint{3.564142in}{5.365515in}}%
\pgfpathclose%
\pgfusepath{stroke,fill}%
\end{pgfscope}%
\begin{pgfscope}%
\pgfpathrectangle{\pgfqpoint{0.481978in}{0.331635in}}{\pgfqpoint{9.300000in}{7.700000in}}%
\pgfusepath{clip}%
\pgfsetbuttcap%
\pgfsetroundjoin%
\definecolor{currentfill}{rgb}{1.000000,0.705882,0.509804}%
\pgfsetfillcolor{currentfill}%
\pgfsetlinewidth{0.481800pt}%
\definecolor{currentstroke}{rgb}{1.000000,1.000000,1.000000}%
\pgfsetstrokecolor{currentstroke}%
\pgfsetdash{}{0pt}%
\pgfpathmoveto{\pgfqpoint{3.840892in}{5.590809in}}%
\pgfpathcurveto{\pgfqpoint{3.851942in}{5.590809in}}{\pgfqpoint{3.862541in}{5.595199in}}{\pgfqpoint{3.870355in}{5.603013in}}%
\pgfpathcurveto{\pgfqpoint{3.878168in}{5.610826in}}{\pgfqpoint{3.882559in}{5.621425in}}{\pgfqpoint{3.882559in}{5.632476in}}%
\pgfpathcurveto{\pgfqpoint{3.882559in}{5.643526in}}{\pgfqpoint{3.878168in}{5.654125in}}{\pgfqpoint{3.870355in}{5.661938in}}%
\pgfpathcurveto{\pgfqpoint{3.862541in}{5.669752in}}{\pgfqpoint{3.851942in}{5.674142in}}{\pgfqpoint{3.840892in}{5.674142in}}%
\pgfpathcurveto{\pgfqpoint{3.829842in}{5.674142in}}{\pgfqpoint{3.819243in}{5.669752in}}{\pgfqpoint{3.811429in}{5.661938in}}%
\pgfpathcurveto{\pgfqpoint{3.803616in}{5.654125in}}{\pgfqpoint{3.799225in}{5.643526in}}{\pgfqpoint{3.799225in}{5.632476in}}%
\pgfpathcurveto{\pgfqpoint{3.799225in}{5.621425in}}{\pgfqpoint{3.803616in}{5.610826in}}{\pgfqpoint{3.811429in}{5.603013in}}%
\pgfpathcurveto{\pgfqpoint{3.819243in}{5.595199in}}{\pgfqpoint{3.829842in}{5.590809in}}{\pgfqpoint{3.840892in}{5.590809in}}%
\pgfpathclose%
\pgfusepath{stroke,fill}%
\end{pgfscope}%
\begin{pgfscope}%
\pgfpathrectangle{\pgfqpoint{0.481978in}{0.331635in}}{\pgfqpoint{9.300000in}{7.700000in}}%
\pgfusepath{clip}%
\pgfsetbuttcap%
\pgfsetroundjoin%
\definecolor{currentfill}{rgb}{1.000000,0.705882,0.509804}%
\pgfsetfillcolor{currentfill}%
\pgfsetlinewidth{0.481800pt}%
\definecolor{currentstroke}{rgb}{1.000000,1.000000,1.000000}%
\pgfsetstrokecolor{currentstroke}%
\pgfsetdash{}{0pt}%
\pgfpathmoveto{\pgfqpoint{5.376976in}{4.611624in}}%
\pgfpathcurveto{\pgfqpoint{5.388026in}{4.611624in}}{\pgfqpoint{5.398625in}{4.616014in}}{\pgfqpoint{5.406439in}{4.623827in}}%
\pgfpathcurveto{\pgfqpoint{5.414252in}{4.631641in}}{\pgfqpoint{5.418643in}{4.642240in}}{\pgfqpoint{5.418643in}{4.653290in}}%
\pgfpathcurveto{\pgfqpoint{5.418643in}{4.664340in}}{\pgfqpoint{5.414252in}{4.674939in}}{\pgfqpoint{5.406439in}{4.682753in}}%
\pgfpathcurveto{\pgfqpoint{5.398625in}{4.690567in}}{\pgfqpoint{5.388026in}{4.694957in}}{\pgfqpoint{5.376976in}{4.694957in}}%
\pgfpathcurveto{\pgfqpoint{5.365926in}{4.694957in}}{\pgfqpoint{5.355327in}{4.690567in}}{\pgfqpoint{5.347513in}{4.682753in}}%
\pgfpathcurveto{\pgfqpoint{5.339699in}{4.674939in}}{\pgfqpoint{5.335309in}{4.664340in}}{\pgfqpoint{5.335309in}{4.653290in}}%
\pgfpathcurveto{\pgfqpoint{5.335309in}{4.642240in}}{\pgfqpoint{5.339699in}{4.631641in}}{\pgfqpoint{5.347513in}{4.623827in}}%
\pgfpathcurveto{\pgfqpoint{5.355327in}{4.616014in}}{\pgfqpoint{5.365926in}{4.611624in}}{\pgfqpoint{5.376976in}{4.611624in}}%
\pgfpathclose%
\pgfusepath{stroke,fill}%
\end{pgfscope}%
\begin{pgfscope}%
\pgfpathrectangle{\pgfqpoint{0.481978in}{0.331635in}}{\pgfqpoint{9.300000in}{7.700000in}}%
\pgfusepath{clip}%
\pgfsetbuttcap%
\pgfsetroundjoin%
\definecolor{currentfill}{rgb}{1.000000,0.705882,0.509804}%
\pgfsetfillcolor{currentfill}%
\pgfsetlinewidth{0.481800pt}%
\definecolor{currentstroke}{rgb}{1.000000,1.000000,1.000000}%
\pgfsetstrokecolor{currentstroke}%
\pgfsetdash{}{0pt}%
\pgfpathmoveto{\pgfqpoint{4.185210in}{4.176863in}}%
\pgfpathcurveto{\pgfqpoint{4.196260in}{4.176863in}}{\pgfqpoint{4.206859in}{4.181253in}}{\pgfqpoint{4.214673in}{4.189066in}}%
\pgfpathcurveto{\pgfqpoint{4.222486in}{4.196880in}}{\pgfqpoint{4.226877in}{4.207479in}}{\pgfqpoint{4.226877in}{4.218529in}}%
\pgfpathcurveto{\pgfqpoint{4.226877in}{4.229579in}}{\pgfqpoint{4.222486in}{4.240178in}}{\pgfqpoint{4.214673in}{4.247992in}}%
\pgfpathcurveto{\pgfqpoint{4.206859in}{4.255806in}}{\pgfqpoint{4.196260in}{4.260196in}}{\pgfqpoint{4.185210in}{4.260196in}}%
\pgfpathcurveto{\pgfqpoint{4.174160in}{4.260196in}}{\pgfqpoint{4.163561in}{4.255806in}}{\pgfqpoint{4.155747in}{4.247992in}}%
\pgfpathcurveto{\pgfqpoint{4.147933in}{4.240178in}}{\pgfqpoint{4.143543in}{4.229579in}}{\pgfqpoint{4.143543in}{4.218529in}}%
\pgfpathcurveto{\pgfqpoint{4.143543in}{4.207479in}}{\pgfqpoint{4.147933in}{4.196880in}}{\pgfqpoint{4.155747in}{4.189066in}}%
\pgfpathcurveto{\pgfqpoint{4.163561in}{4.181253in}}{\pgfqpoint{4.174160in}{4.176863in}}{\pgfqpoint{4.185210in}{4.176863in}}%
\pgfpathclose%
\pgfusepath{stroke,fill}%
\end{pgfscope}%
\begin{pgfscope}%
\pgfpathrectangle{\pgfqpoint{0.481978in}{0.331635in}}{\pgfqpoint{9.300000in}{7.700000in}}%
\pgfusepath{clip}%
\pgfsetbuttcap%
\pgfsetroundjoin%
\definecolor{currentfill}{rgb}{1.000000,0.705882,0.509804}%
\pgfsetfillcolor{currentfill}%
\pgfsetlinewidth{0.481800pt}%
\definecolor{currentstroke}{rgb}{1.000000,1.000000,1.000000}%
\pgfsetstrokecolor{currentstroke}%
\pgfsetdash{}{0pt}%
\pgfpathmoveto{\pgfqpoint{4.500422in}{3.003250in}}%
\pgfpathcurveto{\pgfqpoint{4.511472in}{3.003250in}}{\pgfqpoint{4.522071in}{3.007640in}}{\pgfqpoint{4.529885in}{3.015454in}}%
\pgfpathcurveto{\pgfqpoint{4.537699in}{3.023267in}}{\pgfqpoint{4.542089in}{3.033866in}}{\pgfqpoint{4.542089in}{3.044916in}}%
\pgfpathcurveto{\pgfqpoint{4.542089in}{3.055967in}}{\pgfqpoint{4.537699in}{3.066566in}}{\pgfqpoint{4.529885in}{3.074379in}}%
\pgfpathcurveto{\pgfqpoint{4.522071in}{3.082193in}}{\pgfqpoint{4.511472in}{3.086583in}}{\pgfqpoint{4.500422in}{3.086583in}}%
\pgfpathcurveto{\pgfqpoint{4.489372in}{3.086583in}}{\pgfqpoint{4.478773in}{3.082193in}}{\pgfqpoint{4.470959in}{3.074379in}}%
\pgfpathcurveto{\pgfqpoint{4.463146in}{3.066566in}}{\pgfqpoint{4.458755in}{3.055967in}}{\pgfqpoint{4.458755in}{3.044916in}}%
\pgfpathcurveto{\pgfqpoint{4.458755in}{3.033866in}}{\pgfqpoint{4.463146in}{3.023267in}}{\pgfqpoint{4.470959in}{3.015454in}}%
\pgfpathcurveto{\pgfqpoint{4.478773in}{3.007640in}}{\pgfqpoint{4.489372in}{3.003250in}}{\pgfqpoint{4.500422in}{3.003250in}}%
\pgfpathclose%
\pgfusepath{stroke,fill}%
\end{pgfscope}%
\begin{pgfscope}%
\pgfpathrectangle{\pgfqpoint{0.481978in}{0.331635in}}{\pgfqpoint{9.300000in}{7.700000in}}%
\pgfusepath{clip}%
\pgfsetbuttcap%
\pgfsetroundjoin%
\definecolor{currentfill}{rgb}{1.000000,0.705882,0.509804}%
\pgfsetfillcolor{currentfill}%
\pgfsetlinewidth{0.481800pt}%
\definecolor{currentstroke}{rgb}{1.000000,1.000000,1.000000}%
\pgfsetstrokecolor{currentstroke}%
\pgfsetdash{}{0pt}%
\pgfpathmoveto{\pgfqpoint{4.103119in}{4.252690in}}%
\pgfpathcurveto{\pgfqpoint{4.114169in}{4.252690in}}{\pgfqpoint{4.124768in}{4.257080in}}{\pgfqpoint{4.132582in}{4.264893in}}%
\pgfpathcurveto{\pgfqpoint{4.140396in}{4.272707in}}{\pgfqpoint{4.144786in}{4.283306in}}{\pgfqpoint{4.144786in}{4.294356in}}%
\pgfpathcurveto{\pgfqpoint{4.144786in}{4.305406in}}{\pgfqpoint{4.140396in}{4.316005in}}{\pgfqpoint{4.132582in}{4.323819in}}%
\pgfpathcurveto{\pgfqpoint{4.124768in}{4.331633in}}{\pgfqpoint{4.114169in}{4.336023in}}{\pgfqpoint{4.103119in}{4.336023in}}%
\pgfpathcurveto{\pgfqpoint{4.092069in}{4.336023in}}{\pgfqpoint{4.081470in}{4.331633in}}{\pgfqpoint{4.073656in}{4.323819in}}%
\pgfpathcurveto{\pgfqpoint{4.065843in}{4.316005in}}{\pgfqpoint{4.061452in}{4.305406in}}{\pgfqpoint{4.061452in}{4.294356in}}%
\pgfpathcurveto{\pgfqpoint{4.061452in}{4.283306in}}{\pgfqpoint{4.065843in}{4.272707in}}{\pgfqpoint{4.073656in}{4.264893in}}%
\pgfpathcurveto{\pgfqpoint{4.081470in}{4.257080in}}{\pgfqpoint{4.092069in}{4.252690in}}{\pgfqpoint{4.103119in}{4.252690in}}%
\pgfpathclose%
\pgfusepath{stroke,fill}%
\end{pgfscope}%
\begin{pgfscope}%
\pgfpathrectangle{\pgfqpoint{0.481978in}{0.331635in}}{\pgfqpoint{9.300000in}{7.700000in}}%
\pgfusepath{clip}%
\pgfsetbuttcap%
\pgfsetroundjoin%
\definecolor{currentfill}{rgb}{1.000000,0.705882,0.509804}%
\pgfsetfillcolor{currentfill}%
\pgfsetlinewidth{0.481800pt}%
\definecolor{currentstroke}{rgb}{1.000000,1.000000,1.000000}%
\pgfsetstrokecolor{currentstroke}%
\pgfsetdash{}{0pt}%
\pgfpathmoveto{\pgfqpoint{3.235019in}{4.465313in}}%
\pgfpathcurveto{\pgfqpoint{3.246069in}{4.465313in}}{\pgfqpoint{3.256668in}{4.469703in}}{\pgfqpoint{3.264482in}{4.477517in}}%
\pgfpathcurveto{\pgfqpoint{3.272295in}{4.485331in}}{\pgfqpoint{3.276686in}{4.495930in}}{\pgfqpoint{3.276686in}{4.506980in}}%
\pgfpathcurveto{\pgfqpoint{3.276686in}{4.518030in}}{\pgfqpoint{3.272295in}{4.528629in}}{\pgfqpoint{3.264482in}{4.536443in}}%
\pgfpathcurveto{\pgfqpoint{3.256668in}{4.544256in}}{\pgfqpoint{3.246069in}{4.548647in}}{\pgfqpoint{3.235019in}{4.548647in}}%
\pgfpathcurveto{\pgfqpoint{3.223969in}{4.548647in}}{\pgfqpoint{3.213370in}{4.544256in}}{\pgfqpoint{3.205556in}{4.536443in}}%
\pgfpathcurveto{\pgfqpoint{3.197743in}{4.528629in}}{\pgfqpoint{3.193352in}{4.518030in}}{\pgfqpoint{3.193352in}{4.506980in}}%
\pgfpathcurveto{\pgfqpoint{3.193352in}{4.495930in}}{\pgfqpoint{3.197743in}{4.485331in}}{\pgfqpoint{3.205556in}{4.477517in}}%
\pgfpathcurveto{\pgfqpoint{3.213370in}{4.469703in}}{\pgfqpoint{3.223969in}{4.465313in}}{\pgfqpoint{3.235019in}{4.465313in}}%
\pgfpathclose%
\pgfusepath{stroke,fill}%
\end{pgfscope}%
\begin{pgfscope}%
\pgfpathrectangle{\pgfqpoint{0.481978in}{0.331635in}}{\pgfqpoint{9.300000in}{7.700000in}}%
\pgfusepath{clip}%
\pgfsetbuttcap%
\pgfsetroundjoin%
\definecolor{currentfill}{rgb}{1.000000,0.705882,0.509804}%
\pgfsetfillcolor{currentfill}%
\pgfsetlinewidth{0.481800pt}%
\definecolor{currentstroke}{rgb}{1.000000,1.000000,1.000000}%
\pgfsetstrokecolor{currentstroke}%
\pgfsetdash{}{0pt}%
\pgfpathmoveto{\pgfqpoint{1.466431in}{3.069920in}}%
\pgfpathcurveto{\pgfqpoint{1.477481in}{3.069920in}}{\pgfqpoint{1.488080in}{3.074310in}}{\pgfqpoint{1.495894in}{3.082124in}}%
\pgfpathcurveto{\pgfqpoint{1.503708in}{3.089937in}}{\pgfqpoint{1.508098in}{3.100536in}}{\pgfqpoint{1.508098in}{3.111586in}}%
\pgfpathcurveto{\pgfqpoint{1.508098in}{3.122636in}}{\pgfqpoint{1.503708in}{3.133235in}}{\pgfqpoint{1.495894in}{3.141049in}}%
\pgfpathcurveto{\pgfqpoint{1.488080in}{3.148863in}}{\pgfqpoint{1.477481in}{3.153253in}}{\pgfqpoint{1.466431in}{3.153253in}}%
\pgfpathcurveto{\pgfqpoint{1.455381in}{3.153253in}}{\pgfqpoint{1.444782in}{3.148863in}}{\pgfqpoint{1.436968in}{3.141049in}}%
\pgfpathcurveto{\pgfqpoint{1.429155in}{3.133235in}}{\pgfqpoint{1.424764in}{3.122636in}}{\pgfqpoint{1.424764in}{3.111586in}}%
\pgfpathcurveto{\pgfqpoint{1.424764in}{3.100536in}}{\pgfqpoint{1.429155in}{3.089937in}}{\pgfqpoint{1.436968in}{3.082124in}}%
\pgfpathcurveto{\pgfqpoint{1.444782in}{3.074310in}}{\pgfqpoint{1.455381in}{3.069920in}}{\pgfqpoint{1.466431in}{3.069920in}}%
\pgfpathclose%
\pgfusepath{stroke,fill}%
\end{pgfscope}%
\begin{pgfscope}%
\pgfpathrectangle{\pgfqpoint{0.481978in}{0.331635in}}{\pgfqpoint{9.300000in}{7.700000in}}%
\pgfusepath{clip}%
\pgfsetbuttcap%
\pgfsetroundjoin%
\definecolor{currentfill}{rgb}{1.000000,0.705882,0.509804}%
\pgfsetfillcolor{currentfill}%
\pgfsetlinewidth{0.481800pt}%
\definecolor{currentstroke}{rgb}{1.000000,1.000000,1.000000}%
\pgfsetstrokecolor{currentstroke}%
\pgfsetdash{}{0pt}%
\pgfpathmoveto{\pgfqpoint{5.649061in}{4.441319in}}%
\pgfpathcurveto{\pgfqpoint{5.660112in}{4.441319in}}{\pgfqpoint{5.670711in}{4.445709in}}{\pgfqpoint{5.678524in}{4.453523in}}%
\pgfpathcurveto{\pgfqpoint{5.686338in}{4.461336in}}{\pgfqpoint{5.690728in}{4.471935in}}{\pgfqpoint{5.690728in}{4.482985in}}%
\pgfpathcurveto{\pgfqpoint{5.690728in}{4.494036in}}{\pgfqpoint{5.686338in}{4.504635in}}{\pgfqpoint{5.678524in}{4.512448in}}%
\pgfpathcurveto{\pgfqpoint{5.670711in}{4.520262in}}{\pgfqpoint{5.660112in}{4.524652in}}{\pgfqpoint{5.649061in}{4.524652in}}%
\pgfpathcurveto{\pgfqpoint{5.638011in}{4.524652in}}{\pgfqpoint{5.627412in}{4.520262in}}{\pgfqpoint{5.619599in}{4.512448in}}%
\pgfpathcurveto{\pgfqpoint{5.611785in}{4.504635in}}{\pgfqpoint{5.607395in}{4.494036in}}{\pgfqpoint{5.607395in}{4.482985in}}%
\pgfpathcurveto{\pgfqpoint{5.607395in}{4.471935in}}{\pgfqpoint{5.611785in}{4.461336in}}{\pgfqpoint{5.619599in}{4.453523in}}%
\pgfpathcurveto{\pgfqpoint{5.627412in}{4.445709in}}{\pgfqpoint{5.638011in}{4.441319in}}{\pgfqpoint{5.649061in}{4.441319in}}%
\pgfpathclose%
\pgfusepath{stroke,fill}%
\end{pgfscope}%
\begin{pgfscope}%
\pgfpathrectangle{\pgfqpoint{0.481978in}{0.331635in}}{\pgfqpoint{9.300000in}{7.700000in}}%
\pgfusepath{clip}%
\pgfsetbuttcap%
\pgfsetroundjoin%
\definecolor{currentfill}{rgb}{1.000000,0.705882,0.509804}%
\pgfsetfillcolor{currentfill}%
\pgfsetlinewidth{0.481800pt}%
\definecolor{currentstroke}{rgb}{1.000000,1.000000,1.000000}%
\pgfsetstrokecolor{currentstroke}%
\pgfsetdash{}{0pt}%
\pgfpathmoveto{\pgfqpoint{3.679062in}{2.651300in}}%
\pgfpathcurveto{\pgfqpoint{3.690112in}{2.651300in}}{\pgfqpoint{3.700711in}{2.655690in}}{\pgfqpoint{3.708525in}{2.663503in}}%
\pgfpathcurveto{\pgfqpoint{3.716338in}{2.671317in}}{\pgfqpoint{3.720729in}{2.681916in}}{\pgfqpoint{3.720729in}{2.692966in}}%
\pgfpathcurveto{\pgfqpoint{3.720729in}{2.704016in}}{\pgfqpoint{3.716338in}{2.714615in}}{\pgfqpoint{3.708525in}{2.722429in}}%
\pgfpathcurveto{\pgfqpoint{3.700711in}{2.730243in}}{\pgfqpoint{3.690112in}{2.734633in}}{\pgfqpoint{3.679062in}{2.734633in}}%
\pgfpathcurveto{\pgfqpoint{3.668012in}{2.734633in}}{\pgfqpoint{3.657413in}{2.730243in}}{\pgfqpoint{3.649599in}{2.722429in}}%
\pgfpathcurveto{\pgfqpoint{3.641786in}{2.714615in}}{\pgfqpoint{3.637395in}{2.704016in}}{\pgfqpoint{3.637395in}{2.692966in}}%
\pgfpathcurveto{\pgfqpoint{3.637395in}{2.681916in}}{\pgfqpoint{3.641786in}{2.671317in}}{\pgfqpoint{3.649599in}{2.663503in}}%
\pgfpathcurveto{\pgfqpoint{3.657413in}{2.655690in}}{\pgfqpoint{3.668012in}{2.651300in}}{\pgfqpoint{3.679062in}{2.651300in}}%
\pgfpathclose%
\pgfusepath{stroke,fill}%
\end{pgfscope}%
\begin{pgfscope}%
\pgfpathrectangle{\pgfqpoint{0.481978in}{0.331635in}}{\pgfqpoint{9.300000in}{7.700000in}}%
\pgfusepath{clip}%
\pgfsetbuttcap%
\pgfsetroundjoin%
\definecolor{currentfill}{rgb}{1.000000,0.705882,0.509804}%
\pgfsetfillcolor{currentfill}%
\pgfsetlinewidth{0.481800pt}%
\definecolor{currentstroke}{rgb}{1.000000,1.000000,1.000000}%
\pgfsetstrokecolor{currentstroke}%
\pgfsetdash{}{0pt}%
\pgfpathmoveto{\pgfqpoint{2.616545in}{4.527795in}}%
\pgfpathcurveto{\pgfqpoint{2.627596in}{4.527795in}}{\pgfqpoint{2.638195in}{4.532186in}}{\pgfqpoint{2.646008in}{4.539999in}}%
\pgfpathcurveto{\pgfqpoint{2.653822in}{4.547813in}}{\pgfqpoint{2.658212in}{4.558412in}}{\pgfqpoint{2.658212in}{4.569462in}}%
\pgfpathcurveto{\pgfqpoint{2.658212in}{4.580512in}}{\pgfqpoint{2.653822in}{4.591111in}}{\pgfqpoint{2.646008in}{4.598925in}}%
\pgfpathcurveto{\pgfqpoint{2.638195in}{4.606739in}}{\pgfqpoint{2.627596in}{4.611129in}}{\pgfqpoint{2.616545in}{4.611129in}}%
\pgfpathcurveto{\pgfqpoint{2.605495in}{4.611129in}}{\pgfqpoint{2.594896in}{4.606739in}}{\pgfqpoint{2.587083in}{4.598925in}}%
\pgfpathcurveto{\pgfqpoint{2.579269in}{4.591111in}}{\pgfqpoint{2.574879in}{4.580512in}}{\pgfqpoint{2.574879in}{4.569462in}}%
\pgfpathcurveto{\pgfqpoint{2.574879in}{4.558412in}}{\pgfqpoint{2.579269in}{4.547813in}}{\pgfqpoint{2.587083in}{4.539999in}}%
\pgfpathcurveto{\pgfqpoint{2.594896in}{4.532186in}}{\pgfqpoint{2.605495in}{4.527795in}}{\pgfqpoint{2.616545in}{4.527795in}}%
\pgfpathclose%
\pgfusepath{stroke,fill}%
\end{pgfscope}%
\begin{pgfscope}%
\pgfpathrectangle{\pgfqpoint{0.481978in}{0.331635in}}{\pgfqpoint{9.300000in}{7.700000in}}%
\pgfusepath{clip}%
\pgfsetbuttcap%
\pgfsetroundjoin%
\definecolor{currentfill}{rgb}{1.000000,0.705882,0.509804}%
\pgfsetfillcolor{currentfill}%
\pgfsetlinewidth{0.481800pt}%
\definecolor{currentstroke}{rgb}{1.000000,1.000000,1.000000}%
\pgfsetstrokecolor{currentstroke}%
\pgfsetdash{}{0pt}%
\pgfpathmoveto{\pgfqpoint{5.192724in}{4.988152in}}%
\pgfpathcurveto{\pgfqpoint{5.203774in}{4.988152in}}{\pgfqpoint{5.214373in}{4.992542in}}{\pgfqpoint{5.222187in}{5.000355in}}%
\pgfpathcurveto{\pgfqpoint{5.230000in}{5.008169in}}{\pgfqpoint{5.234391in}{5.018768in}}{\pgfqpoint{5.234391in}{5.029818in}}%
\pgfpathcurveto{\pgfqpoint{5.234391in}{5.040868in}}{\pgfqpoint{5.230000in}{5.051467in}}{\pgfqpoint{5.222187in}{5.059281in}}%
\pgfpathcurveto{\pgfqpoint{5.214373in}{5.067095in}}{\pgfqpoint{5.203774in}{5.071485in}}{\pgfqpoint{5.192724in}{5.071485in}}%
\pgfpathcurveto{\pgfqpoint{5.181674in}{5.071485in}}{\pgfqpoint{5.171075in}{5.067095in}}{\pgfqpoint{5.163261in}{5.059281in}}%
\pgfpathcurveto{\pgfqpoint{5.155447in}{5.051467in}}{\pgfqpoint{5.151057in}{5.040868in}}{\pgfqpoint{5.151057in}{5.029818in}}%
\pgfpathcurveto{\pgfqpoint{5.151057in}{5.018768in}}{\pgfqpoint{5.155447in}{5.008169in}}{\pgfqpoint{5.163261in}{5.000355in}}%
\pgfpathcurveto{\pgfqpoint{5.171075in}{4.992542in}}{\pgfqpoint{5.181674in}{4.988152in}}{\pgfqpoint{5.192724in}{4.988152in}}%
\pgfpathclose%
\pgfusepath{stroke,fill}%
\end{pgfscope}%
\begin{pgfscope}%
\pgfpathrectangle{\pgfqpoint{0.481978in}{0.331635in}}{\pgfqpoint{9.300000in}{7.700000in}}%
\pgfusepath{clip}%
\pgfsetbuttcap%
\pgfsetroundjoin%
\definecolor{currentfill}{rgb}{1.000000,0.705882,0.509804}%
\pgfsetfillcolor{currentfill}%
\pgfsetlinewidth{0.481800pt}%
\definecolor{currentstroke}{rgb}{1.000000,1.000000,1.000000}%
\pgfsetstrokecolor{currentstroke}%
\pgfsetdash{}{0pt}%
\pgfpathmoveto{\pgfqpoint{3.374008in}{2.945401in}}%
\pgfpathcurveto{\pgfqpoint{3.385058in}{2.945401in}}{\pgfqpoint{3.395657in}{2.949791in}}{\pgfqpoint{3.403471in}{2.957605in}}%
\pgfpathcurveto{\pgfqpoint{3.411284in}{2.965419in}}{\pgfqpoint{3.415675in}{2.976018in}}{\pgfqpoint{3.415675in}{2.987068in}}%
\pgfpathcurveto{\pgfqpoint{3.415675in}{2.998118in}}{\pgfqpoint{3.411284in}{3.008717in}}{\pgfqpoint{3.403471in}{3.016531in}}%
\pgfpathcurveto{\pgfqpoint{3.395657in}{3.024344in}}{\pgfqpoint{3.385058in}{3.028734in}}{\pgfqpoint{3.374008in}{3.028734in}}%
\pgfpathcurveto{\pgfqpoint{3.362958in}{3.028734in}}{\pgfqpoint{3.352359in}{3.024344in}}{\pgfqpoint{3.344545in}{3.016531in}}%
\pgfpathcurveto{\pgfqpoint{3.336732in}{3.008717in}}{\pgfqpoint{3.332341in}{2.998118in}}{\pgfqpoint{3.332341in}{2.987068in}}%
\pgfpathcurveto{\pgfqpoint{3.332341in}{2.976018in}}{\pgfqpoint{3.336732in}{2.965419in}}{\pgfqpoint{3.344545in}{2.957605in}}%
\pgfpathcurveto{\pgfqpoint{3.352359in}{2.949791in}}{\pgfqpoint{3.362958in}{2.945401in}}{\pgfqpoint{3.374008in}{2.945401in}}%
\pgfpathclose%
\pgfusepath{stroke,fill}%
\end{pgfscope}%
\begin{pgfscope}%
\pgfpathrectangle{\pgfqpoint{0.481978in}{0.331635in}}{\pgfqpoint{9.300000in}{7.700000in}}%
\pgfusepath{clip}%
\pgfsetbuttcap%
\pgfsetroundjoin%
\definecolor{currentfill}{rgb}{1.000000,0.705882,0.509804}%
\pgfsetfillcolor{currentfill}%
\pgfsetlinewidth{0.481800pt}%
\definecolor{currentstroke}{rgb}{1.000000,1.000000,1.000000}%
\pgfsetstrokecolor{currentstroke}%
\pgfsetdash{}{0pt}%
\pgfpathmoveto{\pgfqpoint{4.257082in}{4.276759in}}%
\pgfpathcurveto{\pgfqpoint{4.268132in}{4.276759in}}{\pgfqpoint{4.278731in}{4.281149in}}{\pgfqpoint{4.286544in}{4.288962in}}%
\pgfpathcurveto{\pgfqpoint{4.294358in}{4.296776in}}{\pgfqpoint{4.298748in}{4.307375in}}{\pgfqpoint{4.298748in}{4.318425in}}%
\pgfpathcurveto{\pgfqpoint{4.298748in}{4.329475in}}{\pgfqpoint{4.294358in}{4.340074in}}{\pgfqpoint{4.286544in}{4.347888in}}%
\pgfpathcurveto{\pgfqpoint{4.278731in}{4.355702in}}{\pgfqpoint{4.268132in}{4.360092in}}{\pgfqpoint{4.257082in}{4.360092in}}%
\pgfpathcurveto{\pgfqpoint{4.246031in}{4.360092in}}{\pgfqpoint{4.235432in}{4.355702in}}{\pgfqpoint{4.227619in}{4.347888in}}%
\pgfpathcurveto{\pgfqpoint{4.219805in}{4.340074in}}{\pgfqpoint{4.215415in}{4.329475in}}{\pgfqpoint{4.215415in}{4.318425in}}%
\pgfpathcurveto{\pgfqpoint{4.215415in}{4.307375in}}{\pgfqpoint{4.219805in}{4.296776in}}{\pgfqpoint{4.227619in}{4.288962in}}%
\pgfpathcurveto{\pgfqpoint{4.235432in}{4.281149in}}{\pgfqpoint{4.246031in}{4.276759in}}{\pgfqpoint{4.257082in}{4.276759in}}%
\pgfpathclose%
\pgfusepath{stroke,fill}%
\end{pgfscope}%
\begin{pgfscope}%
\pgfpathrectangle{\pgfqpoint{0.481978in}{0.331635in}}{\pgfqpoint{9.300000in}{7.700000in}}%
\pgfusepath{clip}%
\pgfsetbuttcap%
\pgfsetroundjoin%
\definecolor{currentfill}{rgb}{1.000000,0.705882,0.509804}%
\pgfsetfillcolor{currentfill}%
\pgfsetlinewidth{0.481800pt}%
\definecolor{currentstroke}{rgb}{1.000000,1.000000,1.000000}%
\pgfsetstrokecolor{currentstroke}%
\pgfsetdash{}{0pt}%
\pgfpathmoveto{\pgfqpoint{1.027576in}{3.744338in}}%
\pgfpathcurveto{\pgfqpoint{1.038626in}{3.744338in}}{\pgfqpoint{1.049225in}{3.748728in}}{\pgfqpoint{1.057038in}{3.756542in}}%
\pgfpathcurveto{\pgfqpoint{1.064852in}{3.764355in}}{\pgfqpoint{1.069242in}{3.774954in}}{\pgfqpoint{1.069242in}{3.786004in}}%
\pgfpathcurveto{\pgfqpoint{1.069242in}{3.797054in}}{\pgfqpoint{1.064852in}{3.807653in}}{\pgfqpoint{1.057038in}{3.815467in}}%
\pgfpathcurveto{\pgfqpoint{1.049225in}{3.823281in}}{\pgfqpoint{1.038626in}{3.827671in}}{\pgfqpoint{1.027576in}{3.827671in}}%
\pgfpathcurveto{\pgfqpoint{1.016525in}{3.827671in}}{\pgfqpoint{1.005926in}{3.823281in}}{\pgfqpoint{0.998113in}{3.815467in}}%
\pgfpathcurveto{\pgfqpoint{0.990299in}{3.807653in}}{\pgfqpoint{0.985909in}{3.797054in}}{\pgfqpoint{0.985909in}{3.786004in}}%
\pgfpathcurveto{\pgfqpoint{0.985909in}{3.774954in}}{\pgfqpoint{0.990299in}{3.764355in}}{\pgfqpoint{0.998113in}{3.756542in}}%
\pgfpathcurveto{\pgfqpoint{1.005926in}{3.748728in}}{\pgfqpoint{1.016525in}{3.744338in}}{\pgfqpoint{1.027576in}{3.744338in}}%
\pgfpathclose%
\pgfusepath{stroke,fill}%
\end{pgfscope}%
\begin{pgfscope}%
\pgfpathrectangle{\pgfqpoint{0.481978in}{0.331635in}}{\pgfqpoint{9.300000in}{7.700000in}}%
\pgfusepath{clip}%
\pgfsetbuttcap%
\pgfsetroundjoin%
\definecolor{currentfill}{rgb}{1.000000,0.705882,0.509804}%
\pgfsetfillcolor{currentfill}%
\pgfsetlinewidth{0.481800pt}%
\definecolor{currentstroke}{rgb}{1.000000,1.000000,1.000000}%
\pgfsetstrokecolor{currentstroke}%
\pgfsetdash{}{0pt}%
\pgfpathmoveto{\pgfqpoint{2.111519in}{4.599199in}}%
\pgfpathcurveto{\pgfqpoint{2.122569in}{4.599199in}}{\pgfqpoint{2.133168in}{4.603589in}}{\pgfqpoint{2.140982in}{4.611403in}}%
\pgfpathcurveto{\pgfqpoint{2.148795in}{4.619216in}}{\pgfqpoint{2.153186in}{4.629815in}}{\pgfqpoint{2.153186in}{4.640866in}}%
\pgfpathcurveto{\pgfqpoint{2.153186in}{4.651916in}}{\pgfqpoint{2.148795in}{4.662515in}}{\pgfqpoint{2.140982in}{4.670328in}}%
\pgfpathcurveto{\pgfqpoint{2.133168in}{4.678142in}}{\pgfqpoint{2.122569in}{4.682532in}}{\pgfqpoint{2.111519in}{4.682532in}}%
\pgfpathcurveto{\pgfqpoint{2.100469in}{4.682532in}}{\pgfqpoint{2.089870in}{4.678142in}}{\pgfqpoint{2.082056in}{4.670328in}}%
\pgfpathcurveto{\pgfqpoint{2.074243in}{4.662515in}}{\pgfqpoint{2.069852in}{4.651916in}}{\pgfqpoint{2.069852in}{4.640866in}}%
\pgfpathcurveto{\pgfqpoint{2.069852in}{4.629815in}}{\pgfqpoint{2.074243in}{4.619216in}}{\pgfqpoint{2.082056in}{4.611403in}}%
\pgfpathcurveto{\pgfqpoint{2.089870in}{4.603589in}}{\pgfqpoint{2.100469in}{4.599199in}}{\pgfqpoint{2.111519in}{4.599199in}}%
\pgfpathclose%
\pgfusepath{stroke,fill}%
\end{pgfscope}%
\begin{pgfscope}%
\pgfpathrectangle{\pgfqpoint{0.481978in}{0.331635in}}{\pgfqpoint{9.300000in}{7.700000in}}%
\pgfusepath{clip}%
\pgfsetbuttcap%
\pgfsetroundjoin%
\definecolor{currentfill}{rgb}{1.000000,0.705882,0.509804}%
\pgfsetfillcolor{currentfill}%
\pgfsetlinewidth{0.481800pt}%
\definecolor{currentstroke}{rgb}{1.000000,1.000000,1.000000}%
\pgfsetstrokecolor{currentstroke}%
\pgfsetdash{}{0pt}%
\pgfpathmoveto{\pgfqpoint{2.540709in}{6.106130in}}%
\pgfpathcurveto{\pgfqpoint{2.551759in}{6.106130in}}{\pgfqpoint{2.562358in}{6.110521in}}{\pgfqpoint{2.570172in}{6.118334in}}%
\pgfpathcurveto{\pgfqpoint{2.577985in}{6.126148in}}{\pgfqpoint{2.582376in}{6.136747in}}{\pgfqpoint{2.582376in}{6.147797in}}%
\pgfpathcurveto{\pgfqpoint{2.582376in}{6.158847in}}{\pgfqpoint{2.577985in}{6.169446in}}{\pgfqpoint{2.570172in}{6.177260in}}%
\pgfpathcurveto{\pgfqpoint{2.562358in}{6.185073in}}{\pgfqpoint{2.551759in}{6.189464in}}{\pgfqpoint{2.540709in}{6.189464in}}%
\pgfpathcurveto{\pgfqpoint{2.529659in}{6.189464in}}{\pgfqpoint{2.519060in}{6.185073in}}{\pgfqpoint{2.511246in}{6.177260in}}%
\pgfpathcurveto{\pgfqpoint{2.503432in}{6.169446in}}{\pgfqpoint{2.499042in}{6.158847in}}{\pgfqpoint{2.499042in}{6.147797in}}%
\pgfpathcurveto{\pgfqpoint{2.499042in}{6.136747in}}{\pgfqpoint{2.503432in}{6.126148in}}{\pgfqpoint{2.511246in}{6.118334in}}%
\pgfpathcurveto{\pgfqpoint{2.519060in}{6.110521in}}{\pgfqpoint{2.529659in}{6.106130in}}{\pgfqpoint{2.540709in}{6.106130in}}%
\pgfpathclose%
\pgfusepath{stroke,fill}%
\end{pgfscope}%
\begin{pgfscope}%
\pgfpathrectangle{\pgfqpoint{0.481978in}{0.331635in}}{\pgfqpoint{9.300000in}{7.700000in}}%
\pgfusepath{clip}%
\pgfsetbuttcap%
\pgfsetroundjoin%
\definecolor{currentfill}{rgb}{1.000000,0.705882,0.509804}%
\pgfsetfillcolor{currentfill}%
\pgfsetlinewidth{0.481800pt}%
\definecolor{currentstroke}{rgb}{1.000000,1.000000,1.000000}%
\pgfsetstrokecolor{currentstroke}%
\pgfsetdash{}{0pt}%
\pgfpathmoveto{\pgfqpoint{2.593367in}{2.323294in}}%
\pgfpathcurveto{\pgfqpoint{2.604417in}{2.323294in}}{\pgfqpoint{2.615016in}{2.327684in}}{\pgfqpoint{2.622829in}{2.335498in}}%
\pgfpathcurveto{\pgfqpoint{2.630643in}{2.343311in}}{\pgfqpoint{2.635033in}{2.353910in}}{\pgfqpoint{2.635033in}{2.364960in}}%
\pgfpathcurveto{\pgfqpoint{2.635033in}{2.376011in}}{\pgfqpoint{2.630643in}{2.386610in}}{\pgfqpoint{2.622829in}{2.394423in}}%
\pgfpathcurveto{\pgfqpoint{2.615016in}{2.402237in}}{\pgfqpoint{2.604417in}{2.406627in}}{\pgfqpoint{2.593367in}{2.406627in}}%
\pgfpathcurveto{\pgfqpoint{2.582317in}{2.406627in}}{\pgfqpoint{2.571717in}{2.402237in}}{\pgfqpoint{2.563904in}{2.394423in}}%
\pgfpathcurveto{\pgfqpoint{2.556090in}{2.386610in}}{\pgfqpoint{2.551700in}{2.376011in}}{\pgfqpoint{2.551700in}{2.364960in}}%
\pgfpathcurveto{\pgfqpoint{2.551700in}{2.353910in}}{\pgfqpoint{2.556090in}{2.343311in}}{\pgfqpoint{2.563904in}{2.335498in}}%
\pgfpathcurveto{\pgfqpoint{2.571717in}{2.327684in}}{\pgfqpoint{2.582317in}{2.323294in}}{\pgfqpoint{2.593367in}{2.323294in}}%
\pgfpathclose%
\pgfusepath{stroke,fill}%
\end{pgfscope}%
\begin{pgfscope}%
\pgfpathrectangle{\pgfqpoint{0.481978in}{0.331635in}}{\pgfqpoint{9.300000in}{7.700000in}}%
\pgfusepath{clip}%
\pgfsetbuttcap%
\pgfsetroundjoin%
\definecolor{currentfill}{rgb}{1.000000,0.705882,0.509804}%
\pgfsetfillcolor{currentfill}%
\pgfsetlinewidth{0.481800pt}%
\definecolor{currentstroke}{rgb}{1.000000,1.000000,1.000000}%
\pgfsetstrokecolor{currentstroke}%
\pgfsetdash{}{0pt}%
\pgfpathmoveto{\pgfqpoint{1.941607in}{3.916070in}}%
\pgfpathcurveto{\pgfqpoint{1.952657in}{3.916070in}}{\pgfqpoint{1.963256in}{3.920460in}}{\pgfqpoint{1.971070in}{3.928274in}}%
\pgfpathcurveto{\pgfqpoint{1.978883in}{3.936087in}}{\pgfqpoint{1.983274in}{3.946686in}}{\pgfqpoint{1.983274in}{3.957736in}}%
\pgfpathcurveto{\pgfqpoint{1.983274in}{3.968787in}}{\pgfqpoint{1.978883in}{3.979386in}}{\pgfqpoint{1.971070in}{3.987199in}}%
\pgfpathcurveto{\pgfqpoint{1.963256in}{3.995013in}}{\pgfqpoint{1.952657in}{3.999403in}}{\pgfqpoint{1.941607in}{3.999403in}}%
\pgfpathcurveto{\pgfqpoint{1.930557in}{3.999403in}}{\pgfqpoint{1.919958in}{3.995013in}}{\pgfqpoint{1.912144in}{3.987199in}}%
\pgfpathcurveto{\pgfqpoint{1.904331in}{3.979386in}}{\pgfqpoint{1.899940in}{3.968787in}}{\pgfqpoint{1.899940in}{3.957736in}}%
\pgfpathcurveto{\pgfqpoint{1.899940in}{3.946686in}}{\pgfqpoint{1.904331in}{3.936087in}}{\pgfqpoint{1.912144in}{3.928274in}}%
\pgfpathcurveto{\pgfqpoint{1.919958in}{3.920460in}}{\pgfqpoint{1.930557in}{3.916070in}}{\pgfqpoint{1.941607in}{3.916070in}}%
\pgfpathclose%
\pgfusepath{stroke,fill}%
\end{pgfscope}%
\begin{pgfscope}%
\pgfpathrectangle{\pgfqpoint{0.481978in}{0.331635in}}{\pgfqpoint{9.300000in}{7.700000in}}%
\pgfusepath{clip}%
\pgfsetbuttcap%
\pgfsetroundjoin%
\definecolor{currentfill}{rgb}{1.000000,0.705882,0.509804}%
\pgfsetfillcolor{currentfill}%
\pgfsetlinewidth{0.481800pt}%
\definecolor{currentstroke}{rgb}{1.000000,1.000000,1.000000}%
\pgfsetstrokecolor{currentstroke}%
\pgfsetdash{}{0pt}%
\pgfpathmoveto{\pgfqpoint{2.880395in}{3.273104in}}%
\pgfpathcurveto{\pgfqpoint{2.891445in}{3.273104in}}{\pgfqpoint{2.902044in}{3.277494in}}{\pgfqpoint{2.909858in}{3.285308in}}%
\pgfpathcurveto{\pgfqpoint{2.917671in}{3.293122in}}{\pgfqpoint{2.922061in}{3.303721in}}{\pgfqpoint{2.922061in}{3.314771in}}%
\pgfpathcurveto{\pgfqpoint{2.922061in}{3.325821in}}{\pgfqpoint{2.917671in}{3.336420in}}{\pgfqpoint{2.909858in}{3.344234in}}%
\pgfpathcurveto{\pgfqpoint{2.902044in}{3.352047in}}{\pgfqpoint{2.891445in}{3.356438in}}{\pgfqpoint{2.880395in}{3.356438in}}%
\pgfpathcurveto{\pgfqpoint{2.869345in}{3.356438in}}{\pgfqpoint{2.858746in}{3.352047in}}{\pgfqpoint{2.850932in}{3.344234in}}%
\pgfpathcurveto{\pgfqpoint{2.843118in}{3.336420in}}{\pgfqpoint{2.838728in}{3.325821in}}{\pgfqpoint{2.838728in}{3.314771in}}%
\pgfpathcurveto{\pgfqpoint{2.838728in}{3.303721in}}{\pgfqpoint{2.843118in}{3.293122in}}{\pgfqpoint{2.850932in}{3.285308in}}%
\pgfpathcurveto{\pgfqpoint{2.858746in}{3.277494in}}{\pgfqpoint{2.869345in}{3.273104in}}{\pgfqpoint{2.880395in}{3.273104in}}%
\pgfpathclose%
\pgfusepath{stroke,fill}%
\end{pgfscope}%
\begin{pgfscope}%
\pgfpathrectangle{\pgfqpoint{0.481978in}{0.331635in}}{\pgfqpoint{9.300000in}{7.700000in}}%
\pgfusepath{clip}%
\pgfsetbuttcap%
\pgfsetroundjoin%
\definecolor{currentfill}{rgb}{1.000000,0.705882,0.509804}%
\pgfsetfillcolor{currentfill}%
\pgfsetlinewidth{0.481800pt}%
\definecolor{currentstroke}{rgb}{1.000000,1.000000,1.000000}%
\pgfsetstrokecolor{currentstroke}%
\pgfsetdash{}{0pt}%
\pgfpathmoveto{\pgfqpoint{3.084100in}{2.910420in}}%
\pgfpathcurveto{\pgfqpoint{3.095150in}{2.910420in}}{\pgfqpoint{3.105749in}{2.914810in}}{\pgfqpoint{3.113563in}{2.922623in}}%
\pgfpathcurveto{\pgfqpoint{3.121377in}{2.930437in}}{\pgfqpoint{3.125767in}{2.941036in}}{\pgfqpoint{3.125767in}{2.952086in}}%
\pgfpathcurveto{\pgfqpoint{3.125767in}{2.963136in}}{\pgfqpoint{3.121377in}{2.973735in}}{\pgfqpoint{3.113563in}{2.981549in}}%
\pgfpathcurveto{\pgfqpoint{3.105749in}{2.989363in}}{\pgfqpoint{3.095150in}{2.993753in}}{\pgfqpoint{3.084100in}{2.993753in}}%
\pgfpathcurveto{\pgfqpoint{3.073050in}{2.993753in}}{\pgfqpoint{3.062451in}{2.989363in}}{\pgfqpoint{3.054637in}{2.981549in}}%
\pgfpathcurveto{\pgfqpoint{3.046824in}{2.973735in}}{\pgfqpoint{3.042434in}{2.963136in}}{\pgfqpoint{3.042434in}{2.952086in}}%
\pgfpathcurveto{\pgfqpoint{3.042434in}{2.941036in}}{\pgfqpoint{3.046824in}{2.930437in}}{\pgfqpoint{3.054637in}{2.922623in}}%
\pgfpathcurveto{\pgfqpoint{3.062451in}{2.914810in}}{\pgfqpoint{3.073050in}{2.910420in}}{\pgfqpoint{3.084100in}{2.910420in}}%
\pgfpathclose%
\pgfusepath{stroke,fill}%
\end{pgfscope}%
\begin{pgfscope}%
\pgfpathrectangle{\pgfqpoint{0.481978in}{0.331635in}}{\pgfqpoint{9.300000in}{7.700000in}}%
\pgfusepath{clip}%
\pgfsetbuttcap%
\pgfsetroundjoin%
\definecolor{currentfill}{rgb}{1.000000,0.705882,0.509804}%
\pgfsetfillcolor{currentfill}%
\pgfsetlinewidth{0.481800pt}%
\definecolor{currentstroke}{rgb}{1.000000,1.000000,1.000000}%
\pgfsetstrokecolor{currentstroke}%
\pgfsetdash{}{0pt}%
\pgfpathmoveto{\pgfqpoint{5.368141in}{2.996376in}}%
\pgfpathcurveto{\pgfqpoint{5.379191in}{2.996376in}}{\pgfqpoint{5.389790in}{3.000766in}}{\pgfqpoint{5.397604in}{3.008580in}}%
\pgfpathcurveto{\pgfqpoint{5.405417in}{3.016394in}}{\pgfqpoint{5.409807in}{3.026993in}}{\pgfqpoint{5.409807in}{3.038043in}}%
\pgfpathcurveto{\pgfqpoint{5.409807in}{3.049093in}}{\pgfqpoint{5.405417in}{3.059692in}}{\pgfqpoint{5.397604in}{3.067506in}}%
\pgfpathcurveto{\pgfqpoint{5.389790in}{3.075319in}}{\pgfqpoint{5.379191in}{3.079709in}}{\pgfqpoint{5.368141in}{3.079709in}}%
\pgfpathcurveto{\pgfqpoint{5.357091in}{3.079709in}}{\pgfqpoint{5.346492in}{3.075319in}}{\pgfqpoint{5.338678in}{3.067506in}}%
\pgfpathcurveto{\pgfqpoint{5.330864in}{3.059692in}}{\pgfqpoint{5.326474in}{3.049093in}}{\pgfqpoint{5.326474in}{3.038043in}}%
\pgfpathcurveto{\pgfqpoint{5.326474in}{3.026993in}}{\pgfqpoint{5.330864in}{3.016394in}}{\pgfqpoint{5.338678in}{3.008580in}}%
\pgfpathcurveto{\pgfqpoint{5.346492in}{3.000766in}}{\pgfqpoint{5.357091in}{2.996376in}}{\pgfqpoint{5.368141in}{2.996376in}}%
\pgfpathclose%
\pgfusepath{stroke,fill}%
\end{pgfscope}%
\begin{pgfscope}%
\pgfpathrectangle{\pgfqpoint{0.481978in}{0.331635in}}{\pgfqpoint{9.300000in}{7.700000in}}%
\pgfusepath{clip}%
\pgfsetbuttcap%
\pgfsetroundjoin%
\definecolor{currentfill}{rgb}{1.000000,0.705882,0.509804}%
\pgfsetfillcolor{currentfill}%
\pgfsetlinewidth{0.481800pt}%
\definecolor{currentstroke}{rgb}{1.000000,1.000000,1.000000}%
\pgfsetstrokecolor{currentstroke}%
\pgfsetdash{}{0pt}%
\pgfpathmoveto{\pgfqpoint{2.886916in}{3.885707in}}%
\pgfpathcurveto{\pgfqpoint{2.897966in}{3.885707in}}{\pgfqpoint{2.908565in}{3.890097in}}{\pgfqpoint{2.916379in}{3.897911in}}%
\pgfpathcurveto{\pgfqpoint{2.924192in}{3.905724in}}{\pgfqpoint{2.928583in}{3.916323in}}{\pgfqpoint{2.928583in}{3.927373in}}%
\pgfpathcurveto{\pgfqpoint{2.928583in}{3.938424in}}{\pgfqpoint{2.924192in}{3.949023in}}{\pgfqpoint{2.916379in}{3.956836in}}%
\pgfpathcurveto{\pgfqpoint{2.908565in}{3.964650in}}{\pgfqpoint{2.897966in}{3.969040in}}{\pgfqpoint{2.886916in}{3.969040in}}%
\pgfpathcurveto{\pgfqpoint{2.875866in}{3.969040in}}{\pgfqpoint{2.865267in}{3.964650in}}{\pgfqpoint{2.857453in}{3.956836in}}%
\pgfpathcurveto{\pgfqpoint{2.849640in}{3.949023in}}{\pgfqpoint{2.845249in}{3.938424in}}{\pgfqpoint{2.845249in}{3.927373in}}%
\pgfpathcurveto{\pgfqpoint{2.845249in}{3.916323in}}{\pgfqpoint{2.849640in}{3.905724in}}{\pgfqpoint{2.857453in}{3.897911in}}%
\pgfpathcurveto{\pgfqpoint{2.865267in}{3.890097in}}{\pgfqpoint{2.875866in}{3.885707in}}{\pgfqpoint{2.886916in}{3.885707in}}%
\pgfpathclose%
\pgfusepath{stroke,fill}%
\end{pgfscope}%
\begin{pgfscope}%
\pgfpathrectangle{\pgfqpoint{0.481978in}{0.331635in}}{\pgfqpoint{9.300000in}{7.700000in}}%
\pgfusepath{clip}%
\pgfsetbuttcap%
\pgfsetroundjoin%
\definecolor{currentfill}{rgb}{1.000000,0.705882,0.509804}%
\pgfsetfillcolor{currentfill}%
\pgfsetlinewidth{0.481800pt}%
\definecolor{currentstroke}{rgb}{1.000000,1.000000,1.000000}%
\pgfsetstrokecolor{currentstroke}%
\pgfsetdash{}{0pt}%
\pgfpathmoveto{\pgfqpoint{4.450031in}{2.732253in}}%
\pgfpathcurveto{\pgfqpoint{4.461082in}{2.732253in}}{\pgfqpoint{4.471681in}{2.736643in}}{\pgfqpoint{4.479494in}{2.744457in}}%
\pgfpathcurveto{\pgfqpoint{4.487308in}{2.752270in}}{\pgfqpoint{4.491698in}{2.762869in}}{\pgfqpoint{4.491698in}{2.773919in}}%
\pgfpathcurveto{\pgfqpoint{4.491698in}{2.784970in}}{\pgfqpoint{4.487308in}{2.795569in}}{\pgfqpoint{4.479494in}{2.803382in}}%
\pgfpathcurveto{\pgfqpoint{4.471681in}{2.811196in}}{\pgfqpoint{4.461082in}{2.815586in}}{\pgfqpoint{4.450031in}{2.815586in}}%
\pgfpathcurveto{\pgfqpoint{4.438981in}{2.815586in}}{\pgfqpoint{4.428382in}{2.811196in}}{\pgfqpoint{4.420569in}{2.803382in}}%
\pgfpathcurveto{\pgfqpoint{4.412755in}{2.795569in}}{\pgfqpoint{4.408365in}{2.784970in}}{\pgfqpoint{4.408365in}{2.773919in}}%
\pgfpathcurveto{\pgfqpoint{4.408365in}{2.762869in}}{\pgfqpoint{4.412755in}{2.752270in}}{\pgfqpoint{4.420569in}{2.744457in}}%
\pgfpathcurveto{\pgfqpoint{4.428382in}{2.736643in}}{\pgfqpoint{4.438981in}{2.732253in}}{\pgfqpoint{4.450031in}{2.732253in}}%
\pgfpathclose%
\pgfusepath{stroke,fill}%
\end{pgfscope}%
\begin{pgfscope}%
\pgfpathrectangle{\pgfqpoint{0.481978in}{0.331635in}}{\pgfqpoint{9.300000in}{7.700000in}}%
\pgfusepath{clip}%
\pgfsetbuttcap%
\pgfsetroundjoin%
\definecolor{currentfill}{rgb}{1.000000,0.705882,0.509804}%
\pgfsetfillcolor{currentfill}%
\pgfsetlinewidth{0.481800pt}%
\definecolor{currentstroke}{rgb}{1.000000,1.000000,1.000000}%
\pgfsetstrokecolor{currentstroke}%
\pgfsetdash{}{0pt}%
\pgfpathmoveto{\pgfqpoint{2.491094in}{5.161273in}}%
\pgfpathcurveto{\pgfqpoint{2.502144in}{5.161273in}}{\pgfqpoint{2.512743in}{5.165663in}}{\pgfqpoint{2.520557in}{5.173476in}}%
\pgfpathcurveto{\pgfqpoint{2.528370in}{5.181290in}}{\pgfqpoint{2.532760in}{5.191889in}}{\pgfqpoint{2.532760in}{5.202939in}}%
\pgfpathcurveto{\pgfqpoint{2.532760in}{5.213989in}}{\pgfqpoint{2.528370in}{5.224588in}}{\pgfqpoint{2.520557in}{5.232402in}}%
\pgfpathcurveto{\pgfqpoint{2.512743in}{5.240216in}}{\pgfqpoint{2.502144in}{5.244606in}}{\pgfqpoint{2.491094in}{5.244606in}}%
\pgfpathcurveto{\pgfqpoint{2.480044in}{5.244606in}}{\pgfqpoint{2.469445in}{5.240216in}}{\pgfqpoint{2.461631in}{5.232402in}}%
\pgfpathcurveto{\pgfqpoint{2.453817in}{5.224588in}}{\pgfqpoint{2.449427in}{5.213989in}}{\pgfqpoint{2.449427in}{5.202939in}}%
\pgfpathcurveto{\pgfqpoint{2.449427in}{5.191889in}}{\pgfqpoint{2.453817in}{5.181290in}}{\pgfqpoint{2.461631in}{5.173476in}}%
\pgfpathcurveto{\pgfqpoint{2.469445in}{5.165663in}}{\pgfqpoint{2.480044in}{5.161273in}}{\pgfqpoint{2.491094in}{5.161273in}}%
\pgfpathclose%
\pgfusepath{stroke,fill}%
\end{pgfscope}%
\begin{pgfscope}%
\pgfpathrectangle{\pgfqpoint{0.481978in}{0.331635in}}{\pgfqpoint{9.300000in}{7.700000in}}%
\pgfusepath{clip}%
\pgfsetbuttcap%
\pgfsetroundjoin%
\definecolor{currentfill}{rgb}{1.000000,0.705882,0.509804}%
\pgfsetfillcolor{currentfill}%
\pgfsetlinewidth{0.481800pt}%
\definecolor{currentstroke}{rgb}{1.000000,1.000000,1.000000}%
\pgfsetstrokecolor{currentstroke}%
\pgfsetdash{}{0pt}%
\pgfpathmoveto{\pgfqpoint{2.662048in}{3.580893in}}%
\pgfpathcurveto{\pgfqpoint{2.673099in}{3.580893in}}{\pgfqpoint{2.683698in}{3.585283in}}{\pgfqpoint{2.691511in}{3.593097in}}%
\pgfpathcurveto{\pgfqpoint{2.699325in}{3.600911in}}{\pgfqpoint{2.703715in}{3.611510in}}{\pgfqpoint{2.703715in}{3.622560in}}%
\pgfpathcurveto{\pgfqpoint{2.703715in}{3.633610in}}{\pgfqpoint{2.699325in}{3.644209in}}{\pgfqpoint{2.691511in}{3.652023in}}%
\pgfpathcurveto{\pgfqpoint{2.683698in}{3.659836in}}{\pgfqpoint{2.673099in}{3.664226in}}{\pgfqpoint{2.662048in}{3.664226in}}%
\pgfpathcurveto{\pgfqpoint{2.650998in}{3.664226in}}{\pgfqpoint{2.640399in}{3.659836in}}{\pgfqpoint{2.632586in}{3.652023in}}%
\pgfpathcurveto{\pgfqpoint{2.624772in}{3.644209in}}{\pgfqpoint{2.620382in}{3.633610in}}{\pgfqpoint{2.620382in}{3.622560in}}%
\pgfpathcurveto{\pgfqpoint{2.620382in}{3.611510in}}{\pgfqpoint{2.624772in}{3.600911in}}{\pgfqpoint{2.632586in}{3.593097in}}%
\pgfpathcurveto{\pgfqpoint{2.640399in}{3.585283in}}{\pgfqpoint{2.650998in}{3.580893in}}{\pgfqpoint{2.662048in}{3.580893in}}%
\pgfpathclose%
\pgfusepath{stroke,fill}%
\end{pgfscope}%
\begin{pgfscope}%
\pgfpathrectangle{\pgfqpoint{0.481978in}{0.331635in}}{\pgfqpoint{9.300000in}{7.700000in}}%
\pgfusepath{clip}%
\pgfsetbuttcap%
\pgfsetroundjoin%
\definecolor{currentfill}{rgb}{1.000000,0.705882,0.509804}%
\pgfsetfillcolor{currentfill}%
\pgfsetlinewidth{0.481800pt}%
\definecolor{currentstroke}{rgb}{1.000000,1.000000,1.000000}%
\pgfsetstrokecolor{currentstroke}%
\pgfsetdash{}{0pt}%
\pgfpathmoveto{\pgfqpoint{4.025822in}{3.570760in}}%
\pgfpathcurveto{\pgfqpoint{4.036872in}{3.570760in}}{\pgfqpoint{4.047471in}{3.575151in}}{\pgfqpoint{4.055285in}{3.582964in}}%
\pgfpathcurveto{\pgfqpoint{4.063098in}{3.590778in}}{\pgfqpoint{4.067489in}{3.601377in}}{\pgfqpoint{4.067489in}{3.612427in}}%
\pgfpathcurveto{\pgfqpoint{4.067489in}{3.623477in}}{\pgfqpoint{4.063098in}{3.634076in}}{\pgfqpoint{4.055285in}{3.641890in}}%
\pgfpathcurveto{\pgfqpoint{4.047471in}{3.649703in}}{\pgfqpoint{4.036872in}{3.654094in}}{\pgfqpoint{4.025822in}{3.654094in}}%
\pgfpathcurveto{\pgfqpoint{4.014772in}{3.654094in}}{\pgfqpoint{4.004173in}{3.649703in}}{\pgfqpoint{3.996359in}{3.641890in}}%
\pgfpathcurveto{\pgfqpoint{3.988546in}{3.634076in}}{\pgfqpoint{3.984155in}{3.623477in}}{\pgfqpoint{3.984155in}{3.612427in}}%
\pgfpathcurveto{\pgfqpoint{3.984155in}{3.601377in}}{\pgfqpoint{3.988546in}{3.590778in}}{\pgfqpoint{3.996359in}{3.582964in}}%
\pgfpathcurveto{\pgfqpoint{4.004173in}{3.575151in}}{\pgfqpoint{4.014772in}{3.570760in}}{\pgfqpoint{4.025822in}{3.570760in}}%
\pgfpathclose%
\pgfusepath{stroke,fill}%
\end{pgfscope}%
\begin{pgfscope}%
\pgfpathrectangle{\pgfqpoint{0.481978in}{0.331635in}}{\pgfqpoint{9.300000in}{7.700000in}}%
\pgfusepath{clip}%
\pgfsetbuttcap%
\pgfsetroundjoin%
\definecolor{currentfill}{rgb}{1.000000,0.705882,0.509804}%
\pgfsetfillcolor{currentfill}%
\pgfsetlinewidth{0.481800pt}%
\definecolor{currentstroke}{rgb}{1.000000,1.000000,1.000000}%
\pgfsetstrokecolor{currentstroke}%
\pgfsetdash{}{0pt}%
\pgfpathmoveto{\pgfqpoint{3.490511in}{5.055258in}}%
\pgfpathcurveto{\pgfqpoint{3.501561in}{5.055258in}}{\pgfqpoint{3.512160in}{5.059649in}}{\pgfqpoint{3.519974in}{5.067462in}}%
\pgfpathcurveto{\pgfqpoint{3.527788in}{5.075276in}}{\pgfqpoint{3.532178in}{5.085875in}}{\pgfqpoint{3.532178in}{5.096925in}}%
\pgfpathcurveto{\pgfqpoint{3.532178in}{5.107975in}}{\pgfqpoint{3.527788in}{5.118574in}}{\pgfqpoint{3.519974in}{5.126388in}}%
\pgfpathcurveto{\pgfqpoint{3.512160in}{5.134202in}}{\pgfqpoint{3.501561in}{5.138592in}}{\pgfqpoint{3.490511in}{5.138592in}}%
\pgfpathcurveto{\pgfqpoint{3.479461in}{5.138592in}}{\pgfqpoint{3.468862in}{5.134202in}}{\pgfqpoint{3.461048in}{5.126388in}}%
\pgfpathcurveto{\pgfqpoint{3.453235in}{5.118574in}}{\pgfqpoint{3.448844in}{5.107975in}}{\pgfqpoint{3.448844in}{5.096925in}}%
\pgfpathcurveto{\pgfqpoint{3.448844in}{5.085875in}}{\pgfqpoint{3.453235in}{5.075276in}}{\pgfqpoint{3.461048in}{5.067462in}}%
\pgfpathcurveto{\pgfqpoint{3.468862in}{5.059649in}}{\pgfqpoint{3.479461in}{5.055258in}}{\pgfqpoint{3.490511in}{5.055258in}}%
\pgfpathclose%
\pgfusepath{stroke,fill}%
\end{pgfscope}%
\begin{pgfscope}%
\pgfpathrectangle{\pgfqpoint{0.481978in}{0.331635in}}{\pgfqpoint{9.300000in}{7.700000in}}%
\pgfusepath{clip}%
\pgfsetbuttcap%
\pgfsetroundjoin%
\definecolor{currentfill}{rgb}{1.000000,0.705882,0.509804}%
\pgfsetfillcolor{currentfill}%
\pgfsetlinewidth{0.481800pt}%
\definecolor{currentstroke}{rgb}{1.000000,1.000000,1.000000}%
\pgfsetstrokecolor{currentstroke}%
\pgfsetdash{}{0pt}%
\pgfpathmoveto{\pgfqpoint{3.394544in}{3.150995in}}%
\pgfpathcurveto{\pgfqpoint{3.405594in}{3.150995in}}{\pgfqpoint{3.416193in}{3.155386in}}{\pgfqpoint{3.424007in}{3.163199in}}%
\pgfpathcurveto{\pgfqpoint{3.431820in}{3.171013in}}{\pgfqpoint{3.436211in}{3.181612in}}{\pgfqpoint{3.436211in}{3.192662in}}%
\pgfpathcurveto{\pgfqpoint{3.436211in}{3.203712in}}{\pgfqpoint{3.431820in}{3.214311in}}{\pgfqpoint{3.424007in}{3.222125in}}%
\pgfpathcurveto{\pgfqpoint{3.416193in}{3.229939in}}{\pgfqpoint{3.405594in}{3.234329in}}{\pgfqpoint{3.394544in}{3.234329in}}%
\pgfpathcurveto{\pgfqpoint{3.383494in}{3.234329in}}{\pgfqpoint{3.372895in}{3.229939in}}{\pgfqpoint{3.365081in}{3.222125in}}%
\pgfpathcurveto{\pgfqpoint{3.357268in}{3.214311in}}{\pgfqpoint{3.352877in}{3.203712in}}{\pgfqpoint{3.352877in}{3.192662in}}%
\pgfpathcurveto{\pgfqpoint{3.352877in}{3.181612in}}{\pgfqpoint{3.357268in}{3.171013in}}{\pgfqpoint{3.365081in}{3.163199in}}%
\pgfpathcurveto{\pgfqpoint{3.372895in}{3.155386in}}{\pgfqpoint{3.383494in}{3.150995in}}{\pgfqpoint{3.394544in}{3.150995in}}%
\pgfpathclose%
\pgfusepath{stroke,fill}%
\end{pgfscope}%
\begin{pgfscope}%
\pgfpathrectangle{\pgfqpoint{0.481978in}{0.331635in}}{\pgfqpoint{9.300000in}{7.700000in}}%
\pgfusepath{clip}%
\pgfsetbuttcap%
\pgfsetroundjoin%
\definecolor{currentfill}{rgb}{1.000000,0.705882,0.509804}%
\pgfsetfillcolor{currentfill}%
\pgfsetlinewidth{0.481800pt}%
\definecolor{currentstroke}{rgb}{1.000000,1.000000,1.000000}%
\pgfsetstrokecolor{currentstroke}%
\pgfsetdash{}{0pt}%
\pgfpathmoveto{\pgfqpoint{3.295196in}{3.631675in}}%
\pgfpathcurveto{\pgfqpoint{3.306246in}{3.631675in}}{\pgfqpoint{3.316845in}{3.636065in}}{\pgfqpoint{3.324659in}{3.643879in}}%
\pgfpathcurveto{\pgfqpoint{3.332473in}{3.651692in}}{\pgfqpoint{3.336863in}{3.662291in}}{\pgfqpoint{3.336863in}{3.673341in}}%
\pgfpathcurveto{\pgfqpoint{3.336863in}{3.684391in}}{\pgfqpoint{3.332473in}{3.694991in}}{\pgfqpoint{3.324659in}{3.702804in}}%
\pgfpathcurveto{\pgfqpoint{3.316845in}{3.710618in}}{\pgfqpoint{3.306246in}{3.715008in}}{\pgfqpoint{3.295196in}{3.715008in}}%
\pgfpathcurveto{\pgfqpoint{3.284146in}{3.715008in}}{\pgfqpoint{3.273547in}{3.710618in}}{\pgfqpoint{3.265733in}{3.702804in}}%
\pgfpathcurveto{\pgfqpoint{3.257920in}{3.694991in}}{\pgfqpoint{3.253529in}{3.684391in}}{\pgfqpoint{3.253529in}{3.673341in}}%
\pgfpathcurveto{\pgfqpoint{3.253529in}{3.662291in}}{\pgfqpoint{3.257920in}{3.651692in}}{\pgfqpoint{3.265733in}{3.643879in}}%
\pgfpathcurveto{\pgfqpoint{3.273547in}{3.636065in}}{\pgfqpoint{3.284146in}{3.631675in}}{\pgfqpoint{3.295196in}{3.631675in}}%
\pgfpathclose%
\pgfusepath{stroke,fill}%
\end{pgfscope}%
\begin{pgfscope}%
\pgfpathrectangle{\pgfqpoint{0.481978in}{0.331635in}}{\pgfqpoint{9.300000in}{7.700000in}}%
\pgfusepath{clip}%
\pgfsetbuttcap%
\pgfsetroundjoin%
\definecolor{currentfill}{rgb}{1.000000,0.705882,0.509804}%
\pgfsetfillcolor{currentfill}%
\pgfsetlinewidth{0.481800pt}%
\definecolor{currentstroke}{rgb}{1.000000,1.000000,1.000000}%
\pgfsetstrokecolor{currentstroke}%
\pgfsetdash{}{0pt}%
\pgfpathmoveto{\pgfqpoint{3.905090in}{4.963082in}}%
\pgfpathcurveto{\pgfqpoint{3.916140in}{4.963082in}}{\pgfqpoint{3.926739in}{4.967472in}}{\pgfqpoint{3.934553in}{4.975286in}}%
\pgfpathcurveto{\pgfqpoint{3.942366in}{4.983099in}}{\pgfqpoint{3.946757in}{4.993698in}}{\pgfqpoint{3.946757in}{5.004749in}}%
\pgfpathcurveto{\pgfqpoint{3.946757in}{5.015799in}}{\pgfqpoint{3.942366in}{5.026398in}}{\pgfqpoint{3.934553in}{5.034211in}}%
\pgfpathcurveto{\pgfqpoint{3.926739in}{5.042025in}}{\pgfqpoint{3.916140in}{5.046415in}}{\pgfqpoint{3.905090in}{5.046415in}}%
\pgfpathcurveto{\pgfqpoint{3.894040in}{5.046415in}}{\pgfqpoint{3.883441in}{5.042025in}}{\pgfqpoint{3.875627in}{5.034211in}}%
\pgfpathcurveto{\pgfqpoint{3.867814in}{5.026398in}}{\pgfqpoint{3.863423in}{5.015799in}}{\pgfqpoint{3.863423in}{5.004749in}}%
\pgfpathcurveto{\pgfqpoint{3.863423in}{4.993698in}}{\pgfqpoint{3.867814in}{4.983099in}}{\pgfqpoint{3.875627in}{4.975286in}}%
\pgfpathcurveto{\pgfqpoint{3.883441in}{4.967472in}}{\pgfqpoint{3.894040in}{4.963082in}}{\pgfqpoint{3.905090in}{4.963082in}}%
\pgfpathclose%
\pgfusepath{stroke,fill}%
\end{pgfscope}%
\begin{pgfscope}%
\pgfpathrectangle{\pgfqpoint{0.481978in}{0.331635in}}{\pgfqpoint{9.300000in}{7.700000in}}%
\pgfusepath{clip}%
\pgfsetbuttcap%
\pgfsetroundjoin%
\definecolor{currentfill}{rgb}{1.000000,0.705882,0.509804}%
\pgfsetfillcolor{currentfill}%
\pgfsetlinewidth{0.481800pt}%
\definecolor{currentstroke}{rgb}{1.000000,1.000000,1.000000}%
\pgfsetstrokecolor{currentstroke}%
\pgfsetdash{}{0pt}%
\pgfpathmoveto{\pgfqpoint{4.067534in}{3.361895in}}%
\pgfpathcurveto{\pgfqpoint{4.078584in}{3.361895in}}{\pgfqpoint{4.089183in}{3.366285in}}{\pgfqpoint{4.096996in}{3.374098in}}%
\pgfpathcurveto{\pgfqpoint{4.104810in}{3.381912in}}{\pgfqpoint{4.109200in}{3.392511in}}{\pgfqpoint{4.109200in}{3.403561in}}%
\pgfpathcurveto{\pgfqpoint{4.109200in}{3.414611in}}{\pgfqpoint{4.104810in}{3.425210in}}{\pgfqpoint{4.096996in}{3.433024in}}%
\pgfpathcurveto{\pgfqpoint{4.089183in}{3.440838in}}{\pgfqpoint{4.078584in}{3.445228in}}{\pgfqpoint{4.067534in}{3.445228in}}%
\pgfpathcurveto{\pgfqpoint{4.056483in}{3.445228in}}{\pgfqpoint{4.045884in}{3.440838in}}{\pgfqpoint{4.038071in}{3.433024in}}%
\pgfpathcurveto{\pgfqpoint{4.030257in}{3.425210in}}{\pgfqpoint{4.025867in}{3.414611in}}{\pgfqpoint{4.025867in}{3.403561in}}%
\pgfpathcurveto{\pgfqpoint{4.025867in}{3.392511in}}{\pgfqpoint{4.030257in}{3.381912in}}{\pgfqpoint{4.038071in}{3.374098in}}%
\pgfpathcurveto{\pgfqpoint{4.045884in}{3.366285in}}{\pgfqpoint{4.056483in}{3.361895in}}{\pgfqpoint{4.067534in}{3.361895in}}%
\pgfpathclose%
\pgfusepath{stroke,fill}%
\end{pgfscope}%
\begin{pgfscope}%
\pgfpathrectangle{\pgfqpoint{0.481978in}{0.331635in}}{\pgfqpoint{9.300000in}{7.700000in}}%
\pgfusepath{clip}%
\pgfsetbuttcap%
\pgfsetroundjoin%
\definecolor{currentfill}{rgb}{1.000000,0.705882,0.509804}%
\pgfsetfillcolor{currentfill}%
\pgfsetlinewidth{0.481800pt}%
\definecolor{currentstroke}{rgb}{1.000000,1.000000,1.000000}%
\pgfsetstrokecolor{currentstroke}%
\pgfsetdash{}{0pt}%
\pgfpathmoveto{\pgfqpoint{1.713787in}{3.836311in}}%
\pgfpathcurveto{\pgfqpoint{1.724837in}{3.836311in}}{\pgfqpoint{1.735436in}{3.840701in}}{\pgfqpoint{1.743250in}{3.848515in}}%
\pgfpathcurveto{\pgfqpoint{1.751064in}{3.856329in}}{\pgfqpoint{1.755454in}{3.866928in}}{\pgfqpoint{1.755454in}{3.877978in}}%
\pgfpathcurveto{\pgfqpoint{1.755454in}{3.889028in}}{\pgfqpoint{1.751064in}{3.899627in}}{\pgfqpoint{1.743250in}{3.907440in}}%
\pgfpathcurveto{\pgfqpoint{1.735436in}{3.915254in}}{\pgfqpoint{1.724837in}{3.919644in}}{\pgfqpoint{1.713787in}{3.919644in}}%
\pgfpathcurveto{\pgfqpoint{1.702737in}{3.919644in}}{\pgfqpoint{1.692138in}{3.915254in}}{\pgfqpoint{1.684324in}{3.907440in}}%
\pgfpathcurveto{\pgfqpoint{1.676511in}{3.899627in}}{\pgfqpoint{1.672121in}{3.889028in}}{\pgfqpoint{1.672121in}{3.877978in}}%
\pgfpathcurveto{\pgfqpoint{1.672121in}{3.866928in}}{\pgfqpoint{1.676511in}{3.856329in}}{\pgfqpoint{1.684324in}{3.848515in}}%
\pgfpathcurveto{\pgfqpoint{1.692138in}{3.840701in}}{\pgfqpoint{1.702737in}{3.836311in}}{\pgfqpoint{1.713787in}{3.836311in}}%
\pgfpathclose%
\pgfusepath{stroke,fill}%
\end{pgfscope}%
\begin{pgfscope}%
\pgfpathrectangle{\pgfqpoint{0.481978in}{0.331635in}}{\pgfqpoint{9.300000in}{7.700000in}}%
\pgfusepath{clip}%
\pgfsetbuttcap%
\pgfsetroundjoin%
\definecolor{currentfill}{rgb}{1.000000,0.705882,0.509804}%
\pgfsetfillcolor{currentfill}%
\pgfsetlinewidth{0.481800pt}%
\definecolor{currentstroke}{rgb}{1.000000,1.000000,1.000000}%
\pgfsetstrokecolor{currentstroke}%
\pgfsetdash{}{0pt}%
\pgfpathmoveto{\pgfqpoint{2.059946in}{4.470058in}}%
\pgfpathcurveto{\pgfqpoint{2.070996in}{4.470058in}}{\pgfqpoint{2.081595in}{4.474448in}}{\pgfqpoint{2.089408in}{4.482262in}}%
\pgfpathcurveto{\pgfqpoint{2.097222in}{4.490075in}}{\pgfqpoint{2.101612in}{4.500674in}}{\pgfqpoint{2.101612in}{4.511724in}}%
\pgfpathcurveto{\pgfqpoint{2.101612in}{4.522775in}}{\pgfqpoint{2.097222in}{4.533374in}}{\pgfqpoint{2.089408in}{4.541187in}}%
\pgfpathcurveto{\pgfqpoint{2.081595in}{4.549001in}}{\pgfqpoint{2.070996in}{4.553391in}}{\pgfqpoint{2.059946in}{4.553391in}}%
\pgfpathcurveto{\pgfqpoint{2.048895in}{4.553391in}}{\pgfqpoint{2.038296in}{4.549001in}}{\pgfqpoint{2.030483in}{4.541187in}}%
\pgfpathcurveto{\pgfqpoint{2.022669in}{4.533374in}}{\pgfqpoint{2.018279in}{4.522775in}}{\pgfqpoint{2.018279in}{4.511724in}}%
\pgfpathcurveto{\pgfqpoint{2.018279in}{4.500674in}}{\pgfqpoint{2.022669in}{4.490075in}}{\pgfqpoint{2.030483in}{4.482262in}}%
\pgfpathcurveto{\pgfqpoint{2.038296in}{4.474448in}}{\pgfqpoint{2.048895in}{4.470058in}}{\pgfqpoint{2.059946in}{4.470058in}}%
\pgfpathclose%
\pgfusepath{stroke,fill}%
\end{pgfscope}%
\begin{pgfscope}%
\pgfpathrectangle{\pgfqpoint{0.481978in}{0.331635in}}{\pgfqpoint{9.300000in}{7.700000in}}%
\pgfusepath{clip}%
\pgfsetbuttcap%
\pgfsetroundjoin%
\definecolor{currentfill}{rgb}{1.000000,0.705882,0.509804}%
\pgfsetfillcolor{currentfill}%
\pgfsetlinewidth{0.481800pt}%
\definecolor{currentstroke}{rgb}{1.000000,1.000000,1.000000}%
\pgfsetstrokecolor{currentstroke}%
\pgfsetdash{}{0pt}%
\pgfpathmoveto{\pgfqpoint{3.097846in}{5.814328in}}%
\pgfpathcurveto{\pgfqpoint{3.108896in}{5.814328in}}{\pgfqpoint{3.119495in}{5.818719in}}{\pgfqpoint{3.127309in}{5.826532in}}%
\pgfpathcurveto{\pgfqpoint{3.135122in}{5.834346in}}{\pgfqpoint{3.139513in}{5.844945in}}{\pgfqpoint{3.139513in}{5.855995in}}%
\pgfpathcurveto{\pgfqpoint{3.139513in}{5.867045in}}{\pgfqpoint{3.135122in}{5.877644in}}{\pgfqpoint{3.127309in}{5.885458in}}%
\pgfpathcurveto{\pgfqpoint{3.119495in}{5.893271in}}{\pgfqpoint{3.108896in}{5.897662in}}{\pgfqpoint{3.097846in}{5.897662in}}%
\pgfpathcurveto{\pgfqpoint{3.086796in}{5.897662in}}{\pgfqpoint{3.076197in}{5.893271in}}{\pgfqpoint{3.068383in}{5.885458in}}%
\pgfpathcurveto{\pgfqpoint{3.060570in}{5.877644in}}{\pgfqpoint{3.056179in}{5.867045in}}{\pgfqpoint{3.056179in}{5.855995in}}%
\pgfpathcurveto{\pgfqpoint{3.056179in}{5.844945in}}{\pgfqpoint{3.060570in}{5.834346in}}{\pgfqpoint{3.068383in}{5.826532in}}%
\pgfpathcurveto{\pgfqpoint{3.076197in}{5.818719in}}{\pgfqpoint{3.086796in}{5.814328in}}{\pgfqpoint{3.097846in}{5.814328in}}%
\pgfpathclose%
\pgfusepath{stroke,fill}%
\end{pgfscope}%
\begin{pgfscope}%
\pgfpathrectangle{\pgfqpoint{0.481978in}{0.331635in}}{\pgfqpoint{9.300000in}{7.700000in}}%
\pgfusepath{clip}%
\pgfsetbuttcap%
\pgfsetroundjoin%
\definecolor{currentfill}{rgb}{1.000000,0.705882,0.509804}%
\pgfsetfillcolor{currentfill}%
\pgfsetlinewidth{0.481800pt}%
\definecolor{currentstroke}{rgb}{1.000000,1.000000,1.000000}%
\pgfsetstrokecolor{currentstroke}%
\pgfsetdash{}{0pt}%
\pgfpathmoveto{\pgfqpoint{2.416512in}{2.668881in}}%
\pgfpathcurveto{\pgfqpoint{2.427562in}{2.668881in}}{\pgfqpoint{2.438161in}{2.673271in}}{\pgfqpoint{2.445975in}{2.681085in}}%
\pgfpathcurveto{\pgfqpoint{2.453789in}{2.688899in}}{\pgfqpoint{2.458179in}{2.699498in}}{\pgfqpoint{2.458179in}{2.710548in}}%
\pgfpathcurveto{\pgfqpoint{2.458179in}{2.721598in}}{\pgfqpoint{2.453789in}{2.732197in}}{\pgfqpoint{2.445975in}{2.740011in}}%
\pgfpathcurveto{\pgfqpoint{2.438161in}{2.747824in}}{\pgfqpoint{2.427562in}{2.752214in}}{\pgfqpoint{2.416512in}{2.752214in}}%
\pgfpathcurveto{\pgfqpoint{2.405462in}{2.752214in}}{\pgfqpoint{2.394863in}{2.747824in}}{\pgfqpoint{2.387049in}{2.740011in}}%
\pgfpathcurveto{\pgfqpoint{2.379236in}{2.732197in}}{\pgfqpoint{2.374846in}{2.721598in}}{\pgfqpoint{2.374846in}{2.710548in}}%
\pgfpathcurveto{\pgfqpoint{2.374846in}{2.699498in}}{\pgfqpoint{2.379236in}{2.688899in}}{\pgfqpoint{2.387049in}{2.681085in}}%
\pgfpathcurveto{\pgfqpoint{2.394863in}{2.673271in}}{\pgfqpoint{2.405462in}{2.668881in}}{\pgfqpoint{2.416512in}{2.668881in}}%
\pgfpathclose%
\pgfusepath{stroke,fill}%
\end{pgfscope}%
\begin{pgfscope}%
\pgfpathrectangle{\pgfqpoint{0.481978in}{0.331635in}}{\pgfqpoint{9.300000in}{7.700000in}}%
\pgfusepath{clip}%
\pgfsetbuttcap%
\pgfsetroundjoin%
\definecolor{currentfill}{rgb}{1.000000,0.705882,0.509804}%
\pgfsetfillcolor{currentfill}%
\pgfsetlinewidth{0.481800pt}%
\definecolor{currentstroke}{rgb}{1.000000,1.000000,1.000000}%
\pgfsetstrokecolor{currentstroke}%
\pgfsetdash{}{0pt}%
\pgfpathmoveto{\pgfqpoint{1.548965in}{4.218503in}}%
\pgfpathcurveto{\pgfqpoint{1.560015in}{4.218503in}}{\pgfqpoint{1.570614in}{4.222894in}}{\pgfqpoint{1.578428in}{4.230707in}}%
\pgfpathcurveto{\pgfqpoint{1.586241in}{4.238521in}}{\pgfqpoint{1.590632in}{4.249120in}}{\pgfqpoint{1.590632in}{4.260170in}}%
\pgfpathcurveto{\pgfqpoint{1.590632in}{4.271220in}}{\pgfqpoint{1.586241in}{4.281819in}}{\pgfqpoint{1.578428in}{4.289633in}}%
\pgfpathcurveto{\pgfqpoint{1.570614in}{4.297447in}}{\pgfqpoint{1.560015in}{4.301837in}}{\pgfqpoint{1.548965in}{4.301837in}}%
\pgfpathcurveto{\pgfqpoint{1.537915in}{4.301837in}}{\pgfqpoint{1.527316in}{4.297447in}}{\pgfqpoint{1.519502in}{4.289633in}}%
\pgfpathcurveto{\pgfqpoint{1.511689in}{4.281819in}}{\pgfqpoint{1.507298in}{4.271220in}}{\pgfqpoint{1.507298in}{4.260170in}}%
\pgfpathcurveto{\pgfqpoint{1.507298in}{4.249120in}}{\pgfqpoint{1.511689in}{4.238521in}}{\pgfqpoint{1.519502in}{4.230707in}}%
\pgfpathcurveto{\pgfqpoint{1.527316in}{4.222894in}}{\pgfqpoint{1.537915in}{4.218503in}}{\pgfqpoint{1.548965in}{4.218503in}}%
\pgfpathclose%
\pgfusepath{stroke,fill}%
\end{pgfscope}%
\begin{pgfscope}%
\pgfpathrectangle{\pgfqpoint{0.481978in}{0.331635in}}{\pgfqpoint{9.300000in}{7.700000in}}%
\pgfusepath{clip}%
\pgfsetbuttcap%
\pgfsetroundjoin%
\definecolor{currentfill}{rgb}{1.000000,0.705882,0.509804}%
\pgfsetfillcolor{currentfill}%
\pgfsetlinewidth{0.481800pt}%
\definecolor{currentstroke}{rgb}{1.000000,1.000000,1.000000}%
\pgfsetstrokecolor{currentstroke}%
\pgfsetdash{}{0pt}%
\pgfpathmoveto{\pgfqpoint{4.223289in}{4.599896in}}%
\pgfpathcurveto{\pgfqpoint{4.234340in}{4.599896in}}{\pgfqpoint{4.244939in}{4.604286in}}{\pgfqpoint{4.252752in}{4.612099in}}%
\pgfpathcurveto{\pgfqpoint{4.260566in}{4.619913in}}{\pgfqpoint{4.264956in}{4.630512in}}{\pgfqpoint{4.264956in}{4.641562in}}%
\pgfpathcurveto{\pgfqpoint{4.264956in}{4.652612in}}{\pgfqpoint{4.260566in}{4.663211in}}{\pgfqpoint{4.252752in}{4.671025in}}%
\pgfpathcurveto{\pgfqpoint{4.244939in}{4.678839in}}{\pgfqpoint{4.234340in}{4.683229in}}{\pgfqpoint{4.223289in}{4.683229in}}%
\pgfpathcurveto{\pgfqpoint{4.212239in}{4.683229in}}{\pgfqpoint{4.201640in}{4.678839in}}{\pgfqpoint{4.193827in}{4.671025in}}%
\pgfpathcurveto{\pgfqpoint{4.186013in}{4.663211in}}{\pgfqpoint{4.181623in}{4.652612in}}{\pgfqpoint{4.181623in}{4.641562in}}%
\pgfpathcurveto{\pgfqpoint{4.181623in}{4.630512in}}{\pgfqpoint{4.186013in}{4.619913in}}{\pgfqpoint{4.193827in}{4.612099in}}%
\pgfpathcurveto{\pgfqpoint{4.201640in}{4.604286in}}{\pgfqpoint{4.212239in}{4.599896in}}{\pgfqpoint{4.223289in}{4.599896in}}%
\pgfpathclose%
\pgfusepath{stroke,fill}%
\end{pgfscope}%
\begin{pgfscope}%
\pgfpathrectangle{\pgfqpoint{0.481978in}{0.331635in}}{\pgfqpoint{9.300000in}{7.700000in}}%
\pgfusepath{clip}%
\pgfsetbuttcap%
\pgfsetroundjoin%
\definecolor{currentfill}{rgb}{1.000000,0.705882,0.509804}%
\pgfsetfillcolor{currentfill}%
\pgfsetlinewidth{0.481800pt}%
\definecolor{currentstroke}{rgb}{1.000000,1.000000,1.000000}%
\pgfsetstrokecolor{currentstroke}%
\pgfsetdash{}{0pt}%
\pgfpathmoveto{\pgfqpoint{3.078663in}{3.560810in}}%
\pgfpathcurveto{\pgfqpoint{3.089713in}{3.560810in}}{\pgfqpoint{3.100312in}{3.565200in}}{\pgfqpoint{3.108126in}{3.573014in}}%
\pgfpathcurveto{\pgfqpoint{3.115940in}{3.580828in}}{\pgfqpoint{3.120330in}{3.591427in}}{\pgfqpoint{3.120330in}{3.602477in}}%
\pgfpathcurveto{\pgfqpoint{3.120330in}{3.613527in}}{\pgfqpoint{3.115940in}{3.624126in}}{\pgfqpoint{3.108126in}{3.631940in}}%
\pgfpathcurveto{\pgfqpoint{3.100312in}{3.639753in}}{\pgfqpoint{3.089713in}{3.644144in}}{\pgfqpoint{3.078663in}{3.644144in}}%
\pgfpathcurveto{\pgfqpoint{3.067613in}{3.644144in}}{\pgfqpoint{3.057014in}{3.639753in}}{\pgfqpoint{3.049200in}{3.631940in}}%
\pgfpathcurveto{\pgfqpoint{3.041387in}{3.624126in}}{\pgfqpoint{3.036997in}{3.613527in}}{\pgfqpoint{3.036997in}{3.602477in}}%
\pgfpathcurveto{\pgfqpoint{3.036997in}{3.591427in}}{\pgfqpoint{3.041387in}{3.580828in}}{\pgfqpoint{3.049200in}{3.573014in}}%
\pgfpathcurveto{\pgfqpoint{3.057014in}{3.565200in}}{\pgfqpoint{3.067613in}{3.560810in}}{\pgfqpoint{3.078663in}{3.560810in}}%
\pgfpathclose%
\pgfusepath{stroke,fill}%
\end{pgfscope}%
\begin{pgfscope}%
\pgfpathrectangle{\pgfqpoint{0.481978in}{0.331635in}}{\pgfqpoint{9.300000in}{7.700000in}}%
\pgfusepath{clip}%
\pgfsetbuttcap%
\pgfsetroundjoin%
\definecolor{currentfill}{rgb}{1.000000,0.705882,0.509804}%
\pgfsetfillcolor{currentfill}%
\pgfsetlinewidth{0.481800pt}%
\definecolor{currentstroke}{rgb}{1.000000,1.000000,1.000000}%
\pgfsetstrokecolor{currentstroke}%
\pgfsetdash{}{0pt}%
\pgfpathmoveto{\pgfqpoint{4.163791in}{4.017675in}}%
\pgfpathcurveto{\pgfqpoint{4.174841in}{4.017675in}}{\pgfqpoint{4.185441in}{4.022066in}}{\pgfqpoint{4.193254in}{4.029879in}}%
\pgfpathcurveto{\pgfqpoint{4.201068in}{4.037693in}}{\pgfqpoint{4.205458in}{4.048292in}}{\pgfqpoint{4.205458in}{4.059342in}}%
\pgfpathcurveto{\pgfqpoint{4.205458in}{4.070392in}}{\pgfqpoint{4.201068in}{4.080991in}}{\pgfqpoint{4.193254in}{4.088805in}}%
\pgfpathcurveto{\pgfqpoint{4.185441in}{4.096618in}}{\pgfqpoint{4.174841in}{4.101009in}}{\pgfqpoint{4.163791in}{4.101009in}}%
\pgfpathcurveto{\pgfqpoint{4.152741in}{4.101009in}}{\pgfqpoint{4.142142in}{4.096618in}}{\pgfqpoint{4.134329in}{4.088805in}}%
\pgfpathcurveto{\pgfqpoint{4.126515in}{4.080991in}}{\pgfqpoint{4.122125in}{4.070392in}}{\pgfqpoint{4.122125in}{4.059342in}}%
\pgfpathcurveto{\pgfqpoint{4.122125in}{4.048292in}}{\pgfqpoint{4.126515in}{4.037693in}}{\pgfqpoint{4.134329in}{4.029879in}}%
\pgfpathcurveto{\pgfqpoint{4.142142in}{4.022066in}}{\pgfqpoint{4.152741in}{4.017675in}}{\pgfqpoint{4.163791in}{4.017675in}}%
\pgfpathclose%
\pgfusepath{stroke,fill}%
\end{pgfscope}%
\begin{pgfscope}%
\pgfpathrectangle{\pgfqpoint{0.481978in}{0.331635in}}{\pgfqpoint{9.300000in}{7.700000in}}%
\pgfusepath{clip}%
\pgfsetbuttcap%
\pgfsetroundjoin%
\definecolor{currentfill}{rgb}{1.000000,0.705882,0.509804}%
\pgfsetfillcolor{currentfill}%
\pgfsetlinewidth{0.481800pt}%
\definecolor{currentstroke}{rgb}{1.000000,1.000000,1.000000}%
\pgfsetstrokecolor{currentstroke}%
\pgfsetdash{}{0pt}%
\pgfpathmoveto{\pgfqpoint{1.998764in}{4.788927in}}%
\pgfpathcurveto{\pgfqpoint{2.009814in}{4.788927in}}{\pgfqpoint{2.020413in}{4.793317in}}{\pgfqpoint{2.028227in}{4.801131in}}%
\pgfpathcurveto{\pgfqpoint{2.036040in}{4.808945in}}{\pgfqpoint{2.040431in}{4.819544in}}{\pgfqpoint{2.040431in}{4.830594in}}%
\pgfpathcurveto{\pgfqpoint{2.040431in}{4.841644in}}{\pgfqpoint{2.036040in}{4.852243in}}{\pgfqpoint{2.028227in}{4.860057in}}%
\pgfpathcurveto{\pgfqpoint{2.020413in}{4.867870in}}{\pgfqpoint{2.009814in}{4.872261in}}{\pgfqpoint{1.998764in}{4.872261in}}%
\pgfpathcurveto{\pgfqpoint{1.987714in}{4.872261in}}{\pgfqpoint{1.977115in}{4.867870in}}{\pgfqpoint{1.969301in}{4.860057in}}%
\pgfpathcurveto{\pgfqpoint{1.961487in}{4.852243in}}{\pgfqpoint{1.957097in}{4.841644in}}{\pgfqpoint{1.957097in}{4.830594in}}%
\pgfpathcurveto{\pgfqpoint{1.957097in}{4.819544in}}{\pgfqpoint{1.961487in}{4.808945in}}{\pgfqpoint{1.969301in}{4.801131in}}%
\pgfpathcurveto{\pgfqpoint{1.977115in}{4.793317in}}{\pgfqpoint{1.987714in}{4.788927in}}{\pgfqpoint{1.998764in}{4.788927in}}%
\pgfpathclose%
\pgfusepath{stroke,fill}%
\end{pgfscope}%
\begin{pgfscope}%
\pgfpathrectangle{\pgfqpoint{0.481978in}{0.331635in}}{\pgfqpoint{9.300000in}{7.700000in}}%
\pgfusepath{clip}%
\pgfsetbuttcap%
\pgfsetroundjoin%
\definecolor{currentfill}{rgb}{1.000000,0.705882,0.509804}%
\pgfsetfillcolor{currentfill}%
\pgfsetlinewidth{0.481800pt}%
\definecolor{currentstroke}{rgb}{1.000000,1.000000,1.000000}%
\pgfsetstrokecolor{currentstroke}%
\pgfsetdash{}{0pt}%
\pgfpathmoveto{\pgfqpoint{4.371217in}{3.333039in}}%
\pgfpathcurveto{\pgfqpoint{4.382267in}{3.333039in}}{\pgfqpoint{4.392866in}{3.337430in}}{\pgfqpoint{4.400680in}{3.345243in}}%
\pgfpathcurveto{\pgfqpoint{4.408494in}{3.353057in}}{\pgfqpoint{4.412884in}{3.363656in}}{\pgfqpoint{4.412884in}{3.374706in}}%
\pgfpathcurveto{\pgfqpoint{4.412884in}{3.385756in}}{\pgfqpoint{4.408494in}{3.396355in}}{\pgfqpoint{4.400680in}{3.404169in}}%
\pgfpathcurveto{\pgfqpoint{4.392866in}{3.411982in}}{\pgfqpoint{4.382267in}{3.416373in}}{\pgfqpoint{4.371217in}{3.416373in}}%
\pgfpathcurveto{\pgfqpoint{4.360167in}{3.416373in}}{\pgfqpoint{4.349568in}{3.411982in}}{\pgfqpoint{4.341755in}{3.404169in}}%
\pgfpathcurveto{\pgfqpoint{4.333941in}{3.396355in}}{\pgfqpoint{4.329551in}{3.385756in}}{\pgfqpoint{4.329551in}{3.374706in}}%
\pgfpathcurveto{\pgfqpoint{4.329551in}{3.363656in}}{\pgfqpoint{4.333941in}{3.353057in}}{\pgfqpoint{4.341755in}{3.345243in}}%
\pgfpathcurveto{\pgfqpoint{4.349568in}{3.337430in}}{\pgfqpoint{4.360167in}{3.333039in}}{\pgfqpoint{4.371217in}{3.333039in}}%
\pgfpathclose%
\pgfusepath{stroke,fill}%
\end{pgfscope}%
\begin{pgfscope}%
\pgfpathrectangle{\pgfqpoint{0.481978in}{0.331635in}}{\pgfqpoint{9.300000in}{7.700000in}}%
\pgfusepath{clip}%
\pgfsetbuttcap%
\pgfsetroundjoin%
\definecolor{currentfill}{rgb}{1.000000,0.705882,0.509804}%
\pgfsetfillcolor{currentfill}%
\pgfsetlinewidth{0.481800pt}%
\definecolor{currentstroke}{rgb}{1.000000,1.000000,1.000000}%
\pgfsetstrokecolor{currentstroke}%
\pgfsetdash{}{0pt}%
\pgfpathmoveto{\pgfqpoint{2.452590in}{4.204860in}}%
\pgfpathcurveto{\pgfqpoint{2.463640in}{4.204860in}}{\pgfqpoint{2.474239in}{4.209250in}}{\pgfqpoint{2.482053in}{4.217064in}}%
\pgfpathcurveto{\pgfqpoint{2.489866in}{4.224877in}}{\pgfqpoint{2.494257in}{4.235476in}}{\pgfqpoint{2.494257in}{4.246526in}}%
\pgfpathcurveto{\pgfqpoint{2.494257in}{4.257577in}}{\pgfqpoint{2.489866in}{4.268176in}}{\pgfqpoint{2.482053in}{4.275989in}}%
\pgfpathcurveto{\pgfqpoint{2.474239in}{4.283803in}}{\pgfqpoint{2.463640in}{4.288193in}}{\pgfqpoint{2.452590in}{4.288193in}}%
\pgfpathcurveto{\pgfqpoint{2.441540in}{4.288193in}}{\pgfqpoint{2.430941in}{4.283803in}}{\pgfqpoint{2.423127in}{4.275989in}}%
\pgfpathcurveto{\pgfqpoint{2.415314in}{4.268176in}}{\pgfqpoint{2.410923in}{4.257577in}}{\pgfqpoint{2.410923in}{4.246526in}}%
\pgfpathcurveto{\pgfqpoint{2.410923in}{4.235476in}}{\pgfqpoint{2.415314in}{4.224877in}}{\pgfqpoint{2.423127in}{4.217064in}}%
\pgfpathcurveto{\pgfqpoint{2.430941in}{4.209250in}}{\pgfqpoint{2.441540in}{4.204860in}}{\pgfqpoint{2.452590in}{4.204860in}}%
\pgfpathclose%
\pgfusepath{stroke,fill}%
\end{pgfscope}%
\begin{pgfscope}%
\pgfpathrectangle{\pgfqpoint{0.481978in}{0.331635in}}{\pgfqpoint{9.300000in}{7.700000in}}%
\pgfusepath{clip}%
\pgfsetbuttcap%
\pgfsetroundjoin%
\definecolor{currentfill}{rgb}{1.000000,0.705882,0.509804}%
\pgfsetfillcolor{currentfill}%
\pgfsetlinewidth{0.481800pt}%
\definecolor{currentstroke}{rgb}{1.000000,1.000000,1.000000}%
\pgfsetstrokecolor{currentstroke}%
\pgfsetdash{}{0pt}%
\pgfpathmoveto{\pgfqpoint{4.281738in}{4.785661in}}%
\pgfpathcurveto{\pgfqpoint{4.292788in}{4.785661in}}{\pgfqpoint{4.303387in}{4.790051in}}{\pgfqpoint{4.311201in}{4.797865in}}%
\pgfpathcurveto{\pgfqpoint{4.319014in}{4.805679in}}{\pgfqpoint{4.323405in}{4.816278in}}{\pgfqpoint{4.323405in}{4.827328in}}%
\pgfpathcurveto{\pgfqpoint{4.323405in}{4.838378in}}{\pgfqpoint{4.319014in}{4.848977in}}{\pgfqpoint{4.311201in}{4.856790in}}%
\pgfpathcurveto{\pgfqpoint{4.303387in}{4.864604in}}{\pgfqpoint{4.292788in}{4.868994in}}{\pgfqpoint{4.281738in}{4.868994in}}%
\pgfpathcurveto{\pgfqpoint{4.270688in}{4.868994in}}{\pgfqpoint{4.260089in}{4.864604in}}{\pgfqpoint{4.252275in}{4.856790in}}%
\pgfpathcurveto{\pgfqpoint{4.244462in}{4.848977in}}{\pgfqpoint{4.240071in}{4.838378in}}{\pgfqpoint{4.240071in}{4.827328in}}%
\pgfpathcurveto{\pgfqpoint{4.240071in}{4.816278in}}{\pgfqpoint{4.244462in}{4.805679in}}{\pgfqpoint{4.252275in}{4.797865in}}%
\pgfpathcurveto{\pgfqpoint{4.260089in}{4.790051in}}{\pgfqpoint{4.270688in}{4.785661in}}{\pgfqpoint{4.281738in}{4.785661in}}%
\pgfpathclose%
\pgfusepath{stroke,fill}%
\end{pgfscope}%
\begin{pgfscope}%
\pgfpathrectangle{\pgfqpoint{0.481978in}{0.331635in}}{\pgfqpoint{9.300000in}{7.700000in}}%
\pgfusepath{clip}%
\pgfsetbuttcap%
\pgfsetroundjoin%
\definecolor{currentfill}{rgb}{1.000000,0.705882,0.509804}%
\pgfsetfillcolor{currentfill}%
\pgfsetlinewidth{0.481800pt}%
\definecolor{currentstroke}{rgb}{1.000000,1.000000,1.000000}%
\pgfsetstrokecolor{currentstroke}%
\pgfsetdash{}{0pt}%
\pgfpathmoveto{\pgfqpoint{3.234311in}{3.396812in}}%
\pgfpathcurveto{\pgfqpoint{3.245361in}{3.396812in}}{\pgfqpoint{3.255960in}{3.401202in}}{\pgfqpoint{3.263774in}{3.409016in}}%
\pgfpathcurveto{\pgfqpoint{3.271587in}{3.416830in}}{\pgfqpoint{3.275977in}{3.427429in}}{\pgfqpoint{3.275977in}{3.438479in}}%
\pgfpathcurveto{\pgfqpoint{3.275977in}{3.449529in}}{\pgfqpoint{3.271587in}{3.460128in}}{\pgfqpoint{3.263774in}{3.467941in}}%
\pgfpathcurveto{\pgfqpoint{3.255960in}{3.475755in}}{\pgfqpoint{3.245361in}{3.480145in}}{\pgfqpoint{3.234311in}{3.480145in}}%
\pgfpathcurveto{\pgfqpoint{3.223261in}{3.480145in}}{\pgfqpoint{3.212662in}{3.475755in}}{\pgfqpoint{3.204848in}{3.467941in}}%
\pgfpathcurveto{\pgfqpoint{3.197034in}{3.460128in}}{\pgfqpoint{3.192644in}{3.449529in}}{\pgfqpoint{3.192644in}{3.438479in}}%
\pgfpathcurveto{\pgfqpoint{3.192644in}{3.427429in}}{\pgfqpoint{3.197034in}{3.416830in}}{\pgfqpoint{3.204848in}{3.409016in}}%
\pgfpathcurveto{\pgfqpoint{3.212662in}{3.401202in}}{\pgfqpoint{3.223261in}{3.396812in}}{\pgfqpoint{3.234311in}{3.396812in}}%
\pgfpathclose%
\pgfusepath{stroke,fill}%
\end{pgfscope}%
\begin{pgfscope}%
\pgfpathrectangle{\pgfqpoint{0.481978in}{0.331635in}}{\pgfqpoint{9.300000in}{7.700000in}}%
\pgfusepath{clip}%
\pgfsetbuttcap%
\pgfsetroundjoin%
\definecolor{currentfill}{rgb}{1.000000,0.705882,0.509804}%
\pgfsetfillcolor{currentfill}%
\pgfsetlinewidth{0.481800pt}%
\definecolor{currentstroke}{rgb}{1.000000,1.000000,1.000000}%
\pgfsetstrokecolor{currentstroke}%
\pgfsetdash{}{0pt}%
\pgfpathmoveto{\pgfqpoint{3.915243in}{6.654693in}}%
\pgfpathcurveto{\pgfqpoint{3.926293in}{6.654693in}}{\pgfqpoint{3.936892in}{6.659084in}}{\pgfqpoint{3.944705in}{6.666897in}}%
\pgfpathcurveto{\pgfqpoint{3.952519in}{6.674711in}}{\pgfqpoint{3.956909in}{6.685310in}}{\pgfqpoint{3.956909in}{6.696360in}}%
\pgfpathcurveto{\pgfqpoint{3.956909in}{6.707410in}}{\pgfqpoint{3.952519in}{6.718009in}}{\pgfqpoint{3.944705in}{6.725823in}}%
\pgfpathcurveto{\pgfqpoint{3.936892in}{6.733637in}}{\pgfqpoint{3.926293in}{6.738027in}}{\pgfqpoint{3.915243in}{6.738027in}}%
\pgfpathcurveto{\pgfqpoint{3.904192in}{6.738027in}}{\pgfqpoint{3.893593in}{6.733637in}}{\pgfqpoint{3.885780in}{6.725823in}}%
\pgfpathcurveto{\pgfqpoint{3.877966in}{6.718009in}}{\pgfqpoint{3.873576in}{6.707410in}}{\pgfqpoint{3.873576in}{6.696360in}}%
\pgfpathcurveto{\pgfqpoint{3.873576in}{6.685310in}}{\pgfqpoint{3.877966in}{6.674711in}}{\pgfqpoint{3.885780in}{6.666897in}}%
\pgfpathcurveto{\pgfqpoint{3.893593in}{6.659084in}}{\pgfqpoint{3.904192in}{6.654693in}}{\pgfqpoint{3.915243in}{6.654693in}}%
\pgfpathclose%
\pgfusepath{stroke,fill}%
\end{pgfscope}%
\begin{pgfscope}%
\pgfpathrectangle{\pgfqpoint{0.481978in}{0.331635in}}{\pgfqpoint{9.300000in}{7.700000in}}%
\pgfusepath{clip}%
\pgfsetbuttcap%
\pgfsetroundjoin%
\definecolor{currentfill}{rgb}{1.000000,0.705882,0.509804}%
\pgfsetfillcolor{currentfill}%
\pgfsetlinewidth{0.481800pt}%
\definecolor{currentstroke}{rgb}{1.000000,1.000000,1.000000}%
\pgfsetstrokecolor{currentstroke}%
\pgfsetdash{}{0pt}%
\pgfpathmoveto{\pgfqpoint{4.358866in}{3.143475in}}%
\pgfpathcurveto{\pgfqpoint{4.369916in}{3.143475in}}{\pgfqpoint{4.380515in}{3.147866in}}{\pgfqpoint{4.388328in}{3.155679in}}%
\pgfpathcurveto{\pgfqpoint{4.396142in}{3.163493in}}{\pgfqpoint{4.400532in}{3.174092in}}{\pgfqpoint{4.400532in}{3.185142in}}%
\pgfpathcurveto{\pgfqpoint{4.400532in}{3.196192in}}{\pgfqpoint{4.396142in}{3.206791in}}{\pgfqpoint{4.388328in}{3.214605in}}%
\pgfpathcurveto{\pgfqpoint{4.380515in}{3.222418in}}{\pgfqpoint{4.369916in}{3.226809in}}{\pgfqpoint{4.358866in}{3.226809in}}%
\pgfpathcurveto{\pgfqpoint{4.347816in}{3.226809in}}{\pgfqpoint{4.337217in}{3.222418in}}{\pgfqpoint{4.329403in}{3.214605in}}%
\pgfpathcurveto{\pgfqpoint{4.321589in}{3.206791in}}{\pgfqpoint{4.317199in}{3.196192in}}{\pgfqpoint{4.317199in}{3.185142in}}%
\pgfpathcurveto{\pgfqpoint{4.317199in}{3.174092in}}{\pgfqpoint{4.321589in}{3.163493in}}{\pgfqpoint{4.329403in}{3.155679in}}%
\pgfpathcurveto{\pgfqpoint{4.337217in}{3.147866in}}{\pgfqpoint{4.347816in}{3.143475in}}{\pgfqpoint{4.358866in}{3.143475in}}%
\pgfpathclose%
\pgfusepath{stroke,fill}%
\end{pgfscope}%
\begin{pgfscope}%
\pgfpathrectangle{\pgfqpoint{0.481978in}{0.331635in}}{\pgfqpoint{9.300000in}{7.700000in}}%
\pgfusepath{clip}%
\pgfsetbuttcap%
\pgfsetroundjoin%
\definecolor{currentfill}{rgb}{1.000000,0.705882,0.509804}%
\pgfsetfillcolor{currentfill}%
\pgfsetlinewidth{0.481800pt}%
\definecolor{currentstroke}{rgb}{1.000000,1.000000,1.000000}%
\pgfsetstrokecolor{currentstroke}%
\pgfsetdash{}{0pt}%
\pgfpathmoveto{\pgfqpoint{4.512704in}{5.547197in}}%
\pgfpathcurveto{\pgfqpoint{4.523754in}{5.547197in}}{\pgfqpoint{4.534353in}{5.551587in}}{\pgfqpoint{4.542167in}{5.559401in}}%
\pgfpathcurveto{\pgfqpoint{4.549981in}{5.567214in}}{\pgfqpoint{4.554371in}{5.577813in}}{\pgfqpoint{4.554371in}{5.588863in}}%
\pgfpathcurveto{\pgfqpoint{4.554371in}{5.599914in}}{\pgfqpoint{4.549981in}{5.610513in}}{\pgfqpoint{4.542167in}{5.618326in}}%
\pgfpathcurveto{\pgfqpoint{4.534353in}{5.626140in}}{\pgfqpoint{4.523754in}{5.630530in}}{\pgfqpoint{4.512704in}{5.630530in}}%
\pgfpathcurveto{\pgfqpoint{4.501654in}{5.630530in}}{\pgfqpoint{4.491055in}{5.626140in}}{\pgfqpoint{4.483242in}{5.618326in}}%
\pgfpathcurveto{\pgfqpoint{4.475428in}{5.610513in}}{\pgfqpoint{4.471038in}{5.599914in}}{\pgfqpoint{4.471038in}{5.588863in}}%
\pgfpathcurveto{\pgfqpoint{4.471038in}{5.577813in}}{\pgfqpoint{4.475428in}{5.567214in}}{\pgfqpoint{4.483242in}{5.559401in}}%
\pgfpathcurveto{\pgfqpoint{4.491055in}{5.551587in}}{\pgfqpoint{4.501654in}{5.547197in}}{\pgfqpoint{4.512704in}{5.547197in}}%
\pgfpathclose%
\pgfusepath{stroke,fill}%
\end{pgfscope}%
\begin{pgfscope}%
\pgfpathrectangle{\pgfqpoint{0.481978in}{0.331635in}}{\pgfqpoint{9.300000in}{7.700000in}}%
\pgfusepath{clip}%
\pgfsetbuttcap%
\pgfsetroundjoin%
\definecolor{currentfill}{rgb}{1.000000,0.705882,0.509804}%
\pgfsetfillcolor{currentfill}%
\pgfsetlinewidth{0.481800pt}%
\definecolor{currentstroke}{rgb}{1.000000,1.000000,1.000000}%
\pgfsetstrokecolor{currentstroke}%
\pgfsetdash{}{0pt}%
\pgfpathmoveto{\pgfqpoint{2.811663in}{4.114795in}}%
\pgfpathcurveto{\pgfqpoint{2.822714in}{4.114795in}}{\pgfqpoint{2.833313in}{4.119185in}}{\pgfqpoint{2.841126in}{4.126999in}}%
\pgfpathcurveto{\pgfqpoint{2.848940in}{4.134813in}}{\pgfqpoint{2.853330in}{4.145412in}}{\pgfqpoint{2.853330in}{4.156462in}}%
\pgfpathcurveto{\pgfqpoint{2.853330in}{4.167512in}}{\pgfqpoint{2.848940in}{4.178111in}}{\pgfqpoint{2.841126in}{4.185925in}}%
\pgfpathcurveto{\pgfqpoint{2.833313in}{4.193738in}}{\pgfqpoint{2.822714in}{4.198128in}}{\pgfqpoint{2.811663in}{4.198128in}}%
\pgfpathcurveto{\pgfqpoint{2.800613in}{4.198128in}}{\pgfqpoint{2.790014in}{4.193738in}}{\pgfqpoint{2.782201in}{4.185925in}}%
\pgfpathcurveto{\pgfqpoint{2.774387in}{4.178111in}}{\pgfqpoint{2.769997in}{4.167512in}}{\pgfqpoint{2.769997in}{4.156462in}}%
\pgfpathcurveto{\pgfqpoint{2.769997in}{4.145412in}}{\pgfqpoint{2.774387in}{4.134813in}}{\pgfqpoint{2.782201in}{4.126999in}}%
\pgfpathcurveto{\pgfqpoint{2.790014in}{4.119185in}}{\pgfqpoint{2.800613in}{4.114795in}}{\pgfqpoint{2.811663in}{4.114795in}}%
\pgfpathclose%
\pgfusepath{stroke,fill}%
\end{pgfscope}%
\begin{pgfscope}%
\pgfpathrectangle{\pgfqpoint{0.481978in}{0.331635in}}{\pgfqpoint{9.300000in}{7.700000in}}%
\pgfusepath{clip}%
\pgfsetbuttcap%
\pgfsetroundjoin%
\definecolor{currentfill}{rgb}{1.000000,0.705882,0.509804}%
\pgfsetfillcolor{currentfill}%
\pgfsetlinewidth{0.481800pt}%
\definecolor{currentstroke}{rgb}{1.000000,1.000000,1.000000}%
\pgfsetstrokecolor{currentstroke}%
\pgfsetdash{}{0pt}%
\pgfpathmoveto{\pgfqpoint{4.849105in}{2.668321in}}%
\pgfpathcurveto{\pgfqpoint{4.860155in}{2.668321in}}{\pgfqpoint{4.870754in}{2.672711in}}{\pgfqpoint{4.878568in}{2.680525in}}%
\pgfpathcurveto{\pgfqpoint{4.886382in}{2.688339in}}{\pgfqpoint{4.890772in}{2.698938in}}{\pgfqpoint{4.890772in}{2.709988in}}%
\pgfpathcurveto{\pgfqpoint{4.890772in}{2.721038in}}{\pgfqpoint{4.886382in}{2.731637in}}{\pgfqpoint{4.878568in}{2.739450in}}%
\pgfpathcurveto{\pgfqpoint{4.870754in}{2.747264in}}{\pgfqpoint{4.860155in}{2.751654in}}{\pgfqpoint{4.849105in}{2.751654in}}%
\pgfpathcurveto{\pgfqpoint{4.838055in}{2.751654in}}{\pgfqpoint{4.827456in}{2.747264in}}{\pgfqpoint{4.819642in}{2.739450in}}%
\pgfpathcurveto{\pgfqpoint{4.811829in}{2.731637in}}{\pgfqpoint{4.807438in}{2.721038in}}{\pgfqpoint{4.807438in}{2.709988in}}%
\pgfpathcurveto{\pgfqpoint{4.807438in}{2.698938in}}{\pgfqpoint{4.811829in}{2.688339in}}{\pgfqpoint{4.819642in}{2.680525in}}%
\pgfpathcurveto{\pgfqpoint{4.827456in}{2.672711in}}{\pgfqpoint{4.838055in}{2.668321in}}{\pgfqpoint{4.849105in}{2.668321in}}%
\pgfpathclose%
\pgfusepath{stroke,fill}%
\end{pgfscope}%
\begin{pgfscope}%
\pgfpathrectangle{\pgfqpoint{0.481978in}{0.331635in}}{\pgfqpoint{9.300000in}{7.700000in}}%
\pgfusepath{clip}%
\pgfsetbuttcap%
\pgfsetroundjoin%
\definecolor{currentfill}{rgb}{1.000000,0.705882,0.509804}%
\pgfsetfillcolor{currentfill}%
\pgfsetlinewidth{0.481800pt}%
\definecolor{currentstroke}{rgb}{1.000000,1.000000,1.000000}%
\pgfsetstrokecolor{currentstroke}%
\pgfsetdash{}{0pt}%
\pgfpathmoveto{\pgfqpoint{3.186199in}{3.136786in}}%
\pgfpathcurveto{\pgfqpoint{3.197249in}{3.136786in}}{\pgfqpoint{3.207848in}{3.141177in}}{\pgfqpoint{3.215662in}{3.148990in}}%
\pgfpathcurveto{\pgfqpoint{3.223475in}{3.156804in}}{\pgfqpoint{3.227866in}{3.167403in}}{\pgfqpoint{3.227866in}{3.178453in}}%
\pgfpathcurveto{\pgfqpoint{3.227866in}{3.189503in}}{\pgfqpoint{3.223475in}{3.200102in}}{\pgfqpoint{3.215662in}{3.207916in}}%
\pgfpathcurveto{\pgfqpoint{3.207848in}{3.215729in}}{\pgfqpoint{3.197249in}{3.220120in}}{\pgfqpoint{3.186199in}{3.220120in}}%
\pgfpathcurveto{\pgfqpoint{3.175149in}{3.220120in}}{\pgfqpoint{3.164550in}{3.215729in}}{\pgfqpoint{3.156736in}{3.207916in}}%
\pgfpathcurveto{\pgfqpoint{3.148922in}{3.200102in}}{\pgfqpoint{3.144532in}{3.189503in}}{\pgfqpoint{3.144532in}{3.178453in}}%
\pgfpathcurveto{\pgfqpoint{3.144532in}{3.167403in}}{\pgfqpoint{3.148922in}{3.156804in}}{\pgfqpoint{3.156736in}{3.148990in}}%
\pgfpathcurveto{\pgfqpoint{3.164550in}{3.141177in}}{\pgfqpoint{3.175149in}{3.136786in}}{\pgfqpoint{3.186199in}{3.136786in}}%
\pgfpathclose%
\pgfusepath{stroke,fill}%
\end{pgfscope}%
\begin{pgfscope}%
\pgfpathrectangle{\pgfqpoint{0.481978in}{0.331635in}}{\pgfqpoint{9.300000in}{7.700000in}}%
\pgfusepath{clip}%
\pgfsetbuttcap%
\pgfsetroundjoin%
\definecolor{currentfill}{rgb}{1.000000,0.705882,0.509804}%
\pgfsetfillcolor{currentfill}%
\pgfsetlinewidth{0.481800pt}%
\definecolor{currentstroke}{rgb}{1.000000,1.000000,1.000000}%
\pgfsetstrokecolor{currentstroke}%
\pgfsetdash{}{0pt}%
\pgfpathmoveto{\pgfqpoint{2.733970in}{2.873017in}}%
\pgfpathcurveto{\pgfqpoint{2.745020in}{2.873017in}}{\pgfqpoint{2.755619in}{2.877407in}}{\pgfqpoint{2.763433in}{2.885221in}}%
\pgfpathcurveto{\pgfqpoint{2.771246in}{2.893034in}}{\pgfqpoint{2.775637in}{2.903633in}}{\pgfqpoint{2.775637in}{2.914683in}}%
\pgfpathcurveto{\pgfqpoint{2.775637in}{2.925734in}}{\pgfqpoint{2.771246in}{2.936333in}}{\pgfqpoint{2.763433in}{2.944146in}}%
\pgfpathcurveto{\pgfqpoint{2.755619in}{2.951960in}}{\pgfqpoint{2.745020in}{2.956350in}}{\pgfqpoint{2.733970in}{2.956350in}}%
\pgfpathcurveto{\pgfqpoint{2.722920in}{2.956350in}}{\pgfqpoint{2.712321in}{2.951960in}}{\pgfqpoint{2.704507in}{2.944146in}}%
\pgfpathcurveto{\pgfqpoint{2.696694in}{2.936333in}}{\pgfqpoint{2.692303in}{2.925734in}}{\pgfqpoint{2.692303in}{2.914683in}}%
\pgfpathcurveto{\pgfqpoint{2.692303in}{2.903633in}}{\pgfqpoint{2.696694in}{2.893034in}}{\pgfqpoint{2.704507in}{2.885221in}}%
\pgfpathcurveto{\pgfqpoint{2.712321in}{2.877407in}}{\pgfqpoint{2.722920in}{2.873017in}}{\pgfqpoint{2.733970in}{2.873017in}}%
\pgfpathclose%
\pgfusepath{stroke,fill}%
\end{pgfscope}%
\begin{pgfscope}%
\pgfpathrectangle{\pgfqpoint{0.481978in}{0.331635in}}{\pgfqpoint{9.300000in}{7.700000in}}%
\pgfusepath{clip}%
\pgfsetbuttcap%
\pgfsetroundjoin%
\definecolor{currentfill}{rgb}{1.000000,0.705882,0.509804}%
\pgfsetfillcolor{currentfill}%
\pgfsetlinewidth{0.481800pt}%
\definecolor{currentstroke}{rgb}{1.000000,1.000000,1.000000}%
\pgfsetstrokecolor{currentstroke}%
\pgfsetdash{}{0pt}%
\pgfpathmoveto{\pgfqpoint{4.054746in}{2.658697in}}%
\pgfpathcurveto{\pgfqpoint{4.065796in}{2.658697in}}{\pgfqpoint{4.076395in}{2.663087in}}{\pgfqpoint{4.084209in}{2.670901in}}%
\pgfpathcurveto{\pgfqpoint{4.092023in}{2.678714in}}{\pgfqpoint{4.096413in}{2.689313in}}{\pgfqpoint{4.096413in}{2.700364in}}%
\pgfpathcurveto{\pgfqpoint{4.096413in}{2.711414in}}{\pgfqpoint{4.092023in}{2.722013in}}{\pgfqpoint{4.084209in}{2.729826in}}%
\pgfpathcurveto{\pgfqpoint{4.076395in}{2.737640in}}{\pgfqpoint{4.065796in}{2.742030in}}{\pgfqpoint{4.054746in}{2.742030in}}%
\pgfpathcurveto{\pgfqpoint{4.043696in}{2.742030in}}{\pgfqpoint{4.033097in}{2.737640in}}{\pgfqpoint{4.025283in}{2.729826in}}%
\pgfpathcurveto{\pgfqpoint{4.017470in}{2.722013in}}{\pgfqpoint{4.013080in}{2.711414in}}{\pgfqpoint{4.013080in}{2.700364in}}%
\pgfpathcurveto{\pgfqpoint{4.013080in}{2.689313in}}{\pgfqpoint{4.017470in}{2.678714in}}{\pgfqpoint{4.025283in}{2.670901in}}%
\pgfpathcurveto{\pgfqpoint{4.033097in}{2.663087in}}{\pgfqpoint{4.043696in}{2.658697in}}{\pgfqpoint{4.054746in}{2.658697in}}%
\pgfpathclose%
\pgfusepath{stroke,fill}%
\end{pgfscope}%
\begin{pgfscope}%
\pgfpathrectangle{\pgfqpoint{0.481978in}{0.331635in}}{\pgfqpoint{9.300000in}{7.700000in}}%
\pgfusepath{clip}%
\pgfsetbuttcap%
\pgfsetroundjoin%
\definecolor{currentfill}{rgb}{1.000000,0.705882,0.509804}%
\pgfsetfillcolor{currentfill}%
\pgfsetlinewidth{0.481800pt}%
\definecolor{currentstroke}{rgb}{1.000000,1.000000,1.000000}%
\pgfsetstrokecolor{currentstroke}%
\pgfsetdash{}{0pt}%
\pgfpathmoveto{\pgfqpoint{3.726283in}{6.087306in}}%
\pgfpathcurveto{\pgfqpoint{3.737334in}{6.087306in}}{\pgfqpoint{3.747933in}{6.091696in}}{\pgfqpoint{3.755746in}{6.099510in}}%
\pgfpathcurveto{\pgfqpoint{3.763560in}{6.107324in}}{\pgfqpoint{3.767950in}{6.117923in}}{\pgfqpoint{3.767950in}{6.128973in}}%
\pgfpathcurveto{\pgfqpoint{3.767950in}{6.140023in}}{\pgfqpoint{3.763560in}{6.150622in}}{\pgfqpoint{3.755746in}{6.158435in}}%
\pgfpathcurveto{\pgfqpoint{3.747933in}{6.166249in}}{\pgfqpoint{3.737334in}{6.170639in}}{\pgfqpoint{3.726283in}{6.170639in}}%
\pgfpathcurveto{\pgfqpoint{3.715233in}{6.170639in}}{\pgfqpoint{3.704634in}{6.166249in}}{\pgfqpoint{3.696821in}{6.158435in}}%
\pgfpathcurveto{\pgfqpoint{3.689007in}{6.150622in}}{\pgfqpoint{3.684617in}{6.140023in}}{\pgfqpoint{3.684617in}{6.128973in}}%
\pgfpathcurveto{\pgfqpoint{3.684617in}{6.117923in}}{\pgfqpoint{3.689007in}{6.107324in}}{\pgfqpoint{3.696821in}{6.099510in}}%
\pgfpathcurveto{\pgfqpoint{3.704634in}{6.091696in}}{\pgfqpoint{3.715233in}{6.087306in}}{\pgfqpoint{3.726283in}{6.087306in}}%
\pgfpathclose%
\pgfusepath{stroke,fill}%
\end{pgfscope}%
\begin{pgfscope}%
\pgfpathrectangle{\pgfqpoint{0.481978in}{0.331635in}}{\pgfqpoint{9.300000in}{7.700000in}}%
\pgfusepath{clip}%
\pgfsetbuttcap%
\pgfsetroundjoin%
\definecolor{currentfill}{rgb}{1.000000,0.705882,0.509804}%
\pgfsetfillcolor{currentfill}%
\pgfsetlinewidth{0.481800pt}%
\definecolor{currentstroke}{rgb}{1.000000,1.000000,1.000000}%
\pgfsetstrokecolor{currentstroke}%
\pgfsetdash{}{0pt}%
\pgfpathmoveto{\pgfqpoint{3.284360in}{6.066090in}}%
\pgfpathcurveto{\pgfqpoint{3.295410in}{6.066090in}}{\pgfqpoint{3.306009in}{6.070480in}}{\pgfqpoint{3.313823in}{6.078294in}}%
\pgfpathcurveto{\pgfqpoint{3.321637in}{6.086108in}}{\pgfqpoint{3.326027in}{6.096707in}}{\pgfqpoint{3.326027in}{6.107757in}}%
\pgfpathcurveto{\pgfqpoint{3.326027in}{6.118807in}}{\pgfqpoint{3.321637in}{6.129406in}}{\pgfqpoint{3.313823in}{6.137220in}}%
\pgfpathcurveto{\pgfqpoint{3.306009in}{6.145033in}}{\pgfqpoint{3.295410in}{6.149423in}}{\pgfqpoint{3.284360in}{6.149423in}}%
\pgfpathcurveto{\pgfqpoint{3.273310in}{6.149423in}}{\pgfqpoint{3.262711in}{6.145033in}}{\pgfqpoint{3.254897in}{6.137220in}}%
\pgfpathcurveto{\pgfqpoint{3.247084in}{6.129406in}}{\pgfqpoint{3.242694in}{6.118807in}}{\pgfqpoint{3.242694in}{6.107757in}}%
\pgfpathcurveto{\pgfqpoint{3.242694in}{6.096707in}}{\pgfqpoint{3.247084in}{6.086108in}}{\pgfqpoint{3.254897in}{6.078294in}}%
\pgfpathcurveto{\pgfqpoint{3.262711in}{6.070480in}}{\pgfqpoint{3.273310in}{6.066090in}}{\pgfqpoint{3.284360in}{6.066090in}}%
\pgfpathclose%
\pgfusepath{stroke,fill}%
\end{pgfscope}%
\begin{pgfscope}%
\pgfpathrectangle{\pgfqpoint{0.481978in}{0.331635in}}{\pgfqpoint{9.300000in}{7.700000in}}%
\pgfusepath{clip}%
\pgfsetbuttcap%
\pgfsetroundjoin%
\definecolor{currentfill}{rgb}{1.000000,0.705882,0.509804}%
\pgfsetfillcolor{currentfill}%
\pgfsetlinewidth{0.481800pt}%
\definecolor{currentstroke}{rgb}{1.000000,1.000000,1.000000}%
\pgfsetstrokecolor{currentstroke}%
\pgfsetdash{}{0pt}%
\pgfpathmoveto{\pgfqpoint{4.144311in}{2.733847in}}%
\pgfpathcurveto{\pgfqpoint{4.155361in}{2.733847in}}{\pgfqpoint{4.165960in}{2.738237in}}{\pgfqpoint{4.173773in}{2.746051in}}%
\pgfpathcurveto{\pgfqpoint{4.181587in}{2.753865in}}{\pgfqpoint{4.185977in}{2.764464in}}{\pgfqpoint{4.185977in}{2.775514in}}%
\pgfpathcurveto{\pgfqpoint{4.185977in}{2.786564in}}{\pgfqpoint{4.181587in}{2.797163in}}{\pgfqpoint{4.173773in}{2.804977in}}%
\pgfpathcurveto{\pgfqpoint{4.165960in}{2.812790in}}{\pgfqpoint{4.155361in}{2.817181in}}{\pgfqpoint{4.144311in}{2.817181in}}%
\pgfpathcurveto{\pgfqpoint{4.133261in}{2.817181in}}{\pgfqpoint{4.122661in}{2.812790in}}{\pgfqpoint{4.114848in}{2.804977in}}%
\pgfpathcurveto{\pgfqpoint{4.107034in}{2.797163in}}{\pgfqpoint{4.102644in}{2.786564in}}{\pgfqpoint{4.102644in}{2.775514in}}%
\pgfpathcurveto{\pgfqpoint{4.102644in}{2.764464in}}{\pgfqpoint{4.107034in}{2.753865in}}{\pgfqpoint{4.114848in}{2.746051in}}%
\pgfpathcurveto{\pgfqpoint{4.122661in}{2.738237in}}{\pgfqpoint{4.133261in}{2.733847in}}{\pgfqpoint{4.144311in}{2.733847in}}%
\pgfpathclose%
\pgfusepath{stroke,fill}%
\end{pgfscope}%
\begin{pgfscope}%
\pgfpathrectangle{\pgfqpoint{0.481978in}{0.331635in}}{\pgfqpoint{9.300000in}{7.700000in}}%
\pgfusepath{clip}%
\pgfsetbuttcap%
\pgfsetroundjoin%
\definecolor{currentfill}{rgb}{1.000000,0.705882,0.509804}%
\pgfsetfillcolor{currentfill}%
\pgfsetlinewidth{0.481800pt}%
\definecolor{currentstroke}{rgb}{1.000000,1.000000,1.000000}%
\pgfsetstrokecolor{currentstroke}%
\pgfsetdash{}{0pt}%
\pgfpathmoveto{\pgfqpoint{2.660738in}{2.575194in}}%
\pgfpathcurveto{\pgfqpoint{2.671788in}{2.575194in}}{\pgfqpoint{2.682387in}{2.579585in}}{\pgfqpoint{2.690201in}{2.587398in}}%
\pgfpathcurveto{\pgfqpoint{2.698014in}{2.595212in}}{\pgfqpoint{2.702404in}{2.605811in}}{\pgfqpoint{2.702404in}{2.616861in}}%
\pgfpathcurveto{\pgfqpoint{2.702404in}{2.627911in}}{\pgfqpoint{2.698014in}{2.638510in}}{\pgfqpoint{2.690201in}{2.646324in}}%
\pgfpathcurveto{\pgfqpoint{2.682387in}{2.654137in}}{\pgfqpoint{2.671788in}{2.658528in}}{\pgfqpoint{2.660738in}{2.658528in}}%
\pgfpathcurveto{\pgfqpoint{2.649688in}{2.658528in}}{\pgfqpoint{2.639089in}{2.654137in}}{\pgfqpoint{2.631275in}{2.646324in}}%
\pgfpathcurveto{\pgfqpoint{2.623461in}{2.638510in}}{\pgfqpoint{2.619071in}{2.627911in}}{\pgfqpoint{2.619071in}{2.616861in}}%
\pgfpathcurveto{\pgfqpoint{2.619071in}{2.605811in}}{\pgfqpoint{2.623461in}{2.595212in}}{\pgfqpoint{2.631275in}{2.587398in}}%
\pgfpathcurveto{\pgfqpoint{2.639089in}{2.579585in}}{\pgfqpoint{2.649688in}{2.575194in}}{\pgfqpoint{2.660738in}{2.575194in}}%
\pgfpathclose%
\pgfusepath{stroke,fill}%
\end{pgfscope}%
\begin{pgfscope}%
\pgfpathrectangle{\pgfqpoint{0.481978in}{0.331635in}}{\pgfqpoint{9.300000in}{7.700000in}}%
\pgfusepath{clip}%
\pgfsetbuttcap%
\pgfsetroundjoin%
\definecolor{currentfill}{rgb}{1.000000,0.705882,0.509804}%
\pgfsetfillcolor{currentfill}%
\pgfsetlinewidth{0.481800pt}%
\definecolor{currentstroke}{rgb}{1.000000,1.000000,1.000000}%
\pgfsetstrokecolor{currentstroke}%
\pgfsetdash{}{0pt}%
\pgfpathmoveto{\pgfqpoint{3.685446in}{2.354103in}}%
\pgfpathcurveto{\pgfqpoint{3.696496in}{2.354103in}}{\pgfqpoint{3.707095in}{2.358493in}}{\pgfqpoint{3.714908in}{2.366307in}}%
\pgfpathcurveto{\pgfqpoint{3.722722in}{2.374121in}}{\pgfqpoint{3.727112in}{2.384720in}}{\pgfqpoint{3.727112in}{2.395770in}}%
\pgfpathcurveto{\pgfqpoint{3.727112in}{2.406820in}}{\pgfqpoint{3.722722in}{2.417419in}}{\pgfqpoint{3.714908in}{2.425233in}}%
\pgfpathcurveto{\pgfqpoint{3.707095in}{2.433046in}}{\pgfqpoint{3.696496in}{2.437436in}}{\pgfqpoint{3.685446in}{2.437436in}}%
\pgfpathcurveto{\pgfqpoint{3.674395in}{2.437436in}}{\pgfqpoint{3.663796in}{2.433046in}}{\pgfqpoint{3.655983in}{2.425233in}}%
\pgfpathcurveto{\pgfqpoint{3.648169in}{2.417419in}}{\pgfqpoint{3.643779in}{2.406820in}}{\pgfqpoint{3.643779in}{2.395770in}}%
\pgfpathcurveto{\pgfqpoint{3.643779in}{2.384720in}}{\pgfqpoint{3.648169in}{2.374121in}}{\pgfqpoint{3.655983in}{2.366307in}}%
\pgfpathcurveto{\pgfqpoint{3.663796in}{2.358493in}}{\pgfqpoint{3.674395in}{2.354103in}}{\pgfqpoint{3.685446in}{2.354103in}}%
\pgfpathclose%
\pgfusepath{stroke,fill}%
\end{pgfscope}%
\begin{pgfscope}%
\pgfpathrectangle{\pgfqpoint{0.481978in}{0.331635in}}{\pgfqpoint{9.300000in}{7.700000in}}%
\pgfusepath{clip}%
\pgfsetbuttcap%
\pgfsetroundjoin%
\definecolor{currentfill}{rgb}{1.000000,0.705882,0.509804}%
\pgfsetfillcolor{currentfill}%
\pgfsetlinewidth{0.481800pt}%
\definecolor{currentstroke}{rgb}{1.000000,1.000000,1.000000}%
\pgfsetstrokecolor{currentstroke}%
\pgfsetdash{}{0pt}%
\pgfpathmoveto{\pgfqpoint{3.030052in}{5.719688in}}%
\pgfpathcurveto{\pgfqpoint{3.041102in}{5.719688in}}{\pgfqpoint{3.051701in}{5.724078in}}{\pgfqpoint{3.059514in}{5.731892in}}%
\pgfpathcurveto{\pgfqpoint{3.067328in}{5.739706in}}{\pgfqpoint{3.071718in}{5.750305in}}{\pgfqpoint{3.071718in}{5.761355in}}%
\pgfpathcurveto{\pgfqpoint{3.071718in}{5.772405in}}{\pgfqpoint{3.067328in}{5.783004in}}{\pgfqpoint{3.059514in}{5.790818in}}%
\pgfpathcurveto{\pgfqpoint{3.051701in}{5.798631in}}{\pgfqpoint{3.041102in}{5.803022in}}{\pgfqpoint{3.030052in}{5.803022in}}%
\pgfpathcurveto{\pgfqpoint{3.019001in}{5.803022in}}{\pgfqpoint{3.008402in}{5.798631in}}{\pgfqpoint{3.000589in}{5.790818in}}%
\pgfpathcurveto{\pgfqpoint{2.992775in}{5.783004in}}{\pgfqpoint{2.988385in}{5.772405in}}{\pgfqpoint{2.988385in}{5.761355in}}%
\pgfpathcurveto{\pgfqpoint{2.988385in}{5.750305in}}{\pgfqpoint{2.992775in}{5.739706in}}{\pgfqpoint{3.000589in}{5.731892in}}%
\pgfpathcurveto{\pgfqpoint{3.008402in}{5.724078in}}{\pgfqpoint{3.019001in}{5.719688in}}{\pgfqpoint{3.030052in}{5.719688in}}%
\pgfpathclose%
\pgfusepath{stroke,fill}%
\end{pgfscope}%
\begin{pgfscope}%
\pgfpathrectangle{\pgfqpoint{0.481978in}{0.331635in}}{\pgfqpoint{9.300000in}{7.700000in}}%
\pgfusepath{clip}%
\pgfsetbuttcap%
\pgfsetroundjoin%
\definecolor{currentfill}{rgb}{1.000000,0.705882,0.509804}%
\pgfsetfillcolor{currentfill}%
\pgfsetlinewidth{0.481800pt}%
\definecolor{currentstroke}{rgb}{1.000000,1.000000,1.000000}%
\pgfsetstrokecolor{currentstroke}%
\pgfsetdash{}{0pt}%
\pgfpathmoveto{\pgfqpoint{1.530784in}{2.492756in}}%
\pgfpathcurveto{\pgfqpoint{1.541834in}{2.492756in}}{\pgfqpoint{1.552434in}{2.497146in}}{\pgfqpoint{1.560247in}{2.504960in}}%
\pgfpathcurveto{\pgfqpoint{1.568061in}{2.512773in}}{\pgfqpoint{1.572451in}{2.523373in}}{\pgfqpoint{1.572451in}{2.534423in}}%
\pgfpathcurveto{\pgfqpoint{1.572451in}{2.545473in}}{\pgfqpoint{1.568061in}{2.556072in}}{\pgfqpoint{1.560247in}{2.563885in}}%
\pgfpathcurveto{\pgfqpoint{1.552434in}{2.571699in}}{\pgfqpoint{1.541834in}{2.576089in}}{\pgfqpoint{1.530784in}{2.576089in}}%
\pgfpathcurveto{\pgfqpoint{1.519734in}{2.576089in}}{\pgfqpoint{1.509135in}{2.571699in}}{\pgfqpoint{1.501322in}{2.563885in}}%
\pgfpathcurveto{\pgfqpoint{1.493508in}{2.556072in}}{\pgfqpoint{1.489118in}{2.545473in}}{\pgfqpoint{1.489118in}{2.534423in}}%
\pgfpathcurveto{\pgfqpoint{1.489118in}{2.523373in}}{\pgfqpoint{1.493508in}{2.512773in}}{\pgfqpoint{1.501322in}{2.504960in}}%
\pgfpathcurveto{\pgfqpoint{1.509135in}{2.497146in}}{\pgfqpoint{1.519734in}{2.492756in}}{\pgfqpoint{1.530784in}{2.492756in}}%
\pgfpathclose%
\pgfusepath{stroke,fill}%
\end{pgfscope}%
\begin{pgfscope}%
\pgfpathrectangle{\pgfqpoint{0.481978in}{0.331635in}}{\pgfqpoint{9.300000in}{7.700000in}}%
\pgfusepath{clip}%
\pgfsetbuttcap%
\pgfsetroundjoin%
\definecolor{currentfill}{rgb}{1.000000,0.705882,0.509804}%
\pgfsetfillcolor{currentfill}%
\pgfsetlinewidth{0.481800pt}%
\definecolor{currentstroke}{rgb}{1.000000,1.000000,1.000000}%
\pgfsetstrokecolor{currentstroke}%
\pgfsetdash{}{0pt}%
\pgfpathmoveto{\pgfqpoint{1.565141in}{5.059110in}}%
\pgfpathcurveto{\pgfqpoint{1.576191in}{5.059110in}}{\pgfqpoint{1.586790in}{5.063500in}}{\pgfqpoint{1.594604in}{5.071313in}}%
\pgfpathcurveto{\pgfqpoint{1.602418in}{5.079127in}}{\pgfqpoint{1.606808in}{5.089726in}}{\pgfqpoint{1.606808in}{5.100776in}}%
\pgfpathcurveto{\pgfqpoint{1.606808in}{5.111826in}}{\pgfqpoint{1.602418in}{5.122425in}}{\pgfqpoint{1.594604in}{5.130239in}}%
\pgfpathcurveto{\pgfqpoint{1.586790in}{5.138053in}}{\pgfqpoint{1.576191in}{5.142443in}}{\pgfqpoint{1.565141in}{5.142443in}}%
\pgfpathcurveto{\pgfqpoint{1.554091in}{5.142443in}}{\pgfqpoint{1.543492in}{5.138053in}}{\pgfqpoint{1.535678in}{5.130239in}}%
\pgfpathcurveto{\pgfqpoint{1.527865in}{5.122425in}}{\pgfqpoint{1.523475in}{5.111826in}}{\pgfqpoint{1.523475in}{5.100776in}}%
\pgfpathcurveto{\pgfqpoint{1.523475in}{5.089726in}}{\pgfqpoint{1.527865in}{5.079127in}}{\pgfqpoint{1.535678in}{5.071313in}}%
\pgfpathcurveto{\pgfqpoint{1.543492in}{5.063500in}}{\pgfqpoint{1.554091in}{5.059110in}}{\pgfqpoint{1.565141in}{5.059110in}}%
\pgfpathclose%
\pgfusepath{stroke,fill}%
\end{pgfscope}%
\begin{pgfscope}%
\pgfpathrectangle{\pgfqpoint{0.481978in}{0.331635in}}{\pgfqpoint{9.300000in}{7.700000in}}%
\pgfusepath{clip}%
\pgfsetbuttcap%
\pgfsetroundjoin%
\definecolor{currentfill}{rgb}{1.000000,0.705882,0.509804}%
\pgfsetfillcolor{currentfill}%
\pgfsetlinewidth{0.481800pt}%
\definecolor{currentstroke}{rgb}{1.000000,1.000000,1.000000}%
\pgfsetstrokecolor{currentstroke}%
\pgfsetdash{}{0pt}%
\pgfpathmoveto{\pgfqpoint{4.138549in}{4.903543in}}%
\pgfpathcurveto{\pgfqpoint{4.149599in}{4.903543in}}{\pgfqpoint{4.160198in}{4.907934in}}{\pgfqpoint{4.168011in}{4.915747in}}%
\pgfpathcurveto{\pgfqpoint{4.175825in}{4.923561in}}{\pgfqpoint{4.180215in}{4.934160in}}{\pgfqpoint{4.180215in}{4.945210in}}%
\pgfpathcurveto{\pgfqpoint{4.180215in}{4.956260in}}{\pgfqpoint{4.175825in}{4.966859in}}{\pgfqpoint{4.168011in}{4.974673in}}%
\pgfpathcurveto{\pgfqpoint{4.160198in}{4.982487in}}{\pgfqpoint{4.149599in}{4.986877in}}{\pgfqpoint{4.138549in}{4.986877in}}%
\pgfpathcurveto{\pgfqpoint{4.127498in}{4.986877in}}{\pgfqpoint{4.116899in}{4.982487in}}{\pgfqpoint{4.109086in}{4.974673in}}%
\pgfpathcurveto{\pgfqpoint{4.101272in}{4.966859in}}{\pgfqpoint{4.096882in}{4.956260in}}{\pgfqpoint{4.096882in}{4.945210in}}%
\pgfpathcurveto{\pgfqpoint{4.096882in}{4.934160in}}{\pgfqpoint{4.101272in}{4.923561in}}{\pgfqpoint{4.109086in}{4.915747in}}%
\pgfpathcurveto{\pgfqpoint{4.116899in}{4.907934in}}{\pgfqpoint{4.127498in}{4.903543in}}{\pgfqpoint{4.138549in}{4.903543in}}%
\pgfpathclose%
\pgfusepath{stroke,fill}%
\end{pgfscope}%
\begin{pgfscope}%
\pgfpathrectangle{\pgfqpoint{0.481978in}{0.331635in}}{\pgfqpoint{9.300000in}{7.700000in}}%
\pgfusepath{clip}%
\pgfsetbuttcap%
\pgfsetroundjoin%
\definecolor{currentfill}{rgb}{1.000000,0.705882,0.509804}%
\pgfsetfillcolor{currentfill}%
\pgfsetlinewidth{0.481800pt}%
\definecolor{currentstroke}{rgb}{1.000000,1.000000,1.000000}%
\pgfsetstrokecolor{currentstroke}%
\pgfsetdash{}{0pt}%
\pgfpathmoveto{\pgfqpoint{4.409605in}{4.724804in}}%
\pgfpathcurveto{\pgfqpoint{4.420655in}{4.724804in}}{\pgfqpoint{4.431254in}{4.729194in}}{\pgfqpoint{4.439068in}{4.737007in}}%
\pgfpathcurveto{\pgfqpoint{4.446882in}{4.744821in}}{\pgfqpoint{4.451272in}{4.755420in}}{\pgfqpoint{4.451272in}{4.766470in}}%
\pgfpathcurveto{\pgfqpoint{4.451272in}{4.777520in}}{\pgfqpoint{4.446882in}{4.788119in}}{\pgfqpoint{4.439068in}{4.795933in}}%
\pgfpathcurveto{\pgfqpoint{4.431254in}{4.803747in}}{\pgfqpoint{4.420655in}{4.808137in}}{\pgfqpoint{4.409605in}{4.808137in}}%
\pgfpathcurveto{\pgfqpoint{4.398555in}{4.808137in}}{\pgfqpoint{4.387956in}{4.803747in}}{\pgfqpoint{4.380142in}{4.795933in}}%
\pgfpathcurveto{\pgfqpoint{4.372329in}{4.788119in}}{\pgfqpoint{4.367939in}{4.777520in}}{\pgfqpoint{4.367939in}{4.766470in}}%
\pgfpathcurveto{\pgfqpoint{4.367939in}{4.755420in}}{\pgfqpoint{4.372329in}{4.744821in}}{\pgfqpoint{4.380142in}{4.737007in}}%
\pgfpathcurveto{\pgfqpoint{4.387956in}{4.729194in}}{\pgfqpoint{4.398555in}{4.724804in}}{\pgfqpoint{4.409605in}{4.724804in}}%
\pgfpathclose%
\pgfusepath{stroke,fill}%
\end{pgfscope}%
\begin{pgfscope}%
\pgfpathrectangle{\pgfqpoint{0.481978in}{0.331635in}}{\pgfqpoint{9.300000in}{7.700000in}}%
\pgfusepath{clip}%
\pgfsetbuttcap%
\pgfsetroundjoin%
\definecolor{currentfill}{rgb}{1.000000,0.705882,0.509804}%
\pgfsetfillcolor{currentfill}%
\pgfsetlinewidth{0.481800pt}%
\definecolor{currentstroke}{rgb}{1.000000,1.000000,1.000000}%
\pgfsetstrokecolor{currentstroke}%
\pgfsetdash{}{0pt}%
\pgfpathmoveto{\pgfqpoint{4.134948in}{3.124159in}}%
\pgfpathcurveto{\pgfqpoint{4.145998in}{3.124159in}}{\pgfqpoint{4.156597in}{3.128549in}}{\pgfqpoint{4.164410in}{3.136363in}}%
\pgfpathcurveto{\pgfqpoint{4.172224in}{3.144177in}}{\pgfqpoint{4.176614in}{3.154776in}}{\pgfqpoint{4.176614in}{3.165826in}}%
\pgfpathcurveto{\pgfqpoint{4.176614in}{3.176876in}}{\pgfqpoint{4.172224in}{3.187475in}}{\pgfqpoint{4.164410in}{3.195289in}}%
\pgfpathcurveto{\pgfqpoint{4.156597in}{3.203102in}}{\pgfqpoint{4.145998in}{3.207493in}}{\pgfqpoint{4.134948in}{3.207493in}}%
\pgfpathcurveto{\pgfqpoint{4.123898in}{3.207493in}}{\pgfqpoint{4.113298in}{3.203102in}}{\pgfqpoint{4.105485in}{3.195289in}}%
\pgfpathcurveto{\pgfqpoint{4.097671in}{3.187475in}}{\pgfqpoint{4.093281in}{3.176876in}}{\pgfqpoint{4.093281in}{3.165826in}}%
\pgfpathcurveto{\pgfqpoint{4.093281in}{3.154776in}}{\pgfqpoint{4.097671in}{3.144177in}}{\pgfqpoint{4.105485in}{3.136363in}}%
\pgfpathcurveto{\pgfqpoint{4.113298in}{3.128549in}}{\pgfqpoint{4.123898in}{3.124159in}}{\pgfqpoint{4.134948in}{3.124159in}}%
\pgfpathclose%
\pgfusepath{stroke,fill}%
\end{pgfscope}%
\begin{pgfscope}%
\pgfpathrectangle{\pgfqpoint{0.481978in}{0.331635in}}{\pgfqpoint{9.300000in}{7.700000in}}%
\pgfusepath{clip}%
\pgfsetbuttcap%
\pgfsetroundjoin%
\definecolor{currentfill}{rgb}{1.000000,0.705882,0.509804}%
\pgfsetfillcolor{currentfill}%
\pgfsetlinewidth{0.481800pt}%
\definecolor{currentstroke}{rgb}{1.000000,1.000000,1.000000}%
\pgfsetstrokecolor{currentstroke}%
\pgfsetdash{}{0pt}%
\pgfpathmoveto{\pgfqpoint{2.632233in}{4.333030in}}%
\pgfpathcurveto{\pgfqpoint{2.643283in}{4.333030in}}{\pgfqpoint{2.653882in}{4.337420in}}{\pgfqpoint{2.661695in}{4.345234in}}%
\pgfpathcurveto{\pgfqpoint{2.669509in}{4.353047in}}{\pgfqpoint{2.673899in}{4.363646in}}{\pgfqpoint{2.673899in}{4.374697in}}%
\pgfpathcurveto{\pgfqpoint{2.673899in}{4.385747in}}{\pgfqpoint{2.669509in}{4.396346in}}{\pgfqpoint{2.661695in}{4.404159in}}%
\pgfpathcurveto{\pgfqpoint{2.653882in}{4.411973in}}{\pgfqpoint{2.643283in}{4.416363in}}{\pgfqpoint{2.632233in}{4.416363in}}%
\pgfpathcurveto{\pgfqpoint{2.621183in}{4.416363in}}{\pgfqpoint{2.610584in}{4.411973in}}{\pgfqpoint{2.602770in}{4.404159in}}%
\pgfpathcurveto{\pgfqpoint{2.594956in}{4.396346in}}{\pgfqpoint{2.590566in}{4.385747in}}{\pgfqpoint{2.590566in}{4.374697in}}%
\pgfpathcurveto{\pgfqpoint{2.590566in}{4.363646in}}{\pgfqpoint{2.594956in}{4.353047in}}{\pgfqpoint{2.602770in}{4.345234in}}%
\pgfpathcurveto{\pgfqpoint{2.610584in}{4.337420in}}{\pgfqpoint{2.621183in}{4.333030in}}{\pgfqpoint{2.632233in}{4.333030in}}%
\pgfpathclose%
\pgfusepath{stroke,fill}%
\end{pgfscope}%
\begin{pgfscope}%
\pgfpathrectangle{\pgfqpoint{0.481978in}{0.331635in}}{\pgfqpoint{9.300000in}{7.700000in}}%
\pgfusepath{clip}%
\pgfsetbuttcap%
\pgfsetroundjoin%
\definecolor{currentfill}{rgb}{1.000000,0.705882,0.509804}%
\pgfsetfillcolor{currentfill}%
\pgfsetlinewidth{0.481800pt}%
\definecolor{currentstroke}{rgb}{1.000000,1.000000,1.000000}%
\pgfsetstrokecolor{currentstroke}%
\pgfsetdash{}{0pt}%
\pgfpathmoveto{\pgfqpoint{3.332817in}{4.260900in}}%
\pgfpathcurveto{\pgfqpoint{3.343867in}{4.260900in}}{\pgfqpoint{3.354466in}{4.265290in}}{\pgfqpoint{3.362280in}{4.273104in}}%
\pgfpathcurveto{\pgfqpoint{3.370094in}{4.280917in}}{\pgfqpoint{3.374484in}{4.291516in}}{\pgfqpoint{3.374484in}{4.302567in}}%
\pgfpathcurveto{\pgfqpoint{3.374484in}{4.313617in}}{\pgfqpoint{3.370094in}{4.324216in}}{\pgfqpoint{3.362280in}{4.332029in}}%
\pgfpathcurveto{\pgfqpoint{3.354466in}{4.339843in}}{\pgfqpoint{3.343867in}{4.344233in}}{\pgfqpoint{3.332817in}{4.344233in}}%
\pgfpathcurveto{\pgfqpoint{3.321767in}{4.344233in}}{\pgfqpoint{3.311168in}{4.339843in}}{\pgfqpoint{3.303354in}{4.332029in}}%
\pgfpathcurveto{\pgfqpoint{3.295541in}{4.324216in}}{\pgfqpoint{3.291150in}{4.313617in}}{\pgfqpoint{3.291150in}{4.302567in}}%
\pgfpathcurveto{\pgfqpoint{3.291150in}{4.291516in}}{\pgfqpoint{3.295541in}{4.280917in}}{\pgfqpoint{3.303354in}{4.273104in}}%
\pgfpathcurveto{\pgfqpoint{3.311168in}{4.265290in}}{\pgfqpoint{3.321767in}{4.260900in}}{\pgfqpoint{3.332817in}{4.260900in}}%
\pgfpathclose%
\pgfusepath{stroke,fill}%
\end{pgfscope}%
\begin{pgfscope}%
\pgfpathrectangle{\pgfqpoint{0.481978in}{0.331635in}}{\pgfqpoint{9.300000in}{7.700000in}}%
\pgfusepath{clip}%
\pgfsetbuttcap%
\pgfsetroundjoin%
\definecolor{currentfill}{rgb}{1.000000,0.705882,0.509804}%
\pgfsetfillcolor{currentfill}%
\pgfsetlinewidth{0.481800pt}%
\definecolor{currentstroke}{rgb}{1.000000,1.000000,1.000000}%
\pgfsetstrokecolor{currentstroke}%
\pgfsetdash{}{0pt}%
\pgfpathmoveto{\pgfqpoint{3.045518in}{4.242593in}}%
\pgfpathcurveto{\pgfqpoint{3.056568in}{4.242593in}}{\pgfqpoint{3.067167in}{4.246983in}}{\pgfqpoint{3.074980in}{4.254797in}}%
\pgfpathcurveto{\pgfqpoint{3.082794in}{4.262610in}}{\pgfqpoint{3.087184in}{4.273209in}}{\pgfqpoint{3.087184in}{4.284260in}}%
\pgfpathcurveto{\pgfqpoint{3.087184in}{4.295310in}}{\pgfqpoint{3.082794in}{4.305909in}}{\pgfqpoint{3.074980in}{4.313722in}}%
\pgfpathcurveto{\pgfqpoint{3.067167in}{4.321536in}}{\pgfqpoint{3.056568in}{4.325926in}}{\pgfqpoint{3.045518in}{4.325926in}}%
\pgfpathcurveto{\pgfqpoint{3.034468in}{4.325926in}}{\pgfqpoint{3.023869in}{4.321536in}}{\pgfqpoint{3.016055in}{4.313722in}}%
\pgfpathcurveto{\pgfqpoint{3.008241in}{4.305909in}}{\pgfqpoint{3.003851in}{4.295310in}}{\pgfqpoint{3.003851in}{4.284260in}}%
\pgfpathcurveto{\pgfqpoint{3.003851in}{4.273209in}}{\pgfqpoint{3.008241in}{4.262610in}}{\pgfqpoint{3.016055in}{4.254797in}}%
\pgfpathcurveto{\pgfqpoint{3.023869in}{4.246983in}}{\pgfqpoint{3.034468in}{4.242593in}}{\pgfqpoint{3.045518in}{4.242593in}}%
\pgfpathclose%
\pgfusepath{stroke,fill}%
\end{pgfscope}%
\begin{pgfscope}%
\pgfpathrectangle{\pgfqpoint{0.481978in}{0.331635in}}{\pgfqpoint{9.300000in}{7.700000in}}%
\pgfusepath{clip}%
\pgfsetbuttcap%
\pgfsetroundjoin%
\definecolor{currentfill}{rgb}{1.000000,0.705882,0.509804}%
\pgfsetfillcolor{currentfill}%
\pgfsetlinewidth{0.481800pt}%
\definecolor{currentstroke}{rgb}{1.000000,1.000000,1.000000}%
\pgfsetstrokecolor{currentstroke}%
\pgfsetdash{}{0pt}%
\pgfpathmoveto{\pgfqpoint{2.948753in}{3.408484in}}%
\pgfpathcurveto{\pgfqpoint{2.959803in}{3.408484in}}{\pgfqpoint{2.970402in}{3.412875in}}{\pgfqpoint{2.978216in}{3.420688in}}%
\pgfpathcurveto{\pgfqpoint{2.986030in}{3.428502in}}{\pgfqpoint{2.990420in}{3.439101in}}{\pgfqpoint{2.990420in}{3.450151in}}%
\pgfpathcurveto{\pgfqpoint{2.990420in}{3.461201in}}{\pgfqpoint{2.986030in}{3.471800in}}{\pgfqpoint{2.978216in}{3.479614in}}%
\pgfpathcurveto{\pgfqpoint{2.970402in}{3.487427in}}{\pgfqpoint{2.959803in}{3.491818in}}{\pgfqpoint{2.948753in}{3.491818in}}%
\pgfpathcurveto{\pgfqpoint{2.937703in}{3.491818in}}{\pgfqpoint{2.927104in}{3.487427in}}{\pgfqpoint{2.919290in}{3.479614in}}%
\pgfpathcurveto{\pgfqpoint{2.911477in}{3.471800in}}{\pgfqpoint{2.907087in}{3.461201in}}{\pgfqpoint{2.907087in}{3.450151in}}%
\pgfpathcurveto{\pgfqpoint{2.907087in}{3.439101in}}{\pgfqpoint{2.911477in}{3.428502in}}{\pgfqpoint{2.919290in}{3.420688in}}%
\pgfpathcurveto{\pgfqpoint{2.927104in}{3.412875in}}{\pgfqpoint{2.937703in}{3.408484in}}{\pgfqpoint{2.948753in}{3.408484in}}%
\pgfpathclose%
\pgfusepath{stroke,fill}%
\end{pgfscope}%
\begin{pgfscope}%
\pgfpathrectangle{\pgfqpoint{0.481978in}{0.331635in}}{\pgfqpoint{9.300000in}{7.700000in}}%
\pgfusepath{clip}%
\pgfsetbuttcap%
\pgfsetroundjoin%
\definecolor{currentfill}{rgb}{1.000000,0.705882,0.509804}%
\pgfsetfillcolor{currentfill}%
\pgfsetlinewidth{0.481800pt}%
\definecolor{currentstroke}{rgb}{1.000000,1.000000,1.000000}%
\pgfsetstrokecolor{currentstroke}%
\pgfsetdash{}{0pt}%
\pgfpathmoveto{\pgfqpoint{3.979638in}{3.574715in}}%
\pgfpathcurveto{\pgfqpoint{3.990688in}{3.574715in}}{\pgfqpoint{4.001287in}{3.579106in}}{\pgfqpoint{4.009101in}{3.586919in}}%
\pgfpathcurveto{\pgfqpoint{4.016914in}{3.594733in}}{\pgfqpoint{4.021304in}{3.605332in}}{\pgfqpoint{4.021304in}{3.616382in}}%
\pgfpathcurveto{\pgfqpoint{4.021304in}{3.627432in}}{\pgfqpoint{4.016914in}{3.638031in}}{\pgfqpoint{4.009101in}{3.645845in}}%
\pgfpathcurveto{\pgfqpoint{4.001287in}{3.653659in}}{\pgfqpoint{3.990688in}{3.658049in}}{\pgfqpoint{3.979638in}{3.658049in}}%
\pgfpathcurveto{\pgfqpoint{3.968588in}{3.658049in}}{\pgfqpoint{3.957989in}{3.653659in}}{\pgfqpoint{3.950175in}{3.645845in}}%
\pgfpathcurveto{\pgfqpoint{3.942361in}{3.638031in}}{\pgfqpoint{3.937971in}{3.627432in}}{\pgfqpoint{3.937971in}{3.616382in}}%
\pgfpathcurveto{\pgfqpoint{3.937971in}{3.605332in}}{\pgfqpoint{3.942361in}{3.594733in}}{\pgfqpoint{3.950175in}{3.586919in}}%
\pgfpathcurveto{\pgfqpoint{3.957989in}{3.579106in}}{\pgfqpoint{3.968588in}{3.574715in}}{\pgfqpoint{3.979638in}{3.574715in}}%
\pgfpathclose%
\pgfusepath{stroke,fill}%
\end{pgfscope}%
\begin{pgfscope}%
\pgfpathrectangle{\pgfqpoint{0.481978in}{0.331635in}}{\pgfqpoint{9.300000in}{7.700000in}}%
\pgfusepath{clip}%
\pgfsetbuttcap%
\pgfsetroundjoin%
\definecolor{currentfill}{rgb}{1.000000,0.705882,0.509804}%
\pgfsetfillcolor{currentfill}%
\pgfsetlinewidth{0.481800pt}%
\definecolor{currentstroke}{rgb}{1.000000,1.000000,1.000000}%
\pgfsetstrokecolor{currentstroke}%
\pgfsetdash{}{0pt}%
\pgfpathmoveto{\pgfqpoint{4.333852in}{2.410081in}}%
\pgfpathcurveto{\pgfqpoint{4.344902in}{2.410081in}}{\pgfqpoint{4.355501in}{2.414471in}}{\pgfqpoint{4.363315in}{2.422284in}}%
\pgfpathcurveto{\pgfqpoint{4.371129in}{2.430098in}}{\pgfqpoint{4.375519in}{2.440697in}}{\pgfqpoint{4.375519in}{2.451747in}}%
\pgfpathcurveto{\pgfqpoint{4.375519in}{2.462797in}}{\pgfqpoint{4.371129in}{2.473396in}}{\pgfqpoint{4.363315in}{2.481210in}}%
\pgfpathcurveto{\pgfqpoint{4.355501in}{2.489024in}}{\pgfqpoint{4.344902in}{2.493414in}}{\pgfqpoint{4.333852in}{2.493414in}}%
\pgfpathcurveto{\pgfqpoint{4.322802in}{2.493414in}}{\pgfqpoint{4.312203in}{2.489024in}}{\pgfqpoint{4.304389in}{2.481210in}}%
\pgfpathcurveto{\pgfqpoint{4.296576in}{2.473396in}}{\pgfqpoint{4.292186in}{2.462797in}}{\pgfqpoint{4.292186in}{2.451747in}}%
\pgfpathcurveto{\pgfqpoint{4.292186in}{2.440697in}}{\pgfqpoint{4.296576in}{2.430098in}}{\pgfqpoint{4.304389in}{2.422284in}}%
\pgfpathcurveto{\pgfqpoint{4.312203in}{2.414471in}}{\pgfqpoint{4.322802in}{2.410081in}}{\pgfqpoint{4.333852in}{2.410081in}}%
\pgfpathclose%
\pgfusepath{stroke,fill}%
\end{pgfscope}%
\begin{pgfscope}%
\pgfpathrectangle{\pgfqpoint{0.481978in}{0.331635in}}{\pgfqpoint{9.300000in}{7.700000in}}%
\pgfusepath{clip}%
\pgfsetbuttcap%
\pgfsetroundjoin%
\definecolor{currentfill}{rgb}{1.000000,0.705882,0.509804}%
\pgfsetfillcolor{currentfill}%
\pgfsetlinewidth{0.481800pt}%
\definecolor{currentstroke}{rgb}{1.000000,1.000000,1.000000}%
\pgfsetstrokecolor{currentstroke}%
\pgfsetdash{}{0pt}%
\pgfpathmoveto{\pgfqpoint{1.939275in}{4.182873in}}%
\pgfpathcurveto{\pgfqpoint{1.950326in}{4.182873in}}{\pgfqpoint{1.960925in}{4.187263in}}{\pgfqpoint{1.968738in}{4.195077in}}%
\pgfpathcurveto{\pgfqpoint{1.976552in}{4.202890in}}{\pgfqpoint{1.980942in}{4.213489in}}{\pgfqpoint{1.980942in}{4.224539in}}%
\pgfpathcurveto{\pgfqpoint{1.980942in}{4.235590in}}{\pgfqpoint{1.976552in}{4.246189in}}{\pgfqpoint{1.968738in}{4.254002in}}%
\pgfpathcurveto{\pgfqpoint{1.960925in}{4.261816in}}{\pgfqpoint{1.950326in}{4.266206in}}{\pgfqpoint{1.939275in}{4.266206in}}%
\pgfpathcurveto{\pgfqpoint{1.928225in}{4.266206in}}{\pgfqpoint{1.917626in}{4.261816in}}{\pgfqpoint{1.909813in}{4.254002in}}%
\pgfpathcurveto{\pgfqpoint{1.901999in}{4.246189in}}{\pgfqpoint{1.897609in}{4.235590in}}{\pgfqpoint{1.897609in}{4.224539in}}%
\pgfpathcurveto{\pgfqpoint{1.897609in}{4.213489in}}{\pgfqpoint{1.901999in}{4.202890in}}{\pgfqpoint{1.909813in}{4.195077in}}%
\pgfpathcurveto{\pgfqpoint{1.917626in}{4.187263in}}{\pgfqpoint{1.928225in}{4.182873in}}{\pgfqpoint{1.939275in}{4.182873in}}%
\pgfpathclose%
\pgfusepath{stroke,fill}%
\end{pgfscope}%
\begin{pgfscope}%
\pgfpathrectangle{\pgfqpoint{0.481978in}{0.331635in}}{\pgfqpoint{9.300000in}{7.700000in}}%
\pgfusepath{clip}%
\pgfsetbuttcap%
\pgfsetroundjoin%
\definecolor{currentfill}{rgb}{1.000000,0.705882,0.509804}%
\pgfsetfillcolor{currentfill}%
\pgfsetlinewidth{0.481800pt}%
\definecolor{currentstroke}{rgb}{1.000000,1.000000,1.000000}%
\pgfsetstrokecolor{currentstroke}%
\pgfsetdash{}{0pt}%
\pgfpathmoveto{\pgfqpoint{4.479280in}{2.869645in}}%
\pgfpathcurveto{\pgfqpoint{4.490330in}{2.869645in}}{\pgfqpoint{4.500929in}{2.874035in}}{\pgfqpoint{4.508743in}{2.881849in}}%
\pgfpathcurveto{\pgfqpoint{4.516557in}{2.889662in}}{\pgfqpoint{4.520947in}{2.900261in}}{\pgfqpoint{4.520947in}{2.911311in}}%
\pgfpathcurveto{\pgfqpoint{4.520947in}{2.922362in}}{\pgfqpoint{4.516557in}{2.932961in}}{\pgfqpoint{4.508743in}{2.940774in}}%
\pgfpathcurveto{\pgfqpoint{4.500929in}{2.948588in}}{\pgfqpoint{4.490330in}{2.952978in}}{\pgfqpoint{4.479280in}{2.952978in}}%
\pgfpathcurveto{\pgfqpoint{4.468230in}{2.952978in}}{\pgfqpoint{4.457631in}{2.948588in}}{\pgfqpoint{4.449817in}{2.940774in}}%
\pgfpathcurveto{\pgfqpoint{4.442004in}{2.932961in}}{\pgfqpoint{4.437614in}{2.922362in}}{\pgfqpoint{4.437614in}{2.911311in}}%
\pgfpathcurveto{\pgfqpoint{4.437614in}{2.900261in}}{\pgfqpoint{4.442004in}{2.889662in}}{\pgfqpoint{4.449817in}{2.881849in}}%
\pgfpathcurveto{\pgfqpoint{4.457631in}{2.874035in}}{\pgfqpoint{4.468230in}{2.869645in}}{\pgfqpoint{4.479280in}{2.869645in}}%
\pgfpathclose%
\pgfusepath{stroke,fill}%
\end{pgfscope}%
\begin{pgfscope}%
\pgfpathrectangle{\pgfqpoint{0.481978in}{0.331635in}}{\pgfqpoint{9.300000in}{7.700000in}}%
\pgfusepath{clip}%
\pgfsetbuttcap%
\pgfsetroundjoin%
\definecolor{currentfill}{rgb}{1.000000,0.705882,0.509804}%
\pgfsetfillcolor{currentfill}%
\pgfsetlinewidth{0.481800pt}%
\definecolor{currentstroke}{rgb}{1.000000,1.000000,1.000000}%
\pgfsetstrokecolor{currentstroke}%
\pgfsetdash{}{0pt}%
\pgfpathmoveto{\pgfqpoint{3.013348in}{2.602675in}}%
\pgfpathcurveto{\pgfqpoint{3.024398in}{2.602675in}}{\pgfqpoint{3.034997in}{2.607065in}}{\pgfqpoint{3.042811in}{2.614879in}}%
\pgfpathcurveto{\pgfqpoint{3.050624in}{2.622692in}}{\pgfqpoint{3.055014in}{2.633291in}}{\pgfqpoint{3.055014in}{2.644342in}}%
\pgfpathcurveto{\pgfqpoint{3.055014in}{2.655392in}}{\pgfqpoint{3.050624in}{2.665991in}}{\pgfqpoint{3.042811in}{2.673804in}}%
\pgfpathcurveto{\pgfqpoint{3.034997in}{2.681618in}}{\pgfqpoint{3.024398in}{2.686008in}}{\pgfqpoint{3.013348in}{2.686008in}}%
\pgfpathcurveto{\pgfqpoint{3.002298in}{2.686008in}}{\pgfqpoint{2.991699in}{2.681618in}}{\pgfqpoint{2.983885in}{2.673804in}}%
\pgfpathcurveto{\pgfqpoint{2.976071in}{2.665991in}}{\pgfqpoint{2.971681in}{2.655392in}}{\pgfqpoint{2.971681in}{2.644342in}}%
\pgfpathcurveto{\pgfqpoint{2.971681in}{2.633291in}}{\pgfqpoint{2.976071in}{2.622692in}}{\pgfqpoint{2.983885in}{2.614879in}}%
\pgfpathcurveto{\pgfqpoint{2.991699in}{2.607065in}}{\pgfqpoint{3.002298in}{2.602675in}}{\pgfqpoint{3.013348in}{2.602675in}}%
\pgfpathclose%
\pgfusepath{stroke,fill}%
\end{pgfscope}%
\begin{pgfscope}%
\pgfpathrectangle{\pgfqpoint{0.481978in}{0.331635in}}{\pgfqpoint{9.300000in}{7.700000in}}%
\pgfusepath{clip}%
\pgfsetbuttcap%
\pgfsetroundjoin%
\definecolor{currentfill}{rgb}{1.000000,0.705882,0.509804}%
\pgfsetfillcolor{currentfill}%
\pgfsetlinewidth{0.481800pt}%
\definecolor{currentstroke}{rgb}{1.000000,1.000000,1.000000}%
\pgfsetstrokecolor{currentstroke}%
\pgfsetdash{}{0pt}%
\pgfpathmoveto{\pgfqpoint{4.165514in}{6.486612in}}%
\pgfpathcurveto{\pgfqpoint{4.176565in}{6.486612in}}{\pgfqpoint{4.187164in}{6.491002in}}{\pgfqpoint{4.194977in}{6.498816in}}%
\pgfpathcurveto{\pgfqpoint{4.202791in}{6.506629in}}{\pgfqpoint{4.207181in}{6.517228in}}{\pgfqpoint{4.207181in}{6.528278in}}%
\pgfpathcurveto{\pgfqpoint{4.207181in}{6.539328in}}{\pgfqpoint{4.202791in}{6.549927in}}{\pgfqpoint{4.194977in}{6.557741in}}%
\pgfpathcurveto{\pgfqpoint{4.187164in}{6.565555in}}{\pgfqpoint{4.176565in}{6.569945in}}{\pgfqpoint{4.165514in}{6.569945in}}%
\pgfpathcurveto{\pgfqpoint{4.154464in}{6.569945in}}{\pgfqpoint{4.143865in}{6.565555in}}{\pgfqpoint{4.136052in}{6.557741in}}%
\pgfpathcurveto{\pgfqpoint{4.128238in}{6.549927in}}{\pgfqpoint{4.123848in}{6.539328in}}{\pgfqpoint{4.123848in}{6.528278in}}%
\pgfpathcurveto{\pgfqpoint{4.123848in}{6.517228in}}{\pgfqpoint{4.128238in}{6.506629in}}{\pgfqpoint{4.136052in}{6.498816in}}%
\pgfpathcurveto{\pgfqpoint{4.143865in}{6.491002in}}{\pgfqpoint{4.154464in}{6.486612in}}{\pgfqpoint{4.165514in}{6.486612in}}%
\pgfpathclose%
\pgfusepath{stroke,fill}%
\end{pgfscope}%
\begin{pgfscope}%
\pgfpathrectangle{\pgfqpoint{0.481978in}{0.331635in}}{\pgfqpoint{9.300000in}{7.700000in}}%
\pgfusepath{clip}%
\pgfsetbuttcap%
\pgfsetroundjoin%
\definecolor{currentfill}{rgb}{1.000000,0.705882,0.509804}%
\pgfsetfillcolor{currentfill}%
\pgfsetlinewidth{0.481800pt}%
\definecolor{currentstroke}{rgb}{1.000000,1.000000,1.000000}%
\pgfsetstrokecolor{currentstroke}%
\pgfsetdash{}{0pt}%
\pgfpathmoveto{\pgfqpoint{2.157737in}{4.110593in}}%
\pgfpathcurveto{\pgfqpoint{2.168787in}{4.110593in}}{\pgfqpoint{2.179387in}{4.114984in}}{\pgfqpoint{2.187200in}{4.122797in}}%
\pgfpathcurveto{\pgfqpoint{2.195014in}{4.130611in}}{\pgfqpoint{2.199404in}{4.141210in}}{\pgfqpoint{2.199404in}{4.152260in}}%
\pgfpathcurveto{\pgfqpoint{2.199404in}{4.163310in}}{\pgfqpoint{2.195014in}{4.173909in}}{\pgfqpoint{2.187200in}{4.181723in}}%
\pgfpathcurveto{\pgfqpoint{2.179387in}{4.189536in}}{\pgfqpoint{2.168787in}{4.193927in}}{\pgfqpoint{2.157737in}{4.193927in}}%
\pgfpathcurveto{\pgfqpoint{2.146687in}{4.193927in}}{\pgfqpoint{2.136088in}{4.189536in}}{\pgfqpoint{2.128275in}{4.181723in}}%
\pgfpathcurveto{\pgfqpoint{2.120461in}{4.173909in}}{\pgfqpoint{2.116071in}{4.163310in}}{\pgfqpoint{2.116071in}{4.152260in}}%
\pgfpathcurveto{\pgfqpoint{2.116071in}{4.141210in}}{\pgfqpoint{2.120461in}{4.130611in}}{\pgfqpoint{2.128275in}{4.122797in}}%
\pgfpathcurveto{\pgfqpoint{2.136088in}{4.114984in}}{\pgfqpoint{2.146687in}{4.110593in}}{\pgfqpoint{2.157737in}{4.110593in}}%
\pgfpathclose%
\pgfusepath{stroke,fill}%
\end{pgfscope}%
\begin{pgfscope}%
\pgfpathrectangle{\pgfqpoint{0.481978in}{0.331635in}}{\pgfqpoint{9.300000in}{7.700000in}}%
\pgfusepath{clip}%
\pgfsetbuttcap%
\pgfsetroundjoin%
\definecolor{currentfill}{rgb}{1.000000,0.705882,0.509804}%
\pgfsetfillcolor{currentfill}%
\pgfsetlinewidth{0.481800pt}%
\definecolor{currentstroke}{rgb}{1.000000,1.000000,1.000000}%
\pgfsetstrokecolor{currentstroke}%
\pgfsetdash{}{0pt}%
\pgfpathmoveto{\pgfqpoint{2.216691in}{5.366568in}}%
\pgfpathcurveto{\pgfqpoint{2.227741in}{5.366568in}}{\pgfqpoint{2.238340in}{5.370958in}}{\pgfqpoint{2.246154in}{5.378772in}}%
\pgfpathcurveto{\pgfqpoint{2.253968in}{5.386586in}}{\pgfqpoint{2.258358in}{5.397185in}}{\pgfqpoint{2.258358in}{5.408235in}}%
\pgfpathcurveto{\pgfqpoint{2.258358in}{5.419285in}}{\pgfqpoint{2.253968in}{5.429884in}}{\pgfqpoint{2.246154in}{5.437697in}}%
\pgfpathcurveto{\pgfqpoint{2.238340in}{5.445511in}}{\pgfqpoint{2.227741in}{5.449901in}}{\pgfqpoint{2.216691in}{5.449901in}}%
\pgfpathcurveto{\pgfqpoint{2.205641in}{5.449901in}}{\pgfqpoint{2.195042in}{5.445511in}}{\pgfqpoint{2.187228in}{5.437697in}}%
\pgfpathcurveto{\pgfqpoint{2.179415in}{5.429884in}}{\pgfqpoint{2.175025in}{5.419285in}}{\pgfqpoint{2.175025in}{5.408235in}}%
\pgfpathcurveto{\pgfqpoint{2.175025in}{5.397185in}}{\pgfqpoint{2.179415in}{5.386586in}}{\pgfqpoint{2.187228in}{5.378772in}}%
\pgfpathcurveto{\pgfqpoint{2.195042in}{5.370958in}}{\pgfqpoint{2.205641in}{5.366568in}}{\pgfqpoint{2.216691in}{5.366568in}}%
\pgfpathclose%
\pgfusepath{stroke,fill}%
\end{pgfscope}%
\begin{pgfscope}%
\pgfpathrectangle{\pgfqpoint{0.481978in}{0.331635in}}{\pgfqpoint{9.300000in}{7.700000in}}%
\pgfusepath{clip}%
\pgfsetbuttcap%
\pgfsetroundjoin%
\definecolor{currentfill}{rgb}{1.000000,0.705882,0.509804}%
\pgfsetfillcolor{currentfill}%
\pgfsetlinewidth{0.481800pt}%
\definecolor{currentstroke}{rgb}{1.000000,1.000000,1.000000}%
\pgfsetstrokecolor{currentstroke}%
\pgfsetdash{}{0pt}%
\pgfpathmoveto{\pgfqpoint{4.655686in}{2.205125in}}%
\pgfpathcurveto{\pgfqpoint{4.666736in}{2.205125in}}{\pgfqpoint{4.677335in}{2.209515in}}{\pgfqpoint{4.685149in}{2.217329in}}%
\pgfpathcurveto{\pgfqpoint{4.692962in}{2.225143in}}{\pgfqpoint{4.697353in}{2.235742in}}{\pgfqpoint{4.697353in}{2.246792in}}%
\pgfpathcurveto{\pgfqpoint{4.697353in}{2.257842in}}{\pgfqpoint{4.692962in}{2.268441in}}{\pgfqpoint{4.685149in}{2.276255in}}%
\pgfpathcurveto{\pgfqpoint{4.677335in}{2.284068in}}{\pgfqpoint{4.666736in}{2.288458in}}{\pgfqpoint{4.655686in}{2.288458in}}%
\pgfpathcurveto{\pgfqpoint{4.644636in}{2.288458in}}{\pgfqpoint{4.634037in}{2.284068in}}{\pgfqpoint{4.626223in}{2.276255in}}%
\pgfpathcurveto{\pgfqpoint{4.618410in}{2.268441in}}{\pgfqpoint{4.614019in}{2.257842in}}{\pgfqpoint{4.614019in}{2.246792in}}%
\pgfpathcurveto{\pgfqpoint{4.614019in}{2.235742in}}{\pgfqpoint{4.618410in}{2.225143in}}{\pgfqpoint{4.626223in}{2.217329in}}%
\pgfpathcurveto{\pgfqpoint{4.634037in}{2.209515in}}{\pgfqpoint{4.644636in}{2.205125in}}{\pgfqpoint{4.655686in}{2.205125in}}%
\pgfpathclose%
\pgfusepath{stroke,fill}%
\end{pgfscope}%
\begin{pgfscope}%
\pgfpathrectangle{\pgfqpoint{0.481978in}{0.331635in}}{\pgfqpoint{9.300000in}{7.700000in}}%
\pgfusepath{clip}%
\pgfsetbuttcap%
\pgfsetroundjoin%
\definecolor{currentfill}{rgb}{1.000000,0.705882,0.509804}%
\pgfsetfillcolor{currentfill}%
\pgfsetlinewidth{0.481800pt}%
\definecolor{currentstroke}{rgb}{1.000000,1.000000,1.000000}%
\pgfsetstrokecolor{currentstroke}%
\pgfsetdash{}{0pt}%
\pgfpathmoveto{\pgfqpoint{3.175062in}{3.307064in}}%
\pgfpathcurveto{\pgfqpoint{3.186112in}{3.307064in}}{\pgfqpoint{3.196711in}{3.311454in}}{\pgfqpoint{3.204525in}{3.319268in}}%
\pgfpathcurveto{\pgfqpoint{3.212339in}{3.327081in}}{\pgfqpoint{3.216729in}{3.337681in}}{\pgfqpoint{3.216729in}{3.348731in}}%
\pgfpathcurveto{\pgfqpoint{3.216729in}{3.359781in}}{\pgfqpoint{3.212339in}{3.370380in}}{\pgfqpoint{3.204525in}{3.378193in}}%
\pgfpathcurveto{\pgfqpoint{3.196711in}{3.386007in}}{\pgfqpoint{3.186112in}{3.390397in}}{\pgfqpoint{3.175062in}{3.390397in}}%
\pgfpathcurveto{\pgfqpoint{3.164012in}{3.390397in}}{\pgfqpoint{3.153413in}{3.386007in}}{\pgfqpoint{3.145599in}{3.378193in}}%
\pgfpathcurveto{\pgfqpoint{3.137786in}{3.370380in}}{\pgfqpoint{3.133396in}{3.359781in}}{\pgfqpoint{3.133396in}{3.348731in}}%
\pgfpathcurveto{\pgfqpoint{3.133396in}{3.337681in}}{\pgfqpoint{3.137786in}{3.327081in}}{\pgfqpoint{3.145599in}{3.319268in}}%
\pgfpathcurveto{\pgfqpoint{3.153413in}{3.311454in}}{\pgfqpoint{3.164012in}{3.307064in}}{\pgfqpoint{3.175062in}{3.307064in}}%
\pgfpathclose%
\pgfusepath{stroke,fill}%
\end{pgfscope}%
\begin{pgfscope}%
\pgfpathrectangle{\pgfqpoint{0.481978in}{0.331635in}}{\pgfqpoint{9.300000in}{7.700000in}}%
\pgfusepath{clip}%
\pgfsetbuttcap%
\pgfsetroundjoin%
\definecolor{currentfill}{rgb}{1.000000,0.705882,0.509804}%
\pgfsetfillcolor{currentfill}%
\pgfsetlinewidth{0.481800pt}%
\definecolor{currentstroke}{rgb}{1.000000,1.000000,1.000000}%
\pgfsetstrokecolor{currentstroke}%
\pgfsetdash{}{0pt}%
\pgfpathmoveto{\pgfqpoint{2.367140in}{4.674797in}}%
\pgfpathcurveto{\pgfqpoint{2.378190in}{4.674797in}}{\pgfqpoint{2.388789in}{4.679187in}}{\pgfqpoint{2.396603in}{4.687001in}}%
\pgfpathcurveto{\pgfqpoint{2.404417in}{4.694815in}}{\pgfqpoint{2.408807in}{4.705414in}}{\pgfqpoint{2.408807in}{4.716464in}}%
\pgfpathcurveto{\pgfqpoint{2.408807in}{4.727514in}}{\pgfqpoint{2.404417in}{4.738113in}}{\pgfqpoint{2.396603in}{4.745927in}}%
\pgfpathcurveto{\pgfqpoint{2.388789in}{4.753740in}}{\pgfqpoint{2.378190in}{4.758131in}}{\pgfqpoint{2.367140in}{4.758131in}}%
\pgfpathcurveto{\pgfqpoint{2.356090in}{4.758131in}}{\pgfqpoint{2.345491in}{4.753740in}}{\pgfqpoint{2.337678in}{4.745927in}}%
\pgfpathcurveto{\pgfqpoint{2.329864in}{4.738113in}}{\pgfqpoint{2.325474in}{4.727514in}}{\pgfqpoint{2.325474in}{4.716464in}}%
\pgfpathcurveto{\pgfqpoint{2.325474in}{4.705414in}}{\pgfqpoint{2.329864in}{4.694815in}}{\pgfqpoint{2.337678in}{4.687001in}}%
\pgfpathcurveto{\pgfqpoint{2.345491in}{4.679187in}}{\pgfqpoint{2.356090in}{4.674797in}}{\pgfqpoint{2.367140in}{4.674797in}}%
\pgfpathclose%
\pgfusepath{stroke,fill}%
\end{pgfscope}%
\begin{pgfscope}%
\pgfpathrectangle{\pgfqpoint{0.481978in}{0.331635in}}{\pgfqpoint{9.300000in}{7.700000in}}%
\pgfusepath{clip}%
\pgfsetbuttcap%
\pgfsetroundjoin%
\definecolor{currentfill}{rgb}{1.000000,0.705882,0.509804}%
\pgfsetfillcolor{currentfill}%
\pgfsetlinewidth{0.481800pt}%
\definecolor{currentstroke}{rgb}{1.000000,1.000000,1.000000}%
\pgfsetstrokecolor{currentstroke}%
\pgfsetdash{}{0pt}%
\pgfpathmoveto{\pgfqpoint{4.838472in}{2.994339in}}%
\pgfpathcurveto{\pgfqpoint{4.849522in}{2.994339in}}{\pgfqpoint{4.860121in}{2.998729in}}{\pgfqpoint{4.867935in}{3.006543in}}%
\pgfpathcurveto{\pgfqpoint{4.875748in}{3.014357in}}{\pgfqpoint{4.880139in}{3.024956in}}{\pgfqpoint{4.880139in}{3.036006in}}%
\pgfpathcurveto{\pgfqpoint{4.880139in}{3.047056in}}{\pgfqpoint{4.875748in}{3.057655in}}{\pgfqpoint{4.867935in}{3.065468in}}%
\pgfpathcurveto{\pgfqpoint{4.860121in}{3.073282in}}{\pgfqpoint{4.849522in}{3.077672in}}{\pgfqpoint{4.838472in}{3.077672in}}%
\pgfpathcurveto{\pgfqpoint{4.827422in}{3.077672in}}{\pgfqpoint{4.816823in}{3.073282in}}{\pgfqpoint{4.809009in}{3.065468in}}%
\pgfpathcurveto{\pgfqpoint{4.801196in}{3.057655in}}{\pgfqpoint{4.796805in}{3.047056in}}{\pgfqpoint{4.796805in}{3.036006in}}%
\pgfpathcurveto{\pgfqpoint{4.796805in}{3.024956in}}{\pgfqpoint{4.801196in}{3.014357in}}{\pgfqpoint{4.809009in}{3.006543in}}%
\pgfpathcurveto{\pgfqpoint{4.816823in}{2.998729in}}{\pgfqpoint{4.827422in}{2.994339in}}{\pgfqpoint{4.838472in}{2.994339in}}%
\pgfpathclose%
\pgfusepath{stroke,fill}%
\end{pgfscope}%
\begin{pgfscope}%
\pgfpathrectangle{\pgfqpoint{0.481978in}{0.331635in}}{\pgfqpoint{9.300000in}{7.700000in}}%
\pgfusepath{clip}%
\pgfsetbuttcap%
\pgfsetroundjoin%
\definecolor{currentfill}{rgb}{1.000000,0.705882,0.509804}%
\pgfsetfillcolor{currentfill}%
\pgfsetlinewidth{0.481800pt}%
\definecolor{currentstroke}{rgb}{1.000000,1.000000,1.000000}%
\pgfsetstrokecolor{currentstroke}%
\pgfsetdash{}{0pt}%
\pgfpathmoveto{\pgfqpoint{4.322216in}{5.005904in}}%
\pgfpathcurveto{\pgfqpoint{4.333266in}{5.005904in}}{\pgfqpoint{4.343866in}{5.010294in}}{\pgfqpoint{4.351679in}{5.018108in}}%
\pgfpathcurveto{\pgfqpoint{4.359493in}{5.025921in}}{\pgfqpoint{4.363883in}{5.036520in}}{\pgfqpoint{4.363883in}{5.047571in}}%
\pgfpathcurveto{\pgfqpoint{4.363883in}{5.058621in}}{\pgfqpoint{4.359493in}{5.069220in}}{\pgfqpoint{4.351679in}{5.077033in}}%
\pgfpathcurveto{\pgfqpoint{4.343866in}{5.084847in}}{\pgfqpoint{4.333266in}{5.089237in}}{\pgfqpoint{4.322216in}{5.089237in}}%
\pgfpathcurveto{\pgfqpoint{4.311166in}{5.089237in}}{\pgfqpoint{4.300567in}{5.084847in}}{\pgfqpoint{4.292754in}{5.077033in}}%
\pgfpathcurveto{\pgfqpoint{4.284940in}{5.069220in}}{\pgfqpoint{4.280550in}{5.058621in}}{\pgfqpoint{4.280550in}{5.047571in}}%
\pgfpathcurveto{\pgfqpoint{4.280550in}{5.036520in}}{\pgfqpoint{4.284940in}{5.025921in}}{\pgfqpoint{4.292754in}{5.018108in}}%
\pgfpathcurveto{\pgfqpoint{4.300567in}{5.010294in}}{\pgfqpoint{4.311166in}{5.005904in}}{\pgfqpoint{4.322216in}{5.005904in}}%
\pgfpathclose%
\pgfusepath{stroke,fill}%
\end{pgfscope}%
\begin{pgfscope}%
\pgfpathrectangle{\pgfqpoint{0.481978in}{0.331635in}}{\pgfqpoint{9.300000in}{7.700000in}}%
\pgfusepath{clip}%
\pgfsetbuttcap%
\pgfsetroundjoin%
\definecolor{currentfill}{rgb}{1.000000,0.705882,0.509804}%
\pgfsetfillcolor{currentfill}%
\pgfsetlinewidth{0.481800pt}%
\definecolor{currentstroke}{rgb}{1.000000,1.000000,1.000000}%
\pgfsetstrokecolor{currentstroke}%
\pgfsetdash{}{0pt}%
\pgfpathmoveto{\pgfqpoint{3.615208in}{3.422279in}}%
\pgfpathcurveto{\pgfqpoint{3.626258in}{3.422279in}}{\pgfqpoint{3.636857in}{3.426669in}}{\pgfqpoint{3.644670in}{3.434483in}}%
\pgfpathcurveto{\pgfqpoint{3.652484in}{3.442296in}}{\pgfqpoint{3.656874in}{3.452895in}}{\pgfqpoint{3.656874in}{3.463945in}}%
\pgfpathcurveto{\pgfqpoint{3.656874in}{3.474995in}}{\pgfqpoint{3.652484in}{3.485595in}}{\pgfqpoint{3.644670in}{3.493408in}}%
\pgfpathcurveto{\pgfqpoint{3.636857in}{3.501222in}}{\pgfqpoint{3.626258in}{3.505612in}}{\pgfqpoint{3.615208in}{3.505612in}}%
\pgfpathcurveto{\pgfqpoint{3.604158in}{3.505612in}}{\pgfqpoint{3.593559in}{3.501222in}}{\pgfqpoint{3.585745in}{3.493408in}}%
\pgfpathcurveto{\pgfqpoint{3.577931in}{3.485595in}}{\pgfqpoint{3.573541in}{3.474995in}}{\pgfqpoint{3.573541in}{3.463945in}}%
\pgfpathcurveto{\pgfqpoint{3.573541in}{3.452895in}}{\pgfqpoint{3.577931in}{3.442296in}}{\pgfqpoint{3.585745in}{3.434483in}}%
\pgfpathcurveto{\pgfqpoint{3.593559in}{3.426669in}}{\pgfqpoint{3.604158in}{3.422279in}}{\pgfqpoint{3.615208in}{3.422279in}}%
\pgfpathclose%
\pgfusepath{stroke,fill}%
\end{pgfscope}%
\begin{pgfscope}%
\pgfpathrectangle{\pgfqpoint{0.481978in}{0.331635in}}{\pgfqpoint{9.300000in}{7.700000in}}%
\pgfusepath{clip}%
\pgfsetbuttcap%
\pgfsetroundjoin%
\definecolor{currentfill}{rgb}{1.000000,0.705882,0.509804}%
\pgfsetfillcolor{currentfill}%
\pgfsetlinewidth{0.481800pt}%
\definecolor{currentstroke}{rgb}{1.000000,1.000000,1.000000}%
\pgfsetstrokecolor{currentstroke}%
\pgfsetdash{}{0pt}%
\pgfpathmoveto{\pgfqpoint{3.424608in}{6.905454in}}%
\pgfpathcurveto{\pgfqpoint{3.435658in}{6.905454in}}{\pgfqpoint{3.446257in}{6.909844in}}{\pgfqpoint{3.454071in}{6.917658in}}%
\pgfpathcurveto{\pgfqpoint{3.461884in}{6.925471in}}{\pgfqpoint{3.466274in}{6.936070in}}{\pgfqpoint{3.466274in}{6.947120in}}%
\pgfpathcurveto{\pgfqpoint{3.466274in}{6.958171in}}{\pgfqpoint{3.461884in}{6.968770in}}{\pgfqpoint{3.454071in}{6.976583in}}%
\pgfpathcurveto{\pgfqpoint{3.446257in}{6.984397in}}{\pgfqpoint{3.435658in}{6.988787in}}{\pgfqpoint{3.424608in}{6.988787in}}%
\pgfpathcurveto{\pgfqpoint{3.413558in}{6.988787in}}{\pgfqpoint{3.402959in}{6.984397in}}{\pgfqpoint{3.395145in}{6.976583in}}%
\pgfpathcurveto{\pgfqpoint{3.387331in}{6.968770in}}{\pgfqpoint{3.382941in}{6.958171in}}{\pgfqpoint{3.382941in}{6.947120in}}%
\pgfpathcurveto{\pgfqpoint{3.382941in}{6.936070in}}{\pgfqpoint{3.387331in}{6.925471in}}{\pgfqpoint{3.395145in}{6.917658in}}%
\pgfpathcurveto{\pgfqpoint{3.402959in}{6.909844in}}{\pgfqpoint{3.413558in}{6.905454in}}{\pgfqpoint{3.424608in}{6.905454in}}%
\pgfpathclose%
\pgfusepath{stroke,fill}%
\end{pgfscope}%
\begin{pgfscope}%
\pgfpathrectangle{\pgfqpoint{0.481978in}{0.331635in}}{\pgfqpoint{9.300000in}{7.700000in}}%
\pgfusepath{clip}%
\pgfsetbuttcap%
\pgfsetroundjoin%
\definecolor{currentfill}{rgb}{1.000000,0.705882,0.509804}%
\pgfsetfillcolor{currentfill}%
\pgfsetlinewidth{0.481800pt}%
\definecolor{currentstroke}{rgb}{1.000000,1.000000,1.000000}%
\pgfsetstrokecolor{currentstroke}%
\pgfsetdash{}{0pt}%
\pgfpathmoveto{\pgfqpoint{5.585402in}{5.198457in}}%
\pgfpathcurveto{\pgfqpoint{5.596452in}{5.198457in}}{\pgfqpoint{5.607051in}{5.202848in}}{\pgfqpoint{5.614865in}{5.210661in}}%
\pgfpathcurveto{\pgfqpoint{5.622678in}{5.218475in}}{\pgfqpoint{5.627069in}{5.229074in}}{\pgfqpoint{5.627069in}{5.240124in}}%
\pgfpathcurveto{\pgfqpoint{5.627069in}{5.251174in}}{\pgfqpoint{5.622678in}{5.261773in}}{\pgfqpoint{5.614865in}{5.269587in}}%
\pgfpathcurveto{\pgfqpoint{5.607051in}{5.277400in}}{\pgfqpoint{5.596452in}{5.281791in}}{\pgfqpoint{5.585402in}{5.281791in}}%
\pgfpathcurveto{\pgfqpoint{5.574352in}{5.281791in}}{\pgfqpoint{5.563753in}{5.277400in}}{\pgfqpoint{5.555939in}{5.269587in}}%
\pgfpathcurveto{\pgfqpoint{5.548125in}{5.261773in}}{\pgfqpoint{5.543735in}{5.251174in}}{\pgfqpoint{5.543735in}{5.240124in}}%
\pgfpathcurveto{\pgfqpoint{5.543735in}{5.229074in}}{\pgfqpoint{5.548125in}{5.218475in}}{\pgfqpoint{5.555939in}{5.210661in}}%
\pgfpathcurveto{\pgfqpoint{5.563753in}{5.202848in}}{\pgfqpoint{5.574352in}{5.198457in}}{\pgfqpoint{5.585402in}{5.198457in}}%
\pgfpathclose%
\pgfusepath{stroke,fill}%
\end{pgfscope}%
\begin{pgfscope}%
\pgfpathrectangle{\pgfqpoint{0.481978in}{0.331635in}}{\pgfqpoint{9.300000in}{7.700000in}}%
\pgfusepath{clip}%
\pgfsetbuttcap%
\pgfsetroundjoin%
\definecolor{currentfill}{rgb}{1.000000,0.705882,0.509804}%
\pgfsetfillcolor{currentfill}%
\pgfsetlinewidth{0.481800pt}%
\definecolor{currentstroke}{rgb}{1.000000,1.000000,1.000000}%
\pgfsetstrokecolor{currentstroke}%
\pgfsetdash{}{0pt}%
\pgfpathmoveto{\pgfqpoint{3.153342in}{5.068040in}}%
\pgfpathcurveto{\pgfqpoint{3.164392in}{5.068040in}}{\pgfqpoint{3.174991in}{5.072431in}}{\pgfqpoint{3.182805in}{5.080244in}}%
\pgfpathcurveto{\pgfqpoint{3.190618in}{5.088058in}}{\pgfqpoint{3.195008in}{5.098657in}}{\pgfqpoint{3.195008in}{5.109707in}}%
\pgfpathcurveto{\pgfqpoint{3.195008in}{5.120757in}}{\pgfqpoint{3.190618in}{5.131356in}}{\pgfqpoint{3.182805in}{5.139170in}}%
\pgfpathcurveto{\pgfqpoint{3.174991in}{5.146983in}}{\pgfqpoint{3.164392in}{5.151374in}}{\pgfqpoint{3.153342in}{5.151374in}}%
\pgfpathcurveto{\pgfqpoint{3.142292in}{5.151374in}}{\pgfqpoint{3.131693in}{5.146983in}}{\pgfqpoint{3.123879in}{5.139170in}}%
\pgfpathcurveto{\pgfqpoint{3.116065in}{5.131356in}}{\pgfqpoint{3.111675in}{5.120757in}}{\pgfqpoint{3.111675in}{5.109707in}}%
\pgfpathcurveto{\pgfqpoint{3.111675in}{5.098657in}}{\pgfqpoint{3.116065in}{5.088058in}}{\pgfqpoint{3.123879in}{5.080244in}}%
\pgfpathcurveto{\pgfqpoint{3.131693in}{5.072431in}}{\pgfqpoint{3.142292in}{5.068040in}}{\pgfqpoint{3.153342in}{5.068040in}}%
\pgfpathclose%
\pgfusepath{stroke,fill}%
\end{pgfscope}%
\begin{pgfscope}%
\pgfpathrectangle{\pgfqpoint{0.481978in}{0.331635in}}{\pgfqpoint{9.300000in}{7.700000in}}%
\pgfusepath{clip}%
\pgfsetbuttcap%
\pgfsetroundjoin%
\definecolor{currentfill}{rgb}{1.000000,0.705882,0.509804}%
\pgfsetfillcolor{currentfill}%
\pgfsetlinewidth{0.481800pt}%
\definecolor{currentstroke}{rgb}{1.000000,1.000000,1.000000}%
\pgfsetstrokecolor{currentstroke}%
\pgfsetdash{}{0pt}%
\pgfpathmoveto{\pgfqpoint{3.297548in}{5.440827in}}%
\pgfpathcurveto{\pgfqpoint{3.308598in}{5.440827in}}{\pgfqpoint{3.319197in}{5.445218in}}{\pgfqpoint{3.327011in}{5.453031in}}%
\pgfpathcurveto{\pgfqpoint{3.334825in}{5.460845in}}{\pgfqpoint{3.339215in}{5.471444in}}{\pgfqpoint{3.339215in}{5.482494in}}%
\pgfpathcurveto{\pgfqpoint{3.339215in}{5.493544in}}{\pgfqpoint{3.334825in}{5.504143in}}{\pgfqpoint{3.327011in}{5.511957in}}%
\pgfpathcurveto{\pgfqpoint{3.319197in}{5.519770in}}{\pgfqpoint{3.308598in}{5.524161in}}{\pgfqpoint{3.297548in}{5.524161in}}%
\pgfpathcurveto{\pgfqpoint{3.286498in}{5.524161in}}{\pgfqpoint{3.275899in}{5.519770in}}{\pgfqpoint{3.268086in}{5.511957in}}%
\pgfpathcurveto{\pgfqpoint{3.260272in}{5.504143in}}{\pgfqpoint{3.255882in}{5.493544in}}{\pgfqpoint{3.255882in}{5.482494in}}%
\pgfpathcurveto{\pgfqpoint{3.255882in}{5.471444in}}{\pgfqpoint{3.260272in}{5.460845in}}{\pgfqpoint{3.268086in}{5.453031in}}%
\pgfpathcurveto{\pgfqpoint{3.275899in}{5.445218in}}{\pgfqpoint{3.286498in}{5.440827in}}{\pgfqpoint{3.297548in}{5.440827in}}%
\pgfpathclose%
\pgfusepath{stroke,fill}%
\end{pgfscope}%
\begin{pgfscope}%
\pgfpathrectangle{\pgfqpoint{0.481978in}{0.331635in}}{\pgfqpoint{9.300000in}{7.700000in}}%
\pgfusepath{clip}%
\pgfsetbuttcap%
\pgfsetroundjoin%
\definecolor{currentfill}{rgb}{1.000000,0.705882,0.509804}%
\pgfsetfillcolor{currentfill}%
\pgfsetlinewidth{0.481800pt}%
\definecolor{currentstroke}{rgb}{1.000000,1.000000,1.000000}%
\pgfsetstrokecolor{currentstroke}%
\pgfsetdash{}{0pt}%
\pgfpathmoveto{\pgfqpoint{4.669793in}{2.217934in}}%
\pgfpathcurveto{\pgfqpoint{4.680844in}{2.217934in}}{\pgfqpoint{4.691443in}{2.222324in}}{\pgfqpoint{4.699256in}{2.230138in}}%
\pgfpathcurveto{\pgfqpoint{4.707070in}{2.237951in}}{\pgfqpoint{4.711460in}{2.248550in}}{\pgfqpoint{4.711460in}{2.259600in}}%
\pgfpathcurveto{\pgfqpoint{4.711460in}{2.270650in}}{\pgfqpoint{4.707070in}{2.281250in}}{\pgfqpoint{4.699256in}{2.289063in}}%
\pgfpathcurveto{\pgfqpoint{4.691443in}{2.296877in}}{\pgfqpoint{4.680844in}{2.301267in}}{\pgfqpoint{4.669793in}{2.301267in}}%
\pgfpathcurveto{\pgfqpoint{4.658743in}{2.301267in}}{\pgfqpoint{4.648144in}{2.296877in}}{\pgfqpoint{4.640331in}{2.289063in}}%
\pgfpathcurveto{\pgfqpoint{4.632517in}{2.281250in}}{\pgfqpoint{4.628127in}{2.270650in}}{\pgfqpoint{4.628127in}{2.259600in}}%
\pgfpathcurveto{\pgfqpoint{4.628127in}{2.248550in}}{\pgfqpoint{4.632517in}{2.237951in}}{\pgfqpoint{4.640331in}{2.230138in}}%
\pgfpathcurveto{\pgfqpoint{4.648144in}{2.222324in}}{\pgfqpoint{4.658743in}{2.217934in}}{\pgfqpoint{4.669793in}{2.217934in}}%
\pgfpathclose%
\pgfusepath{stroke,fill}%
\end{pgfscope}%
\begin{pgfscope}%
\pgfpathrectangle{\pgfqpoint{0.481978in}{0.331635in}}{\pgfqpoint{9.300000in}{7.700000in}}%
\pgfusepath{clip}%
\pgfsetbuttcap%
\pgfsetroundjoin%
\definecolor{currentfill}{rgb}{1.000000,0.705882,0.509804}%
\pgfsetfillcolor{currentfill}%
\pgfsetlinewidth{0.481800pt}%
\definecolor{currentstroke}{rgb}{1.000000,1.000000,1.000000}%
\pgfsetstrokecolor{currentstroke}%
\pgfsetdash{}{0pt}%
\pgfpathmoveto{\pgfqpoint{1.546741in}{5.103666in}}%
\pgfpathcurveto{\pgfqpoint{1.557791in}{5.103666in}}{\pgfqpoint{1.568390in}{5.108056in}}{\pgfqpoint{1.576204in}{5.115870in}}%
\pgfpathcurveto{\pgfqpoint{1.584017in}{5.123683in}}{\pgfqpoint{1.588408in}{5.134282in}}{\pgfqpoint{1.588408in}{5.145333in}}%
\pgfpathcurveto{\pgfqpoint{1.588408in}{5.156383in}}{\pgfqpoint{1.584017in}{5.166982in}}{\pgfqpoint{1.576204in}{5.174795in}}%
\pgfpathcurveto{\pgfqpoint{1.568390in}{5.182609in}}{\pgfqpoint{1.557791in}{5.186999in}}{\pgfqpoint{1.546741in}{5.186999in}}%
\pgfpathcurveto{\pgfqpoint{1.535691in}{5.186999in}}{\pgfqpoint{1.525092in}{5.182609in}}{\pgfqpoint{1.517278in}{5.174795in}}%
\pgfpathcurveto{\pgfqpoint{1.509465in}{5.166982in}}{\pgfqpoint{1.505074in}{5.156383in}}{\pgfqpoint{1.505074in}{5.145333in}}%
\pgfpathcurveto{\pgfqpoint{1.505074in}{5.134282in}}{\pgfqpoint{1.509465in}{5.123683in}}{\pgfqpoint{1.517278in}{5.115870in}}%
\pgfpathcurveto{\pgfqpoint{1.525092in}{5.108056in}}{\pgfqpoint{1.535691in}{5.103666in}}{\pgfqpoint{1.546741in}{5.103666in}}%
\pgfpathclose%
\pgfusepath{stroke,fill}%
\end{pgfscope}%
\begin{pgfscope}%
\pgfpathrectangle{\pgfqpoint{0.481978in}{0.331635in}}{\pgfqpoint{9.300000in}{7.700000in}}%
\pgfusepath{clip}%
\pgfsetbuttcap%
\pgfsetroundjoin%
\definecolor{currentfill}{rgb}{1.000000,0.705882,0.509804}%
\pgfsetfillcolor{currentfill}%
\pgfsetlinewidth{0.481800pt}%
\definecolor{currentstroke}{rgb}{1.000000,1.000000,1.000000}%
\pgfsetstrokecolor{currentstroke}%
\pgfsetdash{}{0pt}%
\pgfpathmoveto{\pgfqpoint{4.426621in}{5.353124in}}%
\pgfpathcurveto{\pgfqpoint{4.437671in}{5.353124in}}{\pgfqpoint{4.448270in}{5.357514in}}{\pgfqpoint{4.456084in}{5.365328in}}%
\pgfpathcurveto{\pgfqpoint{4.463897in}{5.373142in}}{\pgfqpoint{4.468287in}{5.383741in}}{\pgfqpoint{4.468287in}{5.394791in}}%
\pgfpathcurveto{\pgfqpoint{4.468287in}{5.405841in}}{\pgfqpoint{4.463897in}{5.416440in}}{\pgfqpoint{4.456084in}{5.424254in}}%
\pgfpathcurveto{\pgfqpoint{4.448270in}{5.432067in}}{\pgfqpoint{4.437671in}{5.436458in}}{\pgfqpoint{4.426621in}{5.436458in}}%
\pgfpathcurveto{\pgfqpoint{4.415571in}{5.436458in}}{\pgfqpoint{4.404972in}{5.432067in}}{\pgfqpoint{4.397158in}{5.424254in}}%
\pgfpathcurveto{\pgfqpoint{4.389344in}{5.416440in}}{\pgfqpoint{4.384954in}{5.405841in}}{\pgfqpoint{4.384954in}{5.394791in}}%
\pgfpathcurveto{\pgfqpoint{4.384954in}{5.383741in}}{\pgfqpoint{4.389344in}{5.373142in}}{\pgfqpoint{4.397158in}{5.365328in}}%
\pgfpathcurveto{\pgfqpoint{4.404972in}{5.357514in}}{\pgfqpoint{4.415571in}{5.353124in}}{\pgfqpoint{4.426621in}{5.353124in}}%
\pgfpathclose%
\pgfusepath{stroke,fill}%
\end{pgfscope}%
\begin{pgfscope}%
\pgfpathrectangle{\pgfqpoint{0.481978in}{0.331635in}}{\pgfqpoint{9.300000in}{7.700000in}}%
\pgfusepath{clip}%
\pgfsetbuttcap%
\pgfsetroundjoin%
\definecolor{currentfill}{rgb}{1.000000,0.705882,0.509804}%
\pgfsetfillcolor{currentfill}%
\pgfsetlinewidth{0.481800pt}%
\definecolor{currentstroke}{rgb}{1.000000,1.000000,1.000000}%
\pgfsetstrokecolor{currentstroke}%
\pgfsetdash{}{0pt}%
\pgfpathmoveto{\pgfqpoint{2.946822in}{5.311992in}}%
\pgfpathcurveto{\pgfqpoint{2.957872in}{5.311992in}}{\pgfqpoint{2.968471in}{5.316382in}}{\pgfqpoint{2.976285in}{5.324196in}}%
\pgfpathcurveto{\pgfqpoint{2.984098in}{5.332009in}}{\pgfqpoint{2.988489in}{5.342608in}}{\pgfqpoint{2.988489in}{5.353658in}}%
\pgfpathcurveto{\pgfqpoint{2.988489in}{5.364709in}}{\pgfqpoint{2.984098in}{5.375308in}}{\pgfqpoint{2.976285in}{5.383121in}}%
\pgfpathcurveto{\pgfqpoint{2.968471in}{5.390935in}}{\pgfqpoint{2.957872in}{5.395325in}}{\pgfqpoint{2.946822in}{5.395325in}}%
\pgfpathcurveto{\pgfqpoint{2.935772in}{5.395325in}}{\pgfqpoint{2.925173in}{5.390935in}}{\pgfqpoint{2.917359in}{5.383121in}}%
\pgfpathcurveto{\pgfqpoint{2.909546in}{5.375308in}}{\pgfqpoint{2.905155in}{5.364709in}}{\pgfqpoint{2.905155in}{5.353658in}}%
\pgfpathcurveto{\pgfqpoint{2.905155in}{5.342608in}}{\pgfqpoint{2.909546in}{5.332009in}}{\pgfqpoint{2.917359in}{5.324196in}}%
\pgfpathcurveto{\pgfqpoint{2.925173in}{5.316382in}}{\pgfqpoint{2.935772in}{5.311992in}}{\pgfqpoint{2.946822in}{5.311992in}}%
\pgfpathclose%
\pgfusepath{stroke,fill}%
\end{pgfscope}%
\begin{pgfscope}%
\pgfpathrectangle{\pgfqpoint{0.481978in}{0.331635in}}{\pgfqpoint{9.300000in}{7.700000in}}%
\pgfusepath{clip}%
\pgfsetbuttcap%
\pgfsetroundjoin%
\definecolor{currentfill}{rgb}{1.000000,0.705882,0.509804}%
\pgfsetfillcolor{currentfill}%
\pgfsetlinewidth{0.481800pt}%
\definecolor{currentstroke}{rgb}{1.000000,1.000000,1.000000}%
\pgfsetstrokecolor{currentstroke}%
\pgfsetdash{}{0pt}%
\pgfpathmoveto{\pgfqpoint{9.359251in}{1.323480in}}%
\pgfpathcurveto{\pgfqpoint{9.370301in}{1.323480in}}{\pgfqpoint{9.380900in}{1.327870in}}{\pgfqpoint{9.388713in}{1.335684in}}%
\pgfpathcurveto{\pgfqpoint{9.396527in}{1.343497in}}{\pgfqpoint{9.400917in}{1.354096in}}{\pgfqpoint{9.400917in}{1.365147in}}%
\pgfpathcurveto{\pgfqpoint{9.400917in}{1.376197in}}{\pgfqpoint{9.396527in}{1.386796in}}{\pgfqpoint{9.388713in}{1.394609in}}%
\pgfpathcurveto{\pgfqpoint{9.380900in}{1.402423in}}{\pgfqpoint{9.370301in}{1.406813in}}{\pgfqpoint{9.359251in}{1.406813in}}%
\pgfpathcurveto{\pgfqpoint{9.348201in}{1.406813in}}{\pgfqpoint{9.337601in}{1.402423in}}{\pgfqpoint{9.329788in}{1.394609in}}%
\pgfpathcurveto{\pgfqpoint{9.321974in}{1.386796in}}{\pgfqpoint{9.317584in}{1.376197in}}{\pgfqpoint{9.317584in}{1.365147in}}%
\pgfpathcurveto{\pgfqpoint{9.317584in}{1.354096in}}{\pgfqpoint{9.321974in}{1.343497in}}{\pgfqpoint{9.329788in}{1.335684in}}%
\pgfpathcurveto{\pgfqpoint{9.337601in}{1.327870in}}{\pgfqpoint{9.348201in}{1.323480in}}{\pgfqpoint{9.359251in}{1.323480in}}%
\pgfpathclose%
\pgfusepath{stroke,fill}%
\end{pgfscope}%
\begin{pgfscope}%
\pgfpathrectangle{\pgfqpoint{0.481978in}{0.331635in}}{\pgfqpoint{9.300000in}{7.700000in}}%
\pgfusepath{clip}%
\pgfsetbuttcap%
\pgfsetroundjoin%
\definecolor{currentfill}{rgb}{1.000000,0.705882,0.509804}%
\pgfsetfillcolor{currentfill}%
\pgfsetlinewidth{0.481800pt}%
\definecolor{currentstroke}{rgb}{1.000000,1.000000,1.000000}%
\pgfsetstrokecolor{currentstroke}%
\pgfsetdash{}{0pt}%
\pgfpathmoveto{\pgfqpoint{3.311383in}{5.655787in}}%
\pgfpathcurveto{\pgfqpoint{3.322433in}{5.655787in}}{\pgfqpoint{3.333032in}{5.660177in}}{\pgfqpoint{3.340846in}{5.667990in}}%
\pgfpathcurveto{\pgfqpoint{3.348659in}{5.675804in}}{\pgfqpoint{3.353049in}{5.686403in}}{\pgfqpoint{3.353049in}{5.697453in}}%
\pgfpathcurveto{\pgfqpoint{3.353049in}{5.708503in}}{\pgfqpoint{3.348659in}{5.719102in}}{\pgfqpoint{3.340846in}{5.726916in}}%
\pgfpathcurveto{\pgfqpoint{3.333032in}{5.734730in}}{\pgfqpoint{3.322433in}{5.739120in}}{\pgfqpoint{3.311383in}{5.739120in}}%
\pgfpathcurveto{\pgfqpoint{3.300333in}{5.739120in}}{\pgfqpoint{3.289734in}{5.734730in}}{\pgfqpoint{3.281920in}{5.726916in}}%
\pgfpathcurveto{\pgfqpoint{3.274106in}{5.719102in}}{\pgfqpoint{3.269716in}{5.708503in}}{\pgfqpoint{3.269716in}{5.697453in}}%
\pgfpathcurveto{\pgfqpoint{3.269716in}{5.686403in}}{\pgfqpoint{3.274106in}{5.675804in}}{\pgfqpoint{3.281920in}{5.667990in}}%
\pgfpathcurveto{\pgfqpoint{3.289734in}{5.660177in}}{\pgfqpoint{3.300333in}{5.655787in}}{\pgfqpoint{3.311383in}{5.655787in}}%
\pgfpathclose%
\pgfusepath{stroke,fill}%
\end{pgfscope}%
\begin{pgfscope}%
\pgfpathrectangle{\pgfqpoint{0.481978in}{0.331635in}}{\pgfqpoint{9.300000in}{7.700000in}}%
\pgfusepath{clip}%
\pgfsetbuttcap%
\pgfsetroundjoin%
\definecolor{currentfill}{rgb}{1.000000,0.705882,0.509804}%
\pgfsetfillcolor{currentfill}%
\pgfsetlinewidth{0.481800pt}%
\definecolor{currentstroke}{rgb}{1.000000,1.000000,1.000000}%
\pgfsetstrokecolor{currentstroke}%
\pgfsetdash{}{0pt}%
\pgfpathmoveto{\pgfqpoint{3.768020in}{3.816422in}}%
\pgfpathcurveto{\pgfqpoint{3.779070in}{3.816422in}}{\pgfqpoint{3.789669in}{3.820813in}}{\pgfqpoint{3.797483in}{3.828626in}}%
\pgfpathcurveto{\pgfqpoint{3.805296in}{3.836440in}}{\pgfqpoint{3.809687in}{3.847039in}}{\pgfqpoint{3.809687in}{3.858089in}}%
\pgfpathcurveto{\pgfqpoint{3.809687in}{3.869139in}}{\pgfqpoint{3.805296in}{3.879738in}}{\pgfqpoint{3.797483in}{3.887552in}}%
\pgfpathcurveto{\pgfqpoint{3.789669in}{3.895365in}}{\pgfqpoint{3.779070in}{3.899756in}}{\pgfqpoint{3.768020in}{3.899756in}}%
\pgfpathcurveto{\pgfqpoint{3.756970in}{3.899756in}}{\pgfqpoint{3.746371in}{3.895365in}}{\pgfqpoint{3.738557in}{3.887552in}}%
\pgfpathcurveto{\pgfqpoint{3.730744in}{3.879738in}}{\pgfqpoint{3.726353in}{3.869139in}}{\pgfqpoint{3.726353in}{3.858089in}}%
\pgfpathcurveto{\pgfqpoint{3.726353in}{3.847039in}}{\pgfqpoint{3.730744in}{3.836440in}}{\pgfqpoint{3.738557in}{3.828626in}}%
\pgfpathcurveto{\pgfqpoint{3.746371in}{3.820813in}}{\pgfqpoint{3.756970in}{3.816422in}}{\pgfqpoint{3.768020in}{3.816422in}}%
\pgfpathclose%
\pgfusepath{stroke,fill}%
\end{pgfscope}%
\begin{pgfscope}%
\pgfpathrectangle{\pgfqpoint{0.481978in}{0.331635in}}{\pgfqpoint{9.300000in}{7.700000in}}%
\pgfusepath{clip}%
\pgfsetbuttcap%
\pgfsetroundjoin%
\definecolor{currentfill}{rgb}{1.000000,0.705882,0.509804}%
\pgfsetfillcolor{currentfill}%
\pgfsetlinewidth{0.481800pt}%
\definecolor{currentstroke}{rgb}{1.000000,1.000000,1.000000}%
\pgfsetstrokecolor{currentstroke}%
\pgfsetdash{}{0pt}%
\pgfpathmoveto{\pgfqpoint{3.827532in}{5.079630in}}%
\pgfpathcurveto{\pgfqpoint{3.838582in}{5.079630in}}{\pgfqpoint{3.849181in}{5.084020in}}{\pgfqpoint{3.856995in}{5.091834in}}%
\pgfpathcurveto{\pgfqpoint{3.864808in}{5.099647in}}{\pgfqpoint{3.869199in}{5.110246in}}{\pgfqpoint{3.869199in}{5.121296in}}%
\pgfpathcurveto{\pgfqpoint{3.869199in}{5.132346in}}{\pgfqpoint{3.864808in}{5.142945in}}{\pgfqpoint{3.856995in}{5.150759in}}%
\pgfpathcurveto{\pgfqpoint{3.849181in}{5.158573in}}{\pgfqpoint{3.838582in}{5.162963in}}{\pgfqpoint{3.827532in}{5.162963in}}%
\pgfpathcurveto{\pgfqpoint{3.816482in}{5.162963in}}{\pgfqpoint{3.805883in}{5.158573in}}{\pgfqpoint{3.798069in}{5.150759in}}%
\pgfpathcurveto{\pgfqpoint{3.790255in}{5.142945in}}{\pgfqpoint{3.785865in}{5.132346in}}{\pgfqpoint{3.785865in}{5.121296in}}%
\pgfpathcurveto{\pgfqpoint{3.785865in}{5.110246in}}{\pgfqpoint{3.790255in}{5.099647in}}{\pgfqpoint{3.798069in}{5.091834in}}%
\pgfpathcurveto{\pgfqpoint{3.805883in}{5.084020in}}{\pgfqpoint{3.816482in}{5.079630in}}{\pgfqpoint{3.827532in}{5.079630in}}%
\pgfpathclose%
\pgfusepath{stroke,fill}%
\end{pgfscope}%
\begin{pgfscope}%
\pgfpathrectangle{\pgfqpoint{0.481978in}{0.331635in}}{\pgfqpoint{9.300000in}{7.700000in}}%
\pgfusepath{clip}%
\pgfsetbuttcap%
\pgfsetroundjoin%
\definecolor{currentfill}{rgb}{1.000000,0.705882,0.509804}%
\pgfsetfillcolor{currentfill}%
\pgfsetlinewidth{0.481800pt}%
\definecolor{currentstroke}{rgb}{1.000000,1.000000,1.000000}%
\pgfsetstrokecolor{currentstroke}%
\pgfsetdash{}{0pt}%
\pgfpathmoveto{\pgfqpoint{1.059884in}{2.150352in}}%
\pgfpathcurveto{\pgfqpoint{1.070934in}{2.150352in}}{\pgfqpoint{1.081533in}{2.154743in}}{\pgfqpoint{1.089346in}{2.162556in}}%
\pgfpathcurveto{\pgfqpoint{1.097160in}{2.170370in}}{\pgfqpoint{1.101550in}{2.180969in}}{\pgfqpoint{1.101550in}{2.192019in}}%
\pgfpathcurveto{\pgfqpoint{1.101550in}{2.203069in}}{\pgfqpoint{1.097160in}{2.213668in}}{\pgfqpoint{1.089346in}{2.221482in}}%
\pgfpathcurveto{\pgfqpoint{1.081533in}{2.229295in}}{\pgfqpoint{1.070934in}{2.233686in}}{\pgfqpoint{1.059884in}{2.233686in}}%
\pgfpathcurveto{\pgfqpoint{1.048833in}{2.233686in}}{\pgfqpoint{1.038234in}{2.229295in}}{\pgfqpoint{1.030421in}{2.221482in}}%
\pgfpathcurveto{\pgfqpoint{1.022607in}{2.213668in}}{\pgfqpoint{1.018217in}{2.203069in}}{\pgfqpoint{1.018217in}{2.192019in}}%
\pgfpathcurveto{\pgfqpoint{1.018217in}{2.180969in}}{\pgfqpoint{1.022607in}{2.170370in}}{\pgfqpoint{1.030421in}{2.162556in}}%
\pgfpathcurveto{\pgfqpoint{1.038234in}{2.154743in}}{\pgfqpoint{1.048833in}{2.150352in}}{\pgfqpoint{1.059884in}{2.150352in}}%
\pgfpathclose%
\pgfusepath{stroke,fill}%
\end{pgfscope}%
\begin{pgfscope}%
\pgfpathrectangle{\pgfqpoint{0.481978in}{0.331635in}}{\pgfqpoint{9.300000in}{7.700000in}}%
\pgfusepath{clip}%
\pgfsetbuttcap%
\pgfsetroundjoin%
\definecolor{currentfill}{rgb}{1.000000,0.705882,0.509804}%
\pgfsetfillcolor{currentfill}%
\pgfsetlinewidth{0.481800pt}%
\definecolor{currentstroke}{rgb}{1.000000,1.000000,1.000000}%
\pgfsetstrokecolor{currentstroke}%
\pgfsetdash{}{0pt}%
\pgfpathmoveto{\pgfqpoint{4.209058in}{3.207223in}}%
\pgfpathcurveto{\pgfqpoint{4.220108in}{3.207223in}}{\pgfqpoint{4.230707in}{3.211614in}}{\pgfqpoint{4.238521in}{3.219427in}}%
\pgfpathcurveto{\pgfqpoint{4.246334in}{3.227241in}}{\pgfqpoint{4.250725in}{3.237840in}}{\pgfqpoint{4.250725in}{3.248890in}}%
\pgfpathcurveto{\pgfqpoint{4.250725in}{3.259940in}}{\pgfqpoint{4.246334in}{3.270539in}}{\pgfqpoint{4.238521in}{3.278353in}}%
\pgfpathcurveto{\pgfqpoint{4.230707in}{3.286166in}}{\pgfqpoint{4.220108in}{3.290557in}}{\pgfqpoint{4.209058in}{3.290557in}}%
\pgfpathcurveto{\pgfqpoint{4.198008in}{3.290557in}}{\pgfqpoint{4.187409in}{3.286166in}}{\pgfqpoint{4.179595in}{3.278353in}}%
\pgfpathcurveto{\pgfqpoint{4.171781in}{3.270539in}}{\pgfqpoint{4.167391in}{3.259940in}}{\pgfqpoint{4.167391in}{3.248890in}}%
\pgfpathcurveto{\pgfqpoint{4.167391in}{3.237840in}}{\pgfqpoint{4.171781in}{3.227241in}}{\pgfqpoint{4.179595in}{3.219427in}}%
\pgfpathcurveto{\pgfqpoint{4.187409in}{3.211614in}}{\pgfqpoint{4.198008in}{3.207223in}}{\pgfqpoint{4.209058in}{3.207223in}}%
\pgfpathclose%
\pgfusepath{stroke,fill}%
\end{pgfscope}%
\begin{pgfscope}%
\pgfpathrectangle{\pgfqpoint{0.481978in}{0.331635in}}{\pgfqpoint{9.300000in}{7.700000in}}%
\pgfusepath{clip}%
\pgfsetbuttcap%
\pgfsetroundjoin%
\definecolor{currentfill}{rgb}{1.000000,0.705882,0.509804}%
\pgfsetfillcolor{currentfill}%
\pgfsetlinewidth{0.481800pt}%
\definecolor{currentstroke}{rgb}{1.000000,1.000000,1.000000}%
\pgfsetstrokecolor{currentstroke}%
\pgfsetdash{}{0pt}%
\pgfpathmoveto{\pgfqpoint{3.072069in}{2.744087in}}%
\pgfpathcurveto{\pgfqpoint{3.083120in}{2.744087in}}{\pgfqpoint{3.093719in}{2.748477in}}{\pgfqpoint{3.101532in}{2.756291in}}%
\pgfpathcurveto{\pgfqpoint{3.109346in}{2.764104in}}{\pgfqpoint{3.113736in}{2.774703in}}{\pgfqpoint{3.113736in}{2.785754in}}%
\pgfpathcurveto{\pgfqpoint{3.113736in}{2.796804in}}{\pgfqpoint{3.109346in}{2.807403in}}{\pgfqpoint{3.101532in}{2.815216in}}%
\pgfpathcurveto{\pgfqpoint{3.093719in}{2.823030in}}{\pgfqpoint{3.083120in}{2.827420in}}{\pgfqpoint{3.072069in}{2.827420in}}%
\pgfpathcurveto{\pgfqpoint{3.061019in}{2.827420in}}{\pgfqpoint{3.050420in}{2.823030in}}{\pgfqpoint{3.042607in}{2.815216in}}%
\pgfpathcurveto{\pgfqpoint{3.034793in}{2.807403in}}{\pgfqpoint{3.030403in}{2.796804in}}{\pgfqpoint{3.030403in}{2.785754in}}%
\pgfpathcurveto{\pgfqpoint{3.030403in}{2.774703in}}{\pgfqpoint{3.034793in}{2.764104in}}{\pgfqpoint{3.042607in}{2.756291in}}%
\pgfpathcurveto{\pgfqpoint{3.050420in}{2.748477in}}{\pgfqpoint{3.061019in}{2.744087in}}{\pgfqpoint{3.072069in}{2.744087in}}%
\pgfpathclose%
\pgfusepath{stroke,fill}%
\end{pgfscope}%
\begin{pgfscope}%
\pgfpathrectangle{\pgfqpoint{0.481978in}{0.331635in}}{\pgfqpoint{9.300000in}{7.700000in}}%
\pgfusepath{clip}%
\pgfsetbuttcap%
\pgfsetroundjoin%
\definecolor{currentfill}{rgb}{1.000000,0.705882,0.509804}%
\pgfsetfillcolor{currentfill}%
\pgfsetlinewidth{0.481800pt}%
\definecolor{currentstroke}{rgb}{1.000000,1.000000,1.000000}%
\pgfsetstrokecolor{currentstroke}%
\pgfsetdash{}{0pt}%
\pgfpathmoveto{\pgfqpoint{4.167601in}{3.771028in}}%
\pgfpathcurveto{\pgfqpoint{4.178652in}{3.771028in}}{\pgfqpoint{4.189251in}{3.775418in}}{\pgfqpoint{4.197064in}{3.783232in}}%
\pgfpathcurveto{\pgfqpoint{4.204878in}{3.791046in}}{\pgfqpoint{4.209268in}{3.801645in}}{\pgfqpoint{4.209268in}{3.812695in}}%
\pgfpathcurveto{\pgfqpoint{4.209268in}{3.823745in}}{\pgfqpoint{4.204878in}{3.834344in}}{\pgfqpoint{4.197064in}{3.842158in}}%
\pgfpathcurveto{\pgfqpoint{4.189251in}{3.849971in}}{\pgfqpoint{4.178652in}{3.854361in}}{\pgfqpoint{4.167601in}{3.854361in}}%
\pgfpathcurveto{\pgfqpoint{4.156551in}{3.854361in}}{\pgfqpoint{4.145952in}{3.849971in}}{\pgfqpoint{4.138139in}{3.842158in}}%
\pgfpathcurveto{\pgfqpoint{4.130325in}{3.834344in}}{\pgfqpoint{4.125935in}{3.823745in}}{\pgfqpoint{4.125935in}{3.812695in}}%
\pgfpathcurveto{\pgfqpoint{4.125935in}{3.801645in}}{\pgfqpoint{4.130325in}{3.791046in}}{\pgfqpoint{4.138139in}{3.783232in}}%
\pgfpathcurveto{\pgfqpoint{4.145952in}{3.775418in}}{\pgfqpoint{4.156551in}{3.771028in}}{\pgfqpoint{4.167601in}{3.771028in}}%
\pgfpathclose%
\pgfusepath{stroke,fill}%
\end{pgfscope}%
\begin{pgfscope}%
\pgfpathrectangle{\pgfqpoint{0.481978in}{0.331635in}}{\pgfqpoint{9.300000in}{7.700000in}}%
\pgfusepath{clip}%
\pgfsetbuttcap%
\pgfsetroundjoin%
\definecolor{currentfill}{rgb}{1.000000,0.705882,0.509804}%
\pgfsetfillcolor{currentfill}%
\pgfsetlinewidth{0.481800pt}%
\definecolor{currentstroke}{rgb}{1.000000,1.000000,1.000000}%
\pgfsetstrokecolor{currentstroke}%
\pgfsetdash{}{0pt}%
\pgfpathmoveto{\pgfqpoint{4.458251in}{2.508756in}}%
\pgfpathcurveto{\pgfqpoint{4.469301in}{2.508756in}}{\pgfqpoint{4.479900in}{2.513146in}}{\pgfqpoint{4.487713in}{2.520960in}}%
\pgfpathcurveto{\pgfqpoint{4.495527in}{2.528774in}}{\pgfqpoint{4.499917in}{2.539373in}}{\pgfqpoint{4.499917in}{2.550423in}}%
\pgfpathcurveto{\pgfqpoint{4.499917in}{2.561473in}}{\pgfqpoint{4.495527in}{2.572072in}}{\pgfqpoint{4.487713in}{2.579886in}}%
\pgfpathcurveto{\pgfqpoint{4.479900in}{2.587699in}}{\pgfqpoint{4.469301in}{2.592090in}}{\pgfqpoint{4.458251in}{2.592090in}}%
\pgfpathcurveto{\pgfqpoint{4.447201in}{2.592090in}}{\pgfqpoint{4.436602in}{2.587699in}}{\pgfqpoint{4.428788in}{2.579886in}}%
\pgfpathcurveto{\pgfqpoint{4.420974in}{2.572072in}}{\pgfqpoint{4.416584in}{2.561473in}}{\pgfqpoint{4.416584in}{2.550423in}}%
\pgfpathcurveto{\pgfqpoint{4.416584in}{2.539373in}}{\pgfqpoint{4.420974in}{2.528774in}}{\pgfqpoint{4.428788in}{2.520960in}}%
\pgfpathcurveto{\pgfqpoint{4.436602in}{2.513146in}}{\pgfqpoint{4.447201in}{2.508756in}}{\pgfqpoint{4.458251in}{2.508756in}}%
\pgfpathclose%
\pgfusepath{stroke,fill}%
\end{pgfscope}%
\begin{pgfscope}%
\pgfpathrectangle{\pgfqpoint{0.481978in}{0.331635in}}{\pgfqpoint{9.300000in}{7.700000in}}%
\pgfusepath{clip}%
\pgfsetbuttcap%
\pgfsetroundjoin%
\definecolor{currentfill}{rgb}{1.000000,0.705882,0.509804}%
\pgfsetfillcolor{currentfill}%
\pgfsetlinewidth{0.481800pt}%
\definecolor{currentstroke}{rgb}{1.000000,1.000000,1.000000}%
\pgfsetstrokecolor{currentstroke}%
\pgfsetdash{}{0pt}%
\pgfpathmoveto{\pgfqpoint{3.705775in}{4.566842in}}%
\pgfpathcurveto{\pgfqpoint{3.716825in}{4.566842in}}{\pgfqpoint{3.727424in}{4.571233in}}{\pgfqpoint{3.735238in}{4.579046in}}%
\pgfpathcurveto{\pgfqpoint{3.743052in}{4.586860in}}{\pgfqpoint{3.747442in}{4.597459in}}{\pgfqpoint{3.747442in}{4.608509in}}%
\pgfpathcurveto{\pgfqpoint{3.747442in}{4.619559in}}{\pgfqpoint{3.743052in}{4.630158in}}{\pgfqpoint{3.735238in}{4.637972in}}%
\pgfpathcurveto{\pgfqpoint{3.727424in}{4.645785in}}{\pgfqpoint{3.716825in}{4.650176in}}{\pgfqpoint{3.705775in}{4.650176in}}%
\pgfpathcurveto{\pgfqpoint{3.694725in}{4.650176in}}{\pgfqpoint{3.684126in}{4.645785in}}{\pgfqpoint{3.676313in}{4.637972in}}%
\pgfpathcurveto{\pgfqpoint{3.668499in}{4.630158in}}{\pgfqpoint{3.664109in}{4.619559in}}{\pgfqpoint{3.664109in}{4.608509in}}%
\pgfpathcurveto{\pgfqpoint{3.664109in}{4.597459in}}{\pgfqpoint{3.668499in}{4.586860in}}{\pgfqpoint{3.676313in}{4.579046in}}%
\pgfpathcurveto{\pgfqpoint{3.684126in}{4.571233in}}{\pgfqpoint{3.694725in}{4.566842in}}{\pgfqpoint{3.705775in}{4.566842in}}%
\pgfpathclose%
\pgfusepath{stroke,fill}%
\end{pgfscope}%
\begin{pgfscope}%
\pgfpathrectangle{\pgfqpoint{0.481978in}{0.331635in}}{\pgfqpoint{9.300000in}{7.700000in}}%
\pgfusepath{clip}%
\pgfsetbuttcap%
\pgfsetroundjoin%
\definecolor{currentfill}{rgb}{1.000000,0.705882,0.509804}%
\pgfsetfillcolor{currentfill}%
\pgfsetlinewidth{0.481800pt}%
\definecolor{currentstroke}{rgb}{1.000000,1.000000,1.000000}%
\pgfsetstrokecolor{currentstroke}%
\pgfsetdash{}{0pt}%
\pgfpathmoveto{\pgfqpoint{3.840708in}{2.812768in}}%
\pgfpathcurveto{\pgfqpoint{3.851758in}{2.812768in}}{\pgfqpoint{3.862357in}{2.817158in}}{\pgfqpoint{3.870171in}{2.824972in}}%
\pgfpathcurveto{\pgfqpoint{3.877985in}{2.832785in}}{\pgfqpoint{3.882375in}{2.843385in}}{\pgfqpoint{3.882375in}{2.854435in}}%
\pgfpathcurveto{\pgfqpoint{3.882375in}{2.865485in}}{\pgfqpoint{3.877985in}{2.876084in}}{\pgfqpoint{3.870171in}{2.883897in}}%
\pgfpathcurveto{\pgfqpoint{3.862357in}{2.891711in}}{\pgfqpoint{3.851758in}{2.896101in}}{\pgfqpoint{3.840708in}{2.896101in}}%
\pgfpathcurveto{\pgfqpoint{3.829658in}{2.896101in}}{\pgfqpoint{3.819059in}{2.891711in}}{\pgfqpoint{3.811245in}{2.883897in}}%
\pgfpathcurveto{\pgfqpoint{3.803432in}{2.876084in}}{\pgfqpoint{3.799042in}{2.865485in}}{\pgfqpoint{3.799042in}{2.854435in}}%
\pgfpathcurveto{\pgfqpoint{3.799042in}{2.843385in}}{\pgfqpoint{3.803432in}{2.832785in}}{\pgfqpoint{3.811245in}{2.824972in}}%
\pgfpathcurveto{\pgfqpoint{3.819059in}{2.817158in}}{\pgfqpoint{3.829658in}{2.812768in}}{\pgfqpoint{3.840708in}{2.812768in}}%
\pgfpathclose%
\pgfusepath{stroke,fill}%
\end{pgfscope}%
\begin{pgfscope}%
\pgfpathrectangle{\pgfqpoint{0.481978in}{0.331635in}}{\pgfqpoint{9.300000in}{7.700000in}}%
\pgfusepath{clip}%
\pgfsetbuttcap%
\pgfsetroundjoin%
\definecolor{currentfill}{rgb}{1.000000,0.705882,0.509804}%
\pgfsetfillcolor{currentfill}%
\pgfsetlinewidth{0.481800pt}%
\definecolor{currentstroke}{rgb}{1.000000,1.000000,1.000000}%
\pgfsetstrokecolor{currentstroke}%
\pgfsetdash{}{0pt}%
\pgfpathmoveto{\pgfqpoint{2.841012in}{4.126690in}}%
\pgfpathcurveto{\pgfqpoint{2.852062in}{4.126690in}}{\pgfqpoint{2.862661in}{4.131080in}}{\pgfqpoint{2.870475in}{4.138894in}}%
\pgfpathcurveto{\pgfqpoint{2.878288in}{4.146708in}}{\pgfqpoint{2.882679in}{4.157307in}}{\pgfqpoint{2.882679in}{4.168357in}}%
\pgfpathcurveto{\pgfqpoint{2.882679in}{4.179407in}}{\pgfqpoint{2.878288in}{4.190006in}}{\pgfqpoint{2.870475in}{4.197819in}}%
\pgfpathcurveto{\pgfqpoint{2.862661in}{4.205633in}}{\pgfqpoint{2.852062in}{4.210023in}}{\pgfqpoint{2.841012in}{4.210023in}}%
\pgfpathcurveto{\pgfqpoint{2.829962in}{4.210023in}}{\pgfqpoint{2.819363in}{4.205633in}}{\pgfqpoint{2.811549in}{4.197819in}}%
\pgfpathcurveto{\pgfqpoint{2.803736in}{4.190006in}}{\pgfqpoint{2.799345in}{4.179407in}}{\pgfqpoint{2.799345in}{4.168357in}}%
\pgfpathcurveto{\pgfqpoint{2.799345in}{4.157307in}}{\pgfqpoint{2.803736in}{4.146708in}}{\pgfqpoint{2.811549in}{4.138894in}}%
\pgfpathcurveto{\pgfqpoint{2.819363in}{4.131080in}}{\pgfqpoint{2.829962in}{4.126690in}}{\pgfqpoint{2.841012in}{4.126690in}}%
\pgfpathclose%
\pgfusepath{stroke,fill}%
\end{pgfscope}%
\begin{pgfscope}%
\pgfpathrectangle{\pgfqpoint{0.481978in}{0.331635in}}{\pgfqpoint{9.300000in}{7.700000in}}%
\pgfusepath{clip}%
\pgfsetbuttcap%
\pgfsetroundjoin%
\definecolor{currentfill}{rgb}{1.000000,0.705882,0.509804}%
\pgfsetfillcolor{currentfill}%
\pgfsetlinewidth{0.481800pt}%
\definecolor{currentstroke}{rgb}{1.000000,1.000000,1.000000}%
\pgfsetstrokecolor{currentstroke}%
\pgfsetdash{}{0pt}%
\pgfpathmoveto{\pgfqpoint{2.668476in}{1.607002in}}%
\pgfpathcurveto{\pgfqpoint{2.679526in}{1.607002in}}{\pgfqpoint{2.690125in}{1.611392in}}{\pgfqpoint{2.697939in}{1.619206in}}%
\pgfpathcurveto{\pgfqpoint{2.705752in}{1.627020in}}{\pgfqpoint{2.710143in}{1.637619in}}{\pgfqpoint{2.710143in}{1.648669in}}%
\pgfpathcurveto{\pgfqpoint{2.710143in}{1.659719in}}{\pgfqpoint{2.705752in}{1.670318in}}{\pgfqpoint{2.697939in}{1.678131in}}%
\pgfpathcurveto{\pgfqpoint{2.690125in}{1.685945in}}{\pgfqpoint{2.679526in}{1.690335in}}{\pgfqpoint{2.668476in}{1.690335in}}%
\pgfpathcurveto{\pgfqpoint{2.657426in}{1.690335in}}{\pgfqpoint{2.646827in}{1.685945in}}{\pgfqpoint{2.639013in}{1.678131in}}%
\pgfpathcurveto{\pgfqpoint{2.631199in}{1.670318in}}{\pgfqpoint{2.626809in}{1.659719in}}{\pgfqpoint{2.626809in}{1.648669in}}%
\pgfpathcurveto{\pgfqpoint{2.626809in}{1.637619in}}{\pgfqpoint{2.631199in}{1.627020in}}{\pgfqpoint{2.639013in}{1.619206in}}%
\pgfpathcurveto{\pgfqpoint{2.646827in}{1.611392in}}{\pgfqpoint{2.657426in}{1.607002in}}{\pgfqpoint{2.668476in}{1.607002in}}%
\pgfpathclose%
\pgfusepath{stroke,fill}%
\end{pgfscope}%
\begin{pgfscope}%
\pgfpathrectangle{\pgfqpoint{0.481978in}{0.331635in}}{\pgfqpoint{9.300000in}{7.700000in}}%
\pgfusepath{clip}%
\pgfsetbuttcap%
\pgfsetroundjoin%
\definecolor{currentfill}{rgb}{1.000000,0.705882,0.509804}%
\pgfsetfillcolor{currentfill}%
\pgfsetlinewidth{0.481800pt}%
\definecolor{currentstroke}{rgb}{1.000000,1.000000,1.000000}%
\pgfsetstrokecolor{currentstroke}%
\pgfsetdash{}{0pt}%
\pgfpathmoveto{\pgfqpoint{3.108258in}{5.387808in}}%
\pgfpathcurveto{\pgfqpoint{3.119308in}{5.387808in}}{\pgfqpoint{3.129907in}{5.392198in}}{\pgfqpoint{3.137720in}{5.400012in}}%
\pgfpathcurveto{\pgfqpoint{3.145534in}{5.407826in}}{\pgfqpoint{3.149924in}{5.418425in}}{\pgfqpoint{3.149924in}{5.429475in}}%
\pgfpathcurveto{\pgfqpoint{3.149924in}{5.440525in}}{\pgfqpoint{3.145534in}{5.451124in}}{\pgfqpoint{3.137720in}{5.458938in}}%
\pgfpathcurveto{\pgfqpoint{3.129907in}{5.466751in}}{\pgfqpoint{3.119308in}{5.471142in}}{\pgfqpoint{3.108258in}{5.471142in}}%
\pgfpathcurveto{\pgfqpoint{3.097207in}{5.471142in}}{\pgfqpoint{3.086608in}{5.466751in}}{\pgfqpoint{3.078795in}{5.458938in}}%
\pgfpathcurveto{\pgfqpoint{3.070981in}{5.451124in}}{\pgfqpoint{3.066591in}{5.440525in}}{\pgfqpoint{3.066591in}{5.429475in}}%
\pgfpathcurveto{\pgfqpoint{3.066591in}{5.418425in}}{\pgfqpoint{3.070981in}{5.407826in}}{\pgfqpoint{3.078795in}{5.400012in}}%
\pgfpathcurveto{\pgfqpoint{3.086608in}{5.392198in}}{\pgfqpoint{3.097207in}{5.387808in}}{\pgfqpoint{3.108258in}{5.387808in}}%
\pgfpathclose%
\pgfusepath{stroke,fill}%
\end{pgfscope}%
\begin{pgfscope}%
\pgfpathrectangle{\pgfqpoint{0.481978in}{0.331635in}}{\pgfqpoint{9.300000in}{7.700000in}}%
\pgfusepath{clip}%
\pgfsetbuttcap%
\pgfsetroundjoin%
\definecolor{currentfill}{rgb}{1.000000,0.705882,0.509804}%
\pgfsetfillcolor{currentfill}%
\pgfsetlinewidth{0.481800pt}%
\definecolor{currentstroke}{rgb}{1.000000,1.000000,1.000000}%
\pgfsetstrokecolor{currentstroke}%
\pgfsetdash{}{0pt}%
\pgfpathmoveto{\pgfqpoint{1.529852in}{4.755458in}}%
\pgfpathcurveto{\pgfqpoint{1.540902in}{4.755458in}}{\pgfqpoint{1.551501in}{4.759848in}}{\pgfqpoint{1.559315in}{4.767662in}}%
\pgfpathcurveto{\pgfqpoint{1.567128in}{4.775476in}}{\pgfqpoint{1.571519in}{4.786075in}}{\pgfqpoint{1.571519in}{4.797125in}}%
\pgfpathcurveto{\pgfqpoint{1.571519in}{4.808175in}}{\pgfqpoint{1.567128in}{4.818774in}}{\pgfqpoint{1.559315in}{4.826588in}}%
\pgfpathcurveto{\pgfqpoint{1.551501in}{4.834401in}}{\pgfqpoint{1.540902in}{4.838791in}}{\pgfqpoint{1.529852in}{4.838791in}}%
\pgfpathcurveto{\pgfqpoint{1.518802in}{4.838791in}}{\pgfqpoint{1.508203in}{4.834401in}}{\pgfqpoint{1.500389in}{4.826588in}}%
\pgfpathcurveto{\pgfqpoint{1.492576in}{4.818774in}}{\pgfqpoint{1.488185in}{4.808175in}}{\pgfqpoint{1.488185in}{4.797125in}}%
\pgfpathcurveto{\pgfqpoint{1.488185in}{4.786075in}}{\pgfqpoint{1.492576in}{4.775476in}}{\pgfqpoint{1.500389in}{4.767662in}}%
\pgfpathcurveto{\pgfqpoint{1.508203in}{4.759848in}}{\pgfqpoint{1.518802in}{4.755458in}}{\pgfqpoint{1.529852in}{4.755458in}}%
\pgfpathclose%
\pgfusepath{stroke,fill}%
\end{pgfscope}%
\begin{pgfscope}%
\pgfpathrectangle{\pgfqpoint{0.481978in}{0.331635in}}{\pgfqpoint{9.300000in}{7.700000in}}%
\pgfusepath{clip}%
\pgfsetbuttcap%
\pgfsetroundjoin%
\definecolor{currentfill}{rgb}{1.000000,0.705882,0.509804}%
\pgfsetfillcolor{currentfill}%
\pgfsetlinewidth{0.481800pt}%
\definecolor{currentstroke}{rgb}{1.000000,1.000000,1.000000}%
\pgfsetstrokecolor{currentstroke}%
\pgfsetdash{}{0pt}%
\pgfpathmoveto{\pgfqpoint{4.765650in}{4.912183in}}%
\pgfpathcurveto{\pgfqpoint{4.776700in}{4.912183in}}{\pgfqpoint{4.787299in}{4.916573in}}{\pgfqpoint{4.795113in}{4.924387in}}%
\pgfpathcurveto{\pgfqpoint{4.802926in}{4.932201in}}{\pgfqpoint{4.807316in}{4.942800in}}{\pgfqpoint{4.807316in}{4.953850in}}%
\pgfpathcurveto{\pgfqpoint{4.807316in}{4.964900in}}{\pgfqpoint{4.802926in}{4.975499in}}{\pgfqpoint{4.795113in}{4.983313in}}%
\pgfpathcurveto{\pgfqpoint{4.787299in}{4.991126in}}{\pgfqpoint{4.776700in}{4.995516in}}{\pgfqpoint{4.765650in}{4.995516in}}%
\pgfpathcurveto{\pgfqpoint{4.754600in}{4.995516in}}{\pgfqpoint{4.744001in}{4.991126in}}{\pgfqpoint{4.736187in}{4.983313in}}%
\pgfpathcurveto{\pgfqpoint{4.728373in}{4.975499in}}{\pgfqpoint{4.723983in}{4.964900in}}{\pgfqpoint{4.723983in}{4.953850in}}%
\pgfpathcurveto{\pgfqpoint{4.723983in}{4.942800in}}{\pgfqpoint{4.728373in}{4.932201in}}{\pgfqpoint{4.736187in}{4.924387in}}%
\pgfpathcurveto{\pgfqpoint{4.744001in}{4.916573in}}{\pgfqpoint{4.754600in}{4.912183in}}{\pgfqpoint{4.765650in}{4.912183in}}%
\pgfpathclose%
\pgfusepath{stroke,fill}%
\end{pgfscope}%
\begin{pgfscope}%
\pgfpathrectangle{\pgfqpoint{0.481978in}{0.331635in}}{\pgfqpoint{9.300000in}{7.700000in}}%
\pgfusepath{clip}%
\pgfsetbuttcap%
\pgfsetroundjoin%
\definecolor{currentfill}{rgb}{1.000000,0.705882,0.509804}%
\pgfsetfillcolor{currentfill}%
\pgfsetlinewidth{0.481800pt}%
\definecolor{currentstroke}{rgb}{1.000000,1.000000,1.000000}%
\pgfsetstrokecolor{currentstroke}%
\pgfsetdash{}{0pt}%
\pgfpathmoveto{\pgfqpoint{1.272867in}{4.296402in}}%
\pgfpathcurveto{\pgfqpoint{1.283917in}{4.296402in}}{\pgfqpoint{1.294516in}{4.300793in}}{\pgfqpoint{1.302330in}{4.308606in}}%
\pgfpathcurveto{\pgfqpoint{1.310144in}{4.316420in}}{\pgfqpoint{1.314534in}{4.327019in}}{\pgfqpoint{1.314534in}{4.338069in}}%
\pgfpathcurveto{\pgfqpoint{1.314534in}{4.349119in}}{\pgfqpoint{1.310144in}{4.359718in}}{\pgfqpoint{1.302330in}{4.367532in}}%
\pgfpathcurveto{\pgfqpoint{1.294516in}{4.375346in}}{\pgfqpoint{1.283917in}{4.379736in}}{\pgfqpoint{1.272867in}{4.379736in}}%
\pgfpathcurveto{\pgfqpoint{1.261817in}{4.379736in}}{\pgfqpoint{1.251218in}{4.375346in}}{\pgfqpoint{1.243405in}{4.367532in}}%
\pgfpathcurveto{\pgfqpoint{1.235591in}{4.359718in}}{\pgfqpoint{1.231201in}{4.349119in}}{\pgfqpoint{1.231201in}{4.338069in}}%
\pgfpathcurveto{\pgfqpoint{1.231201in}{4.327019in}}{\pgfqpoint{1.235591in}{4.316420in}}{\pgfqpoint{1.243405in}{4.308606in}}%
\pgfpathcurveto{\pgfqpoint{1.251218in}{4.300793in}}{\pgfqpoint{1.261817in}{4.296402in}}{\pgfqpoint{1.272867in}{4.296402in}}%
\pgfpathclose%
\pgfusepath{stroke,fill}%
\end{pgfscope}%
\begin{pgfscope}%
\pgfpathrectangle{\pgfqpoint{0.481978in}{0.331635in}}{\pgfqpoint{9.300000in}{7.700000in}}%
\pgfusepath{clip}%
\pgfsetbuttcap%
\pgfsetroundjoin%
\definecolor{currentfill}{rgb}{1.000000,0.705882,0.509804}%
\pgfsetfillcolor{currentfill}%
\pgfsetlinewidth{0.481800pt}%
\definecolor{currentstroke}{rgb}{1.000000,1.000000,1.000000}%
\pgfsetstrokecolor{currentstroke}%
\pgfsetdash{}{0pt}%
\pgfpathmoveto{\pgfqpoint{3.381281in}{4.075176in}}%
\pgfpathcurveto{\pgfqpoint{3.392331in}{4.075176in}}{\pgfqpoint{3.402930in}{4.079567in}}{\pgfqpoint{3.410743in}{4.087380in}}%
\pgfpathcurveto{\pgfqpoint{3.418557in}{4.095194in}}{\pgfqpoint{3.422947in}{4.105793in}}{\pgfqpoint{3.422947in}{4.116843in}}%
\pgfpathcurveto{\pgfqpoint{3.422947in}{4.127893in}}{\pgfqpoint{3.418557in}{4.138492in}}{\pgfqpoint{3.410743in}{4.146306in}}%
\pgfpathcurveto{\pgfqpoint{3.402930in}{4.154119in}}{\pgfqpoint{3.392331in}{4.158510in}}{\pgfqpoint{3.381281in}{4.158510in}}%
\pgfpathcurveto{\pgfqpoint{3.370230in}{4.158510in}}{\pgfqpoint{3.359631in}{4.154119in}}{\pgfqpoint{3.351818in}{4.146306in}}%
\pgfpathcurveto{\pgfqpoint{3.344004in}{4.138492in}}{\pgfqpoint{3.339614in}{4.127893in}}{\pgfqpoint{3.339614in}{4.116843in}}%
\pgfpathcurveto{\pgfqpoint{3.339614in}{4.105793in}}{\pgfqpoint{3.344004in}{4.095194in}}{\pgfqpoint{3.351818in}{4.087380in}}%
\pgfpathcurveto{\pgfqpoint{3.359631in}{4.079567in}}{\pgfqpoint{3.370230in}{4.075176in}}{\pgfqpoint{3.381281in}{4.075176in}}%
\pgfpathclose%
\pgfusepath{stroke,fill}%
\end{pgfscope}%
\begin{pgfscope}%
\pgfpathrectangle{\pgfqpoint{0.481978in}{0.331635in}}{\pgfqpoint{9.300000in}{7.700000in}}%
\pgfusepath{clip}%
\pgfsetbuttcap%
\pgfsetroundjoin%
\definecolor{currentfill}{rgb}{1.000000,0.705882,0.509804}%
\pgfsetfillcolor{currentfill}%
\pgfsetlinewidth{0.481800pt}%
\definecolor{currentstroke}{rgb}{1.000000,1.000000,1.000000}%
\pgfsetstrokecolor{currentstroke}%
\pgfsetdash{}{0pt}%
\pgfpathmoveto{\pgfqpoint{4.049596in}{2.490832in}}%
\pgfpathcurveto{\pgfqpoint{4.060646in}{2.490832in}}{\pgfqpoint{4.071245in}{2.495222in}}{\pgfqpoint{4.079059in}{2.503036in}}%
\pgfpathcurveto{\pgfqpoint{4.086873in}{2.510850in}}{\pgfqpoint{4.091263in}{2.521449in}}{\pgfqpoint{4.091263in}{2.532499in}}%
\pgfpathcurveto{\pgfqpoint{4.091263in}{2.543549in}}{\pgfqpoint{4.086873in}{2.554148in}}{\pgfqpoint{4.079059in}{2.561961in}}%
\pgfpathcurveto{\pgfqpoint{4.071245in}{2.569775in}}{\pgfqpoint{4.060646in}{2.574165in}}{\pgfqpoint{4.049596in}{2.574165in}}%
\pgfpathcurveto{\pgfqpoint{4.038546in}{2.574165in}}{\pgfqpoint{4.027947in}{2.569775in}}{\pgfqpoint{4.020133in}{2.561961in}}%
\pgfpathcurveto{\pgfqpoint{4.012320in}{2.554148in}}{\pgfqpoint{4.007929in}{2.543549in}}{\pgfqpoint{4.007929in}{2.532499in}}%
\pgfpathcurveto{\pgfqpoint{4.007929in}{2.521449in}}{\pgfqpoint{4.012320in}{2.510850in}}{\pgfqpoint{4.020133in}{2.503036in}}%
\pgfpathcurveto{\pgfqpoint{4.027947in}{2.495222in}}{\pgfqpoint{4.038546in}{2.490832in}}{\pgfqpoint{4.049596in}{2.490832in}}%
\pgfpathclose%
\pgfusepath{stroke,fill}%
\end{pgfscope}%
\begin{pgfscope}%
\pgfpathrectangle{\pgfqpoint{0.481978in}{0.331635in}}{\pgfqpoint{9.300000in}{7.700000in}}%
\pgfusepath{clip}%
\pgfsetbuttcap%
\pgfsetroundjoin%
\definecolor{currentfill}{rgb}{1.000000,0.705882,0.509804}%
\pgfsetfillcolor{currentfill}%
\pgfsetlinewidth{0.481800pt}%
\definecolor{currentstroke}{rgb}{1.000000,1.000000,1.000000}%
\pgfsetstrokecolor{currentstroke}%
\pgfsetdash{}{0pt}%
\pgfpathmoveto{\pgfqpoint{3.917035in}{4.780865in}}%
\pgfpathcurveto{\pgfqpoint{3.928085in}{4.780865in}}{\pgfqpoint{3.938684in}{4.785255in}}{\pgfqpoint{3.946497in}{4.793069in}}%
\pgfpathcurveto{\pgfqpoint{3.954311in}{4.800883in}}{\pgfqpoint{3.958701in}{4.811482in}}{\pgfqpoint{3.958701in}{4.822532in}}%
\pgfpathcurveto{\pgfqpoint{3.958701in}{4.833582in}}{\pgfqpoint{3.954311in}{4.844181in}}{\pgfqpoint{3.946497in}{4.851995in}}%
\pgfpathcurveto{\pgfqpoint{3.938684in}{4.859808in}}{\pgfqpoint{3.928085in}{4.864199in}}{\pgfqpoint{3.917035in}{4.864199in}}%
\pgfpathcurveto{\pgfqpoint{3.905984in}{4.864199in}}{\pgfqpoint{3.895385in}{4.859808in}}{\pgfqpoint{3.887572in}{4.851995in}}%
\pgfpathcurveto{\pgfqpoint{3.879758in}{4.844181in}}{\pgfqpoint{3.875368in}{4.833582in}}{\pgfqpoint{3.875368in}{4.822532in}}%
\pgfpathcurveto{\pgfqpoint{3.875368in}{4.811482in}}{\pgfqpoint{3.879758in}{4.800883in}}{\pgfqpoint{3.887572in}{4.793069in}}%
\pgfpathcurveto{\pgfqpoint{3.895385in}{4.785255in}}{\pgfqpoint{3.905984in}{4.780865in}}{\pgfqpoint{3.917035in}{4.780865in}}%
\pgfpathclose%
\pgfusepath{stroke,fill}%
\end{pgfscope}%
\begin{pgfscope}%
\pgfpathrectangle{\pgfqpoint{0.481978in}{0.331635in}}{\pgfqpoint{9.300000in}{7.700000in}}%
\pgfusepath{clip}%
\pgfsetbuttcap%
\pgfsetroundjoin%
\definecolor{currentfill}{rgb}{1.000000,0.705882,0.509804}%
\pgfsetfillcolor{currentfill}%
\pgfsetlinewidth{0.481800pt}%
\definecolor{currentstroke}{rgb}{1.000000,1.000000,1.000000}%
\pgfsetstrokecolor{currentstroke}%
\pgfsetdash{}{0pt}%
\pgfpathmoveto{\pgfqpoint{2.834471in}{5.928469in}}%
\pgfpathcurveto{\pgfqpoint{2.845521in}{5.928469in}}{\pgfqpoint{2.856120in}{5.932859in}}{\pgfqpoint{2.863933in}{5.940673in}}%
\pgfpathcurveto{\pgfqpoint{2.871747in}{5.948486in}}{\pgfqpoint{2.876137in}{5.959086in}}{\pgfqpoint{2.876137in}{5.970136in}}%
\pgfpathcurveto{\pgfqpoint{2.876137in}{5.981186in}}{\pgfqpoint{2.871747in}{5.991785in}}{\pgfqpoint{2.863933in}{5.999598in}}%
\pgfpathcurveto{\pgfqpoint{2.856120in}{6.007412in}}{\pgfqpoint{2.845521in}{6.011802in}}{\pgfqpoint{2.834471in}{6.011802in}}%
\pgfpathcurveto{\pgfqpoint{2.823420in}{6.011802in}}{\pgfqpoint{2.812821in}{6.007412in}}{\pgfqpoint{2.805008in}{5.999598in}}%
\pgfpathcurveto{\pgfqpoint{2.797194in}{5.991785in}}{\pgfqpoint{2.792804in}{5.981186in}}{\pgfqpoint{2.792804in}{5.970136in}}%
\pgfpathcurveto{\pgfqpoint{2.792804in}{5.959086in}}{\pgfqpoint{2.797194in}{5.948486in}}{\pgfqpoint{2.805008in}{5.940673in}}%
\pgfpathcurveto{\pgfqpoint{2.812821in}{5.932859in}}{\pgfqpoint{2.823420in}{5.928469in}}{\pgfqpoint{2.834471in}{5.928469in}}%
\pgfpathclose%
\pgfusepath{stroke,fill}%
\end{pgfscope}%
\begin{pgfscope}%
\pgfpathrectangle{\pgfqpoint{0.481978in}{0.331635in}}{\pgfqpoint{9.300000in}{7.700000in}}%
\pgfusepath{clip}%
\pgfsetbuttcap%
\pgfsetroundjoin%
\definecolor{currentfill}{rgb}{1.000000,0.705882,0.509804}%
\pgfsetfillcolor{currentfill}%
\pgfsetlinewidth{0.481800pt}%
\definecolor{currentstroke}{rgb}{1.000000,1.000000,1.000000}%
\pgfsetstrokecolor{currentstroke}%
\pgfsetdash{}{0pt}%
\pgfpathmoveto{\pgfqpoint{1.856829in}{3.973348in}}%
\pgfpathcurveto{\pgfqpoint{1.867879in}{3.973348in}}{\pgfqpoint{1.878478in}{3.977738in}}{\pgfqpoint{1.886291in}{3.985552in}}%
\pgfpathcurveto{\pgfqpoint{1.894105in}{3.993366in}}{\pgfqpoint{1.898495in}{4.003965in}}{\pgfqpoint{1.898495in}{4.015015in}}%
\pgfpathcurveto{\pgfqpoint{1.898495in}{4.026065in}}{\pgfqpoint{1.894105in}{4.036664in}}{\pgfqpoint{1.886291in}{4.044478in}}%
\pgfpathcurveto{\pgfqpoint{1.878478in}{4.052291in}}{\pgfqpoint{1.867879in}{4.056681in}}{\pgfqpoint{1.856829in}{4.056681in}}%
\pgfpathcurveto{\pgfqpoint{1.845778in}{4.056681in}}{\pgfqpoint{1.835179in}{4.052291in}}{\pgfqpoint{1.827366in}{4.044478in}}%
\pgfpathcurveto{\pgfqpoint{1.819552in}{4.036664in}}{\pgfqpoint{1.815162in}{4.026065in}}{\pgfqpoint{1.815162in}{4.015015in}}%
\pgfpathcurveto{\pgfqpoint{1.815162in}{4.003965in}}{\pgfqpoint{1.819552in}{3.993366in}}{\pgfqpoint{1.827366in}{3.985552in}}%
\pgfpathcurveto{\pgfqpoint{1.835179in}{3.977738in}}{\pgfqpoint{1.845778in}{3.973348in}}{\pgfqpoint{1.856829in}{3.973348in}}%
\pgfpathclose%
\pgfusepath{stroke,fill}%
\end{pgfscope}%
\begin{pgfscope}%
\pgfpathrectangle{\pgfqpoint{0.481978in}{0.331635in}}{\pgfqpoint{9.300000in}{7.700000in}}%
\pgfusepath{clip}%
\pgfsetbuttcap%
\pgfsetroundjoin%
\definecolor{currentfill}{rgb}{1.000000,0.705882,0.509804}%
\pgfsetfillcolor{currentfill}%
\pgfsetlinewidth{0.481800pt}%
\definecolor{currentstroke}{rgb}{1.000000,1.000000,1.000000}%
\pgfsetstrokecolor{currentstroke}%
\pgfsetdash{}{0pt}%
\pgfpathmoveto{\pgfqpoint{4.069841in}{2.077850in}}%
\pgfpathcurveto{\pgfqpoint{4.080891in}{2.077850in}}{\pgfqpoint{4.091490in}{2.082240in}}{\pgfqpoint{4.099303in}{2.090054in}}%
\pgfpathcurveto{\pgfqpoint{4.107117in}{2.097867in}}{\pgfqpoint{4.111507in}{2.108466in}}{\pgfqpoint{4.111507in}{2.119516in}}%
\pgfpathcurveto{\pgfqpoint{4.111507in}{2.130566in}}{\pgfqpoint{4.107117in}{2.141165in}}{\pgfqpoint{4.099303in}{2.148979in}}%
\pgfpathcurveto{\pgfqpoint{4.091490in}{2.156793in}}{\pgfqpoint{4.080891in}{2.161183in}}{\pgfqpoint{4.069841in}{2.161183in}}%
\pgfpathcurveto{\pgfqpoint{4.058790in}{2.161183in}}{\pgfqpoint{4.048191in}{2.156793in}}{\pgfqpoint{4.040378in}{2.148979in}}%
\pgfpathcurveto{\pgfqpoint{4.032564in}{2.141165in}}{\pgfqpoint{4.028174in}{2.130566in}}{\pgfqpoint{4.028174in}{2.119516in}}%
\pgfpathcurveto{\pgfqpoint{4.028174in}{2.108466in}}{\pgfqpoint{4.032564in}{2.097867in}}{\pgfqpoint{4.040378in}{2.090054in}}%
\pgfpathcurveto{\pgfqpoint{4.048191in}{2.082240in}}{\pgfqpoint{4.058790in}{2.077850in}}{\pgfqpoint{4.069841in}{2.077850in}}%
\pgfpathclose%
\pgfusepath{stroke,fill}%
\end{pgfscope}%
\begin{pgfscope}%
\pgfpathrectangle{\pgfqpoint{0.481978in}{0.331635in}}{\pgfqpoint{9.300000in}{7.700000in}}%
\pgfusepath{clip}%
\pgfsetbuttcap%
\pgfsetroundjoin%
\definecolor{currentfill}{rgb}{1.000000,0.705882,0.509804}%
\pgfsetfillcolor{currentfill}%
\pgfsetlinewidth{0.481800pt}%
\definecolor{currentstroke}{rgb}{1.000000,1.000000,1.000000}%
\pgfsetstrokecolor{currentstroke}%
\pgfsetdash{}{0pt}%
\pgfpathmoveto{\pgfqpoint{3.968076in}{5.367897in}}%
\pgfpathcurveto{\pgfqpoint{3.979127in}{5.367897in}}{\pgfqpoint{3.989726in}{5.372288in}}{\pgfqpoint{3.997539in}{5.380101in}}%
\pgfpathcurveto{\pgfqpoint{4.005353in}{5.387915in}}{\pgfqpoint{4.009743in}{5.398514in}}{\pgfqpoint{4.009743in}{5.409564in}}%
\pgfpathcurveto{\pgfqpoint{4.009743in}{5.420614in}}{\pgfqpoint{4.005353in}{5.431213in}}{\pgfqpoint{3.997539in}{5.439027in}}%
\pgfpathcurveto{\pgfqpoint{3.989726in}{5.446840in}}{\pgfqpoint{3.979127in}{5.451231in}}{\pgfqpoint{3.968076in}{5.451231in}}%
\pgfpathcurveto{\pgfqpoint{3.957026in}{5.451231in}}{\pgfqpoint{3.946427in}{5.446840in}}{\pgfqpoint{3.938614in}{5.439027in}}%
\pgfpathcurveto{\pgfqpoint{3.930800in}{5.431213in}}{\pgfqpoint{3.926410in}{5.420614in}}{\pgfqpoint{3.926410in}{5.409564in}}%
\pgfpathcurveto{\pgfqpoint{3.926410in}{5.398514in}}{\pgfqpoint{3.930800in}{5.387915in}}{\pgfqpoint{3.938614in}{5.380101in}}%
\pgfpathcurveto{\pgfqpoint{3.946427in}{5.372288in}}{\pgfqpoint{3.957026in}{5.367897in}}{\pgfqpoint{3.968076in}{5.367897in}}%
\pgfpathclose%
\pgfusepath{stroke,fill}%
\end{pgfscope}%
\begin{pgfscope}%
\pgfpathrectangle{\pgfqpoint{0.481978in}{0.331635in}}{\pgfqpoint{9.300000in}{7.700000in}}%
\pgfusepath{clip}%
\pgfsetbuttcap%
\pgfsetroundjoin%
\definecolor{currentfill}{rgb}{1.000000,0.705882,0.509804}%
\pgfsetfillcolor{currentfill}%
\pgfsetlinewidth{0.481800pt}%
\definecolor{currentstroke}{rgb}{1.000000,1.000000,1.000000}%
\pgfsetstrokecolor{currentstroke}%
\pgfsetdash{}{0pt}%
\pgfpathmoveto{\pgfqpoint{3.839235in}{2.937156in}}%
\pgfpathcurveto{\pgfqpoint{3.850285in}{2.937156in}}{\pgfqpoint{3.860884in}{2.941546in}}{\pgfqpoint{3.868698in}{2.949360in}}%
\pgfpathcurveto{\pgfqpoint{3.876511in}{2.957173in}}{\pgfqpoint{3.880901in}{2.967772in}}{\pgfqpoint{3.880901in}{2.978822in}}%
\pgfpathcurveto{\pgfqpoint{3.880901in}{2.989872in}}{\pgfqpoint{3.876511in}{3.000471in}}{\pgfqpoint{3.868698in}{3.008285in}}%
\pgfpathcurveto{\pgfqpoint{3.860884in}{3.016099in}}{\pgfqpoint{3.850285in}{3.020489in}}{\pgfqpoint{3.839235in}{3.020489in}}%
\pgfpathcurveto{\pgfqpoint{3.828185in}{3.020489in}}{\pgfqpoint{3.817586in}{3.016099in}}{\pgfqpoint{3.809772in}{3.008285in}}%
\pgfpathcurveto{\pgfqpoint{3.801958in}{3.000471in}}{\pgfqpoint{3.797568in}{2.989872in}}{\pgfqpoint{3.797568in}{2.978822in}}%
\pgfpathcurveto{\pgfqpoint{3.797568in}{2.967772in}}{\pgfqpoint{3.801958in}{2.957173in}}{\pgfqpoint{3.809772in}{2.949360in}}%
\pgfpathcurveto{\pgfqpoint{3.817586in}{2.941546in}}{\pgfqpoint{3.828185in}{2.937156in}}{\pgfqpoint{3.839235in}{2.937156in}}%
\pgfpathclose%
\pgfusepath{stroke,fill}%
\end{pgfscope}%
\begin{pgfscope}%
\pgfpathrectangle{\pgfqpoint{0.481978in}{0.331635in}}{\pgfqpoint{9.300000in}{7.700000in}}%
\pgfusepath{clip}%
\pgfsetbuttcap%
\pgfsetroundjoin%
\definecolor{currentfill}{rgb}{1.000000,0.705882,0.509804}%
\pgfsetfillcolor{currentfill}%
\pgfsetlinewidth{0.481800pt}%
\definecolor{currentstroke}{rgb}{1.000000,1.000000,1.000000}%
\pgfsetstrokecolor{currentstroke}%
\pgfsetdash{}{0pt}%
\pgfpathmoveto{\pgfqpoint{4.658407in}{6.466627in}}%
\pgfpathcurveto{\pgfqpoint{4.669457in}{6.466627in}}{\pgfqpoint{4.680056in}{6.471018in}}{\pgfqpoint{4.687869in}{6.478831in}}%
\pgfpathcurveto{\pgfqpoint{4.695683in}{6.486645in}}{\pgfqpoint{4.700073in}{6.497244in}}{\pgfqpoint{4.700073in}{6.508294in}}%
\pgfpathcurveto{\pgfqpoint{4.700073in}{6.519344in}}{\pgfqpoint{4.695683in}{6.529943in}}{\pgfqpoint{4.687869in}{6.537757in}}%
\pgfpathcurveto{\pgfqpoint{4.680056in}{6.545570in}}{\pgfqpoint{4.669457in}{6.549961in}}{\pgfqpoint{4.658407in}{6.549961in}}%
\pgfpathcurveto{\pgfqpoint{4.647356in}{6.549961in}}{\pgfqpoint{4.636757in}{6.545570in}}{\pgfqpoint{4.628944in}{6.537757in}}%
\pgfpathcurveto{\pgfqpoint{4.621130in}{6.529943in}}{\pgfqpoint{4.616740in}{6.519344in}}{\pgfqpoint{4.616740in}{6.508294in}}%
\pgfpathcurveto{\pgfqpoint{4.616740in}{6.497244in}}{\pgfqpoint{4.621130in}{6.486645in}}{\pgfqpoint{4.628944in}{6.478831in}}%
\pgfpathcurveto{\pgfqpoint{4.636757in}{6.471018in}}{\pgfqpoint{4.647356in}{6.466627in}}{\pgfqpoint{4.658407in}{6.466627in}}%
\pgfpathclose%
\pgfusepath{stroke,fill}%
\end{pgfscope}%
\begin{pgfscope}%
\pgfpathrectangle{\pgfqpoint{0.481978in}{0.331635in}}{\pgfqpoint{9.300000in}{7.700000in}}%
\pgfusepath{clip}%
\pgfsetbuttcap%
\pgfsetroundjoin%
\definecolor{currentfill}{rgb}{1.000000,0.705882,0.509804}%
\pgfsetfillcolor{currentfill}%
\pgfsetlinewidth{0.481800pt}%
\definecolor{currentstroke}{rgb}{1.000000,1.000000,1.000000}%
\pgfsetstrokecolor{currentstroke}%
\pgfsetdash{}{0pt}%
\pgfpathmoveto{\pgfqpoint{4.784824in}{5.879172in}}%
\pgfpathcurveto{\pgfqpoint{4.795874in}{5.879172in}}{\pgfqpoint{4.806473in}{5.883562in}}{\pgfqpoint{4.814287in}{5.891376in}}%
\pgfpathcurveto{\pgfqpoint{4.822100in}{5.899189in}}{\pgfqpoint{4.826491in}{5.909788in}}{\pgfqpoint{4.826491in}{5.920838in}}%
\pgfpathcurveto{\pgfqpoint{4.826491in}{5.931889in}}{\pgfqpoint{4.822100in}{5.942488in}}{\pgfqpoint{4.814287in}{5.950301in}}%
\pgfpathcurveto{\pgfqpoint{4.806473in}{5.958115in}}{\pgfqpoint{4.795874in}{5.962505in}}{\pgfqpoint{4.784824in}{5.962505in}}%
\pgfpathcurveto{\pgfqpoint{4.773774in}{5.962505in}}{\pgfqpoint{4.763175in}{5.958115in}}{\pgfqpoint{4.755361in}{5.950301in}}%
\pgfpathcurveto{\pgfqpoint{4.747548in}{5.942488in}}{\pgfqpoint{4.743157in}{5.931889in}}{\pgfqpoint{4.743157in}{5.920838in}}%
\pgfpathcurveto{\pgfqpoint{4.743157in}{5.909788in}}{\pgfqpoint{4.747548in}{5.899189in}}{\pgfqpoint{4.755361in}{5.891376in}}%
\pgfpathcurveto{\pgfqpoint{4.763175in}{5.883562in}}{\pgfqpoint{4.773774in}{5.879172in}}{\pgfqpoint{4.784824in}{5.879172in}}%
\pgfpathclose%
\pgfusepath{stroke,fill}%
\end{pgfscope}%
\begin{pgfscope}%
\pgfpathrectangle{\pgfqpoint{0.481978in}{0.331635in}}{\pgfqpoint{9.300000in}{7.700000in}}%
\pgfusepath{clip}%
\pgfsetbuttcap%
\pgfsetroundjoin%
\definecolor{currentfill}{rgb}{1.000000,0.705882,0.509804}%
\pgfsetfillcolor{currentfill}%
\pgfsetlinewidth{0.481800pt}%
\definecolor{currentstroke}{rgb}{1.000000,1.000000,1.000000}%
\pgfsetstrokecolor{currentstroke}%
\pgfsetdash{}{0pt}%
\pgfpathmoveto{\pgfqpoint{5.305166in}{4.606714in}}%
\pgfpathcurveto{\pgfqpoint{5.316216in}{4.606714in}}{\pgfqpoint{5.326815in}{4.611104in}}{\pgfqpoint{5.334629in}{4.618918in}}%
\pgfpathcurveto{\pgfqpoint{5.342442in}{4.626731in}}{\pgfqpoint{5.346833in}{4.637330in}}{\pgfqpoint{5.346833in}{4.648380in}}%
\pgfpathcurveto{\pgfqpoint{5.346833in}{4.659431in}}{\pgfqpoint{5.342442in}{4.670030in}}{\pgfqpoint{5.334629in}{4.677843in}}%
\pgfpathcurveto{\pgfqpoint{5.326815in}{4.685657in}}{\pgfqpoint{5.316216in}{4.690047in}}{\pgfqpoint{5.305166in}{4.690047in}}%
\pgfpathcurveto{\pgfqpoint{5.294116in}{4.690047in}}{\pgfqpoint{5.283517in}{4.685657in}}{\pgfqpoint{5.275703in}{4.677843in}}%
\pgfpathcurveto{\pgfqpoint{5.267890in}{4.670030in}}{\pgfqpoint{5.263499in}{4.659431in}}{\pgfqpoint{5.263499in}{4.648380in}}%
\pgfpathcurveto{\pgfqpoint{5.263499in}{4.637330in}}{\pgfqpoint{5.267890in}{4.626731in}}{\pgfqpoint{5.275703in}{4.618918in}}%
\pgfpathcurveto{\pgfqpoint{5.283517in}{4.611104in}}{\pgfqpoint{5.294116in}{4.606714in}}{\pgfqpoint{5.305166in}{4.606714in}}%
\pgfpathclose%
\pgfusepath{stroke,fill}%
\end{pgfscope}%
\begin{pgfscope}%
\pgfpathrectangle{\pgfqpoint{0.481978in}{0.331635in}}{\pgfqpoint{9.300000in}{7.700000in}}%
\pgfusepath{clip}%
\pgfsetbuttcap%
\pgfsetroundjoin%
\definecolor{currentfill}{rgb}{1.000000,0.705882,0.509804}%
\pgfsetfillcolor{currentfill}%
\pgfsetlinewidth{0.481800pt}%
\definecolor{currentstroke}{rgb}{1.000000,1.000000,1.000000}%
\pgfsetstrokecolor{currentstroke}%
\pgfsetdash{}{0pt}%
\pgfpathmoveto{\pgfqpoint{3.719294in}{4.720429in}}%
\pgfpathcurveto{\pgfqpoint{3.730344in}{4.720429in}}{\pgfqpoint{3.740943in}{4.724819in}}{\pgfqpoint{3.748756in}{4.732633in}}%
\pgfpathcurveto{\pgfqpoint{3.756570in}{4.740446in}}{\pgfqpoint{3.760960in}{4.751045in}}{\pgfqpoint{3.760960in}{4.762095in}}%
\pgfpathcurveto{\pgfqpoint{3.760960in}{4.773145in}}{\pgfqpoint{3.756570in}{4.783745in}}{\pgfqpoint{3.748756in}{4.791558in}}%
\pgfpathcurveto{\pgfqpoint{3.740943in}{4.799372in}}{\pgfqpoint{3.730344in}{4.803762in}}{\pgfqpoint{3.719294in}{4.803762in}}%
\pgfpathcurveto{\pgfqpoint{3.708243in}{4.803762in}}{\pgfqpoint{3.697644in}{4.799372in}}{\pgfqpoint{3.689831in}{4.791558in}}%
\pgfpathcurveto{\pgfqpoint{3.682017in}{4.783745in}}{\pgfqpoint{3.677627in}{4.773145in}}{\pgfqpoint{3.677627in}{4.762095in}}%
\pgfpathcurveto{\pgfqpoint{3.677627in}{4.751045in}}{\pgfqpoint{3.682017in}{4.740446in}}{\pgfqpoint{3.689831in}{4.732633in}}%
\pgfpathcurveto{\pgfqpoint{3.697644in}{4.724819in}}{\pgfqpoint{3.708243in}{4.720429in}}{\pgfqpoint{3.719294in}{4.720429in}}%
\pgfpathclose%
\pgfusepath{stroke,fill}%
\end{pgfscope}%
\begin{pgfscope}%
\pgfpathrectangle{\pgfqpoint{0.481978in}{0.331635in}}{\pgfqpoint{9.300000in}{7.700000in}}%
\pgfusepath{clip}%
\pgfsetbuttcap%
\pgfsetroundjoin%
\definecolor{currentfill}{rgb}{1.000000,0.705882,0.509804}%
\pgfsetfillcolor{currentfill}%
\pgfsetlinewidth{0.481800pt}%
\definecolor{currentstroke}{rgb}{1.000000,1.000000,1.000000}%
\pgfsetstrokecolor{currentstroke}%
\pgfsetdash{}{0pt}%
\pgfpathmoveto{\pgfqpoint{6.457903in}{1.796543in}}%
\pgfpathcurveto{\pgfqpoint{6.468953in}{1.796543in}}{\pgfqpoint{6.479552in}{1.800933in}}{\pgfqpoint{6.487366in}{1.808747in}}%
\pgfpathcurveto{\pgfqpoint{6.495179in}{1.816561in}}{\pgfqpoint{6.499570in}{1.827160in}}{\pgfqpoint{6.499570in}{1.838210in}}%
\pgfpathcurveto{\pgfqpoint{6.499570in}{1.849260in}}{\pgfqpoint{6.495179in}{1.859859in}}{\pgfqpoint{6.487366in}{1.867673in}}%
\pgfpathcurveto{\pgfqpoint{6.479552in}{1.875486in}}{\pgfqpoint{6.468953in}{1.879876in}}{\pgfqpoint{6.457903in}{1.879876in}}%
\pgfpathcurveto{\pgfqpoint{6.446853in}{1.879876in}}{\pgfqpoint{6.436254in}{1.875486in}}{\pgfqpoint{6.428440in}{1.867673in}}%
\pgfpathcurveto{\pgfqpoint{6.420627in}{1.859859in}}{\pgfqpoint{6.416236in}{1.849260in}}{\pgfqpoint{6.416236in}{1.838210in}}%
\pgfpathcurveto{\pgfqpoint{6.416236in}{1.827160in}}{\pgfqpoint{6.420627in}{1.816561in}}{\pgfqpoint{6.428440in}{1.808747in}}%
\pgfpathcurveto{\pgfqpoint{6.436254in}{1.800933in}}{\pgfqpoint{6.446853in}{1.796543in}}{\pgfqpoint{6.457903in}{1.796543in}}%
\pgfpathclose%
\pgfusepath{stroke,fill}%
\end{pgfscope}%
\begin{pgfscope}%
\pgfpathrectangle{\pgfqpoint{0.481978in}{0.331635in}}{\pgfqpoint{9.300000in}{7.700000in}}%
\pgfusepath{clip}%
\pgfsetbuttcap%
\pgfsetroundjoin%
\definecolor{currentfill}{rgb}{1.000000,0.705882,0.509804}%
\pgfsetfillcolor{currentfill}%
\pgfsetlinewidth{0.481800pt}%
\definecolor{currentstroke}{rgb}{1.000000,1.000000,1.000000}%
\pgfsetstrokecolor{currentstroke}%
\pgfsetdash{}{0pt}%
\pgfpathmoveto{\pgfqpoint{3.091678in}{5.600061in}}%
\pgfpathcurveto{\pgfqpoint{3.102728in}{5.600061in}}{\pgfqpoint{3.113327in}{5.604452in}}{\pgfqpoint{3.121141in}{5.612265in}}%
\pgfpathcurveto{\pgfqpoint{3.128955in}{5.620079in}}{\pgfqpoint{3.133345in}{5.630678in}}{\pgfqpoint{3.133345in}{5.641728in}}%
\pgfpathcurveto{\pgfqpoint{3.133345in}{5.652778in}}{\pgfqpoint{3.128955in}{5.663377in}}{\pgfqpoint{3.121141in}{5.671191in}}%
\pgfpathcurveto{\pgfqpoint{3.113327in}{5.679004in}}{\pgfqpoint{3.102728in}{5.683395in}}{\pgfqpoint{3.091678in}{5.683395in}}%
\pgfpathcurveto{\pgfqpoint{3.080628in}{5.683395in}}{\pgfqpoint{3.070029in}{5.679004in}}{\pgfqpoint{3.062215in}{5.671191in}}%
\pgfpathcurveto{\pgfqpoint{3.054402in}{5.663377in}}{\pgfqpoint{3.050012in}{5.652778in}}{\pgfqpoint{3.050012in}{5.641728in}}%
\pgfpathcurveto{\pgfqpoint{3.050012in}{5.630678in}}{\pgfqpoint{3.054402in}{5.620079in}}{\pgfqpoint{3.062215in}{5.612265in}}%
\pgfpathcurveto{\pgfqpoint{3.070029in}{5.604452in}}{\pgfqpoint{3.080628in}{5.600061in}}{\pgfqpoint{3.091678in}{5.600061in}}%
\pgfpathclose%
\pgfusepath{stroke,fill}%
\end{pgfscope}%
\begin{pgfscope}%
\pgfpathrectangle{\pgfqpoint{0.481978in}{0.331635in}}{\pgfqpoint{9.300000in}{7.700000in}}%
\pgfusepath{clip}%
\pgfsetbuttcap%
\pgfsetroundjoin%
\definecolor{currentfill}{rgb}{1.000000,0.705882,0.509804}%
\pgfsetfillcolor{currentfill}%
\pgfsetlinewidth{0.481800pt}%
\definecolor{currentstroke}{rgb}{1.000000,1.000000,1.000000}%
\pgfsetstrokecolor{currentstroke}%
\pgfsetdash{}{0pt}%
\pgfpathmoveto{\pgfqpoint{0.904705in}{4.489588in}}%
\pgfpathcurveto{\pgfqpoint{0.915755in}{4.489588in}}{\pgfqpoint{0.926354in}{4.493978in}}{\pgfqpoint{0.934168in}{4.501792in}}%
\pgfpathcurveto{\pgfqpoint{0.941982in}{4.509605in}}{\pgfqpoint{0.946372in}{4.520204in}}{\pgfqpoint{0.946372in}{4.531254in}}%
\pgfpathcurveto{\pgfqpoint{0.946372in}{4.542305in}}{\pgfqpoint{0.941982in}{4.552904in}}{\pgfqpoint{0.934168in}{4.560717in}}%
\pgfpathcurveto{\pgfqpoint{0.926354in}{4.568531in}}{\pgfqpoint{0.915755in}{4.572921in}}{\pgfqpoint{0.904705in}{4.572921in}}%
\pgfpathcurveto{\pgfqpoint{0.893655in}{4.572921in}}{\pgfqpoint{0.883056in}{4.568531in}}{\pgfqpoint{0.875242in}{4.560717in}}%
\pgfpathcurveto{\pgfqpoint{0.867429in}{4.552904in}}{\pgfqpoint{0.863039in}{4.542305in}}{\pgfqpoint{0.863039in}{4.531254in}}%
\pgfpathcurveto{\pgfqpoint{0.863039in}{4.520204in}}{\pgfqpoint{0.867429in}{4.509605in}}{\pgfqpoint{0.875242in}{4.501792in}}%
\pgfpathcurveto{\pgfqpoint{0.883056in}{4.493978in}}{\pgfqpoint{0.893655in}{4.489588in}}{\pgfqpoint{0.904705in}{4.489588in}}%
\pgfpathclose%
\pgfusepath{stroke,fill}%
\end{pgfscope}%
\begin{pgfscope}%
\pgfpathrectangle{\pgfqpoint{0.481978in}{0.331635in}}{\pgfqpoint{9.300000in}{7.700000in}}%
\pgfusepath{clip}%
\pgfsetbuttcap%
\pgfsetroundjoin%
\definecolor{currentfill}{rgb}{1.000000,0.705882,0.509804}%
\pgfsetfillcolor{currentfill}%
\pgfsetlinewidth{0.481800pt}%
\definecolor{currentstroke}{rgb}{1.000000,1.000000,1.000000}%
\pgfsetstrokecolor{currentstroke}%
\pgfsetdash{}{0pt}%
\pgfpathmoveto{\pgfqpoint{5.515150in}{3.272450in}}%
\pgfpathcurveto{\pgfqpoint{5.526200in}{3.272450in}}{\pgfqpoint{5.536799in}{3.276841in}}{\pgfqpoint{5.544612in}{3.284654in}}%
\pgfpathcurveto{\pgfqpoint{5.552426in}{3.292468in}}{\pgfqpoint{5.556816in}{3.303067in}}{\pgfqpoint{5.556816in}{3.314117in}}%
\pgfpathcurveto{\pgfqpoint{5.556816in}{3.325167in}}{\pgfqpoint{5.552426in}{3.335766in}}{\pgfqpoint{5.544612in}{3.343580in}}%
\pgfpathcurveto{\pgfqpoint{5.536799in}{3.351393in}}{\pgfqpoint{5.526200in}{3.355784in}}{\pgfqpoint{5.515150in}{3.355784in}}%
\pgfpathcurveto{\pgfqpoint{5.504099in}{3.355784in}}{\pgfqpoint{5.493500in}{3.351393in}}{\pgfqpoint{5.485687in}{3.343580in}}%
\pgfpathcurveto{\pgfqpoint{5.477873in}{3.335766in}}{\pgfqpoint{5.473483in}{3.325167in}}{\pgfqpoint{5.473483in}{3.314117in}}%
\pgfpathcurveto{\pgfqpoint{5.473483in}{3.303067in}}{\pgfqpoint{5.477873in}{3.292468in}}{\pgfqpoint{5.485687in}{3.284654in}}%
\pgfpathcurveto{\pgfqpoint{5.493500in}{3.276841in}}{\pgfqpoint{5.504099in}{3.272450in}}{\pgfqpoint{5.515150in}{3.272450in}}%
\pgfpathclose%
\pgfusepath{stroke,fill}%
\end{pgfscope}%
\begin{pgfscope}%
\pgfpathrectangle{\pgfqpoint{0.481978in}{0.331635in}}{\pgfqpoint{9.300000in}{7.700000in}}%
\pgfusepath{clip}%
\pgfsetbuttcap%
\pgfsetroundjoin%
\definecolor{currentfill}{rgb}{1.000000,0.705882,0.509804}%
\pgfsetfillcolor{currentfill}%
\pgfsetlinewidth{0.481800pt}%
\definecolor{currentstroke}{rgb}{1.000000,1.000000,1.000000}%
\pgfsetstrokecolor{currentstroke}%
\pgfsetdash{}{0pt}%
\pgfpathmoveto{\pgfqpoint{3.655066in}{5.193021in}}%
\pgfpathcurveto{\pgfqpoint{3.666116in}{5.193021in}}{\pgfqpoint{3.676715in}{5.197411in}}{\pgfqpoint{3.684528in}{5.205225in}}%
\pgfpathcurveto{\pgfqpoint{3.692342in}{5.213038in}}{\pgfqpoint{3.696732in}{5.223637in}}{\pgfqpoint{3.696732in}{5.234687in}}%
\pgfpathcurveto{\pgfqpoint{3.696732in}{5.245738in}}{\pgfqpoint{3.692342in}{5.256337in}}{\pgfqpoint{3.684528in}{5.264150in}}%
\pgfpathcurveto{\pgfqpoint{3.676715in}{5.271964in}}{\pgfqpoint{3.666116in}{5.276354in}}{\pgfqpoint{3.655066in}{5.276354in}}%
\pgfpathcurveto{\pgfqpoint{3.644016in}{5.276354in}}{\pgfqpoint{3.633417in}{5.271964in}}{\pgfqpoint{3.625603in}{5.264150in}}%
\pgfpathcurveto{\pgfqpoint{3.617789in}{5.256337in}}{\pgfqpoint{3.613399in}{5.245738in}}{\pgfqpoint{3.613399in}{5.234687in}}%
\pgfpathcurveto{\pgfqpoint{3.613399in}{5.223637in}}{\pgfqpoint{3.617789in}{5.213038in}}{\pgfqpoint{3.625603in}{5.205225in}}%
\pgfpathcurveto{\pgfqpoint{3.633417in}{5.197411in}}{\pgfqpoint{3.644016in}{5.193021in}}{\pgfqpoint{3.655066in}{5.193021in}}%
\pgfpathclose%
\pgfusepath{stroke,fill}%
\end{pgfscope}%
\begin{pgfscope}%
\pgfpathrectangle{\pgfqpoint{0.481978in}{0.331635in}}{\pgfqpoint{9.300000in}{7.700000in}}%
\pgfusepath{clip}%
\pgfsetbuttcap%
\pgfsetroundjoin%
\definecolor{currentfill}{rgb}{1.000000,0.705882,0.509804}%
\pgfsetfillcolor{currentfill}%
\pgfsetlinewidth{0.481800pt}%
\definecolor{currentstroke}{rgb}{1.000000,1.000000,1.000000}%
\pgfsetstrokecolor{currentstroke}%
\pgfsetdash{}{0pt}%
\pgfpathmoveto{\pgfqpoint{4.606001in}{2.677957in}}%
\pgfpathcurveto{\pgfqpoint{4.617051in}{2.677957in}}{\pgfqpoint{4.627650in}{2.682348in}}{\pgfqpoint{4.635464in}{2.690161in}}%
\pgfpathcurveto{\pgfqpoint{4.643277in}{2.697975in}}{\pgfqpoint{4.647667in}{2.708574in}}{\pgfqpoint{4.647667in}{2.719624in}}%
\pgfpathcurveto{\pgfqpoint{4.647667in}{2.730674in}}{\pgfqpoint{4.643277in}{2.741273in}}{\pgfqpoint{4.635464in}{2.749087in}}%
\pgfpathcurveto{\pgfqpoint{4.627650in}{2.756901in}}{\pgfqpoint{4.617051in}{2.761291in}}{\pgfqpoint{4.606001in}{2.761291in}}%
\pgfpathcurveto{\pgfqpoint{4.594951in}{2.761291in}}{\pgfqpoint{4.584352in}{2.756901in}}{\pgfqpoint{4.576538in}{2.749087in}}%
\pgfpathcurveto{\pgfqpoint{4.568724in}{2.741273in}}{\pgfqpoint{4.564334in}{2.730674in}}{\pgfqpoint{4.564334in}{2.719624in}}%
\pgfpathcurveto{\pgfqpoint{4.564334in}{2.708574in}}{\pgfqpoint{4.568724in}{2.697975in}}{\pgfqpoint{4.576538in}{2.690161in}}%
\pgfpathcurveto{\pgfqpoint{4.584352in}{2.682348in}}{\pgfqpoint{4.594951in}{2.677957in}}{\pgfqpoint{4.606001in}{2.677957in}}%
\pgfpathclose%
\pgfusepath{stroke,fill}%
\end{pgfscope}%
\begin{pgfscope}%
\pgfpathrectangle{\pgfqpoint{0.481978in}{0.331635in}}{\pgfqpoint{9.300000in}{7.700000in}}%
\pgfusepath{clip}%
\pgfsetbuttcap%
\pgfsetroundjoin%
\definecolor{currentfill}{rgb}{1.000000,0.705882,0.509804}%
\pgfsetfillcolor{currentfill}%
\pgfsetlinewidth{0.481800pt}%
\definecolor{currentstroke}{rgb}{1.000000,1.000000,1.000000}%
\pgfsetstrokecolor{currentstroke}%
\pgfsetdash{}{0pt}%
\pgfpathmoveto{\pgfqpoint{2.788909in}{4.513266in}}%
\pgfpathcurveto{\pgfqpoint{2.799959in}{4.513266in}}{\pgfqpoint{2.810558in}{4.517656in}}{\pgfqpoint{2.818372in}{4.525470in}}%
\pgfpathcurveto{\pgfqpoint{2.826185in}{4.533283in}}{\pgfqpoint{2.830575in}{4.543882in}}{\pgfqpoint{2.830575in}{4.554933in}}%
\pgfpathcurveto{\pgfqpoint{2.830575in}{4.565983in}}{\pgfqpoint{2.826185in}{4.576582in}}{\pgfqpoint{2.818372in}{4.584395in}}%
\pgfpathcurveto{\pgfqpoint{2.810558in}{4.592209in}}{\pgfqpoint{2.799959in}{4.596599in}}{\pgfqpoint{2.788909in}{4.596599in}}%
\pgfpathcurveto{\pgfqpoint{2.777859in}{4.596599in}}{\pgfqpoint{2.767260in}{4.592209in}}{\pgfqpoint{2.759446in}{4.584395in}}%
\pgfpathcurveto{\pgfqpoint{2.751632in}{4.576582in}}{\pgfqpoint{2.747242in}{4.565983in}}{\pgfqpoint{2.747242in}{4.554933in}}%
\pgfpathcurveto{\pgfqpoint{2.747242in}{4.543882in}}{\pgfqpoint{2.751632in}{4.533283in}}{\pgfqpoint{2.759446in}{4.525470in}}%
\pgfpathcurveto{\pgfqpoint{2.767260in}{4.517656in}}{\pgfqpoint{2.777859in}{4.513266in}}{\pgfqpoint{2.788909in}{4.513266in}}%
\pgfpathclose%
\pgfusepath{stroke,fill}%
\end{pgfscope}%
\begin{pgfscope}%
\pgfpathrectangle{\pgfqpoint{0.481978in}{0.331635in}}{\pgfqpoint{9.300000in}{7.700000in}}%
\pgfusepath{clip}%
\pgfsetbuttcap%
\pgfsetroundjoin%
\definecolor{currentfill}{rgb}{1.000000,0.705882,0.509804}%
\pgfsetfillcolor{currentfill}%
\pgfsetlinewidth{0.481800pt}%
\definecolor{currentstroke}{rgb}{1.000000,1.000000,1.000000}%
\pgfsetstrokecolor{currentstroke}%
\pgfsetdash{}{0pt}%
\pgfpathmoveto{\pgfqpoint{2.529879in}{3.143449in}}%
\pgfpathcurveto{\pgfqpoint{2.540929in}{3.143449in}}{\pgfqpoint{2.551528in}{3.147839in}}{\pgfqpoint{2.559342in}{3.155653in}}%
\pgfpathcurveto{\pgfqpoint{2.567155in}{3.163466in}}{\pgfqpoint{2.571546in}{3.174065in}}{\pgfqpoint{2.571546in}{3.185116in}}%
\pgfpathcurveto{\pgfqpoint{2.571546in}{3.196166in}}{\pgfqpoint{2.567155in}{3.206765in}}{\pgfqpoint{2.559342in}{3.214578in}}%
\pgfpathcurveto{\pgfqpoint{2.551528in}{3.222392in}}{\pgfqpoint{2.540929in}{3.226782in}}{\pgfqpoint{2.529879in}{3.226782in}}%
\pgfpathcurveto{\pgfqpoint{2.518829in}{3.226782in}}{\pgfqpoint{2.508230in}{3.222392in}}{\pgfqpoint{2.500416in}{3.214578in}}%
\pgfpathcurveto{\pgfqpoint{2.492602in}{3.206765in}}{\pgfqpoint{2.488212in}{3.196166in}}{\pgfqpoint{2.488212in}{3.185116in}}%
\pgfpathcurveto{\pgfqpoint{2.488212in}{3.174065in}}{\pgfqpoint{2.492602in}{3.163466in}}{\pgfqpoint{2.500416in}{3.155653in}}%
\pgfpathcurveto{\pgfqpoint{2.508230in}{3.147839in}}{\pgfqpoint{2.518829in}{3.143449in}}{\pgfqpoint{2.529879in}{3.143449in}}%
\pgfpathclose%
\pgfusepath{stroke,fill}%
\end{pgfscope}%
\begin{pgfscope}%
\pgfpathrectangle{\pgfqpoint{0.481978in}{0.331635in}}{\pgfqpoint{9.300000in}{7.700000in}}%
\pgfusepath{clip}%
\pgfsetbuttcap%
\pgfsetroundjoin%
\definecolor{currentfill}{rgb}{1.000000,0.705882,0.509804}%
\pgfsetfillcolor{currentfill}%
\pgfsetlinewidth{0.481800pt}%
\definecolor{currentstroke}{rgb}{1.000000,1.000000,1.000000}%
\pgfsetstrokecolor{currentstroke}%
\pgfsetdash{}{0pt}%
\pgfpathmoveto{\pgfqpoint{2.960587in}{2.499432in}}%
\pgfpathcurveto{\pgfqpoint{2.971637in}{2.499432in}}{\pgfqpoint{2.982236in}{2.503822in}}{\pgfqpoint{2.990050in}{2.511636in}}%
\pgfpathcurveto{\pgfqpoint{2.997864in}{2.519449in}}{\pgfqpoint{3.002254in}{2.530048in}}{\pgfqpoint{3.002254in}{2.541098in}}%
\pgfpathcurveto{\pgfqpoint{3.002254in}{2.552148in}}{\pgfqpoint{2.997864in}{2.562747in}}{\pgfqpoint{2.990050in}{2.570561in}}%
\pgfpathcurveto{\pgfqpoint{2.982236in}{2.578375in}}{\pgfqpoint{2.971637in}{2.582765in}}{\pgfqpoint{2.960587in}{2.582765in}}%
\pgfpathcurveto{\pgfqpoint{2.949537in}{2.582765in}}{\pgfqpoint{2.938938in}{2.578375in}}{\pgfqpoint{2.931124in}{2.570561in}}%
\pgfpathcurveto{\pgfqpoint{2.923311in}{2.562747in}}{\pgfqpoint{2.918920in}{2.552148in}}{\pgfqpoint{2.918920in}{2.541098in}}%
\pgfpathcurveto{\pgfqpoint{2.918920in}{2.530048in}}{\pgfqpoint{2.923311in}{2.519449in}}{\pgfqpoint{2.931124in}{2.511636in}}%
\pgfpathcurveto{\pgfqpoint{2.938938in}{2.503822in}}{\pgfqpoint{2.949537in}{2.499432in}}{\pgfqpoint{2.960587in}{2.499432in}}%
\pgfpathclose%
\pgfusepath{stroke,fill}%
\end{pgfscope}%
\begin{pgfscope}%
\pgfpathrectangle{\pgfqpoint{0.481978in}{0.331635in}}{\pgfqpoint{9.300000in}{7.700000in}}%
\pgfusepath{clip}%
\pgfsetbuttcap%
\pgfsetroundjoin%
\definecolor{currentfill}{rgb}{1.000000,0.705882,0.509804}%
\pgfsetfillcolor{currentfill}%
\pgfsetlinewidth{0.481800pt}%
\definecolor{currentstroke}{rgb}{1.000000,1.000000,1.000000}%
\pgfsetstrokecolor{currentstroke}%
\pgfsetdash{}{0pt}%
\pgfpathmoveto{\pgfqpoint{8.052568in}{3.832077in}}%
\pgfpathcurveto{\pgfqpoint{8.063618in}{3.832077in}}{\pgfqpoint{8.074217in}{3.836468in}}{\pgfqpoint{8.082031in}{3.844281in}}%
\pgfpathcurveto{\pgfqpoint{8.089844in}{3.852095in}}{\pgfqpoint{8.094234in}{3.862694in}}{\pgfqpoint{8.094234in}{3.873744in}}%
\pgfpathcurveto{\pgfqpoint{8.094234in}{3.884794in}}{\pgfqpoint{8.089844in}{3.895393in}}{\pgfqpoint{8.082031in}{3.903207in}}%
\pgfpathcurveto{\pgfqpoint{8.074217in}{3.911021in}}{\pgfqpoint{8.063618in}{3.915411in}}{\pgfqpoint{8.052568in}{3.915411in}}%
\pgfpathcurveto{\pgfqpoint{8.041518in}{3.915411in}}{\pgfqpoint{8.030919in}{3.911021in}}{\pgfqpoint{8.023105in}{3.903207in}}%
\pgfpathcurveto{\pgfqpoint{8.015291in}{3.895393in}}{\pgfqpoint{8.010901in}{3.884794in}}{\pgfqpoint{8.010901in}{3.873744in}}%
\pgfpathcurveto{\pgfqpoint{8.010901in}{3.862694in}}{\pgfqpoint{8.015291in}{3.852095in}}{\pgfqpoint{8.023105in}{3.844281in}}%
\pgfpathcurveto{\pgfqpoint{8.030919in}{3.836468in}}{\pgfqpoint{8.041518in}{3.832077in}}{\pgfqpoint{8.052568in}{3.832077in}}%
\pgfpathclose%
\pgfusepath{stroke,fill}%
\end{pgfscope}%
\begin{pgfscope}%
\pgfpathrectangle{\pgfqpoint{0.481978in}{0.331635in}}{\pgfqpoint{9.300000in}{7.700000in}}%
\pgfusepath{clip}%
\pgfsetbuttcap%
\pgfsetroundjoin%
\definecolor{currentfill}{rgb}{1.000000,0.705882,0.509804}%
\pgfsetfillcolor{currentfill}%
\pgfsetlinewidth{0.481800pt}%
\definecolor{currentstroke}{rgb}{1.000000,1.000000,1.000000}%
\pgfsetstrokecolor{currentstroke}%
\pgfsetdash{}{0pt}%
\pgfpathmoveto{\pgfqpoint{1.457328in}{2.495820in}}%
\pgfpathcurveto{\pgfqpoint{1.468378in}{2.495820in}}{\pgfqpoint{1.478977in}{2.500210in}}{\pgfqpoint{1.486791in}{2.508024in}}%
\pgfpathcurveto{\pgfqpoint{1.494604in}{2.515837in}}{\pgfqpoint{1.498994in}{2.526436in}}{\pgfqpoint{1.498994in}{2.537487in}}%
\pgfpathcurveto{\pgfqpoint{1.498994in}{2.548537in}}{\pgfqpoint{1.494604in}{2.559136in}}{\pgfqpoint{1.486791in}{2.566949in}}%
\pgfpathcurveto{\pgfqpoint{1.478977in}{2.574763in}}{\pgfqpoint{1.468378in}{2.579153in}}{\pgfqpoint{1.457328in}{2.579153in}}%
\pgfpathcurveto{\pgfqpoint{1.446278in}{2.579153in}}{\pgfqpoint{1.435679in}{2.574763in}}{\pgfqpoint{1.427865in}{2.566949in}}%
\pgfpathcurveto{\pgfqpoint{1.420051in}{2.559136in}}{\pgfqpoint{1.415661in}{2.548537in}}{\pgfqpoint{1.415661in}{2.537487in}}%
\pgfpathcurveto{\pgfqpoint{1.415661in}{2.526436in}}{\pgfqpoint{1.420051in}{2.515837in}}{\pgfqpoint{1.427865in}{2.508024in}}%
\pgfpathcurveto{\pgfqpoint{1.435679in}{2.500210in}}{\pgfqpoint{1.446278in}{2.495820in}}{\pgfqpoint{1.457328in}{2.495820in}}%
\pgfpathclose%
\pgfusepath{stroke,fill}%
\end{pgfscope}%
\begin{pgfscope}%
\pgfpathrectangle{\pgfqpoint{0.481978in}{0.331635in}}{\pgfqpoint{9.300000in}{7.700000in}}%
\pgfusepath{clip}%
\pgfsetbuttcap%
\pgfsetroundjoin%
\definecolor{currentfill}{rgb}{1.000000,0.705882,0.509804}%
\pgfsetfillcolor{currentfill}%
\pgfsetlinewidth{0.481800pt}%
\definecolor{currentstroke}{rgb}{1.000000,1.000000,1.000000}%
\pgfsetstrokecolor{currentstroke}%
\pgfsetdash{}{0pt}%
\pgfpathmoveto{\pgfqpoint{1.959498in}{3.611662in}}%
\pgfpathcurveto{\pgfqpoint{1.970548in}{3.611662in}}{\pgfqpoint{1.981147in}{3.616053in}}{\pgfqpoint{1.988961in}{3.623866in}}%
\pgfpathcurveto{\pgfqpoint{1.996775in}{3.631680in}}{\pgfqpoint{2.001165in}{3.642279in}}{\pgfqpoint{2.001165in}{3.653329in}}%
\pgfpathcurveto{\pgfqpoint{2.001165in}{3.664379in}}{\pgfqpoint{1.996775in}{3.674978in}}{\pgfqpoint{1.988961in}{3.682792in}}%
\pgfpathcurveto{\pgfqpoint{1.981147in}{3.690605in}}{\pgfqpoint{1.970548in}{3.694996in}}{\pgfqpoint{1.959498in}{3.694996in}}%
\pgfpathcurveto{\pgfqpoint{1.948448in}{3.694996in}}{\pgfqpoint{1.937849in}{3.690605in}}{\pgfqpoint{1.930035in}{3.682792in}}%
\pgfpathcurveto{\pgfqpoint{1.922222in}{3.674978in}}{\pgfqpoint{1.917831in}{3.664379in}}{\pgfqpoint{1.917831in}{3.653329in}}%
\pgfpathcurveto{\pgfqpoint{1.917831in}{3.642279in}}{\pgfqpoint{1.922222in}{3.631680in}}{\pgfqpoint{1.930035in}{3.623866in}}%
\pgfpathcurveto{\pgfqpoint{1.937849in}{3.616053in}}{\pgfqpoint{1.948448in}{3.611662in}}{\pgfqpoint{1.959498in}{3.611662in}}%
\pgfpathclose%
\pgfusepath{stroke,fill}%
\end{pgfscope}%
\begin{pgfscope}%
\pgfpathrectangle{\pgfqpoint{0.481978in}{0.331635in}}{\pgfqpoint{9.300000in}{7.700000in}}%
\pgfusepath{clip}%
\pgfsetbuttcap%
\pgfsetroundjoin%
\definecolor{currentfill}{rgb}{1.000000,0.705882,0.509804}%
\pgfsetfillcolor{currentfill}%
\pgfsetlinewidth{0.481800pt}%
\definecolor{currentstroke}{rgb}{1.000000,1.000000,1.000000}%
\pgfsetstrokecolor{currentstroke}%
\pgfsetdash{}{0pt}%
\pgfpathmoveto{\pgfqpoint{5.141137in}{2.853956in}}%
\pgfpathcurveto{\pgfqpoint{5.152187in}{2.853956in}}{\pgfqpoint{5.162786in}{2.858347in}}{\pgfqpoint{5.170600in}{2.866160in}}%
\pgfpathcurveto{\pgfqpoint{5.178414in}{2.873974in}}{\pgfqpoint{5.182804in}{2.884573in}}{\pgfqpoint{5.182804in}{2.895623in}}%
\pgfpathcurveto{\pgfqpoint{5.182804in}{2.906673in}}{\pgfqpoint{5.178414in}{2.917272in}}{\pgfqpoint{5.170600in}{2.925086in}}%
\pgfpathcurveto{\pgfqpoint{5.162786in}{2.932899in}}{\pgfqpoint{5.152187in}{2.937290in}}{\pgfqpoint{5.141137in}{2.937290in}}%
\pgfpathcurveto{\pgfqpoint{5.130087in}{2.937290in}}{\pgfqpoint{5.119488in}{2.932899in}}{\pgfqpoint{5.111674in}{2.925086in}}%
\pgfpathcurveto{\pgfqpoint{5.103861in}{2.917272in}}{\pgfqpoint{5.099470in}{2.906673in}}{\pgfqpoint{5.099470in}{2.895623in}}%
\pgfpathcurveto{\pgfqpoint{5.099470in}{2.884573in}}{\pgfqpoint{5.103861in}{2.873974in}}{\pgfqpoint{5.111674in}{2.866160in}}%
\pgfpathcurveto{\pgfqpoint{5.119488in}{2.858347in}}{\pgfqpoint{5.130087in}{2.853956in}}{\pgfqpoint{5.141137in}{2.853956in}}%
\pgfpathclose%
\pgfusepath{stroke,fill}%
\end{pgfscope}%
\begin{pgfscope}%
\pgfpathrectangle{\pgfqpoint{0.481978in}{0.331635in}}{\pgfqpoint{9.300000in}{7.700000in}}%
\pgfusepath{clip}%
\pgfsetbuttcap%
\pgfsetroundjoin%
\definecolor{currentfill}{rgb}{1.000000,0.705882,0.509804}%
\pgfsetfillcolor{currentfill}%
\pgfsetlinewidth{0.481800pt}%
\definecolor{currentstroke}{rgb}{1.000000,1.000000,1.000000}%
\pgfsetstrokecolor{currentstroke}%
\pgfsetdash{}{0pt}%
\pgfpathmoveto{\pgfqpoint{5.415961in}{3.905778in}}%
\pgfpathcurveto{\pgfqpoint{5.427011in}{3.905778in}}{\pgfqpoint{5.437610in}{3.910169in}}{\pgfqpoint{5.445423in}{3.917982in}}%
\pgfpathcurveto{\pgfqpoint{5.453237in}{3.925796in}}{\pgfqpoint{5.457627in}{3.936395in}}{\pgfqpoint{5.457627in}{3.947445in}}%
\pgfpathcurveto{\pgfqpoint{5.457627in}{3.958495in}}{\pgfqpoint{5.453237in}{3.969094in}}{\pgfqpoint{5.445423in}{3.976908in}}%
\pgfpathcurveto{\pgfqpoint{5.437610in}{3.984721in}}{\pgfqpoint{5.427011in}{3.989112in}}{\pgfqpoint{5.415961in}{3.989112in}}%
\pgfpathcurveto{\pgfqpoint{5.404910in}{3.989112in}}{\pgfqpoint{5.394311in}{3.984721in}}{\pgfqpoint{5.386498in}{3.976908in}}%
\pgfpathcurveto{\pgfqpoint{5.378684in}{3.969094in}}{\pgfqpoint{5.374294in}{3.958495in}}{\pgfqpoint{5.374294in}{3.947445in}}%
\pgfpathcurveto{\pgfqpoint{5.374294in}{3.936395in}}{\pgfqpoint{5.378684in}{3.925796in}}{\pgfqpoint{5.386498in}{3.917982in}}%
\pgfpathcurveto{\pgfqpoint{5.394311in}{3.910169in}}{\pgfqpoint{5.404910in}{3.905778in}}{\pgfqpoint{5.415961in}{3.905778in}}%
\pgfpathclose%
\pgfusepath{stroke,fill}%
\end{pgfscope}%
\begin{pgfscope}%
\pgfpathrectangle{\pgfqpoint{0.481978in}{0.331635in}}{\pgfqpoint{9.300000in}{7.700000in}}%
\pgfusepath{clip}%
\pgfsetbuttcap%
\pgfsetroundjoin%
\definecolor{currentfill}{rgb}{1.000000,0.705882,0.509804}%
\pgfsetfillcolor{currentfill}%
\pgfsetlinewidth{0.481800pt}%
\definecolor{currentstroke}{rgb}{1.000000,1.000000,1.000000}%
\pgfsetstrokecolor{currentstroke}%
\pgfsetdash{}{0pt}%
\pgfpathmoveto{\pgfqpoint{3.418504in}{2.566381in}}%
\pgfpathcurveto{\pgfqpoint{3.429554in}{2.566381in}}{\pgfqpoint{3.440153in}{2.570771in}}{\pgfqpoint{3.447967in}{2.578584in}}%
\pgfpathcurveto{\pgfqpoint{3.455781in}{2.586398in}}{\pgfqpoint{3.460171in}{2.596997in}}{\pgfqpoint{3.460171in}{2.608047in}}%
\pgfpathcurveto{\pgfqpoint{3.460171in}{2.619097in}}{\pgfqpoint{3.455781in}{2.629696in}}{\pgfqpoint{3.447967in}{2.637510in}}%
\pgfpathcurveto{\pgfqpoint{3.440153in}{2.645324in}}{\pgfqpoint{3.429554in}{2.649714in}}{\pgfqpoint{3.418504in}{2.649714in}}%
\pgfpathcurveto{\pgfqpoint{3.407454in}{2.649714in}}{\pgfqpoint{3.396855in}{2.645324in}}{\pgfqpoint{3.389041in}{2.637510in}}%
\pgfpathcurveto{\pgfqpoint{3.381228in}{2.629696in}}{\pgfqpoint{3.376838in}{2.619097in}}{\pgfqpoint{3.376838in}{2.608047in}}%
\pgfpathcurveto{\pgfqpoint{3.376838in}{2.596997in}}{\pgfqpoint{3.381228in}{2.586398in}}{\pgfqpoint{3.389041in}{2.578584in}}%
\pgfpathcurveto{\pgfqpoint{3.396855in}{2.570771in}}{\pgfqpoint{3.407454in}{2.566381in}}{\pgfqpoint{3.418504in}{2.566381in}}%
\pgfpathclose%
\pgfusepath{stroke,fill}%
\end{pgfscope}%
\begin{pgfscope}%
\pgfpathrectangle{\pgfqpoint{0.481978in}{0.331635in}}{\pgfqpoint{9.300000in}{7.700000in}}%
\pgfusepath{clip}%
\pgfsetbuttcap%
\pgfsetroundjoin%
\definecolor{currentfill}{rgb}{1.000000,0.705882,0.509804}%
\pgfsetfillcolor{currentfill}%
\pgfsetlinewidth{0.481800pt}%
\definecolor{currentstroke}{rgb}{1.000000,1.000000,1.000000}%
\pgfsetstrokecolor{currentstroke}%
\pgfsetdash{}{0pt}%
\pgfpathmoveto{\pgfqpoint{1.418868in}{3.681751in}}%
\pgfpathcurveto{\pgfqpoint{1.429918in}{3.681751in}}{\pgfqpoint{1.440517in}{3.686141in}}{\pgfqpoint{1.448331in}{3.693954in}}%
\pgfpathcurveto{\pgfqpoint{1.456145in}{3.701768in}}{\pgfqpoint{1.460535in}{3.712367in}}{\pgfqpoint{1.460535in}{3.723417in}}%
\pgfpathcurveto{\pgfqpoint{1.460535in}{3.734467in}}{\pgfqpoint{1.456145in}{3.745066in}}{\pgfqpoint{1.448331in}{3.752880in}}%
\pgfpathcurveto{\pgfqpoint{1.440517in}{3.760694in}}{\pgfqpoint{1.429918in}{3.765084in}}{\pgfqpoint{1.418868in}{3.765084in}}%
\pgfpathcurveto{\pgfqpoint{1.407818in}{3.765084in}}{\pgfqpoint{1.397219in}{3.760694in}}{\pgfqpoint{1.389405in}{3.752880in}}%
\pgfpathcurveto{\pgfqpoint{1.381592in}{3.745066in}}{\pgfqpoint{1.377202in}{3.734467in}}{\pgfqpoint{1.377202in}{3.723417in}}%
\pgfpathcurveto{\pgfqpoint{1.377202in}{3.712367in}}{\pgfqpoint{1.381592in}{3.701768in}}{\pgfqpoint{1.389405in}{3.693954in}}%
\pgfpathcurveto{\pgfqpoint{1.397219in}{3.686141in}}{\pgfqpoint{1.407818in}{3.681751in}}{\pgfqpoint{1.418868in}{3.681751in}}%
\pgfpathclose%
\pgfusepath{stroke,fill}%
\end{pgfscope}%
\begin{pgfscope}%
\pgfpathrectangle{\pgfqpoint{0.481978in}{0.331635in}}{\pgfqpoint{9.300000in}{7.700000in}}%
\pgfusepath{clip}%
\pgfsetbuttcap%
\pgfsetroundjoin%
\definecolor{currentfill}{rgb}{1.000000,0.705882,0.509804}%
\pgfsetfillcolor{currentfill}%
\pgfsetlinewidth{0.481800pt}%
\definecolor{currentstroke}{rgb}{1.000000,1.000000,1.000000}%
\pgfsetstrokecolor{currentstroke}%
\pgfsetdash{}{0pt}%
\pgfpathmoveto{\pgfqpoint{4.390885in}{4.205008in}}%
\pgfpathcurveto{\pgfqpoint{4.401935in}{4.205008in}}{\pgfqpoint{4.412535in}{4.209399in}}{\pgfqpoint{4.420348in}{4.217212in}}%
\pgfpathcurveto{\pgfqpoint{4.428162in}{4.225026in}}{\pgfqpoint{4.432552in}{4.235625in}}{\pgfqpoint{4.432552in}{4.246675in}}%
\pgfpathcurveto{\pgfqpoint{4.432552in}{4.257725in}}{\pgfqpoint{4.428162in}{4.268324in}}{\pgfqpoint{4.420348in}{4.276138in}}%
\pgfpathcurveto{\pgfqpoint{4.412535in}{4.283951in}}{\pgfqpoint{4.401935in}{4.288342in}}{\pgfqpoint{4.390885in}{4.288342in}}%
\pgfpathcurveto{\pgfqpoint{4.379835in}{4.288342in}}{\pgfqpoint{4.369236in}{4.283951in}}{\pgfqpoint{4.361423in}{4.276138in}}%
\pgfpathcurveto{\pgfqpoint{4.353609in}{4.268324in}}{\pgfqpoint{4.349219in}{4.257725in}}{\pgfqpoint{4.349219in}{4.246675in}}%
\pgfpathcurveto{\pgfqpoint{4.349219in}{4.235625in}}{\pgfqpoint{4.353609in}{4.225026in}}{\pgfqpoint{4.361423in}{4.217212in}}%
\pgfpathcurveto{\pgfqpoint{4.369236in}{4.209399in}}{\pgfqpoint{4.379835in}{4.205008in}}{\pgfqpoint{4.390885in}{4.205008in}}%
\pgfpathclose%
\pgfusepath{stroke,fill}%
\end{pgfscope}%
\begin{pgfscope}%
\pgfpathrectangle{\pgfqpoint{0.481978in}{0.331635in}}{\pgfqpoint{9.300000in}{7.700000in}}%
\pgfusepath{clip}%
\pgfsetbuttcap%
\pgfsetroundjoin%
\definecolor{currentfill}{rgb}{1.000000,0.705882,0.509804}%
\pgfsetfillcolor{currentfill}%
\pgfsetlinewidth{0.481800pt}%
\definecolor{currentstroke}{rgb}{1.000000,1.000000,1.000000}%
\pgfsetstrokecolor{currentstroke}%
\pgfsetdash{}{0pt}%
\pgfpathmoveto{\pgfqpoint{2.103399in}{2.744470in}}%
\pgfpathcurveto{\pgfqpoint{2.114449in}{2.744470in}}{\pgfqpoint{2.125048in}{2.748860in}}{\pgfqpoint{2.132862in}{2.756674in}}%
\pgfpathcurveto{\pgfqpoint{2.140676in}{2.764488in}}{\pgfqpoint{2.145066in}{2.775087in}}{\pgfqpoint{2.145066in}{2.786137in}}%
\pgfpathcurveto{\pgfqpoint{2.145066in}{2.797187in}}{\pgfqpoint{2.140676in}{2.807786in}}{\pgfqpoint{2.132862in}{2.815599in}}%
\pgfpathcurveto{\pgfqpoint{2.125048in}{2.823413in}}{\pgfqpoint{2.114449in}{2.827803in}}{\pgfqpoint{2.103399in}{2.827803in}}%
\pgfpathcurveto{\pgfqpoint{2.092349in}{2.827803in}}{\pgfqpoint{2.081750in}{2.823413in}}{\pgfqpoint{2.073937in}{2.815599in}}%
\pgfpathcurveto{\pgfqpoint{2.066123in}{2.807786in}}{\pgfqpoint{2.061733in}{2.797187in}}{\pgfqpoint{2.061733in}{2.786137in}}%
\pgfpathcurveto{\pgfqpoint{2.061733in}{2.775087in}}{\pgfqpoint{2.066123in}{2.764488in}}{\pgfqpoint{2.073937in}{2.756674in}}%
\pgfpathcurveto{\pgfqpoint{2.081750in}{2.748860in}}{\pgfqpoint{2.092349in}{2.744470in}}{\pgfqpoint{2.103399in}{2.744470in}}%
\pgfpathclose%
\pgfusepath{stroke,fill}%
\end{pgfscope}%
\begin{pgfscope}%
\pgfpathrectangle{\pgfqpoint{0.481978in}{0.331635in}}{\pgfqpoint{9.300000in}{7.700000in}}%
\pgfusepath{clip}%
\pgfsetbuttcap%
\pgfsetroundjoin%
\definecolor{currentfill}{rgb}{1.000000,0.705882,0.509804}%
\pgfsetfillcolor{currentfill}%
\pgfsetlinewidth{0.481800pt}%
\definecolor{currentstroke}{rgb}{1.000000,1.000000,1.000000}%
\pgfsetstrokecolor{currentstroke}%
\pgfsetdash{}{0pt}%
\pgfpathmoveto{\pgfqpoint{3.156352in}{3.653836in}}%
\pgfpathcurveto{\pgfqpoint{3.167402in}{3.653836in}}{\pgfqpoint{3.178001in}{3.658226in}}{\pgfqpoint{3.185815in}{3.666040in}}%
\pgfpathcurveto{\pgfqpoint{3.193628in}{3.673854in}}{\pgfqpoint{3.198018in}{3.684453in}}{\pgfqpoint{3.198018in}{3.695503in}}%
\pgfpathcurveto{\pgfqpoint{3.198018in}{3.706553in}}{\pgfqpoint{3.193628in}{3.717152in}}{\pgfqpoint{3.185815in}{3.724966in}}%
\pgfpathcurveto{\pgfqpoint{3.178001in}{3.732779in}}{\pgfqpoint{3.167402in}{3.737169in}}{\pgfqpoint{3.156352in}{3.737169in}}%
\pgfpathcurveto{\pgfqpoint{3.145302in}{3.737169in}}{\pgfqpoint{3.134703in}{3.732779in}}{\pgfqpoint{3.126889in}{3.724966in}}%
\pgfpathcurveto{\pgfqpoint{3.119075in}{3.717152in}}{\pgfqpoint{3.114685in}{3.706553in}}{\pgfqpoint{3.114685in}{3.695503in}}%
\pgfpathcurveto{\pgfqpoint{3.114685in}{3.684453in}}{\pgfqpoint{3.119075in}{3.673854in}}{\pgfqpoint{3.126889in}{3.666040in}}%
\pgfpathcurveto{\pgfqpoint{3.134703in}{3.658226in}}{\pgfqpoint{3.145302in}{3.653836in}}{\pgfqpoint{3.156352in}{3.653836in}}%
\pgfpathclose%
\pgfusepath{stroke,fill}%
\end{pgfscope}%
\begin{pgfscope}%
\pgfpathrectangle{\pgfqpoint{0.481978in}{0.331635in}}{\pgfqpoint{9.300000in}{7.700000in}}%
\pgfusepath{clip}%
\pgfsetbuttcap%
\pgfsetroundjoin%
\definecolor{currentfill}{rgb}{1.000000,0.705882,0.509804}%
\pgfsetfillcolor{currentfill}%
\pgfsetlinewidth{0.481800pt}%
\definecolor{currentstroke}{rgb}{1.000000,1.000000,1.000000}%
\pgfsetstrokecolor{currentstroke}%
\pgfsetdash{}{0pt}%
\pgfpathmoveto{\pgfqpoint{4.782434in}{5.369392in}}%
\pgfpathcurveto{\pgfqpoint{4.793484in}{5.369392in}}{\pgfqpoint{4.804083in}{5.373782in}}{\pgfqpoint{4.811897in}{5.381596in}}%
\pgfpathcurveto{\pgfqpoint{4.819711in}{5.389410in}}{\pgfqpoint{4.824101in}{5.400009in}}{\pgfqpoint{4.824101in}{5.411059in}}%
\pgfpathcurveto{\pgfqpoint{4.824101in}{5.422109in}}{\pgfqpoint{4.819711in}{5.432708in}}{\pgfqpoint{4.811897in}{5.440522in}}%
\pgfpathcurveto{\pgfqpoint{4.804083in}{5.448335in}}{\pgfqpoint{4.793484in}{5.452725in}}{\pgfqpoint{4.782434in}{5.452725in}}%
\pgfpathcurveto{\pgfqpoint{4.771384in}{5.452725in}}{\pgfqpoint{4.760785in}{5.448335in}}{\pgfqpoint{4.752972in}{5.440522in}}%
\pgfpathcurveto{\pgfqpoint{4.745158in}{5.432708in}}{\pgfqpoint{4.740768in}{5.422109in}}{\pgfqpoint{4.740768in}{5.411059in}}%
\pgfpathcurveto{\pgfqpoint{4.740768in}{5.400009in}}{\pgfqpoint{4.745158in}{5.389410in}}{\pgfqpoint{4.752972in}{5.381596in}}%
\pgfpathcurveto{\pgfqpoint{4.760785in}{5.373782in}}{\pgfqpoint{4.771384in}{5.369392in}}{\pgfqpoint{4.782434in}{5.369392in}}%
\pgfpathclose%
\pgfusepath{stroke,fill}%
\end{pgfscope}%
\begin{pgfscope}%
\pgfpathrectangle{\pgfqpoint{0.481978in}{0.331635in}}{\pgfqpoint{9.300000in}{7.700000in}}%
\pgfusepath{clip}%
\pgfsetbuttcap%
\pgfsetroundjoin%
\definecolor{currentfill}{rgb}{1.000000,0.705882,0.509804}%
\pgfsetfillcolor{currentfill}%
\pgfsetlinewidth{0.481800pt}%
\definecolor{currentstroke}{rgb}{1.000000,1.000000,1.000000}%
\pgfsetstrokecolor{currentstroke}%
\pgfsetdash{}{0pt}%
\pgfpathmoveto{\pgfqpoint{4.313894in}{2.562157in}}%
\pgfpathcurveto{\pgfqpoint{4.324944in}{2.562157in}}{\pgfqpoint{4.335543in}{2.566547in}}{\pgfqpoint{4.343356in}{2.574361in}}%
\pgfpathcurveto{\pgfqpoint{4.351170in}{2.582174in}}{\pgfqpoint{4.355560in}{2.592773in}}{\pgfqpoint{4.355560in}{2.603823in}}%
\pgfpathcurveto{\pgfqpoint{4.355560in}{2.614874in}}{\pgfqpoint{4.351170in}{2.625473in}}{\pgfqpoint{4.343356in}{2.633286in}}%
\pgfpathcurveto{\pgfqpoint{4.335543in}{2.641100in}}{\pgfqpoint{4.324944in}{2.645490in}}{\pgfqpoint{4.313894in}{2.645490in}}%
\pgfpathcurveto{\pgfqpoint{4.302843in}{2.645490in}}{\pgfqpoint{4.292244in}{2.641100in}}{\pgfqpoint{4.284431in}{2.633286in}}%
\pgfpathcurveto{\pgfqpoint{4.276617in}{2.625473in}}{\pgfqpoint{4.272227in}{2.614874in}}{\pgfqpoint{4.272227in}{2.603823in}}%
\pgfpathcurveto{\pgfqpoint{4.272227in}{2.592773in}}{\pgfqpoint{4.276617in}{2.582174in}}{\pgfqpoint{4.284431in}{2.574361in}}%
\pgfpathcurveto{\pgfqpoint{4.292244in}{2.566547in}}{\pgfqpoint{4.302843in}{2.562157in}}{\pgfqpoint{4.313894in}{2.562157in}}%
\pgfpathclose%
\pgfusepath{stroke,fill}%
\end{pgfscope}%
\begin{pgfscope}%
\pgfpathrectangle{\pgfqpoint{0.481978in}{0.331635in}}{\pgfqpoint{9.300000in}{7.700000in}}%
\pgfusepath{clip}%
\pgfsetbuttcap%
\pgfsetroundjoin%
\definecolor{currentfill}{rgb}{1.000000,0.705882,0.509804}%
\pgfsetfillcolor{currentfill}%
\pgfsetlinewidth{0.481800pt}%
\definecolor{currentstroke}{rgb}{1.000000,1.000000,1.000000}%
\pgfsetstrokecolor{currentstroke}%
\pgfsetdash{}{0pt}%
\pgfpathmoveto{\pgfqpoint{3.534223in}{5.881213in}}%
\pgfpathcurveto{\pgfqpoint{3.545273in}{5.881213in}}{\pgfqpoint{3.555872in}{5.885603in}}{\pgfqpoint{3.563686in}{5.893417in}}%
\pgfpathcurveto{\pgfqpoint{3.571500in}{5.901230in}}{\pgfqpoint{3.575890in}{5.911829in}}{\pgfqpoint{3.575890in}{5.922879in}}%
\pgfpathcurveto{\pgfqpoint{3.575890in}{5.933929in}}{\pgfqpoint{3.571500in}{5.944528in}}{\pgfqpoint{3.563686in}{5.952342in}}%
\pgfpathcurveto{\pgfqpoint{3.555872in}{5.960156in}}{\pgfqpoint{3.545273in}{5.964546in}}{\pgfqpoint{3.534223in}{5.964546in}}%
\pgfpathcurveto{\pgfqpoint{3.523173in}{5.964546in}}{\pgfqpoint{3.512574in}{5.960156in}}{\pgfqpoint{3.504760in}{5.952342in}}%
\pgfpathcurveto{\pgfqpoint{3.496947in}{5.944528in}}{\pgfqpoint{3.492556in}{5.933929in}}{\pgfqpoint{3.492556in}{5.922879in}}%
\pgfpathcurveto{\pgfqpoint{3.492556in}{5.911829in}}{\pgfqpoint{3.496947in}{5.901230in}}{\pgfqpoint{3.504760in}{5.893417in}}%
\pgfpathcurveto{\pgfqpoint{3.512574in}{5.885603in}}{\pgfqpoint{3.523173in}{5.881213in}}{\pgfqpoint{3.534223in}{5.881213in}}%
\pgfpathclose%
\pgfusepath{stroke,fill}%
\end{pgfscope}%
\begin{pgfscope}%
\pgfpathrectangle{\pgfqpoint{0.481978in}{0.331635in}}{\pgfqpoint{9.300000in}{7.700000in}}%
\pgfusepath{clip}%
\pgfsetbuttcap%
\pgfsetroundjoin%
\definecolor{currentfill}{rgb}{1.000000,0.705882,0.509804}%
\pgfsetfillcolor{currentfill}%
\pgfsetlinewidth{0.481800pt}%
\definecolor{currentstroke}{rgb}{1.000000,1.000000,1.000000}%
\pgfsetstrokecolor{currentstroke}%
\pgfsetdash{}{0pt}%
\pgfpathmoveto{\pgfqpoint{2.718188in}{5.658172in}}%
\pgfpathcurveto{\pgfqpoint{2.729238in}{5.658172in}}{\pgfqpoint{2.739837in}{5.662562in}}{\pgfqpoint{2.747651in}{5.670376in}}%
\pgfpathcurveto{\pgfqpoint{2.755465in}{5.678190in}}{\pgfqpoint{2.759855in}{5.688789in}}{\pgfqpoint{2.759855in}{5.699839in}}%
\pgfpathcurveto{\pgfqpoint{2.759855in}{5.710889in}}{\pgfqpoint{2.755465in}{5.721488in}}{\pgfqpoint{2.747651in}{5.729302in}}%
\pgfpathcurveto{\pgfqpoint{2.739837in}{5.737115in}}{\pgfqpoint{2.729238in}{5.741506in}}{\pgfqpoint{2.718188in}{5.741506in}}%
\pgfpathcurveto{\pgfqpoint{2.707138in}{5.741506in}}{\pgfqpoint{2.696539in}{5.737115in}}{\pgfqpoint{2.688726in}{5.729302in}}%
\pgfpathcurveto{\pgfqpoint{2.680912in}{5.721488in}}{\pgfqpoint{2.676522in}{5.710889in}}{\pgfqpoint{2.676522in}{5.699839in}}%
\pgfpathcurveto{\pgfqpoint{2.676522in}{5.688789in}}{\pgfqpoint{2.680912in}{5.678190in}}{\pgfqpoint{2.688726in}{5.670376in}}%
\pgfpathcurveto{\pgfqpoint{2.696539in}{5.662562in}}{\pgfqpoint{2.707138in}{5.658172in}}{\pgfqpoint{2.718188in}{5.658172in}}%
\pgfpathclose%
\pgfusepath{stroke,fill}%
\end{pgfscope}%
\begin{pgfscope}%
\pgfpathrectangle{\pgfqpoint{0.481978in}{0.331635in}}{\pgfqpoint{9.300000in}{7.700000in}}%
\pgfusepath{clip}%
\pgfsetbuttcap%
\pgfsetroundjoin%
\definecolor{currentfill}{rgb}{1.000000,0.705882,0.509804}%
\pgfsetfillcolor{currentfill}%
\pgfsetlinewidth{0.481800pt}%
\definecolor{currentstroke}{rgb}{1.000000,1.000000,1.000000}%
\pgfsetstrokecolor{currentstroke}%
\pgfsetdash{}{0pt}%
\pgfpathmoveto{\pgfqpoint{4.798223in}{3.912894in}}%
\pgfpathcurveto{\pgfqpoint{4.809274in}{3.912894in}}{\pgfqpoint{4.819873in}{3.917284in}}{\pgfqpoint{4.827686in}{3.925098in}}%
\pgfpathcurveto{\pgfqpoint{4.835500in}{3.932911in}}{\pgfqpoint{4.839890in}{3.943510in}}{\pgfqpoint{4.839890in}{3.954560in}}%
\pgfpathcurveto{\pgfqpoint{4.839890in}{3.965610in}}{\pgfqpoint{4.835500in}{3.976210in}}{\pgfqpoint{4.827686in}{3.984023in}}%
\pgfpathcurveto{\pgfqpoint{4.819873in}{3.991837in}}{\pgfqpoint{4.809274in}{3.996227in}}{\pgfqpoint{4.798223in}{3.996227in}}%
\pgfpathcurveto{\pgfqpoint{4.787173in}{3.996227in}}{\pgfqpoint{4.776574in}{3.991837in}}{\pgfqpoint{4.768761in}{3.984023in}}%
\pgfpathcurveto{\pgfqpoint{4.760947in}{3.976210in}}{\pgfqpoint{4.756557in}{3.965610in}}{\pgfqpoint{4.756557in}{3.954560in}}%
\pgfpathcurveto{\pgfqpoint{4.756557in}{3.943510in}}{\pgfqpoint{4.760947in}{3.932911in}}{\pgfqpoint{4.768761in}{3.925098in}}%
\pgfpathcurveto{\pgfqpoint{4.776574in}{3.917284in}}{\pgfqpoint{4.787173in}{3.912894in}}{\pgfqpoint{4.798223in}{3.912894in}}%
\pgfpathclose%
\pgfusepath{stroke,fill}%
\end{pgfscope}%
\begin{pgfscope}%
\pgfpathrectangle{\pgfqpoint{0.481978in}{0.331635in}}{\pgfqpoint{9.300000in}{7.700000in}}%
\pgfusepath{clip}%
\pgfsetbuttcap%
\pgfsetroundjoin%
\definecolor{currentfill}{rgb}{1.000000,0.705882,0.509804}%
\pgfsetfillcolor{currentfill}%
\pgfsetlinewidth{0.481800pt}%
\definecolor{currentstroke}{rgb}{1.000000,1.000000,1.000000}%
\pgfsetstrokecolor{currentstroke}%
\pgfsetdash{}{0pt}%
\pgfpathmoveto{\pgfqpoint{2.688115in}{4.275920in}}%
\pgfpathcurveto{\pgfqpoint{2.699165in}{4.275920in}}{\pgfqpoint{2.709764in}{4.280311in}}{\pgfqpoint{2.717578in}{4.288124in}}%
\pgfpathcurveto{\pgfqpoint{2.725392in}{4.295938in}}{\pgfqpoint{2.729782in}{4.306537in}}{\pgfqpoint{2.729782in}{4.317587in}}%
\pgfpathcurveto{\pgfqpoint{2.729782in}{4.328637in}}{\pgfqpoint{2.725392in}{4.339236in}}{\pgfqpoint{2.717578in}{4.347050in}}%
\pgfpathcurveto{\pgfqpoint{2.709764in}{4.354863in}}{\pgfqpoint{2.699165in}{4.359254in}}{\pgfqpoint{2.688115in}{4.359254in}}%
\pgfpathcurveto{\pgfqpoint{2.677065in}{4.359254in}}{\pgfqpoint{2.666466in}{4.354863in}}{\pgfqpoint{2.658652in}{4.347050in}}%
\pgfpathcurveto{\pgfqpoint{2.650839in}{4.339236in}}{\pgfqpoint{2.646449in}{4.328637in}}{\pgfqpoint{2.646449in}{4.317587in}}%
\pgfpathcurveto{\pgfqpoint{2.646449in}{4.306537in}}{\pgfqpoint{2.650839in}{4.295938in}}{\pgfqpoint{2.658652in}{4.288124in}}%
\pgfpathcurveto{\pgfqpoint{2.666466in}{4.280311in}}{\pgfqpoint{2.677065in}{4.275920in}}{\pgfqpoint{2.688115in}{4.275920in}}%
\pgfpathclose%
\pgfusepath{stroke,fill}%
\end{pgfscope}%
\begin{pgfscope}%
\pgfpathrectangle{\pgfqpoint{0.481978in}{0.331635in}}{\pgfqpoint{9.300000in}{7.700000in}}%
\pgfusepath{clip}%
\pgfsetbuttcap%
\pgfsetroundjoin%
\definecolor{currentfill}{rgb}{1.000000,0.705882,0.509804}%
\pgfsetfillcolor{currentfill}%
\pgfsetlinewidth{0.481800pt}%
\definecolor{currentstroke}{rgb}{1.000000,1.000000,1.000000}%
\pgfsetstrokecolor{currentstroke}%
\pgfsetdash{}{0pt}%
\pgfpathmoveto{\pgfqpoint{3.356009in}{3.291155in}}%
\pgfpathcurveto{\pgfqpoint{3.367060in}{3.291155in}}{\pgfqpoint{3.377659in}{3.295545in}}{\pgfqpoint{3.385472in}{3.303359in}}%
\pgfpathcurveto{\pgfqpoint{3.393286in}{3.311173in}}{\pgfqpoint{3.397676in}{3.321772in}}{\pgfqpoint{3.397676in}{3.332822in}}%
\pgfpathcurveto{\pgfqpoint{3.397676in}{3.343872in}}{\pgfqpoint{3.393286in}{3.354471in}}{\pgfqpoint{3.385472in}{3.362285in}}%
\pgfpathcurveto{\pgfqpoint{3.377659in}{3.370098in}}{\pgfqpoint{3.367060in}{3.374488in}}{\pgfqpoint{3.356009in}{3.374488in}}%
\pgfpathcurveto{\pgfqpoint{3.344959in}{3.374488in}}{\pgfqpoint{3.334360in}{3.370098in}}{\pgfqpoint{3.326547in}{3.362285in}}%
\pgfpathcurveto{\pgfqpoint{3.318733in}{3.354471in}}{\pgfqpoint{3.314343in}{3.343872in}}{\pgfqpoint{3.314343in}{3.332822in}}%
\pgfpathcurveto{\pgfqpoint{3.314343in}{3.321772in}}{\pgfqpoint{3.318733in}{3.311173in}}{\pgfqpoint{3.326547in}{3.303359in}}%
\pgfpathcurveto{\pgfqpoint{3.334360in}{3.295545in}}{\pgfqpoint{3.344959in}{3.291155in}}{\pgfqpoint{3.356009in}{3.291155in}}%
\pgfpathclose%
\pgfusepath{stroke,fill}%
\end{pgfscope}%
\begin{pgfscope}%
\pgfpathrectangle{\pgfqpoint{0.481978in}{0.331635in}}{\pgfqpoint{9.300000in}{7.700000in}}%
\pgfusepath{clip}%
\pgfsetbuttcap%
\pgfsetroundjoin%
\definecolor{currentfill}{rgb}{1.000000,0.705882,0.509804}%
\pgfsetfillcolor{currentfill}%
\pgfsetlinewidth{0.481800pt}%
\definecolor{currentstroke}{rgb}{1.000000,1.000000,1.000000}%
\pgfsetstrokecolor{currentstroke}%
\pgfsetdash{}{0pt}%
\pgfpathmoveto{\pgfqpoint{2.750103in}{4.713490in}}%
\pgfpathcurveto{\pgfqpoint{2.761153in}{4.713490in}}{\pgfqpoint{2.771752in}{4.717880in}}{\pgfqpoint{2.779565in}{4.725694in}}%
\pgfpathcurveto{\pgfqpoint{2.787379in}{4.733507in}}{\pgfqpoint{2.791769in}{4.744106in}}{\pgfqpoint{2.791769in}{4.755156in}}%
\pgfpathcurveto{\pgfqpoint{2.791769in}{4.766206in}}{\pgfqpoint{2.787379in}{4.776805in}}{\pgfqpoint{2.779565in}{4.784619in}}%
\pgfpathcurveto{\pgfqpoint{2.771752in}{4.792433in}}{\pgfqpoint{2.761153in}{4.796823in}}{\pgfqpoint{2.750103in}{4.796823in}}%
\pgfpathcurveto{\pgfqpoint{2.739052in}{4.796823in}}{\pgfqpoint{2.728453in}{4.792433in}}{\pgfqpoint{2.720640in}{4.784619in}}%
\pgfpathcurveto{\pgfqpoint{2.712826in}{4.776805in}}{\pgfqpoint{2.708436in}{4.766206in}}{\pgfqpoint{2.708436in}{4.755156in}}%
\pgfpathcurveto{\pgfqpoint{2.708436in}{4.744106in}}{\pgfqpoint{2.712826in}{4.733507in}}{\pgfqpoint{2.720640in}{4.725694in}}%
\pgfpathcurveto{\pgfqpoint{2.728453in}{4.717880in}}{\pgfqpoint{2.739052in}{4.713490in}}{\pgfqpoint{2.750103in}{4.713490in}}%
\pgfpathclose%
\pgfusepath{stroke,fill}%
\end{pgfscope}%
\begin{pgfscope}%
\pgfpathrectangle{\pgfqpoint{0.481978in}{0.331635in}}{\pgfqpoint{9.300000in}{7.700000in}}%
\pgfusepath{clip}%
\pgfsetbuttcap%
\pgfsetroundjoin%
\definecolor{currentfill}{rgb}{1.000000,0.705882,0.509804}%
\pgfsetfillcolor{currentfill}%
\pgfsetlinewidth{0.481800pt}%
\definecolor{currentstroke}{rgb}{1.000000,1.000000,1.000000}%
\pgfsetstrokecolor{currentstroke}%
\pgfsetdash{}{0pt}%
\pgfpathmoveto{\pgfqpoint{3.502659in}{2.513832in}}%
\pgfpathcurveto{\pgfqpoint{3.513709in}{2.513832in}}{\pgfqpoint{3.524308in}{2.518222in}}{\pgfqpoint{3.532122in}{2.526036in}}%
\pgfpathcurveto{\pgfqpoint{3.539935in}{2.533850in}}{\pgfqpoint{3.544326in}{2.544449in}}{\pgfqpoint{3.544326in}{2.555499in}}%
\pgfpathcurveto{\pgfqpoint{3.544326in}{2.566549in}}{\pgfqpoint{3.539935in}{2.577148in}}{\pgfqpoint{3.532122in}{2.584962in}}%
\pgfpathcurveto{\pgfqpoint{3.524308in}{2.592775in}}{\pgfqpoint{3.513709in}{2.597166in}}{\pgfqpoint{3.502659in}{2.597166in}}%
\pgfpathcurveto{\pgfqpoint{3.491609in}{2.597166in}}{\pgfqpoint{3.481010in}{2.592775in}}{\pgfqpoint{3.473196in}{2.584962in}}%
\pgfpathcurveto{\pgfqpoint{3.465383in}{2.577148in}}{\pgfqpoint{3.460992in}{2.566549in}}{\pgfqpoint{3.460992in}{2.555499in}}%
\pgfpathcurveto{\pgfqpoint{3.460992in}{2.544449in}}{\pgfqpoint{3.465383in}{2.533850in}}{\pgfqpoint{3.473196in}{2.526036in}}%
\pgfpathcurveto{\pgfqpoint{3.481010in}{2.518222in}}{\pgfqpoint{3.491609in}{2.513832in}}{\pgfqpoint{3.502659in}{2.513832in}}%
\pgfpathclose%
\pgfusepath{stroke,fill}%
\end{pgfscope}%
\begin{pgfscope}%
\pgfpathrectangle{\pgfqpoint{0.481978in}{0.331635in}}{\pgfqpoint{9.300000in}{7.700000in}}%
\pgfusepath{clip}%
\pgfsetbuttcap%
\pgfsetroundjoin%
\definecolor{currentfill}{rgb}{1.000000,0.705882,0.509804}%
\pgfsetfillcolor{currentfill}%
\pgfsetlinewidth{0.481800pt}%
\definecolor{currentstroke}{rgb}{1.000000,1.000000,1.000000}%
\pgfsetstrokecolor{currentstroke}%
\pgfsetdash{}{0pt}%
\pgfpathmoveto{\pgfqpoint{3.883915in}{4.705845in}}%
\pgfpathcurveto{\pgfqpoint{3.894965in}{4.705845in}}{\pgfqpoint{3.905564in}{4.710236in}}{\pgfqpoint{3.913378in}{4.718049in}}%
\pgfpathcurveto{\pgfqpoint{3.921192in}{4.725863in}}{\pgfqpoint{3.925582in}{4.736462in}}{\pgfqpoint{3.925582in}{4.747512in}}%
\pgfpathcurveto{\pgfqpoint{3.925582in}{4.758562in}}{\pgfqpoint{3.921192in}{4.769161in}}{\pgfqpoint{3.913378in}{4.776975in}}%
\pgfpathcurveto{\pgfqpoint{3.905564in}{4.784788in}}{\pgfqpoint{3.894965in}{4.789179in}}{\pgfqpoint{3.883915in}{4.789179in}}%
\pgfpathcurveto{\pgfqpoint{3.872865in}{4.789179in}}{\pgfqpoint{3.862266in}{4.784788in}}{\pgfqpoint{3.854452in}{4.776975in}}%
\pgfpathcurveto{\pgfqpoint{3.846639in}{4.769161in}}{\pgfqpoint{3.842249in}{4.758562in}}{\pgfqpoint{3.842249in}{4.747512in}}%
\pgfpathcurveto{\pgfqpoint{3.842249in}{4.736462in}}{\pgfqpoint{3.846639in}{4.725863in}}{\pgfqpoint{3.854452in}{4.718049in}}%
\pgfpathcurveto{\pgfqpoint{3.862266in}{4.710236in}}{\pgfqpoint{3.872865in}{4.705845in}}{\pgfqpoint{3.883915in}{4.705845in}}%
\pgfpathclose%
\pgfusepath{stroke,fill}%
\end{pgfscope}%
\begin{pgfscope}%
\pgfpathrectangle{\pgfqpoint{0.481978in}{0.331635in}}{\pgfqpoint{9.300000in}{7.700000in}}%
\pgfusepath{clip}%
\pgfsetbuttcap%
\pgfsetroundjoin%
\definecolor{currentfill}{rgb}{1.000000,0.705882,0.509804}%
\pgfsetfillcolor{currentfill}%
\pgfsetlinewidth{0.481800pt}%
\definecolor{currentstroke}{rgb}{1.000000,1.000000,1.000000}%
\pgfsetstrokecolor{currentstroke}%
\pgfsetdash{}{0pt}%
\pgfpathmoveto{\pgfqpoint{4.261705in}{2.302620in}}%
\pgfpathcurveto{\pgfqpoint{4.272755in}{2.302620in}}{\pgfqpoint{4.283354in}{2.307010in}}{\pgfqpoint{4.291168in}{2.314824in}}%
\pgfpathcurveto{\pgfqpoint{4.298982in}{2.322638in}}{\pgfqpoint{4.303372in}{2.333237in}}{\pgfqpoint{4.303372in}{2.344287in}}%
\pgfpathcurveto{\pgfqpoint{4.303372in}{2.355337in}}{\pgfqpoint{4.298982in}{2.365936in}}{\pgfqpoint{4.291168in}{2.373750in}}%
\pgfpathcurveto{\pgfqpoint{4.283354in}{2.381563in}}{\pgfqpoint{4.272755in}{2.385953in}}{\pgfqpoint{4.261705in}{2.385953in}}%
\pgfpathcurveto{\pgfqpoint{4.250655in}{2.385953in}}{\pgfqpoint{4.240056in}{2.381563in}}{\pgfqpoint{4.232243in}{2.373750in}}%
\pgfpathcurveto{\pgfqpoint{4.224429in}{2.365936in}}{\pgfqpoint{4.220039in}{2.355337in}}{\pgfqpoint{4.220039in}{2.344287in}}%
\pgfpathcurveto{\pgfqpoint{4.220039in}{2.333237in}}{\pgfqpoint{4.224429in}{2.322638in}}{\pgfqpoint{4.232243in}{2.314824in}}%
\pgfpathcurveto{\pgfqpoint{4.240056in}{2.307010in}}{\pgfqpoint{4.250655in}{2.302620in}}{\pgfqpoint{4.261705in}{2.302620in}}%
\pgfpathclose%
\pgfusepath{stroke,fill}%
\end{pgfscope}%
\begin{pgfscope}%
\pgfpathrectangle{\pgfqpoint{0.481978in}{0.331635in}}{\pgfqpoint{9.300000in}{7.700000in}}%
\pgfusepath{clip}%
\pgfsetbuttcap%
\pgfsetroundjoin%
\definecolor{currentfill}{rgb}{1.000000,0.705882,0.509804}%
\pgfsetfillcolor{currentfill}%
\pgfsetlinewidth{0.481800pt}%
\definecolor{currentstroke}{rgb}{1.000000,1.000000,1.000000}%
\pgfsetstrokecolor{currentstroke}%
\pgfsetdash{}{0pt}%
\pgfpathmoveto{\pgfqpoint{1.767331in}{4.627608in}}%
\pgfpathcurveto{\pgfqpoint{1.778382in}{4.627608in}}{\pgfqpoint{1.788981in}{4.631998in}}{\pgfqpoint{1.796794in}{4.639811in}}%
\pgfpathcurveto{\pgfqpoint{1.804608in}{4.647625in}}{\pgfqpoint{1.808998in}{4.658224in}}{\pgfqpoint{1.808998in}{4.669274in}}%
\pgfpathcurveto{\pgfqpoint{1.808998in}{4.680324in}}{\pgfqpoint{1.804608in}{4.690923in}}{\pgfqpoint{1.796794in}{4.698737in}}%
\pgfpathcurveto{\pgfqpoint{1.788981in}{4.706551in}}{\pgfqpoint{1.778382in}{4.710941in}}{\pgfqpoint{1.767331in}{4.710941in}}%
\pgfpathcurveto{\pgfqpoint{1.756281in}{4.710941in}}{\pgfqpoint{1.745682in}{4.706551in}}{\pgfqpoint{1.737869in}{4.698737in}}%
\pgfpathcurveto{\pgfqpoint{1.730055in}{4.690923in}}{\pgfqpoint{1.725665in}{4.680324in}}{\pgfqpoint{1.725665in}{4.669274in}}%
\pgfpathcurveto{\pgfqpoint{1.725665in}{4.658224in}}{\pgfqpoint{1.730055in}{4.647625in}}{\pgfqpoint{1.737869in}{4.639811in}}%
\pgfpathcurveto{\pgfqpoint{1.745682in}{4.631998in}}{\pgfqpoint{1.756281in}{4.627608in}}{\pgfqpoint{1.767331in}{4.627608in}}%
\pgfpathclose%
\pgfusepath{stroke,fill}%
\end{pgfscope}%
\begin{pgfscope}%
\pgfpathrectangle{\pgfqpoint{0.481978in}{0.331635in}}{\pgfqpoint{9.300000in}{7.700000in}}%
\pgfusepath{clip}%
\pgfsetbuttcap%
\pgfsetroundjoin%
\definecolor{currentfill}{rgb}{1.000000,0.705882,0.509804}%
\pgfsetfillcolor{currentfill}%
\pgfsetlinewidth{0.481800pt}%
\definecolor{currentstroke}{rgb}{1.000000,1.000000,1.000000}%
\pgfsetstrokecolor{currentstroke}%
\pgfsetdash{}{0pt}%
\pgfpathmoveto{\pgfqpoint{1.457321in}{4.643687in}}%
\pgfpathcurveto{\pgfqpoint{1.468371in}{4.643687in}}{\pgfqpoint{1.478970in}{4.648077in}}{\pgfqpoint{1.486784in}{4.655891in}}%
\pgfpathcurveto{\pgfqpoint{1.494597in}{4.663705in}}{\pgfqpoint{1.498988in}{4.674304in}}{\pgfqpoint{1.498988in}{4.685354in}}%
\pgfpathcurveto{\pgfqpoint{1.498988in}{4.696404in}}{\pgfqpoint{1.494597in}{4.707003in}}{\pgfqpoint{1.486784in}{4.714817in}}%
\pgfpathcurveto{\pgfqpoint{1.478970in}{4.722630in}}{\pgfqpoint{1.468371in}{4.727020in}}{\pgfqpoint{1.457321in}{4.727020in}}%
\pgfpathcurveto{\pgfqpoint{1.446271in}{4.727020in}}{\pgfqpoint{1.435672in}{4.722630in}}{\pgfqpoint{1.427858in}{4.714817in}}%
\pgfpathcurveto{\pgfqpoint{1.420045in}{4.707003in}}{\pgfqpoint{1.415654in}{4.696404in}}{\pgfqpoint{1.415654in}{4.685354in}}%
\pgfpathcurveto{\pgfqpoint{1.415654in}{4.674304in}}{\pgfqpoint{1.420045in}{4.663705in}}{\pgfqpoint{1.427858in}{4.655891in}}%
\pgfpathcurveto{\pgfqpoint{1.435672in}{4.648077in}}{\pgfqpoint{1.446271in}{4.643687in}}{\pgfqpoint{1.457321in}{4.643687in}}%
\pgfpathclose%
\pgfusepath{stroke,fill}%
\end{pgfscope}%
\begin{pgfscope}%
\pgfpathrectangle{\pgfqpoint{0.481978in}{0.331635in}}{\pgfqpoint{9.300000in}{7.700000in}}%
\pgfusepath{clip}%
\pgfsetbuttcap%
\pgfsetroundjoin%
\definecolor{currentfill}{rgb}{1.000000,0.705882,0.509804}%
\pgfsetfillcolor{currentfill}%
\pgfsetlinewidth{0.481800pt}%
\definecolor{currentstroke}{rgb}{1.000000,1.000000,1.000000}%
\pgfsetstrokecolor{currentstroke}%
\pgfsetdash{}{0pt}%
\pgfpathmoveto{\pgfqpoint{3.872133in}{3.246559in}}%
\pgfpathcurveto{\pgfqpoint{3.883183in}{3.246559in}}{\pgfqpoint{3.893782in}{3.250949in}}{\pgfqpoint{3.901596in}{3.258763in}}%
\pgfpathcurveto{\pgfqpoint{3.909410in}{3.266576in}}{\pgfqpoint{3.913800in}{3.277175in}}{\pgfqpoint{3.913800in}{3.288225in}}%
\pgfpathcurveto{\pgfqpoint{3.913800in}{3.299276in}}{\pgfqpoint{3.909410in}{3.309875in}}{\pgfqpoint{3.901596in}{3.317688in}}%
\pgfpathcurveto{\pgfqpoint{3.893782in}{3.325502in}}{\pgfqpoint{3.883183in}{3.329892in}}{\pgfqpoint{3.872133in}{3.329892in}}%
\pgfpathcurveto{\pgfqpoint{3.861083in}{3.329892in}}{\pgfqpoint{3.850484in}{3.325502in}}{\pgfqpoint{3.842670in}{3.317688in}}%
\pgfpathcurveto{\pgfqpoint{3.834857in}{3.309875in}}{\pgfqpoint{3.830466in}{3.299276in}}{\pgfqpoint{3.830466in}{3.288225in}}%
\pgfpathcurveto{\pgfqpoint{3.830466in}{3.277175in}}{\pgfqpoint{3.834857in}{3.266576in}}{\pgfqpoint{3.842670in}{3.258763in}}%
\pgfpathcurveto{\pgfqpoint{3.850484in}{3.250949in}}{\pgfqpoint{3.861083in}{3.246559in}}{\pgfqpoint{3.872133in}{3.246559in}}%
\pgfpathclose%
\pgfusepath{stroke,fill}%
\end{pgfscope}%
\begin{pgfscope}%
\pgfpathrectangle{\pgfqpoint{0.481978in}{0.331635in}}{\pgfqpoint{9.300000in}{7.700000in}}%
\pgfusepath{clip}%
\pgfsetbuttcap%
\pgfsetroundjoin%
\definecolor{currentfill}{rgb}{1.000000,0.705882,0.509804}%
\pgfsetfillcolor{currentfill}%
\pgfsetlinewidth{0.481800pt}%
\definecolor{currentstroke}{rgb}{1.000000,1.000000,1.000000}%
\pgfsetstrokecolor{currentstroke}%
\pgfsetdash{}{0pt}%
\pgfpathmoveto{\pgfqpoint{3.604393in}{3.059359in}}%
\pgfpathcurveto{\pgfqpoint{3.615443in}{3.059359in}}{\pgfqpoint{3.626042in}{3.063750in}}{\pgfqpoint{3.633855in}{3.071563in}}%
\pgfpathcurveto{\pgfqpoint{3.641669in}{3.079377in}}{\pgfqpoint{3.646059in}{3.089976in}}{\pgfqpoint{3.646059in}{3.101026in}}%
\pgfpathcurveto{\pgfqpoint{3.646059in}{3.112076in}}{\pgfqpoint{3.641669in}{3.122675in}}{\pgfqpoint{3.633855in}{3.130489in}}%
\pgfpathcurveto{\pgfqpoint{3.626042in}{3.138302in}}{\pgfqpoint{3.615443in}{3.142693in}}{\pgfqpoint{3.604393in}{3.142693in}}%
\pgfpathcurveto{\pgfqpoint{3.593343in}{3.142693in}}{\pgfqpoint{3.582744in}{3.138302in}}{\pgfqpoint{3.574930in}{3.130489in}}%
\pgfpathcurveto{\pgfqpoint{3.567116in}{3.122675in}}{\pgfqpoint{3.562726in}{3.112076in}}{\pgfqpoint{3.562726in}{3.101026in}}%
\pgfpathcurveto{\pgfqpoint{3.562726in}{3.089976in}}{\pgfqpoint{3.567116in}{3.079377in}}{\pgfqpoint{3.574930in}{3.071563in}}%
\pgfpathcurveto{\pgfqpoint{3.582744in}{3.063750in}}{\pgfqpoint{3.593343in}{3.059359in}}{\pgfqpoint{3.604393in}{3.059359in}}%
\pgfpathclose%
\pgfusepath{stroke,fill}%
\end{pgfscope}%
\begin{pgfscope}%
\pgfpathrectangle{\pgfqpoint{0.481978in}{0.331635in}}{\pgfqpoint{9.300000in}{7.700000in}}%
\pgfusepath{clip}%
\pgfsetbuttcap%
\pgfsetroundjoin%
\definecolor{currentfill}{rgb}{1.000000,0.705882,0.509804}%
\pgfsetfillcolor{currentfill}%
\pgfsetlinewidth{0.481800pt}%
\definecolor{currentstroke}{rgb}{1.000000,1.000000,1.000000}%
\pgfsetstrokecolor{currentstroke}%
\pgfsetdash{}{0pt}%
\pgfpathmoveto{\pgfqpoint{3.157682in}{4.400131in}}%
\pgfpathcurveto{\pgfqpoint{3.168733in}{4.400131in}}{\pgfqpoint{3.179332in}{4.404521in}}{\pgfqpoint{3.187145in}{4.412334in}}%
\pgfpathcurveto{\pgfqpoint{3.194959in}{4.420148in}}{\pgfqpoint{3.199349in}{4.430747in}}{\pgfqpoint{3.199349in}{4.441797in}}%
\pgfpathcurveto{\pgfqpoint{3.199349in}{4.452847in}}{\pgfqpoint{3.194959in}{4.463446in}}{\pgfqpoint{3.187145in}{4.471260in}}%
\pgfpathcurveto{\pgfqpoint{3.179332in}{4.479074in}}{\pgfqpoint{3.168733in}{4.483464in}}{\pgfqpoint{3.157682in}{4.483464in}}%
\pgfpathcurveto{\pgfqpoint{3.146632in}{4.483464in}}{\pgfqpoint{3.136033in}{4.479074in}}{\pgfqpoint{3.128220in}{4.471260in}}%
\pgfpathcurveto{\pgfqpoint{3.120406in}{4.463446in}}{\pgfqpoint{3.116016in}{4.452847in}}{\pgfqpoint{3.116016in}{4.441797in}}%
\pgfpathcurveto{\pgfqpoint{3.116016in}{4.430747in}}{\pgfqpoint{3.120406in}{4.420148in}}{\pgfqpoint{3.128220in}{4.412334in}}%
\pgfpathcurveto{\pgfqpoint{3.136033in}{4.404521in}}{\pgfqpoint{3.146632in}{4.400131in}}{\pgfqpoint{3.157682in}{4.400131in}}%
\pgfpathclose%
\pgfusepath{stroke,fill}%
\end{pgfscope}%
\begin{pgfscope}%
\pgfpathrectangle{\pgfqpoint{0.481978in}{0.331635in}}{\pgfqpoint{9.300000in}{7.700000in}}%
\pgfusepath{clip}%
\pgfsetbuttcap%
\pgfsetroundjoin%
\definecolor{currentfill}{rgb}{1.000000,0.705882,0.509804}%
\pgfsetfillcolor{currentfill}%
\pgfsetlinewidth{0.481800pt}%
\definecolor{currentstroke}{rgb}{1.000000,1.000000,1.000000}%
\pgfsetstrokecolor{currentstroke}%
\pgfsetdash{}{0pt}%
\pgfpathmoveto{\pgfqpoint{3.462770in}{5.070375in}}%
\pgfpathcurveto{\pgfqpoint{3.473820in}{5.070375in}}{\pgfqpoint{3.484419in}{5.074765in}}{\pgfqpoint{3.492233in}{5.082579in}}%
\pgfpathcurveto{\pgfqpoint{3.500047in}{5.090392in}}{\pgfqpoint{3.504437in}{5.100991in}}{\pgfqpoint{3.504437in}{5.112041in}}%
\pgfpathcurveto{\pgfqpoint{3.504437in}{5.123092in}}{\pgfqpoint{3.500047in}{5.133691in}}{\pgfqpoint{3.492233in}{5.141504in}}%
\pgfpathcurveto{\pgfqpoint{3.484419in}{5.149318in}}{\pgfqpoint{3.473820in}{5.153708in}}{\pgfqpoint{3.462770in}{5.153708in}}%
\pgfpathcurveto{\pgfqpoint{3.451720in}{5.153708in}}{\pgfqpoint{3.441121in}{5.149318in}}{\pgfqpoint{3.433307in}{5.141504in}}%
\pgfpathcurveto{\pgfqpoint{3.425494in}{5.133691in}}{\pgfqpoint{3.421104in}{5.123092in}}{\pgfqpoint{3.421104in}{5.112041in}}%
\pgfpathcurveto{\pgfqpoint{3.421104in}{5.100991in}}{\pgfqpoint{3.425494in}{5.090392in}}{\pgfqpoint{3.433307in}{5.082579in}}%
\pgfpathcurveto{\pgfqpoint{3.441121in}{5.074765in}}{\pgfqpoint{3.451720in}{5.070375in}}{\pgfqpoint{3.462770in}{5.070375in}}%
\pgfpathclose%
\pgfusepath{stroke,fill}%
\end{pgfscope}%
\begin{pgfscope}%
\pgfpathrectangle{\pgfqpoint{0.481978in}{0.331635in}}{\pgfqpoint{9.300000in}{7.700000in}}%
\pgfusepath{clip}%
\pgfsetbuttcap%
\pgfsetroundjoin%
\definecolor{currentfill}{rgb}{1.000000,0.705882,0.509804}%
\pgfsetfillcolor{currentfill}%
\pgfsetlinewidth{0.481800pt}%
\definecolor{currentstroke}{rgb}{1.000000,1.000000,1.000000}%
\pgfsetstrokecolor{currentstroke}%
\pgfsetdash{}{0pt}%
\pgfpathmoveto{\pgfqpoint{4.591082in}{3.522891in}}%
\pgfpathcurveto{\pgfqpoint{4.602132in}{3.522891in}}{\pgfqpoint{4.612731in}{3.527282in}}{\pgfqpoint{4.620545in}{3.535095in}}%
\pgfpathcurveto{\pgfqpoint{4.628358in}{3.542909in}}{\pgfqpoint{4.632749in}{3.553508in}}{\pgfqpoint{4.632749in}{3.564558in}}%
\pgfpathcurveto{\pgfqpoint{4.632749in}{3.575608in}}{\pgfqpoint{4.628358in}{3.586207in}}{\pgfqpoint{4.620545in}{3.594021in}}%
\pgfpathcurveto{\pgfqpoint{4.612731in}{3.601834in}}{\pgfqpoint{4.602132in}{3.606225in}}{\pgfqpoint{4.591082in}{3.606225in}}%
\pgfpathcurveto{\pgfqpoint{4.580032in}{3.606225in}}{\pgfqpoint{4.569433in}{3.601834in}}{\pgfqpoint{4.561619in}{3.594021in}}%
\pgfpathcurveto{\pgfqpoint{4.553806in}{3.586207in}}{\pgfqpoint{4.549415in}{3.575608in}}{\pgfqpoint{4.549415in}{3.564558in}}%
\pgfpathcurveto{\pgfqpoint{4.549415in}{3.553508in}}{\pgfqpoint{4.553806in}{3.542909in}}{\pgfqpoint{4.561619in}{3.535095in}}%
\pgfpathcurveto{\pgfqpoint{4.569433in}{3.527282in}}{\pgfqpoint{4.580032in}{3.522891in}}{\pgfqpoint{4.591082in}{3.522891in}}%
\pgfpathclose%
\pgfusepath{stroke,fill}%
\end{pgfscope}%
\begin{pgfscope}%
\pgfpathrectangle{\pgfqpoint{0.481978in}{0.331635in}}{\pgfqpoint{9.300000in}{7.700000in}}%
\pgfusepath{clip}%
\pgfsetbuttcap%
\pgfsetroundjoin%
\definecolor{currentfill}{rgb}{1.000000,0.705882,0.509804}%
\pgfsetfillcolor{currentfill}%
\pgfsetlinewidth{0.481800pt}%
\definecolor{currentstroke}{rgb}{1.000000,1.000000,1.000000}%
\pgfsetstrokecolor{currentstroke}%
\pgfsetdash{}{0pt}%
\pgfpathmoveto{\pgfqpoint{5.774624in}{4.557577in}}%
\pgfpathcurveto{\pgfqpoint{5.785674in}{4.557577in}}{\pgfqpoint{5.796274in}{4.561967in}}{\pgfqpoint{5.804087in}{4.569781in}}%
\pgfpathcurveto{\pgfqpoint{5.811901in}{4.577595in}}{\pgfqpoint{5.816291in}{4.588194in}}{\pgfqpoint{5.816291in}{4.599244in}}%
\pgfpathcurveto{\pgfqpoint{5.816291in}{4.610294in}}{\pgfqpoint{5.811901in}{4.620893in}}{\pgfqpoint{5.804087in}{4.628707in}}%
\pgfpathcurveto{\pgfqpoint{5.796274in}{4.636520in}}{\pgfqpoint{5.785674in}{4.640911in}}{\pgfqpoint{5.774624in}{4.640911in}}%
\pgfpathcurveto{\pgfqpoint{5.763574in}{4.640911in}}{\pgfqpoint{5.752975in}{4.636520in}}{\pgfqpoint{5.745162in}{4.628707in}}%
\pgfpathcurveto{\pgfqpoint{5.737348in}{4.620893in}}{\pgfqpoint{5.732958in}{4.610294in}}{\pgfqpoint{5.732958in}{4.599244in}}%
\pgfpathcurveto{\pgfqpoint{5.732958in}{4.588194in}}{\pgfqpoint{5.737348in}{4.577595in}}{\pgfqpoint{5.745162in}{4.569781in}}%
\pgfpathcurveto{\pgfqpoint{5.752975in}{4.561967in}}{\pgfqpoint{5.763574in}{4.557577in}}{\pgfqpoint{5.774624in}{4.557577in}}%
\pgfpathclose%
\pgfusepath{stroke,fill}%
\end{pgfscope}%
\begin{pgfscope}%
\pgfpathrectangle{\pgfqpoint{0.481978in}{0.331635in}}{\pgfqpoint{9.300000in}{7.700000in}}%
\pgfusepath{clip}%
\pgfsetbuttcap%
\pgfsetroundjoin%
\definecolor{currentfill}{rgb}{1.000000,0.705882,0.509804}%
\pgfsetfillcolor{currentfill}%
\pgfsetlinewidth{0.481800pt}%
\definecolor{currentstroke}{rgb}{1.000000,1.000000,1.000000}%
\pgfsetstrokecolor{currentstroke}%
\pgfsetdash{}{0pt}%
\pgfpathmoveto{\pgfqpoint{3.755577in}{2.666678in}}%
\pgfpathcurveto{\pgfqpoint{3.766628in}{2.666678in}}{\pgfqpoint{3.777227in}{2.671068in}}{\pgfqpoint{3.785040in}{2.678882in}}%
\pgfpathcurveto{\pgfqpoint{3.792854in}{2.686696in}}{\pgfqpoint{3.797244in}{2.697295in}}{\pgfqpoint{3.797244in}{2.708345in}}%
\pgfpathcurveto{\pgfqpoint{3.797244in}{2.719395in}}{\pgfqpoint{3.792854in}{2.729994in}}{\pgfqpoint{3.785040in}{2.737808in}}%
\pgfpathcurveto{\pgfqpoint{3.777227in}{2.745621in}}{\pgfqpoint{3.766628in}{2.750012in}}{\pgfqpoint{3.755577in}{2.750012in}}%
\pgfpathcurveto{\pgfqpoint{3.744527in}{2.750012in}}{\pgfqpoint{3.733928in}{2.745621in}}{\pgfqpoint{3.726115in}{2.737808in}}%
\pgfpathcurveto{\pgfqpoint{3.718301in}{2.729994in}}{\pgfqpoint{3.713911in}{2.719395in}}{\pgfqpoint{3.713911in}{2.708345in}}%
\pgfpathcurveto{\pgfqpoint{3.713911in}{2.697295in}}{\pgfqpoint{3.718301in}{2.686696in}}{\pgfqpoint{3.726115in}{2.678882in}}%
\pgfpathcurveto{\pgfqpoint{3.733928in}{2.671068in}}{\pgfqpoint{3.744527in}{2.666678in}}{\pgfqpoint{3.755577in}{2.666678in}}%
\pgfpathclose%
\pgfusepath{stroke,fill}%
\end{pgfscope}%
\begin{pgfscope}%
\pgfpathrectangle{\pgfqpoint{0.481978in}{0.331635in}}{\pgfqpoint{9.300000in}{7.700000in}}%
\pgfusepath{clip}%
\pgfsetbuttcap%
\pgfsetroundjoin%
\definecolor{currentfill}{rgb}{1.000000,0.705882,0.509804}%
\pgfsetfillcolor{currentfill}%
\pgfsetlinewidth{0.481800pt}%
\definecolor{currentstroke}{rgb}{1.000000,1.000000,1.000000}%
\pgfsetstrokecolor{currentstroke}%
\pgfsetdash{}{0pt}%
\pgfpathmoveto{\pgfqpoint{2.355130in}{2.560250in}}%
\pgfpathcurveto{\pgfqpoint{2.366180in}{2.560250in}}{\pgfqpoint{2.376779in}{2.564640in}}{\pgfqpoint{2.384593in}{2.572454in}}%
\pgfpathcurveto{\pgfqpoint{2.392406in}{2.580267in}}{\pgfqpoint{2.396797in}{2.590866in}}{\pgfqpoint{2.396797in}{2.601916in}}%
\pgfpathcurveto{\pgfqpoint{2.396797in}{2.612967in}}{\pgfqpoint{2.392406in}{2.623566in}}{\pgfqpoint{2.384593in}{2.631379in}}%
\pgfpathcurveto{\pgfqpoint{2.376779in}{2.639193in}}{\pgfqpoint{2.366180in}{2.643583in}}{\pgfqpoint{2.355130in}{2.643583in}}%
\pgfpathcurveto{\pgfqpoint{2.344080in}{2.643583in}}{\pgfqpoint{2.333481in}{2.639193in}}{\pgfqpoint{2.325667in}{2.631379in}}%
\pgfpathcurveto{\pgfqpoint{2.317854in}{2.623566in}}{\pgfqpoint{2.313463in}{2.612967in}}{\pgfqpoint{2.313463in}{2.601916in}}%
\pgfpathcurveto{\pgfqpoint{2.313463in}{2.590866in}}{\pgfqpoint{2.317854in}{2.580267in}}{\pgfqpoint{2.325667in}{2.572454in}}%
\pgfpathcurveto{\pgfqpoint{2.333481in}{2.564640in}}{\pgfqpoint{2.344080in}{2.560250in}}{\pgfqpoint{2.355130in}{2.560250in}}%
\pgfpathclose%
\pgfusepath{stroke,fill}%
\end{pgfscope}%
\begin{pgfscope}%
\pgfpathrectangle{\pgfqpoint{0.481978in}{0.331635in}}{\pgfqpoint{9.300000in}{7.700000in}}%
\pgfusepath{clip}%
\pgfsetbuttcap%
\pgfsetroundjoin%
\definecolor{currentfill}{rgb}{1.000000,0.705882,0.509804}%
\pgfsetfillcolor{currentfill}%
\pgfsetlinewidth{0.481800pt}%
\definecolor{currentstroke}{rgb}{1.000000,1.000000,1.000000}%
\pgfsetstrokecolor{currentstroke}%
\pgfsetdash{}{0pt}%
\pgfpathmoveto{\pgfqpoint{3.074624in}{3.245468in}}%
\pgfpathcurveto{\pgfqpoint{3.085674in}{3.245468in}}{\pgfqpoint{3.096273in}{3.249858in}}{\pgfqpoint{3.104087in}{3.257672in}}%
\pgfpathcurveto{\pgfqpoint{3.111901in}{3.265485in}}{\pgfqpoint{3.116291in}{3.276084in}}{\pgfqpoint{3.116291in}{3.287134in}}%
\pgfpathcurveto{\pgfqpoint{3.116291in}{3.298185in}}{\pgfqpoint{3.111901in}{3.308784in}}{\pgfqpoint{3.104087in}{3.316597in}}%
\pgfpathcurveto{\pgfqpoint{3.096273in}{3.324411in}}{\pgfqpoint{3.085674in}{3.328801in}}{\pgfqpoint{3.074624in}{3.328801in}}%
\pgfpathcurveto{\pgfqpoint{3.063574in}{3.328801in}}{\pgfqpoint{3.052975in}{3.324411in}}{\pgfqpoint{3.045161in}{3.316597in}}%
\pgfpathcurveto{\pgfqpoint{3.037348in}{3.308784in}}{\pgfqpoint{3.032957in}{3.298185in}}{\pgfqpoint{3.032957in}{3.287134in}}%
\pgfpathcurveto{\pgfqpoint{3.032957in}{3.276084in}}{\pgfqpoint{3.037348in}{3.265485in}}{\pgfqpoint{3.045161in}{3.257672in}}%
\pgfpathcurveto{\pgfqpoint{3.052975in}{3.249858in}}{\pgfqpoint{3.063574in}{3.245468in}}{\pgfqpoint{3.074624in}{3.245468in}}%
\pgfpathclose%
\pgfusepath{stroke,fill}%
\end{pgfscope}%
\begin{pgfscope}%
\pgfpathrectangle{\pgfqpoint{0.481978in}{0.331635in}}{\pgfqpoint{9.300000in}{7.700000in}}%
\pgfusepath{clip}%
\pgfsetbuttcap%
\pgfsetroundjoin%
\definecolor{currentfill}{rgb}{1.000000,0.705882,0.509804}%
\pgfsetfillcolor{currentfill}%
\pgfsetlinewidth{0.481800pt}%
\definecolor{currentstroke}{rgb}{1.000000,1.000000,1.000000}%
\pgfsetstrokecolor{currentstroke}%
\pgfsetdash{}{0pt}%
\pgfpathmoveto{\pgfqpoint{2.232057in}{4.434934in}}%
\pgfpathcurveto{\pgfqpoint{2.243108in}{4.434934in}}{\pgfqpoint{2.253707in}{4.439325in}}{\pgfqpoint{2.261520in}{4.447138in}}%
\pgfpathcurveto{\pgfqpoint{2.269334in}{4.454952in}}{\pgfqpoint{2.273724in}{4.465551in}}{\pgfqpoint{2.273724in}{4.476601in}}%
\pgfpathcurveto{\pgfqpoint{2.273724in}{4.487651in}}{\pgfqpoint{2.269334in}{4.498250in}}{\pgfqpoint{2.261520in}{4.506064in}}%
\pgfpathcurveto{\pgfqpoint{2.253707in}{4.513877in}}{\pgfqpoint{2.243108in}{4.518268in}}{\pgfqpoint{2.232057in}{4.518268in}}%
\pgfpathcurveto{\pgfqpoint{2.221007in}{4.518268in}}{\pgfqpoint{2.210408in}{4.513877in}}{\pgfqpoint{2.202595in}{4.506064in}}%
\pgfpathcurveto{\pgfqpoint{2.194781in}{4.498250in}}{\pgfqpoint{2.190391in}{4.487651in}}{\pgfqpoint{2.190391in}{4.476601in}}%
\pgfpathcurveto{\pgfqpoint{2.190391in}{4.465551in}}{\pgfqpoint{2.194781in}{4.454952in}}{\pgfqpoint{2.202595in}{4.447138in}}%
\pgfpathcurveto{\pgfqpoint{2.210408in}{4.439325in}}{\pgfqpoint{2.221007in}{4.434934in}}{\pgfqpoint{2.232057in}{4.434934in}}%
\pgfpathclose%
\pgfusepath{stroke,fill}%
\end{pgfscope}%
\begin{pgfscope}%
\pgfpathrectangle{\pgfqpoint{0.481978in}{0.331635in}}{\pgfqpoint{9.300000in}{7.700000in}}%
\pgfusepath{clip}%
\pgfsetbuttcap%
\pgfsetroundjoin%
\definecolor{currentfill}{rgb}{1.000000,0.705882,0.509804}%
\pgfsetfillcolor{currentfill}%
\pgfsetlinewidth{0.481800pt}%
\definecolor{currentstroke}{rgb}{1.000000,1.000000,1.000000}%
\pgfsetstrokecolor{currentstroke}%
\pgfsetdash{}{0pt}%
\pgfpathmoveto{\pgfqpoint{3.954754in}{3.700619in}}%
\pgfpathcurveto{\pgfqpoint{3.965804in}{3.700619in}}{\pgfqpoint{3.976403in}{3.705009in}}{\pgfqpoint{3.984217in}{3.712823in}}%
\pgfpathcurveto{\pgfqpoint{3.992031in}{3.720636in}}{\pgfqpoint{3.996421in}{3.731235in}}{\pgfqpoint{3.996421in}{3.742286in}}%
\pgfpathcurveto{\pgfqpoint{3.996421in}{3.753336in}}{\pgfqpoint{3.992031in}{3.763935in}}{\pgfqpoint{3.984217in}{3.771748in}}%
\pgfpathcurveto{\pgfqpoint{3.976403in}{3.779562in}}{\pgfqpoint{3.965804in}{3.783952in}}{\pgfqpoint{3.954754in}{3.783952in}}%
\pgfpathcurveto{\pgfqpoint{3.943704in}{3.783952in}}{\pgfqpoint{3.933105in}{3.779562in}}{\pgfqpoint{3.925291in}{3.771748in}}%
\pgfpathcurveto{\pgfqpoint{3.917478in}{3.763935in}}{\pgfqpoint{3.913088in}{3.753336in}}{\pgfqpoint{3.913088in}{3.742286in}}%
\pgfpathcurveto{\pgfqpoint{3.913088in}{3.731235in}}{\pgfqpoint{3.917478in}{3.720636in}}{\pgfqpoint{3.925291in}{3.712823in}}%
\pgfpathcurveto{\pgfqpoint{3.933105in}{3.705009in}}{\pgfqpoint{3.943704in}{3.700619in}}{\pgfqpoint{3.954754in}{3.700619in}}%
\pgfpathclose%
\pgfusepath{stroke,fill}%
\end{pgfscope}%
\begin{pgfscope}%
\pgfpathrectangle{\pgfqpoint{0.481978in}{0.331635in}}{\pgfqpoint{9.300000in}{7.700000in}}%
\pgfusepath{clip}%
\pgfsetbuttcap%
\pgfsetroundjoin%
\definecolor{currentfill}{rgb}{1.000000,0.705882,0.509804}%
\pgfsetfillcolor{currentfill}%
\pgfsetlinewidth{0.481800pt}%
\definecolor{currentstroke}{rgb}{1.000000,1.000000,1.000000}%
\pgfsetstrokecolor{currentstroke}%
\pgfsetdash{}{0pt}%
\pgfpathmoveto{\pgfqpoint{1.921313in}{5.061256in}}%
\pgfpathcurveto{\pgfqpoint{1.932363in}{5.061256in}}{\pgfqpoint{1.942962in}{5.065646in}}{\pgfqpoint{1.950776in}{5.073460in}}%
\pgfpathcurveto{\pgfqpoint{1.958590in}{5.081274in}}{\pgfqpoint{1.962980in}{5.091873in}}{\pgfqpoint{1.962980in}{5.102923in}}%
\pgfpathcurveto{\pgfqpoint{1.962980in}{5.113973in}}{\pgfqpoint{1.958590in}{5.124572in}}{\pgfqpoint{1.950776in}{5.132386in}}%
\pgfpathcurveto{\pgfqpoint{1.942962in}{5.140199in}}{\pgfqpoint{1.932363in}{5.144590in}}{\pgfqpoint{1.921313in}{5.144590in}}%
\pgfpathcurveto{\pgfqpoint{1.910263in}{5.144590in}}{\pgfqpoint{1.899664in}{5.140199in}}{\pgfqpoint{1.891850in}{5.132386in}}%
\pgfpathcurveto{\pgfqpoint{1.884037in}{5.124572in}}{\pgfqpoint{1.879647in}{5.113973in}}{\pgfqpoint{1.879647in}{5.102923in}}%
\pgfpathcurveto{\pgfqpoint{1.879647in}{5.091873in}}{\pgfqpoint{1.884037in}{5.081274in}}{\pgfqpoint{1.891850in}{5.073460in}}%
\pgfpathcurveto{\pgfqpoint{1.899664in}{5.065646in}}{\pgfqpoint{1.910263in}{5.061256in}}{\pgfqpoint{1.921313in}{5.061256in}}%
\pgfpathclose%
\pgfusepath{stroke,fill}%
\end{pgfscope}%
\begin{pgfscope}%
\pgfpathrectangle{\pgfqpoint{0.481978in}{0.331635in}}{\pgfqpoint{9.300000in}{7.700000in}}%
\pgfusepath{clip}%
\pgfsetbuttcap%
\pgfsetroundjoin%
\definecolor{currentfill}{rgb}{1.000000,0.705882,0.509804}%
\pgfsetfillcolor{currentfill}%
\pgfsetlinewidth{0.481800pt}%
\definecolor{currentstroke}{rgb}{1.000000,1.000000,1.000000}%
\pgfsetstrokecolor{currentstroke}%
\pgfsetdash{}{0pt}%
\pgfpathmoveto{\pgfqpoint{3.918352in}{4.402192in}}%
\pgfpathcurveto{\pgfqpoint{3.929402in}{4.402192in}}{\pgfqpoint{3.940001in}{4.406582in}}{\pgfqpoint{3.947815in}{4.414396in}}%
\pgfpathcurveto{\pgfqpoint{3.955628in}{4.422209in}}{\pgfqpoint{3.960018in}{4.432808in}}{\pgfqpoint{3.960018in}{4.443858in}}%
\pgfpathcurveto{\pgfqpoint{3.960018in}{4.454909in}}{\pgfqpoint{3.955628in}{4.465508in}}{\pgfqpoint{3.947815in}{4.473321in}}%
\pgfpathcurveto{\pgfqpoint{3.940001in}{4.481135in}}{\pgfqpoint{3.929402in}{4.485525in}}{\pgfqpoint{3.918352in}{4.485525in}}%
\pgfpathcurveto{\pgfqpoint{3.907302in}{4.485525in}}{\pgfqpoint{3.896703in}{4.481135in}}{\pgfqpoint{3.888889in}{4.473321in}}%
\pgfpathcurveto{\pgfqpoint{3.881075in}{4.465508in}}{\pgfqpoint{3.876685in}{4.454909in}}{\pgfqpoint{3.876685in}{4.443858in}}%
\pgfpathcurveto{\pgfqpoint{3.876685in}{4.432808in}}{\pgfqpoint{3.881075in}{4.422209in}}{\pgfqpoint{3.888889in}{4.414396in}}%
\pgfpathcurveto{\pgfqpoint{3.896703in}{4.406582in}}{\pgfqpoint{3.907302in}{4.402192in}}{\pgfqpoint{3.918352in}{4.402192in}}%
\pgfpathclose%
\pgfusepath{stroke,fill}%
\end{pgfscope}%
\begin{pgfscope}%
\pgfpathrectangle{\pgfqpoint{0.481978in}{0.331635in}}{\pgfqpoint{9.300000in}{7.700000in}}%
\pgfusepath{clip}%
\pgfsetbuttcap%
\pgfsetroundjoin%
\definecolor{currentfill}{rgb}{1.000000,0.705882,0.509804}%
\pgfsetfillcolor{currentfill}%
\pgfsetlinewidth{0.481800pt}%
\definecolor{currentstroke}{rgb}{1.000000,1.000000,1.000000}%
\pgfsetstrokecolor{currentstroke}%
\pgfsetdash{}{0pt}%
\pgfpathmoveto{\pgfqpoint{1.874082in}{4.402216in}}%
\pgfpathcurveto{\pgfqpoint{1.885132in}{4.402216in}}{\pgfqpoint{1.895731in}{4.406606in}}{\pgfqpoint{1.903545in}{4.414420in}}%
\pgfpathcurveto{\pgfqpoint{1.911358in}{4.422234in}}{\pgfqpoint{1.915749in}{4.432833in}}{\pgfqpoint{1.915749in}{4.443883in}}%
\pgfpathcurveto{\pgfqpoint{1.915749in}{4.454933in}}{\pgfqpoint{1.911358in}{4.465532in}}{\pgfqpoint{1.903545in}{4.473346in}}%
\pgfpathcurveto{\pgfqpoint{1.895731in}{4.481159in}}{\pgfqpoint{1.885132in}{4.485549in}}{\pgfqpoint{1.874082in}{4.485549in}}%
\pgfpathcurveto{\pgfqpoint{1.863032in}{4.485549in}}{\pgfqpoint{1.852433in}{4.481159in}}{\pgfqpoint{1.844619in}{4.473346in}}%
\pgfpathcurveto{\pgfqpoint{1.836806in}{4.465532in}}{\pgfqpoint{1.832415in}{4.454933in}}{\pgfqpoint{1.832415in}{4.443883in}}%
\pgfpathcurveto{\pgfqpoint{1.832415in}{4.432833in}}{\pgfqpoint{1.836806in}{4.422234in}}{\pgfqpoint{1.844619in}{4.414420in}}%
\pgfpathcurveto{\pgfqpoint{1.852433in}{4.406606in}}{\pgfqpoint{1.863032in}{4.402216in}}{\pgfqpoint{1.874082in}{4.402216in}}%
\pgfpathclose%
\pgfusepath{stroke,fill}%
\end{pgfscope}%
\begin{pgfscope}%
\pgfpathrectangle{\pgfqpoint{0.481978in}{0.331635in}}{\pgfqpoint{9.300000in}{7.700000in}}%
\pgfusepath{clip}%
\pgfsetbuttcap%
\pgfsetroundjoin%
\definecolor{currentfill}{rgb}{1.000000,0.705882,0.509804}%
\pgfsetfillcolor{currentfill}%
\pgfsetlinewidth{0.481800pt}%
\definecolor{currentstroke}{rgb}{1.000000,1.000000,1.000000}%
\pgfsetstrokecolor{currentstroke}%
\pgfsetdash{}{0pt}%
\pgfpathmoveto{\pgfqpoint{4.966201in}{4.808664in}}%
\pgfpathcurveto{\pgfqpoint{4.977251in}{4.808664in}}{\pgfqpoint{4.987850in}{4.813054in}}{\pgfqpoint{4.995663in}{4.820868in}}%
\pgfpathcurveto{\pgfqpoint{5.003477in}{4.828681in}}{\pgfqpoint{5.007867in}{4.839280in}}{\pgfqpoint{5.007867in}{4.850330in}}%
\pgfpathcurveto{\pgfqpoint{5.007867in}{4.861380in}}{\pgfqpoint{5.003477in}{4.871979in}}{\pgfqpoint{4.995663in}{4.879793in}}%
\pgfpathcurveto{\pgfqpoint{4.987850in}{4.887607in}}{\pgfqpoint{4.977251in}{4.891997in}}{\pgfqpoint{4.966201in}{4.891997in}}%
\pgfpathcurveto{\pgfqpoint{4.955151in}{4.891997in}}{\pgfqpoint{4.944552in}{4.887607in}}{\pgfqpoint{4.936738in}{4.879793in}}%
\pgfpathcurveto{\pgfqpoint{4.928924in}{4.871979in}}{\pgfqpoint{4.924534in}{4.861380in}}{\pgfqpoint{4.924534in}{4.850330in}}%
\pgfpathcurveto{\pgfqpoint{4.924534in}{4.839280in}}{\pgfqpoint{4.928924in}{4.828681in}}{\pgfqpoint{4.936738in}{4.820868in}}%
\pgfpathcurveto{\pgfqpoint{4.944552in}{4.813054in}}{\pgfqpoint{4.955151in}{4.808664in}}{\pgfqpoint{4.966201in}{4.808664in}}%
\pgfpathclose%
\pgfusepath{stroke,fill}%
\end{pgfscope}%
\begin{pgfscope}%
\pgfpathrectangle{\pgfqpoint{0.481978in}{0.331635in}}{\pgfqpoint{9.300000in}{7.700000in}}%
\pgfusepath{clip}%
\pgfsetbuttcap%
\pgfsetroundjoin%
\definecolor{currentfill}{rgb}{0.631373,0.788235,0.956863}%
\pgfsetfillcolor{currentfill}%
\pgfsetlinewidth{1.003750pt}%
\definecolor{currentstroke}{rgb}{0.631373,0.788235,0.956863}%
\pgfsetstrokecolor{currentstroke}%
\pgfsetdash{}{0pt}%
\pgfsys@defobject{currentmarker}{\pgfqpoint{-0.041667in}{-0.041667in}}{\pgfqpoint{0.041667in}{0.041667in}}{%
\pgfpathmoveto{\pgfqpoint{0.000000in}{-0.041667in}}%
\pgfpathcurveto{\pgfqpoint{0.011050in}{-0.041667in}}{\pgfqpoint{0.021649in}{-0.037276in}}{\pgfqpoint{0.029463in}{-0.029463in}}%
\pgfpathcurveto{\pgfqpoint{0.037276in}{-0.021649in}}{\pgfqpoint{0.041667in}{-0.011050in}}{\pgfqpoint{0.041667in}{0.000000in}}%
\pgfpathcurveto{\pgfqpoint{0.041667in}{0.011050in}}{\pgfqpoint{0.037276in}{0.021649in}}{\pgfqpoint{0.029463in}{0.029463in}}%
\pgfpathcurveto{\pgfqpoint{0.021649in}{0.037276in}}{\pgfqpoint{0.011050in}{0.041667in}}{\pgfqpoint{0.000000in}{0.041667in}}%
\pgfpathcurveto{\pgfqpoint{-0.011050in}{0.041667in}}{\pgfqpoint{-0.021649in}{0.037276in}}{\pgfqpoint{-0.029463in}{0.029463in}}%
\pgfpathcurveto{\pgfqpoint{-0.037276in}{0.021649in}}{\pgfqpoint{-0.041667in}{0.011050in}}{\pgfqpoint{-0.041667in}{0.000000in}}%
\pgfpathcurveto{\pgfqpoint{-0.041667in}{-0.011050in}}{\pgfqpoint{-0.037276in}{-0.021649in}}{\pgfqpoint{-0.029463in}{-0.029463in}}%
\pgfpathcurveto{\pgfqpoint{-0.021649in}{-0.037276in}}{\pgfqpoint{-0.011050in}{-0.041667in}}{\pgfqpoint{0.000000in}{-0.041667in}}%
\pgfpathclose%
\pgfusepath{stroke,fill}%
}%
\end{pgfscope}%
\begin{pgfscope}%
\pgfpathrectangle{\pgfqpoint{0.481978in}{0.331635in}}{\pgfqpoint{9.300000in}{7.700000in}}%
\pgfusepath{clip}%
\pgfsetbuttcap%
\pgfsetroundjoin%
\definecolor{currentfill}{rgb}{1.000000,0.705882,0.509804}%
\pgfsetfillcolor{currentfill}%
\pgfsetlinewidth{1.003750pt}%
\definecolor{currentstroke}{rgb}{1.000000,0.705882,0.509804}%
\pgfsetstrokecolor{currentstroke}%
\pgfsetdash{}{0pt}%
\pgfsys@defobject{currentmarker}{\pgfqpoint{-0.041667in}{-0.041667in}}{\pgfqpoint{0.041667in}{0.041667in}}{%
\pgfpathmoveto{\pgfqpoint{0.000000in}{-0.041667in}}%
\pgfpathcurveto{\pgfqpoint{0.011050in}{-0.041667in}}{\pgfqpoint{0.021649in}{-0.037276in}}{\pgfqpoint{0.029463in}{-0.029463in}}%
\pgfpathcurveto{\pgfqpoint{0.037276in}{-0.021649in}}{\pgfqpoint{0.041667in}{-0.011050in}}{\pgfqpoint{0.041667in}{0.000000in}}%
\pgfpathcurveto{\pgfqpoint{0.041667in}{0.011050in}}{\pgfqpoint{0.037276in}{0.021649in}}{\pgfqpoint{0.029463in}{0.029463in}}%
\pgfpathcurveto{\pgfqpoint{0.021649in}{0.037276in}}{\pgfqpoint{0.011050in}{0.041667in}}{\pgfqpoint{0.000000in}{0.041667in}}%
\pgfpathcurveto{\pgfqpoint{-0.011050in}{0.041667in}}{\pgfqpoint{-0.021649in}{0.037276in}}{\pgfqpoint{-0.029463in}{0.029463in}}%
\pgfpathcurveto{\pgfqpoint{-0.037276in}{0.021649in}}{\pgfqpoint{-0.041667in}{0.011050in}}{\pgfqpoint{-0.041667in}{0.000000in}}%
\pgfpathcurveto{\pgfqpoint{-0.041667in}{-0.011050in}}{\pgfqpoint{-0.037276in}{-0.021649in}}{\pgfqpoint{-0.029463in}{-0.029463in}}%
\pgfpathcurveto{\pgfqpoint{-0.021649in}{-0.037276in}}{\pgfqpoint{-0.011050in}{-0.041667in}}{\pgfqpoint{0.000000in}{-0.041667in}}%
\pgfpathclose%
\pgfusepath{stroke,fill}%
}%
\end{pgfscope}%
\begin{pgfscope}%
\pgfsetbuttcap%
\pgfsetroundjoin%
\definecolor{currentfill}{rgb}{0.000000,0.000000,0.000000}%
\pgfsetfillcolor{currentfill}%
\pgfsetlinewidth{0.803000pt}%
\definecolor{currentstroke}{rgb}{0.000000,0.000000,0.000000}%
\pgfsetstrokecolor{currentstroke}%
\pgfsetdash{}{0pt}%
\pgfsys@defobject{currentmarker}{\pgfqpoint{0.000000in}{-0.048611in}}{\pgfqpoint{0.000000in}{0.000000in}}{%
\pgfpathmoveto{\pgfqpoint{0.000000in}{0.000000in}}%
\pgfpathlineto{\pgfqpoint{0.000000in}{-0.048611in}}%
\pgfusepath{stroke,fill}%
}%
\begin{pgfscope}%
\pgfsys@transformshift{0.539104in}{0.331635in}%
\pgfsys@useobject{currentmarker}{}%
\end{pgfscope}%
\end{pgfscope}%
\begin{pgfscope}%
\definecolor{textcolor}{rgb}{0.000000,0.000000,0.000000}%
\pgfsetstrokecolor{textcolor}%
\pgfsetfillcolor{textcolor}%
\pgftext[x=0.539104in,y=0.234413in,,top]{\color{textcolor}\sffamily\fontsize{10.000000}{12.000000}\selectfont \ensuremath{-}30}%
\end{pgfscope}%
\begin{pgfscope}%
\pgfsetbuttcap%
\pgfsetroundjoin%
\definecolor{currentfill}{rgb}{0.000000,0.000000,0.000000}%
\pgfsetfillcolor{currentfill}%
\pgfsetlinewidth{0.803000pt}%
\definecolor{currentstroke}{rgb}{0.000000,0.000000,0.000000}%
\pgfsetstrokecolor{currentstroke}%
\pgfsetdash{}{0pt}%
\pgfsys@defobject{currentmarker}{\pgfqpoint{0.000000in}{-0.048611in}}{\pgfqpoint{0.000000in}{0.000000in}}{%
\pgfpathmoveto{\pgfqpoint{0.000000in}{0.000000in}}%
\pgfpathlineto{\pgfqpoint{0.000000in}{-0.048611in}}%
\pgfusepath{stroke,fill}%
}%
\begin{pgfscope}%
\pgfsys@transformshift{2.011781in}{0.331635in}%
\pgfsys@useobject{currentmarker}{}%
\end{pgfscope}%
\end{pgfscope}%
\begin{pgfscope}%
\definecolor{textcolor}{rgb}{0.000000,0.000000,0.000000}%
\pgfsetstrokecolor{textcolor}%
\pgfsetfillcolor{textcolor}%
\pgftext[x=2.011781in,y=0.234413in,,top]{\color{textcolor}\sffamily\fontsize{10.000000}{12.000000}\selectfont \ensuremath{-}20}%
\end{pgfscope}%
\begin{pgfscope}%
\pgfsetbuttcap%
\pgfsetroundjoin%
\definecolor{currentfill}{rgb}{0.000000,0.000000,0.000000}%
\pgfsetfillcolor{currentfill}%
\pgfsetlinewidth{0.803000pt}%
\definecolor{currentstroke}{rgb}{0.000000,0.000000,0.000000}%
\pgfsetstrokecolor{currentstroke}%
\pgfsetdash{}{0pt}%
\pgfsys@defobject{currentmarker}{\pgfqpoint{0.000000in}{-0.048611in}}{\pgfqpoint{0.000000in}{0.000000in}}{%
\pgfpathmoveto{\pgfqpoint{0.000000in}{0.000000in}}%
\pgfpathlineto{\pgfqpoint{0.000000in}{-0.048611in}}%
\pgfusepath{stroke,fill}%
}%
\begin{pgfscope}%
\pgfsys@transformshift{3.484457in}{0.331635in}%
\pgfsys@useobject{currentmarker}{}%
\end{pgfscope}%
\end{pgfscope}%
\begin{pgfscope}%
\definecolor{textcolor}{rgb}{0.000000,0.000000,0.000000}%
\pgfsetstrokecolor{textcolor}%
\pgfsetfillcolor{textcolor}%
\pgftext[x=3.484457in,y=0.234413in,,top]{\color{textcolor}\sffamily\fontsize{10.000000}{12.000000}\selectfont \ensuremath{-}10}%
\end{pgfscope}%
\begin{pgfscope}%
\pgfsetbuttcap%
\pgfsetroundjoin%
\definecolor{currentfill}{rgb}{0.000000,0.000000,0.000000}%
\pgfsetfillcolor{currentfill}%
\pgfsetlinewidth{0.803000pt}%
\definecolor{currentstroke}{rgb}{0.000000,0.000000,0.000000}%
\pgfsetstrokecolor{currentstroke}%
\pgfsetdash{}{0pt}%
\pgfsys@defobject{currentmarker}{\pgfqpoint{0.000000in}{-0.048611in}}{\pgfqpoint{0.000000in}{0.000000in}}{%
\pgfpathmoveto{\pgfqpoint{0.000000in}{0.000000in}}%
\pgfpathlineto{\pgfqpoint{0.000000in}{-0.048611in}}%
\pgfusepath{stroke,fill}%
}%
\begin{pgfscope}%
\pgfsys@transformshift{4.957134in}{0.331635in}%
\pgfsys@useobject{currentmarker}{}%
\end{pgfscope}%
\end{pgfscope}%
\begin{pgfscope}%
\definecolor{textcolor}{rgb}{0.000000,0.000000,0.000000}%
\pgfsetstrokecolor{textcolor}%
\pgfsetfillcolor{textcolor}%
\pgftext[x=4.957134in,y=0.234413in,,top]{\color{textcolor}\sffamily\fontsize{10.000000}{12.000000}\selectfont 0}%
\end{pgfscope}%
\begin{pgfscope}%
\pgfsetbuttcap%
\pgfsetroundjoin%
\definecolor{currentfill}{rgb}{0.000000,0.000000,0.000000}%
\pgfsetfillcolor{currentfill}%
\pgfsetlinewidth{0.803000pt}%
\definecolor{currentstroke}{rgb}{0.000000,0.000000,0.000000}%
\pgfsetstrokecolor{currentstroke}%
\pgfsetdash{}{0pt}%
\pgfsys@defobject{currentmarker}{\pgfqpoint{0.000000in}{-0.048611in}}{\pgfqpoint{0.000000in}{0.000000in}}{%
\pgfpathmoveto{\pgfqpoint{0.000000in}{0.000000in}}%
\pgfpathlineto{\pgfqpoint{0.000000in}{-0.048611in}}%
\pgfusepath{stroke,fill}%
}%
\begin{pgfscope}%
\pgfsys@transformshift{6.429811in}{0.331635in}%
\pgfsys@useobject{currentmarker}{}%
\end{pgfscope}%
\end{pgfscope}%
\begin{pgfscope}%
\definecolor{textcolor}{rgb}{0.000000,0.000000,0.000000}%
\pgfsetstrokecolor{textcolor}%
\pgfsetfillcolor{textcolor}%
\pgftext[x=6.429811in,y=0.234413in,,top]{\color{textcolor}\sffamily\fontsize{10.000000}{12.000000}\selectfont 10}%
\end{pgfscope}%
\begin{pgfscope}%
\pgfsetbuttcap%
\pgfsetroundjoin%
\definecolor{currentfill}{rgb}{0.000000,0.000000,0.000000}%
\pgfsetfillcolor{currentfill}%
\pgfsetlinewidth{0.803000pt}%
\definecolor{currentstroke}{rgb}{0.000000,0.000000,0.000000}%
\pgfsetstrokecolor{currentstroke}%
\pgfsetdash{}{0pt}%
\pgfsys@defobject{currentmarker}{\pgfqpoint{0.000000in}{-0.048611in}}{\pgfqpoint{0.000000in}{0.000000in}}{%
\pgfpathmoveto{\pgfqpoint{0.000000in}{0.000000in}}%
\pgfpathlineto{\pgfqpoint{0.000000in}{-0.048611in}}%
\pgfusepath{stroke,fill}%
}%
\begin{pgfscope}%
\pgfsys@transformshift{7.902487in}{0.331635in}%
\pgfsys@useobject{currentmarker}{}%
\end{pgfscope}%
\end{pgfscope}%
\begin{pgfscope}%
\definecolor{textcolor}{rgb}{0.000000,0.000000,0.000000}%
\pgfsetstrokecolor{textcolor}%
\pgfsetfillcolor{textcolor}%
\pgftext[x=7.902487in,y=0.234413in,,top]{\color{textcolor}\sffamily\fontsize{10.000000}{12.000000}\selectfont 20}%
\end{pgfscope}%
\begin{pgfscope}%
\pgfsetbuttcap%
\pgfsetroundjoin%
\definecolor{currentfill}{rgb}{0.000000,0.000000,0.000000}%
\pgfsetfillcolor{currentfill}%
\pgfsetlinewidth{0.803000pt}%
\definecolor{currentstroke}{rgb}{0.000000,0.000000,0.000000}%
\pgfsetstrokecolor{currentstroke}%
\pgfsetdash{}{0pt}%
\pgfsys@defobject{currentmarker}{\pgfqpoint{0.000000in}{-0.048611in}}{\pgfqpoint{0.000000in}{0.000000in}}{%
\pgfpathmoveto{\pgfqpoint{0.000000in}{0.000000in}}%
\pgfpathlineto{\pgfqpoint{0.000000in}{-0.048611in}}%
\pgfusepath{stroke,fill}%
}%
\begin{pgfscope}%
\pgfsys@transformshift{9.375164in}{0.331635in}%
\pgfsys@useobject{currentmarker}{}%
\end{pgfscope}%
\end{pgfscope}%
\begin{pgfscope}%
\definecolor{textcolor}{rgb}{0.000000,0.000000,0.000000}%
\pgfsetstrokecolor{textcolor}%
\pgfsetfillcolor{textcolor}%
\pgftext[x=9.375164in,y=0.234413in,,top]{\color{textcolor}\sffamily\fontsize{10.000000}{12.000000}\selectfont 30}%
\end{pgfscope}%
\begin{pgfscope}%
\pgfsetbuttcap%
\pgfsetroundjoin%
\definecolor{currentfill}{rgb}{0.000000,0.000000,0.000000}%
\pgfsetfillcolor{currentfill}%
\pgfsetlinewidth{0.803000pt}%
\definecolor{currentstroke}{rgb}{0.000000,0.000000,0.000000}%
\pgfsetstrokecolor{currentstroke}%
\pgfsetdash{}{0pt}%
\pgfsys@defobject{currentmarker}{\pgfqpoint{-0.048611in}{0.000000in}}{\pgfqpoint{-0.000000in}{0.000000in}}{%
\pgfpathmoveto{\pgfqpoint{-0.000000in}{0.000000in}}%
\pgfpathlineto{\pgfqpoint{-0.048611in}{0.000000in}}%
\pgfusepath{stroke,fill}%
}%
\begin{pgfscope}%
\pgfsys@transformshift{0.481978in}{1.322252in}%
\pgfsys@useobject{currentmarker}{}%
\end{pgfscope}%
\end{pgfscope}%
\begin{pgfscope}%
\definecolor{textcolor}{rgb}{0.000000,0.000000,0.000000}%
\pgfsetstrokecolor{textcolor}%
\pgfsetfillcolor{textcolor}%
\pgftext[x=0.100000in, y=1.269491in, left, base]{\color{textcolor}\sffamily\fontsize{10.000000}{12.000000}\selectfont \ensuremath{-}20}%
\end{pgfscope}%
\begin{pgfscope}%
\pgfsetbuttcap%
\pgfsetroundjoin%
\definecolor{currentfill}{rgb}{0.000000,0.000000,0.000000}%
\pgfsetfillcolor{currentfill}%
\pgfsetlinewidth{0.803000pt}%
\definecolor{currentstroke}{rgb}{0.000000,0.000000,0.000000}%
\pgfsetstrokecolor{currentstroke}%
\pgfsetdash{}{0pt}%
\pgfsys@defobject{currentmarker}{\pgfqpoint{-0.048611in}{0.000000in}}{\pgfqpoint{-0.000000in}{0.000000in}}{%
\pgfpathmoveto{\pgfqpoint{-0.000000in}{0.000000in}}%
\pgfpathlineto{\pgfqpoint{-0.048611in}{0.000000in}}%
\pgfusepath{stroke,fill}%
}%
\begin{pgfscope}%
\pgfsys@transformshift{0.481978in}{2.613640in}%
\pgfsys@useobject{currentmarker}{}%
\end{pgfscope}%
\end{pgfscope}%
\begin{pgfscope}%
\definecolor{textcolor}{rgb}{0.000000,0.000000,0.000000}%
\pgfsetstrokecolor{textcolor}%
\pgfsetfillcolor{textcolor}%
\pgftext[x=0.100000in, y=2.560879in, left, base]{\color{textcolor}\sffamily\fontsize{10.000000}{12.000000}\selectfont \ensuremath{-}10}%
\end{pgfscope}%
\begin{pgfscope}%
\pgfsetbuttcap%
\pgfsetroundjoin%
\definecolor{currentfill}{rgb}{0.000000,0.000000,0.000000}%
\pgfsetfillcolor{currentfill}%
\pgfsetlinewidth{0.803000pt}%
\definecolor{currentstroke}{rgb}{0.000000,0.000000,0.000000}%
\pgfsetstrokecolor{currentstroke}%
\pgfsetdash{}{0pt}%
\pgfsys@defobject{currentmarker}{\pgfqpoint{-0.048611in}{0.000000in}}{\pgfqpoint{-0.000000in}{0.000000in}}{%
\pgfpathmoveto{\pgfqpoint{-0.000000in}{0.000000in}}%
\pgfpathlineto{\pgfqpoint{-0.048611in}{0.000000in}}%
\pgfusepath{stroke,fill}%
}%
\begin{pgfscope}%
\pgfsys@transformshift{0.481978in}{3.905028in}%
\pgfsys@useobject{currentmarker}{}%
\end{pgfscope}%
\end{pgfscope}%
\begin{pgfscope}%
\definecolor{textcolor}{rgb}{0.000000,0.000000,0.000000}%
\pgfsetstrokecolor{textcolor}%
\pgfsetfillcolor{textcolor}%
\pgftext[x=0.296390in, y=3.852267in, left, base]{\color{textcolor}\sffamily\fontsize{10.000000}{12.000000}\selectfont 0}%
\end{pgfscope}%
\begin{pgfscope}%
\pgfsetbuttcap%
\pgfsetroundjoin%
\definecolor{currentfill}{rgb}{0.000000,0.000000,0.000000}%
\pgfsetfillcolor{currentfill}%
\pgfsetlinewidth{0.803000pt}%
\definecolor{currentstroke}{rgb}{0.000000,0.000000,0.000000}%
\pgfsetstrokecolor{currentstroke}%
\pgfsetdash{}{0pt}%
\pgfsys@defobject{currentmarker}{\pgfqpoint{-0.048611in}{0.000000in}}{\pgfqpoint{-0.000000in}{0.000000in}}{%
\pgfpathmoveto{\pgfqpoint{-0.000000in}{0.000000in}}%
\pgfpathlineto{\pgfqpoint{-0.048611in}{0.000000in}}%
\pgfusepath{stroke,fill}%
}%
\begin{pgfscope}%
\pgfsys@transformshift{0.481978in}{5.196416in}%
\pgfsys@useobject{currentmarker}{}%
\end{pgfscope}%
\end{pgfscope}%
\begin{pgfscope}%
\definecolor{textcolor}{rgb}{0.000000,0.000000,0.000000}%
\pgfsetstrokecolor{textcolor}%
\pgfsetfillcolor{textcolor}%
\pgftext[x=0.208025in, y=5.143654in, left, base]{\color{textcolor}\sffamily\fontsize{10.000000}{12.000000}\selectfont 10}%
\end{pgfscope}%
\begin{pgfscope}%
\pgfsetbuttcap%
\pgfsetroundjoin%
\definecolor{currentfill}{rgb}{0.000000,0.000000,0.000000}%
\pgfsetfillcolor{currentfill}%
\pgfsetlinewidth{0.803000pt}%
\definecolor{currentstroke}{rgb}{0.000000,0.000000,0.000000}%
\pgfsetstrokecolor{currentstroke}%
\pgfsetdash{}{0pt}%
\pgfsys@defobject{currentmarker}{\pgfqpoint{-0.048611in}{0.000000in}}{\pgfqpoint{-0.000000in}{0.000000in}}{%
\pgfpathmoveto{\pgfqpoint{-0.000000in}{0.000000in}}%
\pgfpathlineto{\pgfqpoint{-0.048611in}{0.000000in}}%
\pgfusepath{stroke,fill}%
}%
\begin{pgfscope}%
\pgfsys@transformshift{0.481978in}{6.487804in}%
\pgfsys@useobject{currentmarker}{}%
\end{pgfscope}%
\end{pgfscope}%
\begin{pgfscope}%
\definecolor{textcolor}{rgb}{0.000000,0.000000,0.000000}%
\pgfsetstrokecolor{textcolor}%
\pgfsetfillcolor{textcolor}%
\pgftext[x=0.208025in, y=6.435042in, left, base]{\color{textcolor}\sffamily\fontsize{10.000000}{12.000000}\selectfont 20}%
\end{pgfscope}%
\begin{pgfscope}%
\pgfsetbuttcap%
\pgfsetroundjoin%
\definecolor{currentfill}{rgb}{0.000000,0.000000,0.000000}%
\pgfsetfillcolor{currentfill}%
\pgfsetlinewidth{0.803000pt}%
\definecolor{currentstroke}{rgb}{0.000000,0.000000,0.000000}%
\pgfsetstrokecolor{currentstroke}%
\pgfsetdash{}{0pt}%
\pgfsys@defobject{currentmarker}{\pgfqpoint{-0.048611in}{0.000000in}}{\pgfqpoint{-0.000000in}{0.000000in}}{%
\pgfpathmoveto{\pgfqpoint{-0.000000in}{0.000000in}}%
\pgfpathlineto{\pgfqpoint{-0.048611in}{0.000000in}}%
\pgfusepath{stroke,fill}%
}%
\begin{pgfscope}%
\pgfsys@transformshift{0.481978in}{7.779192in}%
\pgfsys@useobject{currentmarker}{}%
\end{pgfscope}%
\end{pgfscope}%
\begin{pgfscope}%
\definecolor{textcolor}{rgb}{0.000000,0.000000,0.000000}%
\pgfsetstrokecolor{textcolor}%
\pgfsetfillcolor{textcolor}%
\pgftext[x=0.208025in, y=7.726430in, left, base]{\color{textcolor}\sffamily\fontsize{10.000000}{12.000000}\selectfont 30}%
\end{pgfscope}%
\begin{pgfscope}%
\pgfpathrectangle{\pgfqpoint{0.481978in}{0.331635in}}{\pgfqpoint{9.300000in}{7.700000in}}%
\pgfusepath{clip}%
\pgfsetrectcap%
\pgfsetroundjoin%
\pgfsetlinewidth{1.505625pt}%
\definecolor{currentstroke}{rgb}{0.631373,0.788235,0.956863}%
\pgfsetstrokecolor{currentstroke}%
\pgfsetstrokeopacity{0.800000}%
\pgfsetdash{}{0pt}%
\pgfpathmoveto{\pgfqpoint{5.562292in}{2.046179in}}%
\pgfpathlineto{\pgfqpoint{5.711094in}{3.773662in}}%
\pgfusepath{stroke}%
\end{pgfscope}%
\begin{pgfscope}%
\pgfpathrectangle{\pgfqpoint{0.481978in}{0.331635in}}{\pgfqpoint{9.300000in}{7.700000in}}%
\pgfusepath{clip}%
\pgfsetrectcap%
\pgfsetroundjoin%
\pgfsetlinewidth{1.505625pt}%
\definecolor{currentstroke}{rgb}{0.631373,0.788235,0.956863}%
\pgfsetstrokecolor{currentstroke}%
\pgfsetstrokeopacity{0.800000}%
\pgfsetdash{}{0pt}%
\pgfpathmoveto{\pgfqpoint{6.909912in}{4.460102in}}%
\pgfpathlineto{\pgfqpoint{5.711094in}{3.773662in}}%
\pgfusepath{stroke}%
\end{pgfscope}%
\begin{pgfscope}%
\pgfpathrectangle{\pgfqpoint{0.481978in}{0.331635in}}{\pgfqpoint{9.300000in}{7.700000in}}%
\pgfusepath{clip}%
\pgfsetrectcap%
\pgfsetroundjoin%
\pgfsetlinewidth{1.505625pt}%
\definecolor{currentstroke}{rgb}{0.631373,0.788235,0.956863}%
\pgfsetstrokecolor{currentstroke}%
\pgfsetstrokeopacity{0.800000}%
\pgfsetdash{}{0pt}%
\pgfpathmoveto{\pgfqpoint{7.565874in}{5.883101in}}%
\pgfpathlineto{\pgfqpoint{5.711094in}{3.773662in}}%
\pgfusepath{stroke}%
\end{pgfscope}%
\begin{pgfscope}%
\pgfpathrectangle{\pgfqpoint{0.481978in}{0.331635in}}{\pgfqpoint{9.300000in}{7.700000in}}%
\pgfusepath{clip}%
\pgfsetrectcap%
\pgfsetroundjoin%
\pgfsetlinewidth{1.505625pt}%
\definecolor{currentstroke}{rgb}{0.631373,0.788235,0.956863}%
\pgfsetstrokecolor{currentstroke}%
\pgfsetstrokeopacity{0.800000}%
\pgfsetdash{}{0pt}%
\pgfpathmoveto{\pgfqpoint{8.104384in}{5.189132in}}%
\pgfpathlineto{\pgfqpoint{5.711094in}{3.773662in}}%
\pgfusepath{stroke}%
\end{pgfscope}%
\begin{pgfscope}%
\pgfpathrectangle{\pgfqpoint{0.481978in}{0.331635in}}{\pgfqpoint{9.300000in}{7.700000in}}%
\pgfusepath{clip}%
\pgfsetrectcap%
\pgfsetroundjoin%
\pgfsetlinewidth{1.505625pt}%
\definecolor{currentstroke}{rgb}{0.631373,0.788235,0.956863}%
\pgfsetstrokecolor{currentstroke}%
\pgfsetstrokeopacity{0.800000}%
\pgfsetdash{}{0pt}%
\pgfpathmoveto{\pgfqpoint{7.967295in}{4.825787in}}%
\pgfpathlineto{\pgfqpoint{5.711094in}{3.773662in}}%
\pgfusepath{stroke}%
\end{pgfscope}%
\begin{pgfscope}%
\pgfpathrectangle{\pgfqpoint{0.481978in}{0.331635in}}{\pgfqpoint{9.300000in}{7.700000in}}%
\pgfusepath{clip}%
\pgfsetrectcap%
\pgfsetroundjoin%
\pgfsetlinewidth{1.505625pt}%
\definecolor{currentstroke}{rgb}{0.631373,0.788235,0.956863}%
\pgfsetstrokecolor{currentstroke}%
\pgfsetstrokeopacity{0.800000}%
\pgfsetdash{}{0pt}%
\pgfpathmoveto{\pgfqpoint{5.184843in}{3.705537in}}%
\pgfpathlineto{\pgfqpoint{5.711094in}{3.773662in}}%
\pgfusepath{stroke}%
\end{pgfscope}%
\begin{pgfscope}%
\pgfpathrectangle{\pgfqpoint{0.481978in}{0.331635in}}{\pgfqpoint{9.300000in}{7.700000in}}%
\pgfusepath{clip}%
\pgfsetrectcap%
\pgfsetroundjoin%
\pgfsetlinewidth{1.505625pt}%
\definecolor{currentstroke}{rgb}{0.631373,0.788235,0.956863}%
\pgfsetstrokecolor{currentstroke}%
\pgfsetstrokeopacity{0.800000}%
\pgfsetdash{}{0pt}%
\pgfpathmoveto{\pgfqpoint{2.850031in}{2.147835in}}%
\pgfpathlineto{\pgfqpoint{5.711094in}{3.773662in}}%
\pgfusepath{stroke}%
\end{pgfscope}%
\begin{pgfscope}%
\pgfpathrectangle{\pgfqpoint{0.481978in}{0.331635in}}{\pgfqpoint{9.300000in}{7.700000in}}%
\pgfusepath{clip}%
\pgfsetrectcap%
\pgfsetroundjoin%
\pgfsetlinewidth{1.505625pt}%
\definecolor{currentstroke}{rgb}{0.631373,0.788235,0.956863}%
\pgfsetstrokecolor{currentstroke}%
\pgfsetstrokeopacity{0.800000}%
\pgfsetdash{}{0pt}%
\pgfpathmoveto{\pgfqpoint{5.765949in}{2.940193in}}%
\pgfpathlineto{\pgfqpoint{5.711094in}{3.773662in}}%
\pgfusepath{stroke}%
\end{pgfscope}%
\begin{pgfscope}%
\pgfpathrectangle{\pgfqpoint{0.481978in}{0.331635in}}{\pgfqpoint{9.300000in}{7.700000in}}%
\pgfusepath{clip}%
\pgfsetrectcap%
\pgfsetroundjoin%
\pgfsetlinewidth{1.505625pt}%
\definecolor{currentstroke}{rgb}{0.631373,0.788235,0.956863}%
\pgfsetstrokecolor{currentstroke}%
\pgfsetstrokeopacity{0.800000}%
\pgfsetdash{}{0pt}%
\pgfpathmoveto{\pgfqpoint{6.395914in}{3.230225in}}%
\pgfpathlineto{\pgfqpoint{5.711094in}{3.773662in}}%
\pgfusepath{stroke}%
\end{pgfscope}%
\begin{pgfscope}%
\pgfpathrectangle{\pgfqpoint{0.481978in}{0.331635in}}{\pgfqpoint{9.300000in}{7.700000in}}%
\pgfusepath{clip}%
\pgfsetrectcap%
\pgfsetroundjoin%
\pgfsetlinewidth{1.505625pt}%
\definecolor{currentstroke}{rgb}{0.631373,0.788235,0.956863}%
\pgfsetstrokecolor{currentstroke}%
\pgfsetstrokeopacity{0.800000}%
\pgfsetdash{}{0pt}%
\pgfpathmoveto{\pgfqpoint{4.098452in}{6.271448in}}%
\pgfpathlineto{\pgfqpoint{5.711094in}{3.773662in}}%
\pgfusepath{stroke}%
\end{pgfscope}%
\begin{pgfscope}%
\pgfpathrectangle{\pgfqpoint{0.481978in}{0.331635in}}{\pgfqpoint{9.300000in}{7.700000in}}%
\pgfusepath{clip}%
\pgfsetrectcap%
\pgfsetroundjoin%
\pgfsetlinewidth{1.505625pt}%
\definecolor{currentstroke}{rgb}{0.631373,0.788235,0.956863}%
\pgfsetstrokecolor{currentstroke}%
\pgfsetstrokeopacity{0.800000}%
\pgfsetdash{}{0pt}%
\pgfpathmoveto{\pgfqpoint{6.735020in}{4.508870in}}%
\pgfpathlineto{\pgfqpoint{5.711094in}{3.773662in}}%
\pgfusepath{stroke}%
\end{pgfscope}%
\begin{pgfscope}%
\pgfpathrectangle{\pgfqpoint{0.481978in}{0.331635in}}{\pgfqpoint{9.300000in}{7.700000in}}%
\pgfusepath{clip}%
\pgfsetrectcap%
\pgfsetroundjoin%
\pgfsetlinewidth{1.505625pt}%
\definecolor{currentstroke}{rgb}{0.631373,0.788235,0.956863}%
\pgfsetstrokecolor{currentstroke}%
\pgfsetstrokeopacity{0.800000}%
\pgfsetdash{}{0pt}%
\pgfpathmoveto{\pgfqpoint{5.876608in}{2.528955in}}%
\pgfpathlineto{\pgfqpoint{5.711094in}{3.773662in}}%
\pgfusepath{stroke}%
\end{pgfscope}%
\begin{pgfscope}%
\pgfpathrectangle{\pgfqpoint{0.481978in}{0.331635in}}{\pgfqpoint{9.300000in}{7.700000in}}%
\pgfusepath{clip}%
\pgfsetrectcap%
\pgfsetroundjoin%
\pgfsetlinewidth{1.505625pt}%
\definecolor{currentstroke}{rgb}{0.631373,0.788235,0.956863}%
\pgfsetstrokecolor{currentstroke}%
\pgfsetstrokeopacity{0.800000}%
\pgfsetdash{}{0pt}%
\pgfpathmoveto{\pgfqpoint{4.683187in}{4.425480in}}%
\pgfpathlineto{\pgfqpoint{5.711094in}{3.773662in}}%
\pgfusepath{stroke}%
\end{pgfscope}%
\begin{pgfscope}%
\pgfpathrectangle{\pgfqpoint{0.481978in}{0.331635in}}{\pgfqpoint{9.300000in}{7.700000in}}%
\pgfusepath{clip}%
\pgfsetrectcap%
\pgfsetroundjoin%
\pgfsetlinewidth{1.505625pt}%
\definecolor{currentstroke}{rgb}{0.631373,0.788235,0.956863}%
\pgfsetstrokecolor{currentstroke}%
\pgfsetstrokeopacity{0.800000}%
\pgfsetdash{}{0pt}%
\pgfpathmoveto{\pgfqpoint{2.854341in}{2.041662in}}%
\pgfpathlineto{\pgfqpoint{5.711094in}{3.773662in}}%
\pgfusepath{stroke}%
\end{pgfscope}%
\begin{pgfscope}%
\pgfpathrectangle{\pgfqpoint{0.481978in}{0.331635in}}{\pgfqpoint{9.300000in}{7.700000in}}%
\pgfusepath{clip}%
\pgfsetrectcap%
\pgfsetroundjoin%
\pgfsetlinewidth{1.505625pt}%
\definecolor{currentstroke}{rgb}{0.631373,0.788235,0.956863}%
\pgfsetstrokecolor{currentstroke}%
\pgfsetstrokeopacity{0.800000}%
\pgfsetdash{}{0pt}%
\pgfpathmoveto{\pgfqpoint{7.738499in}{4.320193in}}%
\pgfpathlineto{\pgfqpoint{5.711094in}{3.773662in}}%
\pgfusepath{stroke}%
\end{pgfscope}%
\begin{pgfscope}%
\pgfpathrectangle{\pgfqpoint{0.481978in}{0.331635in}}{\pgfqpoint{9.300000in}{7.700000in}}%
\pgfusepath{clip}%
\pgfsetrectcap%
\pgfsetroundjoin%
\pgfsetlinewidth{1.505625pt}%
\definecolor{currentstroke}{rgb}{0.631373,0.788235,0.956863}%
\pgfsetstrokecolor{currentstroke}%
\pgfsetstrokeopacity{0.800000}%
\pgfsetdash{}{0pt}%
\pgfpathmoveto{\pgfqpoint{4.181272in}{7.669721in}}%
\pgfpathlineto{\pgfqpoint{5.711094in}{3.773662in}}%
\pgfusepath{stroke}%
\end{pgfscope}%
\begin{pgfscope}%
\pgfpathrectangle{\pgfqpoint{0.481978in}{0.331635in}}{\pgfqpoint{9.300000in}{7.700000in}}%
\pgfusepath{clip}%
\pgfsetrectcap%
\pgfsetroundjoin%
\pgfsetlinewidth{1.505625pt}%
\definecolor{currentstroke}{rgb}{0.631373,0.788235,0.956863}%
\pgfsetstrokecolor{currentstroke}%
\pgfsetstrokeopacity{0.800000}%
\pgfsetdash{}{0pt}%
\pgfpathmoveto{\pgfqpoint{6.285628in}{4.457219in}}%
\pgfpathlineto{\pgfqpoint{5.711094in}{3.773662in}}%
\pgfusepath{stroke}%
\end{pgfscope}%
\begin{pgfscope}%
\pgfpathrectangle{\pgfqpoint{0.481978in}{0.331635in}}{\pgfqpoint{9.300000in}{7.700000in}}%
\pgfusepath{clip}%
\pgfsetrectcap%
\pgfsetroundjoin%
\pgfsetlinewidth{1.505625pt}%
\definecolor{currentstroke}{rgb}{0.631373,0.788235,0.956863}%
\pgfsetstrokecolor{currentstroke}%
\pgfsetstrokeopacity{0.800000}%
\pgfsetdash{}{0pt}%
\pgfpathmoveto{\pgfqpoint{7.326306in}{5.231850in}}%
\pgfpathlineto{\pgfqpoint{5.711094in}{3.773662in}}%
\pgfusepath{stroke}%
\end{pgfscope}%
\begin{pgfscope}%
\pgfpathrectangle{\pgfqpoint{0.481978in}{0.331635in}}{\pgfqpoint{9.300000in}{7.700000in}}%
\pgfusepath{clip}%
\pgfsetrectcap%
\pgfsetroundjoin%
\pgfsetlinewidth{1.505625pt}%
\definecolor{currentstroke}{rgb}{0.631373,0.788235,0.956863}%
\pgfsetstrokecolor{currentstroke}%
\pgfsetstrokeopacity{0.800000}%
\pgfsetdash{}{0pt}%
\pgfpathmoveto{\pgfqpoint{5.231346in}{2.275215in}}%
\pgfpathlineto{\pgfqpoint{5.711094in}{3.773662in}}%
\pgfusepath{stroke}%
\end{pgfscope}%
\begin{pgfscope}%
\pgfpathrectangle{\pgfqpoint{0.481978in}{0.331635in}}{\pgfqpoint{9.300000in}{7.700000in}}%
\pgfusepath{clip}%
\pgfsetrectcap%
\pgfsetroundjoin%
\pgfsetlinewidth{1.505625pt}%
\definecolor{currentstroke}{rgb}{0.631373,0.788235,0.956863}%
\pgfsetstrokecolor{currentstroke}%
\pgfsetstrokeopacity{0.800000}%
\pgfsetdash{}{0pt}%
\pgfpathmoveto{\pgfqpoint{2.924270in}{1.932016in}}%
\pgfpathlineto{\pgfqpoint{5.711094in}{3.773662in}}%
\pgfusepath{stroke}%
\end{pgfscope}%
\begin{pgfscope}%
\pgfpathrectangle{\pgfqpoint{0.481978in}{0.331635in}}{\pgfqpoint{9.300000in}{7.700000in}}%
\pgfusepath{clip}%
\pgfsetrectcap%
\pgfsetroundjoin%
\pgfsetlinewidth{1.505625pt}%
\definecolor{currentstroke}{rgb}{0.631373,0.788235,0.956863}%
\pgfsetstrokecolor{currentstroke}%
\pgfsetstrokeopacity{0.800000}%
\pgfsetdash{}{0pt}%
\pgfpathmoveto{\pgfqpoint{5.426953in}{7.077580in}}%
\pgfpathlineto{\pgfqpoint{5.711094in}{3.773662in}}%
\pgfusepath{stroke}%
\end{pgfscope}%
\begin{pgfscope}%
\pgfpathrectangle{\pgfqpoint{0.481978in}{0.331635in}}{\pgfqpoint{9.300000in}{7.700000in}}%
\pgfusepath{clip}%
\pgfsetrectcap%
\pgfsetroundjoin%
\pgfsetlinewidth{1.505625pt}%
\definecolor{currentstroke}{rgb}{0.631373,0.788235,0.956863}%
\pgfsetstrokecolor{currentstroke}%
\pgfsetstrokeopacity{0.800000}%
\pgfsetdash{}{0pt}%
\pgfpathmoveto{\pgfqpoint{8.241979in}{5.269247in}}%
\pgfpathlineto{\pgfqpoint{5.711094in}{3.773662in}}%
\pgfusepath{stroke}%
\end{pgfscope}%
\begin{pgfscope}%
\pgfpathrectangle{\pgfqpoint{0.481978in}{0.331635in}}{\pgfqpoint{9.300000in}{7.700000in}}%
\pgfusepath{clip}%
\pgfsetrectcap%
\pgfsetroundjoin%
\pgfsetlinewidth{1.505625pt}%
\definecolor{currentstroke}{rgb}{0.631373,0.788235,0.956863}%
\pgfsetstrokecolor{currentstroke}%
\pgfsetstrokeopacity{0.800000}%
\pgfsetdash{}{0pt}%
\pgfpathmoveto{\pgfqpoint{5.988042in}{2.740024in}}%
\pgfpathlineto{\pgfqpoint{5.711094in}{3.773662in}}%
\pgfusepath{stroke}%
\end{pgfscope}%
\begin{pgfscope}%
\pgfpathrectangle{\pgfqpoint{0.481978in}{0.331635in}}{\pgfqpoint{9.300000in}{7.700000in}}%
\pgfusepath{clip}%
\pgfsetrectcap%
\pgfsetroundjoin%
\pgfsetlinewidth{1.505625pt}%
\definecolor{currentstroke}{rgb}{0.631373,0.788235,0.956863}%
\pgfsetstrokecolor{currentstroke}%
\pgfsetstrokeopacity{0.800000}%
\pgfsetdash{}{0pt}%
\pgfpathmoveto{\pgfqpoint{6.404379in}{5.407426in}}%
\pgfpathlineto{\pgfqpoint{5.711094in}{3.773662in}}%
\pgfusepath{stroke}%
\end{pgfscope}%
\begin{pgfscope}%
\pgfpathrectangle{\pgfqpoint{0.481978in}{0.331635in}}{\pgfqpoint{9.300000in}{7.700000in}}%
\pgfusepath{clip}%
\pgfsetrectcap%
\pgfsetroundjoin%
\pgfsetlinewidth{1.505625pt}%
\definecolor{currentstroke}{rgb}{0.631373,0.788235,0.956863}%
\pgfsetstrokecolor{currentstroke}%
\pgfsetstrokeopacity{0.800000}%
\pgfsetdash{}{0pt}%
\pgfpathmoveto{\pgfqpoint{7.465449in}{4.156782in}}%
\pgfpathlineto{\pgfqpoint{5.711094in}{3.773662in}}%
\pgfusepath{stroke}%
\end{pgfscope}%
\begin{pgfscope}%
\pgfpathrectangle{\pgfqpoint{0.481978in}{0.331635in}}{\pgfqpoint{9.300000in}{7.700000in}}%
\pgfusepath{clip}%
\pgfsetrectcap%
\pgfsetroundjoin%
\pgfsetlinewidth{1.505625pt}%
\definecolor{currentstroke}{rgb}{0.631373,0.788235,0.956863}%
\pgfsetstrokecolor{currentstroke}%
\pgfsetstrokeopacity{0.800000}%
\pgfsetdash{}{0pt}%
\pgfpathmoveto{\pgfqpoint{6.297588in}{4.859032in}}%
\pgfpathlineto{\pgfqpoint{5.711094in}{3.773662in}}%
\pgfusepath{stroke}%
\end{pgfscope}%
\begin{pgfscope}%
\pgfpathrectangle{\pgfqpoint{0.481978in}{0.331635in}}{\pgfqpoint{9.300000in}{7.700000in}}%
\pgfusepath{clip}%
\pgfsetrectcap%
\pgfsetroundjoin%
\pgfsetlinewidth{1.505625pt}%
\definecolor{currentstroke}{rgb}{0.631373,0.788235,0.956863}%
\pgfsetstrokecolor{currentstroke}%
\pgfsetstrokeopacity{0.800000}%
\pgfsetdash{}{0pt}%
\pgfpathmoveto{\pgfqpoint{7.598727in}{4.532028in}}%
\pgfpathlineto{\pgfqpoint{5.711094in}{3.773662in}}%
\pgfusepath{stroke}%
\end{pgfscope}%
\begin{pgfscope}%
\pgfpathrectangle{\pgfqpoint{0.481978in}{0.331635in}}{\pgfqpoint{9.300000in}{7.700000in}}%
\pgfusepath{clip}%
\pgfsetrectcap%
\pgfsetroundjoin%
\pgfsetlinewidth{1.505625pt}%
\definecolor{currentstroke}{rgb}{0.631373,0.788235,0.956863}%
\pgfsetstrokecolor{currentstroke}%
\pgfsetstrokeopacity{0.800000}%
\pgfsetdash{}{0pt}%
\pgfpathmoveto{\pgfqpoint{4.778349in}{5.151613in}}%
\pgfpathlineto{\pgfqpoint{5.711094in}{3.773662in}}%
\pgfusepath{stroke}%
\end{pgfscope}%
\begin{pgfscope}%
\pgfpathrectangle{\pgfqpoint{0.481978in}{0.331635in}}{\pgfqpoint{9.300000in}{7.700000in}}%
\pgfusepath{clip}%
\pgfsetrectcap%
\pgfsetroundjoin%
\pgfsetlinewidth{1.505625pt}%
\definecolor{currentstroke}{rgb}{0.631373,0.788235,0.956863}%
\pgfsetstrokecolor{currentstroke}%
\pgfsetstrokeopacity{0.800000}%
\pgfsetdash{}{0pt}%
\pgfpathmoveto{\pgfqpoint{5.249516in}{5.519220in}}%
\pgfpathlineto{\pgfqpoint{5.711094in}{3.773662in}}%
\pgfusepath{stroke}%
\end{pgfscope}%
\begin{pgfscope}%
\pgfpathrectangle{\pgfqpoint{0.481978in}{0.331635in}}{\pgfqpoint{9.300000in}{7.700000in}}%
\pgfusepath{clip}%
\pgfsetrectcap%
\pgfsetroundjoin%
\pgfsetlinewidth{1.505625pt}%
\definecolor{currentstroke}{rgb}{0.631373,0.788235,0.956863}%
\pgfsetstrokecolor{currentstroke}%
\pgfsetstrokeopacity{0.800000}%
\pgfsetdash{}{0pt}%
\pgfpathmoveto{\pgfqpoint{6.692326in}{1.689848in}}%
\pgfpathlineto{\pgfqpoint{5.711094in}{3.773662in}}%
\pgfusepath{stroke}%
\end{pgfscope}%
\begin{pgfscope}%
\pgfpathrectangle{\pgfqpoint{0.481978in}{0.331635in}}{\pgfqpoint{9.300000in}{7.700000in}}%
\pgfusepath{clip}%
\pgfsetrectcap%
\pgfsetroundjoin%
\pgfsetlinewidth{1.505625pt}%
\definecolor{currentstroke}{rgb}{0.631373,0.788235,0.956863}%
\pgfsetstrokecolor{currentstroke}%
\pgfsetstrokeopacity{0.800000}%
\pgfsetdash{}{0pt}%
\pgfpathmoveto{\pgfqpoint{4.317631in}{1.948751in}}%
\pgfpathlineto{\pgfqpoint{5.711094in}{3.773662in}}%
\pgfusepath{stroke}%
\end{pgfscope}%
\begin{pgfscope}%
\pgfpathrectangle{\pgfqpoint{0.481978in}{0.331635in}}{\pgfqpoint{9.300000in}{7.700000in}}%
\pgfusepath{clip}%
\pgfsetrectcap%
\pgfsetroundjoin%
\pgfsetlinewidth{1.505625pt}%
\definecolor{currentstroke}{rgb}{0.631373,0.788235,0.956863}%
\pgfsetstrokecolor{currentstroke}%
\pgfsetstrokeopacity{0.800000}%
\pgfsetdash{}{0pt}%
\pgfpathmoveto{\pgfqpoint{7.680936in}{6.049583in}}%
\pgfpathlineto{\pgfqpoint{5.711094in}{3.773662in}}%
\pgfusepath{stroke}%
\end{pgfscope}%
\begin{pgfscope}%
\pgfpathrectangle{\pgfqpoint{0.481978in}{0.331635in}}{\pgfqpoint{9.300000in}{7.700000in}}%
\pgfusepath{clip}%
\pgfsetrectcap%
\pgfsetroundjoin%
\pgfsetlinewidth{1.505625pt}%
\definecolor{currentstroke}{rgb}{0.631373,0.788235,0.956863}%
\pgfsetstrokecolor{currentstroke}%
\pgfsetstrokeopacity{0.800000}%
\pgfsetdash{}{0pt}%
\pgfpathmoveto{\pgfqpoint{4.088639in}{6.016536in}}%
\pgfpathlineto{\pgfqpoint{5.711094in}{3.773662in}}%
\pgfusepath{stroke}%
\end{pgfscope}%
\begin{pgfscope}%
\pgfpathrectangle{\pgfqpoint{0.481978in}{0.331635in}}{\pgfqpoint{9.300000in}{7.700000in}}%
\pgfusepath{clip}%
\pgfsetrectcap%
\pgfsetroundjoin%
\pgfsetlinewidth{1.505625pt}%
\definecolor{currentstroke}{rgb}{0.631373,0.788235,0.956863}%
\pgfsetstrokecolor{currentstroke}%
\pgfsetstrokeopacity{0.800000}%
\pgfsetdash{}{0pt}%
\pgfpathmoveto{\pgfqpoint{2.271824in}{3.813978in}}%
\pgfpathlineto{\pgfqpoint{5.711094in}{3.773662in}}%
\pgfusepath{stroke}%
\end{pgfscope}%
\begin{pgfscope}%
\pgfpathrectangle{\pgfqpoint{0.481978in}{0.331635in}}{\pgfqpoint{9.300000in}{7.700000in}}%
\pgfusepath{clip}%
\pgfsetrectcap%
\pgfsetroundjoin%
\pgfsetlinewidth{1.505625pt}%
\definecolor{currentstroke}{rgb}{0.631373,0.788235,0.956863}%
\pgfsetstrokecolor{currentstroke}%
\pgfsetstrokeopacity{0.800000}%
\pgfsetdash{}{0pt}%
\pgfpathmoveto{\pgfqpoint{6.175909in}{2.684328in}}%
\pgfpathlineto{\pgfqpoint{5.711094in}{3.773662in}}%
\pgfusepath{stroke}%
\end{pgfscope}%
\begin{pgfscope}%
\pgfpathrectangle{\pgfqpoint{0.481978in}{0.331635in}}{\pgfqpoint{9.300000in}{7.700000in}}%
\pgfusepath{clip}%
\pgfsetrectcap%
\pgfsetroundjoin%
\pgfsetlinewidth{1.505625pt}%
\definecolor{currentstroke}{rgb}{0.631373,0.788235,0.956863}%
\pgfsetstrokecolor{currentstroke}%
\pgfsetstrokeopacity{0.800000}%
\pgfsetdash{}{0pt}%
\pgfpathmoveto{\pgfqpoint{5.076404in}{1.764299in}}%
\pgfpathlineto{\pgfqpoint{5.711094in}{3.773662in}}%
\pgfusepath{stroke}%
\end{pgfscope}%
\begin{pgfscope}%
\pgfpathrectangle{\pgfqpoint{0.481978in}{0.331635in}}{\pgfqpoint{9.300000in}{7.700000in}}%
\pgfusepath{clip}%
\pgfsetrectcap%
\pgfsetroundjoin%
\pgfsetlinewidth{1.505625pt}%
\definecolor{currentstroke}{rgb}{0.631373,0.788235,0.956863}%
\pgfsetstrokecolor{currentstroke}%
\pgfsetstrokeopacity{0.800000}%
\pgfsetdash{}{0pt}%
\pgfpathmoveto{\pgfqpoint{8.665313in}{4.858730in}}%
\pgfpathlineto{\pgfqpoint{5.711094in}{3.773662in}}%
\pgfusepath{stroke}%
\end{pgfscope}%
\begin{pgfscope}%
\pgfpathrectangle{\pgfqpoint{0.481978in}{0.331635in}}{\pgfqpoint{9.300000in}{7.700000in}}%
\pgfusepath{clip}%
\pgfsetrectcap%
\pgfsetroundjoin%
\pgfsetlinewidth{1.505625pt}%
\definecolor{currentstroke}{rgb}{0.631373,0.788235,0.956863}%
\pgfsetstrokecolor{currentstroke}%
\pgfsetstrokeopacity{0.800000}%
\pgfsetdash{}{0pt}%
\pgfpathmoveto{\pgfqpoint{7.138229in}{2.600769in}}%
\pgfpathlineto{\pgfqpoint{5.711094in}{3.773662in}}%
\pgfusepath{stroke}%
\end{pgfscope}%
\begin{pgfscope}%
\pgfpathrectangle{\pgfqpoint{0.481978in}{0.331635in}}{\pgfqpoint{9.300000in}{7.700000in}}%
\pgfusepath{clip}%
\pgfsetrectcap%
\pgfsetroundjoin%
\pgfsetlinewidth{1.505625pt}%
\definecolor{currentstroke}{rgb}{0.631373,0.788235,0.956863}%
\pgfsetstrokecolor{currentstroke}%
\pgfsetstrokeopacity{0.800000}%
\pgfsetdash{}{0pt}%
\pgfpathmoveto{\pgfqpoint{6.963010in}{1.855391in}}%
\pgfpathlineto{\pgfqpoint{5.711094in}{3.773662in}}%
\pgfusepath{stroke}%
\end{pgfscope}%
\begin{pgfscope}%
\pgfpathrectangle{\pgfqpoint{0.481978in}{0.331635in}}{\pgfqpoint{9.300000in}{7.700000in}}%
\pgfusepath{clip}%
\pgfsetrectcap%
\pgfsetroundjoin%
\pgfsetlinewidth{1.505625pt}%
\definecolor{currentstroke}{rgb}{0.631373,0.788235,0.956863}%
\pgfsetstrokecolor{currentstroke}%
\pgfsetstrokeopacity{0.800000}%
\pgfsetdash{}{0pt}%
\pgfpathmoveto{\pgfqpoint{8.113494in}{5.309463in}}%
\pgfpathlineto{\pgfqpoint{5.711094in}{3.773662in}}%
\pgfusepath{stroke}%
\end{pgfscope}%
\begin{pgfscope}%
\pgfpathrectangle{\pgfqpoint{0.481978in}{0.331635in}}{\pgfqpoint{9.300000in}{7.700000in}}%
\pgfusepath{clip}%
\pgfsetrectcap%
\pgfsetroundjoin%
\pgfsetlinewidth{1.505625pt}%
\definecolor{currentstroke}{rgb}{0.631373,0.788235,0.956863}%
\pgfsetstrokecolor{currentstroke}%
\pgfsetstrokeopacity{0.800000}%
\pgfsetdash{}{0pt}%
\pgfpathmoveto{\pgfqpoint{5.616150in}{1.874161in}}%
\pgfpathlineto{\pgfqpoint{5.711094in}{3.773662in}}%
\pgfusepath{stroke}%
\end{pgfscope}%
\begin{pgfscope}%
\pgfpathrectangle{\pgfqpoint{0.481978in}{0.331635in}}{\pgfqpoint{9.300000in}{7.700000in}}%
\pgfusepath{clip}%
\pgfsetrectcap%
\pgfsetroundjoin%
\pgfsetlinewidth{1.505625pt}%
\definecolor{currentstroke}{rgb}{0.631373,0.788235,0.956863}%
\pgfsetstrokecolor{currentstroke}%
\pgfsetstrokeopacity{0.800000}%
\pgfsetdash{}{0pt}%
\pgfpathmoveto{\pgfqpoint{4.785427in}{1.312445in}}%
\pgfpathlineto{\pgfqpoint{5.711094in}{3.773662in}}%
\pgfusepath{stroke}%
\end{pgfscope}%
\begin{pgfscope}%
\pgfpathrectangle{\pgfqpoint{0.481978in}{0.331635in}}{\pgfqpoint{9.300000in}{7.700000in}}%
\pgfusepath{clip}%
\pgfsetrectcap%
\pgfsetroundjoin%
\pgfsetlinewidth{1.505625pt}%
\definecolor{currentstroke}{rgb}{0.631373,0.788235,0.956863}%
\pgfsetstrokecolor{currentstroke}%
\pgfsetstrokeopacity{0.800000}%
\pgfsetdash{}{0pt}%
\pgfpathmoveto{\pgfqpoint{8.275701in}{5.009576in}}%
\pgfpathlineto{\pgfqpoint{5.711094in}{3.773662in}}%
\pgfusepath{stroke}%
\end{pgfscope}%
\begin{pgfscope}%
\pgfpathrectangle{\pgfqpoint{0.481978in}{0.331635in}}{\pgfqpoint{9.300000in}{7.700000in}}%
\pgfusepath{clip}%
\pgfsetrectcap%
\pgfsetroundjoin%
\pgfsetlinewidth{1.505625pt}%
\definecolor{currentstroke}{rgb}{0.631373,0.788235,0.956863}%
\pgfsetstrokecolor{currentstroke}%
\pgfsetstrokeopacity{0.800000}%
\pgfsetdash{}{0pt}%
\pgfpathmoveto{\pgfqpoint{7.698866in}{5.118187in}}%
\pgfpathlineto{\pgfqpoint{5.711094in}{3.773662in}}%
\pgfusepath{stroke}%
\end{pgfscope}%
\begin{pgfscope}%
\pgfpathrectangle{\pgfqpoint{0.481978in}{0.331635in}}{\pgfqpoint{9.300000in}{7.700000in}}%
\pgfusepath{clip}%
\pgfsetrectcap%
\pgfsetroundjoin%
\pgfsetlinewidth{1.505625pt}%
\definecolor{currentstroke}{rgb}{0.631373,0.788235,0.956863}%
\pgfsetstrokecolor{currentstroke}%
\pgfsetstrokeopacity{0.800000}%
\pgfsetdash{}{0pt}%
\pgfpathmoveto{\pgfqpoint{2.744094in}{5.050535in}}%
\pgfpathlineto{\pgfqpoint{5.711094in}{3.773662in}}%
\pgfusepath{stroke}%
\end{pgfscope}%
\begin{pgfscope}%
\pgfpathrectangle{\pgfqpoint{0.481978in}{0.331635in}}{\pgfqpoint{9.300000in}{7.700000in}}%
\pgfusepath{clip}%
\pgfsetrectcap%
\pgfsetroundjoin%
\pgfsetlinewidth{1.505625pt}%
\definecolor{currentstroke}{rgb}{0.631373,0.788235,0.956863}%
\pgfsetstrokecolor{currentstroke}%
\pgfsetstrokeopacity{0.800000}%
\pgfsetdash{}{0pt}%
\pgfpathmoveto{\pgfqpoint{7.761828in}{4.319095in}}%
\pgfpathlineto{\pgfqpoint{5.711094in}{3.773662in}}%
\pgfusepath{stroke}%
\end{pgfscope}%
\begin{pgfscope}%
\pgfpathrectangle{\pgfqpoint{0.481978in}{0.331635in}}{\pgfqpoint{9.300000in}{7.700000in}}%
\pgfusepath{clip}%
\pgfsetrectcap%
\pgfsetroundjoin%
\pgfsetlinewidth{1.505625pt}%
\definecolor{currentstroke}{rgb}{0.631373,0.788235,0.956863}%
\pgfsetstrokecolor{currentstroke}%
\pgfsetstrokeopacity{0.800000}%
\pgfsetdash{}{0pt}%
\pgfpathmoveto{\pgfqpoint{6.999874in}{2.249399in}}%
\pgfpathlineto{\pgfqpoint{5.711094in}{3.773662in}}%
\pgfusepath{stroke}%
\end{pgfscope}%
\begin{pgfscope}%
\pgfpathrectangle{\pgfqpoint{0.481978in}{0.331635in}}{\pgfqpoint{9.300000in}{7.700000in}}%
\pgfusepath{clip}%
\pgfsetrectcap%
\pgfsetroundjoin%
\pgfsetlinewidth{1.505625pt}%
\definecolor{currentstroke}{rgb}{0.631373,0.788235,0.956863}%
\pgfsetstrokecolor{currentstroke}%
\pgfsetstrokeopacity{0.800000}%
\pgfsetdash{}{0pt}%
\pgfpathmoveto{\pgfqpoint{5.604350in}{5.430750in}}%
\pgfpathlineto{\pgfqpoint{5.711094in}{3.773662in}}%
\pgfusepath{stroke}%
\end{pgfscope}%
\begin{pgfscope}%
\pgfpathrectangle{\pgfqpoint{0.481978in}{0.331635in}}{\pgfqpoint{9.300000in}{7.700000in}}%
\pgfusepath{clip}%
\pgfsetrectcap%
\pgfsetroundjoin%
\pgfsetlinewidth{1.505625pt}%
\definecolor{currentstroke}{rgb}{0.631373,0.788235,0.956863}%
\pgfsetstrokecolor{currentstroke}%
\pgfsetstrokeopacity{0.800000}%
\pgfsetdash{}{0pt}%
\pgfpathmoveto{\pgfqpoint{2.349323in}{2.274882in}}%
\pgfpathlineto{\pgfqpoint{5.711094in}{3.773662in}}%
\pgfusepath{stroke}%
\end{pgfscope}%
\begin{pgfscope}%
\pgfpathrectangle{\pgfqpoint{0.481978in}{0.331635in}}{\pgfqpoint{9.300000in}{7.700000in}}%
\pgfusepath{clip}%
\pgfsetrectcap%
\pgfsetroundjoin%
\pgfsetlinewidth{1.505625pt}%
\definecolor{currentstroke}{rgb}{0.631373,0.788235,0.956863}%
\pgfsetstrokecolor{currentstroke}%
\pgfsetstrokeopacity{0.800000}%
\pgfsetdash{}{0pt}%
\pgfpathmoveto{\pgfqpoint{6.569928in}{3.257001in}}%
\pgfpathlineto{\pgfqpoint{5.711094in}{3.773662in}}%
\pgfusepath{stroke}%
\end{pgfscope}%
\begin{pgfscope}%
\pgfpathrectangle{\pgfqpoint{0.481978in}{0.331635in}}{\pgfqpoint{9.300000in}{7.700000in}}%
\pgfusepath{clip}%
\pgfsetrectcap%
\pgfsetroundjoin%
\pgfsetlinewidth{1.505625pt}%
\definecolor{currentstroke}{rgb}{0.631373,0.788235,0.956863}%
\pgfsetstrokecolor{currentstroke}%
\pgfsetstrokeopacity{0.800000}%
\pgfsetdash{}{0pt}%
\pgfpathmoveto{\pgfqpoint{5.147663in}{2.612815in}}%
\pgfpathlineto{\pgfqpoint{5.711094in}{3.773662in}}%
\pgfusepath{stroke}%
\end{pgfscope}%
\begin{pgfscope}%
\pgfpathrectangle{\pgfqpoint{0.481978in}{0.331635in}}{\pgfqpoint{9.300000in}{7.700000in}}%
\pgfusepath{clip}%
\pgfsetrectcap%
\pgfsetroundjoin%
\pgfsetlinewidth{1.505625pt}%
\definecolor{currentstroke}{rgb}{0.631373,0.788235,0.956863}%
\pgfsetstrokecolor{currentstroke}%
\pgfsetstrokeopacity{0.800000}%
\pgfsetdash{}{0pt}%
\pgfpathmoveto{\pgfqpoint{6.704806in}{2.233579in}}%
\pgfpathlineto{\pgfqpoint{5.711094in}{3.773662in}}%
\pgfusepath{stroke}%
\end{pgfscope}%
\begin{pgfscope}%
\pgfpathrectangle{\pgfqpoint{0.481978in}{0.331635in}}{\pgfqpoint{9.300000in}{7.700000in}}%
\pgfusepath{clip}%
\pgfsetrectcap%
\pgfsetroundjoin%
\pgfsetlinewidth{1.505625pt}%
\definecolor{currentstroke}{rgb}{0.631373,0.788235,0.956863}%
\pgfsetstrokecolor{currentstroke}%
\pgfsetstrokeopacity{0.800000}%
\pgfsetdash{}{0pt}%
\pgfpathmoveto{\pgfqpoint{2.794290in}{5.372826in}}%
\pgfpathlineto{\pgfqpoint{5.711094in}{3.773662in}}%
\pgfusepath{stroke}%
\end{pgfscope}%
\begin{pgfscope}%
\pgfpathrectangle{\pgfqpoint{0.481978in}{0.331635in}}{\pgfqpoint{9.300000in}{7.700000in}}%
\pgfusepath{clip}%
\pgfsetrectcap%
\pgfsetroundjoin%
\pgfsetlinewidth{1.505625pt}%
\definecolor{currentstroke}{rgb}{0.631373,0.788235,0.956863}%
\pgfsetstrokecolor{currentstroke}%
\pgfsetstrokeopacity{0.800000}%
\pgfsetdash{}{0pt}%
\pgfpathmoveto{\pgfqpoint{8.083200in}{5.007083in}}%
\pgfpathlineto{\pgfqpoint{5.711094in}{3.773662in}}%
\pgfusepath{stroke}%
\end{pgfscope}%
\begin{pgfscope}%
\pgfpathrectangle{\pgfqpoint{0.481978in}{0.331635in}}{\pgfqpoint{9.300000in}{7.700000in}}%
\pgfusepath{clip}%
\pgfsetrectcap%
\pgfsetroundjoin%
\pgfsetlinewidth{1.505625pt}%
\definecolor{currentstroke}{rgb}{0.631373,0.788235,0.956863}%
\pgfsetstrokecolor{currentstroke}%
\pgfsetstrokeopacity{0.800000}%
\pgfsetdash{}{0pt}%
\pgfpathmoveto{\pgfqpoint{4.500321in}{1.322703in}}%
\pgfpathlineto{\pgfqpoint{5.711094in}{3.773662in}}%
\pgfusepath{stroke}%
\end{pgfscope}%
\begin{pgfscope}%
\pgfpathrectangle{\pgfqpoint{0.481978in}{0.331635in}}{\pgfqpoint{9.300000in}{7.700000in}}%
\pgfusepath{clip}%
\pgfsetrectcap%
\pgfsetroundjoin%
\pgfsetlinewidth{1.505625pt}%
\definecolor{currentstroke}{rgb}{0.631373,0.788235,0.956863}%
\pgfsetstrokecolor{currentstroke}%
\pgfsetstrokeopacity{0.800000}%
\pgfsetdash{}{0pt}%
\pgfpathmoveto{\pgfqpoint{3.168751in}{1.375498in}}%
\pgfpathlineto{\pgfqpoint{5.711094in}{3.773662in}}%
\pgfusepath{stroke}%
\end{pgfscope}%
\begin{pgfscope}%
\pgfpathrectangle{\pgfqpoint{0.481978in}{0.331635in}}{\pgfqpoint{9.300000in}{7.700000in}}%
\pgfusepath{clip}%
\pgfsetrectcap%
\pgfsetroundjoin%
\pgfsetlinewidth{1.505625pt}%
\definecolor{currentstroke}{rgb}{0.631373,0.788235,0.956863}%
\pgfsetstrokecolor{currentstroke}%
\pgfsetstrokeopacity{0.800000}%
\pgfsetdash{}{0pt}%
\pgfpathmoveto{\pgfqpoint{4.177608in}{7.681635in}}%
\pgfpathlineto{\pgfqpoint{5.711094in}{3.773662in}}%
\pgfusepath{stroke}%
\end{pgfscope}%
\begin{pgfscope}%
\pgfpathrectangle{\pgfqpoint{0.481978in}{0.331635in}}{\pgfqpoint{9.300000in}{7.700000in}}%
\pgfusepath{clip}%
\pgfsetrectcap%
\pgfsetroundjoin%
\pgfsetlinewidth{1.505625pt}%
\definecolor{currentstroke}{rgb}{0.631373,0.788235,0.956863}%
\pgfsetstrokecolor{currentstroke}%
\pgfsetstrokeopacity{0.800000}%
\pgfsetdash{}{0pt}%
\pgfpathmoveto{\pgfqpoint{6.895459in}{3.175574in}}%
\pgfpathlineto{\pgfqpoint{5.711094in}{3.773662in}}%
\pgfusepath{stroke}%
\end{pgfscope}%
\begin{pgfscope}%
\pgfpathrectangle{\pgfqpoint{0.481978in}{0.331635in}}{\pgfqpoint{9.300000in}{7.700000in}}%
\pgfusepath{clip}%
\pgfsetrectcap%
\pgfsetroundjoin%
\pgfsetlinewidth{1.505625pt}%
\definecolor{currentstroke}{rgb}{0.631373,0.788235,0.956863}%
\pgfsetstrokecolor{currentstroke}%
\pgfsetstrokeopacity{0.800000}%
\pgfsetdash{}{0pt}%
\pgfpathmoveto{\pgfqpoint{7.647915in}{1.908788in}}%
\pgfpathlineto{\pgfqpoint{5.711094in}{3.773662in}}%
\pgfusepath{stroke}%
\end{pgfscope}%
\begin{pgfscope}%
\pgfpathrectangle{\pgfqpoint{0.481978in}{0.331635in}}{\pgfqpoint{9.300000in}{7.700000in}}%
\pgfusepath{clip}%
\pgfsetrectcap%
\pgfsetroundjoin%
\pgfsetlinewidth{1.505625pt}%
\definecolor{currentstroke}{rgb}{0.631373,0.788235,0.956863}%
\pgfsetstrokecolor{currentstroke}%
\pgfsetstrokeopacity{0.800000}%
\pgfsetdash{}{0pt}%
\pgfpathmoveto{\pgfqpoint{7.913291in}{4.980495in}}%
\pgfpathlineto{\pgfqpoint{5.711094in}{3.773662in}}%
\pgfusepath{stroke}%
\end{pgfscope}%
\begin{pgfscope}%
\pgfpathrectangle{\pgfqpoint{0.481978in}{0.331635in}}{\pgfqpoint{9.300000in}{7.700000in}}%
\pgfusepath{clip}%
\pgfsetrectcap%
\pgfsetroundjoin%
\pgfsetlinewidth{1.505625pt}%
\definecolor{currentstroke}{rgb}{0.631373,0.788235,0.956863}%
\pgfsetstrokecolor{currentstroke}%
\pgfsetstrokeopacity{0.800000}%
\pgfsetdash{}{0pt}%
\pgfpathmoveto{\pgfqpoint{5.605585in}{2.864452in}}%
\pgfpathlineto{\pgfqpoint{5.711094in}{3.773662in}}%
\pgfusepath{stroke}%
\end{pgfscope}%
\begin{pgfscope}%
\pgfpathrectangle{\pgfqpoint{0.481978in}{0.331635in}}{\pgfqpoint{9.300000in}{7.700000in}}%
\pgfusepath{clip}%
\pgfsetrectcap%
\pgfsetroundjoin%
\pgfsetlinewidth{1.505625pt}%
\definecolor{currentstroke}{rgb}{0.631373,0.788235,0.956863}%
\pgfsetstrokecolor{currentstroke}%
\pgfsetstrokeopacity{0.800000}%
\pgfsetdash{}{0pt}%
\pgfpathmoveto{\pgfqpoint{4.066101in}{5.156275in}}%
\pgfpathlineto{\pgfqpoint{5.711094in}{3.773662in}}%
\pgfusepath{stroke}%
\end{pgfscope}%
\begin{pgfscope}%
\pgfpathrectangle{\pgfqpoint{0.481978in}{0.331635in}}{\pgfqpoint{9.300000in}{7.700000in}}%
\pgfusepath{clip}%
\pgfsetrectcap%
\pgfsetroundjoin%
\pgfsetlinewidth{1.505625pt}%
\definecolor{currentstroke}{rgb}{0.631373,0.788235,0.956863}%
\pgfsetstrokecolor{currentstroke}%
\pgfsetstrokeopacity{0.800000}%
\pgfsetdash{}{0pt}%
\pgfpathmoveto{\pgfqpoint{6.162715in}{2.553984in}}%
\pgfpathlineto{\pgfqpoint{5.711094in}{3.773662in}}%
\pgfusepath{stroke}%
\end{pgfscope}%
\begin{pgfscope}%
\pgfpathrectangle{\pgfqpoint{0.481978in}{0.331635in}}{\pgfqpoint{9.300000in}{7.700000in}}%
\pgfusepath{clip}%
\pgfsetrectcap%
\pgfsetroundjoin%
\pgfsetlinewidth{1.505625pt}%
\definecolor{currentstroke}{rgb}{0.631373,0.788235,0.956863}%
\pgfsetstrokecolor{currentstroke}%
\pgfsetstrokeopacity{0.800000}%
\pgfsetdash{}{0pt}%
\pgfpathmoveto{\pgfqpoint{6.160576in}{3.132339in}}%
\pgfpathlineto{\pgfqpoint{5.711094in}{3.773662in}}%
\pgfusepath{stroke}%
\end{pgfscope}%
\begin{pgfscope}%
\pgfpathrectangle{\pgfqpoint{0.481978in}{0.331635in}}{\pgfqpoint{9.300000in}{7.700000in}}%
\pgfusepath{clip}%
\pgfsetrectcap%
\pgfsetroundjoin%
\pgfsetlinewidth{1.505625pt}%
\definecolor{currentstroke}{rgb}{0.631373,0.788235,0.956863}%
\pgfsetstrokecolor{currentstroke}%
\pgfsetstrokeopacity{0.800000}%
\pgfsetdash{}{0pt}%
\pgfpathmoveto{\pgfqpoint{4.893800in}{6.410049in}}%
\pgfpathlineto{\pgfqpoint{5.711094in}{3.773662in}}%
\pgfusepath{stroke}%
\end{pgfscope}%
\begin{pgfscope}%
\pgfpathrectangle{\pgfqpoint{0.481978in}{0.331635in}}{\pgfqpoint{9.300000in}{7.700000in}}%
\pgfusepath{clip}%
\pgfsetrectcap%
\pgfsetroundjoin%
\pgfsetlinewidth{1.505625pt}%
\definecolor{currentstroke}{rgb}{0.631373,0.788235,0.956863}%
\pgfsetstrokecolor{currentstroke}%
\pgfsetstrokeopacity{0.800000}%
\pgfsetdash{}{0pt}%
\pgfpathmoveto{\pgfqpoint{3.428502in}{1.854777in}}%
\pgfpathlineto{\pgfqpoint{5.711094in}{3.773662in}}%
\pgfusepath{stroke}%
\end{pgfscope}%
\begin{pgfscope}%
\pgfpathrectangle{\pgfqpoint{0.481978in}{0.331635in}}{\pgfqpoint{9.300000in}{7.700000in}}%
\pgfusepath{clip}%
\pgfsetrectcap%
\pgfsetroundjoin%
\pgfsetlinewidth{1.505625pt}%
\definecolor{currentstroke}{rgb}{0.631373,0.788235,0.956863}%
\pgfsetstrokecolor{currentstroke}%
\pgfsetstrokeopacity{0.800000}%
\pgfsetdash{}{0pt}%
\pgfpathmoveto{\pgfqpoint{3.370561in}{7.155344in}}%
\pgfpathlineto{\pgfqpoint{5.711094in}{3.773662in}}%
\pgfusepath{stroke}%
\end{pgfscope}%
\begin{pgfscope}%
\pgfpathrectangle{\pgfqpoint{0.481978in}{0.331635in}}{\pgfqpoint{9.300000in}{7.700000in}}%
\pgfusepath{clip}%
\pgfsetrectcap%
\pgfsetroundjoin%
\pgfsetlinewidth{1.505625pt}%
\definecolor{currentstroke}{rgb}{0.631373,0.788235,0.956863}%
\pgfsetstrokecolor{currentstroke}%
\pgfsetstrokeopacity{0.800000}%
\pgfsetdash{}{0pt}%
\pgfpathmoveto{\pgfqpoint{6.915286in}{4.198455in}}%
\pgfpathlineto{\pgfqpoint{5.711094in}{3.773662in}}%
\pgfusepath{stroke}%
\end{pgfscope}%
\begin{pgfscope}%
\pgfpathrectangle{\pgfqpoint{0.481978in}{0.331635in}}{\pgfqpoint{9.300000in}{7.700000in}}%
\pgfusepath{clip}%
\pgfsetrectcap%
\pgfsetroundjoin%
\pgfsetlinewidth{1.505625pt}%
\definecolor{currentstroke}{rgb}{0.631373,0.788235,0.956863}%
\pgfsetstrokecolor{currentstroke}%
\pgfsetstrokeopacity{0.800000}%
\pgfsetdash{}{0pt}%
\pgfpathmoveto{\pgfqpoint{3.604723in}{5.526537in}}%
\pgfpathlineto{\pgfqpoint{5.711094in}{3.773662in}}%
\pgfusepath{stroke}%
\end{pgfscope}%
\begin{pgfscope}%
\pgfpathrectangle{\pgfqpoint{0.481978in}{0.331635in}}{\pgfqpoint{9.300000in}{7.700000in}}%
\pgfusepath{clip}%
\pgfsetrectcap%
\pgfsetroundjoin%
\pgfsetlinewidth{1.505625pt}%
\definecolor{currentstroke}{rgb}{0.631373,0.788235,0.956863}%
\pgfsetstrokecolor{currentstroke}%
\pgfsetstrokeopacity{0.800000}%
\pgfsetdash{}{0pt}%
\pgfpathmoveto{\pgfqpoint{4.283816in}{3.539087in}}%
\pgfpathlineto{\pgfqpoint{5.711094in}{3.773662in}}%
\pgfusepath{stroke}%
\end{pgfscope}%
\begin{pgfscope}%
\pgfpathrectangle{\pgfqpoint{0.481978in}{0.331635in}}{\pgfqpoint{9.300000in}{7.700000in}}%
\pgfusepath{clip}%
\pgfsetrectcap%
\pgfsetroundjoin%
\pgfsetlinewidth{1.505625pt}%
\definecolor{currentstroke}{rgb}{0.631373,0.788235,0.956863}%
\pgfsetstrokecolor{currentstroke}%
\pgfsetstrokeopacity{0.800000}%
\pgfsetdash{}{0pt}%
\pgfpathmoveto{\pgfqpoint{7.064950in}{1.944367in}}%
\pgfpathlineto{\pgfqpoint{5.711094in}{3.773662in}}%
\pgfusepath{stroke}%
\end{pgfscope}%
\begin{pgfscope}%
\pgfpathrectangle{\pgfqpoint{0.481978in}{0.331635in}}{\pgfqpoint{9.300000in}{7.700000in}}%
\pgfusepath{clip}%
\pgfsetrectcap%
\pgfsetroundjoin%
\pgfsetlinewidth{1.505625pt}%
\definecolor{currentstroke}{rgb}{0.631373,0.788235,0.956863}%
\pgfsetstrokecolor{currentstroke}%
\pgfsetstrokeopacity{0.800000}%
\pgfsetdash{}{0pt}%
\pgfpathmoveto{\pgfqpoint{3.497950in}{6.643419in}}%
\pgfpathlineto{\pgfqpoint{5.711094in}{3.773662in}}%
\pgfusepath{stroke}%
\end{pgfscope}%
\begin{pgfscope}%
\pgfpathrectangle{\pgfqpoint{0.481978in}{0.331635in}}{\pgfqpoint{9.300000in}{7.700000in}}%
\pgfusepath{clip}%
\pgfsetrectcap%
\pgfsetroundjoin%
\pgfsetlinewidth{1.505625pt}%
\definecolor{currentstroke}{rgb}{0.631373,0.788235,0.956863}%
\pgfsetstrokecolor{currentstroke}%
\pgfsetstrokeopacity{0.800000}%
\pgfsetdash{}{0pt}%
\pgfpathmoveto{\pgfqpoint{5.579377in}{3.207767in}}%
\pgfpathlineto{\pgfqpoint{5.711094in}{3.773662in}}%
\pgfusepath{stroke}%
\end{pgfscope}%
\begin{pgfscope}%
\pgfpathrectangle{\pgfqpoint{0.481978in}{0.331635in}}{\pgfqpoint{9.300000in}{7.700000in}}%
\pgfusepath{clip}%
\pgfsetrectcap%
\pgfsetroundjoin%
\pgfsetlinewidth{1.505625pt}%
\definecolor{currentstroke}{rgb}{0.631373,0.788235,0.956863}%
\pgfsetstrokecolor{currentstroke}%
\pgfsetstrokeopacity{0.800000}%
\pgfsetdash{}{0pt}%
\pgfpathmoveto{\pgfqpoint{8.844215in}{5.072545in}}%
\pgfpathlineto{\pgfqpoint{5.711094in}{3.773662in}}%
\pgfusepath{stroke}%
\end{pgfscope}%
\begin{pgfscope}%
\pgfpathrectangle{\pgfqpoint{0.481978in}{0.331635in}}{\pgfqpoint{9.300000in}{7.700000in}}%
\pgfusepath{clip}%
\pgfsetrectcap%
\pgfsetroundjoin%
\pgfsetlinewidth{1.505625pt}%
\definecolor{currentstroke}{rgb}{0.631373,0.788235,0.956863}%
\pgfsetstrokecolor{currentstroke}%
\pgfsetstrokeopacity{0.800000}%
\pgfsetdash{}{0pt}%
\pgfpathmoveto{\pgfqpoint{4.858261in}{1.852850in}}%
\pgfpathlineto{\pgfqpoint{5.711094in}{3.773662in}}%
\pgfusepath{stroke}%
\end{pgfscope}%
\begin{pgfscope}%
\pgfpathrectangle{\pgfqpoint{0.481978in}{0.331635in}}{\pgfqpoint{9.300000in}{7.700000in}}%
\pgfusepath{clip}%
\pgfsetrectcap%
\pgfsetroundjoin%
\pgfsetlinewidth{1.505625pt}%
\definecolor{currentstroke}{rgb}{0.631373,0.788235,0.956863}%
\pgfsetstrokecolor{currentstroke}%
\pgfsetstrokeopacity{0.800000}%
\pgfsetdash{}{0pt}%
\pgfpathmoveto{\pgfqpoint{7.438148in}{1.970168in}}%
\pgfpathlineto{\pgfqpoint{5.711094in}{3.773662in}}%
\pgfusepath{stroke}%
\end{pgfscope}%
\begin{pgfscope}%
\pgfpathrectangle{\pgfqpoint{0.481978in}{0.331635in}}{\pgfqpoint{9.300000in}{7.700000in}}%
\pgfusepath{clip}%
\pgfsetrectcap%
\pgfsetroundjoin%
\pgfsetlinewidth{1.505625pt}%
\definecolor{currentstroke}{rgb}{0.631373,0.788235,0.956863}%
\pgfsetstrokecolor{currentstroke}%
\pgfsetstrokeopacity{0.800000}%
\pgfsetdash{}{0pt}%
\pgfpathmoveto{\pgfqpoint{3.422792in}{1.040261in}}%
\pgfpathlineto{\pgfqpoint{5.711094in}{3.773662in}}%
\pgfusepath{stroke}%
\end{pgfscope}%
\begin{pgfscope}%
\pgfpathrectangle{\pgfqpoint{0.481978in}{0.331635in}}{\pgfqpoint{9.300000in}{7.700000in}}%
\pgfusepath{clip}%
\pgfsetrectcap%
\pgfsetroundjoin%
\pgfsetlinewidth{1.505625pt}%
\definecolor{currentstroke}{rgb}{0.631373,0.788235,0.956863}%
\pgfsetstrokecolor{currentstroke}%
\pgfsetstrokeopacity{0.800000}%
\pgfsetdash{}{0pt}%
\pgfpathmoveto{\pgfqpoint{2.139723in}{2.484932in}}%
\pgfpathlineto{\pgfqpoint{5.711094in}{3.773662in}}%
\pgfusepath{stroke}%
\end{pgfscope}%
\begin{pgfscope}%
\pgfpathrectangle{\pgfqpoint{0.481978in}{0.331635in}}{\pgfqpoint{9.300000in}{7.700000in}}%
\pgfusepath{clip}%
\pgfsetrectcap%
\pgfsetroundjoin%
\pgfsetlinewidth{1.505625pt}%
\definecolor{currentstroke}{rgb}{0.631373,0.788235,0.956863}%
\pgfsetstrokecolor{currentstroke}%
\pgfsetstrokeopacity{0.800000}%
\pgfsetdash{}{0pt}%
\pgfpathmoveto{\pgfqpoint{8.049554in}{5.466103in}}%
\pgfpathlineto{\pgfqpoint{5.711094in}{3.773662in}}%
\pgfusepath{stroke}%
\end{pgfscope}%
\begin{pgfscope}%
\pgfpathrectangle{\pgfqpoint{0.481978in}{0.331635in}}{\pgfqpoint{9.300000in}{7.700000in}}%
\pgfusepath{clip}%
\pgfsetrectcap%
\pgfsetroundjoin%
\pgfsetlinewidth{1.505625pt}%
\definecolor{currentstroke}{rgb}{0.631373,0.788235,0.956863}%
\pgfsetstrokecolor{currentstroke}%
\pgfsetstrokeopacity{0.800000}%
\pgfsetdash{}{0pt}%
\pgfpathmoveto{\pgfqpoint{3.389691in}{1.627458in}}%
\pgfpathlineto{\pgfqpoint{5.711094in}{3.773662in}}%
\pgfusepath{stroke}%
\end{pgfscope}%
\begin{pgfscope}%
\pgfpathrectangle{\pgfqpoint{0.481978in}{0.331635in}}{\pgfqpoint{9.300000in}{7.700000in}}%
\pgfusepath{clip}%
\pgfsetrectcap%
\pgfsetroundjoin%
\pgfsetlinewidth{1.505625pt}%
\definecolor{currentstroke}{rgb}{0.631373,0.788235,0.956863}%
\pgfsetstrokecolor{currentstroke}%
\pgfsetstrokeopacity{0.800000}%
\pgfsetdash{}{0pt}%
\pgfpathmoveto{\pgfqpoint{6.131345in}{5.250597in}}%
\pgfpathlineto{\pgfqpoint{5.711094in}{3.773662in}}%
\pgfusepath{stroke}%
\end{pgfscope}%
\begin{pgfscope}%
\pgfpathrectangle{\pgfqpoint{0.481978in}{0.331635in}}{\pgfqpoint{9.300000in}{7.700000in}}%
\pgfusepath{clip}%
\pgfsetrectcap%
\pgfsetroundjoin%
\pgfsetlinewidth{1.505625pt}%
\definecolor{currentstroke}{rgb}{0.631373,0.788235,0.956863}%
\pgfsetstrokecolor{currentstroke}%
\pgfsetstrokeopacity{0.800000}%
\pgfsetdash{}{0pt}%
\pgfpathmoveto{\pgfqpoint{4.856079in}{6.415647in}}%
\pgfpathlineto{\pgfqpoint{5.711094in}{3.773662in}}%
\pgfusepath{stroke}%
\end{pgfscope}%
\begin{pgfscope}%
\pgfpathrectangle{\pgfqpoint{0.481978in}{0.331635in}}{\pgfqpoint{9.300000in}{7.700000in}}%
\pgfusepath{clip}%
\pgfsetrectcap%
\pgfsetroundjoin%
\pgfsetlinewidth{1.505625pt}%
\definecolor{currentstroke}{rgb}{0.631373,0.788235,0.956863}%
\pgfsetstrokecolor{currentstroke}%
\pgfsetstrokeopacity{0.800000}%
\pgfsetdash{}{0pt}%
\pgfpathmoveto{\pgfqpoint{4.975752in}{4.620168in}}%
\pgfpathlineto{\pgfqpoint{5.711094in}{3.773662in}}%
\pgfusepath{stroke}%
\end{pgfscope}%
\begin{pgfscope}%
\pgfpathrectangle{\pgfqpoint{0.481978in}{0.331635in}}{\pgfqpoint{9.300000in}{7.700000in}}%
\pgfusepath{clip}%
\pgfsetrectcap%
\pgfsetroundjoin%
\pgfsetlinewidth{1.505625pt}%
\definecolor{currentstroke}{rgb}{0.631373,0.788235,0.956863}%
\pgfsetstrokecolor{currentstroke}%
\pgfsetstrokeopacity{0.800000}%
\pgfsetdash{}{0pt}%
\pgfpathmoveto{\pgfqpoint{5.729756in}{5.746283in}}%
\pgfpathlineto{\pgfqpoint{5.711094in}{3.773662in}}%
\pgfusepath{stroke}%
\end{pgfscope}%
\begin{pgfscope}%
\pgfpathrectangle{\pgfqpoint{0.481978in}{0.331635in}}{\pgfqpoint{9.300000in}{7.700000in}}%
\pgfusepath{clip}%
\pgfsetrectcap%
\pgfsetroundjoin%
\pgfsetlinewidth{1.505625pt}%
\definecolor{currentstroke}{rgb}{0.631373,0.788235,0.956863}%
\pgfsetstrokecolor{currentstroke}%
\pgfsetstrokeopacity{0.800000}%
\pgfsetdash{}{0pt}%
\pgfpathmoveto{\pgfqpoint{5.526353in}{1.663412in}}%
\pgfpathlineto{\pgfqpoint{5.711094in}{3.773662in}}%
\pgfusepath{stroke}%
\end{pgfscope}%
\begin{pgfscope}%
\pgfpathrectangle{\pgfqpoint{0.481978in}{0.331635in}}{\pgfqpoint{9.300000in}{7.700000in}}%
\pgfusepath{clip}%
\pgfsetrectcap%
\pgfsetroundjoin%
\pgfsetlinewidth{1.505625pt}%
\definecolor{currentstroke}{rgb}{0.631373,0.788235,0.956863}%
\pgfsetstrokecolor{currentstroke}%
\pgfsetstrokeopacity{0.800000}%
\pgfsetdash{}{0pt}%
\pgfpathmoveto{\pgfqpoint{6.278735in}{2.430417in}}%
\pgfpathlineto{\pgfqpoint{5.711094in}{3.773662in}}%
\pgfusepath{stroke}%
\end{pgfscope}%
\begin{pgfscope}%
\pgfpathrectangle{\pgfqpoint{0.481978in}{0.331635in}}{\pgfqpoint{9.300000in}{7.700000in}}%
\pgfusepath{clip}%
\pgfsetrectcap%
\pgfsetroundjoin%
\pgfsetlinewidth{1.505625pt}%
\definecolor{currentstroke}{rgb}{0.631373,0.788235,0.956863}%
\pgfsetstrokecolor{currentstroke}%
\pgfsetstrokeopacity{0.800000}%
\pgfsetdash{}{0pt}%
\pgfpathmoveto{\pgfqpoint{2.008732in}{2.146558in}}%
\pgfpathlineto{\pgfqpoint{5.711094in}{3.773662in}}%
\pgfusepath{stroke}%
\end{pgfscope}%
\begin{pgfscope}%
\pgfpathrectangle{\pgfqpoint{0.481978in}{0.331635in}}{\pgfqpoint{9.300000in}{7.700000in}}%
\pgfusepath{clip}%
\pgfsetrectcap%
\pgfsetroundjoin%
\pgfsetlinewidth{1.505625pt}%
\definecolor{currentstroke}{rgb}{0.631373,0.788235,0.956863}%
\pgfsetstrokecolor{currentstroke}%
\pgfsetstrokeopacity{0.800000}%
\pgfsetdash{}{0pt}%
\pgfpathmoveto{\pgfqpoint{5.928515in}{1.775836in}}%
\pgfpathlineto{\pgfqpoint{5.711094in}{3.773662in}}%
\pgfusepath{stroke}%
\end{pgfscope}%
\begin{pgfscope}%
\pgfpathrectangle{\pgfqpoint{0.481978in}{0.331635in}}{\pgfqpoint{9.300000in}{7.700000in}}%
\pgfusepath{clip}%
\pgfsetrectcap%
\pgfsetroundjoin%
\pgfsetlinewidth{1.505625pt}%
\definecolor{currentstroke}{rgb}{0.631373,0.788235,0.956863}%
\pgfsetstrokecolor{currentstroke}%
\pgfsetstrokeopacity{0.800000}%
\pgfsetdash{}{0pt}%
\pgfpathmoveto{\pgfqpoint{6.134955in}{2.473624in}}%
\pgfpathlineto{\pgfqpoint{5.711094in}{3.773662in}}%
\pgfusepath{stroke}%
\end{pgfscope}%
\begin{pgfscope}%
\pgfpathrectangle{\pgfqpoint{0.481978in}{0.331635in}}{\pgfqpoint{9.300000in}{7.700000in}}%
\pgfusepath{clip}%
\pgfsetrectcap%
\pgfsetroundjoin%
\pgfsetlinewidth{1.505625pt}%
\definecolor{currentstroke}{rgb}{0.631373,0.788235,0.956863}%
\pgfsetstrokecolor{currentstroke}%
\pgfsetstrokeopacity{0.800000}%
\pgfsetdash{}{0pt}%
\pgfpathmoveto{\pgfqpoint{5.647865in}{3.664159in}}%
\pgfpathlineto{\pgfqpoint{5.711094in}{3.773662in}}%
\pgfusepath{stroke}%
\end{pgfscope}%
\begin{pgfscope}%
\pgfpathrectangle{\pgfqpoint{0.481978in}{0.331635in}}{\pgfqpoint{9.300000in}{7.700000in}}%
\pgfusepath{clip}%
\pgfsetrectcap%
\pgfsetroundjoin%
\pgfsetlinewidth{1.505625pt}%
\definecolor{currentstroke}{rgb}{0.631373,0.788235,0.956863}%
\pgfsetstrokecolor{currentstroke}%
\pgfsetstrokeopacity{0.800000}%
\pgfsetdash{}{0pt}%
\pgfpathmoveto{\pgfqpoint{6.505006in}{5.447387in}}%
\pgfpathlineto{\pgfqpoint{5.711094in}{3.773662in}}%
\pgfusepath{stroke}%
\end{pgfscope}%
\begin{pgfscope}%
\pgfpathrectangle{\pgfqpoint{0.481978in}{0.331635in}}{\pgfqpoint{9.300000in}{7.700000in}}%
\pgfusepath{clip}%
\pgfsetrectcap%
\pgfsetroundjoin%
\pgfsetlinewidth{1.505625pt}%
\definecolor{currentstroke}{rgb}{0.631373,0.788235,0.956863}%
\pgfsetstrokecolor{currentstroke}%
\pgfsetstrokeopacity{0.800000}%
\pgfsetdash{}{0pt}%
\pgfpathmoveto{\pgfqpoint{6.819209in}{3.684591in}}%
\pgfpathlineto{\pgfqpoint{5.711094in}{3.773662in}}%
\pgfusepath{stroke}%
\end{pgfscope}%
\begin{pgfscope}%
\pgfpathrectangle{\pgfqpoint{0.481978in}{0.331635in}}{\pgfqpoint{9.300000in}{7.700000in}}%
\pgfusepath{clip}%
\pgfsetrectcap%
\pgfsetroundjoin%
\pgfsetlinewidth{1.505625pt}%
\definecolor{currentstroke}{rgb}{0.631373,0.788235,0.956863}%
\pgfsetstrokecolor{currentstroke}%
\pgfsetstrokeopacity{0.800000}%
\pgfsetdash{}{0pt}%
\pgfpathmoveto{\pgfqpoint{7.024686in}{3.334185in}}%
\pgfpathlineto{\pgfqpoint{5.711094in}{3.773662in}}%
\pgfusepath{stroke}%
\end{pgfscope}%
\begin{pgfscope}%
\pgfpathrectangle{\pgfqpoint{0.481978in}{0.331635in}}{\pgfqpoint{9.300000in}{7.700000in}}%
\pgfusepath{clip}%
\pgfsetrectcap%
\pgfsetroundjoin%
\pgfsetlinewidth{1.505625pt}%
\definecolor{currentstroke}{rgb}{0.631373,0.788235,0.956863}%
\pgfsetstrokecolor{currentstroke}%
\pgfsetstrokeopacity{0.800000}%
\pgfsetdash{}{0pt}%
\pgfpathmoveto{\pgfqpoint{2.950033in}{6.709299in}}%
\pgfpathlineto{\pgfqpoint{5.711094in}{3.773662in}}%
\pgfusepath{stroke}%
\end{pgfscope}%
\begin{pgfscope}%
\pgfpathrectangle{\pgfqpoint{0.481978in}{0.331635in}}{\pgfqpoint{9.300000in}{7.700000in}}%
\pgfusepath{clip}%
\pgfsetrectcap%
\pgfsetroundjoin%
\pgfsetlinewidth{1.505625pt}%
\definecolor{currentstroke}{rgb}{0.631373,0.788235,0.956863}%
\pgfsetstrokecolor{currentstroke}%
\pgfsetstrokeopacity{0.800000}%
\pgfsetdash{}{0pt}%
\pgfpathmoveto{\pgfqpoint{4.721839in}{7.179361in}}%
\pgfpathlineto{\pgfqpoint{5.711094in}{3.773662in}}%
\pgfusepath{stroke}%
\end{pgfscope}%
\begin{pgfscope}%
\pgfpathrectangle{\pgfqpoint{0.481978in}{0.331635in}}{\pgfqpoint{9.300000in}{7.700000in}}%
\pgfusepath{clip}%
\pgfsetrectcap%
\pgfsetroundjoin%
\pgfsetlinewidth{1.505625pt}%
\definecolor{currentstroke}{rgb}{0.631373,0.788235,0.956863}%
\pgfsetstrokecolor{currentstroke}%
\pgfsetstrokeopacity{0.800000}%
\pgfsetdash{}{0pt}%
\pgfpathmoveto{\pgfqpoint{6.257325in}{5.337340in}}%
\pgfpathlineto{\pgfqpoint{5.711094in}{3.773662in}}%
\pgfusepath{stroke}%
\end{pgfscope}%
\begin{pgfscope}%
\pgfpathrectangle{\pgfqpoint{0.481978in}{0.331635in}}{\pgfqpoint{9.300000in}{7.700000in}}%
\pgfusepath{clip}%
\pgfsetrectcap%
\pgfsetroundjoin%
\pgfsetlinewidth{1.505625pt}%
\definecolor{currentstroke}{rgb}{0.631373,0.788235,0.956863}%
\pgfsetstrokecolor{currentstroke}%
\pgfsetstrokeopacity{0.800000}%
\pgfsetdash{}{0pt}%
\pgfpathmoveto{\pgfqpoint{7.727200in}{4.854891in}}%
\pgfpathlineto{\pgfqpoint{5.711094in}{3.773662in}}%
\pgfusepath{stroke}%
\end{pgfscope}%
\begin{pgfscope}%
\pgfpathrectangle{\pgfqpoint{0.481978in}{0.331635in}}{\pgfqpoint{9.300000in}{7.700000in}}%
\pgfusepath{clip}%
\pgfsetrectcap%
\pgfsetroundjoin%
\pgfsetlinewidth{1.505625pt}%
\definecolor{currentstroke}{rgb}{0.631373,0.788235,0.956863}%
\pgfsetstrokecolor{currentstroke}%
\pgfsetstrokeopacity{0.800000}%
\pgfsetdash{}{0pt}%
\pgfpathmoveto{\pgfqpoint{3.030295in}{1.728271in}}%
\pgfpathlineto{\pgfqpoint{5.711094in}{3.773662in}}%
\pgfusepath{stroke}%
\end{pgfscope}%
\begin{pgfscope}%
\pgfpathrectangle{\pgfqpoint{0.481978in}{0.331635in}}{\pgfqpoint{9.300000in}{7.700000in}}%
\pgfusepath{clip}%
\pgfsetrectcap%
\pgfsetroundjoin%
\pgfsetlinewidth{1.505625pt}%
\definecolor{currentstroke}{rgb}{0.631373,0.788235,0.956863}%
\pgfsetstrokecolor{currentstroke}%
\pgfsetstrokeopacity{0.800000}%
\pgfsetdash{}{0pt}%
\pgfpathmoveto{\pgfqpoint{4.665732in}{6.497651in}}%
\pgfpathlineto{\pgfqpoint{5.711094in}{3.773662in}}%
\pgfusepath{stroke}%
\end{pgfscope}%
\begin{pgfscope}%
\pgfpathrectangle{\pgfqpoint{0.481978in}{0.331635in}}{\pgfqpoint{9.300000in}{7.700000in}}%
\pgfusepath{clip}%
\pgfsetrectcap%
\pgfsetroundjoin%
\pgfsetlinewidth{1.505625pt}%
\definecolor{currentstroke}{rgb}{0.631373,0.788235,0.956863}%
\pgfsetstrokecolor{currentstroke}%
\pgfsetstrokeopacity{0.800000}%
\pgfsetdash{}{0pt}%
\pgfpathmoveto{\pgfqpoint{6.123240in}{2.041304in}}%
\pgfpathlineto{\pgfqpoint{5.711094in}{3.773662in}}%
\pgfusepath{stroke}%
\end{pgfscope}%
\begin{pgfscope}%
\pgfpathrectangle{\pgfqpoint{0.481978in}{0.331635in}}{\pgfqpoint{9.300000in}{7.700000in}}%
\pgfusepath{clip}%
\pgfsetrectcap%
\pgfsetroundjoin%
\pgfsetlinewidth{1.505625pt}%
\definecolor{currentstroke}{rgb}{0.631373,0.788235,0.956863}%
\pgfsetstrokecolor{currentstroke}%
\pgfsetstrokeopacity{0.800000}%
\pgfsetdash{}{0pt}%
\pgfpathmoveto{\pgfqpoint{2.813753in}{6.523043in}}%
\pgfpathlineto{\pgfqpoint{5.711094in}{3.773662in}}%
\pgfusepath{stroke}%
\end{pgfscope}%
\begin{pgfscope}%
\pgfpathrectangle{\pgfqpoint{0.481978in}{0.331635in}}{\pgfqpoint{9.300000in}{7.700000in}}%
\pgfusepath{clip}%
\pgfsetrectcap%
\pgfsetroundjoin%
\pgfsetlinewidth{1.505625pt}%
\definecolor{currentstroke}{rgb}{0.631373,0.788235,0.956863}%
\pgfsetstrokecolor{currentstroke}%
\pgfsetstrokeopacity{0.800000}%
\pgfsetdash{}{0pt}%
\pgfpathmoveto{\pgfqpoint{8.071034in}{4.423951in}}%
\pgfpathlineto{\pgfqpoint{5.711094in}{3.773662in}}%
\pgfusepath{stroke}%
\end{pgfscope}%
\begin{pgfscope}%
\pgfpathrectangle{\pgfqpoint{0.481978in}{0.331635in}}{\pgfqpoint{9.300000in}{7.700000in}}%
\pgfusepath{clip}%
\pgfsetrectcap%
\pgfsetroundjoin%
\pgfsetlinewidth{1.505625pt}%
\definecolor{currentstroke}{rgb}{0.631373,0.788235,0.956863}%
\pgfsetstrokecolor{currentstroke}%
\pgfsetstrokeopacity{0.800000}%
\pgfsetdash{}{0pt}%
\pgfpathmoveto{\pgfqpoint{6.467051in}{2.365045in}}%
\pgfpathlineto{\pgfqpoint{5.711094in}{3.773662in}}%
\pgfusepath{stroke}%
\end{pgfscope}%
\begin{pgfscope}%
\pgfpathrectangle{\pgfqpoint{0.481978in}{0.331635in}}{\pgfqpoint{9.300000in}{7.700000in}}%
\pgfusepath{clip}%
\pgfsetrectcap%
\pgfsetroundjoin%
\pgfsetlinewidth{1.505625pt}%
\definecolor{currentstroke}{rgb}{0.631373,0.788235,0.956863}%
\pgfsetstrokecolor{currentstroke}%
\pgfsetstrokeopacity{0.800000}%
\pgfsetdash{}{0pt}%
\pgfpathmoveto{\pgfqpoint{7.133111in}{4.714051in}}%
\pgfpathlineto{\pgfqpoint{5.711094in}{3.773662in}}%
\pgfusepath{stroke}%
\end{pgfscope}%
\begin{pgfscope}%
\pgfpathrectangle{\pgfqpoint{0.481978in}{0.331635in}}{\pgfqpoint{9.300000in}{7.700000in}}%
\pgfusepath{clip}%
\pgfsetrectcap%
\pgfsetroundjoin%
\pgfsetlinewidth{1.505625pt}%
\definecolor{currentstroke}{rgb}{0.631373,0.788235,0.956863}%
\pgfsetstrokecolor{currentstroke}%
\pgfsetstrokeopacity{0.800000}%
\pgfsetdash{}{0pt}%
\pgfpathmoveto{\pgfqpoint{5.228856in}{2.192608in}}%
\pgfpathlineto{\pgfqpoint{5.711094in}{3.773662in}}%
\pgfusepath{stroke}%
\end{pgfscope}%
\begin{pgfscope}%
\pgfpathrectangle{\pgfqpoint{0.481978in}{0.331635in}}{\pgfqpoint{9.300000in}{7.700000in}}%
\pgfusepath{clip}%
\pgfsetrectcap%
\pgfsetroundjoin%
\pgfsetlinewidth{1.505625pt}%
\definecolor{currentstroke}{rgb}{0.631373,0.788235,0.956863}%
\pgfsetstrokecolor{currentstroke}%
\pgfsetstrokeopacity{0.800000}%
\pgfsetdash{}{0pt}%
\pgfpathmoveto{\pgfqpoint{6.173996in}{3.318811in}}%
\pgfpathlineto{\pgfqpoint{5.711094in}{3.773662in}}%
\pgfusepath{stroke}%
\end{pgfscope}%
\begin{pgfscope}%
\pgfpathrectangle{\pgfqpoint{0.481978in}{0.331635in}}{\pgfqpoint{9.300000in}{7.700000in}}%
\pgfusepath{clip}%
\pgfsetrectcap%
\pgfsetroundjoin%
\pgfsetlinewidth{1.505625pt}%
\definecolor{currentstroke}{rgb}{0.631373,0.788235,0.956863}%
\pgfsetstrokecolor{currentstroke}%
\pgfsetstrokeopacity{0.800000}%
\pgfsetdash{}{0pt}%
\pgfpathmoveto{\pgfqpoint{2.961409in}{1.537320in}}%
\pgfpathlineto{\pgfqpoint{5.711094in}{3.773662in}}%
\pgfusepath{stroke}%
\end{pgfscope}%
\begin{pgfscope}%
\pgfpathrectangle{\pgfqpoint{0.481978in}{0.331635in}}{\pgfqpoint{9.300000in}{7.700000in}}%
\pgfusepath{clip}%
\pgfsetrectcap%
\pgfsetroundjoin%
\pgfsetlinewidth{1.505625pt}%
\definecolor{currentstroke}{rgb}{0.631373,0.788235,0.956863}%
\pgfsetstrokecolor{currentstroke}%
\pgfsetstrokeopacity{0.800000}%
\pgfsetdash{}{0pt}%
\pgfpathmoveto{\pgfqpoint{5.678749in}{4.888232in}}%
\pgfpathlineto{\pgfqpoint{5.711094in}{3.773662in}}%
\pgfusepath{stroke}%
\end{pgfscope}%
\begin{pgfscope}%
\pgfpathrectangle{\pgfqpoint{0.481978in}{0.331635in}}{\pgfqpoint{9.300000in}{7.700000in}}%
\pgfusepath{clip}%
\pgfsetrectcap%
\pgfsetroundjoin%
\pgfsetlinewidth{1.505625pt}%
\definecolor{currentstroke}{rgb}{0.631373,0.788235,0.956863}%
\pgfsetstrokecolor{currentstroke}%
\pgfsetstrokeopacity{0.800000}%
\pgfsetdash{}{0pt}%
\pgfpathmoveto{\pgfqpoint{4.278462in}{1.312611in}}%
\pgfpathlineto{\pgfqpoint{5.711094in}{3.773662in}}%
\pgfusepath{stroke}%
\end{pgfscope}%
\begin{pgfscope}%
\pgfpathrectangle{\pgfqpoint{0.481978in}{0.331635in}}{\pgfqpoint{9.300000in}{7.700000in}}%
\pgfusepath{clip}%
\pgfsetrectcap%
\pgfsetroundjoin%
\pgfsetlinewidth{1.505625pt}%
\definecolor{currentstroke}{rgb}{0.631373,0.788235,0.956863}%
\pgfsetstrokecolor{currentstroke}%
\pgfsetstrokeopacity{0.800000}%
\pgfsetdash{}{0pt}%
\pgfpathmoveto{\pgfqpoint{6.739909in}{2.297981in}}%
\pgfpathlineto{\pgfqpoint{5.711094in}{3.773662in}}%
\pgfusepath{stroke}%
\end{pgfscope}%
\begin{pgfscope}%
\pgfpathrectangle{\pgfqpoint{0.481978in}{0.331635in}}{\pgfqpoint{9.300000in}{7.700000in}}%
\pgfusepath{clip}%
\pgfsetrectcap%
\pgfsetroundjoin%
\pgfsetlinewidth{1.505625pt}%
\definecolor{currentstroke}{rgb}{0.631373,0.788235,0.956863}%
\pgfsetstrokecolor{currentstroke}%
\pgfsetstrokeopacity{0.800000}%
\pgfsetdash{}{0pt}%
\pgfpathmoveto{\pgfqpoint{5.837904in}{2.580252in}}%
\pgfpathlineto{\pgfqpoint{5.711094in}{3.773662in}}%
\pgfusepath{stroke}%
\end{pgfscope}%
\begin{pgfscope}%
\pgfpathrectangle{\pgfqpoint{0.481978in}{0.331635in}}{\pgfqpoint{9.300000in}{7.700000in}}%
\pgfusepath{clip}%
\pgfsetrectcap%
\pgfsetroundjoin%
\pgfsetlinewidth{1.505625pt}%
\definecolor{currentstroke}{rgb}{0.631373,0.788235,0.956863}%
\pgfsetstrokecolor{currentstroke}%
\pgfsetstrokeopacity{0.800000}%
\pgfsetdash{}{0pt}%
\pgfpathmoveto{\pgfqpoint{7.682296in}{2.311061in}}%
\pgfpathlineto{\pgfqpoint{5.711094in}{3.773662in}}%
\pgfusepath{stroke}%
\end{pgfscope}%
\begin{pgfscope}%
\pgfpathrectangle{\pgfqpoint{0.481978in}{0.331635in}}{\pgfqpoint{9.300000in}{7.700000in}}%
\pgfusepath{clip}%
\pgfsetrectcap%
\pgfsetroundjoin%
\pgfsetlinewidth{1.505625pt}%
\definecolor{currentstroke}{rgb}{0.631373,0.788235,0.956863}%
\pgfsetstrokecolor{currentstroke}%
\pgfsetstrokeopacity{0.800000}%
\pgfsetdash{}{0pt}%
\pgfpathmoveto{\pgfqpoint{4.017749in}{1.899805in}}%
\pgfpathlineto{\pgfqpoint{5.711094in}{3.773662in}}%
\pgfusepath{stroke}%
\end{pgfscope}%
\begin{pgfscope}%
\pgfpathrectangle{\pgfqpoint{0.481978in}{0.331635in}}{\pgfqpoint{9.300000in}{7.700000in}}%
\pgfusepath{clip}%
\pgfsetrectcap%
\pgfsetroundjoin%
\pgfsetlinewidth{1.505625pt}%
\definecolor{currentstroke}{rgb}{0.631373,0.788235,0.956863}%
\pgfsetstrokecolor{currentstroke}%
\pgfsetstrokeopacity{0.800000}%
\pgfsetdash{}{0pt}%
\pgfpathmoveto{\pgfqpoint{6.127545in}{4.884465in}}%
\pgfpathlineto{\pgfqpoint{5.711094in}{3.773662in}}%
\pgfusepath{stroke}%
\end{pgfscope}%
\begin{pgfscope}%
\pgfpathrectangle{\pgfqpoint{0.481978in}{0.331635in}}{\pgfqpoint{9.300000in}{7.700000in}}%
\pgfusepath{clip}%
\pgfsetrectcap%
\pgfsetroundjoin%
\pgfsetlinewidth{1.505625pt}%
\definecolor{currentstroke}{rgb}{0.631373,0.788235,0.956863}%
\pgfsetstrokecolor{currentstroke}%
\pgfsetstrokeopacity{0.800000}%
\pgfsetdash{}{0pt}%
\pgfpathmoveto{\pgfqpoint{7.971575in}{5.289470in}}%
\pgfpathlineto{\pgfqpoint{5.711094in}{3.773662in}}%
\pgfusepath{stroke}%
\end{pgfscope}%
\begin{pgfscope}%
\pgfpathrectangle{\pgfqpoint{0.481978in}{0.331635in}}{\pgfqpoint{9.300000in}{7.700000in}}%
\pgfusepath{clip}%
\pgfsetrectcap%
\pgfsetroundjoin%
\pgfsetlinewidth{1.505625pt}%
\definecolor{currentstroke}{rgb}{0.631373,0.788235,0.956863}%
\pgfsetstrokecolor{currentstroke}%
\pgfsetstrokeopacity{0.800000}%
\pgfsetdash{}{0pt}%
\pgfpathmoveto{\pgfqpoint{5.508663in}{2.419584in}}%
\pgfpathlineto{\pgfqpoint{5.711094in}{3.773662in}}%
\pgfusepath{stroke}%
\end{pgfscope}%
\begin{pgfscope}%
\pgfpathrectangle{\pgfqpoint{0.481978in}{0.331635in}}{\pgfqpoint{9.300000in}{7.700000in}}%
\pgfusepath{clip}%
\pgfsetrectcap%
\pgfsetroundjoin%
\pgfsetlinewidth{1.505625pt}%
\definecolor{currentstroke}{rgb}{0.631373,0.788235,0.956863}%
\pgfsetstrokecolor{currentstroke}%
\pgfsetstrokeopacity{0.800000}%
\pgfsetdash{}{0pt}%
\pgfpathmoveto{\pgfqpoint{4.730932in}{7.181326in}}%
\pgfpathlineto{\pgfqpoint{5.711094in}{3.773662in}}%
\pgfusepath{stroke}%
\end{pgfscope}%
\begin{pgfscope}%
\pgfpathrectangle{\pgfqpoint{0.481978in}{0.331635in}}{\pgfqpoint{9.300000in}{7.700000in}}%
\pgfusepath{clip}%
\pgfsetrectcap%
\pgfsetroundjoin%
\pgfsetlinewidth{1.505625pt}%
\definecolor{currentstroke}{rgb}{0.631373,0.788235,0.956863}%
\pgfsetstrokecolor{currentstroke}%
\pgfsetstrokeopacity{0.800000}%
\pgfsetdash{}{0pt}%
\pgfpathmoveto{\pgfqpoint{5.337518in}{5.951535in}}%
\pgfpathlineto{\pgfqpoint{5.711094in}{3.773662in}}%
\pgfusepath{stroke}%
\end{pgfscope}%
\begin{pgfscope}%
\pgfpathrectangle{\pgfqpoint{0.481978in}{0.331635in}}{\pgfqpoint{9.300000in}{7.700000in}}%
\pgfusepath{clip}%
\pgfsetrectcap%
\pgfsetroundjoin%
\pgfsetlinewidth{1.505625pt}%
\definecolor{currentstroke}{rgb}{0.631373,0.788235,0.956863}%
\pgfsetstrokecolor{currentstroke}%
\pgfsetstrokeopacity{0.800000}%
\pgfsetdash{}{0pt}%
\pgfpathmoveto{\pgfqpoint{5.165140in}{3.130223in}}%
\pgfpathlineto{\pgfqpoint{5.711094in}{3.773662in}}%
\pgfusepath{stroke}%
\end{pgfscope}%
\begin{pgfscope}%
\pgfpathrectangle{\pgfqpoint{0.481978in}{0.331635in}}{\pgfqpoint{9.300000in}{7.700000in}}%
\pgfusepath{clip}%
\pgfsetrectcap%
\pgfsetroundjoin%
\pgfsetlinewidth{1.505625pt}%
\definecolor{currentstroke}{rgb}{0.631373,0.788235,0.956863}%
\pgfsetstrokecolor{currentstroke}%
\pgfsetstrokeopacity{0.800000}%
\pgfsetdash{}{0pt}%
\pgfpathmoveto{\pgfqpoint{6.635187in}{2.624141in}}%
\pgfpathlineto{\pgfqpoint{5.711094in}{3.773662in}}%
\pgfusepath{stroke}%
\end{pgfscope}%
\begin{pgfscope}%
\pgfpathrectangle{\pgfqpoint{0.481978in}{0.331635in}}{\pgfqpoint{9.300000in}{7.700000in}}%
\pgfusepath{clip}%
\pgfsetrectcap%
\pgfsetroundjoin%
\pgfsetlinewidth{1.505625pt}%
\definecolor{currentstroke}{rgb}{0.631373,0.788235,0.956863}%
\pgfsetstrokecolor{currentstroke}%
\pgfsetstrokeopacity{0.800000}%
\pgfsetdash{}{0pt}%
\pgfpathmoveto{\pgfqpoint{2.269929in}{6.475456in}}%
\pgfpathlineto{\pgfqpoint{5.711094in}{3.773662in}}%
\pgfusepath{stroke}%
\end{pgfscope}%
\begin{pgfscope}%
\pgfpathrectangle{\pgfqpoint{0.481978in}{0.331635in}}{\pgfqpoint{9.300000in}{7.700000in}}%
\pgfusepath{clip}%
\pgfsetrectcap%
\pgfsetroundjoin%
\pgfsetlinewidth{1.505625pt}%
\definecolor{currentstroke}{rgb}{0.631373,0.788235,0.956863}%
\pgfsetstrokecolor{currentstroke}%
\pgfsetstrokeopacity{0.800000}%
\pgfsetdash{}{0pt}%
\pgfpathmoveto{\pgfqpoint{6.041629in}{4.151018in}}%
\pgfpathlineto{\pgfqpoint{5.711094in}{3.773662in}}%
\pgfusepath{stroke}%
\end{pgfscope}%
\begin{pgfscope}%
\pgfpathrectangle{\pgfqpoint{0.481978in}{0.331635in}}{\pgfqpoint{9.300000in}{7.700000in}}%
\pgfusepath{clip}%
\pgfsetrectcap%
\pgfsetroundjoin%
\pgfsetlinewidth{1.505625pt}%
\definecolor{currentstroke}{rgb}{0.631373,0.788235,0.956863}%
\pgfsetstrokecolor{currentstroke}%
\pgfsetstrokeopacity{0.800000}%
\pgfsetdash{}{0pt}%
\pgfpathmoveto{\pgfqpoint{7.965678in}{4.024809in}}%
\pgfpathlineto{\pgfqpoint{5.711094in}{3.773662in}}%
\pgfusepath{stroke}%
\end{pgfscope}%
\begin{pgfscope}%
\pgfpathrectangle{\pgfqpoint{0.481978in}{0.331635in}}{\pgfqpoint{9.300000in}{7.700000in}}%
\pgfusepath{clip}%
\pgfsetrectcap%
\pgfsetroundjoin%
\pgfsetlinewidth{1.505625pt}%
\definecolor{currentstroke}{rgb}{0.631373,0.788235,0.956863}%
\pgfsetstrokecolor{currentstroke}%
\pgfsetstrokeopacity{0.800000}%
\pgfsetdash{}{0pt}%
\pgfpathmoveto{\pgfqpoint{6.739115in}{1.489329in}}%
\pgfpathlineto{\pgfqpoint{5.711094in}{3.773662in}}%
\pgfusepath{stroke}%
\end{pgfscope}%
\begin{pgfscope}%
\pgfpathrectangle{\pgfqpoint{0.481978in}{0.331635in}}{\pgfqpoint{9.300000in}{7.700000in}}%
\pgfusepath{clip}%
\pgfsetrectcap%
\pgfsetroundjoin%
\pgfsetlinewidth{1.505625pt}%
\definecolor{currentstroke}{rgb}{0.631373,0.788235,0.956863}%
\pgfsetstrokecolor{currentstroke}%
\pgfsetstrokeopacity{0.800000}%
\pgfsetdash{}{0pt}%
\pgfpathmoveto{\pgfqpoint{5.908920in}{1.565271in}}%
\pgfpathlineto{\pgfqpoint{5.711094in}{3.773662in}}%
\pgfusepath{stroke}%
\end{pgfscope}%
\begin{pgfscope}%
\pgfpathrectangle{\pgfqpoint{0.481978in}{0.331635in}}{\pgfqpoint{9.300000in}{7.700000in}}%
\pgfusepath{clip}%
\pgfsetrectcap%
\pgfsetroundjoin%
\pgfsetlinewidth{1.505625pt}%
\definecolor{currentstroke}{rgb}{0.631373,0.788235,0.956863}%
\pgfsetstrokecolor{currentstroke}%
\pgfsetstrokeopacity{0.800000}%
\pgfsetdash{}{0pt}%
\pgfpathmoveto{\pgfqpoint{4.782057in}{1.328628in}}%
\pgfpathlineto{\pgfqpoint{5.711094in}{3.773662in}}%
\pgfusepath{stroke}%
\end{pgfscope}%
\begin{pgfscope}%
\pgfpathrectangle{\pgfqpoint{0.481978in}{0.331635in}}{\pgfqpoint{9.300000in}{7.700000in}}%
\pgfusepath{clip}%
\pgfsetrectcap%
\pgfsetroundjoin%
\pgfsetlinewidth{1.505625pt}%
\definecolor{currentstroke}{rgb}{0.631373,0.788235,0.956863}%
\pgfsetstrokecolor{currentstroke}%
\pgfsetstrokeopacity{0.800000}%
\pgfsetdash{}{0pt}%
\pgfpathmoveto{\pgfqpoint{2.762676in}{1.438822in}}%
\pgfpathlineto{\pgfqpoint{5.711094in}{3.773662in}}%
\pgfusepath{stroke}%
\end{pgfscope}%
\begin{pgfscope}%
\pgfpathrectangle{\pgfqpoint{0.481978in}{0.331635in}}{\pgfqpoint{9.300000in}{7.700000in}}%
\pgfusepath{clip}%
\pgfsetrectcap%
\pgfsetroundjoin%
\pgfsetlinewidth{1.505625pt}%
\definecolor{currentstroke}{rgb}{0.631373,0.788235,0.956863}%
\pgfsetstrokecolor{currentstroke}%
\pgfsetstrokeopacity{0.800000}%
\pgfsetdash{}{0pt}%
\pgfpathmoveto{\pgfqpoint{6.674444in}{1.875958in}}%
\pgfpathlineto{\pgfqpoint{5.711094in}{3.773662in}}%
\pgfusepath{stroke}%
\end{pgfscope}%
\begin{pgfscope}%
\pgfpathrectangle{\pgfqpoint{0.481978in}{0.331635in}}{\pgfqpoint{9.300000in}{7.700000in}}%
\pgfusepath{clip}%
\pgfsetrectcap%
\pgfsetroundjoin%
\pgfsetlinewidth{1.505625pt}%
\definecolor{currentstroke}{rgb}{0.631373,0.788235,0.956863}%
\pgfsetstrokecolor{currentstroke}%
\pgfsetstrokeopacity{0.800000}%
\pgfsetdash{}{0pt}%
\pgfpathmoveto{\pgfqpoint{7.190919in}{2.117314in}}%
\pgfpathlineto{\pgfqpoint{5.711094in}{3.773662in}}%
\pgfusepath{stroke}%
\end{pgfscope}%
\begin{pgfscope}%
\pgfpathrectangle{\pgfqpoint{0.481978in}{0.331635in}}{\pgfqpoint{9.300000in}{7.700000in}}%
\pgfusepath{clip}%
\pgfsetrectcap%
\pgfsetroundjoin%
\pgfsetlinewidth{1.505625pt}%
\definecolor{currentstroke}{rgb}{0.631373,0.788235,0.956863}%
\pgfsetstrokecolor{currentstroke}%
\pgfsetstrokeopacity{0.800000}%
\pgfsetdash{}{0pt}%
\pgfpathmoveto{\pgfqpoint{5.231440in}{6.892073in}}%
\pgfpathlineto{\pgfqpoint{5.711094in}{3.773662in}}%
\pgfusepath{stroke}%
\end{pgfscope}%
\begin{pgfscope}%
\pgfpathrectangle{\pgfqpoint{0.481978in}{0.331635in}}{\pgfqpoint{9.300000in}{7.700000in}}%
\pgfusepath{clip}%
\pgfsetrectcap%
\pgfsetroundjoin%
\pgfsetlinewidth{1.505625pt}%
\definecolor{currentstroke}{rgb}{0.631373,0.788235,0.956863}%
\pgfsetstrokecolor{currentstroke}%
\pgfsetstrokeopacity{0.800000}%
\pgfsetdash{}{0pt}%
\pgfpathmoveto{\pgfqpoint{7.641158in}{4.391700in}}%
\pgfpathlineto{\pgfqpoint{5.711094in}{3.773662in}}%
\pgfusepath{stroke}%
\end{pgfscope}%
\begin{pgfscope}%
\pgfpathrectangle{\pgfqpoint{0.481978in}{0.331635in}}{\pgfqpoint{9.300000in}{7.700000in}}%
\pgfusepath{clip}%
\pgfsetrectcap%
\pgfsetroundjoin%
\pgfsetlinewidth{1.505625pt}%
\definecolor{currentstroke}{rgb}{0.631373,0.788235,0.956863}%
\pgfsetstrokecolor{currentstroke}%
\pgfsetstrokeopacity{0.800000}%
\pgfsetdash{}{0pt}%
\pgfpathmoveto{\pgfqpoint{5.917655in}{4.377356in}}%
\pgfpathlineto{\pgfqpoint{5.711094in}{3.773662in}}%
\pgfusepath{stroke}%
\end{pgfscope}%
\begin{pgfscope}%
\pgfpathrectangle{\pgfqpoint{0.481978in}{0.331635in}}{\pgfqpoint{9.300000in}{7.700000in}}%
\pgfusepath{clip}%
\pgfsetrectcap%
\pgfsetroundjoin%
\pgfsetlinewidth{1.505625pt}%
\definecolor{currentstroke}{rgb}{0.631373,0.788235,0.956863}%
\pgfsetstrokecolor{currentstroke}%
\pgfsetstrokeopacity{0.800000}%
\pgfsetdash{}{0pt}%
\pgfpathmoveto{\pgfqpoint{6.915293in}{3.968555in}}%
\pgfpathlineto{\pgfqpoint{5.711094in}{3.773662in}}%
\pgfusepath{stroke}%
\end{pgfscope}%
\begin{pgfscope}%
\pgfpathrectangle{\pgfqpoint{0.481978in}{0.331635in}}{\pgfqpoint{9.300000in}{7.700000in}}%
\pgfusepath{clip}%
\pgfsetrectcap%
\pgfsetroundjoin%
\pgfsetlinewidth{1.505625pt}%
\definecolor{currentstroke}{rgb}{0.631373,0.788235,0.956863}%
\pgfsetstrokecolor{currentstroke}%
\pgfsetstrokeopacity{0.800000}%
\pgfsetdash{}{0pt}%
\pgfpathmoveto{\pgfqpoint{5.971393in}{2.386439in}}%
\pgfpathlineto{\pgfqpoint{5.711094in}{3.773662in}}%
\pgfusepath{stroke}%
\end{pgfscope}%
\begin{pgfscope}%
\pgfpathrectangle{\pgfqpoint{0.481978in}{0.331635in}}{\pgfqpoint{9.300000in}{7.700000in}}%
\pgfusepath{clip}%
\pgfsetrectcap%
\pgfsetroundjoin%
\pgfsetlinewidth{1.505625pt}%
\definecolor{currentstroke}{rgb}{0.631373,0.788235,0.956863}%
\pgfsetstrokecolor{currentstroke}%
\pgfsetstrokeopacity{0.800000}%
\pgfsetdash{}{0pt}%
\pgfpathmoveto{\pgfqpoint{2.928552in}{6.619267in}}%
\pgfpathlineto{\pgfqpoint{5.711094in}{3.773662in}}%
\pgfusepath{stroke}%
\end{pgfscope}%
\begin{pgfscope}%
\pgfpathrectangle{\pgfqpoint{0.481978in}{0.331635in}}{\pgfqpoint{9.300000in}{7.700000in}}%
\pgfusepath{clip}%
\pgfsetrectcap%
\pgfsetroundjoin%
\pgfsetlinewidth{1.505625pt}%
\definecolor{currentstroke}{rgb}{0.631373,0.788235,0.956863}%
\pgfsetstrokecolor{currentstroke}%
\pgfsetstrokeopacity{0.800000}%
\pgfsetdash{}{0pt}%
\pgfpathmoveto{\pgfqpoint{5.743990in}{2.614242in}}%
\pgfpathlineto{\pgfqpoint{5.711094in}{3.773662in}}%
\pgfusepath{stroke}%
\end{pgfscope}%
\begin{pgfscope}%
\pgfpathrectangle{\pgfqpoint{0.481978in}{0.331635in}}{\pgfqpoint{9.300000in}{7.700000in}}%
\pgfusepath{clip}%
\pgfsetrectcap%
\pgfsetroundjoin%
\pgfsetlinewidth{1.505625pt}%
\definecolor{currentstroke}{rgb}{0.631373,0.788235,0.956863}%
\pgfsetstrokecolor{currentstroke}%
\pgfsetstrokeopacity{0.800000}%
\pgfsetdash{}{0pt}%
\pgfpathmoveto{\pgfqpoint{8.326824in}{4.923794in}}%
\pgfpathlineto{\pgfqpoint{5.711094in}{3.773662in}}%
\pgfusepath{stroke}%
\end{pgfscope}%
\begin{pgfscope}%
\pgfpathrectangle{\pgfqpoint{0.481978in}{0.331635in}}{\pgfqpoint{9.300000in}{7.700000in}}%
\pgfusepath{clip}%
\pgfsetrectcap%
\pgfsetroundjoin%
\pgfsetlinewidth{1.505625pt}%
\definecolor{currentstroke}{rgb}{0.631373,0.788235,0.956863}%
\pgfsetstrokecolor{currentstroke}%
\pgfsetstrokeopacity{0.800000}%
\pgfsetdash{}{0pt}%
\pgfpathmoveto{\pgfqpoint{4.333517in}{5.783751in}}%
\pgfpathlineto{\pgfqpoint{5.711094in}{3.773662in}}%
\pgfusepath{stroke}%
\end{pgfscope}%
\begin{pgfscope}%
\pgfpathrectangle{\pgfqpoint{0.481978in}{0.331635in}}{\pgfqpoint{9.300000in}{7.700000in}}%
\pgfusepath{clip}%
\pgfsetrectcap%
\pgfsetroundjoin%
\pgfsetlinewidth{1.505625pt}%
\definecolor{currentstroke}{rgb}{0.631373,0.788235,0.956863}%
\pgfsetstrokecolor{currentstroke}%
\pgfsetstrokeopacity{0.800000}%
\pgfsetdash{}{0pt}%
\pgfpathmoveto{\pgfqpoint{5.146855in}{2.060560in}}%
\pgfpathlineto{\pgfqpoint{5.711094in}{3.773662in}}%
\pgfusepath{stroke}%
\end{pgfscope}%
\begin{pgfscope}%
\pgfpathrectangle{\pgfqpoint{0.481978in}{0.331635in}}{\pgfqpoint{9.300000in}{7.700000in}}%
\pgfusepath{clip}%
\pgfsetrectcap%
\pgfsetroundjoin%
\pgfsetlinewidth{1.505625pt}%
\definecolor{currentstroke}{rgb}{0.631373,0.788235,0.956863}%
\pgfsetstrokecolor{currentstroke}%
\pgfsetstrokeopacity{0.800000}%
\pgfsetdash{}{0pt}%
\pgfpathmoveto{\pgfqpoint{6.726675in}{5.410343in}}%
\pgfpathlineto{\pgfqpoint{5.711094in}{3.773662in}}%
\pgfusepath{stroke}%
\end{pgfscope}%
\begin{pgfscope}%
\pgfpathrectangle{\pgfqpoint{0.481978in}{0.331635in}}{\pgfqpoint{9.300000in}{7.700000in}}%
\pgfusepath{clip}%
\pgfsetrectcap%
\pgfsetroundjoin%
\pgfsetlinewidth{1.505625pt}%
\definecolor{currentstroke}{rgb}{0.631373,0.788235,0.956863}%
\pgfsetstrokecolor{currentstroke}%
\pgfsetstrokeopacity{0.800000}%
\pgfsetdash{}{0pt}%
\pgfpathmoveto{\pgfqpoint{5.444209in}{2.575038in}}%
\pgfpathlineto{\pgfqpoint{5.711094in}{3.773662in}}%
\pgfusepath{stroke}%
\end{pgfscope}%
\begin{pgfscope}%
\pgfpathrectangle{\pgfqpoint{0.481978in}{0.331635in}}{\pgfqpoint{9.300000in}{7.700000in}}%
\pgfusepath{clip}%
\pgfsetrectcap%
\pgfsetroundjoin%
\pgfsetlinewidth{1.505625pt}%
\definecolor{currentstroke}{rgb}{0.631373,0.788235,0.956863}%
\pgfsetstrokecolor{currentstroke}%
\pgfsetstrokeopacity{0.800000}%
\pgfsetdash{}{0pt}%
\pgfpathmoveto{\pgfqpoint{6.317681in}{2.783974in}}%
\pgfpathlineto{\pgfqpoint{5.711094in}{3.773662in}}%
\pgfusepath{stroke}%
\end{pgfscope}%
\begin{pgfscope}%
\pgfpathrectangle{\pgfqpoint{0.481978in}{0.331635in}}{\pgfqpoint{9.300000in}{7.700000in}}%
\pgfusepath{clip}%
\pgfsetrectcap%
\pgfsetroundjoin%
\pgfsetlinewidth{1.505625pt}%
\definecolor{currentstroke}{rgb}{0.631373,0.788235,0.956863}%
\pgfsetstrokecolor{currentstroke}%
\pgfsetstrokeopacity{0.800000}%
\pgfsetdash{}{0pt}%
\pgfpathmoveto{\pgfqpoint{5.001156in}{3.894549in}}%
\pgfpathlineto{\pgfqpoint{5.711094in}{3.773662in}}%
\pgfusepath{stroke}%
\end{pgfscope}%
\begin{pgfscope}%
\pgfpathrectangle{\pgfqpoint{0.481978in}{0.331635in}}{\pgfqpoint{9.300000in}{7.700000in}}%
\pgfusepath{clip}%
\pgfsetrectcap%
\pgfsetroundjoin%
\pgfsetlinewidth{1.505625pt}%
\definecolor{currentstroke}{rgb}{0.631373,0.788235,0.956863}%
\pgfsetstrokecolor{currentstroke}%
\pgfsetstrokeopacity{0.800000}%
\pgfsetdash{}{0pt}%
\pgfpathmoveto{\pgfqpoint{5.223174in}{3.276248in}}%
\pgfpathlineto{\pgfqpoint{5.711094in}{3.773662in}}%
\pgfusepath{stroke}%
\end{pgfscope}%
\begin{pgfscope}%
\pgfpathrectangle{\pgfqpoint{0.481978in}{0.331635in}}{\pgfqpoint{9.300000in}{7.700000in}}%
\pgfusepath{clip}%
\pgfsetrectcap%
\pgfsetroundjoin%
\pgfsetlinewidth{1.505625pt}%
\definecolor{currentstroke}{rgb}{0.631373,0.788235,0.956863}%
\pgfsetstrokecolor{currentstroke}%
\pgfsetstrokeopacity{0.800000}%
\pgfsetdash{}{0pt}%
\pgfpathmoveto{\pgfqpoint{5.990771in}{2.551896in}}%
\pgfpathlineto{\pgfqpoint{5.711094in}{3.773662in}}%
\pgfusepath{stroke}%
\end{pgfscope}%
\begin{pgfscope}%
\pgfpathrectangle{\pgfqpoint{0.481978in}{0.331635in}}{\pgfqpoint{9.300000in}{7.700000in}}%
\pgfusepath{clip}%
\pgfsetrectcap%
\pgfsetroundjoin%
\pgfsetlinewidth{1.505625pt}%
\definecolor{currentstroke}{rgb}{0.631373,0.788235,0.956863}%
\pgfsetstrokecolor{currentstroke}%
\pgfsetstrokeopacity{0.800000}%
\pgfsetdash{}{0pt}%
\pgfpathmoveto{\pgfqpoint{5.037316in}{4.329956in}}%
\pgfpathlineto{\pgfqpoint{5.711094in}{3.773662in}}%
\pgfusepath{stroke}%
\end{pgfscope}%
\begin{pgfscope}%
\pgfpathrectangle{\pgfqpoint{0.481978in}{0.331635in}}{\pgfqpoint{9.300000in}{7.700000in}}%
\pgfusepath{clip}%
\pgfsetrectcap%
\pgfsetroundjoin%
\pgfsetlinewidth{1.505625pt}%
\definecolor{currentstroke}{rgb}{0.631373,0.788235,0.956863}%
\pgfsetstrokecolor{currentstroke}%
\pgfsetstrokeopacity{0.800000}%
\pgfsetdash{}{0pt}%
\pgfpathmoveto{\pgfqpoint{4.961622in}{0.885246in}}%
\pgfpathlineto{\pgfqpoint{5.711094in}{3.773662in}}%
\pgfusepath{stroke}%
\end{pgfscope}%
\begin{pgfscope}%
\pgfpathrectangle{\pgfqpoint{0.481978in}{0.331635in}}{\pgfqpoint{9.300000in}{7.700000in}}%
\pgfusepath{clip}%
\pgfsetrectcap%
\pgfsetroundjoin%
\pgfsetlinewidth{1.505625pt}%
\definecolor{currentstroke}{rgb}{0.631373,0.788235,0.956863}%
\pgfsetstrokecolor{currentstroke}%
\pgfsetstrokeopacity{0.800000}%
\pgfsetdash{}{0pt}%
\pgfpathmoveto{\pgfqpoint{5.509130in}{2.091290in}}%
\pgfpathlineto{\pgfqpoint{5.711094in}{3.773662in}}%
\pgfusepath{stroke}%
\end{pgfscope}%
\begin{pgfscope}%
\pgfpathrectangle{\pgfqpoint{0.481978in}{0.331635in}}{\pgfqpoint{9.300000in}{7.700000in}}%
\pgfusepath{clip}%
\pgfsetrectcap%
\pgfsetroundjoin%
\pgfsetlinewidth{1.505625pt}%
\definecolor{currentstroke}{rgb}{0.631373,0.788235,0.956863}%
\pgfsetstrokecolor{currentstroke}%
\pgfsetstrokeopacity{0.800000}%
\pgfsetdash{}{0pt}%
\pgfpathmoveto{\pgfqpoint{2.965360in}{6.422179in}}%
\pgfpathlineto{\pgfqpoint{5.711094in}{3.773662in}}%
\pgfusepath{stroke}%
\end{pgfscope}%
\begin{pgfscope}%
\pgfpathrectangle{\pgfqpoint{0.481978in}{0.331635in}}{\pgfqpoint{9.300000in}{7.700000in}}%
\pgfusepath{clip}%
\pgfsetrectcap%
\pgfsetroundjoin%
\pgfsetlinewidth{1.505625pt}%
\definecolor{currentstroke}{rgb}{0.631373,0.788235,0.956863}%
\pgfsetstrokecolor{currentstroke}%
\pgfsetstrokeopacity{0.800000}%
\pgfsetdash{}{0pt}%
\pgfpathmoveto{\pgfqpoint{6.394060in}{2.566866in}}%
\pgfpathlineto{\pgfqpoint{5.711094in}{3.773662in}}%
\pgfusepath{stroke}%
\end{pgfscope}%
\begin{pgfscope}%
\pgfpathrectangle{\pgfqpoint{0.481978in}{0.331635in}}{\pgfqpoint{9.300000in}{7.700000in}}%
\pgfusepath{clip}%
\pgfsetrectcap%
\pgfsetroundjoin%
\pgfsetlinewidth{1.505625pt}%
\definecolor{currentstroke}{rgb}{0.631373,0.788235,0.956863}%
\pgfsetstrokecolor{currentstroke}%
\pgfsetstrokeopacity{0.800000}%
\pgfsetdash{}{0pt}%
\pgfpathmoveto{\pgfqpoint{2.186468in}{6.813277in}}%
\pgfpathlineto{\pgfqpoint{5.711094in}{3.773662in}}%
\pgfusepath{stroke}%
\end{pgfscope}%
\begin{pgfscope}%
\pgfpathrectangle{\pgfqpoint{0.481978in}{0.331635in}}{\pgfqpoint{9.300000in}{7.700000in}}%
\pgfusepath{clip}%
\pgfsetrectcap%
\pgfsetroundjoin%
\pgfsetlinewidth{1.505625pt}%
\definecolor{currentstroke}{rgb}{0.631373,0.788235,0.956863}%
\pgfsetstrokecolor{currentstroke}%
\pgfsetstrokeopacity{0.800000}%
\pgfsetdash{}{0pt}%
\pgfpathmoveto{\pgfqpoint{7.121234in}{2.917579in}}%
\pgfpathlineto{\pgfqpoint{5.711094in}{3.773662in}}%
\pgfusepath{stroke}%
\end{pgfscope}%
\begin{pgfscope}%
\pgfpathrectangle{\pgfqpoint{0.481978in}{0.331635in}}{\pgfqpoint{9.300000in}{7.700000in}}%
\pgfusepath{clip}%
\pgfsetrectcap%
\pgfsetroundjoin%
\pgfsetlinewidth{1.505625pt}%
\definecolor{currentstroke}{rgb}{0.631373,0.788235,0.956863}%
\pgfsetstrokecolor{currentstroke}%
\pgfsetstrokeopacity{0.800000}%
\pgfsetdash{}{0pt}%
\pgfpathmoveto{\pgfqpoint{8.167603in}{5.004497in}}%
\pgfpathlineto{\pgfqpoint{5.711094in}{3.773662in}}%
\pgfusepath{stroke}%
\end{pgfscope}%
\begin{pgfscope}%
\pgfpathrectangle{\pgfqpoint{0.481978in}{0.331635in}}{\pgfqpoint{9.300000in}{7.700000in}}%
\pgfusepath{clip}%
\pgfsetrectcap%
\pgfsetroundjoin%
\pgfsetlinewidth{1.505625pt}%
\definecolor{currentstroke}{rgb}{0.631373,0.788235,0.956863}%
\pgfsetstrokecolor{currentstroke}%
\pgfsetstrokeopacity{0.800000}%
\pgfsetdash{}{0pt}%
\pgfpathmoveto{\pgfqpoint{6.293261in}{4.446775in}}%
\pgfpathlineto{\pgfqpoint{5.711094in}{3.773662in}}%
\pgfusepath{stroke}%
\end{pgfscope}%
\begin{pgfscope}%
\pgfpathrectangle{\pgfqpoint{0.481978in}{0.331635in}}{\pgfqpoint{9.300000in}{7.700000in}}%
\pgfusepath{clip}%
\pgfsetrectcap%
\pgfsetroundjoin%
\pgfsetlinewidth{1.505625pt}%
\definecolor{currentstroke}{rgb}{0.631373,0.788235,0.956863}%
\pgfsetstrokecolor{currentstroke}%
\pgfsetstrokeopacity{0.800000}%
\pgfsetdash{}{0pt}%
\pgfpathmoveto{\pgfqpoint{4.855717in}{7.204103in}}%
\pgfpathlineto{\pgfqpoint{5.711094in}{3.773662in}}%
\pgfusepath{stroke}%
\end{pgfscope}%
\begin{pgfscope}%
\pgfpathrectangle{\pgfqpoint{0.481978in}{0.331635in}}{\pgfqpoint{9.300000in}{7.700000in}}%
\pgfusepath{clip}%
\pgfsetrectcap%
\pgfsetroundjoin%
\pgfsetlinewidth{1.505625pt}%
\definecolor{currentstroke}{rgb}{0.631373,0.788235,0.956863}%
\pgfsetstrokecolor{currentstroke}%
\pgfsetstrokeopacity{0.800000}%
\pgfsetdash{}{0pt}%
\pgfpathmoveto{\pgfqpoint{8.228148in}{4.830010in}}%
\pgfpathlineto{\pgfqpoint{5.711094in}{3.773662in}}%
\pgfusepath{stroke}%
\end{pgfscope}%
\begin{pgfscope}%
\pgfpathrectangle{\pgfqpoint{0.481978in}{0.331635in}}{\pgfqpoint{9.300000in}{7.700000in}}%
\pgfusepath{clip}%
\pgfsetrectcap%
\pgfsetroundjoin%
\pgfsetlinewidth{1.505625pt}%
\definecolor{currentstroke}{rgb}{0.631373,0.788235,0.956863}%
\pgfsetstrokecolor{currentstroke}%
\pgfsetstrokeopacity{0.800000}%
\pgfsetdash{}{0pt}%
\pgfpathmoveto{\pgfqpoint{7.375231in}{1.952235in}}%
\pgfpathlineto{\pgfqpoint{5.711094in}{3.773662in}}%
\pgfusepath{stroke}%
\end{pgfscope}%
\begin{pgfscope}%
\pgfpathrectangle{\pgfqpoint{0.481978in}{0.331635in}}{\pgfqpoint{9.300000in}{7.700000in}}%
\pgfusepath{clip}%
\pgfsetrectcap%
\pgfsetroundjoin%
\pgfsetlinewidth{1.505625pt}%
\definecolor{currentstroke}{rgb}{0.631373,0.788235,0.956863}%
\pgfsetstrokecolor{currentstroke}%
\pgfsetstrokeopacity{0.800000}%
\pgfsetdash{}{0pt}%
\pgfpathmoveto{\pgfqpoint{6.949155in}{1.649974in}}%
\pgfpathlineto{\pgfqpoint{5.711094in}{3.773662in}}%
\pgfusepath{stroke}%
\end{pgfscope}%
\begin{pgfscope}%
\pgfpathrectangle{\pgfqpoint{0.481978in}{0.331635in}}{\pgfqpoint{9.300000in}{7.700000in}}%
\pgfusepath{clip}%
\pgfsetrectcap%
\pgfsetroundjoin%
\pgfsetlinewidth{1.505625pt}%
\definecolor{currentstroke}{rgb}{0.631373,0.788235,0.956863}%
\pgfsetstrokecolor{currentstroke}%
\pgfsetstrokeopacity{0.800000}%
\pgfsetdash{}{0pt}%
\pgfpathmoveto{\pgfqpoint{7.251466in}{2.520062in}}%
\pgfpathlineto{\pgfqpoint{5.711094in}{3.773662in}}%
\pgfusepath{stroke}%
\end{pgfscope}%
\begin{pgfscope}%
\pgfpathrectangle{\pgfqpoint{0.481978in}{0.331635in}}{\pgfqpoint{9.300000in}{7.700000in}}%
\pgfusepath{clip}%
\pgfsetrectcap%
\pgfsetroundjoin%
\pgfsetlinewidth{1.505625pt}%
\definecolor{currentstroke}{rgb}{0.631373,0.788235,0.956863}%
\pgfsetstrokecolor{currentstroke}%
\pgfsetstrokeopacity{0.800000}%
\pgfsetdash{}{0pt}%
\pgfpathmoveto{\pgfqpoint{5.500453in}{2.173795in}}%
\pgfpathlineto{\pgfqpoint{5.711094in}{3.773662in}}%
\pgfusepath{stroke}%
\end{pgfscope}%
\begin{pgfscope}%
\pgfpathrectangle{\pgfqpoint{0.481978in}{0.331635in}}{\pgfqpoint{9.300000in}{7.700000in}}%
\pgfusepath{clip}%
\pgfsetrectcap%
\pgfsetroundjoin%
\pgfsetlinewidth{1.505625pt}%
\definecolor{currentstroke}{rgb}{0.631373,0.788235,0.956863}%
\pgfsetstrokecolor{currentstroke}%
\pgfsetstrokeopacity{0.800000}%
\pgfsetdash{}{0pt}%
\pgfpathmoveto{\pgfqpoint{6.149200in}{2.319484in}}%
\pgfpathlineto{\pgfqpoint{5.711094in}{3.773662in}}%
\pgfusepath{stroke}%
\end{pgfscope}%
\begin{pgfscope}%
\pgfpathrectangle{\pgfqpoint{0.481978in}{0.331635in}}{\pgfqpoint{9.300000in}{7.700000in}}%
\pgfusepath{clip}%
\pgfsetrectcap%
\pgfsetroundjoin%
\pgfsetlinewidth{1.505625pt}%
\definecolor{currentstroke}{rgb}{0.631373,0.788235,0.956863}%
\pgfsetstrokecolor{currentstroke}%
\pgfsetstrokeopacity{0.800000}%
\pgfsetdash{}{0pt}%
\pgfpathmoveto{\pgfqpoint{2.951748in}{1.154804in}}%
\pgfpathlineto{\pgfqpoint{5.711094in}{3.773662in}}%
\pgfusepath{stroke}%
\end{pgfscope}%
\begin{pgfscope}%
\pgfpathrectangle{\pgfqpoint{0.481978in}{0.331635in}}{\pgfqpoint{9.300000in}{7.700000in}}%
\pgfusepath{clip}%
\pgfsetrectcap%
\pgfsetroundjoin%
\pgfsetlinewidth{1.505625pt}%
\definecolor{currentstroke}{rgb}{0.631373,0.788235,0.956863}%
\pgfsetstrokecolor{currentstroke}%
\pgfsetstrokeopacity{0.800000}%
\pgfsetdash{}{0pt}%
\pgfpathmoveto{\pgfqpoint{6.416375in}{3.235325in}}%
\pgfpathlineto{\pgfqpoint{5.711094in}{3.773662in}}%
\pgfusepath{stroke}%
\end{pgfscope}%
\begin{pgfscope}%
\pgfpathrectangle{\pgfqpoint{0.481978in}{0.331635in}}{\pgfqpoint{9.300000in}{7.700000in}}%
\pgfusepath{clip}%
\pgfsetrectcap%
\pgfsetroundjoin%
\pgfsetlinewidth{1.505625pt}%
\definecolor{currentstroke}{rgb}{0.631373,0.788235,0.956863}%
\pgfsetstrokecolor{currentstroke}%
\pgfsetstrokeopacity{0.800000}%
\pgfsetdash{}{0pt}%
\pgfpathmoveto{\pgfqpoint{2.918703in}{1.265777in}}%
\pgfpathlineto{\pgfqpoint{5.711094in}{3.773662in}}%
\pgfusepath{stroke}%
\end{pgfscope}%
\begin{pgfscope}%
\pgfpathrectangle{\pgfqpoint{0.481978in}{0.331635in}}{\pgfqpoint{9.300000in}{7.700000in}}%
\pgfusepath{clip}%
\pgfsetrectcap%
\pgfsetroundjoin%
\pgfsetlinewidth{1.505625pt}%
\definecolor{currentstroke}{rgb}{0.631373,0.788235,0.956863}%
\pgfsetstrokecolor{currentstroke}%
\pgfsetstrokeopacity{0.800000}%
\pgfsetdash{}{0pt}%
\pgfpathmoveto{\pgfqpoint{2.176114in}{5.871183in}}%
\pgfpathlineto{\pgfqpoint{5.711094in}{3.773662in}}%
\pgfusepath{stroke}%
\end{pgfscope}%
\begin{pgfscope}%
\pgfpathrectangle{\pgfqpoint{0.481978in}{0.331635in}}{\pgfqpoint{9.300000in}{7.700000in}}%
\pgfusepath{clip}%
\pgfsetrectcap%
\pgfsetroundjoin%
\pgfsetlinewidth{1.505625pt}%
\definecolor{currentstroke}{rgb}{0.631373,0.788235,0.956863}%
\pgfsetstrokecolor{currentstroke}%
\pgfsetstrokeopacity{0.800000}%
\pgfsetdash{}{0pt}%
\pgfpathmoveto{\pgfqpoint{6.830655in}{1.576246in}}%
\pgfpathlineto{\pgfqpoint{5.711094in}{3.773662in}}%
\pgfusepath{stroke}%
\end{pgfscope}%
\begin{pgfscope}%
\pgfpathrectangle{\pgfqpoint{0.481978in}{0.331635in}}{\pgfqpoint{9.300000in}{7.700000in}}%
\pgfusepath{clip}%
\pgfsetrectcap%
\pgfsetroundjoin%
\pgfsetlinewidth{1.505625pt}%
\definecolor{currentstroke}{rgb}{0.631373,0.788235,0.956863}%
\pgfsetstrokecolor{currentstroke}%
\pgfsetstrokeopacity{0.800000}%
\pgfsetdash{}{0pt}%
\pgfpathmoveto{\pgfqpoint{6.368029in}{1.379909in}}%
\pgfpathlineto{\pgfqpoint{5.711094in}{3.773662in}}%
\pgfusepath{stroke}%
\end{pgfscope}%
\begin{pgfscope}%
\pgfpathrectangle{\pgfqpoint{0.481978in}{0.331635in}}{\pgfqpoint{9.300000in}{7.700000in}}%
\pgfusepath{clip}%
\pgfsetrectcap%
\pgfsetroundjoin%
\pgfsetlinewidth{1.505625pt}%
\definecolor{currentstroke}{rgb}{0.631373,0.788235,0.956863}%
\pgfsetstrokecolor{currentstroke}%
\pgfsetstrokeopacity{0.800000}%
\pgfsetdash{}{0pt}%
\pgfpathmoveto{\pgfqpoint{7.095890in}{2.699343in}}%
\pgfpathlineto{\pgfqpoint{5.711094in}{3.773662in}}%
\pgfusepath{stroke}%
\end{pgfscope}%
\begin{pgfscope}%
\pgfpathrectangle{\pgfqpoint{0.481978in}{0.331635in}}{\pgfqpoint{9.300000in}{7.700000in}}%
\pgfusepath{clip}%
\pgfsetrectcap%
\pgfsetroundjoin%
\pgfsetlinewidth{1.505625pt}%
\definecolor{currentstroke}{rgb}{0.631373,0.788235,0.956863}%
\pgfsetstrokecolor{currentstroke}%
\pgfsetstrokeopacity{0.800000}%
\pgfsetdash{}{0pt}%
\pgfpathmoveto{\pgfqpoint{5.900690in}{3.673565in}}%
\pgfpathlineto{\pgfqpoint{5.711094in}{3.773662in}}%
\pgfusepath{stroke}%
\end{pgfscope}%
\begin{pgfscope}%
\pgfpathrectangle{\pgfqpoint{0.481978in}{0.331635in}}{\pgfqpoint{9.300000in}{7.700000in}}%
\pgfusepath{clip}%
\pgfsetrectcap%
\pgfsetroundjoin%
\pgfsetlinewidth{1.505625pt}%
\definecolor{currentstroke}{rgb}{0.631373,0.788235,0.956863}%
\pgfsetstrokecolor{currentstroke}%
\pgfsetstrokeopacity{0.800000}%
\pgfsetdash{}{0pt}%
\pgfpathmoveto{\pgfqpoint{6.724845in}{4.610887in}}%
\pgfpathlineto{\pgfqpoint{5.711094in}{3.773662in}}%
\pgfusepath{stroke}%
\end{pgfscope}%
\begin{pgfscope}%
\pgfpathrectangle{\pgfqpoint{0.481978in}{0.331635in}}{\pgfqpoint{9.300000in}{7.700000in}}%
\pgfusepath{clip}%
\pgfsetrectcap%
\pgfsetroundjoin%
\pgfsetlinewidth{1.505625pt}%
\definecolor{currentstroke}{rgb}{0.631373,0.788235,0.956863}%
\pgfsetstrokecolor{currentstroke}%
\pgfsetstrokeopacity{0.800000}%
\pgfsetdash{}{0pt}%
\pgfpathmoveto{\pgfqpoint{8.573001in}{5.152498in}}%
\pgfpathlineto{\pgfqpoint{5.711094in}{3.773662in}}%
\pgfusepath{stroke}%
\end{pgfscope}%
\begin{pgfscope}%
\pgfpathrectangle{\pgfqpoint{0.481978in}{0.331635in}}{\pgfqpoint{9.300000in}{7.700000in}}%
\pgfusepath{clip}%
\pgfsetrectcap%
\pgfsetroundjoin%
\pgfsetlinewidth{1.505625pt}%
\definecolor{currentstroke}{rgb}{0.631373,0.788235,0.956863}%
\pgfsetstrokecolor{currentstroke}%
\pgfsetstrokeopacity{0.800000}%
\pgfsetdash{}{0pt}%
\pgfpathmoveto{\pgfqpoint{5.469587in}{7.033042in}}%
\pgfpathlineto{\pgfqpoint{5.711094in}{3.773662in}}%
\pgfusepath{stroke}%
\end{pgfscope}%
\begin{pgfscope}%
\pgfpathrectangle{\pgfqpoint{0.481978in}{0.331635in}}{\pgfqpoint{9.300000in}{7.700000in}}%
\pgfusepath{clip}%
\pgfsetrectcap%
\pgfsetroundjoin%
\pgfsetlinewidth{1.505625pt}%
\definecolor{currentstroke}{rgb}{0.631373,0.788235,0.956863}%
\pgfsetstrokecolor{currentstroke}%
\pgfsetstrokeopacity{0.800000}%
\pgfsetdash{}{0pt}%
\pgfpathmoveto{\pgfqpoint{4.990883in}{4.358746in}}%
\pgfpathlineto{\pgfqpoint{5.711094in}{3.773662in}}%
\pgfusepath{stroke}%
\end{pgfscope}%
\begin{pgfscope}%
\pgfpathrectangle{\pgfqpoint{0.481978in}{0.331635in}}{\pgfqpoint{9.300000in}{7.700000in}}%
\pgfusepath{clip}%
\pgfsetrectcap%
\pgfsetroundjoin%
\pgfsetlinewidth{1.505625pt}%
\definecolor{currentstroke}{rgb}{0.631373,0.788235,0.956863}%
\pgfsetstrokecolor{currentstroke}%
\pgfsetstrokeopacity{0.800000}%
\pgfsetdash{}{0pt}%
\pgfpathmoveto{\pgfqpoint{6.589980in}{1.687564in}}%
\pgfpathlineto{\pgfqpoint{5.711094in}{3.773662in}}%
\pgfusepath{stroke}%
\end{pgfscope}%
\begin{pgfscope}%
\pgfpathrectangle{\pgfqpoint{0.481978in}{0.331635in}}{\pgfqpoint{9.300000in}{7.700000in}}%
\pgfusepath{clip}%
\pgfsetrectcap%
\pgfsetroundjoin%
\pgfsetlinewidth{1.505625pt}%
\definecolor{currentstroke}{rgb}{0.631373,0.788235,0.956863}%
\pgfsetstrokecolor{currentstroke}%
\pgfsetstrokeopacity{0.800000}%
\pgfsetdash{}{0pt}%
\pgfpathmoveto{\pgfqpoint{5.765884in}{2.327113in}}%
\pgfpathlineto{\pgfqpoint{5.711094in}{3.773662in}}%
\pgfusepath{stroke}%
\end{pgfscope}%
\begin{pgfscope}%
\pgfpathrectangle{\pgfqpoint{0.481978in}{0.331635in}}{\pgfqpoint{9.300000in}{7.700000in}}%
\pgfusepath{clip}%
\pgfsetrectcap%
\pgfsetroundjoin%
\pgfsetlinewidth{1.505625pt}%
\definecolor{currentstroke}{rgb}{0.631373,0.788235,0.956863}%
\pgfsetstrokecolor{currentstroke}%
\pgfsetstrokeopacity{0.800000}%
\pgfsetdash{}{0pt}%
\pgfpathmoveto{\pgfqpoint{4.097733in}{6.308336in}}%
\pgfpathlineto{\pgfqpoint{5.711094in}{3.773662in}}%
\pgfusepath{stroke}%
\end{pgfscope}%
\begin{pgfscope}%
\pgfpathrectangle{\pgfqpoint{0.481978in}{0.331635in}}{\pgfqpoint{9.300000in}{7.700000in}}%
\pgfusepath{clip}%
\pgfsetrectcap%
\pgfsetroundjoin%
\pgfsetlinewidth{1.505625pt}%
\definecolor{currentstroke}{rgb}{0.631373,0.788235,0.956863}%
\pgfsetstrokecolor{currentstroke}%
\pgfsetstrokeopacity{0.800000}%
\pgfsetdash{}{0pt}%
\pgfpathmoveto{\pgfqpoint{7.842322in}{5.306029in}}%
\pgfpathlineto{\pgfqpoint{5.711094in}{3.773662in}}%
\pgfusepath{stroke}%
\end{pgfscope}%
\begin{pgfscope}%
\pgfpathrectangle{\pgfqpoint{0.481978in}{0.331635in}}{\pgfqpoint{9.300000in}{7.700000in}}%
\pgfusepath{clip}%
\pgfsetrectcap%
\pgfsetroundjoin%
\pgfsetlinewidth{1.505625pt}%
\definecolor{currentstroke}{rgb}{0.631373,0.788235,0.956863}%
\pgfsetstrokecolor{currentstroke}%
\pgfsetstrokeopacity{0.800000}%
\pgfsetdash{}{0pt}%
\pgfpathmoveto{\pgfqpoint{3.960820in}{4.623230in}}%
\pgfpathlineto{\pgfqpoint{5.711094in}{3.773662in}}%
\pgfusepath{stroke}%
\end{pgfscope}%
\begin{pgfscope}%
\pgfpathrectangle{\pgfqpoint{0.481978in}{0.331635in}}{\pgfqpoint{9.300000in}{7.700000in}}%
\pgfusepath{clip}%
\pgfsetrectcap%
\pgfsetroundjoin%
\pgfsetlinewidth{1.505625pt}%
\definecolor{currentstroke}{rgb}{0.631373,0.788235,0.956863}%
\pgfsetstrokecolor{currentstroke}%
\pgfsetstrokeopacity{0.800000}%
\pgfsetdash{}{0pt}%
\pgfpathmoveto{\pgfqpoint{3.426685in}{4.665956in}}%
\pgfpathlineto{\pgfqpoint{5.711094in}{3.773662in}}%
\pgfusepath{stroke}%
\end{pgfscope}%
\begin{pgfscope}%
\pgfpathrectangle{\pgfqpoint{0.481978in}{0.331635in}}{\pgfqpoint{9.300000in}{7.700000in}}%
\pgfusepath{clip}%
\pgfsetrectcap%
\pgfsetroundjoin%
\pgfsetlinewidth{1.505625pt}%
\definecolor{currentstroke}{rgb}{0.631373,0.788235,0.956863}%
\pgfsetstrokecolor{currentstroke}%
\pgfsetstrokeopacity{0.800000}%
\pgfsetdash{}{0pt}%
\pgfpathmoveto{\pgfqpoint{6.445366in}{2.091249in}}%
\pgfpathlineto{\pgfqpoint{5.711094in}{3.773662in}}%
\pgfusepath{stroke}%
\end{pgfscope}%
\begin{pgfscope}%
\pgfpathrectangle{\pgfqpoint{0.481978in}{0.331635in}}{\pgfqpoint{9.300000in}{7.700000in}}%
\pgfusepath{clip}%
\pgfsetrectcap%
\pgfsetroundjoin%
\pgfsetlinewidth{1.505625pt}%
\definecolor{currentstroke}{rgb}{0.631373,0.788235,0.956863}%
\pgfsetstrokecolor{currentstroke}%
\pgfsetstrokeopacity{0.800000}%
\pgfsetdash{}{0pt}%
\pgfpathmoveto{\pgfqpoint{8.216289in}{4.738244in}}%
\pgfpathlineto{\pgfqpoint{5.711094in}{3.773662in}}%
\pgfusepath{stroke}%
\end{pgfscope}%
\begin{pgfscope}%
\pgfpathrectangle{\pgfqpoint{0.481978in}{0.331635in}}{\pgfqpoint{9.300000in}{7.700000in}}%
\pgfusepath{clip}%
\pgfsetrectcap%
\pgfsetroundjoin%
\pgfsetlinewidth{1.505625pt}%
\definecolor{currentstroke}{rgb}{0.631373,0.788235,0.956863}%
\pgfsetstrokecolor{currentstroke}%
\pgfsetstrokeopacity{0.800000}%
\pgfsetdash{}{0pt}%
\pgfpathmoveto{\pgfqpoint{8.236668in}{4.419102in}}%
\pgfpathlineto{\pgfqpoint{5.711094in}{3.773662in}}%
\pgfusepath{stroke}%
\end{pgfscope}%
\begin{pgfscope}%
\pgfpathrectangle{\pgfqpoint{0.481978in}{0.331635in}}{\pgfqpoint{9.300000in}{7.700000in}}%
\pgfusepath{clip}%
\pgfsetrectcap%
\pgfsetroundjoin%
\pgfsetlinewidth{1.505625pt}%
\definecolor{currentstroke}{rgb}{0.631373,0.788235,0.956863}%
\pgfsetstrokecolor{currentstroke}%
\pgfsetstrokeopacity{0.800000}%
\pgfsetdash{}{0pt}%
\pgfpathmoveto{\pgfqpoint{7.008316in}{2.365537in}}%
\pgfpathlineto{\pgfqpoint{5.711094in}{3.773662in}}%
\pgfusepath{stroke}%
\end{pgfscope}%
\begin{pgfscope}%
\pgfpathrectangle{\pgfqpoint{0.481978in}{0.331635in}}{\pgfqpoint{9.300000in}{7.700000in}}%
\pgfusepath{clip}%
\pgfsetrectcap%
\pgfsetroundjoin%
\pgfsetlinewidth{1.505625pt}%
\definecolor{currentstroke}{rgb}{0.631373,0.788235,0.956863}%
\pgfsetstrokecolor{currentstroke}%
\pgfsetstrokeopacity{0.800000}%
\pgfsetdash{}{0pt}%
\pgfpathmoveto{\pgfqpoint{8.317641in}{5.164385in}}%
\pgfpathlineto{\pgfqpoint{5.711094in}{3.773662in}}%
\pgfusepath{stroke}%
\end{pgfscope}%
\begin{pgfscope}%
\pgfpathrectangle{\pgfqpoint{0.481978in}{0.331635in}}{\pgfqpoint{9.300000in}{7.700000in}}%
\pgfusepath{clip}%
\pgfsetrectcap%
\pgfsetroundjoin%
\pgfsetlinewidth{1.505625pt}%
\definecolor{currentstroke}{rgb}{0.631373,0.788235,0.956863}%
\pgfsetstrokecolor{currentstroke}%
\pgfsetstrokeopacity{0.800000}%
\pgfsetdash{}{0pt}%
\pgfpathmoveto{\pgfqpoint{7.082145in}{1.740727in}}%
\pgfpathlineto{\pgfqpoint{5.711094in}{3.773662in}}%
\pgfusepath{stroke}%
\end{pgfscope}%
\begin{pgfscope}%
\pgfpathrectangle{\pgfqpoint{0.481978in}{0.331635in}}{\pgfqpoint{9.300000in}{7.700000in}}%
\pgfusepath{clip}%
\pgfsetrectcap%
\pgfsetroundjoin%
\pgfsetlinewidth{1.505625pt}%
\definecolor{currentstroke}{rgb}{0.631373,0.788235,0.956863}%
\pgfsetstrokecolor{currentstroke}%
\pgfsetstrokeopacity{0.800000}%
\pgfsetdash{}{0pt}%
\pgfpathmoveto{\pgfqpoint{5.724298in}{1.804236in}}%
\pgfpathlineto{\pgfqpoint{5.711094in}{3.773662in}}%
\pgfusepath{stroke}%
\end{pgfscope}%
\begin{pgfscope}%
\pgfpathrectangle{\pgfqpoint{0.481978in}{0.331635in}}{\pgfqpoint{9.300000in}{7.700000in}}%
\pgfusepath{clip}%
\pgfsetrectcap%
\pgfsetroundjoin%
\pgfsetlinewidth{1.505625pt}%
\definecolor{currentstroke}{rgb}{0.631373,0.788235,0.956863}%
\pgfsetstrokecolor{currentstroke}%
\pgfsetstrokeopacity{0.800000}%
\pgfsetdash{}{0pt}%
\pgfpathmoveto{\pgfqpoint{2.561461in}{2.024440in}}%
\pgfpathlineto{\pgfqpoint{5.711094in}{3.773662in}}%
\pgfusepath{stroke}%
\end{pgfscope}%
\begin{pgfscope}%
\pgfpathrectangle{\pgfqpoint{0.481978in}{0.331635in}}{\pgfqpoint{9.300000in}{7.700000in}}%
\pgfusepath{clip}%
\pgfsetrectcap%
\pgfsetroundjoin%
\pgfsetlinewidth{1.505625pt}%
\definecolor{currentstroke}{rgb}{0.631373,0.788235,0.956863}%
\pgfsetstrokecolor{currentstroke}%
\pgfsetstrokeopacity{0.800000}%
\pgfsetdash{}{0pt}%
\pgfpathmoveto{\pgfqpoint{6.177807in}{1.877434in}}%
\pgfpathlineto{\pgfqpoint{5.711094in}{3.773662in}}%
\pgfusepath{stroke}%
\end{pgfscope}%
\begin{pgfscope}%
\pgfpathrectangle{\pgfqpoint{0.481978in}{0.331635in}}{\pgfqpoint{9.300000in}{7.700000in}}%
\pgfusepath{clip}%
\pgfsetrectcap%
\pgfsetroundjoin%
\pgfsetlinewidth{1.505625pt}%
\definecolor{currentstroke}{rgb}{0.631373,0.788235,0.956863}%
\pgfsetstrokecolor{currentstroke}%
\pgfsetstrokeopacity{0.800000}%
\pgfsetdash{}{0pt}%
\pgfpathmoveto{\pgfqpoint{6.336424in}{0.681635in}}%
\pgfpathlineto{\pgfqpoint{5.711094in}{3.773662in}}%
\pgfusepath{stroke}%
\end{pgfscope}%
\begin{pgfscope}%
\pgfpathrectangle{\pgfqpoint{0.481978in}{0.331635in}}{\pgfqpoint{9.300000in}{7.700000in}}%
\pgfusepath{clip}%
\pgfsetrectcap%
\pgfsetroundjoin%
\pgfsetlinewidth{1.505625pt}%
\definecolor{currentstroke}{rgb}{0.631373,0.788235,0.956863}%
\pgfsetstrokecolor{currentstroke}%
\pgfsetstrokeopacity{0.800000}%
\pgfsetdash{}{0pt}%
\pgfpathmoveto{\pgfqpoint{7.682062in}{5.953421in}}%
\pgfpathlineto{\pgfqpoint{5.711094in}{3.773662in}}%
\pgfusepath{stroke}%
\end{pgfscope}%
\begin{pgfscope}%
\pgfpathrectangle{\pgfqpoint{0.481978in}{0.331635in}}{\pgfqpoint{9.300000in}{7.700000in}}%
\pgfusepath{clip}%
\pgfsetrectcap%
\pgfsetroundjoin%
\pgfsetlinewidth{1.505625pt}%
\definecolor{currentstroke}{rgb}{0.631373,0.788235,0.956863}%
\pgfsetstrokecolor{currentstroke}%
\pgfsetstrokeopacity{0.800000}%
\pgfsetdash{}{0pt}%
\pgfpathmoveto{\pgfqpoint{6.477636in}{3.880545in}}%
\pgfpathlineto{\pgfqpoint{5.711094in}{3.773662in}}%
\pgfusepath{stroke}%
\end{pgfscope}%
\begin{pgfscope}%
\pgfpathrectangle{\pgfqpoint{0.481978in}{0.331635in}}{\pgfqpoint{9.300000in}{7.700000in}}%
\pgfusepath{clip}%
\pgfsetrectcap%
\pgfsetroundjoin%
\pgfsetlinewidth{1.505625pt}%
\definecolor{currentstroke}{rgb}{0.631373,0.788235,0.956863}%
\pgfsetstrokecolor{currentstroke}%
\pgfsetstrokeopacity{0.800000}%
\pgfsetdash{}{0pt}%
\pgfpathmoveto{\pgfqpoint{6.604622in}{2.950776in}}%
\pgfpathlineto{\pgfqpoint{5.711094in}{3.773662in}}%
\pgfusepath{stroke}%
\end{pgfscope}%
\begin{pgfscope}%
\pgfpathrectangle{\pgfqpoint{0.481978in}{0.331635in}}{\pgfqpoint{9.300000in}{7.700000in}}%
\pgfusepath{clip}%
\pgfsetrectcap%
\pgfsetroundjoin%
\pgfsetlinewidth{1.505625pt}%
\definecolor{currentstroke}{rgb}{0.631373,0.788235,0.956863}%
\pgfsetstrokecolor{currentstroke}%
\pgfsetstrokeopacity{0.800000}%
\pgfsetdash{}{0pt}%
\pgfpathmoveto{\pgfqpoint{7.441282in}{2.708973in}}%
\pgfpathlineto{\pgfqpoint{5.711094in}{3.773662in}}%
\pgfusepath{stroke}%
\end{pgfscope}%
\begin{pgfscope}%
\pgfpathrectangle{\pgfqpoint{0.481978in}{0.331635in}}{\pgfqpoint{9.300000in}{7.700000in}}%
\pgfusepath{clip}%
\pgfsetrectcap%
\pgfsetroundjoin%
\pgfsetlinewidth{1.505625pt}%
\definecolor{currentstroke}{rgb}{0.631373,0.788235,0.956863}%
\pgfsetstrokecolor{currentstroke}%
\pgfsetstrokeopacity{0.800000}%
\pgfsetdash{}{0pt}%
\pgfpathmoveto{\pgfqpoint{5.021448in}{5.598038in}}%
\pgfpathlineto{\pgfqpoint{5.711094in}{3.773662in}}%
\pgfusepath{stroke}%
\end{pgfscope}%
\begin{pgfscope}%
\pgfpathrectangle{\pgfqpoint{0.481978in}{0.331635in}}{\pgfqpoint{9.300000in}{7.700000in}}%
\pgfusepath{clip}%
\pgfsetrectcap%
\pgfsetroundjoin%
\pgfsetlinewidth{1.505625pt}%
\definecolor{currentstroke}{rgb}{0.631373,0.788235,0.956863}%
\pgfsetstrokecolor{currentstroke}%
\pgfsetstrokeopacity{0.800000}%
\pgfsetdash{}{0pt}%
\pgfpathmoveto{\pgfqpoint{6.840343in}{4.443893in}}%
\pgfpathlineto{\pgfqpoint{5.711094in}{3.773662in}}%
\pgfusepath{stroke}%
\end{pgfscope}%
\begin{pgfscope}%
\pgfpathrectangle{\pgfqpoint{0.481978in}{0.331635in}}{\pgfqpoint{9.300000in}{7.700000in}}%
\pgfusepath{clip}%
\pgfsetrectcap%
\pgfsetroundjoin%
\pgfsetlinewidth{1.505625pt}%
\definecolor{currentstroke}{rgb}{0.631373,0.788235,0.956863}%
\pgfsetstrokecolor{currentstroke}%
\pgfsetstrokeopacity{0.800000}%
\pgfsetdash{}{0pt}%
\pgfpathmoveto{\pgfqpoint{6.128623in}{6.002066in}}%
\pgfpathlineto{\pgfqpoint{5.711094in}{3.773662in}}%
\pgfusepath{stroke}%
\end{pgfscope}%
\begin{pgfscope}%
\pgfpathrectangle{\pgfqpoint{0.481978in}{0.331635in}}{\pgfqpoint{9.300000in}{7.700000in}}%
\pgfusepath{clip}%
\pgfsetrectcap%
\pgfsetroundjoin%
\pgfsetlinewidth{1.505625pt}%
\definecolor{currentstroke}{rgb}{0.631373,0.788235,0.956863}%
\pgfsetstrokecolor{currentstroke}%
\pgfsetstrokeopacity{0.800000}%
\pgfsetdash{}{0pt}%
\pgfpathmoveto{\pgfqpoint{6.130665in}{5.997020in}}%
\pgfpathlineto{\pgfqpoint{5.711094in}{3.773662in}}%
\pgfusepath{stroke}%
\end{pgfscope}%
\begin{pgfscope}%
\pgfpathrectangle{\pgfqpoint{0.481978in}{0.331635in}}{\pgfqpoint{9.300000in}{7.700000in}}%
\pgfusepath{clip}%
\pgfsetrectcap%
\pgfsetroundjoin%
\pgfsetlinewidth{1.505625pt}%
\definecolor{currentstroke}{rgb}{0.631373,0.788235,0.956863}%
\pgfsetstrokecolor{currentstroke}%
\pgfsetstrokeopacity{0.800000}%
\pgfsetdash{}{0pt}%
\pgfpathmoveto{\pgfqpoint{5.250013in}{5.339556in}}%
\pgfpathlineto{\pgfqpoint{5.711094in}{3.773662in}}%
\pgfusepath{stroke}%
\end{pgfscope}%
\begin{pgfscope}%
\pgfpathrectangle{\pgfqpoint{0.481978in}{0.331635in}}{\pgfqpoint{9.300000in}{7.700000in}}%
\pgfusepath{clip}%
\pgfsetrectcap%
\pgfsetroundjoin%
\pgfsetlinewidth{1.505625pt}%
\definecolor{currentstroke}{rgb}{0.631373,0.788235,0.956863}%
\pgfsetstrokecolor{currentstroke}%
\pgfsetstrokeopacity{0.800000}%
\pgfsetdash{}{0pt}%
\pgfpathmoveto{\pgfqpoint{7.725554in}{5.826615in}}%
\pgfpathlineto{\pgfqpoint{5.711094in}{3.773662in}}%
\pgfusepath{stroke}%
\end{pgfscope}%
\begin{pgfscope}%
\pgfpathrectangle{\pgfqpoint{0.481978in}{0.331635in}}{\pgfqpoint{9.300000in}{7.700000in}}%
\pgfusepath{clip}%
\pgfsetrectcap%
\pgfsetroundjoin%
\pgfsetlinewidth{1.505625pt}%
\definecolor{currentstroke}{rgb}{0.631373,0.788235,0.956863}%
\pgfsetstrokecolor{currentstroke}%
\pgfsetstrokeopacity{0.800000}%
\pgfsetdash{}{0pt}%
\pgfpathmoveto{\pgfqpoint{5.804196in}{2.934150in}}%
\pgfpathlineto{\pgfqpoint{5.711094in}{3.773662in}}%
\pgfusepath{stroke}%
\end{pgfscope}%
\begin{pgfscope}%
\pgfpathrectangle{\pgfqpoint{0.481978in}{0.331635in}}{\pgfqpoint{9.300000in}{7.700000in}}%
\pgfusepath{clip}%
\pgfsetrectcap%
\pgfsetroundjoin%
\pgfsetlinewidth{1.505625pt}%
\definecolor{currentstroke}{rgb}{0.631373,0.788235,0.956863}%
\pgfsetstrokecolor{currentstroke}%
\pgfsetstrokeopacity{0.800000}%
\pgfsetdash{}{0pt}%
\pgfpathmoveto{\pgfqpoint{4.361412in}{6.142763in}}%
\pgfpathlineto{\pgfqpoint{5.711094in}{3.773662in}}%
\pgfusepath{stroke}%
\end{pgfscope}%
\begin{pgfscope}%
\pgfpathrectangle{\pgfqpoint{0.481978in}{0.331635in}}{\pgfqpoint{9.300000in}{7.700000in}}%
\pgfusepath{clip}%
\pgfsetrectcap%
\pgfsetroundjoin%
\pgfsetlinewidth{1.505625pt}%
\definecolor{currentstroke}{rgb}{0.631373,0.788235,0.956863}%
\pgfsetstrokecolor{currentstroke}%
\pgfsetstrokeopacity{0.800000}%
\pgfsetdash{}{0pt}%
\pgfpathmoveto{\pgfqpoint{7.515857in}{5.150238in}}%
\pgfpathlineto{\pgfqpoint{5.711094in}{3.773662in}}%
\pgfusepath{stroke}%
\end{pgfscope}%
\begin{pgfscope}%
\pgfpathrectangle{\pgfqpoint{0.481978in}{0.331635in}}{\pgfqpoint{9.300000in}{7.700000in}}%
\pgfusepath{clip}%
\pgfsetrectcap%
\pgfsetroundjoin%
\pgfsetlinewidth{1.505625pt}%
\definecolor{currentstroke}{rgb}{0.631373,0.788235,0.956863}%
\pgfsetstrokecolor{currentstroke}%
\pgfsetstrokeopacity{0.800000}%
\pgfsetdash{}{0pt}%
\pgfpathmoveto{\pgfqpoint{8.605266in}{4.835347in}}%
\pgfpathlineto{\pgfqpoint{5.711094in}{3.773662in}}%
\pgfusepath{stroke}%
\end{pgfscope}%
\begin{pgfscope}%
\pgfpathrectangle{\pgfqpoint{0.481978in}{0.331635in}}{\pgfqpoint{9.300000in}{7.700000in}}%
\pgfusepath{clip}%
\pgfsetrectcap%
\pgfsetroundjoin%
\pgfsetlinewidth{1.505625pt}%
\definecolor{currentstroke}{rgb}{0.631373,0.788235,0.956863}%
\pgfsetstrokecolor{currentstroke}%
\pgfsetstrokeopacity{0.800000}%
\pgfsetdash{}{0pt}%
\pgfpathmoveto{\pgfqpoint{6.196847in}{1.949591in}}%
\pgfpathlineto{\pgfqpoint{5.711094in}{3.773662in}}%
\pgfusepath{stroke}%
\end{pgfscope}%
\begin{pgfscope}%
\pgfpathrectangle{\pgfqpoint{0.481978in}{0.331635in}}{\pgfqpoint{9.300000in}{7.700000in}}%
\pgfusepath{clip}%
\pgfsetrectcap%
\pgfsetroundjoin%
\pgfsetlinewidth{1.505625pt}%
\definecolor{currentstroke}{rgb}{0.631373,0.788235,0.956863}%
\pgfsetstrokecolor{currentstroke}%
\pgfsetstrokeopacity{0.800000}%
\pgfsetdash{}{0pt}%
\pgfpathmoveto{\pgfqpoint{4.336260in}{2.062934in}}%
\pgfpathlineto{\pgfqpoint{5.711094in}{3.773662in}}%
\pgfusepath{stroke}%
\end{pgfscope}%
\begin{pgfscope}%
\pgfpathrectangle{\pgfqpoint{0.481978in}{0.331635in}}{\pgfqpoint{9.300000in}{7.700000in}}%
\pgfusepath{clip}%
\pgfsetrectcap%
\pgfsetroundjoin%
\pgfsetlinewidth{1.505625pt}%
\definecolor{currentstroke}{rgb}{0.631373,0.788235,0.956863}%
\pgfsetstrokecolor{currentstroke}%
\pgfsetstrokeopacity{0.800000}%
\pgfsetdash{}{0pt}%
\pgfpathmoveto{\pgfqpoint{4.441169in}{5.411759in}}%
\pgfpathlineto{\pgfqpoint{5.711094in}{3.773662in}}%
\pgfusepath{stroke}%
\end{pgfscope}%
\begin{pgfscope}%
\pgfpathrectangle{\pgfqpoint{0.481978in}{0.331635in}}{\pgfqpoint{9.300000in}{7.700000in}}%
\pgfusepath{clip}%
\pgfsetrectcap%
\pgfsetroundjoin%
\pgfsetlinewidth{1.505625pt}%
\definecolor{currentstroke}{rgb}{0.631373,0.788235,0.956863}%
\pgfsetstrokecolor{currentstroke}%
\pgfsetstrokeopacity{0.800000}%
\pgfsetdash{}{0pt}%
\pgfpathmoveto{\pgfqpoint{6.227427in}{1.697900in}}%
\pgfpathlineto{\pgfqpoint{5.711094in}{3.773662in}}%
\pgfusepath{stroke}%
\end{pgfscope}%
\begin{pgfscope}%
\pgfpathrectangle{\pgfqpoint{0.481978in}{0.331635in}}{\pgfqpoint{9.300000in}{7.700000in}}%
\pgfusepath{clip}%
\pgfsetrectcap%
\pgfsetroundjoin%
\pgfsetlinewidth{1.505625pt}%
\definecolor{currentstroke}{rgb}{0.631373,0.788235,0.956863}%
\pgfsetstrokecolor{currentstroke}%
\pgfsetstrokeopacity{0.800000}%
\pgfsetdash{}{0pt}%
\pgfpathmoveto{\pgfqpoint{7.013858in}{2.259947in}}%
\pgfpathlineto{\pgfqpoint{5.711094in}{3.773662in}}%
\pgfusepath{stroke}%
\end{pgfscope}%
\begin{pgfscope}%
\pgfpathrectangle{\pgfqpoint{0.481978in}{0.331635in}}{\pgfqpoint{9.300000in}{7.700000in}}%
\pgfusepath{clip}%
\pgfsetrectcap%
\pgfsetroundjoin%
\pgfsetlinewidth{1.505625pt}%
\definecolor{currentstroke}{rgb}{0.631373,0.788235,0.956863}%
\pgfsetstrokecolor{currentstroke}%
\pgfsetstrokeopacity{0.800000}%
\pgfsetdash{}{0pt}%
\pgfpathmoveto{\pgfqpoint{3.180339in}{6.512223in}}%
\pgfpathlineto{\pgfqpoint{5.711094in}{3.773662in}}%
\pgfusepath{stroke}%
\end{pgfscope}%
\begin{pgfscope}%
\pgfpathrectangle{\pgfqpoint{0.481978in}{0.331635in}}{\pgfqpoint{9.300000in}{7.700000in}}%
\pgfusepath{clip}%
\pgfsetrectcap%
\pgfsetroundjoin%
\pgfsetlinewidth{1.505625pt}%
\definecolor{currentstroke}{rgb}{0.631373,0.788235,0.956863}%
\pgfsetstrokecolor{currentstroke}%
\pgfsetstrokeopacity{0.800000}%
\pgfsetdash{}{0pt}%
\pgfpathmoveto{\pgfqpoint{4.845738in}{0.865290in}}%
\pgfpathlineto{\pgfqpoint{5.711094in}{3.773662in}}%
\pgfusepath{stroke}%
\end{pgfscope}%
\begin{pgfscope}%
\pgfpathrectangle{\pgfqpoint{0.481978in}{0.331635in}}{\pgfqpoint{9.300000in}{7.700000in}}%
\pgfusepath{clip}%
\pgfsetrectcap%
\pgfsetroundjoin%
\pgfsetlinewidth{1.505625pt}%
\definecolor{currentstroke}{rgb}{0.631373,0.788235,0.956863}%
\pgfsetstrokecolor{currentstroke}%
\pgfsetstrokeopacity{0.800000}%
\pgfsetdash{}{0pt}%
\pgfpathmoveto{\pgfqpoint{4.394203in}{1.848846in}}%
\pgfpathlineto{\pgfqpoint{5.711094in}{3.773662in}}%
\pgfusepath{stroke}%
\end{pgfscope}%
\begin{pgfscope}%
\pgfpathrectangle{\pgfqpoint{0.481978in}{0.331635in}}{\pgfqpoint{9.300000in}{7.700000in}}%
\pgfusepath{clip}%
\pgfsetrectcap%
\pgfsetroundjoin%
\pgfsetlinewidth{1.505625pt}%
\definecolor{currentstroke}{rgb}{0.631373,0.788235,0.956863}%
\pgfsetstrokecolor{currentstroke}%
\pgfsetstrokeopacity{0.800000}%
\pgfsetdash{}{0pt}%
\pgfpathmoveto{\pgfqpoint{5.179568in}{5.355810in}}%
\pgfpathlineto{\pgfqpoint{5.711094in}{3.773662in}}%
\pgfusepath{stroke}%
\end{pgfscope}%
\begin{pgfscope}%
\pgfpathrectangle{\pgfqpoint{0.481978in}{0.331635in}}{\pgfqpoint{9.300000in}{7.700000in}}%
\pgfusepath{clip}%
\pgfsetrectcap%
\pgfsetroundjoin%
\pgfsetlinewidth{1.505625pt}%
\definecolor{currentstroke}{rgb}{0.631373,0.788235,0.956863}%
\pgfsetstrokecolor{currentstroke}%
\pgfsetstrokeopacity{0.800000}%
\pgfsetdash{}{0pt}%
\pgfpathmoveto{\pgfqpoint{7.941055in}{4.986937in}}%
\pgfpathlineto{\pgfqpoint{5.711094in}{3.773662in}}%
\pgfusepath{stroke}%
\end{pgfscope}%
\begin{pgfscope}%
\pgfpathrectangle{\pgfqpoint{0.481978in}{0.331635in}}{\pgfqpoint{9.300000in}{7.700000in}}%
\pgfusepath{clip}%
\pgfsetrectcap%
\pgfsetroundjoin%
\pgfsetlinewidth{1.505625pt}%
\definecolor{currentstroke}{rgb}{0.631373,0.788235,0.956863}%
\pgfsetstrokecolor{currentstroke}%
\pgfsetstrokeopacity{0.800000}%
\pgfsetdash{}{0pt}%
\pgfpathmoveto{\pgfqpoint{3.294139in}{1.577402in}}%
\pgfpathlineto{\pgfqpoint{5.711094in}{3.773662in}}%
\pgfusepath{stroke}%
\end{pgfscope}%
\begin{pgfscope}%
\pgfpathrectangle{\pgfqpoint{0.481978in}{0.331635in}}{\pgfqpoint{9.300000in}{7.700000in}}%
\pgfusepath{clip}%
\pgfsetrectcap%
\pgfsetroundjoin%
\pgfsetlinewidth{1.505625pt}%
\definecolor{currentstroke}{rgb}{0.631373,0.788235,0.956863}%
\pgfsetstrokecolor{currentstroke}%
\pgfsetstrokeopacity{0.800000}%
\pgfsetdash{}{0pt}%
\pgfpathmoveto{\pgfqpoint{7.359994in}{4.031512in}}%
\pgfpathlineto{\pgfqpoint{5.711094in}{3.773662in}}%
\pgfusepath{stroke}%
\end{pgfscope}%
\begin{pgfscope}%
\pgfpathrectangle{\pgfqpoint{0.481978in}{0.331635in}}{\pgfqpoint{9.300000in}{7.700000in}}%
\pgfusepath{clip}%
\pgfsetrectcap%
\pgfsetroundjoin%
\pgfsetlinewidth{1.505625pt}%
\definecolor{currentstroke}{rgb}{0.631373,0.788235,0.956863}%
\pgfsetstrokecolor{currentstroke}%
\pgfsetstrokeopacity{0.800000}%
\pgfsetdash{}{0pt}%
\pgfpathmoveto{\pgfqpoint{7.314751in}{4.915613in}}%
\pgfpathlineto{\pgfqpoint{5.711094in}{3.773662in}}%
\pgfusepath{stroke}%
\end{pgfscope}%
\begin{pgfscope}%
\pgfpathrectangle{\pgfqpoint{0.481978in}{0.331635in}}{\pgfqpoint{9.300000in}{7.700000in}}%
\pgfusepath{clip}%
\pgfsetrectcap%
\pgfsetroundjoin%
\pgfsetlinewidth{1.505625pt}%
\definecolor{currentstroke}{rgb}{0.631373,0.788235,0.956863}%
\pgfsetstrokecolor{currentstroke}%
\pgfsetstrokeopacity{0.800000}%
\pgfsetdash{}{0pt}%
\pgfpathmoveto{\pgfqpoint{3.066979in}{6.682028in}}%
\pgfpathlineto{\pgfqpoint{5.711094in}{3.773662in}}%
\pgfusepath{stroke}%
\end{pgfscope}%
\begin{pgfscope}%
\pgfpathrectangle{\pgfqpoint{0.481978in}{0.331635in}}{\pgfqpoint{9.300000in}{7.700000in}}%
\pgfusepath{clip}%
\pgfsetrectcap%
\pgfsetroundjoin%
\pgfsetlinewidth{1.505625pt}%
\definecolor{currentstroke}{rgb}{0.631373,0.788235,0.956863}%
\pgfsetstrokecolor{currentstroke}%
\pgfsetstrokeopacity{0.800000}%
\pgfsetdash{}{0pt}%
\pgfpathmoveto{\pgfqpoint{2.849648in}{6.910141in}}%
\pgfpathlineto{\pgfqpoint{5.711094in}{3.773662in}}%
\pgfusepath{stroke}%
\end{pgfscope}%
\begin{pgfscope}%
\pgfpathrectangle{\pgfqpoint{0.481978in}{0.331635in}}{\pgfqpoint{9.300000in}{7.700000in}}%
\pgfusepath{clip}%
\pgfsetrectcap%
\pgfsetroundjoin%
\pgfsetlinewidth{1.505625pt}%
\definecolor{currentstroke}{rgb}{0.631373,0.788235,0.956863}%
\pgfsetstrokecolor{currentstroke}%
\pgfsetstrokeopacity{0.800000}%
\pgfsetdash{}{0pt}%
\pgfpathmoveto{\pgfqpoint{2.771056in}{6.665431in}}%
\pgfpathlineto{\pgfqpoint{5.711094in}{3.773662in}}%
\pgfusepath{stroke}%
\end{pgfscope}%
\begin{pgfscope}%
\pgfpathrectangle{\pgfqpoint{0.481978in}{0.331635in}}{\pgfqpoint{9.300000in}{7.700000in}}%
\pgfusepath{clip}%
\pgfsetrectcap%
\pgfsetroundjoin%
\pgfsetlinewidth{1.505625pt}%
\definecolor{currentstroke}{rgb}{0.631373,0.788235,0.956863}%
\pgfsetstrokecolor{currentstroke}%
\pgfsetstrokeopacity{0.800000}%
\pgfsetdash{}{0pt}%
\pgfpathmoveto{\pgfqpoint{4.853047in}{5.922135in}}%
\pgfpathlineto{\pgfqpoint{5.711094in}{3.773662in}}%
\pgfusepath{stroke}%
\end{pgfscope}%
\begin{pgfscope}%
\pgfpathrectangle{\pgfqpoint{0.481978in}{0.331635in}}{\pgfqpoint{9.300000in}{7.700000in}}%
\pgfusepath{clip}%
\pgfsetrectcap%
\pgfsetroundjoin%
\pgfsetlinewidth{1.505625pt}%
\definecolor{currentstroke}{rgb}{0.631373,0.788235,0.956863}%
\pgfsetstrokecolor{currentstroke}%
\pgfsetstrokeopacity{0.800000}%
\pgfsetdash{}{0pt}%
\pgfpathmoveto{\pgfqpoint{5.879927in}{0.844058in}}%
\pgfpathlineto{\pgfqpoint{5.711094in}{3.773662in}}%
\pgfusepath{stroke}%
\end{pgfscope}%
\begin{pgfscope}%
\pgfpathrectangle{\pgfqpoint{0.481978in}{0.331635in}}{\pgfqpoint{9.300000in}{7.700000in}}%
\pgfusepath{clip}%
\pgfsetrectcap%
\pgfsetroundjoin%
\pgfsetlinewidth{1.505625pt}%
\definecolor{currentstroke}{rgb}{0.631373,0.788235,0.956863}%
\pgfsetstrokecolor{currentstroke}%
\pgfsetstrokeopacity{0.800000}%
\pgfsetdash{}{0pt}%
\pgfpathmoveto{\pgfqpoint{6.834691in}{2.083391in}}%
\pgfpathlineto{\pgfqpoint{5.711094in}{3.773662in}}%
\pgfusepath{stroke}%
\end{pgfscope}%
\begin{pgfscope}%
\pgfpathrectangle{\pgfqpoint{0.481978in}{0.331635in}}{\pgfqpoint{9.300000in}{7.700000in}}%
\pgfusepath{clip}%
\pgfsetrectcap%
\pgfsetroundjoin%
\pgfsetlinewidth{1.505625pt}%
\definecolor{currentstroke}{rgb}{0.631373,0.788235,0.956863}%
\pgfsetstrokecolor{currentstroke}%
\pgfsetstrokeopacity{0.800000}%
\pgfsetdash{}{0pt}%
\pgfpathmoveto{\pgfqpoint{5.297141in}{2.779134in}}%
\pgfpathlineto{\pgfqpoint{5.711094in}{3.773662in}}%
\pgfusepath{stroke}%
\end{pgfscope}%
\begin{pgfscope}%
\pgfpathrectangle{\pgfqpoint{0.481978in}{0.331635in}}{\pgfqpoint{9.300000in}{7.700000in}}%
\pgfusepath{clip}%
\pgfsetrectcap%
\pgfsetroundjoin%
\pgfsetlinewidth{1.505625pt}%
\definecolor{currentstroke}{rgb}{0.631373,0.788235,0.956863}%
\pgfsetstrokecolor{currentstroke}%
\pgfsetstrokeopacity{0.800000}%
\pgfsetdash{}{0pt}%
\pgfpathmoveto{\pgfqpoint{6.713307in}{5.182723in}}%
\pgfpathlineto{\pgfqpoint{5.711094in}{3.773662in}}%
\pgfusepath{stroke}%
\end{pgfscope}%
\begin{pgfscope}%
\pgfpathrectangle{\pgfqpoint{0.481978in}{0.331635in}}{\pgfqpoint{9.300000in}{7.700000in}}%
\pgfusepath{clip}%
\pgfsetrectcap%
\pgfsetroundjoin%
\pgfsetlinewidth{1.505625pt}%
\definecolor{currentstroke}{rgb}{0.631373,0.788235,0.956863}%
\pgfsetstrokecolor{currentstroke}%
\pgfsetstrokeopacity{0.800000}%
\pgfsetdash{}{0pt}%
\pgfpathmoveto{\pgfqpoint{5.117952in}{2.351880in}}%
\pgfpathlineto{\pgfqpoint{5.711094in}{3.773662in}}%
\pgfusepath{stroke}%
\end{pgfscope}%
\begin{pgfscope}%
\pgfpathrectangle{\pgfqpoint{0.481978in}{0.331635in}}{\pgfqpoint{9.300000in}{7.700000in}}%
\pgfusepath{clip}%
\pgfsetrectcap%
\pgfsetroundjoin%
\pgfsetlinewidth{1.505625pt}%
\definecolor{currentstroke}{rgb}{0.631373,0.788235,0.956863}%
\pgfsetstrokecolor{currentstroke}%
\pgfsetstrokeopacity{0.800000}%
\pgfsetdash{}{0pt}%
\pgfpathmoveto{\pgfqpoint{7.076892in}{3.094365in}}%
\pgfpathlineto{\pgfqpoint{5.711094in}{3.773662in}}%
\pgfusepath{stroke}%
\end{pgfscope}%
\begin{pgfscope}%
\pgfpathrectangle{\pgfqpoint{0.481978in}{0.331635in}}{\pgfqpoint{9.300000in}{7.700000in}}%
\pgfusepath{clip}%
\pgfsetrectcap%
\pgfsetroundjoin%
\pgfsetlinewidth{1.505625pt}%
\definecolor{currentstroke}{rgb}{0.631373,0.788235,0.956863}%
\pgfsetstrokecolor{currentstroke}%
\pgfsetstrokeopacity{0.800000}%
\pgfsetdash{}{0pt}%
\pgfpathmoveto{\pgfqpoint{7.507895in}{1.953530in}}%
\pgfpathlineto{\pgfqpoint{5.711094in}{3.773662in}}%
\pgfusepath{stroke}%
\end{pgfscope}%
\begin{pgfscope}%
\pgfpathrectangle{\pgfqpoint{0.481978in}{0.331635in}}{\pgfqpoint{9.300000in}{7.700000in}}%
\pgfusepath{clip}%
\pgfsetrectcap%
\pgfsetroundjoin%
\pgfsetlinewidth{1.505625pt}%
\definecolor{currentstroke}{rgb}{0.631373,0.788235,0.956863}%
\pgfsetstrokecolor{currentstroke}%
\pgfsetstrokeopacity{0.800000}%
\pgfsetdash{}{0pt}%
\pgfpathmoveto{\pgfqpoint{4.102323in}{5.830387in}}%
\pgfpathlineto{\pgfqpoint{5.711094in}{3.773662in}}%
\pgfusepath{stroke}%
\end{pgfscope}%
\begin{pgfscope}%
\pgfpathrectangle{\pgfqpoint{0.481978in}{0.331635in}}{\pgfqpoint{9.300000in}{7.700000in}}%
\pgfusepath{clip}%
\pgfsetrectcap%
\pgfsetroundjoin%
\pgfsetlinewidth{1.505625pt}%
\definecolor{currentstroke}{rgb}{0.631373,0.788235,0.956863}%
\pgfsetstrokecolor{currentstroke}%
\pgfsetstrokeopacity{0.800000}%
\pgfsetdash{}{0pt}%
\pgfpathmoveto{\pgfqpoint{6.020143in}{5.316402in}}%
\pgfpathlineto{\pgfqpoint{5.711094in}{3.773662in}}%
\pgfusepath{stroke}%
\end{pgfscope}%
\begin{pgfscope}%
\pgfpathrectangle{\pgfqpoint{0.481978in}{0.331635in}}{\pgfqpoint{9.300000in}{7.700000in}}%
\pgfusepath{clip}%
\pgfsetrectcap%
\pgfsetroundjoin%
\pgfsetlinewidth{1.505625pt}%
\definecolor{currentstroke}{rgb}{0.631373,0.788235,0.956863}%
\pgfsetstrokecolor{currentstroke}%
\pgfsetstrokeopacity{0.800000}%
\pgfsetdash{}{0pt}%
\pgfpathmoveto{\pgfqpoint{7.865979in}{5.806222in}}%
\pgfpathlineto{\pgfqpoint{5.711094in}{3.773662in}}%
\pgfusepath{stroke}%
\end{pgfscope}%
\begin{pgfscope}%
\pgfpathrectangle{\pgfqpoint{0.481978in}{0.331635in}}{\pgfqpoint{9.300000in}{7.700000in}}%
\pgfusepath{clip}%
\pgfsetrectcap%
\pgfsetroundjoin%
\pgfsetlinewidth{1.505625pt}%
\definecolor{currentstroke}{rgb}{0.631373,0.788235,0.956863}%
\pgfsetstrokecolor{currentstroke}%
\pgfsetstrokeopacity{0.800000}%
\pgfsetdash{}{0pt}%
\pgfpathmoveto{\pgfqpoint{5.392504in}{5.413544in}}%
\pgfpathlineto{\pgfqpoint{5.711094in}{3.773662in}}%
\pgfusepath{stroke}%
\end{pgfscope}%
\begin{pgfscope}%
\pgfpathrectangle{\pgfqpoint{0.481978in}{0.331635in}}{\pgfqpoint{9.300000in}{7.700000in}}%
\pgfusepath{clip}%
\pgfsetrectcap%
\pgfsetroundjoin%
\pgfsetlinewidth{1.505625pt}%
\definecolor{currentstroke}{rgb}{0.631373,0.788235,0.956863}%
\pgfsetstrokecolor{currentstroke}%
\pgfsetstrokeopacity{0.800000}%
\pgfsetdash{}{0pt}%
\pgfpathmoveto{\pgfqpoint{2.932058in}{5.424089in}}%
\pgfpathlineto{\pgfqpoint{5.711094in}{3.773662in}}%
\pgfusepath{stroke}%
\end{pgfscope}%
\begin{pgfscope}%
\pgfpathrectangle{\pgfqpoint{0.481978in}{0.331635in}}{\pgfqpoint{9.300000in}{7.700000in}}%
\pgfusepath{clip}%
\pgfsetrectcap%
\pgfsetroundjoin%
\pgfsetlinewidth{1.505625pt}%
\definecolor{currentstroke}{rgb}{0.631373,0.788235,0.956863}%
\pgfsetstrokecolor{currentstroke}%
\pgfsetstrokeopacity{0.800000}%
\pgfsetdash{}{0pt}%
\pgfpathmoveto{\pgfqpoint{7.779256in}{5.079699in}}%
\pgfpathlineto{\pgfqpoint{5.711094in}{3.773662in}}%
\pgfusepath{stroke}%
\end{pgfscope}%
\begin{pgfscope}%
\pgfpathrectangle{\pgfqpoint{0.481978in}{0.331635in}}{\pgfqpoint{9.300000in}{7.700000in}}%
\pgfusepath{clip}%
\pgfsetrectcap%
\pgfsetroundjoin%
\pgfsetlinewidth{1.505625pt}%
\definecolor{currentstroke}{rgb}{0.631373,0.788235,0.956863}%
\pgfsetstrokecolor{currentstroke}%
\pgfsetstrokeopacity{0.800000}%
\pgfsetdash{}{0pt}%
\pgfpathmoveto{\pgfqpoint{7.196974in}{1.631105in}}%
\pgfpathlineto{\pgfqpoint{5.711094in}{3.773662in}}%
\pgfusepath{stroke}%
\end{pgfscope}%
\begin{pgfscope}%
\pgfpathrectangle{\pgfqpoint{0.481978in}{0.331635in}}{\pgfqpoint{9.300000in}{7.700000in}}%
\pgfusepath{clip}%
\pgfsetrectcap%
\pgfsetroundjoin%
\pgfsetlinewidth{1.505625pt}%
\definecolor{currentstroke}{rgb}{0.631373,0.788235,0.956863}%
\pgfsetstrokecolor{currentstroke}%
\pgfsetstrokeopacity{0.800000}%
\pgfsetdash{}{0pt}%
\pgfpathmoveto{\pgfqpoint{3.418933in}{1.347084in}}%
\pgfpathlineto{\pgfqpoint{5.711094in}{3.773662in}}%
\pgfusepath{stroke}%
\end{pgfscope}%
\begin{pgfscope}%
\pgfpathrectangle{\pgfqpoint{0.481978in}{0.331635in}}{\pgfqpoint{9.300000in}{7.700000in}}%
\pgfusepath{clip}%
\pgfsetrectcap%
\pgfsetroundjoin%
\pgfsetlinewidth{1.505625pt}%
\definecolor{currentstroke}{rgb}{0.631373,0.788235,0.956863}%
\pgfsetstrokecolor{currentstroke}%
\pgfsetstrokeopacity{0.800000}%
\pgfsetdash{}{0pt}%
\pgfpathmoveto{\pgfqpoint{5.333628in}{3.495257in}}%
\pgfpathlineto{\pgfqpoint{5.711094in}{3.773662in}}%
\pgfusepath{stroke}%
\end{pgfscope}%
\begin{pgfscope}%
\pgfpathrectangle{\pgfqpoint{0.481978in}{0.331635in}}{\pgfqpoint{9.300000in}{7.700000in}}%
\pgfusepath{clip}%
\pgfsetrectcap%
\pgfsetroundjoin%
\pgfsetlinewidth{1.505625pt}%
\definecolor{currentstroke}{rgb}{0.631373,0.788235,0.956863}%
\pgfsetstrokecolor{currentstroke}%
\pgfsetstrokeopacity{0.800000}%
\pgfsetdash{}{0pt}%
\pgfpathmoveto{\pgfqpoint{6.946844in}{2.835271in}}%
\pgfpathlineto{\pgfqpoint{5.711094in}{3.773662in}}%
\pgfusepath{stroke}%
\end{pgfscope}%
\begin{pgfscope}%
\pgfpathrectangle{\pgfqpoint{0.481978in}{0.331635in}}{\pgfqpoint{9.300000in}{7.700000in}}%
\pgfusepath{clip}%
\pgfsetrectcap%
\pgfsetroundjoin%
\pgfsetlinewidth{1.505625pt}%
\definecolor{currentstroke}{rgb}{0.631373,0.788235,0.956863}%
\pgfsetstrokecolor{currentstroke}%
\pgfsetstrokeopacity{0.800000}%
\pgfsetdash{}{0pt}%
\pgfpathmoveto{\pgfqpoint{8.414129in}{5.020721in}}%
\pgfpathlineto{\pgfqpoint{5.711094in}{3.773662in}}%
\pgfusepath{stroke}%
\end{pgfscope}%
\begin{pgfscope}%
\pgfpathrectangle{\pgfqpoint{0.481978in}{0.331635in}}{\pgfqpoint{9.300000in}{7.700000in}}%
\pgfusepath{clip}%
\pgfsetrectcap%
\pgfsetroundjoin%
\pgfsetlinewidth{1.505625pt}%
\definecolor{currentstroke}{rgb}{0.631373,0.788235,0.956863}%
\pgfsetstrokecolor{currentstroke}%
\pgfsetstrokeopacity{0.800000}%
\pgfsetdash{}{0pt}%
\pgfpathmoveto{\pgfqpoint{4.957931in}{2.138162in}}%
\pgfpathlineto{\pgfqpoint{5.711094in}{3.773662in}}%
\pgfusepath{stroke}%
\end{pgfscope}%
\begin{pgfscope}%
\pgfpathrectangle{\pgfqpoint{0.481978in}{0.331635in}}{\pgfqpoint{9.300000in}{7.700000in}}%
\pgfusepath{clip}%
\pgfsetrectcap%
\pgfsetroundjoin%
\pgfsetlinewidth{1.505625pt}%
\definecolor{currentstroke}{rgb}{0.631373,0.788235,0.956863}%
\pgfsetstrokecolor{currentstroke}%
\pgfsetstrokeopacity{0.800000}%
\pgfsetdash{}{0pt}%
\pgfpathmoveto{\pgfqpoint{7.947630in}{5.393707in}}%
\pgfpathlineto{\pgfqpoint{5.711094in}{3.773662in}}%
\pgfusepath{stroke}%
\end{pgfscope}%
\begin{pgfscope}%
\pgfpathrectangle{\pgfqpoint{0.481978in}{0.331635in}}{\pgfqpoint{9.300000in}{7.700000in}}%
\pgfusepath{clip}%
\pgfsetrectcap%
\pgfsetroundjoin%
\pgfsetlinewidth{1.505625pt}%
\definecolor{currentstroke}{rgb}{0.631373,0.788235,0.956863}%
\pgfsetstrokecolor{currentstroke}%
\pgfsetstrokeopacity{0.800000}%
\pgfsetdash{}{0pt}%
\pgfpathmoveto{\pgfqpoint{3.205045in}{1.571422in}}%
\pgfpathlineto{\pgfqpoint{5.711094in}{3.773662in}}%
\pgfusepath{stroke}%
\end{pgfscope}%
\begin{pgfscope}%
\pgfpathrectangle{\pgfqpoint{0.481978in}{0.331635in}}{\pgfqpoint{9.300000in}{7.700000in}}%
\pgfusepath{clip}%
\pgfsetrectcap%
\pgfsetroundjoin%
\pgfsetlinewidth{1.505625pt}%
\definecolor{currentstroke}{rgb}{0.631373,0.788235,0.956863}%
\pgfsetstrokecolor{currentstroke}%
\pgfsetstrokeopacity{0.800000}%
\pgfsetdash{}{0pt}%
\pgfpathmoveto{\pgfqpoint{7.723488in}{5.525665in}}%
\pgfpathlineto{\pgfqpoint{5.711094in}{3.773662in}}%
\pgfusepath{stroke}%
\end{pgfscope}%
\begin{pgfscope}%
\pgfpathrectangle{\pgfqpoint{0.481978in}{0.331635in}}{\pgfqpoint{9.300000in}{7.700000in}}%
\pgfusepath{clip}%
\pgfsetrectcap%
\pgfsetroundjoin%
\pgfsetlinewidth{1.505625pt}%
\definecolor{currentstroke}{rgb}{0.631373,0.788235,0.956863}%
\pgfsetstrokecolor{currentstroke}%
\pgfsetstrokeopacity{0.800000}%
\pgfsetdash{}{0pt}%
\pgfpathmoveto{\pgfqpoint{4.705109in}{1.086625in}}%
\pgfpathlineto{\pgfqpoint{5.711094in}{3.773662in}}%
\pgfusepath{stroke}%
\end{pgfscope}%
\begin{pgfscope}%
\pgfpathrectangle{\pgfqpoint{0.481978in}{0.331635in}}{\pgfqpoint{9.300000in}{7.700000in}}%
\pgfusepath{clip}%
\pgfsetrectcap%
\pgfsetroundjoin%
\pgfsetlinewidth{1.505625pt}%
\definecolor{currentstroke}{rgb}{0.631373,0.788235,0.956863}%
\pgfsetstrokecolor{currentstroke}%
\pgfsetstrokeopacity{0.800000}%
\pgfsetdash{}{0pt}%
\pgfpathmoveto{\pgfqpoint{3.932688in}{3.913553in}}%
\pgfpathlineto{\pgfqpoint{5.711094in}{3.773662in}}%
\pgfusepath{stroke}%
\end{pgfscope}%
\begin{pgfscope}%
\pgfpathrectangle{\pgfqpoint{0.481978in}{0.331635in}}{\pgfqpoint{9.300000in}{7.700000in}}%
\pgfusepath{clip}%
\pgfsetrectcap%
\pgfsetroundjoin%
\pgfsetlinewidth{1.505625pt}%
\definecolor{currentstroke}{rgb}{0.631373,0.788235,0.956863}%
\pgfsetstrokecolor{currentstroke}%
\pgfsetstrokeopacity{0.800000}%
\pgfsetdash{}{0pt}%
\pgfpathmoveto{\pgfqpoint{7.599765in}{5.966116in}}%
\pgfpathlineto{\pgfqpoint{5.711094in}{3.773662in}}%
\pgfusepath{stroke}%
\end{pgfscope}%
\begin{pgfscope}%
\pgfpathrectangle{\pgfqpoint{0.481978in}{0.331635in}}{\pgfqpoint{9.300000in}{7.700000in}}%
\pgfusepath{clip}%
\pgfsetrectcap%
\pgfsetroundjoin%
\pgfsetlinewidth{1.505625pt}%
\definecolor{currentstroke}{rgb}{0.631373,0.788235,0.956863}%
\pgfsetstrokecolor{currentstroke}%
\pgfsetstrokeopacity{0.800000}%
\pgfsetdash{}{0pt}%
\pgfpathmoveto{\pgfqpoint{6.870443in}{1.703003in}}%
\pgfpathlineto{\pgfqpoint{5.711094in}{3.773662in}}%
\pgfusepath{stroke}%
\end{pgfscope}%
\begin{pgfscope}%
\pgfpathrectangle{\pgfqpoint{0.481978in}{0.331635in}}{\pgfqpoint{9.300000in}{7.700000in}}%
\pgfusepath{clip}%
\pgfsetrectcap%
\pgfsetroundjoin%
\pgfsetlinewidth{1.505625pt}%
\definecolor{currentstroke}{rgb}{0.631373,0.788235,0.956863}%
\pgfsetstrokecolor{currentstroke}%
\pgfsetstrokeopacity{0.800000}%
\pgfsetdash{}{0pt}%
\pgfpathmoveto{\pgfqpoint{6.112529in}{2.152402in}}%
\pgfpathlineto{\pgfqpoint{5.711094in}{3.773662in}}%
\pgfusepath{stroke}%
\end{pgfscope}%
\begin{pgfscope}%
\pgfpathrectangle{\pgfqpoint{0.481978in}{0.331635in}}{\pgfqpoint{9.300000in}{7.700000in}}%
\pgfusepath{clip}%
\pgfsetrectcap%
\pgfsetroundjoin%
\pgfsetlinewidth{1.505625pt}%
\definecolor{currentstroke}{rgb}{0.631373,0.788235,0.956863}%
\pgfsetstrokecolor{currentstroke}%
\pgfsetstrokeopacity{0.800000}%
\pgfsetdash{}{0pt}%
\pgfpathmoveto{\pgfqpoint{7.807922in}{5.795876in}}%
\pgfpathlineto{\pgfqpoint{5.711094in}{3.773662in}}%
\pgfusepath{stroke}%
\end{pgfscope}%
\begin{pgfscope}%
\pgfpathrectangle{\pgfqpoint{0.481978in}{0.331635in}}{\pgfqpoint{9.300000in}{7.700000in}}%
\pgfusepath{clip}%
\pgfsetrectcap%
\pgfsetroundjoin%
\pgfsetlinewidth{1.505625pt}%
\definecolor{currentstroke}{rgb}{0.631373,0.788235,0.956863}%
\pgfsetstrokecolor{currentstroke}%
\pgfsetstrokeopacity{0.800000}%
\pgfsetdash{}{0pt}%
\pgfpathmoveto{\pgfqpoint{3.326485in}{2.058650in}}%
\pgfpathlineto{\pgfqpoint{5.711094in}{3.773662in}}%
\pgfusepath{stroke}%
\end{pgfscope}%
\begin{pgfscope}%
\pgfpathrectangle{\pgfqpoint{0.481978in}{0.331635in}}{\pgfqpoint{9.300000in}{7.700000in}}%
\pgfusepath{clip}%
\pgfsetrectcap%
\pgfsetroundjoin%
\pgfsetlinewidth{1.505625pt}%
\definecolor{currentstroke}{rgb}{0.631373,0.788235,0.956863}%
\pgfsetstrokecolor{currentstroke}%
\pgfsetstrokeopacity{0.800000}%
\pgfsetdash{}{0pt}%
\pgfpathmoveto{\pgfqpoint{1.925994in}{5.507668in}}%
\pgfpathlineto{\pgfqpoint{5.711094in}{3.773662in}}%
\pgfusepath{stroke}%
\end{pgfscope}%
\begin{pgfscope}%
\pgfpathrectangle{\pgfqpoint{0.481978in}{0.331635in}}{\pgfqpoint{9.300000in}{7.700000in}}%
\pgfusepath{clip}%
\pgfsetrectcap%
\pgfsetroundjoin%
\pgfsetlinewidth{1.505625pt}%
\definecolor{currentstroke}{rgb}{0.631373,0.788235,0.956863}%
\pgfsetstrokecolor{currentstroke}%
\pgfsetstrokeopacity{0.800000}%
\pgfsetdash{}{0pt}%
\pgfpathmoveto{\pgfqpoint{5.878648in}{0.848903in}}%
\pgfpathlineto{\pgfqpoint{5.711094in}{3.773662in}}%
\pgfusepath{stroke}%
\end{pgfscope}%
\begin{pgfscope}%
\pgfpathrectangle{\pgfqpoint{0.481978in}{0.331635in}}{\pgfqpoint{9.300000in}{7.700000in}}%
\pgfusepath{clip}%
\pgfsetrectcap%
\pgfsetroundjoin%
\pgfsetlinewidth{1.505625pt}%
\definecolor{currentstroke}{rgb}{0.631373,0.788235,0.956863}%
\pgfsetstrokecolor{currentstroke}%
\pgfsetstrokeopacity{0.800000}%
\pgfsetdash{}{0pt}%
\pgfpathmoveto{\pgfqpoint{8.178842in}{4.605790in}}%
\pgfpathlineto{\pgfqpoint{5.711094in}{3.773662in}}%
\pgfusepath{stroke}%
\end{pgfscope}%
\begin{pgfscope}%
\pgfpathrectangle{\pgfqpoint{0.481978in}{0.331635in}}{\pgfqpoint{9.300000in}{7.700000in}}%
\pgfusepath{clip}%
\pgfsetrectcap%
\pgfsetroundjoin%
\pgfsetlinewidth{1.505625pt}%
\definecolor{currentstroke}{rgb}{0.631373,0.788235,0.956863}%
\pgfsetstrokecolor{currentstroke}%
\pgfsetstrokeopacity{0.800000}%
\pgfsetdash{}{0pt}%
\pgfpathmoveto{\pgfqpoint{6.339836in}{2.882881in}}%
\pgfpathlineto{\pgfqpoint{5.711094in}{3.773662in}}%
\pgfusepath{stroke}%
\end{pgfscope}%
\begin{pgfscope}%
\pgfpathrectangle{\pgfqpoint{0.481978in}{0.331635in}}{\pgfqpoint{9.300000in}{7.700000in}}%
\pgfusepath{clip}%
\pgfsetrectcap%
\pgfsetroundjoin%
\pgfsetlinewidth{1.505625pt}%
\definecolor{currentstroke}{rgb}{0.631373,0.788235,0.956863}%
\pgfsetstrokecolor{currentstroke}%
\pgfsetstrokeopacity{0.800000}%
\pgfsetdash{}{0pt}%
\pgfpathmoveto{\pgfqpoint{5.971524in}{2.077066in}}%
\pgfpathlineto{\pgfqpoint{5.711094in}{3.773662in}}%
\pgfusepath{stroke}%
\end{pgfscope}%
\begin{pgfscope}%
\pgfpathrectangle{\pgfqpoint{0.481978in}{0.331635in}}{\pgfqpoint{9.300000in}{7.700000in}}%
\pgfusepath{clip}%
\pgfsetrectcap%
\pgfsetroundjoin%
\pgfsetlinewidth{1.505625pt}%
\definecolor{currentstroke}{rgb}{0.631373,0.788235,0.956863}%
\pgfsetstrokecolor{currentstroke}%
\pgfsetstrokeopacity{0.800000}%
\pgfsetdash{}{0pt}%
\pgfpathmoveto{\pgfqpoint{3.002811in}{6.491469in}}%
\pgfpathlineto{\pgfqpoint{5.711094in}{3.773662in}}%
\pgfusepath{stroke}%
\end{pgfscope}%
\begin{pgfscope}%
\pgfpathrectangle{\pgfqpoint{0.481978in}{0.331635in}}{\pgfqpoint{9.300000in}{7.700000in}}%
\pgfusepath{clip}%
\pgfsetrectcap%
\pgfsetroundjoin%
\pgfsetlinewidth{1.505625pt}%
\definecolor{currentstroke}{rgb}{0.631373,0.788235,0.956863}%
\pgfsetstrokecolor{currentstroke}%
\pgfsetstrokeopacity{0.800000}%
\pgfsetdash{}{0pt}%
\pgfpathmoveto{\pgfqpoint{5.207999in}{6.891529in}}%
\pgfpathlineto{\pgfqpoint{5.711094in}{3.773662in}}%
\pgfusepath{stroke}%
\end{pgfscope}%
\begin{pgfscope}%
\pgfpathrectangle{\pgfqpoint{0.481978in}{0.331635in}}{\pgfqpoint{9.300000in}{7.700000in}}%
\pgfusepath{clip}%
\pgfsetrectcap%
\pgfsetroundjoin%
\pgfsetlinewidth{1.505625pt}%
\definecolor{currentstroke}{rgb}{0.631373,0.788235,0.956863}%
\pgfsetstrokecolor{currentstroke}%
\pgfsetstrokeopacity{0.800000}%
\pgfsetdash{}{0pt}%
\pgfpathmoveto{\pgfqpoint{4.854525in}{3.614784in}}%
\pgfpathlineto{\pgfqpoint{5.711094in}{3.773662in}}%
\pgfusepath{stroke}%
\end{pgfscope}%
\begin{pgfscope}%
\pgfpathrectangle{\pgfqpoint{0.481978in}{0.331635in}}{\pgfqpoint{9.300000in}{7.700000in}}%
\pgfusepath{clip}%
\pgfsetrectcap%
\pgfsetroundjoin%
\pgfsetlinewidth{1.505625pt}%
\definecolor{currentstroke}{rgb}{0.631373,0.788235,0.956863}%
\pgfsetstrokecolor{currentstroke}%
\pgfsetstrokeopacity{0.800000}%
\pgfsetdash{}{0pt}%
\pgfpathmoveto{\pgfqpoint{3.734095in}{1.504608in}}%
\pgfpathlineto{\pgfqpoint{5.711094in}{3.773662in}}%
\pgfusepath{stroke}%
\end{pgfscope}%
\begin{pgfscope}%
\pgfpathrectangle{\pgfqpoint{0.481978in}{0.331635in}}{\pgfqpoint{9.300000in}{7.700000in}}%
\pgfusepath{clip}%
\pgfsetrectcap%
\pgfsetroundjoin%
\pgfsetlinewidth{1.505625pt}%
\definecolor{currentstroke}{rgb}{0.631373,0.788235,0.956863}%
\pgfsetstrokecolor{currentstroke}%
\pgfsetstrokeopacity{0.800000}%
\pgfsetdash{}{0pt}%
\pgfpathmoveto{\pgfqpoint{3.327053in}{1.432012in}}%
\pgfpathlineto{\pgfqpoint{5.711094in}{3.773662in}}%
\pgfusepath{stroke}%
\end{pgfscope}%
\begin{pgfscope}%
\pgfpathrectangle{\pgfqpoint{0.481978in}{0.331635in}}{\pgfqpoint{9.300000in}{7.700000in}}%
\pgfusepath{clip}%
\pgfsetrectcap%
\pgfsetroundjoin%
\pgfsetlinewidth{1.505625pt}%
\definecolor{currentstroke}{rgb}{0.631373,0.788235,0.956863}%
\pgfsetstrokecolor{currentstroke}%
\pgfsetstrokeopacity{0.800000}%
\pgfsetdash{}{0pt}%
\pgfpathmoveto{\pgfqpoint{5.992810in}{4.059757in}}%
\pgfpathlineto{\pgfqpoint{5.711094in}{3.773662in}}%
\pgfusepath{stroke}%
\end{pgfscope}%
\begin{pgfscope}%
\pgfpathrectangle{\pgfqpoint{0.481978in}{0.331635in}}{\pgfqpoint{9.300000in}{7.700000in}}%
\pgfusepath{clip}%
\pgfsetrectcap%
\pgfsetroundjoin%
\pgfsetlinewidth{1.505625pt}%
\definecolor{currentstroke}{rgb}{0.631373,0.788235,0.956863}%
\pgfsetstrokecolor{currentstroke}%
\pgfsetstrokeopacity{0.800000}%
\pgfsetdash{}{0pt}%
\pgfpathmoveto{\pgfqpoint{5.648838in}{3.554140in}}%
\pgfpathlineto{\pgfqpoint{5.711094in}{3.773662in}}%
\pgfusepath{stroke}%
\end{pgfscope}%
\begin{pgfscope}%
\pgfpathrectangle{\pgfqpoint{0.481978in}{0.331635in}}{\pgfqpoint{9.300000in}{7.700000in}}%
\pgfusepath{clip}%
\pgfsetrectcap%
\pgfsetroundjoin%
\pgfsetlinewidth{1.505625pt}%
\definecolor{currentstroke}{rgb}{0.631373,0.788235,0.956863}%
\pgfsetstrokecolor{currentstroke}%
\pgfsetstrokeopacity{0.800000}%
\pgfsetdash{}{0pt}%
\pgfpathmoveto{\pgfqpoint{6.635439in}{5.393842in}}%
\pgfpathlineto{\pgfqpoint{5.711094in}{3.773662in}}%
\pgfusepath{stroke}%
\end{pgfscope}%
\begin{pgfscope}%
\pgfpathrectangle{\pgfqpoint{0.481978in}{0.331635in}}{\pgfqpoint{9.300000in}{7.700000in}}%
\pgfusepath{clip}%
\pgfsetrectcap%
\pgfsetroundjoin%
\pgfsetlinewidth{1.505625pt}%
\definecolor{currentstroke}{rgb}{0.631373,0.788235,0.956863}%
\pgfsetstrokecolor{currentstroke}%
\pgfsetstrokeopacity{0.800000}%
\pgfsetdash{}{0pt}%
\pgfpathmoveto{\pgfqpoint{6.399047in}{2.071266in}}%
\pgfpathlineto{\pgfqpoint{5.711094in}{3.773662in}}%
\pgfusepath{stroke}%
\end{pgfscope}%
\begin{pgfscope}%
\pgfpathrectangle{\pgfqpoint{0.481978in}{0.331635in}}{\pgfqpoint{9.300000in}{7.700000in}}%
\pgfusepath{clip}%
\pgfsetrectcap%
\pgfsetroundjoin%
\pgfsetlinewidth{1.505625pt}%
\definecolor{currentstroke}{rgb}{0.631373,0.788235,0.956863}%
\pgfsetstrokecolor{currentstroke}%
\pgfsetstrokeopacity{0.800000}%
\pgfsetdash{}{0pt}%
\pgfpathmoveto{\pgfqpoint{2.913660in}{6.864512in}}%
\pgfpathlineto{\pgfqpoint{5.711094in}{3.773662in}}%
\pgfusepath{stroke}%
\end{pgfscope}%
\begin{pgfscope}%
\pgfpathrectangle{\pgfqpoint{0.481978in}{0.331635in}}{\pgfqpoint{9.300000in}{7.700000in}}%
\pgfusepath{clip}%
\pgfsetrectcap%
\pgfsetroundjoin%
\pgfsetlinewidth{1.505625pt}%
\definecolor{currentstroke}{rgb}{0.631373,0.788235,0.956863}%
\pgfsetstrokecolor{currentstroke}%
\pgfsetstrokeopacity{0.800000}%
\pgfsetdash{}{0pt}%
\pgfpathmoveto{\pgfqpoint{4.618112in}{4.376225in}}%
\pgfpathlineto{\pgfqpoint{5.711094in}{3.773662in}}%
\pgfusepath{stroke}%
\end{pgfscope}%
\begin{pgfscope}%
\pgfpathrectangle{\pgfqpoint{0.481978in}{0.331635in}}{\pgfqpoint{9.300000in}{7.700000in}}%
\pgfusepath{clip}%
\pgfsetrectcap%
\pgfsetroundjoin%
\pgfsetlinewidth{1.505625pt}%
\definecolor{currentstroke}{rgb}{0.631373,0.788235,0.956863}%
\pgfsetstrokecolor{currentstroke}%
\pgfsetstrokeopacity{0.800000}%
\pgfsetdash{}{0pt}%
\pgfpathmoveto{\pgfqpoint{7.244029in}{2.471548in}}%
\pgfpathlineto{\pgfqpoint{5.711094in}{3.773662in}}%
\pgfusepath{stroke}%
\end{pgfscope}%
\begin{pgfscope}%
\pgfpathrectangle{\pgfqpoint{0.481978in}{0.331635in}}{\pgfqpoint{9.300000in}{7.700000in}}%
\pgfusepath{clip}%
\pgfsetrectcap%
\pgfsetroundjoin%
\pgfsetlinewidth{1.505625pt}%
\definecolor{currentstroke}{rgb}{0.631373,0.788235,0.956863}%
\pgfsetstrokecolor{currentstroke}%
\pgfsetstrokeopacity{0.800000}%
\pgfsetdash{}{0pt}%
\pgfpathmoveto{\pgfqpoint{6.714656in}{1.567114in}}%
\pgfpathlineto{\pgfqpoint{5.711094in}{3.773662in}}%
\pgfusepath{stroke}%
\end{pgfscope}%
\begin{pgfscope}%
\pgfpathrectangle{\pgfqpoint{0.481978in}{0.331635in}}{\pgfqpoint{9.300000in}{7.700000in}}%
\pgfusepath{clip}%
\pgfsetrectcap%
\pgfsetroundjoin%
\pgfsetlinewidth{1.505625pt}%
\definecolor{currentstroke}{rgb}{0.631373,0.788235,0.956863}%
\pgfsetstrokecolor{currentstroke}%
\pgfsetstrokeopacity{0.800000}%
\pgfsetdash{}{0pt}%
\pgfpathmoveto{\pgfqpoint{3.591337in}{1.309324in}}%
\pgfpathlineto{\pgfqpoint{5.711094in}{3.773662in}}%
\pgfusepath{stroke}%
\end{pgfscope}%
\begin{pgfscope}%
\pgfpathrectangle{\pgfqpoint{0.481978in}{0.331635in}}{\pgfqpoint{9.300000in}{7.700000in}}%
\pgfusepath{clip}%
\pgfsetrectcap%
\pgfsetroundjoin%
\pgfsetlinewidth{1.505625pt}%
\definecolor{currentstroke}{rgb}{0.631373,0.788235,0.956863}%
\pgfsetstrokecolor{currentstroke}%
\pgfsetstrokeopacity{0.800000}%
\pgfsetdash{}{0pt}%
\pgfpathmoveto{\pgfqpoint{8.476836in}{5.273389in}}%
\pgfpathlineto{\pgfqpoint{5.711094in}{3.773662in}}%
\pgfusepath{stroke}%
\end{pgfscope}%
\begin{pgfscope}%
\pgfpathrectangle{\pgfqpoint{0.481978in}{0.331635in}}{\pgfqpoint{9.300000in}{7.700000in}}%
\pgfusepath{clip}%
\pgfsetrectcap%
\pgfsetroundjoin%
\pgfsetlinewidth{1.505625pt}%
\definecolor{currentstroke}{rgb}{0.631373,0.788235,0.956863}%
\pgfsetstrokecolor{currentstroke}%
\pgfsetstrokeopacity{0.800000}%
\pgfsetdash{}{0pt}%
\pgfpathmoveto{\pgfqpoint{6.886735in}{1.453604in}}%
\pgfpathlineto{\pgfqpoint{5.711094in}{3.773662in}}%
\pgfusepath{stroke}%
\end{pgfscope}%
\begin{pgfscope}%
\pgfpathrectangle{\pgfqpoint{0.481978in}{0.331635in}}{\pgfqpoint{9.300000in}{7.700000in}}%
\pgfusepath{clip}%
\pgfsetrectcap%
\pgfsetroundjoin%
\pgfsetlinewidth{1.505625pt}%
\definecolor{currentstroke}{rgb}{0.631373,0.788235,0.956863}%
\pgfsetstrokecolor{currentstroke}%
\pgfsetstrokeopacity{0.800000}%
\pgfsetdash{}{0pt}%
\pgfpathmoveto{\pgfqpoint{5.148908in}{1.672237in}}%
\pgfpathlineto{\pgfqpoint{5.711094in}{3.773662in}}%
\pgfusepath{stroke}%
\end{pgfscope}%
\begin{pgfscope}%
\pgfpathrectangle{\pgfqpoint{0.481978in}{0.331635in}}{\pgfqpoint{9.300000in}{7.700000in}}%
\pgfusepath{clip}%
\pgfsetrectcap%
\pgfsetroundjoin%
\pgfsetlinewidth{1.505625pt}%
\definecolor{currentstroke}{rgb}{0.631373,0.788235,0.956863}%
\pgfsetstrokecolor{currentstroke}%
\pgfsetstrokeopacity{0.800000}%
\pgfsetdash{}{0pt}%
\pgfpathmoveto{\pgfqpoint{5.442394in}{6.854337in}}%
\pgfpathlineto{\pgfqpoint{5.711094in}{3.773662in}}%
\pgfusepath{stroke}%
\end{pgfscope}%
\begin{pgfscope}%
\pgfpathrectangle{\pgfqpoint{0.481978in}{0.331635in}}{\pgfqpoint{9.300000in}{7.700000in}}%
\pgfusepath{clip}%
\pgfsetrectcap%
\pgfsetroundjoin%
\pgfsetlinewidth{1.505625pt}%
\definecolor{currentstroke}{rgb}{0.631373,0.788235,0.956863}%
\pgfsetstrokecolor{currentstroke}%
\pgfsetstrokeopacity{0.800000}%
\pgfsetdash{}{0pt}%
\pgfpathmoveto{\pgfqpoint{6.133252in}{3.376659in}}%
\pgfpathlineto{\pgfqpoint{5.711094in}{3.773662in}}%
\pgfusepath{stroke}%
\end{pgfscope}%
\begin{pgfscope}%
\pgfpathrectangle{\pgfqpoint{0.481978in}{0.331635in}}{\pgfqpoint{9.300000in}{7.700000in}}%
\pgfusepath{clip}%
\pgfsetrectcap%
\pgfsetroundjoin%
\pgfsetlinewidth{1.505625pt}%
\definecolor{currentstroke}{rgb}{0.631373,0.788235,0.956863}%
\pgfsetstrokecolor{currentstroke}%
\pgfsetstrokeopacity{0.800000}%
\pgfsetdash{}{0pt}%
\pgfpathmoveto{\pgfqpoint{6.565040in}{5.067481in}}%
\pgfpathlineto{\pgfqpoint{5.711094in}{3.773662in}}%
\pgfusepath{stroke}%
\end{pgfscope}%
\begin{pgfscope}%
\pgfpathrectangle{\pgfqpoint{0.481978in}{0.331635in}}{\pgfqpoint{9.300000in}{7.700000in}}%
\pgfusepath{clip}%
\pgfsetrectcap%
\pgfsetroundjoin%
\pgfsetlinewidth{1.505625pt}%
\definecolor{currentstroke}{rgb}{0.631373,0.788235,0.956863}%
\pgfsetstrokecolor{currentstroke}%
\pgfsetstrokeopacity{0.800000}%
\pgfsetdash{}{0pt}%
\pgfpathmoveto{\pgfqpoint{6.488247in}{4.439323in}}%
\pgfpathlineto{\pgfqpoint{5.711094in}{3.773662in}}%
\pgfusepath{stroke}%
\end{pgfscope}%
\begin{pgfscope}%
\pgfpathrectangle{\pgfqpoint{0.481978in}{0.331635in}}{\pgfqpoint{9.300000in}{7.700000in}}%
\pgfusepath{clip}%
\pgfsetrectcap%
\pgfsetroundjoin%
\pgfsetlinewidth{1.505625pt}%
\definecolor{currentstroke}{rgb}{0.631373,0.788235,0.956863}%
\pgfsetstrokecolor{currentstroke}%
\pgfsetstrokeopacity{0.800000}%
\pgfsetdash{}{0pt}%
\pgfpathmoveto{\pgfqpoint{6.070621in}{5.046173in}}%
\pgfpathlineto{\pgfqpoint{5.711094in}{3.773662in}}%
\pgfusepath{stroke}%
\end{pgfscope}%
\begin{pgfscope}%
\pgfpathrectangle{\pgfqpoint{0.481978in}{0.331635in}}{\pgfqpoint{9.300000in}{7.700000in}}%
\pgfusepath{clip}%
\pgfsetrectcap%
\pgfsetroundjoin%
\pgfsetlinewidth{1.505625pt}%
\definecolor{currentstroke}{rgb}{0.631373,0.788235,0.956863}%
\pgfsetstrokecolor{currentstroke}%
\pgfsetstrokeopacity{0.800000}%
\pgfsetdash{}{0pt}%
\pgfpathmoveto{\pgfqpoint{4.466261in}{1.325503in}}%
\pgfpathlineto{\pgfqpoint{5.711094in}{3.773662in}}%
\pgfusepath{stroke}%
\end{pgfscope}%
\begin{pgfscope}%
\pgfpathrectangle{\pgfqpoint{0.481978in}{0.331635in}}{\pgfqpoint{9.300000in}{7.700000in}}%
\pgfusepath{clip}%
\pgfsetrectcap%
\pgfsetroundjoin%
\pgfsetlinewidth{1.505625pt}%
\definecolor{currentstroke}{rgb}{0.631373,0.788235,0.956863}%
\pgfsetstrokecolor{currentstroke}%
\pgfsetstrokeopacity{0.800000}%
\pgfsetdash{}{0pt}%
\pgfpathmoveto{\pgfqpoint{3.012102in}{4.653388in}}%
\pgfpathlineto{\pgfqpoint{5.711094in}{3.773662in}}%
\pgfusepath{stroke}%
\end{pgfscope}%
\begin{pgfscope}%
\pgfpathrectangle{\pgfqpoint{0.481978in}{0.331635in}}{\pgfqpoint{9.300000in}{7.700000in}}%
\pgfusepath{clip}%
\pgfsetrectcap%
\pgfsetroundjoin%
\pgfsetlinewidth{1.505625pt}%
\definecolor{currentstroke}{rgb}{0.631373,0.788235,0.956863}%
\pgfsetstrokecolor{currentstroke}%
\pgfsetstrokeopacity{0.800000}%
\pgfsetdash{}{0pt}%
\pgfpathmoveto{\pgfqpoint{6.293539in}{3.626500in}}%
\pgfpathlineto{\pgfqpoint{5.711094in}{3.773662in}}%
\pgfusepath{stroke}%
\end{pgfscope}%
\begin{pgfscope}%
\pgfpathrectangle{\pgfqpoint{0.481978in}{0.331635in}}{\pgfqpoint{9.300000in}{7.700000in}}%
\pgfusepath{clip}%
\pgfsetrectcap%
\pgfsetroundjoin%
\pgfsetlinewidth{1.505625pt}%
\definecolor{currentstroke}{rgb}{0.631373,0.788235,0.956863}%
\pgfsetstrokecolor{currentstroke}%
\pgfsetstrokeopacity{0.800000}%
\pgfsetdash{}{0pt}%
\pgfpathmoveto{\pgfqpoint{2.026106in}{1.649106in}}%
\pgfpathlineto{\pgfqpoint{5.711094in}{3.773662in}}%
\pgfusepath{stroke}%
\end{pgfscope}%
\begin{pgfscope}%
\pgfpathrectangle{\pgfqpoint{0.481978in}{0.331635in}}{\pgfqpoint{9.300000in}{7.700000in}}%
\pgfusepath{clip}%
\pgfsetrectcap%
\pgfsetroundjoin%
\pgfsetlinewidth{1.505625pt}%
\definecolor{currentstroke}{rgb}{1.000000,0.705882,0.509804}%
\pgfsetstrokecolor{currentstroke}%
\pgfsetstrokeopacity{0.800000}%
\pgfsetdash{}{0pt}%
\pgfpathmoveto{\pgfqpoint{4.850856in}{2.854692in}}%
\pgfpathlineto{\pgfqpoint{3.502867in}{4.072531in}}%
\pgfusepath{stroke}%
\end{pgfscope}%
\begin{pgfscope}%
\pgfpathrectangle{\pgfqpoint{0.481978in}{0.331635in}}{\pgfqpoint{9.300000in}{7.700000in}}%
\pgfusepath{clip}%
\pgfsetrectcap%
\pgfsetroundjoin%
\pgfsetlinewidth{1.505625pt}%
\definecolor{currentstroke}{rgb}{1.000000,0.705882,0.509804}%
\pgfsetstrokecolor{currentstroke}%
\pgfsetstrokeopacity{0.800000}%
\pgfsetdash{}{0pt}%
\pgfpathmoveto{\pgfqpoint{2.916721in}{3.120345in}}%
\pgfpathlineto{\pgfqpoint{3.502867in}{4.072531in}}%
\pgfusepath{stroke}%
\end{pgfscope}%
\begin{pgfscope}%
\pgfpathrectangle{\pgfqpoint{0.481978in}{0.331635in}}{\pgfqpoint{9.300000in}{7.700000in}}%
\pgfusepath{clip}%
\pgfsetrectcap%
\pgfsetroundjoin%
\pgfsetlinewidth{1.505625pt}%
\definecolor{currentstroke}{rgb}{1.000000,0.705882,0.509804}%
\pgfsetstrokecolor{currentstroke}%
\pgfsetstrokeopacity{0.800000}%
\pgfsetdash{}{0pt}%
\pgfpathmoveto{\pgfqpoint{3.594827in}{5.702842in}}%
\pgfpathlineto{\pgfqpoint{3.502867in}{4.072531in}}%
\pgfusepath{stroke}%
\end{pgfscope}%
\begin{pgfscope}%
\pgfpathrectangle{\pgfqpoint{0.481978in}{0.331635in}}{\pgfqpoint{9.300000in}{7.700000in}}%
\pgfusepath{clip}%
\pgfsetrectcap%
\pgfsetroundjoin%
\pgfsetlinewidth{1.505625pt}%
\definecolor{currentstroke}{rgb}{1.000000,0.705882,0.509804}%
\pgfsetstrokecolor{currentstroke}%
\pgfsetstrokeopacity{0.800000}%
\pgfsetdash{}{0pt}%
\pgfpathmoveto{\pgfqpoint{3.147191in}{3.809748in}}%
\pgfpathlineto{\pgfqpoint{3.502867in}{4.072531in}}%
\pgfusepath{stroke}%
\end{pgfscope}%
\begin{pgfscope}%
\pgfpathrectangle{\pgfqpoint{0.481978in}{0.331635in}}{\pgfqpoint{9.300000in}{7.700000in}}%
\pgfusepath{clip}%
\pgfsetrectcap%
\pgfsetroundjoin%
\pgfsetlinewidth{1.505625pt}%
\definecolor{currentstroke}{rgb}{1.000000,0.705882,0.509804}%
\pgfsetstrokecolor{currentstroke}%
\pgfsetstrokeopacity{0.800000}%
\pgfsetdash{}{0pt}%
\pgfpathmoveto{\pgfqpoint{3.148403in}{2.315312in}}%
\pgfpathlineto{\pgfqpoint{3.502867in}{4.072531in}}%
\pgfusepath{stroke}%
\end{pgfscope}%
\begin{pgfscope}%
\pgfpathrectangle{\pgfqpoint{0.481978in}{0.331635in}}{\pgfqpoint{9.300000in}{7.700000in}}%
\pgfusepath{clip}%
\pgfsetrectcap%
\pgfsetroundjoin%
\pgfsetlinewidth{1.505625pt}%
\definecolor{currentstroke}{rgb}{1.000000,0.705882,0.509804}%
\pgfsetstrokecolor{currentstroke}%
\pgfsetstrokeopacity{0.800000}%
\pgfsetdash{}{0pt}%
\pgfpathmoveto{\pgfqpoint{3.951193in}{0.954754in}}%
\pgfpathlineto{\pgfqpoint{3.502867in}{4.072531in}}%
\pgfusepath{stroke}%
\end{pgfscope}%
\begin{pgfscope}%
\pgfpathrectangle{\pgfqpoint{0.481978in}{0.331635in}}{\pgfqpoint{9.300000in}{7.700000in}}%
\pgfusepath{clip}%
\pgfsetrectcap%
\pgfsetroundjoin%
\pgfsetlinewidth{1.505625pt}%
\definecolor{currentstroke}{rgb}{1.000000,0.705882,0.509804}%
\pgfsetstrokecolor{currentstroke}%
\pgfsetstrokeopacity{0.800000}%
\pgfsetdash{}{0pt}%
\pgfpathmoveto{\pgfqpoint{5.646419in}{4.320008in}}%
\pgfpathlineto{\pgfqpoint{3.502867in}{4.072531in}}%
\pgfusepath{stroke}%
\end{pgfscope}%
\begin{pgfscope}%
\pgfpathrectangle{\pgfqpoint{0.481978in}{0.331635in}}{\pgfqpoint{9.300000in}{7.700000in}}%
\pgfusepath{clip}%
\pgfsetrectcap%
\pgfsetroundjoin%
\pgfsetlinewidth{1.505625pt}%
\definecolor{currentstroke}{rgb}{1.000000,0.705882,0.509804}%
\pgfsetstrokecolor{currentstroke}%
\pgfsetstrokeopacity{0.800000}%
\pgfsetdash{}{0pt}%
\pgfpathmoveto{\pgfqpoint{2.847205in}{4.348681in}}%
\pgfpathlineto{\pgfqpoint{3.502867in}{4.072531in}}%
\pgfusepath{stroke}%
\end{pgfscope}%
\begin{pgfscope}%
\pgfpathrectangle{\pgfqpoint{0.481978in}{0.331635in}}{\pgfqpoint{9.300000in}{7.700000in}}%
\pgfusepath{clip}%
\pgfsetrectcap%
\pgfsetroundjoin%
\pgfsetlinewidth{1.505625pt}%
\definecolor{currentstroke}{rgb}{1.000000,0.705882,0.509804}%
\pgfsetstrokecolor{currentstroke}%
\pgfsetstrokeopacity{0.800000}%
\pgfsetdash{}{0pt}%
\pgfpathmoveto{\pgfqpoint{3.926575in}{2.377543in}}%
\pgfpathlineto{\pgfqpoint{3.502867in}{4.072531in}}%
\pgfusepath{stroke}%
\end{pgfscope}%
\begin{pgfscope}%
\pgfpathrectangle{\pgfqpoint{0.481978in}{0.331635in}}{\pgfqpoint{9.300000in}{7.700000in}}%
\pgfusepath{clip}%
\pgfsetrectcap%
\pgfsetroundjoin%
\pgfsetlinewidth{1.505625pt}%
\definecolor{currentstroke}{rgb}{1.000000,0.705882,0.509804}%
\pgfsetstrokecolor{currentstroke}%
\pgfsetstrokeopacity{0.800000}%
\pgfsetdash{}{0pt}%
\pgfpathmoveto{\pgfqpoint{2.406478in}{5.903770in}}%
\pgfpathlineto{\pgfqpoint{3.502867in}{4.072531in}}%
\pgfusepath{stroke}%
\end{pgfscope}%
\begin{pgfscope}%
\pgfpathrectangle{\pgfqpoint{0.481978in}{0.331635in}}{\pgfqpoint{9.300000in}{7.700000in}}%
\pgfusepath{clip}%
\pgfsetrectcap%
\pgfsetroundjoin%
\pgfsetlinewidth{1.505625pt}%
\definecolor{currentstroke}{rgb}{1.000000,0.705882,0.509804}%
\pgfsetstrokecolor{currentstroke}%
\pgfsetstrokeopacity{0.800000}%
\pgfsetdash{}{0pt}%
\pgfpathmoveto{\pgfqpoint{4.065253in}{2.372474in}}%
\pgfpathlineto{\pgfqpoint{3.502867in}{4.072531in}}%
\pgfusepath{stroke}%
\end{pgfscope}%
\begin{pgfscope}%
\pgfpathrectangle{\pgfqpoint{0.481978in}{0.331635in}}{\pgfqpoint{9.300000in}{7.700000in}}%
\pgfusepath{clip}%
\pgfsetrectcap%
\pgfsetroundjoin%
\pgfsetlinewidth{1.505625pt}%
\definecolor{currentstroke}{rgb}{1.000000,0.705882,0.509804}%
\pgfsetstrokecolor{currentstroke}%
\pgfsetstrokeopacity{0.800000}%
\pgfsetdash{}{0pt}%
\pgfpathmoveto{\pgfqpoint{4.470303in}{5.206116in}}%
\pgfpathlineto{\pgfqpoint{3.502867in}{4.072531in}}%
\pgfusepath{stroke}%
\end{pgfscope}%
\begin{pgfscope}%
\pgfpathrectangle{\pgfqpoint{0.481978in}{0.331635in}}{\pgfqpoint{9.300000in}{7.700000in}}%
\pgfusepath{clip}%
\pgfsetrectcap%
\pgfsetroundjoin%
\pgfsetlinewidth{1.505625pt}%
\definecolor{currentstroke}{rgb}{1.000000,0.705882,0.509804}%
\pgfsetstrokecolor{currentstroke}%
\pgfsetstrokeopacity{0.800000}%
\pgfsetdash{}{0pt}%
\pgfpathmoveto{\pgfqpoint{2.649493in}{4.102821in}}%
\pgfpathlineto{\pgfqpoint{3.502867in}{4.072531in}}%
\pgfusepath{stroke}%
\end{pgfscope}%
\begin{pgfscope}%
\pgfpathrectangle{\pgfqpoint{0.481978in}{0.331635in}}{\pgfqpoint{9.300000in}{7.700000in}}%
\pgfusepath{clip}%
\pgfsetrectcap%
\pgfsetroundjoin%
\pgfsetlinewidth{1.505625pt}%
\definecolor{currentstroke}{rgb}{1.000000,0.705882,0.509804}%
\pgfsetstrokecolor{currentstroke}%
\pgfsetstrokeopacity{0.800000}%
\pgfsetdash{}{0pt}%
\pgfpathmoveto{\pgfqpoint{2.136818in}{3.314846in}}%
\pgfpathlineto{\pgfqpoint{3.502867in}{4.072531in}}%
\pgfusepath{stroke}%
\end{pgfscope}%
\begin{pgfscope}%
\pgfpathrectangle{\pgfqpoint{0.481978in}{0.331635in}}{\pgfqpoint{9.300000in}{7.700000in}}%
\pgfusepath{clip}%
\pgfsetrectcap%
\pgfsetroundjoin%
\pgfsetlinewidth{1.505625pt}%
\definecolor{currentstroke}{rgb}{1.000000,0.705882,0.509804}%
\pgfsetstrokecolor{currentstroke}%
\pgfsetstrokeopacity{0.800000}%
\pgfsetdash{}{0pt}%
\pgfpathmoveto{\pgfqpoint{3.304744in}{3.445980in}}%
\pgfpathlineto{\pgfqpoint{3.502867in}{4.072531in}}%
\pgfusepath{stroke}%
\end{pgfscope}%
\begin{pgfscope}%
\pgfpathrectangle{\pgfqpoint{0.481978in}{0.331635in}}{\pgfqpoint{9.300000in}{7.700000in}}%
\pgfusepath{clip}%
\pgfsetrectcap%
\pgfsetroundjoin%
\pgfsetlinewidth{1.505625pt}%
\definecolor{currentstroke}{rgb}{1.000000,0.705882,0.509804}%
\pgfsetstrokecolor{currentstroke}%
\pgfsetstrokeopacity{0.800000}%
\pgfsetdash{}{0pt}%
\pgfpathmoveto{\pgfqpoint{1.284269in}{4.746461in}}%
\pgfpathlineto{\pgfqpoint{3.502867in}{4.072531in}}%
\pgfusepath{stroke}%
\end{pgfscope}%
\begin{pgfscope}%
\pgfpathrectangle{\pgfqpoint{0.481978in}{0.331635in}}{\pgfqpoint{9.300000in}{7.700000in}}%
\pgfusepath{clip}%
\pgfsetrectcap%
\pgfsetroundjoin%
\pgfsetlinewidth{1.505625pt}%
\definecolor{currentstroke}{rgb}{1.000000,0.705882,0.509804}%
\pgfsetstrokecolor{currentstroke}%
\pgfsetstrokeopacity{0.800000}%
\pgfsetdash{}{0pt}%
\pgfpathmoveto{\pgfqpoint{3.921640in}{6.537801in}}%
\pgfpathlineto{\pgfqpoint{3.502867in}{4.072531in}}%
\pgfusepath{stroke}%
\end{pgfscope}%
\begin{pgfscope}%
\pgfpathrectangle{\pgfqpoint{0.481978in}{0.331635in}}{\pgfqpoint{9.300000in}{7.700000in}}%
\pgfusepath{clip}%
\pgfsetrectcap%
\pgfsetroundjoin%
\pgfsetlinewidth{1.505625pt}%
\definecolor{currentstroke}{rgb}{1.000000,0.705882,0.509804}%
\pgfsetstrokecolor{currentstroke}%
\pgfsetstrokeopacity{0.800000}%
\pgfsetdash{}{0pt}%
\pgfpathmoveto{\pgfqpoint{4.551669in}{3.889138in}}%
\pgfpathlineto{\pgfqpoint{3.502867in}{4.072531in}}%
\pgfusepath{stroke}%
\end{pgfscope}%
\begin{pgfscope}%
\pgfpathrectangle{\pgfqpoint{0.481978in}{0.331635in}}{\pgfqpoint{9.300000in}{7.700000in}}%
\pgfusepath{clip}%
\pgfsetrectcap%
\pgfsetroundjoin%
\pgfsetlinewidth{1.505625pt}%
\definecolor{currentstroke}{rgb}{1.000000,0.705882,0.509804}%
\pgfsetstrokecolor{currentstroke}%
\pgfsetstrokeopacity{0.800000}%
\pgfsetdash{}{0pt}%
\pgfpathmoveto{\pgfqpoint{2.003717in}{3.814665in}}%
\pgfpathlineto{\pgfqpoint{3.502867in}{4.072531in}}%
\pgfusepath{stroke}%
\end{pgfscope}%
\begin{pgfscope}%
\pgfpathrectangle{\pgfqpoint{0.481978in}{0.331635in}}{\pgfqpoint{9.300000in}{7.700000in}}%
\pgfusepath{clip}%
\pgfsetrectcap%
\pgfsetroundjoin%
\pgfsetlinewidth{1.505625pt}%
\definecolor{currentstroke}{rgb}{1.000000,0.705882,0.509804}%
\pgfsetstrokecolor{currentstroke}%
\pgfsetstrokeopacity{0.800000}%
\pgfsetdash{}{0pt}%
\pgfpathmoveto{\pgfqpoint{2.821317in}{3.728489in}}%
\pgfpathlineto{\pgfqpoint{3.502867in}{4.072531in}}%
\pgfusepath{stroke}%
\end{pgfscope}%
\begin{pgfscope}%
\pgfpathrectangle{\pgfqpoint{0.481978in}{0.331635in}}{\pgfqpoint{9.300000in}{7.700000in}}%
\pgfusepath{clip}%
\pgfsetrectcap%
\pgfsetroundjoin%
\pgfsetlinewidth{1.505625pt}%
\definecolor{currentstroke}{rgb}{1.000000,0.705882,0.509804}%
\pgfsetstrokecolor{currentstroke}%
\pgfsetstrokeopacity{0.800000}%
\pgfsetdash{}{0pt}%
\pgfpathmoveto{\pgfqpoint{2.335285in}{3.880841in}}%
\pgfpathlineto{\pgfqpoint{3.502867in}{4.072531in}}%
\pgfusepath{stroke}%
\end{pgfscope}%
\begin{pgfscope}%
\pgfpathrectangle{\pgfqpoint{0.481978in}{0.331635in}}{\pgfqpoint{9.300000in}{7.700000in}}%
\pgfusepath{clip}%
\pgfsetrectcap%
\pgfsetroundjoin%
\pgfsetlinewidth{1.505625pt}%
\definecolor{currentstroke}{rgb}{1.000000,0.705882,0.509804}%
\pgfsetstrokecolor{currentstroke}%
\pgfsetstrokeopacity{0.800000}%
\pgfsetdash{}{0pt}%
\pgfpathmoveto{\pgfqpoint{4.485108in}{3.865983in}}%
\pgfpathlineto{\pgfqpoint{3.502867in}{4.072531in}}%
\pgfusepath{stroke}%
\end{pgfscope}%
\begin{pgfscope}%
\pgfpathrectangle{\pgfqpoint{0.481978in}{0.331635in}}{\pgfqpoint{9.300000in}{7.700000in}}%
\pgfusepath{clip}%
\pgfsetrectcap%
\pgfsetroundjoin%
\pgfsetlinewidth{1.505625pt}%
\definecolor{currentstroke}{rgb}{1.000000,0.705882,0.509804}%
\pgfsetstrokecolor{currentstroke}%
\pgfsetstrokeopacity{0.800000}%
\pgfsetdash{}{0pt}%
\pgfpathmoveto{\pgfqpoint{1.703134in}{4.289761in}}%
\pgfpathlineto{\pgfqpoint{3.502867in}{4.072531in}}%
\pgfusepath{stroke}%
\end{pgfscope}%
\begin{pgfscope}%
\pgfpathrectangle{\pgfqpoint{0.481978in}{0.331635in}}{\pgfqpoint{9.300000in}{7.700000in}}%
\pgfusepath{clip}%
\pgfsetrectcap%
\pgfsetroundjoin%
\pgfsetlinewidth{1.505625pt}%
\definecolor{currentstroke}{rgb}{1.000000,0.705882,0.509804}%
\pgfsetstrokecolor{currentstroke}%
\pgfsetstrokeopacity{0.800000}%
\pgfsetdash{}{0pt}%
\pgfpathmoveto{\pgfqpoint{4.082596in}{4.780623in}}%
\pgfpathlineto{\pgfqpoint{3.502867in}{4.072531in}}%
\pgfusepath{stroke}%
\end{pgfscope}%
\begin{pgfscope}%
\pgfpathrectangle{\pgfqpoint{0.481978in}{0.331635in}}{\pgfqpoint{9.300000in}{7.700000in}}%
\pgfusepath{clip}%
\pgfsetrectcap%
\pgfsetroundjoin%
\pgfsetlinewidth{1.505625pt}%
\definecolor{currentstroke}{rgb}{1.000000,0.705882,0.509804}%
\pgfsetstrokecolor{currentstroke}%
\pgfsetstrokeopacity{0.800000}%
\pgfsetdash{}{0pt}%
\pgfpathmoveto{\pgfqpoint{3.376694in}{3.830492in}}%
\pgfpathlineto{\pgfqpoint{3.502867in}{4.072531in}}%
\pgfusepath{stroke}%
\end{pgfscope}%
\begin{pgfscope}%
\pgfpathrectangle{\pgfqpoint{0.481978in}{0.331635in}}{\pgfqpoint{9.300000in}{7.700000in}}%
\pgfusepath{clip}%
\pgfsetrectcap%
\pgfsetroundjoin%
\pgfsetlinewidth{1.505625pt}%
\definecolor{currentstroke}{rgb}{1.000000,0.705882,0.509804}%
\pgfsetstrokecolor{currentstroke}%
\pgfsetstrokeopacity{0.800000}%
\pgfsetdash{}{0pt}%
\pgfpathmoveto{\pgfqpoint{4.094855in}{5.282243in}}%
\pgfpathlineto{\pgfqpoint{3.502867in}{4.072531in}}%
\pgfusepath{stroke}%
\end{pgfscope}%
\begin{pgfscope}%
\pgfpathrectangle{\pgfqpoint{0.481978in}{0.331635in}}{\pgfqpoint{9.300000in}{7.700000in}}%
\pgfusepath{clip}%
\pgfsetrectcap%
\pgfsetroundjoin%
\pgfsetlinewidth{1.505625pt}%
\definecolor{currentstroke}{rgb}{1.000000,0.705882,0.509804}%
\pgfsetstrokecolor{currentstroke}%
\pgfsetstrokeopacity{0.800000}%
\pgfsetdash{}{0pt}%
\pgfpathmoveto{\pgfqpoint{4.055791in}{5.348899in}}%
\pgfpathlineto{\pgfqpoint{3.502867in}{4.072531in}}%
\pgfusepath{stroke}%
\end{pgfscope}%
\begin{pgfscope}%
\pgfpathrectangle{\pgfqpoint{0.481978in}{0.331635in}}{\pgfqpoint{9.300000in}{7.700000in}}%
\pgfusepath{clip}%
\pgfsetrectcap%
\pgfsetroundjoin%
\pgfsetlinewidth{1.505625pt}%
\definecolor{currentstroke}{rgb}{1.000000,0.705882,0.509804}%
\pgfsetstrokecolor{currentstroke}%
\pgfsetstrokeopacity{0.800000}%
\pgfsetdash{}{0pt}%
\pgfpathmoveto{\pgfqpoint{4.975557in}{3.385448in}}%
\pgfpathlineto{\pgfqpoint{3.502867in}{4.072531in}}%
\pgfusepath{stroke}%
\end{pgfscope}%
\begin{pgfscope}%
\pgfpathrectangle{\pgfqpoint{0.481978in}{0.331635in}}{\pgfqpoint{9.300000in}{7.700000in}}%
\pgfusepath{clip}%
\pgfsetrectcap%
\pgfsetroundjoin%
\pgfsetlinewidth{1.505625pt}%
\definecolor{currentstroke}{rgb}{1.000000,0.705882,0.509804}%
\pgfsetstrokecolor{currentstroke}%
\pgfsetstrokeopacity{0.800000}%
\pgfsetdash{}{0pt}%
\pgfpathmoveto{\pgfqpoint{2.986906in}{3.719025in}}%
\pgfpathlineto{\pgfqpoint{3.502867in}{4.072531in}}%
\pgfusepath{stroke}%
\end{pgfscope}%
\begin{pgfscope}%
\pgfpathrectangle{\pgfqpoint{0.481978in}{0.331635in}}{\pgfqpoint{9.300000in}{7.700000in}}%
\pgfusepath{clip}%
\pgfsetrectcap%
\pgfsetroundjoin%
\pgfsetlinewidth{1.505625pt}%
\definecolor{currentstroke}{rgb}{1.000000,0.705882,0.509804}%
\pgfsetstrokecolor{currentstroke}%
\pgfsetstrokeopacity{0.800000}%
\pgfsetdash{}{0pt}%
\pgfpathmoveto{\pgfqpoint{3.977136in}{2.715901in}}%
\pgfpathlineto{\pgfqpoint{3.502867in}{4.072531in}}%
\pgfusepath{stroke}%
\end{pgfscope}%
\begin{pgfscope}%
\pgfpathrectangle{\pgfqpoint{0.481978in}{0.331635in}}{\pgfqpoint{9.300000in}{7.700000in}}%
\pgfusepath{clip}%
\pgfsetrectcap%
\pgfsetroundjoin%
\pgfsetlinewidth{1.505625pt}%
\definecolor{currentstroke}{rgb}{1.000000,0.705882,0.509804}%
\pgfsetstrokecolor{currentstroke}%
\pgfsetstrokeopacity{0.800000}%
\pgfsetdash{}{0pt}%
\pgfpathmoveto{\pgfqpoint{3.838833in}{6.961369in}}%
\pgfpathlineto{\pgfqpoint{3.502867in}{4.072531in}}%
\pgfusepath{stroke}%
\end{pgfscope}%
\begin{pgfscope}%
\pgfpathrectangle{\pgfqpoint{0.481978in}{0.331635in}}{\pgfqpoint{9.300000in}{7.700000in}}%
\pgfusepath{clip}%
\pgfsetrectcap%
\pgfsetroundjoin%
\pgfsetlinewidth{1.505625pt}%
\definecolor{currentstroke}{rgb}{1.000000,0.705882,0.509804}%
\pgfsetstrokecolor{currentstroke}%
\pgfsetstrokeopacity{0.800000}%
\pgfsetdash{}{0pt}%
\pgfpathmoveto{\pgfqpoint{2.868567in}{3.813763in}}%
\pgfpathlineto{\pgfqpoint{3.502867in}{4.072531in}}%
\pgfusepath{stroke}%
\end{pgfscope}%
\begin{pgfscope}%
\pgfpathrectangle{\pgfqpoint{0.481978in}{0.331635in}}{\pgfqpoint{9.300000in}{7.700000in}}%
\pgfusepath{clip}%
\pgfsetrectcap%
\pgfsetroundjoin%
\pgfsetlinewidth{1.505625pt}%
\definecolor{currentstroke}{rgb}{1.000000,0.705882,0.509804}%
\pgfsetstrokecolor{currentstroke}%
\pgfsetstrokeopacity{0.800000}%
\pgfsetdash{}{0pt}%
\pgfpathmoveto{\pgfqpoint{5.264259in}{4.933464in}}%
\pgfpathlineto{\pgfqpoint{3.502867in}{4.072531in}}%
\pgfusepath{stroke}%
\end{pgfscope}%
\begin{pgfscope}%
\pgfpathrectangle{\pgfqpoint{0.481978in}{0.331635in}}{\pgfqpoint{9.300000in}{7.700000in}}%
\pgfusepath{clip}%
\pgfsetrectcap%
\pgfsetroundjoin%
\pgfsetlinewidth{1.505625pt}%
\definecolor{currentstroke}{rgb}{1.000000,0.705882,0.509804}%
\pgfsetstrokecolor{currentstroke}%
\pgfsetstrokeopacity{0.800000}%
\pgfsetdash{}{0pt}%
\pgfpathmoveto{\pgfqpoint{3.655520in}{5.762264in}}%
\pgfpathlineto{\pgfqpoint{3.502867in}{4.072531in}}%
\pgfusepath{stroke}%
\end{pgfscope}%
\begin{pgfscope}%
\pgfpathrectangle{\pgfqpoint{0.481978in}{0.331635in}}{\pgfqpoint{9.300000in}{7.700000in}}%
\pgfusepath{clip}%
\pgfsetrectcap%
\pgfsetroundjoin%
\pgfsetlinewidth{1.505625pt}%
\definecolor{currentstroke}{rgb}{1.000000,0.705882,0.509804}%
\pgfsetstrokecolor{currentstroke}%
\pgfsetstrokeopacity{0.800000}%
\pgfsetdash{}{0pt}%
\pgfpathmoveto{\pgfqpoint{1.251672in}{4.064003in}}%
\pgfpathlineto{\pgfqpoint{3.502867in}{4.072531in}}%
\pgfusepath{stroke}%
\end{pgfscope}%
\begin{pgfscope}%
\pgfpathrectangle{\pgfqpoint{0.481978in}{0.331635in}}{\pgfqpoint{9.300000in}{7.700000in}}%
\pgfusepath{clip}%
\pgfsetrectcap%
\pgfsetroundjoin%
\pgfsetlinewidth{1.505625pt}%
\definecolor{currentstroke}{rgb}{1.000000,0.705882,0.509804}%
\pgfsetstrokecolor{currentstroke}%
\pgfsetstrokeopacity{0.800000}%
\pgfsetdash{}{0pt}%
\pgfpathmoveto{\pgfqpoint{1.875951in}{4.198061in}}%
\pgfpathlineto{\pgfqpoint{3.502867in}{4.072531in}}%
\pgfusepath{stroke}%
\end{pgfscope}%
\begin{pgfscope}%
\pgfpathrectangle{\pgfqpoint{0.481978in}{0.331635in}}{\pgfqpoint{9.300000in}{7.700000in}}%
\pgfusepath{clip}%
\pgfsetrectcap%
\pgfsetroundjoin%
\pgfsetlinewidth{1.505625pt}%
\definecolor{currentstroke}{rgb}{1.000000,0.705882,0.509804}%
\pgfsetstrokecolor{currentstroke}%
\pgfsetstrokeopacity{0.800000}%
\pgfsetdash{}{0pt}%
\pgfpathmoveto{\pgfqpoint{2.496147in}{3.990985in}}%
\pgfpathlineto{\pgfqpoint{3.502867in}{4.072531in}}%
\pgfusepath{stroke}%
\end{pgfscope}%
\begin{pgfscope}%
\pgfpathrectangle{\pgfqpoint{0.481978in}{0.331635in}}{\pgfqpoint{9.300000in}{7.700000in}}%
\pgfusepath{clip}%
\pgfsetrectcap%
\pgfsetroundjoin%
\pgfsetlinewidth{1.505625pt}%
\definecolor{currentstroke}{rgb}{1.000000,0.705882,0.509804}%
\pgfsetstrokecolor{currentstroke}%
\pgfsetstrokeopacity{0.800000}%
\pgfsetdash{}{0pt}%
\pgfpathmoveto{\pgfqpoint{3.452404in}{6.280804in}}%
\pgfpathlineto{\pgfqpoint{3.502867in}{4.072531in}}%
\pgfusepath{stroke}%
\end{pgfscope}%
\begin{pgfscope}%
\pgfpathrectangle{\pgfqpoint{0.481978in}{0.331635in}}{\pgfqpoint{9.300000in}{7.700000in}}%
\pgfusepath{clip}%
\pgfsetrectcap%
\pgfsetroundjoin%
\pgfsetlinewidth{1.505625pt}%
\definecolor{currentstroke}{rgb}{1.000000,0.705882,0.509804}%
\pgfsetstrokecolor{currentstroke}%
\pgfsetstrokeopacity{0.800000}%
\pgfsetdash{}{0pt}%
\pgfpathmoveto{\pgfqpoint{1.475053in}{4.585541in}}%
\pgfpathlineto{\pgfqpoint{3.502867in}{4.072531in}}%
\pgfusepath{stroke}%
\end{pgfscope}%
\begin{pgfscope}%
\pgfpathrectangle{\pgfqpoint{0.481978in}{0.331635in}}{\pgfqpoint{9.300000in}{7.700000in}}%
\pgfusepath{clip}%
\pgfsetrectcap%
\pgfsetroundjoin%
\pgfsetlinewidth{1.505625pt}%
\definecolor{currentstroke}{rgb}{1.000000,0.705882,0.509804}%
\pgfsetstrokecolor{currentstroke}%
\pgfsetstrokeopacity{0.800000}%
\pgfsetdash{}{0pt}%
\pgfpathmoveto{\pgfqpoint{2.980194in}{4.503252in}}%
\pgfpathlineto{\pgfqpoint{3.502867in}{4.072531in}}%
\pgfusepath{stroke}%
\end{pgfscope}%
\begin{pgfscope}%
\pgfpathrectangle{\pgfqpoint{0.481978in}{0.331635in}}{\pgfqpoint{9.300000in}{7.700000in}}%
\pgfusepath{clip}%
\pgfsetrectcap%
\pgfsetroundjoin%
\pgfsetlinewidth{1.505625pt}%
\definecolor{currentstroke}{rgb}{1.000000,0.705882,0.509804}%
\pgfsetstrokecolor{currentstroke}%
\pgfsetstrokeopacity{0.800000}%
\pgfsetdash{}{0pt}%
\pgfpathmoveto{\pgfqpoint{4.618269in}{4.700871in}}%
\pgfpathlineto{\pgfqpoint{3.502867in}{4.072531in}}%
\pgfusepath{stroke}%
\end{pgfscope}%
\begin{pgfscope}%
\pgfpathrectangle{\pgfqpoint{0.481978in}{0.331635in}}{\pgfqpoint{9.300000in}{7.700000in}}%
\pgfusepath{clip}%
\pgfsetrectcap%
\pgfsetroundjoin%
\pgfsetlinewidth{1.505625pt}%
\definecolor{currentstroke}{rgb}{1.000000,0.705882,0.509804}%
\pgfsetstrokecolor{currentstroke}%
\pgfsetstrokeopacity{0.800000}%
\pgfsetdash{}{0pt}%
\pgfpathmoveto{\pgfqpoint{3.693817in}{4.084468in}}%
\pgfpathlineto{\pgfqpoint{3.502867in}{4.072531in}}%
\pgfusepath{stroke}%
\end{pgfscope}%
\begin{pgfscope}%
\pgfpathrectangle{\pgfqpoint{0.481978in}{0.331635in}}{\pgfqpoint{9.300000in}{7.700000in}}%
\pgfusepath{clip}%
\pgfsetrectcap%
\pgfsetroundjoin%
\pgfsetlinewidth{1.505625pt}%
\definecolor{currentstroke}{rgb}{1.000000,0.705882,0.509804}%
\pgfsetstrokecolor{currentstroke}%
\pgfsetstrokeopacity{0.800000}%
\pgfsetdash{}{0pt}%
\pgfpathmoveto{\pgfqpoint{4.823817in}{3.134124in}}%
\pgfpathlineto{\pgfqpoint{3.502867in}{4.072531in}}%
\pgfusepath{stroke}%
\end{pgfscope}%
\begin{pgfscope}%
\pgfpathrectangle{\pgfqpoint{0.481978in}{0.331635in}}{\pgfqpoint{9.300000in}{7.700000in}}%
\pgfusepath{clip}%
\pgfsetrectcap%
\pgfsetroundjoin%
\pgfsetlinewidth{1.505625pt}%
\definecolor{currentstroke}{rgb}{1.000000,0.705882,0.509804}%
\pgfsetstrokecolor{currentstroke}%
\pgfsetstrokeopacity{0.800000}%
\pgfsetdash{}{0pt}%
\pgfpathmoveto{\pgfqpoint{2.627728in}{3.915305in}}%
\pgfpathlineto{\pgfqpoint{3.502867in}{4.072531in}}%
\pgfusepath{stroke}%
\end{pgfscope}%
\begin{pgfscope}%
\pgfpathrectangle{\pgfqpoint{0.481978in}{0.331635in}}{\pgfqpoint{9.300000in}{7.700000in}}%
\pgfusepath{clip}%
\pgfsetrectcap%
\pgfsetroundjoin%
\pgfsetlinewidth{1.505625pt}%
\definecolor{currentstroke}{rgb}{1.000000,0.705882,0.509804}%
\pgfsetstrokecolor{currentstroke}%
\pgfsetstrokeopacity{0.800000}%
\pgfsetdash{}{0pt}%
\pgfpathmoveto{\pgfqpoint{2.782878in}{3.370487in}}%
\pgfpathlineto{\pgfqpoint{3.502867in}{4.072531in}}%
\pgfusepath{stroke}%
\end{pgfscope}%
\begin{pgfscope}%
\pgfpathrectangle{\pgfqpoint{0.481978in}{0.331635in}}{\pgfqpoint{9.300000in}{7.700000in}}%
\pgfusepath{clip}%
\pgfsetrectcap%
\pgfsetroundjoin%
\pgfsetlinewidth{1.505625pt}%
\definecolor{currentstroke}{rgb}{1.000000,0.705882,0.509804}%
\pgfsetstrokecolor{currentstroke}%
\pgfsetstrokeopacity{0.800000}%
\pgfsetdash{}{0pt}%
\pgfpathmoveto{\pgfqpoint{3.755167in}{4.881969in}}%
\pgfpathlineto{\pgfqpoint{3.502867in}{4.072531in}}%
\pgfusepath{stroke}%
\end{pgfscope}%
\begin{pgfscope}%
\pgfpathrectangle{\pgfqpoint{0.481978in}{0.331635in}}{\pgfqpoint{9.300000in}{7.700000in}}%
\pgfusepath{clip}%
\pgfsetrectcap%
\pgfsetroundjoin%
\pgfsetlinewidth{1.505625pt}%
\definecolor{currentstroke}{rgb}{1.000000,0.705882,0.509804}%
\pgfsetstrokecolor{currentstroke}%
\pgfsetstrokeopacity{0.800000}%
\pgfsetdash{}{0pt}%
\pgfpathmoveto{\pgfqpoint{4.616665in}{2.897911in}}%
\pgfpathlineto{\pgfqpoint{3.502867in}{4.072531in}}%
\pgfusepath{stroke}%
\end{pgfscope}%
\begin{pgfscope}%
\pgfpathrectangle{\pgfqpoint{0.481978in}{0.331635in}}{\pgfqpoint{9.300000in}{7.700000in}}%
\pgfusepath{clip}%
\pgfsetrectcap%
\pgfsetroundjoin%
\pgfsetlinewidth{1.505625pt}%
\definecolor{currentstroke}{rgb}{1.000000,0.705882,0.509804}%
\pgfsetstrokecolor{currentstroke}%
\pgfsetstrokeopacity{0.800000}%
\pgfsetdash{}{0pt}%
\pgfpathmoveto{\pgfqpoint{2.176301in}{2.922426in}}%
\pgfpathlineto{\pgfqpoint{3.502867in}{4.072531in}}%
\pgfusepath{stroke}%
\end{pgfscope}%
\begin{pgfscope}%
\pgfpathrectangle{\pgfqpoint{0.481978in}{0.331635in}}{\pgfqpoint{9.300000in}{7.700000in}}%
\pgfusepath{clip}%
\pgfsetrectcap%
\pgfsetroundjoin%
\pgfsetlinewidth{1.505625pt}%
\definecolor{currentstroke}{rgb}{1.000000,0.705882,0.509804}%
\pgfsetstrokecolor{currentstroke}%
\pgfsetstrokeopacity{0.800000}%
\pgfsetdash{}{0pt}%
\pgfpathmoveto{\pgfqpoint{4.459267in}{4.587523in}}%
\pgfpathlineto{\pgfqpoint{3.502867in}{4.072531in}}%
\pgfusepath{stroke}%
\end{pgfscope}%
\begin{pgfscope}%
\pgfpathrectangle{\pgfqpoint{0.481978in}{0.331635in}}{\pgfqpoint{9.300000in}{7.700000in}}%
\pgfusepath{clip}%
\pgfsetrectcap%
\pgfsetroundjoin%
\pgfsetlinewidth{1.505625pt}%
\definecolor{currentstroke}{rgb}{1.000000,0.705882,0.509804}%
\pgfsetstrokecolor{currentstroke}%
\pgfsetstrokeopacity{0.800000}%
\pgfsetdash{}{0pt}%
\pgfpathmoveto{\pgfqpoint{4.335059in}{3.019883in}}%
\pgfpathlineto{\pgfqpoint{3.502867in}{4.072531in}}%
\pgfusepath{stroke}%
\end{pgfscope}%
\begin{pgfscope}%
\pgfpathrectangle{\pgfqpoint{0.481978in}{0.331635in}}{\pgfqpoint{9.300000in}{7.700000in}}%
\pgfusepath{clip}%
\pgfsetrectcap%
\pgfsetroundjoin%
\pgfsetlinewidth{1.505625pt}%
\definecolor{currentstroke}{rgb}{1.000000,0.705882,0.509804}%
\pgfsetstrokecolor{currentstroke}%
\pgfsetstrokeopacity{0.800000}%
\pgfsetdash{}{0pt}%
\pgfpathmoveto{\pgfqpoint{4.103001in}{2.989415in}}%
\pgfpathlineto{\pgfqpoint{3.502867in}{4.072531in}}%
\pgfusepath{stroke}%
\end{pgfscope}%
\begin{pgfscope}%
\pgfpathrectangle{\pgfqpoint{0.481978in}{0.331635in}}{\pgfqpoint{9.300000in}{7.700000in}}%
\pgfusepath{clip}%
\pgfsetrectcap%
\pgfsetroundjoin%
\pgfsetlinewidth{1.505625pt}%
\definecolor{currentstroke}{rgb}{1.000000,0.705882,0.509804}%
\pgfsetstrokecolor{currentstroke}%
\pgfsetstrokeopacity{0.800000}%
\pgfsetdash{}{0pt}%
\pgfpathmoveto{\pgfqpoint{1.240028in}{3.789606in}}%
\pgfpathlineto{\pgfqpoint{3.502867in}{4.072531in}}%
\pgfusepath{stroke}%
\end{pgfscope}%
\begin{pgfscope}%
\pgfpathrectangle{\pgfqpoint{0.481978in}{0.331635in}}{\pgfqpoint{9.300000in}{7.700000in}}%
\pgfusepath{clip}%
\pgfsetrectcap%
\pgfsetroundjoin%
\pgfsetlinewidth{1.505625pt}%
\definecolor{currentstroke}{rgb}{1.000000,0.705882,0.509804}%
\pgfsetstrokecolor{currentstroke}%
\pgfsetstrokeopacity{0.800000}%
\pgfsetdash{}{0pt}%
\pgfpathmoveto{\pgfqpoint{3.665627in}{3.711742in}}%
\pgfpathlineto{\pgfqpoint{3.502867in}{4.072531in}}%
\pgfusepath{stroke}%
\end{pgfscope}%
\begin{pgfscope}%
\pgfpathrectangle{\pgfqpoint{0.481978in}{0.331635in}}{\pgfqpoint{9.300000in}{7.700000in}}%
\pgfusepath{clip}%
\pgfsetrectcap%
\pgfsetroundjoin%
\pgfsetlinewidth{1.505625pt}%
\definecolor{currentstroke}{rgb}{1.000000,0.705882,0.509804}%
\pgfsetstrokecolor{currentstroke}%
\pgfsetstrokeopacity{0.800000}%
\pgfsetdash{}{0pt}%
\pgfpathmoveto{\pgfqpoint{3.637232in}{4.421791in}}%
\pgfpathlineto{\pgfqpoint{3.502867in}{4.072531in}}%
\pgfusepath{stroke}%
\end{pgfscope}%
\begin{pgfscope}%
\pgfpathrectangle{\pgfqpoint{0.481978in}{0.331635in}}{\pgfqpoint{9.300000in}{7.700000in}}%
\pgfusepath{clip}%
\pgfsetrectcap%
\pgfsetroundjoin%
\pgfsetlinewidth{1.505625pt}%
\definecolor{currentstroke}{rgb}{1.000000,0.705882,0.509804}%
\pgfsetstrokecolor{currentstroke}%
\pgfsetstrokeopacity{0.800000}%
\pgfsetdash{}{0pt}%
\pgfpathmoveto{\pgfqpoint{1.338259in}{3.743025in}}%
\pgfpathlineto{\pgfqpoint{3.502867in}{4.072531in}}%
\pgfusepath{stroke}%
\end{pgfscope}%
\begin{pgfscope}%
\pgfpathrectangle{\pgfqpoint{0.481978in}{0.331635in}}{\pgfqpoint{9.300000in}{7.700000in}}%
\pgfusepath{clip}%
\pgfsetrectcap%
\pgfsetroundjoin%
\pgfsetlinewidth{1.505625pt}%
\definecolor{currentstroke}{rgb}{1.000000,0.705882,0.509804}%
\pgfsetstrokecolor{currentstroke}%
\pgfsetstrokeopacity{0.800000}%
\pgfsetdash{}{0pt}%
\pgfpathmoveto{\pgfqpoint{1.995054in}{3.031987in}}%
\pgfpathlineto{\pgfqpoint{3.502867in}{4.072531in}}%
\pgfusepath{stroke}%
\end{pgfscope}%
\begin{pgfscope}%
\pgfpathrectangle{\pgfqpoint{0.481978in}{0.331635in}}{\pgfqpoint{9.300000in}{7.700000in}}%
\pgfusepath{clip}%
\pgfsetrectcap%
\pgfsetroundjoin%
\pgfsetlinewidth{1.505625pt}%
\definecolor{currentstroke}{rgb}{1.000000,0.705882,0.509804}%
\pgfsetstrokecolor{currentstroke}%
\pgfsetstrokeopacity{0.800000}%
\pgfsetdash{}{0pt}%
\pgfpathmoveto{\pgfqpoint{1.350306in}{4.003250in}}%
\pgfpathlineto{\pgfqpoint{3.502867in}{4.072531in}}%
\pgfusepath{stroke}%
\end{pgfscope}%
\begin{pgfscope}%
\pgfpathrectangle{\pgfqpoint{0.481978in}{0.331635in}}{\pgfqpoint{9.300000in}{7.700000in}}%
\pgfusepath{clip}%
\pgfsetrectcap%
\pgfsetroundjoin%
\pgfsetlinewidth{1.505625pt}%
\definecolor{currentstroke}{rgb}{1.000000,0.705882,0.509804}%
\pgfsetstrokecolor{currentstroke}%
\pgfsetstrokeopacity{0.800000}%
\pgfsetdash{}{0pt}%
\pgfpathmoveto{\pgfqpoint{1.701822in}{4.066580in}}%
\pgfpathlineto{\pgfqpoint{3.502867in}{4.072531in}}%
\pgfusepath{stroke}%
\end{pgfscope}%
\begin{pgfscope}%
\pgfpathrectangle{\pgfqpoint{0.481978in}{0.331635in}}{\pgfqpoint{9.300000in}{7.700000in}}%
\pgfusepath{clip}%
\pgfsetrectcap%
\pgfsetroundjoin%
\pgfsetlinewidth{1.505625pt}%
\definecolor{currentstroke}{rgb}{1.000000,0.705882,0.509804}%
\pgfsetstrokecolor{currentstroke}%
\pgfsetstrokeopacity{0.800000}%
\pgfsetdash{}{0pt}%
\pgfpathmoveto{\pgfqpoint{3.815204in}{4.006611in}}%
\pgfpathlineto{\pgfqpoint{3.502867in}{4.072531in}}%
\pgfusepath{stroke}%
\end{pgfscope}%
\begin{pgfscope}%
\pgfpathrectangle{\pgfqpoint{0.481978in}{0.331635in}}{\pgfqpoint{9.300000in}{7.700000in}}%
\pgfusepath{clip}%
\pgfsetrectcap%
\pgfsetroundjoin%
\pgfsetlinewidth{1.505625pt}%
\definecolor{currentstroke}{rgb}{1.000000,0.705882,0.509804}%
\pgfsetstrokecolor{currentstroke}%
\pgfsetstrokeopacity{0.800000}%
\pgfsetdash{}{0pt}%
\pgfpathmoveto{\pgfqpoint{3.592234in}{4.973285in}}%
\pgfpathlineto{\pgfqpoint{3.502867in}{4.072531in}}%
\pgfusepath{stroke}%
\end{pgfscope}%
\begin{pgfscope}%
\pgfpathrectangle{\pgfqpoint{0.481978in}{0.331635in}}{\pgfqpoint{9.300000in}{7.700000in}}%
\pgfusepath{clip}%
\pgfsetrectcap%
\pgfsetroundjoin%
\pgfsetlinewidth{1.505625pt}%
\definecolor{currentstroke}{rgb}{1.000000,0.705882,0.509804}%
\pgfsetstrokecolor{currentstroke}%
\pgfsetstrokeopacity{0.800000}%
\pgfsetdash{}{0pt}%
\pgfpathmoveto{\pgfqpoint{4.400900in}{2.398264in}}%
\pgfpathlineto{\pgfqpoint{3.502867in}{4.072531in}}%
\pgfusepath{stroke}%
\end{pgfscope}%
\begin{pgfscope}%
\pgfpathrectangle{\pgfqpoint{0.481978in}{0.331635in}}{\pgfqpoint{9.300000in}{7.700000in}}%
\pgfusepath{clip}%
\pgfsetrectcap%
\pgfsetroundjoin%
\pgfsetlinewidth{1.505625pt}%
\definecolor{currentstroke}{rgb}{1.000000,0.705882,0.509804}%
\pgfsetstrokecolor{currentstroke}%
\pgfsetstrokeopacity{0.800000}%
\pgfsetdash{}{0pt}%
\pgfpathmoveto{\pgfqpoint{4.677099in}{3.495443in}}%
\pgfpathlineto{\pgfqpoint{3.502867in}{4.072531in}}%
\pgfusepath{stroke}%
\end{pgfscope}%
\begin{pgfscope}%
\pgfpathrectangle{\pgfqpoint{0.481978in}{0.331635in}}{\pgfqpoint{9.300000in}{7.700000in}}%
\pgfusepath{clip}%
\pgfsetrectcap%
\pgfsetroundjoin%
\pgfsetlinewidth{1.505625pt}%
\definecolor{currentstroke}{rgb}{1.000000,0.705882,0.509804}%
\pgfsetstrokecolor{currentstroke}%
\pgfsetstrokeopacity{0.800000}%
\pgfsetdash{}{0pt}%
\pgfpathmoveto{\pgfqpoint{2.595817in}{5.503093in}}%
\pgfpathlineto{\pgfqpoint{3.502867in}{4.072531in}}%
\pgfusepath{stroke}%
\end{pgfscope}%
\begin{pgfscope}%
\pgfpathrectangle{\pgfqpoint{0.481978in}{0.331635in}}{\pgfqpoint{9.300000in}{7.700000in}}%
\pgfusepath{clip}%
\pgfsetrectcap%
\pgfsetroundjoin%
\pgfsetlinewidth{1.505625pt}%
\definecolor{currentstroke}{rgb}{1.000000,0.705882,0.509804}%
\pgfsetstrokecolor{currentstroke}%
\pgfsetstrokeopacity{0.800000}%
\pgfsetdash{}{0pt}%
\pgfpathmoveto{\pgfqpoint{3.380362in}{4.128920in}}%
\pgfpathlineto{\pgfqpoint{3.502867in}{4.072531in}}%
\pgfusepath{stroke}%
\end{pgfscope}%
\begin{pgfscope}%
\pgfpathrectangle{\pgfqpoint{0.481978in}{0.331635in}}{\pgfqpoint{9.300000in}{7.700000in}}%
\pgfusepath{clip}%
\pgfsetrectcap%
\pgfsetroundjoin%
\pgfsetlinewidth{1.505625pt}%
\definecolor{currentstroke}{rgb}{1.000000,0.705882,0.509804}%
\pgfsetstrokecolor{currentstroke}%
\pgfsetstrokeopacity{0.800000}%
\pgfsetdash{}{0pt}%
\pgfpathmoveto{\pgfqpoint{8.094040in}{3.915369in}}%
\pgfpathlineto{\pgfqpoint{3.502867in}{4.072531in}}%
\pgfusepath{stroke}%
\end{pgfscope}%
\begin{pgfscope}%
\pgfpathrectangle{\pgfqpoint{0.481978in}{0.331635in}}{\pgfqpoint{9.300000in}{7.700000in}}%
\pgfusepath{clip}%
\pgfsetrectcap%
\pgfsetroundjoin%
\pgfsetlinewidth{1.505625pt}%
\definecolor{currentstroke}{rgb}{1.000000,0.705882,0.509804}%
\pgfsetstrokecolor{currentstroke}%
\pgfsetstrokeopacity{0.800000}%
\pgfsetdash{}{0pt}%
\pgfpathmoveto{\pgfqpoint{3.449664in}{4.288307in}}%
\pgfpathlineto{\pgfqpoint{3.502867in}{4.072531in}}%
\pgfusepath{stroke}%
\end{pgfscope}%
\begin{pgfscope}%
\pgfpathrectangle{\pgfqpoint{0.481978in}{0.331635in}}{\pgfqpoint{9.300000in}{7.700000in}}%
\pgfusepath{clip}%
\pgfsetrectcap%
\pgfsetroundjoin%
\pgfsetlinewidth{1.505625pt}%
\definecolor{currentstroke}{rgb}{1.000000,0.705882,0.509804}%
\pgfsetstrokecolor{currentstroke}%
\pgfsetstrokeopacity{0.800000}%
\pgfsetdash{}{0pt}%
\pgfpathmoveto{\pgfqpoint{3.549059in}{7.246059in}}%
\pgfpathlineto{\pgfqpoint{3.502867in}{4.072531in}}%
\pgfusepath{stroke}%
\end{pgfscope}%
\begin{pgfscope}%
\pgfpathrectangle{\pgfqpoint{0.481978in}{0.331635in}}{\pgfqpoint{9.300000in}{7.700000in}}%
\pgfusepath{clip}%
\pgfsetrectcap%
\pgfsetroundjoin%
\pgfsetlinewidth{1.505625pt}%
\definecolor{currentstroke}{rgb}{1.000000,0.705882,0.509804}%
\pgfsetstrokecolor{currentstroke}%
\pgfsetstrokeopacity{0.800000}%
\pgfsetdash{}{0pt}%
\pgfpathmoveto{\pgfqpoint{1.737569in}{3.322770in}}%
\pgfpathlineto{\pgfqpoint{3.502867in}{4.072531in}}%
\pgfusepath{stroke}%
\end{pgfscope}%
\begin{pgfscope}%
\pgfpathrectangle{\pgfqpoint{0.481978in}{0.331635in}}{\pgfqpoint{9.300000in}{7.700000in}}%
\pgfusepath{clip}%
\pgfsetrectcap%
\pgfsetroundjoin%
\pgfsetlinewidth{1.505625pt}%
\definecolor{currentstroke}{rgb}{1.000000,0.705882,0.509804}%
\pgfsetstrokecolor{currentstroke}%
\pgfsetstrokeopacity{0.800000}%
\pgfsetdash{}{0pt}%
\pgfpathmoveto{\pgfqpoint{3.522445in}{3.146007in}}%
\pgfpathlineto{\pgfqpoint{3.502867in}{4.072531in}}%
\pgfusepath{stroke}%
\end{pgfscope}%
\begin{pgfscope}%
\pgfpathrectangle{\pgfqpoint{0.481978in}{0.331635in}}{\pgfqpoint{9.300000in}{7.700000in}}%
\pgfusepath{clip}%
\pgfsetrectcap%
\pgfsetroundjoin%
\pgfsetlinewidth{1.505625pt}%
\definecolor{currentstroke}{rgb}{1.000000,0.705882,0.509804}%
\pgfsetstrokecolor{currentstroke}%
\pgfsetstrokeopacity{0.800000}%
\pgfsetdash{}{0pt}%
\pgfpathmoveto{\pgfqpoint{5.442526in}{4.591847in}}%
\pgfpathlineto{\pgfqpoint{3.502867in}{4.072531in}}%
\pgfusepath{stroke}%
\end{pgfscope}%
\begin{pgfscope}%
\pgfpathrectangle{\pgfqpoint{0.481978in}{0.331635in}}{\pgfqpoint{9.300000in}{7.700000in}}%
\pgfusepath{clip}%
\pgfsetrectcap%
\pgfsetroundjoin%
\pgfsetlinewidth{1.505625pt}%
\definecolor{currentstroke}{rgb}{1.000000,0.705882,0.509804}%
\pgfsetstrokecolor{currentstroke}%
\pgfsetstrokeopacity{0.800000}%
\pgfsetdash{}{0pt}%
\pgfpathmoveto{\pgfqpoint{3.182729in}{4.203491in}}%
\pgfpathlineto{\pgfqpoint{3.502867in}{4.072531in}}%
\pgfusepath{stroke}%
\end{pgfscope}%
\begin{pgfscope}%
\pgfpathrectangle{\pgfqpoint{0.481978in}{0.331635in}}{\pgfqpoint{9.300000in}{7.700000in}}%
\pgfusepath{clip}%
\pgfsetrectcap%
\pgfsetroundjoin%
\pgfsetlinewidth{1.505625pt}%
\definecolor{currentstroke}{rgb}{1.000000,0.705882,0.509804}%
\pgfsetstrokecolor{currentstroke}%
\pgfsetstrokeopacity{0.800000}%
\pgfsetdash{}{0pt}%
\pgfpathmoveto{\pgfqpoint{5.580105in}{6.325768in}}%
\pgfpathlineto{\pgfqpoint{3.502867in}{4.072531in}}%
\pgfusepath{stroke}%
\end{pgfscope}%
\begin{pgfscope}%
\pgfpathrectangle{\pgfqpoint{0.481978in}{0.331635in}}{\pgfqpoint{9.300000in}{7.700000in}}%
\pgfusepath{clip}%
\pgfsetrectcap%
\pgfsetroundjoin%
\pgfsetlinewidth{1.505625pt}%
\definecolor{currentstroke}{rgb}{1.000000,0.705882,0.509804}%
\pgfsetstrokecolor{currentstroke}%
\pgfsetstrokeopacity{0.800000}%
\pgfsetdash{}{0pt}%
\pgfpathmoveto{\pgfqpoint{3.383682in}{4.926698in}}%
\pgfpathlineto{\pgfqpoint{3.502867in}{4.072531in}}%
\pgfusepath{stroke}%
\end{pgfscope}%
\begin{pgfscope}%
\pgfpathrectangle{\pgfqpoint{0.481978in}{0.331635in}}{\pgfqpoint{9.300000in}{7.700000in}}%
\pgfusepath{clip}%
\pgfsetrectcap%
\pgfsetroundjoin%
\pgfsetlinewidth{1.505625pt}%
\definecolor{currentstroke}{rgb}{1.000000,0.705882,0.509804}%
\pgfsetstrokecolor{currentstroke}%
\pgfsetstrokeopacity{0.800000}%
\pgfsetdash{}{0pt}%
\pgfpathmoveto{\pgfqpoint{3.010244in}{5.167031in}}%
\pgfpathlineto{\pgfqpoint{3.502867in}{4.072531in}}%
\pgfusepath{stroke}%
\end{pgfscope}%
\begin{pgfscope}%
\pgfpathrectangle{\pgfqpoint{0.481978in}{0.331635in}}{\pgfqpoint{9.300000in}{7.700000in}}%
\pgfusepath{clip}%
\pgfsetrectcap%
\pgfsetroundjoin%
\pgfsetlinewidth{1.505625pt}%
\definecolor{currentstroke}{rgb}{1.000000,0.705882,0.509804}%
\pgfsetstrokecolor{currentstroke}%
\pgfsetstrokeopacity{0.800000}%
\pgfsetdash{}{0pt}%
\pgfpathmoveto{\pgfqpoint{4.553603in}{4.039119in}}%
\pgfpathlineto{\pgfqpoint{3.502867in}{4.072531in}}%
\pgfusepath{stroke}%
\end{pgfscope}%
\begin{pgfscope}%
\pgfpathrectangle{\pgfqpoint{0.481978in}{0.331635in}}{\pgfqpoint{9.300000in}{7.700000in}}%
\pgfusepath{clip}%
\pgfsetrectcap%
\pgfsetroundjoin%
\pgfsetlinewidth{1.505625pt}%
\definecolor{currentstroke}{rgb}{1.000000,0.705882,0.509804}%
\pgfsetstrokecolor{currentstroke}%
\pgfsetstrokeopacity{0.800000}%
\pgfsetdash{}{0pt}%
\pgfpathmoveto{\pgfqpoint{9.359202in}{1.365364in}}%
\pgfpathlineto{\pgfqpoint{3.502867in}{4.072531in}}%
\pgfusepath{stroke}%
\end{pgfscope}%
\begin{pgfscope}%
\pgfpathrectangle{\pgfqpoint{0.481978in}{0.331635in}}{\pgfqpoint{9.300000in}{7.700000in}}%
\pgfusepath{clip}%
\pgfsetrectcap%
\pgfsetroundjoin%
\pgfsetlinewidth{1.505625pt}%
\definecolor{currentstroke}{rgb}{1.000000,0.705882,0.509804}%
\pgfsetstrokecolor{currentstroke}%
\pgfsetstrokeopacity{0.800000}%
\pgfsetdash{}{0pt}%
\pgfpathmoveto{\pgfqpoint{3.953659in}{2.911613in}}%
\pgfpathlineto{\pgfqpoint{3.502867in}{4.072531in}}%
\pgfusepath{stroke}%
\end{pgfscope}%
\begin{pgfscope}%
\pgfpathrectangle{\pgfqpoint{0.481978in}{0.331635in}}{\pgfqpoint{9.300000in}{7.700000in}}%
\pgfusepath{clip}%
\pgfsetrectcap%
\pgfsetroundjoin%
\pgfsetlinewidth{1.505625pt}%
\definecolor{currentstroke}{rgb}{1.000000,0.705882,0.509804}%
\pgfsetstrokecolor{currentstroke}%
\pgfsetstrokeopacity{0.800000}%
\pgfsetdash{}{0pt}%
\pgfpathmoveto{\pgfqpoint{4.408411in}{2.639990in}}%
\pgfpathlineto{\pgfqpoint{3.502867in}{4.072531in}}%
\pgfusepath{stroke}%
\end{pgfscope}%
\begin{pgfscope}%
\pgfpathrectangle{\pgfqpoint{0.481978in}{0.331635in}}{\pgfqpoint{9.300000in}{7.700000in}}%
\pgfusepath{clip}%
\pgfsetrectcap%
\pgfsetroundjoin%
\pgfsetlinewidth{1.505625pt}%
\definecolor{currentstroke}{rgb}{1.000000,0.705882,0.509804}%
\pgfsetstrokecolor{currentstroke}%
\pgfsetstrokeopacity{0.800000}%
\pgfsetdash{}{0pt}%
\pgfpathmoveto{\pgfqpoint{1.290181in}{5.575688in}}%
\pgfpathlineto{\pgfqpoint{3.502867in}{4.072531in}}%
\pgfusepath{stroke}%
\end{pgfscope}%
\begin{pgfscope}%
\pgfpathrectangle{\pgfqpoint{0.481978in}{0.331635in}}{\pgfqpoint{9.300000in}{7.700000in}}%
\pgfusepath{clip}%
\pgfsetrectcap%
\pgfsetroundjoin%
\pgfsetlinewidth{1.505625pt}%
\definecolor{currentstroke}{rgb}{1.000000,0.705882,0.509804}%
\pgfsetstrokecolor{currentstroke}%
\pgfsetstrokeopacity{0.800000}%
\pgfsetdash{}{0pt}%
\pgfpathmoveto{\pgfqpoint{2.624493in}{3.262294in}}%
\pgfpathlineto{\pgfqpoint{3.502867in}{4.072531in}}%
\pgfusepath{stroke}%
\end{pgfscope}%
\begin{pgfscope}%
\pgfpathrectangle{\pgfqpoint{0.481978in}{0.331635in}}{\pgfqpoint{9.300000in}{7.700000in}}%
\pgfusepath{clip}%
\pgfsetrectcap%
\pgfsetroundjoin%
\pgfsetlinewidth{1.505625pt}%
\definecolor{currentstroke}{rgb}{1.000000,0.705882,0.509804}%
\pgfsetstrokecolor{currentstroke}%
\pgfsetstrokeopacity{0.800000}%
\pgfsetdash{}{0pt}%
\pgfpathmoveto{\pgfqpoint{5.315161in}{4.162619in}}%
\pgfpathlineto{\pgfqpoint{3.502867in}{4.072531in}}%
\pgfusepath{stroke}%
\end{pgfscope}%
\begin{pgfscope}%
\pgfpathrectangle{\pgfqpoint{0.481978in}{0.331635in}}{\pgfqpoint{9.300000in}{7.700000in}}%
\pgfusepath{clip}%
\pgfsetrectcap%
\pgfsetroundjoin%
\pgfsetlinewidth{1.505625pt}%
\definecolor{currentstroke}{rgb}{1.000000,0.705882,0.509804}%
\pgfsetstrokecolor{currentstroke}%
\pgfsetstrokeopacity{0.800000}%
\pgfsetdash{}{0pt}%
\pgfpathmoveto{\pgfqpoint{4.062917in}{3.260250in}}%
\pgfpathlineto{\pgfqpoint{3.502867in}{4.072531in}}%
\pgfusepath{stroke}%
\end{pgfscope}%
\begin{pgfscope}%
\pgfpathrectangle{\pgfqpoint{0.481978in}{0.331635in}}{\pgfqpoint{9.300000in}{7.700000in}}%
\pgfusepath{clip}%
\pgfsetrectcap%
\pgfsetroundjoin%
\pgfsetlinewidth{1.505625pt}%
\definecolor{currentstroke}{rgb}{1.000000,0.705882,0.509804}%
\pgfsetstrokecolor{currentstroke}%
\pgfsetstrokeopacity{0.800000}%
\pgfsetdash{}{0pt}%
\pgfpathmoveto{\pgfqpoint{4.828561in}{4.070130in}}%
\pgfpathlineto{\pgfqpoint{3.502867in}{4.072531in}}%
\pgfusepath{stroke}%
\end{pgfscope}%
\begin{pgfscope}%
\pgfpathrectangle{\pgfqpoint{0.481978in}{0.331635in}}{\pgfqpoint{9.300000in}{7.700000in}}%
\pgfusepath{clip}%
\pgfsetrectcap%
\pgfsetroundjoin%
\pgfsetlinewidth{1.505625pt}%
\definecolor{currentstroke}{rgb}{1.000000,0.705882,0.509804}%
\pgfsetstrokecolor{currentstroke}%
\pgfsetstrokeopacity{0.800000}%
\pgfsetdash{}{0pt}%
\pgfpathmoveto{\pgfqpoint{2.387650in}{3.525751in}}%
\pgfpathlineto{\pgfqpoint{3.502867in}{4.072531in}}%
\pgfusepath{stroke}%
\end{pgfscope}%
\begin{pgfscope}%
\pgfpathrectangle{\pgfqpoint{0.481978in}{0.331635in}}{\pgfqpoint{9.300000in}{7.700000in}}%
\pgfusepath{clip}%
\pgfsetrectcap%
\pgfsetroundjoin%
\pgfsetlinewidth{1.505625pt}%
\definecolor{currentstroke}{rgb}{1.000000,0.705882,0.509804}%
\pgfsetstrokecolor{currentstroke}%
\pgfsetstrokeopacity{0.800000}%
\pgfsetdash{}{0pt}%
\pgfpathmoveto{\pgfqpoint{3.134591in}{4.927538in}}%
\pgfpathlineto{\pgfqpoint{3.502867in}{4.072531in}}%
\pgfusepath{stroke}%
\end{pgfscope}%
\begin{pgfscope}%
\pgfpathrectangle{\pgfqpoint{0.481978in}{0.331635in}}{\pgfqpoint{9.300000in}{7.700000in}}%
\pgfusepath{clip}%
\pgfsetrectcap%
\pgfsetroundjoin%
\pgfsetlinewidth{1.505625pt}%
\definecolor{currentstroke}{rgb}{1.000000,0.705882,0.509804}%
\pgfsetstrokecolor{currentstroke}%
\pgfsetstrokeopacity{0.800000}%
\pgfsetdash{}{0pt}%
\pgfpathmoveto{\pgfqpoint{1.638268in}{4.702300in}}%
\pgfpathlineto{\pgfqpoint{3.502867in}{4.072531in}}%
\pgfusepath{stroke}%
\end{pgfscope}%
\begin{pgfscope}%
\pgfpathrectangle{\pgfqpoint{0.481978in}{0.331635in}}{\pgfqpoint{9.300000in}{7.700000in}}%
\pgfusepath{clip}%
\pgfsetrectcap%
\pgfsetroundjoin%
\pgfsetlinewidth{1.505625pt}%
\definecolor{currentstroke}{rgb}{1.000000,0.705882,0.509804}%
\pgfsetstrokecolor{currentstroke}%
\pgfsetstrokeopacity{0.800000}%
\pgfsetdash{}{0pt}%
\pgfpathmoveto{\pgfqpoint{4.067183in}{6.837238in}}%
\pgfpathlineto{\pgfqpoint{3.502867in}{4.072531in}}%
\pgfusepath{stroke}%
\end{pgfscope}%
\begin{pgfscope}%
\pgfpathrectangle{\pgfqpoint{0.481978in}{0.331635in}}{\pgfqpoint{9.300000in}{7.700000in}}%
\pgfusepath{clip}%
\pgfsetrectcap%
\pgfsetroundjoin%
\pgfsetlinewidth{1.505625pt}%
\definecolor{currentstroke}{rgb}{1.000000,0.705882,0.509804}%
\pgfsetstrokecolor{currentstroke}%
\pgfsetstrokeopacity{0.800000}%
\pgfsetdash{}{0pt}%
\pgfpathmoveto{\pgfqpoint{3.079878in}{3.918994in}}%
\pgfpathlineto{\pgfqpoint{3.502867in}{4.072531in}}%
\pgfusepath{stroke}%
\end{pgfscope}%
\begin{pgfscope}%
\pgfpathrectangle{\pgfqpoint{0.481978in}{0.331635in}}{\pgfqpoint{9.300000in}{7.700000in}}%
\pgfusepath{clip}%
\pgfsetrectcap%
\pgfsetroundjoin%
\pgfsetlinewidth{1.505625pt}%
\definecolor{currentstroke}{rgb}{1.000000,0.705882,0.509804}%
\pgfsetstrokecolor{currentstroke}%
\pgfsetstrokeopacity{0.800000}%
\pgfsetdash{}{0pt}%
\pgfpathmoveto{\pgfqpoint{3.762277in}{2.634602in}}%
\pgfpathlineto{\pgfqpoint{3.502867in}{4.072531in}}%
\pgfusepath{stroke}%
\end{pgfscope}%
\begin{pgfscope}%
\pgfpathrectangle{\pgfqpoint{0.481978in}{0.331635in}}{\pgfqpoint{9.300000in}{7.700000in}}%
\pgfusepath{clip}%
\pgfsetrectcap%
\pgfsetroundjoin%
\pgfsetlinewidth{1.505625pt}%
\definecolor{currentstroke}{rgb}{1.000000,0.705882,0.509804}%
\pgfsetstrokecolor{currentstroke}%
\pgfsetstrokeopacity{0.800000}%
\pgfsetdash{}{0pt}%
\pgfpathmoveto{\pgfqpoint{3.580814in}{3.904227in}}%
\pgfpathlineto{\pgfqpoint{3.502867in}{4.072531in}}%
\pgfusepath{stroke}%
\end{pgfscope}%
\begin{pgfscope}%
\pgfpathrectangle{\pgfqpoint{0.481978in}{0.331635in}}{\pgfqpoint{9.300000in}{7.700000in}}%
\pgfusepath{clip}%
\pgfsetrectcap%
\pgfsetroundjoin%
\pgfsetlinewidth{1.505625pt}%
\definecolor{currentstroke}{rgb}{1.000000,0.705882,0.509804}%
\pgfsetstrokecolor{currentstroke}%
\pgfsetstrokeopacity{0.800000}%
\pgfsetdash{}{0pt}%
\pgfpathmoveto{\pgfqpoint{4.361455in}{2.831609in}}%
\pgfpathlineto{\pgfqpoint{3.502867in}{4.072531in}}%
\pgfusepath{stroke}%
\end{pgfscope}%
\begin{pgfscope}%
\pgfpathrectangle{\pgfqpoint{0.481978in}{0.331635in}}{\pgfqpoint{9.300000in}{7.700000in}}%
\pgfusepath{clip}%
\pgfsetrectcap%
\pgfsetroundjoin%
\pgfsetlinewidth{1.505625pt}%
\definecolor{currentstroke}{rgb}{1.000000,0.705882,0.509804}%
\pgfsetstrokecolor{currentstroke}%
\pgfsetstrokeopacity{0.800000}%
\pgfsetdash{}{0pt}%
\pgfpathmoveto{\pgfqpoint{3.821252in}{6.921355in}}%
\pgfpathlineto{\pgfqpoint{3.502867in}{4.072531in}}%
\pgfusepath{stroke}%
\end{pgfscope}%
\begin{pgfscope}%
\pgfpathrectangle{\pgfqpoint{0.481978in}{0.331635in}}{\pgfqpoint{9.300000in}{7.700000in}}%
\pgfusepath{clip}%
\pgfsetrectcap%
\pgfsetroundjoin%
\pgfsetlinewidth{1.505625pt}%
\definecolor{currentstroke}{rgb}{1.000000,0.705882,0.509804}%
\pgfsetstrokecolor{currentstroke}%
\pgfsetstrokeopacity{0.800000}%
\pgfsetdash{}{0pt}%
\pgfpathmoveto{\pgfqpoint{2.933293in}{5.640936in}}%
\pgfpathlineto{\pgfqpoint{3.502867in}{4.072531in}}%
\pgfusepath{stroke}%
\end{pgfscope}%
\begin{pgfscope}%
\pgfpathrectangle{\pgfqpoint{0.481978in}{0.331635in}}{\pgfqpoint{9.300000in}{7.700000in}}%
\pgfusepath{clip}%
\pgfsetrectcap%
\pgfsetroundjoin%
\pgfsetlinewidth{1.505625pt}%
\definecolor{currentstroke}{rgb}{1.000000,0.705882,0.509804}%
\pgfsetstrokecolor{currentstroke}%
\pgfsetstrokeopacity{0.800000}%
\pgfsetdash{}{0pt}%
\pgfpathmoveto{\pgfqpoint{4.420071in}{4.768249in}}%
\pgfpathlineto{\pgfqpoint{3.502867in}{4.072531in}}%
\pgfusepath{stroke}%
\end{pgfscope}%
\begin{pgfscope}%
\pgfpathrectangle{\pgfqpoint{0.481978in}{0.331635in}}{\pgfqpoint{9.300000in}{7.700000in}}%
\pgfusepath{clip}%
\pgfsetrectcap%
\pgfsetroundjoin%
\pgfsetlinewidth{1.505625pt}%
\definecolor{currentstroke}{rgb}{1.000000,0.705882,0.509804}%
\pgfsetstrokecolor{currentstroke}%
\pgfsetstrokeopacity{0.800000}%
\pgfsetdash{}{0pt}%
\pgfpathmoveto{\pgfqpoint{2.246703in}{5.008506in}}%
\pgfpathlineto{\pgfqpoint{3.502867in}{4.072531in}}%
\pgfusepath{stroke}%
\end{pgfscope}%
\begin{pgfscope}%
\pgfpathrectangle{\pgfqpoint{0.481978in}{0.331635in}}{\pgfqpoint{9.300000in}{7.700000in}}%
\pgfusepath{clip}%
\pgfsetrectcap%
\pgfsetroundjoin%
\pgfsetlinewidth{1.505625pt}%
\definecolor{currentstroke}{rgb}{1.000000,0.705882,0.509804}%
\pgfsetstrokecolor{currentstroke}%
\pgfsetstrokeopacity{0.800000}%
\pgfsetdash{}{0pt}%
\pgfpathmoveto{\pgfqpoint{4.205087in}{5.636892in}}%
\pgfpathlineto{\pgfqpoint{3.502867in}{4.072531in}}%
\pgfusepath{stroke}%
\end{pgfscope}%
\begin{pgfscope}%
\pgfpathrectangle{\pgfqpoint{0.481978in}{0.331635in}}{\pgfqpoint{9.300000in}{7.700000in}}%
\pgfusepath{clip}%
\pgfsetrectcap%
\pgfsetroundjoin%
\pgfsetlinewidth{1.505625pt}%
\definecolor{currentstroke}{rgb}{1.000000,0.705882,0.509804}%
\pgfsetstrokecolor{currentstroke}%
\pgfsetstrokeopacity{0.800000}%
\pgfsetdash{}{0pt}%
\pgfpathmoveto{\pgfqpoint{3.838690in}{4.225761in}}%
\pgfpathlineto{\pgfqpoint{3.502867in}{4.072531in}}%
\pgfusepath{stroke}%
\end{pgfscope}%
\begin{pgfscope}%
\pgfpathrectangle{\pgfqpoint{0.481978in}{0.331635in}}{\pgfqpoint{9.300000in}{7.700000in}}%
\pgfusepath{clip}%
\pgfsetrectcap%
\pgfsetroundjoin%
\pgfsetlinewidth{1.505625pt}%
\definecolor{currentstroke}{rgb}{1.000000,0.705882,0.509804}%
\pgfsetstrokecolor{currentstroke}%
\pgfsetstrokeopacity{0.800000}%
\pgfsetdash{}{0pt}%
\pgfpathmoveto{\pgfqpoint{1.293480in}{5.571862in}}%
\pgfpathlineto{\pgfqpoint{3.502867in}{4.072531in}}%
\pgfusepath{stroke}%
\end{pgfscope}%
\begin{pgfscope}%
\pgfpathrectangle{\pgfqpoint{0.481978in}{0.331635in}}{\pgfqpoint{9.300000in}{7.700000in}}%
\pgfusepath{clip}%
\pgfsetrectcap%
\pgfsetroundjoin%
\pgfsetlinewidth{1.505625pt}%
\definecolor{currentstroke}{rgb}{1.000000,0.705882,0.509804}%
\pgfsetstrokecolor{currentstroke}%
\pgfsetstrokeopacity{0.800000}%
\pgfsetdash{}{0pt}%
\pgfpathmoveto{\pgfqpoint{4.735330in}{2.540766in}}%
\pgfpathlineto{\pgfqpoint{3.502867in}{4.072531in}}%
\pgfusepath{stroke}%
\end{pgfscope}%
\begin{pgfscope}%
\pgfpathrectangle{\pgfqpoint{0.481978in}{0.331635in}}{\pgfqpoint{9.300000in}{7.700000in}}%
\pgfusepath{clip}%
\pgfsetrectcap%
\pgfsetroundjoin%
\pgfsetlinewidth{1.505625pt}%
\definecolor{currentstroke}{rgb}{1.000000,0.705882,0.509804}%
\pgfsetstrokecolor{currentstroke}%
\pgfsetstrokeopacity{0.800000}%
\pgfsetdash{}{0pt}%
\pgfpathmoveto{\pgfqpoint{4.464515in}{5.013803in}}%
\pgfpathlineto{\pgfqpoint{3.502867in}{4.072531in}}%
\pgfusepath{stroke}%
\end{pgfscope}%
\begin{pgfscope}%
\pgfpathrectangle{\pgfqpoint{0.481978in}{0.331635in}}{\pgfqpoint{9.300000in}{7.700000in}}%
\pgfusepath{clip}%
\pgfsetrectcap%
\pgfsetroundjoin%
\pgfsetlinewidth{1.505625pt}%
\definecolor{currentstroke}{rgb}{1.000000,0.705882,0.509804}%
\pgfsetstrokecolor{currentstroke}%
\pgfsetstrokeopacity{0.800000}%
\pgfsetdash{}{0pt}%
\pgfpathmoveto{\pgfqpoint{3.760896in}{2.909223in}}%
\pgfpathlineto{\pgfqpoint{3.502867in}{4.072531in}}%
\pgfusepath{stroke}%
\end{pgfscope}%
\begin{pgfscope}%
\pgfpathrectangle{\pgfqpoint{0.481978in}{0.331635in}}{\pgfqpoint{9.300000in}{7.700000in}}%
\pgfusepath{clip}%
\pgfsetrectcap%
\pgfsetroundjoin%
\pgfsetlinewidth{1.505625pt}%
\definecolor{currentstroke}{rgb}{1.000000,0.705882,0.509804}%
\pgfsetstrokecolor{currentstroke}%
\pgfsetstrokeopacity{0.800000}%
\pgfsetdash{}{0pt}%
\pgfpathmoveto{\pgfqpoint{2.336586in}{3.518177in}}%
\pgfpathlineto{\pgfqpoint{3.502867in}{4.072531in}}%
\pgfusepath{stroke}%
\end{pgfscope}%
\begin{pgfscope}%
\pgfpathrectangle{\pgfqpoint{0.481978in}{0.331635in}}{\pgfqpoint{9.300000in}{7.700000in}}%
\pgfusepath{clip}%
\pgfsetrectcap%
\pgfsetroundjoin%
\pgfsetlinewidth{1.505625pt}%
\definecolor{currentstroke}{rgb}{1.000000,0.705882,0.509804}%
\pgfsetstrokecolor{currentstroke}%
\pgfsetstrokeopacity{0.800000}%
\pgfsetdash{}{0pt}%
\pgfpathmoveto{\pgfqpoint{3.224347in}{6.045836in}}%
\pgfpathlineto{\pgfqpoint{3.502867in}{4.072531in}}%
\pgfusepath{stroke}%
\end{pgfscope}%
\begin{pgfscope}%
\pgfpathrectangle{\pgfqpoint{0.481978in}{0.331635in}}{\pgfqpoint{9.300000in}{7.700000in}}%
\pgfusepath{clip}%
\pgfsetrectcap%
\pgfsetroundjoin%
\pgfsetlinewidth{1.505625pt}%
\definecolor{currentstroke}{rgb}{1.000000,0.705882,0.509804}%
\pgfsetstrokecolor{currentstroke}%
\pgfsetstrokeopacity{0.800000}%
\pgfsetdash{}{0pt}%
\pgfpathmoveto{\pgfqpoint{3.710549in}{6.562804in}}%
\pgfpathlineto{\pgfqpoint{3.502867in}{4.072531in}}%
\pgfusepath{stroke}%
\end{pgfscope}%
\begin{pgfscope}%
\pgfpathrectangle{\pgfqpoint{0.481978in}{0.331635in}}{\pgfqpoint{9.300000in}{7.700000in}}%
\pgfusepath{clip}%
\pgfsetrectcap%
\pgfsetroundjoin%
\pgfsetlinewidth{1.505625pt}%
\definecolor{currentstroke}{rgb}{1.000000,0.705882,0.509804}%
\pgfsetstrokecolor{currentstroke}%
\pgfsetstrokeopacity{0.800000}%
\pgfsetdash{}{0pt}%
\pgfpathmoveto{\pgfqpoint{3.320504in}{5.288824in}}%
\pgfpathlineto{\pgfqpoint{3.502867in}{4.072531in}}%
\pgfusepath{stroke}%
\end{pgfscope}%
\begin{pgfscope}%
\pgfpathrectangle{\pgfqpoint{0.481978in}{0.331635in}}{\pgfqpoint{9.300000in}{7.700000in}}%
\pgfusepath{clip}%
\pgfsetrectcap%
\pgfsetroundjoin%
\pgfsetlinewidth{1.505625pt}%
\definecolor{currentstroke}{rgb}{1.000000,0.705882,0.509804}%
\pgfsetstrokecolor{currentstroke}%
\pgfsetstrokeopacity{0.800000}%
\pgfsetdash{}{0pt}%
\pgfpathmoveto{\pgfqpoint{5.389021in}{1.280411in}}%
\pgfpathlineto{\pgfqpoint{3.502867in}{4.072531in}}%
\pgfusepath{stroke}%
\end{pgfscope}%
\begin{pgfscope}%
\pgfpathrectangle{\pgfqpoint{0.481978in}{0.331635in}}{\pgfqpoint{9.300000in}{7.700000in}}%
\pgfusepath{clip}%
\pgfsetrectcap%
\pgfsetroundjoin%
\pgfsetlinewidth{1.505625pt}%
\definecolor{currentstroke}{rgb}{1.000000,0.705882,0.509804}%
\pgfsetstrokecolor{currentstroke}%
\pgfsetstrokeopacity{0.800000}%
\pgfsetdash{}{0pt}%
\pgfpathmoveto{\pgfqpoint{1.225270in}{3.197032in}}%
\pgfpathlineto{\pgfqpoint{3.502867in}{4.072531in}}%
\pgfusepath{stroke}%
\end{pgfscope}%
\begin{pgfscope}%
\pgfpathrectangle{\pgfqpoint{0.481978in}{0.331635in}}{\pgfqpoint{9.300000in}{7.700000in}}%
\pgfusepath{clip}%
\pgfsetrectcap%
\pgfsetroundjoin%
\pgfsetlinewidth{1.505625pt}%
\definecolor{currentstroke}{rgb}{1.000000,0.705882,0.509804}%
\pgfsetstrokecolor{currentstroke}%
\pgfsetstrokeopacity{0.800000}%
\pgfsetdash{}{0pt}%
\pgfpathmoveto{\pgfqpoint{4.019367in}{2.972140in}}%
\pgfpathlineto{\pgfqpoint{3.502867in}{4.072531in}}%
\pgfusepath{stroke}%
\end{pgfscope}%
\begin{pgfscope}%
\pgfpathrectangle{\pgfqpoint{0.481978in}{0.331635in}}{\pgfqpoint{9.300000in}{7.700000in}}%
\pgfusepath{clip}%
\pgfsetrectcap%
\pgfsetroundjoin%
\pgfsetlinewidth{1.505625pt}%
\definecolor{currentstroke}{rgb}{1.000000,0.705882,0.509804}%
\pgfsetstrokecolor{currentstroke}%
\pgfsetstrokeopacity{0.800000}%
\pgfsetdash{}{0pt}%
\pgfpathmoveto{\pgfqpoint{3.098846in}{4.090458in}}%
\pgfpathlineto{\pgfqpoint{3.502867in}{4.072531in}}%
\pgfusepath{stroke}%
\end{pgfscope}%
\begin{pgfscope}%
\pgfpathrectangle{\pgfqpoint{0.481978in}{0.331635in}}{\pgfqpoint{9.300000in}{7.700000in}}%
\pgfusepath{clip}%
\pgfsetrectcap%
\pgfsetroundjoin%
\pgfsetlinewidth{1.505625pt}%
\definecolor{currentstroke}{rgb}{1.000000,0.705882,0.509804}%
\pgfsetstrokecolor{currentstroke}%
\pgfsetstrokeopacity{0.800000}%
\pgfsetdash{}{0pt}%
\pgfpathmoveto{\pgfqpoint{2.491460in}{4.541702in}}%
\pgfpathlineto{\pgfqpoint{3.502867in}{4.072531in}}%
\pgfusepath{stroke}%
\end{pgfscope}%
\begin{pgfscope}%
\pgfpathrectangle{\pgfqpoint{0.481978in}{0.331635in}}{\pgfqpoint{9.300000in}{7.700000in}}%
\pgfusepath{clip}%
\pgfsetrectcap%
\pgfsetroundjoin%
\pgfsetlinewidth{1.505625pt}%
\definecolor{currentstroke}{rgb}{1.000000,0.705882,0.509804}%
\pgfsetstrokecolor{currentstroke}%
\pgfsetstrokeopacity{0.800000}%
\pgfsetdash{}{0pt}%
\pgfpathmoveto{\pgfqpoint{4.722330in}{3.392277in}}%
\pgfpathlineto{\pgfqpoint{3.502867in}{4.072531in}}%
\pgfusepath{stroke}%
\end{pgfscope}%
\begin{pgfscope}%
\pgfpathrectangle{\pgfqpoint{0.481978in}{0.331635in}}{\pgfqpoint{9.300000in}{7.700000in}}%
\pgfusepath{clip}%
\pgfsetrectcap%
\pgfsetroundjoin%
\pgfsetlinewidth{1.505625pt}%
\definecolor{currentstroke}{rgb}{1.000000,0.705882,0.509804}%
\pgfsetstrokecolor{currentstroke}%
\pgfsetstrokeopacity{0.800000}%
\pgfsetdash{}{0pt}%
\pgfpathmoveto{\pgfqpoint{4.962053in}{4.763970in}}%
\pgfpathlineto{\pgfqpoint{3.502867in}{4.072531in}}%
\pgfusepath{stroke}%
\end{pgfscope}%
\begin{pgfscope}%
\pgfpathrectangle{\pgfqpoint{0.481978in}{0.331635in}}{\pgfqpoint{9.300000in}{7.700000in}}%
\pgfusepath{clip}%
\pgfsetrectcap%
\pgfsetroundjoin%
\pgfsetlinewidth{1.505625pt}%
\definecolor{currentstroke}{rgb}{1.000000,0.705882,0.509804}%
\pgfsetstrokecolor{currentstroke}%
\pgfsetstrokeopacity{0.800000}%
\pgfsetdash{}{0pt}%
\pgfpathmoveto{\pgfqpoint{4.756086in}{2.522358in}}%
\pgfpathlineto{\pgfqpoint{3.502867in}{4.072531in}}%
\pgfusepath{stroke}%
\end{pgfscope}%
\begin{pgfscope}%
\pgfpathrectangle{\pgfqpoint{0.481978in}{0.331635in}}{\pgfqpoint{9.300000in}{7.700000in}}%
\pgfusepath{clip}%
\pgfsetrectcap%
\pgfsetroundjoin%
\pgfsetlinewidth{1.505625pt}%
\definecolor{currentstroke}{rgb}{1.000000,0.705882,0.509804}%
\pgfsetstrokecolor{currentstroke}%
\pgfsetstrokeopacity{0.800000}%
\pgfsetdash{}{0pt}%
\pgfpathmoveto{\pgfqpoint{2.604559in}{2.358183in}}%
\pgfpathlineto{\pgfqpoint{3.502867in}{4.072531in}}%
\pgfusepath{stroke}%
\end{pgfscope}%
\begin{pgfscope}%
\pgfpathrectangle{\pgfqpoint{0.481978in}{0.331635in}}{\pgfqpoint{9.300000in}{7.700000in}}%
\pgfusepath{clip}%
\pgfsetrectcap%
\pgfsetroundjoin%
\pgfsetlinewidth{1.505625pt}%
\definecolor{currentstroke}{rgb}{1.000000,0.705882,0.509804}%
\pgfsetstrokecolor{currentstroke}%
\pgfsetstrokeopacity{0.800000}%
\pgfsetdash{}{0pt}%
\pgfpathmoveto{\pgfqpoint{3.564142in}{5.407182in}}%
\pgfpathlineto{\pgfqpoint{3.502867in}{4.072531in}}%
\pgfusepath{stroke}%
\end{pgfscope}%
\begin{pgfscope}%
\pgfpathrectangle{\pgfqpoint{0.481978in}{0.331635in}}{\pgfqpoint{9.300000in}{7.700000in}}%
\pgfusepath{clip}%
\pgfsetrectcap%
\pgfsetroundjoin%
\pgfsetlinewidth{1.505625pt}%
\definecolor{currentstroke}{rgb}{1.000000,0.705882,0.509804}%
\pgfsetstrokecolor{currentstroke}%
\pgfsetstrokeopacity{0.800000}%
\pgfsetdash{}{0pt}%
\pgfpathmoveto{\pgfqpoint{3.840892in}{5.632476in}}%
\pgfpathlineto{\pgfqpoint{3.502867in}{4.072531in}}%
\pgfusepath{stroke}%
\end{pgfscope}%
\begin{pgfscope}%
\pgfpathrectangle{\pgfqpoint{0.481978in}{0.331635in}}{\pgfqpoint{9.300000in}{7.700000in}}%
\pgfusepath{clip}%
\pgfsetrectcap%
\pgfsetroundjoin%
\pgfsetlinewidth{1.505625pt}%
\definecolor{currentstroke}{rgb}{1.000000,0.705882,0.509804}%
\pgfsetstrokecolor{currentstroke}%
\pgfsetstrokeopacity{0.800000}%
\pgfsetdash{}{0pt}%
\pgfpathmoveto{\pgfqpoint{5.376976in}{4.653290in}}%
\pgfpathlineto{\pgfqpoint{3.502867in}{4.072531in}}%
\pgfusepath{stroke}%
\end{pgfscope}%
\begin{pgfscope}%
\pgfpathrectangle{\pgfqpoint{0.481978in}{0.331635in}}{\pgfqpoint{9.300000in}{7.700000in}}%
\pgfusepath{clip}%
\pgfsetrectcap%
\pgfsetroundjoin%
\pgfsetlinewidth{1.505625pt}%
\definecolor{currentstroke}{rgb}{1.000000,0.705882,0.509804}%
\pgfsetstrokecolor{currentstroke}%
\pgfsetstrokeopacity{0.800000}%
\pgfsetdash{}{0pt}%
\pgfpathmoveto{\pgfqpoint{4.185210in}{4.218529in}}%
\pgfpathlineto{\pgfqpoint{3.502867in}{4.072531in}}%
\pgfusepath{stroke}%
\end{pgfscope}%
\begin{pgfscope}%
\pgfpathrectangle{\pgfqpoint{0.481978in}{0.331635in}}{\pgfqpoint{9.300000in}{7.700000in}}%
\pgfusepath{clip}%
\pgfsetrectcap%
\pgfsetroundjoin%
\pgfsetlinewidth{1.505625pt}%
\definecolor{currentstroke}{rgb}{1.000000,0.705882,0.509804}%
\pgfsetstrokecolor{currentstroke}%
\pgfsetstrokeopacity{0.800000}%
\pgfsetdash{}{0pt}%
\pgfpathmoveto{\pgfqpoint{4.500422in}{3.044916in}}%
\pgfpathlineto{\pgfqpoint{3.502867in}{4.072531in}}%
\pgfusepath{stroke}%
\end{pgfscope}%
\begin{pgfscope}%
\pgfpathrectangle{\pgfqpoint{0.481978in}{0.331635in}}{\pgfqpoint{9.300000in}{7.700000in}}%
\pgfusepath{clip}%
\pgfsetrectcap%
\pgfsetroundjoin%
\pgfsetlinewidth{1.505625pt}%
\definecolor{currentstroke}{rgb}{1.000000,0.705882,0.509804}%
\pgfsetstrokecolor{currentstroke}%
\pgfsetstrokeopacity{0.800000}%
\pgfsetdash{}{0pt}%
\pgfpathmoveto{\pgfqpoint{4.103119in}{4.294356in}}%
\pgfpathlineto{\pgfqpoint{3.502867in}{4.072531in}}%
\pgfusepath{stroke}%
\end{pgfscope}%
\begin{pgfscope}%
\pgfpathrectangle{\pgfqpoint{0.481978in}{0.331635in}}{\pgfqpoint{9.300000in}{7.700000in}}%
\pgfusepath{clip}%
\pgfsetrectcap%
\pgfsetroundjoin%
\pgfsetlinewidth{1.505625pt}%
\definecolor{currentstroke}{rgb}{1.000000,0.705882,0.509804}%
\pgfsetstrokecolor{currentstroke}%
\pgfsetstrokeopacity{0.800000}%
\pgfsetdash{}{0pt}%
\pgfpathmoveto{\pgfqpoint{3.235019in}{4.506980in}}%
\pgfpathlineto{\pgfqpoint{3.502867in}{4.072531in}}%
\pgfusepath{stroke}%
\end{pgfscope}%
\begin{pgfscope}%
\pgfpathrectangle{\pgfqpoint{0.481978in}{0.331635in}}{\pgfqpoint{9.300000in}{7.700000in}}%
\pgfusepath{clip}%
\pgfsetrectcap%
\pgfsetroundjoin%
\pgfsetlinewidth{1.505625pt}%
\definecolor{currentstroke}{rgb}{1.000000,0.705882,0.509804}%
\pgfsetstrokecolor{currentstroke}%
\pgfsetstrokeopacity{0.800000}%
\pgfsetdash{}{0pt}%
\pgfpathmoveto{\pgfqpoint{1.466431in}{3.111586in}}%
\pgfpathlineto{\pgfqpoint{3.502867in}{4.072531in}}%
\pgfusepath{stroke}%
\end{pgfscope}%
\begin{pgfscope}%
\pgfpathrectangle{\pgfqpoint{0.481978in}{0.331635in}}{\pgfqpoint{9.300000in}{7.700000in}}%
\pgfusepath{clip}%
\pgfsetrectcap%
\pgfsetroundjoin%
\pgfsetlinewidth{1.505625pt}%
\definecolor{currentstroke}{rgb}{1.000000,0.705882,0.509804}%
\pgfsetstrokecolor{currentstroke}%
\pgfsetstrokeopacity{0.800000}%
\pgfsetdash{}{0pt}%
\pgfpathmoveto{\pgfqpoint{5.649061in}{4.482985in}}%
\pgfpathlineto{\pgfqpoint{3.502867in}{4.072531in}}%
\pgfusepath{stroke}%
\end{pgfscope}%
\begin{pgfscope}%
\pgfpathrectangle{\pgfqpoint{0.481978in}{0.331635in}}{\pgfqpoint{9.300000in}{7.700000in}}%
\pgfusepath{clip}%
\pgfsetrectcap%
\pgfsetroundjoin%
\pgfsetlinewidth{1.505625pt}%
\definecolor{currentstroke}{rgb}{1.000000,0.705882,0.509804}%
\pgfsetstrokecolor{currentstroke}%
\pgfsetstrokeopacity{0.800000}%
\pgfsetdash{}{0pt}%
\pgfpathmoveto{\pgfqpoint{3.679062in}{2.692966in}}%
\pgfpathlineto{\pgfqpoint{3.502867in}{4.072531in}}%
\pgfusepath{stroke}%
\end{pgfscope}%
\begin{pgfscope}%
\pgfpathrectangle{\pgfqpoint{0.481978in}{0.331635in}}{\pgfqpoint{9.300000in}{7.700000in}}%
\pgfusepath{clip}%
\pgfsetrectcap%
\pgfsetroundjoin%
\pgfsetlinewidth{1.505625pt}%
\definecolor{currentstroke}{rgb}{1.000000,0.705882,0.509804}%
\pgfsetstrokecolor{currentstroke}%
\pgfsetstrokeopacity{0.800000}%
\pgfsetdash{}{0pt}%
\pgfpathmoveto{\pgfqpoint{2.616545in}{4.569462in}}%
\pgfpathlineto{\pgfqpoint{3.502867in}{4.072531in}}%
\pgfusepath{stroke}%
\end{pgfscope}%
\begin{pgfscope}%
\pgfpathrectangle{\pgfqpoint{0.481978in}{0.331635in}}{\pgfqpoint{9.300000in}{7.700000in}}%
\pgfusepath{clip}%
\pgfsetrectcap%
\pgfsetroundjoin%
\pgfsetlinewidth{1.505625pt}%
\definecolor{currentstroke}{rgb}{1.000000,0.705882,0.509804}%
\pgfsetstrokecolor{currentstroke}%
\pgfsetstrokeopacity{0.800000}%
\pgfsetdash{}{0pt}%
\pgfpathmoveto{\pgfqpoint{5.192724in}{5.029818in}}%
\pgfpathlineto{\pgfqpoint{3.502867in}{4.072531in}}%
\pgfusepath{stroke}%
\end{pgfscope}%
\begin{pgfscope}%
\pgfpathrectangle{\pgfqpoint{0.481978in}{0.331635in}}{\pgfqpoint{9.300000in}{7.700000in}}%
\pgfusepath{clip}%
\pgfsetrectcap%
\pgfsetroundjoin%
\pgfsetlinewidth{1.505625pt}%
\definecolor{currentstroke}{rgb}{1.000000,0.705882,0.509804}%
\pgfsetstrokecolor{currentstroke}%
\pgfsetstrokeopacity{0.800000}%
\pgfsetdash{}{0pt}%
\pgfpathmoveto{\pgfqpoint{3.374008in}{2.987068in}}%
\pgfpathlineto{\pgfqpoint{3.502867in}{4.072531in}}%
\pgfusepath{stroke}%
\end{pgfscope}%
\begin{pgfscope}%
\pgfpathrectangle{\pgfqpoint{0.481978in}{0.331635in}}{\pgfqpoint{9.300000in}{7.700000in}}%
\pgfusepath{clip}%
\pgfsetrectcap%
\pgfsetroundjoin%
\pgfsetlinewidth{1.505625pt}%
\definecolor{currentstroke}{rgb}{1.000000,0.705882,0.509804}%
\pgfsetstrokecolor{currentstroke}%
\pgfsetstrokeopacity{0.800000}%
\pgfsetdash{}{0pt}%
\pgfpathmoveto{\pgfqpoint{4.257082in}{4.318425in}}%
\pgfpathlineto{\pgfqpoint{3.502867in}{4.072531in}}%
\pgfusepath{stroke}%
\end{pgfscope}%
\begin{pgfscope}%
\pgfpathrectangle{\pgfqpoint{0.481978in}{0.331635in}}{\pgfqpoint{9.300000in}{7.700000in}}%
\pgfusepath{clip}%
\pgfsetrectcap%
\pgfsetroundjoin%
\pgfsetlinewidth{1.505625pt}%
\definecolor{currentstroke}{rgb}{1.000000,0.705882,0.509804}%
\pgfsetstrokecolor{currentstroke}%
\pgfsetstrokeopacity{0.800000}%
\pgfsetdash{}{0pt}%
\pgfpathmoveto{\pgfqpoint{1.027576in}{3.786004in}}%
\pgfpathlineto{\pgfqpoint{3.502867in}{4.072531in}}%
\pgfusepath{stroke}%
\end{pgfscope}%
\begin{pgfscope}%
\pgfpathrectangle{\pgfqpoint{0.481978in}{0.331635in}}{\pgfqpoint{9.300000in}{7.700000in}}%
\pgfusepath{clip}%
\pgfsetrectcap%
\pgfsetroundjoin%
\pgfsetlinewidth{1.505625pt}%
\definecolor{currentstroke}{rgb}{1.000000,0.705882,0.509804}%
\pgfsetstrokecolor{currentstroke}%
\pgfsetstrokeopacity{0.800000}%
\pgfsetdash{}{0pt}%
\pgfpathmoveto{\pgfqpoint{2.111519in}{4.640866in}}%
\pgfpathlineto{\pgfqpoint{3.502867in}{4.072531in}}%
\pgfusepath{stroke}%
\end{pgfscope}%
\begin{pgfscope}%
\pgfpathrectangle{\pgfqpoint{0.481978in}{0.331635in}}{\pgfqpoint{9.300000in}{7.700000in}}%
\pgfusepath{clip}%
\pgfsetrectcap%
\pgfsetroundjoin%
\pgfsetlinewidth{1.505625pt}%
\definecolor{currentstroke}{rgb}{1.000000,0.705882,0.509804}%
\pgfsetstrokecolor{currentstroke}%
\pgfsetstrokeopacity{0.800000}%
\pgfsetdash{}{0pt}%
\pgfpathmoveto{\pgfqpoint{2.540709in}{6.147797in}}%
\pgfpathlineto{\pgfqpoint{3.502867in}{4.072531in}}%
\pgfusepath{stroke}%
\end{pgfscope}%
\begin{pgfscope}%
\pgfpathrectangle{\pgfqpoint{0.481978in}{0.331635in}}{\pgfqpoint{9.300000in}{7.700000in}}%
\pgfusepath{clip}%
\pgfsetrectcap%
\pgfsetroundjoin%
\pgfsetlinewidth{1.505625pt}%
\definecolor{currentstroke}{rgb}{1.000000,0.705882,0.509804}%
\pgfsetstrokecolor{currentstroke}%
\pgfsetstrokeopacity{0.800000}%
\pgfsetdash{}{0pt}%
\pgfpathmoveto{\pgfqpoint{2.593367in}{2.364960in}}%
\pgfpathlineto{\pgfqpoint{3.502867in}{4.072531in}}%
\pgfusepath{stroke}%
\end{pgfscope}%
\begin{pgfscope}%
\pgfpathrectangle{\pgfqpoint{0.481978in}{0.331635in}}{\pgfqpoint{9.300000in}{7.700000in}}%
\pgfusepath{clip}%
\pgfsetrectcap%
\pgfsetroundjoin%
\pgfsetlinewidth{1.505625pt}%
\definecolor{currentstroke}{rgb}{1.000000,0.705882,0.509804}%
\pgfsetstrokecolor{currentstroke}%
\pgfsetstrokeopacity{0.800000}%
\pgfsetdash{}{0pt}%
\pgfpathmoveto{\pgfqpoint{1.941607in}{3.957736in}}%
\pgfpathlineto{\pgfqpoint{3.502867in}{4.072531in}}%
\pgfusepath{stroke}%
\end{pgfscope}%
\begin{pgfscope}%
\pgfpathrectangle{\pgfqpoint{0.481978in}{0.331635in}}{\pgfqpoint{9.300000in}{7.700000in}}%
\pgfusepath{clip}%
\pgfsetrectcap%
\pgfsetroundjoin%
\pgfsetlinewidth{1.505625pt}%
\definecolor{currentstroke}{rgb}{1.000000,0.705882,0.509804}%
\pgfsetstrokecolor{currentstroke}%
\pgfsetstrokeopacity{0.800000}%
\pgfsetdash{}{0pt}%
\pgfpathmoveto{\pgfqpoint{2.880395in}{3.314771in}}%
\pgfpathlineto{\pgfqpoint{3.502867in}{4.072531in}}%
\pgfusepath{stroke}%
\end{pgfscope}%
\begin{pgfscope}%
\pgfpathrectangle{\pgfqpoint{0.481978in}{0.331635in}}{\pgfqpoint{9.300000in}{7.700000in}}%
\pgfusepath{clip}%
\pgfsetrectcap%
\pgfsetroundjoin%
\pgfsetlinewidth{1.505625pt}%
\definecolor{currentstroke}{rgb}{1.000000,0.705882,0.509804}%
\pgfsetstrokecolor{currentstroke}%
\pgfsetstrokeopacity{0.800000}%
\pgfsetdash{}{0pt}%
\pgfpathmoveto{\pgfqpoint{3.084100in}{2.952086in}}%
\pgfpathlineto{\pgfqpoint{3.502867in}{4.072531in}}%
\pgfusepath{stroke}%
\end{pgfscope}%
\begin{pgfscope}%
\pgfpathrectangle{\pgfqpoint{0.481978in}{0.331635in}}{\pgfqpoint{9.300000in}{7.700000in}}%
\pgfusepath{clip}%
\pgfsetrectcap%
\pgfsetroundjoin%
\pgfsetlinewidth{1.505625pt}%
\definecolor{currentstroke}{rgb}{1.000000,0.705882,0.509804}%
\pgfsetstrokecolor{currentstroke}%
\pgfsetstrokeopacity{0.800000}%
\pgfsetdash{}{0pt}%
\pgfpathmoveto{\pgfqpoint{5.368141in}{3.038043in}}%
\pgfpathlineto{\pgfqpoint{3.502867in}{4.072531in}}%
\pgfusepath{stroke}%
\end{pgfscope}%
\begin{pgfscope}%
\pgfpathrectangle{\pgfqpoint{0.481978in}{0.331635in}}{\pgfqpoint{9.300000in}{7.700000in}}%
\pgfusepath{clip}%
\pgfsetrectcap%
\pgfsetroundjoin%
\pgfsetlinewidth{1.505625pt}%
\definecolor{currentstroke}{rgb}{1.000000,0.705882,0.509804}%
\pgfsetstrokecolor{currentstroke}%
\pgfsetstrokeopacity{0.800000}%
\pgfsetdash{}{0pt}%
\pgfpathmoveto{\pgfqpoint{2.886916in}{3.927373in}}%
\pgfpathlineto{\pgfqpoint{3.502867in}{4.072531in}}%
\pgfusepath{stroke}%
\end{pgfscope}%
\begin{pgfscope}%
\pgfpathrectangle{\pgfqpoint{0.481978in}{0.331635in}}{\pgfqpoint{9.300000in}{7.700000in}}%
\pgfusepath{clip}%
\pgfsetrectcap%
\pgfsetroundjoin%
\pgfsetlinewidth{1.505625pt}%
\definecolor{currentstroke}{rgb}{1.000000,0.705882,0.509804}%
\pgfsetstrokecolor{currentstroke}%
\pgfsetstrokeopacity{0.800000}%
\pgfsetdash{}{0pt}%
\pgfpathmoveto{\pgfqpoint{4.450031in}{2.773919in}}%
\pgfpathlineto{\pgfqpoint{3.502867in}{4.072531in}}%
\pgfusepath{stroke}%
\end{pgfscope}%
\begin{pgfscope}%
\pgfpathrectangle{\pgfqpoint{0.481978in}{0.331635in}}{\pgfqpoint{9.300000in}{7.700000in}}%
\pgfusepath{clip}%
\pgfsetrectcap%
\pgfsetroundjoin%
\pgfsetlinewidth{1.505625pt}%
\definecolor{currentstroke}{rgb}{1.000000,0.705882,0.509804}%
\pgfsetstrokecolor{currentstroke}%
\pgfsetstrokeopacity{0.800000}%
\pgfsetdash{}{0pt}%
\pgfpathmoveto{\pgfqpoint{2.491094in}{5.202939in}}%
\pgfpathlineto{\pgfqpoint{3.502867in}{4.072531in}}%
\pgfusepath{stroke}%
\end{pgfscope}%
\begin{pgfscope}%
\pgfpathrectangle{\pgfqpoint{0.481978in}{0.331635in}}{\pgfqpoint{9.300000in}{7.700000in}}%
\pgfusepath{clip}%
\pgfsetrectcap%
\pgfsetroundjoin%
\pgfsetlinewidth{1.505625pt}%
\definecolor{currentstroke}{rgb}{1.000000,0.705882,0.509804}%
\pgfsetstrokecolor{currentstroke}%
\pgfsetstrokeopacity{0.800000}%
\pgfsetdash{}{0pt}%
\pgfpathmoveto{\pgfqpoint{2.662048in}{3.622560in}}%
\pgfpathlineto{\pgfqpoint{3.502867in}{4.072531in}}%
\pgfusepath{stroke}%
\end{pgfscope}%
\begin{pgfscope}%
\pgfpathrectangle{\pgfqpoint{0.481978in}{0.331635in}}{\pgfqpoint{9.300000in}{7.700000in}}%
\pgfusepath{clip}%
\pgfsetrectcap%
\pgfsetroundjoin%
\pgfsetlinewidth{1.505625pt}%
\definecolor{currentstroke}{rgb}{1.000000,0.705882,0.509804}%
\pgfsetstrokecolor{currentstroke}%
\pgfsetstrokeopacity{0.800000}%
\pgfsetdash{}{0pt}%
\pgfpathmoveto{\pgfqpoint{4.025822in}{3.612427in}}%
\pgfpathlineto{\pgfqpoint{3.502867in}{4.072531in}}%
\pgfusepath{stroke}%
\end{pgfscope}%
\begin{pgfscope}%
\pgfpathrectangle{\pgfqpoint{0.481978in}{0.331635in}}{\pgfqpoint{9.300000in}{7.700000in}}%
\pgfusepath{clip}%
\pgfsetrectcap%
\pgfsetroundjoin%
\pgfsetlinewidth{1.505625pt}%
\definecolor{currentstroke}{rgb}{1.000000,0.705882,0.509804}%
\pgfsetstrokecolor{currentstroke}%
\pgfsetstrokeopacity{0.800000}%
\pgfsetdash{}{0pt}%
\pgfpathmoveto{\pgfqpoint{3.490511in}{5.096925in}}%
\pgfpathlineto{\pgfqpoint{3.502867in}{4.072531in}}%
\pgfusepath{stroke}%
\end{pgfscope}%
\begin{pgfscope}%
\pgfpathrectangle{\pgfqpoint{0.481978in}{0.331635in}}{\pgfqpoint{9.300000in}{7.700000in}}%
\pgfusepath{clip}%
\pgfsetrectcap%
\pgfsetroundjoin%
\pgfsetlinewidth{1.505625pt}%
\definecolor{currentstroke}{rgb}{1.000000,0.705882,0.509804}%
\pgfsetstrokecolor{currentstroke}%
\pgfsetstrokeopacity{0.800000}%
\pgfsetdash{}{0pt}%
\pgfpathmoveto{\pgfqpoint{3.394544in}{3.192662in}}%
\pgfpathlineto{\pgfqpoint{3.502867in}{4.072531in}}%
\pgfusepath{stroke}%
\end{pgfscope}%
\begin{pgfscope}%
\pgfpathrectangle{\pgfqpoint{0.481978in}{0.331635in}}{\pgfqpoint{9.300000in}{7.700000in}}%
\pgfusepath{clip}%
\pgfsetrectcap%
\pgfsetroundjoin%
\pgfsetlinewidth{1.505625pt}%
\definecolor{currentstroke}{rgb}{1.000000,0.705882,0.509804}%
\pgfsetstrokecolor{currentstroke}%
\pgfsetstrokeopacity{0.800000}%
\pgfsetdash{}{0pt}%
\pgfpathmoveto{\pgfqpoint{3.295196in}{3.673341in}}%
\pgfpathlineto{\pgfqpoint{3.502867in}{4.072531in}}%
\pgfusepath{stroke}%
\end{pgfscope}%
\begin{pgfscope}%
\pgfpathrectangle{\pgfqpoint{0.481978in}{0.331635in}}{\pgfqpoint{9.300000in}{7.700000in}}%
\pgfusepath{clip}%
\pgfsetrectcap%
\pgfsetroundjoin%
\pgfsetlinewidth{1.505625pt}%
\definecolor{currentstroke}{rgb}{1.000000,0.705882,0.509804}%
\pgfsetstrokecolor{currentstroke}%
\pgfsetstrokeopacity{0.800000}%
\pgfsetdash{}{0pt}%
\pgfpathmoveto{\pgfqpoint{3.905090in}{5.004749in}}%
\pgfpathlineto{\pgfqpoint{3.502867in}{4.072531in}}%
\pgfusepath{stroke}%
\end{pgfscope}%
\begin{pgfscope}%
\pgfpathrectangle{\pgfqpoint{0.481978in}{0.331635in}}{\pgfqpoint{9.300000in}{7.700000in}}%
\pgfusepath{clip}%
\pgfsetrectcap%
\pgfsetroundjoin%
\pgfsetlinewidth{1.505625pt}%
\definecolor{currentstroke}{rgb}{1.000000,0.705882,0.509804}%
\pgfsetstrokecolor{currentstroke}%
\pgfsetstrokeopacity{0.800000}%
\pgfsetdash{}{0pt}%
\pgfpathmoveto{\pgfqpoint{4.067534in}{3.403561in}}%
\pgfpathlineto{\pgfqpoint{3.502867in}{4.072531in}}%
\pgfusepath{stroke}%
\end{pgfscope}%
\begin{pgfscope}%
\pgfpathrectangle{\pgfqpoint{0.481978in}{0.331635in}}{\pgfqpoint{9.300000in}{7.700000in}}%
\pgfusepath{clip}%
\pgfsetrectcap%
\pgfsetroundjoin%
\pgfsetlinewidth{1.505625pt}%
\definecolor{currentstroke}{rgb}{1.000000,0.705882,0.509804}%
\pgfsetstrokecolor{currentstroke}%
\pgfsetstrokeopacity{0.800000}%
\pgfsetdash{}{0pt}%
\pgfpathmoveto{\pgfqpoint{1.713787in}{3.877978in}}%
\pgfpathlineto{\pgfqpoint{3.502867in}{4.072531in}}%
\pgfusepath{stroke}%
\end{pgfscope}%
\begin{pgfscope}%
\pgfpathrectangle{\pgfqpoint{0.481978in}{0.331635in}}{\pgfqpoint{9.300000in}{7.700000in}}%
\pgfusepath{clip}%
\pgfsetrectcap%
\pgfsetroundjoin%
\pgfsetlinewidth{1.505625pt}%
\definecolor{currentstroke}{rgb}{1.000000,0.705882,0.509804}%
\pgfsetstrokecolor{currentstroke}%
\pgfsetstrokeopacity{0.800000}%
\pgfsetdash{}{0pt}%
\pgfpathmoveto{\pgfqpoint{2.059946in}{4.511724in}}%
\pgfpathlineto{\pgfqpoint{3.502867in}{4.072531in}}%
\pgfusepath{stroke}%
\end{pgfscope}%
\begin{pgfscope}%
\pgfpathrectangle{\pgfqpoint{0.481978in}{0.331635in}}{\pgfqpoint{9.300000in}{7.700000in}}%
\pgfusepath{clip}%
\pgfsetrectcap%
\pgfsetroundjoin%
\pgfsetlinewidth{1.505625pt}%
\definecolor{currentstroke}{rgb}{1.000000,0.705882,0.509804}%
\pgfsetstrokecolor{currentstroke}%
\pgfsetstrokeopacity{0.800000}%
\pgfsetdash{}{0pt}%
\pgfpathmoveto{\pgfqpoint{3.097846in}{5.855995in}}%
\pgfpathlineto{\pgfqpoint{3.502867in}{4.072531in}}%
\pgfusepath{stroke}%
\end{pgfscope}%
\begin{pgfscope}%
\pgfpathrectangle{\pgfqpoint{0.481978in}{0.331635in}}{\pgfqpoint{9.300000in}{7.700000in}}%
\pgfusepath{clip}%
\pgfsetrectcap%
\pgfsetroundjoin%
\pgfsetlinewidth{1.505625pt}%
\definecolor{currentstroke}{rgb}{1.000000,0.705882,0.509804}%
\pgfsetstrokecolor{currentstroke}%
\pgfsetstrokeopacity{0.800000}%
\pgfsetdash{}{0pt}%
\pgfpathmoveto{\pgfqpoint{2.416512in}{2.710548in}}%
\pgfpathlineto{\pgfqpoint{3.502867in}{4.072531in}}%
\pgfusepath{stroke}%
\end{pgfscope}%
\begin{pgfscope}%
\pgfpathrectangle{\pgfqpoint{0.481978in}{0.331635in}}{\pgfqpoint{9.300000in}{7.700000in}}%
\pgfusepath{clip}%
\pgfsetrectcap%
\pgfsetroundjoin%
\pgfsetlinewidth{1.505625pt}%
\definecolor{currentstroke}{rgb}{1.000000,0.705882,0.509804}%
\pgfsetstrokecolor{currentstroke}%
\pgfsetstrokeopacity{0.800000}%
\pgfsetdash{}{0pt}%
\pgfpathmoveto{\pgfqpoint{1.548965in}{4.260170in}}%
\pgfpathlineto{\pgfqpoint{3.502867in}{4.072531in}}%
\pgfusepath{stroke}%
\end{pgfscope}%
\begin{pgfscope}%
\pgfpathrectangle{\pgfqpoint{0.481978in}{0.331635in}}{\pgfqpoint{9.300000in}{7.700000in}}%
\pgfusepath{clip}%
\pgfsetrectcap%
\pgfsetroundjoin%
\pgfsetlinewidth{1.505625pt}%
\definecolor{currentstroke}{rgb}{1.000000,0.705882,0.509804}%
\pgfsetstrokecolor{currentstroke}%
\pgfsetstrokeopacity{0.800000}%
\pgfsetdash{}{0pt}%
\pgfpathmoveto{\pgfqpoint{4.223289in}{4.641562in}}%
\pgfpathlineto{\pgfqpoint{3.502867in}{4.072531in}}%
\pgfusepath{stroke}%
\end{pgfscope}%
\begin{pgfscope}%
\pgfpathrectangle{\pgfqpoint{0.481978in}{0.331635in}}{\pgfqpoint{9.300000in}{7.700000in}}%
\pgfusepath{clip}%
\pgfsetrectcap%
\pgfsetroundjoin%
\pgfsetlinewidth{1.505625pt}%
\definecolor{currentstroke}{rgb}{1.000000,0.705882,0.509804}%
\pgfsetstrokecolor{currentstroke}%
\pgfsetstrokeopacity{0.800000}%
\pgfsetdash{}{0pt}%
\pgfpathmoveto{\pgfqpoint{3.078663in}{3.602477in}}%
\pgfpathlineto{\pgfqpoint{3.502867in}{4.072531in}}%
\pgfusepath{stroke}%
\end{pgfscope}%
\begin{pgfscope}%
\pgfpathrectangle{\pgfqpoint{0.481978in}{0.331635in}}{\pgfqpoint{9.300000in}{7.700000in}}%
\pgfusepath{clip}%
\pgfsetrectcap%
\pgfsetroundjoin%
\pgfsetlinewidth{1.505625pt}%
\definecolor{currentstroke}{rgb}{1.000000,0.705882,0.509804}%
\pgfsetstrokecolor{currentstroke}%
\pgfsetstrokeopacity{0.800000}%
\pgfsetdash{}{0pt}%
\pgfpathmoveto{\pgfqpoint{4.163791in}{4.059342in}}%
\pgfpathlineto{\pgfqpoint{3.502867in}{4.072531in}}%
\pgfusepath{stroke}%
\end{pgfscope}%
\begin{pgfscope}%
\pgfpathrectangle{\pgfqpoint{0.481978in}{0.331635in}}{\pgfqpoint{9.300000in}{7.700000in}}%
\pgfusepath{clip}%
\pgfsetrectcap%
\pgfsetroundjoin%
\pgfsetlinewidth{1.505625pt}%
\definecolor{currentstroke}{rgb}{1.000000,0.705882,0.509804}%
\pgfsetstrokecolor{currentstroke}%
\pgfsetstrokeopacity{0.800000}%
\pgfsetdash{}{0pt}%
\pgfpathmoveto{\pgfqpoint{1.998764in}{4.830594in}}%
\pgfpathlineto{\pgfqpoint{3.502867in}{4.072531in}}%
\pgfusepath{stroke}%
\end{pgfscope}%
\begin{pgfscope}%
\pgfpathrectangle{\pgfqpoint{0.481978in}{0.331635in}}{\pgfqpoint{9.300000in}{7.700000in}}%
\pgfusepath{clip}%
\pgfsetrectcap%
\pgfsetroundjoin%
\pgfsetlinewidth{1.505625pt}%
\definecolor{currentstroke}{rgb}{1.000000,0.705882,0.509804}%
\pgfsetstrokecolor{currentstroke}%
\pgfsetstrokeopacity{0.800000}%
\pgfsetdash{}{0pt}%
\pgfpathmoveto{\pgfqpoint{4.371217in}{3.374706in}}%
\pgfpathlineto{\pgfqpoint{3.502867in}{4.072531in}}%
\pgfusepath{stroke}%
\end{pgfscope}%
\begin{pgfscope}%
\pgfpathrectangle{\pgfqpoint{0.481978in}{0.331635in}}{\pgfqpoint{9.300000in}{7.700000in}}%
\pgfusepath{clip}%
\pgfsetrectcap%
\pgfsetroundjoin%
\pgfsetlinewidth{1.505625pt}%
\definecolor{currentstroke}{rgb}{1.000000,0.705882,0.509804}%
\pgfsetstrokecolor{currentstroke}%
\pgfsetstrokeopacity{0.800000}%
\pgfsetdash{}{0pt}%
\pgfpathmoveto{\pgfqpoint{2.452590in}{4.246526in}}%
\pgfpathlineto{\pgfqpoint{3.502867in}{4.072531in}}%
\pgfusepath{stroke}%
\end{pgfscope}%
\begin{pgfscope}%
\pgfpathrectangle{\pgfqpoint{0.481978in}{0.331635in}}{\pgfqpoint{9.300000in}{7.700000in}}%
\pgfusepath{clip}%
\pgfsetrectcap%
\pgfsetroundjoin%
\pgfsetlinewidth{1.505625pt}%
\definecolor{currentstroke}{rgb}{1.000000,0.705882,0.509804}%
\pgfsetstrokecolor{currentstroke}%
\pgfsetstrokeopacity{0.800000}%
\pgfsetdash{}{0pt}%
\pgfpathmoveto{\pgfqpoint{4.281738in}{4.827328in}}%
\pgfpathlineto{\pgfqpoint{3.502867in}{4.072531in}}%
\pgfusepath{stroke}%
\end{pgfscope}%
\begin{pgfscope}%
\pgfpathrectangle{\pgfqpoint{0.481978in}{0.331635in}}{\pgfqpoint{9.300000in}{7.700000in}}%
\pgfusepath{clip}%
\pgfsetrectcap%
\pgfsetroundjoin%
\pgfsetlinewidth{1.505625pt}%
\definecolor{currentstroke}{rgb}{1.000000,0.705882,0.509804}%
\pgfsetstrokecolor{currentstroke}%
\pgfsetstrokeopacity{0.800000}%
\pgfsetdash{}{0pt}%
\pgfpathmoveto{\pgfqpoint{3.234311in}{3.438479in}}%
\pgfpathlineto{\pgfqpoint{3.502867in}{4.072531in}}%
\pgfusepath{stroke}%
\end{pgfscope}%
\begin{pgfscope}%
\pgfpathrectangle{\pgfqpoint{0.481978in}{0.331635in}}{\pgfqpoint{9.300000in}{7.700000in}}%
\pgfusepath{clip}%
\pgfsetrectcap%
\pgfsetroundjoin%
\pgfsetlinewidth{1.505625pt}%
\definecolor{currentstroke}{rgb}{1.000000,0.705882,0.509804}%
\pgfsetstrokecolor{currentstroke}%
\pgfsetstrokeopacity{0.800000}%
\pgfsetdash{}{0pt}%
\pgfpathmoveto{\pgfqpoint{3.915243in}{6.696360in}}%
\pgfpathlineto{\pgfqpoint{3.502867in}{4.072531in}}%
\pgfusepath{stroke}%
\end{pgfscope}%
\begin{pgfscope}%
\pgfpathrectangle{\pgfqpoint{0.481978in}{0.331635in}}{\pgfqpoint{9.300000in}{7.700000in}}%
\pgfusepath{clip}%
\pgfsetrectcap%
\pgfsetroundjoin%
\pgfsetlinewidth{1.505625pt}%
\definecolor{currentstroke}{rgb}{1.000000,0.705882,0.509804}%
\pgfsetstrokecolor{currentstroke}%
\pgfsetstrokeopacity{0.800000}%
\pgfsetdash{}{0pt}%
\pgfpathmoveto{\pgfqpoint{4.358866in}{3.185142in}}%
\pgfpathlineto{\pgfqpoint{3.502867in}{4.072531in}}%
\pgfusepath{stroke}%
\end{pgfscope}%
\begin{pgfscope}%
\pgfpathrectangle{\pgfqpoint{0.481978in}{0.331635in}}{\pgfqpoint{9.300000in}{7.700000in}}%
\pgfusepath{clip}%
\pgfsetrectcap%
\pgfsetroundjoin%
\pgfsetlinewidth{1.505625pt}%
\definecolor{currentstroke}{rgb}{1.000000,0.705882,0.509804}%
\pgfsetstrokecolor{currentstroke}%
\pgfsetstrokeopacity{0.800000}%
\pgfsetdash{}{0pt}%
\pgfpathmoveto{\pgfqpoint{4.512704in}{5.588863in}}%
\pgfpathlineto{\pgfqpoint{3.502867in}{4.072531in}}%
\pgfusepath{stroke}%
\end{pgfscope}%
\begin{pgfscope}%
\pgfpathrectangle{\pgfqpoint{0.481978in}{0.331635in}}{\pgfqpoint{9.300000in}{7.700000in}}%
\pgfusepath{clip}%
\pgfsetrectcap%
\pgfsetroundjoin%
\pgfsetlinewidth{1.505625pt}%
\definecolor{currentstroke}{rgb}{1.000000,0.705882,0.509804}%
\pgfsetstrokecolor{currentstroke}%
\pgfsetstrokeopacity{0.800000}%
\pgfsetdash{}{0pt}%
\pgfpathmoveto{\pgfqpoint{2.811663in}{4.156462in}}%
\pgfpathlineto{\pgfqpoint{3.502867in}{4.072531in}}%
\pgfusepath{stroke}%
\end{pgfscope}%
\begin{pgfscope}%
\pgfpathrectangle{\pgfqpoint{0.481978in}{0.331635in}}{\pgfqpoint{9.300000in}{7.700000in}}%
\pgfusepath{clip}%
\pgfsetrectcap%
\pgfsetroundjoin%
\pgfsetlinewidth{1.505625pt}%
\definecolor{currentstroke}{rgb}{1.000000,0.705882,0.509804}%
\pgfsetstrokecolor{currentstroke}%
\pgfsetstrokeopacity{0.800000}%
\pgfsetdash{}{0pt}%
\pgfpathmoveto{\pgfqpoint{4.849105in}{2.709988in}}%
\pgfpathlineto{\pgfqpoint{3.502867in}{4.072531in}}%
\pgfusepath{stroke}%
\end{pgfscope}%
\begin{pgfscope}%
\pgfpathrectangle{\pgfqpoint{0.481978in}{0.331635in}}{\pgfqpoint{9.300000in}{7.700000in}}%
\pgfusepath{clip}%
\pgfsetrectcap%
\pgfsetroundjoin%
\pgfsetlinewidth{1.505625pt}%
\definecolor{currentstroke}{rgb}{1.000000,0.705882,0.509804}%
\pgfsetstrokecolor{currentstroke}%
\pgfsetstrokeopacity{0.800000}%
\pgfsetdash{}{0pt}%
\pgfpathmoveto{\pgfqpoint{3.186199in}{3.178453in}}%
\pgfpathlineto{\pgfqpoint{3.502867in}{4.072531in}}%
\pgfusepath{stroke}%
\end{pgfscope}%
\begin{pgfscope}%
\pgfpathrectangle{\pgfqpoint{0.481978in}{0.331635in}}{\pgfqpoint{9.300000in}{7.700000in}}%
\pgfusepath{clip}%
\pgfsetrectcap%
\pgfsetroundjoin%
\pgfsetlinewidth{1.505625pt}%
\definecolor{currentstroke}{rgb}{1.000000,0.705882,0.509804}%
\pgfsetstrokecolor{currentstroke}%
\pgfsetstrokeopacity{0.800000}%
\pgfsetdash{}{0pt}%
\pgfpathmoveto{\pgfqpoint{2.733970in}{2.914683in}}%
\pgfpathlineto{\pgfqpoint{3.502867in}{4.072531in}}%
\pgfusepath{stroke}%
\end{pgfscope}%
\begin{pgfscope}%
\pgfpathrectangle{\pgfqpoint{0.481978in}{0.331635in}}{\pgfqpoint{9.300000in}{7.700000in}}%
\pgfusepath{clip}%
\pgfsetrectcap%
\pgfsetroundjoin%
\pgfsetlinewidth{1.505625pt}%
\definecolor{currentstroke}{rgb}{1.000000,0.705882,0.509804}%
\pgfsetstrokecolor{currentstroke}%
\pgfsetstrokeopacity{0.800000}%
\pgfsetdash{}{0pt}%
\pgfpathmoveto{\pgfqpoint{4.054746in}{2.700364in}}%
\pgfpathlineto{\pgfqpoint{3.502867in}{4.072531in}}%
\pgfusepath{stroke}%
\end{pgfscope}%
\begin{pgfscope}%
\pgfpathrectangle{\pgfqpoint{0.481978in}{0.331635in}}{\pgfqpoint{9.300000in}{7.700000in}}%
\pgfusepath{clip}%
\pgfsetrectcap%
\pgfsetroundjoin%
\pgfsetlinewidth{1.505625pt}%
\definecolor{currentstroke}{rgb}{1.000000,0.705882,0.509804}%
\pgfsetstrokecolor{currentstroke}%
\pgfsetstrokeopacity{0.800000}%
\pgfsetdash{}{0pt}%
\pgfpathmoveto{\pgfqpoint{3.726283in}{6.128973in}}%
\pgfpathlineto{\pgfqpoint{3.502867in}{4.072531in}}%
\pgfusepath{stroke}%
\end{pgfscope}%
\begin{pgfscope}%
\pgfpathrectangle{\pgfqpoint{0.481978in}{0.331635in}}{\pgfqpoint{9.300000in}{7.700000in}}%
\pgfusepath{clip}%
\pgfsetrectcap%
\pgfsetroundjoin%
\pgfsetlinewidth{1.505625pt}%
\definecolor{currentstroke}{rgb}{1.000000,0.705882,0.509804}%
\pgfsetstrokecolor{currentstroke}%
\pgfsetstrokeopacity{0.800000}%
\pgfsetdash{}{0pt}%
\pgfpathmoveto{\pgfqpoint{3.284360in}{6.107757in}}%
\pgfpathlineto{\pgfqpoint{3.502867in}{4.072531in}}%
\pgfusepath{stroke}%
\end{pgfscope}%
\begin{pgfscope}%
\pgfpathrectangle{\pgfqpoint{0.481978in}{0.331635in}}{\pgfqpoint{9.300000in}{7.700000in}}%
\pgfusepath{clip}%
\pgfsetrectcap%
\pgfsetroundjoin%
\pgfsetlinewidth{1.505625pt}%
\definecolor{currentstroke}{rgb}{1.000000,0.705882,0.509804}%
\pgfsetstrokecolor{currentstroke}%
\pgfsetstrokeopacity{0.800000}%
\pgfsetdash{}{0pt}%
\pgfpathmoveto{\pgfqpoint{4.144311in}{2.775514in}}%
\pgfpathlineto{\pgfqpoint{3.502867in}{4.072531in}}%
\pgfusepath{stroke}%
\end{pgfscope}%
\begin{pgfscope}%
\pgfpathrectangle{\pgfqpoint{0.481978in}{0.331635in}}{\pgfqpoint{9.300000in}{7.700000in}}%
\pgfusepath{clip}%
\pgfsetrectcap%
\pgfsetroundjoin%
\pgfsetlinewidth{1.505625pt}%
\definecolor{currentstroke}{rgb}{1.000000,0.705882,0.509804}%
\pgfsetstrokecolor{currentstroke}%
\pgfsetstrokeopacity{0.800000}%
\pgfsetdash{}{0pt}%
\pgfpathmoveto{\pgfqpoint{2.660738in}{2.616861in}}%
\pgfpathlineto{\pgfqpoint{3.502867in}{4.072531in}}%
\pgfusepath{stroke}%
\end{pgfscope}%
\begin{pgfscope}%
\pgfpathrectangle{\pgfqpoint{0.481978in}{0.331635in}}{\pgfqpoint{9.300000in}{7.700000in}}%
\pgfusepath{clip}%
\pgfsetrectcap%
\pgfsetroundjoin%
\pgfsetlinewidth{1.505625pt}%
\definecolor{currentstroke}{rgb}{1.000000,0.705882,0.509804}%
\pgfsetstrokecolor{currentstroke}%
\pgfsetstrokeopacity{0.800000}%
\pgfsetdash{}{0pt}%
\pgfpathmoveto{\pgfqpoint{3.685446in}{2.395770in}}%
\pgfpathlineto{\pgfqpoint{3.502867in}{4.072531in}}%
\pgfusepath{stroke}%
\end{pgfscope}%
\begin{pgfscope}%
\pgfpathrectangle{\pgfqpoint{0.481978in}{0.331635in}}{\pgfqpoint{9.300000in}{7.700000in}}%
\pgfusepath{clip}%
\pgfsetrectcap%
\pgfsetroundjoin%
\pgfsetlinewidth{1.505625pt}%
\definecolor{currentstroke}{rgb}{1.000000,0.705882,0.509804}%
\pgfsetstrokecolor{currentstroke}%
\pgfsetstrokeopacity{0.800000}%
\pgfsetdash{}{0pt}%
\pgfpathmoveto{\pgfqpoint{3.030052in}{5.761355in}}%
\pgfpathlineto{\pgfqpoint{3.502867in}{4.072531in}}%
\pgfusepath{stroke}%
\end{pgfscope}%
\begin{pgfscope}%
\pgfpathrectangle{\pgfqpoint{0.481978in}{0.331635in}}{\pgfqpoint{9.300000in}{7.700000in}}%
\pgfusepath{clip}%
\pgfsetrectcap%
\pgfsetroundjoin%
\pgfsetlinewidth{1.505625pt}%
\definecolor{currentstroke}{rgb}{1.000000,0.705882,0.509804}%
\pgfsetstrokecolor{currentstroke}%
\pgfsetstrokeopacity{0.800000}%
\pgfsetdash{}{0pt}%
\pgfpathmoveto{\pgfqpoint{1.530784in}{2.534423in}}%
\pgfpathlineto{\pgfqpoint{3.502867in}{4.072531in}}%
\pgfusepath{stroke}%
\end{pgfscope}%
\begin{pgfscope}%
\pgfpathrectangle{\pgfqpoint{0.481978in}{0.331635in}}{\pgfqpoint{9.300000in}{7.700000in}}%
\pgfusepath{clip}%
\pgfsetrectcap%
\pgfsetroundjoin%
\pgfsetlinewidth{1.505625pt}%
\definecolor{currentstroke}{rgb}{1.000000,0.705882,0.509804}%
\pgfsetstrokecolor{currentstroke}%
\pgfsetstrokeopacity{0.800000}%
\pgfsetdash{}{0pt}%
\pgfpathmoveto{\pgfqpoint{1.565141in}{5.100776in}}%
\pgfpathlineto{\pgfqpoint{3.502867in}{4.072531in}}%
\pgfusepath{stroke}%
\end{pgfscope}%
\begin{pgfscope}%
\pgfpathrectangle{\pgfqpoint{0.481978in}{0.331635in}}{\pgfqpoint{9.300000in}{7.700000in}}%
\pgfusepath{clip}%
\pgfsetrectcap%
\pgfsetroundjoin%
\pgfsetlinewidth{1.505625pt}%
\definecolor{currentstroke}{rgb}{1.000000,0.705882,0.509804}%
\pgfsetstrokecolor{currentstroke}%
\pgfsetstrokeopacity{0.800000}%
\pgfsetdash{}{0pt}%
\pgfpathmoveto{\pgfqpoint{4.138549in}{4.945210in}}%
\pgfpathlineto{\pgfqpoint{3.502867in}{4.072531in}}%
\pgfusepath{stroke}%
\end{pgfscope}%
\begin{pgfscope}%
\pgfpathrectangle{\pgfqpoint{0.481978in}{0.331635in}}{\pgfqpoint{9.300000in}{7.700000in}}%
\pgfusepath{clip}%
\pgfsetrectcap%
\pgfsetroundjoin%
\pgfsetlinewidth{1.505625pt}%
\definecolor{currentstroke}{rgb}{1.000000,0.705882,0.509804}%
\pgfsetstrokecolor{currentstroke}%
\pgfsetstrokeopacity{0.800000}%
\pgfsetdash{}{0pt}%
\pgfpathmoveto{\pgfqpoint{4.409605in}{4.766470in}}%
\pgfpathlineto{\pgfqpoint{3.502867in}{4.072531in}}%
\pgfusepath{stroke}%
\end{pgfscope}%
\begin{pgfscope}%
\pgfpathrectangle{\pgfqpoint{0.481978in}{0.331635in}}{\pgfqpoint{9.300000in}{7.700000in}}%
\pgfusepath{clip}%
\pgfsetrectcap%
\pgfsetroundjoin%
\pgfsetlinewidth{1.505625pt}%
\definecolor{currentstroke}{rgb}{1.000000,0.705882,0.509804}%
\pgfsetstrokecolor{currentstroke}%
\pgfsetstrokeopacity{0.800000}%
\pgfsetdash{}{0pt}%
\pgfpathmoveto{\pgfqpoint{4.134948in}{3.165826in}}%
\pgfpathlineto{\pgfqpoint{3.502867in}{4.072531in}}%
\pgfusepath{stroke}%
\end{pgfscope}%
\begin{pgfscope}%
\pgfpathrectangle{\pgfqpoint{0.481978in}{0.331635in}}{\pgfqpoint{9.300000in}{7.700000in}}%
\pgfusepath{clip}%
\pgfsetrectcap%
\pgfsetroundjoin%
\pgfsetlinewidth{1.505625pt}%
\definecolor{currentstroke}{rgb}{1.000000,0.705882,0.509804}%
\pgfsetstrokecolor{currentstroke}%
\pgfsetstrokeopacity{0.800000}%
\pgfsetdash{}{0pt}%
\pgfpathmoveto{\pgfqpoint{2.632233in}{4.374697in}}%
\pgfpathlineto{\pgfqpoint{3.502867in}{4.072531in}}%
\pgfusepath{stroke}%
\end{pgfscope}%
\begin{pgfscope}%
\pgfpathrectangle{\pgfqpoint{0.481978in}{0.331635in}}{\pgfqpoint{9.300000in}{7.700000in}}%
\pgfusepath{clip}%
\pgfsetrectcap%
\pgfsetroundjoin%
\pgfsetlinewidth{1.505625pt}%
\definecolor{currentstroke}{rgb}{1.000000,0.705882,0.509804}%
\pgfsetstrokecolor{currentstroke}%
\pgfsetstrokeopacity{0.800000}%
\pgfsetdash{}{0pt}%
\pgfpathmoveto{\pgfqpoint{3.332817in}{4.302567in}}%
\pgfpathlineto{\pgfqpoint{3.502867in}{4.072531in}}%
\pgfusepath{stroke}%
\end{pgfscope}%
\begin{pgfscope}%
\pgfpathrectangle{\pgfqpoint{0.481978in}{0.331635in}}{\pgfqpoint{9.300000in}{7.700000in}}%
\pgfusepath{clip}%
\pgfsetrectcap%
\pgfsetroundjoin%
\pgfsetlinewidth{1.505625pt}%
\definecolor{currentstroke}{rgb}{1.000000,0.705882,0.509804}%
\pgfsetstrokecolor{currentstroke}%
\pgfsetstrokeopacity{0.800000}%
\pgfsetdash{}{0pt}%
\pgfpathmoveto{\pgfqpoint{3.045518in}{4.284260in}}%
\pgfpathlineto{\pgfqpoint{3.502867in}{4.072531in}}%
\pgfusepath{stroke}%
\end{pgfscope}%
\begin{pgfscope}%
\pgfpathrectangle{\pgfqpoint{0.481978in}{0.331635in}}{\pgfqpoint{9.300000in}{7.700000in}}%
\pgfusepath{clip}%
\pgfsetrectcap%
\pgfsetroundjoin%
\pgfsetlinewidth{1.505625pt}%
\definecolor{currentstroke}{rgb}{1.000000,0.705882,0.509804}%
\pgfsetstrokecolor{currentstroke}%
\pgfsetstrokeopacity{0.800000}%
\pgfsetdash{}{0pt}%
\pgfpathmoveto{\pgfqpoint{2.948753in}{3.450151in}}%
\pgfpathlineto{\pgfqpoint{3.502867in}{4.072531in}}%
\pgfusepath{stroke}%
\end{pgfscope}%
\begin{pgfscope}%
\pgfpathrectangle{\pgfqpoint{0.481978in}{0.331635in}}{\pgfqpoint{9.300000in}{7.700000in}}%
\pgfusepath{clip}%
\pgfsetrectcap%
\pgfsetroundjoin%
\pgfsetlinewidth{1.505625pt}%
\definecolor{currentstroke}{rgb}{1.000000,0.705882,0.509804}%
\pgfsetstrokecolor{currentstroke}%
\pgfsetstrokeopacity{0.800000}%
\pgfsetdash{}{0pt}%
\pgfpathmoveto{\pgfqpoint{3.979638in}{3.616382in}}%
\pgfpathlineto{\pgfqpoint{3.502867in}{4.072531in}}%
\pgfusepath{stroke}%
\end{pgfscope}%
\begin{pgfscope}%
\pgfpathrectangle{\pgfqpoint{0.481978in}{0.331635in}}{\pgfqpoint{9.300000in}{7.700000in}}%
\pgfusepath{clip}%
\pgfsetrectcap%
\pgfsetroundjoin%
\pgfsetlinewidth{1.505625pt}%
\definecolor{currentstroke}{rgb}{1.000000,0.705882,0.509804}%
\pgfsetstrokecolor{currentstroke}%
\pgfsetstrokeopacity{0.800000}%
\pgfsetdash{}{0pt}%
\pgfpathmoveto{\pgfqpoint{4.333852in}{2.451747in}}%
\pgfpathlineto{\pgfqpoint{3.502867in}{4.072531in}}%
\pgfusepath{stroke}%
\end{pgfscope}%
\begin{pgfscope}%
\pgfpathrectangle{\pgfqpoint{0.481978in}{0.331635in}}{\pgfqpoint{9.300000in}{7.700000in}}%
\pgfusepath{clip}%
\pgfsetrectcap%
\pgfsetroundjoin%
\pgfsetlinewidth{1.505625pt}%
\definecolor{currentstroke}{rgb}{1.000000,0.705882,0.509804}%
\pgfsetstrokecolor{currentstroke}%
\pgfsetstrokeopacity{0.800000}%
\pgfsetdash{}{0pt}%
\pgfpathmoveto{\pgfqpoint{1.939275in}{4.224539in}}%
\pgfpathlineto{\pgfqpoint{3.502867in}{4.072531in}}%
\pgfusepath{stroke}%
\end{pgfscope}%
\begin{pgfscope}%
\pgfpathrectangle{\pgfqpoint{0.481978in}{0.331635in}}{\pgfqpoint{9.300000in}{7.700000in}}%
\pgfusepath{clip}%
\pgfsetrectcap%
\pgfsetroundjoin%
\pgfsetlinewidth{1.505625pt}%
\definecolor{currentstroke}{rgb}{1.000000,0.705882,0.509804}%
\pgfsetstrokecolor{currentstroke}%
\pgfsetstrokeopacity{0.800000}%
\pgfsetdash{}{0pt}%
\pgfpathmoveto{\pgfqpoint{4.479280in}{2.911311in}}%
\pgfpathlineto{\pgfqpoint{3.502867in}{4.072531in}}%
\pgfusepath{stroke}%
\end{pgfscope}%
\begin{pgfscope}%
\pgfpathrectangle{\pgfqpoint{0.481978in}{0.331635in}}{\pgfqpoint{9.300000in}{7.700000in}}%
\pgfusepath{clip}%
\pgfsetrectcap%
\pgfsetroundjoin%
\pgfsetlinewidth{1.505625pt}%
\definecolor{currentstroke}{rgb}{1.000000,0.705882,0.509804}%
\pgfsetstrokecolor{currentstroke}%
\pgfsetstrokeopacity{0.800000}%
\pgfsetdash{}{0pt}%
\pgfpathmoveto{\pgfqpoint{3.013348in}{2.644342in}}%
\pgfpathlineto{\pgfqpoint{3.502867in}{4.072531in}}%
\pgfusepath{stroke}%
\end{pgfscope}%
\begin{pgfscope}%
\pgfpathrectangle{\pgfqpoint{0.481978in}{0.331635in}}{\pgfqpoint{9.300000in}{7.700000in}}%
\pgfusepath{clip}%
\pgfsetrectcap%
\pgfsetroundjoin%
\pgfsetlinewidth{1.505625pt}%
\definecolor{currentstroke}{rgb}{1.000000,0.705882,0.509804}%
\pgfsetstrokecolor{currentstroke}%
\pgfsetstrokeopacity{0.800000}%
\pgfsetdash{}{0pt}%
\pgfpathmoveto{\pgfqpoint{4.165514in}{6.528278in}}%
\pgfpathlineto{\pgfqpoint{3.502867in}{4.072531in}}%
\pgfusepath{stroke}%
\end{pgfscope}%
\begin{pgfscope}%
\pgfpathrectangle{\pgfqpoint{0.481978in}{0.331635in}}{\pgfqpoint{9.300000in}{7.700000in}}%
\pgfusepath{clip}%
\pgfsetrectcap%
\pgfsetroundjoin%
\pgfsetlinewidth{1.505625pt}%
\definecolor{currentstroke}{rgb}{1.000000,0.705882,0.509804}%
\pgfsetstrokecolor{currentstroke}%
\pgfsetstrokeopacity{0.800000}%
\pgfsetdash{}{0pt}%
\pgfpathmoveto{\pgfqpoint{2.157737in}{4.152260in}}%
\pgfpathlineto{\pgfqpoint{3.502867in}{4.072531in}}%
\pgfusepath{stroke}%
\end{pgfscope}%
\begin{pgfscope}%
\pgfpathrectangle{\pgfqpoint{0.481978in}{0.331635in}}{\pgfqpoint{9.300000in}{7.700000in}}%
\pgfusepath{clip}%
\pgfsetrectcap%
\pgfsetroundjoin%
\pgfsetlinewidth{1.505625pt}%
\definecolor{currentstroke}{rgb}{1.000000,0.705882,0.509804}%
\pgfsetstrokecolor{currentstroke}%
\pgfsetstrokeopacity{0.800000}%
\pgfsetdash{}{0pt}%
\pgfpathmoveto{\pgfqpoint{2.216691in}{5.408235in}}%
\pgfpathlineto{\pgfqpoint{3.502867in}{4.072531in}}%
\pgfusepath{stroke}%
\end{pgfscope}%
\begin{pgfscope}%
\pgfpathrectangle{\pgfqpoint{0.481978in}{0.331635in}}{\pgfqpoint{9.300000in}{7.700000in}}%
\pgfusepath{clip}%
\pgfsetrectcap%
\pgfsetroundjoin%
\pgfsetlinewidth{1.505625pt}%
\definecolor{currentstroke}{rgb}{1.000000,0.705882,0.509804}%
\pgfsetstrokecolor{currentstroke}%
\pgfsetstrokeopacity{0.800000}%
\pgfsetdash{}{0pt}%
\pgfpathmoveto{\pgfqpoint{4.655686in}{2.246792in}}%
\pgfpathlineto{\pgfqpoint{3.502867in}{4.072531in}}%
\pgfusepath{stroke}%
\end{pgfscope}%
\begin{pgfscope}%
\pgfpathrectangle{\pgfqpoint{0.481978in}{0.331635in}}{\pgfqpoint{9.300000in}{7.700000in}}%
\pgfusepath{clip}%
\pgfsetrectcap%
\pgfsetroundjoin%
\pgfsetlinewidth{1.505625pt}%
\definecolor{currentstroke}{rgb}{1.000000,0.705882,0.509804}%
\pgfsetstrokecolor{currentstroke}%
\pgfsetstrokeopacity{0.800000}%
\pgfsetdash{}{0pt}%
\pgfpathmoveto{\pgfqpoint{3.175062in}{3.348731in}}%
\pgfpathlineto{\pgfqpoint{3.502867in}{4.072531in}}%
\pgfusepath{stroke}%
\end{pgfscope}%
\begin{pgfscope}%
\pgfpathrectangle{\pgfqpoint{0.481978in}{0.331635in}}{\pgfqpoint{9.300000in}{7.700000in}}%
\pgfusepath{clip}%
\pgfsetrectcap%
\pgfsetroundjoin%
\pgfsetlinewidth{1.505625pt}%
\definecolor{currentstroke}{rgb}{1.000000,0.705882,0.509804}%
\pgfsetstrokecolor{currentstroke}%
\pgfsetstrokeopacity{0.800000}%
\pgfsetdash{}{0pt}%
\pgfpathmoveto{\pgfqpoint{2.367140in}{4.716464in}}%
\pgfpathlineto{\pgfqpoint{3.502867in}{4.072531in}}%
\pgfusepath{stroke}%
\end{pgfscope}%
\begin{pgfscope}%
\pgfpathrectangle{\pgfqpoint{0.481978in}{0.331635in}}{\pgfqpoint{9.300000in}{7.700000in}}%
\pgfusepath{clip}%
\pgfsetrectcap%
\pgfsetroundjoin%
\pgfsetlinewidth{1.505625pt}%
\definecolor{currentstroke}{rgb}{1.000000,0.705882,0.509804}%
\pgfsetstrokecolor{currentstroke}%
\pgfsetstrokeopacity{0.800000}%
\pgfsetdash{}{0pt}%
\pgfpathmoveto{\pgfqpoint{4.838472in}{3.036006in}}%
\pgfpathlineto{\pgfqpoint{3.502867in}{4.072531in}}%
\pgfusepath{stroke}%
\end{pgfscope}%
\begin{pgfscope}%
\pgfpathrectangle{\pgfqpoint{0.481978in}{0.331635in}}{\pgfqpoint{9.300000in}{7.700000in}}%
\pgfusepath{clip}%
\pgfsetrectcap%
\pgfsetroundjoin%
\pgfsetlinewidth{1.505625pt}%
\definecolor{currentstroke}{rgb}{1.000000,0.705882,0.509804}%
\pgfsetstrokecolor{currentstroke}%
\pgfsetstrokeopacity{0.800000}%
\pgfsetdash{}{0pt}%
\pgfpathmoveto{\pgfqpoint{4.322216in}{5.047571in}}%
\pgfpathlineto{\pgfqpoint{3.502867in}{4.072531in}}%
\pgfusepath{stroke}%
\end{pgfscope}%
\begin{pgfscope}%
\pgfpathrectangle{\pgfqpoint{0.481978in}{0.331635in}}{\pgfqpoint{9.300000in}{7.700000in}}%
\pgfusepath{clip}%
\pgfsetrectcap%
\pgfsetroundjoin%
\pgfsetlinewidth{1.505625pt}%
\definecolor{currentstroke}{rgb}{1.000000,0.705882,0.509804}%
\pgfsetstrokecolor{currentstroke}%
\pgfsetstrokeopacity{0.800000}%
\pgfsetdash{}{0pt}%
\pgfpathmoveto{\pgfqpoint{3.615208in}{3.463945in}}%
\pgfpathlineto{\pgfqpoint{3.502867in}{4.072531in}}%
\pgfusepath{stroke}%
\end{pgfscope}%
\begin{pgfscope}%
\pgfpathrectangle{\pgfqpoint{0.481978in}{0.331635in}}{\pgfqpoint{9.300000in}{7.700000in}}%
\pgfusepath{clip}%
\pgfsetrectcap%
\pgfsetroundjoin%
\pgfsetlinewidth{1.505625pt}%
\definecolor{currentstroke}{rgb}{1.000000,0.705882,0.509804}%
\pgfsetstrokecolor{currentstroke}%
\pgfsetstrokeopacity{0.800000}%
\pgfsetdash{}{0pt}%
\pgfpathmoveto{\pgfqpoint{3.424608in}{6.947120in}}%
\pgfpathlineto{\pgfqpoint{3.502867in}{4.072531in}}%
\pgfusepath{stroke}%
\end{pgfscope}%
\begin{pgfscope}%
\pgfpathrectangle{\pgfqpoint{0.481978in}{0.331635in}}{\pgfqpoint{9.300000in}{7.700000in}}%
\pgfusepath{clip}%
\pgfsetrectcap%
\pgfsetroundjoin%
\pgfsetlinewidth{1.505625pt}%
\definecolor{currentstroke}{rgb}{1.000000,0.705882,0.509804}%
\pgfsetstrokecolor{currentstroke}%
\pgfsetstrokeopacity{0.800000}%
\pgfsetdash{}{0pt}%
\pgfpathmoveto{\pgfqpoint{5.585402in}{5.240124in}}%
\pgfpathlineto{\pgfqpoint{3.502867in}{4.072531in}}%
\pgfusepath{stroke}%
\end{pgfscope}%
\begin{pgfscope}%
\pgfpathrectangle{\pgfqpoint{0.481978in}{0.331635in}}{\pgfqpoint{9.300000in}{7.700000in}}%
\pgfusepath{clip}%
\pgfsetrectcap%
\pgfsetroundjoin%
\pgfsetlinewidth{1.505625pt}%
\definecolor{currentstroke}{rgb}{1.000000,0.705882,0.509804}%
\pgfsetstrokecolor{currentstroke}%
\pgfsetstrokeopacity{0.800000}%
\pgfsetdash{}{0pt}%
\pgfpathmoveto{\pgfqpoint{3.153342in}{5.109707in}}%
\pgfpathlineto{\pgfqpoint{3.502867in}{4.072531in}}%
\pgfusepath{stroke}%
\end{pgfscope}%
\begin{pgfscope}%
\pgfpathrectangle{\pgfqpoint{0.481978in}{0.331635in}}{\pgfqpoint{9.300000in}{7.700000in}}%
\pgfusepath{clip}%
\pgfsetrectcap%
\pgfsetroundjoin%
\pgfsetlinewidth{1.505625pt}%
\definecolor{currentstroke}{rgb}{1.000000,0.705882,0.509804}%
\pgfsetstrokecolor{currentstroke}%
\pgfsetstrokeopacity{0.800000}%
\pgfsetdash{}{0pt}%
\pgfpathmoveto{\pgfqpoint{3.297548in}{5.482494in}}%
\pgfpathlineto{\pgfqpoint{3.502867in}{4.072531in}}%
\pgfusepath{stroke}%
\end{pgfscope}%
\begin{pgfscope}%
\pgfpathrectangle{\pgfqpoint{0.481978in}{0.331635in}}{\pgfqpoint{9.300000in}{7.700000in}}%
\pgfusepath{clip}%
\pgfsetrectcap%
\pgfsetroundjoin%
\pgfsetlinewidth{1.505625pt}%
\definecolor{currentstroke}{rgb}{1.000000,0.705882,0.509804}%
\pgfsetstrokecolor{currentstroke}%
\pgfsetstrokeopacity{0.800000}%
\pgfsetdash{}{0pt}%
\pgfpathmoveto{\pgfqpoint{4.669793in}{2.259600in}}%
\pgfpathlineto{\pgfqpoint{3.502867in}{4.072531in}}%
\pgfusepath{stroke}%
\end{pgfscope}%
\begin{pgfscope}%
\pgfpathrectangle{\pgfqpoint{0.481978in}{0.331635in}}{\pgfqpoint{9.300000in}{7.700000in}}%
\pgfusepath{clip}%
\pgfsetrectcap%
\pgfsetroundjoin%
\pgfsetlinewidth{1.505625pt}%
\definecolor{currentstroke}{rgb}{1.000000,0.705882,0.509804}%
\pgfsetstrokecolor{currentstroke}%
\pgfsetstrokeopacity{0.800000}%
\pgfsetdash{}{0pt}%
\pgfpathmoveto{\pgfqpoint{1.546741in}{5.145333in}}%
\pgfpathlineto{\pgfqpoint{3.502867in}{4.072531in}}%
\pgfusepath{stroke}%
\end{pgfscope}%
\begin{pgfscope}%
\pgfpathrectangle{\pgfqpoint{0.481978in}{0.331635in}}{\pgfqpoint{9.300000in}{7.700000in}}%
\pgfusepath{clip}%
\pgfsetrectcap%
\pgfsetroundjoin%
\pgfsetlinewidth{1.505625pt}%
\definecolor{currentstroke}{rgb}{1.000000,0.705882,0.509804}%
\pgfsetstrokecolor{currentstroke}%
\pgfsetstrokeopacity{0.800000}%
\pgfsetdash{}{0pt}%
\pgfpathmoveto{\pgfqpoint{4.426621in}{5.394791in}}%
\pgfpathlineto{\pgfqpoint{3.502867in}{4.072531in}}%
\pgfusepath{stroke}%
\end{pgfscope}%
\begin{pgfscope}%
\pgfpathrectangle{\pgfqpoint{0.481978in}{0.331635in}}{\pgfqpoint{9.300000in}{7.700000in}}%
\pgfusepath{clip}%
\pgfsetrectcap%
\pgfsetroundjoin%
\pgfsetlinewidth{1.505625pt}%
\definecolor{currentstroke}{rgb}{1.000000,0.705882,0.509804}%
\pgfsetstrokecolor{currentstroke}%
\pgfsetstrokeopacity{0.800000}%
\pgfsetdash{}{0pt}%
\pgfpathmoveto{\pgfqpoint{2.946822in}{5.353658in}}%
\pgfpathlineto{\pgfqpoint{3.502867in}{4.072531in}}%
\pgfusepath{stroke}%
\end{pgfscope}%
\begin{pgfscope}%
\pgfpathrectangle{\pgfqpoint{0.481978in}{0.331635in}}{\pgfqpoint{9.300000in}{7.700000in}}%
\pgfusepath{clip}%
\pgfsetrectcap%
\pgfsetroundjoin%
\pgfsetlinewidth{1.505625pt}%
\definecolor{currentstroke}{rgb}{1.000000,0.705882,0.509804}%
\pgfsetstrokecolor{currentstroke}%
\pgfsetstrokeopacity{0.800000}%
\pgfsetdash{}{0pt}%
\pgfpathmoveto{\pgfqpoint{9.359251in}{1.365147in}}%
\pgfpathlineto{\pgfqpoint{3.502867in}{4.072531in}}%
\pgfusepath{stroke}%
\end{pgfscope}%
\begin{pgfscope}%
\pgfpathrectangle{\pgfqpoint{0.481978in}{0.331635in}}{\pgfqpoint{9.300000in}{7.700000in}}%
\pgfusepath{clip}%
\pgfsetrectcap%
\pgfsetroundjoin%
\pgfsetlinewidth{1.505625pt}%
\definecolor{currentstroke}{rgb}{1.000000,0.705882,0.509804}%
\pgfsetstrokecolor{currentstroke}%
\pgfsetstrokeopacity{0.800000}%
\pgfsetdash{}{0pt}%
\pgfpathmoveto{\pgfqpoint{3.311383in}{5.697453in}}%
\pgfpathlineto{\pgfqpoint{3.502867in}{4.072531in}}%
\pgfusepath{stroke}%
\end{pgfscope}%
\begin{pgfscope}%
\pgfpathrectangle{\pgfqpoint{0.481978in}{0.331635in}}{\pgfqpoint{9.300000in}{7.700000in}}%
\pgfusepath{clip}%
\pgfsetrectcap%
\pgfsetroundjoin%
\pgfsetlinewidth{1.505625pt}%
\definecolor{currentstroke}{rgb}{1.000000,0.705882,0.509804}%
\pgfsetstrokecolor{currentstroke}%
\pgfsetstrokeopacity{0.800000}%
\pgfsetdash{}{0pt}%
\pgfpathmoveto{\pgfqpoint{3.768020in}{3.858089in}}%
\pgfpathlineto{\pgfqpoint{3.502867in}{4.072531in}}%
\pgfusepath{stroke}%
\end{pgfscope}%
\begin{pgfscope}%
\pgfpathrectangle{\pgfqpoint{0.481978in}{0.331635in}}{\pgfqpoint{9.300000in}{7.700000in}}%
\pgfusepath{clip}%
\pgfsetrectcap%
\pgfsetroundjoin%
\pgfsetlinewidth{1.505625pt}%
\definecolor{currentstroke}{rgb}{1.000000,0.705882,0.509804}%
\pgfsetstrokecolor{currentstroke}%
\pgfsetstrokeopacity{0.800000}%
\pgfsetdash{}{0pt}%
\pgfpathmoveto{\pgfqpoint{3.827532in}{5.121296in}}%
\pgfpathlineto{\pgfqpoint{3.502867in}{4.072531in}}%
\pgfusepath{stroke}%
\end{pgfscope}%
\begin{pgfscope}%
\pgfpathrectangle{\pgfqpoint{0.481978in}{0.331635in}}{\pgfqpoint{9.300000in}{7.700000in}}%
\pgfusepath{clip}%
\pgfsetrectcap%
\pgfsetroundjoin%
\pgfsetlinewidth{1.505625pt}%
\definecolor{currentstroke}{rgb}{1.000000,0.705882,0.509804}%
\pgfsetstrokecolor{currentstroke}%
\pgfsetstrokeopacity{0.800000}%
\pgfsetdash{}{0pt}%
\pgfpathmoveto{\pgfqpoint{1.059884in}{2.192019in}}%
\pgfpathlineto{\pgfqpoint{3.502867in}{4.072531in}}%
\pgfusepath{stroke}%
\end{pgfscope}%
\begin{pgfscope}%
\pgfpathrectangle{\pgfqpoint{0.481978in}{0.331635in}}{\pgfqpoint{9.300000in}{7.700000in}}%
\pgfusepath{clip}%
\pgfsetrectcap%
\pgfsetroundjoin%
\pgfsetlinewidth{1.505625pt}%
\definecolor{currentstroke}{rgb}{1.000000,0.705882,0.509804}%
\pgfsetstrokecolor{currentstroke}%
\pgfsetstrokeopacity{0.800000}%
\pgfsetdash{}{0pt}%
\pgfpathmoveto{\pgfqpoint{4.209058in}{3.248890in}}%
\pgfpathlineto{\pgfqpoint{3.502867in}{4.072531in}}%
\pgfusepath{stroke}%
\end{pgfscope}%
\begin{pgfscope}%
\pgfpathrectangle{\pgfqpoint{0.481978in}{0.331635in}}{\pgfqpoint{9.300000in}{7.700000in}}%
\pgfusepath{clip}%
\pgfsetrectcap%
\pgfsetroundjoin%
\pgfsetlinewidth{1.505625pt}%
\definecolor{currentstroke}{rgb}{1.000000,0.705882,0.509804}%
\pgfsetstrokecolor{currentstroke}%
\pgfsetstrokeopacity{0.800000}%
\pgfsetdash{}{0pt}%
\pgfpathmoveto{\pgfqpoint{3.072069in}{2.785754in}}%
\pgfpathlineto{\pgfqpoint{3.502867in}{4.072531in}}%
\pgfusepath{stroke}%
\end{pgfscope}%
\begin{pgfscope}%
\pgfpathrectangle{\pgfqpoint{0.481978in}{0.331635in}}{\pgfqpoint{9.300000in}{7.700000in}}%
\pgfusepath{clip}%
\pgfsetrectcap%
\pgfsetroundjoin%
\pgfsetlinewidth{1.505625pt}%
\definecolor{currentstroke}{rgb}{1.000000,0.705882,0.509804}%
\pgfsetstrokecolor{currentstroke}%
\pgfsetstrokeopacity{0.800000}%
\pgfsetdash{}{0pt}%
\pgfpathmoveto{\pgfqpoint{4.167601in}{3.812695in}}%
\pgfpathlineto{\pgfqpoint{3.502867in}{4.072531in}}%
\pgfusepath{stroke}%
\end{pgfscope}%
\begin{pgfscope}%
\pgfpathrectangle{\pgfqpoint{0.481978in}{0.331635in}}{\pgfqpoint{9.300000in}{7.700000in}}%
\pgfusepath{clip}%
\pgfsetrectcap%
\pgfsetroundjoin%
\pgfsetlinewidth{1.505625pt}%
\definecolor{currentstroke}{rgb}{1.000000,0.705882,0.509804}%
\pgfsetstrokecolor{currentstroke}%
\pgfsetstrokeopacity{0.800000}%
\pgfsetdash{}{0pt}%
\pgfpathmoveto{\pgfqpoint{4.458251in}{2.550423in}}%
\pgfpathlineto{\pgfqpoint{3.502867in}{4.072531in}}%
\pgfusepath{stroke}%
\end{pgfscope}%
\begin{pgfscope}%
\pgfpathrectangle{\pgfqpoint{0.481978in}{0.331635in}}{\pgfqpoint{9.300000in}{7.700000in}}%
\pgfusepath{clip}%
\pgfsetrectcap%
\pgfsetroundjoin%
\pgfsetlinewidth{1.505625pt}%
\definecolor{currentstroke}{rgb}{1.000000,0.705882,0.509804}%
\pgfsetstrokecolor{currentstroke}%
\pgfsetstrokeopacity{0.800000}%
\pgfsetdash{}{0pt}%
\pgfpathmoveto{\pgfqpoint{3.705775in}{4.608509in}}%
\pgfpathlineto{\pgfqpoint{3.502867in}{4.072531in}}%
\pgfusepath{stroke}%
\end{pgfscope}%
\begin{pgfscope}%
\pgfpathrectangle{\pgfqpoint{0.481978in}{0.331635in}}{\pgfqpoint{9.300000in}{7.700000in}}%
\pgfusepath{clip}%
\pgfsetrectcap%
\pgfsetroundjoin%
\pgfsetlinewidth{1.505625pt}%
\definecolor{currentstroke}{rgb}{1.000000,0.705882,0.509804}%
\pgfsetstrokecolor{currentstroke}%
\pgfsetstrokeopacity{0.800000}%
\pgfsetdash{}{0pt}%
\pgfpathmoveto{\pgfqpoint{3.840708in}{2.854435in}}%
\pgfpathlineto{\pgfqpoint{3.502867in}{4.072531in}}%
\pgfusepath{stroke}%
\end{pgfscope}%
\begin{pgfscope}%
\pgfpathrectangle{\pgfqpoint{0.481978in}{0.331635in}}{\pgfqpoint{9.300000in}{7.700000in}}%
\pgfusepath{clip}%
\pgfsetrectcap%
\pgfsetroundjoin%
\pgfsetlinewidth{1.505625pt}%
\definecolor{currentstroke}{rgb}{1.000000,0.705882,0.509804}%
\pgfsetstrokecolor{currentstroke}%
\pgfsetstrokeopacity{0.800000}%
\pgfsetdash{}{0pt}%
\pgfpathmoveto{\pgfqpoint{2.841012in}{4.168357in}}%
\pgfpathlineto{\pgfqpoint{3.502867in}{4.072531in}}%
\pgfusepath{stroke}%
\end{pgfscope}%
\begin{pgfscope}%
\pgfpathrectangle{\pgfqpoint{0.481978in}{0.331635in}}{\pgfqpoint{9.300000in}{7.700000in}}%
\pgfusepath{clip}%
\pgfsetrectcap%
\pgfsetroundjoin%
\pgfsetlinewidth{1.505625pt}%
\definecolor{currentstroke}{rgb}{1.000000,0.705882,0.509804}%
\pgfsetstrokecolor{currentstroke}%
\pgfsetstrokeopacity{0.800000}%
\pgfsetdash{}{0pt}%
\pgfpathmoveto{\pgfqpoint{2.668476in}{1.648669in}}%
\pgfpathlineto{\pgfqpoint{3.502867in}{4.072531in}}%
\pgfusepath{stroke}%
\end{pgfscope}%
\begin{pgfscope}%
\pgfpathrectangle{\pgfqpoint{0.481978in}{0.331635in}}{\pgfqpoint{9.300000in}{7.700000in}}%
\pgfusepath{clip}%
\pgfsetrectcap%
\pgfsetroundjoin%
\pgfsetlinewidth{1.505625pt}%
\definecolor{currentstroke}{rgb}{1.000000,0.705882,0.509804}%
\pgfsetstrokecolor{currentstroke}%
\pgfsetstrokeopacity{0.800000}%
\pgfsetdash{}{0pt}%
\pgfpathmoveto{\pgfqpoint{3.108258in}{5.429475in}}%
\pgfpathlineto{\pgfqpoint{3.502867in}{4.072531in}}%
\pgfusepath{stroke}%
\end{pgfscope}%
\begin{pgfscope}%
\pgfpathrectangle{\pgfqpoint{0.481978in}{0.331635in}}{\pgfqpoint{9.300000in}{7.700000in}}%
\pgfusepath{clip}%
\pgfsetrectcap%
\pgfsetroundjoin%
\pgfsetlinewidth{1.505625pt}%
\definecolor{currentstroke}{rgb}{1.000000,0.705882,0.509804}%
\pgfsetstrokecolor{currentstroke}%
\pgfsetstrokeopacity{0.800000}%
\pgfsetdash{}{0pt}%
\pgfpathmoveto{\pgfqpoint{1.529852in}{4.797125in}}%
\pgfpathlineto{\pgfqpoint{3.502867in}{4.072531in}}%
\pgfusepath{stroke}%
\end{pgfscope}%
\begin{pgfscope}%
\pgfpathrectangle{\pgfqpoint{0.481978in}{0.331635in}}{\pgfqpoint{9.300000in}{7.700000in}}%
\pgfusepath{clip}%
\pgfsetrectcap%
\pgfsetroundjoin%
\pgfsetlinewidth{1.505625pt}%
\definecolor{currentstroke}{rgb}{1.000000,0.705882,0.509804}%
\pgfsetstrokecolor{currentstroke}%
\pgfsetstrokeopacity{0.800000}%
\pgfsetdash{}{0pt}%
\pgfpathmoveto{\pgfqpoint{4.765650in}{4.953850in}}%
\pgfpathlineto{\pgfqpoint{3.502867in}{4.072531in}}%
\pgfusepath{stroke}%
\end{pgfscope}%
\begin{pgfscope}%
\pgfpathrectangle{\pgfqpoint{0.481978in}{0.331635in}}{\pgfqpoint{9.300000in}{7.700000in}}%
\pgfusepath{clip}%
\pgfsetrectcap%
\pgfsetroundjoin%
\pgfsetlinewidth{1.505625pt}%
\definecolor{currentstroke}{rgb}{1.000000,0.705882,0.509804}%
\pgfsetstrokecolor{currentstroke}%
\pgfsetstrokeopacity{0.800000}%
\pgfsetdash{}{0pt}%
\pgfpathmoveto{\pgfqpoint{1.272867in}{4.338069in}}%
\pgfpathlineto{\pgfqpoint{3.502867in}{4.072531in}}%
\pgfusepath{stroke}%
\end{pgfscope}%
\begin{pgfscope}%
\pgfpathrectangle{\pgfqpoint{0.481978in}{0.331635in}}{\pgfqpoint{9.300000in}{7.700000in}}%
\pgfusepath{clip}%
\pgfsetrectcap%
\pgfsetroundjoin%
\pgfsetlinewidth{1.505625pt}%
\definecolor{currentstroke}{rgb}{1.000000,0.705882,0.509804}%
\pgfsetstrokecolor{currentstroke}%
\pgfsetstrokeopacity{0.800000}%
\pgfsetdash{}{0pt}%
\pgfpathmoveto{\pgfqpoint{3.381281in}{4.116843in}}%
\pgfpathlineto{\pgfqpoint{3.502867in}{4.072531in}}%
\pgfusepath{stroke}%
\end{pgfscope}%
\begin{pgfscope}%
\pgfpathrectangle{\pgfqpoint{0.481978in}{0.331635in}}{\pgfqpoint{9.300000in}{7.700000in}}%
\pgfusepath{clip}%
\pgfsetrectcap%
\pgfsetroundjoin%
\pgfsetlinewidth{1.505625pt}%
\definecolor{currentstroke}{rgb}{1.000000,0.705882,0.509804}%
\pgfsetstrokecolor{currentstroke}%
\pgfsetstrokeopacity{0.800000}%
\pgfsetdash{}{0pt}%
\pgfpathmoveto{\pgfqpoint{4.049596in}{2.532499in}}%
\pgfpathlineto{\pgfqpoint{3.502867in}{4.072531in}}%
\pgfusepath{stroke}%
\end{pgfscope}%
\begin{pgfscope}%
\pgfpathrectangle{\pgfqpoint{0.481978in}{0.331635in}}{\pgfqpoint{9.300000in}{7.700000in}}%
\pgfusepath{clip}%
\pgfsetrectcap%
\pgfsetroundjoin%
\pgfsetlinewidth{1.505625pt}%
\definecolor{currentstroke}{rgb}{1.000000,0.705882,0.509804}%
\pgfsetstrokecolor{currentstroke}%
\pgfsetstrokeopacity{0.800000}%
\pgfsetdash{}{0pt}%
\pgfpathmoveto{\pgfqpoint{3.917035in}{4.822532in}}%
\pgfpathlineto{\pgfqpoint{3.502867in}{4.072531in}}%
\pgfusepath{stroke}%
\end{pgfscope}%
\begin{pgfscope}%
\pgfpathrectangle{\pgfqpoint{0.481978in}{0.331635in}}{\pgfqpoint{9.300000in}{7.700000in}}%
\pgfusepath{clip}%
\pgfsetrectcap%
\pgfsetroundjoin%
\pgfsetlinewidth{1.505625pt}%
\definecolor{currentstroke}{rgb}{1.000000,0.705882,0.509804}%
\pgfsetstrokecolor{currentstroke}%
\pgfsetstrokeopacity{0.800000}%
\pgfsetdash{}{0pt}%
\pgfpathmoveto{\pgfqpoint{2.834471in}{5.970136in}}%
\pgfpathlineto{\pgfqpoint{3.502867in}{4.072531in}}%
\pgfusepath{stroke}%
\end{pgfscope}%
\begin{pgfscope}%
\pgfpathrectangle{\pgfqpoint{0.481978in}{0.331635in}}{\pgfqpoint{9.300000in}{7.700000in}}%
\pgfusepath{clip}%
\pgfsetrectcap%
\pgfsetroundjoin%
\pgfsetlinewidth{1.505625pt}%
\definecolor{currentstroke}{rgb}{1.000000,0.705882,0.509804}%
\pgfsetstrokecolor{currentstroke}%
\pgfsetstrokeopacity{0.800000}%
\pgfsetdash{}{0pt}%
\pgfpathmoveto{\pgfqpoint{1.856829in}{4.015015in}}%
\pgfpathlineto{\pgfqpoint{3.502867in}{4.072531in}}%
\pgfusepath{stroke}%
\end{pgfscope}%
\begin{pgfscope}%
\pgfpathrectangle{\pgfqpoint{0.481978in}{0.331635in}}{\pgfqpoint{9.300000in}{7.700000in}}%
\pgfusepath{clip}%
\pgfsetrectcap%
\pgfsetroundjoin%
\pgfsetlinewidth{1.505625pt}%
\definecolor{currentstroke}{rgb}{1.000000,0.705882,0.509804}%
\pgfsetstrokecolor{currentstroke}%
\pgfsetstrokeopacity{0.800000}%
\pgfsetdash{}{0pt}%
\pgfpathmoveto{\pgfqpoint{4.069841in}{2.119516in}}%
\pgfpathlineto{\pgfqpoint{3.502867in}{4.072531in}}%
\pgfusepath{stroke}%
\end{pgfscope}%
\begin{pgfscope}%
\pgfpathrectangle{\pgfqpoint{0.481978in}{0.331635in}}{\pgfqpoint{9.300000in}{7.700000in}}%
\pgfusepath{clip}%
\pgfsetrectcap%
\pgfsetroundjoin%
\pgfsetlinewidth{1.505625pt}%
\definecolor{currentstroke}{rgb}{1.000000,0.705882,0.509804}%
\pgfsetstrokecolor{currentstroke}%
\pgfsetstrokeopacity{0.800000}%
\pgfsetdash{}{0pt}%
\pgfpathmoveto{\pgfqpoint{3.968076in}{5.409564in}}%
\pgfpathlineto{\pgfqpoint{3.502867in}{4.072531in}}%
\pgfusepath{stroke}%
\end{pgfscope}%
\begin{pgfscope}%
\pgfpathrectangle{\pgfqpoint{0.481978in}{0.331635in}}{\pgfqpoint{9.300000in}{7.700000in}}%
\pgfusepath{clip}%
\pgfsetrectcap%
\pgfsetroundjoin%
\pgfsetlinewidth{1.505625pt}%
\definecolor{currentstroke}{rgb}{1.000000,0.705882,0.509804}%
\pgfsetstrokecolor{currentstroke}%
\pgfsetstrokeopacity{0.800000}%
\pgfsetdash{}{0pt}%
\pgfpathmoveto{\pgfqpoint{3.839235in}{2.978822in}}%
\pgfpathlineto{\pgfqpoint{3.502867in}{4.072531in}}%
\pgfusepath{stroke}%
\end{pgfscope}%
\begin{pgfscope}%
\pgfpathrectangle{\pgfqpoint{0.481978in}{0.331635in}}{\pgfqpoint{9.300000in}{7.700000in}}%
\pgfusepath{clip}%
\pgfsetrectcap%
\pgfsetroundjoin%
\pgfsetlinewidth{1.505625pt}%
\definecolor{currentstroke}{rgb}{1.000000,0.705882,0.509804}%
\pgfsetstrokecolor{currentstroke}%
\pgfsetstrokeopacity{0.800000}%
\pgfsetdash{}{0pt}%
\pgfpathmoveto{\pgfqpoint{4.658407in}{6.508294in}}%
\pgfpathlineto{\pgfqpoint{3.502867in}{4.072531in}}%
\pgfusepath{stroke}%
\end{pgfscope}%
\begin{pgfscope}%
\pgfpathrectangle{\pgfqpoint{0.481978in}{0.331635in}}{\pgfqpoint{9.300000in}{7.700000in}}%
\pgfusepath{clip}%
\pgfsetrectcap%
\pgfsetroundjoin%
\pgfsetlinewidth{1.505625pt}%
\definecolor{currentstroke}{rgb}{1.000000,0.705882,0.509804}%
\pgfsetstrokecolor{currentstroke}%
\pgfsetstrokeopacity{0.800000}%
\pgfsetdash{}{0pt}%
\pgfpathmoveto{\pgfqpoint{4.784824in}{5.920838in}}%
\pgfpathlineto{\pgfqpoint{3.502867in}{4.072531in}}%
\pgfusepath{stroke}%
\end{pgfscope}%
\begin{pgfscope}%
\pgfpathrectangle{\pgfqpoint{0.481978in}{0.331635in}}{\pgfqpoint{9.300000in}{7.700000in}}%
\pgfusepath{clip}%
\pgfsetrectcap%
\pgfsetroundjoin%
\pgfsetlinewidth{1.505625pt}%
\definecolor{currentstroke}{rgb}{1.000000,0.705882,0.509804}%
\pgfsetstrokecolor{currentstroke}%
\pgfsetstrokeopacity{0.800000}%
\pgfsetdash{}{0pt}%
\pgfpathmoveto{\pgfqpoint{5.305166in}{4.648380in}}%
\pgfpathlineto{\pgfqpoint{3.502867in}{4.072531in}}%
\pgfusepath{stroke}%
\end{pgfscope}%
\begin{pgfscope}%
\pgfpathrectangle{\pgfqpoint{0.481978in}{0.331635in}}{\pgfqpoint{9.300000in}{7.700000in}}%
\pgfusepath{clip}%
\pgfsetrectcap%
\pgfsetroundjoin%
\pgfsetlinewidth{1.505625pt}%
\definecolor{currentstroke}{rgb}{1.000000,0.705882,0.509804}%
\pgfsetstrokecolor{currentstroke}%
\pgfsetstrokeopacity{0.800000}%
\pgfsetdash{}{0pt}%
\pgfpathmoveto{\pgfqpoint{3.719294in}{4.762095in}}%
\pgfpathlineto{\pgfqpoint{3.502867in}{4.072531in}}%
\pgfusepath{stroke}%
\end{pgfscope}%
\begin{pgfscope}%
\pgfpathrectangle{\pgfqpoint{0.481978in}{0.331635in}}{\pgfqpoint{9.300000in}{7.700000in}}%
\pgfusepath{clip}%
\pgfsetrectcap%
\pgfsetroundjoin%
\pgfsetlinewidth{1.505625pt}%
\definecolor{currentstroke}{rgb}{1.000000,0.705882,0.509804}%
\pgfsetstrokecolor{currentstroke}%
\pgfsetstrokeopacity{0.800000}%
\pgfsetdash{}{0pt}%
\pgfpathmoveto{\pgfqpoint{6.457903in}{1.838210in}}%
\pgfpathlineto{\pgfqpoint{3.502867in}{4.072531in}}%
\pgfusepath{stroke}%
\end{pgfscope}%
\begin{pgfscope}%
\pgfpathrectangle{\pgfqpoint{0.481978in}{0.331635in}}{\pgfqpoint{9.300000in}{7.700000in}}%
\pgfusepath{clip}%
\pgfsetrectcap%
\pgfsetroundjoin%
\pgfsetlinewidth{1.505625pt}%
\definecolor{currentstroke}{rgb}{1.000000,0.705882,0.509804}%
\pgfsetstrokecolor{currentstroke}%
\pgfsetstrokeopacity{0.800000}%
\pgfsetdash{}{0pt}%
\pgfpathmoveto{\pgfqpoint{3.091678in}{5.641728in}}%
\pgfpathlineto{\pgfqpoint{3.502867in}{4.072531in}}%
\pgfusepath{stroke}%
\end{pgfscope}%
\begin{pgfscope}%
\pgfpathrectangle{\pgfqpoint{0.481978in}{0.331635in}}{\pgfqpoint{9.300000in}{7.700000in}}%
\pgfusepath{clip}%
\pgfsetrectcap%
\pgfsetroundjoin%
\pgfsetlinewidth{1.505625pt}%
\definecolor{currentstroke}{rgb}{1.000000,0.705882,0.509804}%
\pgfsetstrokecolor{currentstroke}%
\pgfsetstrokeopacity{0.800000}%
\pgfsetdash{}{0pt}%
\pgfpathmoveto{\pgfqpoint{0.904705in}{4.531254in}}%
\pgfpathlineto{\pgfqpoint{3.502867in}{4.072531in}}%
\pgfusepath{stroke}%
\end{pgfscope}%
\begin{pgfscope}%
\pgfpathrectangle{\pgfqpoint{0.481978in}{0.331635in}}{\pgfqpoint{9.300000in}{7.700000in}}%
\pgfusepath{clip}%
\pgfsetrectcap%
\pgfsetroundjoin%
\pgfsetlinewidth{1.505625pt}%
\definecolor{currentstroke}{rgb}{1.000000,0.705882,0.509804}%
\pgfsetstrokecolor{currentstroke}%
\pgfsetstrokeopacity{0.800000}%
\pgfsetdash{}{0pt}%
\pgfpathmoveto{\pgfqpoint{5.515150in}{3.314117in}}%
\pgfpathlineto{\pgfqpoint{3.502867in}{4.072531in}}%
\pgfusepath{stroke}%
\end{pgfscope}%
\begin{pgfscope}%
\pgfpathrectangle{\pgfqpoint{0.481978in}{0.331635in}}{\pgfqpoint{9.300000in}{7.700000in}}%
\pgfusepath{clip}%
\pgfsetrectcap%
\pgfsetroundjoin%
\pgfsetlinewidth{1.505625pt}%
\definecolor{currentstroke}{rgb}{1.000000,0.705882,0.509804}%
\pgfsetstrokecolor{currentstroke}%
\pgfsetstrokeopacity{0.800000}%
\pgfsetdash{}{0pt}%
\pgfpathmoveto{\pgfqpoint{3.655066in}{5.234687in}}%
\pgfpathlineto{\pgfqpoint{3.502867in}{4.072531in}}%
\pgfusepath{stroke}%
\end{pgfscope}%
\begin{pgfscope}%
\pgfpathrectangle{\pgfqpoint{0.481978in}{0.331635in}}{\pgfqpoint{9.300000in}{7.700000in}}%
\pgfusepath{clip}%
\pgfsetrectcap%
\pgfsetroundjoin%
\pgfsetlinewidth{1.505625pt}%
\definecolor{currentstroke}{rgb}{1.000000,0.705882,0.509804}%
\pgfsetstrokecolor{currentstroke}%
\pgfsetstrokeopacity{0.800000}%
\pgfsetdash{}{0pt}%
\pgfpathmoveto{\pgfqpoint{4.606001in}{2.719624in}}%
\pgfpathlineto{\pgfqpoint{3.502867in}{4.072531in}}%
\pgfusepath{stroke}%
\end{pgfscope}%
\begin{pgfscope}%
\pgfpathrectangle{\pgfqpoint{0.481978in}{0.331635in}}{\pgfqpoint{9.300000in}{7.700000in}}%
\pgfusepath{clip}%
\pgfsetrectcap%
\pgfsetroundjoin%
\pgfsetlinewidth{1.505625pt}%
\definecolor{currentstroke}{rgb}{1.000000,0.705882,0.509804}%
\pgfsetstrokecolor{currentstroke}%
\pgfsetstrokeopacity{0.800000}%
\pgfsetdash{}{0pt}%
\pgfpathmoveto{\pgfqpoint{2.788909in}{4.554933in}}%
\pgfpathlineto{\pgfqpoint{3.502867in}{4.072531in}}%
\pgfusepath{stroke}%
\end{pgfscope}%
\begin{pgfscope}%
\pgfpathrectangle{\pgfqpoint{0.481978in}{0.331635in}}{\pgfqpoint{9.300000in}{7.700000in}}%
\pgfusepath{clip}%
\pgfsetrectcap%
\pgfsetroundjoin%
\pgfsetlinewidth{1.505625pt}%
\definecolor{currentstroke}{rgb}{1.000000,0.705882,0.509804}%
\pgfsetstrokecolor{currentstroke}%
\pgfsetstrokeopacity{0.800000}%
\pgfsetdash{}{0pt}%
\pgfpathmoveto{\pgfqpoint{2.529879in}{3.185116in}}%
\pgfpathlineto{\pgfqpoint{3.502867in}{4.072531in}}%
\pgfusepath{stroke}%
\end{pgfscope}%
\begin{pgfscope}%
\pgfpathrectangle{\pgfqpoint{0.481978in}{0.331635in}}{\pgfqpoint{9.300000in}{7.700000in}}%
\pgfusepath{clip}%
\pgfsetrectcap%
\pgfsetroundjoin%
\pgfsetlinewidth{1.505625pt}%
\definecolor{currentstroke}{rgb}{1.000000,0.705882,0.509804}%
\pgfsetstrokecolor{currentstroke}%
\pgfsetstrokeopacity{0.800000}%
\pgfsetdash{}{0pt}%
\pgfpathmoveto{\pgfqpoint{2.960587in}{2.541098in}}%
\pgfpathlineto{\pgfqpoint{3.502867in}{4.072531in}}%
\pgfusepath{stroke}%
\end{pgfscope}%
\begin{pgfscope}%
\pgfpathrectangle{\pgfqpoint{0.481978in}{0.331635in}}{\pgfqpoint{9.300000in}{7.700000in}}%
\pgfusepath{clip}%
\pgfsetrectcap%
\pgfsetroundjoin%
\pgfsetlinewidth{1.505625pt}%
\definecolor{currentstroke}{rgb}{1.000000,0.705882,0.509804}%
\pgfsetstrokecolor{currentstroke}%
\pgfsetstrokeopacity{0.800000}%
\pgfsetdash{}{0pt}%
\pgfpathmoveto{\pgfqpoint{8.052568in}{3.873744in}}%
\pgfpathlineto{\pgfqpoint{3.502867in}{4.072531in}}%
\pgfusepath{stroke}%
\end{pgfscope}%
\begin{pgfscope}%
\pgfpathrectangle{\pgfqpoint{0.481978in}{0.331635in}}{\pgfqpoint{9.300000in}{7.700000in}}%
\pgfusepath{clip}%
\pgfsetrectcap%
\pgfsetroundjoin%
\pgfsetlinewidth{1.505625pt}%
\definecolor{currentstroke}{rgb}{1.000000,0.705882,0.509804}%
\pgfsetstrokecolor{currentstroke}%
\pgfsetstrokeopacity{0.800000}%
\pgfsetdash{}{0pt}%
\pgfpathmoveto{\pgfqpoint{1.457328in}{2.537487in}}%
\pgfpathlineto{\pgfqpoint{3.502867in}{4.072531in}}%
\pgfusepath{stroke}%
\end{pgfscope}%
\begin{pgfscope}%
\pgfpathrectangle{\pgfqpoint{0.481978in}{0.331635in}}{\pgfqpoint{9.300000in}{7.700000in}}%
\pgfusepath{clip}%
\pgfsetrectcap%
\pgfsetroundjoin%
\pgfsetlinewidth{1.505625pt}%
\definecolor{currentstroke}{rgb}{1.000000,0.705882,0.509804}%
\pgfsetstrokecolor{currentstroke}%
\pgfsetstrokeopacity{0.800000}%
\pgfsetdash{}{0pt}%
\pgfpathmoveto{\pgfqpoint{1.959498in}{3.653329in}}%
\pgfpathlineto{\pgfqpoint{3.502867in}{4.072531in}}%
\pgfusepath{stroke}%
\end{pgfscope}%
\begin{pgfscope}%
\pgfpathrectangle{\pgfqpoint{0.481978in}{0.331635in}}{\pgfqpoint{9.300000in}{7.700000in}}%
\pgfusepath{clip}%
\pgfsetrectcap%
\pgfsetroundjoin%
\pgfsetlinewidth{1.505625pt}%
\definecolor{currentstroke}{rgb}{1.000000,0.705882,0.509804}%
\pgfsetstrokecolor{currentstroke}%
\pgfsetstrokeopacity{0.800000}%
\pgfsetdash{}{0pt}%
\pgfpathmoveto{\pgfqpoint{5.141137in}{2.895623in}}%
\pgfpathlineto{\pgfqpoint{3.502867in}{4.072531in}}%
\pgfusepath{stroke}%
\end{pgfscope}%
\begin{pgfscope}%
\pgfpathrectangle{\pgfqpoint{0.481978in}{0.331635in}}{\pgfqpoint{9.300000in}{7.700000in}}%
\pgfusepath{clip}%
\pgfsetrectcap%
\pgfsetroundjoin%
\pgfsetlinewidth{1.505625pt}%
\definecolor{currentstroke}{rgb}{1.000000,0.705882,0.509804}%
\pgfsetstrokecolor{currentstroke}%
\pgfsetstrokeopacity{0.800000}%
\pgfsetdash{}{0pt}%
\pgfpathmoveto{\pgfqpoint{5.415961in}{3.947445in}}%
\pgfpathlineto{\pgfqpoint{3.502867in}{4.072531in}}%
\pgfusepath{stroke}%
\end{pgfscope}%
\begin{pgfscope}%
\pgfpathrectangle{\pgfqpoint{0.481978in}{0.331635in}}{\pgfqpoint{9.300000in}{7.700000in}}%
\pgfusepath{clip}%
\pgfsetrectcap%
\pgfsetroundjoin%
\pgfsetlinewidth{1.505625pt}%
\definecolor{currentstroke}{rgb}{1.000000,0.705882,0.509804}%
\pgfsetstrokecolor{currentstroke}%
\pgfsetstrokeopacity{0.800000}%
\pgfsetdash{}{0pt}%
\pgfpathmoveto{\pgfqpoint{3.418504in}{2.608047in}}%
\pgfpathlineto{\pgfqpoint{3.502867in}{4.072531in}}%
\pgfusepath{stroke}%
\end{pgfscope}%
\begin{pgfscope}%
\pgfpathrectangle{\pgfqpoint{0.481978in}{0.331635in}}{\pgfqpoint{9.300000in}{7.700000in}}%
\pgfusepath{clip}%
\pgfsetrectcap%
\pgfsetroundjoin%
\pgfsetlinewidth{1.505625pt}%
\definecolor{currentstroke}{rgb}{1.000000,0.705882,0.509804}%
\pgfsetstrokecolor{currentstroke}%
\pgfsetstrokeopacity{0.800000}%
\pgfsetdash{}{0pt}%
\pgfpathmoveto{\pgfqpoint{1.418868in}{3.723417in}}%
\pgfpathlineto{\pgfqpoint{3.502867in}{4.072531in}}%
\pgfusepath{stroke}%
\end{pgfscope}%
\begin{pgfscope}%
\pgfpathrectangle{\pgfqpoint{0.481978in}{0.331635in}}{\pgfqpoint{9.300000in}{7.700000in}}%
\pgfusepath{clip}%
\pgfsetrectcap%
\pgfsetroundjoin%
\pgfsetlinewidth{1.505625pt}%
\definecolor{currentstroke}{rgb}{1.000000,0.705882,0.509804}%
\pgfsetstrokecolor{currentstroke}%
\pgfsetstrokeopacity{0.800000}%
\pgfsetdash{}{0pt}%
\pgfpathmoveto{\pgfqpoint{4.390885in}{4.246675in}}%
\pgfpathlineto{\pgfqpoint{3.502867in}{4.072531in}}%
\pgfusepath{stroke}%
\end{pgfscope}%
\begin{pgfscope}%
\pgfpathrectangle{\pgfqpoint{0.481978in}{0.331635in}}{\pgfqpoint{9.300000in}{7.700000in}}%
\pgfusepath{clip}%
\pgfsetrectcap%
\pgfsetroundjoin%
\pgfsetlinewidth{1.505625pt}%
\definecolor{currentstroke}{rgb}{1.000000,0.705882,0.509804}%
\pgfsetstrokecolor{currentstroke}%
\pgfsetstrokeopacity{0.800000}%
\pgfsetdash{}{0pt}%
\pgfpathmoveto{\pgfqpoint{2.103399in}{2.786137in}}%
\pgfpathlineto{\pgfqpoint{3.502867in}{4.072531in}}%
\pgfusepath{stroke}%
\end{pgfscope}%
\begin{pgfscope}%
\pgfpathrectangle{\pgfqpoint{0.481978in}{0.331635in}}{\pgfqpoint{9.300000in}{7.700000in}}%
\pgfusepath{clip}%
\pgfsetrectcap%
\pgfsetroundjoin%
\pgfsetlinewidth{1.505625pt}%
\definecolor{currentstroke}{rgb}{1.000000,0.705882,0.509804}%
\pgfsetstrokecolor{currentstroke}%
\pgfsetstrokeopacity{0.800000}%
\pgfsetdash{}{0pt}%
\pgfpathmoveto{\pgfqpoint{3.156352in}{3.695503in}}%
\pgfpathlineto{\pgfqpoint{3.502867in}{4.072531in}}%
\pgfusepath{stroke}%
\end{pgfscope}%
\begin{pgfscope}%
\pgfpathrectangle{\pgfqpoint{0.481978in}{0.331635in}}{\pgfqpoint{9.300000in}{7.700000in}}%
\pgfusepath{clip}%
\pgfsetrectcap%
\pgfsetroundjoin%
\pgfsetlinewidth{1.505625pt}%
\definecolor{currentstroke}{rgb}{1.000000,0.705882,0.509804}%
\pgfsetstrokecolor{currentstroke}%
\pgfsetstrokeopacity{0.800000}%
\pgfsetdash{}{0pt}%
\pgfpathmoveto{\pgfqpoint{4.782434in}{5.411059in}}%
\pgfpathlineto{\pgfqpoint{3.502867in}{4.072531in}}%
\pgfusepath{stroke}%
\end{pgfscope}%
\begin{pgfscope}%
\pgfpathrectangle{\pgfqpoint{0.481978in}{0.331635in}}{\pgfqpoint{9.300000in}{7.700000in}}%
\pgfusepath{clip}%
\pgfsetrectcap%
\pgfsetroundjoin%
\pgfsetlinewidth{1.505625pt}%
\definecolor{currentstroke}{rgb}{1.000000,0.705882,0.509804}%
\pgfsetstrokecolor{currentstroke}%
\pgfsetstrokeopacity{0.800000}%
\pgfsetdash{}{0pt}%
\pgfpathmoveto{\pgfqpoint{4.313894in}{2.603823in}}%
\pgfpathlineto{\pgfqpoint{3.502867in}{4.072531in}}%
\pgfusepath{stroke}%
\end{pgfscope}%
\begin{pgfscope}%
\pgfpathrectangle{\pgfqpoint{0.481978in}{0.331635in}}{\pgfqpoint{9.300000in}{7.700000in}}%
\pgfusepath{clip}%
\pgfsetrectcap%
\pgfsetroundjoin%
\pgfsetlinewidth{1.505625pt}%
\definecolor{currentstroke}{rgb}{1.000000,0.705882,0.509804}%
\pgfsetstrokecolor{currentstroke}%
\pgfsetstrokeopacity{0.800000}%
\pgfsetdash{}{0pt}%
\pgfpathmoveto{\pgfqpoint{3.534223in}{5.922879in}}%
\pgfpathlineto{\pgfqpoint{3.502867in}{4.072531in}}%
\pgfusepath{stroke}%
\end{pgfscope}%
\begin{pgfscope}%
\pgfpathrectangle{\pgfqpoint{0.481978in}{0.331635in}}{\pgfqpoint{9.300000in}{7.700000in}}%
\pgfusepath{clip}%
\pgfsetrectcap%
\pgfsetroundjoin%
\pgfsetlinewidth{1.505625pt}%
\definecolor{currentstroke}{rgb}{1.000000,0.705882,0.509804}%
\pgfsetstrokecolor{currentstroke}%
\pgfsetstrokeopacity{0.800000}%
\pgfsetdash{}{0pt}%
\pgfpathmoveto{\pgfqpoint{2.718188in}{5.699839in}}%
\pgfpathlineto{\pgfqpoint{3.502867in}{4.072531in}}%
\pgfusepath{stroke}%
\end{pgfscope}%
\begin{pgfscope}%
\pgfpathrectangle{\pgfqpoint{0.481978in}{0.331635in}}{\pgfqpoint{9.300000in}{7.700000in}}%
\pgfusepath{clip}%
\pgfsetrectcap%
\pgfsetroundjoin%
\pgfsetlinewidth{1.505625pt}%
\definecolor{currentstroke}{rgb}{1.000000,0.705882,0.509804}%
\pgfsetstrokecolor{currentstroke}%
\pgfsetstrokeopacity{0.800000}%
\pgfsetdash{}{0pt}%
\pgfpathmoveto{\pgfqpoint{4.798223in}{3.954560in}}%
\pgfpathlineto{\pgfqpoint{3.502867in}{4.072531in}}%
\pgfusepath{stroke}%
\end{pgfscope}%
\begin{pgfscope}%
\pgfpathrectangle{\pgfqpoint{0.481978in}{0.331635in}}{\pgfqpoint{9.300000in}{7.700000in}}%
\pgfusepath{clip}%
\pgfsetrectcap%
\pgfsetroundjoin%
\pgfsetlinewidth{1.505625pt}%
\definecolor{currentstroke}{rgb}{1.000000,0.705882,0.509804}%
\pgfsetstrokecolor{currentstroke}%
\pgfsetstrokeopacity{0.800000}%
\pgfsetdash{}{0pt}%
\pgfpathmoveto{\pgfqpoint{2.688115in}{4.317587in}}%
\pgfpathlineto{\pgfqpoint{3.502867in}{4.072531in}}%
\pgfusepath{stroke}%
\end{pgfscope}%
\begin{pgfscope}%
\pgfpathrectangle{\pgfqpoint{0.481978in}{0.331635in}}{\pgfqpoint{9.300000in}{7.700000in}}%
\pgfusepath{clip}%
\pgfsetrectcap%
\pgfsetroundjoin%
\pgfsetlinewidth{1.505625pt}%
\definecolor{currentstroke}{rgb}{1.000000,0.705882,0.509804}%
\pgfsetstrokecolor{currentstroke}%
\pgfsetstrokeopacity{0.800000}%
\pgfsetdash{}{0pt}%
\pgfpathmoveto{\pgfqpoint{3.356009in}{3.332822in}}%
\pgfpathlineto{\pgfqpoint{3.502867in}{4.072531in}}%
\pgfusepath{stroke}%
\end{pgfscope}%
\begin{pgfscope}%
\pgfpathrectangle{\pgfqpoint{0.481978in}{0.331635in}}{\pgfqpoint{9.300000in}{7.700000in}}%
\pgfusepath{clip}%
\pgfsetrectcap%
\pgfsetroundjoin%
\pgfsetlinewidth{1.505625pt}%
\definecolor{currentstroke}{rgb}{1.000000,0.705882,0.509804}%
\pgfsetstrokecolor{currentstroke}%
\pgfsetstrokeopacity{0.800000}%
\pgfsetdash{}{0pt}%
\pgfpathmoveto{\pgfqpoint{2.750103in}{4.755156in}}%
\pgfpathlineto{\pgfqpoint{3.502867in}{4.072531in}}%
\pgfusepath{stroke}%
\end{pgfscope}%
\begin{pgfscope}%
\pgfpathrectangle{\pgfqpoint{0.481978in}{0.331635in}}{\pgfqpoint{9.300000in}{7.700000in}}%
\pgfusepath{clip}%
\pgfsetrectcap%
\pgfsetroundjoin%
\pgfsetlinewidth{1.505625pt}%
\definecolor{currentstroke}{rgb}{1.000000,0.705882,0.509804}%
\pgfsetstrokecolor{currentstroke}%
\pgfsetstrokeopacity{0.800000}%
\pgfsetdash{}{0pt}%
\pgfpathmoveto{\pgfqpoint{3.502659in}{2.555499in}}%
\pgfpathlineto{\pgfqpoint{3.502867in}{4.072531in}}%
\pgfusepath{stroke}%
\end{pgfscope}%
\begin{pgfscope}%
\pgfpathrectangle{\pgfqpoint{0.481978in}{0.331635in}}{\pgfqpoint{9.300000in}{7.700000in}}%
\pgfusepath{clip}%
\pgfsetrectcap%
\pgfsetroundjoin%
\pgfsetlinewidth{1.505625pt}%
\definecolor{currentstroke}{rgb}{1.000000,0.705882,0.509804}%
\pgfsetstrokecolor{currentstroke}%
\pgfsetstrokeopacity{0.800000}%
\pgfsetdash{}{0pt}%
\pgfpathmoveto{\pgfqpoint{3.883915in}{4.747512in}}%
\pgfpathlineto{\pgfqpoint{3.502867in}{4.072531in}}%
\pgfusepath{stroke}%
\end{pgfscope}%
\begin{pgfscope}%
\pgfpathrectangle{\pgfqpoint{0.481978in}{0.331635in}}{\pgfqpoint{9.300000in}{7.700000in}}%
\pgfusepath{clip}%
\pgfsetrectcap%
\pgfsetroundjoin%
\pgfsetlinewidth{1.505625pt}%
\definecolor{currentstroke}{rgb}{1.000000,0.705882,0.509804}%
\pgfsetstrokecolor{currentstroke}%
\pgfsetstrokeopacity{0.800000}%
\pgfsetdash{}{0pt}%
\pgfpathmoveto{\pgfqpoint{4.261705in}{2.344287in}}%
\pgfpathlineto{\pgfqpoint{3.502867in}{4.072531in}}%
\pgfusepath{stroke}%
\end{pgfscope}%
\begin{pgfscope}%
\pgfpathrectangle{\pgfqpoint{0.481978in}{0.331635in}}{\pgfqpoint{9.300000in}{7.700000in}}%
\pgfusepath{clip}%
\pgfsetrectcap%
\pgfsetroundjoin%
\pgfsetlinewidth{1.505625pt}%
\definecolor{currentstroke}{rgb}{1.000000,0.705882,0.509804}%
\pgfsetstrokecolor{currentstroke}%
\pgfsetstrokeopacity{0.800000}%
\pgfsetdash{}{0pt}%
\pgfpathmoveto{\pgfqpoint{1.767331in}{4.669274in}}%
\pgfpathlineto{\pgfqpoint{3.502867in}{4.072531in}}%
\pgfusepath{stroke}%
\end{pgfscope}%
\begin{pgfscope}%
\pgfpathrectangle{\pgfqpoint{0.481978in}{0.331635in}}{\pgfqpoint{9.300000in}{7.700000in}}%
\pgfusepath{clip}%
\pgfsetrectcap%
\pgfsetroundjoin%
\pgfsetlinewidth{1.505625pt}%
\definecolor{currentstroke}{rgb}{1.000000,0.705882,0.509804}%
\pgfsetstrokecolor{currentstroke}%
\pgfsetstrokeopacity{0.800000}%
\pgfsetdash{}{0pt}%
\pgfpathmoveto{\pgfqpoint{1.457321in}{4.685354in}}%
\pgfpathlineto{\pgfqpoint{3.502867in}{4.072531in}}%
\pgfusepath{stroke}%
\end{pgfscope}%
\begin{pgfscope}%
\pgfpathrectangle{\pgfqpoint{0.481978in}{0.331635in}}{\pgfqpoint{9.300000in}{7.700000in}}%
\pgfusepath{clip}%
\pgfsetrectcap%
\pgfsetroundjoin%
\pgfsetlinewidth{1.505625pt}%
\definecolor{currentstroke}{rgb}{1.000000,0.705882,0.509804}%
\pgfsetstrokecolor{currentstroke}%
\pgfsetstrokeopacity{0.800000}%
\pgfsetdash{}{0pt}%
\pgfpathmoveto{\pgfqpoint{3.872133in}{3.288225in}}%
\pgfpathlineto{\pgfqpoint{3.502867in}{4.072531in}}%
\pgfusepath{stroke}%
\end{pgfscope}%
\begin{pgfscope}%
\pgfpathrectangle{\pgfqpoint{0.481978in}{0.331635in}}{\pgfqpoint{9.300000in}{7.700000in}}%
\pgfusepath{clip}%
\pgfsetrectcap%
\pgfsetroundjoin%
\pgfsetlinewidth{1.505625pt}%
\definecolor{currentstroke}{rgb}{1.000000,0.705882,0.509804}%
\pgfsetstrokecolor{currentstroke}%
\pgfsetstrokeopacity{0.800000}%
\pgfsetdash{}{0pt}%
\pgfpathmoveto{\pgfqpoint{3.604393in}{3.101026in}}%
\pgfpathlineto{\pgfqpoint{3.502867in}{4.072531in}}%
\pgfusepath{stroke}%
\end{pgfscope}%
\begin{pgfscope}%
\pgfpathrectangle{\pgfqpoint{0.481978in}{0.331635in}}{\pgfqpoint{9.300000in}{7.700000in}}%
\pgfusepath{clip}%
\pgfsetrectcap%
\pgfsetroundjoin%
\pgfsetlinewidth{1.505625pt}%
\definecolor{currentstroke}{rgb}{1.000000,0.705882,0.509804}%
\pgfsetstrokecolor{currentstroke}%
\pgfsetstrokeopacity{0.800000}%
\pgfsetdash{}{0pt}%
\pgfpathmoveto{\pgfqpoint{3.157682in}{4.441797in}}%
\pgfpathlineto{\pgfqpoint{3.502867in}{4.072531in}}%
\pgfusepath{stroke}%
\end{pgfscope}%
\begin{pgfscope}%
\pgfpathrectangle{\pgfqpoint{0.481978in}{0.331635in}}{\pgfqpoint{9.300000in}{7.700000in}}%
\pgfusepath{clip}%
\pgfsetrectcap%
\pgfsetroundjoin%
\pgfsetlinewidth{1.505625pt}%
\definecolor{currentstroke}{rgb}{1.000000,0.705882,0.509804}%
\pgfsetstrokecolor{currentstroke}%
\pgfsetstrokeopacity{0.800000}%
\pgfsetdash{}{0pt}%
\pgfpathmoveto{\pgfqpoint{3.462770in}{5.112041in}}%
\pgfpathlineto{\pgfqpoint{3.502867in}{4.072531in}}%
\pgfusepath{stroke}%
\end{pgfscope}%
\begin{pgfscope}%
\pgfpathrectangle{\pgfqpoint{0.481978in}{0.331635in}}{\pgfqpoint{9.300000in}{7.700000in}}%
\pgfusepath{clip}%
\pgfsetrectcap%
\pgfsetroundjoin%
\pgfsetlinewidth{1.505625pt}%
\definecolor{currentstroke}{rgb}{1.000000,0.705882,0.509804}%
\pgfsetstrokecolor{currentstroke}%
\pgfsetstrokeopacity{0.800000}%
\pgfsetdash{}{0pt}%
\pgfpathmoveto{\pgfqpoint{4.591082in}{3.564558in}}%
\pgfpathlineto{\pgfqpoint{3.502867in}{4.072531in}}%
\pgfusepath{stroke}%
\end{pgfscope}%
\begin{pgfscope}%
\pgfpathrectangle{\pgfqpoint{0.481978in}{0.331635in}}{\pgfqpoint{9.300000in}{7.700000in}}%
\pgfusepath{clip}%
\pgfsetrectcap%
\pgfsetroundjoin%
\pgfsetlinewidth{1.505625pt}%
\definecolor{currentstroke}{rgb}{1.000000,0.705882,0.509804}%
\pgfsetstrokecolor{currentstroke}%
\pgfsetstrokeopacity{0.800000}%
\pgfsetdash{}{0pt}%
\pgfpathmoveto{\pgfqpoint{5.774624in}{4.599244in}}%
\pgfpathlineto{\pgfqpoint{3.502867in}{4.072531in}}%
\pgfusepath{stroke}%
\end{pgfscope}%
\begin{pgfscope}%
\pgfpathrectangle{\pgfqpoint{0.481978in}{0.331635in}}{\pgfqpoint{9.300000in}{7.700000in}}%
\pgfusepath{clip}%
\pgfsetrectcap%
\pgfsetroundjoin%
\pgfsetlinewidth{1.505625pt}%
\definecolor{currentstroke}{rgb}{1.000000,0.705882,0.509804}%
\pgfsetstrokecolor{currentstroke}%
\pgfsetstrokeopacity{0.800000}%
\pgfsetdash{}{0pt}%
\pgfpathmoveto{\pgfqpoint{3.755577in}{2.708345in}}%
\pgfpathlineto{\pgfqpoint{3.502867in}{4.072531in}}%
\pgfusepath{stroke}%
\end{pgfscope}%
\begin{pgfscope}%
\pgfpathrectangle{\pgfqpoint{0.481978in}{0.331635in}}{\pgfqpoint{9.300000in}{7.700000in}}%
\pgfusepath{clip}%
\pgfsetrectcap%
\pgfsetroundjoin%
\pgfsetlinewidth{1.505625pt}%
\definecolor{currentstroke}{rgb}{1.000000,0.705882,0.509804}%
\pgfsetstrokecolor{currentstroke}%
\pgfsetstrokeopacity{0.800000}%
\pgfsetdash{}{0pt}%
\pgfpathmoveto{\pgfqpoint{2.355130in}{2.601916in}}%
\pgfpathlineto{\pgfqpoint{3.502867in}{4.072531in}}%
\pgfusepath{stroke}%
\end{pgfscope}%
\begin{pgfscope}%
\pgfpathrectangle{\pgfqpoint{0.481978in}{0.331635in}}{\pgfqpoint{9.300000in}{7.700000in}}%
\pgfusepath{clip}%
\pgfsetrectcap%
\pgfsetroundjoin%
\pgfsetlinewidth{1.505625pt}%
\definecolor{currentstroke}{rgb}{1.000000,0.705882,0.509804}%
\pgfsetstrokecolor{currentstroke}%
\pgfsetstrokeopacity{0.800000}%
\pgfsetdash{}{0pt}%
\pgfpathmoveto{\pgfqpoint{3.074624in}{3.287134in}}%
\pgfpathlineto{\pgfqpoint{3.502867in}{4.072531in}}%
\pgfusepath{stroke}%
\end{pgfscope}%
\begin{pgfscope}%
\pgfpathrectangle{\pgfqpoint{0.481978in}{0.331635in}}{\pgfqpoint{9.300000in}{7.700000in}}%
\pgfusepath{clip}%
\pgfsetrectcap%
\pgfsetroundjoin%
\pgfsetlinewidth{1.505625pt}%
\definecolor{currentstroke}{rgb}{1.000000,0.705882,0.509804}%
\pgfsetstrokecolor{currentstroke}%
\pgfsetstrokeopacity{0.800000}%
\pgfsetdash{}{0pt}%
\pgfpathmoveto{\pgfqpoint{2.232057in}{4.476601in}}%
\pgfpathlineto{\pgfqpoint{3.502867in}{4.072531in}}%
\pgfusepath{stroke}%
\end{pgfscope}%
\begin{pgfscope}%
\pgfpathrectangle{\pgfqpoint{0.481978in}{0.331635in}}{\pgfqpoint{9.300000in}{7.700000in}}%
\pgfusepath{clip}%
\pgfsetrectcap%
\pgfsetroundjoin%
\pgfsetlinewidth{1.505625pt}%
\definecolor{currentstroke}{rgb}{1.000000,0.705882,0.509804}%
\pgfsetstrokecolor{currentstroke}%
\pgfsetstrokeopacity{0.800000}%
\pgfsetdash{}{0pt}%
\pgfpathmoveto{\pgfqpoint{3.954754in}{3.742286in}}%
\pgfpathlineto{\pgfqpoint{3.502867in}{4.072531in}}%
\pgfusepath{stroke}%
\end{pgfscope}%
\begin{pgfscope}%
\pgfpathrectangle{\pgfqpoint{0.481978in}{0.331635in}}{\pgfqpoint{9.300000in}{7.700000in}}%
\pgfusepath{clip}%
\pgfsetrectcap%
\pgfsetroundjoin%
\pgfsetlinewidth{1.505625pt}%
\definecolor{currentstroke}{rgb}{1.000000,0.705882,0.509804}%
\pgfsetstrokecolor{currentstroke}%
\pgfsetstrokeopacity{0.800000}%
\pgfsetdash{}{0pt}%
\pgfpathmoveto{\pgfqpoint{1.921313in}{5.102923in}}%
\pgfpathlineto{\pgfqpoint{3.502867in}{4.072531in}}%
\pgfusepath{stroke}%
\end{pgfscope}%
\begin{pgfscope}%
\pgfpathrectangle{\pgfqpoint{0.481978in}{0.331635in}}{\pgfqpoint{9.300000in}{7.700000in}}%
\pgfusepath{clip}%
\pgfsetrectcap%
\pgfsetroundjoin%
\pgfsetlinewidth{1.505625pt}%
\definecolor{currentstroke}{rgb}{1.000000,0.705882,0.509804}%
\pgfsetstrokecolor{currentstroke}%
\pgfsetstrokeopacity{0.800000}%
\pgfsetdash{}{0pt}%
\pgfpathmoveto{\pgfqpoint{3.918352in}{4.443858in}}%
\pgfpathlineto{\pgfqpoint{3.502867in}{4.072531in}}%
\pgfusepath{stroke}%
\end{pgfscope}%
\begin{pgfscope}%
\pgfpathrectangle{\pgfqpoint{0.481978in}{0.331635in}}{\pgfqpoint{9.300000in}{7.700000in}}%
\pgfusepath{clip}%
\pgfsetrectcap%
\pgfsetroundjoin%
\pgfsetlinewidth{1.505625pt}%
\definecolor{currentstroke}{rgb}{1.000000,0.705882,0.509804}%
\pgfsetstrokecolor{currentstroke}%
\pgfsetstrokeopacity{0.800000}%
\pgfsetdash{}{0pt}%
\pgfpathmoveto{\pgfqpoint{1.874082in}{4.443883in}}%
\pgfpathlineto{\pgfqpoint{3.502867in}{4.072531in}}%
\pgfusepath{stroke}%
\end{pgfscope}%
\begin{pgfscope}%
\pgfpathrectangle{\pgfqpoint{0.481978in}{0.331635in}}{\pgfqpoint{9.300000in}{7.700000in}}%
\pgfusepath{clip}%
\pgfsetrectcap%
\pgfsetroundjoin%
\pgfsetlinewidth{1.505625pt}%
\definecolor{currentstroke}{rgb}{1.000000,0.705882,0.509804}%
\pgfsetstrokecolor{currentstroke}%
\pgfsetstrokeopacity{0.800000}%
\pgfsetdash{}{0pt}%
\pgfpathmoveto{\pgfqpoint{4.966201in}{4.850330in}}%
\pgfpathlineto{\pgfqpoint{3.502867in}{4.072531in}}%
\pgfusepath{stroke}%
\end{pgfscope}%
\begin{pgfscope}%
\pgfsetrectcap%
\pgfsetmiterjoin%
\pgfsetlinewidth{0.803000pt}%
\definecolor{currentstroke}{rgb}{0.000000,0.000000,0.000000}%
\pgfsetstrokecolor{currentstroke}%
\pgfsetdash{}{0pt}%
\pgfpathmoveto{\pgfqpoint{0.481978in}{0.331635in}}%
\pgfpathlineto{\pgfqpoint{0.481978in}{8.031635in}}%
\pgfusepath{stroke}%
\end{pgfscope}%
\begin{pgfscope}%
\pgfsetrectcap%
\pgfsetmiterjoin%
\pgfsetlinewidth{0.803000pt}%
\definecolor{currentstroke}{rgb}{0.000000,0.000000,0.000000}%
\pgfsetstrokecolor{currentstroke}%
\pgfsetdash{}{0pt}%
\pgfpathmoveto{\pgfqpoint{9.781978in}{0.331635in}}%
\pgfpathlineto{\pgfqpoint{9.781978in}{8.031635in}}%
\pgfusepath{stroke}%
\end{pgfscope}%
\begin{pgfscope}%
\pgfsetrectcap%
\pgfsetmiterjoin%
\pgfsetlinewidth{0.803000pt}%
\definecolor{currentstroke}{rgb}{0.000000,0.000000,0.000000}%
\pgfsetstrokecolor{currentstroke}%
\pgfsetdash{}{0pt}%
\pgfpathmoveto{\pgfqpoint{0.481978in}{0.331635in}}%
\pgfpathlineto{\pgfqpoint{9.781978in}{0.331635in}}%
\pgfusepath{stroke}%
\end{pgfscope}%
\begin{pgfscope}%
\pgfsetrectcap%
\pgfsetmiterjoin%
\pgfsetlinewidth{0.803000pt}%
\definecolor{currentstroke}{rgb}{0.000000,0.000000,0.000000}%
\pgfsetstrokecolor{currentstroke}%
\pgfsetdash{}{0pt}%
\pgfpathmoveto{\pgfqpoint{0.481978in}{8.031635in}}%
\pgfpathlineto{\pgfqpoint{9.781978in}{8.031635in}}%
\pgfusepath{stroke}%
\end{pgfscope}%
\begin{pgfscope}%
\definecolor{textcolor}{rgb}{0.000000,0.000000,0.000000}%
\pgfsetstrokecolor{textcolor}%
\pgfsetfillcolor{textcolor}%
\pgftext[x=5.131978in,y=8.114968in,,base]{\color{textcolor}\sffamily\fontsize{12.000000}{14.400000}\selectfont T-SNE for Pix3D and S2R:3DFREE}%
\end{pgfscope}%
\begin{pgfscope}%
\pgfsetbuttcap%
\pgfsetmiterjoin%
\definecolor{currentfill}{rgb}{1.000000,1.000000,1.000000}%
\pgfsetfillcolor{currentfill}%
\pgfsetfillopacity{0.800000}%
\pgfsetlinewidth{1.003750pt}%
\definecolor{currentstroke}{rgb}{0.800000,0.800000,0.800000}%
\pgfsetstrokecolor{currentstroke}%
\pgfsetstrokeopacity{0.800000}%
\pgfsetdash{}{0pt}%
\pgfpathmoveto{\pgfqpoint{9.879200in}{3.956944in}}%
\pgfpathlineto{\pgfqpoint{11.186410in}{3.956944in}}%
\pgfpathquadraticcurveto{\pgfqpoint{11.214188in}{3.956944in}}{\pgfqpoint{11.214188in}{3.984722in}}%
\pgfpathlineto{\pgfqpoint{11.214188in}{4.378548in}}%
\pgfpathquadraticcurveto{\pgfqpoint{11.214188in}{4.406326in}}{\pgfqpoint{11.186410in}{4.406326in}}%
\pgfpathlineto{\pgfqpoint{9.879200in}{4.406326in}}%
\pgfpathquadraticcurveto{\pgfqpoint{9.851422in}{4.406326in}}{\pgfqpoint{9.851422in}{4.378548in}}%
\pgfpathlineto{\pgfqpoint{9.851422in}{3.984722in}}%
\pgfpathquadraticcurveto{\pgfqpoint{9.851422in}{3.956944in}}{\pgfqpoint{9.879200in}{3.956944in}}%
\pgfpathclose%
\pgfusepath{stroke,fill}%
\end{pgfscope}%
\begin{pgfscope}%
\pgfsetbuttcap%
\pgfsetroundjoin%
\definecolor{currentfill}{rgb}{0.631373,0.788235,0.956863}%
\pgfsetfillcolor{currentfill}%
\pgfsetlinewidth{1.003750pt}%
\definecolor{currentstroke}{rgb}{0.631373,0.788235,0.956863}%
\pgfsetstrokecolor{currentstroke}%
\pgfsetdash{}{0pt}%
\pgfsys@defobject{currentmarker}{\pgfqpoint{-0.041667in}{-0.041667in}}{\pgfqpoint{0.041667in}{0.041667in}}{%
\pgfpathmoveto{\pgfqpoint{0.000000in}{-0.041667in}}%
\pgfpathcurveto{\pgfqpoint{0.011050in}{-0.041667in}}{\pgfqpoint{0.021649in}{-0.037276in}}{\pgfqpoint{0.029463in}{-0.029463in}}%
\pgfpathcurveto{\pgfqpoint{0.037276in}{-0.021649in}}{\pgfqpoint{0.041667in}{-0.011050in}}{\pgfqpoint{0.041667in}{0.000000in}}%
\pgfpathcurveto{\pgfqpoint{0.041667in}{0.011050in}}{\pgfqpoint{0.037276in}{0.021649in}}{\pgfqpoint{0.029463in}{0.029463in}}%
\pgfpathcurveto{\pgfqpoint{0.021649in}{0.037276in}}{\pgfqpoint{0.011050in}{0.041667in}}{\pgfqpoint{0.000000in}{0.041667in}}%
\pgfpathcurveto{\pgfqpoint{-0.011050in}{0.041667in}}{\pgfqpoint{-0.021649in}{0.037276in}}{\pgfqpoint{-0.029463in}{0.029463in}}%
\pgfpathcurveto{\pgfqpoint{-0.037276in}{0.021649in}}{\pgfqpoint{-0.041667in}{0.011050in}}{\pgfqpoint{-0.041667in}{0.000000in}}%
\pgfpathcurveto{\pgfqpoint{-0.041667in}{-0.011050in}}{\pgfqpoint{-0.037276in}{-0.021649in}}{\pgfqpoint{-0.029463in}{-0.029463in}}%
\pgfpathcurveto{\pgfqpoint{-0.021649in}{-0.037276in}}{\pgfqpoint{-0.011050in}{-0.041667in}}{\pgfqpoint{0.000000in}{-0.041667in}}%
\pgfpathclose%
\pgfusepath{stroke,fill}%
}%
\begin{pgfscope}%
\pgfsys@transformshift{10.045867in}{4.281705in}%
\pgfsys@useobject{currentmarker}{}%
\end{pgfscope}%
\end{pgfscope}%
\begin{pgfscope}%
\definecolor{textcolor}{rgb}{0.000000,0.000000,0.000000}%
\pgfsetstrokecolor{textcolor}%
\pgfsetfillcolor{textcolor}%
\pgftext[x=10.295867in,y=4.245247in,left,base]{\color{textcolor}\sffamily\fontsize{10.000000}{12.000000}\selectfont Pix3D}%
\end{pgfscope}%
\begin{pgfscope}%
\pgfsetbuttcap%
\pgfsetroundjoin%
\definecolor{currentfill}{rgb}{1.000000,0.705882,0.509804}%
\pgfsetfillcolor{currentfill}%
\pgfsetlinewidth{1.003750pt}%
\definecolor{currentstroke}{rgb}{1.000000,0.705882,0.509804}%
\pgfsetstrokecolor{currentstroke}%
\pgfsetdash{}{0pt}%
\pgfsys@defobject{currentmarker}{\pgfqpoint{-0.041667in}{-0.041667in}}{\pgfqpoint{0.041667in}{0.041667in}}{%
\pgfpathmoveto{\pgfqpoint{0.000000in}{-0.041667in}}%
\pgfpathcurveto{\pgfqpoint{0.011050in}{-0.041667in}}{\pgfqpoint{0.021649in}{-0.037276in}}{\pgfqpoint{0.029463in}{-0.029463in}}%
\pgfpathcurveto{\pgfqpoint{0.037276in}{-0.021649in}}{\pgfqpoint{0.041667in}{-0.011050in}}{\pgfqpoint{0.041667in}{0.000000in}}%
\pgfpathcurveto{\pgfqpoint{0.041667in}{0.011050in}}{\pgfqpoint{0.037276in}{0.021649in}}{\pgfqpoint{0.029463in}{0.029463in}}%
\pgfpathcurveto{\pgfqpoint{0.021649in}{0.037276in}}{\pgfqpoint{0.011050in}{0.041667in}}{\pgfqpoint{0.000000in}{0.041667in}}%
\pgfpathcurveto{\pgfqpoint{-0.011050in}{0.041667in}}{\pgfqpoint{-0.021649in}{0.037276in}}{\pgfqpoint{-0.029463in}{0.029463in}}%
\pgfpathcurveto{\pgfqpoint{-0.037276in}{0.021649in}}{\pgfqpoint{-0.041667in}{0.011050in}}{\pgfqpoint{-0.041667in}{0.000000in}}%
\pgfpathcurveto{\pgfqpoint{-0.041667in}{-0.011050in}}{\pgfqpoint{-0.037276in}{-0.021649in}}{\pgfqpoint{-0.029463in}{-0.029463in}}%
\pgfpathcurveto{\pgfqpoint{-0.021649in}{-0.037276in}}{\pgfqpoint{-0.011050in}{-0.041667in}}{\pgfqpoint{0.000000in}{-0.041667in}}%
\pgfpathclose%
\pgfusepath{stroke,fill}%
}%
\begin{pgfscope}%
\pgfsys@transformshift{10.045867in}{4.077848in}%
\pgfsys@useobject{currentmarker}{}%
\end{pgfscope}%
\end{pgfscope}%
\begin{pgfscope}%
\definecolor{textcolor}{rgb}{0.000000,0.000000,0.000000}%
\pgfsetstrokecolor{textcolor}%
\pgfsetfillcolor{textcolor}%
\pgftext[x=10.295867in,y=4.041390in,left,base]{\color{textcolor}\sffamily\fontsize{10.000000}{12.000000}\selectfont S2R:3DFREE}%
\end{pgfscope}%
\end{pgfpicture}%
\makeatother%
\endgroup%
}
    \caption{T-SNE visualisation for chair images from Pix3d and \gls{free} dataset.}
    \label{fig:pix3d_s2r3dfree}
\end{figure}

\begin{figure}[!ht]
    \centering
    \resizebox{0.49\linewidth}{5cm}{%% Creator: Matplotlib, PGF backend
%%
%% To include the figure in your LaTeX document, write
%%   \input{<filename>.pgf}
%%
%% Make sure the required packages are loaded in your preamble
%%   \usepackage{pgf}
%%
%% Figures using additional raster images can only be included by \input if
%% they are in the same directory as the main LaTeX file. For loading figures
%% from other directories you can use the `import` package
%%   \usepackage{import}
%%
%% and then include the figures with
%%   \import{<path to file>}{<filename>.pgf}
%%
%% Matplotlib used the following preamble
%%   \usepackage{fontspec}
%%   \setmainfont{DejaVuSerif.ttf}[Path=\detokenize{/Users/apple/opt/anaconda3/envs/kaolin/lib/python3.7/site-packages/matplotlib/mpl-data/fonts/ttf/}]
%%   \setsansfont{DejaVuSans.ttf}[Path=\detokenize{/Users/apple/opt/anaconda3/envs/kaolin/lib/python3.7/site-packages/matplotlib/mpl-data/fonts/ttf/}]
%%   \setmonofont{DejaVuSansMono.ttf}[Path=\detokenize{/Users/apple/opt/anaconda3/envs/kaolin/lib/python3.7/site-packages/matplotlib/mpl-data/fonts/ttf/}]
%%
\begingroup%
\makeatletter%
\begin{pgfpicture}%
\pgfpathrectangle{\pgfpointorigin}{\pgfqpoint{11.229688in}{8.341596in}}%
\pgfusepath{use as bounding box, clip}%
\begin{pgfscope}%
\pgfsetbuttcap%
\pgfsetmiterjoin%
\definecolor{currentfill}{rgb}{1.000000,1.000000,1.000000}%
\pgfsetfillcolor{currentfill}%
\pgfsetlinewidth{0.000000pt}%
\definecolor{currentstroke}{rgb}{1.000000,1.000000,1.000000}%
\pgfsetstrokecolor{currentstroke}%
\pgfsetdash{}{0pt}%
\pgfpathmoveto{\pgfqpoint{0.000000in}{0.000000in}}%
\pgfpathlineto{\pgfqpoint{11.229688in}{0.000000in}}%
\pgfpathlineto{\pgfqpoint{11.229688in}{8.341596in}}%
\pgfpathlineto{\pgfqpoint{0.000000in}{8.341596in}}%
\pgfpathclose%
\pgfusepath{fill}%
\end{pgfscope}%
\begin{pgfscope}%
\pgfsetbuttcap%
\pgfsetmiterjoin%
\definecolor{currentfill}{rgb}{1.000000,1.000000,1.000000}%
\pgfsetfillcolor{currentfill}%
\pgfsetlinewidth{0.000000pt}%
\definecolor{currentstroke}{rgb}{0.000000,0.000000,0.000000}%
\pgfsetstrokecolor{currentstroke}%
\pgfsetstrokeopacity{0.000000}%
\pgfsetdash{}{0pt}%
\pgfpathmoveto{\pgfqpoint{0.481978in}{0.331635in}}%
\pgfpathlineto{\pgfqpoint{9.781978in}{0.331635in}}%
\pgfpathlineto{\pgfqpoint{9.781978in}{8.031635in}}%
\pgfpathlineto{\pgfqpoint{0.481978in}{8.031635in}}%
\pgfpathclose%
\pgfusepath{fill}%
\end{pgfscope}%
\begin{pgfscope}%
\pgfpathrectangle{\pgfqpoint{0.481978in}{0.331635in}}{\pgfqpoint{9.300000in}{7.700000in}}%
\pgfusepath{clip}%
\pgfsetbuttcap%
\pgfsetroundjoin%
\definecolor{currentfill}{rgb}{0.631373,0.788235,0.956863}%
\pgfsetfillcolor{currentfill}%
\pgfsetlinewidth{0.481800pt}%
\definecolor{currentstroke}{rgb}{1.000000,1.000000,1.000000}%
\pgfsetstrokecolor{currentstroke}%
\pgfsetdash{}{0pt}%
\pgfpathmoveto{\pgfqpoint{5.968461in}{3.642287in}}%
\pgfpathcurveto{\pgfqpoint{5.979511in}{3.642287in}}{\pgfqpoint{5.990110in}{3.646677in}}{\pgfqpoint{5.997924in}{3.654491in}}%
\pgfpathcurveto{\pgfqpoint{6.005737in}{3.662304in}}{\pgfqpoint{6.010127in}{3.672903in}}{\pgfqpoint{6.010127in}{3.683953in}}%
\pgfpathcurveto{\pgfqpoint{6.010127in}{3.695004in}}{\pgfqpoint{6.005737in}{3.705603in}}{\pgfqpoint{5.997924in}{3.713416in}}%
\pgfpathcurveto{\pgfqpoint{5.990110in}{3.721230in}}{\pgfqpoint{5.979511in}{3.725620in}}{\pgfqpoint{5.968461in}{3.725620in}}%
\pgfpathcurveto{\pgfqpoint{5.957411in}{3.725620in}}{\pgfqpoint{5.946812in}{3.721230in}}{\pgfqpoint{5.938998in}{3.713416in}}%
\pgfpathcurveto{\pgfqpoint{5.931184in}{3.705603in}}{\pgfqpoint{5.926794in}{3.695004in}}{\pgfqpoint{5.926794in}{3.683953in}}%
\pgfpathcurveto{\pgfqpoint{5.926794in}{3.672903in}}{\pgfqpoint{5.931184in}{3.662304in}}{\pgfqpoint{5.938998in}{3.654491in}}%
\pgfpathcurveto{\pgfqpoint{5.946812in}{3.646677in}}{\pgfqpoint{5.957411in}{3.642287in}}{\pgfqpoint{5.968461in}{3.642287in}}%
\pgfpathclose%
\pgfusepath{stroke,fill}%
\end{pgfscope}%
\begin{pgfscope}%
\pgfpathrectangle{\pgfqpoint{0.481978in}{0.331635in}}{\pgfqpoint{9.300000in}{7.700000in}}%
\pgfusepath{clip}%
\pgfsetbuttcap%
\pgfsetroundjoin%
\definecolor{currentfill}{rgb}{0.631373,0.788235,0.956863}%
\pgfsetfillcolor{currentfill}%
\pgfsetlinewidth{0.481800pt}%
\definecolor{currentstroke}{rgb}{1.000000,1.000000,1.000000}%
\pgfsetstrokecolor{currentstroke}%
\pgfsetdash{}{0pt}%
\pgfpathmoveto{\pgfqpoint{6.716760in}{0.639968in}}%
\pgfpathcurveto{\pgfqpoint{6.727810in}{0.639968in}}{\pgfqpoint{6.738409in}{0.644359in}}{\pgfqpoint{6.746222in}{0.652172in}}%
\pgfpathcurveto{\pgfqpoint{6.754036in}{0.659986in}}{\pgfqpoint{6.758426in}{0.670585in}}{\pgfqpoint{6.758426in}{0.681635in}}%
\pgfpathcurveto{\pgfqpoint{6.758426in}{0.692685in}}{\pgfqpoint{6.754036in}{0.703284in}}{\pgfqpoint{6.746222in}{0.711098in}}%
\pgfpathcurveto{\pgfqpoint{6.738409in}{0.718911in}}{\pgfqpoint{6.727810in}{0.723302in}}{\pgfqpoint{6.716760in}{0.723302in}}%
\pgfpathcurveto{\pgfqpoint{6.705710in}{0.723302in}}{\pgfqpoint{6.695110in}{0.718911in}}{\pgfqpoint{6.687297in}{0.711098in}}%
\pgfpathcurveto{\pgfqpoint{6.679483in}{0.703284in}}{\pgfqpoint{6.675093in}{0.692685in}}{\pgfqpoint{6.675093in}{0.681635in}}%
\pgfpathcurveto{\pgfqpoint{6.675093in}{0.670585in}}{\pgfqpoint{6.679483in}{0.659986in}}{\pgfqpoint{6.687297in}{0.652172in}}%
\pgfpathcurveto{\pgfqpoint{6.695110in}{0.644359in}}{\pgfqpoint{6.705710in}{0.639968in}}{\pgfqpoint{6.716760in}{0.639968in}}%
\pgfpathclose%
\pgfusepath{stroke,fill}%
\end{pgfscope}%
\begin{pgfscope}%
\pgfpathrectangle{\pgfqpoint{0.481978in}{0.331635in}}{\pgfqpoint{9.300000in}{7.700000in}}%
\pgfusepath{clip}%
\pgfsetbuttcap%
\pgfsetroundjoin%
\definecolor{currentfill}{rgb}{0.631373,0.788235,0.956863}%
\pgfsetfillcolor{currentfill}%
\pgfsetlinewidth{0.481800pt}%
\definecolor{currentstroke}{rgb}{1.000000,1.000000,1.000000}%
\pgfsetstrokecolor{currentstroke}%
\pgfsetdash{}{0pt}%
\pgfpathmoveto{\pgfqpoint{4.030414in}{3.987941in}}%
\pgfpathcurveto{\pgfqpoint{4.041464in}{3.987941in}}{\pgfqpoint{4.052063in}{3.992331in}}{\pgfqpoint{4.059877in}{4.000145in}}%
\pgfpathcurveto{\pgfqpoint{4.067690in}{4.007958in}}{\pgfqpoint{4.072080in}{4.018557in}}{\pgfqpoint{4.072080in}{4.029607in}}%
\pgfpathcurveto{\pgfqpoint{4.072080in}{4.040658in}}{\pgfqpoint{4.067690in}{4.051257in}}{\pgfqpoint{4.059877in}{4.059070in}}%
\pgfpathcurveto{\pgfqpoint{4.052063in}{4.066884in}}{\pgfqpoint{4.041464in}{4.071274in}}{\pgfqpoint{4.030414in}{4.071274in}}%
\pgfpathcurveto{\pgfqpoint{4.019364in}{4.071274in}}{\pgfqpoint{4.008765in}{4.066884in}}{\pgfqpoint{4.000951in}{4.059070in}}%
\pgfpathcurveto{\pgfqpoint{3.993137in}{4.051257in}}{\pgfqpoint{3.988747in}{4.040658in}}{\pgfqpoint{3.988747in}{4.029607in}}%
\pgfpathcurveto{\pgfqpoint{3.988747in}{4.018557in}}{\pgfqpoint{3.993137in}{4.007958in}}{\pgfqpoint{4.000951in}{4.000145in}}%
\pgfpathcurveto{\pgfqpoint{4.008765in}{3.992331in}}{\pgfqpoint{4.019364in}{3.987941in}}{\pgfqpoint{4.030414in}{3.987941in}}%
\pgfpathclose%
\pgfusepath{stroke,fill}%
\end{pgfscope}%
\begin{pgfscope}%
\pgfpathrectangle{\pgfqpoint{0.481978in}{0.331635in}}{\pgfqpoint{9.300000in}{7.700000in}}%
\pgfusepath{clip}%
\pgfsetbuttcap%
\pgfsetroundjoin%
\definecolor{currentfill}{rgb}{0.631373,0.788235,0.956863}%
\pgfsetfillcolor{currentfill}%
\pgfsetlinewidth{0.481800pt}%
\definecolor{currentstroke}{rgb}{1.000000,1.000000,1.000000}%
\pgfsetstrokecolor{currentstroke}%
\pgfsetdash{}{0pt}%
\pgfpathmoveto{\pgfqpoint{3.807365in}{2.300877in}}%
\pgfpathcurveto{\pgfqpoint{3.818415in}{2.300877in}}{\pgfqpoint{3.829014in}{2.305268in}}{\pgfqpoint{3.836828in}{2.313081in}}%
\pgfpathcurveto{\pgfqpoint{3.844642in}{2.320895in}}{\pgfqpoint{3.849032in}{2.331494in}}{\pgfqpoint{3.849032in}{2.342544in}}%
\pgfpathcurveto{\pgfqpoint{3.849032in}{2.353594in}}{\pgfqpoint{3.844642in}{2.364193in}}{\pgfqpoint{3.836828in}{2.372007in}}%
\pgfpathcurveto{\pgfqpoint{3.829014in}{2.379820in}}{\pgfqpoint{3.818415in}{2.384211in}}{\pgfqpoint{3.807365in}{2.384211in}}%
\pgfpathcurveto{\pgfqpoint{3.796315in}{2.384211in}}{\pgfqpoint{3.785716in}{2.379820in}}{\pgfqpoint{3.777903in}{2.372007in}}%
\pgfpathcurveto{\pgfqpoint{3.770089in}{2.364193in}}{\pgfqpoint{3.765699in}{2.353594in}}{\pgfqpoint{3.765699in}{2.342544in}}%
\pgfpathcurveto{\pgfqpoint{3.765699in}{2.331494in}}{\pgfqpoint{3.770089in}{2.320895in}}{\pgfqpoint{3.777903in}{2.313081in}}%
\pgfpathcurveto{\pgfqpoint{3.785716in}{2.305268in}}{\pgfqpoint{3.796315in}{2.300877in}}{\pgfqpoint{3.807365in}{2.300877in}}%
\pgfpathclose%
\pgfusepath{stroke,fill}%
\end{pgfscope}%
\begin{pgfscope}%
\pgfpathrectangle{\pgfqpoint{0.481978in}{0.331635in}}{\pgfqpoint{9.300000in}{7.700000in}}%
\pgfusepath{clip}%
\pgfsetbuttcap%
\pgfsetroundjoin%
\definecolor{currentfill}{rgb}{0.631373,0.788235,0.956863}%
\pgfsetfillcolor{currentfill}%
\pgfsetlinewidth{0.481800pt}%
\definecolor{currentstroke}{rgb}{1.000000,1.000000,1.000000}%
\pgfsetstrokecolor{currentstroke}%
\pgfsetdash{}{0pt}%
\pgfpathmoveto{\pgfqpoint{2.350199in}{2.814087in}}%
\pgfpathcurveto{\pgfqpoint{2.361249in}{2.814087in}}{\pgfqpoint{2.371848in}{2.818477in}}{\pgfqpoint{2.379662in}{2.826291in}}%
\pgfpathcurveto{\pgfqpoint{2.387475in}{2.834104in}}{\pgfqpoint{2.391866in}{2.844703in}}{\pgfqpoint{2.391866in}{2.855754in}}%
\pgfpathcurveto{\pgfqpoint{2.391866in}{2.866804in}}{\pgfqpoint{2.387475in}{2.877403in}}{\pgfqpoint{2.379662in}{2.885216in}}%
\pgfpathcurveto{\pgfqpoint{2.371848in}{2.893030in}}{\pgfqpoint{2.361249in}{2.897420in}}{\pgfqpoint{2.350199in}{2.897420in}}%
\pgfpathcurveto{\pgfqpoint{2.339149in}{2.897420in}}{\pgfqpoint{2.328550in}{2.893030in}}{\pgfqpoint{2.320736in}{2.885216in}}%
\pgfpathcurveto{\pgfqpoint{2.312923in}{2.877403in}}{\pgfqpoint{2.308532in}{2.866804in}}{\pgfqpoint{2.308532in}{2.855754in}}%
\pgfpathcurveto{\pgfqpoint{2.308532in}{2.844703in}}{\pgfqpoint{2.312923in}{2.834104in}}{\pgfqpoint{2.320736in}{2.826291in}}%
\pgfpathcurveto{\pgfqpoint{2.328550in}{2.818477in}}{\pgfqpoint{2.339149in}{2.814087in}}{\pgfqpoint{2.350199in}{2.814087in}}%
\pgfpathclose%
\pgfusepath{stroke,fill}%
\end{pgfscope}%
\begin{pgfscope}%
\pgfpathrectangle{\pgfqpoint{0.481978in}{0.331635in}}{\pgfqpoint{9.300000in}{7.700000in}}%
\pgfusepath{clip}%
\pgfsetbuttcap%
\pgfsetroundjoin%
\definecolor{currentfill}{rgb}{0.631373,0.788235,0.956863}%
\pgfsetfillcolor{currentfill}%
\pgfsetlinewidth{0.481800pt}%
\definecolor{currentstroke}{rgb}{1.000000,1.000000,1.000000}%
\pgfsetstrokecolor{currentstroke}%
\pgfsetdash{}{0pt}%
\pgfpathmoveto{\pgfqpoint{5.589222in}{5.930958in}}%
\pgfpathcurveto{\pgfqpoint{5.600272in}{5.930958in}}{\pgfqpoint{5.610871in}{5.935348in}}{\pgfqpoint{5.618685in}{5.943162in}}%
\pgfpathcurveto{\pgfqpoint{5.626498in}{5.950975in}}{\pgfqpoint{5.630888in}{5.961574in}}{\pgfqpoint{5.630888in}{5.972624in}}%
\pgfpathcurveto{\pgfqpoint{5.630888in}{5.983674in}}{\pgfqpoint{5.626498in}{5.994274in}}{\pgfqpoint{5.618685in}{6.002087in}}%
\pgfpathcurveto{\pgfqpoint{5.610871in}{6.009901in}}{\pgfqpoint{5.600272in}{6.014291in}}{\pgfqpoint{5.589222in}{6.014291in}}%
\pgfpathcurveto{\pgfqpoint{5.578172in}{6.014291in}}{\pgfqpoint{5.567573in}{6.009901in}}{\pgfqpoint{5.559759in}{6.002087in}}%
\pgfpathcurveto{\pgfqpoint{5.551945in}{5.994274in}}{\pgfqpoint{5.547555in}{5.983674in}}{\pgfqpoint{5.547555in}{5.972624in}}%
\pgfpathcurveto{\pgfqpoint{5.547555in}{5.961574in}}{\pgfqpoint{5.551945in}{5.950975in}}{\pgfqpoint{5.559759in}{5.943162in}}%
\pgfpathcurveto{\pgfqpoint{5.567573in}{5.935348in}}{\pgfqpoint{5.578172in}{5.930958in}}{\pgfqpoint{5.589222in}{5.930958in}}%
\pgfpathclose%
\pgfusepath{stroke,fill}%
\end{pgfscope}%
\begin{pgfscope}%
\pgfpathrectangle{\pgfqpoint{0.481978in}{0.331635in}}{\pgfqpoint{9.300000in}{7.700000in}}%
\pgfusepath{clip}%
\pgfsetbuttcap%
\pgfsetroundjoin%
\definecolor{currentfill}{rgb}{0.631373,0.788235,0.956863}%
\pgfsetfillcolor{currentfill}%
\pgfsetlinewidth{0.481800pt}%
\definecolor{currentstroke}{rgb}{1.000000,1.000000,1.000000}%
\pgfsetstrokecolor{currentstroke}%
\pgfsetdash{}{0pt}%
\pgfpathmoveto{\pgfqpoint{9.359251in}{4.404130in}}%
\pgfpathcurveto{\pgfqpoint{9.370301in}{4.404130in}}{\pgfqpoint{9.380900in}{4.408521in}}{\pgfqpoint{9.388713in}{4.416334in}}%
\pgfpathcurveto{\pgfqpoint{9.396527in}{4.424148in}}{\pgfqpoint{9.400917in}{4.434747in}}{\pgfqpoint{9.400917in}{4.445797in}}%
\pgfpathcurveto{\pgfqpoint{9.400917in}{4.456847in}}{\pgfqpoint{9.396527in}{4.467446in}}{\pgfqpoint{9.388713in}{4.475260in}}%
\pgfpathcurveto{\pgfqpoint{9.380900in}{4.483074in}}{\pgfqpoint{9.370301in}{4.487464in}}{\pgfqpoint{9.359251in}{4.487464in}}%
\pgfpathcurveto{\pgfqpoint{9.348201in}{4.487464in}}{\pgfqpoint{9.337601in}{4.483074in}}{\pgfqpoint{9.329788in}{4.475260in}}%
\pgfpathcurveto{\pgfqpoint{9.321974in}{4.467446in}}{\pgfqpoint{9.317584in}{4.456847in}}{\pgfqpoint{9.317584in}{4.445797in}}%
\pgfpathcurveto{\pgfqpoint{9.317584in}{4.434747in}}{\pgfqpoint{9.321974in}{4.424148in}}{\pgfqpoint{9.329788in}{4.416334in}}%
\pgfpathcurveto{\pgfqpoint{9.337601in}{4.408521in}}{\pgfqpoint{9.348201in}{4.404130in}}{\pgfqpoint{9.359251in}{4.404130in}}%
\pgfpathclose%
\pgfusepath{stroke,fill}%
\end{pgfscope}%
\begin{pgfscope}%
\pgfpathrectangle{\pgfqpoint{0.481978in}{0.331635in}}{\pgfqpoint{9.300000in}{7.700000in}}%
\pgfusepath{clip}%
\pgfsetbuttcap%
\pgfsetroundjoin%
\definecolor{currentfill}{rgb}{0.631373,0.788235,0.956863}%
\pgfsetfillcolor{currentfill}%
\pgfsetlinewidth{0.481800pt}%
\definecolor{currentstroke}{rgb}{1.000000,1.000000,1.000000}%
\pgfsetstrokecolor{currentstroke}%
\pgfsetdash{}{0pt}%
\pgfpathmoveto{\pgfqpoint{5.412406in}{5.045738in}}%
\pgfpathcurveto{\pgfqpoint{5.423456in}{5.045738in}}{\pgfqpoint{5.434055in}{5.050128in}}{\pgfqpoint{5.441869in}{5.057942in}}%
\pgfpathcurveto{\pgfqpoint{5.449682in}{5.065756in}}{\pgfqpoint{5.454073in}{5.076355in}}{\pgfqpoint{5.454073in}{5.087405in}}%
\pgfpathcurveto{\pgfqpoint{5.454073in}{5.098455in}}{\pgfqpoint{5.449682in}{5.109054in}}{\pgfqpoint{5.441869in}{5.116868in}}%
\pgfpathcurveto{\pgfqpoint{5.434055in}{5.124681in}}{\pgfqpoint{5.423456in}{5.129071in}}{\pgfqpoint{5.412406in}{5.129071in}}%
\pgfpathcurveto{\pgfqpoint{5.401356in}{5.129071in}}{\pgfqpoint{5.390757in}{5.124681in}}{\pgfqpoint{5.382943in}{5.116868in}}%
\pgfpathcurveto{\pgfqpoint{5.375130in}{5.109054in}}{\pgfqpoint{5.370739in}{5.098455in}}{\pgfqpoint{5.370739in}{5.087405in}}%
\pgfpathcurveto{\pgfqpoint{5.370739in}{5.076355in}}{\pgfqpoint{5.375130in}{5.065756in}}{\pgfqpoint{5.382943in}{5.057942in}}%
\pgfpathcurveto{\pgfqpoint{5.390757in}{5.050128in}}{\pgfqpoint{5.401356in}{5.045738in}}{\pgfqpoint{5.412406in}{5.045738in}}%
\pgfpathclose%
\pgfusepath{stroke,fill}%
\end{pgfscope}%
\begin{pgfscope}%
\pgfpathrectangle{\pgfqpoint{0.481978in}{0.331635in}}{\pgfqpoint{9.300000in}{7.700000in}}%
\pgfusepath{clip}%
\pgfsetbuttcap%
\pgfsetroundjoin%
\definecolor{currentfill}{rgb}{0.631373,0.788235,0.956863}%
\pgfsetfillcolor{currentfill}%
\pgfsetlinewidth{0.481800pt}%
\definecolor{currentstroke}{rgb}{1.000000,1.000000,1.000000}%
\pgfsetstrokecolor{currentstroke}%
\pgfsetdash{}{0pt}%
\pgfpathmoveto{\pgfqpoint{3.390760in}{6.231096in}}%
\pgfpathcurveto{\pgfqpoint{3.401810in}{6.231096in}}{\pgfqpoint{3.412409in}{6.235486in}}{\pgfqpoint{3.420223in}{6.243300in}}%
\pgfpathcurveto{\pgfqpoint{3.428037in}{6.251113in}}{\pgfqpoint{3.432427in}{6.261712in}}{\pgfqpoint{3.432427in}{6.272762in}}%
\pgfpathcurveto{\pgfqpoint{3.432427in}{6.283813in}}{\pgfqpoint{3.428037in}{6.294412in}}{\pgfqpoint{3.420223in}{6.302225in}}%
\pgfpathcurveto{\pgfqpoint{3.412409in}{6.310039in}}{\pgfqpoint{3.401810in}{6.314429in}}{\pgfqpoint{3.390760in}{6.314429in}}%
\pgfpathcurveto{\pgfqpoint{3.379710in}{6.314429in}}{\pgfqpoint{3.369111in}{6.310039in}}{\pgfqpoint{3.361297in}{6.302225in}}%
\pgfpathcurveto{\pgfqpoint{3.353484in}{6.294412in}}{\pgfqpoint{3.349094in}{6.283813in}}{\pgfqpoint{3.349094in}{6.272762in}}%
\pgfpathcurveto{\pgfqpoint{3.349094in}{6.261712in}}{\pgfqpoint{3.353484in}{6.251113in}}{\pgfqpoint{3.361297in}{6.243300in}}%
\pgfpathcurveto{\pgfqpoint{3.369111in}{6.235486in}}{\pgfqpoint{3.379710in}{6.231096in}}{\pgfqpoint{3.390760in}{6.231096in}}%
\pgfpathclose%
\pgfusepath{stroke,fill}%
\end{pgfscope}%
\begin{pgfscope}%
\pgfpathrectangle{\pgfqpoint{0.481978in}{0.331635in}}{\pgfqpoint{9.300000in}{7.700000in}}%
\pgfusepath{clip}%
\pgfsetbuttcap%
\pgfsetroundjoin%
\definecolor{currentfill}{rgb}{0.631373,0.788235,0.956863}%
\pgfsetfillcolor{currentfill}%
\pgfsetlinewidth{0.481800pt}%
\definecolor{currentstroke}{rgb}{1.000000,1.000000,1.000000}%
\pgfsetstrokecolor{currentstroke}%
\pgfsetdash{}{0pt}%
\pgfpathmoveto{\pgfqpoint{7.065739in}{4.651877in}}%
\pgfpathcurveto{\pgfqpoint{7.076789in}{4.651877in}}{\pgfqpoint{7.087388in}{4.656267in}}{\pgfqpoint{7.095202in}{4.664081in}}%
\pgfpathcurveto{\pgfqpoint{7.103015in}{4.671894in}}{\pgfqpoint{7.107405in}{4.682493in}}{\pgfqpoint{7.107405in}{4.693543in}}%
\pgfpathcurveto{\pgfqpoint{7.107405in}{4.704594in}}{\pgfqpoint{7.103015in}{4.715193in}}{\pgfqpoint{7.095202in}{4.723006in}}%
\pgfpathcurveto{\pgfqpoint{7.087388in}{4.730820in}}{\pgfqpoint{7.076789in}{4.735210in}}{\pgfqpoint{7.065739in}{4.735210in}}%
\pgfpathcurveto{\pgfqpoint{7.054689in}{4.735210in}}{\pgfqpoint{7.044090in}{4.730820in}}{\pgfqpoint{7.036276in}{4.723006in}}%
\pgfpathcurveto{\pgfqpoint{7.028462in}{4.715193in}}{\pgfqpoint{7.024072in}{4.704594in}}{\pgfqpoint{7.024072in}{4.693543in}}%
\pgfpathcurveto{\pgfqpoint{7.024072in}{4.682493in}}{\pgfqpoint{7.028462in}{4.671894in}}{\pgfqpoint{7.036276in}{4.664081in}}%
\pgfpathcurveto{\pgfqpoint{7.044090in}{4.656267in}}{\pgfqpoint{7.054689in}{4.651877in}}{\pgfqpoint{7.065739in}{4.651877in}}%
\pgfpathclose%
\pgfusepath{stroke,fill}%
\end{pgfscope}%
\begin{pgfscope}%
\pgfpathrectangle{\pgfqpoint{0.481978in}{0.331635in}}{\pgfqpoint{9.300000in}{7.700000in}}%
\pgfusepath{clip}%
\pgfsetbuttcap%
\pgfsetroundjoin%
\definecolor{currentfill}{rgb}{0.631373,0.788235,0.956863}%
\pgfsetfillcolor{currentfill}%
\pgfsetlinewidth{0.481800pt}%
\definecolor{currentstroke}{rgb}{1.000000,1.000000,1.000000}%
\pgfsetstrokecolor{currentstroke}%
\pgfsetdash{}{0pt}%
\pgfpathmoveto{\pgfqpoint{6.796596in}{5.826111in}}%
\pgfpathcurveto{\pgfqpoint{6.807646in}{5.826111in}}{\pgfqpoint{6.818245in}{5.830501in}}{\pgfqpoint{6.826058in}{5.838315in}}%
\pgfpathcurveto{\pgfqpoint{6.833872in}{5.846128in}}{\pgfqpoint{6.838262in}{5.856728in}}{\pgfqpoint{6.838262in}{5.867778in}}%
\pgfpathcurveto{\pgfqpoint{6.838262in}{5.878828in}}{\pgfqpoint{6.833872in}{5.889427in}}{\pgfqpoint{6.826058in}{5.897240in}}%
\pgfpathcurveto{\pgfqpoint{6.818245in}{5.905054in}}{\pgfqpoint{6.807646in}{5.909444in}}{\pgfqpoint{6.796596in}{5.909444in}}%
\pgfpathcurveto{\pgfqpoint{6.785545in}{5.909444in}}{\pgfqpoint{6.774946in}{5.905054in}}{\pgfqpoint{6.767133in}{5.897240in}}%
\pgfpathcurveto{\pgfqpoint{6.759319in}{5.889427in}}{\pgfqpoint{6.754929in}{5.878828in}}{\pgfqpoint{6.754929in}{5.867778in}}%
\pgfpathcurveto{\pgfqpoint{6.754929in}{5.856728in}}{\pgfqpoint{6.759319in}{5.846128in}}{\pgfqpoint{6.767133in}{5.838315in}}%
\pgfpathcurveto{\pgfqpoint{6.774946in}{5.830501in}}{\pgfqpoint{6.785545in}{5.826111in}}{\pgfqpoint{6.796596in}{5.826111in}}%
\pgfpathclose%
\pgfusepath{stroke,fill}%
\end{pgfscope}%
\begin{pgfscope}%
\pgfpathrectangle{\pgfqpoint{0.481978in}{0.331635in}}{\pgfqpoint{9.300000in}{7.700000in}}%
\pgfusepath{clip}%
\pgfsetbuttcap%
\pgfsetroundjoin%
\definecolor{currentfill}{rgb}{0.631373,0.788235,0.956863}%
\pgfsetfillcolor{currentfill}%
\pgfsetlinewidth{0.481800pt}%
\definecolor{currentstroke}{rgb}{1.000000,1.000000,1.000000}%
\pgfsetstrokecolor{currentstroke}%
\pgfsetdash{}{0pt}%
\pgfpathmoveto{\pgfqpoint{6.170106in}{5.271216in}}%
\pgfpathcurveto{\pgfqpoint{6.181156in}{5.271216in}}{\pgfqpoint{6.191755in}{5.275606in}}{\pgfqpoint{6.199569in}{5.283419in}}%
\pgfpathcurveto{\pgfqpoint{6.207382in}{5.291233in}}{\pgfqpoint{6.211772in}{5.301832in}}{\pgfqpoint{6.211772in}{5.312882in}}%
\pgfpathcurveto{\pgfqpoint{6.211772in}{5.323932in}}{\pgfqpoint{6.207382in}{5.334531in}}{\pgfqpoint{6.199569in}{5.342345in}}%
\pgfpathcurveto{\pgfqpoint{6.191755in}{5.350159in}}{\pgfqpoint{6.181156in}{5.354549in}}{\pgfqpoint{6.170106in}{5.354549in}}%
\pgfpathcurveto{\pgfqpoint{6.159056in}{5.354549in}}{\pgfqpoint{6.148457in}{5.350159in}}{\pgfqpoint{6.140643in}{5.342345in}}%
\pgfpathcurveto{\pgfqpoint{6.132829in}{5.334531in}}{\pgfqpoint{6.128439in}{5.323932in}}{\pgfqpoint{6.128439in}{5.312882in}}%
\pgfpathcurveto{\pgfqpoint{6.128439in}{5.301832in}}{\pgfqpoint{6.132829in}{5.291233in}}{\pgfqpoint{6.140643in}{5.283419in}}%
\pgfpathcurveto{\pgfqpoint{6.148457in}{5.275606in}}{\pgfqpoint{6.159056in}{5.271216in}}{\pgfqpoint{6.170106in}{5.271216in}}%
\pgfpathclose%
\pgfusepath{stroke,fill}%
\end{pgfscope}%
\begin{pgfscope}%
\pgfpathrectangle{\pgfqpoint{0.481978in}{0.331635in}}{\pgfqpoint{9.300000in}{7.700000in}}%
\pgfusepath{clip}%
\pgfsetbuttcap%
\pgfsetroundjoin%
\definecolor{currentfill}{rgb}{0.631373,0.788235,0.956863}%
\pgfsetfillcolor{currentfill}%
\pgfsetlinewidth{0.481800pt}%
\definecolor{currentstroke}{rgb}{1.000000,1.000000,1.000000}%
\pgfsetstrokecolor{currentstroke}%
\pgfsetdash{}{0pt}%
\pgfpathmoveto{\pgfqpoint{7.510480in}{3.033169in}}%
\pgfpathcurveto{\pgfqpoint{7.521530in}{3.033169in}}{\pgfqpoint{7.532130in}{3.037559in}}{\pgfqpoint{7.539943in}{3.045372in}}%
\pgfpathcurveto{\pgfqpoint{7.547757in}{3.053186in}}{\pgfqpoint{7.552147in}{3.063785in}}{\pgfqpoint{7.552147in}{3.074835in}}%
\pgfpathcurveto{\pgfqpoint{7.552147in}{3.085885in}}{\pgfqpoint{7.547757in}{3.096484in}}{\pgfqpoint{7.539943in}{3.104298in}}%
\pgfpathcurveto{\pgfqpoint{7.532130in}{3.112112in}}{\pgfqpoint{7.521530in}{3.116502in}}{\pgfqpoint{7.510480in}{3.116502in}}%
\pgfpathcurveto{\pgfqpoint{7.499430in}{3.116502in}}{\pgfqpoint{7.488831in}{3.112112in}}{\pgfqpoint{7.481018in}{3.104298in}}%
\pgfpathcurveto{\pgfqpoint{7.473204in}{3.096484in}}{\pgfqpoint{7.468814in}{3.085885in}}{\pgfqpoint{7.468814in}{3.074835in}}%
\pgfpathcurveto{\pgfqpoint{7.468814in}{3.063785in}}{\pgfqpoint{7.473204in}{3.053186in}}{\pgfqpoint{7.481018in}{3.045372in}}%
\pgfpathcurveto{\pgfqpoint{7.488831in}{3.037559in}}{\pgfqpoint{7.499430in}{3.033169in}}{\pgfqpoint{7.510480in}{3.033169in}}%
\pgfpathclose%
\pgfusepath{stroke,fill}%
\end{pgfscope}%
\begin{pgfscope}%
\pgfpathrectangle{\pgfqpoint{0.481978in}{0.331635in}}{\pgfqpoint{9.300000in}{7.700000in}}%
\pgfusepath{clip}%
\pgfsetbuttcap%
\pgfsetroundjoin%
\definecolor{currentfill}{rgb}{0.631373,0.788235,0.956863}%
\pgfsetfillcolor{currentfill}%
\pgfsetlinewidth{0.481800pt}%
\definecolor{currentstroke}{rgb}{1.000000,1.000000,1.000000}%
\pgfsetstrokecolor{currentstroke}%
\pgfsetdash{}{0pt}%
\pgfpathmoveto{\pgfqpoint{6.540018in}{1.774081in}}%
\pgfpathcurveto{\pgfqpoint{6.551068in}{1.774081in}}{\pgfqpoint{6.561667in}{1.778471in}}{\pgfqpoint{6.569480in}{1.786285in}}%
\pgfpathcurveto{\pgfqpoint{6.577294in}{1.794098in}}{\pgfqpoint{6.581684in}{1.804697in}}{\pgfqpoint{6.581684in}{1.815747in}}%
\pgfpathcurveto{\pgfqpoint{6.581684in}{1.826797in}}{\pgfqpoint{6.577294in}{1.837396in}}{\pgfqpoint{6.569480in}{1.845210in}}%
\pgfpathcurveto{\pgfqpoint{6.561667in}{1.853024in}}{\pgfqpoint{6.551068in}{1.857414in}}{\pgfqpoint{6.540018in}{1.857414in}}%
\pgfpathcurveto{\pgfqpoint{6.528968in}{1.857414in}}{\pgfqpoint{6.518369in}{1.853024in}}{\pgfqpoint{6.510555in}{1.845210in}}%
\pgfpathcurveto{\pgfqpoint{6.502741in}{1.837396in}}{\pgfqpoint{6.498351in}{1.826797in}}{\pgfqpoint{6.498351in}{1.815747in}}%
\pgfpathcurveto{\pgfqpoint{6.498351in}{1.804697in}}{\pgfqpoint{6.502741in}{1.794098in}}{\pgfqpoint{6.510555in}{1.786285in}}%
\pgfpathcurveto{\pgfqpoint{6.518369in}{1.778471in}}{\pgfqpoint{6.528968in}{1.774081in}}{\pgfqpoint{6.540018in}{1.774081in}}%
\pgfpathclose%
\pgfusepath{stroke,fill}%
\end{pgfscope}%
\begin{pgfscope}%
\pgfpathrectangle{\pgfqpoint{0.481978in}{0.331635in}}{\pgfqpoint{9.300000in}{7.700000in}}%
\pgfusepath{clip}%
\pgfsetbuttcap%
\pgfsetroundjoin%
\definecolor{currentfill}{rgb}{0.631373,0.788235,0.956863}%
\pgfsetfillcolor{currentfill}%
\pgfsetlinewidth{0.481800pt}%
\definecolor{currentstroke}{rgb}{1.000000,1.000000,1.000000}%
\pgfsetstrokecolor{currentstroke}%
\pgfsetdash{}{0pt}%
\pgfpathmoveto{\pgfqpoint{6.818751in}{2.766920in}}%
\pgfpathcurveto{\pgfqpoint{6.829801in}{2.766920in}}{\pgfqpoint{6.840400in}{2.771310in}}{\pgfqpoint{6.848214in}{2.779124in}}%
\pgfpathcurveto{\pgfqpoint{6.856027in}{2.786938in}}{\pgfqpoint{6.860418in}{2.797537in}}{\pgfqpoint{6.860418in}{2.808587in}}%
\pgfpathcurveto{\pgfqpoint{6.860418in}{2.819637in}}{\pgfqpoint{6.856027in}{2.830236in}}{\pgfqpoint{6.848214in}{2.838050in}}%
\pgfpathcurveto{\pgfqpoint{6.840400in}{2.845863in}}{\pgfqpoint{6.829801in}{2.850254in}}{\pgfqpoint{6.818751in}{2.850254in}}%
\pgfpathcurveto{\pgfqpoint{6.807701in}{2.850254in}}{\pgfqpoint{6.797102in}{2.845863in}}{\pgfqpoint{6.789288in}{2.838050in}}%
\pgfpathcurveto{\pgfqpoint{6.781474in}{2.830236in}}{\pgfqpoint{6.777084in}{2.819637in}}{\pgfqpoint{6.777084in}{2.808587in}}%
\pgfpathcurveto{\pgfqpoint{6.777084in}{2.797537in}}{\pgfqpoint{6.781474in}{2.786938in}}{\pgfqpoint{6.789288in}{2.779124in}}%
\pgfpathcurveto{\pgfqpoint{6.797102in}{2.771310in}}{\pgfqpoint{6.807701in}{2.766920in}}{\pgfqpoint{6.818751in}{2.766920in}}%
\pgfpathclose%
\pgfusepath{stroke,fill}%
\end{pgfscope}%
\begin{pgfscope}%
\pgfpathrectangle{\pgfqpoint{0.481978in}{0.331635in}}{\pgfqpoint{9.300000in}{7.700000in}}%
\pgfusepath{clip}%
\pgfsetbuttcap%
\pgfsetroundjoin%
\definecolor{currentfill}{rgb}{0.631373,0.788235,0.956863}%
\pgfsetfillcolor{currentfill}%
\pgfsetlinewidth{0.481800pt}%
\definecolor{currentstroke}{rgb}{1.000000,1.000000,1.000000}%
\pgfsetstrokecolor{currentstroke}%
\pgfsetdash{}{0pt}%
\pgfpathmoveto{\pgfqpoint{4.820436in}{6.196549in}}%
\pgfpathcurveto{\pgfqpoint{4.831487in}{6.196549in}}{\pgfqpoint{4.842086in}{6.200939in}}{\pgfqpoint{4.849899in}{6.208752in}}%
\pgfpathcurveto{\pgfqpoint{4.857713in}{6.216566in}}{\pgfqpoint{4.862103in}{6.227165in}}{\pgfqpoint{4.862103in}{6.238215in}}%
\pgfpathcurveto{\pgfqpoint{4.862103in}{6.249265in}}{\pgfqpoint{4.857713in}{6.259864in}}{\pgfqpoint{4.849899in}{6.267678in}}%
\pgfpathcurveto{\pgfqpoint{4.842086in}{6.275492in}}{\pgfqpoint{4.831487in}{6.279882in}}{\pgfqpoint{4.820436in}{6.279882in}}%
\pgfpathcurveto{\pgfqpoint{4.809386in}{6.279882in}}{\pgfqpoint{4.798787in}{6.275492in}}{\pgfqpoint{4.790974in}{6.267678in}}%
\pgfpathcurveto{\pgfqpoint{4.783160in}{6.259864in}}{\pgfqpoint{4.778770in}{6.249265in}}{\pgfqpoint{4.778770in}{6.238215in}}%
\pgfpathcurveto{\pgfqpoint{4.778770in}{6.227165in}}{\pgfqpoint{4.783160in}{6.216566in}}{\pgfqpoint{4.790974in}{6.208752in}}%
\pgfpathcurveto{\pgfqpoint{4.798787in}{6.200939in}}{\pgfqpoint{4.809386in}{6.196549in}}{\pgfqpoint{4.820436in}{6.196549in}}%
\pgfpathclose%
\pgfusepath{stroke,fill}%
\end{pgfscope}%
\begin{pgfscope}%
\pgfpathrectangle{\pgfqpoint{0.481978in}{0.331635in}}{\pgfqpoint{9.300000in}{7.700000in}}%
\pgfusepath{clip}%
\pgfsetbuttcap%
\pgfsetroundjoin%
\definecolor{currentfill}{rgb}{0.631373,0.788235,0.956863}%
\pgfsetfillcolor{currentfill}%
\pgfsetlinewidth{0.481800pt}%
\definecolor{currentstroke}{rgb}{1.000000,1.000000,1.000000}%
\pgfsetstrokecolor{currentstroke}%
\pgfsetdash{}{0pt}%
\pgfpathmoveto{\pgfqpoint{5.842861in}{4.427537in}}%
\pgfpathcurveto{\pgfqpoint{5.853911in}{4.427537in}}{\pgfqpoint{5.864510in}{4.431928in}}{\pgfqpoint{5.872324in}{4.439741in}}%
\pgfpathcurveto{\pgfqpoint{5.880137in}{4.447555in}}{\pgfqpoint{5.884528in}{4.458154in}}{\pgfqpoint{5.884528in}{4.469204in}}%
\pgfpathcurveto{\pgfqpoint{5.884528in}{4.480254in}}{\pgfqpoint{5.880137in}{4.490853in}}{\pgfqpoint{5.872324in}{4.498667in}}%
\pgfpathcurveto{\pgfqpoint{5.864510in}{4.506481in}}{\pgfqpoint{5.853911in}{4.510871in}}{\pgfqpoint{5.842861in}{4.510871in}}%
\pgfpathcurveto{\pgfqpoint{5.831811in}{4.510871in}}{\pgfqpoint{5.821212in}{4.506481in}}{\pgfqpoint{5.813398in}{4.498667in}}%
\pgfpathcurveto{\pgfqpoint{5.805585in}{4.490853in}}{\pgfqpoint{5.801194in}{4.480254in}}{\pgfqpoint{5.801194in}{4.469204in}}%
\pgfpathcurveto{\pgfqpoint{5.801194in}{4.458154in}}{\pgfqpoint{5.805585in}{4.447555in}}{\pgfqpoint{5.813398in}{4.439741in}}%
\pgfpathcurveto{\pgfqpoint{5.821212in}{4.431928in}}{\pgfqpoint{5.831811in}{4.427537in}}{\pgfqpoint{5.842861in}{4.427537in}}%
\pgfpathclose%
\pgfusepath{stroke,fill}%
\end{pgfscope}%
\begin{pgfscope}%
\pgfpathrectangle{\pgfqpoint{0.481978in}{0.331635in}}{\pgfqpoint{9.300000in}{7.700000in}}%
\pgfusepath{clip}%
\pgfsetbuttcap%
\pgfsetroundjoin%
\definecolor{currentfill}{rgb}{0.631373,0.788235,0.956863}%
\pgfsetfillcolor{currentfill}%
\pgfsetlinewidth{0.481800pt}%
\definecolor{currentstroke}{rgb}{1.000000,1.000000,1.000000}%
\pgfsetstrokecolor{currentstroke}%
\pgfsetdash{}{0pt}%
\pgfpathmoveto{\pgfqpoint{6.322821in}{6.783128in}}%
\pgfpathcurveto{\pgfqpoint{6.333871in}{6.783128in}}{\pgfqpoint{6.344470in}{6.787518in}}{\pgfqpoint{6.352284in}{6.795331in}}%
\pgfpathcurveto{\pgfqpoint{6.360097in}{6.803145in}}{\pgfqpoint{6.364487in}{6.813744in}}{\pgfqpoint{6.364487in}{6.824794in}}%
\pgfpathcurveto{\pgfqpoint{6.364487in}{6.835844in}}{\pgfqpoint{6.360097in}{6.846443in}}{\pgfqpoint{6.352284in}{6.854257in}}%
\pgfpathcurveto{\pgfqpoint{6.344470in}{6.862071in}}{\pgfqpoint{6.333871in}{6.866461in}}{\pgfqpoint{6.322821in}{6.866461in}}%
\pgfpathcurveto{\pgfqpoint{6.311771in}{6.866461in}}{\pgfqpoint{6.301172in}{6.862071in}}{\pgfqpoint{6.293358in}{6.854257in}}%
\pgfpathcurveto{\pgfqpoint{6.285544in}{6.846443in}}{\pgfqpoint{6.281154in}{6.835844in}}{\pgfqpoint{6.281154in}{6.824794in}}%
\pgfpathcurveto{\pgfqpoint{6.281154in}{6.813744in}}{\pgfqpoint{6.285544in}{6.803145in}}{\pgfqpoint{6.293358in}{6.795331in}}%
\pgfpathcurveto{\pgfqpoint{6.301172in}{6.787518in}}{\pgfqpoint{6.311771in}{6.783128in}}{\pgfqpoint{6.322821in}{6.783128in}}%
\pgfpathclose%
\pgfusepath{stroke,fill}%
\end{pgfscope}%
\begin{pgfscope}%
\pgfpathrectangle{\pgfqpoint{0.481978in}{0.331635in}}{\pgfqpoint{9.300000in}{7.700000in}}%
\pgfusepath{clip}%
\pgfsetbuttcap%
\pgfsetroundjoin%
\definecolor{currentfill}{rgb}{0.631373,0.788235,0.956863}%
\pgfsetfillcolor{currentfill}%
\pgfsetlinewidth{0.481800pt}%
\definecolor{currentstroke}{rgb}{1.000000,1.000000,1.000000}%
\pgfsetstrokecolor{currentstroke}%
\pgfsetdash{}{0pt}%
\pgfpathmoveto{\pgfqpoint{7.830146in}{0.979624in}}%
\pgfpathcurveto{\pgfqpoint{7.841196in}{0.979624in}}{\pgfqpoint{7.851795in}{0.984014in}}{\pgfqpoint{7.859608in}{0.991828in}}%
\pgfpathcurveto{\pgfqpoint{7.867422in}{0.999641in}}{\pgfqpoint{7.871812in}{1.010240in}}{\pgfqpoint{7.871812in}{1.021290in}}%
\pgfpathcurveto{\pgfqpoint{7.871812in}{1.032340in}}{\pgfqpoint{7.867422in}{1.042939in}}{\pgfqpoint{7.859608in}{1.050753in}}%
\pgfpathcurveto{\pgfqpoint{7.851795in}{1.058567in}}{\pgfqpoint{7.841196in}{1.062957in}}{\pgfqpoint{7.830146in}{1.062957in}}%
\pgfpathcurveto{\pgfqpoint{7.819095in}{1.062957in}}{\pgfqpoint{7.808496in}{1.058567in}}{\pgfqpoint{7.800683in}{1.050753in}}%
\pgfpathcurveto{\pgfqpoint{7.792869in}{1.042939in}}{\pgfqpoint{7.788479in}{1.032340in}}{\pgfqpoint{7.788479in}{1.021290in}}%
\pgfpathcurveto{\pgfqpoint{7.788479in}{1.010240in}}{\pgfqpoint{7.792869in}{0.999641in}}{\pgfqpoint{7.800683in}{0.991828in}}%
\pgfpathcurveto{\pgfqpoint{7.808496in}{0.984014in}}{\pgfqpoint{7.819095in}{0.979624in}}{\pgfqpoint{7.830146in}{0.979624in}}%
\pgfpathclose%
\pgfusepath{stroke,fill}%
\end{pgfscope}%
\begin{pgfscope}%
\pgfpathrectangle{\pgfqpoint{0.481978in}{0.331635in}}{\pgfqpoint{9.300000in}{7.700000in}}%
\pgfusepath{clip}%
\pgfsetbuttcap%
\pgfsetroundjoin%
\definecolor{currentfill}{rgb}{0.631373,0.788235,0.956863}%
\pgfsetfillcolor{currentfill}%
\pgfsetlinewidth{0.481800pt}%
\definecolor{currentstroke}{rgb}{1.000000,1.000000,1.000000}%
\pgfsetstrokecolor{currentstroke}%
\pgfsetdash{}{0pt}%
\pgfpathmoveto{\pgfqpoint{5.762941in}{2.425077in}}%
\pgfpathcurveto{\pgfqpoint{5.773991in}{2.425077in}}{\pgfqpoint{5.784590in}{2.429467in}}{\pgfqpoint{5.792404in}{2.437281in}}%
\pgfpathcurveto{\pgfqpoint{5.800217in}{2.445094in}}{\pgfqpoint{5.804608in}{2.455693in}}{\pgfqpoint{5.804608in}{2.466744in}}%
\pgfpathcurveto{\pgfqpoint{5.804608in}{2.477794in}}{\pgfqpoint{5.800217in}{2.488393in}}{\pgfqpoint{5.792404in}{2.496206in}}%
\pgfpathcurveto{\pgfqpoint{5.784590in}{2.504020in}}{\pgfqpoint{5.773991in}{2.508410in}}{\pgfqpoint{5.762941in}{2.508410in}}%
\pgfpathcurveto{\pgfqpoint{5.751891in}{2.508410in}}{\pgfqpoint{5.741292in}{2.504020in}}{\pgfqpoint{5.733478in}{2.496206in}}%
\pgfpathcurveto{\pgfqpoint{5.725665in}{2.488393in}}{\pgfqpoint{5.721274in}{2.477794in}}{\pgfqpoint{5.721274in}{2.466744in}}%
\pgfpathcurveto{\pgfqpoint{5.721274in}{2.455693in}}{\pgfqpoint{5.725665in}{2.445094in}}{\pgfqpoint{5.733478in}{2.437281in}}%
\pgfpathcurveto{\pgfqpoint{5.741292in}{2.429467in}}{\pgfqpoint{5.751891in}{2.425077in}}{\pgfqpoint{5.762941in}{2.425077in}}%
\pgfpathclose%
\pgfusepath{stroke,fill}%
\end{pgfscope}%
\begin{pgfscope}%
\pgfpathrectangle{\pgfqpoint{0.481978in}{0.331635in}}{\pgfqpoint{9.300000in}{7.700000in}}%
\pgfusepath{clip}%
\pgfsetbuttcap%
\pgfsetroundjoin%
\definecolor{currentfill}{rgb}{0.631373,0.788235,0.956863}%
\pgfsetfillcolor{currentfill}%
\pgfsetlinewidth{0.481800pt}%
\definecolor{currentstroke}{rgb}{1.000000,1.000000,1.000000}%
\pgfsetstrokecolor{currentstroke}%
\pgfsetdash{}{0pt}%
\pgfpathmoveto{\pgfqpoint{6.414379in}{4.456302in}}%
\pgfpathcurveto{\pgfqpoint{6.425429in}{4.456302in}}{\pgfqpoint{6.436028in}{4.460692in}}{\pgfqpoint{6.443842in}{4.468506in}}%
\pgfpathcurveto{\pgfqpoint{6.451656in}{4.476319in}}{\pgfqpoint{6.456046in}{4.486918in}}{\pgfqpoint{6.456046in}{4.497968in}}%
\pgfpathcurveto{\pgfqpoint{6.456046in}{4.509018in}}{\pgfqpoint{6.451656in}{4.519618in}}{\pgfqpoint{6.443842in}{4.527431in}}%
\pgfpathcurveto{\pgfqpoint{6.436028in}{4.535245in}}{\pgfqpoint{6.425429in}{4.539635in}}{\pgfqpoint{6.414379in}{4.539635in}}%
\pgfpathcurveto{\pgfqpoint{6.403329in}{4.539635in}}{\pgfqpoint{6.392730in}{4.535245in}}{\pgfqpoint{6.384916in}{4.527431in}}%
\pgfpathcurveto{\pgfqpoint{6.377103in}{4.519618in}}{\pgfqpoint{6.372713in}{4.509018in}}{\pgfqpoint{6.372713in}{4.497968in}}%
\pgfpathcurveto{\pgfqpoint{6.372713in}{4.486918in}}{\pgfqpoint{6.377103in}{4.476319in}}{\pgfqpoint{6.384916in}{4.468506in}}%
\pgfpathcurveto{\pgfqpoint{6.392730in}{4.460692in}}{\pgfqpoint{6.403329in}{4.456302in}}{\pgfqpoint{6.414379in}{4.456302in}}%
\pgfpathclose%
\pgfusepath{stroke,fill}%
\end{pgfscope}%
\begin{pgfscope}%
\pgfpathrectangle{\pgfqpoint{0.481978in}{0.331635in}}{\pgfqpoint{9.300000in}{7.700000in}}%
\pgfusepath{clip}%
\pgfsetbuttcap%
\pgfsetroundjoin%
\definecolor{currentfill}{rgb}{0.631373,0.788235,0.956863}%
\pgfsetfillcolor{currentfill}%
\pgfsetlinewidth{0.481800pt}%
\definecolor{currentstroke}{rgb}{1.000000,1.000000,1.000000}%
\pgfsetstrokecolor{currentstroke}%
\pgfsetdash{}{0pt}%
\pgfpathmoveto{\pgfqpoint{5.330843in}{3.109933in}}%
\pgfpathcurveto{\pgfqpoint{5.341893in}{3.109933in}}{\pgfqpoint{5.352492in}{3.114323in}}{\pgfqpoint{5.360306in}{3.122137in}}%
\pgfpathcurveto{\pgfqpoint{5.368120in}{3.129950in}}{\pgfqpoint{5.372510in}{3.140549in}}{\pgfqpoint{5.372510in}{3.151599in}}%
\pgfpathcurveto{\pgfqpoint{5.372510in}{3.162650in}}{\pgfqpoint{5.368120in}{3.173249in}}{\pgfqpoint{5.360306in}{3.181062in}}%
\pgfpathcurveto{\pgfqpoint{5.352492in}{3.188876in}}{\pgfqpoint{5.341893in}{3.193266in}}{\pgfqpoint{5.330843in}{3.193266in}}%
\pgfpathcurveto{\pgfqpoint{5.319793in}{3.193266in}}{\pgfqpoint{5.309194in}{3.188876in}}{\pgfqpoint{5.301380in}{3.181062in}}%
\pgfpathcurveto{\pgfqpoint{5.293567in}{3.173249in}}{\pgfqpoint{5.289177in}{3.162650in}}{\pgfqpoint{5.289177in}{3.151599in}}%
\pgfpathcurveto{\pgfqpoint{5.289177in}{3.140549in}}{\pgfqpoint{5.293567in}{3.129950in}}{\pgfqpoint{5.301380in}{3.122137in}}%
\pgfpathcurveto{\pgfqpoint{5.309194in}{3.114323in}}{\pgfqpoint{5.319793in}{3.109933in}}{\pgfqpoint{5.330843in}{3.109933in}}%
\pgfpathclose%
\pgfusepath{stroke,fill}%
\end{pgfscope}%
\begin{pgfscope}%
\pgfpathrectangle{\pgfqpoint{0.481978in}{0.331635in}}{\pgfqpoint{9.300000in}{7.700000in}}%
\pgfusepath{clip}%
\pgfsetbuttcap%
\pgfsetroundjoin%
\definecolor{currentfill}{rgb}{0.631373,0.788235,0.956863}%
\pgfsetfillcolor{currentfill}%
\pgfsetlinewidth{0.481800pt}%
\definecolor{currentstroke}{rgb}{1.000000,1.000000,1.000000}%
\pgfsetstrokecolor{currentstroke}%
\pgfsetdash{}{0pt}%
\pgfpathmoveto{\pgfqpoint{5.723444in}{1.091644in}}%
\pgfpathcurveto{\pgfqpoint{5.734494in}{1.091644in}}{\pgfqpoint{5.745093in}{1.096034in}}{\pgfqpoint{5.752907in}{1.103848in}}%
\pgfpathcurveto{\pgfqpoint{5.760720in}{1.111661in}}{\pgfqpoint{5.765110in}{1.122260in}}{\pgfqpoint{5.765110in}{1.133310in}}%
\pgfpathcurveto{\pgfqpoint{5.765110in}{1.144361in}}{\pgfqpoint{5.760720in}{1.154960in}}{\pgfqpoint{5.752907in}{1.162773in}}%
\pgfpathcurveto{\pgfqpoint{5.745093in}{1.170587in}}{\pgfqpoint{5.734494in}{1.174977in}}{\pgfqpoint{5.723444in}{1.174977in}}%
\pgfpathcurveto{\pgfqpoint{5.712394in}{1.174977in}}{\pgfqpoint{5.701795in}{1.170587in}}{\pgfqpoint{5.693981in}{1.162773in}}%
\pgfpathcurveto{\pgfqpoint{5.686167in}{1.154960in}}{\pgfqpoint{5.681777in}{1.144361in}}{\pgfqpoint{5.681777in}{1.133310in}}%
\pgfpathcurveto{\pgfqpoint{5.681777in}{1.122260in}}{\pgfqpoint{5.686167in}{1.111661in}}{\pgfqpoint{5.693981in}{1.103848in}}%
\pgfpathcurveto{\pgfqpoint{5.701795in}{1.096034in}}{\pgfqpoint{5.712394in}{1.091644in}}{\pgfqpoint{5.723444in}{1.091644in}}%
\pgfpathclose%
\pgfusepath{stroke,fill}%
\end{pgfscope}%
\begin{pgfscope}%
\pgfpathrectangle{\pgfqpoint{0.481978in}{0.331635in}}{\pgfqpoint{9.300000in}{7.700000in}}%
\pgfusepath{clip}%
\pgfsetbuttcap%
\pgfsetroundjoin%
\definecolor{currentfill}{rgb}{0.631373,0.788235,0.956863}%
\pgfsetfillcolor{currentfill}%
\pgfsetlinewidth{0.481800pt}%
\definecolor{currentstroke}{rgb}{1.000000,1.000000,1.000000}%
\pgfsetstrokecolor{currentstroke}%
\pgfsetdash{}{0pt}%
\pgfpathmoveto{\pgfqpoint{3.540419in}{3.324689in}}%
\pgfpathcurveto{\pgfqpoint{3.551469in}{3.324689in}}{\pgfqpoint{3.562068in}{3.329079in}}{\pgfqpoint{3.569881in}{3.336893in}}%
\pgfpathcurveto{\pgfqpoint{3.577695in}{3.344707in}}{\pgfqpoint{3.582085in}{3.355306in}}{\pgfqpoint{3.582085in}{3.366356in}}%
\pgfpathcurveto{\pgfqpoint{3.582085in}{3.377406in}}{\pgfqpoint{3.577695in}{3.388005in}}{\pgfqpoint{3.569881in}{3.395819in}}%
\pgfpathcurveto{\pgfqpoint{3.562068in}{3.403632in}}{\pgfqpoint{3.551469in}{3.408022in}}{\pgfqpoint{3.540419in}{3.408022in}}%
\pgfpathcurveto{\pgfqpoint{3.529368in}{3.408022in}}{\pgfqpoint{3.518769in}{3.403632in}}{\pgfqpoint{3.510956in}{3.395819in}}%
\pgfpathcurveto{\pgfqpoint{3.503142in}{3.388005in}}{\pgfqpoint{3.498752in}{3.377406in}}{\pgfqpoint{3.498752in}{3.366356in}}%
\pgfpathcurveto{\pgfqpoint{3.498752in}{3.355306in}}{\pgfqpoint{3.503142in}{3.344707in}}{\pgfqpoint{3.510956in}{3.336893in}}%
\pgfpathcurveto{\pgfqpoint{3.518769in}{3.329079in}}{\pgfqpoint{3.529368in}{3.324689in}}{\pgfqpoint{3.540419in}{3.324689in}}%
\pgfpathclose%
\pgfusepath{stroke,fill}%
\end{pgfscope}%
\begin{pgfscope}%
\pgfpathrectangle{\pgfqpoint{0.481978in}{0.331635in}}{\pgfqpoint{9.300000in}{7.700000in}}%
\pgfusepath{clip}%
\pgfsetbuttcap%
\pgfsetroundjoin%
\definecolor{currentfill}{rgb}{0.631373,0.788235,0.956863}%
\pgfsetfillcolor{currentfill}%
\pgfsetlinewidth{0.481800pt}%
\definecolor{currentstroke}{rgb}{1.000000,1.000000,1.000000}%
\pgfsetstrokecolor{currentstroke}%
\pgfsetdash{}{0pt}%
\pgfpathmoveto{\pgfqpoint{7.439774in}{5.499083in}}%
\pgfpathcurveto{\pgfqpoint{7.450825in}{5.499083in}}{\pgfqpoint{7.461424in}{5.503473in}}{\pgfqpoint{7.469237in}{5.511287in}}%
\pgfpathcurveto{\pgfqpoint{7.477051in}{5.519101in}}{\pgfqpoint{7.481441in}{5.529700in}}{\pgfqpoint{7.481441in}{5.540750in}}%
\pgfpathcurveto{\pgfqpoint{7.481441in}{5.551800in}}{\pgfqpoint{7.477051in}{5.562399in}}{\pgfqpoint{7.469237in}{5.570213in}}%
\pgfpathcurveto{\pgfqpoint{7.461424in}{5.578026in}}{\pgfqpoint{7.450825in}{5.582417in}}{\pgfqpoint{7.439774in}{5.582417in}}%
\pgfpathcurveto{\pgfqpoint{7.428724in}{5.582417in}}{\pgfqpoint{7.418125in}{5.578026in}}{\pgfqpoint{7.410312in}{5.570213in}}%
\pgfpathcurveto{\pgfqpoint{7.402498in}{5.562399in}}{\pgfqpoint{7.398108in}{5.551800in}}{\pgfqpoint{7.398108in}{5.540750in}}%
\pgfpathcurveto{\pgfqpoint{7.398108in}{5.529700in}}{\pgfqpoint{7.402498in}{5.519101in}}{\pgfqpoint{7.410312in}{5.511287in}}%
\pgfpathcurveto{\pgfqpoint{7.418125in}{5.503473in}}{\pgfqpoint{7.428724in}{5.499083in}}{\pgfqpoint{7.439774in}{5.499083in}}%
\pgfpathclose%
\pgfusepath{stroke,fill}%
\end{pgfscope}%
\begin{pgfscope}%
\pgfpathrectangle{\pgfqpoint{0.481978in}{0.331635in}}{\pgfqpoint{9.300000in}{7.700000in}}%
\pgfusepath{clip}%
\pgfsetbuttcap%
\pgfsetroundjoin%
\definecolor{currentfill}{rgb}{0.631373,0.788235,0.956863}%
\pgfsetfillcolor{currentfill}%
\pgfsetlinewidth{0.481800pt}%
\definecolor{currentstroke}{rgb}{1.000000,1.000000,1.000000}%
\pgfsetstrokecolor{currentstroke}%
\pgfsetdash{}{0pt}%
\pgfpathmoveto{\pgfqpoint{8.272345in}{2.122362in}}%
\pgfpathcurveto{\pgfqpoint{8.283395in}{2.122362in}}{\pgfqpoint{8.293994in}{2.126752in}}{\pgfqpoint{8.301808in}{2.134566in}}%
\pgfpathcurveto{\pgfqpoint{8.309621in}{2.142380in}}{\pgfqpoint{8.314011in}{2.152979in}}{\pgfqpoint{8.314011in}{2.164029in}}%
\pgfpathcurveto{\pgfqpoint{8.314011in}{2.175079in}}{\pgfqpoint{8.309621in}{2.185678in}}{\pgfqpoint{8.301808in}{2.193492in}}%
\pgfpathcurveto{\pgfqpoint{8.293994in}{2.201305in}}{\pgfqpoint{8.283395in}{2.205696in}}{\pgfqpoint{8.272345in}{2.205696in}}%
\pgfpathcurveto{\pgfqpoint{8.261295in}{2.205696in}}{\pgfqpoint{8.250696in}{2.201305in}}{\pgfqpoint{8.242882in}{2.193492in}}%
\pgfpathcurveto{\pgfqpoint{8.235068in}{2.185678in}}{\pgfqpoint{8.230678in}{2.175079in}}{\pgfqpoint{8.230678in}{2.164029in}}%
\pgfpathcurveto{\pgfqpoint{8.230678in}{2.152979in}}{\pgfqpoint{8.235068in}{2.142380in}}{\pgfqpoint{8.242882in}{2.134566in}}%
\pgfpathcurveto{\pgfqpoint{8.250696in}{2.126752in}}{\pgfqpoint{8.261295in}{2.122362in}}{\pgfqpoint{8.272345in}{2.122362in}}%
\pgfpathclose%
\pgfusepath{stroke,fill}%
\end{pgfscope}%
\begin{pgfscope}%
\pgfpathrectangle{\pgfqpoint{0.481978in}{0.331635in}}{\pgfqpoint{9.300000in}{7.700000in}}%
\pgfusepath{clip}%
\pgfsetbuttcap%
\pgfsetroundjoin%
\definecolor{currentfill}{rgb}{0.631373,0.788235,0.956863}%
\pgfsetfillcolor{currentfill}%
\pgfsetlinewidth{0.481800pt}%
\definecolor{currentstroke}{rgb}{1.000000,1.000000,1.000000}%
\pgfsetstrokecolor{currentstroke}%
\pgfsetdash{}{0pt}%
\pgfpathmoveto{\pgfqpoint{7.470382in}{1.755818in}}%
\pgfpathcurveto{\pgfqpoint{7.481433in}{1.755818in}}{\pgfqpoint{7.492032in}{1.760209in}}{\pgfqpoint{7.499845in}{1.768022in}}%
\pgfpathcurveto{\pgfqpoint{7.507659in}{1.775836in}}{\pgfqpoint{7.512049in}{1.786435in}}{\pgfqpoint{7.512049in}{1.797485in}}%
\pgfpathcurveto{\pgfqpoint{7.512049in}{1.808535in}}{\pgfqpoint{7.507659in}{1.819134in}}{\pgfqpoint{7.499845in}{1.826948in}}%
\pgfpathcurveto{\pgfqpoint{7.492032in}{1.834761in}}{\pgfqpoint{7.481433in}{1.839152in}}{\pgfqpoint{7.470382in}{1.839152in}}%
\pgfpathcurveto{\pgfqpoint{7.459332in}{1.839152in}}{\pgfqpoint{7.448733in}{1.834761in}}{\pgfqpoint{7.440920in}{1.826948in}}%
\pgfpathcurveto{\pgfqpoint{7.433106in}{1.819134in}}{\pgfqpoint{7.428716in}{1.808535in}}{\pgfqpoint{7.428716in}{1.797485in}}%
\pgfpathcurveto{\pgfqpoint{7.428716in}{1.786435in}}{\pgfqpoint{7.433106in}{1.775836in}}{\pgfqpoint{7.440920in}{1.768022in}}%
\pgfpathcurveto{\pgfqpoint{7.448733in}{1.760209in}}{\pgfqpoint{7.459332in}{1.755818in}}{\pgfqpoint{7.470382in}{1.755818in}}%
\pgfpathclose%
\pgfusepath{stroke,fill}%
\end{pgfscope}%
\begin{pgfscope}%
\pgfpathrectangle{\pgfqpoint{0.481978in}{0.331635in}}{\pgfqpoint{9.300000in}{7.700000in}}%
\pgfusepath{clip}%
\pgfsetbuttcap%
\pgfsetroundjoin%
\definecolor{currentfill}{rgb}{0.631373,0.788235,0.956863}%
\pgfsetfillcolor{currentfill}%
\pgfsetlinewidth{0.481800pt}%
\definecolor{currentstroke}{rgb}{1.000000,1.000000,1.000000}%
\pgfsetstrokecolor{currentstroke}%
\pgfsetdash{}{0pt}%
\pgfpathmoveto{\pgfqpoint{7.403075in}{6.729377in}}%
\pgfpathcurveto{\pgfqpoint{7.414125in}{6.729377in}}{\pgfqpoint{7.424724in}{6.733767in}}{\pgfqpoint{7.432538in}{6.741580in}}%
\pgfpathcurveto{\pgfqpoint{7.440351in}{6.749394in}}{\pgfqpoint{7.444741in}{6.759993in}}{\pgfqpoint{7.444741in}{6.771043in}}%
\pgfpathcurveto{\pgfqpoint{7.444741in}{6.782093in}}{\pgfqpoint{7.440351in}{6.792692in}}{\pgfqpoint{7.432538in}{6.800506in}}%
\pgfpathcurveto{\pgfqpoint{7.424724in}{6.808320in}}{\pgfqpoint{7.414125in}{6.812710in}}{\pgfqpoint{7.403075in}{6.812710in}}%
\pgfpathcurveto{\pgfqpoint{7.392025in}{6.812710in}}{\pgfqpoint{7.381426in}{6.808320in}}{\pgfqpoint{7.373612in}{6.800506in}}%
\pgfpathcurveto{\pgfqpoint{7.365798in}{6.792692in}}{\pgfqpoint{7.361408in}{6.782093in}}{\pgfqpoint{7.361408in}{6.771043in}}%
\pgfpathcurveto{\pgfqpoint{7.361408in}{6.759993in}}{\pgfqpoint{7.365798in}{6.749394in}}{\pgfqpoint{7.373612in}{6.741580in}}%
\pgfpathcurveto{\pgfqpoint{7.381426in}{6.733767in}}{\pgfqpoint{7.392025in}{6.729377in}}{\pgfqpoint{7.403075in}{6.729377in}}%
\pgfpathclose%
\pgfusepath{stroke,fill}%
\end{pgfscope}%
\begin{pgfscope}%
\pgfpathrectangle{\pgfqpoint{0.481978in}{0.331635in}}{\pgfqpoint{9.300000in}{7.700000in}}%
\pgfusepath{clip}%
\pgfsetbuttcap%
\pgfsetroundjoin%
\definecolor{currentfill}{rgb}{1.000000,0.705882,0.509804}%
\pgfsetfillcolor{currentfill}%
\pgfsetlinewidth{0.481800pt}%
\definecolor{currentstroke}{rgb}{1.000000,1.000000,1.000000}%
\pgfsetstrokecolor{currentstroke}%
\pgfsetdash{}{0pt}%
\pgfpathmoveto{\pgfqpoint{4.737094in}{4.504340in}}%
\pgfpathcurveto{\pgfqpoint{4.748144in}{4.504340in}}{\pgfqpoint{4.758743in}{4.508730in}}{\pgfqpoint{4.766557in}{4.516544in}}%
\pgfpathcurveto{\pgfqpoint{4.774370in}{4.524357in}}{\pgfqpoint{4.778761in}{4.534956in}}{\pgfqpoint{4.778761in}{4.546006in}}%
\pgfpathcurveto{\pgfqpoint{4.778761in}{4.557057in}}{\pgfqpoint{4.774370in}{4.567656in}}{\pgfqpoint{4.766557in}{4.575469in}}%
\pgfpathcurveto{\pgfqpoint{4.758743in}{4.583283in}}{\pgfqpoint{4.748144in}{4.587673in}}{\pgfqpoint{4.737094in}{4.587673in}}%
\pgfpathcurveto{\pgfqpoint{4.726044in}{4.587673in}}{\pgfqpoint{4.715445in}{4.583283in}}{\pgfqpoint{4.707631in}{4.575469in}}%
\pgfpathcurveto{\pgfqpoint{4.699817in}{4.567656in}}{\pgfqpoint{4.695427in}{4.557057in}}{\pgfqpoint{4.695427in}{4.546006in}}%
\pgfpathcurveto{\pgfqpoint{4.695427in}{4.534956in}}{\pgfqpoint{4.699817in}{4.524357in}}{\pgfqpoint{4.707631in}{4.516544in}}%
\pgfpathcurveto{\pgfqpoint{4.715445in}{4.508730in}}{\pgfqpoint{4.726044in}{4.504340in}}{\pgfqpoint{4.737094in}{4.504340in}}%
\pgfpathclose%
\pgfusepath{stroke,fill}%
\end{pgfscope}%
\begin{pgfscope}%
\pgfpathrectangle{\pgfqpoint{0.481978in}{0.331635in}}{\pgfqpoint{9.300000in}{7.700000in}}%
\pgfusepath{clip}%
\pgfsetbuttcap%
\pgfsetroundjoin%
\definecolor{currentfill}{rgb}{1.000000,0.705882,0.509804}%
\pgfsetfillcolor{currentfill}%
\pgfsetlinewidth{0.481800pt}%
\definecolor{currentstroke}{rgb}{1.000000,1.000000,1.000000}%
\pgfsetstrokecolor{currentstroke}%
\pgfsetdash{}{0pt}%
\pgfpathmoveto{\pgfqpoint{4.102620in}{7.153439in}}%
\pgfpathcurveto{\pgfqpoint{4.113670in}{7.153439in}}{\pgfqpoint{4.124269in}{7.157829in}}{\pgfqpoint{4.132083in}{7.165642in}}%
\pgfpathcurveto{\pgfqpoint{4.139896in}{7.173456in}}{\pgfqpoint{4.144287in}{7.184055in}}{\pgfqpoint{4.144287in}{7.195105in}}%
\pgfpathcurveto{\pgfqpoint{4.144287in}{7.206155in}}{\pgfqpoint{4.139896in}{7.216754in}}{\pgfqpoint{4.132083in}{7.224568in}}%
\pgfpathcurveto{\pgfqpoint{4.124269in}{7.232382in}}{\pgfqpoint{4.113670in}{7.236772in}}{\pgfqpoint{4.102620in}{7.236772in}}%
\pgfpathcurveto{\pgfqpoint{4.091570in}{7.236772in}}{\pgfqpoint{4.080971in}{7.232382in}}{\pgfqpoint{4.073157in}{7.224568in}}%
\pgfpathcurveto{\pgfqpoint{4.065344in}{7.216754in}}{\pgfqpoint{4.060953in}{7.206155in}}{\pgfqpoint{4.060953in}{7.195105in}}%
\pgfpathcurveto{\pgfqpoint{4.060953in}{7.184055in}}{\pgfqpoint{4.065344in}{7.173456in}}{\pgfqpoint{4.073157in}{7.165642in}}%
\pgfpathcurveto{\pgfqpoint{4.080971in}{7.157829in}}{\pgfqpoint{4.091570in}{7.153439in}}{\pgfqpoint{4.102620in}{7.153439in}}%
\pgfpathclose%
\pgfusepath{stroke,fill}%
\end{pgfscope}%
\begin{pgfscope}%
\pgfpathrectangle{\pgfqpoint{0.481978in}{0.331635in}}{\pgfqpoint{9.300000in}{7.700000in}}%
\pgfusepath{clip}%
\pgfsetbuttcap%
\pgfsetroundjoin%
\definecolor{currentfill}{rgb}{1.000000,0.705882,0.509804}%
\pgfsetfillcolor{currentfill}%
\pgfsetlinewidth{0.481800pt}%
\definecolor{currentstroke}{rgb}{1.000000,1.000000,1.000000}%
\pgfsetstrokecolor{currentstroke}%
\pgfsetdash{}{0pt}%
\pgfpathmoveto{\pgfqpoint{3.960956in}{4.752480in}}%
\pgfpathcurveto{\pgfqpoint{3.972006in}{4.752480in}}{\pgfqpoint{3.982605in}{4.756870in}}{\pgfqpoint{3.990418in}{4.764684in}}%
\pgfpathcurveto{\pgfqpoint{3.998232in}{4.772497in}}{\pgfqpoint{4.002622in}{4.783097in}}{\pgfqpoint{4.002622in}{4.794147in}}%
\pgfpathcurveto{\pgfqpoint{4.002622in}{4.805197in}}{\pgfqpoint{3.998232in}{4.815796in}}{\pgfqpoint{3.990418in}{4.823609in}}%
\pgfpathcurveto{\pgfqpoint{3.982605in}{4.831423in}}{\pgfqpoint{3.972006in}{4.835813in}}{\pgfqpoint{3.960956in}{4.835813in}}%
\pgfpathcurveto{\pgfqpoint{3.949905in}{4.835813in}}{\pgfqpoint{3.939306in}{4.831423in}}{\pgfqpoint{3.931493in}{4.823609in}}%
\pgfpathcurveto{\pgfqpoint{3.923679in}{4.815796in}}{\pgfqpoint{3.919289in}{4.805197in}}{\pgfqpoint{3.919289in}{4.794147in}}%
\pgfpathcurveto{\pgfqpoint{3.919289in}{4.783097in}}{\pgfqpoint{3.923679in}{4.772497in}}{\pgfqpoint{3.931493in}{4.764684in}}%
\pgfpathcurveto{\pgfqpoint{3.939306in}{4.756870in}}{\pgfqpoint{3.949905in}{4.752480in}}{\pgfqpoint{3.960956in}{4.752480in}}%
\pgfpathclose%
\pgfusepath{stroke,fill}%
\end{pgfscope}%
\begin{pgfscope}%
\pgfpathrectangle{\pgfqpoint{0.481978in}{0.331635in}}{\pgfqpoint{9.300000in}{7.700000in}}%
\pgfusepath{clip}%
\pgfsetbuttcap%
\pgfsetroundjoin%
\definecolor{currentfill}{rgb}{1.000000,0.705882,0.509804}%
\pgfsetfillcolor{currentfill}%
\pgfsetlinewidth{0.481800pt}%
\definecolor{currentstroke}{rgb}{1.000000,1.000000,1.000000}%
\pgfsetstrokecolor{currentstroke}%
\pgfsetdash{}{0pt}%
\pgfpathmoveto{\pgfqpoint{3.386233in}{5.143212in}}%
\pgfpathcurveto{\pgfqpoint{3.397283in}{5.143212in}}{\pgfqpoint{3.407882in}{5.147603in}}{\pgfqpoint{3.415695in}{5.155416in}}%
\pgfpathcurveto{\pgfqpoint{3.423509in}{5.163230in}}{\pgfqpoint{3.427899in}{5.173829in}}{\pgfqpoint{3.427899in}{5.184879in}}%
\pgfpathcurveto{\pgfqpoint{3.427899in}{5.195929in}}{\pgfqpoint{3.423509in}{5.206528in}}{\pgfqpoint{3.415695in}{5.214342in}}%
\pgfpathcurveto{\pgfqpoint{3.407882in}{5.222155in}}{\pgfqpoint{3.397283in}{5.226546in}}{\pgfqpoint{3.386233in}{5.226546in}}%
\pgfpathcurveto{\pgfqpoint{3.375183in}{5.226546in}}{\pgfqpoint{3.364584in}{5.222155in}}{\pgfqpoint{3.356770in}{5.214342in}}%
\pgfpathcurveto{\pgfqpoint{3.348956in}{5.206528in}}{\pgfqpoint{3.344566in}{5.195929in}}{\pgfqpoint{3.344566in}{5.184879in}}%
\pgfpathcurveto{\pgfqpoint{3.344566in}{5.173829in}}{\pgfqpoint{3.348956in}{5.163230in}}{\pgfqpoint{3.356770in}{5.155416in}}%
\pgfpathcurveto{\pgfqpoint{3.364584in}{5.147603in}}{\pgfqpoint{3.375183in}{5.143212in}}{\pgfqpoint{3.386233in}{5.143212in}}%
\pgfpathclose%
\pgfusepath{stroke,fill}%
\end{pgfscope}%
\begin{pgfscope}%
\pgfpathrectangle{\pgfqpoint{0.481978in}{0.331635in}}{\pgfqpoint{9.300000in}{7.700000in}}%
\pgfusepath{clip}%
\pgfsetbuttcap%
\pgfsetroundjoin%
\definecolor{currentfill}{rgb}{1.000000,0.705882,0.509804}%
\pgfsetfillcolor{currentfill}%
\pgfsetlinewidth{0.481800pt}%
\definecolor{currentstroke}{rgb}{1.000000,1.000000,1.000000}%
\pgfsetstrokecolor{currentstroke}%
\pgfsetdash{}{0pt}%
\pgfpathmoveto{\pgfqpoint{2.506952in}{3.859071in}}%
\pgfpathcurveto{\pgfqpoint{2.518002in}{3.859071in}}{\pgfqpoint{2.528601in}{3.863462in}}{\pgfqpoint{2.536415in}{3.871275in}}%
\pgfpathcurveto{\pgfqpoint{2.544228in}{3.879089in}}{\pgfqpoint{2.548618in}{3.889688in}}{\pgfqpoint{2.548618in}{3.900738in}}%
\pgfpathcurveto{\pgfqpoint{2.548618in}{3.911788in}}{\pgfqpoint{2.544228in}{3.922387in}}{\pgfqpoint{2.536415in}{3.930201in}}%
\pgfpathcurveto{\pgfqpoint{2.528601in}{3.938015in}}{\pgfqpoint{2.518002in}{3.942405in}}{\pgfqpoint{2.506952in}{3.942405in}}%
\pgfpathcurveto{\pgfqpoint{2.495902in}{3.942405in}}{\pgfqpoint{2.485303in}{3.938015in}}{\pgfqpoint{2.477489in}{3.930201in}}%
\pgfpathcurveto{\pgfqpoint{2.469675in}{3.922387in}}{\pgfqpoint{2.465285in}{3.911788in}}{\pgfqpoint{2.465285in}{3.900738in}}%
\pgfpathcurveto{\pgfqpoint{2.465285in}{3.889688in}}{\pgfqpoint{2.469675in}{3.879089in}}{\pgfqpoint{2.477489in}{3.871275in}}%
\pgfpathcurveto{\pgfqpoint{2.485303in}{3.863462in}}{\pgfqpoint{2.495902in}{3.859071in}}{\pgfqpoint{2.506952in}{3.859071in}}%
\pgfpathclose%
\pgfusepath{stroke,fill}%
\end{pgfscope}%
\begin{pgfscope}%
\pgfpathrectangle{\pgfqpoint{0.481978in}{0.331635in}}{\pgfqpoint{9.300000in}{7.700000in}}%
\pgfusepath{clip}%
\pgfsetbuttcap%
\pgfsetroundjoin%
\definecolor{currentfill}{rgb}{1.000000,0.705882,0.509804}%
\pgfsetfillcolor{currentfill}%
\pgfsetlinewidth{0.481800pt}%
\definecolor{currentstroke}{rgb}{1.000000,1.000000,1.000000}%
\pgfsetstrokecolor{currentstroke}%
\pgfsetdash{}{0pt}%
\pgfpathmoveto{\pgfqpoint{7.641446in}{3.984517in}}%
\pgfpathcurveto{\pgfqpoint{7.652496in}{3.984517in}}{\pgfqpoint{7.663095in}{3.988907in}}{\pgfqpoint{7.670909in}{3.996720in}}%
\pgfpathcurveto{\pgfqpoint{7.678722in}{4.004534in}}{\pgfqpoint{7.683113in}{4.015133in}}{\pgfqpoint{7.683113in}{4.026183in}}%
\pgfpathcurveto{\pgfqpoint{7.683113in}{4.037233in}}{\pgfqpoint{7.678722in}{4.047832in}}{\pgfqpoint{7.670909in}{4.055646in}}%
\pgfpathcurveto{\pgfqpoint{7.663095in}{4.063460in}}{\pgfqpoint{7.652496in}{4.067850in}}{\pgfqpoint{7.641446in}{4.067850in}}%
\pgfpathcurveto{\pgfqpoint{7.630396in}{4.067850in}}{\pgfqpoint{7.619797in}{4.063460in}}{\pgfqpoint{7.611983in}{4.055646in}}%
\pgfpathcurveto{\pgfqpoint{7.604170in}{4.047832in}}{\pgfqpoint{7.599779in}{4.037233in}}{\pgfqpoint{7.599779in}{4.026183in}}%
\pgfpathcurveto{\pgfqpoint{7.599779in}{4.015133in}}{\pgfqpoint{7.604170in}{4.004534in}}{\pgfqpoint{7.611983in}{3.996720in}}%
\pgfpathcurveto{\pgfqpoint{7.619797in}{3.988907in}}{\pgfqpoint{7.630396in}{3.984517in}}{\pgfqpoint{7.641446in}{3.984517in}}%
\pgfpathclose%
\pgfusepath{stroke,fill}%
\end{pgfscope}%
\begin{pgfscope}%
\pgfpathrectangle{\pgfqpoint{0.481978in}{0.331635in}}{\pgfqpoint{9.300000in}{7.700000in}}%
\pgfusepath{clip}%
\pgfsetbuttcap%
\pgfsetroundjoin%
\definecolor{currentfill}{rgb}{1.000000,0.705882,0.509804}%
\pgfsetfillcolor{currentfill}%
\pgfsetlinewidth{0.481800pt}%
\definecolor{currentstroke}{rgb}{1.000000,1.000000,1.000000}%
\pgfsetstrokecolor{currentstroke}%
\pgfsetdash{}{0pt}%
\pgfpathmoveto{\pgfqpoint{2.626934in}{4.727491in}}%
\pgfpathcurveto{\pgfqpoint{2.637984in}{4.727491in}}{\pgfqpoint{2.648583in}{4.731882in}}{\pgfqpoint{2.656397in}{4.739695in}}%
\pgfpathcurveto{\pgfqpoint{2.664210in}{4.747509in}}{\pgfqpoint{2.668601in}{4.758108in}}{\pgfqpoint{2.668601in}{4.769158in}}%
\pgfpathcurveto{\pgfqpoint{2.668601in}{4.780208in}}{\pgfqpoint{2.664210in}{4.790807in}}{\pgfqpoint{2.656397in}{4.798621in}}%
\pgfpathcurveto{\pgfqpoint{2.648583in}{4.806434in}}{\pgfqpoint{2.637984in}{4.810825in}}{\pgfqpoint{2.626934in}{4.810825in}}%
\pgfpathcurveto{\pgfqpoint{2.615884in}{4.810825in}}{\pgfqpoint{2.605285in}{4.806434in}}{\pgfqpoint{2.597471in}{4.798621in}}%
\pgfpathcurveto{\pgfqpoint{2.589658in}{4.790807in}}{\pgfqpoint{2.585267in}{4.780208in}}{\pgfqpoint{2.585267in}{4.769158in}}%
\pgfpathcurveto{\pgfqpoint{2.585267in}{4.758108in}}{\pgfqpoint{2.589658in}{4.747509in}}{\pgfqpoint{2.597471in}{4.739695in}}%
\pgfpathcurveto{\pgfqpoint{2.605285in}{4.731882in}}{\pgfqpoint{2.615884in}{4.727491in}}{\pgfqpoint{2.626934in}{4.727491in}}%
\pgfpathclose%
\pgfusepath{stroke,fill}%
\end{pgfscope}%
\begin{pgfscope}%
\pgfpathrectangle{\pgfqpoint{0.481978in}{0.331635in}}{\pgfqpoint{9.300000in}{7.700000in}}%
\pgfusepath{clip}%
\pgfsetbuttcap%
\pgfsetroundjoin%
\definecolor{currentfill}{rgb}{1.000000,0.705882,0.509804}%
\pgfsetfillcolor{currentfill}%
\pgfsetlinewidth{0.481800pt}%
\definecolor{currentstroke}{rgb}{1.000000,1.000000,1.000000}%
\pgfsetstrokecolor{currentstroke}%
\pgfsetdash{}{0pt}%
\pgfpathmoveto{\pgfqpoint{4.292588in}{0.962886in}}%
\pgfpathcurveto{\pgfqpoint{4.303638in}{0.962886in}}{\pgfqpoint{4.314237in}{0.967276in}}{\pgfqpoint{4.322051in}{0.975090in}}%
\pgfpathcurveto{\pgfqpoint{4.329865in}{0.982903in}}{\pgfqpoint{4.334255in}{0.993502in}}{\pgfqpoint{4.334255in}{1.004552in}}%
\pgfpathcurveto{\pgfqpoint{4.334255in}{1.015602in}}{\pgfqpoint{4.329865in}{1.026202in}}{\pgfqpoint{4.322051in}{1.034015in}}%
\pgfpathcurveto{\pgfqpoint{4.314237in}{1.041829in}}{\pgfqpoint{4.303638in}{1.046219in}}{\pgfqpoint{4.292588in}{1.046219in}}%
\pgfpathcurveto{\pgfqpoint{4.281538in}{1.046219in}}{\pgfqpoint{4.270939in}{1.041829in}}{\pgfqpoint{4.263125in}{1.034015in}}%
\pgfpathcurveto{\pgfqpoint{4.255312in}{1.026202in}}{\pgfqpoint{4.250922in}{1.015602in}}{\pgfqpoint{4.250922in}{1.004552in}}%
\pgfpathcurveto{\pgfqpoint{4.250922in}{0.993502in}}{\pgfqpoint{4.255312in}{0.982903in}}{\pgfqpoint{4.263125in}{0.975090in}}%
\pgfpathcurveto{\pgfqpoint{4.270939in}{0.967276in}}{\pgfqpoint{4.281538in}{0.962886in}}{\pgfqpoint{4.292588in}{0.962886in}}%
\pgfpathclose%
\pgfusepath{stroke,fill}%
\end{pgfscope}%
\begin{pgfscope}%
\pgfpathrectangle{\pgfqpoint{0.481978in}{0.331635in}}{\pgfqpoint{9.300000in}{7.700000in}}%
\pgfusepath{clip}%
\pgfsetbuttcap%
\pgfsetroundjoin%
\definecolor{currentfill}{rgb}{1.000000,0.705882,0.509804}%
\pgfsetfillcolor{currentfill}%
\pgfsetlinewidth{0.481800pt}%
\definecolor{currentstroke}{rgb}{1.000000,1.000000,1.000000}%
\pgfsetstrokecolor{currentstroke}%
\pgfsetdash{}{0pt}%
\pgfpathmoveto{\pgfqpoint{4.914941in}{2.209440in}}%
\pgfpathcurveto{\pgfqpoint{4.925991in}{2.209440in}}{\pgfqpoint{4.936590in}{2.213830in}}{\pgfqpoint{4.944403in}{2.221643in}}%
\pgfpathcurveto{\pgfqpoint{4.952217in}{2.229457in}}{\pgfqpoint{4.956607in}{2.240056in}}{\pgfqpoint{4.956607in}{2.251106in}}%
\pgfpathcurveto{\pgfqpoint{4.956607in}{2.262156in}}{\pgfqpoint{4.952217in}{2.272755in}}{\pgfqpoint{4.944403in}{2.280569in}}%
\pgfpathcurveto{\pgfqpoint{4.936590in}{2.288383in}}{\pgfqpoint{4.925991in}{2.292773in}}{\pgfqpoint{4.914941in}{2.292773in}}%
\pgfpathcurveto{\pgfqpoint{4.903891in}{2.292773in}}{\pgfqpoint{4.893292in}{2.288383in}}{\pgfqpoint{4.885478in}{2.280569in}}%
\pgfpathcurveto{\pgfqpoint{4.877664in}{2.272755in}}{\pgfqpoint{4.873274in}{2.262156in}}{\pgfqpoint{4.873274in}{2.251106in}}%
\pgfpathcurveto{\pgfqpoint{4.873274in}{2.240056in}}{\pgfqpoint{4.877664in}{2.229457in}}{\pgfqpoint{4.885478in}{2.221643in}}%
\pgfpathcurveto{\pgfqpoint{4.893292in}{2.213830in}}{\pgfqpoint{4.903891in}{2.209440in}}{\pgfqpoint{4.914941in}{2.209440in}}%
\pgfpathclose%
\pgfusepath{stroke,fill}%
\end{pgfscope}%
\begin{pgfscope}%
\pgfpathrectangle{\pgfqpoint{0.481978in}{0.331635in}}{\pgfqpoint{9.300000in}{7.700000in}}%
\pgfusepath{clip}%
\pgfsetbuttcap%
\pgfsetroundjoin%
\definecolor{currentfill}{rgb}{1.000000,0.705882,0.509804}%
\pgfsetfillcolor{currentfill}%
\pgfsetlinewidth{0.481800pt}%
\definecolor{currentstroke}{rgb}{1.000000,1.000000,1.000000}%
\pgfsetstrokecolor{currentstroke}%
\pgfsetdash{}{0pt}%
\pgfpathmoveto{\pgfqpoint{1.360419in}{3.188475in}}%
\pgfpathcurveto{\pgfqpoint{1.371469in}{3.188475in}}{\pgfqpoint{1.382068in}{3.192866in}}{\pgfqpoint{1.389882in}{3.200679in}}%
\pgfpathcurveto{\pgfqpoint{1.397695in}{3.208493in}}{\pgfqpoint{1.402086in}{3.219092in}}{\pgfqpoint{1.402086in}{3.230142in}}%
\pgfpathcurveto{\pgfqpoint{1.402086in}{3.241192in}}{\pgfqpoint{1.397695in}{3.251791in}}{\pgfqpoint{1.389882in}{3.259605in}}%
\pgfpathcurveto{\pgfqpoint{1.382068in}{3.267418in}}{\pgfqpoint{1.371469in}{3.271809in}}{\pgfqpoint{1.360419in}{3.271809in}}%
\pgfpathcurveto{\pgfqpoint{1.349369in}{3.271809in}}{\pgfqpoint{1.338770in}{3.267418in}}{\pgfqpoint{1.330956in}{3.259605in}}%
\pgfpathcurveto{\pgfqpoint{1.323143in}{3.251791in}}{\pgfqpoint{1.318752in}{3.241192in}}{\pgfqpoint{1.318752in}{3.230142in}}%
\pgfpathcurveto{\pgfqpoint{1.318752in}{3.219092in}}{\pgfqpoint{1.323143in}{3.208493in}}{\pgfqpoint{1.330956in}{3.200679in}}%
\pgfpathcurveto{\pgfqpoint{1.338770in}{3.192866in}}{\pgfqpoint{1.349369in}{3.188475in}}{\pgfqpoint{1.360419in}{3.188475in}}%
\pgfpathclose%
\pgfusepath{stroke,fill}%
\end{pgfscope}%
\begin{pgfscope}%
\pgfpathrectangle{\pgfqpoint{0.481978in}{0.331635in}}{\pgfqpoint{9.300000in}{7.700000in}}%
\pgfusepath{clip}%
\pgfsetbuttcap%
\pgfsetroundjoin%
\definecolor{currentfill}{rgb}{1.000000,0.705882,0.509804}%
\pgfsetfillcolor{currentfill}%
\pgfsetlinewidth{0.481800pt}%
\definecolor{currentstroke}{rgb}{1.000000,1.000000,1.000000}%
\pgfsetstrokecolor{currentstroke}%
\pgfsetdash{}{0pt}%
\pgfpathmoveto{\pgfqpoint{2.680958in}{5.597608in}}%
\pgfpathcurveto{\pgfqpoint{2.692008in}{5.597608in}}{\pgfqpoint{2.702607in}{5.601998in}}{\pgfqpoint{2.710420in}{5.609812in}}%
\pgfpathcurveto{\pgfqpoint{2.718234in}{5.617625in}}{\pgfqpoint{2.722624in}{5.628224in}}{\pgfqpoint{2.722624in}{5.639274in}}%
\pgfpathcurveto{\pgfqpoint{2.722624in}{5.650325in}}{\pgfqpoint{2.718234in}{5.660924in}}{\pgfqpoint{2.710420in}{5.668737in}}%
\pgfpathcurveto{\pgfqpoint{2.702607in}{5.676551in}}{\pgfqpoint{2.692008in}{5.680941in}}{\pgfqpoint{2.680958in}{5.680941in}}%
\pgfpathcurveto{\pgfqpoint{2.669908in}{5.680941in}}{\pgfqpoint{2.659309in}{5.676551in}}{\pgfqpoint{2.651495in}{5.668737in}}%
\pgfpathcurveto{\pgfqpoint{2.643681in}{5.660924in}}{\pgfqpoint{2.639291in}{5.650325in}}{\pgfqpoint{2.639291in}{5.639274in}}%
\pgfpathcurveto{\pgfqpoint{2.639291in}{5.628224in}}{\pgfqpoint{2.643681in}{5.617625in}}{\pgfqpoint{2.651495in}{5.609812in}}%
\pgfpathcurveto{\pgfqpoint{2.659309in}{5.601998in}}{\pgfqpoint{2.669908in}{5.597608in}}{\pgfqpoint{2.680958in}{5.597608in}}%
\pgfpathclose%
\pgfusepath{stroke,fill}%
\end{pgfscope}%
\begin{pgfscope}%
\pgfpathrectangle{\pgfqpoint{0.481978in}{0.331635in}}{\pgfqpoint{9.300000in}{7.700000in}}%
\pgfusepath{clip}%
\pgfsetbuttcap%
\pgfsetroundjoin%
\definecolor{currentfill}{rgb}{1.000000,0.705882,0.509804}%
\pgfsetfillcolor{currentfill}%
\pgfsetlinewidth{0.481800pt}%
\definecolor{currentstroke}{rgb}{1.000000,1.000000,1.000000}%
\pgfsetstrokecolor{currentstroke}%
\pgfsetdash{}{0pt}%
\pgfpathmoveto{\pgfqpoint{8.439887in}{6.135148in}}%
\pgfpathcurveto{\pgfqpoint{8.450937in}{6.135148in}}{\pgfqpoint{8.461536in}{6.139538in}}{\pgfqpoint{8.469350in}{6.147352in}}%
\pgfpathcurveto{\pgfqpoint{8.477163in}{6.155166in}}{\pgfqpoint{8.481553in}{6.165765in}}{\pgfqpoint{8.481553in}{6.176815in}}%
\pgfpathcurveto{\pgfqpoint{8.481553in}{6.187865in}}{\pgfqpoint{8.477163in}{6.198464in}}{\pgfqpoint{8.469350in}{6.206278in}}%
\pgfpathcurveto{\pgfqpoint{8.461536in}{6.214091in}}{\pgfqpoint{8.450937in}{6.218481in}}{\pgfqpoint{8.439887in}{6.218481in}}%
\pgfpathcurveto{\pgfqpoint{8.428837in}{6.218481in}}{\pgfqpoint{8.418238in}{6.214091in}}{\pgfqpoint{8.410424in}{6.206278in}}%
\pgfpathcurveto{\pgfqpoint{8.402610in}{6.198464in}}{\pgfqpoint{8.398220in}{6.187865in}}{\pgfqpoint{8.398220in}{6.176815in}}%
\pgfpathcurveto{\pgfqpoint{8.398220in}{6.165765in}}{\pgfqpoint{8.402610in}{6.155166in}}{\pgfqpoint{8.410424in}{6.147352in}}%
\pgfpathcurveto{\pgfqpoint{8.418238in}{6.139538in}}{\pgfqpoint{8.428837in}{6.135148in}}{\pgfqpoint{8.439887in}{6.135148in}}%
\pgfpathclose%
\pgfusepath{stroke,fill}%
\end{pgfscope}%
\begin{pgfscope}%
\pgfpathrectangle{\pgfqpoint{0.481978in}{0.331635in}}{\pgfqpoint{9.300000in}{7.700000in}}%
\pgfusepath{clip}%
\pgfsetbuttcap%
\pgfsetroundjoin%
\definecolor{currentfill}{rgb}{1.000000,0.705882,0.509804}%
\pgfsetfillcolor{currentfill}%
\pgfsetlinewidth{0.481800pt}%
\definecolor{currentstroke}{rgb}{1.000000,1.000000,1.000000}%
\pgfsetstrokecolor{currentstroke}%
\pgfsetdash{}{0pt}%
\pgfpathmoveto{\pgfqpoint{5.343023in}{7.639968in}}%
\pgfpathcurveto{\pgfqpoint{5.354073in}{7.639968in}}{\pgfqpoint{5.364672in}{7.644359in}}{\pgfqpoint{5.372486in}{7.652172in}}%
\pgfpathcurveto{\pgfqpoint{5.380299in}{7.659986in}}{\pgfqpoint{5.384690in}{7.670585in}}{\pgfqpoint{5.384690in}{7.681635in}}%
\pgfpathcurveto{\pgfqpoint{5.384690in}{7.692685in}}{\pgfqpoint{5.380299in}{7.703284in}}{\pgfqpoint{5.372486in}{7.711098in}}%
\pgfpathcurveto{\pgfqpoint{5.364672in}{7.718911in}}{\pgfqpoint{5.354073in}{7.723302in}}{\pgfqpoint{5.343023in}{7.723302in}}%
\pgfpathcurveto{\pgfqpoint{5.331973in}{7.723302in}}{\pgfqpoint{5.321374in}{7.718911in}}{\pgfqpoint{5.313560in}{7.711098in}}%
\pgfpathcurveto{\pgfqpoint{5.305747in}{7.703284in}}{\pgfqpoint{5.301356in}{7.692685in}}{\pgfqpoint{5.301356in}{7.681635in}}%
\pgfpathcurveto{\pgfqpoint{5.301356in}{7.670585in}}{\pgfqpoint{5.305747in}{7.659986in}}{\pgfqpoint{5.313560in}{7.652172in}}%
\pgfpathcurveto{\pgfqpoint{5.321374in}{7.644359in}}{\pgfqpoint{5.331973in}{7.639968in}}{\pgfqpoint{5.343023in}{7.639968in}}%
\pgfpathclose%
\pgfusepath{stroke,fill}%
\end{pgfscope}%
\begin{pgfscope}%
\pgfpathrectangle{\pgfqpoint{0.481978in}{0.331635in}}{\pgfqpoint{9.300000in}{7.700000in}}%
\pgfusepath{clip}%
\pgfsetbuttcap%
\pgfsetroundjoin%
\definecolor{currentfill}{rgb}{1.000000,0.705882,0.509804}%
\pgfsetfillcolor{currentfill}%
\pgfsetlinewidth{0.481800pt}%
\definecolor{currentstroke}{rgb}{1.000000,1.000000,1.000000}%
\pgfsetstrokecolor{currentstroke}%
\pgfsetdash{}{0pt}%
\pgfpathmoveto{\pgfqpoint{1.984154in}{4.127682in}}%
\pgfpathcurveto{\pgfqpoint{1.995204in}{4.127682in}}{\pgfqpoint{2.005803in}{4.132072in}}{\pgfqpoint{2.013617in}{4.139886in}}%
\pgfpathcurveto{\pgfqpoint{2.021431in}{4.147700in}}{\pgfqpoint{2.025821in}{4.158299in}}{\pgfqpoint{2.025821in}{4.169349in}}%
\pgfpathcurveto{\pgfqpoint{2.025821in}{4.180399in}}{\pgfqpoint{2.021431in}{4.190998in}}{\pgfqpoint{2.013617in}{4.198812in}}%
\pgfpathcurveto{\pgfqpoint{2.005803in}{4.206625in}}{\pgfqpoint{1.995204in}{4.211015in}}{\pgfqpoint{1.984154in}{4.211015in}}%
\pgfpathcurveto{\pgfqpoint{1.973104in}{4.211015in}}{\pgfqpoint{1.962505in}{4.206625in}}{\pgfqpoint{1.954691in}{4.198812in}}%
\pgfpathcurveto{\pgfqpoint{1.946878in}{4.190998in}}{\pgfqpoint{1.942487in}{4.180399in}}{\pgfqpoint{1.942487in}{4.169349in}}%
\pgfpathcurveto{\pgfqpoint{1.942487in}{4.158299in}}{\pgfqpoint{1.946878in}{4.147700in}}{\pgfqpoint{1.954691in}{4.139886in}}%
\pgfpathcurveto{\pgfqpoint{1.962505in}{4.132072in}}{\pgfqpoint{1.973104in}{4.127682in}}{\pgfqpoint{1.984154in}{4.127682in}}%
\pgfpathclose%
\pgfusepath{stroke,fill}%
\end{pgfscope}%
\begin{pgfscope}%
\pgfpathrectangle{\pgfqpoint{0.481978in}{0.331635in}}{\pgfqpoint{9.300000in}{7.700000in}}%
\pgfusepath{clip}%
\pgfsetbuttcap%
\pgfsetroundjoin%
\definecolor{currentfill}{rgb}{1.000000,0.705882,0.509804}%
\pgfsetfillcolor{currentfill}%
\pgfsetlinewidth{0.481800pt}%
\definecolor{currentstroke}{rgb}{1.000000,1.000000,1.000000}%
\pgfsetstrokecolor{currentstroke}%
\pgfsetdash{}{0pt}%
\pgfpathmoveto{\pgfqpoint{8.035769in}{4.906239in}}%
\pgfpathcurveto{\pgfqpoint{8.046819in}{4.906239in}}{\pgfqpoint{8.057418in}{4.910629in}}{\pgfqpoint{8.065232in}{4.918443in}}%
\pgfpathcurveto{\pgfqpoint{8.073046in}{4.926256in}}{\pgfqpoint{8.077436in}{4.936855in}}{\pgfqpoint{8.077436in}{4.947906in}}%
\pgfpathcurveto{\pgfqpoint{8.077436in}{4.958956in}}{\pgfqpoint{8.073046in}{4.969555in}}{\pgfqpoint{8.065232in}{4.977368in}}%
\pgfpathcurveto{\pgfqpoint{8.057418in}{4.985182in}}{\pgfqpoint{8.046819in}{4.989572in}}{\pgfqpoint{8.035769in}{4.989572in}}%
\pgfpathcurveto{\pgfqpoint{8.024719in}{4.989572in}}{\pgfqpoint{8.014120in}{4.985182in}}{\pgfqpoint{8.006307in}{4.977368in}}%
\pgfpathcurveto{\pgfqpoint{7.998493in}{4.969555in}}{\pgfqpoint{7.994103in}{4.958956in}}{\pgfqpoint{7.994103in}{4.947906in}}%
\pgfpathcurveto{\pgfqpoint{7.994103in}{4.936855in}}{\pgfqpoint{7.998493in}{4.926256in}}{\pgfqpoint{8.006307in}{4.918443in}}%
\pgfpathcurveto{\pgfqpoint{8.014120in}{4.910629in}}{\pgfqpoint{8.024719in}{4.906239in}}{\pgfqpoint{8.035769in}{4.906239in}}%
\pgfpathclose%
\pgfusepath{stroke,fill}%
\end{pgfscope}%
\begin{pgfscope}%
\pgfpathrectangle{\pgfqpoint{0.481978in}{0.331635in}}{\pgfqpoint{9.300000in}{7.700000in}}%
\pgfusepath{clip}%
\pgfsetbuttcap%
\pgfsetroundjoin%
\definecolor{currentfill}{rgb}{1.000000,0.705882,0.509804}%
\pgfsetfillcolor{currentfill}%
\pgfsetlinewidth{0.481800pt}%
\definecolor{currentstroke}{rgb}{1.000000,1.000000,1.000000}%
\pgfsetstrokecolor{currentstroke}%
\pgfsetdash{}{0pt}%
\pgfpathmoveto{\pgfqpoint{6.113857in}{3.264397in}}%
\pgfpathcurveto{\pgfqpoint{6.124907in}{3.264397in}}{\pgfqpoint{6.135506in}{3.268787in}}{\pgfqpoint{6.143320in}{3.276601in}}%
\pgfpathcurveto{\pgfqpoint{6.151133in}{3.284414in}}{\pgfqpoint{6.155524in}{3.295013in}}{\pgfqpoint{6.155524in}{3.306063in}}%
\pgfpathcurveto{\pgfqpoint{6.155524in}{3.317114in}}{\pgfqpoint{6.151133in}{3.327713in}}{\pgfqpoint{6.143320in}{3.335526in}}%
\pgfpathcurveto{\pgfqpoint{6.135506in}{3.343340in}}{\pgfqpoint{6.124907in}{3.347730in}}{\pgfqpoint{6.113857in}{3.347730in}}%
\pgfpathcurveto{\pgfqpoint{6.102807in}{3.347730in}}{\pgfqpoint{6.092208in}{3.343340in}}{\pgfqpoint{6.084394in}{3.335526in}}%
\pgfpathcurveto{\pgfqpoint{6.076581in}{3.327713in}}{\pgfqpoint{6.072190in}{3.317114in}}{\pgfqpoint{6.072190in}{3.306063in}}%
\pgfpathcurveto{\pgfqpoint{6.072190in}{3.295013in}}{\pgfqpoint{6.076581in}{3.284414in}}{\pgfqpoint{6.084394in}{3.276601in}}%
\pgfpathcurveto{\pgfqpoint{6.092208in}{3.268787in}}{\pgfqpoint{6.102807in}{3.264397in}}{\pgfqpoint{6.113857in}{3.264397in}}%
\pgfpathclose%
\pgfusepath{stroke,fill}%
\end{pgfscope}%
\begin{pgfscope}%
\pgfpathrectangle{\pgfqpoint{0.481978in}{0.331635in}}{\pgfqpoint{9.300000in}{7.700000in}}%
\pgfusepath{clip}%
\pgfsetbuttcap%
\pgfsetroundjoin%
\definecolor{currentfill}{rgb}{1.000000,0.705882,0.509804}%
\pgfsetfillcolor{currentfill}%
\pgfsetlinewidth{0.481800pt}%
\definecolor{currentstroke}{rgb}{1.000000,1.000000,1.000000}%
\pgfsetstrokecolor{currentstroke}%
\pgfsetdash{}{0pt}%
\pgfpathmoveto{\pgfqpoint{5.267481in}{4.055813in}}%
\pgfpathcurveto{\pgfqpoint{5.278531in}{4.055813in}}{\pgfqpoint{5.289130in}{4.060203in}}{\pgfqpoint{5.296944in}{4.068017in}}%
\pgfpathcurveto{\pgfqpoint{5.304758in}{4.075831in}}{\pgfqpoint{5.309148in}{4.086430in}}{\pgfqpoint{5.309148in}{4.097480in}}%
\pgfpathcurveto{\pgfqpoint{5.309148in}{4.108530in}}{\pgfqpoint{5.304758in}{4.119129in}}{\pgfqpoint{5.296944in}{4.126943in}}%
\pgfpathcurveto{\pgfqpoint{5.289130in}{4.134756in}}{\pgfqpoint{5.278531in}{4.139146in}}{\pgfqpoint{5.267481in}{4.139146in}}%
\pgfpathcurveto{\pgfqpoint{5.256431in}{4.139146in}}{\pgfqpoint{5.245832in}{4.134756in}}{\pgfqpoint{5.238018in}{4.126943in}}%
\pgfpathcurveto{\pgfqpoint{5.230205in}{4.119129in}}{\pgfqpoint{5.225815in}{4.108530in}}{\pgfqpoint{5.225815in}{4.097480in}}%
\pgfpathcurveto{\pgfqpoint{5.225815in}{4.086430in}}{\pgfqpoint{5.230205in}{4.075831in}}{\pgfqpoint{5.238018in}{4.068017in}}%
\pgfpathcurveto{\pgfqpoint{5.245832in}{4.060203in}}{\pgfqpoint{5.256431in}{4.055813in}}{\pgfqpoint{5.267481in}{4.055813in}}%
\pgfpathclose%
\pgfusepath{stroke,fill}%
\end{pgfscope}%
\begin{pgfscope}%
\pgfpathrectangle{\pgfqpoint{0.481978in}{0.331635in}}{\pgfqpoint{9.300000in}{7.700000in}}%
\pgfusepath{clip}%
\pgfsetbuttcap%
\pgfsetroundjoin%
\definecolor{currentfill}{rgb}{1.000000,0.705882,0.509804}%
\pgfsetfillcolor{currentfill}%
\pgfsetlinewidth{0.481800pt}%
\definecolor{currentstroke}{rgb}{1.000000,1.000000,1.000000}%
\pgfsetstrokecolor{currentstroke}%
\pgfsetdash{}{0pt}%
\pgfpathmoveto{\pgfqpoint{2.715921in}{1.829469in}}%
\pgfpathcurveto{\pgfqpoint{2.726972in}{1.829469in}}{\pgfqpoint{2.737571in}{1.833859in}}{\pgfqpoint{2.745384in}{1.841673in}}%
\pgfpathcurveto{\pgfqpoint{2.753198in}{1.849487in}}{\pgfqpoint{2.757588in}{1.860086in}}{\pgfqpoint{2.757588in}{1.871136in}}%
\pgfpathcurveto{\pgfqpoint{2.757588in}{1.882186in}}{\pgfqpoint{2.753198in}{1.892785in}}{\pgfqpoint{2.745384in}{1.900599in}}%
\pgfpathcurveto{\pgfqpoint{2.737571in}{1.908412in}}{\pgfqpoint{2.726972in}{1.912803in}}{\pgfqpoint{2.715921in}{1.912803in}}%
\pgfpathcurveto{\pgfqpoint{2.704871in}{1.912803in}}{\pgfqpoint{2.694272in}{1.908412in}}{\pgfqpoint{2.686459in}{1.900599in}}%
\pgfpathcurveto{\pgfqpoint{2.678645in}{1.892785in}}{\pgfqpoint{2.674255in}{1.882186in}}{\pgfqpoint{2.674255in}{1.871136in}}%
\pgfpathcurveto{\pgfqpoint{2.674255in}{1.860086in}}{\pgfqpoint{2.678645in}{1.849487in}}{\pgfqpoint{2.686459in}{1.841673in}}%
\pgfpathcurveto{\pgfqpoint{2.694272in}{1.833859in}}{\pgfqpoint{2.704871in}{1.829469in}}{\pgfqpoint{2.715921in}{1.829469in}}%
\pgfpathclose%
\pgfusepath{stroke,fill}%
\end{pgfscope}%
\begin{pgfscope}%
\pgfpathrectangle{\pgfqpoint{0.481978in}{0.331635in}}{\pgfqpoint{9.300000in}{7.700000in}}%
\pgfusepath{clip}%
\pgfsetbuttcap%
\pgfsetroundjoin%
\definecolor{currentfill}{rgb}{1.000000,0.705882,0.509804}%
\pgfsetfillcolor{currentfill}%
\pgfsetlinewidth{0.481800pt}%
\definecolor{currentstroke}{rgb}{1.000000,1.000000,1.000000}%
\pgfsetstrokecolor{currentstroke}%
\pgfsetdash{}{0pt}%
\pgfpathmoveto{\pgfqpoint{0.904705in}{5.054642in}}%
\pgfpathcurveto{\pgfqpoint{0.915755in}{5.054642in}}{\pgfqpoint{0.926354in}{5.059032in}}{\pgfqpoint{0.934168in}{5.066846in}}%
\pgfpathcurveto{\pgfqpoint{0.941982in}{5.074660in}}{\pgfqpoint{0.946372in}{5.085259in}}{\pgfqpoint{0.946372in}{5.096309in}}%
\pgfpathcurveto{\pgfqpoint{0.946372in}{5.107359in}}{\pgfqpoint{0.941982in}{5.117958in}}{\pgfqpoint{0.934168in}{5.125772in}}%
\pgfpathcurveto{\pgfqpoint{0.926354in}{5.133585in}}{\pgfqpoint{0.915755in}{5.137975in}}{\pgfqpoint{0.904705in}{5.137975in}}%
\pgfpathcurveto{\pgfqpoint{0.893655in}{5.137975in}}{\pgfqpoint{0.883056in}{5.133585in}}{\pgfqpoint{0.875242in}{5.125772in}}%
\pgfpathcurveto{\pgfqpoint{0.867429in}{5.117958in}}{\pgfqpoint{0.863039in}{5.107359in}}{\pgfqpoint{0.863039in}{5.096309in}}%
\pgfpathcurveto{\pgfqpoint{0.863039in}{5.085259in}}{\pgfqpoint{0.867429in}{5.074660in}}{\pgfqpoint{0.875242in}{5.066846in}}%
\pgfpathcurveto{\pgfqpoint{0.883056in}{5.059032in}}{\pgfqpoint{0.893655in}{5.054642in}}{\pgfqpoint{0.904705in}{5.054642in}}%
\pgfpathclose%
\pgfusepath{stroke,fill}%
\end{pgfscope}%
\begin{pgfscope}%
\pgfpathrectangle{\pgfqpoint{0.481978in}{0.331635in}}{\pgfqpoint{9.300000in}{7.700000in}}%
\pgfusepath{clip}%
\pgfsetbuttcap%
\pgfsetroundjoin%
\definecolor{currentfill}{rgb}{1.000000,0.705882,0.509804}%
\pgfsetfillcolor{currentfill}%
\pgfsetlinewidth{0.481800pt}%
\definecolor{currentstroke}{rgb}{1.000000,1.000000,1.000000}%
\pgfsetstrokecolor{currentstroke}%
\pgfsetdash{}{0pt}%
\pgfpathmoveto{\pgfqpoint{8.533110in}{3.540648in}}%
\pgfpathcurveto{\pgfqpoint{8.544160in}{3.540648in}}{\pgfqpoint{8.554759in}{3.545038in}}{\pgfqpoint{8.562572in}{3.552852in}}%
\pgfpathcurveto{\pgfqpoint{8.570386in}{3.560666in}}{\pgfqpoint{8.574776in}{3.571265in}}{\pgfqpoint{8.574776in}{3.582315in}}%
\pgfpathcurveto{\pgfqpoint{8.574776in}{3.593365in}}{\pgfqpoint{8.570386in}{3.603964in}}{\pgfqpoint{8.562572in}{3.611777in}}%
\pgfpathcurveto{\pgfqpoint{8.554759in}{3.619591in}}{\pgfqpoint{8.544160in}{3.623981in}}{\pgfqpoint{8.533110in}{3.623981in}}%
\pgfpathcurveto{\pgfqpoint{8.522059in}{3.623981in}}{\pgfqpoint{8.511460in}{3.619591in}}{\pgfqpoint{8.503647in}{3.611777in}}%
\pgfpathcurveto{\pgfqpoint{8.495833in}{3.603964in}}{\pgfqpoint{8.491443in}{3.593365in}}{\pgfqpoint{8.491443in}{3.582315in}}%
\pgfpathcurveto{\pgfqpoint{8.491443in}{3.571265in}}{\pgfqpoint{8.495833in}{3.560666in}}{\pgfqpoint{8.503647in}{3.552852in}}%
\pgfpathcurveto{\pgfqpoint{8.511460in}{3.545038in}}{\pgfqpoint{8.522059in}{3.540648in}}{\pgfqpoint{8.533110in}{3.540648in}}%
\pgfpathclose%
\pgfusepath{stroke,fill}%
\end{pgfscope}%
\begin{pgfscope}%
\pgfpathrectangle{\pgfqpoint{0.481978in}{0.331635in}}{\pgfqpoint{9.300000in}{7.700000in}}%
\pgfusepath{clip}%
\pgfsetbuttcap%
\pgfsetroundjoin%
\definecolor{currentfill}{rgb}{1.000000,0.705882,0.509804}%
\pgfsetfillcolor{currentfill}%
\pgfsetlinewidth{0.481800pt}%
\definecolor{currentstroke}{rgb}{1.000000,1.000000,1.000000}%
\pgfsetstrokecolor{currentstroke}%
\pgfsetdash{}{0pt}%
\pgfpathmoveto{\pgfqpoint{4.391134in}{3.007273in}}%
\pgfpathcurveto{\pgfqpoint{4.402184in}{3.007273in}}{\pgfqpoint{4.412783in}{3.011663in}}{\pgfqpoint{4.420597in}{3.019477in}}%
\pgfpathcurveto{\pgfqpoint{4.428410in}{3.027290in}}{\pgfqpoint{4.432800in}{3.037889in}}{\pgfqpoint{4.432800in}{3.048940in}}%
\pgfpathcurveto{\pgfqpoint{4.432800in}{3.059990in}}{\pgfqpoint{4.428410in}{3.070589in}}{\pgfqpoint{4.420597in}{3.078402in}}%
\pgfpathcurveto{\pgfqpoint{4.412783in}{3.086216in}}{\pgfqpoint{4.402184in}{3.090606in}}{\pgfqpoint{4.391134in}{3.090606in}}%
\pgfpathcurveto{\pgfqpoint{4.380084in}{3.090606in}}{\pgfqpoint{4.369485in}{3.086216in}}{\pgfqpoint{4.361671in}{3.078402in}}%
\pgfpathcurveto{\pgfqpoint{4.353857in}{3.070589in}}{\pgfqpoint{4.349467in}{3.059990in}}{\pgfqpoint{4.349467in}{3.048940in}}%
\pgfpathcurveto{\pgfqpoint{4.349467in}{3.037889in}}{\pgfqpoint{4.353857in}{3.027290in}}{\pgfqpoint{4.361671in}{3.019477in}}%
\pgfpathcurveto{\pgfqpoint{4.369485in}{3.011663in}}{\pgfqpoint{4.380084in}{3.007273in}}{\pgfqpoint{4.391134in}{3.007273in}}%
\pgfpathclose%
\pgfusepath{stroke,fill}%
\end{pgfscope}%
\begin{pgfscope}%
\pgfpathrectangle{\pgfqpoint{0.481978in}{0.331635in}}{\pgfqpoint{9.300000in}{7.700000in}}%
\pgfusepath{clip}%
\pgfsetbuttcap%
\pgfsetroundjoin%
\definecolor{currentfill}{rgb}{1.000000,0.705882,0.509804}%
\pgfsetfillcolor{currentfill}%
\pgfsetlinewidth{0.481800pt}%
\definecolor{currentstroke}{rgb}{1.000000,1.000000,1.000000}%
\pgfsetstrokecolor{currentstroke}%
\pgfsetdash{}{0pt}%
\pgfpathmoveto{\pgfqpoint{1.537381in}{5.295533in}}%
\pgfpathcurveto{\pgfqpoint{1.548432in}{5.295533in}}{\pgfqpoint{1.559031in}{5.299923in}}{\pgfqpoint{1.566844in}{5.307737in}}%
\pgfpathcurveto{\pgfqpoint{1.574658in}{5.315551in}}{\pgfqpoint{1.579048in}{5.326150in}}{\pgfqpoint{1.579048in}{5.337200in}}%
\pgfpathcurveto{\pgfqpoint{1.579048in}{5.348250in}}{\pgfqpoint{1.574658in}{5.358849in}}{\pgfqpoint{1.566844in}{5.366662in}}%
\pgfpathcurveto{\pgfqpoint{1.559031in}{5.374476in}}{\pgfqpoint{1.548432in}{5.378866in}}{\pgfqpoint{1.537381in}{5.378866in}}%
\pgfpathcurveto{\pgfqpoint{1.526331in}{5.378866in}}{\pgfqpoint{1.515732in}{5.374476in}}{\pgfqpoint{1.507919in}{5.366662in}}%
\pgfpathcurveto{\pgfqpoint{1.500105in}{5.358849in}}{\pgfqpoint{1.495715in}{5.348250in}}{\pgfqpoint{1.495715in}{5.337200in}}%
\pgfpathcurveto{\pgfqpoint{1.495715in}{5.326150in}}{\pgfqpoint{1.500105in}{5.315551in}}{\pgfqpoint{1.507919in}{5.307737in}}%
\pgfpathcurveto{\pgfqpoint{1.515732in}{5.299923in}}{\pgfqpoint{1.526331in}{5.295533in}}{\pgfqpoint{1.537381in}{5.295533in}}%
\pgfpathclose%
\pgfusepath{stroke,fill}%
\end{pgfscope}%
\begin{pgfscope}%
\pgfpathrectangle{\pgfqpoint{0.481978in}{0.331635in}}{\pgfqpoint{9.300000in}{7.700000in}}%
\pgfusepath{clip}%
\pgfsetbuttcap%
\pgfsetroundjoin%
\definecolor{currentfill}{rgb}{1.000000,0.705882,0.509804}%
\pgfsetfillcolor{currentfill}%
\pgfsetlinewidth{0.481800pt}%
\definecolor{currentstroke}{rgb}{1.000000,1.000000,1.000000}%
\pgfsetstrokecolor{currentstroke}%
\pgfsetdash{}{0pt}%
\pgfpathmoveto{\pgfqpoint{6.879175in}{3.698562in}}%
\pgfpathcurveto{\pgfqpoint{6.890225in}{3.698562in}}{\pgfqpoint{6.900824in}{3.702952in}}{\pgfqpoint{6.908638in}{3.710766in}}%
\pgfpathcurveto{\pgfqpoint{6.916452in}{3.718579in}}{\pgfqpoint{6.920842in}{3.729178in}}{\pgfqpoint{6.920842in}{3.740228in}}%
\pgfpathcurveto{\pgfqpoint{6.920842in}{3.751278in}}{\pgfqpoint{6.916452in}{3.761877in}}{\pgfqpoint{6.908638in}{3.769691in}}%
\pgfpathcurveto{\pgfqpoint{6.900824in}{3.777505in}}{\pgfqpoint{6.890225in}{3.781895in}}{\pgfqpoint{6.879175in}{3.781895in}}%
\pgfpathcurveto{\pgfqpoint{6.868125in}{3.781895in}}{\pgfqpoint{6.857526in}{3.777505in}}{\pgfqpoint{6.849712in}{3.769691in}}%
\pgfpathcurveto{\pgfqpoint{6.841899in}{3.761877in}}{\pgfqpoint{6.837509in}{3.751278in}}{\pgfqpoint{6.837509in}{3.740228in}}%
\pgfpathcurveto{\pgfqpoint{6.837509in}{3.729178in}}{\pgfqpoint{6.841899in}{3.718579in}}{\pgfqpoint{6.849712in}{3.710766in}}%
\pgfpathcurveto{\pgfqpoint{6.857526in}{3.702952in}}{\pgfqpoint{6.868125in}{3.698562in}}{\pgfqpoint{6.879175in}{3.698562in}}%
\pgfpathclose%
\pgfusepath{stroke,fill}%
\end{pgfscope}%
\begin{pgfscope}%
\pgfpathrectangle{\pgfqpoint{0.481978in}{0.331635in}}{\pgfqpoint{9.300000in}{7.700000in}}%
\pgfusepath{clip}%
\pgfsetbuttcap%
\pgfsetroundjoin%
\definecolor{currentfill}{rgb}{1.000000,0.705882,0.509804}%
\pgfsetfillcolor{currentfill}%
\pgfsetlinewidth{0.481800pt}%
\definecolor{currentstroke}{rgb}{1.000000,1.000000,1.000000}%
\pgfsetstrokecolor{currentstroke}%
\pgfsetdash{}{0pt}%
\pgfpathmoveto{\pgfqpoint{4.800335in}{3.680630in}}%
\pgfpathcurveto{\pgfqpoint{4.811385in}{3.680630in}}{\pgfqpoint{4.821984in}{3.685020in}}{\pgfqpoint{4.829798in}{3.692834in}}%
\pgfpathcurveto{\pgfqpoint{4.837611in}{3.700647in}}{\pgfqpoint{4.842002in}{3.711246in}}{\pgfqpoint{4.842002in}{3.722296in}}%
\pgfpathcurveto{\pgfqpoint{4.842002in}{3.733346in}}{\pgfqpoint{4.837611in}{3.743945in}}{\pgfqpoint{4.829798in}{3.751759in}}%
\pgfpathcurveto{\pgfqpoint{4.821984in}{3.759573in}}{\pgfqpoint{4.811385in}{3.763963in}}{\pgfqpoint{4.800335in}{3.763963in}}%
\pgfpathcurveto{\pgfqpoint{4.789285in}{3.763963in}}{\pgfqpoint{4.778686in}{3.759573in}}{\pgfqpoint{4.770872in}{3.751759in}}%
\pgfpathcurveto{\pgfqpoint{4.763059in}{3.743945in}}{\pgfqpoint{4.758668in}{3.733346in}}{\pgfqpoint{4.758668in}{3.722296in}}%
\pgfpathcurveto{\pgfqpoint{4.758668in}{3.711246in}}{\pgfqpoint{4.763059in}{3.700647in}}{\pgfqpoint{4.770872in}{3.692834in}}%
\pgfpathcurveto{\pgfqpoint{4.778686in}{3.685020in}}{\pgfqpoint{4.789285in}{3.680630in}}{\pgfqpoint{4.800335in}{3.680630in}}%
\pgfpathclose%
\pgfusepath{stroke,fill}%
\end{pgfscope}%
\begin{pgfscope}%
\pgfpathrectangle{\pgfqpoint{0.481978in}{0.331635in}}{\pgfqpoint{9.300000in}{7.700000in}}%
\pgfusepath{clip}%
\pgfsetbuttcap%
\pgfsetroundjoin%
\definecolor{currentfill}{rgb}{1.000000,0.705882,0.509804}%
\pgfsetfillcolor{currentfill}%
\pgfsetlinewidth{0.481800pt}%
\definecolor{currentstroke}{rgb}{1.000000,1.000000,1.000000}%
\pgfsetstrokecolor{currentstroke}%
\pgfsetdash{}{0pt}%
\pgfpathmoveto{\pgfqpoint{4.757910in}{5.288221in}}%
\pgfpathcurveto{\pgfqpoint{4.768960in}{5.288221in}}{\pgfqpoint{4.779559in}{5.292611in}}{\pgfqpoint{4.787373in}{5.300424in}}%
\pgfpathcurveto{\pgfqpoint{4.795186in}{5.308238in}}{\pgfqpoint{4.799577in}{5.318837in}}{\pgfqpoint{4.799577in}{5.329887in}}%
\pgfpathcurveto{\pgfqpoint{4.799577in}{5.340937in}}{\pgfqpoint{4.795186in}{5.351536in}}{\pgfqpoint{4.787373in}{5.359350in}}%
\pgfpathcurveto{\pgfqpoint{4.779559in}{5.367164in}}{\pgfqpoint{4.768960in}{5.371554in}}{\pgfqpoint{4.757910in}{5.371554in}}%
\pgfpathcurveto{\pgfqpoint{4.746860in}{5.371554in}}{\pgfqpoint{4.736261in}{5.367164in}}{\pgfqpoint{4.728447in}{5.359350in}}%
\pgfpathcurveto{\pgfqpoint{4.720633in}{5.351536in}}{\pgfqpoint{4.716243in}{5.340937in}}{\pgfqpoint{4.716243in}{5.329887in}}%
\pgfpathcurveto{\pgfqpoint{4.716243in}{5.318837in}}{\pgfqpoint{4.720633in}{5.308238in}}{\pgfqpoint{4.728447in}{5.300424in}}%
\pgfpathcurveto{\pgfqpoint{4.736261in}{5.292611in}}{\pgfqpoint{4.746860in}{5.288221in}}{\pgfqpoint{4.757910in}{5.288221in}}%
\pgfpathclose%
\pgfusepath{stroke,fill}%
\end{pgfscope}%
\begin{pgfscope}%
\pgfpathrectangle{\pgfqpoint{0.481978in}{0.331635in}}{\pgfqpoint{9.300000in}{7.700000in}}%
\pgfusepath{clip}%
\pgfsetbuttcap%
\pgfsetroundjoin%
\definecolor{currentfill}{rgb}{1.000000,0.705882,0.509804}%
\pgfsetfillcolor{currentfill}%
\pgfsetlinewidth{0.481800pt}%
\definecolor{currentstroke}{rgb}{1.000000,1.000000,1.000000}%
\pgfsetstrokecolor{currentstroke}%
\pgfsetdash{}{0pt}%
\pgfpathmoveto{\pgfqpoint{4.104994in}{5.646590in}}%
\pgfpathcurveto{\pgfqpoint{4.116044in}{5.646590in}}{\pgfqpoint{4.126643in}{5.650980in}}{\pgfqpoint{4.134457in}{5.658794in}}%
\pgfpathcurveto{\pgfqpoint{4.142271in}{5.666607in}}{\pgfqpoint{4.146661in}{5.677206in}}{\pgfqpoint{4.146661in}{5.688256in}}%
\pgfpathcurveto{\pgfqpoint{4.146661in}{5.699306in}}{\pgfqpoint{4.142271in}{5.709906in}}{\pgfqpoint{4.134457in}{5.717719in}}%
\pgfpathcurveto{\pgfqpoint{4.126643in}{5.725533in}}{\pgfqpoint{4.116044in}{5.729923in}}{\pgfqpoint{4.104994in}{5.729923in}}%
\pgfpathcurveto{\pgfqpoint{4.093944in}{5.729923in}}{\pgfqpoint{4.083345in}{5.725533in}}{\pgfqpoint{4.075531in}{5.717719in}}%
\pgfpathcurveto{\pgfqpoint{4.067718in}{5.709906in}}{\pgfqpoint{4.063328in}{5.699306in}}{\pgfqpoint{4.063328in}{5.688256in}}%
\pgfpathcurveto{\pgfqpoint{4.063328in}{5.677206in}}{\pgfqpoint{4.067718in}{5.666607in}}{\pgfqpoint{4.075531in}{5.658794in}}%
\pgfpathcurveto{\pgfqpoint{4.083345in}{5.650980in}}{\pgfqpoint{4.093944in}{5.646590in}}{\pgfqpoint{4.104994in}{5.646590in}}%
\pgfpathclose%
\pgfusepath{stroke,fill}%
\end{pgfscope}%
\begin{pgfscope}%
\pgfpathrectangle{\pgfqpoint{0.481978in}{0.331635in}}{\pgfqpoint{9.300000in}{7.700000in}}%
\pgfusepath{clip}%
\pgfsetbuttcap%
\pgfsetroundjoin%
\definecolor{currentfill}{rgb}{1.000000,0.705882,0.509804}%
\pgfsetfillcolor{currentfill}%
\pgfsetlinewidth{0.481800pt}%
\definecolor{currentstroke}{rgb}{1.000000,1.000000,1.000000}%
\pgfsetstrokecolor{currentstroke}%
\pgfsetdash{}{0pt}%
\pgfpathmoveto{\pgfqpoint{1.344247in}{4.328959in}}%
\pgfpathcurveto{\pgfqpoint{1.355297in}{4.328959in}}{\pgfqpoint{1.365896in}{4.333349in}}{\pgfqpoint{1.373710in}{4.341163in}}%
\pgfpathcurveto{\pgfqpoint{1.381524in}{4.348977in}}{\pgfqpoint{1.385914in}{4.359576in}}{\pgfqpoint{1.385914in}{4.370626in}}%
\pgfpathcurveto{\pgfqpoint{1.385914in}{4.381676in}}{\pgfqpoint{1.381524in}{4.392275in}}{\pgfqpoint{1.373710in}{4.400089in}}%
\pgfpathcurveto{\pgfqpoint{1.365896in}{4.407902in}}{\pgfqpoint{1.355297in}{4.412292in}}{\pgfqpoint{1.344247in}{4.412292in}}%
\pgfpathcurveto{\pgfqpoint{1.333197in}{4.412292in}}{\pgfqpoint{1.322598in}{4.407902in}}{\pgfqpoint{1.314784in}{4.400089in}}%
\pgfpathcurveto{\pgfqpoint{1.306971in}{4.392275in}}{\pgfqpoint{1.302580in}{4.381676in}}{\pgfqpoint{1.302580in}{4.370626in}}%
\pgfpathcurveto{\pgfqpoint{1.302580in}{4.359576in}}{\pgfqpoint{1.306971in}{4.348977in}}{\pgfqpoint{1.314784in}{4.341163in}}%
\pgfpathcurveto{\pgfqpoint{1.322598in}{4.333349in}}{\pgfqpoint{1.333197in}{4.328959in}}{\pgfqpoint{1.344247in}{4.328959in}}%
\pgfpathclose%
\pgfusepath{stroke,fill}%
\end{pgfscope}%
\begin{pgfscope}%
\pgfpathrectangle{\pgfqpoint{0.481978in}{0.331635in}}{\pgfqpoint{9.300000in}{7.700000in}}%
\pgfusepath{clip}%
\pgfsetbuttcap%
\pgfsetroundjoin%
\definecolor{currentfill}{rgb}{1.000000,0.705882,0.509804}%
\pgfsetfillcolor{currentfill}%
\pgfsetlinewidth{0.481800pt}%
\definecolor{currentstroke}{rgb}{1.000000,1.000000,1.000000}%
\pgfsetstrokecolor{currentstroke}%
\pgfsetdash{}{0pt}%
\pgfpathmoveto{\pgfqpoint{3.248962in}{4.199832in}}%
\pgfpathcurveto{\pgfqpoint{3.260013in}{4.199832in}}{\pgfqpoint{3.270612in}{4.204223in}}{\pgfqpoint{3.278425in}{4.212036in}}%
\pgfpathcurveto{\pgfqpoint{3.286239in}{4.219850in}}{\pgfqpoint{3.290629in}{4.230449in}}{\pgfqpoint{3.290629in}{4.241499in}}%
\pgfpathcurveto{\pgfqpoint{3.290629in}{4.252549in}}{\pgfqpoint{3.286239in}{4.263148in}}{\pgfqpoint{3.278425in}{4.270962in}}%
\pgfpathcurveto{\pgfqpoint{3.270612in}{4.278775in}}{\pgfqpoint{3.260013in}{4.283166in}}{\pgfqpoint{3.248962in}{4.283166in}}%
\pgfpathcurveto{\pgfqpoint{3.237912in}{4.283166in}}{\pgfqpoint{3.227313in}{4.278775in}}{\pgfqpoint{3.219500in}{4.270962in}}%
\pgfpathcurveto{\pgfqpoint{3.211686in}{4.263148in}}{\pgfqpoint{3.207296in}{4.252549in}}{\pgfqpoint{3.207296in}{4.241499in}}%
\pgfpathcurveto{\pgfqpoint{3.207296in}{4.230449in}}{\pgfqpoint{3.211686in}{4.219850in}}{\pgfqpoint{3.219500in}{4.212036in}}%
\pgfpathcurveto{\pgfqpoint{3.227313in}{4.204223in}}{\pgfqpoint{3.237912in}{4.199832in}}{\pgfqpoint{3.248962in}{4.199832in}}%
\pgfpathclose%
\pgfusepath{stroke,fill}%
\end{pgfscope}%
\begin{pgfscope}%
\pgfpathrectangle{\pgfqpoint{0.481978in}{0.331635in}}{\pgfqpoint{9.300000in}{7.700000in}}%
\pgfusepath{clip}%
\pgfsetbuttcap%
\pgfsetroundjoin%
\definecolor{currentfill}{rgb}{0.631373,0.788235,0.956863}%
\pgfsetfillcolor{currentfill}%
\pgfsetlinewidth{1.003750pt}%
\definecolor{currentstroke}{rgb}{0.631373,0.788235,0.956863}%
\pgfsetstrokecolor{currentstroke}%
\pgfsetdash{}{0pt}%
\pgfsys@defobject{currentmarker}{\pgfqpoint{-0.041667in}{-0.041667in}}{\pgfqpoint{0.041667in}{0.041667in}}{%
\pgfpathmoveto{\pgfqpoint{0.000000in}{-0.041667in}}%
\pgfpathcurveto{\pgfqpoint{0.011050in}{-0.041667in}}{\pgfqpoint{0.021649in}{-0.037276in}}{\pgfqpoint{0.029463in}{-0.029463in}}%
\pgfpathcurveto{\pgfqpoint{0.037276in}{-0.021649in}}{\pgfqpoint{0.041667in}{-0.011050in}}{\pgfqpoint{0.041667in}{0.000000in}}%
\pgfpathcurveto{\pgfqpoint{0.041667in}{0.011050in}}{\pgfqpoint{0.037276in}{0.021649in}}{\pgfqpoint{0.029463in}{0.029463in}}%
\pgfpathcurveto{\pgfqpoint{0.021649in}{0.037276in}}{\pgfqpoint{0.011050in}{0.041667in}}{\pgfqpoint{0.000000in}{0.041667in}}%
\pgfpathcurveto{\pgfqpoint{-0.011050in}{0.041667in}}{\pgfqpoint{-0.021649in}{0.037276in}}{\pgfqpoint{-0.029463in}{0.029463in}}%
\pgfpathcurveto{\pgfqpoint{-0.037276in}{0.021649in}}{\pgfqpoint{-0.041667in}{0.011050in}}{\pgfqpoint{-0.041667in}{0.000000in}}%
\pgfpathcurveto{\pgfqpoint{-0.041667in}{-0.011050in}}{\pgfqpoint{-0.037276in}{-0.021649in}}{\pgfqpoint{-0.029463in}{-0.029463in}}%
\pgfpathcurveto{\pgfqpoint{-0.021649in}{-0.037276in}}{\pgfqpoint{-0.011050in}{-0.041667in}}{\pgfqpoint{0.000000in}{-0.041667in}}%
\pgfpathclose%
\pgfusepath{stroke,fill}%
}%
\end{pgfscope}%
\begin{pgfscope}%
\pgfpathrectangle{\pgfqpoint{0.481978in}{0.331635in}}{\pgfqpoint{9.300000in}{7.700000in}}%
\pgfusepath{clip}%
\pgfsetbuttcap%
\pgfsetroundjoin%
\definecolor{currentfill}{rgb}{1.000000,0.705882,0.509804}%
\pgfsetfillcolor{currentfill}%
\pgfsetlinewidth{1.003750pt}%
\definecolor{currentstroke}{rgb}{1.000000,0.705882,0.509804}%
\pgfsetstrokecolor{currentstroke}%
\pgfsetdash{}{0pt}%
\pgfsys@defobject{currentmarker}{\pgfqpoint{-0.041667in}{-0.041667in}}{\pgfqpoint{0.041667in}{0.041667in}}{%
\pgfpathmoveto{\pgfqpoint{0.000000in}{-0.041667in}}%
\pgfpathcurveto{\pgfqpoint{0.011050in}{-0.041667in}}{\pgfqpoint{0.021649in}{-0.037276in}}{\pgfqpoint{0.029463in}{-0.029463in}}%
\pgfpathcurveto{\pgfqpoint{0.037276in}{-0.021649in}}{\pgfqpoint{0.041667in}{-0.011050in}}{\pgfqpoint{0.041667in}{0.000000in}}%
\pgfpathcurveto{\pgfqpoint{0.041667in}{0.011050in}}{\pgfqpoint{0.037276in}{0.021649in}}{\pgfqpoint{0.029463in}{0.029463in}}%
\pgfpathcurveto{\pgfqpoint{0.021649in}{0.037276in}}{\pgfqpoint{0.011050in}{0.041667in}}{\pgfqpoint{0.000000in}{0.041667in}}%
\pgfpathcurveto{\pgfqpoint{-0.011050in}{0.041667in}}{\pgfqpoint{-0.021649in}{0.037276in}}{\pgfqpoint{-0.029463in}{0.029463in}}%
\pgfpathcurveto{\pgfqpoint{-0.037276in}{0.021649in}}{\pgfqpoint{-0.041667in}{0.011050in}}{\pgfqpoint{-0.041667in}{0.000000in}}%
\pgfpathcurveto{\pgfqpoint{-0.041667in}{-0.011050in}}{\pgfqpoint{-0.037276in}{-0.021649in}}{\pgfqpoint{-0.029463in}{-0.029463in}}%
\pgfpathcurveto{\pgfqpoint{-0.021649in}{-0.037276in}}{\pgfqpoint{-0.011050in}{-0.041667in}}{\pgfqpoint{0.000000in}{-0.041667in}}%
\pgfpathclose%
\pgfusepath{stroke,fill}%
}%
\end{pgfscope}%
\begin{pgfscope}%
\pgfsetbuttcap%
\pgfsetroundjoin%
\definecolor{currentfill}{rgb}{0.000000,0.000000,0.000000}%
\pgfsetfillcolor{currentfill}%
\pgfsetlinewidth{0.803000pt}%
\definecolor{currentstroke}{rgb}{0.000000,0.000000,0.000000}%
\pgfsetstrokecolor{currentstroke}%
\pgfsetdash{}{0pt}%
\pgfsys@defobject{currentmarker}{\pgfqpoint{0.000000in}{-0.048611in}}{\pgfqpoint{0.000000in}{0.000000in}}{%
\pgfpathmoveto{\pgfqpoint{0.000000in}{0.000000in}}%
\pgfpathlineto{\pgfqpoint{0.000000in}{-0.048611in}}%
\pgfusepath{stroke,fill}%
}%
\begin{pgfscope}%
\pgfsys@transformshift{1.232596in}{0.331635in}%
\pgfsys@useobject{currentmarker}{}%
\end{pgfscope}%
\end{pgfscope}%
\begin{pgfscope}%
\definecolor{textcolor}{rgb}{0.000000,0.000000,0.000000}%
\pgfsetstrokecolor{textcolor}%
\pgfsetfillcolor{textcolor}%
\pgftext[x=1.232596in,y=0.234413in,,top]{\color{textcolor}\sffamily\fontsize{10.000000}{12.000000}\selectfont \ensuremath{-}100}%
\end{pgfscope}%
\begin{pgfscope}%
\pgfsetbuttcap%
\pgfsetroundjoin%
\definecolor{currentfill}{rgb}{0.000000,0.000000,0.000000}%
\pgfsetfillcolor{currentfill}%
\pgfsetlinewidth{0.803000pt}%
\definecolor{currentstroke}{rgb}{0.000000,0.000000,0.000000}%
\pgfsetstrokecolor{currentstroke}%
\pgfsetdash{}{0pt}%
\pgfsys@defobject{currentmarker}{\pgfqpoint{0.000000in}{-0.048611in}}{\pgfqpoint{0.000000in}{0.000000in}}{%
\pgfpathmoveto{\pgfqpoint{0.000000in}{0.000000in}}%
\pgfpathlineto{\pgfqpoint{0.000000in}{-0.048611in}}%
\pgfusepath{stroke,fill}%
}%
\begin{pgfscope}%
\pgfsys@transformshift{2.286320in}{0.331635in}%
\pgfsys@useobject{currentmarker}{}%
\end{pgfscope}%
\end{pgfscope}%
\begin{pgfscope}%
\definecolor{textcolor}{rgb}{0.000000,0.000000,0.000000}%
\pgfsetstrokecolor{textcolor}%
\pgfsetfillcolor{textcolor}%
\pgftext[x=2.286320in,y=0.234413in,,top]{\color{textcolor}\sffamily\fontsize{10.000000}{12.000000}\selectfont \ensuremath{-}75}%
\end{pgfscope}%
\begin{pgfscope}%
\pgfsetbuttcap%
\pgfsetroundjoin%
\definecolor{currentfill}{rgb}{0.000000,0.000000,0.000000}%
\pgfsetfillcolor{currentfill}%
\pgfsetlinewidth{0.803000pt}%
\definecolor{currentstroke}{rgb}{0.000000,0.000000,0.000000}%
\pgfsetstrokecolor{currentstroke}%
\pgfsetdash{}{0pt}%
\pgfsys@defobject{currentmarker}{\pgfqpoint{0.000000in}{-0.048611in}}{\pgfqpoint{0.000000in}{0.000000in}}{%
\pgfpathmoveto{\pgfqpoint{0.000000in}{0.000000in}}%
\pgfpathlineto{\pgfqpoint{0.000000in}{-0.048611in}}%
\pgfusepath{stroke,fill}%
}%
\begin{pgfscope}%
\pgfsys@transformshift{3.340044in}{0.331635in}%
\pgfsys@useobject{currentmarker}{}%
\end{pgfscope}%
\end{pgfscope}%
\begin{pgfscope}%
\definecolor{textcolor}{rgb}{0.000000,0.000000,0.000000}%
\pgfsetstrokecolor{textcolor}%
\pgfsetfillcolor{textcolor}%
\pgftext[x=3.340044in,y=0.234413in,,top]{\color{textcolor}\sffamily\fontsize{10.000000}{12.000000}\selectfont \ensuremath{-}50}%
\end{pgfscope}%
\begin{pgfscope}%
\pgfsetbuttcap%
\pgfsetroundjoin%
\definecolor{currentfill}{rgb}{0.000000,0.000000,0.000000}%
\pgfsetfillcolor{currentfill}%
\pgfsetlinewidth{0.803000pt}%
\definecolor{currentstroke}{rgb}{0.000000,0.000000,0.000000}%
\pgfsetstrokecolor{currentstroke}%
\pgfsetdash{}{0pt}%
\pgfsys@defobject{currentmarker}{\pgfqpoint{0.000000in}{-0.048611in}}{\pgfqpoint{0.000000in}{0.000000in}}{%
\pgfpathmoveto{\pgfqpoint{0.000000in}{0.000000in}}%
\pgfpathlineto{\pgfqpoint{0.000000in}{-0.048611in}}%
\pgfusepath{stroke,fill}%
}%
\begin{pgfscope}%
\pgfsys@transformshift{4.393768in}{0.331635in}%
\pgfsys@useobject{currentmarker}{}%
\end{pgfscope}%
\end{pgfscope}%
\begin{pgfscope}%
\definecolor{textcolor}{rgb}{0.000000,0.000000,0.000000}%
\pgfsetstrokecolor{textcolor}%
\pgfsetfillcolor{textcolor}%
\pgftext[x=4.393768in,y=0.234413in,,top]{\color{textcolor}\sffamily\fontsize{10.000000}{12.000000}\selectfont \ensuremath{-}25}%
\end{pgfscope}%
\begin{pgfscope}%
\pgfsetbuttcap%
\pgfsetroundjoin%
\definecolor{currentfill}{rgb}{0.000000,0.000000,0.000000}%
\pgfsetfillcolor{currentfill}%
\pgfsetlinewidth{0.803000pt}%
\definecolor{currentstroke}{rgb}{0.000000,0.000000,0.000000}%
\pgfsetstrokecolor{currentstroke}%
\pgfsetdash{}{0pt}%
\pgfsys@defobject{currentmarker}{\pgfqpoint{0.000000in}{-0.048611in}}{\pgfqpoint{0.000000in}{0.000000in}}{%
\pgfpathmoveto{\pgfqpoint{0.000000in}{0.000000in}}%
\pgfpathlineto{\pgfqpoint{0.000000in}{-0.048611in}}%
\pgfusepath{stroke,fill}%
}%
\begin{pgfscope}%
\pgfsys@transformshift{5.447492in}{0.331635in}%
\pgfsys@useobject{currentmarker}{}%
\end{pgfscope}%
\end{pgfscope}%
\begin{pgfscope}%
\definecolor{textcolor}{rgb}{0.000000,0.000000,0.000000}%
\pgfsetstrokecolor{textcolor}%
\pgfsetfillcolor{textcolor}%
\pgftext[x=5.447492in,y=0.234413in,,top]{\color{textcolor}\sffamily\fontsize{10.000000}{12.000000}\selectfont 0}%
\end{pgfscope}%
\begin{pgfscope}%
\pgfsetbuttcap%
\pgfsetroundjoin%
\definecolor{currentfill}{rgb}{0.000000,0.000000,0.000000}%
\pgfsetfillcolor{currentfill}%
\pgfsetlinewidth{0.803000pt}%
\definecolor{currentstroke}{rgb}{0.000000,0.000000,0.000000}%
\pgfsetstrokecolor{currentstroke}%
\pgfsetdash{}{0pt}%
\pgfsys@defobject{currentmarker}{\pgfqpoint{0.000000in}{-0.048611in}}{\pgfqpoint{0.000000in}{0.000000in}}{%
\pgfpathmoveto{\pgfqpoint{0.000000in}{0.000000in}}%
\pgfpathlineto{\pgfqpoint{0.000000in}{-0.048611in}}%
\pgfusepath{stroke,fill}%
}%
\begin{pgfscope}%
\pgfsys@transformshift{6.501216in}{0.331635in}%
\pgfsys@useobject{currentmarker}{}%
\end{pgfscope}%
\end{pgfscope}%
\begin{pgfscope}%
\definecolor{textcolor}{rgb}{0.000000,0.000000,0.000000}%
\pgfsetstrokecolor{textcolor}%
\pgfsetfillcolor{textcolor}%
\pgftext[x=6.501216in,y=0.234413in,,top]{\color{textcolor}\sffamily\fontsize{10.000000}{12.000000}\selectfont 25}%
\end{pgfscope}%
\begin{pgfscope}%
\pgfsetbuttcap%
\pgfsetroundjoin%
\definecolor{currentfill}{rgb}{0.000000,0.000000,0.000000}%
\pgfsetfillcolor{currentfill}%
\pgfsetlinewidth{0.803000pt}%
\definecolor{currentstroke}{rgb}{0.000000,0.000000,0.000000}%
\pgfsetstrokecolor{currentstroke}%
\pgfsetdash{}{0pt}%
\pgfsys@defobject{currentmarker}{\pgfqpoint{0.000000in}{-0.048611in}}{\pgfqpoint{0.000000in}{0.000000in}}{%
\pgfpathmoveto{\pgfqpoint{0.000000in}{0.000000in}}%
\pgfpathlineto{\pgfqpoint{0.000000in}{-0.048611in}}%
\pgfusepath{stroke,fill}%
}%
\begin{pgfscope}%
\pgfsys@transformshift{7.554940in}{0.331635in}%
\pgfsys@useobject{currentmarker}{}%
\end{pgfscope}%
\end{pgfscope}%
\begin{pgfscope}%
\definecolor{textcolor}{rgb}{0.000000,0.000000,0.000000}%
\pgfsetstrokecolor{textcolor}%
\pgfsetfillcolor{textcolor}%
\pgftext[x=7.554940in,y=0.234413in,,top]{\color{textcolor}\sffamily\fontsize{10.000000}{12.000000}\selectfont 50}%
\end{pgfscope}%
\begin{pgfscope}%
\pgfsetbuttcap%
\pgfsetroundjoin%
\definecolor{currentfill}{rgb}{0.000000,0.000000,0.000000}%
\pgfsetfillcolor{currentfill}%
\pgfsetlinewidth{0.803000pt}%
\definecolor{currentstroke}{rgb}{0.000000,0.000000,0.000000}%
\pgfsetstrokecolor{currentstroke}%
\pgfsetdash{}{0pt}%
\pgfsys@defobject{currentmarker}{\pgfqpoint{0.000000in}{-0.048611in}}{\pgfqpoint{0.000000in}{0.000000in}}{%
\pgfpathmoveto{\pgfqpoint{0.000000in}{0.000000in}}%
\pgfpathlineto{\pgfqpoint{0.000000in}{-0.048611in}}%
\pgfusepath{stroke,fill}%
}%
\begin{pgfscope}%
\pgfsys@transformshift{8.608664in}{0.331635in}%
\pgfsys@useobject{currentmarker}{}%
\end{pgfscope}%
\end{pgfscope}%
\begin{pgfscope}%
\definecolor{textcolor}{rgb}{0.000000,0.000000,0.000000}%
\pgfsetstrokecolor{textcolor}%
\pgfsetfillcolor{textcolor}%
\pgftext[x=8.608664in,y=0.234413in,,top]{\color{textcolor}\sffamily\fontsize{10.000000}{12.000000}\selectfont 75}%
\end{pgfscope}%
\begin{pgfscope}%
\pgfsetbuttcap%
\pgfsetroundjoin%
\definecolor{currentfill}{rgb}{0.000000,0.000000,0.000000}%
\pgfsetfillcolor{currentfill}%
\pgfsetlinewidth{0.803000pt}%
\definecolor{currentstroke}{rgb}{0.000000,0.000000,0.000000}%
\pgfsetstrokecolor{currentstroke}%
\pgfsetdash{}{0pt}%
\pgfsys@defobject{currentmarker}{\pgfqpoint{0.000000in}{-0.048611in}}{\pgfqpoint{0.000000in}{0.000000in}}{%
\pgfpathmoveto{\pgfqpoint{0.000000in}{0.000000in}}%
\pgfpathlineto{\pgfqpoint{0.000000in}{-0.048611in}}%
\pgfusepath{stroke,fill}%
}%
\begin{pgfscope}%
\pgfsys@transformshift{9.662388in}{0.331635in}%
\pgfsys@useobject{currentmarker}{}%
\end{pgfscope}%
\end{pgfscope}%
\begin{pgfscope}%
\definecolor{textcolor}{rgb}{0.000000,0.000000,0.000000}%
\pgfsetstrokecolor{textcolor}%
\pgfsetfillcolor{textcolor}%
\pgftext[x=9.662388in,y=0.234413in,,top]{\color{textcolor}\sffamily\fontsize{10.000000}{12.000000}\selectfont 100}%
\end{pgfscope}%
\begin{pgfscope}%
\pgfsetbuttcap%
\pgfsetroundjoin%
\definecolor{currentfill}{rgb}{0.000000,0.000000,0.000000}%
\pgfsetfillcolor{currentfill}%
\pgfsetlinewidth{0.803000pt}%
\definecolor{currentstroke}{rgb}{0.000000,0.000000,0.000000}%
\pgfsetstrokecolor{currentstroke}%
\pgfsetdash{}{0pt}%
\pgfsys@defobject{currentmarker}{\pgfqpoint{-0.048611in}{0.000000in}}{\pgfqpoint{-0.000000in}{0.000000in}}{%
\pgfpathmoveto{\pgfqpoint{-0.000000in}{0.000000in}}%
\pgfpathlineto{\pgfqpoint{-0.048611in}{0.000000in}}%
\pgfusepath{stroke,fill}%
}%
\begin{pgfscope}%
\pgfsys@transformshift{0.481978in}{0.363094in}%
\pgfsys@useobject{currentmarker}{}%
\end{pgfscope}%
\end{pgfscope}%
\begin{pgfscope}%
\definecolor{textcolor}{rgb}{0.000000,0.000000,0.000000}%
\pgfsetstrokecolor{textcolor}%
\pgfsetfillcolor{textcolor}%
\pgftext[x=0.100000in, y=0.310333in, left, base]{\color{textcolor}\sffamily\fontsize{10.000000}{12.000000}\selectfont \ensuremath{-}80}%
\end{pgfscope}%
\begin{pgfscope}%
\pgfsetbuttcap%
\pgfsetroundjoin%
\definecolor{currentfill}{rgb}{0.000000,0.000000,0.000000}%
\pgfsetfillcolor{currentfill}%
\pgfsetlinewidth{0.803000pt}%
\definecolor{currentstroke}{rgb}{0.000000,0.000000,0.000000}%
\pgfsetstrokecolor{currentstroke}%
\pgfsetdash{}{0pt}%
\pgfsys@defobject{currentmarker}{\pgfqpoint{-0.048611in}{0.000000in}}{\pgfqpoint{-0.000000in}{0.000000in}}{%
\pgfpathmoveto{\pgfqpoint{-0.000000in}{0.000000in}}%
\pgfpathlineto{\pgfqpoint{-0.048611in}{0.000000in}}%
\pgfusepath{stroke,fill}%
}%
\begin{pgfscope}%
\pgfsys@transformshift{0.481978in}{1.398454in}%
\pgfsys@useobject{currentmarker}{}%
\end{pgfscope}%
\end{pgfscope}%
\begin{pgfscope}%
\definecolor{textcolor}{rgb}{0.000000,0.000000,0.000000}%
\pgfsetstrokecolor{textcolor}%
\pgfsetfillcolor{textcolor}%
\pgftext[x=0.100000in, y=1.345693in, left, base]{\color{textcolor}\sffamily\fontsize{10.000000}{12.000000}\selectfont \ensuremath{-}60}%
\end{pgfscope}%
\begin{pgfscope}%
\pgfsetbuttcap%
\pgfsetroundjoin%
\definecolor{currentfill}{rgb}{0.000000,0.000000,0.000000}%
\pgfsetfillcolor{currentfill}%
\pgfsetlinewidth{0.803000pt}%
\definecolor{currentstroke}{rgb}{0.000000,0.000000,0.000000}%
\pgfsetstrokecolor{currentstroke}%
\pgfsetdash{}{0pt}%
\pgfsys@defobject{currentmarker}{\pgfqpoint{-0.048611in}{0.000000in}}{\pgfqpoint{-0.000000in}{0.000000in}}{%
\pgfpathmoveto{\pgfqpoint{-0.000000in}{0.000000in}}%
\pgfpathlineto{\pgfqpoint{-0.048611in}{0.000000in}}%
\pgfusepath{stroke,fill}%
}%
\begin{pgfscope}%
\pgfsys@transformshift{0.481978in}{2.433815in}%
\pgfsys@useobject{currentmarker}{}%
\end{pgfscope}%
\end{pgfscope}%
\begin{pgfscope}%
\definecolor{textcolor}{rgb}{0.000000,0.000000,0.000000}%
\pgfsetstrokecolor{textcolor}%
\pgfsetfillcolor{textcolor}%
\pgftext[x=0.100000in, y=2.381053in, left, base]{\color{textcolor}\sffamily\fontsize{10.000000}{12.000000}\selectfont \ensuremath{-}40}%
\end{pgfscope}%
\begin{pgfscope}%
\pgfsetbuttcap%
\pgfsetroundjoin%
\definecolor{currentfill}{rgb}{0.000000,0.000000,0.000000}%
\pgfsetfillcolor{currentfill}%
\pgfsetlinewidth{0.803000pt}%
\definecolor{currentstroke}{rgb}{0.000000,0.000000,0.000000}%
\pgfsetstrokecolor{currentstroke}%
\pgfsetdash{}{0pt}%
\pgfsys@defobject{currentmarker}{\pgfqpoint{-0.048611in}{0.000000in}}{\pgfqpoint{-0.000000in}{0.000000in}}{%
\pgfpathmoveto{\pgfqpoint{-0.000000in}{0.000000in}}%
\pgfpathlineto{\pgfqpoint{-0.048611in}{0.000000in}}%
\pgfusepath{stroke,fill}%
}%
\begin{pgfscope}%
\pgfsys@transformshift{0.481978in}{3.469175in}%
\pgfsys@useobject{currentmarker}{}%
\end{pgfscope}%
\end{pgfscope}%
\begin{pgfscope}%
\definecolor{textcolor}{rgb}{0.000000,0.000000,0.000000}%
\pgfsetstrokecolor{textcolor}%
\pgfsetfillcolor{textcolor}%
\pgftext[x=0.100000in, y=3.416413in, left, base]{\color{textcolor}\sffamily\fontsize{10.000000}{12.000000}\selectfont \ensuremath{-}20}%
\end{pgfscope}%
\begin{pgfscope}%
\pgfsetbuttcap%
\pgfsetroundjoin%
\definecolor{currentfill}{rgb}{0.000000,0.000000,0.000000}%
\pgfsetfillcolor{currentfill}%
\pgfsetlinewidth{0.803000pt}%
\definecolor{currentstroke}{rgb}{0.000000,0.000000,0.000000}%
\pgfsetstrokecolor{currentstroke}%
\pgfsetdash{}{0pt}%
\pgfsys@defobject{currentmarker}{\pgfqpoint{-0.048611in}{0.000000in}}{\pgfqpoint{-0.000000in}{0.000000in}}{%
\pgfpathmoveto{\pgfqpoint{-0.000000in}{0.000000in}}%
\pgfpathlineto{\pgfqpoint{-0.048611in}{0.000000in}}%
\pgfusepath{stroke,fill}%
}%
\begin{pgfscope}%
\pgfsys@transformshift{0.481978in}{4.504535in}%
\pgfsys@useobject{currentmarker}{}%
\end{pgfscope}%
\end{pgfscope}%
\begin{pgfscope}%
\definecolor{textcolor}{rgb}{0.000000,0.000000,0.000000}%
\pgfsetstrokecolor{textcolor}%
\pgfsetfillcolor{textcolor}%
\pgftext[x=0.296390in, y=4.451774in, left, base]{\color{textcolor}\sffamily\fontsize{10.000000}{12.000000}\selectfont 0}%
\end{pgfscope}%
\begin{pgfscope}%
\pgfsetbuttcap%
\pgfsetroundjoin%
\definecolor{currentfill}{rgb}{0.000000,0.000000,0.000000}%
\pgfsetfillcolor{currentfill}%
\pgfsetlinewidth{0.803000pt}%
\definecolor{currentstroke}{rgb}{0.000000,0.000000,0.000000}%
\pgfsetstrokecolor{currentstroke}%
\pgfsetdash{}{0pt}%
\pgfsys@defobject{currentmarker}{\pgfqpoint{-0.048611in}{0.000000in}}{\pgfqpoint{-0.000000in}{0.000000in}}{%
\pgfpathmoveto{\pgfqpoint{-0.000000in}{0.000000in}}%
\pgfpathlineto{\pgfqpoint{-0.048611in}{0.000000in}}%
\pgfusepath{stroke,fill}%
}%
\begin{pgfscope}%
\pgfsys@transformshift{0.481978in}{5.539895in}%
\pgfsys@useobject{currentmarker}{}%
\end{pgfscope}%
\end{pgfscope}%
\begin{pgfscope}%
\definecolor{textcolor}{rgb}{0.000000,0.000000,0.000000}%
\pgfsetstrokecolor{textcolor}%
\pgfsetfillcolor{textcolor}%
\pgftext[x=0.208025in, y=5.487134in, left, base]{\color{textcolor}\sffamily\fontsize{10.000000}{12.000000}\selectfont 20}%
\end{pgfscope}%
\begin{pgfscope}%
\pgfsetbuttcap%
\pgfsetroundjoin%
\definecolor{currentfill}{rgb}{0.000000,0.000000,0.000000}%
\pgfsetfillcolor{currentfill}%
\pgfsetlinewidth{0.803000pt}%
\definecolor{currentstroke}{rgb}{0.000000,0.000000,0.000000}%
\pgfsetstrokecolor{currentstroke}%
\pgfsetdash{}{0pt}%
\pgfsys@defobject{currentmarker}{\pgfqpoint{-0.048611in}{0.000000in}}{\pgfqpoint{-0.000000in}{0.000000in}}{%
\pgfpathmoveto{\pgfqpoint{-0.000000in}{0.000000in}}%
\pgfpathlineto{\pgfqpoint{-0.048611in}{0.000000in}}%
\pgfusepath{stroke,fill}%
}%
\begin{pgfscope}%
\pgfsys@transformshift{0.481978in}{6.575256in}%
\pgfsys@useobject{currentmarker}{}%
\end{pgfscope}%
\end{pgfscope}%
\begin{pgfscope}%
\definecolor{textcolor}{rgb}{0.000000,0.000000,0.000000}%
\pgfsetstrokecolor{textcolor}%
\pgfsetfillcolor{textcolor}%
\pgftext[x=0.208025in, y=6.522494in, left, base]{\color{textcolor}\sffamily\fontsize{10.000000}{12.000000}\selectfont 40}%
\end{pgfscope}%
\begin{pgfscope}%
\pgfsetbuttcap%
\pgfsetroundjoin%
\definecolor{currentfill}{rgb}{0.000000,0.000000,0.000000}%
\pgfsetfillcolor{currentfill}%
\pgfsetlinewidth{0.803000pt}%
\definecolor{currentstroke}{rgb}{0.000000,0.000000,0.000000}%
\pgfsetstrokecolor{currentstroke}%
\pgfsetdash{}{0pt}%
\pgfsys@defobject{currentmarker}{\pgfqpoint{-0.048611in}{0.000000in}}{\pgfqpoint{-0.000000in}{0.000000in}}{%
\pgfpathmoveto{\pgfqpoint{-0.000000in}{0.000000in}}%
\pgfpathlineto{\pgfqpoint{-0.048611in}{0.000000in}}%
\pgfusepath{stroke,fill}%
}%
\begin{pgfscope}%
\pgfsys@transformshift{0.481978in}{7.610616in}%
\pgfsys@useobject{currentmarker}{}%
\end{pgfscope}%
\end{pgfscope}%
\begin{pgfscope}%
\definecolor{textcolor}{rgb}{0.000000,0.000000,0.000000}%
\pgfsetstrokecolor{textcolor}%
\pgfsetfillcolor{textcolor}%
\pgftext[x=0.208025in, y=7.557854in, left, base]{\color{textcolor}\sffamily\fontsize{10.000000}{12.000000}\selectfont 60}%
\end{pgfscope}%
\begin{pgfscope}%
\pgfpathrectangle{\pgfqpoint{0.481978in}{0.331635in}}{\pgfqpoint{9.300000in}{7.700000in}}%
\pgfusepath{clip}%
\pgfsetrectcap%
\pgfsetroundjoin%
\pgfsetlinewidth{1.505625pt}%
\definecolor{currentstroke}{rgb}{0.631373,0.788235,0.956863}%
\pgfsetstrokecolor{currentstroke}%
\pgfsetstrokeopacity{0.800000}%
\pgfsetdash{}{0pt}%
\pgfpathmoveto{\pgfqpoint{5.968461in}{3.683953in}}%
\pgfpathlineto{\pgfqpoint{6.060728in}{3.871009in}}%
\pgfusepath{stroke}%
\end{pgfscope}%
\begin{pgfscope}%
\pgfpathrectangle{\pgfqpoint{0.481978in}{0.331635in}}{\pgfqpoint{9.300000in}{7.700000in}}%
\pgfusepath{clip}%
\pgfsetrectcap%
\pgfsetroundjoin%
\pgfsetlinewidth{1.505625pt}%
\definecolor{currentstroke}{rgb}{0.631373,0.788235,0.956863}%
\pgfsetstrokecolor{currentstroke}%
\pgfsetstrokeopacity{0.800000}%
\pgfsetdash{}{0pt}%
\pgfpathmoveto{\pgfqpoint{6.716760in}{0.681635in}}%
\pgfpathlineto{\pgfqpoint{6.060728in}{3.871009in}}%
\pgfusepath{stroke}%
\end{pgfscope}%
\begin{pgfscope}%
\pgfpathrectangle{\pgfqpoint{0.481978in}{0.331635in}}{\pgfqpoint{9.300000in}{7.700000in}}%
\pgfusepath{clip}%
\pgfsetrectcap%
\pgfsetroundjoin%
\pgfsetlinewidth{1.505625pt}%
\definecolor{currentstroke}{rgb}{0.631373,0.788235,0.956863}%
\pgfsetstrokecolor{currentstroke}%
\pgfsetstrokeopacity{0.800000}%
\pgfsetdash{}{0pt}%
\pgfpathmoveto{\pgfqpoint{4.030414in}{4.029607in}}%
\pgfpathlineto{\pgfqpoint{6.060728in}{3.871009in}}%
\pgfusepath{stroke}%
\end{pgfscope}%
\begin{pgfscope}%
\pgfpathrectangle{\pgfqpoint{0.481978in}{0.331635in}}{\pgfqpoint{9.300000in}{7.700000in}}%
\pgfusepath{clip}%
\pgfsetrectcap%
\pgfsetroundjoin%
\pgfsetlinewidth{1.505625pt}%
\definecolor{currentstroke}{rgb}{0.631373,0.788235,0.956863}%
\pgfsetstrokecolor{currentstroke}%
\pgfsetstrokeopacity{0.800000}%
\pgfsetdash{}{0pt}%
\pgfpathmoveto{\pgfqpoint{3.807365in}{2.342544in}}%
\pgfpathlineto{\pgfqpoint{6.060728in}{3.871009in}}%
\pgfusepath{stroke}%
\end{pgfscope}%
\begin{pgfscope}%
\pgfpathrectangle{\pgfqpoint{0.481978in}{0.331635in}}{\pgfqpoint{9.300000in}{7.700000in}}%
\pgfusepath{clip}%
\pgfsetrectcap%
\pgfsetroundjoin%
\pgfsetlinewidth{1.505625pt}%
\definecolor{currentstroke}{rgb}{0.631373,0.788235,0.956863}%
\pgfsetstrokecolor{currentstroke}%
\pgfsetstrokeopacity{0.800000}%
\pgfsetdash{}{0pt}%
\pgfpathmoveto{\pgfqpoint{2.350199in}{2.855754in}}%
\pgfpathlineto{\pgfqpoint{6.060728in}{3.871009in}}%
\pgfusepath{stroke}%
\end{pgfscope}%
\begin{pgfscope}%
\pgfpathrectangle{\pgfqpoint{0.481978in}{0.331635in}}{\pgfqpoint{9.300000in}{7.700000in}}%
\pgfusepath{clip}%
\pgfsetrectcap%
\pgfsetroundjoin%
\pgfsetlinewidth{1.505625pt}%
\definecolor{currentstroke}{rgb}{0.631373,0.788235,0.956863}%
\pgfsetstrokecolor{currentstroke}%
\pgfsetstrokeopacity{0.800000}%
\pgfsetdash{}{0pt}%
\pgfpathmoveto{\pgfqpoint{5.589222in}{5.972624in}}%
\pgfpathlineto{\pgfqpoint{6.060728in}{3.871009in}}%
\pgfusepath{stroke}%
\end{pgfscope}%
\begin{pgfscope}%
\pgfpathrectangle{\pgfqpoint{0.481978in}{0.331635in}}{\pgfqpoint{9.300000in}{7.700000in}}%
\pgfusepath{clip}%
\pgfsetrectcap%
\pgfsetroundjoin%
\pgfsetlinewidth{1.505625pt}%
\definecolor{currentstroke}{rgb}{0.631373,0.788235,0.956863}%
\pgfsetstrokecolor{currentstroke}%
\pgfsetstrokeopacity{0.800000}%
\pgfsetdash{}{0pt}%
\pgfpathmoveto{\pgfqpoint{9.359251in}{4.445797in}}%
\pgfpathlineto{\pgfqpoint{6.060728in}{3.871009in}}%
\pgfusepath{stroke}%
\end{pgfscope}%
\begin{pgfscope}%
\pgfpathrectangle{\pgfqpoint{0.481978in}{0.331635in}}{\pgfqpoint{9.300000in}{7.700000in}}%
\pgfusepath{clip}%
\pgfsetrectcap%
\pgfsetroundjoin%
\pgfsetlinewidth{1.505625pt}%
\definecolor{currentstroke}{rgb}{0.631373,0.788235,0.956863}%
\pgfsetstrokecolor{currentstroke}%
\pgfsetstrokeopacity{0.800000}%
\pgfsetdash{}{0pt}%
\pgfpathmoveto{\pgfqpoint{5.412406in}{5.087405in}}%
\pgfpathlineto{\pgfqpoint{6.060728in}{3.871009in}}%
\pgfusepath{stroke}%
\end{pgfscope}%
\begin{pgfscope}%
\pgfpathrectangle{\pgfqpoint{0.481978in}{0.331635in}}{\pgfqpoint{9.300000in}{7.700000in}}%
\pgfusepath{clip}%
\pgfsetrectcap%
\pgfsetroundjoin%
\pgfsetlinewidth{1.505625pt}%
\definecolor{currentstroke}{rgb}{0.631373,0.788235,0.956863}%
\pgfsetstrokecolor{currentstroke}%
\pgfsetstrokeopacity{0.800000}%
\pgfsetdash{}{0pt}%
\pgfpathmoveto{\pgfqpoint{3.390760in}{6.272762in}}%
\pgfpathlineto{\pgfqpoint{6.060728in}{3.871009in}}%
\pgfusepath{stroke}%
\end{pgfscope}%
\begin{pgfscope}%
\pgfpathrectangle{\pgfqpoint{0.481978in}{0.331635in}}{\pgfqpoint{9.300000in}{7.700000in}}%
\pgfusepath{clip}%
\pgfsetrectcap%
\pgfsetroundjoin%
\pgfsetlinewidth{1.505625pt}%
\definecolor{currentstroke}{rgb}{0.631373,0.788235,0.956863}%
\pgfsetstrokecolor{currentstroke}%
\pgfsetstrokeopacity{0.800000}%
\pgfsetdash{}{0pt}%
\pgfpathmoveto{\pgfqpoint{7.065739in}{4.693543in}}%
\pgfpathlineto{\pgfqpoint{6.060728in}{3.871009in}}%
\pgfusepath{stroke}%
\end{pgfscope}%
\begin{pgfscope}%
\pgfpathrectangle{\pgfqpoint{0.481978in}{0.331635in}}{\pgfqpoint{9.300000in}{7.700000in}}%
\pgfusepath{clip}%
\pgfsetrectcap%
\pgfsetroundjoin%
\pgfsetlinewidth{1.505625pt}%
\definecolor{currentstroke}{rgb}{0.631373,0.788235,0.956863}%
\pgfsetstrokecolor{currentstroke}%
\pgfsetstrokeopacity{0.800000}%
\pgfsetdash{}{0pt}%
\pgfpathmoveto{\pgfqpoint{6.796596in}{5.867778in}}%
\pgfpathlineto{\pgfqpoint{6.060728in}{3.871009in}}%
\pgfusepath{stroke}%
\end{pgfscope}%
\begin{pgfscope}%
\pgfpathrectangle{\pgfqpoint{0.481978in}{0.331635in}}{\pgfqpoint{9.300000in}{7.700000in}}%
\pgfusepath{clip}%
\pgfsetrectcap%
\pgfsetroundjoin%
\pgfsetlinewidth{1.505625pt}%
\definecolor{currentstroke}{rgb}{0.631373,0.788235,0.956863}%
\pgfsetstrokecolor{currentstroke}%
\pgfsetstrokeopacity{0.800000}%
\pgfsetdash{}{0pt}%
\pgfpathmoveto{\pgfqpoint{6.170106in}{5.312882in}}%
\pgfpathlineto{\pgfqpoint{6.060728in}{3.871009in}}%
\pgfusepath{stroke}%
\end{pgfscope}%
\begin{pgfscope}%
\pgfpathrectangle{\pgfqpoint{0.481978in}{0.331635in}}{\pgfqpoint{9.300000in}{7.700000in}}%
\pgfusepath{clip}%
\pgfsetrectcap%
\pgfsetroundjoin%
\pgfsetlinewidth{1.505625pt}%
\definecolor{currentstroke}{rgb}{0.631373,0.788235,0.956863}%
\pgfsetstrokecolor{currentstroke}%
\pgfsetstrokeopacity{0.800000}%
\pgfsetdash{}{0pt}%
\pgfpathmoveto{\pgfqpoint{7.510480in}{3.074835in}}%
\pgfpathlineto{\pgfqpoint{6.060728in}{3.871009in}}%
\pgfusepath{stroke}%
\end{pgfscope}%
\begin{pgfscope}%
\pgfpathrectangle{\pgfqpoint{0.481978in}{0.331635in}}{\pgfqpoint{9.300000in}{7.700000in}}%
\pgfusepath{clip}%
\pgfsetrectcap%
\pgfsetroundjoin%
\pgfsetlinewidth{1.505625pt}%
\definecolor{currentstroke}{rgb}{0.631373,0.788235,0.956863}%
\pgfsetstrokecolor{currentstroke}%
\pgfsetstrokeopacity{0.800000}%
\pgfsetdash{}{0pt}%
\pgfpathmoveto{\pgfqpoint{6.540018in}{1.815747in}}%
\pgfpathlineto{\pgfqpoint{6.060728in}{3.871009in}}%
\pgfusepath{stroke}%
\end{pgfscope}%
\begin{pgfscope}%
\pgfpathrectangle{\pgfqpoint{0.481978in}{0.331635in}}{\pgfqpoint{9.300000in}{7.700000in}}%
\pgfusepath{clip}%
\pgfsetrectcap%
\pgfsetroundjoin%
\pgfsetlinewidth{1.505625pt}%
\definecolor{currentstroke}{rgb}{0.631373,0.788235,0.956863}%
\pgfsetstrokecolor{currentstroke}%
\pgfsetstrokeopacity{0.800000}%
\pgfsetdash{}{0pt}%
\pgfpathmoveto{\pgfqpoint{6.818751in}{2.808587in}}%
\pgfpathlineto{\pgfqpoint{6.060728in}{3.871009in}}%
\pgfusepath{stroke}%
\end{pgfscope}%
\begin{pgfscope}%
\pgfpathrectangle{\pgfqpoint{0.481978in}{0.331635in}}{\pgfqpoint{9.300000in}{7.700000in}}%
\pgfusepath{clip}%
\pgfsetrectcap%
\pgfsetroundjoin%
\pgfsetlinewidth{1.505625pt}%
\definecolor{currentstroke}{rgb}{0.631373,0.788235,0.956863}%
\pgfsetstrokecolor{currentstroke}%
\pgfsetstrokeopacity{0.800000}%
\pgfsetdash{}{0pt}%
\pgfpathmoveto{\pgfqpoint{4.820436in}{6.238215in}}%
\pgfpathlineto{\pgfqpoint{6.060728in}{3.871009in}}%
\pgfusepath{stroke}%
\end{pgfscope}%
\begin{pgfscope}%
\pgfpathrectangle{\pgfqpoint{0.481978in}{0.331635in}}{\pgfqpoint{9.300000in}{7.700000in}}%
\pgfusepath{clip}%
\pgfsetrectcap%
\pgfsetroundjoin%
\pgfsetlinewidth{1.505625pt}%
\definecolor{currentstroke}{rgb}{0.631373,0.788235,0.956863}%
\pgfsetstrokecolor{currentstroke}%
\pgfsetstrokeopacity{0.800000}%
\pgfsetdash{}{0pt}%
\pgfpathmoveto{\pgfqpoint{5.842861in}{4.469204in}}%
\pgfpathlineto{\pgfqpoint{6.060728in}{3.871009in}}%
\pgfusepath{stroke}%
\end{pgfscope}%
\begin{pgfscope}%
\pgfpathrectangle{\pgfqpoint{0.481978in}{0.331635in}}{\pgfqpoint{9.300000in}{7.700000in}}%
\pgfusepath{clip}%
\pgfsetrectcap%
\pgfsetroundjoin%
\pgfsetlinewidth{1.505625pt}%
\definecolor{currentstroke}{rgb}{0.631373,0.788235,0.956863}%
\pgfsetstrokecolor{currentstroke}%
\pgfsetstrokeopacity{0.800000}%
\pgfsetdash{}{0pt}%
\pgfpathmoveto{\pgfqpoint{6.322821in}{6.824794in}}%
\pgfpathlineto{\pgfqpoint{6.060728in}{3.871009in}}%
\pgfusepath{stroke}%
\end{pgfscope}%
\begin{pgfscope}%
\pgfpathrectangle{\pgfqpoint{0.481978in}{0.331635in}}{\pgfqpoint{9.300000in}{7.700000in}}%
\pgfusepath{clip}%
\pgfsetrectcap%
\pgfsetroundjoin%
\pgfsetlinewidth{1.505625pt}%
\definecolor{currentstroke}{rgb}{0.631373,0.788235,0.956863}%
\pgfsetstrokecolor{currentstroke}%
\pgfsetstrokeopacity{0.800000}%
\pgfsetdash{}{0pt}%
\pgfpathmoveto{\pgfqpoint{7.830146in}{1.021290in}}%
\pgfpathlineto{\pgfqpoint{6.060728in}{3.871009in}}%
\pgfusepath{stroke}%
\end{pgfscope}%
\begin{pgfscope}%
\pgfpathrectangle{\pgfqpoint{0.481978in}{0.331635in}}{\pgfqpoint{9.300000in}{7.700000in}}%
\pgfusepath{clip}%
\pgfsetrectcap%
\pgfsetroundjoin%
\pgfsetlinewidth{1.505625pt}%
\definecolor{currentstroke}{rgb}{0.631373,0.788235,0.956863}%
\pgfsetstrokecolor{currentstroke}%
\pgfsetstrokeopacity{0.800000}%
\pgfsetdash{}{0pt}%
\pgfpathmoveto{\pgfqpoint{5.762941in}{2.466744in}}%
\pgfpathlineto{\pgfqpoint{6.060728in}{3.871009in}}%
\pgfusepath{stroke}%
\end{pgfscope}%
\begin{pgfscope}%
\pgfpathrectangle{\pgfqpoint{0.481978in}{0.331635in}}{\pgfqpoint{9.300000in}{7.700000in}}%
\pgfusepath{clip}%
\pgfsetrectcap%
\pgfsetroundjoin%
\pgfsetlinewidth{1.505625pt}%
\definecolor{currentstroke}{rgb}{0.631373,0.788235,0.956863}%
\pgfsetstrokecolor{currentstroke}%
\pgfsetstrokeopacity{0.800000}%
\pgfsetdash{}{0pt}%
\pgfpathmoveto{\pgfqpoint{6.414379in}{4.497968in}}%
\pgfpathlineto{\pgfqpoint{6.060728in}{3.871009in}}%
\pgfusepath{stroke}%
\end{pgfscope}%
\begin{pgfscope}%
\pgfpathrectangle{\pgfqpoint{0.481978in}{0.331635in}}{\pgfqpoint{9.300000in}{7.700000in}}%
\pgfusepath{clip}%
\pgfsetrectcap%
\pgfsetroundjoin%
\pgfsetlinewidth{1.505625pt}%
\definecolor{currentstroke}{rgb}{0.631373,0.788235,0.956863}%
\pgfsetstrokecolor{currentstroke}%
\pgfsetstrokeopacity{0.800000}%
\pgfsetdash{}{0pt}%
\pgfpathmoveto{\pgfqpoint{5.330843in}{3.151599in}}%
\pgfpathlineto{\pgfqpoint{6.060728in}{3.871009in}}%
\pgfusepath{stroke}%
\end{pgfscope}%
\begin{pgfscope}%
\pgfpathrectangle{\pgfqpoint{0.481978in}{0.331635in}}{\pgfqpoint{9.300000in}{7.700000in}}%
\pgfusepath{clip}%
\pgfsetrectcap%
\pgfsetroundjoin%
\pgfsetlinewidth{1.505625pt}%
\definecolor{currentstroke}{rgb}{0.631373,0.788235,0.956863}%
\pgfsetstrokecolor{currentstroke}%
\pgfsetstrokeopacity{0.800000}%
\pgfsetdash{}{0pt}%
\pgfpathmoveto{\pgfqpoint{5.723444in}{1.133310in}}%
\pgfpathlineto{\pgfqpoint{6.060728in}{3.871009in}}%
\pgfusepath{stroke}%
\end{pgfscope}%
\begin{pgfscope}%
\pgfpathrectangle{\pgfqpoint{0.481978in}{0.331635in}}{\pgfqpoint{9.300000in}{7.700000in}}%
\pgfusepath{clip}%
\pgfsetrectcap%
\pgfsetroundjoin%
\pgfsetlinewidth{1.505625pt}%
\definecolor{currentstroke}{rgb}{0.631373,0.788235,0.956863}%
\pgfsetstrokecolor{currentstroke}%
\pgfsetstrokeopacity{0.800000}%
\pgfsetdash{}{0pt}%
\pgfpathmoveto{\pgfqpoint{3.540419in}{3.366356in}}%
\pgfpathlineto{\pgfqpoint{6.060728in}{3.871009in}}%
\pgfusepath{stroke}%
\end{pgfscope}%
\begin{pgfscope}%
\pgfpathrectangle{\pgfqpoint{0.481978in}{0.331635in}}{\pgfqpoint{9.300000in}{7.700000in}}%
\pgfusepath{clip}%
\pgfsetrectcap%
\pgfsetroundjoin%
\pgfsetlinewidth{1.505625pt}%
\definecolor{currentstroke}{rgb}{0.631373,0.788235,0.956863}%
\pgfsetstrokecolor{currentstroke}%
\pgfsetstrokeopacity{0.800000}%
\pgfsetdash{}{0pt}%
\pgfpathmoveto{\pgfqpoint{7.439774in}{5.540750in}}%
\pgfpathlineto{\pgfqpoint{6.060728in}{3.871009in}}%
\pgfusepath{stroke}%
\end{pgfscope}%
\begin{pgfscope}%
\pgfpathrectangle{\pgfqpoint{0.481978in}{0.331635in}}{\pgfqpoint{9.300000in}{7.700000in}}%
\pgfusepath{clip}%
\pgfsetrectcap%
\pgfsetroundjoin%
\pgfsetlinewidth{1.505625pt}%
\definecolor{currentstroke}{rgb}{0.631373,0.788235,0.956863}%
\pgfsetstrokecolor{currentstroke}%
\pgfsetstrokeopacity{0.800000}%
\pgfsetdash{}{0pt}%
\pgfpathmoveto{\pgfqpoint{8.272345in}{2.164029in}}%
\pgfpathlineto{\pgfqpoint{6.060728in}{3.871009in}}%
\pgfusepath{stroke}%
\end{pgfscope}%
\begin{pgfscope}%
\pgfpathrectangle{\pgfqpoint{0.481978in}{0.331635in}}{\pgfqpoint{9.300000in}{7.700000in}}%
\pgfusepath{clip}%
\pgfsetrectcap%
\pgfsetroundjoin%
\pgfsetlinewidth{1.505625pt}%
\definecolor{currentstroke}{rgb}{0.631373,0.788235,0.956863}%
\pgfsetstrokecolor{currentstroke}%
\pgfsetstrokeopacity{0.800000}%
\pgfsetdash{}{0pt}%
\pgfpathmoveto{\pgfqpoint{7.470382in}{1.797485in}}%
\pgfpathlineto{\pgfqpoint{6.060728in}{3.871009in}}%
\pgfusepath{stroke}%
\end{pgfscope}%
\begin{pgfscope}%
\pgfpathrectangle{\pgfqpoint{0.481978in}{0.331635in}}{\pgfqpoint{9.300000in}{7.700000in}}%
\pgfusepath{clip}%
\pgfsetrectcap%
\pgfsetroundjoin%
\pgfsetlinewidth{1.505625pt}%
\definecolor{currentstroke}{rgb}{0.631373,0.788235,0.956863}%
\pgfsetstrokecolor{currentstroke}%
\pgfsetstrokeopacity{0.800000}%
\pgfsetdash{}{0pt}%
\pgfpathmoveto{\pgfqpoint{7.403075in}{6.771043in}}%
\pgfpathlineto{\pgfqpoint{6.060728in}{3.871009in}}%
\pgfusepath{stroke}%
\end{pgfscope}%
\begin{pgfscope}%
\pgfpathrectangle{\pgfqpoint{0.481978in}{0.331635in}}{\pgfqpoint{9.300000in}{7.700000in}}%
\pgfusepath{clip}%
\pgfsetrectcap%
\pgfsetroundjoin%
\pgfsetlinewidth{1.505625pt}%
\definecolor{currentstroke}{rgb}{1.000000,0.705882,0.509804}%
\pgfsetstrokecolor{currentstroke}%
\pgfsetstrokeopacity{0.800000}%
\pgfsetdash{}{0pt}%
\pgfpathmoveto{\pgfqpoint{4.737094in}{4.546006in}}%
\pgfpathlineto{\pgfqpoint{4.307614in}{4.391044in}}%
\pgfusepath{stroke}%
\end{pgfscope}%
\begin{pgfscope}%
\pgfpathrectangle{\pgfqpoint{0.481978in}{0.331635in}}{\pgfqpoint{9.300000in}{7.700000in}}%
\pgfusepath{clip}%
\pgfsetrectcap%
\pgfsetroundjoin%
\pgfsetlinewidth{1.505625pt}%
\definecolor{currentstroke}{rgb}{1.000000,0.705882,0.509804}%
\pgfsetstrokecolor{currentstroke}%
\pgfsetstrokeopacity{0.800000}%
\pgfsetdash{}{0pt}%
\pgfpathmoveto{\pgfqpoint{4.102620in}{7.195105in}}%
\pgfpathlineto{\pgfqpoint{4.307614in}{4.391044in}}%
\pgfusepath{stroke}%
\end{pgfscope}%
\begin{pgfscope}%
\pgfpathrectangle{\pgfqpoint{0.481978in}{0.331635in}}{\pgfqpoint{9.300000in}{7.700000in}}%
\pgfusepath{clip}%
\pgfsetrectcap%
\pgfsetroundjoin%
\pgfsetlinewidth{1.505625pt}%
\definecolor{currentstroke}{rgb}{1.000000,0.705882,0.509804}%
\pgfsetstrokecolor{currentstroke}%
\pgfsetstrokeopacity{0.800000}%
\pgfsetdash{}{0pt}%
\pgfpathmoveto{\pgfqpoint{3.960956in}{4.794147in}}%
\pgfpathlineto{\pgfqpoint{4.307614in}{4.391044in}}%
\pgfusepath{stroke}%
\end{pgfscope}%
\begin{pgfscope}%
\pgfpathrectangle{\pgfqpoint{0.481978in}{0.331635in}}{\pgfqpoint{9.300000in}{7.700000in}}%
\pgfusepath{clip}%
\pgfsetrectcap%
\pgfsetroundjoin%
\pgfsetlinewidth{1.505625pt}%
\definecolor{currentstroke}{rgb}{1.000000,0.705882,0.509804}%
\pgfsetstrokecolor{currentstroke}%
\pgfsetstrokeopacity{0.800000}%
\pgfsetdash{}{0pt}%
\pgfpathmoveto{\pgfqpoint{3.386233in}{5.184879in}}%
\pgfpathlineto{\pgfqpoint{4.307614in}{4.391044in}}%
\pgfusepath{stroke}%
\end{pgfscope}%
\begin{pgfscope}%
\pgfpathrectangle{\pgfqpoint{0.481978in}{0.331635in}}{\pgfqpoint{9.300000in}{7.700000in}}%
\pgfusepath{clip}%
\pgfsetrectcap%
\pgfsetroundjoin%
\pgfsetlinewidth{1.505625pt}%
\definecolor{currentstroke}{rgb}{1.000000,0.705882,0.509804}%
\pgfsetstrokecolor{currentstroke}%
\pgfsetstrokeopacity{0.800000}%
\pgfsetdash{}{0pt}%
\pgfpathmoveto{\pgfqpoint{2.506952in}{3.900738in}}%
\pgfpathlineto{\pgfqpoint{4.307614in}{4.391044in}}%
\pgfusepath{stroke}%
\end{pgfscope}%
\begin{pgfscope}%
\pgfpathrectangle{\pgfqpoint{0.481978in}{0.331635in}}{\pgfqpoint{9.300000in}{7.700000in}}%
\pgfusepath{clip}%
\pgfsetrectcap%
\pgfsetroundjoin%
\pgfsetlinewidth{1.505625pt}%
\definecolor{currentstroke}{rgb}{1.000000,0.705882,0.509804}%
\pgfsetstrokecolor{currentstroke}%
\pgfsetstrokeopacity{0.800000}%
\pgfsetdash{}{0pt}%
\pgfpathmoveto{\pgfqpoint{7.641446in}{4.026183in}}%
\pgfpathlineto{\pgfqpoint{4.307614in}{4.391044in}}%
\pgfusepath{stroke}%
\end{pgfscope}%
\begin{pgfscope}%
\pgfpathrectangle{\pgfqpoint{0.481978in}{0.331635in}}{\pgfqpoint{9.300000in}{7.700000in}}%
\pgfusepath{clip}%
\pgfsetrectcap%
\pgfsetroundjoin%
\pgfsetlinewidth{1.505625pt}%
\definecolor{currentstroke}{rgb}{1.000000,0.705882,0.509804}%
\pgfsetstrokecolor{currentstroke}%
\pgfsetstrokeopacity{0.800000}%
\pgfsetdash{}{0pt}%
\pgfpathmoveto{\pgfqpoint{2.626934in}{4.769158in}}%
\pgfpathlineto{\pgfqpoint{4.307614in}{4.391044in}}%
\pgfusepath{stroke}%
\end{pgfscope}%
\begin{pgfscope}%
\pgfpathrectangle{\pgfqpoint{0.481978in}{0.331635in}}{\pgfqpoint{9.300000in}{7.700000in}}%
\pgfusepath{clip}%
\pgfsetrectcap%
\pgfsetroundjoin%
\pgfsetlinewidth{1.505625pt}%
\definecolor{currentstroke}{rgb}{1.000000,0.705882,0.509804}%
\pgfsetstrokecolor{currentstroke}%
\pgfsetstrokeopacity{0.800000}%
\pgfsetdash{}{0pt}%
\pgfpathmoveto{\pgfqpoint{4.292588in}{1.004552in}}%
\pgfpathlineto{\pgfqpoint{4.307614in}{4.391044in}}%
\pgfusepath{stroke}%
\end{pgfscope}%
\begin{pgfscope}%
\pgfpathrectangle{\pgfqpoint{0.481978in}{0.331635in}}{\pgfqpoint{9.300000in}{7.700000in}}%
\pgfusepath{clip}%
\pgfsetrectcap%
\pgfsetroundjoin%
\pgfsetlinewidth{1.505625pt}%
\definecolor{currentstroke}{rgb}{1.000000,0.705882,0.509804}%
\pgfsetstrokecolor{currentstroke}%
\pgfsetstrokeopacity{0.800000}%
\pgfsetdash{}{0pt}%
\pgfpathmoveto{\pgfqpoint{4.914941in}{2.251106in}}%
\pgfpathlineto{\pgfqpoint{4.307614in}{4.391044in}}%
\pgfusepath{stroke}%
\end{pgfscope}%
\begin{pgfscope}%
\pgfpathrectangle{\pgfqpoint{0.481978in}{0.331635in}}{\pgfqpoint{9.300000in}{7.700000in}}%
\pgfusepath{clip}%
\pgfsetrectcap%
\pgfsetroundjoin%
\pgfsetlinewidth{1.505625pt}%
\definecolor{currentstroke}{rgb}{1.000000,0.705882,0.509804}%
\pgfsetstrokecolor{currentstroke}%
\pgfsetstrokeopacity{0.800000}%
\pgfsetdash{}{0pt}%
\pgfpathmoveto{\pgfqpoint{1.360419in}{3.230142in}}%
\pgfpathlineto{\pgfqpoint{4.307614in}{4.391044in}}%
\pgfusepath{stroke}%
\end{pgfscope}%
\begin{pgfscope}%
\pgfpathrectangle{\pgfqpoint{0.481978in}{0.331635in}}{\pgfqpoint{9.300000in}{7.700000in}}%
\pgfusepath{clip}%
\pgfsetrectcap%
\pgfsetroundjoin%
\pgfsetlinewidth{1.505625pt}%
\definecolor{currentstroke}{rgb}{1.000000,0.705882,0.509804}%
\pgfsetstrokecolor{currentstroke}%
\pgfsetstrokeopacity{0.800000}%
\pgfsetdash{}{0pt}%
\pgfpathmoveto{\pgfqpoint{2.680958in}{5.639274in}}%
\pgfpathlineto{\pgfqpoint{4.307614in}{4.391044in}}%
\pgfusepath{stroke}%
\end{pgfscope}%
\begin{pgfscope}%
\pgfpathrectangle{\pgfqpoint{0.481978in}{0.331635in}}{\pgfqpoint{9.300000in}{7.700000in}}%
\pgfusepath{clip}%
\pgfsetrectcap%
\pgfsetroundjoin%
\pgfsetlinewidth{1.505625pt}%
\definecolor{currentstroke}{rgb}{1.000000,0.705882,0.509804}%
\pgfsetstrokecolor{currentstroke}%
\pgfsetstrokeopacity{0.800000}%
\pgfsetdash{}{0pt}%
\pgfpathmoveto{\pgfqpoint{8.439887in}{6.176815in}}%
\pgfpathlineto{\pgfqpoint{4.307614in}{4.391044in}}%
\pgfusepath{stroke}%
\end{pgfscope}%
\begin{pgfscope}%
\pgfpathrectangle{\pgfqpoint{0.481978in}{0.331635in}}{\pgfqpoint{9.300000in}{7.700000in}}%
\pgfusepath{clip}%
\pgfsetrectcap%
\pgfsetroundjoin%
\pgfsetlinewidth{1.505625pt}%
\definecolor{currentstroke}{rgb}{1.000000,0.705882,0.509804}%
\pgfsetstrokecolor{currentstroke}%
\pgfsetstrokeopacity{0.800000}%
\pgfsetdash{}{0pt}%
\pgfpathmoveto{\pgfqpoint{5.343023in}{7.681635in}}%
\pgfpathlineto{\pgfqpoint{4.307614in}{4.391044in}}%
\pgfusepath{stroke}%
\end{pgfscope}%
\begin{pgfscope}%
\pgfpathrectangle{\pgfqpoint{0.481978in}{0.331635in}}{\pgfqpoint{9.300000in}{7.700000in}}%
\pgfusepath{clip}%
\pgfsetrectcap%
\pgfsetroundjoin%
\pgfsetlinewidth{1.505625pt}%
\definecolor{currentstroke}{rgb}{1.000000,0.705882,0.509804}%
\pgfsetstrokecolor{currentstroke}%
\pgfsetstrokeopacity{0.800000}%
\pgfsetdash{}{0pt}%
\pgfpathmoveto{\pgfqpoint{1.984154in}{4.169349in}}%
\pgfpathlineto{\pgfqpoint{4.307614in}{4.391044in}}%
\pgfusepath{stroke}%
\end{pgfscope}%
\begin{pgfscope}%
\pgfpathrectangle{\pgfqpoint{0.481978in}{0.331635in}}{\pgfqpoint{9.300000in}{7.700000in}}%
\pgfusepath{clip}%
\pgfsetrectcap%
\pgfsetroundjoin%
\pgfsetlinewidth{1.505625pt}%
\definecolor{currentstroke}{rgb}{1.000000,0.705882,0.509804}%
\pgfsetstrokecolor{currentstroke}%
\pgfsetstrokeopacity{0.800000}%
\pgfsetdash{}{0pt}%
\pgfpathmoveto{\pgfqpoint{8.035769in}{4.947906in}}%
\pgfpathlineto{\pgfqpoint{4.307614in}{4.391044in}}%
\pgfusepath{stroke}%
\end{pgfscope}%
\begin{pgfscope}%
\pgfpathrectangle{\pgfqpoint{0.481978in}{0.331635in}}{\pgfqpoint{9.300000in}{7.700000in}}%
\pgfusepath{clip}%
\pgfsetrectcap%
\pgfsetroundjoin%
\pgfsetlinewidth{1.505625pt}%
\definecolor{currentstroke}{rgb}{1.000000,0.705882,0.509804}%
\pgfsetstrokecolor{currentstroke}%
\pgfsetstrokeopacity{0.800000}%
\pgfsetdash{}{0pt}%
\pgfpathmoveto{\pgfqpoint{6.113857in}{3.306063in}}%
\pgfpathlineto{\pgfqpoint{4.307614in}{4.391044in}}%
\pgfusepath{stroke}%
\end{pgfscope}%
\begin{pgfscope}%
\pgfpathrectangle{\pgfqpoint{0.481978in}{0.331635in}}{\pgfqpoint{9.300000in}{7.700000in}}%
\pgfusepath{clip}%
\pgfsetrectcap%
\pgfsetroundjoin%
\pgfsetlinewidth{1.505625pt}%
\definecolor{currentstroke}{rgb}{1.000000,0.705882,0.509804}%
\pgfsetstrokecolor{currentstroke}%
\pgfsetstrokeopacity{0.800000}%
\pgfsetdash{}{0pt}%
\pgfpathmoveto{\pgfqpoint{5.267481in}{4.097480in}}%
\pgfpathlineto{\pgfqpoint{4.307614in}{4.391044in}}%
\pgfusepath{stroke}%
\end{pgfscope}%
\begin{pgfscope}%
\pgfpathrectangle{\pgfqpoint{0.481978in}{0.331635in}}{\pgfqpoint{9.300000in}{7.700000in}}%
\pgfusepath{clip}%
\pgfsetrectcap%
\pgfsetroundjoin%
\pgfsetlinewidth{1.505625pt}%
\definecolor{currentstroke}{rgb}{1.000000,0.705882,0.509804}%
\pgfsetstrokecolor{currentstroke}%
\pgfsetstrokeopacity{0.800000}%
\pgfsetdash{}{0pt}%
\pgfpathmoveto{\pgfqpoint{2.715921in}{1.871136in}}%
\pgfpathlineto{\pgfqpoint{4.307614in}{4.391044in}}%
\pgfusepath{stroke}%
\end{pgfscope}%
\begin{pgfscope}%
\pgfpathrectangle{\pgfqpoint{0.481978in}{0.331635in}}{\pgfqpoint{9.300000in}{7.700000in}}%
\pgfusepath{clip}%
\pgfsetrectcap%
\pgfsetroundjoin%
\pgfsetlinewidth{1.505625pt}%
\definecolor{currentstroke}{rgb}{1.000000,0.705882,0.509804}%
\pgfsetstrokecolor{currentstroke}%
\pgfsetstrokeopacity{0.800000}%
\pgfsetdash{}{0pt}%
\pgfpathmoveto{\pgfqpoint{0.904705in}{5.096309in}}%
\pgfpathlineto{\pgfqpoint{4.307614in}{4.391044in}}%
\pgfusepath{stroke}%
\end{pgfscope}%
\begin{pgfscope}%
\pgfpathrectangle{\pgfqpoint{0.481978in}{0.331635in}}{\pgfqpoint{9.300000in}{7.700000in}}%
\pgfusepath{clip}%
\pgfsetrectcap%
\pgfsetroundjoin%
\pgfsetlinewidth{1.505625pt}%
\definecolor{currentstroke}{rgb}{1.000000,0.705882,0.509804}%
\pgfsetstrokecolor{currentstroke}%
\pgfsetstrokeopacity{0.800000}%
\pgfsetdash{}{0pt}%
\pgfpathmoveto{\pgfqpoint{8.533110in}{3.582315in}}%
\pgfpathlineto{\pgfqpoint{4.307614in}{4.391044in}}%
\pgfusepath{stroke}%
\end{pgfscope}%
\begin{pgfscope}%
\pgfpathrectangle{\pgfqpoint{0.481978in}{0.331635in}}{\pgfqpoint{9.300000in}{7.700000in}}%
\pgfusepath{clip}%
\pgfsetrectcap%
\pgfsetroundjoin%
\pgfsetlinewidth{1.505625pt}%
\definecolor{currentstroke}{rgb}{1.000000,0.705882,0.509804}%
\pgfsetstrokecolor{currentstroke}%
\pgfsetstrokeopacity{0.800000}%
\pgfsetdash{}{0pt}%
\pgfpathmoveto{\pgfqpoint{4.391134in}{3.048940in}}%
\pgfpathlineto{\pgfqpoint{4.307614in}{4.391044in}}%
\pgfusepath{stroke}%
\end{pgfscope}%
\begin{pgfscope}%
\pgfpathrectangle{\pgfqpoint{0.481978in}{0.331635in}}{\pgfqpoint{9.300000in}{7.700000in}}%
\pgfusepath{clip}%
\pgfsetrectcap%
\pgfsetroundjoin%
\pgfsetlinewidth{1.505625pt}%
\definecolor{currentstroke}{rgb}{1.000000,0.705882,0.509804}%
\pgfsetstrokecolor{currentstroke}%
\pgfsetstrokeopacity{0.800000}%
\pgfsetdash{}{0pt}%
\pgfpathmoveto{\pgfqpoint{1.537381in}{5.337200in}}%
\pgfpathlineto{\pgfqpoint{4.307614in}{4.391044in}}%
\pgfusepath{stroke}%
\end{pgfscope}%
\begin{pgfscope}%
\pgfpathrectangle{\pgfqpoint{0.481978in}{0.331635in}}{\pgfqpoint{9.300000in}{7.700000in}}%
\pgfusepath{clip}%
\pgfsetrectcap%
\pgfsetroundjoin%
\pgfsetlinewidth{1.505625pt}%
\definecolor{currentstroke}{rgb}{1.000000,0.705882,0.509804}%
\pgfsetstrokecolor{currentstroke}%
\pgfsetstrokeopacity{0.800000}%
\pgfsetdash{}{0pt}%
\pgfpathmoveto{\pgfqpoint{6.879175in}{3.740228in}}%
\pgfpathlineto{\pgfqpoint{4.307614in}{4.391044in}}%
\pgfusepath{stroke}%
\end{pgfscope}%
\begin{pgfscope}%
\pgfpathrectangle{\pgfqpoint{0.481978in}{0.331635in}}{\pgfqpoint{9.300000in}{7.700000in}}%
\pgfusepath{clip}%
\pgfsetrectcap%
\pgfsetroundjoin%
\pgfsetlinewidth{1.505625pt}%
\definecolor{currentstroke}{rgb}{1.000000,0.705882,0.509804}%
\pgfsetstrokecolor{currentstroke}%
\pgfsetstrokeopacity{0.800000}%
\pgfsetdash{}{0pt}%
\pgfpathmoveto{\pgfqpoint{4.800335in}{3.722296in}}%
\pgfpathlineto{\pgfqpoint{4.307614in}{4.391044in}}%
\pgfusepath{stroke}%
\end{pgfscope}%
\begin{pgfscope}%
\pgfpathrectangle{\pgfqpoint{0.481978in}{0.331635in}}{\pgfqpoint{9.300000in}{7.700000in}}%
\pgfusepath{clip}%
\pgfsetrectcap%
\pgfsetroundjoin%
\pgfsetlinewidth{1.505625pt}%
\definecolor{currentstroke}{rgb}{1.000000,0.705882,0.509804}%
\pgfsetstrokecolor{currentstroke}%
\pgfsetstrokeopacity{0.800000}%
\pgfsetdash{}{0pt}%
\pgfpathmoveto{\pgfqpoint{4.757910in}{5.329887in}}%
\pgfpathlineto{\pgfqpoint{4.307614in}{4.391044in}}%
\pgfusepath{stroke}%
\end{pgfscope}%
\begin{pgfscope}%
\pgfpathrectangle{\pgfqpoint{0.481978in}{0.331635in}}{\pgfqpoint{9.300000in}{7.700000in}}%
\pgfusepath{clip}%
\pgfsetrectcap%
\pgfsetroundjoin%
\pgfsetlinewidth{1.505625pt}%
\definecolor{currentstroke}{rgb}{1.000000,0.705882,0.509804}%
\pgfsetstrokecolor{currentstroke}%
\pgfsetstrokeopacity{0.800000}%
\pgfsetdash{}{0pt}%
\pgfpathmoveto{\pgfqpoint{4.104994in}{5.688256in}}%
\pgfpathlineto{\pgfqpoint{4.307614in}{4.391044in}}%
\pgfusepath{stroke}%
\end{pgfscope}%
\begin{pgfscope}%
\pgfpathrectangle{\pgfqpoint{0.481978in}{0.331635in}}{\pgfqpoint{9.300000in}{7.700000in}}%
\pgfusepath{clip}%
\pgfsetrectcap%
\pgfsetroundjoin%
\pgfsetlinewidth{1.505625pt}%
\definecolor{currentstroke}{rgb}{1.000000,0.705882,0.509804}%
\pgfsetstrokecolor{currentstroke}%
\pgfsetstrokeopacity{0.800000}%
\pgfsetdash{}{0pt}%
\pgfpathmoveto{\pgfqpoint{1.344247in}{4.370626in}}%
\pgfpathlineto{\pgfqpoint{4.307614in}{4.391044in}}%
\pgfusepath{stroke}%
\end{pgfscope}%
\begin{pgfscope}%
\pgfpathrectangle{\pgfqpoint{0.481978in}{0.331635in}}{\pgfqpoint{9.300000in}{7.700000in}}%
\pgfusepath{clip}%
\pgfsetrectcap%
\pgfsetroundjoin%
\pgfsetlinewidth{1.505625pt}%
\definecolor{currentstroke}{rgb}{1.000000,0.705882,0.509804}%
\pgfsetstrokecolor{currentstroke}%
\pgfsetstrokeopacity{0.800000}%
\pgfsetdash{}{0pt}%
\pgfpathmoveto{\pgfqpoint{3.248962in}{4.241499in}}%
\pgfpathlineto{\pgfqpoint{4.307614in}{4.391044in}}%
\pgfusepath{stroke}%
\end{pgfscope}%
\begin{pgfscope}%
\pgfsetrectcap%
\pgfsetmiterjoin%
\pgfsetlinewidth{0.803000pt}%
\definecolor{currentstroke}{rgb}{0.000000,0.000000,0.000000}%
\pgfsetstrokecolor{currentstroke}%
\pgfsetdash{}{0pt}%
\pgfpathmoveto{\pgfqpoint{0.481978in}{0.331635in}}%
\pgfpathlineto{\pgfqpoint{0.481978in}{8.031635in}}%
\pgfusepath{stroke}%
\end{pgfscope}%
\begin{pgfscope}%
\pgfsetrectcap%
\pgfsetmiterjoin%
\pgfsetlinewidth{0.803000pt}%
\definecolor{currentstroke}{rgb}{0.000000,0.000000,0.000000}%
\pgfsetstrokecolor{currentstroke}%
\pgfsetdash{}{0pt}%
\pgfpathmoveto{\pgfqpoint{9.781978in}{0.331635in}}%
\pgfpathlineto{\pgfqpoint{9.781978in}{8.031635in}}%
\pgfusepath{stroke}%
\end{pgfscope}%
\begin{pgfscope}%
\pgfsetrectcap%
\pgfsetmiterjoin%
\pgfsetlinewidth{0.803000pt}%
\definecolor{currentstroke}{rgb}{0.000000,0.000000,0.000000}%
\pgfsetstrokecolor{currentstroke}%
\pgfsetdash{}{0pt}%
\pgfpathmoveto{\pgfqpoint{0.481978in}{0.331635in}}%
\pgfpathlineto{\pgfqpoint{9.781978in}{0.331635in}}%
\pgfusepath{stroke}%
\end{pgfscope}%
\begin{pgfscope}%
\pgfsetrectcap%
\pgfsetmiterjoin%
\pgfsetlinewidth{0.803000pt}%
\definecolor{currentstroke}{rgb}{0.000000,0.000000,0.000000}%
\pgfsetstrokecolor{currentstroke}%
\pgfsetdash{}{0pt}%
\pgfpathmoveto{\pgfqpoint{0.481978in}{8.031635in}}%
\pgfpathlineto{\pgfqpoint{9.781978in}{8.031635in}}%
\pgfusepath{stroke}%
\end{pgfscope}%
\begin{pgfscope}%
\definecolor{textcolor}{rgb}{0.000000,0.000000,0.000000}%
\pgfsetstrokecolor{textcolor}%
\pgfsetfillcolor{textcolor}%
\pgftext[x=5.131978in,y=8.114968in,,base]{\color{textcolor}\sffamily\fontsize{12.000000}{14.400000}\selectfont Photo-Realistic Images}%
\end{pgfscope}%
\begin{pgfscope}%
\pgfsetbuttcap%
\pgfsetmiterjoin%
\definecolor{currentfill}{rgb}{1.000000,1.000000,1.000000}%
\pgfsetfillcolor{currentfill}%
\pgfsetfillopacity{0.800000}%
\pgfsetlinewidth{1.003750pt}%
\definecolor{currentstroke}{rgb}{0.800000,0.800000,0.800000}%
\pgfsetstrokecolor{currentstroke}%
\pgfsetstrokeopacity{0.800000}%
\pgfsetdash{}{0pt}%
\pgfpathmoveto{\pgfqpoint{9.879200in}{3.956944in}}%
\pgfpathlineto{\pgfqpoint{11.101911in}{3.956944in}}%
\pgfpathquadraticcurveto{\pgfqpoint{11.129688in}{3.956944in}}{\pgfqpoint{11.129688in}{3.984722in}}%
\pgfpathlineto{\pgfqpoint{11.129688in}{4.378548in}}%
\pgfpathquadraticcurveto{\pgfqpoint{11.129688in}{4.406326in}}{\pgfqpoint{11.101911in}{4.406326in}}%
\pgfpathlineto{\pgfqpoint{9.879200in}{4.406326in}}%
\pgfpathquadraticcurveto{\pgfqpoint{9.851422in}{4.406326in}}{\pgfqpoint{9.851422in}{4.378548in}}%
\pgfpathlineto{\pgfqpoint{9.851422in}{3.984722in}}%
\pgfpathquadraticcurveto{\pgfqpoint{9.851422in}{3.956944in}}{\pgfqpoint{9.879200in}{3.956944in}}%
\pgfpathclose%
\pgfusepath{stroke,fill}%
\end{pgfscope}%
\begin{pgfscope}%
\pgfsetbuttcap%
\pgfsetroundjoin%
\definecolor{currentfill}{rgb}{0.631373,0.788235,0.956863}%
\pgfsetfillcolor{currentfill}%
\pgfsetlinewidth{1.003750pt}%
\definecolor{currentstroke}{rgb}{0.631373,0.788235,0.956863}%
\pgfsetstrokecolor{currentstroke}%
\pgfsetdash{}{0pt}%
\pgfsys@defobject{currentmarker}{\pgfqpoint{-0.041667in}{-0.041667in}}{\pgfqpoint{0.041667in}{0.041667in}}{%
\pgfpathmoveto{\pgfqpoint{0.000000in}{-0.041667in}}%
\pgfpathcurveto{\pgfqpoint{0.011050in}{-0.041667in}}{\pgfqpoint{0.021649in}{-0.037276in}}{\pgfqpoint{0.029463in}{-0.029463in}}%
\pgfpathcurveto{\pgfqpoint{0.037276in}{-0.021649in}}{\pgfqpoint{0.041667in}{-0.011050in}}{\pgfqpoint{0.041667in}{0.000000in}}%
\pgfpathcurveto{\pgfqpoint{0.041667in}{0.011050in}}{\pgfqpoint{0.037276in}{0.021649in}}{\pgfqpoint{0.029463in}{0.029463in}}%
\pgfpathcurveto{\pgfqpoint{0.021649in}{0.037276in}}{\pgfqpoint{0.011050in}{0.041667in}}{\pgfqpoint{0.000000in}{0.041667in}}%
\pgfpathcurveto{\pgfqpoint{-0.011050in}{0.041667in}}{\pgfqpoint{-0.021649in}{0.037276in}}{\pgfqpoint{-0.029463in}{0.029463in}}%
\pgfpathcurveto{\pgfqpoint{-0.037276in}{0.021649in}}{\pgfqpoint{-0.041667in}{0.011050in}}{\pgfqpoint{-0.041667in}{0.000000in}}%
\pgfpathcurveto{\pgfqpoint{-0.041667in}{-0.011050in}}{\pgfqpoint{-0.037276in}{-0.021649in}}{\pgfqpoint{-0.029463in}{-0.029463in}}%
\pgfpathcurveto{\pgfqpoint{-0.021649in}{-0.037276in}}{\pgfqpoint{-0.011050in}{-0.041667in}}{\pgfqpoint{0.000000in}{-0.041667in}}%
\pgfpathclose%
\pgfusepath{stroke,fill}%
}%
\begin{pgfscope}%
\pgfsys@transformshift{10.045867in}{4.281705in}%
\pgfsys@useobject{currentmarker}{}%
\end{pgfscope}%
\end{pgfscope}%
\begin{pgfscope}%
\definecolor{textcolor}{rgb}{0.000000,0.000000,0.000000}%
\pgfsetstrokecolor{textcolor}%
\pgfsetfillcolor{textcolor}%
\pgftext[x=10.295867in,y=4.245247in,left,base]{\color{textcolor}\sffamily\fontsize{10.000000}{12.000000}\selectfont openrooms}%
\end{pgfscope}%
\begin{pgfscope}%
\pgfsetbuttcap%
\pgfsetroundjoin%
\definecolor{currentfill}{rgb}{1.000000,0.705882,0.509804}%
\pgfsetfillcolor{currentfill}%
\pgfsetlinewidth{1.003750pt}%
\definecolor{currentstroke}{rgb}{1.000000,0.705882,0.509804}%
\pgfsetstrokecolor{currentstroke}%
\pgfsetdash{}{0pt}%
\pgfsys@defobject{currentmarker}{\pgfqpoint{-0.041667in}{-0.041667in}}{\pgfqpoint{0.041667in}{0.041667in}}{%
\pgfpathmoveto{\pgfqpoint{0.000000in}{-0.041667in}}%
\pgfpathcurveto{\pgfqpoint{0.011050in}{-0.041667in}}{\pgfqpoint{0.021649in}{-0.037276in}}{\pgfqpoint{0.029463in}{-0.029463in}}%
\pgfpathcurveto{\pgfqpoint{0.037276in}{-0.021649in}}{\pgfqpoint{0.041667in}{-0.011050in}}{\pgfqpoint{0.041667in}{0.000000in}}%
\pgfpathcurveto{\pgfqpoint{0.041667in}{0.011050in}}{\pgfqpoint{0.037276in}{0.021649in}}{\pgfqpoint{0.029463in}{0.029463in}}%
\pgfpathcurveto{\pgfqpoint{0.021649in}{0.037276in}}{\pgfqpoint{0.011050in}{0.041667in}}{\pgfqpoint{0.000000in}{0.041667in}}%
\pgfpathcurveto{\pgfqpoint{-0.011050in}{0.041667in}}{\pgfqpoint{-0.021649in}{0.037276in}}{\pgfqpoint{-0.029463in}{0.029463in}}%
\pgfpathcurveto{\pgfqpoint{-0.037276in}{0.021649in}}{\pgfqpoint{-0.041667in}{0.011050in}}{\pgfqpoint{-0.041667in}{0.000000in}}%
\pgfpathcurveto{\pgfqpoint{-0.041667in}{-0.011050in}}{\pgfqpoint{-0.037276in}{-0.021649in}}{\pgfqpoint{-0.029463in}{-0.029463in}}%
\pgfpathcurveto{\pgfqpoint{-0.021649in}{-0.037276in}}{\pgfqpoint{-0.011050in}{-0.041667in}}{\pgfqpoint{0.000000in}{-0.041667in}}%
\pgfpathclose%
\pgfusepath{stroke,fill}%
}%
\begin{pgfscope}%
\pgfsys@transformshift{10.045867in}{4.077848in}%
\pgfsys@useobject{currentmarker}{}%
\end{pgfscope}%
\end{pgfscope}%
\begin{pgfscope}%
\definecolor{textcolor}{rgb}{0.000000,0.000000,0.000000}%
\pgfsetstrokecolor{textcolor}%
\pgfsetfillcolor{textcolor}%
\pgftext[x=10.295867in,y=4.041390in,left,base]{\color{textcolor}\sffamily\fontsize{10.000000}{12.000000}\selectfont pix3d}%
\end{pgfscope}%
\end{pgfpicture}%
\makeatother%
\endgroup%
}
    \resizebox{0.49\linewidth}{5cm}{%% Creator: Matplotlib, PGF backend
%%
%% To include the figure in your LaTeX document, write
%%   \input{<filename>.pgf}
%%
%% Make sure the required packages are loaded in your preamble
%%   \usepackage{pgf}
%%
%% Figures using additional raster images can only be included by \input if
%% they are in the same directory as the main LaTeX file. For loading figures
%% from other directories you can use the `import` package
%%   \usepackage{import}
%%
%% and then include the figures with
%%   \import{<path to file>}{<filename>.pgf}
%%
%% Matplotlib used the following preamble
%%   \usepackage{fontspec}
%%   \setmainfont{DejaVuSerif.ttf}[Path=\detokenize{/Users/apple/opt/anaconda3/envs/kaolin/lib/python3.7/site-packages/matplotlib/mpl-data/fonts/ttf/}]
%%   \setsansfont{DejaVuSans.ttf}[Path=\detokenize{/Users/apple/opt/anaconda3/envs/kaolin/lib/python3.7/site-packages/matplotlib/mpl-data/fonts/ttf/}]
%%   \setmonofont{DejaVuSansMono.ttf}[Path=\detokenize{/Users/apple/opt/anaconda3/envs/kaolin/lib/python3.7/site-packages/matplotlib/mpl-data/fonts/ttf/}]
%%
\begingroup%
\makeatletter%
\begin{pgfpicture}%
\pgfpathrectangle{\pgfpointorigin}{\pgfqpoint{5.541978in}{4.337596in}}%
\pgfusepath{use as bounding box, clip}%
\begin{pgfscope}%
\pgfsetbuttcap%
\pgfsetmiterjoin%
\definecolor{currentfill}{rgb}{1.000000,1.000000,1.000000}%
\pgfsetfillcolor{currentfill}%
\pgfsetlinewidth{0.000000pt}%
\definecolor{currentstroke}{rgb}{1.000000,1.000000,1.000000}%
\pgfsetstrokecolor{currentstroke}%
\pgfsetdash{}{0pt}%
\pgfpathmoveto{\pgfqpoint{0.000000in}{0.000000in}}%
\pgfpathlineto{\pgfqpoint{5.541978in}{0.000000in}}%
\pgfpathlineto{\pgfqpoint{5.541978in}{4.337596in}}%
\pgfpathlineto{\pgfqpoint{0.000000in}{4.337596in}}%
\pgfpathclose%
\pgfusepath{fill}%
\end{pgfscope}%
\begin{pgfscope}%
\pgfsetbuttcap%
\pgfsetmiterjoin%
\definecolor{currentfill}{rgb}{1.000000,1.000000,1.000000}%
\pgfsetfillcolor{currentfill}%
\pgfsetlinewidth{0.000000pt}%
\definecolor{currentstroke}{rgb}{0.000000,0.000000,0.000000}%
\pgfsetstrokecolor{currentstroke}%
\pgfsetstrokeopacity{0.000000}%
\pgfsetdash{}{0pt}%
\pgfpathmoveto{\pgfqpoint{0.481978in}{0.331635in}}%
\pgfpathlineto{\pgfqpoint{5.441978in}{0.331635in}}%
\pgfpathlineto{\pgfqpoint{5.441978in}{4.027635in}}%
\pgfpathlineto{\pgfqpoint{0.481978in}{4.027635in}}%
\pgfpathclose%
\pgfusepath{fill}%
\end{pgfscope}%
\begin{pgfscope}%
\pgfpathrectangle{\pgfqpoint{0.481978in}{0.331635in}}{\pgfqpoint{4.960000in}{3.696000in}}%
\pgfusepath{clip}%
\pgfsetbuttcap%
\pgfsetroundjoin%
\definecolor{currentfill}{rgb}{0.631373,0.788235,0.956863}%
\pgfsetfillcolor{currentfill}%
\pgfsetlinewidth{0.481800pt}%
\definecolor{currentstroke}{rgb}{1.000000,1.000000,1.000000}%
\pgfsetstrokecolor{currentstroke}%
\pgfsetdash{}{0pt}%
\pgfpathmoveto{\pgfqpoint{3.170276in}{2.389021in}}%
\pgfpathcurveto{\pgfqpoint{3.181326in}{2.389021in}}{\pgfqpoint{3.191925in}{2.393411in}}{\pgfqpoint{3.199739in}{2.401225in}}%
\pgfpathcurveto{\pgfqpoint{3.207552in}{2.409038in}}{\pgfqpoint{3.211942in}{2.419637in}}{\pgfqpoint{3.211942in}{2.430688in}}%
\pgfpathcurveto{\pgfqpoint{3.211942in}{2.441738in}}{\pgfqpoint{3.207552in}{2.452337in}}{\pgfqpoint{3.199739in}{2.460150in}}%
\pgfpathcurveto{\pgfqpoint{3.191925in}{2.467964in}}{\pgfqpoint{3.181326in}{2.472354in}}{\pgfqpoint{3.170276in}{2.472354in}}%
\pgfpathcurveto{\pgfqpoint{3.159226in}{2.472354in}}{\pgfqpoint{3.148627in}{2.467964in}}{\pgfqpoint{3.140813in}{2.460150in}}%
\pgfpathcurveto{\pgfqpoint{3.132999in}{2.452337in}}{\pgfqpoint{3.128609in}{2.441738in}}{\pgfqpoint{3.128609in}{2.430688in}}%
\pgfpathcurveto{\pgfqpoint{3.128609in}{2.419637in}}{\pgfqpoint{3.132999in}{2.409038in}}{\pgfqpoint{3.140813in}{2.401225in}}%
\pgfpathcurveto{\pgfqpoint{3.148627in}{2.393411in}}{\pgfqpoint{3.159226in}{2.389021in}}{\pgfqpoint{3.170276in}{2.389021in}}%
\pgfpathclose%
\pgfusepath{stroke,fill}%
\end{pgfscope}%
\begin{pgfscope}%
\pgfpathrectangle{\pgfqpoint{0.481978in}{0.331635in}}{\pgfqpoint{4.960000in}{3.696000in}}%
\pgfusepath{clip}%
\pgfsetbuttcap%
\pgfsetroundjoin%
\definecolor{currentfill}{rgb}{0.631373,0.788235,0.956863}%
\pgfsetfillcolor{currentfill}%
\pgfsetlinewidth{0.481800pt}%
\definecolor{currentstroke}{rgb}{1.000000,1.000000,1.000000}%
\pgfsetstrokecolor{currentstroke}%
\pgfsetdash{}{0pt}%
\pgfpathmoveto{\pgfqpoint{3.797277in}{1.208051in}}%
\pgfpathcurveto{\pgfqpoint{3.808327in}{1.208051in}}{\pgfqpoint{3.818926in}{1.212442in}}{\pgfqpoint{3.826739in}{1.220255in}}%
\pgfpathcurveto{\pgfqpoint{3.834553in}{1.228069in}}{\pgfqpoint{3.838943in}{1.238668in}}{\pgfqpoint{3.838943in}{1.249718in}}%
\pgfpathcurveto{\pgfqpoint{3.838943in}{1.260768in}}{\pgfqpoint{3.834553in}{1.271367in}}{\pgfqpoint{3.826739in}{1.279181in}}%
\pgfpathcurveto{\pgfqpoint{3.818926in}{1.286995in}}{\pgfqpoint{3.808327in}{1.291385in}}{\pgfqpoint{3.797277in}{1.291385in}}%
\pgfpathcurveto{\pgfqpoint{3.786227in}{1.291385in}}{\pgfqpoint{3.775627in}{1.286995in}}{\pgfqpoint{3.767814in}{1.279181in}}%
\pgfpathcurveto{\pgfqpoint{3.760000in}{1.271367in}}{\pgfqpoint{3.755610in}{1.260768in}}{\pgfqpoint{3.755610in}{1.249718in}}%
\pgfpathcurveto{\pgfqpoint{3.755610in}{1.238668in}}{\pgfqpoint{3.760000in}{1.228069in}}{\pgfqpoint{3.767814in}{1.220255in}}%
\pgfpathcurveto{\pgfqpoint{3.775627in}{1.212442in}}{\pgfqpoint{3.786227in}{1.208051in}}{\pgfqpoint{3.797277in}{1.208051in}}%
\pgfpathclose%
\pgfusepath{stroke,fill}%
\end{pgfscope}%
\begin{pgfscope}%
\pgfpathrectangle{\pgfqpoint{0.481978in}{0.331635in}}{\pgfqpoint{4.960000in}{3.696000in}}%
\pgfusepath{clip}%
\pgfsetbuttcap%
\pgfsetroundjoin%
\definecolor{currentfill}{rgb}{0.631373,0.788235,0.956863}%
\pgfsetfillcolor{currentfill}%
\pgfsetlinewidth{0.481800pt}%
\definecolor{currentstroke}{rgb}{1.000000,1.000000,1.000000}%
\pgfsetstrokecolor{currentstroke}%
\pgfsetdash{}{0pt}%
\pgfpathmoveto{\pgfqpoint{2.431161in}{2.118790in}}%
\pgfpathcurveto{\pgfqpoint{2.442211in}{2.118790in}}{\pgfqpoint{2.452810in}{2.123181in}}{\pgfqpoint{2.460624in}{2.130994in}}%
\pgfpathcurveto{\pgfqpoint{2.468437in}{2.138808in}}{\pgfqpoint{2.472828in}{2.149407in}}{\pgfqpoint{2.472828in}{2.160457in}}%
\pgfpathcurveto{\pgfqpoint{2.472828in}{2.171507in}}{\pgfqpoint{2.468437in}{2.182106in}}{\pgfqpoint{2.460624in}{2.189920in}}%
\pgfpathcurveto{\pgfqpoint{2.452810in}{2.197733in}}{\pgfqpoint{2.442211in}{2.202124in}}{\pgfqpoint{2.431161in}{2.202124in}}%
\pgfpathcurveto{\pgfqpoint{2.420111in}{2.202124in}}{\pgfqpoint{2.409512in}{2.197733in}}{\pgfqpoint{2.401698in}{2.189920in}}%
\pgfpathcurveto{\pgfqpoint{2.393884in}{2.182106in}}{\pgfqpoint{2.389494in}{2.171507in}}{\pgfqpoint{2.389494in}{2.160457in}}%
\pgfpathcurveto{\pgfqpoint{2.389494in}{2.149407in}}{\pgfqpoint{2.393884in}{2.138808in}}{\pgfqpoint{2.401698in}{2.130994in}}%
\pgfpathcurveto{\pgfqpoint{2.409512in}{2.123181in}}{\pgfqpoint{2.420111in}{2.118790in}}{\pgfqpoint{2.431161in}{2.118790in}}%
\pgfpathclose%
\pgfusepath{stroke,fill}%
\end{pgfscope}%
\begin{pgfscope}%
\pgfpathrectangle{\pgfqpoint{0.481978in}{0.331635in}}{\pgfqpoint{4.960000in}{3.696000in}}%
\pgfusepath{clip}%
\pgfsetbuttcap%
\pgfsetroundjoin%
\definecolor{currentfill}{rgb}{0.631373,0.788235,0.956863}%
\pgfsetfillcolor{currentfill}%
\pgfsetlinewidth{0.481800pt}%
\definecolor{currentstroke}{rgb}{1.000000,1.000000,1.000000}%
\pgfsetstrokecolor{currentstroke}%
\pgfsetdash{}{0pt}%
\pgfpathmoveto{\pgfqpoint{3.988216in}{3.317337in}}%
\pgfpathcurveto{\pgfqpoint{3.999266in}{3.317337in}}{\pgfqpoint{4.009865in}{3.321727in}}{\pgfqpoint{4.017678in}{3.329541in}}%
\pgfpathcurveto{\pgfqpoint{4.025492in}{3.337354in}}{\pgfqpoint{4.029882in}{3.347953in}}{\pgfqpoint{4.029882in}{3.359003in}}%
\pgfpathcurveto{\pgfqpoint{4.029882in}{3.370053in}}{\pgfqpoint{4.025492in}{3.380652in}}{\pgfqpoint{4.017678in}{3.388466in}}%
\pgfpathcurveto{\pgfqpoint{4.009865in}{3.396280in}}{\pgfqpoint{3.999266in}{3.400670in}}{\pgfqpoint{3.988216in}{3.400670in}}%
\pgfpathcurveto{\pgfqpoint{3.977165in}{3.400670in}}{\pgfqpoint{3.966566in}{3.396280in}}{\pgfqpoint{3.958753in}{3.388466in}}%
\pgfpathcurveto{\pgfqpoint{3.950939in}{3.380652in}}{\pgfqpoint{3.946549in}{3.370053in}}{\pgfqpoint{3.946549in}{3.359003in}}%
\pgfpathcurveto{\pgfqpoint{3.946549in}{3.347953in}}{\pgfqpoint{3.950939in}{3.337354in}}{\pgfqpoint{3.958753in}{3.329541in}}%
\pgfpathcurveto{\pgfqpoint{3.966566in}{3.321727in}}{\pgfqpoint{3.977165in}{3.317337in}}{\pgfqpoint{3.988216in}{3.317337in}}%
\pgfpathclose%
\pgfusepath{stroke,fill}%
\end{pgfscope}%
\begin{pgfscope}%
\pgfpathrectangle{\pgfqpoint{0.481978in}{0.331635in}}{\pgfqpoint{4.960000in}{3.696000in}}%
\pgfusepath{clip}%
\pgfsetbuttcap%
\pgfsetroundjoin%
\definecolor{currentfill}{rgb}{0.631373,0.788235,0.956863}%
\pgfsetfillcolor{currentfill}%
\pgfsetlinewidth{0.481800pt}%
\definecolor{currentstroke}{rgb}{1.000000,1.000000,1.000000}%
\pgfsetstrokecolor{currentstroke}%
\pgfsetdash{}{0pt}%
\pgfpathmoveto{\pgfqpoint{4.509417in}{1.343708in}}%
\pgfpathcurveto{\pgfqpoint{4.520467in}{1.343708in}}{\pgfqpoint{4.531066in}{1.348099in}}{\pgfqpoint{4.538879in}{1.355912in}}%
\pgfpathcurveto{\pgfqpoint{4.546693in}{1.363726in}}{\pgfqpoint{4.551083in}{1.374325in}}{\pgfqpoint{4.551083in}{1.385375in}}%
\pgfpathcurveto{\pgfqpoint{4.551083in}{1.396425in}}{\pgfqpoint{4.546693in}{1.407024in}}{\pgfqpoint{4.538879in}{1.414838in}}%
\pgfpathcurveto{\pgfqpoint{4.531066in}{1.422651in}}{\pgfqpoint{4.520467in}{1.427042in}}{\pgfqpoint{4.509417in}{1.427042in}}%
\pgfpathcurveto{\pgfqpoint{4.498367in}{1.427042in}}{\pgfqpoint{4.487767in}{1.422651in}}{\pgfqpoint{4.479954in}{1.414838in}}%
\pgfpathcurveto{\pgfqpoint{4.472140in}{1.407024in}}{\pgfqpoint{4.467750in}{1.396425in}}{\pgfqpoint{4.467750in}{1.385375in}}%
\pgfpathcurveto{\pgfqpoint{4.467750in}{1.374325in}}{\pgfqpoint{4.472140in}{1.363726in}}{\pgfqpoint{4.479954in}{1.355912in}}%
\pgfpathcurveto{\pgfqpoint{4.487767in}{1.348099in}}{\pgfqpoint{4.498367in}{1.343708in}}{\pgfqpoint{4.509417in}{1.343708in}}%
\pgfpathclose%
\pgfusepath{stroke,fill}%
\end{pgfscope}%
\begin{pgfscope}%
\pgfpathrectangle{\pgfqpoint{0.481978in}{0.331635in}}{\pgfqpoint{4.960000in}{3.696000in}}%
\pgfusepath{clip}%
\pgfsetbuttcap%
\pgfsetroundjoin%
\definecolor{currentfill}{rgb}{0.631373,0.788235,0.956863}%
\pgfsetfillcolor{currentfill}%
\pgfsetlinewidth{0.481800pt}%
\definecolor{currentstroke}{rgb}{1.000000,1.000000,1.000000}%
\pgfsetstrokecolor{currentstroke}%
\pgfsetdash{}{0pt}%
\pgfpathmoveto{\pgfqpoint{3.252727in}{2.098088in}}%
\pgfpathcurveto{\pgfqpoint{3.263777in}{2.098088in}}{\pgfqpoint{3.274376in}{2.102478in}}{\pgfqpoint{3.282190in}{2.110292in}}%
\pgfpathcurveto{\pgfqpoint{3.290003in}{2.118105in}}{\pgfqpoint{3.294394in}{2.128704in}}{\pgfqpoint{3.294394in}{2.139754in}}%
\pgfpathcurveto{\pgfqpoint{3.294394in}{2.150805in}}{\pgfqpoint{3.290003in}{2.161404in}}{\pgfqpoint{3.282190in}{2.169217in}}%
\pgfpathcurveto{\pgfqpoint{3.274376in}{2.177031in}}{\pgfqpoint{3.263777in}{2.181421in}}{\pgfqpoint{3.252727in}{2.181421in}}%
\pgfpathcurveto{\pgfqpoint{3.241677in}{2.181421in}}{\pgfqpoint{3.231078in}{2.177031in}}{\pgfqpoint{3.223264in}{2.169217in}}%
\pgfpathcurveto{\pgfqpoint{3.215450in}{2.161404in}}{\pgfqpoint{3.211060in}{2.150805in}}{\pgfqpoint{3.211060in}{2.139754in}}%
\pgfpathcurveto{\pgfqpoint{3.211060in}{2.128704in}}{\pgfqpoint{3.215450in}{2.118105in}}{\pgfqpoint{3.223264in}{2.110292in}}%
\pgfpathcurveto{\pgfqpoint{3.231078in}{2.102478in}}{\pgfqpoint{3.241677in}{2.098088in}}{\pgfqpoint{3.252727in}{2.098088in}}%
\pgfpathclose%
\pgfusepath{stroke,fill}%
\end{pgfscope}%
\begin{pgfscope}%
\pgfpathrectangle{\pgfqpoint{0.481978in}{0.331635in}}{\pgfqpoint{4.960000in}{3.696000in}}%
\pgfusepath{clip}%
\pgfsetbuttcap%
\pgfsetroundjoin%
\definecolor{currentfill}{rgb}{0.631373,0.788235,0.956863}%
\pgfsetfillcolor{currentfill}%
\pgfsetlinewidth{0.481800pt}%
\definecolor{currentstroke}{rgb}{1.000000,1.000000,1.000000}%
\pgfsetstrokecolor{currentstroke}%
\pgfsetdash{}{0pt}%
\pgfpathmoveto{\pgfqpoint{4.536927in}{3.153019in}}%
\pgfpathcurveto{\pgfqpoint{4.547977in}{3.153019in}}{\pgfqpoint{4.558576in}{3.157409in}}{\pgfqpoint{4.566389in}{3.165223in}}%
\pgfpathcurveto{\pgfqpoint{4.574203in}{3.173036in}}{\pgfqpoint{4.578593in}{3.183635in}}{\pgfqpoint{4.578593in}{3.194685in}}%
\pgfpathcurveto{\pgfqpoint{4.578593in}{3.205735in}}{\pgfqpoint{4.574203in}{3.216334in}}{\pgfqpoint{4.566389in}{3.224148in}}%
\pgfpathcurveto{\pgfqpoint{4.558576in}{3.231962in}}{\pgfqpoint{4.547977in}{3.236352in}}{\pgfqpoint{4.536927in}{3.236352in}}%
\pgfpathcurveto{\pgfqpoint{4.525877in}{3.236352in}}{\pgfqpoint{4.515278in}{3.231962in}}{\pgfqpoint{4.507464in}{3.224148in}}%
\pgfpathcurveto{\pgfqpoint{4.499650in}{3.216334in}}{\pgfqpoint{4.495260in}{3.205735in}}{\pgfqpoint{4.495260in}{3.194685in}}%
\pgfpathcurveto{\pgfqpoint{4.495260in}{3.183635in}}{\pgfqpoint{4.499650in}{3.173036in}}{\pgfqpoint{4.507464in}{3.165223in}}%
\pgfpathcurveto{\pgfqpoint{4.515278in}{3.157409in}}{\pgfqpoint{4.525877in}{3.153019in}}{\pgfqpoint{4.536927in}{3.153019in}}%
\pgfpathclose%
\pgfusepath{stroke,fill}%
\end{pgfscope}%
\begin{pgfscope}%
\pgfpathrectangle{\pgfqpoint{0.481978in}{0.331635in}}{\pgfqpoint{4.960000in}{3.696000in}}%
\pgfusepath{clip}%
\pgfsetbuttcap%
\pgfsetroundjoin%
\definecolor{currentfill}{rgb}{0.631373,0.788235,0.956863}%
\pgfsetfillcolor{currentfill}%
\pgfsetlinewidth{0.481800pt}%
\definecolor{currentstroke}{rgb}{1.000000,1.000000,1.000000}%
\pgfsetstrokecolor{currentstroke}%
\pgfsetdash{}{0pt}%
\pgfpathmoveto{\pgfqpoint{3.471510in}{2.570085in}}%
\pgfpathcurveto{\pgfqpoint{3.482560in}{2.570085in}}{\pgfqpoint{3.493159in}{2.574475in}}{\pgfqpoint{3.500973in}{2.582289in}}%
\pgfpathcurveto{\pgfqpoint{3.508786in}{2.590102in}}{\pgfqpoint{3.513177in}{2.600701in}}{\pgfqpoint{3.513177in}{2.611751in}}%
\pgfpathcurveto{\pgfqpoint{3.513177in}{2.622802in}}{\pgfqpoint{3.508786in}{2.633401in}}{\pgfqpoint{3.500973in}{2.641214in}}%
\pgfpathcurveto{\pgfqpoint{3.493159in}{2.649028in}}{\pgfqpoint{3.482560in}{2.653418in}}{\pgfqpoint{3.471510in}{2.653418in}}%
\pgfpathcurveto{\pgfqpoint{3.460460in}{2.653418in}}{\pgfqpoint{3.449861in}{2.649028in}}{\pgfqpoint{3.442047in}{2.641214in}}%
\pgfpathcurveto{\pgfqpoint{3.434234in}{2.633401in}}{\pgfqpoint{3.429843in}{2.622802in}}{\pgfqpoint{3.429843in}{2.611751in}}%
\pgfpathcurveto{\pgfqpoint{3.429843in}{2.600701in}}{\pgfqpoint{3.434234in}{2.590102in}}{\pgfqpoint{3.442047in}{2.582289in}}%
\pgfpathcurveto{\pgfqpoint{3.449861in}{2.574475in}}{\pgfqpoint{3.460460in}{2.570085in}}{\pgfqpoint{3.471510in}{2.570085in}}%
\pgfpathclose%
\pgfusepath{stroke,fill}%
\end{pgfscope}%
\begin{pgfscope}%
\pgfpathrectangle{\pgfqpoint{0.481978in}{0.331635in}}{\pgfqpoint{4.960000in}{3.696000in}}%
\pgfusepath{clip}%
\pgfsetbuttcap%
\pgfsetroundjoin%
\definecolor{currentfill}{rgb}{0.631373,0.788235,0.956863}%
\pgfsetfillcolor{currentfill}%
\pgfsetlinewidth{0.481800pt}%
\definecolor{currentstroke}{rgb}{1.000000,1.000000,1.000000}%
\pgfsetstrokecolor{currentstroke}%
\pgfsetdash{}{0pt}%
\pgfpathmoveto{\pgfqpoint{2.459931in}{1.717452in}}%
\pgfpathcurveto{\pgfqpoint{2.470981in}{1.717452in}}{\pgfqpoint{2.481580in}{1.721842in}}{\pgfqpoint{2.489394in}{1.729656in}}%
\pgfpathcurveto{\pgfqpoint{2.497208in}{1.737470in}}{\pgfqpoint{2.501598in}{1.748069in}}{\pgfqpoint{2.501598in}{1.759119in}}%
\pgfpathcurveto{\pgfqpoint{2.501598in}{1.770169in}}{\pgfqpoint{2.497208in}{1.780768in}}{\pgfqpoint{2.489394in}{1.788582in}}%
\pgfpathcurveto{\pgfqpoint{2.481580in}{1.796395in}}{\pgfqpoint{2.470981in}{1.800785in}}{\pgfqpoint{2.459931in}{1.800785in}}%
\pgfpathcurveto{\pgfqpoint{2.448881in}{1.800785in}}{\pgfqpoint{2.438282in}{1.796395in}}{\pgfqpoint{2.430468in}{1.788582in}}%
\pgfpathcurveto{\pgfqpoint{2.422655in}{1.780768in}}{\pgfqpoint{2.418265in}{1.770169in}}{\pgfqpoint{2.418265in}{1.759119in}}%
\pgfpathcurveto{\pgfqpoint{2.418265in}{1.748069in}}{\pgfqpoint{2.422655in}{1.737470in}}{\pgfqpoint{2.430468in}{1.729656in}}%
\pgfpathcurveto{\pgfqpoint{2.438282in}{1.721842in}}{\pgfqpoint{2.448881in}{1.717452in}}{\pgfqpoint{2.459931in}{1.717452in}}%
\pgfpathclose%
\pgfusepath{stroke,fill}%
\end{pgfscope}%
\begin{pgfscope}%
\pgfpathrectangle{\pgfqpoint{0.481978in}{0.331635in}}{\pgfqpoint{4.960000in}{3.696000in}}%
\pgfusepath{clip}%
\pgfsetbuttcap%
\pgfsetroundjoin%
\definecolor{currentfill}{rgb}{0.631373,0.788235,0.956863}%
\pgfsetfillcolor{currentfill}%
\pgfsetlinewidth{0.481800pt}%
\definecolor{currentstroke}{rgb}{1.000000,1.000000,1.000000}%
\pgfsetstrokecolor{currentstroke}%
\pgfsetdash{}{0pt}%
\pgfpathmoveto{\pgfqpoint{2.937570in}{1.854576in}}%
\pgfpathcurveto{\pgfqpoint{2.948621in}{1.854576in}}{\pgfqpoint{2.959220in}{1.858966in}}{\pgfqpoint{2.967033in}{1.866779in}}%
\pgfpathcurveto{\pgfqpoint{2.974847in}{1.874593in}}{\pgfqpoint{2.979237in}{1.885192in}}{\pgfqpoint{2.979237in}{1.896242in}}%
\pgfpathcurveto{\pgfqpoint{2.979237in}{1.907292in}}{\pgfqpoint{2.974847in}{1.917891in}}{\pgfqpoint{2.967033in}{1.925705in}}%
\pgfpathcurveto{\pgfqpoint{2.959220in}{1.933519in}}{\pgfqpoint{2.948621in}{1.937909in}}{\pgfqpoint{2.937570in}{1.937909in}}%
\pgfpathcurveto{\pgfqpoint{2.926520in}{1.937909in}}{\pgfqpoint{2.915921in}{1.933519in}}{\pgfqpoint{2.908108in}{1.925705in}}%
\pgfpathcurveto{\pgfqpoint{2.900294in}{1.917891in}}{\pgfqpoint{2.895904in}{1.907292in}}{\pgfqpoint{2.895904in}{1.896242in}}%
\pgfpathcurveto{\pgfqpoint{2.895904in}{1.885192in}}{\pgfqpoint{2.900294in}{1.874593in}}{\pgfqpoint{2.908108in}{1.866779in}}%
\pgfpathcurveto{\pgfqpoint{2.915921in}{1.858966in}}{\pgfqpoint{2.926520in}{1.854576in}}{\pgfqpoint{2.937570in}{1.854576in}}%
\pgfpathclose%
\pgfusepath{stroke,fill}%
\end{pgfscope}%
\begin{pgfscope}%
\pgfpathrectangle{\pgfqpoint{0.481978in}{0.331635in}}{\pgfqpoint{4.960000in}{3.696000in}}%
\pgfusepath{clip}%
\pgfsetbuttcap%
\pgfsetroundjoin%
\definecolor{currentfill}{rgb}{0.631373,0.788235,0.956863}%
\pgfsetfillcolor{currentfill}%
\pgfsetlinewidth{0.481800pt}%
\definecolor{currentstroke}{rgb}{1.000000,1.000000,1.000000}%
\pgfsetstrokecolor{currentstroke}%
\pgfsetdash{}{0pt}%
\pgfpathmoveto{\pgfqpoint{3.802689in}{1.746808in}}%
\pgfpathcurveto{\pgfqpoint{3.813739in}{1.746808in}}{\pgfqpoint{3.824338in}{1.751198in}}{\pgfqpoint{3.832152in}{1.759012in}}%
\pgfpathcurveto{\pgfqpoint{3.839965in}{1.766825in}}{\pgfqpoint{3.844355in}{1.777424in}}{\pgfqpoint{3.844355in}{1.788474in}}%
\pgfpathcurveto{\pgfqpoint{3.844355in}{1.799525in}}{\pgfqpoint{3.839965in}{1.810124in}}{\pgfqpoint{3.832152in}{1.817937in}}%
\pgfpathcurveto{\pgfqpoint{3.824338in}{1.825751in}}{\pgfqpoint{3.813739in}{1.830141in}}{\pgfqpoint{3.802689in}{1.830141in}}%
\pgfpathcurveto{\pgfqpoint{3.791639in}{1.830141in}}{\pgfqpoint{3.781040in}{1.825751in}}{\pgfqpoint{3.773226in}{1.817937in}}%
\pgfpathcurveto{\pgfqpoint{3.765412in}{1.810124in}}{\pgfqpoint{3.761022in}{1.799525in}}{\pgfqpoint{3.761022in}{1.788474in}}%
\pgfpathcurveto{\pgfqpoint{3.761022in}{1.777424in}}{\pgfqpoint{3.765412in}{1.766825in}}{\pgfqpoint{3.773226in}{1.759012in}}%
\pgfpathcurveto{\pgfqpoint{3.781040in}{1.751198in}}{\pgfqpoint{3.791639in}{1.746808in}}{\pgfqpoint{3.802689in}{1.746808in}}%
\pgfpathclose%
\pgfusepath{stroke,fill}%
\end{pgfscope}%
\begin{pgfscope}%
\pgfpathrectangle{\pgfqpoint{0.481978in}{0.331635in}}{\pgfqpoint{4.960000in}{3.696000in}}%
\pgfusepath{clip}%
\pgfsetbuttcap%
\pgfsetroundjoin%
\definecolor{currentfill}{rgb}{0.631373,0.788235,0.956863}%
\pgfsetfillcolor{currentfill}%
\pgfsetlinewidth{0.481800pt}%
\definecolor{currentstroke}{rgb}{1.000000,1.000000,1.000000}%
\pgfsetstrokecolor{currentstroke}%
\pgfsetdash{}{0pt}%
\pgfpathmoveto{\pgfqpoint{3.541785in}{1.498159in}}%
\pgfpathcurveto{\pgfqpoint{3.552835in}{1.498159in}}{\pgfqpoint{3.563434in}{1.502550in}}{\pgfqpoint{3.571248in}{1.510363in}}%
\pgfpathcurveto{\pgfqpoint{3.579061in}{1.518177in}}{\pgfqpoint{3.583452in}{1.528776in}}{\pgfqpoint{3.583452in}{1.539826in}}%
\pgfpathcurveto{\pgfqpoint{3.583452in}{1.550876in}}{\pgfqpoint{3.579061in}{1.561475in}}{\pgfqpoint{3.571248in}{1.569289in}}%
\pgfpathcurveto{\pgfqpoint{3.563434in}{1.577102in}}{\pgfqpoint{3.552835in}{1.581493in}}{\pgfqpoint{3.541785in}{1.581493in}}%
\pgfpathcurveto{\pgfqpoint{3.530735in}{1.581493in}}{\pgfqpoint{3.520136in}{1.577102in}}{\pgfqpoint{3.512322in}{1.569289in}}%
\pgfpathcurveto{\pgfqpoint{3.504509in}{1.561475in}}{\pgfqpoint{3.500118in}{1.550876in}}{\pgfqpoint{3.500118in}{1.539826in}}%
\pgfpathcurveto{\pgfqpoint{3.500118in}{1.528776in}}{\pgfqpoint{3.504509in}{1.518177in}}{\pgfqpoint{3.512322in}{1.510363in}}%
\pgfpathcurveto{\pgfqpoint{3.520136in}{1.502550in}}{\pgfqpoint{3.530735in}{1.498159in}}{\pgfqpoint{3.541785in}{1.498159in}}%
\pgfpathclose%
\pgfusepath{stroke,fill}%
\end{pgfscope}%
\begin{pgfscope}%
\pgfpathrectangle{\pgfqpoint{0.481978in}{0.331635in}}{\pgfqpoint{4.960000in}{3.696000in}}%
\pgfusepath{clip}%
\pgfsetbuttcap%
\pgfsetroundjoin%
\definecolor{currentfill}{rgb}{0.631373,0.788235,0.956863}%
\pgfsetfillcolor{currentfill}%
\pgfsetlinewidth{0.481800pt}%
\definecolor{currentstroke}{rgb}{1.000000,1.000000,1.000000}%
\pgfsetstrokecolor{currentstroke}%
\pgfsetdash{}{0pt}%
\pgfpathmoveto{\pgfqpoint{5.216523in}{2.392773in}}%
\pgfpathcurveto{\pgfqpoint{5.227574in}{2.392773in}}{\pgfqpoint{5.238173in}{2.397163in}}{\pgfqpoint{5.245986in}{2.404976in}}%
\pgfpathcurveto{\pgfqpoint{5.253800in}{2.412790in}}{\pgfqpoint{5.258190in}{2.423389in}}{\pgfqpoint{5.258190in}{2.434439in}}%
\pgfpathcurveto{\pgfqpoint{5.258190in}{2.445489in}}{\pgfqpoint{5.253800in}{2.456088in}}{\pgfqpoint{5.245986in}{2.463902in}}%
\pgfpathcurveto{\pgfqpoint{5.238173in}{2.471716in}}{\pgfqpoint{5.227574in}{2.476106in}}{\pgfqpoint{5.216523in}{2.476106in}}%
\pgfpathcurveto{\pgfqpoint{5.205473in}{2.476106in}}{\pgfqpoint{5.194874in}{2.471716in}}{\pgfqpoint{5.187061in}{2.463902in}}%
\pgfpathcurveto{\pgfqpoint{5.179247in}{2.456088in}}{\pgfqpoint{5.174857in}{2.445489in}}{\pgfqpoint{5.174857in}{2.434439in}}%
\pgfpathcurveto{\pgfqpoint{5.174857in}{2.423389in}}{\pgfqpoint{5.179247in}{2.412790in}}{\pgfqpoint{5.187061in}{2.404976in}}%
\pgfpathcurveto{\pgfqpoint{5.194874in}{2.397163in}}{\pgfqpoint{5.205473in}{2.392773in}}{\pgfqpoint{5.216523in}{2.392773in}}%
\pgfpathclose%
\pgfusepath{stroke,fill}%
\end{pgfscope}%
\begin{pgfscope}%
\pgfpathrectangle{\pgfqpoint{0.481978in}{0.331635in}}{\pgfqpoint{4.960000in}{3.696000in}}%
\pgfusepath{clip}%
\pgfsetbuttcap%
\pgfsetroundjoin%
\definecolor{currentfill}{rgb}{0.631373,0.788235,0.956863}%
\pgfsetfillcolor{currentfill}%
\pgfsetlinewidth{0.481800pt}%
\definecolor{currentstroke}{rgb}{1.000000,1.000000,1.000000}%
\pgfsetstrokecolor{currentstroke}%
\pgfsetdash{}{0pt}%
\pgfpathmoveto{\pgfqpoint{3.043134in}{1.544907in}}%
\pgfpathcurveto{\pgfqpoint{3.054184in}{1.544907in}}{\pgfqpoint{3.064783in}{1.549297in}}{\pgfqpoint{3.072597in}{1.557111in}}%
\pgfpathcurveto{\pgfqpoint{3.080410in}{1.564924in}}{\pgfqpoint{3.084801in}{1.575523in}}{\pgfqpoint{3.084801in}{1.586573in}}%
\pgfpathcurveto{\pgfqpoint{3.084801in}{1.597624in}}{\pgfqpoint{3.080410in}{1.608223in}}{\pgfqpoint{3.072597in}{1.616036in}}%
\pgfpathcurveto{\pgfqpoint{3.064783in}{1.623850in}}{\pgfqpoint{3.054184in}{1.628240in}}{\pgfqpoint{3.043134in}{1.628240in}}%
\pgfpathcurveto{\pgfqpoint{3.032084in}{1.628240in}}{\pgfqpoint{3.021485in}{1.623850in}}{\pgfqpoint{3.013671in}{1.616036in}}%
\pgfpathcurveto{\pgfqpoint{3.005858in}{1.608223in}}{\pgfqpoint{3.001467in}{1.597624in}}{\pgfqpoint{3.001467in}{1.586573in}}%
\pgfpathcurveto{\pgfqpoint{3.001467in}{1.575523in}}{\pgfqpoint{3.005858in}{1.564924in}}{\pgfqpoint{3.013671in}{1.557111in}}%
\pgfpathcurveto{\pgfqpoint{3.021485in}{1.549297in}}{\pgfqpoint{3.032084in}{1.544907in}}{\pgfqpoint{3.043134in}{1.544907in}}%
\pgfpathclose%
\pgfusepath{stroke,fill}%
\end{pgfscope}%
\begin{pgfscope}%
\pgfpathrectangle{\pgfqpoint{0.481978in}{0.331635in}}{\pgfqpoint{4.960000in}{3.696000in}}%
\pgfusepath{clip}%
\pgfsetbuttcap%
\pgfsetroundjoin%
\definecolor{currentfill}{rgb}{0.631373,0.788235,0.956863}%
\pgfsetfillcolor{currentfill}%
\pgfsetlinewidth{0.481800pt}%
\definecolor{currentstroke}{rgb}{1.000000,1.000000,1.000000}%
\pgfsetstrokecolor{currentstroke}%
\pgfsetdash{}{0pt}%
\pgfpathmoveto{\pgfqpoint{2.580652in}{3.142864in}}%
\pgfpathcurveto{\pgfqpoint{2.591703in}{3.142864in}}{\pgfqpoint{2.602302in}{3.147254in}}{\pgfqpoint{2.610115in}{3.155068in}}%
\pgfpathcurveto{\pgfqpoint{2.617929in}{3.162881in}}{\pgfqpoint{2.622319in}{3.173480in}}{\pgfqpoint{2.622319in}{3.184530in}}%
\pgfpathcurveto{\pgfqpoint{2.622319in}{3.195580in}}{\pgfqpoint{2.617929in}{3.206180in}}{\pgfqpoint{2.610115in}{3.213993in}}%
\pgfpathcurveto{\pgfqpoint{2.602302in}{3.221807in}}{\pgfqpoint{2.591703in}{3.226197in}}{\pgfqpoint{2.580652in}{3.226197in}}%
\pgfpathcurveto{\pgfqpoint{2.569602in}{3.226197in}}{\pgfqpoint{2.559003in}{3.221807in}}{\pgfqpoint{2.551190in}{3.213993in}}%
\pgfpathcurveto{\pgfqpoint{2.543376in}{3.206180in}}{\pgfqpoint{2.538986in}{3.195580in}}{\pgfqpoint{2.538986in}{3.184530in}}%
\pgfpathcurveto{\pgfqpoint{2.538986in}{3.173480in}}{\pgfqpoint{2.543376in}{3.162881in}}{\pgfqpoint{2.551190in}{3.155068in}}%
\pgfpathcurveto{\pgfqpoint{2.559003in}{3.147254in}}{\pgfqpoint{2.569602in}{3.142864in}}{\pgfqpoint{2.580652in}{3.142864in}}%
\pgfpathclose%
\pgfusepath{stroke,fill}%
\end{pgfscope}%
\begin{pgfscope}%
\pgfpathrectangle{\pgfqpoint{0.481978in}{0.331635in}}{\pgfqpoint{4.960000in}{3.696000in}}%
\pgfusepath{clip}%
\pgfsetbuttcap%
\pgfsetroundjoin%
\definecolor{currentfill}{rgb}{0.631373,0.788235,0.956863}%
\pgfsetfillcolor{currentfill}%
\pgfsetlinewidth{0.481800pt}%
\definecolor{currentstroke}{rgb}{1.000000,1.000000,1.000000}%
\pgfsetstrokecolor{currentstroke}%
\pgfsetdash{}{0pt}%
\pgfpathmoveto{\pgfqpoint{4.001840in}{1.536698in}}%
\pgfpathcurveto{\pgfqpoint{4.012890in}{1.536698in}}{\pgfqpoint{4.023489in}{1.541088in}}{\pgfqpoint{4.031303in}{1.548902in}}%
\pgfpathcurveto{\pgfqpoint{4.039117in}{1.556715in}}{\pgfqpoint{4.043507in}{1.567314in}}{\pgfqpoint{4.043507in}{1.578365in}}%
\pgfpathcurveto{\pgfqpoint{4.043507in}{1.589415in}}{\pgfqpoint{4.039117in}{1.600014in}}{\pgfqpoint{4.031303in}{1.607827in}}%
\pgfpathcurveto{\pgfqpoint{4.023489in}{1.615641in}}{\pgfqpoint{4.012890in}{1.620031in}}{\pgfqpoint{4.001840in}{1.620031in}}%
\pgfpathcurveto{\pgfqpoint{3.990790in}{1.620031in}}{\pgfqpoint{3.980191in}{1.615641in}}{\pgfqpoint{3.972377in}{1.607827in}}%
\pgfpathcurveto{\pgfqpoint{3.964564in}{1.600014in}}{\pgfqpoint{3.960174in}{1.589415in}}{\pgfqpoint{3.960174in}{1.578365in}}%
\pgfpathcurveto{\pgfqpoint{3.960174in}{1.567314in}}{\pgfqpoint{3.964564in}{1.556715in}}{\pgfqpoint{3.972377in}{1.548902in}}%
\pgfpathcurveto{\pgfqpoint{3.980191in}{1.541088in}}{\pgfqpoint{3.990790in}{1.536698in}}{\pgfqpoint{4.001840in}{1.536698in}}%
\pgfpathclose%
\pgfusepath{stroke,fill}%
\end{pgfscope}%
\begin{pgfscope}%
\pgfpathrectangle{\pgfqpoint{0.481978in}{0.331635in}}{\pgfqpoint{4.960000in}{3.696000in}}%
\pgfusepath{clip}%
\pgfsetbuttcap%
\pgfsetroundjoin%
\definecolor{currentfill}{rgb}{0.631373,0.788235,0.956863}%
\pgfsetfillcolor{currentfill}%
\pgfsetlinewidth{0.481800pt}%
\definecolor{currentstroke}{rgb}{1.000000,1.000000,1.000000}%
\pgfsetstrokecolor{currentstroke}%
\pgfsetdash{}{0pt}%
\pgfpathmoveto{\pgfqpoint{3.872808in}{2.514458in}}%
\pgfpathcurveto{\pgfqpoint{3.883858in}{2.514458in}}{\pgfqpoint{3.894457in}{2.518848in}}{\pgfqpoint{3.902271in}{2.526662in}}%
\pgfpathcurveto{\pgfqpoint{3.910084in}{2.534475in}}{\pgfqpoint{3.914475in}{2.545074in}}{\pgfqpoint{3.914475in}{2.556124in}}%
\pgfpathcurveto{\pgfqpoint{3.914475in}{2.567175in}}{\pgfqpoint{3.910084in}{2.577774in}}{\pgfqpoint{3.902271in}{2.585587in}}%
\pgfpathcurveto{\pgfqpoint{3.894457in}{2.593401in}}{\pgfqpoint{3.883858in}{2.597791in}}{\pgfqpoint{3.872808in}{2.597791in}}%
\pgfpathcurveto{\pgfqpoint{3.861758in}{2.597791in}}{\pgfqpoint{3.851159in}{2.593401in}}{\pgfqpoint{3.843345in}{2.585587in}}%
\pgfpathcurveto{\pgfqpoint{3.835532in}{2.577774in}}{\pgfqpoint{3.831141in}{2.567175in}}{\pgfqpoint{3.831141in}{2.556124in}}%
\pgfpathcurveto{\pgfqpoint{3.831141in}{2.545074in}}{\pgfqpoint{3.835532in}{2.534475in}}{\pgfqpoint{3.843345in}{2.526662in}}%
\pgfpathcurveto{\pgfqpoint{3.851159in}{2.518848in}}{\pgfqpoint{3.861758in}{2.514458in}}{\pgfqpoint{3.872808in}{2.514458in}}%
\pgfpathclose%
\pgfusepath{stroke,fill}%
\end{pgfscope}%
\begin{pgfscope}%
\pgfpathrectangle{\pgfqpoint{0.481978in}{0.331635in}}{\pgfqpoint{4.960000in}{3.696000in}}%
\pgfusepath{clip}%
\pgfsetbuttcap%
\pgfsetroundjoin%
\definecolor{currentfill}{rgb}{0.631373,0.788235,0.956863}%
\pgfsetfillcolor{currentfill}%
\pgfsetlinewidth{0.481800pt}%
\definecolor{currentstroke}{rgb}{1.000000,1.000000,1.000000}%
\pgfsetstrokecolor{currentstroke}%
\pgfsetdash{}{0pt}%
\pgfpathmoveto{\pgfqpoint{3.135350in}{2.736846in}}%
\pgfpathcurveto{\pgfqpoint{3.146400in}{2.736846in}}{\pgfqpoint{3.156999in}{2.741237in}}{\pgfqpoint{3.164813in}{2.749050in}}%
\pgfpathcurveto{\pgfqpoint{3.172627in}{2.756864in}}{\pgfqpoint{3.177017in}{2.767463in}}{\pgfqpoint{3.177017in}{2.778513in}}%
\pgfpathcurveto{\pgfqpoint{3.177017in}{2.789563in}}{\pgfqpoint{3.172627in}{2.800162in}}{\pgfqpoint{3.164813in}{2.807976in}}%
\pgfpathcurveto{\pgfqpoint{3.156999in}{2.815789in}}{\pgfqpoint{3.146400in}{2.820180in}}{\pgfqpoint{3.135350in}{2.820180in}}%
\pgfpathcurveto{\pgfqpoint{3.124300in}{2.820180in}}{\pgfqpoint{3.113701in}{2.815789in}}{\pgfqpoint{3.105888in}{2.807976in}}%
\pgfpathcurveto{\pgfqpoint{3.098074in}{2.800162in}}{\pgfqpoint{3.093684in}{2.789563in}}{\pgfqpoint{3.093684in}{2.778513in}}%
\pgfpathcurveto{\pgfqpoint{3.093684in}{2.767463in}}{\pgfqpoint{3.098074in}{2.756864in}}{\pgfqpoint{3.105888in}{2.749050in}}%
\pgfpathcurveto{\pgfqpoint{3.113701in}{2.741237in}}{\pgfqpoint{3.124300in}{2.736846in}}{\pgfqpoint{3.135350in}{2.736846in}}%
\pgfpathclose%
\pgfusepath{stroke,fill}%
\end{pgfscope}%
\begin{pgfscope}%
\pgfpathrectangle{\pgfqpoint{0.481978in}{0.331635in}}{\pgfqpoint{4.960000in}{3.696000in}}%
\pgfusepath{clip}%
\pgfsetbuttcap%
\pgfsetroundjoin%
\definecolor{currentfill}{rgb}{0.631373,0.788235,0.956863}%
\pgfsetfillcolor{currentfill}%
\pgfsetlinewidth{0.481800pt}%
\definecolor{currentstroke}{rgb}{1.000000,1.000000,1.000000}%
\pgfsetstrokecolor{currentstroke}%
\pgfsetdash{}{0pt}%
\pgfpathmoveto{\pgfqpoint{3.378442in}{1.825081in}}%
\pgfpathcurveto{\pgfqpoint{3.389492in}{1.825081in}}{\pgfqpoint{3.400091in}{1.829471in}}{\pgfqpoint{3.407905in}{1.837285in}}%
\pgfpathcurveto{\pgfqpoint{3.415719in}{1.845098in}}{\pgfqpoint{3.420109in}{1.855697in}}{\pgfqpoint{3.420109in}{1.866748in}}%
\pgfpathcurveto{\pgfqpoint{3.420109in}{1.877798in}}{\pgfqpoint{3.415719in}{1.888397in}}{\pgfqpoint{3.407905in}{1.896210in}}%
\pgfpathcurveto{\pgfqpoint{3.400091in}{1.904024in}}{\pgfqpoint{3.389492in}{1.908414in}}{\pgfqpoint{3.378442in}{1.908414in}}%
\pgfpathcurveto{\pgfqpoint{3.367392in}{1.908414in}}{\pgfqpoint{3.356793in}{1.904024in}}{\pgfqpoint{3.348979in}{1.896210in}}%
\pgfpathcurveto{\pgfqpoint{3.341166in}{1.888397in}}{\pgfqpoint{3.336775in}{1.877798in}}{\pgfqpoint{3.336775in}{1.866748in}}%
\pgfpathcurveto{\pgfqpoint{3.336775in}{1.855697in}}{\pgfqpoint{3.341166in}{1.845098in}}{\pgfqpoint{3.348979in}{1.837285in}}%
\pgfpathcurveto{\pgfqpoint{3.356793in}{1.829471in}}{\pgfqpoint{3.367392in}{1.825081in}}{\pgfqpoint{3.378442in}{1.825081in}}%
\pgfpathclose%
\pgfusepath{stroke,fill}%
\end{pgfscope}%
\begin{pgfscope}%
\pgfpathrectangle{\pgfqpoint{0.481978in}{0.331635in}}{\pgfqpoint{4.960000in}{3.696000in}}%
\pgfusepath{clip}%
\pgfsetbuttcap%
\pgfsetroundjoin%
\definecolor{currentfill}{rgb}{0.631373,0.788235,0.956863}%
\pgfsetfillcolor{currentfill}%
\pgfsetlinewidth{0.481800pt}%
\definecolor{currentstroke}{rgb}{1.000000,1.000000,1.000000}%
\pgfsetstrokecolor{currentstroke}%
\pgfsetdash{}{0pt}%
\pgfpathmoveto{\pgfqpoint{3.656219in}{2.033922in}}%
\pgfpathcurveto{\pgfqpoint{3.667269in}{2.033922in}}{\pgfqpoint{3.677868in}{2.038312in}}{\pgfqpoint{3.685682in}{2.046126in}}%
\pgfpathcurveto{\pgfqpoint{3.693495in}{2.053939in}}{\pgfqpoint{3.697885in}{2.064538in}}{\pgfqpoint{3.697885in}{2.075588in}}%
\pgfpathcurveto{\pgfqpoint{3.697885in}{2.086638in}}{\pgfqpoint{3.693495in}{2.097238in}}{\pgfqpoint{3.685682in}{2.105051in}}%
\pgfpathcurveto{\pgfqpoint{3.677868in}{2.112865in}}{\pgfqpoint{3.667269in}{2.117255in}}{\pgfqpoint{3.656219in}{2.117255in}}%
\pgfpathcurveto{\pgfqpoint{3.645169in}{2.117255in}}{\pgfqpoint{3.634570in}{2.112865in}}{\pgfqpoint{3.626756in}{2.105051in}}%
\pgfpathcurveto{\pgfqpoint{3.618942in}{2.097238in}}{\pgfqpoint{3.614552in}{2.086638in}}{\pgfqpoint{3.614552in}{2.075588in}}%
\pgfpathcurveto{\pgfqpoint{3.614552in}{2.064538in}}{\pgfqpoint{3.618942in}{2.053939in}}{\pgfqpoint{3.626756in}{2.046126in}}%
\pgfpathcurveto{\pgfqpoint{3.634570in}{2.038312in}}{\pgfqpoint{3.645169in}{2.033922in}}{\pgfqpoint{3.656219in}{2.033922in}}%
\pgfpathclose%
\pgfusepath{stroke,fill}%
\end{pgfscope}%
\begin{pgfscope}%
\pgfpathrectangle{\pgfqpoint{0.481978in}{0.331635in}}{\pgfqpoint{4.960000in}{3.696000in}}%
\pgfusepath{clip}%
\pgfsetbuttcap%
\pgfsetroundjoin%
\definecolor{currentfill}{rgb}{0.631373,0.788235,0.956863}%
\pgfsetfillcolor{currentfill}%
\pgfsetlinewidth{0.481800pt}%
\definecolor{currentstroke}{rgb}{1.000000,1.000000,1.000000}%
\pgfsetstrokecolor{currentstroke}%
\pgfsetdash{}{0pt}%
\pgfpathmoveto{\pgfqpoint{3.440189in}{2.967561in}}%
\pgfpathcurveto{\pgfqpoint{3.451239in}{2.967561in}}{\pgfqpoint{3.461838in}{2.971951in}}{\pgfqpoint{3.469652in}{2.979765in}}%
\pgfpathcurveto{\pgfqpoint{3.477466in}{2.987578in}}{\pgfqpoint{3.481856in}{2.998177in}}{\pgfqpoint{3.481856in}{3.009227in}}%
\pgfpathcurveto{\pgfqpoint{3.481856in}{3.020278in}}{\pgfqpoint{3.477466in}{3.030877in}}{\pgfqpoint{3.469652in}{3.038690in}}%
\pgfpathcurveto{\pgfqpoint{3.461838in}{3.046504in}}{\pgfqpoint{3.451239in}{3.050894in}}{\pgfqpoint{3.440189in}{3.050894in}}%
\pgfpathcurveto{\pgfqpoint{3.429139in}{3.050894in}}{\pgfqpoint{3.418540in}{3.046504in}}{\pgfqpoint{3.410727in}{3.038690in}}%
\pgfpathcurveto{\pgfqpoint{3.402913in}{3.030877in}}{\pgfqpoint{3.398523in}{3.020278in}}{\pgfqpoint{3.398523in}{3.009227in}}%
\pgfpathcurveto{\pgfqpoint{3.398523in}{2.998177in}}{\pgfqpoint{3.402913in}{2.987578in}}{\pgfqpoint{3.410727in}{2.979765in}}%
\pgfpathcurveto{\pgfqpoint{3.418540in}{2.971951in}}{\pgfqpoint{3.429139in}{2.967561in}}{\pgfqpoint{3.440189in}{2.967561in}}%
\pgfpathclose%
\pgfusepath{stroke,fill}%
\end{pgfscope}%
\begin{pgfscope}%
\pgfpathrectangle{\pgfqpoint{0.481978in}{0.331635in}}{\pgfqpoint{4.960000in}{3.696000in}}%
\pgfusepath{clip}%
\pgfsetbuttcap%
\pgfsetroundjoin%
\definecolor{currentfill}{rgb}{0.631373,0.788235,0.956863}%
\pgfsetfillcolor{currentfill}%
\pgfsetlinewidth{0.481800pt}%
\definecolor{currentstroke}{rgb}{1.000000,1.000000,1.000000}%
\pgfsetstrokecolor{currentstroke}%
\pgfsetdash{}{0pt}%
\pgfpathmoveto{\pgfqpoint{4.280646in}{1.811485in}}%
\pgfpathcurveto{\pgfqpoint{4.291696in}{1.811485in}}{\pgfqpoint{4.302295in}{1.815876in}}{\pgfqpoint{4.310109in}{1.823689in}}%
\pgfpathcurveto{\pgfqpoint{4.317922in}{1.831503in}}{\pgfqpoint{4.322313in}{1.842102in}}{\pgfqpoint{4.322313in}{1.853152in}}%
\pgfpathcurveto{\pgfqpoint{4.322313in}{1.864202in}}{\pgfqpoint{4.317922in}{1.874801in}}{\pgfqpoint{4.310109in}{1.882615in}}%
\pgfpathcurveto{\pgfqpoint{4.302295in}{1.890428in}}{\pgfqpoint{4.291696in}{1.894819in}}{\pgfqpoint{4.280646in}{1.894819in}}%
\pgfpathcurveto{\pgfqpoint{4.269596in}{1.894819in}}{\pgfqpoint{4.258997in}{1.890428in}}{\pgfqpoint{4.251183in}{1.882615in}}%
\pgfpathcurveto{\pgfqpoint{4.243370in}{1.874801in}}{\pgfqpoint{4.238979in}{1.864202in}}{\pgfqpoint{4.238979in}{1.853152in}}%
\pgfpathcurveto{\pgfqpoint{4.238979in}{1.842102in}}{\pgfqpoint{4.243370in}{1.831503in}}{\pgfqpoint{4.251183in}{1.823689in}}%
\pgfpathcurveto{\pgfqpoint{4.258997in}{1.815876in}}{\pgfqpoint{4.269596in}{1.811485in}}{\pgfqpoint{4.280646in}{1.811485in}}%
\pgfpathclose%
\pgfusepath{stroke,fill}%
\end{pgfscope}%
\begin{pgfscope}%
\pgfpathrectangle{\pgfqpoint{0.481978in}{0.331635in}}{\pgfqpoint{4.960000in}{3.696000in}}%
\pgfusepath{clip}%
\pgfsetbuttcap%
\pgfsetroundjoin%
\definecolor{currentfill}{rgb}{0.631373,0.788235,0.956863}%
\pgfsetfillcolor{currentfill}%
\pgfsetlinewidth{0.481800pt}%
\definecolor{currentstroke}{rgb}{1.000000,1.000000,1.000000}%
\pgfsetstrokecolor{currentstroke}%
\pgfsetdash{}{0pt}%
\pgfpathmoveto{\pgfqpoint{2.790063in}{2.195412in}}%
\pgfpathcurveto{\pgfqpoint{2.801113in}{2.195412in}}{\pgfqpoint{2.811712in}{2.199802in}}{\pgfqpoint{2.819526in}{2.207616in}}%
\pgfpathcurveto{\pgfqpoint{2.827340in}{2.215429in}}{\pgfqpoint{2.831730in}{2.226028in}}{\pgfqpoint{2.831730in}{2.237079in}}%
\pgfpathcurveto{\pgfqpoint{2.831730in}{2.248129in}}{\pgfqpoint{2.827340in}{2.258728in}}{\pgfqpoint{2.819526in}{2.266541in}}%
\pgfpathcurveto{\pgfqpoint{2.811712in}{2.274355in}}{\pgfqpoint{2.801113in}{2.278745in}}{\pgfqpoint{2.790063in}{2.278745in}}%
\pgfpathcurveto{\pgfqpoint{2.779013in}{2.278745in}}{\pgfqpoint{2.768414in}{2.274355in}}{\pgfqpoint{2.760600in}{2.266541in}}%
\pgfpathcurveto{\pgfqpoint{2.752787in}{2.258728in}}{\pgfqpoint{2.748397in}{2.248129in}}{\pgfqpoint{2.748397in}{2.237079in}}%
\pgfpathcurveto{\pgfqpoint{2.748397in}{2.226028in}}{\pgfqpoint{2.752787in}{2.215429in}}{\pgfqpoint{2.760600in}{2.207616in}}%
\pgfpathcurveto{\pgfqpoint{2.768414in}{2.199802in}}{\pgfqpoint{2.779013in}{2.195412in}}{\pgfqpoint{2.790063in}{2.195412in}}%
\pgfpathclose%
\pgfusepath{stroke,fill}%
\end{pgfscope}%
\begin{pgfscope}%
\pgfpathrectangle{\pgfqpoint{0.481978in}{0.331635in}}{\pgfqpoint{4.960000in}{3.696000in}}%
\pgfusepath{clip}%
\pgfsetbuttcap%
\pgfsetroundjoin%
\definecolor{currentfill}{rgb}{0.631373,0.788235,0.956863}%
\pgfsetfillcolor{currentfill}%
\pgfsetlinewidth{0.481800pt}%
\definecolor{currentstroke}{rgb}{1.000000,1.000000,1.000000}%
\pgfsetstrokecolor{currentstroke}%
\pgfsetdash{}{0pt}%
\pgfpathmoveto{\pgfqpoint{4.256551in}{2.832602in}}%
\pgfpathcurveto{\pgfqpoint{4.267602in}{2.832602in}}{\pgfqpoint{4.278201in}{2.836992in}}{\pgfqpoint{4.286014in}{2.844806in}}%
\pgfpathcurveto{\pgfqpoint{4.293828in}{2.852619in}}{\pgfqpoint{4.298218in}{2.863218in}}{\pgfqpoint{4.298218in}{2.874269in}}%
\pgfpathcurveto{\pgfqpoint{4.298218in}{2.885319in}}{\pgfqpoint{4.293828in}{2.895918in}}{\pgfqpoint{4.286014in}{2.903731in}}%
\pgfpathcurveto{\pgfqpoint{4.278201in}{2.911545in}}{\pgfqpoint{4.267602in}{2.915935in}}{\pgfqpoint{4.256551in}{2.915935in}}%
\pgfpathcurveto{\pgfqpoint{4.245501in}{2.915935in}}{\pgfqpoint{4.234902in}{2.911545in}}{\pgfqpoint{4.227089in}{2.903731in}}%
\pgfpathcurveto{\pgfqpoint{4.219275in}{2.895918in}}{\pgfqpoint{4.214885in}{2.885319in}}{\pgfqpoint{4.214885in}{2.874269in}}%
\pgfpathcurveto{\pgfqpoint{4.214885in}{2.863218in}}{\pgfqpoint{4.219275in}{2.852619in}}{\pgfqpoint{4.227089in}{2.844806in}}%
\pgfpathcurveto{\pgfqpoint{4.234902in}{2.836992in}}{\pgfqpoint{4.245501in}{2.832602in}}{\pgfqpoint{4.256551in}{2.832602in}}%
\pgfpathclose%
\pgfusepath{stroke,fill}%
\end{pgfscope}%
\begin{pgfscope}%
\pgfpathrectangle{\pgfqpoint{0.481978in}{0.331635in}}{\pgfqpoint{4.960000in}{3.696000in}}%
\pgfusepath{clip}%
\pgfsetbuttcap%
\pgfsetroundjoin%
\definecolor{currentfill}{rgb}{0.631373,0.788235,0.956863}%
\pgfsetfillcolor{currentfill}%
\pgfsetlinewidth{0.481800pt}%
\definecolor{currentstroke}{rgb}{1.000000,1.000000,1.000000}%
\pgfsetstrokecolor{currentstroke}%
\pgfsetdash{}{0pt}%
\pgfpathmoveto{\pgfqpoint{4.575205in}{2.151472in}}%
\pgfpathcurveto{\pgfqpoint{4.586255in}{2.151472in}}{\pgfqpoint{4.596854in}{2.155862in}}{\pgfqpoint{4.604668in}{2.163676in}}%
\pgfpathcurveto{\pgfqpoint{4.612482in}{2.171490in}}{\pgfqpoint{4.616872in}{2.182089in}}{\pgfqpoint{4.616872in}{2.193139in}}%
\pgfpathcurveto{\pgfqpoint{4.616872in}{2.204189in}}{\pgfqpoint{4.612482in}{2.214788in}}{\pgfqpoint{4.604668in}{2.222601in}}%
\pgfpathcurveto{\pgfqpoint{4.596854in}{2.230415in}}{\pgfqpoint{4.586255in}{2.234805in}}{\pgfqpoint{4.575205in}{2.234805in}}%
\pgfpathcurveto{\pgfqpoint{4.564155in}{2.234805in}}{\pgfqpoint{4.553556in}{2.230415in}}{\pgfqpoint{4.545742in}{2.222601in}}%
\pgfpathcurveto{\pgfqpoint{4.537929in}{2.214788in}}{\pgfqpoint{4.533539in}{2.204189in}}{\pgfqpoint{4.533539in}{2.193139in}}%
\pgfpathcurveto{\pgfqpoint{4.533539in}{2.182089in}}{\pgfqpoint{4.537929in}{2.171490in}}{\pgfqpoint{4.545742in}{2.163676in}}%
\pgfpathcurveto{\pgfqpoint{4.553556in}{2.155862in}}{\pgfqpoint{4.564155in}{2.151472in}}{\pgfqpoint{4.575205in}{2.151472in}}%
\pgfpathclose%
\pgfusepath{stroke,fill}%
\end{pgfscope}%
\begin{pgfscope}%
\pgfpathrectangle{\pgfqpoint{0.481978in}{0.331635in}}{\pgfqpoint{4.960000in}{3.696000in}}%
\pgfusepath{clip}%
\pgfsetbuttcap%
\pgfsetroundjoin%
\definecolor{currentfill}{rgb}{0.631373,0.788235,0.956863}%
\pgfsetfillcolor{currentfill}%
\pgfsetlinewidth{0.481800pt}%
\definecolor{currentstroke}{rgb}{1.000000,1.000000,1.000000}%
\pgfsetstrokecolor{currentstroke}%
\pgfsetdash{}{0pt}%
\pgfpathmoveto{\pgfqpoint{3.448245in}{3.271854in}}%
\pgfpathcurveto{\pgfqpoint{3.459295in}{3.271854in}}{\pgfqpoint{3.469894in}{3.276244in}}{\pgfqpoint{3.477708in}{3.284058in}}%
\pgfpathcurveto{\pgfqpoint{3.485521in}{3.291871in}}{\pgfqpoint{3.489911in}{3.302470in}}{\pgfqpoint{3.489911in}{3.313520in}}%
\pgfpathcurveto{\pgfqpoint{3.489911in}{3.324571in}}{\pgfqpoint{3.485521in}{3.335170in}}{\pgfqpoint{3.477708in}{3.342983in}}%
\pgfpathcurveto{\pgfqpoint{3.469894in}{3.350797in}}{\pgfqpoint{3.459295in}{3.355187in}}{\pgfqpoint{3.448245in}{3.355187in}}%
\pgfpathcurveto{\pgfqpoint{3.437195in}{3.355187in}}{\pgfqpoint{3.426596in}{3.350797in}}{\pgfqpoint{3.418782in}{3.342983in}}%
\pgfpathcurveto{\pgfqpoint{3.410968in}{3.335170in}}{\pgfqpoint{3.406578in}{3.324571in}}{\pgfqpoint{3.406578in}{3.313520in}}%
\pgfpathcurveto{\pgfqpoint{3.406578in}{3.302470in}}{\pgfqpoint{3.410968in}{3.291871in}}{\pgfqpoint{3.418782in}{3.284058in}}%
\pgfpathcurveto{\pgfqpoint{3.426596in}{3.276244in}}{\pgfqpoint{3.437195in}{3.271854in}}{\pgfqpoint{3.448245in}{3.271854in}}%
\pgfpathclose%
\pgfusepath{stroke,fill}%
\end{pgfscope}%
\begin{pgfscope}%
\pgfpathrectangle{\pgfqpoint{0.481978in}{0.331635in}}{\pgfqpoint{4.960000in}{3.696000in}}%
\pgfusepath{clip}%
\pgfsetbuttcap%
\pgfsetroundjoin%
\definecolor{currentfill}{rgb}{0.631373,0.788235,0.956863}%
\pgfsetfillcolor{currentfill}%
\pgfsetlinewidth{0.481800pt}%
\definecolor{currentstroke}{rgb}{1.000000,1.000000,1.000000}%
\pgfsetstrokecolor{currentstroke}%
\pgfsetdash{}{0pt}%
\pgfpathmoveto{\pgfqpoint{2.657795in}{1.338091in}}%
\pgfpathcurveto{\pgfqpoint{2.668845in}{1.338091in}}{\pgfqpoint{2.679444in}{1.342481in}}{\pgfqpoint{2.687258in}{1.350295in}}%
\pgfpathcurveto{\pgfqpoint{2.695071in}{1.358109in}}{\pgfqpoint{2.699462in}{1.368708in}}{\pgfqpoint{2.699462in}{1.379758in}}%
\pgfpathcurveto{\pgfqpoint{2.699462in}{1.390808in}}{\pgfqpoint{2.695071in}{1.401407in}}{\pgfqpoint{2.687258in}{1.409221in}}%
\pgfpathcurveto{\pgfqpoint{2.679444in}{1.417034in}}{\pgfqpoint{2.668845in}{1.421424in}}{\pgfqpoint{2.657795in}{1.421424in}}%
\pgfpathcurveto{\pgfqpoint{2.646745in}{1.421424in}}{\pgfqpoint{2.636146in}{1.417034in}}{\pgfqpoint{2.628332in}{1.409221in}}%
\pgfpathcurveto{\pgfqpoint{2.620519in}{1.401407in}}{\pgfqpoint{2.616128in}{1.390808in}}{\pgfqpoint{2.616128in}{1.379758in}}%
\pgfpathcurveto{\pgfqpoint{2.616128in}{1.368708in}}{\pgfqpoint{2.620519in}{1.358109in}}{\pgfqpoint{2.628332in}{1.350295in}}%
\pgfpathcurveto{\pgfqpoint{2.636146in}{1.342481in}}{\pgfqpoint{2.646745in}{1.338091in}}{\pgfqpoint{2.657795in}{1.338091in}}%
\pgfpathclose%
\pgfusepath{stroke,fill}%
\end{pgfscope}%
\begin{pgfscope}%
\pgfpathrectangle{\pgfqpoint{0.481978in}{0.331635in}}{\pgfqpoint{4.960000in}{3.696000in}}%
\pgfusepath{clip}%
\pgfsetbuttcap%
\pgfsetroundjoin%
\definecolor{currentfill}{rgb}{0.631373,0.788235,0.956863}%
\pgfsetfillcolor{currentfill}%
\pgfsetlinewidth{0.481800pt}%
\definecolor{currentstroke}{rgb}{1.000000,1.000000,1.000000}%
\pgfsetstrokecolor{currentstroke}%
\pgfsetdash{}{0pt}%
\pgfpathmoveto{\pgfqpoint{3.546339in}{2.299935in}}%
\pgfpathcurveto{\pgfqpoint{3.557389in}{2.299935in}}{\pgfqpoint{3.567988in}{2.304326in}}{\pgfqpoint{3.575802in}{2.312139in}}%
\pgfpathcurveto{\pgfqpoint{3.583616in}{2.319953in}}{\pgfqpoint{3.588006in}{2.330552in}}{\pgfqpoint{3.588006in}{2.341602in}}%
\pgfpathcurveto{\pgfqpoint{3.588006in}{2.352652in}}{\pgfqpoint{3.583616in}{2.363251in}}{\pgfqpoint{3.575802in}{2.371065in}}%
\pgfpathcurveto{\pgfqpoint{3.567988in}{2.378878in}}{\pgfqpoint{3.557389in}{2.383269in}}{\pgfqpoint{3.546339in}{2.383269in}}%
\pgfpathcurveto{\pgfqpoint{3.535289in}{2.383269in}}{\pgfqpoint{3.524690in}{2.378878in}}{\pgfqpoint{3.516876in}{2.371065in}}%
\pgfpathcurveto{\pgfqpoint{3.509063in}{2.363251in}}{\pgfqpoint{3.504673in}{2.352652in}}{\pgfqpoint{3.504673in}{2.341602in}}%
\pgfpathcurveto{\pgfqpoint{3.504673in}{2.330552in}}{\pgfqpoint{3.509063in}{2.319953in}}{\pgfqpoint{3.516876in}{2.312139in}}%
\pgfpathcurveto{\pgfqpoint{3.524690in}{2.304326in}}{\pgfqpoint{3.535289in}{2.299935in}}{\pgfqpoint{3.546339in}{2.299935in}}%
\pgfpathclose%
\pgfusepath{stroke,fill}%
\end{pgfscope}%
\begin{pgfscope}%
\pgfpathrectangle{\pgfqpoint{0.481978in}{0.331635in}}{\pgfqpoint{4.960000in}{3.696000in}}%
\pgfusepath{clip}%
\pgfsetbuttcap%
\pgfsetroundjoin%
\definecolor{currentfill}{rgb}{1.000000,0.705882,0.509804}%
\pgfsetfillcolor{currentfill}%
\pgfsetlinewidth{0.481800pt}%
\definecolor{currentstroke}{rgb}{1.000000,1.000000,1.000000}%
\pgfsetstrokecolor{currentstroke}%
\pgfsetdash{}{0pt}%
\pgfpathmoveto{\pgfqpoint{3.014752in}{3.153624in}}%
\pgfpathcurveto{\pgfqpoint{3.025802in}{3.153624in}}{\pgfqpoint{3.036401in}{3.158014in}}{\pgfqpoint{3.044215in}{3.165828in}}%
\pgfpathcurveto{\pgfqpoint{3.052028in}{3.173642in}}{\pgfqpoint{3.056418in}{3.184241in}}{\pgfqpoint{3.056418in}{3.195291in}}%
\pgfpathcurveto{\pgfqpoint{3.056418in}{3.206341in}}{\pgfqpoint{3.052028in}{3.216940in}}{\pgfqpoint{3.044215in}{3.224754in}}%
\pgfpathcurveto{\pgfqpoint{3.036401in}{3.232567in}}{\pgfqpoint{3.025802in}{3.236957in}}{\pgfqpoint{3.014752in}{3.236957in}}%
\pgfpathcurveto{\pgfqpoint{3.003702in}{3.236957in}}{\pgfqpoint{2.993103in}{3.232567in}}{\pgfqpoint{2.985289in}{3.224754in}}%
\pgfpathcurveto{\pgfqpoint{2.977475in}{3.216940in}}{\pgfqpoint{2.973085in}{3.206341in}}{\pgfqpoint{2.973085in}{3.195291in}}%
\pgfpathcurveto{\pgfqpoint{2.973085in}{3.184241in}}{\pgfqpoint{2.977475in}{3.173642in}}{\pgfqpoint{2.985289in}{3.165828in}}%
\pgfpathcurveto{\pgfqpoint{2.993103in}{3.158014in}}{\pgfqpoint{3.003702in}{3.153624in}}{\pgfqpoint{3.014752in}{3.153624in}}%
\pgfpathclose%
\pgfusepath{stroke,fill}%
\end{pgfscope}%
\begin{pgfscope}%
\pgfpathrectangle{\pgfqpoint{0.481978in}{0.331635in}}{\pgfqpoint{4.960000in}{3.696000in}}%
\pgfusepath{clip}%
\pgfsetbuttcap%
\pgfsetroundjoin%
\definecolor{currentfill}{rgb}{1.000000,0.705882,0.509804}%
\pgfsetfillcolor{currentfill}%
\pgfsetlinewidth{0.481800pt}%
\definecolor{currentstroke}{rgb}{1.000000,1.000000,1.000000}%
\pgfsetstrokecolor{currentstroke}%
\pgfsetdash{}{0pt}%
\pgfpathmoveto{\pgfqpoint{5.085754in}{1.751706in}}%
\pgfpathcurveto{\pgfqpoint{5.096804in}{1.751706in}}{\pgfqpoint{5.107403in}{1.756096in}}{\pgfqpoint{5.115216in}{1.763910in}}%
\pgfpathcurveto{\pgfqpoint{5.123030in}{1.771723in}}{\pgfqpoint{5.127420in}{1.782322in}}{\pgfqpoint{5.127420in}{1.793372in}}%
\pgfpathcurveto{\pgfqpoint{5.127420in}{1.804422in}}{\pgfqpoint{5.123030in}{1.815021in}}{\pgfqpoint{5.115216in}{1.822835in}}%
\pgfpathcurveto{\pgfqpoint{5.107403in}{1.830649in}}{\pgfqpoint{5.096804in}{1.835039in}}{\pgfqpoint{5.085754in}{1.835039in}}%
\pgfpathcurveto{\pgfqpoint{5.074703in}{1.835039in}}{\pgfqpoint{5.064104in}{1.830649in}}{\pgfqpoint{5.056291in}{1.822835in}}%
\pgfpathcurveto{\pgfqpoint{5.048477in}{1.815021in}}{\pgfqpoint{5.044087in}{1.804422in}}{\pgfqpoint{5.044087in}{1.793372in}}%
\pgfpathcurveto{\pgfqpoint{5.044087in}{1.782322in}}{\pgfqpoint{5.048477in}{1.771723in}}{\pgfqpoint{5.056291in}{1.763910in}}%
\pgfpathcurveto{\pgfqpoint{5.064104in}{1.756096in}}{\pgfqpoint{5.074703in}{1.751706in}}{\pgfqpoint{5.085754in}{1.751706in}}%
\pgfpathclose%
\pgfusepath{stroke,fill}%
\end{pgfscope}%
\begin{pgfscope}%
\pgfpathrectangle{\pgfqpoint{0.481978in}{0.331635in}}{\pgfqpoint{4.960000in}{3.696000in}}%
\pgfusepath{clip}%
\pgfsetbuttcap%
\pgfsetroundjoin%
\definecolor{currentfill}{rgb}{1.000000,0.705882,0.509804}%
\pgfsetfillcolor{currentfill}%
\pgfsetlinewidth{0.481800pt}%
\definecolor{currentstroke}{rgb}{1.000000,1.000000,1.000000}%
\pgfsetstrokecolor{currentstroke}%
\pgfsetdash{}{0pt}%
\pgfpathmoveto{\pgfqpoint{2.284015in}{2.558577in}}%
\pgfpathcurveto{\pgfqpoint{2.295066in}{2.558577in}}{\pgfqpoint{2.305665in}{2.562967in}}{\pgfqpoint{2.313478in}{2.570781in}}%
\pgfpathcurveto{\pgfqpoint{2.321292in}{2.578594in}}{\pgfqpoint{2.325682in}{2.589193in}}{\pgfqpoint{2.325682in}{2.600243in}}%
\pgfpathcurveto{\pgfqpoint{2.325682in}{2.611293in}}{\pgfqpoint{2.321292in}{2.621892in}}{\pgfqpoint{2.313478in}{2.629706in}}%
\pgfpathcurveto{\pgfqpoint{2.305665in}{2.637520in}}{\pgfqpoint{2.295066in}{2.641910in}}{\pgfqpoint{2.284015in}{2.641910in}}%
\pgfpathcurveto{\pgfqpoint{2.272965in}{2.641910in}}{\pgfqpoint{2.262366in}{2.637520in}}{\pgfqpoint{2.254553in}{2.629706in}}%
\pgfpathcurveto{\pgfqpoint{2.246739in}{2.621892in}}{\pgfqpoint{2.242349in}{2.611293in}}{\pgfqpoint{2.242349in}{2.600243in}}%
\pgfpathcurveto{\pgfqpoint{2.242349in}{2.589193in}}{\pgfqpoint{2.246739in}{2.578594in}}{\pgfqpoint{2.254553in}{2.570781in}}%
\pgfpathcurveto{\pgfqpoint{2.262366in}{2.562967in}}{\pgfqpoint{2.272965in}{2.558577in}}{\pgfqpoint{2.284015in}{2.558577in}}%
\pgfpathclose%
\pgfusepath{stroke,fill}%
\end{pgfscope}%
\begin{pgfscope}%
\pgfpathrectangle{\pgfqpoint{0.481978in}{0.331635in}}{\pgfqpoint{4.960000in}{3.696000in}}%
\pgfusepath{clip}%
\pgfsetbuttcap%
\pgfsetroundjoin%
\definecolor{currentfill}{rgb}{1.000000,0.705882,0.509804}%
\pgfsetfillcolor{currentfill}%
\pgfsetlinewidth{0.481800pt}%
\definecolor{currentstroke}{rgb}{1.000000,1.000000,1.000000}%
\pgfsetstrokecolor{currentstroke}%
\pgfsetdash{}{0pt}%
\pgfpathmoveto{\pgfqpoint{1.922934in}{2.324891in}}%
\pgfpathcurveto{\pgfqpoint{1.933984in}{2.324891in}}{\pgfqpoint{1.944583in}{2.329282in}}{\pgfqpoint{1.952397in}{2.337095in}}%
\pgfpathcurveto{\pgfqpoint{1.960211in}{2.344909in}}{\pgfqpoint{1.964601in}{2.355508in}}{\pgfqpoint{1.964601in}{2.366558in}}%
\pgfpathcurveto{\pgfqpoint{1.964601in}{2.377608in}}{\pgfqpoint{1.960211in}{2.388207in}}{\pgfqpoint{1.952397in}{2.396021in}}%
\pgfpathcurveto{\pgfqpoint{1.944583in}{2.403834in}}{\pgfqpoint{1.933984in}{2.408225in}}{\pgfqpoint{1.922934in}{2.408225in}}%
\pgfpathcurveto{\pgfqpoint{1.911884in}{2.408225in}}{\pgfqpoint{1.901285in}{2.403834in}}{\pgfqpoint{1.893472in}{2.396021in}}%
\pgfpathcurveto{\pgfqpoint{1.885658in}{2.388207in}}{\pgfqpoint{1.881268in}{2.377608in}}{\pgfqpoint{1.881268in}{2.366558in}}%
\pgfpathcurveto{\pgfqpoint{1.881268in}{2.355508in}}{\pgfqpoint{1.885658in}{2.344909in}}{\pgfqpoint{1.893472in}{2.337095in}}%
\pgfpathcurveto{\pgfqpoint{1.901285in}{2.329282in}}{\pgfqpoint{1.911884in}{2.324891in}}{\pgfqpoint{1.922934in}{2.324891in}}%
\pgfpathclose%
\pgfusepath{stroke,fill}%
\end{pgfscope}%
\begin{pgfscope}%
\pgfpathrectangle{\pgfqpoint{0.481978in}{0.331635in}}{\pgfqpoint{4.960000in}{3.696000in}}%
\pgfusepath{clip}%
\pgfsetbuttcap%
\pgfsetroundjoin%
\definecolor{currentfill}{rgb}{1.000000,0.705882,0.509804}%
\pgfsetfillcolor{currentfill}%
\pgfsetlinewidth{0.481800pt}%
\definecolor{currentstroke}{rgb}{1.000000,1.000000,1.000000}%
\pgfsetstrokecolor{currentstroke}%
\pgfsetdash{}{0pt}%
\pgfpathmoveto{\pgfqpoint{1.598169in}{2.677784in}}%
\pgfpathcurveto{\pgfqpoint{1.609219in}{2.677784in}}{\pgfqpoint{1.619818in}{2.682175in}}{\pgfqpoint{1.627631in}{2.689988in}}%
\pgfpathcurveto{\pgfqpoint{1.635445in}{2.697802in}}{\pgfqpoint{1.639835in}{2.708401in}}{\pgfqpoint{1.639835in}{2.719451in}}%
\pgfpathcurveto{\pgfqpoint{1.639835in}{2.730501in}}{\pgfqpoint{1.635445in}{2.741100in}}{\pgfqpoint{1.627631in}{2.748914in}}%
\pgfpathcurveto{\pgfqpoint{1.619818in}{2.756728in}}{\pgfqpoint{1.609219in}{2.761118in}}{\pgfqpoint{1.598169in}{2.761118in}}%
\pgfpathcurveto{\pgfqpoint{1.587119in}{2.761118in}}{\pgfqpoint{1.576520in}{2.756728in}}{\pgfqpoint{1.568706in}{2.748914in}}%
\pgfpathcurveto{\pgfqpoint{1.560892in}{2.741100in}}{\pgfqpoint{1.556502in}{2.730501in}}{\pgfqpoint{1.556502in}{2.719451in}}%
\pgfpathcurveto{\pgfqpoint{1.556502in}{2.708401in}}{\pgfqpoint{1.560892in}{2.697802in}}{\pgfqpoint{1.568706in}{2.689988in}}%
\pgfpathcurveto{\pgfqpoint{1.576520in}{2.682175in}}{\pgfqpoint{1.587119in}{2.677784in}}{\pgfqpoint{1.598169in}{2.677784in}}%
\pgfpathclose%
\pgfusepath{stroke,fill}%
\end{pgfscope}%
\begin{pgfscope}%
\pgfpathrectangle{\pgfqpoint{0.481978in}{0.331635in}}{\pgfqpoint{4.960000in}{3.696000in}}%
\pgfusepath{clip}%
\pgfsetbuttcap%
\pgfsetroundjoin%
\definecolor{currentfill}{rgb}{1.000000,0.705882,0.509804}%
\pgfsetfillcolor{currentfill}%
\pgfsetlinewidth{0.481800pt}%
\definecolor{currentstroke}{rgb}{1.000000,1.000000,1.000000}%
\pgfsetstrokecolor{currentstroke}%
\pgfsetdash{}{0pt}%
\pgfpathmoveto{\pgfqpoint{3.105033in}{0.769246in}}%
\pgfpathcurveto{\pgfqpoint{3.116083in}{0.769246in}}{\pgfqpoint{3.126682in}{0.773636in}}{\pgfqpoint{3.134495in}{0.781450in}}%
\pgfpathcurveto{\pgfqpoint{3.142309in}{0.789263in}}{\pgfqpoint{3.146699in}{0.799862in}}{\pgfqpoint{3.146699in}{0.810912in}}%
\pgfpathcurveto{\pgfqpoint{3.146699in}{0.821962in}}{\pgfqpoint{3.142309in}{0.832562in}}{\pgfqpoint{3.134495in}{0.840375in}}%
\pgfpathcurveto{\pgfqpoint{3.126682in}{0.848189in}}{\pgfqpoint{3.116083in}{0.852579in}}{\pgfqpoint{3.105033in}{0.852579in}}%
\pgfpathcurveto{\pgfqpoint{3.093983in}{0.852579in}}{\pgfqpoint{3.083384in}{0.848189in}}{\pgfqpoint{3.075570in}{0.840375in}}%
\pgfpathcurveto{\pgfqpoint{3.067756in}{0.832562in}}{\pgfqpoint{3.063366in}{0.821962in}}{\pgfqpoint{3.063366in}{0.810912in}}%
\pgfpathcurveto{\pgfqpoint{3.063366in}{0.799862in}}{\pgfqpoint{3.067756in}{0.789263in}}{\pgfqpoint{3.075570in}{0.781450in}}%
\pgfpathcurveto{\pgfqpoint{3.083384in}{0.773636in}}{\pgfqpoint{3.093983in}{0.769246in}}{\pgfqpoint{3.105033in}{0.769246in}}%
\pgfpathclose%
\pgfusepath{stroke,fill}%
\end{pgfscope}%
\begin{pgfscope}%
\pgfpathrectangle{\pgfqpoint{0.481978in}{0.331635in}}{\pgfqpoint{4.960000in}{3.696000in}}%
\pgfusepath{clip}%
\pgfsetbuttcap%
\pgfsetroundjoin%
\definecolor{currentfill}{rgb}{1.000000,0.705882,0.509804}%
\pgfsetfillcolor{currentfill}%
\pgfsetlinewidth{0.481800pt}%
\definecolor{currentstroke}{rgb}{1.000000,1.000000,1.000000}%
\pgfsetstrokecolor{currentstroke}%
\pgfsetdash{}{0pt}%
\pgfpathmoveto{\pgfqpoint{1.895234in}{3.273531in}}%
\pgfpathcurveto{\pgfqpoint{1.906284in}{3.273531in}}{\pgfqpoint{1.916883in}{3.277921in}}{\pgfqpoint{1.924697in}{3.285735in}}%
\pgfpathcurveto{\pgfqpoint{1.932510in}{3.293548in}}{\pgfqpoint{1.936901in}{3.304147in}}{\pgfqpoint{1.936901in}{3.315197in}}%
\pgfpathcurveto{\pgfqpoint{1.936901in}{3.326248in}}{\pgfqpoint{1.932510in}{3.336847in}}{\pgfqpoint{1.924697in}{3.344660in}}%
\pgfpathcurveto{\pgfqpoint{1.916883in}{3.352474in}}{\pgfqpoint{1.906284in}{3.356864in}}{\pgfqpoint{1.895234in}{3.356864in}}%
\pgfpathcurveto{\pgfqpoint{1.884184in}{3.356864in}}{\pgfqpoint{1.873585in}{3.352474in}}{\pgfqpoint{1.865771in}{3.344660in}}%
\pgfpathcurveto{\pgfqpoint{1.857958in}{3.336847in}}{\pgfqpoint{1.853567in}{3.326248in}}{\pgfqpoint{1.853567in}{3.315197in}}%
\pgfpathcurveto{\pgfqpoint{1.853567in}{3.304147in}}{\pgfqpoint{1.857958in}{3.293548in}}{\pgfqpoint{1.865771in}{3.285735in}}%
\pgfpathcurveto{\pgfqpoint{1.873585in}{3.277921in}}{\pgfqpoint{1.884184in}{3.273531in}}{\pgfqpoint{1.895234in}{3.273531in}}%
\pgfpathclose%
\pgfusepath{stroke,fill}%
\end{pgfscope}%
\begin{pgfscope}%
\pgfpathrectangle{\pgfqpoint{0.481978in}{0.331635in}}{\pgfqpoint{4.960000in}{3.696000in}}%
\pgfusepath{clip}%
\pgfsetbuttcap%
\pgfsetroundjoin%
\definecolor{currentfill}{rgb}{1.000000,0.705882,0.509804}%
\pgfsetfillcolor{currentfill}%
\pgfsetlinewidth{0.481800pt}%
\definecolor{currentstroke}{rgb}{1.000000,1.000000,1.000000}%
\pgfsetstrokecolor{currentstroke}%
\pgfsetdash{}{0pt}%
\pgfpathmoveto{\pgfqpoint{3.360808in}{1.083842in}}%
\pgfpathcurveto{\pgfqpoint{3.371858in}{1.083842in}}{\pgfqpoint{3.382457in}{1.088232in}}{\pgfqpoint{3.390271in}{1.096046in}}%
\pgfpathcurveto{\pgfqpoint{3.398084in}{1.103859in}}{\pgfqpoint{3.402474in}{1.114458in}}{\pgfqpoint{3.402474in}{1.125508in}}%
\pgfpathcurveto{\pgfqpoint{3.402474in}{1.136558in}}{\pgfqpoint{3.398084in}{1.147157in}}{\pgfqpoint{3.390271in}{1.154971in}}%
\pgfpathcurveto{\pgfqpoint{3.382457in}{1.162785in}}{\pgfqpoint{3.371858in}{1.167175in}}{\pgfqpoint{3.360808in}{1.167175in}}%
\pgfpathcurveto{\pgfqpoint{3.349758in}{1.167175in}}{\pgfqpoint{3.339159in}{1.162785in}}{\pgfqpoint{3.331345in}{1.154971in}}%
\pgfpathcurveto{\pgfqpoint{3.323531in}{1.147157in}}{\pgfqpoint{3.319141in}{1.136558in}}{\pgfqpoint{3.319141in}{1.125508in}}%
\pgfpathcurveto{\pgfqpoint{3.319141in}{1.114458in}}{\pgfqpoint{3.323531in}{1.103859in}}{\pgfqpoint{3.331345in}{1.096046in}}%
\pgfpathcurveto{\pgfqpoint{3.339159in}{1.088232in}}{\pgfqpoint{3.349758in}{1.083842in}}{\pgfqpoint{3.360808in}{1.083842in}}%
\pgfpathclose%
\pgfusepath{stroke,fill}%
\end{pgfscope}%
\begin{pgfscope}%
\pgfpathrectangle{\pgfqpoint{0.481978in}{0.331635in}}{\pgfqpoint{4.960000in}{3.696000in}}%
\pgfusepath{clip}%
\pgfsetbuttcap%
\pgfsetroundjoin%
\definecolor{currentfill}{rgb}{1.000000,0.705882,0.509804}%
\pgfsetfillcolor{currentfill}%
\pgfsetlinewidth{0.481800pt}%
\definecolor{currentstroke}{rgb}{1.000000,1.000000,1.000000}%
\pgfsetstrokecolor{currentstroke}%
\pgfsetdash{}{0pt}%
\pgfpathmoveto{\pgfqpoint{4.027785in}{2.127986in}}%
\pgfpathcurveto{\pgfqpoint{4.038835in}{2.127986in}}{\pgfqpoint{4.049434in}{2.132376in}}{\pgfqpoint{4.057247in}{2.140190in}}%
\pgfpathcurveto{\pgfqpoint{4.065061in}{2.148003in}}{\pgfqpoint{4.069451in}{2.158602in}}{\pgfqpoint{4.069451in}{2.169652in}}%
\pgfpathcurveto{\pgfqpoint{4.069451in}{2.180703in}}{\pgfqpoint{4.065061in}{2.191302in}}{\pgfqpoint{4.057247in}{2.199115in}}%
\pgfpathcurveto{\pgfqpoint{4.049434in}{2.206929in}}{\pgfqpoint{4.038835in}{2.211319in}}{\pgfqpoint{4.027785in}{2.211319in}}%
\pgfpathcurveto{\pgfqpoint{4.016735in}{2.211319in}}{\pgfqpoint{4.006136in}{2.206929in}}{\pgfqpoint{3.998322in}{2.199115in}}%
\pgfpathcurveto{\pgfqpoint{3.990508in}{2.191302in}}{\pgfqpoint{3.986118in}{2.180703in}}{\pgfqpoint{3.986118in}{2.169652in}}%
\pgfpathcurveto{\pgfqpoint{3.986118in}{2.158602in}}{\pgfqpoint{3.990508in}{2.148003in}}{\pgfqpoint{3.998322in}{2.140190in}}%
\pgfpathcurveto{\pgfqpoint{4.006136in}{2.132376in}}{\pgfqpoint{4.016735in}{2.127986in}}{\pgfqpoint{4.027785in}{2.127986in}}%
\pgfpathclose%
\pgfusepath{stroke,fill}%
\end{pgfscope}%
\begin{pgfscope}%
\pgfpathrectangle{\pgfqpoint{0.481978in}{0.331635in}}{\pgfqpoint{4.960000in}{3.696000in}}%
\pgfusepath{clip}%
\pgfsetbuttcap%
\pgfsetroundjoin%
\definecolor{currentfill}{rgb}{1.000000,0.705882,0.509804}%
\pgfsetfillcolor{currentfill}%
\pgfsetlinewidth{0.481800pt}%
\definecolor{currentstroke}{rgb}{1.000000,1.000000,1.000000}%
\pgfsetstrokecolor{currentstroke}%
\pgfsetdash{}{0pt}%
\pgfpathmoveto{\pgfqpoint{1.211108in}{2.423451in}}%
\pgfpathcurveto{\pgfqpoint{1.222158in}{2.423451in}}{\pgfqpoint{1.232757in}{2.427842in}}{\pgfqpoint{1.240570in}{2.435655in}}%
\pgfpathcurveto{\pgfqpoint{1.248384in}{2.443469in}}{\pgfqpoint{1.252774in}{2.454068in}}{\pgfqpoint{1.252774in}{2.465118in}}%
\pgfpathcurveto{\pgfqpoint{1.252774in}{2.476168in}}{\pgfqpoint{1.248384in}{2.486767in}}{\pgfqpoint{1.240570in}{2.494581in}}%
\pgfpathcurveto{\pgfqpoint{1.232757in}{2.502395in}}{\pgfqpoint{1.222158in}{2.506785in}}{\pgfqpoint{1.211108in}{2.506785in}}%
\pgfpathcurveto{\pgfqpoint{1.200057in}{2.506785in}}{\pgfqpoint{1.189458in}{2.502395in}}{\pgfqpoint{1.181645in}{2.494581in}}%
\pgfpathcurveto{\pgfqpoint{1.173831in}{2.486767in}}{\pgfqpoint{1.169441in}{2.476168in}}{\pgfqpoint{1.169441in}{2.465118in}}%
\pgfpathcurveto{\pgfqpoint{1.169441in}{2.454068in}}{\pgfqpoint{1.173831in}{2.443469in}}{\pgfqpoint{1.181645in}{2.435655in}}%
\pgfpathcurveto{\pgfqpoint{1.189458in}{2.427842in}}{\pgfqpoint{1.200057in}{2.423451in}}{\pgfqpoint{1.211108in}{2.423451in}}%
\pgfpathclose%
\pgfusepath{stroke,fill}%
\end{pgfscope}%
\begin{pgfscope}%
\pgfpathrectangle{\pgfqpoint{0.481978in}{0.331635in}}{\pgfqpoint{4.960000in}{3.696000in}}%
\pgfusepath{clip}%
\pgfsetbuttcap%
\pgfsetroundjoin%
\definecolor{currentfill}{rgb}{1.000000,0.705882,0.509804}%
\pgfsetfillcolor{currentfill}%
\pgfsetlinewidth{0.481800pt}%
\definecolor{currentstroke}{rgb}{1.000000,1.000000,1.000000}%
\pgfsetstrokecolor{currentstroke}%
\pgfsetdash{}{0pt}%
\pgfpathmoveto{\pgfqpoint{4.184791in}{0.901415in}}%
\pgfpathcurveto{\pgfqpoint{4.195841in}{0.901415in}}{\pgfqpoint{4.206440in}{0.905805in}}{\pgfqpoint{4.214253in}{0.913619in}}%
\pgfpathcurveto{\pgfqpoint{4.222067in}{0.921433in}}{\pgfqpoint{4.226457in}{0.932032in}}{\pgfqpoint{4.226457in}{0.943082in}}%
\pgfpathcurveto{\pgfqpoint{4.226457in}{0.954132in}}{\pgfqpoint{4.222067in}{0.964731in}}{\pgfqpoint{4.214253in}{0.972545in}}%
\pgfpathcurveto{\pgfqpoint{4.206440in}{0.980358in}}{\pgfqpoint{4.195841in}{0.984749in}}{\pgfqpoint{4.184791in}{0.984749in}}%
\pgfpathcurveto{\pgfqpoint{4.173740in}{0.984749in}}{\pgfqpoint{4.163141in}{0.980358in}}{\pgfqpoint{4.155328in}{0.972545in}}%
\pgfpathcurveto{\pgfqpoint{4.147514in}{0.964731in}}{\pgfqpoint{4.143124in}{0.954132in}}{\pgfqpoint{4.143124in}{0.943082in}}%
\pgfpathcurveto{\pgfqpoint{4.143124in}{0.932032in}}{\pgfqpoint{4.147514in}{0.921433in}}{\pgfqpoint{4.155328in}{0.913619in}}%
\pgfpathcurveto{\pgfqpoint{4.163141in}{0.905805in}}{\pgfqpoint{4.173740in}{0.901415in}}{\pgfqpoint{4.184791in}{0.901415in}}%
\pgfpathclose%
\pgfusepath{stroke,fill}%
\end{pgfscope}%
\begin{pgfscope}%
\pgfpathrectangle{\pgfqpoint{0.481978in}{0.331635in}}{\pgfqpoint{4.960000in}{3.696000in}}%
\pgfusepath{clip}%
\pgfsetbuttcap%
\pgfsetroundjoin%
\definecolor{currentfill}{rgb}{1.000000,0.705882,0.509804}%
\pgfsetfillcolor{currentfill}%
\pgfsetlinewidth{0.481800pt}%
\definecolor{currentstroke}{rgb}{1.000000,1.000000,1.000000}%
\pgfsetstrokecolor{currentstroke}%
\pgfsetdash{}{0pt}%
\pgfpathmoveto{\pgfqpoint{1.789027in}{1.264166in}}%
\pgfpathcurveto{\pgfqpoint{1.800077in}{1.264166in}}{\pgfqpoint{1.810677in}{1.268556in}}{\pgfqpoint{1.818490in}{1.276370in}}%
\pgfpathcurveto{\pgfqpoint{1.826304in}{1.284183in}}{\pgfqpoint{1.830694in}{1.294782in}}{\pgfqpoint{1.830694in}{1.305832in}}%
\pgfpathcurveto{\pgfqpoint{1.830694in}{1.316882in}}{\pgfqpoint{1.826304in}{1.327482in}}{\pgfqpoint{1.818490in}{1.335295in}}%
\pgfpathcurveto{\pgfqpoint{1.810677in}{1.343109in}}{\pgfqpoint{1.800077in}{1.347499in}}{\pgfqpoint{1.789027in}{1.347499in}}%
\pgfpathcurveto{\pgfqpoint{1.777977in}{1.347499in}}{\pgfqpoint{1.767378in}{1.343109in}}{\pgfqpoint{1.759565in}{1.335295in}}%
\pgfpathcurveto{\pgfqpoint{1.751751in}{1.327482in}}{\pgfqpoint{1.747361in}{1.316882in}}{\pgfqpoint{1.747361in}{1.305832in}}%
\pgfpathcurveto{\pgfqpoint{1.747361in}{1.294782in}}{\pgfqpoint{1.751751in}{1.284183in}}{\pgfqpoint{1.759565in}{1.276370in}}%
\pgfpathcurveto{\pgfqpoint{1.767378in}{1.268556in}}{\pgfqpoint{1.777977in}{1.264166in}}{\pgfqpoint{1.789027in}{1.264166in}}%
\pgfpathclose%
\pgfusepath{stroke,fill}%
\end{pgfscope}%
\begin{pgfscope}%
\pgfpathrectangle{\pgfqpoint{0.481978in}{0.331635in}}{\pgfqpoint{4.960000in}{3.696000in}}%
\pgfusepath{clip}%
\pgfsetbuttcap%
\pgfsetroundjoin%
\definecolor{currentfill}{rgb}{1.000000,0.705882,0.509804}%
\pgfsetfillcolor{currentfill}%
\pgfsetlinewidth{0.481800pt}%
\definecolor{currentstroke}{rgb}{1.000000,1.000000,1.000000}%
\pgfsetstrokecolor{currentstroke}%
\pgfsetdash{}{0pt}%
\pgfpathmoveto{\pgfqpoint{2.368419in}{0.963626in}}%
\pgfpathcurveto{\pgfqpoint{2.379470in}{0.963626in}}{\pgfqpoint{2.390069in}{0.968016in}}{\pgfqpoint{2.397882in}{0.975829in}}%
\pgfpathcurveto{\pgfqpoint{2.405696in}{0.983643in}}{\pgfqpoint{2.410086in}{0.994242in}}{\pgfqpoint{2.410086in}{1.005292in}}%
\pgfpathcurveto{\pgfqpoint{2.410086in}{1.016342in}}{\pgfqpoint{2.405696in}{1.026941in}}{\pgfqpoint{2.397882in}{1.034755in}}%
\pgfpathcurveto{\pgfqpoint{2.390069in}{1.042569in}}{\pgfqpoint{2.379470in}{1.046959in}}{\pgfqpoint{2.368419in}{1.046959in}}%
\pgfpathcurveto{\pgfqpoint{2.357369in}{1.046959in}}{\pgfqpoint{2.346770in}{1.042569in}}{\pgfqpoint{2.338957in}{1.034755in}}%
\pgfpathcurveto{\pgfqpoint{2.331143in}{1.026941in}}{\pgfqpoint{2.326753in}{1.016342in}}{\pgfqpoint{2.326753in}{1.005292in}}%
\pgfpathcurveto{\pgfqpoint{2.326753in}{0.994242in}}{\pgfqpoint{2.331143in}{0.983643in}}{\pgfqpoint{2.338957in}{0.975829in}}%
\pgfpathcurveto{\pgfqpoint{2.346770in}{0.968016in}}{\pgfqpoint{2.357369in}{0.963626in}}{\pgfqpoint{2.368419in}{0.963626in}}%
\pgfpathclose%
\pgfusepath{stroke,fill}%
\end{pgfscope}%
\begin{pgfscope}%
\pgfpathrectangle{\pgfqpoint{0.481978in}{0.331635in}}{\pgfqpoint{4.960000in}{3.696000in}}%
\pgfusepath{clip}%
\pgfsetbuttcap%
\pgfsetroundjoin%
\definecolor{currentfill}{rgb}{1.000000,0.705882,0.509804}%
\pgfsetfillcolor{currentfill}%
\pgfsetlinewidth{0.481800pt}%
\definecolor{currentstroke}{rgb}{1.000000,1.000000,1.000000}%
\pgfsetstrokecolor{currentstroke}%
\pgfsetdash{}{0pt}%
\pgfpathmoveto{\pgfqpoint{1.427964in}{2.922774in}}%
\pgfpathcurveto{\pgfqpoint{1.439014in}{2.922774in}}{\pgfqpoint{1.449613in}{2.927164in}}{\pgfqpoint{1.457427in}{2.934978in}}%
\pgfpathcurveto{\pgfqpoint{1.465240in}{2.942791in}}{\pgfqpoint{1.469631in}{2.953390in}}{\pgfqpoint{1.469631in}{2.964440in}}%
\pgfpathcurveto{\pgfqpoint{1.469631in}{2.975490in}}{\pgfqpoint{1.465240in}{2.986089in}}{\pgfqpoint{1.457427in}{2.993903in}}%
\pgfpathcurveto{\pgfqpoint{1.449613in}{3.001717in}}{\pgfqpoint{1.439014in}{3.006107in}}{\pgfqpoint{1.427964in}{3.006107in}}%
\pgfpathcurveto{\pgfqpoint{1.416914in}{3.006107in}}{\pgfqpoint{1.406315in}{3.001717in}}{\pgfqpoint{1.398501in}{2.993903in}}%
\pgfpathcurveto{\pgfqpoint{1.390687in}{2.986089in}}{\pgfqpoint{1.386297in}{2.975490in}}{\pgfqpoint{1.386297in}{2.964440in}}%
\pgfpathcurveto{\pgfqpoint{1.386297in}{2.953390in}}{\pgfqpoint{1.390687in}{2.942791in}}{\pgfqpoint{1.398501in}{2.934978in}}%
\pgfpathcurveto{\pgfqpoint{1.406315in}{2.927164in}}{\pgfqpoint{1.416914in}{2.922774in}}{\pgfqpoint{1.427964in}{2.922774in}}%
\pgfpathclose%
\pgfusepath{stroke,fill}%
\end{pgfscope}%
\begin{pgfscope}%
\pgfpathrectangle{\pgfqpoint{0.481978in}{0.331635in}}{\pgfqpoint{4.960000in}{3.696000in}}%
\pgfusepath{clip}%
\pgfsetbuttcap%
\pgfsetroundjoin%
\definecolor{currentfill}{rgb}{1.000000,0.705882,0.509804}%
\pgfsetfillcolor{currentfill}%
\pgfsetlinewidth{0.481800pt}%
\definecolor{currentstroke}{rgb}{1.000000,1.000000,1.000000}%
\pgfsetstrokecolor{currentstroke}%
\pgfsetdash{}{0pt}%
\pgfpathmoveto{\pgfqpoint{3.393546in}{3.817968in}}%
\pgfpathcurveto{\pgfqpoint{3.404596in}{3.817968in}}{\pgfqpoint{3.415195in}{3.822359in}}{\pgfqpoint{3.423008in}{3.830172in}}%
\pgfpathcurveto{\pgfqpoint{3.430822in}{3.837986in}}{\pgfqpoint{3.435212in}{3.848585in}}{\pgfqpoint{3.435212in}{3.859635in}}%
\pgfpathcurveto{\pgfqpoint{3.435212in}{3.870685in}}{\pgfqpoint{3.430822in}{3.881284in}}{\pgfqpoint{3.423008in}{3.889098in}}%
\pgfpathcurveto{\pgfqpoint{3.415195in}{3.896911in}}{\pgfqpoint{3.404596in}{3.901302in}}{\pgfqpoint{3.393546in}{3.901302in}}%
\pgfpathcurveto{\pgfqpoint{3.382495in}{3.901302in}}{\pgfqpoint{3.371896in}{3.896911in}}{\pgfqpoint{3.364083in}{3.889098in}}%
\pgfpathcurveto{\pgfqpoint{3.356269in}{3.881284in}}{\pgfqpoint{3.351879in}{3.870685in}}{\pgfqpoint{3.351879in}{3.859635in}}%
\pgfpathcurveto{\pgfqpoint{3.351879in}{3.848585in}}{\pgfqpoint{3.356269in}{3.837986in}}{\pgfqpoint{3.364083in}{3.830172in}}%
\pgfpathcurveto{\pgfqpoint{3.371896in}{3.822359in}}{\pgfqpoint{3.382495in}{3.817968in}}{\pgfqpoint{3.393546in}{3.817968in}}%
\pgfpathclose%
\pgfusepath{stroke,fill}%
\end{pgfscope}%
\begin{pgfscope}%
\pgfpathrectangle{\pgfqpoint{0.481978in}{0.331635in}}{\pgfqpoint{4.960000in}{3.696000in}}%
\pgfusepath{clip}%
\pgfsetbuttcap%
\pgfsetroundjoin%
\definecolor{currentfill}{rgb}{1.000000,0.705882,0.509804}%
\pgfsetfillcolor{currentfill}%
\pgfsetlinewidth{0.481800pt}%
\definecolor{currentstroke}{rgb}{1.000000,1.000000,1.000000}%
\pgfsetstrokecolor{currentstroke}%
\pgfsetdash{}{0pt}%
\pgfpathmoveto{\pgfqpoint{1.952110in}{1.586388in}}%
\pgfpathcurveto{\pgfqpoint{1.963160in}{1.586388in}}{\pgfqpoint{1.973759in}{1.590779in}}{\pgfqpoint{1.981573in}{1.598592in}}%
\pgfpathcurveto{\pgfqpoint{1.989386in}{1.606406in}}{\pgfqpoint{1.993777in}{1.617005in}}{\pgfqpoint{1.993777in}{1.628055in}}%
\pgfpathcurveto{\pgfqpoint{1.993777in}{1.639105in}}{\pgfqpoint{1.989386in}{1.649704in}}{\pgfqpoint{1.981573in}{1.657518in}}%
\pgfpathcurveto{\pgfqpoint{1.973759in}{1.665331in}}{\pgfqpoint{1.963160in}{1.669722in}}{\pgfqpoint{1.952110in}{1.669722in}}%
\pgfpathcurveto{\pgfqpoint{1.941060in}{1.669722in}}{\pgfqpoint{1.930461in}{1.665331in}}{\pgfqpoint{1.922647in}{1.657518in}}%
\pgfpathcurveto{\pgfqpoint{1.914833in}{1.649704in}}{\pgfqpoint{1.910443in}{1.639105in}}{\pgfqpoint{1.910443in}{1.628055in}}%
\pgfpathcurveto{\pgfqpoint{1.910443in}{1.617005in}}{\pgfqpoint{1.914833in}{1.606406in}}{\pgfqpoint{1.922647in}{1.598592in}}%
\pgfpathcurveto{\pgfqpoint{1.930461in}{1.590779in}}{\pgfqpoint{1.941060in}{1.586388in}}{\pgfqpoint{1.952110in}{1.586388in}}%
\pgfpathclose%
\pgfusepath{stroke,fill}%
\end{pgfscope}%
\begin{pgfscope}%
\pgfpathrectangle{\pgfqpoint{0.481978in}{0.331635in}}{\pgfqpoint{4.960000in}{3.696000in}}%
\pgfusepath{clip}%
\pgfsetbuttcap%
\pgfsetroundjoin%
\definecolor{currentfill}{rgb}{1.000000,0.705882,0.509804}%
\pgfsetfillcolor{currentfill}%
\pgfsetlinewidth{0.481800pt}%
\definecolor{currentstroke}{rgb}{1.000000,1.000000,1.000000}%
\pgfsetstrokecolor{currentstroke}%
\pgfsetdash{}{0pt}%
\pgfpathmoveto{\pgfqpoint{2.726428in}{2.815925in}}%
\pgfpathcurveto{\pgfqpoint{2.737479in}{2.815925in}}{\pgfqpoint{2.748078in}{2.820315in}}{\pgfqpoint{2.755891in}{2.828128in}}%
\pgfpathcurveto{\pgfqpoint{2.763705in}{2.835942in}}{\pgfqpoint{2.768095in}{2.846541in}}{\pgfqpoint{2.768095in}{2.857591in}}%
\pgfpathcurveto{\pgfqpoint{2.768095in}{2.868641in}}{\pgfqpoint{2.763705in}{2.879240in}}{\pgfqpoint{2.755891in}{2.887054in}}%
\pgfpathcurveto{\pgfqpoint{2.748078in}{2.894868in}}{\pgfqpoint{2.737479in}{2.899258in}}{\pgfqpoint{2.726428in}{2.899258in}}%
\pgfpathcurveto{\pgfqpoint{2.715378in}{2.899258in}}{\pgfqpoint{2.704779in}{2.894868in}}{\pgfqpoint{2.696966in}{2.887054in}}%
\pgfpathcurveto{\pgfqpoint{2.689152in}{2.879240in}}{\pgfqpoint{2.684762in}{2.868641in}}{\pgfqpoint{2.684762in}{2.857591in}}%
\pgfpathcurveto{\pgfqpoint{2.684762in}{2.846541in}}{\pgfqpoint{2.689152in}{2.835942in}}{\pgfqpoint{2.696966in}{2.828128in}}%
\pgfpathcurveto{\pgfqpoint{2.704779in}{2.820315in}}{\pgfqpoint{2.715378in}{2.815925in}}{\pgfqpoint{2.726428in}{2.815925in}}%
\pgfpathclose%
\pgfusepath{stroke,fill}%
\end{pgfscope}%
\begin{pgfscope}%
\pgfpathrectangle{\pgfqpoint{0.481978in}{0.331635in}}{\pgfqpoint{4.960000in}{3.696000in}}%
\pgfusepath{clip}%
\pgfsetbuttcap%
\pgfsetroundjoin%
\definecolor{currentfill}{rgb}{1.000000,0.705882,0.509804}%
\pgfsetfillcolor{currentfill}%
\pgfsetlinewidth{0.481800pt}%
\definecolor{currentstroke}{rgb}{1.000000,1.000000,1.000000}%
\pgfsetstrokecolor{currentstroke}%
\pgfsetdash{}{0pt}%
\pgfpathmoveto{\pgfqpoint{1.372997in}{1.814536in}}%
\pgfpathcurveto{\pgfqpoint{1.384048in}{1.814536in}}{\pgfqpoint{1.394647in}{1.818926in}}{\pgfqpoint{1.402460in}{1.826740in}}%
\pgfpathcurveto{\pgfqpoint{1.410274in}{1.834554in}}{\pgfqpoint{1.414664in}{1.845153in}}{\pgfqpoint{1.414664in}{1.856203in}}%
\pgfpathcurveto{\pgfqpoint{1.414664in}{1.867253in}}{\pgfqpoint{1.410274in}{1.877852in}}{\pgfqpoint{1.402460in}{1.885666in}}%
\pgfpathcurveto{\pgfqpoint{1.394647in}{1.893479in}}{\pgfqpoint{1.384048in}{1.897869in}}{\pgfqpoint{1.372997in}{1.897869in}}%
\pgfpathcurveto{\pgfqpoint{1.361947in}{1.897869in}}{\pgfqpoint{1.351348in}{1.893479in}}{\pgfqpoint{1.343535in}{1.885666in}}%
\pgfpathcurveto{\pgfqpoint{1.335721in}{1.877852in}}{\pgfqpoint{1.331331in}{1.867253in}}{\pgfqpoint{1.331331in}{1.856203in}}%
\pgfpathcurveto{\pgfqpoint{1.331331in}{1.845153in}}{\pgfqpoint{1.335721in}{1.834554in}}{\pgfqpoint{1.343535in}{1.826740in}}%
\pgfpathcurveto{\pgfqpoint{1.351348in}{1.818926in}}{\pgfqpoint{1.361947in}{1.814536in}}{\pgfqpoint{1.372997in}{1.814536in}}%
\pgfpathclose%
\pgfusepath{stroke,fill}%
\end{pgfscope}%
\begin{pgfscope}%
\pgfpathrectangle{\pgfqpoint{0.481978in}{0.331635in}}{\pgfqpoint{4.960000in}{3.696000in}}%
\pgfusepath{clip}%
\pgfsetbuttcap%
\pgfsetroundjoin%
\definecolor{currentfill}{rgb}{1.000000,0.705882,0.509804}%
\pgfsetfillcolor{currentfill}%
\pgfsetlinewidth{0.481800pt}%
\definecolor{currentstroke}{rgb}{1.000000,1.000000,1.000000}%
\pgfsetstrokecolor{currentstroke}%
\pgfsetdash{}{0pt}%
\pgfpathmoveto{\pgfqpoint{0.707432in}{3.102544in}}%
\pgfpathcurveto{\pgfqpoint{0.718483in}{3.102544in}}{\pgfqpoint{0.729082in}{3.106934in}}{\pgfqpoint{0.736895in}{3.114748in}}%
\pgfpathcurveto{\pgfqpoint{0.744709in}{3.122561in}}{\pgfqpoint{0.749099in}{3.133160in}}{\pgfqpoint{0.749099in}{3.144211in}}%
\pgfpathcurveto{\pgfqpoint{0.749099in}{3.155261in}}{\pgfqpoint{0.744709in}{3.165860in}}{\pgfqpoint{0.736895in}{3.173673in}}%
\pgfpathcurveto{\pgfqpoint{0.729082in}{3.181487in}}{\pgfqpoint{0.718483in}{3.185877in}}{\pgfqpoint{0.707432in}{3.185877in}}%
\pgfpathcurveto{\pgfqpoint{0.696382in}{3.185877in}}{\pgfqpoint{0.685783in}{3.181487in}}{\pgfqpoint{0.677970in}{3.173673in}}%
\pgfpathcurveto{\pgfqpoint{0.670156in}{3.165860in}}{\pgfqpoint{0.665766in}{3.155261in}}{\pgfqpoint{0.665766in}{3.144211in}}%
\pgfpathcurveto{\pgfqpoint{0.665766in}{3.133160in}}{\pgfqpoint{0.670156in}{3.122561in}}{\pgfqpoint{0.677970in}{3.114748in}}%
\pgfpathcurveto{\pgfqpoint{0.685783in}{3.106934in}}{\pgfqpoint{0.696382in}{3.102544in}}{\pgfqpoint{0.707432in}{3.102544in}}%
\pgfpathclose%
\pgfusepath{stroke,fill}%
\end{pgfscope}%
\begin{pgfscope}%
\pgfpathrectangle{\pgfqpoint{0.481978in}{0.331635in}}{\pgfqpoint{4.960000in}{3.696000in}}%
\pgfusepath{clip}%
\pgfsetbuttcap%
\pgfsetroundjoin%
\definecolor{currentfill}{rgb}{1.000000,0.705882,0.509804}%
\pgfsetfillcolor{currentfill}%
\pgfsetlinewidth{0.481800pt}%
\definecolor{currentstroke}{rgb}{1.000000,1.000000,1.000000}%
\pgfsetstrokecolor{currentstroke}%
\pgfsetdash{}{0pt}%
\pgfpathmoveto{\pgfqpoint{3.782152in}{0.457968in}}%
\pgfpathcurveto{\pgfqpoint{3.793202in}{0.457968in}}{\pgfqpoint{3.803801in}{0.462359in}}{\pgfqpoint{3.811614in}{0.470172in}}%
\pgfpathcurveto{\pgfqpoint{3.819428in}{0.477986in}}{\pgfqpoint{3.823818in}{0.488585in}}{\pgfqpoint{3.823818in}{0.499635in}}%
\pgfpathcurveto{\pgfqpoint{3.823818in}{0.510685in}}{\pgfqpoint{3.819428in}{0.521284in}}{\pgfqpoint{3.811614in}{0.529098in}}%
\pgfpathcurveto{\pgfqpoint{3.803801in}{0.536911in}}{\pgfqpoint{3.793202in}{0.541302in}}{\pgfqpoint{3.782152in}{0.541302in}}%
\pgfpathcurveto{\pgfqpoint{3.771101in}{0.541302in}}{\pgfqpoint{3.760502in}{0.536911in}}{\pgfqpoint{3.752689in}{0.529098in}}%
\pgfpathcurveto{\pgfqpoint{3.744875in}{0.521284in}}{\pgfqpoint{3.740485in}{0.510685in}}{\pgfqpoint{3.740485in}{0.499635in}}%
\pgfpathcurveto{\pgfqpoint{3.740485in}{0.488585in}}{\pgfqpoint{3.744875in}{0.477986in}}{\pgfqpoint{3.752689in}{0.470172in}}%
\pgfpathcurveto{\pgfqpoint{3.760502in}{0.462359in}}{\pgfqpoint{3.771101in}{0.457968in}}{\pgfqpoint{3.782152in}{0.457968in}}%
\pgfpathclose%
\pgfusepath{stroke,fill}%
\end{pgfscope}%
\begin{pgfscope}%
\pgfpathrectangle{\pgfqpoint{0.481978in}{0.331635in}}{\pgfqpoint{4.960000in}{3.696000in}}%
\pgfusepath{clip}%
\pgfsetbuttcap%
\pgfsetroundjoin%
\definecolor{currentfill}{rgb}{1.000000,0.705882,0.509804}%
\pgfsetfillcolor{currentfill}%
\pgfsetlinewidth{0.481800pt}%
\definecolor{currentstroke}{rgb}{1.000000,1.000000,1.000000}%
\pgfsetstrokecolor{currentstroke}%
\pgfsetdash{}{0pt}%
\pgfpathmoveto{\pgfqpoint{4.764267in}{2.705857in}}%
\pgfpathcurveto{\pgfqpoint{4.775318in}{2.705857in}}{\pgfqpoint{4.785917in}{2.710247in}}{\pgfqpoint{4.793730in}{2.718061in}}%
\pgfpathcurveto{\pgfqpoint{4.801544in}{2.725875in}}{\pgfqpoint{4.805934in}{2.736474in}}{\pgfqpoint{4.805934in}{2.747524in}}%
\pgfpathcurveto{\pgfqpoint{4.805934in}{2.758574in}}{\pgfqpoint{4.801544in}{2.769173in}}{\pgfqpoint{4.793730in}{2.776986in}}%
\pgfpathcurveto{\pgfqpoint{4.785917in}{2.784800in}}{\pgfqpoint{4.775318in}{2.789190in}}{\pgfqpoint{4.764267in}{2.789190in}}%
\pgfpathcurveto{\pgfqpoint{4.753217in}{2.789190in}}{\pgfqpoint{4.742618in}{2.784800in}}{\pgfqpoint{4.734805in}{2.776986in}}%
\pgfpathcurveto{\pgfqpoint{4.726991in}{2.769173in}}{\pgfqpoint{4.722601in}{2.758574in}}{\pgfqpoint{4.722601in}{2.747524in}}%
\pgfpathcurveto{\pgfqpoint{4.722601in}{2.736474in}}{\pgfqpoint{4.726991in}{2.725875in}}{\pgfqpoint{4.734805in}{2.718061in}}%
\pgfpathcurveto{\pgfqpoint{4.742618in}{2.710247in}}{\pgfqpoint{4.753217in}{2.705857in}}{\pgfqpoint{4.764267in}{2.705857in}}%
\pgfpathclose%
\pgfusepath{stroke,fill}%
\end{pgfscope}%
\begin{pgfscope}%
\pgfpathrectangle{\pgfqpoint{0.481978in}{0.331635in}}{\pgfqpoint{4.960000in}{3.696000in}}%
\pgfusepath{clip}%
\pgfsetbuttcap%
\pgfsetroundjoin%
\definecolor{currentfill}{rgb}{1.000000,0.705882,0.509804}%
\pgfsetfillcolor{currentfill}%
\pgfsetlinewidth{0.481800pt}%
\definecolor{currentstroke}{rgb}{1.000000,1.000000,1.000000}%
\pgfsetstrokecolor{currentstroke}%
\pgfsetdash{}{0pt}%
\pgfpathmoveto{\pgfqpoint{1.027884in}{3.268783in}}%
\pgfpathcurveto{\pgfqpoint{1.038934in}{3.268783in}}{\pgfqpoint{1.049533in}{3.273173in}}{\pgfqpoint{1.057347in}{3.280987in}}%
\pgfpathcurveto{\pgfqpoint{1.065160in}{3.288800in}}{\pgfqpoint{1.069550in}{3.299399in}}{\pgfqpoint{1.069550in}{3.310449in}}%
\pgfpathcurveto{\pgfqpoint{1.069550in}{3.321500in}}{\pgfqpoint{1.065160in}{3.332099in}}{\pgfqpoint{1.057347in}{3.339912in}}%
\pgfpathcurveto{\pgfqpoint{1.049533in}{3.347726in}}{\pgfqpoint{1.038934in}{3.352116in}}{\pgfqpoint{1.027884in}{3.352116in}}%
\pgfpathcurveto{\pgfqpoint{1.016834in}{3.352116in}}{\pgfqpoint{1.006235in}{3.347726in}}{\pgfqpoint{0.998421in}{3.339912in}}%
\pgfpathcurveto{\pgfqpoint{0.990607in}{3.332099in}}{\pgfqpoint{0.986217in}{3.321500in}}{\pgfqpoint{0.986217in}{3.310449in}}%
\pgfpathcurveto{\pgfqpoint{0.986217in}{3.299399in}}{\pgfqpoint{0.990607in}{3.288800in}}{\pgfqpoint{0.998421in}{3.280987in}}%
\pgfpathcurveto{\pgfqpoint{1.006235in}{3.273173in}}{\pgfqpoint{1.016834in}{3.268783in}}{\pgfqpoint{1.027884in}{3.268783in}}%
\pgfpathclose%
\pgfusepath{stroke,fill}%
\end{pgfscope}%
\begin{pgfscope}%
\pgfpathrectangle{\pgfqpoint{0.481978in}{0.331635in}}{\pgfqpoint{4.960000in}{3.696000in}}%
\pgfusepath{clip}%
\pgfsetbuttcap%
\pgfsetroundjoin%
\definecolor{currentfill}{rgb}{1.000000,0.705882,0.509804}%
\pgfsetfillcolor{currentfill}%
\pgfsetlinewidth{0.481800pt}%
\definecolor{currentstroke}{rgb}{1.000000,1.000000,1.000000}%
\pgfsetstrokecolor{currentstroke}%
\pgfsetdash{}{0pt}%
\pgfpathmoveto{\pgfqpoint{2.571490in}{3.546079in}}%
\pgfpathcurveto{\pgfqpoint{2.582540in}{3.546079in}}{\pgfqpoint{2.593139in}{3.550469in}}{\pgfqpoint{2.600953in}{3.558283in}}%
\pgfpathcurveto{\pgfqpoint{2.608766in}{3.566096in}}{\pgfqpoint{2.613157in}{3.576695in}}{\pgfqpoint{2.613157in}{3.587745in}}%
\pgfpathcurveto{\pgfqpoint{2.613157in}{3.598796in}}{\pgfqpoint{2.608766in}{3.609395in}}{\pgfqpoint{2.600953in}{3.617208in}}%
\pgfpathcurveto{\pgfqpoint{2.593139in}{3.625022in}}{\pgfqpoint{2.582540in}{3.629412in}}{\pgfqpoint{2.571490in}{3.629412in}}%
\pgfpathcurveto{\pgfqpoint{2.560440in}{3.629412in}}{\pgfqpoint{2.549841in}{3.625022in}}{\pgfqpoint{2.542027in}{3.617208in}}%
\pgfpathcurveto{\pgfqpoint{2.534214in}{3.609395in}}{\pgfqpoint{2.529823in}{3.598796in}}{\pgfqpoint{2.529823in}{3.587745in}}%
\pgfpathcurveto{\pgfqpoint{2.529823in}{3.576695in}}{\pgfqpoint{2.534214in}{3.566096in}}{\pgfqpoint{2.542027in}{3.558283in}}%
\pgfpathcurveto{\pgfqpoint{2.549841in}{3.550469in}}{\pgfqpoint{2.560440in}{3.546079in}}{\pgfqpoint{2.571490in}{3.546079in}}%
\pgfpathclose%
\pgfusepath{stroke,fill}%
\end{pgfscope}%
\begin{pgfscope}%
\pgfpathrectangle{\pgfqpoint{0.481978in}{0.331635in}}{\pgfqpoint{4.960000in}{3.696000in}}%
\pgfusepath{clip}%
\pgfsetbuttcap%
\pgfsetroundjoin%
\definecolor{currentfill}{rgb}{1.000000,0.705882,0.509804}%
\pgfsetfillcolor{currentfill}%
\pgfsetlinewidth{0.481800pt}%
\definecolor{currentstroke}{rgb}{1.000000,1.000000,1.000000}%
\pgfsetstrokecolor{currentstroke}%
\pgfsetdash{}{0pt}%
\pgfpathmoveto{\pgfqpoint{2.788619in}{2.545126in}}%
\pgfpathcurveto{\pgfqpoint{2.799669in}{2.545126in}}{\pgfqpoint{2.810268in}{2.549516in}}{\pgfqpoint{2.818081in}{2.557330in}}%
\pgfpathcurveto{\pgfqpoint{2.825895in}{2.565144in}}{\pgfqpoint{2.830285in}{2.575743in}}{\pgfqpoint{2.830285in}{2.586793in}}%
\pgfpathcurveto{\pgfqpoint{2.830285in}{2.597843in}}{\pgfqpoint{2.825895in}{2.608442in}}{\pgfqpoint{2.818081in}{2.616256in}}%
\pgfpathcurveto{\pgfqpoint{2.810268in}{2.624069in}}{\pgfqpoint{2.799669in}{2.628459in}}{\pgfqpoint{2.788619in}{2.628459in}}%
\pgfpathcurveto{\pgfqpoint{2.777568in}{2.628459in}}{\pgfqpoint{2.766969in}{2.624069in}}{\pgfqpoint{2.759156in}{2.616256in}}%
\pgfpathcurveto{\pgfqpoint{2.751342in}{2.608442in}}{\pgfqpoint{2.746952in}{2.597843in}}{\pgfqpoint{2.746952in}{2.586793in}}%
\pgfpathcurveto{\pgfqpoint{2.746952in}{2.575743in}}{\pgfqpoint{2.751342in}{2.565144in}}{\pgfqpoint{2.759156in}{2.557330in}}%
\pgfpathcurveto{\pgfqpoint{2.766969in}{2.549516in}}{\pgfqpoint{2.777568in}{2.545126in}}{\pgfqpoint{2.788619in}{2.545126in}}%
\pgfpathclose%
\pgfusepath{stroke,fill}%
\end{pgfscope}%
\begin{pgfscope}%
\pgfpathrectangle{\pgfqpoint{0.481978in}{0.331635in}}{\pgfqpoint{4.960000in}{3.696000in}}%
\pgfusepath{clip}%
\pgfsetbuttcap%
\pgfsetroundjoin%
\definecolor{currentfill}{rgb}{1.000000,0.705882,0.509804}%
\pgfsetfillcolor{currentfill}%
\pgfsetlinewidth{0.481800pt}%
\definecolor{currentstroke}{rgb}{1.000000,1.000000,1.000000}%
\pgfsetstrokecolor{currentstroke}%
\pgfsetdash{}{0pt}%
\pgfpathmoveto{\pgfqpoint{3.846440in}{2.897939in}}%
\pgfpathcurveto{\pgfqpoint{3.857490in}{2.897939in}}{\pgfqpoint{3.868089in}{2.902329in}}{\pgfqpoint{3.875902in}{2.910143in}}%
\pgfpathcurveto{\pgfqpoint{3.883716in}{2.917956in}}{\pgfqpoint{3.888106in}{2.928555in}}{\pgfqpoint{3.888106in}{2.939605in}}%
\pgfpathcurveto{\pgfqpoint{3.888106in}{2.950656in}}{\pgfqpoint{3.883716in}{2.961255in}}{\pgfqpoint{3.875902in}{2.969068in}}%
\pgfpathcurveto{\pgfqpoint{3.868089in}{2.976882in}}{\pgfqpoint{3.857490in}{2.981272in}}{\pgfqpoint{3.846440in}{2.981272in}}%
\pgfpathcurveto{\pgfqpoint{3.835390in}{2.981272in}}{\pgfqpoint{3.824790in}{2.976882in}}{\pgfqpoint{3.816977in}{2.969068in}}%
\pgfpathcurveto{\pgfqpoint{3.809163in}{2.961255in}}{\pgfqpoint{3.804773in}{2.950656in}}{\pgfqpoint{3.804773in}{2.939605in}}%
\pgfpathcurveto{\pgfqpoint{3.804773in}{2.928555in}}{\pgfqpoint{3.809163in}{2.917956in}}{\pgfqpoint{3.816977in}{2.910143in}}%
\pgfpathcurveto{\pgfqpoint{3.824790in}{2.902329in}}{\pgfqpoint{3.835390in}{2.897939in}}{\pgfqpoint{3.846440in}{2.897939in}}%
\pgfpathclose%
\pgfusepath{stroke,fill}%
\end{pgfscope}%
\begin{pgfscope}%
\pgfpathrectangle{\pgfqpoint{0.481978in}{0.331635in}}{\pgfqpoint{4.960000in}{3.696000in}}%
\pgfusepath{clip}%
\pgfsetbuttcap%
\pgfsetroundjoin%
\definecolor{currentfill}{rgb}{1.000000,0.705882,0.509804}%
\pgfsetfillcolor{currentfill}%
\pgfsetlinewidth{0.481800pt}%
\definecolor{currentstroke}{rgb}{1.000000,1.000000,1.000000}%
\pgfsetstrokecolor{currentstroke}%
\pgfsetdash{}{0pt}%
\pgfpathmoveto{\pgfqpoint{4.322794in}{2.423244in}}%
\pgfpathcurveto{\pgfqpoint{4.333844in}{2.423244in}}{\pgfqpoint{4.344443in}{2.427634in}}{\pgfqpoint{4.352257in}{2.435448in}}%
\pgfpathcurveto{\pgfqpoint{4.360070in}{2.443261in}}{\pgfqpoint{4.364461in}{2.453860in}}{\pgfqpoint{4.364461in}{2.464911in}}%
\pgfpathcurveto{\pgfqpoint{4.364461in}{2.475961in}}{\pgfqpoint{4.360070in}{2.486560in}}{\pgfqpoint{4.352257in}{2.494373in}}%
\pgfpathcurveto{\pgfqpoint{4.344443in}{2.502187in}}{\pgfqpoint{4.333844in}{2.506577in}}{\pgfqpoint{4.322794in}{2.506577in}}%
\pgfpathcurveto{\pgfqpoint{4.311744in}{2.506577in}}{\pgfqpoint{4.301145in}{2.502187in}}{\pgfqpoint{4.293331in}{2.494373in}}%
\pgfpathcurveto{\pgfqpoint{4.285518in}{2.486560in}}{\pgfqpoint{4.281127in}{2.475961in}}{\pgfqpoint{4.281127in}{2.464911in}}%
\pgfpathcurveto{\pgfqpoint{4.281127in}{2.453860in}}{\pgfqpoint{4.285518in}{2.443261in}}{\pgfqpoint{4.293331in}{2.435448in}}%
\pgfpathcurveto{\pgfqpoint{4.301145in}{2.427634in}}{\pgfqpoint{4.311744in}{2.423244in}}{\pgfqpoint{4.322794in}{2.423244in}}%
\pgfpathclose%
\pgfusepath{stroke,fill}%
\end{pgfscope}%
\begin{pgfscope}%
\pgfpathrectangle{\pgfqpoint{0.481978in}{0.331635in}}{\pgfqpoint{4.960000in}{3.696000in}}%
\pgfusepath{clip}%
\pgfsetbuttcap%
\pgfsetroundjoin%
\definecolor{currentfill}{rgb}{1.000000,0.705882,0.509804}%
\pgfsetfillcolor{currentfill}%
\pgfsetlinewidth{0.481800pt}%
\definecolor{currentstroke}{rgb}{1.000000,1.000000,1.000000}%
\pgfsetstrokecolor{currentstroke}%
\pgfsetdash{}{0pt}%
\pgfpathmoveto{\pgfqpoint{1.013445in}{2.853600in}}%
\pgfpathcurveto{\pgfqpoint{1.024495in}{2.853600in}}{\pgfqpoint{1.035094in}{2.857991in}}{\pgfqpoint{1.042907in}{2.865804in}}%
\pgfpathcurveto{\pgfqpoint{1.050721in}{2.873618in}}{\pgfqpoint{1.055111in}{2.884217in}}{\pgfqpoint{1.055111in}{2.895267in}}%
\pgfpathcurveto{\pgfqpoint{1.055111in}{2.906317in}}{\pgfqpoint{1.050721in}{2.916916in}}{\pgfqpoint{1.042907in}{2.924730in}}%
\pgfpathcurveto{\pgfqpoint{1.035094in}{2.932544in}}{\pgfqpoint{1.024495in}{2.936934in}}{\pgfqpoint{1.013445in}{2.936934in}}%
\pgfpathcurveto{\pgfqpoint{1.002394in}{2.936934in}}{\pgfqpoint{0.991795in}{2.932544in}}{\pgfqpoint{0.983982in}{2.924730in}}%
\pgfpathcurveto{\pgfqpoint{0.976168in}{2.916916in}}{\pgfqpoint{0.971778in}{2.906317in}}{\pgfqpoint{0.971778in}{2.895267in}}%
\pgfpathcurveto{\pgfqpoint{0.971778in}{2.884217in}}{\pgfqpoint{0.976168in}{2.873618in}}{\pgfqpoint{0.983982in}{2.865804in}}%
\pgfpathcurveto{\pgfqpoint{0.991795in}{2.857991in}}{\pgfqpoint{1.002394in}{2.853600in}}{\pgfqpoint{1.013445in}{2.853600in}}%
\pgfpathclose%
\pgfusepath{stroke,fill}%
\end{pgfscope}%
\begin{pgfscope}%
\pgfpathrectangle{\pgfqpoint{0.481978in}{0.331635in}}{\pgfqpoint{4.960000in}{3.696000in}}%
\pgfusepath{clip}%
\pgfsetbuttcap%
\pgfsetroundjoin%
\definecolor{currentfill}{rgb}{1.000000,0.705882,0.509804}%
\pgfsetfillcolor{currentfill}%
\pgfsetlinewidth{0.481800pt}%
\definecolor{currentstroke}{rgb}{1.000000,1.000000,1.000000}%
\pgfsetstrokecolor{currentstroke}%
\pgfsetdash{}{0pt}%
\pgfpathmoveto{\pgfqpoint{2.049477in}{2.898169in}}%
\pgfpathcurveto{\pgfqpoint{2.060527in}{2.898169in}}{\pgfqpoint{2.071126in}{2.902559in}}{\pgfqpoint{2.078940in}{2.910373in}}%
\pgfpathcurveto{\pgfqpoint{2.086754in}{2.918186in}}{\pgfqpoint{2.091144in}{2.928785in}}{\pgfqpoint{2.091144in}{2.939835in}}%
\pgfpathcurveto{\pgfqpoint{2.091144in}{2.950886in}}{\pgfqpoint{2.086754in}{2.961485in}}{\pgfqpoint{2.078940in}{2.969298in}}%
\pgfpathcurveto{\pgfqpoint{2.071126in}{2.977112in}}{\pgfqpoint{2.060527in}{2.981502in}}{\pgfqpoint{2.049477in}{2.981502in}}%
\pgfpathcurveto{\pgfqpoint{2.038427in}{2.981502in}}{\pgfqpoint{2.027828in}{2.977112in}}{\pgfqpoint{2.020014in}{2.969298in}}%
\pgfpathcurveto{\pgfqpoint{2.012201in}{2.961485in}}{\pgfqpoint{2.007810in}{2.950886in}}{\pgfqpoint{2.007810in}{2.939835in}}%
\pgfpathcurveto{\pgfqpoint{2.007810in}{2.928785in}}{\pgfqpoint{2.012201in}{2.918186in}}{\pgfqpoint{2.020014in}{2.910373in}}%
\pgfpathcurveto{\pgfqpoint{2.027828in}{2.902559in}}{\pgfqpoint{2.038427in}{2.898169in}}{\pgfqpoint{2.049477in}{2.898169in}}%
\pgfpathclose%
\pgfusepath{stroke,fill}%
\end{pgfscope}%
\begin{pgfscope}%
\pgfpathrectangle{\pgfqpoint{0.481978in}{0.331635in}}{\pgfqpoint{4.960000in}{3.696000in}}%
\pgfusepath{clip}%
\pgfsetbuttcap%
\pgfsetroundjoin%
\definecolor{currentfill}{rgb}{0.631373,0.788235,0.956863}%
\pgfsetfillcolor{currentfill}%
\pgfsetlinewidth{1.003750pt}%
\definecolor{currentstroke}{rgb}{0.631373,0.788235,0.956863}%
\pgfsetstrokecolor{currentstroke}%
\pgfsetdash{}{0pt}%
\pgfsys@defobject{currentmarker}{\pgfqpoint{-0.041667in}{-0.041667in}}{\pgfqpoint{0.041667in}{0.041667in}}{%
\pgfpathmoveto{\pgfqpoint{0.000000in}{-0.041667in}}%
\pgfpathcurveto{\pgfqpoint{0.011050in}{-0.041667in}}{\pgfqpoint{0.021649in}{-0.037276in}}{\pgfqpoint{0.029463in}{-0.029463in}}%
\pgfpathcurveto{\pgfqpoint{0.037276in}{-0.021649in}}{\pgfqpoint{0.041667in}{-0.011050in}}{\pgfqpoint{0.041667in}{0.000000in}}%
\pgfpathcurveto{\pgfqpoint{0.041667in}{0.011050in}}{\pgfqpoint{0.037276in}{0.021649in}}{\pgfqpoint{0.029463in}{0.029463in}}%
\pgfpathcurveto{\pgfqpoint{0.021649in}{0.037276in}}{\pgfqpoint{0.011050in}{0.041667in}}{\pgfqpoint{0.000000in}{0.041667in}}%
\pgfpathcurveto{\pgfqpoint{-0.011050in}{0.041667in}}{\pgfqpoint{-0.021649in}{0.037276in}}{\pgfqpoint{-0.029463in}{0.029463in}}%
\pgfpathcurveto{\pgfqpoint{-0.037276in}{0.021649in}}{\pgfqpoint{-0.041667in}{0.011050in}}{\pgfqpoint{-0.041667in}{0.000000in}}%
\pgfpathcurveto{\pgfqpoint{-0.041667in}{-0.011050in}}{\pgfqpoint{-0.037276in}{-0.021649in}}{\pgfqpoint{-0.029463in}{-0.029463in}}%
\pgfpathcurveto{\pgfqpoint{-0.021649in}{-0.037276in}}{\pgfqpoint{-0.011050in}{-0.041667in}}{\pgfqpoint{0.000000in}{-0.041667in}}%
\pgfpathclose%
\pgfusepath{stroke,fill}%
}%
\end{pgfscope}%
\begin{pgfscope}%
\pgfpathrectangle{\pgfqpoint{0.481978in}{0.331635in}}{\pgfqpoint{4.960000in}{3.696000in}}%
\pgfusepath{clip}%
\pgfsetbuttcap%
\pgfsetroundjoin%
\definecolor{currentfill}{rgb}{1.000000,0.705882,0.509804}%
\pgfsetfillcolor{currentfill}%
\pgfsetlinewidth{1.003750pt}%
\definecolor{currentstroke}{rgb}{1.000000,0.705882,0.509804}%
\pgfsetstrokecolor{currentstroke}%
\pgfsetdash{}{0pt}%
\pgfsys@defobject{currentmarker}{\pgfqpoint{-0.041667in}{-0.041667in}}{\pgfqpoint{0.041667in}{0.041667in}}{%
\pgfpathmoveto{\pgfqpoint{0.000000in}{-0.041667in}}%
\pgfpathcurveto{\pgfqpoint{0.011050in}{-0.041667in}}{\pgfqpoint{0.021649in}{-0.037276in}}{\pgfqpoint{0.029463in}{-0.029463in}}%
\pgfpathcurveto{\pgfqpoint{0.037276in}{-0.021649in}}{\pgfqpoint{0.041667in}{-0.011050in}}{\pgfqpoint{0.041667in}{0.000000in}}%
\pgfpathcurveto{\pgfqpoint{0.041667in}{0.011050in}}{\pgfqpoint{0.037276in}{0.021649in}}{\pgfqpoint{0.029463in}{0.029463in}}%
\pgfpathcurveto{\pgfqpoint{0.021649in}{0.037276in}}{\pgfqpoint{0.011050in}{0.041667in}}{\pgfqpoint{0.000000in}{0.041667in}}%
\pgfpathcurveto{\pgfqpoint{-0.011050in}{0.041667in}}{\pgfqpoint{-0.021649in}{0.037276in}}{\pgfqpoint{-0.029463in}{0.029463in}}%
\pgfpathcurveto{\pgfqpoint{-0.037276in}{0.021649in}}{\pgfqpoint{-0.041667in}{0.011050in}}{\pgfqpoint{-0.041667in}{0.000000in}}%
\pgfpathcurveto{\pgfqpoint{-0.041667in}{-0.011050in}}{\pgfqpoint{-0.037276in}{-0.021649in}}{\pgfqpoint{-0.029463in}{-0.029463in}}%
\pgfpathcurveto{\pgfqpoint{-0.021649in}{-0.037276in}}{\pgfqpoint{-0.011050in}{-0.041667in}}{\pgfqpoint{0.000000in}{-0.041667in}}%
\pgfpathclose%
\pgfusepath{stroke,fill}%
}%
\end{pgfscope}%
\begin{pgfscope}%
\pgfsetbuttcap%
\pgfsetroundjoin%
\definecolor{currentfill}{rgb}{0.000000,0.000000,0.000000}%
\pgfsetfillcolor{currentfill}%
\pgfsetlinewidth{0.803000pt}%
\definecolor{currentstroke}{rgb}{0.000000,0.000000,0.000000}%
\pgfsetstrokecolor{currentstroke}%
\pgfsetdash{}{0pt}%
\pgfsys@defobject{currentmarker}{\pgfqpoint{0.000000in}{-0.048611in}}{\pgfqpoint{0.000000in}{0.000000in}}{%
\pgfpathmoveto{\pgfqpoint{0.000000in}{0.000000in}}%
\pgfpathlineto{\pgfqpoint{0.000000in}{-0.048611in}}%
\pgfusepath{stroke,fill}%
}%
\begin{pgfscope}%
\pgfsys@transformshift{0.651717in}{0.331635in}%
\pgfsys@useobject{currentmarker}{}%
\end{pgfscope}%
\end{pgfscope}%
\begin{pgfscope}%
\definecolor{textcolor}{rgb}{0.000000,0.000000,0.000000}%
\pgfsetstrokecolor{textcolor}%
\pgfsetfillcolor{textcolor}%
\pgftext[x=0.651717in,y=0.234413in,,top]{\color{textcolor}\sffamily\fontsize{10.000000}{12.000000}\selectfont \ensuremath{-}60}%
\end{pgfscope}%
\begin{pgfscope}%
\pgfsetbuttcap%
\pgfsetroundjoin%
\definecolor{currentfill}{rgb}{0.000000,0.000000,0.000000}%
\pgfsetfillcolor{currentfill}%
\pgfsetlinewidth{0.803000pt}%
\definecolor{currentstroke}{rgb}{0.000000,0.000000,0.000000}%
\pgfsetstrokecolor{currentstroke}%
\pgfsetdash{}{0pt}%
\pgfsys@defobject{currentmarker}{\pgfqpoint{0.000000in}{-0.048611in}}{\pgfqpoint{0.000000in}{0.000000in}}{%
\pgfpathmoveto{\pgfqpoint{0.000000in}{0.000000in}}%
\pgfpathlineto{\pgfqpoint{0.000000in}{-0.048611in}}%
\pgfusepath{stroke,fill}%
}%
\begin{pgfscope}%
\pgfsys@transformshift{1.501372in}{0.331635in}%
\pgfsys@useobject{currentmarker}{}%
\end{pgfscope}%
\end{pgfscope}%
\begin{pgfscope}%
\definecolor{textcolor}{rgb}{0.000000,0.000000,0.000000}%
\pgfsetstrokecolor{textcolor}%
\pgfsetfillcolor{textcolor}%
\pgftext[x=1.501372in,y=0.234413in,,top]{\color{textcolor}\sffamily\fontsize{10.000000}{12.000000}\selectfont \ensuremath{-}40}%
\end{pgfscope}%
\begin{pgfscope}%
\pgfsetbuttcap%
\pgfsetroundjoin%
\definecolor{currentfill}{rgb}{0.000000,0.000000,0.000000}%
\pgfsetfillcolor{currentfill}%
\pgfsetlinewidth{0.803000pt}%
\definecolor{currentstroke}{rgb}{0.000000,0.000000,0.000000}%
\pgfsetstrokecolor{currentstroke}%
\pgfsetdash{}{0pt}%
\pgfsys@defobject{currentmarker}{\pgfqpoint{0.000000in}{-0.048611in}}{\pgfqpoint{0.000000in}{0.000000in}}{%
\pgfpathmoveto{\pgfqpoint{0.000000in}{0.000000in}}%
\pgfpathlineto{\pgfqpoint{0.000000in}{-0.048611in}}%
\pgfusepath{stroke,fill}%
}%
\begin{pgfscope}%
\pgfsys@transformshift{2.351027in}{0.331635in}%
\pgfsys@useobject{currentmarker}{}%
\end{pgfscope}%
\end{pgfscope}%
\begin{pgfscope}%
\definecolor{textcolor}{rgb}{0.000000,0.000000,0.000000}%
\pgfsetstrokecolor{textcolor}%
\pgfsetfillcolor{textcolor}%
\pgftext[x=2.351027in,y=0.234413in,,top]{\color{textcolor}\sffamily\fontsize{10.000000}{12.000000}\selectfont \ensuremath{-}20}%
\end{pgfscope}%
\begin{pgfscope}%
\pgfsetbuttcap%
\pgfsetroundjoin%
\definecolor{currentfill}{rgb}{0.000000,0.000000,0.000000}%
\pgfsetfillcolor{currentfill}%
\pgfsetlinewidth{0.803000pt}%
\definecolor{currentstroke}{rgb}{0.000000,0.000000,0.000000}%
\pgfsetstrokecolor{currentstroke}%
\pgfsetdash{}{0pt}%
\pgfsys@defobject{currentmarker}{\pgfqpoint{0.000000in}{-0.048611in}}{\pgfqpoint{0.000000in}{0.000000in}}{%
\pgfpathmoveto{\pgfqpoint{0.000000in}{0.000000in}}%
\pgfpathlineto{\pgfqpoint{0.000000in}{-0.048611in}}%
\pgfusepath{stroke,fill}%
}%
\begin{pgfscope}%
\pgfsys@transformshift{3.200683in}{0.331635in}%
\pgfsys@useobject{currentmarker}{}%
\end{pgfscope}%
\end{pgfscope}%
\begin{pgfscope}%
\definecolor{textcolor}{rgb}{0.000000,0.000000,0.000000}%
\pgfsetstrokecolor{textcolor}%
\pgfsetfillcolor{textcolor}%
\pgftext[x=3.200683in,y=0.234413in,,top]{\color{textcolor}\sffamily\fontsize{10.000000}{12.000000}\selectfont 0}%
\end{pgfscope}%
\begin{pgfscope}%
\pgfsetbuttcap%
\pgfsetroundjoin%
\definecolor{currentfill}{rgb}{0.000000,0.000000,0.000000}%
\pgfsetfillcolor{currentfill}%
\pgfsetlinewidth{0.803000pt}%
\definecolor{currentstroke}{rgb}{0.000000,0.000000,0.000000}%
\pgfsetstrokecolor{currentstroke}%
\pgfsetdash{}{0pt}%
\pgfsys@defobject{currentmarker}{\pgfqpoint{0.000000in}{-0.048611in}}{\pgfqpoint{0.000000in}{0.000000in}}{%
\pgfpathmoveto{\pgfqpoint{0.000000in}{0.000000in}}%
\pgfpathlineto{\pgfqpoint{0.000000in}{-0.048611in}}%
\pgfusepath{stroke,fill}%
}%
\begin{pgfscope}%
\pgfsys@transformshift{4.050338in}{0.331635in}%
\pgfsys@useobject{currentmarker}{}%
\end{pgfscope}%
\end{pgfscope}%
\begin{pgfscope}%
\definecolor{textcolor}{rgb}{0.000000,0.000000,0.000000}%
\pgfsetstrokecolor{textcolor}%
\pgfsetfillcolor{textcolor}%
\pgftext[x=4.050338in,y=0.234413in,,top]{\color{textcolor}\sffamily\fontsize{10.000000}{12.000000}\selectfont 20}%
\end{pgfscope}%
\begin{pgfscope}%
\pgfsetbuttcap%
\pgfsetroundjoin%
\definecolor{currentfill}{rgb}{0.000000,0.000000,0.000000}%
\pgfsetfillcolor{currentfill}%
\pgfsetlinewidth{0.803000pt}%
\definecolor{currentstroke}{rgb}{0.000000,0.000000,0.000000}%
\pgfsetstrokecolor{currentstroke}%
\pgfsetdash{}{0pt}%
\pgfsys@defobject{currentmarker}{\pgfqpoint{0.000000in}{-0.048611in}}{\pgfqpoint{0.000000in}{0.000000in}}{%
\pgfpathmoveto{\pgfqpoint{0.000000in}{0.000000in}}%
\pgfpathlineto{\pgfqpoint{0.000000in}{-0.048611in}}%
\pgfusepath{stroke,fill}%
}%
\begin{pgfscope}%
\pgfsys@transformshift{4.899993in}{0.331635in}%
\pgfsys@useobject{currentmarker}{}%
\end{pgfscope}%
\end{pgfscope}%
\begin{pgfscope}%
\definecolor{textcolor}{rgb}{0.000000,0.000000,0.000000}%
\pgfsetstrokecolor{textcolor}%
\pgfsetfillcolor{textcolor}%
\pgftext[x=4.899993in,y=0.234413in,,top]{\color{textcolor}\sffamily\fontsize{10.000000}{12.000000}\selectfont 40}%
\end{pgfscope}%
\begin{pgfscope}%
\pgfsetbuttcap%
\pgfsetroundjoin%
\definecolor{currentfill}{rgb}{0.000000,0.000000,0.000000}%
\pgfsetfillcolor{currentfill}%
\pgfsetlinewidth{0.803000pt}%
\definecolor{currentstroke}{rgb}{0.000000,0.000000,0.000000}%
\pgfsetstrokecolor{currentstroke}%
\pgfsetdash{}{0pt}%
\pgfsys@defobject{currentmarker}{\pgfqpoint{-0.048611in}{0.000000in}}{\pgfqpoint{-0.000000in}{0.000000in}}{%
\pgfpathmoveto{\pgfqpoint{-0.000000in}{0.000000in}}%
\pgfpathlineto{\pgfqpoint{-0.048611in}{0.000000in}}%
\pgfusepath{stroke,fill}%
}%
\begin{pgfscope}%
\pgfsys@transformshift{0.481978in}{0.459608in}%
\pgfsys@useobject{currentmarker}{}%
\end{pgfscope}%
\end{pgfscope}%
\begin{pgfscope}%
\definecolor{textcolor}{rgb}{0.000000,0.000000,0.000000}%
\pgfsetstrokecolor{textcolor}%
\pgfsetfillcolor{textcolor}%
\pgftext[x=0.100000in, y=0.406846in, left, base]{\color{textcolor}\sffamily\fontsize{10.000000}{12.000000}\selectfont \ensuremath{-}60}%
\end{pgfscope}%
\begin{pgfscope}%
\pgfsetbuttcap%
\pgfsetroundjoin%
\definecolor{currentfill}{rgb}{0.000000,0.000000,0.000000}%
\pgfsetfillcolor{currentfill}%
\pgfsetlinewidth{0.803000pt}%
\definecolor{currentstroke}{rgb}{0.000000,0.000000,0.000000}%
\pgfsetstrokecolor{currentstroke}%
\pgfsetdash{}{0pt}%
\pgfsys@defobject{currentmarker}{\pgfqpoint{-0.048611in}{0.000000in}}{\pgfqpoint{-0.000000in}{0.000000in}}{%
\pgfpathmoveto{\pgfqpoint{-0.000000in}{0.000000in}}%
\pgfpathlineto{\pgfqpoint{-0.048611in}{0.000000in}}%
\pgfusepath{stroke,fill}%
}%
\begin{pgfscope}%
\pgfsys@transformshift{0.481978in}{1.157602in}%
\pgfsys@useobject{currentmarker}{}%
\end{pgfscope}%
\end{pgfscope}%
\begin{pgfscope}%
\definecolor{textcolor}{rgb}{0.000000,0.000000,0.000000}%
\pgfsetstrokecolor{textcolor}%
\pgfsetfillcolor{textcolor}%
\pgftext[x=0.100000in, y=1.104840in, left, base]{\color{textcolor}\sffamily\fontsize{10.000000}{12.000000}\selectfont \ensuremath{-}40}%
\end{pgfscope}%
\begin{pgfscope}%
\pgfsetbuttcap%
\pgfsetroundjoin%
\definecolor{currentfill}{rgb}{0.000000,0.000000,0.000000}%
\pgfsetfillcolor{currentfill}%
\pgfsetlinewidth{0.803000pt}%
\definecolor{currentstroke}{rgb}{0.000000,0.000000,0.000000}%
\pgfsetstrokecolor{currentstroke}%
\pgfsetdash{}{0pt}%
\pgfsys@defobject{currentmarker}{\pgfqpoint{-0.048611in}{0.000000in}}{\pgfqpoint{-0.000000in}{0.000000in}}{%
\pgfpathmoveto{\pgfqpoint{-0.000000in}{0.000000in}}%
\pgfpathlineto{\pgfqpoint{-0.048611in}{0.000000in}}%
\pgfusepath{stroke,fill}%
}%
\begin{pgfscope}%
\pgfsys@transformshift{0.481978in}{1.855596in}%
\pgfsys@useobject{currentmarker}{}%
\end{pgfscope}%
\end{pgfscope}%
\begin{pgfscope}%
\definecolor{textcolor}{rgb}{0.000000,0.000000,0.000000}%
\pgfsetstrokecolor{textcolor}%
\pgfsetfillcolor{textcolor}%
\pgftext[x=0.100000in, y=1.802834in, left, base]{\color{textcolor}\sffamily\fontsize{10.000000}{12.000000}\selectfont \ensuremath{-}20}%
\end{pgfscope}%
\begin{pgfscope}%
\pgfsetbuttcap%
\pgfsetroundjoin%
\definecolor{currentfill}{rgb}{0.000000,0.000000,0.000000}%
\pgfsetfillcolor{currentfill}%
\pgfsetlinewidth{0.803000pt}%
\definecolor{currentstroke}{rgb}{0.000000,0.000000,0.000000}%
\pgfsetstrokecolor{currentstroke}%
\pgfsetdash{}{0pt}%
\pgfsys@defobject{currentmarker}{\pgfqpoint{-0.048611in}{0.000000in}}{\pgfqpoint{-0.000000in}{0.000000in}}{%
\pgfpathmoveto{\pgfqpoint{-0.000000in}{0.000000in}}%
\pgfpathlineto{\pgfqpoint{-0.048611in}{0.000000in}}%
\pgfusepath{stroke,fill}%
}%
\begin{pgfscope}%
\pgfsys@transformshift{0.481978in}{2.553589in}%
\pgfsys@useobject{currentmarker}{}%
\end{pgfscope}%
\end{pgfscope}%
\begin{pgfscope}%
\definecolor{textcolor}{rgb}{0.000000,0.000000,0.000000}%
\pgfsetstrokecolor{textcolor}%
\pgfsetfillcolor{textcolor}%
\pgftext[x=0.296390in, y=2.500828in, left, base]{\color{textcolor}\sffamily\fontsize{10.000000}{12.000000}\selectfont 0}%
\end{pgfscope}%
\begin{pgfscope}%
\pgfsetbuttcap%
\pgfsetroundjoin%
\definecolor{currentfill}{rgb}{0.000000,0.000000,0.000000}%
\pgfsetfillcolor{currentfill}%
\pgfsetlinewidth{0.803000pt}%
\definecolor{currentstroke}{rgb}{0.000000,0.000000,0.000000}%
\pgfsetstrokecolor{currentstroke}%
\pgfsetdash{}{0pt}%
\pgfsys@defobject{currentmarker}{\pgfqpoint{-0.048611in}{0.000000in}}{\pgfqpoint{-0.000000in}{0.000000in}}{%
\pgfpathmoveto{\pgfqpoint{-0.000000in}{0.000000in}}%
\pgfpathlineto{\pgfqpoint{-0.048611in}{0.000000in}}%
\pgfusepath{stroke,fill}%
}%
\begin{pgfscope}%
\pgfsys@transformshift{0.481978in}{3.251583in}%
\pgfsys@useobject{currentmarker}{}%
\end{pgfscope}%
\end{pgfscope}%
\begin{pgfscope}%
\definecolor{textcolor}{rgb}{0.000000,0.000000,0.000000}%
\pgfsetstrokecolor{textcolor}%
\pgfsetfillcolor{textcolor}%
\pgftext[x=0.208025in, y=3.198822in, left, base]{\color{textcolor}\sffamily\fontsize{10.000000}{12.000000}\selectfont 20}%
\end{pgfscope}%
\begin{pgfscope}%
\pgfsetbuttcap%
\pgfsetroundjoin%
\definecolor{currentfill}{rgb}{0.000000,0.000000,0.000000}%
\pgfsetfillcolor{currentfill}%
\pgfsetlinewidth{0.803000pt}%
\definecolor{currentstroke}{rgb}{0.000000,0.000000,0.000000}%
\pgfsetstrokecolor{currentstroke}%
\pgfsetdash{}{0pt}%
\pgfsys@defobject{currentmarker}{\pgfqpoint{-0.048611in}{0.000000in}}{\pgfqpoint{-0.000000in}{0.000000in}}{%
\pgfpathmoveto{\pgfqpoint{-0.000000in}{0.000000in}}%
\pgfpathlineto{\pgfqpoint{-0.048611in}{0.000000in}}%
\pgfusepath{stroke,fill}%
}%
\begin{pgfscope}%
\pgfsys@transformshift{0.481978in}{3.949577in}%
\pgfsys@useobject{currentmarker}{}%
\end{pgfscope}%
\end{pgfscope}%
\begin{pgfscope}%
\definecolor{textcolor}{rgb}{0.000000,0.000000,0.000000}%
\pgfsetstrokecolor{textcolor}%
\pgfsetfillcolor{textcolor}%
\pgftext[x=0.208025in, y=3.896816in, left, base]{\color{textcolor}\sffamily\fontsize{10.000000}{12.000000}\selectfont 40}%
\end{pgfscope}%
\begin{pgfscope}%
\pgfpathrectangle{\pgfqpoint{0.481978in}{0.331635in}}{\pgfqpoint{4.960000in}{3.696000in}}%
\pgfusepath{clip}%
\pgfsetrectcap%
\pgfsetroundjoin%
\pgfsetlinewidth{1.505625pt}%
\definecolor{currentstroke}{rgb}{0.631373,0.788235,0.956863}%
\pgfsetstrokecolor{currentstroke}%
\pgfsetstrokeopacity{0.800000}%
\pgfsetdash{}{0pt}%
\pgfpathmoveto{\pgfqpoint{3.170276in}{2.430688in}}%
\pgfpathlineto{\pgfqpoint{3.563553in}{2.242061in}}%
\pgfusepath{stroke}%
\end{pgfscope}%
\begin{pgfscope}%
\pgfpathrectangle{\pgfqpoint{0.481978in}{0.331635in}}{\pgfqpoint{4.960000in}{3.696000in}}%
\pgfusepath{clip}%
\pgfsetrectcap%
\pgfsetroundjoin%
\pgfsetlinewidth{1.505625pt}%
\definecolor{currentstroke}{rgb}{0.631373,0.788235,0.956863}%
\pgfsetstrokecolor{currentstroke}%
\pgfsetstrokeopacity{0.800000}%
\pgfsetdash{}{0pt}%
\pgfpathmoveto{\pgfqpoint{3.797277in}{1.249718in}}%
\pgfpathlineto{\pgfqpoint{3.563553in}{2.242061in}}%
\pgfusepath{stroke}%
\end{pgfscope}%
\begin{pgfscope}%
\pgfpathrectangle{\pgfqpoint{0.481978in}{0.331635in}}{\pgfqpoint{4.960000in}{3.696000in}}%
\pgfusepath{clip}%
\pgfsetrectcap%
\pgfsetroundjoin%
\pgfsetlinewidth{1.505625pt}%
\definecolor{currentstroke}{rgb}{0.631373,0.788235,0.956863}%
\pgfsetstrokecolor{currentstroke}%
\pgfsetstrokeopacity{0.800000}%
\pgfsetdash{}{0pt}%
\pgfpathmoveto{\pgfqpoint{2.431161in}{2.160457in}}%
\pgfpathlineto{\pgfqpoint{3.563553in}{2.242061in}}%
\pgfusepath{stroke}%
\end{pgfscope}%
\begin{pgfscope}%
\pgfpathrectangle{\pgfqpoint{0.481978in}{0.331635in}}{\pgfqpoint{4.960000in}{3.696000in}}%
\pgfusepath{clip}%
\pgfsetrectcap%
\pgfsetroundjoin%
\pgfsetlinewidth{1.505625pt}%
\definecolor{currentstroke}{rgb}{0.631373,0.788235,0.956863}%
\pgfsetstrokecolor{currentstroke}%
\pgfsetstrokeopacity{0.800000}%
\pgfsetdash{}{0pt}%
\pgfpathmoveto{\pgfqpoint{3.988216in}{3.359003in}}%
\pgfpathlineto{\pgfqpoint{3.563553in}{2.242061in}}%
\pgfusepath{stroke}%
\end{pgfscope}%
\begin{pgfscope}%
\pgfpathrectangle{\pgfqpoint{0.481978in}{0.331635in}}{\pgfqpoint{4.960000in}{3.696000in}}%
\pgfusepath{clip}%
\pgfsetrectcap%
\pgfsetroundjoin%
\pgfsetlinewidth{1.505625pt}%
\definecolor{currentstroke}{rgb}{0.631373,0.788235,0.956863}%
\pgfsetstrokecolor{currentstroke}%
\pgfsetstrokeopacity{0.800000}%
\pgfsetdash{}{0pt}%
\pgfpathmoveto{\pgfqpoint{4.509417in}{1.385375in}}%
\pgfpathlineto{\pgfqpoint{3.563553in}{2.242061in}}%
\pgfusepath{stroke}%
\end{pgfscope}%
\begin{pgfscope}%
\pgfpathrectangle{\pgfqpoint{0.481978in}{0.331635in}}{\pgfqpoint{4.960000in}{3.696000in}}%
\pgfusepath{clip}%
\pgfsetrectcap%
\pgfsetroundjoin%
\pgfsetlinewidth{1.505625pt}%
\definecolor{currentstroke}{rgb}{0.631373,0.788235,0.956863}%
\pgfsetstrokecolor{currentstroke}%
\pgfsetstrokeopacity{0.800000}%
\pgfsetdash{}{0pt}%
\pgfpathmoveto{\pgfqpoint{3.252727in}{2.139754in}}%
\pgfpathlineto{\pgfqpoint{3.563553in}{2.242061in}}%
\pgfusepath{stroke}%
\end{pgfscope}%
\begin{pgfscope}%
\pgfpathrectangle{\pgfqpoint{0.481978in}{0.331635in}}{\pgfqpoint{4.960000in}{3.696000in}}%
\pgfusepath{clip}%
\pgfsetrectcap%
\pgfsetroundjoin%
\pgfsetlinewidth{1.505625pt}%
\definecolor{currentstroke}{rgb}{0.631373,0.788235,0.956863}%
\pgfsetstrokecolor{currentstroke}%
\pgfsetstrokeopacity{0.800000}%
\pgfsetdash{}{0pt}%
\pgfpathmoveto{\pgfqpoint{4.536927in}{3.194685in}}%
\pgfpathlineto{\pgfqpoint{3.563553in}{2.242061in}}%
\pgfusepath{stroke}%
\end{pgfscope}%
\begin{pgfscope}%
\pgfpathrectangle{\pgfqpoint{0.481978in}{0.331635in}}{\pgfqpoint{4.960000in}{3.696000in}}%
\pgfusepath{clip}%
\pgfsetrectcap%
\pgfsetroundjoin%
\pgfsetlinewidth{1.505625pt}%
\definecolor{currentstroke}{rgb}{0.631373,0.788235,0.956863}%
\pgfsetstrokecolor{currentstroke}%
\pgfsetstrokeopacity{0.800000}%
\pgfsetdash{}{0pt}%
\pgfpathmoveto{\pgfqpoint{3.471510in}{2.611751in}}%
\pgfpathlineto{\pgfqpoint{3.563553in}{2.242061in}}%
\pgfusepath{stroke}%
\end{pgfscope}%
\begin{pgfscope}%
\pgfpathrectangle{\pgfqpoint{0.481978in}{0.331635in}}{\pgfqpoint{4.960000in}{3.696000in}}%
\pgfusepath{clip}%
\pgfsetrectcap%
\pgfsetroundjoin%
\pgfsetlinewidth{1.505625pt}%
\definecolor{currentstroke}{rgb}{0.631373,0.788235,0.956863}%
\pgfsetstrokecolor{currentstroke}%
\pgfsetstrokeopacity{0.800000}%
\pgfsetdash{}{0pt}%
\pgfpathmoveto{\pgfqpoint{2.459931in}{1.759119in}}%
\pgfpathlineto{\pgfqpoint{3.563553in}{2.242061in}}%
\pgfusepath{stroke}%
\end{pgfscope}%
\begin{pgfscope}%
\pgfpathrectangle{\pgfqpoint{0.481978in}{0.331635in}}{\pgfqpoint{4.960000in}{3.696000in}}%
\pgfusepath{clip}%
\pgfsetrectcap%
\pgfsetroundjoin%
\pgfsetlinewidth{1.505625pt}%
\definecolor{currentstroke}{rgb}{0.631373,0.788235,0.956863}%
\pgfsetstrokecolor{currentstroke}%
\pgfsetstrokeopacity{0.800000}%
\pgfsetdash{}{0pt}%
\pgfpathmoveto{\pgfqpoint{2.937570in}{1.896242in}}%
\pgfpathlineto{\pgfqpoint{3.563553in}{2.242061in}}%
\pgfusepath{stroke}%
\end{pgfscope}%
\begin{pgfscope}%
\pgfpathrectangle{\pgfqpoint{0.481978in}{0.331635in}}{\pgfqpoint{4.960000in}{3.696000in}}%
\pgfusepath{clip}%
\pgfsetrectcap%
\pgfsetroundjoin%
\pgfsetlinewidth{1.505625pt}%
\definecolor{currentstroke}{rgb}{0.631373,0.788235,0.956863}%
\pgfsetstrokecolor{currentstroke}%
\pgfsetstrokeopacity{0.800000}%
\pgfsetdash{}{0pt}%
\pgfpathmoveto{\pgfqpoint{3.802689in}{1.788474in}}%
\pgfpathlineto{\pgfqpoint{3.563553in}{2.242061in}}%
\pgfusepath{stroke}%
\end{pgfscope}%
\begin{pgfscope}%
\pgfpathrectangle{\pgfqpoint{0.481978in}{0.331635in}}{\pgfqpoint{4.960000in}{3.696000in}}%
\pgfusepath{clip}%
\pgfsetrectcap%
\pgfsetroundjoin%
\pgfsetlinewidth{1.505625pt}%
\definecolor{currentstroke}{rgb}{0.631373,0.788235,0.956863}%
\pgfsetstrokecolor{currentstroke}%
\pgfsetstrokeopacity{0.800000}%
\pgfsetdash{}{0pt}%
\pgfpathmoveto{\pgfqpoint{3.541785in}{1.539826in}}%
\pgfpathlineto{\pgfqpoint{3.563553in}{2.242061in}}%
\pgfusepath{stroke}%
\end{pgfscope}%
\begin{pgfscope}%
\pgfpathrectangle{\pgfqpoint{0.481978in}{0.331635in}}{\pgfqpoint{4.960000in}{3.696000in}}%
\pgfusepath{clip}%
\pgfsetrectcap%
\pgfsetroundjoin%
\pgfsetlinewidth{1.505625pt}%
\definecolor{currentstroke}{rgb}{0.631373,0.788235,0.956863}%
\pgfsetstrokecolor{currentstroke}%
\pgfsetstrokeopacity{0.800000}%
\pgfsetdash{}{0pt}%
\pgfpathmoveto{\pgfqpoint{5.216523in}{2.434439in}}%
\pgfpathlineto{\pgfqpoint{3.563553in}{2.242061in}}%
\pgfusepath{stroke}%
\end{pgfscope}%
\begin{pgfscope}%
\pgfpathrectangle{\pgfqpoint{0.481978in}{0.331635in}}{\pgfqpoint{4.960000in}{3.696000in}}%
\pgfusepath{clip}%
\pgfsetrectcap%
\pgfsetroundjoin%
\pgfsetlinewidth{1.505625pt}%
\definecolor{currentstroke}{rgb}{0.631373,0.788235,0.956863}%
\pgfsetstrokecolor{currentstroke}%
\pgfsetstrokeopacity{0.800000}%
\pgfsetdash{}{0pt}%
\pgfpathmoveto{\pgfqpoint{3.043134in}{1.586573in}}%
\pgfpathlineto{\pgfqpoint{3.563553in}{2.242061in}}%
\pgfusepath{stroke}%
\end{pgfscope}%
\begin{pgfscope}%
\pgfpathrectangle{\pgfqpoint{0.481978in}{0.331635in}}{\pgfqpoint{4.960000in}{3.696000in}}%
\pgfusepath{clip}%
\pgfsetrectcap%
\pgfsetroundjoin%
\pgfsetlinewidth{1.505625pt}%
\definecolor{currentstroke}{rgb}{0.631373,0.788235,0.956863}%
\pgfsetstrokecolor{currentstroke}%
\pgfsetstrokeopacity{0.800000}%
\pgfsetdash{}{0pt}%
\pgfpathmoveto{\pgfqpoint{2.580652in}{3.184530in}}%
\pgfpathlineto{\pgfqpoint{3.563553in}{2.242061in}}%
\pgfusepath{stroke}%
\end{pgfscope}%
\begin{pgfscope}%
\pgfpathrectangle{\pgfqpoint{0.481978in}{0.331635in}}{\pgfqpoint{4.960000in}{3.696000in}}%
\pgfusepath{clip}%
\pgfsetrectcap%
\pgfsetroundjoin%
\pgfsetlinewidth{1.505625pt}%
\definecolor{currentstroke}{rgb}{0.631373,0.788235,0.956863}%
\pgfsetstrokecolor{currentstroke}%
\pgfsetstrokeopacity{0.800000}%
\pgfsetdash{}{0pt}%
\pgfpathmoveto{\pgfqpoint{4.001840in}{1.578365in}}%
\pgfpathlineto{\pgfqpoint{3.563553in}{2.242061in}}%
\pgfusepath{stroke}%
\end{pgfscope}%
\begin{pgfscope}%
\pgfpathrectangle{\pgfqpoint{0.481978in}{0.331635in}}{\pgfqpoint{4.960000in}{3.696000in}}%
\pgfusepath{clip}%
\pgfsetrectcap%
\pgfsetroundjoin%
\pgfsetlinewidth{1.505625pt}%
\definecolor{currentstroke}{rgb}{0.631373,0.788235,0.956863}%
\pgfsetstrokecolor{currentstroke}%
\pgfsetstrokeopacity{0.800000}%
\pgfsetdash{}{0pt}%
\pgfpathmoveto{\pgfqpoint{3.872808in}{2.556124in}}%
\pgfpathlineto{\pgfqpoint{3.563553in}{2.242061in}}%
\pgfusepath{stroke}%
\end{pgfscope}%
\begin{pgfscope}%
\pgfpathrectangle{\pgfqpoint{0.481978in}{0.331635in}}{\pgfqpoint{4.960000in}{3.696000in}}%
\pgfusepath{clip}%
\pgfsetrectcap%
\pgfsetroundjoin%
\pgfsetlinewidth{1.505625pt}%
\definecolor{currentstroke}{rgb}{0.631373,0.788235,0.956863}%
\pgfsetstrokecolor{currentstroke}%
\pgfsetstrokeopacity{0.800000}%
\pgfsetdash{}{0pt}%
\pgfpathmoveto{\pgfqpoint{3.135350in}{2.778513in}}%
\pgfpathlineto{\pgfqpoint{3.563553in}{2.242061in}}%
\pgfusepath{stroke}%
\end{pgfscope}%
\begin{pgfscope}%
\pgfpathrectangle{\pgfqpoint{0.481978in}{0.331635in}}{\pgfqpoint{4.960000in}{3.696000in}}%
\pgfusepath{clip}%
\pgfsetrectcap%
\pgfsetroundjoin%
\pgfsetlinewidth{1.505625pt}%
\definecolor{currentstroke}{rgb}{0.631373,0.788235,0.956863}%
\pgfsetstrokecolor{currentstroke}%
\pgfsetstrokeopacity{0.800000}%
\pgfsetdash{}{0pt}%
\pgfpathmoveto{\pgfqpoint{3.378442in}{1.866748in}}%
\pgfpathlineto{\pgfqpoint{3.563553in}{2.242061in}}%
\pgfusepath{stroke}%
\end{pgfscope}%
\begin{pgfscope}%
\pgfpathrectangle{\pgfqpoint{0.481978in}{0.331635in}}{\pgfqpoint{4.960000in}{3.696000in}}%
\pgfusepath{clip}%
\pgfsetrectcap%
\pgfsetroundjoin%
\pgfsetlinewidth{1.505625pt}%
\definecolor{currentstroke}{rgb}{0.631373,0.788235,0.956863}%
\pgfsetstrokecolor{currentstroke}%
\pgfsetstrokeopacity{0.800000}%
\pgfsetdash{}{0pt}%
\pgfpathmoveto{\pgfqpoint{3.656219in}{2.075588in}}%
\pgfpathlineto{\pgfqpoint{3.563553in}{2.242061in}}%
\pgfusepath{stroke}%
\end{pgfscope}%
\begin{pgfscope}%
\pgfpathrectangle{\pgfqpoint{0.481978in}{0.331635in}}{\pgfqpoint{4.960000in}{3.696000in}}%
\pgfusepath{clip}%
\pgfsetrectcap%
\pgfsetroundjoin%
\pgfsetlinewidth{1.505625pt}%
\definecolor{currentstroke}{rgb}{0.631373,0.788235,0.956863}%
\pgfsetstrokecolor{currentstroke}%
\pgfsetstrokeopacity{0.800000}%
\pgfsetdash{}{0pt}%
\pgfpathmoveto{\pgfqpoint{3.440189in}{3.009227in}}%
\pgfpathlineto{\pgfqpoint{3.563553in}{2.242061in}}%
\pgfusepath{stroke}%
\end{pgfscope}%
\begin{pgfscope}%
\pgfpathrectangle{\pgfqpoint{0.481978in}{0.331635in}}{\pgfqpoint{4.960000in}{3.696000in}}%
\pgfusepath{clip}%
\pgfsetrectcap%
\pgfsetroundjoin%
\pgfsetlinewidth{1.505625pt}%
\definecolor{currentstroke}{rgb}{0.631373,0.788235,0.956863}%
\pgfsetstrokecolor{currentstroke}%
\pgfsetstrokeopacity{0.800000}%
\pgfsetdash{}{0pt}%
\pgfpathmoveto{\pgfqpoint{4.280646in}{1.853152in}}%
\pgfpathlineto{\pgfqpoint{3.563553in}{2.242061in}}%
\pgfusepath{stroke}%
\end{pgfscope}%
\begin{pgfscope}%
\pgfpathrectangle{\pgfqpoint{0.481978in}{0.331635in}}{\pgfqpoint{4.960000in}{3.696000in}}%
\pgfusepath{clip}%
\pgfsetrectcap%
\pgfsetroundjoin%
\pgfsetlinewidth{1.505625pt}%
\definecolor{currentstroke}{rgb}{0.631373,0.788235,0.956863}%
\pgfsetstrokecolor{currentstroke}%
\pgfsetstrokeopacity{0.800000}%
\pgfsetdash{}{0pt}%
\pgfpathmoveto{\pgfqpoint{2.790063in}{2.237079in}}%
\pgfpathlineto{\pgfqpoint{3.563553in}{2.242061in}}%
\pgfusepath{stroke}%
\end{pgfscope}%
\begin{pgfscope}%
\pgfpathrectangle{\pgfqpoint{0.481978in}{0.331635in}}{\pgfqpoint{4.960000in}{3.696000in}}%
\pgfusepath{clip}%
\pgfsetrectcap%
\pgfsetroundjoin%
\pgfsetlinewidth{1.505625pt}%
\definecolor{currentstroke}{rgb}{0.631373,0.788235,0.956863}%
\pgfsetstrokecolor{currentstroke}%
\pgfsetstrokeopacity{0.800000}%
\pgfsetdash{}{0pt}%
\pgfpathmoveto{\pgfqpoint{4.256551in}{2.874269in}}%
\pgfpathlineto{\pgfqpoint{3.563553in}{2.242061in}}%
\pgfusepath{stroke}%
\end{pgfscope}%
\begin{pgfscope}%
\pgfpathrectangle{\pgfqpoint{0.481978in}{0.331635in}}{\pgfqpoint{4.960000in}{3.696000in}}%
\pgfusepath{clip}%
\pgfsetrectcap%
\pgfsetroundjoin%
\pgfsetlinewidth{1.505625pt}%
\definecolor{currentstroke}{rgb}{0.631373,0.788235,0.956863}%
\pgfsetstrokecolor{currentstroke}%
\pgfsetstrokeopacity{0.800000}%
\pgfsetdash{}{0pt}%
\pgfpathmoveto{\pgfqpoint{4.575205in}{2.193139in}}%
\pgfpathlineto{\pgfqpoint{3.563553in}{2.242061in}}%
\pgfusepath{stroke}%
\end{pgfscope}%
\begin{pgfscope}%
\pgfpathrectangle{\pgfqpoint{0.481978in}{0.331635in}}{\pgfqpoint{4.960000in}{3.696000in}}%
\pgfusepath{clip}%
\pgfsetrectcap%
\pgfsetroundjoin%
\pgfsetlinewidth{1.505625pt}%
\definecolor{currentstroke}{rgb}{0.631373,0.788235,0.956863}%
\pgfsetstrokecolor{currentstroke}%
\pgfsetstrokeopacity{0.800000}%
\pgfsetdash{}{0pt}%
\pgfpathmoveto{\pgfqpoint{3.448245in}{3.313520in}}%
\pgfpathlineto{\pgfqpoint{3.563553in}{2.242061in}}%
\pgfusepath{stroke}%
\end{pgfscope}%
\begin{pgfscope}%
\pgfpathrectangle{\pgfqpoint{0.481978in}{0.331635in}}{\pgfqpoint{4.960000in}{3.696000in}}%
\pgfusepath{clip}%
\pgfsetrectcap%
\pgfsetroundjoin%
\pgfsetlinewidth{1.505625pt}%
\definecolor{currentstroke}{rgb}{0.631373,0.788235,0.956863}%
\pgfsetstrokecolor{currentstroke}%
\pgfsetstrokeopacity{0.800000}%
\pgfsetdash{}{0pt}%
\pgfpathmoveto{\pgfqpoint{2.657795in}{1.379758in}}%
\pgfpathlineto{\pgfqpoint{3.563553in}{2.242061in}}%
\pgfusepath{stroke}%
\end{pgfscope}%
\begin{pgfscope}%
\pgfpathrectangle{\pgfqpoint{0.481978in}{0.331635in}}{\pgfqpoint{4.960000in}{3.696000in}}%
\pgfusepath{clip}%
\pgfsetrectcap%
\pgfsetroundjoin%
\pgfsetlinewidth{1.505625pt}%
\definecolor{currentstroke}{rgb}{0.631373,0.788235,0.956863}%
\pgfsetstrokecolor{currentstroke}%
\pgfsetstrokeopacity{0.800000}%
\pgfsetdash{}{0pt}%
\pgfpathmoveto{\pgfqpoint{3.546339in}{2.341602in}}%
\pgfpathlineto{\pgfqpoint{3.563553in}{2.242061in}}%
\pgfusepath{stroke}%
\end{pgfscope}%
\begin{pgfscope}%
\pgfpathrectangle{\pgfqpoint{0.481978in}{0.331635in}}{\pgfqpoint{4.960000in}{3.696000in}}%
\pgfusepath{clip}%
\pgfsetrectcap%
\pgfsetroundjoin%
\pgfsetlinewidth{1.505625pt}%
\definecolor{currentstroke}{rgb}{1.000000,0.705882,0.509804}%
\pgfsetstrokecolor{currentstroke}%
\pgfsetstrokeopacity{0.800000}%
\pgfsetdash{}{0pt}%
\pgfpathmoveto{\pgfqpoint{3.014752in}{3.195291in}}%
\pgfpathlineto{\pgfqpoint{2.628388in}{2.360622in}}%
\pgfusepath{stroke}%
\end{pgfscope}%
\begin{pgfscope}%
\pgfpathrectangle{\pgfqpoint{0.481978in}{0.331635in}}{\pgfqpoint{4.960000in}{3.696000in}}%
\pgfusepath{clip}%
\pgfsetrectcap%
\pgfsetroundjoin%
\pgfsetlinewidth{1.505625pt}%
\definecolor{currentstroke}{rgb}{1.000000,0.705882,0.509804}%
\pgfsetstrokecolor{currentstroke}%
\pgfsetstrokeopacity{0.800000}%
\pgfsetdash{}{0pt}%
\pgfpathmoveto{\pgfqpoint{5.085754in}{1.793372in}}%
\pgfpathlineto{\pgfqpoint{2.628388in}{2.360622in}}%
\pgfusepath{stroke}%
\end{pgfscope}%
\begin{pgfscope}%
\pgfpathrectangle{\pgfqpoint{0.481978in}{0.331635in}}{\pgfqpoint{4.960000in}{3.696000in}}%
\pgfusepath{clip}%
\pgfsetrectcap%
\pgfsetroundjoin%
\pgfsetlinewidth{1.505625pt}%
\definecolor{currentstroke}{rgb}{1.000000,0.705882,0.509804}%
\pgfsetstrokecolor{currentstroke}%
\pgfsetstrokeopacity{0.800000}%
\pgfsetdash{}{0pt}%
\pgfpathmoveto{\pgfqpoint{2.284015in}{2.600243in}}%
\pgfpathlineto{\pgfqpoint{2.628388in}{2.360622in}}%
\pgfusepath{stroke}%
\end{pgfscope}%
\begin{pgfscope}%
\pgfpathrectangle{\pgfqpoint{0.481978in}{0.331635in}}{\pgfqpoint{4.960000in}{3.696000in}}%
\pgfusepath{clip}%
\pgfsetrectcap%
\pgfsetroundjoin%
\pgfsetlinewidth{1.505625pt}%
\definecolor{currentstroke}{rgb}{1.000000,0.705882,0.509804}%
\pgfsetstrokecolor{currentstroke}%
\pgfsetstrokeopacity{0.800000}%
\pgfsetdash{}{0pt}%
\pgfpathmoveto{\pgfqpoint{1.922934in}{2.366558in}}%
\pgfpathlineto{\pgfqpoint{2.628388in}{2.360622in}}%
\pgfusepath{stroke}%
\end{pgfscope}%
\begin{pgfscope}%
\pgfpathrectangle{\pgfqpoint{0.481978in}{0.331635in}}{\pgfqpoint{4.960000in}{3.696000in}}%
\pgfusepath{clip}%
\pgfsetrectcap%
\pgfsetroundjoin%
\pgfsetlinewidth{1.505625pt}%
\definecolor{currentstroke}{rgb}{1.000000,0.705882,0.509804}%
\pgfsetstrokecolor{currentstroke}%
\pgfsetstrokeopacity{0.800000}%
\pgfsetdash{}{0pt}%
\pgfpathmoveto{\pgfqpoint{1.598169in}{2.719451in}}%
\pgfpathlineto{\pgfqpoint{2.628388in}{2.360622in}}%
\pgfusepath{stroke}%
\end{pgfscope}%
\begin{pgfscope}%
\pgfpathrectangle{\pgfqpoint{0.481978in}{0.331635in}}{\pgfqpoint{4.960000in}{3.696000in}}%
\pgfusepath{clip}%
\pgfsetrectcap%
\pgfsetroundjoin%
\pgfsetlinewidth{1.505625pt}%
\definecolor{currentstroke}{rgb}{1.000000,0.705882,0.509804}%
\pgfsetstrokecolor{currentstroke}%
\pgfsetstrokeopacity{0.800000}%
\pgfsetdash{}{0pt}%
\pgfpathmoveto{\pgfqpoint{3.105033in}{0.810912in}}%
\pgfpathlineto{\pgfqpoint{2.628388in}{2.360622in}}%
\pgfusepath{stroke}%
\end{pgfscope}%
\begin{pgfscope}%
\pgfpathrectangle{\pgfqpoint{0.481978in}{0.331635in}}{\pgfqpoint{4.960000in}{3.696000in}}%
\pgfusepath{clip}%
\pgfsetrectcap%
\pgfsetroundjoin%
\pgfsetlinewidth{1.505625pt}%
\definecolor{currentstroke}{rgb}{1.000000,0.705882,0.509804}%
\pgfsetstrokecolor{currentstroke}%
\pgfsetstrokeopacity{0.800000}%
\pgfsetdash{}{0pt}%
\pgfpathmoveto{\pgfqpoint{1.895234in}{3.315197in}}%
\pgfpathlineto{\pgfqpoint{2.628388in}{2.360622in}}%
\pgfusepath{stroke}%
\end{pgfscope}%
\begin{pgfscope}%
\pgfpathrectangle{\pgfqpoint{0.481978in}{0.331635in}}{\pgfqpoint{4.960000in}{3.696000in}}%
\pgfusepath{clip}%
\pgfsetrectcap%
\pgfsetroundjoin%
\pgfsetlinewidth{1.505625pt}%
\definecolor{currentstroke}{rgb}{1.000000,0.705882,0.509804}%
\pgfsetstrokecolor{currentstroke}%
\pgfsetstrokeopacity{0.800000}%
\pgfsetdash{}{0pt}%
\pgfpathmoveto{\pgfqpoint{3.360808in}{1.125508in}}%
\pgfpathlineto{\pgfqpoint{2.628388in}{2.360622in}}%
\pgfusepath{stroke}%
\end{pgfscope}%
\begin{pgfscope}%
\pgfpathrectangle{\pgfqpoint{0.481978in}{0.331635in}}{\pgfqpoint{4.960000in}{3.696000in}}%
\pgfusepath{clip}%
\pgfsetrectcap%
\pgfsetroundjoin%
\pgfsetlinewidth{1.505625pt}%
\definecolor{currentstroke}{rgb}{1.000000,0.705882,0.509804}%
\pgfsetstrokecolor{currentstroke}%
\pgfsetstrokeopacity{0.800000}%
\pgfsetdash{}{0pt}%
\pgfpathmoveto{\pgfqpoint{4.027785in}{2.169652in}}%
\pgfpathlineto{\pgfqpoint{2.628388in}{2.360622in}}%
\pgfusepath{stroke}%
\end{pgfscope}%
\begin{pgfscope}%
\pgfpathrectangle{\pgfqpoint{0.481978in}{0.331635in}}{\pgfqpoint{4.960000in}{3.696000in}}%
\pgfusepath{clip}%
\pgfsetrectcap%
\pgfsetroundjoin%
\pgfsetlinewidth{1.505625pt}%
\definecolor{currentstroke}{rgb}{1.000000,0.705882,0.509804}%
\pgfsetstrokecolor{currentstroke}%
\pgfsetstrokeopacity{0.800000}%
\pgfsetdash{}{0pt}%
\pgfpathmoveto{\pgfqpoint{1.211108in}{2.465118in}}%
\pgfpathlineto{\pgfqpoint{2.628388in}{2.360622in}}%
\pgfusepath{stroke}%
\end{pgfscope}%
\begin{pgfscope}%
\pgfpathrectangle{\pgfqpoint{0.481978in}{0.331635in}}{\pgfqpoint{4.960000in}{3.696000in}}%
\pgfusepath{clip}%
\pgfsetrectcap%
\pgfsetroundjoin%
\pgfsetlinewidth{1.505625pt}%
\definecolor{currentstroke}{rgb}{1.000000,0.705882,0.509804}%
\pgfsetstrokecolor{currentstroke}%
\pgfsetstrokeopacity{0.800000}%
\pgfsetdash{}{0pt}%
\pgfpathmoveto{\pgfqpoint{4.184791in}{0.943082in}}%
\pgfpathlineto{\pgfqpoint{2.628388in}{2.360622in}}%
\pgfusepath{stroke}%
\end{pgfscope}%
\begin{pgfscope}%
\pgfpathrectangle{\pgfqpoint{0.481978in}{0.331635in}}{\pgfqpoint{4.960000in}{3.696000in}}%
\pgfusepath{clip}%
\pgfsetrectcap%
\pgfsetroundjoin%
\pgfsetlinewidth{1.505625pt}%
\definecolor{currentstroke}{rgb}{1.000000,0.705882,0.509804}%
\pgfsetstrokecolor{currentstroke}%
\pgfsetstrokeopacity{0.800000}%
\pgfsetdash{}{0pt}%
\pgfpathmoveto{\pgfqpoint{1.789027in}{1.305832in}}%
\pgfpathlineto{\pgfqpoint{2.628388in}{2.360622in}}%
\pgfusepath{stroke}%
\end{pgfscope}%
\begin{pgfscope}%
\pgfpathrectangle{\pgfqpoint{0.481978in}{0.331635in}}{\pgfqpoint{4.960000in}{3.696000in}}%
\pgfusepath{clip}%
\pgfsetrectcap%
\pgfsetroundjoin%
\pgfsetlinewidth{1.505625pt}%
\definecolor{currentstroke}{rgb}{1.000000,0.705882,0.509804}%
\pgfsetstrokecolor{currentstroke}%
\pgfsetstrokeopacity{0.800000}%
\pgfsetdash{}{0pt}%
\pgfpathmoveto{\pgfqpoint{2.368419in}{1.005292in}}%
\pgfpathlineto{\pgfqpoint{2.628388in}{2.360622in}}%
\pgfusepath{stroke}%
\end{pgfscope}%
\begin{pgfscope}%
\pgfpathrectangle{\pgfqpoint{0.481978in}{0.331635in}}{\pgfqpoint{4.960000in}{3.696000in}}%
\pgfusepath{clip}%
\pgfsetrectcap%
\pgfsetroundjoin%
\pgfsetlinewidth{1.505625pt}%
\definecolor{currentstroke}{rgb}{1.000000,0.705882,0.509804}%
\pgfsetstrokecolor{currentstroke}%
\pgfsetstrokeopacity{0.800000}%
\pgfsetdash{}{0pt}%
\pgfpathmoveto{\pgfqpoint{1.427964in}{2.964440in}}%
\pgfpathlineto{\pgfqpoint{2.628388in}{2.360622in}}%
\pgfusepath{stroke}%
\end{pgfscope}%
\begin{pgfscope}%
\pgfpathrectangle{\pgfqpoint{0.481978in}{0.331635in}}{\pgfqpoint{4.960000in}{3.696000in}}%
\pgfusepath{clip}%
\pgfsetrectcap%
\pgfsetroundjoin%
\pgfsetlinewidth{1.505625pt}%
\definecolor{currentstroke}{rgb}{1.000000,0.705882,0.509804}%
\pgfsetstrokecolor{currentstroke}%
\pgfsetstrokeopacity{0.800000}%
\pgfsetdash{}{0pt}%
\pgfpathmoveto{\pgfqpoint{3.393546in}{3.859635in}}%
\pgfpathlineto{\pgfqpoint{2.628388in}{2.360622in}}%
\pgfusepath{stroke}%
\end{pgfscope}%
\begin{pgfscope}%
\pgfpathrectangle{\pgfqpoint{0.481978in}{0.331635in}}{\pgfqpoint{4.960000in}{3.696000in}}%
\pgfusepath{clip}%
\pgfsetrectcap%
\pgfsetroundjoin%
\pgfsetlinewidth{1.505625pt}%
\definecolor{currentstroke}{rgb}{1.000000,0.705882,0.509804}%
\pgfsetstrokecolor{currentstroke}%
\pgfsetstrokeopacity{0.800000}%
\pgfsetdash{}{0pt}%
\pgfpathmoveto{\pgfqpoint{1.952110in}{1.628055in}}%
\pgfpathlineto{\pgfqpoint{2.628388in}{2.360622in}}%
\pgfusepath{stroke}%
\end{pgfscope}%
\begin{pgfscope}%
\pgfpathrectangle{\pgfqpoint{0.481978in}{0.331635in}}{\pgfqpoint{4.960000in}{3.696000in}}%
\pgfusepath{clip}%
\pgfsetrectcap%
\pgfsetroundjoin%
\pgfsetlinewidth{1.505625pt}%
\definecolor{currentstroke}{rgb}{1.000000,0.705882,0.509804}%
\pgfsetstrokecolor{currentstroke}%
\pgfsetstrokeopacity{0.800000}%
\pgfsetdash{}{0pt}%
\pgfpathmoveto{\pgfqpoint{2.726428in}{2.857591in}}%
\pgfpathlineto{\pgfqpoint{2.628388in}{2.360622in}}%
\pgfusepath{stroke}%
\end{pgfscope}%
\begin{pgfscope}%
\pgfpathrectangle{\pgfqpoint{0.481978in}{0.331635in}}{\pgfqpoint{4.960000in}{3.696000in}}%
\pgfusepath{clip}%
\pgfsetrectcap%
\pgfsetroundjoin%
\pgfsetlinewidth{1.505625pt}%
\definecolor{currentstroke}{rgb}{1.000000,0.705882,0.509804}%
\pgfsetstrokecolor{currentstroke}%
\pgfsetstrokeopacity{0.800000}%
\pgfsetdash{}{0pt}%
\pgfpathmoveto{\pgfqpoint{1.372997in}{1.856203in}}%
\pgfpathlineto{\pgfqpoint{2.628388in}{2.360622in}}%
\pgfusepath{stroke}%
\end{pgfscope}%
\begin{pgfscope}%
\pgfpathrectangle{\pgfqpoint{0.481978in}{0.331635in}}{\pgfqpoint{4.960000in}{3.696000in}}%
\pgfusepath{clip}%
\pgfsetrectcap%
\pgfsetroundjoin%
\pgfsetlinewidth{1.505625pt}%
\definecolor{currentstroke}{rgb}{1.000000,0.705882,0.509804}%
\pgfsetstrokecolor{currentstroke}%
\pgfsetstrokeopacity{0.800000}%
\pgfsetdash{}{0pt}%
\pgfpathmoveto{\pgfqpoint{0.707432in}{3.144211in}}%
\pgfpathlineto{\pgfqpoint{2.628388in}{2.360622in}}%
\pgfusepath{stroke}%
\end{pgfscope}%
\begin{pgfscope}%
\pgfpathrectangle{\pgfqpoint{0.481978in}{0.331635in}}{\pgfqpoint{4.960000in}{3.696000in}}%
\pgfusepath{clip}%
\pgfsetrectcap%
\pgfsetroundjoin%
\pgfsetlinewidth{1.505625pt}%
\definecolor{currentstroke}{rgb}{1.000000,0.705882,0.509804}%
\pgfsetstrokecolor{currentstroke}%
\pgfsetstrokeopacity{0.800000}%
\pgfsetdash{}{0pt}%
\pgfpathmoveto{\pgfqpoint{3.782152in}{0.499635in}}%
\pgfpathlineto{\pgfqpoint{2.628388in}{2.360622in}}%
\pgfusepath{stroke}%
\end{pgfscope}%
\begin{pgfscope}%
\pgfpathrectangle{\pgfqpoint{0.481978in}{0.331635in}}{\pgfqpoint{4.960000in}{3.696000in}}%
\pgfusepath{clip}%
\pgfsetrectcap%
\pgfsetroundjoin%
\pgfsetlinewidth{1.505625pt}%
\definecolor{currentstroke}{rgb}{1.000000,0.705882,0.509804}%
\pgfsetstrokecolor{currentstroke}%
\pgfsetstrokeopacity{0.800000}%
\pgfsetdash{}{0pt}%
\pgfpathmoveto{\pgfqpoint{4.764267in}{2.747524in}}%
\pgfpathlineto{\pgfqpoint{2.628388in}{2.360622in}}%
\pgfusepath{stroke}%
\end{pgfscope}%
\begin{pgfscope}%
\pgfpathrectangle{\pgfqpoint{0.481978in}{0.331635in}}{\pgfqpoint{4.960000in}{3.696000in}}%
\pgfusepath{clip}%
\pgfsetrectcap%
\pgfsetroundjoin%
\pgfsetlinewidth{1.505625pt}%
\definecolor{currentstroke}{rgb}{1.000000,0.705882,0.509804}%
\pgfsetstrokecolor{currentstroke}%
\pgfsetstrokeopacity{0.800000}%
\pgfsetdash{}{0pt}%
\pgfpathmoveto{\pgfqpoint{1.027884in}{3.310449in}}%
\pgfpathlineto{\pgfqpoint{2.628388in}{2.360622in}}%
\pgfusepath{stroke}%
\end{pgfscope}%
\begin{pgfscope}%
\pgfpathrectangle{\pgfqpoint{0.481978in}{0.331635in}}{\pgfqpoint{4.960000in}{3.696000in}}%
\pgfusepath{clip}%
\pgfsetrectcap%
\pgfsetroundjoin%
\pgfsetlinewidth{1.505625pt}%
\definecolor{currentstroke}{rgb}{1.000000,0.705882,0.509804}%
\pgfsetstrokecolor{currentstroke}%
\pgfsetstrokeopacity{0.800000}%
\pgfsetdash{}{0pt}%
\pgfpathmoveto{\pgfqpoint{2.571490in}{3.587745in}}%
\pgfpathlineto{\pgfqpoint{2.628388in}{2.360622in}}%
\pgfusepath{stroke}%
\end{pgfscope}%
\begin{pgfscope}%
\pgfpathrectangle{\pgfqpoint{0.481978in}{0.331635in}}{\pgfqpoint{4.960000in}{3.696000in}}%
\pgfusepath{clip}%
\pgfsetrectcap%
\pgfsetroundjoin%
\pgfsetlinewidth{1.505625pt}%
\definecolor{currentstroke}{rgb}{1.000000,0.705882,0.509804}%
\pgfsetstrokecolor{currentstroke}%
\pgfsetstrokeopacity{0.800000}%
\pgfsetdash{}{0pt}%
\pgfpathmoveto{\pgfqpoint{2.788619in}{2.586793in}}%
\pgfpathlineto{\pgfqpoint{2.628388in}{2.360622in}}%
\pgfusepath{stroke}%
\end{pgfscope}%
\begin{pgfscope}%
\pgfpathrectangle{\pgfqpoint{0.481978in}{0.331635in}}{\pgfqpoint{4.960000in}{3.696000in}}%
\pgfusepath{clip}%
\pgfsetrectcap%
\pgfsetroundjoin%
\pgfsetlinewidth{1.505625pt}%
\definecolor{currentstroke}{rgb}{1.000000,0.705882,0.509804}%
\pgfsetstrokecolor{currentstroke}%
\pgfsetstrokeopacity{0.800000}%
\pgfsetdash{}{0pt}%
\pgfpathmoveto{\pgfqpoint{3.846440in}{2.939605in}}%
\pgfpathlineto{\pgfqpoint{2.628388in}{2.360622in}}%
\pgfusepath{stroke}%
\end{pgfscope}%
\begin{pgfscope}%
\pgfpathrectangle{\pgfqpoint{0.481978in}{0.331635in}}{\pgfqpoint{4.960000in}{3.696000in}}%
\pgfusepath{clip}%
\pgfsetrectcap%
\pgfsetroundjoin%
\pgfsetlinewidth{1.505625pt}%
\definecolor{currentstroke}{rgb}{1.000000,0.705882,0.509804}%
\pgfsetstrokecolor{currentstroke}%
\pgfsetstrokeopacity{0.800000}%
\pgfsetdash{}{0pt}%
\pgfpathmoveto{\pgfqpoint{4.322794in}{2.464911in}}%
\pgfpathlineto{\pgfqpoint{2.628388in}{2.360622in}}%
\pgfusepath{stroke}%
\end{pgfscope}%
\begin{pgfscope}%
\pgfpathrectangle{\pgfqpoint{0.481978in}{0.331635in}}{\pgfqpoint{4.960000in}{3.696000in}}%
\pgfusepath{clip}%
\pgfsetrectcap%
\pgfsetroundjoin%
\pgfsetlinewidth{1.505625pt}%
\definecolor{currentstroke}{rgb}{1.000000,0.705882,0.509804}%
\pgfsetstrokecolor{currentstroke}%
\pgfsetstrokeopacity{0.800000}%
\pgfsetdash{}{0pt}%
\pgfpathmoveto{\pgfqpoint{1.013445in}{2.895267in}}%
\pgfpathlineto{\pgfqpoint{2.628388in}{2.360622in}}%
\pgfusepath{stroke}%
\end{pgfscope}%
\begin{pgfscope}%
\pgfpathrectangle{\pgfqpoint{0.481978in}{0.331635in}}{\pgfqpoint{4.960000in}{3.696000in}}%
\pgfusepath{clip}%
\pgfsetrectcap%
\pgfsetroundjoin%
\pgfsetlinewidth{1.505625pt}%
\definecolor{currentstroke}{rgb}{1.000000,0.705882,0.509804}%
\pgfsetstrokecolor{currentstroke}%
\pgfsetstrokeopacity{0.800000}%
\pgfsetdash{}{0pt}%
\pgfpathmoveto{\pgfqpoint{2.049477in}{2.939835in}}%
\pgfpathlineto{\pgfqpoint{2.628388in}{2.360622in}}%
\pgfusepath{stroke}%
\end{pgfscope}%
\begin{pgfscope}%
\pgfsetrectcap%
\pgfsetmiterjoin%
\pgfsetlinewidth{0.803000pt}%
\definecolor{currentstroke}{rgb}{0.000000,0.000000,0.000000}%
\pgfsetstrokecolor{currentstroke}%
\pgfsetdash{}{0pt}%
\pgfpathmoveto{\pgfqpoint{0.481978in}{0.331635in}}%
\pgfpathlineto{\pgfqpoint{0.481978in}{4.027635in}}%
\pgfusepath{stroke}%
\end{pgfscope}%
\begin{pgfscope}%
\pgfsetrectcap%
\pgfsetmiterjoin%
\pgfsetlinewidth{0.803000pt}%
\definecolor{currentstroke}{rgb}{0.000000,0.000000,0.000000}%
\pgfsetstrokecolor{currentstroke}%
\pgfsetdash{}{0pt}%
\pgfpathmoveto{\pgfqpoint{5.441978in}{0.331635in}}%
\pgfpathlineto{\pgfqpoint{5.441978in}{4.027635in}}%
\pgfusepath{stroke}%
\end{pgfscope}%
\begin{pgfscope}%
\pgfsetrectcap%
\pgfsetmiterjoin%
\pgfsetlinewidth{0.803000pt}%
\definecolor{currentstroke}{rgb}{0.000000,0.000000,0.000000}%
\pgfsetstrokecolor{currentstroke}%
\pgfsetdash{}{0pt}%
\pgfpathmoveto{\pgfqpoint{0.481978in}{0.331635in}}%
\pgfpathlineto{\pgfqpoint{5.441978in}{0.331635in}}%
\pgfusepath{stroke}%
\end{pgfscope}%
\begin{pgfscope}%
\pgfsetrectcap%
\pgfsetmiterjoin%
\pgfsetlinewidth{0.803000pt}%
\definecolor{currentstroke}{rgb}{0.000000,0.000000,0.000000}%
\pgfsetstrokecolor{currentstroke}%
\pgfsetdash{}{0pt}%
\pgfpathmoveto{\pgfqpoint{0.481978in}{4.027635in}}%
\pgfpathlineto{\pgfqpoint{5.441978in}{4.027635in}}%
\pgfusepath{stroke}%
\end{pgfscope}%
\begin{pgfscope}%
\definecolor{textcolor}{rgb}{0.000000,0.000000,0.000000}%
\pgfsetstrokecolor{textcolor}%
\pgfsetfillcolor{textcolor}%
\pgftext[x=2.961978in,y=4.110968in,,base]{\color{textcolor}\sffamily\fontsize{12.000000}{14.400000}\selectfont t-SNE for pix3d and scenenet}%
\end{pgfscope}%
\begin{pgfscope}%
\pgfsetbuttcap%
\pgfsetmiterjoin%
\definecolor{currentfill}{rgb}{1.000000,1.000000,1.000000}%
\pgfsetfillcolor{currentfill}%
\pgfsetfillopacity{0.800000}%
\pgfsetlinewidth{1.003750pt}%
\definecolor{currentstroke}{rgb}{0.800000,0.800000,0.800000}%
\pgfsetstrokecolor{currentstroke}%
\pgfsetstrokeopacity{0.800000}%
\pgfsetdash{}{0pt}%
\pgfpathmoveto{\pgfqpoint{4.264732in}{3.508809in}}%
\pgfpathlineto{\pgfqpoint{5.344756in}{3.508809in}}%
\pgfpathquadraticcurveto{\pgfqpoint{5.372533in}{3.508809in}}{\pgfqpoint{5.372533in}{3.536587in}}%
\pgfpathlineto{\pgfqpoint{5.372533in}{3.930413in}}%
\pgfpathquadraticcurveto{\pgfqpoint{5.372533in}{3.958191in}}{\pgfqpoint{5.344756in}{3.958191in}}%
\pgfpathlineto{\pgfqpoint{4.264732in}{3.958191in}}%
\pgfpathquadraticcurveto{\pgfqpoint{4.236954in}{3.958191in}}{\pgfqpoint{4.236954in}{3.930413in}}%
\pgfpathlineto{\pgfqpoint{4.236954in}{3.536587in}}%
\pgfpathquadraticcurveto{\pgfqpoint{4.236954in}{3.508809in}}{\pgfqpoint{4.264732in}{3.508809in}}%
\pgfpathclose%
\pgfusepath{stroke,fill}%
\end{pgfscope}%
\begin{pgfscope}%
\pgfsetbuttcap%
\pgfsetroundjoin%
\definecolor{currentfill}{rgb}{0.631373,0.788235,0.956863}%
\pgfsetfillcolor{currentfill}%
\pgfsetlinewidth{1.003750pt}%
\definecolor{currentstroke}{rgb}{0.631373,0.788235,0.956863}%
\pgfsetstrokecolor{currentstroke}%
\pgfsetdash{}{0pt}%
\pgfsys@defobject{currentmarker}{\pgfqpoint{-0.041667in}{-0.041667in}}{\pgfqpoint{0.041667in}{0.041667in}}{%
\pgfpathmoveto{\pgfqpoint{0.000000in}{-0.041667in}}%
\pgfpathcurveto{\pgfqpoint{0.011050in}{-0.041667in}}{\pgfqpoint{0.021649in}{-0.037276in}}{\pgfqpoint{0.029463in}{-0.029463in}}%
\pgfpathcurveto{\pgfqpoint{0.037276in}{-0.021649in}}{\pgfqpoint{0.041667in}{-0.011050in}}{\pgfqpoint{0.041667in}{0.000000in}}%
\pgfpathcurveto{\pgfqpoint{0.041667in}{0.011050in}}{\pgfqpoint{0.037276in}{0.021649in}}{\pgfqpoint{0.029463in}{0.029463in}}%
\pgfpathcurveto{\pgfqpoint{0.021649in}{0.037276in}}{\pgfqpoint{0.011050in}{0.041667in}}{\pgfqpoint{0.000000in}{0.041667in}}%
\pgfpathcurveto{\pgfqpoint{-0.011050in}{0.041667in}}{\pgfqpoint{-0.021649in}{0.037276in}}{\pgfqpoint{-0.029463in}{0.029463in}}%
\pgfpathcurveto{\pgfqpoint{-0.037276in}{0.021649in}}{\pgfqpoint{-0.041667in}{0.011050in}}{\pgfqpoint{-0.041667in}{0.000000in}}%
\pgfpathcurveto{\pgfqpoint{-0.041667in}{-0.011050in}}{\pgfqpoint{-0.037276in}{-0.021649in}}{\pgfqpoint{-0.029463in}{-0.029463in}}%
\pgfpathcurveto{\pgfqpoint{-0.021649in}{-0.037276in}}{\pgfqpoint{-0.011050in}{-0.041667in}}{\pgfqpoint{0.000000in}{-0.041667in}}%
\pgfpathclose%
\pgfusepath{stroke,fill}%
}%
\begin{pgfscope}%
\pgfsys@transformshift{4.431398in}{3.833570in}%
\pgfsys@useobject{currentmarker}{}%
\end{pgfscope}%
\end{pgfscope}%
\begin{pgfscope}%
\definecolor{textcolor}{rgb}{0.000000,0.000000,0.000000}%
\pgfsetstrokecolor{textcolor}%
\pgfsetfillcolor{textcolor}%
\pgftext[x=4.681398in,y=3.797112in,left,base]{\color{textcolor}\sffamily\fontsize{10.000000}{12.000000}\selectfont scenenet}%
\end{pgfscope}%
\begin{pgfscope}%
\pgfsetbuttcap%
\pgfsetroundjoin%
\definecolor{currentfill}{rgb}{1.000000,0.705882,0.509804}%
\pgfsetfillcolor{currentfill}%
\pgfsetlinewidth{1.003750pt}%
\definecolor{currentstroke}{rgb}{1.000000,0.705882,0.509804}%
\pgfsetstrokecolor{currentstroke}%
\pgfsetdash{}{0pt}%
\pgfsys@defobject{currentmarker}{\pgfqpoint{-0.041667in}{-0.041667in}}{\pgfqpoint{0.041667in}{0.041667in}}{%
\pgfpathmoveto{\pgfqpoint{0.000000in}{-0.041667in}}%
\pgfpathcurveto{\pgfqpoint{0.011050in}{-0.041667in}}{\pgfqpoint{0.021649in}{-0.037276in}}{\pgfqpoint{0.029463in}{-0.029463in}}%
\pgfpathcurveto{\pgfqpoint{0.037276in}{-0.021649in}}{\pgfqpoint{0.041667in}{-0.011050in}}{\pgfqpoint{0.041667in}{0.000000in}}%
\pgfpathcurveto{\pgfqpoint{0.041667in}{0.011050in}}{\pgfqpoint{0.037276in}{0.021649in}}{\pgfqpoint{0.029463in}{0.029463in}}%
\pgfpathcurveto{\pgfqpoint{0.021649in}{0.037276in}}{\pgfqpoint{0.011050in}{0.041667in}}{\pgfqpoint{0.000000in}{0.041667in}}%
\pgfpathcurveto{\pgfqpoint{-0.011050in}{0.041667in}}{\pgfqpoint{-0.021649in}{0.037276in}}{\pgfqpoint{-0.029463in}{0.029463in}}%
\pgfpathcurveto{\pgfqpoint{-0.037276in}{0.021649in}}{\pgfqpoint{-0.041667in}{0.011050in}}{\pgfqpoint{-0.041667in}{0.000000in}}%
\pgfpathcurveto{\pgfqpoint{-0.041667in}{-0.011050in}}{\pgfqpoint{-0.037276in}{-0.021649in}}{\pgfqpoint{-0.029463in}{-0.029463in}}%
\pgfpathcurveto{\pgfqpoint{-0.021649in}{-0.037276in}}{\pgfqpoint{-0.011050in}{-0.041667in}}{\pgfqpoint{0.000000in}{-0.041667in}}%
\pgfpathclose%
\pgfusepath{stroke,fill}%
}%
\begin{pgfscope}%
\pgfsys@transformshift{4.431398in}{3.629713in}%
\pgfsys@useobject{currentmarker}{}%
\end{pgfscope}%
\end{pgfscope}%
\begin{pgfscope}%
\definecolor{textcolor}{rgb}{0.000000,0.000000,0.000000}%
\pgfsetstrokecolor{textcolor}%
\pgfsetfillcolor{textcolor}%
\pgftext[x=4.681398in,y=3.593255in,left,base]{\color{textcolor}\sffamily\fontsize{10.000000}{12.000000}\selectfont pix3d}%
\end{pgfscope}%
\end{pgfpicture}%
\makeatother%
\endgroup%
}\\
    \resizebox{0.49\linewidth}{5cm}{%% Creator: Matplotlib, PGF backend
%%
%% To include the figure in your LaTeX document, write
%%   \input{<filename>.pgf}
%%
%% Make sure the required packages are loaded in your preamble
%%   \usepackage{pgf}
%%
%% Figures using additional raster images can only be included by \input if
%% they are in the same directory as the main LaTeX file. For loading figures
%% from other directories you can use the `import` package
%%   \usepackage{import}
%%
%% and then include the figures with
%%   \import{<path to file>}{<filename>.pgf}
%%
%% Matplotlib used the following preamble
%%   \usepackage{fontspec}
%%   \setmainfont{DejaVuSerif.ttf}[Path=\detokenize{/Users/apple/opt/anaconda3/envs/kaolin/lib/python3.7/site-packages/matplotlib/mpl-data/fonts/ttf/}]
%%   \setsansfont{DejaVuSans.ttf}[Path=\detokenize{/Users/apple/opt/anaconda3/envs/kaolin/lib/python3.7/site-packages/matplotlib/mpl-data/fonts/ttf/}]
%%   \setmonofont{DejaVuSansMono.ttf}[Path=\detokenize{/Users/apple/opt/anaconda3/envs/kaolin/lib/python3.7/site-packages/matplotlib/mpl-data/fonts/ttf/}]
%%
\begingroup%
\makeatletter%
\begin{pgfpicture}%
\pgfpathrectangle{\pgfpointorigin}{\pgfqpoint{11.374274in}{8.341596in}}%
\pgfusepath{use as bounding box, clip}%
\begin{pgfscope}%
\pgfsetbuttcap%
\pgfsetmiterjoin%
\definecolor{currentfill}{rgb}{1.000000,1.000000,1.000000}%
\pgfsetfillcolor{currentfill}%
\pgfsetlinewidth{0.000000pt}%
\definecolor{currentstroke}{rgb}{1.000000,1.000000,1.000000}%
\pgfsetstrokecolor{currentstroke}%
\pgfsetdash{}{0pt}%
\pgfpathmoveto{\pgfqpoint{0.000000in}{0.000000in}}%
\pgfpathlineto{\pgfqpoint{11.374274in}{0.000000in}}%
\pgfpathlineto{\pgfqpoint{11.374274in}{8.341596in}}%
\pgfpathlineto{\pgfqpoint{0.000000in}{8.341596in}}%
\pgfpathclose%
\pgfusepath{fill}%
\end{pgfscope}%
\begin{pgfscope}%
\pgfsetbuttcap%
\pgfsetmiterjoin%
\definecolor{currentfill}{rgb}{1.000000,1.000000,1.000000}%
\pgfsetfillcolor{currentfill}%
\pgfsetlinewidth{0.000000pt}%
\definecolor{currentstroke}{rgb}{0.000000,0.000000,0.000000}%
\pgfsetstrokecolor{currentstroke}%
\pgfsetstrokeopacity{0.000000}%
\pgfsetdash{}{0pt}%
\pgfpathmoveto{\pgfqpoint{0.570343in}{0.331635in}}%
\pgfpathlineto{\pgfqpoint{9.870343in}{0.331635in}}%
\pgfpathlineto{\pgfqpoint{9.870343in}{8.031635in}}%
\pgfpathlineto{\pgfqpoint{0.570343in}{8.031635in}}%
\pgfpathclose%
\pgfusepath{fill}%
\end{pgfscope}%
\begin{pgfscope}%
\pgfpathrectangle{\pgfqpoint{0.570343in}{0.331635in}}{\pgfqpoint{9.300000in}{7.700000in}}%
\pgfusepath{clip}%
\pgfsetbuttcap%
\pgfsetroundjoin%
\definecolor{currentfill}{rgb}{0.631373,0.788235,0.956863}%
\pgfsetfillcolor{currentfill}%
\pgfsetlinewidth{0.481800pt}%
\definecolor{currentstroke}{rgb}{1.000000,1.000000,1.000000}%
\pgfsetstrokecolor{currentstroke}%
\pgfsetdash{}{0pt}%
\pgfpathmoveto{\pgfqpoint{2.738718in}{6.340074in}}%
\pgfpathcurveto{\pgfqpoint{2.749768in}{6.340074in}}{\pgfqpoint{2.760367in}{6.344464in}}{\pgfqpoint{2.768181in}{6.352277in}}%
\pgfpathcurveto{\pgfqpoint{2.775994in}{6.360091in}}{\pgfqpoint{2.780384in}{6.370690in}}{\pgfqpoint{2.780384in}{6.381740in}}%
\pgfpathcurveto{\pgfqpoint{2.780384in}{6.392790in}}{\pgfqpoint{2.775994in}{6.403389in}}{\pgfqpoint{2.768181in}{6.411203in}}%
\pgfpathcurveto{\pgfqpoint{2.760367in}{6.419017in}}{\pgfqpoint{2.749768in}{6.423407in}}{\pgfqpoint{2.738718in}{6.423407in}}%
\pgfpathcurveto{\pgfqpoint{2.727668in}{6.423407in}}{\pgfqpoint{2.717069in}{6.419017in}}{\pgfqpoint{2.709255in}{6.411203in}}%
\pgfpathcurveto{\pgfqpoint{2.701441in}{6.403389in}}{\pgfqpoint{2.697051in}{6.392790in}}{\pgfqpoint{2.697051in}{6.381740in}}%
\pgfpathcurveto{\pgfqpoint{2.697051in}{6.370690in}}{\pgfqpoint{2.701441in}{6.360091in}}{\pgfqpoint{2.709255in}{6.352277in}}%
\pgfpathcurveto{\pgfqpoint{2.717069in}{6.344464in}}{\pgfqpoint{2.727668in}{6.340074in}}{\pgfqpoint{2.738718in}{6.340074in}}%
\pgfpathclose%
\pgfusepath{stroke,fill}%
\end{pgfscope}%
\begin{pgfscope}%
\pgfpathrectangle{\pgfqpoint{0.570343in}{0.331635in}}{\pgfqpoint{9.300000in}{7.700000in}}%
\pgfusepath{clip}%
\pgfsetbuttcap%
\pgfsetroundjoin%
\definecolor{currentfill}{rgb}{0.631373,0.788235,0.956863}%
\pgfsetfillcolor{currentfill}%
\pgfsetlinewidth{0.481800pt}%
\definecolor{currentstroke}{rgb}{1.000000,1.000000,1.000000}%
\pgfsetstrokecolor{currentstroke}%
\pgfsetdash{}{0pt}%
\pgfpathmoveto{\pgfqpoint{4.645211in}{3.331282in}}%
\pgfpathcurveto{\pgfqpoint{4.656261in}{3.331282in}}{\pgfqpoint{4.666860in}{3.335672in}}{\pgfqpoint{4.674674in}{3.343485in}}%
\pgfpathcurveto{\pgfqpoint{4.682487in}{3.351299in}}{\pgfqpoint{4.686878in}{3.361898in}}{\pgfqpoint{4.686878in}{3.372948in}}%
\pgfpathcurveto{\pgfqpoint{4.686878in}{3.383998in}}{\pgfqpoint{4.682487in}{3.394597in}}{\pgfqpoint{4.674674in}{3.402411in}}%
\pgfpathcurveto{\pgfqpoint{4.666860in}{3.410225in}}{\pgfqpoint{4.656261in}{3.414615in}}{\pgfqpoint{4.645211in}{3.414615in}}%
\pgfpathcurveto{\pgfqpoint{4.634161in}{3.414615in}}{\pgfqpoint{4.623562in}{3.410225in}}{\pgfqpoint{4.615748in}{3.402411in}}%
\pgfpathcurveto{\pgfqpoint{4.607935in}{3.394597in}}{\pgfqpoint{4.603544in}{3.383998in}}{\pgfqpoint{4.603544in}{3.372948in}}%
\pgfpathcurveto{\pgfqpoint{4.603544in}{3.361898in}}{\pgfqpoint{4.607935in}{3.351299in}}{\pgfqpoint{4.615748in}{3.343485in}}%
\pgfpathcurveto{\pgfqpoint{4.623562in}{3.335672in}}{\pgfqpoint{4.634161in}{3.331282in}}{\pgfqpoint{4.645211in}{3.331282in}}%
\pgfpathclose%
\pgfusepath{stroke,fill}%
\end{pgfscope}%
\begin{pgfscope}%
\pgfpathrectangle{\pgfqpoint{0.570343in}{0.331635in}}{\pgfqpoint{9.300000in}{7.700000in}}%
\pgfusepath{clip}%
\pgfsetbuttcap%
\pgfsetroundjoin%
\definecolor{currentfill}{rgb}{0.631373,0.788235,0.956863}%
\pgfsetfillcolor{currentfill}%
\pgfsetlinewidth{0.481800pt}%
\definecolor{currentstroke}{rgb}{1.000000,1.000000,1.000000}%
\pgfsetstrokecolor{currentstroke}%
\pgfsetdash{}{0pt}%
\pgfpathmoveto{\pgfqpoint{5.785764in}{4.741451in}}%
\pgfpathcurveto{\pgfqpoint{5.796815in}{4.741451in}}{\pgfqpoint{5.807414in}{4.745842in}}{\pgfqpoint{5.815227in}{4.753655in}}%
\pgfpathcurveto{\pgfqpoint{5.823041in}{4.761469in}}{\pgfqpoint{5.827431in}{4.772068in}}{\pgfqpoint{5.827431in}{4.783118in}}%
\pgfpathcurveto{\pgfqpoint{5.827431in}{4.794168in}}{\pgfqpoint{5.823041in}{4.804767in}}{\pgfqpoint{5.815227in}{4.812581in}}%
\pgfpathcurveto{\pgfqpoint{5.807414in}{4.820394in}}{\pgfqpoint{5.796815in}{4.824785in}}{\pgfqpoint{5.785764in}{4.824785in}}%
\pgfpathcurveto{\pgfqpoint{5.774714in}{4.824785in}}{\pgfqpoint{5.764115in}{4.820394in}}{\pgfqpoint{5.756302in}{4.812581in}}%
\pgfpathcurveto{\pgfqpoint{5.748488in}{4.804767in}}{\pgfqpoint{5.744098in}{4.794168in}}{\pgfqpoint{5.744098in}{4.783118in}}%
\pgfpathcurveto{\pgfqpoint{5.744098in}{4.772068in}}{\pgfqpoint{5.748488in}{4.761469in}}{\pgfqpoint{5.756302in}{4.753655in}}%
\pgfpathcurveto{\pgfqpoint{5.764115in}{4.745842in}}{\pgfqpoint{5.774714in}{4.741451in}}{\pgfqpoint{5.785764in}{4.741451in}}%
\pgfpathclose%
\pgfusepath{stroke,fill}%
\end{pgfscope}%
\begin{pgfscope}%
\pgfpathrectangle{\pgfqpoint{0.570343in}{0.331635in}}{\pgfqpoint{9.300000in}{7.700000in}}%
\pgfusepath{clip}%
\pgfsetbuttcap%
\pgfsetroundjoin%
\definecolor{currentfill}{rgb}{0.631373,0.788235,0.956863}%
\pgfsetfillcolor{currentfill}%
\pgfsetlinewidth{0.481800pt}%
\definecolor{currentstroke}{rgb}{1.000000,1.000000,1.000000}%
\pgfsetstrokecolor{currentstroke}%
\pgfsetdash{}{0pt}%
\pgfpathmoveto{\pgfqpoint{5.100277in}{4.600045in}}%
\pgfpathcurveto{\pgfqpoint{5.111327in}{4.600045in}}{\pgfqpoint{5.121926in}{4.604435in}}{\pgfqpoint{5.129740in}{4.612249in}}%
\pgfpathcurveto{\pgfqpoint{5.137553in}{4.620062in}}{\pgfqpoint{5.141944in}{4.630661in}}{\pgfqpoint{5.141944in}{4.641711in}}%
\pgfpathcurveto{\pgfqpoint{5.141944in}{4.652761in}}{\pgfqpoint{5.137553in}{4.663360in}}{\pgfqpoint{5.129740in}{4.671174in}}%
\pgfpathcurveto{\pgfqpoint{5.121926in}{4.678988in}}{\pgfqpoint{5.111327in}{4.683378in}}{\pgfqpoint{5.100277in}{4.683378in}}%
\pgfpathcurveto{\pgfqpoint{5.089227in}{4.683378in}}{\pgfqpoint{5.078628in}{4.678988in}}{\pgfqpoint{5.070814in}{4.671174in}}%
\pgfpathcurveto{\pgfqpoint{5.063001in}{4.663360in}}{\pgfqpoint{5.058610in}{4.652761in}}{\pgfqpoint{5.058610in}{4.641711in}}%
\pgfpathcurveto{\pgfqpoint{5.058610in}{4.630661in}}{\pgfqpoint{5.063001in}{4.620062in}}{\pgfqpoint{5.070814in}{4.612249in}}%
\pgfpathcurveto{\pgfqpoint{5.078628in}{4.604435in}}{\pgfqpoint{5.089227in}{4.600045in}}{\pgfqpoint{5.100277in}{4.600045in}}%
\pgfpathclose%
\pgfusepath{stroke,fill}%
\end{pgfscope}%
\begin{pgfscope}%
\pgfpathrectangle{\pgfqpoint{0.570343in}{0.331635in}}{\pgfqpoint{9.300000in}{7.700000in}}%
\pgfusepath{clip}%
\pgfsetbuttcap%
\pgfsetroundjoin%
\definecolor{currentfill}{rgb}{0.631373,0.788235,0.956863}%
\pgfsetfillcolor{currentfill}%
\pgfsetlinewidth{0.481800pt}%
\definecolor{currentstroke}{rgb}{1.000000,1.000000,1.000000}%
\pgfsetstrokecolor{currentstroke}%
\pgfsetdash{}{0pt}%
\pgfpathmoveto{\pgfqpoint{0.993071in}{4.256195in}}%
\pgfpathcurveto{\pgfqpoint{1.004121in}{4.256195in}}{\pgfqpoint{1.014720in}{4.260585in}}{\pgfqpoint{1.022533in}{4.268399in}}%
\pgfpathcurveto{\pgfqpoint{1.030347in}{4.276213in}}{\pgfqpoint{1.034737in}{4.286812in}}{\pgfqpoint{1.034737in}{4.297862in}}%
\pgfpathcurveto{\pgfqpoint{1.034737in}{4.308912in}}{\pgfqpoint{1.030347in}{4.319511in}}{\pgfqpoint{1.022533in}{4.327325in}}%
\pgfpathcurveto{\pgfqpoint{1.014720in}{4.335138in}}{\pgfqpoint{1.004121in}{4.339528in}}{\pgfqpoint{0.993071in}{4.339528in}}%
\pgfpathcurveto{\pgfqpoint{0.982020in}{4.339528in}}{\pgfqpoint{0.971421in}{4.335138in}}{\pgfqpoint{0.963608in}{4.327325in}}%
\pgfpathcurveto{\pgfqpoint{0.955794in}{4.319511in}}{\pgfqpoint{0.951404in}{4.308912in}}{\pgfqpoint{0.951404in}{4.297862in}}%
\pgfpathcurveto{\pgfqpoint{0.951404in}{4.286812in}}{\pgfqpoint{0.955794in}{4.276213in}}{\pgfqpoint{0.963608in}{4.268399in}}%
\pgfpathcurveto{\pgfqpoint{0.971421in}{4.260585in}}{\pgfqpoint{0.982020in}{4.256195in}}{\pgfqpoint{0.993071in}{4.256195in}}%
\pgfpathclose%
\pgfusepath{stroke,fill}%
\end{pgfscope}%
\begin{pgfscope}%
\pgfpathrectangle{\pgfqpoint{0.570343in}{0.331635in}}{\pgfqpoint{9.300000in}{7.700000in}}%
\pgfusepath{clip}%
\pgfsetbuttcap%
\pgfsetroundjoin%
\definecolor{currentfill}{rgb}{0.631373,0.788235,0.956863}%
\pgfsetfillcolor{currentfill}%
\pgfsetlinewidth{0.481800pt}%
\definecolor{currentstroke}{rgb}{1.000000,1.000000,1.000000}%
\pgfsetstrokecolor{currentstroke}%
\pgfsetdash{}{0pt}%
\pgfpathmoveto{\pgfqpoint{6.176499in}{3.572166in}}%
\pgfpathcurveto{\pgfqpoint{6.187549in}{3.572166in}}{\pgfqpoint{6.198148in}{3.576557in}}{\pgfqpoint{6.205962in}{3.584370in}}%
\pgfpathcurveto{\pgfqpoint{6.213775in}{3.592184in}}{\pgfqpoint{6.218166in}{3.602783in}}{\pgfqpoint{6.218166in}{3.613833in}}%
\pgfpathcurveto{\pgfqpoint{6.218166in}{3.624883in}}{\pgfqpoint{6.213775in}{3.635482in}}{\pgfqpoint{6.205962in}{3.643296in}}%
\pgfpathcurveto{\pgfqpoint{6.198148in}{3.651109in}}{\pgfqpoint{6.187549in}{3.655500in}}{\pgfqpoint{6.176499in}{3.655500in}}%
\pgfpathcurveto{\pgfqpoint{6.165449in}{3.655500in}}{\pgfqpoint{6.154850in}{3.651109in}}{\pgfqpoint{6.147036in}{3.643296in}}%
\pgfpathcurveto{\pgfqpoint{6.139223in}{3.635482in}}{\pgfqpoint{6.134832in}{3.624883in}}{\pgfqpoint{6.134832in}{3.613833in}}%
\pgfpathcurveto{\pgfqpoint{6.134832in}{3.602783in}}{\pgfqpoint{6.139223in}{3.592184in}}{\pgfqpoint{6.147036in}{3.584370in}}%
\pgfpathcurveto{\pgfqpoint{6.154850in}{3.576557in}}{\pgfqpoint{6.165449in}{3.572166in}}{\pgfqpoint{6.176499in}{3.572166in}}%
\pgfpathclose%
\pgfusepath{stroke,fill}%
\end{pgfscope}%
\begin{pgfscope}%
\pgfpathrectangle{\pgfqpoint{0.570343in}{0.331635in}}{\pgfqpoint{9.300000in}{7.700000in}}%
\pgfusepath{clip}%
\pgfsetbuttcap%
\pgfsetroundjoin%
\definecolor{currentfill}{rgb}{0.631373,0.788235,0.956863}%
\pgfsetfillcolor{currentfill}%
\pgfsetlinewidth{0.481800pt}%
\definecolor{currentstroke}{rgb}{1.000000,1.000000,1.000000}%
\pgfsetstrokecolor{currentstroke}%
\pgfsetdash{}{0pt}%
\pgfpathmoveto{\pgfqpoint{3.773953in}{2.538414in}}%
\pgfpathcurveto{\pgfqpoint{3.785003in}{2.538414in}}{\pgfqpoint{3.795602in}{2.542804in}}{\pgfqpoint{3.803416in}{2.550618in}}%
\pgfpathcurveto{\pgfqpoint{3.811229in}{2.558431in}}{\pgfqpoint{3.815620in}{2.569030in}}{\pgfqpoint{3.815620in}{2.580081in}}%
\pgfpathcurveto{\pgfqpoint{3.815620in}{2.591131in}}{\pgfqpoint{3.811229in}{2.601730in}}{\pgfqpoint{3.803416in}{2.609543in}}%
\pgfpathcurveto{\pgfqpoint{3.795602in}{2.617357in}}{\pgfqpoint{3.785003in}{2.621747in}}{\pgfqpoint{3.773953in}{2.621747in}}%
\pgfpathcurveto{\pgfqpoint{3.762903in}{2.621747in}}{\pgfqpoint{3.752304in}{2.617357in}}{\pgfqpoint{3.744490in}{2.609543in}}%
\pgfpathcurveto{\pgfqpoint{3.736677in}{2.601730in}}{\pgfqpoint{3.732286in}{2.591131in}}{\pgfqpoint{3.732286in}{2.580081in}}%
\pgfpathcurveto{\pgfqpoint{3.732286in}{2.569030in}}{\pgfqpoint{3.736677in}{2.558431in}}{\pgfqpoint{3.744490in}{2.550618in}}%
\pgfpathcurveto{\pgfqpoint{3.752304in}{2.542804in}}{\pgfqpoint{3.762903in}{2.538414in}}{\pgfqpoint{3.773953in}{2.538414in}}%
\pgfpathclose%
\pgfusepath{stroke,fill}%
\end{pgfscope}%
\begin{pgfscope}%
\pgfpathrectangle{\pgfqpoint{0.570343in}{0.331635in}}{\pgfqpoint{9.300000in}{7.700000in}}%
\pgfusepath{clip}%
\pgfsetbuttcap%
\pgfsetroundjoin%
\definecolor{currentfill}{rgb}{0.631373,0.788235,0.956863}%
\pgfsetfillcolor{currentfill}%
\pgfsetlinewidth{0.481800pt}%
\definecolor{currentstroke}{rgb}{1.000000,1.000000,1.000000}%
\pgfsetstrokecolor{currentstroke}%
\pgfsetdash{}{0pt}%
\pgfpathmoveto{\pgfqpoint{7.797264in}{4.727564in}}%
\pgfpathcurveto{\pgfqpoint{7.808314in}{4.727564in}}{\pgfqpoint{7.818913in}{4.731954in}}{\pgfqpoint{7.826727in}{4.739768in}}%
\pgfpathcurveto{\pgfqpoint{7.834541in}{4.747581in}}{\pgfqpoint{7.838931in}{4.758180in}}{\pgfqpoint{7.838931in}{4.769230in}}%
\pgfpathcurveto{\pgfqpoint{7.838931in}{4.780280in}}{\pgfqpoint{7.834541in}{4.790880in}}{\pgfqpoint{7.826727in}{4.798693in}}%
\pgfpathcurveto{\pgfqpoint{7.818913in}{4.806507in}}{\pgfqpoint{7.808314in}{4.810897in}}{\pgfqpoint{7.797264in}{4.810897in}}%
\pgfpathcurveto{\pgfqpoint{7.786214in}{4.810897in}}{\pgfqpoint{7.775615in}{4.806507in}}{\pgfqpoint{7.767801in}{4.798693in}}%
\pgfpathcurveto{\pgfqpoint{7.759988in}{4.790880in}}{\pgfqpoint{7.755597in}{4.780280in}}{\pgfqpoint{7.755597in}{4.769230in}}%
\pgfpathcurveto{\pgfqpoint{7.755597in}{4.758180in}}{\pgfqpoint{7.759988in}{4.747581in}}{\pgfqpoint{7.767801in}{4.739768in}}%
\pgfpathcurveto{\pgfqpoint{7.775615in}{4.731954in}}{\pgfqpoint{7.786214in}{4.727564in}}{\pgfqpoint{7.797264in}{4.727564in}}%
\pgfpathclose%
\pgfusepath{stroke,fill}%
\end{pgfscope}%
\begin{pgfscope}%
\pgfpathrectangle{\pgfqpoint{0.570343in}{0.331635in}}{\pgfqpoint{9.300000in}{7.700000in}}%
\pgfusepath{clip}%
\pgfsetbuttcap%
\pgfsetroundjoin%
\definecolor{currentfill}{rgb}{0.631373,0.788235,0.956863}%
\pgfsetfillcolor{currentfill}%
\pgfsetlinewidth{0.481800pt}%
\definecolor{currentstroke}{rgb}{1.000000,1.000000,1.000000}%
\pgfsetstrokecolor{currentstroke}%
\pgfsetdash{}{0pt}%
\pgfpathmoveto{\pgfqpoint{2.799738in}{3.089874in}}%
\pgfpathcurveto{\pgfqpoint{2.810788in}{3.089874in}}{\pgfqpoint{2.821387in}{3.094264in}}{\pgfqpoint{2.829201in}{3.102078in}}%
\pgfpathcurveto{\pgfqpoint{2.837014in}{3.109892in}}{\pgfqpoint{2.841405in}{3.120491in}}{\pgfqpoint{2.841405in}{3.131541in}}%
\pgfpathcurveto{\pgfqpoint{2.841405in}{3.142591in}}{\pgfqpoint{2.837014in}{3.153190in}}{\pgfqpoint{2.829201in}{3.161004in}}%
\pgfpathcurveto{\pgfqpoint{2.821387in}{3.168817in}}{\pgfqpoint{2.810788in}{3.173207in}}{\pgfqpoint{2.799738in}{3.173207in}}%
\pgfpathcurveto{\pgfqpoint{2.788688in}{3.173207in}}{\pgfqpoint{2.778089in}{3.168817in}}{\pgfqpoint{2.770275in}{3.161004in}}%
\pgfpathcurveto{\pgfqpoint{2.762462in}{3.153190in}}{\pgfqpoint{2.758071in}{3.142591in}}{\pgfqpoint{2.758071in}{3.131541in}}%
\pgfpathcurveto{\pgfqpoint{2.758071in}{3.120491in}}{\pgfqpoint{2.762462in}{3.109892in}}{\pgfqpoint{2.770275in}{3.102078in}}%
\pgfpathcurveto{\pgfqpoint{2.778089in}{3.094264in}}{\pgfqpoint{2.788688in}{3.089874in}}{\pgfqpoint{2.799738in}{3.089874in}}%
\pgfpathclose%
\pgfusepath{stroke,fill}%
\end{pgfscope}%
\begin{pgfscope}%
\pgfpathrectangle{\pgfqpoint{0.570343in}{0.331635in}}{\pgfqpoint{9.300000in}{7.700000in}}%
\pgfusepath{clip}%
\pgfsetbuttcap%
\pgfsetroundjoin%
\definecolor{currentfill}{rgb}{0.631373,0.788235,0.956863}%
\pgfsetfillcolor{currentfill}%
\pgfsetlinewidth{0.481800pt}%
\definecolor{currentstroke}{rgb}{1.000000,1.000000,1.000000}%
\pgfsetstrokecolor{currentstroke}%
\pgfsetdash{}{0pt}%
\pgfpathmoveto{\pgfqpoint{6.185154in}{2.850809in}}%
\pgfpathcurveto{\pgfqpoint{6.196204in}{2.850809in}}{\pgfqpoint{6.206803in}{2.855199in}}{\pgfqpoint{6.214617in}{2.863013in}}%
\pgfpathcurveto{\pgfqpoint{6.222431in}{2.870827in}}{\pgfqpoint{6.226821in}{2.881426in}}{\pgfqpoint{6.226821in}{2.892476in}}%
\pgfpathcurveto{\pgfqpoint{6.226821in}{2.903526in}}{\pgfqpoint{6.222431in}{2.914125in}}{\pgfqpoint{6.214617in}{2.921939in}}%
\pgfpathcurveto{\pgfqpoint{6.206803in}{2.929752in}}{\pgfqpoint{6.196204in}{2.934143in}}{\pgfqpoint{6.185154in}{2.934143in}}%
\pgfpathcurveto{\pgfqpoint{6.174104in}{2.934143in}}{\pgfqpoint{6.163505in}{2.929752in}}{\pgfqpoint{6.155691in}{2.921939in}}%
\pgfpathcurveto{\pgfqpoint{6.147878in}{2.914125in}}{\pgfqpoint{6.143487in}{2.903526in}}{\pgfqpoint{6.143487in}{2.892476in}}%
\pgfpathcurveto{\pgfqpoint{6.143487in}{2.881426in}}{\pgfqpoint{6.147878in}{2.870827in}}{\pgfqpoint{6.155691in}{2.863013in}}%
\pgfpathcurveto{\pgfqpoint{6.163505in}{2.855199in}}{\pgfqpoint{6.174104in}{2.850809in}}{\pgfqpoint{6.185154in}{2.850809in}}%
\pgfpathclose%
\pgfusepath{stroke,fill}%
\end{pgfscope}%
\begin{pgfscope}%
\pgfpathrectangle{\pgfqpoint{0.570343in}{0.331635in}}{\pgfqpoint{9.300000in}{7.700000in}}%
\pgfusepath{clip}%
\pgfsetbuttcap%
\pgfsetroundjoin%
\definecolor{currentfill}{rgb}{0.631373,0.788235,0.956863}%
\pgfsetfillcolor{currentfill}%
\pgfsetlinewidth{0.481800pt}%
\definecolor{currentstroke}{rgb}{1.000000,1.000000,1.000000}%
\pgfsetstrokecolor{currentstroke}%
\pgfsetdash{}{0pt}%
\pgfpathmoveto{\pgfqpoint{2.095236in}{3.470631in}}%
\pgfpathcurveto{\pgfqpoint{2.106287in}{3.470631in}}{\pgfqpoint{2.116886in}{3.475022in}}{\pgfqpoint{2.124699in}{3.482835in}}%
\pgfpathcurveto{\pgfqpoint{2.132513in}{3.490649in}}{\pgfqpoint{2.136903in}{3.501248in}}{\pgfqpoint{2.136903in}{3.512298in}}%
\pgfpathcurveto{\pgfqpoint{2.136903in}{3.523348in}}{\pgfqpoint{2.132513in}{3.533947in}}{\pgfqpoint{2.124699in}{3.541761in}}%
\pgfpathcurveto{\pgfqpoint{2.116886in}{3.549574in}}{\pgfqpoint{2.106287in}{3.553965in}}{\pgfqpoint{2.095236in}{3.553965in}}%
\pgfpathcurveto{\pgfqpoint{2.084186in}{3.553965in}}{\pgfqpoint{2.073587in}{3.549574in}}{\pgfqpoint{2.065774in}{3.541761in}}%
\pgfpathcurveto{\pgfqpoint{2.057960in}{3.533947in}}{\pgfqpoint{2.053570in}{3.523348in}}{\pgfqpoint{2.053570in}{3.512298in}}%
\pgfpathcurveto{\pgfqpoint{2.053570in}{3.501248in}}{\pgfqpoint{2.057960in}{3.490649in}}{\pgfqpoint{2.065774in}{3.482835in}}%
\pgfpathcurveto{\pgfqpoint{2.073587in}{3.475022in}}{\pgfqpoint{2.084186in}{3.470631in}}{\pgfqpoint{2.095236in}{3.470631in}}%
\pgfpathclose%
\pgfusepath{stroke,fill}%
\end{pgfscope}%
\begin{pgfscope}%
\pgfpathrectangle{\pgfqpoint{0.570343in}{0.331635in}}{\pgfqpoint{9.300000in}{7.700000in}}%
\pgfusepath{clip}%
\pgfsetbuttcap%
\pgfsetroundjoin%
\definecolor{currentfill}{rgb}{0.631373,0.788235,0.956863}%
\pgfsetfillcolor{currentfill}%
\pgfsetlinewidth{0.481800pt}%
\definecolor{currentstroke}{rgb}{1.000000,1.000000,1.000000}%
\pgfsetstrokecolor{currentstroke}%
\pgfsetdash{}{0pt}%
\pgfpathmoveto{\pgfqpoint{4.706525in}{5.260494in}}%
\pgfpathcurveto{\pgfqpoint{4.717575in}{5.260494in}}{\pgfqpoint{4.728174in}{5.264885in}}{\pgfqpoint{4.735988in}{5.272698in}}%
\pgfpathcurveto{\pgfqpoint{4.743802in}{5.280512in}}{\pgfqpoint{4.748192in}{5.291111in}}{\pgfqpoint{4.748192in}{5.302161in}}%
\pgfpathcurveto{\pgfqpoint{4.748192in}{5.313211in}}{\pgfqpoint{4.743802in}{5.323810in}}{\pgfqpoint{4.735988in}{5.331624in}}%
\pgfpathcurveto{\pgfqpoint{4.728174in}{5.339437in}}{\pgfqpoint{4.717575in}{5.343828in}}{\pgfqpoint{4.706525in}{5.343828in}}%
\pgfpathcurveto{\pgfqpoint{4.695475in}{5.343828in}}{\pgfqpoint{4.684876in}{5.339437in}}{\pgfqpoint{4.677062in}{5.331624in}}%
\pgfpathcurveto{\pgfqpoint{4.669249in}{5.323810in}}{\pgfqpoint{4.664858in}{5.313211in}}{\pgfqpoint{4.664858in}{5.302161in}}%
\pgfpathcurveto{\pgfqpoint{4.664858in}{5.291111in}}{\pgfqpoint{4.669249in}{5.280512in}}{\pgfqpoint{4.677062in}{5.272698in}}%
\pgfpathcurveto{\pgfqpoint{4.684876in}{5.264885in}}{\pgfqpoint{4.695475in}{5.260494in}}{\pgfqpoint{4.706525in}{5.260494in}}%
\pgfpathclose%
\pgfusepath{stroke,fill}%
\end{pgfscope}%
\begin{pgfscope}%
\pgfpathrectangle{\pgfqpoint{0.570343in}{0.331635in}}{\pgfqpoint{9.300000in}{7.700000in}}%
\pgfusepath{clip}%
\pgfsetbuttcap%
\pgfsetroundjoin%
\definecolor{currentfill}{rgb}{0.631373,0.788235,0.956863}%
\pgfsetfillcolor{currentfill}%
\pgfsetlinewidth{0.481800pt}%
\definecolor{currentstroke}{rgb}{1.000000,1.000000,1.000000}%
\pgfsetstrokecolor{currentstroke}%
\pgfsetdash{}{0pt}%
\pgfpathmoveto{\pgfqpoint{3.943202in}{3.894706in}}%
\pgfpathcurveto{\pgfqpoint{3.954252in}{3.894706in}}{\pgfqpoint{3.964851in}{3.899097in}}{\pgfqpoint{3.972665in}{3.906910in}}%
\pgfpathcurveto{\pgfqpoint{3.980479in}{3.914724in}}{\pgfqpoint{3.984869in}{3.925323in}}{\pgfqpoint{3.984869in}{3.936373in}}%
\pgfpathcurveto{\pgfqpoint{3.984869in}{3.947423in}}{\pgfqpoint{3.980479in}{3.958022in}}{\pgfqpoint{3.972665in}{3.965836in}}%
\pgfpathcurveto{\pgfqpoint{3.964851in}{3.973649in}}{\pgfqpoint{3.954252in}{3.978040in}}{\pgfqpoint{3.943202in}{3.978040in}}%
\pgfpathcurveto{\pgfqpoint{3.932152in}{3.978040in}}{\pgfqpoint{3.921553in}{3.973649in}}{\pgfqpoint{3.913740in}{3.965836in}}%
\pgfpathcurveto{\pgfqpoint{3.905926in}{3.958022in}}{\pgfqpoint{3.901536in}{3.947423in}}{\pgfqpoint{3.901536in}{3.936373in}}%
\pgfpathcurveto{\pgfqpoint{3.901536in}{3.925323in}}{\pgfqpoint{3.905926in}{3.914724in}}{\pgfqpoint{3.913740in}{3.906910in}}%
\pgfpathcurveto{\pgfqpoint{3.921553in}{3.899097in}}{\pgfqpoint{3.932152in}{3.894706in}}{\pgfqpoint{3.943202in}{3.894706in}}%
\pgfpathclose%
\pgfusepath{stroke,fill}%
\end{pgfscope}%
\begin{pgfscope}%
\pgfpathrectangle{\pgfqpoint{0.570343in}{0.331635in}}{\pgfqpoint{9.300000in}{7.700000in}}%
\pgfusepath{clip}%
\pgfsetbuttcap%
\pgfsetroundjoin%
\definecolor{currentfill}{rgb}{0.631373,0.788235,0.956863}%
\pgfsetfillcolor{currentfill}%
\pgfsetlinewidth{0.481800pt}%
\definecolor{currentstroke}{rgb}{1.000000,1.000000,1.000000}%
\pgfsetstrokecolor{currentstroke}%
\pgfsetdash{}{0pt}%
\pgfpathmoveto{\pgfqpoint{4.635731in}{3.961694in}}%
\pgfpathcurveto{\pgfqpoint{4.646781in}{3.961694in}}{\pgfqpoint{4.657380in}{3.966084in}}{\pgfqpoint{4.665193in}{3.973897in}}%
\pgfpathcurveto{\pgfqpoint{4.673007in}{3.981711in}}{\pgfqpoint{4.677397in}{3.992310in}}{\pgfqpoint{4.677397in}{4.003360in}}%
\pgfpathcurveto{\pgfqpoint{4.677397in}{4.014410in}}{\pgfqpoint{4.673007in}{4.025009in}}{\pgfqpoint{4.665193in}{4.032823in}}%
\pgfpathcurveto{\pgfqpoint{4.657380in}{4.040637in}}{\pgfqpoint{4.646781in}{4.045027in}}{\pgfqpoint{4.635731in}{4.045027in}}%
\pgfpathcurveto{\pgfqpoint{4.624680in}{4.045027in}}{\pgfqpoint{4.614081in}{4.040637in}}{\pgfqpoint{4.606268in}{4.032823in}}%
\pgfpathcurveto{\pgfqpoint{4.598454in}{4.025009in}}{\pgfqpoint{4.594064in}{4.014410in}}{\pgfqpoint{4.594064in}{4.003360in}}%
\pgfpathcurveto{\pgfqpoint{4.594064in}{3.992310in}}{\pgfqpoint{4.598454in}{3.981711in}}{\pgfqpoint{4.606268in}{3.973897in}}%
\pgfpathcurveto{\pgfqpoint{4.614081in}{3.966084in}}{\pgfqpoint{4.624680in}{3.961694in}}{\pgfqpoint{4.635731in}{3.961694in}}%
\pgfpathclose%
\pgfusepath{stroke,fill}%
\end{pgfscope}%
\begin{pgfscope}%
\pgfpathrectangle{\pgfqpoint{0.570343in}{0.331635in}}{\pgfqpoint{9.300000in}{7.700000in}}%
\pgfusepath{clip}%
\pgfsetbuttcap%
\pgfsetroundjoin%
\definecolor{currentfill}{rgb}{0.631373,0.788235,0.956863}%
\pgfsetfillcolor{currentfill}%
\pgfsetlinewidth{0.481800pt}%
\definecolor{currentstroke}{rgb}{1.000000,1.000000,1.000000}%
\pgfsetstrokecolor{currentstroke}%
\pgfsetdash{}{0pt}%
\pgfpathmoveto{\pgfqpoint{3.827407in}{3.262886in}}%
\pgfpathcurveto{\pgfqpoint{3.838458in}{3.262886in}}{\pgfqpoint{3.849057in}{3.267277in}}{\pgfqpoint{3.856870in}{3.275090in}}%
\pgfpathcurveto{\pgfqpoint{3.864684in}{3.282904in}}{\pgfqpoint{3.869074in}{3.293503in}}{\pgfqpoint{3.869074in}{3.304553in}}%
\pgfpathcurveto{\pgfqpoint{3.869074in}{3.315603in}}{\pgfqpoint{3.864684in}{3.326202in}}{\pgfqpoint{3.856870in}{3.334016in}}%
\pgfpathcurveto{\pgfqpoint{3.849057in}{3.341829in}}{\pgfqpoint{3.838458in}{3.346220in}}{\pgfqpoint{3.827407in}{3.346220in}}%
\pgfpathcurveto{\pgfqpoint{3.816357in}{3.346220in}}{\pgfqpoint{3.805758in}{3.341829in}}{\pgfqpoint{3.797945in}{3.334016in}}%
\pgfpathcurveto{\pgfqpoint{3.790131in}{3.326202in}}{\pgfqpoint{3.785741in}{3.315603in}}{\pgfqpoint{3.785741in}{3.304553in}}%
\pgfpathcurveto{\pgfqpoint{3.785741in}{3.293503in}}{\pgfqpoint{3.790131in}{3.282904in}}{\pgfqpoint{3.797945in}{3.275090in}}%
\pgfpathcurveto{\pgfqpoint{3.805758in}{3.267277in}}{\pgfqpoint{3.816357in}{3.262886in}}{\pgfqpoint{3.827407in}{3.262886in}}%
\pgfpathclose%
\pgfusepath{stroke,fill}%
\end{pgfscope}%
\begin{pgfscope}%
\pgfpathrectangle{\pgfqpoint{0.570343in}{0.331635in}}{\pgfqpoint{9.300000in}{7.700000in}}%
\pgfusepath{clip}%
\pgfsetbuttcap%
\pgfsetroundjoin%
\definecolor{currentfill}{rgb}{0.631373,0.788235,0.956863}%
\pgfsetfillcolor{currentfill}%
\pgfsetlinewidth{0.481800pt}%
\definecolor{currentstroke}{rgb}{1.000000,1.000000,1.000000}%
\pgfsetstrokecolor{currentstroke}%
\pgfsetdash{}{0pt}%
\pgfpathmoveto{\pgfqpoint{3.605912in}{4.449800in}}%
\pgfpathcurveto{\pgfqpoint{3.616962in}{4.449800in}}{\pgfqpoint{3.627562in}{4.454190in}}{\pgfqpoint{3.635375in}{4.462004in}}%
\pgfpathcurveto{\pgfqpoint{3.643189in}{4.469818in}}{\pgfqpoint{3.647579in}{4.480417in}}{\pgfqpoint{3.647579in}{4.491467in}}%
\pgfpathcurveto{\pgfqpoint{3.647579in}{4.502517in}}{\pgfqpoint{3.643189in}{4.513116in}}{\pgfqpoint{3.635375in}{4.520930in}}%
\pgfpathcurveto{\pgfqpoint{3.627562in}{4.528743in}}{\pgfqpoint{3.616962in}{4.533133in}}{\pgfqpoint{3.605912in}{4.533133in}}%
\pgfpathcurveto{\pgfqpoint{3.594862in}{4.533133in}}{\pgfqpoint{3.584263in}{4.528743in}}{\pgfqpoint{3.576450in}{4.520930in}}%
\pgfpathcurveto{\pgfqpoint{3.568636in}{4.513116in}}{\pgfqpoint{3.564246in}{4.502517in}}{\pgfqpoint{3.564246in}{4.491467in}}%
\pgfpathcurveto{\pgfqpoint{3.564246in}{4.480417in}}{\pgfqpoint{3.568636in}{4.469818in}}{\pgfqpoint{3.576450in}{4.462004in}}%
\pgfpathcurveto{\pgfqpoint{3.584263in}{4.454190in}}{\pgfqpoint{3.594862in}{4.449800in}}{\pgfqpoint{3.605912in}{4.449800in}}%
\pgfpathclose%
\pgfusepath{stroke,fill}%
\end{pgfscope}%
\begin{pgfscope}%
\pgfpathrectangle{\pgfqpoint{0.570343in}{0.331635in}}{\pgfqpoint{9.300000in}{7.700000in}}%
\pgfusepath{clip}%
\pgfsetbuttcap%
\pgfsetroundjoin%
\definecolor{currentfill}{rgb}{0.631373,0.788235,0.956863}%
\pgfsetfillcolor{currentfill}%
\pgfsetlinewidth{0.481800pt}%
\definecolor{currentstroke}{rgb}{1.000000,1.000000,1.000000}%
\pgfsetstrokecolor{currentstroke}%
\pgfsetdash{}{0pt}%
\pgfpathmoveto{\pgfqpoint{5.549912in}{4.145442in}}%
\pgfpathcurveto{\pgfqpoint{5.560962in}{4.145442in}}{\pgfqpoint{5.571561in}{4.149832in}}{\pgfqpoint{5.579374in}{4.157646in}}%
\pgfpathcurveto{\pgfqpoint{5.587188in}{4.165459in}}{\pgfqpoint{5.591578in}{4.176058in}}{\pgfqpoint{5.591578in}{4.187109in}}%
\pgfpathcurveto{\pgfqpoint{5.591578in}{4.198159in}}{\pgfqpoint{5.587188in}{4.208758in}}{\pgfqpoint{5.579374in}{4.216571in}}%
\pgfpathcurveto{\pgfqpoint{5.571561in}{4.224385in}}{\pgfqpoint{5.560962in}{4.228775in}}{\pgfqpoint{5.549912in}{4.228775in}}%
\pgfpathcurveto{\pgfqpoint{5.538862in}{4.228775in}}{\pgfqpoint{5.528262in}{4.224385in}}{\pgfqpoint{5.520449in}{4.216571in}}%
\pgfpathcurveto{\pgfqpoint{5.512635in}{4.208758in}}{\pgfqpoint{5.508245in}{4.198159in}}{\pgfqpoint{5.508245in}{4.187109in}}%
\pgfpathcurveto{\pgfqpoint{5.508245in}{4.176058in}}{\pgfqpoint{5.512635in}{4.165459in}}{\pgfqpoint{5.520449in}{4.157646in}}%
\pgfpathcurveto{\pgfqpoint{5.528262in}{4.149832in}}{\pgfqpoint{5.538862in}{4.145442in}}{\pgfqpoint{5.549912in}{4.145442in}}%
\pgfpathclose%
\pgfusepath{stroke,fill}%
\end{pgfscope}%
\begin{pgfscope}%
\pgfpathrectangle{\pgfqpoint{0.570343in}{0.331635in}}{\pgfqpoint{9.300000in}{7.700000in}}%
\pgfusepath{clip}%
\pgfsetbuttcap%
\pgfsetroundjoin%
\definecolor{currentfill}{rgb}{0.631373,0.788235,0.956863}%
\pgfsetfillcolor{currentfill}%
\pgfsetlinewidth{0.481800pt}%
\definecolor{currentstroke}{rgb}{1.000000,1.000000,1.000000}%
\pgfsetstrokecolor{currentstroke}%
\pgfsetdash{}{0pt}%
\pgfpathmoveto{\pgfqpoint{6.429778in}{5.520341in}}%
\pgfpathcurveto{\pgfqpoint{6.440828in}{5.520341in}}{\pgfqpoint{6.451427in}{5.524732in}}{\pgfqpoint{6.459241in}{5.532545in}}%
\pgfpathcurveto{\pgfqpoint{6.467054in}{5.540359in}}{\pgfqpoint{6.471445in}{5.550958in}}{\pgfqpoint{6.471445in}{5.562008in}}%
\pgfpathcurveto{\pgfqpoint{6.471445in}{5.573058in}}{\pgfqpoint{6.467054in}{5.583657in}}{\pgfqpoint{6.459241in}{5.591471in}}%
\pgfpathcurveto{\pgfqpoint{6.451427in}{5.599284in}}{\pgfqpoint{6.440828in}{5.603675in}}{\pgfqpoint{6.429778in}{5.603675in}}%
\pgfpathcurveto{\pgfqpoint{6.418728in}{5.603675in}}{\pgfqpoint{6.408129in}{5.599284in}}{\pgfqpoint{6.400315in}{5.591471in}}%
\pgfpathcurveto{\pgfqpoint{6.392502in}{5.583657in}}{\pgfqpoint{6.388111in}{5.573058in}}{\pgfqpoint{6.388111in}{5.562008in}}%
\pgfpathcurveto{\pgfqpoint{6.388111in}{5.550958in}}{\pgfqpoint{6.392502in}{5.540359in}}{\pgfqpoint{6.400315in}{5.532545in}}%
\pgfpathcurveto{\pgfqpoint{6.408129in}{5.524732in}}{\pgfqpoint{6.418728in}{5.520341in}}{\pgfqpoint{6.429778in}{5.520341in}}%
\pgfpathclose%
\pgfusepath{stroke,fill}%
\end{pgfscope}%
\begin{pgfscope}%
\pgfpathrectangle{\pgfqpoint{0.570343in}{0.331635in}}{\pgfqpoint{9.300000in}{7.700000in}}%
\pgfusepath{clip}%
\pgfsetbuttcap%
\pgfsetroundjoin%
\definecolor{currentfill}{rgb}{0.631373,0.788235,0.956863}%
\pgfsetfillcolor{currentfill}%
\pgfsetlinewidth{0.481800pt}%
\definecolor{currentstroke}{rgb}{1.000000,1.000000,1.000000}%
\pgfsetstrokecolor{currentstroke}%
\pgfsetdash{}{0pt}%
\pgfpathmoveto{\pgfqpoint{4.510732in}{2.688622in}}%
\pgfpathcurveto{\pgfqpoint{4.521782in}{2.688622in}}{\pgfqpoint{4.532381in}{2.693012in}}{\pgfqpoint{4.540195in}{2.700826in}}%
\pgfpathcurveto{\pgfqpoint{4.548008in}{2.708640in}}{\pgfqpoint{4.552399in}{2.719239in}}{\pgfqpoint{4.552399in}{2.730289in}}%
\pgfpathcurveto{\pgfqpoint{4.552399in}{2.741339in}}{\pgfqpoint{4.548008in}{2.751938in}}{\pgfqpoint{4.540195in}{2.759752in}}%
\pgfpathcurveto{\pgfqpoint{4.532381in}{2.767565in}}{\pgfqpoint{4.521782in}{2.771956in}}{\pgfqpoint{4.510732in}{2.771956in}}%
\pgfpathcurveto{\pgfqpoint{4.499682in}{2.771956in}}{\pgfqpoint{4.489083in}{2.767565in}}{\pgfqpoint{4.481269in}{2.759752in}}%
\pgfpathcurveto{\pgfqpoint{4.473456in}{2.751938in}}{\pgfqpoint{4.469065in}{2.741339in}}{\pgfqpoint{4.469065in}{2.730289in}}%
\pgfpathcurveto{\pgfqpoint{4.469065in}{2.719239in}}{\pgfqpoint{4.473456in}{2.708640in}}{\pgfqpoint{4.481269in}{2.700826in}}%
\pgfpathcurveto{\pgfqpoint{4.489083in}{2.693012in}}{\pgfqpoint{4.499682in}{2.688622in}}{\pgfqpoint{4.510732in}{2.688622in}}%
\pgfpathclose%
\pgfusepath{stroke,fill}%
\end{pgfscope}%
\begin{pgfscope}%
\pgfpathrectangle{\pgfqpoint{0.570343in}{0.331635in}}{\pgfqpoint{9.300000in}{7.700000in}}%
\pgfusepath{clip}%
\pgfsetbuttcap%
\pgfsetroundjoin%
\definecolor{currentfill}{rgb}{0.631373,0.788235,0.956863}%
\pgfsetfillcolor{currentfill}%
\pgfsetlinewidth{0.481800pt}%
\definecolor{currentstroke}{rgb}{1.000000,1.000000,1.000000}%
\pgfsetstrokecolor{currentstroke}%
\pgfsetdash{}{0pt}%
\pgfpathmoveto{\pgfqpoint{5.386210in}{3.505671in}}%
\pgfpathcurveto{\pgfqpoint{5.397260in}{3.505671in}}{\pgfqpoint{5.407859in}{3.510061in}}{\pgfqpoint{5.415673in}{3.517875in}}%
\pgfpathcurveto{\pgfqpoint{5.423486in}{3.525688in}}{\pgfqpoint{5.427877in}{3.536287in}}{\pgfqpoint{5.427877in}{3.547338in}}%
\pgfpathcurveto{\pgfqpoint{5.427877in}{3.558388in}}{\pgfqpoint{5.423486in}{3.568987in}}{\pgfqpoint{5.415673in}{3.576800in}}%
\pgfpathcurveto{\pgfqpoint{5.407859in}{3.584614in}}{\pgfqpoint{5.397260in}{3.589004in}}{\pgfqpoint{5.386210in}{3.589004in}}%
\pgfpathcurveto{\pgfqpoint{5.375160in}{3.589004in}}{\pgfqpoint{5.364561in}{3.584614in}}{\pgfqpoint{5.356747in}{3.576800in}}%
\pgfpathcurveto{\pgfqpoint{5.348934in}{3.568987in}}{\pgfqpoint{5.344543in}{3.558388in}}{\pgfqpoint{5.344543in}{3.547338in}}%
\pgfpathcurveto{\pgfqpoint{5.344543in}{3.536287in}}{\pgfqpoint{5.348934in}{3.525688in}}{\pgfqpoint{5.356747in}{3.517875in}}%
\pgfpathcurveto{\pgfqpoint{5.364561in}{3.510061in}}{\pgfqpoint{5.375160in}{3.505671in}}{\pgfqpoint{5.386210in}{3.505671in}}%
\pgfpathclose%
\pgfusepath{stroke,fill}%
\end{pgfscope}%
\begin{pgfscope}%
\pgfpathrectangle{\pgfqpoint{0.570343in}{0.331635in}}{\pgfqpoint{9.300000in}{7.700000in}}%
\pgfusepath{clip}%
\pgfsetbuttcap%
\pgfsetroundjoin%
\definecolor{currentfill}{rgb}{0.631373,0.788235,0.956863}%
\pgfsetfillcolor{currentfill}%
\pgfsetlinewidth{0.481800pt}%
\definecolor{currentstroke}{rgb}{1.000000,1.000000,1.000000}%
\pgfsetstrokecolor{currentstroke}%
\pgfsetdash{}{0pt}%
\pgfpathmoveto{\pgfqpoint{8.348041in}{5.455977in}}%
\pgfpathcurveto{\pgfqpoint{8.359091in}{5.455977in}}{\pgfqpoint{8.369690in}{5.460367in}}{\pgfqpoint{8.377503in}{5.468181in}}%
\pgfpathcurveto{\pgfqpoint{8.385317in}{5.475994in}}{\pgfqpoint{8.389707in}{5.486593in}}{\pgfqpoint{8.389707in}{5.497643in}}%
\pgfpathcurveto{\pgfqpoint{8.389707in}{5.508694in}}{\pgfqpoint{8.385317in}{5.519293in}}{\pgfqpoint{8.377503in}{5.527106in}}%
\pgfpathcurveto{\pgfqpoint{8.369690in}{5.534920in}}{\pgfqpoint{8.359091in}{5.539310in}}{\pgfqpoint{8.348041in}{5.539310in}}%
\pgfpathcurveto{\pgfqpoint{8.336991in}{5.539310in}}{\pgfqpoint{8.326391in}{5.534920in}}{\pgfqpoint{8.318578in}{5.527106in}}%
\pgfpathcurveto{\pgfqpoint{8.310764in}{5.519293in}}{\pgfqpoint{8.306374in}{5.508694in}}{\pgfqpoint{8.306374in}{5.497643in}}%
\pgfpathcurveto{\pgfqpoint{8.306374in}{5.486593in}}{\pgfqpoint{8.310764in}{5.475994in}}{\pgfqpoint{8.318578in}{5.468181in}}%
\pgfpathcurveto{\pgfqpoint{8.326391in}{5.460367in}}{\pgfqpoint{8.336991in}{5.455977in}}{\pgfqpoint{8.348041in}{5.455977in}}%
\pgfpathclose%
\pgfusepath{stroke,fill}%
\end{pgfscope}%
\begin{pgfscope}%
\pgfpathrectangle{\pgfqpoint{0.570343in}{0.331635in}}{\pgfqpoint{9.300000in}{7.700000in}}%
\pgfusepath{clip}%
\pgfsetbuttcap%
\pgfsetroundjoin%
\definecolor{currentfill}{rgb}{0.631373,0.788235,0.956863}%
\pgfsetfillcolor{currentfill}%
\pgfsetlinewidth{0.481800pt}%
\definecolor{currentstroke}{rgb}{1.000000,1.000000,1.000000}%
\pgfsetstrokecolor{currentstroke}%
\pgfsetdash{}{0pt}%
\pgfpathmoveto{\pgfqpoint{5.507412in}{5.364816in}}%
\pgfpathcurveto{\pgfqpoint{5.518463in}{5.364816in}}{\pgfqpoint{5.529062in}{5.369206in}}{\pgfqpoint{5.536875in}{5.377020in}}%
\pgfpathcurveto{\pgfqpoint{5.544689in}{5.384833in}}{\pgfqpoint{5.549079in}{5.395432in}}{\pgfqpoint{5.549079in}{5.406482in}}%
\pgfpathcurveto{\pgfqpoint{5.549079in}{5.417533in}}{\pgfqpoint{5.544689in}{5.428132in}}{\pgfqpoint{5.536875in}{5.435945in}}%
\pgfpathcurveto{\pgfqpoint{5.529062in}{5.443759in}}{\pgfqpoint{5.518463in}{5.448149in}}{\pgfqpoint{5.507412in}{5.448149in}}%
\pgfpathcurveto{\pgfqpoint{5.496362in}{5.448149in}}{\pgfqpoint{5.485763in}{5.443759in}}{\pgfqpoint{5.477950in}{5.435945in}}%
\pgfpathcurveto{\pgfqpoint{5.470136in}{5.428132in}}{\pgfqpoint{5.465746in}{5.417533in}}{\pgfqpoint{5.465746in}{5.406482in}}%
\pgfpathcurveto{\pgfqpoint{5.465746in}{5.395432in}}{\pgfqpoint{5.470136in}{5.384833in}}{\pgfqpoint{5.477950in}{5.377020in}}%
\pgfpathcurveto{\pgfqpoint{5.485763in}{5.369206in}}{\pgfqpoint{5.496362in}{5.364816in}}{\pgfqpoint{5.507412in}{5.364816in}}%
\pgfpathclose%
\pgfusepath{stroke,fill}%
\end{pgfscope}%
\begin{pgfscope}%
\pgfpathrectangle{\pgfqpoint{0.570343in}{0.331635in}}{\pgfqpoint{9.300000in}{7.700000in}}%
\pgfusepath{clip}%
\pgfsetbuttcap%
\pgfsetroundjoin%
\definecolor{currentfill}{rgb}{0.631373,0.788235,0.956863}%
\pgfsetfillcolor{currentfill}%
\pgfsetlinewidth{0.481800pt}%
\definecolor{currentstroke}{rgb}{1.000000,1.000000,1.000000}%
\pgfsetstrokecolor{currentstroke}%
\pgfsetdash{}{0pt}%
\pgfpathmoveto{\pgfqpoint{5.886021in}{6.235301in}}%
\pgfpathcurveto{\pgfqpoint{5.897071in}{6.235301in}}{\pgfqpoint{5.907670in}{6.239691in}}{\pgfqpoint{5.915484in}{6.247505in}}%
\pgfpathcurveto{\pgfqpoint{5.923297in}{6.255319in}}{\pgfqpoint{5.927688in}{6.265918in}}{\pgfqpoint{5.927688in}{6.276968in}}%
\pgfpathcurveto{\pgfqpoint{5.927688in}{6.288018in}}{\pgfqpoint{5.923297in}{6.298617in}}{\pgfqpoint{5.915484in}{6.306430in}}%
\pgfpathcurveto{\pgfqpoint{5.907670in}{6.314244in}}{\pgfqpoint{5.897071in}{6.318634in}}{\pgfqpoint{5.886021in}{6.318634in}}%
\pgfpathcurveto{\pgfqpoint{5.874971in}{6.318634in}}{\pgfqpoint{5.864372in}{6.314244in}}{\pgfqpoint{5.856558in}{6.306430in}}%
\pgfpathcurveto{\pgfqpoint{5.848744in}{6.298617in}}{\pgfqpoint{5.844354in}{6.288018in}}{\pgfqpoint{5.844354in}{6.276968in}}%
\pgfpathcurveto{\pgfqpoint{5.844354in}{6.265918in}}{\pgfqpoint{5.848744in}{6.255319in}}{\pgfqpoint{5.856558in}{6.247505in}}%
\pgfpathcurveto{\pgfqpoint{5.864372in}{6.239691in}}{\pgfqpoint{5.874971in}{6.235301in}}{\pgfqpoint{5.886021in}{6.235301in}}%
\pgfpathclose%
\pgfusepath{stroke,fill}%
\end{pgfscope}%
\begin{pgfscope}%
\pgfpathrectangle{\pgfqpoint{0.570343in}{0.331635in}}{\pgfqpoint{9.300000in}{7.700000in}}%
\pgfusepath{clip}%
\pgfsetbuttcap%
\pgfsetroundjoin%
\definecolor{currentfill}{rgb}{0.631373,0.788235,0.956863}%
\pgfsetfillcolor{currentfill}%
\pgfsetlinewidth{0.481800pt}%
\definecolor{currentstroke}{rgb}{1.000000,1.000000,1.000000}%
\pgfsetstrokecolor{currentstroke}%
\pgfsetdash{}{0pt}%
\pgfpathmoveto{\pgfqpoint{6.703980in}{2.093785in}}%
\pgfpathcurveto{\pgfqpoint{6.715031in}{2.093785in}}{\pgfqpoint{6.725630in}{2.098175in}}{\pgfqpoint{6.733443in}{2.105989in}}%
\pgfpathcurveto{\pgfqpoint{6.741257in}{2.113803in}}{\pgfqpoint{6.745647in}{2.124402in}}{\pgfqpoint{6.745647in}{2.135452in}}%
\pgfpathcurveto{\pgfqpoint{6.745647in}{2.146502in}}{\pgfqpoint{6.741257in}{2.157101in}}{\pgfqpoint{6.733443in}{2.164915in}}%
\pgfpathcurveto{\pgfqpoint{6.725630in}{2.172728in}}{\pgfqpoint{6.715031in}{2.177119in}}{\pgfqpoint{6.703980in}{2.177119in}}%
\pgfpathcurveto{\pgfqpoint{6.692930in}{2.177119in}}{\pgfqpoint{6.682331in}{2.172728in}}{\pgfqpoint{6.674518in}{2.164915in}}%
\pgfpathcurveto{\pgfqpoint{6.666704in}{2.157101in}}{\pgfqpoint{6.662314in}{2.146502in}}{\pgfqpoint{6.662314in}{2.135452in}}%
\pgfpathcurveto{\pgfqpoint{6.662314in}{2.124402in}}{\pgfqpoint{6.666704in}{2.113803in}}{\pgfqpoint{6.674518in}{2.105989in}}%
\pgfpathcurveto{\pgfqpoint{6.682331in}{2.098175in}}{\pgfqpoint{6.692930in}{2.093785in}}{\pgfqpoint{6.703980in}{2.093785in}}%
\pgfpathclose%
\pgfusepath{stroke,fill}%
\end{pgfscope}%
\begin{pgfscope}%
\pgfpathrectangle{\pgfqpoint{0.570343in}{0.331635in}}{\pgfqpoint{9.300000in}{7.700000in}}%
\pgfusepath{clip}%
\pgfsetbuttcap%
\pgfsetroundjoin%
\definecolor{currentfill}{rgb}{0.631373,0.788235,0.956863}%
\pgfsetfillcolor{currentfill}%
\pgfsetlinewidth{0.481800pt}%
\definecolor{currentstroke}{rgb}{1.000000,1.000000,1.000000}%
\pgfsetstrokecolor{currentstroke}%
\pgfsetdash{}{0pt}%
\pgfpathmoveto{\pgfqpoint{3.152829in}{3.769352in}}%
\pgfpathcurveto{\pgfqpoint{3.163879in}{3.769352in}}{\pgfqpoint{3.174478in}{3.773742in}}{\pgfqpoint{3.182292in}{3.781556in}}%
\pgfpathcurveto{\pgfqpoint{3.190105in}{3.789369in}}{\pgfqpoint{3.194496in}{3.799968in}}{\pgfqpoint{3.194496in}{3.811018in}}%
\pgfpathcurveto{\pgfqpoint{3.194496in}{3.822069in}}{\pgfqpoint{3.190105in}{3.832668in}}{\pgfqpoint{3.182292in}{3.840481in}}%
\pgfpathcurveto{\pgfqpoint{3.174478in}{3.848295in}}{\pgfqpoint{3.163879in}{3.852685in}}{\pgfqpoint{3.152829in}{3.852685in}}%
\pgfpathcurveto{\pgfqpoint{3.141779in}{3.852685in}}{\pgfqpoint{3.131180in}{3.848295in}}{\pgfqpoint{3.123366in}{3.840481in}}%
\pgfpathcurveto{\pgfqpoint{3.115552in}{3.832668in}}{\pgfqpoint{3.111162in}{3.822069in}}{\pgfqpoint{3.111162in}{3.811018in}}%
\pgfpathcurveto{\pgfqpoint{3.111162in}{3.799968in}}{\pgfqpoint{3.115552in}{3.789369in}}{\pgfqpoint{3.123366in}{3.781556in}}%
\pgfpathcurveto{\pgfqpoint{3.131180in}{3.773742in}}{\pgfqpoint{3.141779in}{3.769352in}}{\pgfqpoint{3.152829in}{3.769352in}}%
\pgfpathclose%
\pgfusepath{stroke,fill}%
\end{pgfscope}%
\begin{pgfscope}%
\pgfpathrectangle{\pgfqpoint{0.570343in}{0.331635in}}{\pgfqpoint{9.300000in}{7.700000in}}%
\pgfusepath{clip}%
\pgfsetbuttcap%
\pgfsetroundjoin%
\definecolor{currentfill}{rgb}{0.631373,0.788235,0.956863}%
\pgfsetfillcolor{currentfill}%
\pgfsetlinewidth{0.481800pt}%
\definecolor{currentstroke}{rgb}{1.000000,1.000000,1.000000}%
\pgfsetstrokecolor{currentstroke}%
\pgfsetdash{}{0pt}%
\pgfpathmoveto{\pgfqpoint{4.382416in}{4.593957in}}%
\pgfpathcurveto{\pgfqpoint{4.393466in}{4.593957in}}{\pgfqpoint{4.404065in}{4.598348in}}{\pgfqpoint{4.411878in}{4.606161in}}%
\pgfpathcurveto{\pgfqpoint{4.419692in}{4.613975in}}{\pgfqpoint{4.424082in}{4.624574in}}{\pgfqpoint{4.424082in}{4.635624in}}%
\pgfpathcurveto{\pgfqpoint{4.424082in}{4.646674in}}{\pgfqpoint{4.419692in}{4.657273in}}{\pgfqpoint{4.411878in}{4.665087in}}%
\pgfpathcurveto{\pgfqpoint{4.404065in}{4.672900in}}{\pgfqpoint{4.393466in}{4.677291in}}{\pgfqpoint{4.382416in}{4.677291in}}%
\pgfpathcurveto{\pgfqpoint{4.371365in}{4.677291in}}{\pgfqpoint{4.360766in}{4.672900in}}{\pgfqpoint{4.352953in}{4.665087in}}%
\pgfpathcurveto{\pgfqpoint{4.345139in}{4.657273in}}{\pgfqpoint{4.340749in}{4.646674in}}{\pgfqpoint{4.340749in}{4.635624in}}%
\pgfpathcurveto{\pgfqpoint{4.340749in}{4.624574in}}{\pgfqpoint{4.345139in}{4.613975in}}{\pgfqpoint{4.352953in}{4.606161in}}%
\pgfpathcurveto{\pgfqpoint{4.360766in}{4.598348in}}{\pgfqpoint{4.371365in}{4.593957in}}{\pgfqpoint{4.382416in}{4.593957in}}%
\pgfpathclose%
\pgfusepath{stroke,fill}%
\end{pgfscope}%
\begin{pgfscope}%
\pgfpathrectangle{\pgfqpoint{0.570343in}{0.331635in}}{\pgfqpoint{9.300000in}{7.700000in}}%
\pgfusepath{clip}%
\pgfsetbuttcap%
\pgfsetroundjoin%
\definecolor{currentfill}{rgb}{0.631373,0.788235,0.956863}%
\pgfsetfillcolor{currentfill}%
\pgfsetlinewidth{0.481800pt}%
\definecolor{currentstroke}{rgb}{1.000000,1.000000,1.000000}%
\pgfsetstrokecolor{currentstroke}%
\pgfsetdash{}{0pt}%
\pgfpathmoveto{\pgfqpoint{6.950477in}{6.262727in}}%
\pgfpathcurveto{\pgfqpoint{6.961527in}{6.262727in}}{\pgfqpoint{6.972126in}{6.267117in}}{\pgfqpoint{6.979940in}{6.274931in}}%
\pgfpathcurveto{\pgfqpoint{6.987753in}{6.282745in}}{\pgfqpoint{6.992144in}{6.293344in}}{\pgfqpoint{6.992144in}{6.304394in}}%
\pgfpathcurveto{\pgfqpoint{6.992144in}{6.315444in}}{\pgfqpoint{6.987753in}{6.326043in}}{\pgfqpoint{6.979940in}{6.333857in}}%
\pgfpathcurveto{\pgfqpoint{6.972126in}{6.341670in}}{\pgfqpoint{6.961527in}{6.346060in}}{\pgfqpoint{6.950477in}{6.346060in}}%
\pgfpathcurveto{\pgfqpoint{6.939427in}{6.346060in}}{\pgfqpoint{6.928828in}{6.341670in}}{\pgfqpoint{6.921014in}{6.333857in}}%
\pgfpathcurveto{\pgfqpoint{6.913201in}{6.326043in}}{\pgfqpoint{6.908810in}{6.315444in}}{\pgfqpoint{6.908810in}{6.304394in}}%
\pgfpathcurveto{\pgfqpoint{6.908810in}{6.293344in}}{\pgfqpoint{6.913201in}{6.282745in}}{\pgfqpoint{6.921014in}{6.274931in}}%
\pgfpathcurveto{\pgfqpoint{6.928828in}{6.267117in}}{\pgfqpoint{6.939427in}{6.262727in}}{\pgfqpoint{6.950477in}{6.262727in}}%
\pgfpathclose%
\pgfusepath{stroke,fill}%
\end{pgfscope}%
\begin{pgfscope}%
\pgfpathrectangle{\pgfqpoint{0.570343in}{0.331635in}}{\pgfqpoint{9.300000in}{7.700000in}}%
\pgfusepath{clip}%
\pgfsetbuttcap%
\pgfsetroundjoin%
\definecolor{currentfill}{rgb}{0.631373,0.788235,0.956863}%
\pgfsetfillcolor{currentfill}%
\pgfsetlinewidth{0.481800pt}%
\definecolor{currentstroke}{rgb}{1.000000,1.000000,1.000000}%
\pgfsetstrokecolor{currentstroke}%
\pgfsetdash{}{0pt}%
\pgfpathmoveto{\pgfqpoint{2.893408in}{4.607151in}}%
\pgfpathcurveto{\pgfqpoint{2.904458in}{4.607151in}}{\pgfqpoint{2.915057in}{4.611542in}}{\pgfqpoint{2.922871in}{4.619355in}}%
\pgfpathcurveto{\pgfqpoint{2.930685in}{4.627169in}}{\pgfqpoint{2.935075in}{4.637768in}}{\pgfqpoint{2.935075in}{4.648818in}}%
\pgfpathcurveto{\pgfqpoint{2.935075in}{4.659868in}}{\pgfqpoint{2.930685in}{4.670467in}}{\pgfqpoint{2.922871in}{4.678281in}}%
\pgfpathcurveto{\pgfqpoint{2.915057in}{4.686094in}}{\pgfqpoint{2.904458in}{4.690485in}}{\pgfqpoint{2.893408in}{4.690485in}}%
\pgfpathcurveto{\pgfqpoint{2.882358in}{4.690485in}}{\pgfqpoint{2.871759in}{4.686094in}}{\pgfqpoint{2.863945in}{4.678281in}}%
\pgfpathcurveto{\pgfqpoint{2.856132in}{4.670467in}}{\pgfqpoint{2.851741in}{4.659868in}}{\pgfqpoint{2.851741in}{4.648818in}}%
\pgfpathcurveto{\pgfqpoint{2.851741in}{4.637768in}}{\pgfqpoint{2.856132in}{4.627169in}}{\pgfqpoint{2.863945in}{4.619355in}}%
\pgfpathcurveto{\pgfqpoint{2.871759in}{4.611542in}}{\pgfqpoint{2.882358in}{4.607151in}}{\pgfqpoint{2.893408in}{4.607151in}}%
\pgfpathclose%
\pgfusepath{stroke,fill}%
\end{pgfscope}%
\begin{pgfscope}%
\pgfpathrectangle{\pgfqpoint{0.570343in}{0.331635in}}{\pgfqpoint{9.300000in}{7.700000in}}%
\pgfusepath{clip}%
\pgfsetbuttcap%
\pgfsetroundjoin%
\definecolor{currentfill}{rgb}{1.000000,0.705882,0.509804}%
\pgfsetfillcolor{currentfill}%
\pgfsetlinewidth{0.481800pt}%
\definecolor{currentstroke}{rgb}{1.000000,1.000000,1.000000}%
\pgfsetstrokecolor{currentstroke}%
\pgfsetdash{}{0pt}%
\pgfpathmoveto{\pgfqpoint{3.817222in}{5.982446in}}%
\pgfpathcurveto{\pgfqpoint{3.828272in}{5.982446in}}{\pgfqpoint{3.838871in}{5.986836in}}{\pgfqpoint{3.846685in}{5.994650in}}%
\pgfpathcurveto{\pgfqpoint{3.854498in}{6.002464in}}{\pgfqpoint{3.858889in}{6.013063in}}{\pgfqpoint{3.858889in}{6.024113in}}%
\pgfpathcurveto{\pgfqpoint{3.858889in}{6.035163in}}{\pgfqpoint{3.854498in}{6.045762in}}{\pgfqpoint{3.846685in}{6.053576in}}%
\pgfpathcurveto{\pgfqpoint{3.838871in}{6.061389in}}{\pgfqpoint{3.828272in}{6.065779in}}{\pgfqpoint{3.817222in}{6.065779in}}%
\pgfpathcurveto{\pgfqpoint{3.806172in}{6.065779in}}{\pgfqpoint{3.795573in}{6.061389in}}{\pgfqpoint{3.787759in}{6.053576in}}%
\pgfpathcurveto{\pgfqpoint{3.779946in}{6.045762in}}{\pgfqpoint{3.775555in}{6.035163in}}{\pgfqpoint{3.775555in}{6.024113in}}%
\pgfpathcurveto{\pgfqpoint{3.775555in}{6.013063in}}{\pgfqpoint{3.779946in}{6.002464in}}{\pgfqpoint{3.787759in}{5.994650in}}%
\pgfpathcurveto{\pgfqpoint{3.795573in}{5.986836in}}{\pgfqpoint{3.806172in}{5.982446in}}{\pgfqpoint{3.817222in}{5.982446in}}%
\pgfpathclose%
\pgfusepath{stroke,fill}%
\end{pgfscope}%
\begin{pgfscope}%
\pgfpathrectangle{\pgfqpoint{0.570343in}{0.331635in}}{\pgfqpoint{9.300000in}{7.700000in}}%
\pgfusepath{clip}%
\pgfsetbuttcap%
\pgfsetroundjoin%
\definecolor{currentfill}{rgb}{1.000000,0.705882,0.509804}%
\pgfsetfillcolor{currentfill}%
\pgfsetlinewidth{0.481800pt}%
\definecolor{currentstroke}{rgb}{1.000000,1.000000,1.000000}%
\pgfsetstrokecolor{currentstroke}%
\pgfsetdash{}{0pt}%
\pgfpathmoveto{\pgfqpoint{7.659561in}{1.344530in}}%
\pgfpathcurveto{\pgfqpoint{7.670611in}{1.344530in}}{\pgfqpoint{7.681210in}{1.348920in}}{\pgfqpoint{7.689024in}{1.356734in}}%
\pgfpathcurveto{\pgfqpoint{7.696838in}{1.364548in}}{\pgfqpoint{7.701228in}{1.375147in}}{\pgfqpoint{7.701228in}{1.386197in}}%
\pgfpathcurveto{\pgfqpoint{7.701228in}{1.397247in}}{\pgfqpoint{7.696838in}{1.407846in}}{\pgfqpoint{7.689024in}{1.415660in}}%
\pgfpathcurveto{\pgfqpoint{7.681210in}{1.423473in}}{\pgfqpoint{7.670611in}{1.427864in}}{\pgfqpoint{7.659561in}{1.427864in}}%
\pgfpathcurveto{\pgfqpoint{7.648511in}{1.427864in}}{\pgfqpoint{7.637912in}{1.423473in}}{\pgfqpoint{7.630098in}{1.415660in}}%
\pgfpathcurveto{\pgfqpoint{7.622285in}{1.407846in}}{\pgfqpoint{7.617895in}{1.397247in}}{\pgfqpoint{7.617895in}{1.386197in}}%
\pgfpathcurveto{\pgfqpoint{7.617895in}{1.375147in}}{\pgfqpoint{7.622285in}{1.364548in}}{\pgfqpoint{7.630098in}{1.356734in}}%
\pgfpathcurveto{\pgfqpoint{7.637912in}{1.348920in}}{\pgfqpoint{7.648511in}{1.344530in}}{\pgfqpoint{7.659561in}{1.344530in}}%
\pgfpathclose%
\pgfusepath{stroke,fill}%
\end{pgfscope}%
\begin{pgfscope}%
\pgfpathrectangle{\pgfqpoint{0.570343in}{0.331635in}}{\pgfqpoint{9.300000in}{7.700000in}}%
\pgfusepath{clip}%
\pgfsetbuttcap%
\pgfsetroundjoin%
\definecolor{currentfill}{rgb}{1.000000,0.705882,0.509804}%
\pgfsetfillcolor{currentfill}%
\pgfsetlinewidth{0.481800pt}%
\definecolor{currentstroke}{rgb}{1.000000,1.000000,1.000000}%
\pgfsetstrokecolor{currentstroke}%
\pgfsetdash{}{0pt}%
\pgfpathmoveto{\pgfqpoint{6.328447in}{4.263692in}}%
\pgfpathcurveto{\pgfqpoint{6.339497in}{4.263692in}}{\pgfqpoint{6.350096in}{4.268082in}}{\pgfqpoint{6.357910in}{4.275896in}}%
\pgfpathcurveto{\pgfqpoint{6.365724in}{4.283709in}}{\pgfqpoint{6.370114in}{4.294308in}}{\pgfqpoint{6.370114in}{4.305358in}}%
\pgfpathcurveto{\pgfqpoint{6.370114in}{4.316408in}}{\pgfqpoint{6.365724in}{4.327007in}}{\pgfqpoint{6.357910in}{4.334821in}}%
\pgfpathcurveto{\pgfqpoint{6.350096in}{4.342635in}}{\pgfqpoint{6.339497in}{4.347025in}}{\pgfqpoint{6.328447in}{4.347025in}}%
\pgfpathcurveto{\pgfqpoint{6.317397in}{4.347025in}}{\pgfqpoint{6.306798in}{4.342635in}}{\pgfqpoint{6.298984in}{4.334821in}}%
\pgfpathcurveto{\pgfqpoint{6.291171in}{4.327007in}}{\pgfqpoint{6.286780in}{4.316408in}}{\pgfqpoint{6.286780in}{4.305358in}}%
\pgfpathcurveto{\pgfqpoint{6.286780in}{4.294308in}}{\pgfqpoint{6.291171in}{4.283709in}}{\pgfqpoint{6.298984in}{4.275896in}}%
\pgfpathcurveto{\pgfqpoint{6.306798in}{4.268082in}}{\pgfqpoint{6.317397in}{4.263692in}}{\pgfqpoint{6.328447in}{4.263692in}}%
\pgfpathclose%
\pgfusepath{stroke,fill}%
\end{pgfscope}%
\begin{pgfscope}%
\pgfpathrectangle{\pgfqpoint{0.570343in}{0.331635in}}{\pgfqpoint{9.300000in}{7.700000in}}%
\pgfusepath{clip}%
\pgfsetbuttcap%
\pgfsetroundjoin%
\definecolor{currentfill}{rgb}{1.000000,0.705882,0.509804}%
\pgfsetfillcolor{currentfill}%
\pgfsetlinewidth{0.481800pt}%
\definecolor{currentstroke}{rgb}{1.000000,1.000000,1.000000}%
\pgfsetstrokecolor{currentstroke}%
\pgfsetdash{}{0pt}%
\pgfpathmoveto{\pgfqpoint{8.131383in}{3.712424in}}%
\pgfpathcurveto{\pgfqpoint{8.142433in}{3.712424in}}{\pgfqpoint{8.153032in}{3.716814in}}{\pgfqpoint{8.160846in}{3.724627in}}%
\pgfpathcurveto{\pgfqpoint{8.168659in}{3.732441in}}{\pgfqpoint{8.173049in}{3.743040in}}{\pgfqpoint{8.173049in}{3.754090in}}%
\pgfpathcurveto{\pgfqpoint{8.173049in}{3.765140in}}{\pgfqpoint{8.168659in}{3.775739in}}{\pgfqpoint{8.160846in}{3.783553in}}%
\pgfpathcurveto{\pgfqpoint{8.153032in}{3.791367in}}{\pgfqpoint{8.142433in}{3.795757in}}{\pgfqpoint{8.131383in}{3.795757in}}%
\pgfpathcurveto{\pgfqpoint{8.120333in}{3.795757in}}{\pgfqpoint{8.109734in}{3.791367in}}{\pgfqpoint{8.101920in}{3.783553in}}%
\pgfpathcurveto{\pgfqpoint{8.094106in}{3.775739in}}{\pgfqpoint{8.089716in}{3.765140in}}{\pgfqpoint{8.089716in}{3.754090in}}%
\pgfpathcurveto{\pgfqpoint{8.089716in}{3.743040in}}{\pgfqpoint{8.094106in}{3.732441in}}{\pgfqpoint{8.101920in}{3.724627in}}%
\pgfpathcurveto{\pgfqpoint{8.109734in}{3.716814in}}{\pgfqpoint{8.120333in}{3.712424in}}{\pgfqpoint{8.131383in}{3.712424in}}%
\pgfpathclose%
\pgfusepath{stroke,fill}%
\end{pgfscope}%
\begin{pgfscope}%
\pgfpathrectangle{\pgfqpoint{0.570343in}{0.331635in}}{\pgfqpoint{9.300000in}{7.700000in}}%
\pgfusepath{clip}%
\pgfsetbuttcap%
\pgfsetroundjoin%
\definecolor{currentfill}{rgb}{1.000000,0.705882,0.509804}%
\pgfsetfillcolor{currentfill}%
\pgfsetlinewidth{0.481800pt}%
\definecolor{currentstroke}{rgb}{1.000000,1.000000,1.000000}%
\pgfsetstrokecolor{currentstroke}%
\pgfsetdash{}{0pt}%
\pgfpathmoveto{\pgfqpoint{6.702968in}{4.863493in}}%
\pgfpathcurveto{\pgfqpoint{6.714018in}{4.863493in}}{\pgfqpoint{6.724617in}{4.867883in}}{\pgfqpoint{6.732431in}{4.875697in}}%
\pgfpathcurveto{\pgfqpoint{6.740245in}{4.883510in}}{\pgfqpoint{6.744635in}{4.894109in}}{\pgfqpoint{6.744635in}{4.905160in}}%
\pgfpathcurveto{\pgfqpoint{6.744635in}{4.916210in}}{\pgfqpoint{6.740245in}{4.926809in}}{\pgfqpoint{6.732431in}{4.934622in}}%
\pgfpathcurveto{\pgfqpoint{6.724617in}{4.942436in}}{\pgfqpoint{6.714018in}{4.946826in}}{\pgfqpoint{6.702968in}{4.946826in}}%
\pgfpathcurveto{\pgfqpoint{6.691918in}{4.946826in}}{\pgfqpoint{6.681319in}{4.942436in}}{\pgfqpoint{6.673505in}{4.934622in}}%
\pgfpathcurveto{\pgfqpoint{6.665692in}{4.926809in}}{\pgfqpoint{6.661302in}{4.916210in}}{\pgfqpoint{6.661302in}{4.905160in}}%
\pgfpathcurveto{\pgfqpoint{6.661302in}{4.894109in}}{\pgfqpoint{6.665692in}{4.883510in}}{\pgfqpoint{6.673505in}{4.875697in}}%
\pgfpathcurveto{\pgfqpoint{6.681319in}{4.867883in}}{\pgfqpoint{6.691918in}{4.863493in}}{\pgfqpoint{6.702968in}{4.863493in}}%
\pgfpathclose%
\pgfusepath{stroke,fill}%
\end{pgfscope}%
\begin{pgfscope}%
\pgfpathrectangle{\pgfqpoint{0.570343in}{0.331635in}}{\pgfqpoint{9.300000in}{7.700000in}}%
\pgfusepath{clip}%
\pgfsetbuttcap%
\pgfsetroundjoin%
\definecolor{currentfill}{rgb}{1.000000,0.705882,0.509804}%
\pgfsetfillcolor{currentfill}%
\pgfsetlinewidth{0.481800pt}%
\definecolor{currentstroke}{rgb}{1.000000,1.000000,1.000000}%
\pgfsetstrokecolor{currentstroke}%
\pgfsetdash{}{0pt}%
\pgfpathmoveto{\pgfqpoint{5.355699in}{2.026578in}}%
\pgfpathcurveto{\pgfqpoint{5.366750in}{2.026578in}}{\pgfqpoint{5.377349in}{2.030968in}}{\pgfqpoint{5.385162in}{2.038782in}}%
\pgfpathcurveto{\pgfqpoint{5.392976in}{2.046595in}}{\pgfqpoint{5.397366in}{2.057194in}}{\pgfqpoint{5.397366in}{2.068245in}}%
\pgfpathcurveto{\pgfqpoint{5.397366in}{2.079295in}}{\pgfqpoint{5.392976in}{2.089894in}}{\pgfqpoint{5.385162in}{2.097707in}}%
\pgfpathcurveto{\pgfqpoint{5.377349in}{2.105521in}}{\pgfqpoint{5.366750in}{2.109911in}}{\pgfqpoint{5.355699in}{2.109911in}}%
\pgfpathcurveto{\pgfqpoint{5.344649in}{2.109911in}}{\pgfqpoint{5.334050in}{2.105521in}}{\pgfqpoint{5.326237in}{2.097707in}}%
\pgfpathcurveto{\pgfqpoint{5.318423in}{2.089894in}}{\pgfqpoint{5.314033in}{2.079295in}}{\pgfqpoint{5.314033in}{2.068245in}}%
\pgfpathcurveto{\pgfqpoint{5.314033in}{2.057194in}}{\pgfqpoint{5.318423in}{2.046595in}}{\pgfqpoint{5.326237in}{2.038782in}}%
\pgfpathcurveto{\pgfqpoint{5.334050in}{2.030968in}}{\pgfqpoint{5.344649in}{2.026578in}}{\pgfqpoint{5.355699in}{2.026578in}}%
\pgfpathclose%
\pgfusepath{stroke,fill}%
\end{pgfscope}%
\begin{pgfscope}%
\pgfpathrectangle{\pgfqpoint{0.570343in}{0.331635in}}{\pgfqpoint{9.300000in}{7.700000in}}%
\pgfusepath{clip}%
\pgfsetbuttcap%
\pgfsetroundjoin%
\definecolor{currentfill}{rgb}{1.000000,0.705882,0.509804}%
\pgfsetfillcolor{currentfill}%
\pgfsetlinewidth{0.481800pt}%
\definecolor{currentstroke}{rgb}{1.000000,1.000000,1.000000}%
\pgfsetstrokecolor{currentstroke}%
\pgfsetdash{}{0pt}%
\pgfpathmoveto{\pgfqpoint{2.219532in}{1.446127in}}%
\pgfpathcurveto{\pgfqpoint{2.230582in}{1.446127in}}{\pgfqpoint{2.241181in}{1.450517in}}{\pgfqpoint{2.248995in}{1.458330in}}%
\pgfpathcurveto{\pgfqpoint{2.256809in}{1.466144in}}{\pgfqpoint{2.261199in}{1.476743in}}{\pgfqpoint{2.261199in}{1.487793in}}%
\pgfpathcurveto{\pgfqpoint{2.261199in}{1.498843in}}{\pgfqpoint{2.256809in}{1.509442in}}{\pgfqpoint{2.248995in}{1.517256in}}%
\pgfpathcurveto{\pgfqpoint{2.241181in}{1.525070in}}{\pgfqpoint{2.230582in}{1.529460in}}{\pgfqpoint{2.219532in}{1.529460in}}%
\pgfpathcurveto{\pgfqpoint{2.208482in}{1.529460in}}{\pgfqpoint{2.197883in}{1.525070in}}{\pgfqpoint{2.190069in}{1.517256in}}%
\pgfpathcurveto{\pgfqpoint{2.182256in}{1.509442in}}{\pgfqpoint{2.177865in}{1.498843in}}{\pgfqpoint{2.177865in}{1.487793in}}%
\pgfpathcurveto{\pgfqpoint{2.177865in}{1.476743in}}{\pgfqpoint{2.182256in}{1.466144in}}{\pgfqpoint{2.190069in}{1.458330in}}%
\pgfpathcurveto{\pgfqpoint{2.197883in}{1.450517in}}{\pgfqpoint{2.208482in}{1.446127in}}{\pgfqpoint{2.219532in}{1.446127in}}%
\pgfpathclose%
\pgfusepath{stroke,fill}%
\end{pgfscope}%
\begin{pgfscope}%
\pgfpathrectangle{\pgfqpoint{0.570343in}{0.331635in}}{\pgfqpoint{9.300000in}{7.700000in}}%
\pgfusepath{clip}%
\pgfsetbuttcap%
\pgfsetroundjoin%
\definecolor{currentfill}{rgb}{1.000000,0.705882,0.509804}%
\pgfsetfillcolor{currentfill}%
\pgfsetlinewidth{0.481800pt}%
\definecolor{currentstroke}{rgb}{1.000000,1.000000,1.000000}%
\pgfsetstrokecolor{currentstroke}%
\pgfsetdash{}{0pt}%
\pgfpathmoveto{\pgfqpoint{4.514146in}{1.406180in}}%
\pgfpathcurveto{\pgfqpoint{4.525196in}{1.406180in}}{\pgfqpoint{4.535795in}{1.410570in}}{\pgfqpoint{4.543609in}{1.418384in}}%
\pgfpathcurveto{\pgfqpoint{4.551422in}{1.426198in}}{\pgfqpoint{4.555813in}{1.436797in}}{\pgfqpoint{4.555813in}{1.447847in}}%
\pgfpathcurveto{\pgfqpoint{4.555813in}{1.458897in}}{\pgfqpoint{4.551422in}{1.469496in}}{\pgfqpoint{4.543609in}{1.477310in}}%
\pgfpathcurveto{\pgfqpoint{4.535795in}{1.485123in}}{\pgfqpoint{4.525196in}{1.489514in}}{\pgfqpoint{4.514146in}{1.489514in}}%
\pgfpathcurveto{\pgfqpoint{4.503096in}{1.489514in}}{\pgfqpoint{4.492497in}{1.485123in}}{\pgfqpoint{4.484683in}{1.477310in}}%
\pgfpathcurveto{\pgfqpoint{4.476870in}{1.469496in}}{\pgfqpoint{4.472479in}{1.458897in}}{\pgfqpoint{4.472479in}{1.447847in}}%
\pgfpathcurveto{\pgfqpoint{4.472479in}{1.436797in}}{\pgfqpoint{4.476870in}{1.426198in}}{\pgfqpoint{4.484683in}{1.418384in}}%
\pgfpathcurveto{\pgfqpoint{4.492497in}{1.410570in}}{\pgfqpoint{4.503096in}{1.406180in}}{\pgfqpoint{4.514146in}{1.406180in}}%
\pgfpathclose%
\pgfusepath{stroke,fill}%
\end{pgfscope}%
\begin{pgfscope}%
\pgfpathrectangle{\pgfqpoint{0.570343in}{0.331635in}}{\pgfqpoint{9.300000in}{7.700000in}}%
\pgfusepath{clip}%
\pgfsetbuttcap%
\pgfsetroundjoin%
\definecolor{currentfill}{rgb}{1.000000,0.705882,0.509804}%
\pgfsetfillcolor{currentfill}%
\pgfsetlinewidth{0.481800pt}%
\definecolor{currentstroke}{rgb}{1.000000,1.000000,1.000000}%
\pgfsetstrokecolor{currentstroke}%
\pgfsetdash{}{0pt}%
\pgfpathmoveto{\pgfqpoint{5.290381in}{2.832328in}}%
\pgfpathcurveto{\pgfqpoint{5.301431in}{2.832328in}}{\pgfqpoint{5.312030in}{2.836718in}}{\pgfqpoint{5.319843in}{2.844532in}}%
\pgfpathcurveto{\pgfqpoint{5.327657in}{2.852345in}}{\pgfqpoint{5.332047in}{2.862944in}}{\pgfqpoint{5.332047in}{2.873994in}}%
\pgfpathcurveto{\pgfqpoint{5.332047in}{2.885044in}}{\pgfqpoint{5.327657in}{2.895643in}}{\pgfqpoint{5.319843in}{2.903457in}}%
\pgfpathcurveto{\pgfqpoint{5.312030in}{2.911271in}}{\pgfqpoint{5.301431in}{2.915661in}}{\pgfqpoint{5.290381in}{2.915661in}}%
\pgfpathcurveto{\pgfqpoint{5.279331in}{2.915661in}}{\pgfqpoint{5.268731in}{2.911271in}}{\pgfqpoint{5.260918in}{2.903457in}}%
\pgfpathcurveto{\pgfqpoint{5.253104in}{2.895643in}}{\pgfqpoint{5.248714in}{2.885044in}}{\pgfqpoint{5.248714in}{2.873994in}}%
\pgfpathcurveto{\pgfqpoint{5.248714in}{2.862944in}}{\pgfqpoint{5.253104in}{2.852345in}}{\pgfqpoint{5.260918in}{2.844532in}}%
\pgfpathcurveto{\pgfqpoint{5.268731in}{2.836718in}}{\pgfqpoint{5.279331in}{2.832328in}}{\pgfqpoint{5.290381in}{2.832328in}}%
\pgfpathclose%
\pgfusepath{stroke,fill}%
\end{pgfscope}%
\begin{pgfscope}%
\pgfpathrectangle{\pgfqpoint{0.570343in}{0.331635in}}{\pgfqpoint{9.300000in}{7.700000in}}%
\pgfusepath{clip}%
\pgfsetbuttcap%
\pgfsetroundjoin%
\definecolor{currentfill}{rgb}{1.000000,0.705882,0.509804}%
\pgfsetfillcolor{currentfill}%
\pgfsetlinewidth{0.481800pt}%
\definecolor{currentstroke}{rgb}{1.000000,1.000000,1.000000}%
\pgfsetstrokecolor{currentstroke}%
\pgfsetdash{}{0pt}%
\pgfpathmoveto{\pgfqpoint{4.901182in}{6.182179in}}%
\pgfpathcurveto{\pgfqpoint{4.912232in}{6.182179in}}{\pgfqpoint{4.922831in}{6.186569in}}{\pgfqpoint{4.930645in}{6.194383in}}%
\pgfpathcurveto{\pgfqpoint{4.938459in}{6.202197in}}{\pgfqpoint{4.942849in}{6.212796in}}{\pgfqpoint{4.942849in}{6.223846in}}%
\pgfpathcurveto{\pgfqpoint{4.942849in}{6.234896in}}{\pgfqpoint{4.938459in}{6.245495in}}{\pgfqpoint{4.930645in}{6.253308in}}%
\pgfpathcurveto{\pgfqpoint{4.922831in}{6.261122in}}{\pgfqpoint{4.912232in}{6.265512in}}{\pgfqpoint{4.901182in}{6.265512in}}%
\pgfpathcurveto{\pgfqpoint{4.890132in}{6.265512in}}{\pgfqpoint{4.879533in}{6.261122in}}{\pgfqpoint{4.871719in}{6.253308in}}%
\pgfpathcurveto{\pgfqpoint{4.863906in}{6.245495in}}{\pgfqpoint{4.859516in}{6.234896in}}{\pgfqpoint{4.859516in}{6.223846in}}%
\pgfpathcurveto{\pgfqpoint{4.859516in}{6.212796in}}{\pgfqpoint{4.863906in}{6.202197in}}{\pgfqpoint{4.871719in}{6.194383in}}%
\pgfpathcurveto{\pgfqpoint{4.879533in}{6.186569in}}{\pgfqpoint{4.890132in}{6.182179in}}{\pgfqpoint{4.901182in}{6.182179in}}%
\pgfpathclose%
\pgfusepath{stroke,fill}%
\end{pgfscope}%
\begin{pgfscope}%
\pgfpathrectangle{\pgfqpoint{0.570343in}{0.331635in}}{\pgfqpoint{9.300000in}{7.700000in}}%
\pgfusepath{clip}%
\pgfsetbuttcap%
\pgfsetroundjoin%
\definecolor{currentfill}{rgb}{1.000000,0.705882,0.509804}%
\pgfsetfillcolor{currentfill}%
\pgfsetlinewidth{0.481800pt}%
\definecolor{currentstroke}{rgb}{1.000000,1.000000,1.000000}%
\pgfsetstrokecolor{currentstroke}%
\pgfsetdash{}{0pt}%
\pgfpathmoveto{\pgfqpoint{8.177594in}{2.886332in}}%
\pgfpathcurveto{\pgfqpoint{8.188644in}{2.886332in}}{\pgfqpoint{8.199243in}{2.890723in}}{\pgfqpoint{8.207057in}{2.898536in}}%
\pgfpathcurveto{\pgfqpoint{8.214870in}{2.906350in}}{\pgfqpoint{8.219260in}{2.916949in}}{\pgfqpoint{8.219260in}{2.927999in}}%
\pgfpathcurveto{\pgfqpoint{8.219260in}{2.939049in}}{\pgfqpoint{8.214870in}{2.949648in}}{\pgfqpoint{8.207057in}{2.957462in}}%
\pgfpathcurveto{\pgfqpoint{8.199243in}{2.965276in}}{\pgfqpoint{8.188644in}{2.969666in}}{\pgfqpoint{8.177594in}{2.969666in}}%
\pgfpathcurveto{\pgfqpoint{8.166544in}{2.969666in}}{\pgfqpoint{8.155945in}{2.965276in}}{\pgfqpoint{8.148131in}{2.957462in}}%
\pgfpathcurveto{\pgfqpoint{8.140317in}{2.949648in}}{\pgfqpoint{8.135927in}{2.939049in}}{\pgfqpoint{8.135927in}{2.927999in}}%
\pgfpathcurveto{\pgfqpoint{8.135927in}{2.916949in}}{\pgfqpoint{8.140317in}{2.906350in}}{\pgfqpoint{8.148131in}{2.898536in}}%
\pgfpathcurveto{\pgfqpoint{8.155945in}{2.890723in}}{\pgfqpoint{8.166544in}{2.886332in}}{\pgfqpoint{8.177594in}{2.886332in}}%
\pgfpathclose%
\pgfusepath{stroke,fill}%
\end{pgfscope}%
\begin{pgfscope}%
\pgfpathrectangle{\pgfqpoint{0.570343in}{0.331635in}}{\pgfqpoint{9.300000in}{7.700000in}}%
\pgfusepath{clip}%
\pgfsetbuttcap%
\pgfsetroundjoin%
\definecolor{currentfill}{rgb}{1.000000,0.705882,0.509804}%
\pgfsetfillcolor{currentfill}%
\pgfsetlinewidth{0.481800pt}%
\definecolor{currentstroke}{rgb}{1.000000,1.000000,1.000000}%
\pgfsetstrokecolor{currentstroke}%
\pgfsetdash{}{0pt}%
\pgfpathmoveto{\pgfqpoint{7.239046in}{7.639968in}}%
\pgfpathcurveto{\pgfqpoint{7.250096in}{7.639968in}}{\pgfqpoint{7.260695in}{7.644359in}}{\pgfqpoint{7.268508in}{7.652172in}}%
\pgfpathcurveto{\pgfqpoint{7.276322in}{7.659986in}}{\pgfqpoint{7.280712in}{7.670585in}}{\pgfqpoint{7.280712in}{7.681635in}}%
\pgfpathcurveto{\pgfqpoint{7.280712in}{7.692685in}}{\pgfqpoint{7.276322in}{7.703284in}}{\pgfqpoint{7.268508in}{7.711098in}}%
\pgfpathcurveto{\pgfqpoint{7.260695in}{7.718911in}}{\pgfqpoint{7.250096in}{7.723302in}}{\pgfqpoint{7.239046in}{7.723302in}}%
\pgfpathcurveto{\pgfqpoint{7.227996in}{7.723302in}}{\pgfqpoint{7.217397in}{7.718911in}}{\pgfqpoint{7.209583in}{7.711098in}}%
\pgfpathcurveto{\pgfqpoint{7.201769in}{7.703284in}}{\pgfqpoint{7.197379in}{7.692685in}}{\pgfqpoint{7.197379in}{7.681635in}}%
\pgfpathcurveto{\pgfqpoint{7.197379in}{7.670585in}}{\pgfqpoint{7.201769in}{7.659986in}}{\pgfqpoint{7.209583in}{7.652172in}}%
\pgfpathcurveto{\pgfqpoint{7.217397in}{7.644359in}}{\pgfqpoint{7.227996in}{7.639968in}}{\pgfqpoint{7.239046in}{7.639968in}}%
\pgfpathclose%
\pgfusepath{stroke,fill}%
\end{pgfscope}%
\begin{pgfscope}%
\pgfpathrectangle{\pgfqpoint{0.570343in}{0.331635in}}{\pgfqpoint{9.300000in}{7.700000in}}%
\pgfusepath{clip}%
\pgfsetbuttcap%
\pgfsetroundjoin%
\definecolor{currentfill}{rgb}{1.000000,0.705882,0.509804}%
\pgfsetfillcolor{currentfill}%
\pgfsetlinewidth{0.481800pt}%
\definecolor{currentstroke}{rgb}{1.000000,1.000000,1.000000}%
\pgfsetstrokecolor{currentstroke}%
\pgfsetdash{}{0pt}%
\pgfpathmoveto{\pgfqpoint{3.182112in}{0.639968in}}%
\pgfpathcurveto{\pgfqpoint{3.193162in}{0.639968in}}{\pgfqpoint{3.203761in}{0.644359in}}{\pgfqpoint{3.211575in}{0.652172in}}%
\pgfpathcurveto{\pgfqpoint{3.219389in}{0.659986in}}{\pgfqpoint{3.223779in}{0.670585in}}{\pgfqpoint{3.223779in}{0.681635in}}%
\pgfpathcurveto{\pgfqpoint{3.223779in}{0.692685in}}{\pgfqpoint{3.219389in}{0.703284in}}{\pgfqpoint{3.211575in}{0.711098in}}%
\pgfpathcurveto{\pgfqpoint{3.203761in}{0.718911in}}{\pgfqpoint{3.193162in}{0.723302in}}{\pgfqpoint{3.182112in}{0.723302in}}%
\pgfpathcurveto{\pgfqpoint{3.171062in}{0.723302in}}{\pgfqpoint{3.160463in}{0.718911in}}{\pgfqpoint{3.152649in}{0.711098in}}%
\pgfpathcurveto{\pgfqpoint{3.144836in}{0.703284in}}{\pgfqpoint{3.140446in}{0.692685in}}{\pgfqpoint{3.140446in}{0.681635in}}%
\pgfpathcurveto{\pgfqpoint{3.140446in}{0.670585in}}{\pgfqpoint{3.144836in}{0.659986in}}{\pgfqpoint{3.152649in}{0.652172in}}%
\pgfpathcurveto{\pgfqpoint{3.160463in}{0.644359in}}{\pgfqpoint{3.171062in}{0.639968in}}{\pgfqpoint{3.182112in}{0.639968in}}%
\pgfpathclose%
\pgfusepath{stroke,fill}%
\end{pgfscope}%
\begin{pgfscope}%
\pgfpathrectangle{\pgfqpoint{0.570343in}{0.331635in}}{\pgfqpoint{9.300000in}{7.700000in}}%
\pgfusepath{clip}%
\pgfsetbuttcap%
\pgfsetroundjoin%
\definecolor{currentfill}{rgb}{1.000000,0.705882,0.509804}%
\pgfsetfillcolor{currentfill}%
\pgfsetlinewidth{0.481800pt}%
\definecolor{currentstroke}{rgb}{1.000000,1.000000,1.000000}%
\pgfsetstrokecolor{currentstroke}%
\pgfsetdash{}{0pt}%
\pgfpathmoveto{\pgfqpoint{7.238874in}{5.429662in}}%
\pgfpathcurveto{\pgfqpoint{7.249925in}{5.429662in}}{\pgfqpoint{7.260524in}{5.434052in}}{\pgfqpoint{7.268337in}{5.441866in}}%
\pgfpathcurveto{\pgfqpoint{7.276151in}{5.449680in}}{\pgfqpoint{7.280541in}{5.460279in}}{\pgfqpoint{7.280541in}{5.471329in}}%
\pgfpathcurveto{\pgfqpoint{7.280541in}{5.482379in}}{\pgfqpoint{7.276151in}{5.492978in}}{\pgfqpoint{7.268337in}{5.500792in}}%
\pgfpathcurveto{\pgfqpoint{7.260524in}{5.508605in}}{\pgfqpoint{7.249925in}{5.512995in}}{\pgfqpoint{7.238874in}{5.512995in}}%
\pgfpathcurveto{\pgfqpoint{7.227824in}{5.512995in}}{\pgfqpoint{7.217225in}{5.508605in}}{\pgfqpoint{7.209412in}{5.500792in}}%
\pgfpathcurveto{\pgfqpoint{7.201598in}{5.492978in}}{\pgfqpoint{7.197208in}{5.482379in}}{\pgfqpoint{7.197208in}{5.471329in}}%
\pgfpathcurveto{\pgfqpoint{7.197208in}{5.460279in}}{\pgfqpoint{7.201598in}{5.449680in}}{\pgfqpoint{7.209412in}{5.441866in}}%
\pgfpathcurveto{\pgfqpoint{7.217225in}{5.434052in}}{\pgfqpoint{7.227824in}{5.429662in}}{\pgfqpoint{7.238874in}{5.429662in}}%
\pgfpathclose%
\pgfusepath{stroke,fill}%
\end{pgfscope}%
\begin{pgfscope}%
\pgfpathrectangle{\pgfqpoint{0.570343in}{0.331635in}}{\pgfqpoint{9.300000in}{7.700000in}}%
\pgfusepath{clip}%
\pgfsetbuttcap%
\pgfsetroundjoin%
\definecolor{currentfill}{rgb}{1.000000,0.705882,0.509804}%
\pgfsetfillcolor{currentfill}%
\pgfsetlinewidth{0.481800pt}%
\definecolor{currentstroke}{rgb}{1.000000,1.000000,1.000000}%
\pgfsetstrokecolor{currentstroke}%
\pgfsetdash{}{0pt}%
\pgfpathmoveto{\pgfqpoint{3.375695in}{1.877218in}}%
\pgfpathcurveto{\pgfqpoint{3.386745in}{1.877218in}}{\pgfqpoint{3.397344in}{1.881608in}}{\pgfqpoint{3.405158in}{1.889422in}}%
\pgfpathcurveto{\pgfqpoint{3.412971in}{1.897236in}}{\pgfqpoint{3.417362in}{1.907835in}}{\pgfqpoint{3.417362in}{1.918885in}}%
\pgfpathcurveto{\pgfqpoint{3.417362in}{1.929935in}}{\pgfqpoint{3.412971in}{1.940534in}}{\pgfqpoint{3.405158in}{1.948348in}}%
\pgfpathcurveto{\pgfqpoint{3.397344in}{1.956161in}}{\pgfqpoint{3.386745in}{1.960551in}}{\pgfqpoint{3.375695in}{1.960551in}}%
\pgfpathcurveto{\pgfqpoint{3.364645in}{1.960551in}}{\pgfqpoint{3.354046in}{1.956161in}}{\pgfqpoint{3.346232in}{1.948348in}}%
\pgfpathcurveto{\pgfqpoint{3.338419in}{1.940534in}}{\pgfqpoint{3.334028in}{1.929935in}}{\pgfqpoint{3.334028in}{1.918885in}}%
\pgfpathcurveto{\pgfqpoint{3.334028in}{1.907835in}}{\pgfqpoint{3.338419in}{1.897236in}}{\pgfqpoint{3.346232in}{1.889422in}}%
\pgfpathcurveto{\pgfqpoint{3.354046in}{1.881608in}}{\pgfqpoint{3.364645in}{1.877218in}}{\pgfqpoint{3.375695in}{1.877218in}}%
\pgfpathclose%
\pgfusepath{stroke,fill}%
\end{pgfscope}%
\begin{pgfscope}%
\pgfpathrectangle{\pgfqpoint{0.570343in}{0.331635in}}{\pgfqpoint{9.300000in}{7.700000in}}%
\pgfusepath{clip}%
\pgfsetbuttcap%
\pgfsetroundjoin%
\definecolor{currentfill}{rgb}{1.000000,0.705882,0.509804}%
\pgfsetfillcolor{currentfill}%
\pgfsetlinewidth{0.481800pt}%
\definecolor{currentstroke}{rgb}{1.000000,1.000000,1.000000}%
\pgfsetstrokecolor{currentstroke}%
\pgfsetdash{}{0pt}%
\pgfpathmoveto{\pgfqpoint{5.763337in}{1.269409in}}%
\pgfpathcurveto{\pgfqpoint{5.774387in}{1.269409in}}{\pgfqpoint{5.784986in}{1.273799in}}{\pgfqpoint{5.792800in}{1.281613in}}%
\pgfpathcurveto{\pgfqpoint{5.800613in}{1.289427in}}{\pgfqpoint{5.805004in}{1.300026in}}{\pgfqpoint{5.805004in}{1.311076in}}%
\pgfpathcurveto{\pgfqpoint{5.805004in}{1.322126in}}{\pgfqpoint{5.800613in}{1.332725in}}{\pgfqpoint{5.792800in}{1.340539in}}%
\pgfpathcurveto{\pgfqpoint{5.784986in}{1.348352in}}{\pgfqpoint{5.774387in}{1.352742in}}{\pgfqpoint{5.763337in}{1.352742in}}%
\pgfpathcurveto{\pgfqpoint{5.752287in}{1.352742in}}{\pgfqpoint{5.741688in}{1.348352in}}{\pgfqpoint{5.733874in}{1.340539in}}%
\pgfpathcurveto{\pgfqpoint{5.726061in}{1.332725in}}{\pgfqpoint{5.721670in}{1.322126in}}{\pgfqpoint{5.721670in}{1.311076in}}%
\pgfpathcurveto{\pgfqpoint{5.721670in}{1.300026in}}{\pgfqpoint{5.726061in}{1.289427in}}{\pgfqpoint{5.733874in}{1.281613in}}%
\pgfpathcurveto{\pgfqpoint{5.741688in}{1.273799in}}{\pgfqpoint{5.752287in}{1.269409in}}{\pgfqpoint{5.763337in}{1.269409in}}%
\pgfpathclose%
\pgfusepath{stroke,fill}%
\end{pgfscope}%
\begin{pgfscope}%
\pgfpathrectangle{\pgfqpoint{0.570343in}{0.331635in}}{\pgfqpoint{9.300000in}{7.700000in}}%
\pgfusepath{clip}%
\pgfsetbuttcap%
\pgfsetroundjoin%
\definecolor{currentfill}{rgb}{1.000000,0.705882,0.509804}%
\pgfsetfillcolor{currentfill}%
\pgfsetlinewidth{0.481800pt}%
\definecolor{currentstroke}{rgb}{1.000000,1.000000,1.000000}%
\pgfsetstrokecolor{currentstroke}%
\pgfsetdash{}{0pt}%
\pgfpathmoveto{\pgfqpoint{2.235429in}{2.293475in}}%
\pgfpathcurveto{\pgfqpoint{2.246479in}{2.293475in}}{\pgfqpoint{2.257078in}{2.297865in}}{\pgfqpoint{2.264891in}{2.305678in}}%
\pgfpathcurveto{\pgfqpoint{2.272705in}{2.313492in}}{\pgfqpoint{2.277095in}{2.324091in}}{\pgfqpoint{2.277095in}{2.335141in}}%
\pgfpathcurveto{\pgfqpoint{2.277095in}{2.346191in}}{\pgfqpoint{2.272705in}{2.356790in}}{\pgfqpoint{2.264891in}{2.364604in}}%
\pgfpathcurveto{\pgfqpoint{2.257078in}{2.372418in}}{\pgfqpoint{2.246479in}{2.376808in}}{\pgfqpoint{2.235429in}{2.376808in}}%
\pgfpathcurveto{\pgfqpoint{2.224378in}{2.376808in}}{\pgfqpoint{2.213779in}{2.372418in}}{\pgfqpoint{2.205966in}{2.364604in}}%
\pgfpathcurveto{\pgfqpoint{2.198152in}{2.356790in}}{\pgfqpoint{2.193762in}{2.346191in}}{\pgfqpoint{2.193762in}{2.335141in}}%
\pgfpathcurveto{\pgfqpoint{2.193762in}{2.324091in}}{\pgfqpoint{2.198152in}{2.313492in}}{\pgfqpoint{2.205966in}{2.305678in}}%
\pgfpathcurveto{\pgfqpoint{2.213779in}{2.297865in}}{\pgfqpoint{2.224378in}{2.293475in}}{\pgfqpoint{2.235429in}{2.293475in}}%
\pgfpathclose%
\pgfusepath{stroke,fill}%
\end{pgfscope}%
\begin{pgfscope}%
\pgfpathrectangle{\pgfqpoint{0.570343in}{0.331635in}}{\pgfqpoint{9.300000in}{7.700000in}}%
\pgfusepath{clip}%
\pgfsetbuttcap%
\pgfsetroundjoin%
\definecolor{currentfill}{rgb}{1.000000,0.705882,0.509804}%
\pgfsetfillcolor{currentfill}%
\pgfsetlinewidth{0.481800pt}%
\definecolor{currentstroke}{rgb}{1.000000,1.000000,1.000000}%
\pgfsetstrokecolor{currentstroke}%
\pgfsetdash{}{0pt}%
\pgfpathmoveto{\pgfqpoint{8.228735in}{6.525549in}}%
\pgfpathcurveto{\pgfqpoint{8.239785in}{6.525549in}}{\pgfqpoint{8.250384in}{6.529939in}}{\pgfqpoint{8.258198in}{6.537753in}}%
\pgfpathcurveto{\pgfqpoint{8.266011in}{6.545566in}}{\pgfqpoint{8.270402in}{6.556165in}}{\pgfqpoint{8.270402in}{6.567215in}}%
\pgfpathcurveto{\pgfqpoint{8.270402in}{6.578265in}}{\pgfqpoint{8.266011in}{6.588864in}}{\pgfqpoint{8.258198in}{6.596678in}}%
\pgfpathcurveto{\pgfqpoint{8.250384in}{6.604492in}}{\pgfqpoint{8.239785in}{6.608882in}}{\pgfqpoint{8.228735in}{6.608882in}}%
\pgfpathcurveto{\pgfqpoint{8.217685in}{6.608882in}}{\pgfqpoint{8.207086in}{6.604492in}}{\pgfqpoint{8.199272in}{6.596678in}}%
\pgfpathcurveto{\pgfqpoint{8.191459in}{6.588864in}}{\pgfqpoint{8.187068in}{6.578265in}}{\pgfqpoint{8.187068in}{6.567215in}}%
\pgfpathcurveto{\pgfqpoint{8.187068in}{6.556165in}}{\pgfqpoint{8.191459in}{6.545566in}}{\pgfqpoint{8.199272in}{6.537753in}}%
\pgfpathcurveto{\pgfqpoint{8.207086in}{6.529939in}}{\pgfqpoint{8.217685in}{6.525549in}}{\pgfqpoint{8.228735in}{6.525549in}}%
\pgfpathclose%
\pgfusepath{stroke,fill}%
\end{pgfscope}%
\begin{pgfscope}%
\pgfpathrectangle{\pgfqpoint{0.570343in}{0.331635in}}{\pgfqpoint{9.300000in}{7.700000in}}%
\pgfusepath{clip}%
\pgfsetbuttcap%
\pgfsetroundjoin%
\definecolor{currentfill}{rgb}{1.000000,0.705882,0.509804}%
\pgfsetfillcolor{currentfill}%
\pgfsetlinewidth{0.481800pt}%
\definecolor{currentstroke}{rgb}{1.000000,1.000000,1.000000}%
\pgfsetstrokecolor{currentstroke}%
\pgfsetdash{}{0pt}%
\pgfpathmoveto{\pgfqpoint{4.520708in}{7.381404in}}%
\pgfpathcurveto{\pgfqpoint{4.531758in}{7.381404in}}{\pgfqpoint{4.542358in}{7.385794in}}{\pgfqpoint{4.550171in}{7.393608in}}%
\pgfpathcurveto{\pgfqpoint{4.557985in}{7.401421in}}{\pgfqpoint{4.562375in}{7.412020in}}{\pgfqpoint{4.562375in}{7.423071in}}%
\pgfpathcurveto{\pgfqpoint{4.562375in}{7.434121in}}{\pgfqpoint{4.557985in}{7.444720in}}{\pgfqpoint{4.550171in}{7.452533in}}%
\pgfpathcurveto{\pgfqpoint{4.542358in}{7.460347in}}{\pgfqpoint{4.531758in}{7.464737in}}{\pgfqpoint{4.520708in}{7.464737in}}%
\pgfpathcurveto{\pgfqpoint{4.509658in}{7.464737in}}{\pgfqpoint{4.499059in}{7.460347in}}{\pgfqpoint{4.491246in}{7.452533in}}%
\pgfpathcurveto{\pgfqpoint{4.483432in}{7.444720in}}{\pgfqpoint{4.479042in}{7.434121in}}{\pgfqpoint{4.479042in}{7.423071in}}%
\pgfpathcurveto{\pgfqpoint{4.479042in}{7.412020in}}{\pgfqpoint{4.483432in}{7.401421in}}{\pgfqpoint{4.491246in}{7.393608in}}%
\pgfpathcurveto{\pgfqpoint{4.499059in}{7.385794in}}{\pgfqpoint{4.509658in}{7.381404in}}{\pgfqpoint{4.520708in}{7.381404in}}%
\pgfpathclose%
\pgfusepath{stroke,fill}%
\end{pgfscope}%
\begin{pgfscope}%
\pgfpathrectangle{\pgfqpoint{0.570343in}{0.331635in}}{\pgfqpoint{9.300000in}{7.700000in}}%
\pgfusepath{clip}%
\pgfsetbuttcap%
\pgfsetroundjoin%
\definecolor{currentfill}{rgb}{1.000000,0.705882,0.509804}%
\pgfsetfillcolor{currentfill}%
\pgfsetlinewidth{0.481800pt}%
\definecolor{currentstroke}{rgb}{1.000000,1.000000,1.000000}%
\pgfsetstrokecolor{currentstroke}%
\pgfsetdash{}{0pt}%
\pgfpathmoveto{\pgfqpoint{5.816987in}{7.290765in}}%
\pgfpathcurveto{\pgfqpoint{5.828037in}{7.290765in}}{\pgfqpoint{5.838636in}{7.295155in}}{\pgfqpoint{5.846449in}{7.302969in}}%
\pgfpathcurveto{\pgfqpoint{5.854263in}{7.310783in}}{\pgfqpoint{5.858653in}{7.321382in}}{\pgfqpoint{5.858653in}{7.332432in}}%
\pgfpathcurveto{\pgfqpoint{5.858653in}{7.343482in}}{\pgfqpoint{5.854263in}{7.354081in}}{\pgfqpoint{5.846449in}{7.361895in}}%
\pgfpathcurveto{\pgfqpoint{5.838636in}{7.369708in}}{\pgfqpoint{5.828037in}{7.374099in}}{\pgfqpoint{5.816987in}{7.374099in}}%
\pgfpathcurveto{\pgfqpoint{5.805937in}{7.374099in}}{\pgfqpoint{5.795337in}{7.369708in}}{\pgfqpoint{5.787524in}{7.361895in}}%
\pgfpathcurveto{\pgfqpoint{5.779710in}{7.354081in}}{\pgfqpoint{5.775320in}{7.343482in}}{\pgfqpoint{5.775320in}{7.332432in}}%
\pgfpathcurveto{\pgfqpoint{5.775320in}{7.321382in}}{\pgfqpoint{5.779710in}{7.310783in}}{\pgfqpoint{5.787524in}{7.302969in}}%
\pgfpathcurveto{\pgfqpoint{5.795337in}{7.295155in}}{\pgfqpoint{5.805937in}{7.290765in}}{\pgfqpoint{5.816987in}{7.290765in}}%
\pgfpathclose%
\pgfusepath{stroke,fill}%
\end{pgfscope}%
\begin{pgfscope}%
\pgfpathrectangle{\pgfqpoint{0.570343in}{0.331635in}}{\pgfqpoint{9.300000in}{7.700000in}}%
\pgfusepath{clip}%
\pgfsetbuttcap%
\pgfsetroundjoin%
\definecolor{currentfill}{rgb}{1.000000,0.705882,0.509804}%
\pgfsetfillcolor{currentfill}%
\pgfsetlinewidth{0.481800pt}%
\definecolor{currentstroke}{rgb}{1.000000,1.000000,1.000000}%
\pgfsetstrokecolor{currentstroke}%
\pgfsetdash{}{0pt}%
\pgfpathmoveto{\pgfqpoint{1.860590in}{5.242580in}}%
\pgfpathcurveto{\pgfqpoint{1.871640in}{5.242580in}}{\pgfqpoint{1.882239in}{5.246970in}}{\pgfqpoint{1.890053in}{5.254784in}}%
\pgfpathcurveto{\pgfqpoint{1.897867in}{5.262598in}}{\pgfqpoint{1.902257in}{5.273197in}}{\pgfqpoint{1.902257in}{5.284247in}}%
\pgfpathcurveto{\pgfqpoint{1.902257in}{5.295297in}}{\pgfqpoint{1.897867in}{5.305896in}}{\pgfqpoint{1.890053in}{5.313710in}}%
\pgfpathcurveto{\pgfqpoint{1.882239in}{5.321523in}}{\pgfqpoint{1.871640in}{5.325913in}}{\pgfqpoint{1.860590in}{5.325913in}}%
\pgfpathcurveto{\pgfqpoint{1.849540in}{5.325913in}}{\pgfqpoint{1.838941in}{5.321523in}}{\pgfqpoint{1.831127in}{5.313710in}}%
\pgfpathcurveto{\pgfqpoint{1.823314in}{5.305896in}}{\pgfqpoint{1.818923in}{5.295297in}}{\pgfqpoint{1.818923in}{5.284247in}}%
\pgfpathcurveto{\pgfqpoint{1.818923in}{5.273197in}}{\pgfqpoint{1.823314in}{5.262598in}}{\pgfqpoint{1.831127in}{5.254784in}}%
\pgfpathcurveto{\pgfqpoint{1.838941in}{5.246970in}}{\pgfqpoint{1.849540in}{5.242580in}}{\pgfqpoint{1.860590in}{5.242580in}}%
\pgfpathclose%
\pgfusepath{stroke,fill}%
\end{pgfscope}%
\begin{pgfscope}%
\pgfpathrectangle{\pgfqpoint{0.570343in}{0.331635in}}{\pgfqpoint{9.300000in}{7.700000in}}%
\pgfusepath{clip}%
\pgfsetbuttcap%
\pgfsetroundjoin%
\definecolor{currentfill}{rgb}{1.000000,0.705882,0.509804}%
\pgfsetfillcolor{currentfill}%
\pgfsetlinewidth{0.481800pt}%
\definecolor{currentstroke}{rgb}{1.000000,1.000000,1.000000}%
\pgfsetstrokecolor{currentstroke}%
\pgfsetdash{}{0pt}%
\pgfpathmoveto{\pgfqpoint{7.102567in}{4.100170in}}%
\pgfpathcurveto{\pgfqpoint{7.113617in}{4.100170in}}{\pgfqpoint{7.124216in}{4.104560in}}{\pgfqpoint{7.132030in}{4.112374in}}%
\pgfpathcurveto{\pgfqpoint{7.139844in}{4.120188in}}{\pgfqpoint{7.144234in}{4.130787in}}{\pgfqpoint{7.144234in}{4.141837in}}%
\pgfpathcurveto{\pgfqpoint{7.144234in}{4.152887in}}{\pgfqpoint{7.139844in}{4.163486in}}{\pgfqpoint{7.132030in}{4.171300in}}%
\pgfpathcurveto{\pgfqpoint{7.124216in}{4.179113in}}{\pgfqpoint{7.113617in}{4.183503in}}{\pgfqpoint{7.102567in}{4.183503in}}%
\pgfpathcurveto{\pgfqpoint{7.091517in}{4.183503in}}{\pgfqpoint{7.080918in}{4.179113in}}{\pgfqpoint{7.073104in}{4.171300in}}%
\pgfpathcurveto{\pgfqpoint{7.065291in}{4.163486in}}{\pgfqpoint{7.060901in}{4.152887in}}{\pgfqpoint{7.060901in}{4.141837in}}%
\pgfpathcurveto{\pgfqpoint{7.060901in}{4.130787in}}{\pgfqpoint{7.065291in}{4.120188in}}{\pgfqpoint{7.073104in}{4.112374in}}%
\pgfpathcurveto{\pgfqpoint{7.080918in}{4.104560in}}{\pgfqpoint{7.091517in}{4.100170in}}{\pgfqpoint{7.102567in}{4.100170in}}%
\pgfpathclose%
\pgfusepath{stroke,fill}%
\end{pgfscope}%
\begin{pgfscope}%
\pgfpathrectangle{\pgfqpoint{0.570343in}{0.331635in}}{\pgfqpoint{9.300000in}{7.700000in}}%
\pgfusepath{clip}%
\pgfsetbuttcap%
\pgfsetroundjoin%
\definecolor{currentfill}{rgb}{1.000000,0.705882,0.509804}%
\pgfsetfillcolor{currentfill}%
\pgfsetlinewidth{0.481800pt}%
\definecolor{currentstroke}{rgb}{1.000000,1.000000,1.000000}%
\pgfsetstrokecolor{currentstroke}%
\pgfsetdash{}{0pt}%
\pgfpathmoveto{\pgfqpoint{1.082655in}{2.149923in}}%
\pgfpathcurveto{\pgfqpoint{1.093705in}{2.149923in}}{\pgfqpoint{1.104304in}{2.154313in}}{\pgfqpoint{1.112118in}{2.162126in}}%
\pgfpathcurveto{\pgfqpoint{1.119931in}{2.169940in}}{\pgfqpoint{1.124322in}{2.180539in}}{\pgfqpoint{1.124322in}{2.191589in}}%
\pgfpathcurveto{\pgfqpoint{1.124322in}{2.202639in}}{\pgfqpoint{1.119931in}{2.213238in}}{\pgfqpoint{1.112118in}{2.221052in}}%
\pgfpathcurveto{\pgfqpoint{1.104304in}{2.228866in}}{\pgfqpoint{1.093705in}{2.233256in}}{\pgfqpoint{1.082655in}{2.233256in}}%
\pgfpathcurveto{\pgfqpoint{1.071605in}{2.233256in}}{\pgfqpoint{1.061006in}{2.228866in}}{\pgfqpoint{1.053192in}{2.221052in}}%
\pgfpathcurveto{\pgfqpoint{1.045379in}{2.213238in}}{\pgfqpoint{1.040988in}{2.202639in}}{\pgfqpoint{1.040988in}{2.191589in}}%
\pgfpathcurveto{\pgfqpoint{1.040988in}{2.180539in}}{\pgfqpoint{1.045379in}{2.169940in}}{\pgfqpoint{1.053192in}{2.162126in}}%
\pgfpathcurveto{\pgfqpoint{1.061006in}{2.154313in}}{\pgfqpoint{1.071605in}{2.149923in}}{\pgfqpoint{1.082655in}{2.149923in}}%
\pgfpathclose%
\pgfusepath{stroke,fill}%
\end{pgfscope}%
\begin{pgfscope}%
\pgfpathrectangle{\pgfqpoint{0.570343in}{0.331635in}}{\pgfqpoint{9.300000in}{7.700000in}}%
\pgfusepath{clip}%
\pgfsetbuttcap%
\pgfsetroundjoin%
\definecolor{currentfill}{rgb}{1.000000,0.705882,0.509804}%
\pgfsetfillcolor{currentfill}%
\pgfsetlinewidth{0.481800pt}%
\definecolor{currentstroke}{rgb}{1.000000,1.000000,1.000000}%
\pgfsetstrokecolor{currentstroke}%
\pgfsetdash{}{0pt}%
\pgfpathmoveto{\pgfqpoint{9.447616in}{3.296455in}}%
\pgfpathcurveto{\pgfqpoint{9.458666in}{3.296455in}}{\pgfqpoint{9.469265in}{3.300845in}}{\pgfqpoint{9.477079in}{3.308658in}}%
\pgfpathcurveto{\pgfqpoint{9.484892in}{3.316472in}}{\pgfqpoint{9.489283in}{3.327071in}}{\pgfqpoint{9.489283in}{3.338121in}}%
\pgfpathcurveto{\pgfqpoint{9.489283in}{3.349171in}}{\pgfqpoint{9.484892in}{3.359770in}}{\pgfqpoint{9.477079in}{3.367584in}}%
\pgfpathcurveto{\pgfqpoint{9.469265in}{3.375398in}}{\pgfqpoint{9.458666in}{3.379788in}}{\pgfqpoint{9.447616in}{3.379788in}}%
\pgfpathcurveto{\pgfqpoint{9.436566in}{3.379788in}}{\pgfqpoint{9.425967in}{3.375398in}}{\pgfqpoint{9.418153in}{3.367584in}}%
\pgfpathcurveto{\pgfqpoint{9.410340in}{3.359770in}}{\pgfqpoint{9.405949in}{3.349171in}}{\pgfqpoint{9.405949in}{3.338121in}}%
\pgfpathcurveto{\pgfqpoint{9.405949in}{3.327071in}}{\pgfqpoint{9.410340in}{3.316472in}}{\pgfqpoint{9.418153in}{3.308658in}}%
\pgfpathcurveto{\pgfqpoint{9.425967in}{3.300845in}}{\pgfqpoint{9.436566in}{3.296455in}}{\pgfqpoint{9.447616in}{3.296455in}}%
\pgfpathclose%
\pgfusepath{stroke,fill}%
\end{pgfscope}%
\begin{pgfscope}%
\pgfpathrectangle{\pgfqpoint{0.570343in}{0.331635in}}{\pgfqpoint{9.300000in}{7.700000in}}%
\pgfusepath{clip}%
\pgfsetbuttcap%
\pgfsetroundjoin%
\definecolor{currentfill}{rgb}{1.000000,0.705882,0.509804}%
\pgfsetfillcolor{currentfill}%
\pgfsetlinewidth{0.481800pt}%
\definecolor{currentstroke}{rgb}{1.000000,1.000000,1.000000}%
\pgfsetstrokecolor{currentstroke}%
\pgfsetdash{}{0pt}%
\pgfpathmoveto{\pgfqpoint{3.829395in}{5.151485in}}%
\pgfpathcurveto{\pgfqpoint{3.840445in}{5.151485in}}{\pgfqpoint{3.851044in}{5.155875in}}{\pgfqpoint{3.858857in}{5.163689in}}%
\pgfpathcurveto{\pgfqpoint{3.866671in}{5.171502in}}{\pgfqpoint{3.871061in}{5.182101in}}{\pgfqpoint{3.871061in}{5.193151in}}%
\pgfpathcurveto{\pgfqpoint{3.871061in}{5.204202in}}{\pgfqpoint{3.866671in}{5.214801in}}{\pgfqpoint{3.858857in}{5.222614in}}%
\pgfpathcurveto{\pgfqpoint{3.851044in}{5.230428in}}{\pgfqpoint{3.840445in}{5.234818in}}{\pgfqpoint{3.829395in}{5.234818in}}%
\pgfpathcurveto{\pgfqpoint{3.818345in}{5.234818in}}{\pgfqpoint{3.807746in}{5.230428in}}{\pgfqpoint{3.799932in}{5.222614in}}%
\pgfpathcurveto{\pgfqpoint{3.792118in}{5.214801in}}{\pgfqpoint{3.787728in}{5.204202in}}{\pgfqpoint{3.787728in}{5.193151in}}%
\pgfpathcurveto{\pgfqpoint{3.787728in}{5.182101in}}{\pgfqpoint{3.792118in}{5.171502in}}{\pgfqpoint{3.799932in}{5.163689in}}%
\pgfpathcurveto{\pgfqpoint{3.807746in}{5.155875in}}{\pgfqpoint{3.818345in}{5.151485in}}{\pgfqpoint{3.829395in}{5.151485in}}%
\pgfpathclose%
\pgfusepath{stroke,fill}%
\end{pgfscope}%
\begin{pgfscope}%
\pgfpathrectangle{\pgfqpoint{0.570343in}{0.331635in}}{\pgfqpoint{9.300000in}{7.700000in}}%
\pgfusepath{clip}%
\pgfsetbuttcap%
\pgfsetroundjoin%
\definecolor{currentfill}{rgb}{1.000000,0.705882,0.509804}%
\pgfsetfillcolor{currentfill}%
\pgfsetlinewidth{0.481800pt}%
\definecolor{currentstroke}{rgb}{1.000000,1.000000,1.000000}%
\pgfsetstrokecolor{currentstroke}%
\pgfsetdash{}{0pt}%
\pgfpathmoveto{\pgfqpoint{2.176389in}{4.174756in}}%
\pgfpathcurveto{\pgfqpoint{2.187439in}{4.174756in}}{\pgfqpoint{2.198038in}{4.179147in}}{\pgfqpoint{2.205852in}{4.186960in}}%
\pgfpathcurveto{\pgfqpoint{2.213665in}{4.194774in}}{\pgfqpoint{2.218056in}{4.205373in}}{\pgfqpoint{2.218056in}{4.216423in}}%
\pgfpathcurveto{\pgfqpoint{2.218056in}{4.227473in}}{\pgfqpoint{2.213665in}{4.238072in}}{\pgfqpoint{2.205852in}{4.245886in}}%
\pgfpathcurveto{\pgfqpoint{2.198038in}{4.253699in}}{\pgfqpoint{2.187439in}{4.258090in}}{\pgfqpoint{2.176389in}{4.258090in}}%
\pgfpathcurveto{\pgfqpoint{2.165339in}{4.258090in}}{\pgfqpoint{2.154740in}{4.253699in}}{\pgfqpoint{2.146926in}{4.245886in}}%
\pgfpathcurveto{\pgfqpoint{2.139113in}{4.238072in}}{\pgfqpoint{2.134722in}{4.227473in}}{\pgfqpoint{2.134722in}{4.216423in}}%
\pgfpathcurveto{\pgfqpoint{2.134722in}{4.205373in}}{\pgfqpoint{2.139113in}{4.194774in}}{\pgfqpoint{2.146926in}{4.186960in}}%
\pgfpathcurveto{\pgfqpoint{2.154740in}{4.179147in}}{\pgfqpoint{2.165339in}{4.174756in}}{\pgfqpoint{2.176389in}{4.174756in}}%
\pgfpathclose%
\pgfusepath{stroke,fill}%
\end{pgfscope}%
\begin{pgfscope}%
\pgfpathrectangle{\pgfqpoint{0.570343in}{0.331635in}}{\pgfqpoint{9.300000in}{7.700000in}}%
\pgfusepath{clip}%
\pgfsetbuttcap%
\pgfsetroundjoin%
\definecolor{currentfill}{rgb}{1.000000,0.705882,0.509804}%
\pgfsetfillcolor{currentfill}%
\pgfsetlinewidth{0.481800pt}%
\definecolor{currentstroke}{rgb}{1.000000,1.000000,1.000000}%
\pgfsetstrokecolor{currentstroke}%
\pgfsetdash{}{0pt}%
\pgfpathmoveto{\pgfqpoint{7.119633in}{3.356574in}}%
\pgfpathcurveto{\pgfqpoint{7.130684in}{3.356574in}}{\pgfqpoint{7.141283in}{3.360964in}}{\pgfqpoint{7.149096in}{3.368778in}}%
\pgfpathcurveto{\pgfqpoint{7.156910in}{3.376591in}}{\pgfqpoint{7.161300in}{3.387190in}}{\pgfqpoint{7.161300in}{3.398240in}}%
\pgfpathcurveto{\pgfqpoint{7.161300in}{3.409291in}}{\pgfqpoint{7.156910in}{3.419890in}}{\pgfqpoint{7.149096in}{3.427703in}}%
\pgfpathcurveto{\pgfqpoint{7.141283in}{3.435517in}}{\pgfqpoint{7.130684in}{3.439907in}}{\pgfqpoint{7.119633in}{3.439907in}}%
\pgfpathcurveto{\pgfqpoint{7.108583in}{3.439907in}}{\pgfqpoint{7.097984in}{3.435517in}}{\pgfqpoint{7.090171in}{3.427703in}}%
\pgfpathcurveto{\pgfqpoint{7.082357in}{3.419890in}}{\pgfqpoint{7.077967in}{3.409291in}}{\pgfqpoint{7.077967in}{3.398240in}}%
\pgfpathcurveto{\pgfqpoint{7.077967in}{3.387190in}}{\pgfqpoint{7.082357in}{3.376591in}}{\pgfqpoint{7.090171in}{3.368778in}}%
\pgfpathcurveto{\pgfqpoint{7.097984in}{3.360964in}}{\pgfqpoint{7.108583in}{3.356574in}}{\pgfqpoint{7.119633in}{3.356574in}}%
\pgfpathclose%
\pgfusepath{stroke,fill}%
\end{pgfscope}%
\begin{pgfscope}%
\pgfpathrectangle{\pgfqpoint{0.570343in}{0.331635in}}{\pgfqpoint{9.300000in}{7.700000in}}%
\pgfusepath{clip}%
\pgfsetbuttcap%
\pgfsetroundjoin%
\definecolor{currentfill}{rgb}{1.000000,0.705882,0.509804}%
\pgfsetfillcolor{currentfill}%
\pgfsetlinewidth{0.481800pt}%
\definecolor{currentstroke}{rgb}{1.000000,1.000000,1.000000}%
\pgfsetstrokecolor{currentstroke}%
\pgfsetdash{}{0pt}%
\pgfpathmoveto{\pgfqpoint{2.977371in}{5.463396in}}%
\pgfpathcurveto{\pgfqpoint{2.988421in}{5.463396in}}{\pgfqpoint{2.999020in}{5.467787in}}{\pgfqpoint{3.006833in}{5.475600in}}%
\pgfpathcurveto{\pgfqpoint{3.014647in}{5.483414in}}{\pgfqpoint{3.019037in}{5.494013in}}{\pgfqpoint{3.019037in}{5.505063in}}%
\pgfpathcurveto{\pgfqpoint{3.019037in}{5.516113in}}{\pgfqpoint{3.014647in}{5.526712in}}{\pgfqpoint{3.006833in}{5.534526in}}%
\pgfpathcurveto{\pgfqpoint{2.999020in}{5.542339in}}{\pgfqpoint{2.988421in}{5.546730in}}{\pgfqpoint{2.977371in}{5.546730in}}%
\pgfpathcurveto{\pgfqpoint{2.966321in}{5.546730in}}{\pgfqpoint{2.955721in}{5.542339in}}{\pgfqpoint{2.947908in}{5.534526in}}%
\pgfpathcurveto{\pgfqpoint{2.940094in}{5.526712in}}{\pgfqpoint{2.935704in}{5.516113in}}{\pgfqpoint{2.935704in}{5.505063in}}%
\pgfpathcurveto{\pgfqpoint{2.935704in}{5.494013in}}{\pgfqpoint{2.940094in}{5.483414in}}{\pgfqpoint{2.947908in}{5.475600in}}%
\pgfpathcurveto{\pgfqpoint{2.955721in}{5.467787in}}{\pgfqpoint{2.966321in}{5.463396in}}{\pgfqpoint{2.977371in}{5.463396in}}%
\pgfpathclose%
\pgfusepath{stroke,fill}%
\end{pgfscope}%
\begin{pgfscope}%
\pgfpathrectangle{\pgfqpoint{0.570343in}{0.331635in}}{\pgfqpoint{9.300000in}{7.700000in}}%
\pgfusepath{clip}%
\pgfsetbuttcap%
\pgfsetroundjoin%
\definecolor{currentfill}{rgb}{0.631373,0.788235,0.956863}%
\pgfsetfillcolor{currentfill}%
\pgfsetlinewidth{1.003750pt}%
\definecolor{currentstroke}{rgb}{0.631373,0.788235,0.956863}%
\pgfsetstrokecolor{currentstroke}%
\pgfsetdash{}{0pt}%
\pgfsys@defobject{currentmarker}{\pgfqpoint{-0.041667in}{-0.041667in}}{\pgfqpoint{0.041667in}{0.041667in}}{%
\pgfpathmoveto{\pgfqpoint{0.000000in}{-0.041667in}}%
\pgfpathcurveto{\pgfqpoint{0.011050in}{-0.041667in}}{\pgfqpoint{0.021649in}{-0.037276in}}{\pgfqpoint{0.029463in}{-0.029463in}}%
\pgfpathcurveto{\pgfqpoint{0.037276in}{-0.021649in}}{\pgfqpoint{0.041667in}{-0.011050in}}{\pgfqpoint{0.041667in}{0.000000in}}%
\pgfpathcurveto{\pgfqpoint{0.041667in}{0.011050in}}{\pgfqpoint{0.037276in}{0.021649in}}{\pgfqpoint{0.029463in}{0.029463in}}%
\pgfpathcurveto{\pgfqpoint{0.021649in}{0.037276in}}{\pgfqpoint{0.011050in}{0.041667in}}{\pgfqpoint{0.000000in}{0.041667in}}%
\pgfpathcurveto{\pgfqpoint{-0.011050in}{0.041667in}}{\pgfqpoint{-0.021649in}{0.037276in}}{\pgfqpoint{-0.029463in}{0.029463in}}%
\pgfpathcurveto{\pgfqpoint{-0.037276in}{0.021649in}}{\pgfqpoint{-0.041667in}{0.011050in}}{\pgfqpoint{-0.041667in}{0.000000in}}%
\pgfpathcurveto{\pgfqpoint{-0.041667in}{-0.011050in}}{\pgfqpoint{-0.037276in}{-0.021649in}}{\pgfqpoint{-0.029463in}{-0.029463in}}%
\pgfpathcurveto{\pgfqpoint{-0.021649in}{-0.037276in}}{\pgfqpoint{-0.011050in}{-0.041667in}}{\pgfqpoint{0.000000in}{-0.041667in}}%
\pgfpathclose%
\pgfusepath{stroke,fill}%
}%
\end{pgfscope}%
\begin{pgfscope}%
\pgfpathrectangle{\pgfqpoint{0.570343in}{0.331635in}}{\pgfqpoint{9.300000in}{7.700000in}}%
\pgfusepath{clip}%
\pgfsetbuttcap%
\pgfsetroundjoin%
\definecolor{currentfill}{rgb}{1.000000,0.705882,0.509804}%
\pgfsetfillcolor{currentfill}%
\pgfsetlinewidth{1.003750pt}%
\definecolor{currentstroke}{rgb}{1.000000,0.705882,0.509804}%
\pgfsetstrokecolor{currentstroke}%
\pgfsetdash{}{0pt}%
\pgfsys@defobject{currentmarker}{\pgfqpoint{-0.041667in}{-0.041667in}}{\pgfqpoint{0.041667in}{0.041667in}}{%
\pgfpathmoveto{\pgfqpoint{0.000000in}{-0.041667in}}%
\pgfpathcurveto{\pgfqpoint{0.011050in}{-0.041667in}}{\pgfqpoint{0.021649in}{-0.037276in}}{\pgfqpoint{0.029463in}{-0.029463in}}%
\pgfpathcurveto{\pgfqpoint{0.037276in}{-0.021649in}}{\pgfqpoint{0.041667in}{-0.011050in}}{\pgfqpoint{0.041667in}{0.000000in}}%
\pgfpathcurveto{\pgfqpoint{0.041667in}{0.011050in}}{\pgfqpoint{0.037276in}{0.021649in}}{\pgfqpoint{0.029463in}{0.029463in}}%
\pgfpathcurveto{\pgfqpoint{0.021649in}{0.037276in}}{\pgfqpoint{0.011050in}{0.041667in}}{\pgfqpoint{0.000000in}{0.041667in}}%
\pgfpathcurveto{\pgfqpoint{-0.011050in}{0.041667in}}{\pgfqpoint{-0.021649in}{0.037276in}}{\pgfqpoint{-0.029463in}{0.029463in}}%
\pgfpathcurveto{\pgfqpoint{-0.037276in}{0.021649in}}{\pgfqpoint{-0.041667in}{0.011050in}}{\pgfqpoint{-0.041667in}{0.000000in}}%
\pgfpathcurveto{\pgfqpoint{-0.041667in}{-0.011050in}}{\pgfqpoint{-0.037276in}{-0.021649in}}{\pgfqpoint{-0.029463in}{-0.029463in}}%
\pgfpathcurveto{\pgfqpoint{-0.021649in}{-0.037276in}}{\pgfqpoint{-0.011050in}{-0.041667in}}{\pgfqpoint{0.000000in}{-0.041667in}}%
\pgfpathclose%
\pgfusepath{stroke,fill}%
}%
\end{pgfscope}%
\begin{pgfscope}%
\pgfsetbuttcap%
\pgfsetroundjoin%
\definecolor{currentfill}{rgb}{0.000000,0.000000,0.000000}%
\pgfsetfillcolor{currentfill}%
\pgfsetlinewidth{0.803000pt}%
\definecolor{currentstroke}{rgb}{0.000000,0.000000,0.000000}%
\pgfsetstrokecolor{currentstroke}%
\pgfsetdash{}{0pt}%
\pgfsys@defobject{currentmarker}{\pgfqpoint{0.000000in}{-0.048611in}}{\pgfqpoint{0.000000in}{0.000000in}}{%
\pgfpathmoveto{\pgfqpoint{0.000000in}{0.000000in}}%
\pgfpathlineto{\pgfqpoint{0.000000in}{-0.048611in}}%
\pgfusepath{stroke,fill}%
}%
\begin{pgfscope}%
\pgfsys@transformshift{1.089085in}{0.331635in}%
\pgfsys@useobject{currentmarker}{}%
\end{pgfscope}%
\end{pgfscope}%
\begin{pgfscope}%
\definecolor{textcolor}{rgb}{0.000000,0.000000,0.000000}%
\pgfsetstrokecolor{textcolor}%
\pgfsetfillcolor{textcolor}%
\pgftext[x=1.089085in,y=0.234413in,,top]{\color{textcolor}\sffamily\fontsize{10.000000}{12.000000}\selectfont \ensuremath{-}150}%
\end{pgfscope}%
\begin{pgfscope}%
\pgfsetbuttcap%
\pgfsetroundjoin%
\definecolor{currentfill}{rgb}{0.000000,0.000000,0.000000}%
\pgfsetfillcolor{currentfill}%
\pgfsetlinewidth{0.803000pt}%
\definecolor{currentstroke}{rgb}{0.000000,0.000000,0.000000}%
\pgfsetstrokecolor{currentstroke}%
\pgfsetdash{}{0pt}%
\pgfsys@defobject{currentmarker}{\pgfqpoint{0.000000in}{-0.048611in}}{\pgfqpoint{0.000000in}{0.000000in}}{%
\pgfpathmoveto{\pgfqpoint{0.000000in}{0.000000in}}%
\pgfpathlineto{\pgfqpoint{0.000000in}{-0.048611in}}%
\pgfusepath{stroke,fill}%
}%
\begin{pgfscope}%
\pgfsys@transformshift{2.469704in}{0.331635in}%
\pgfsys@useobject{currentmarker}{}%
\end{pgfscope}%
\end{pgfscope}%
\begin{pgfscope}%
\definecolor{textcolor}{rgb}{0.000000,0.000000,0.000000}%
\pgfsetstrokecolor{textcolor}%
\pgfsetfillcolor{textcolor}%
\pgftext[x=2.469704in,y=0.234413in,,top]{\color{textcolor}\sffamily\fontsize{10.000000}{12.000000}\selectfont \ensuremath{-}100}%
\end{pgfscope}%
\begin{pgfscope}%
\pgfsetbuttcap%
\pgfsetroundjoin%
\definecolor{currentfill}{rgb}{0.000000,0.000000,0.000000}%
\pgfsetfillcolor{currentfill}%
\pgfsetlinewidth{0.803000pt}%
\definecolor{currentstroke}{rgb}{0.000000,0.000000,0.000000}%
\pgfsetstrokecolor{currentstroke}%
\pgfsetdash{}{0pt}%
\pgfsys@defobject{currentmarker}{\pgfqpoint{0.000000in}{-0.048611in}}{\pgfqpoint{0.000000in}{0.000000in}}{%
\pgfpathmoveto{\pgfqpoint{0.000000in}{0.000000in}}%
\pgfpathlineto{\pgfqpoint{0.000000in}{-0.048611in}}%
\pgfusepath{stroke,fill}%
}%
\begin{pgfscope}%
\pgfsys@transformshift{3.850323in}{0.331635in}%
\pgfsys@useobject{currentmarker}{}%
\end{pgfscope}%
\end{pgfscope}%
\begin{pgfscope}%
\definecolor{textcolor}{rgb}{0.000000,0.000000,0.000000}%
\pgfsetstrokecolor{textcolor}%
\pgfsetfillcolor{textcolor}%
\pgftext[x=3.850323in,y=0.234413in,,top]{\color{textcolor}\sffamily\fontsize{10.000000}{12.000000}\selectfont \ensuremath{-}50}%
\end{pgfscope}%
\begin{pgfscope}%
\pgfsetbuttcap%
\pgfsetroundjoin%
\definecolor{currentfill}{rgb}{0.000000,0.000000,0.000000}%
\pgfsetfillcolor{currentfill}%
\pgfsetlinewidth{0.803000pt}%
\definecolor{currentstroke}{rgb}{0.000000,0.000000,0.000000}%
\pgfsetstrokecolor{currentstroke}%
\pgfsetdash{}{0pt}%
\pgfsys@defobject{currentmarker}{\pgfqpoint{0.000000in}{-0.048611in}}{\pgfqpoint{0.000000in}{0.000000in}}{%
\pgfpathmoveto{\pgfqpoint{0.000000in}{0.000000in}}%
\pgfpathlineto{\pgfqpoint{0.000000in}{-0.048611in}}%
\pgfusepath{stroke,fill}%
}%
\begin{pgfscope}%
\pgfsys@transformshift{5.230942in}{0.331635in}%
\pgfsys@useobject{currentmarker}{}%
\end{pgfscope}%
\end{pgfscope}%
\begin{pgfscope}%
\definecolor{textcolor}{rgb}{0.000000,0.000000,0.000000}%
\pgfsetstrokecolor{textcolor}%
\pgfsetfillcolor{textcolor}%
\pgftext[x=5.230942in,y=0.234413in,,top]{\color{textcolor}\sffamily\fontsize{10.000000}{12.000000}\selectfont 0}%
\end{pgfscope}%
\begin{pgfscope}%
\pgfsetbuttcap%
\pgfsetroundjoin%
\definecolor{currentfill}{rgb}{0.000000,0.000000,0.000000}%
\pgfsetfillcolor{currentfill}%
\pgfsetlinewidth{0.803000pt}%
\definecolor{currentstroke}{rgb}{0.000000,0.000000,0.000000}%
\pgfsetstrokecolor{currentstroke}%
\pgfsetdash{}{0pt}%
\pgfsys@defobject{currentmarker}{\pgfqpoint{0.000000in}{-0.048611in}}{\pgfqpoint{0.000000in}{0.000000in}}{%
\pgfpathmoveto{\pgfqpoint{0.000000in}{0.000000in}}%
\pgfpathlineto{\pgfqpoint{0.000000in}{-0.048611in}}%
\pgfusepath{stroke,fill}%
}%
\begin{pgfscope}%
\pgfsys@transformshift{6.611561in}{0.331635in}%
\pgfsys@useobject{currentmarker}{}%
\end{pgfscope}%
\end{pgfscope}%
\begin{pgfscope}%
\definecolor{textcolor}{rgb}{0.000000,0.000000,0.000000}%
\pgfsetstrokecolor{textcolor}%
\pgfsetfillcolor{textcolor}%
\pgftext[x=6.611561in,y=0.234413in,,top]{\color{textcolor}\sffamily\fontsize{10.000000}{12.000000}\selectfont 50}%
\end{pgfscope}%
\begin{pgfscope}%
\pgfsetbuttcap%
\pgfsetroundjoin%
\definecolor{currentfill}{rgb}{0.000000,0.000000,0.000000}%
\pgfsetfillcolor{currentfill}%
\pgfsetlinewidth{0.803000pt}%
\definecolor{currentstroke}{rgb}{0.000000,0.000000,0.000000}%
\pgfsetstrokecolor{currentstroke}%
\pgfsetdash{}{0pt}%
\pgfsys@defobject{currentmarker}{\pgfqpoint{0.000000in}{-0.048611in}}{\pgfqpoint{0.000000in}{0.000000in}}{%
\pgfpathmoveto{\pgfqpoint{0.000000in}{0.000000in}}%
\pgfpathlineto{\pgfqpoint{0.000000in}{-0.048611in}}%
\pgfusepath{stroke,fill}%
}%
\begin{pgfscope}%
\pgfsys@transformshift{7.992180in}{0.331635in}%
\pgfsys@useobject{currentmarker}{}%
\end{pgfscope}%
\end{pgfscope}%
\begin{pgfscope}%
\definecolor{textcolor}{rgb}{0.000000,0.000000,0.000000}%
\pgfsetstrokecolor{textcolor}%
\pgfsetfillcolor{textcolor}%
\pgftext[x=7.992180in,y=0.234413in,,top]{\color{textcolor}\sffamily\fontsize{10.000000}{12.000000}\selectfont 100}%
\end{pgfscope}%
\begin{pgfscope}%
\pgfsetbuttcap%
\pgfsetroundjoin%
\definecolor{currentfill}{rgb}{0.000000,0.000000,0.000000}%
\pgfsetfillcolor{currentfill}%
\pgfsetlinewidth{0.803000pt}%
\definecolor{currentstroke}{rgb}{0.000000,0.000000,0.000000}%
\pgfsetstrokecolor{currentstroke}%
\pgfsetdash{}{0pt}%
\pgfsys@defobject{currentmarker}{\pgfqpoint{0.000000in}{-0.048611in}}{\pgfqpoint{0.000000in}{0.000000in}}{%
\pgfpathmoveto{\pgfqpoint{0.000000in}{0.000000in}}%
\pgfpathlineto{\pgfqpoint{0.000000in}{-0.048611in}}%
\pgfusepath{stroke,fill}%
}%
\begin{pgfscope}%
\pgfsys@transformshift{9.372799in}{0.331635in}%
\pgfsys@useobject{currentmarker}{}%
\end{pgfscope}%
\end{pgfscope}%
\begin{pgfscope}%
\definecolor{textcolor}{rgb}{0.000000,0.000000,0.000000}%
\pgfsetstrokecolor{textcolor}%
\pgfsetfillcolor{textcolor}%
\pgftext[x=9.372799in,y=0.234413in,,top]{\color{textcolor}\sffamily\fontsize{10.000000}{12.000000}\selectfont 150}%
\end{pgfscope}%
\begin{pgfscope}%
\pgfsetbuttcap%
\pgfsetroundjoin%
\definecolor{currentfill}{rgb}{0.000000,0.000000,0.000000}%
\pgfsetfillcolor{currentfill}%
\pgfsetlinewidth{0.803000pt}%
\definecolor{currentstroke}{rgb}{0.000000,0.000000,0.000000}%
\pgfsetstrokecolor{currentstroke}%
\pgfsetdash{}{0pt}%
\pgfsys@defobject{currentmarker}{\pgfqpoint{-0.048611in}{0.000000in}}{\pgfqpoint{-0.000000in}{0.000000in}}{%
\pgfpathmoveto{\pgfqpoint{-0.000000in}{0.000000in}}%
\pgfpathlineto{\pgfqpoint{-0.048611in}{0.000000in}}%
\pgfusepath{stroke,fill}%
}%
\begin{pgfscope}%
\pgfsys@transformshift{0.570343in}{0.447685in}%
\pgfsys@useobject{currentmarker}{}%
\end{pgfscope}%
\end{pgfscope}%
\begin{pgfscope}%
\definecolor{textcolor}{rgb}{0.000000,0.000000,0.000000}%
\pgfsetstrokecolor{textcolor}%
\pgfsetfillcolor{textcolor}%
\pgftext[x=0.100000in, y=0.394924in, left, base]{\color{textcolor}\sffamily\fontsize{10.000000}{12.000000}\selectfont \ensuremath{-}150}%
\end{pgfscope}%
\begin{pgfscope}%
\pgfsetbuttcap%
\pgfsetroundjoin%
\definecolor{currentfill}{rgb}{0.000000,0.000000,0.000000}%
\pgfsetfillcolor{currentfill}%
\pgfsetlinewidth{0.803000pt}%
\definecolor{currentstroke}{rgb}{0.000000,0.000000,0.000000}%
\pgfsetstrokecolor{currentstroke}%
\pgfsetdash{}{0pt}%
\pgfsys@defobject{currentmarker}{\pgfqpoint{-0.048611in}{0.000000in}}{\pgfqpoint{-0.000000in}{0.000000in}}{%
\pgfpathmoveto{\pgfqpoint{-0.000000in}{0.000000in}}%
\pgfpathlineto{\pgfqpoint{-0.048611in}{0.000000in}}%
\pgfusepath{stroke,fill}%
}%
\begin{pgfscope}%
\pgfsys@transformshift{0.570343in}{1.694204in}%
\pgfsys@useobject{currentmarker}{}%
\end{pgfscope}%
\end{pgfscope}%
\begin{pgfscope}%
\definecolor{textcolor}{rgb}{0.000000,0.000000,0.000000}%
\pgfsetstrokecolor{textcolor}%
\pgfsetfillcolor{textcolor}%
\pgftext[x=0.100000in, y=1.641443in, left, base]{\color{textcolor}\sffamily\fontsize{10.000000}{12.000000}\selectfont \ensuremath{-}100}%
\end{pgfscope}%
\begin{pgfscope}%
\pgfsetbuttcap%
\pgfsetroundjoin%
\definecolor{currentfill}{rgb}{0.000000,0.000000,0.000000}%
\pgfsetfillcolor{currentfill}%
\pgfsetlinewidth{0.803000pt}%
\definecolor{currentstroke}{rgb}{0.000000,0.000000,0.000000}%
\pgfsetstrokecolor{currentstroke}%
\pgfsetdash{}{0pt}%
\pgfsys@defobject{currentmarker}{\pgfqpoint{-0.048611in}{0.000000in}}{\pgfqpoint{-0.000000in}{0.000000in}}{%
\pgfpathmoveto{\pgfqpoint{-0.000000in}{0.000000in}}%
\pgfpathlineto{\pgfqpoint{-0.048611in}{0.000000in}}%
\pgfusepath{stroke,fill}%
}%
\begin{pgfscope}%
\pgfsys@transformshift{0.570343in}{2.940723in}%
\pgfsys@useobject{currentmarker}{}%
\end{pgfscope}%
\end{pgfscope}%
\begin{pgfscope}%
\definecolor{textcolor}{rgb}{0.000000,0.000000,0.000000}%
\pgfsetstrokecolor{textcolor}%
\pgfsetfillcolor{textcolor}%
\pgftext[x=0.188365in, y=2.887962in, left, base]{\color{textcolor}\sffamily\fontsize{10.000000}{12.000000}\selectfont \ensuremath{-}50}%
\end{pgfscope}%
\begin{pgfscope}%
\pgfsetbuttcap%
\pgfsetroundjoin%
\definecolor{currentfill}{rgb}{0.000000,0.000000,0.000000}%
\pgfsetfillcolor{currentfill}%
\pgfsetlinewidth{0.803000pt}%
\definecolor{currentstroke}{rgb}{0.000000,0.000000,0.000000}%
\pgfsetstrokecolor{currentstroke}%
\pgfsetdash{}{0pt}%
\pgfsys@defobject{currentmarker}{\pgfqpoint{-0.048611in}{0.000000in}}{\pgfqpoint{-0.000000in}{0.000000in}}{%
\pgfpathmoveto{\pgfqpoint{-0.000000in}{0.000000in}}%
\pgfpathlineto{\pgfqpoint{-0.048611in}{0.000000in}}%
\pgfusepath{stroke,fill}%
}%
\begin{pgfscope}%
\pgfsys@transformshift{0.570343in}{4.187242in}%
\pgfsys@useobject{currentmarker}{}%
\end{pgfscope}%
\end{pgfscope}%
\begin{pgfscope}%
\definecolor{textcolor}{rgb}{0.000000,0.000000,0.000000}%
\pgfsetstrokecolor{textcolor}%
\pgfsetfillcolor{textcolor}%
\pgftext[x=0.384756in, y=4.134481in, left, base]{\color{textcolor}\sffamily\fontsize{10.000000}{12.000000}\selectfont 0}%
\end{pgfscope}%
\begin{pgfscope}%
\pgfsetbuttcap%
\pgfsetroundjoin%
\definecolor{currentfill}{rgb}{0.000000,0.000000,0.000000}%
\pgfsetfillcolor{currentfill}%
\pgfsetlinewidth{0.803000pt}%
\definecolor{currentstroke}{rgb}{0.000000,0.000000,0.000000}%
\pgfsetstrokecolor{currentstroke}%
\pgfsetdash{}{0pt}%
\pgfsys@defobject{currentmarker}{\pgfqpoint{-0.048611in}{0.000000in}}{\pgfqpoint{-0.000000in}{0.000000in}}{%
\pgfpathmoveto{\pgfqpoint{-0.000000in}{0.000000in}}%
\pgfpathlineto{\pgfqpoint{-0.048611in}{0.000000in}}%
\pgfusepath{stroke,fill}%
}%
\begin{pgfscope}%
\pgfsys@transformshift{0.570343in}{5.433761in}%
\pgfsys@useobject{currentmarker}{}%
\end{pgfscope}%
\end{pgfscope}%
\begin{pgfscope}%
\definecolor{textcolor}{rgb}{0.000000,0.000000,0.000000}%
\pgfsetstrokecolor{textcolor}%
\pgfsetfillcolor{textcolor}%
\pgftext[x=0.296390in, y=5.381000in, left, base]{\color{textcolor}\sffamily\fontsize{10.000000}{12.000000}\selectfont 50}%
\end{pgfscope}%
\begin{pgfscope}%
\pgfsetbuttcap%
\pgfsetroundjoin%
\definecolor{currentfill}{rgb}{0.000000,0.000000,0.000000}%
\pgfsetfillcolor{currentfill}%
\pgfsetlinewidth{0.803000pt}%
\definecolor{currentstroke}{rgb}{0.000000,0.000000,0.000000}%
\pgfsetstrokecolor{currentstroke}%
\pgfsetdash{}{0pt}%
\pgfsys@defobject{currentmarker}{\pgfqpoint{-0.048611in}{0.000000in}}{\pgfqpoint{-0.000000in}{0.000000in}}{%
\pgfpathmoveto{\pgfqpoint{-0.000000in}{0.000000in}}%
\pgfpathlineto{\pgfqpoint{-0.048611in}{0.000000in}}%
\pgfusepath{stroke,fill}%
}%
\begin{pgfscope}%
\pgfsys@transformshift{0.570343in}{6.680280in}%
\pgfsys@useobject{currentmarker}{}%
\end{pgfscope}%
\end{pgfscope}%
\begin{pgfscope}%
\definecolor{textcolor}{rgb}{0.000000,0.000000,0.000000}%
\pgfsetstrokecolor{textcolor}%
\pgfsetfillcolor{textcolor}%
\pgftext[x=0.208025in, y=6.627519in, left, base]{\color{textcolor}\sffamily\fontsize{10.000000}{12.000000}\selectfont 100}%
\end{pgfscope}%
\begin{pgfscope}%
\pgfsetbuttcap%
\pgfsetroundjoin%
\definecolor{currentfill}{rgb}{0.000000,0.000000,0.000000}%
\pgfsetfillcolor{currentfill}%
\pgfsetlinewidth{0.803000pt}%
\definecolor{currentstroke}{rgb}{0.000000,0.000000,0.000000}%
\pgfsetstrokecolor{currentstroke}%
\pgfsetdash{}{0pt}%
\pgfsys@defobject{currentmarker}{\pgfqpoint{-0.048611in}{0.000000in}}{\pgfqpoint{-0.000000in}{0.000000in}}{%
\pgfpathmoveto{\pgfqpoint{-0.000000in}{0.000000in}}%
\pgfpathlineto{\pgfqpoint{-0.048611in}{0.000000in}}%
\pgfusepath{stroke,fill}%
}%
\begin{pgfscope}%
\pgfsys@transformshift{0.570343in}{7.926799in}%
\pgfsys@useobject{currentmarker}{}%
\end{pgfscope}%
\end{pgfscope}%
\begin{pgfscope}%
\definecolor{textcolor}{rgb}{0.000000,0.000000,0.000000}%
\pgfsetstrokecolor{textcolor}%
\pgfsetfillcolor{textcolor}%
\pgftext[x=0.208025in, y=7.874038in, left, base]{\color{textcolor}\sffamily\fontsize{10.000000}{12.000000}\selectfont 150}%
\end{pgfscope}%
\begin{pgfscope}%
\pgfpathrectangle{\pgfqpoint{0.570343in}{0.331635in}}{\pgfqpoint{9.300000in}{7.700000in}}%
\pgfusepath{clip}%
\pgfsetrectcap%
\pgfsetroundjoin%
\pgfsetlinewidth{1.505625pt}%
\definecolor{currentstroke}{rgb}{0.631373,0.788235,0.956863}%
\pgfsetstrokecolor{currentstroke}%
\pgfsetstrokeopacity{0.800000}%
\pgfsetdash{}{0pt}%
\pgfpathmoveto{\pgfqpoint{2.738718in}{6.381740in}}%
\pgfpathlineto{\pgfqpoint{4.803960in}{4.277068in}}%
\pgfusepath{stroke}%
\end{pgfscope}%
\begin{pgfscope}%
\pgfpathrectangle{\pgfqpoint{0.570343in}{0.331635in}}{\pgfqpoint{9.300000in}{7.700000in}}%
\pgfusepath{clip}%
\pgfsetrectcap%
\pgfsetroundjoin%
\pgfsetlinewidth{1.505625pt}%
\definecolor{currentstroke}{rgb}{0.631373,0.788235,0.956863}%
\pgfsetstrokecolor{currentstroke}%
\pgfsetstrokeopacity{0.800000}%
\pgfsetdash{}{0pt}%
\pgfpathmoveto{\pgfqpoint{4.645211in}{3.372948in}}%
\pgfpathlineto{\pgfqpoint{4.803960in}{4.277068in}}%
\pgfusepath{stroke}%
\end{pgfscope}%
\begin{pgfscope}%
\pgfpathrectangle{\pgfqpoint{0.570343in}{0.331635in}}{\pgfqpoint{9.300000in}{7.700000in}}%
\pgfusepath{clip}%
\pgfsetrectcap%
\pgfsetroundjoin%
\pgfsetlinewidth{1.505625pt}%
\definecolor{currentstroke}{rgb}{0.631373,0.788235,0.956863}%
\pgfsetstrokecolor{currentstroke}%
\pgfsetstrokeopacity{0.800000}%
\pgfsetdash{}{0pt}%
\pgfpathmoveto{\pgfqpoint{5.785764in}{4.783118in}}%
\pgfpathlineto{\pgfqpoint{4.803960in}{4.277068in}}%
\pgfusepath{stroke}%
\end{pgfscope}%
\begin{pgfscope}%
\pgfpathrectangle{\pgfqpoint{0.570343in}{0.331635in}}{\pgfqpoint{9.300000in}{7.700000in}}%
\pgfusepath{clip}%
\pgfsetrectcap%
\pgfsetroundjoin%
\pgfsetlinewidth{1.505625pt}%
\definecolor{currentstroke}{rgb}{0.631373,0.788235,0.956863}%
\pgfsetstrokecolor{currentstroke}%
\pgfsetstrokeopacity{0.800000}%
\pgfsetdash{}{0pt}%
\pgfpathmoveto{\pgfqpoint{5.100277in}{4.641711in}}%
\pgfpathlineto{\pgfqpoint{4.803960in}{4.277068in}}%
\pgfusepath{stroke}%
\end{pgfscope}%
\begin{pgfscope}%
\pgfpathrectangle{\pgfqpoint{0.570343in}{0.331635in}}{\pgfqpoint{9.300000in}{7.700000in}}%
\pgfusepath{clip}%
\pgfsetrectcap%
\pgfsetroundjoin%
\pgfsetlinewidth{1.505625pt}%
\definecolor{currentstroke}{rgb}{0.631373,0.788235,0.956863}%
\pgfsetstrokecolor{currentstroke}%
\pgfsetstrokeopacity{0.800000}%
\pgfsetdash{}{0pt}%
\pgfpathmoveto{\pgfqpoint{0.993071in}{4.297862in}}%
\pgfpathlineto{\pgfqpoint{4.803960in}{4.277068in}}%
\pgfusepath{stroke}%
\end{pgfscope}%
\begin{pgfscope}%
\pgfpathrectangle{\pgfqpoint{0.570343in}{0.331635in}}{\pgfqpoint{9.300000in}{7.700000in}}%
\pgfusepath{clip}%
\pgfsetrectcap%
\pgfsetroundjoin%
\pgfsetlinewidth{1.505625pt}%
\definecolor{currentstroke}{rgb}{0.631373,0.788235,0.956863}%
\pgfsetstrokecolor{currentstroke}%
\pgfsetstrokeopacity{0.800000}%
\pgfsetdash{}{0pt}%
\pgfpathmoveto{\pgfqpoint{6.176499in}{3.613833in}}%
\pgfpathlineto{\pgfqpoint{4.803960in}{4.277068in}}%
\pgfusepath{stroke}%
\end{pgfscope}%
\begin{pgfscope}%
\pgfpathrectangle{\pgfqpoint{0.570343in}{0.331635in}}{\pgfqpoint{9.300000in}{7.700000in}}%
\pgfusepath{clip}%
\pgfsetrectcap%
\pgfsetroundjoin%
\pgfsetlinewidth{1.505625pt}%
\definecolor{currentstroke}{rgb}{0.631373,0.788235,0.956863}%
\pgfsetstrokecolor{currentstroke}%
\pgfsetstrokeopacity{0.800000}%
\pgfsetdash{}{0pt}%
\pgfpathmoveto{\pgfqpoint{3.773953in}{2.580081in}}%
\pgfpathlineto{\pgfqpoint{4.803960in}{4.277068in}}%
\pgfusepath{stroke}%
\end{pgfscope}%
\begin{pgfscope}%
\pgfpathrectangle{\pgfqpoint{0.570343in}{0.331635in}}{\pgfqpoint{9.300000in}{7.700000in}}%
\pgfusepath{clip}%
\pgfsetrectcap%
\pgfsetroundjoin%
\pgfsetlinewidth{1.505625pt}%
\definecolor{currentstroke}{rgb}{0.631373,0.788235,0.956863}%
\pgfsetstrokecolor{currentstroke}%
\pgfsetstrokeopacity{0.800000}%
\pgfsetdash{}{0pt}%
\pgfpathmoveto{\pgfqpoint{7.797264in}{4.769230in}}%
\pgfpathlineto{\pgfqpoint{4.803960in}{4.277068in}}%
\pgfusepath{stroke}%
\end{pgfscope}%
\begin{pgfscope}%
\pgfpathrectangle{\pgfqpoint{0.570343in}{0.331635in}}{\pgfqpoint{9.300000in}{7.700000in}}%
\pgfusepath{clip}%
\pgfsetrectcap%
\pgfsetroundjoin%
\pgfsetlinewidth{1.505625pt}%
\definecolor{currentstroke}{rgb}{0.631373,0.788235,0.956863}%
\pgfsetstrokecolor{currentstroke}%
\pgfsetstrokeopacity{0.800000}%
\pgfsetdash{}{0pt}%
\pgfpathmoveto{\pgfqpoint{2.799738in}{3.131541in}}%
\pgfpathlineto{\pgfqpoint{4.803960in}{4.277068in}}%
\pgfusepath{stroke}%
\end{pgfscope}%
\begin{pgfscope}%
\pgfpathrectangle{\pgfqpoint{0.570343in}{0.331635in}}{\pgfqpoint{9.300000in}{7.700000in}}%
\pgfusepath{clip}%
\pgfsetrectcap%
\pgfsetroundjoin%
\pgfsetlinewidth{1.505625pt}%
\definecolor{currentstroke}{rgb}{0.631373,0.788235,0.956863}%
\pgfsetstrokecolor{currentstroke}%
\pgfsetstrokeopacity{0.800000}%
\pgfsetdash{}{0pt}%
\pgfpathmoveto{\pgfqpoint{6.185154in}{2.892476in}}%
\pgfpathlineto{\pgfqpoint{4.803960in}{4.277068in}}%
\pgfusepath{stroke}%
\end{pgfscope}%
\begin{pgfscope}%
\pgfpathrectangle{\pgfqpoint{0.570343in}{0.331635in}}{\pgfqpoint{9.300000in}{7.700000in}}%
\pgfusepath{clip}%
\pgfsetrectcap%
\pgfsetroundjoin%
\pgfsetlinewidth{1.505625pt}%
\definecolor{currentstroke}{rgb}{0.631373,0.788235,0.956863}%
\pgfsetstrokecolor{currentstroke}%
\pgfsetstrokeopacity{0.800000}%
\pgfsetdash{}{0pt}%
\pgfpathmoveto{\pgfqpoint{2.095236in}{3.512298in}}%
\pgfpathlineto{\pgfqpoint{4.803960in}{4.277068in}}%
\pgfusepath{stroke}%
\end{pgfscope}%
\begin{pgfscope}%
\pgfpathrectangle{\pgfqpoint{0.570343in}{0.331635in}}{\pgfqpoint{9.300000in}{7.700000in}}%
\pgfusepath{clip}%
\pgfsetrectcap%
\pgfsetroundjoin%
\pgfsetlinewidth{1.505625pt}%
\definecolor{currentstroke}{rgb}{0.631373,0.788235,0.956863}%
\pgfsetstrokecolor{currentstroke}%
\pgfsetstrokeopacity{0.800000}%
\pgfsetdash{}{0pt}%
\pgfpathmoveto{\pgfqpoint{4.706525in}{5.302161in}}%
\pgfpathlineto{\pgfqpoint{4.803960in}{4.277068in}}%
\pgfusepath{stroke}%
\end{pgfscope}%
\begin{pgfscope}%
\pgfpathrectangle{\pgfqpoint{0.570343in}{0.331635in}}{\pgfqpoint{9.300000in}{7.700000in}}%
\pgfusepath{clip}%
\pgfsetrectcap%
\pgfsetroundjoin%
\pgfsetlinewidth{1.505625pt}%
\definecolor{currentstroke}{rgb}{0.631373,0.788235,0.956863}%
\pgfsetstrokecolor{currentstroke}%
\pgfsetstrokeopacity{0.800000}%
\pgfsetdash{}{0pt}%
\pgfpathmoveto{\pgfqpoint{3.943202in}{3.936373in}}%
\pgfpathlineto{\pgfqpoint{4.803960in}{4.277068in}}%
\pgfusepath{stroke}%
\end{pgfscope}%
\begin{pgfscope}%
\pgfpathrectangle{\pgfqpoint{0.570343in}{0.331635in}}{\pgfqpoint{9.300000in}{7.700000in}}%
\pgfusepath{clip}%
\pgfsetrectcap%
\pgfsetroundjoin%
\pgfsetlinewidth{1.505625pt}%
\definecolor{currentstroke}{rgb}{0.631373,0.788235,0.956863}%
\pgfsetstrokecolor{currentstroke}%
\pgfsetstrokeopacity{0.800000}%
\pgfsetdash{}{0pt}%
\pgfpathmoveto{\pgfqpoint{4.635731in}{4.003360in}}%
\pgfpathlineto{\pgfqpoint{4.803960in}{4.277068in}}%
\pgfusepath{stroke}%
\end{pgfscope}%
\begin{pgfscope}%
\pgfpathrectangle{\pgfqpoint{0.570343in}{0.331635in}}{\pgfqpoint{9.300000in}{7.700000in}}%
\pgfusepath{clip}%
\pgfsetrectcap%
\pgfsetroundjoin%
\pgfsetlinewidth{1.505625pt}%
\definecolor{currentstroke}{rgb}{0.631373,0.788235,0.956863}%
\pgfsetstrokecolor{currentstroke}%
\pgfsetstrokeopacity{0.800000}%
\pgfsetdash{}{0pt}%
\pgfpathmoveto{\pgfqpoint{3.827407in}{3.304553in}}%
\pgfpathlineto{\pgfqpoint{4.803960in}{4.277068in}}%
\pgfusepath{stroke}%
\end{pgfscope}%
\begin{pgfscope}%
\pgfpathrectangle{\pgfqpoint{0.570343in}{0.331635in}}{\pgfqpoint{9.300000in}{7.700000in}}%
\pgfusepath{clip}%
\pgfsetrectcap%
\pgfsetroundjoin%
\pgfsetlinewidth{1.505625pt}%
\definecolor{currentstroke}{rgb}{0.631373,0.788235,0.956863}%
\pgfsetstrokecolor{currentstroke}%
\pgfsetstrokeopacity{0.800000}%
\pgfsetdash{}{0pt}%
\pgfpathmoveto{\pgfqpoint{3.605912in}{4.491467in}}%
\pgfpathlineto{\pgfqpoint{4.803960in}{4.277068in}}%
\pgfusepath{stroke}%
\end{pgfscope}%
\begin{pgfscope}%
\pgfpathrectangle{\pgfqpoint{0.570343in}{0.331635in}}{\pgfqpoint{9.300000in}{7.700000in}}%
\pgfusepath{clip}%
\pgfsetrectcap%
\pgfsetroundjoin%
\pgfsetlinewidth{1.505625pt}%
\definecolor{currentstroke}{rgb}{0.631373,0.788235,0.956863}%
\pgfsetstrokecolor{currentstroke}%
\pgfsetstrokeopacity{0.800000}%
\pgfsetdash{}{0pt}%
\pgfpathmoveto{\pgfqpoint{5.549912in}{4.187109in}}%
\pgfpathlineto{\pgfqpoint{4.803960in}{4.277068in}}%
\pgfusepath{stroke}%
\end{pgfscope}%
\begin{pgfscope}%
\pgfpathrectangle{\pgfqpoint{0.570343in}{0.331635in}}{\pgfqpoint{9.300000in}{7.700000in}}%
\pgfusepath{clip}%
\pgfsetrectcap%
\pgfsetroundjoin%
\pgfsetlinewidth{1.505625pt}%
\definecolor{currentstroke}{rgb}{0.631373,0.788235,0.956863}%
\pgfsetstrokecolor{currentstroke}%
\pgfsetstrokeopacity{0.800000}%
\pgfsetdash{}{0pt}%
\pgfpathmoveto{\pgfqpoint{6.429778in}{5.562008in}}%
\pgfpathlineto{\pgfqpoint{4.803960in}{4.277068in}}%
\pgfusepath{stroke}%
\end{pgfscope}%
\begin{pgfscope}%
\pgfpathrectangle{\pgfqpoint{0.570343in}{0.331635in}}{\pgfqpoint{9.300000in}{7.700000in}}%
\pgfusepath{clip}%
\pgfsetrectcap%
\pgfsetroundjoin%
\pgfsetlinewidth{1.505625pt}%
\definecolor{currentstroke}{rgb}{0.631373,0.788235,0.956863}%
\pgfsetstrokecolor{currentstroke}%
\pgfsetstrokeopacity{0.800000}%
\pgfsetdash{}{0pt}%
\pgfpathmoveto{\pgfqpoint{4.510732in}{2.730289in}}%
\pgfpathlineto{\pgfqpoint{4.803960in}{4.277068in}}%
\pgfusepath{stroke}%
\end{pgfscope}%
\begin{pgfscope}%
\pgfpathrectangle{\pgfqpoint{0.570343in}{0.331635in}}{\pgfqpoint{9.300000in}{7.700000in}}%
\pgfusepath{clip}%
\pgfsetrectcap%
\pgfsetroundjoin%
\pgfsetlinewidth{1.505625pt}%
\definecolor{currentstroke}{rgb}{0.631373,0.788235,0.956863}%
\pgfsetstrokecolor{currentstroke}%
\pgfsetstrokeopacity{0.800000}%
\pgfsetdash{}{0pt}%
\pgfpathmoveto{\pgfqpoint{5.386210in}{3.547338in}}%
\pgfpathlineto{\pgfqpoint{4.803960in}{4.277068in}}%
\pgfusepath{stroke}%
\end{pgfscope}%
\begin{pgfscope}%
\pgfpathrectangle{\pgfqpoint{0.570343in}{0.331635in}}{\pgfqpoint{9.300000in}{7.700000in}}%
\pgfusepath{clip}%
\pgfsetrectcap%
\pgfsetroundjoin%
\pgfsetlinewidth{1.505625pt}%
\definecolor{currentstroke}{rgb}{0.631373,0.788235,0.956863}%
\pgfsetstrokecolor{currentstroke}%
\pgfsetstrokeopacity{0.800000}%
\pgfsetdash{}{0pt}%
\pgfpathmoveto{\pgfqpoint{8.348041in}{5.497643in}}%
\pgfpathlineto{\pgfqpoint{4.803960in}{4.277068in}}%
\pgfusepath{stroke}%
\end{pgfscope}%
\begin{pgfscope}%
\pgfpathrectangle{\pgfqpoint{0.570343in}{0.331635in}}{\pgfqpoint{9.300000in}{7.700000in}}%
\pgfusepath{clip}%
\pgfsetrectcap%
\pgfsetroundjoin%
\pgfsetlinewidth{1.505625pt}%
\definecolor{currentstroke}{rgb}{0.631373,0.788235,0.956863}%
\pgfsetstrokecolor{currentstroke}%
\pgfsetstrokeopacity{0.800000}%
\pgfsetdash{}{0pt}%
\pgfpathmoveto{\pgfqpoint{5.507412in}{5.406482in}}%
\pgfpathlineto{\pgfqpoint{4.803960in}{4.277068in}}%
\pgfusepath{stroke}%
\end{pgfscope}%
\begin{pgfscope}%
\pgfpathrectangle{\pgfqpoint{0.570343in}{0.331635in}}{\pgfqpoint{9.300000in}{7.700000in}}%
\pgfusepath{clip}%
\pgfsetrectcap%
\pgfsetroundjoin%
\pgfsetlinewidth{1.505625pt}%
\definecolor{currentstroke}{rgb}{0.631373,0.788235,0.956863}%
\pgfsetstrokecolor{currentstroke}%
\pgfsetstrokeopacity{0.800000}%
\pgfsetdash{}{0pt}%
\pgfpathmoveto{\pgfqpoint{5.886021in}{6.276968in}}%
\pgfpathlineto{\pgfqpoint{4.803960in}{4.277068in}}%
\pgfusepath{stroke}%
\end{pgfscope}%
\begin{pgfscope}%
\pgfpathrectangle{\pgfqpoint{0.570343in}{0.331635in}}{\pgfqpoint{9.300000in}{7.700000in}}%
\pgfusepath{clip}%
\pgfsetrectcap%
\pgfsetroundjoin%
\pgfsetlinewidth{1.505625pt}%
\definecolor{currentstroke}{rgb}{0.631373,0.788235,0.956863}%
\pgfsetstrokecolor{currentstroke}%
\pgfsetstrokeopacity{0.800000}%
\pgfsetdash{}{0pt}%
\pgfpathmoveto{\pgfqpoint{6.703980in}{2.135452in}}%
\pgfpathlineto{\pgfqpoint{4.803960in}{4.277068in}}%
\pgfusepath{stroke}%
\end{pgfscope}%
\begin{pgfscope}%
\pgfpathrectangle{\pgfqpoint{0.570343in}{0.331635in}}{\pgfqpoint{9.300000in}{7.700000in}}%
\pgfusepath{clip}%
\pgfsetrectcap%
\pgfsetroundjoin%
\pgfsetlinewidth{1.505625pt}%
\definecolor{currentstroke}{rgb}{0.631373,0.788235,0.956863}%
\pgfsetstrokecolor{currentstroke}%
\pgfsetstrokeopacity{0.800000}%
\pgfsetdash{}{0pt}%
\pgfpathmoveto{\pgfqpoint{3.152829in}{3.811018in}}%
\pgfpathlineto{\pgfqpoint{4.803960in}{4.277068in}}%
\pgfusepath{stroke}%
\end{pgfscope}%
\begin{pgfscope}%
\pgfpathrectangle{\pgfqpoint{0.570343in}{0.331635in}}{\pgfqpoint{9.300000in}{7.700000in}}%
\pgfusepath{clip}%
\pgfsetrectcap%
\pgfsetroundjoin%
\pgfsetlinewidth{1.505625pt}%
\definecolor{currentstroke}{rgb}{0.631373,0.788235,0.956863}%
\pgfsetstrokecolor{currentstroke}%
\pgfsetstrokeopacity{0.800000}%
\pgfsetdash{}{0pt}%
\pgfpathmoveto{\pgfqpoint{4.382416in}{4.635624in}}%
\pgfpathlineto{\pgfqpoint{4.803960in}{4.277068in}}%
\pgfusepath{stroke}%
\end{pgfscope}%
\begin{pgfscope}%
\pgfpathrectangle{\pgfqpoint{0.570343in}{0.331635in}}{\pgfqpoint{9.300000in}{7.700000in}}%
\pgfusepath{clip}%
\pgfsetrectcap%
\pgfsetroundjoin%
\pgfsetlinewidth{1.505625pt}%
\definecolor{currentstroke}{rgb}{0.631373,0.788235,0.956863}%
\pgfsetstrokecolor{currentstroke}%
\pgfsetstrokeopacity{0.800000}%
\pgfsetdash{}{0pt}%
\pgfpathmoveto{\pgfqpoint{6.950477in}{6.304394in}}%
\pgfpathlineto{\pgfqpoint{4.803960in}{4.277068in}}%
\pgfusepath{stroke}%
\end{pgfscope}%
\begin{pgfscope}%
\pgfpathrectangle{\pgfqpoint{0.570343in}{0.331635in}}{\pgfqpoint{9.300000in}{7.700000in}}%
\pgfusepath{clip}%
\pgfsetrectcap%
\pgfsetroundjoin%
\pgfsetlinewidth{1.505625pt}%
\definecolor{currentstroke}{rgb}{0.631373,0.788235,0.956863}%
\pgfsetstrokecolor{currentstroke}%
\pgfsetstrokeopacity{0.800000}%
\pgfsetdash{}{0pt}%
\pgfpathmoveto{\pgfqpoint{2.893408in}{4.648818in}}%
\pgfpathlineto{\pgfqpoint{4.803960in}{4.277068in}}%
\pgfusepath{stroke}%
\end{pgfscope}%
\begin{pgfscope}%
\pgfpathrectangle{\pgfqpoint{0.570343in}{0.331635in}}{\pgfqpoint{9.300000in}{7.700000in}}%
\pgfusepath{clip}%
\pgfsetrectcap%
\pgfsetroundjoin%
\pgfsetlinewidth{1.505625pt}%
\definecolor{currentstroke}{rgb}{1.000000,0.705882,0.509804}%
\pgfsetstrokecolor{currentstroke}%
\pgfsetstrokeopacity{0.800000}%
\pgfsetdash{}{0pt}%
\pgfpathmoveto{\pgfqpoint{3.817222in}{6.024113in}}%
\pgfpathlineto{\pgfqpoint{5.224831in}{3.978419in}}%
\pgfusepath{stroke}%
\end{pgfscope}%
\begin{pgfscope}%
\pgfpathrectangle{\pgfqpoint{0.570343in}{0.331635in}}{\pgfqpoint{9.300000in}{7.700000in}}%
\pgfusepath{clip}%
\pgfsetrectcap%
\pgfsetroundjoin%
\pgfsetlinewidth{1.505625pt}%
\definecolor{currentstroke}{rgb}{1.000000,0.705882,0.509804}%
\pgfsetstrokecolor{currentstroke}%
\pgfsetstrokeopacity{0.800000}%
\pgfsetdash{}{0pt}%
\pgfpathmoveto{\pgfqpoint{7.659561in}{1.386197in}}%
\pgfpathlineto{\pgfqpoint{5.224831in}{3.978419in}}%
\pgfusepath{stroke}%
\end{pgfscope}%
\begin{pgfscope}%
\pgfpathrectangle{\pgfqpoint{0.570343in}{0.331635in}}{\pgfqpoint{9.300000in}{7.700000in}}%
\pgfusepath{clip}%
\pgfsetrectcap%
\pgfsetroundjoin%
\pgfsetlinewidth{1.505625pt}%
\definecolor{currentstroke}{rgb}{1.000000,0.705882,0.509804}%
\pgfsetstrokecolor{currentstroke}%
\pgfsetstrokeopacity{0.800000}%
\pgfsetdash{}{0pt}%
\pgfpathmoveto{\pgfqpoint{6.328447in}{4.305358in}}%
\pgfpathlineto{\pgfqpoint{5.224831in}{3.978419in}}%
\pgfusepath{stroke}%
\end{pgfscope}%
\begin{pgfscope}%
\pgfpathrectangle{\pgfqpoint{0.570343in}{0.331635in}}{\pgfqpoint{9.300000in}{7.700000in}}%
\pgfusepath{clip}%
\pgfsetrectcap%
\pgfsetroundjoin%
\pgfsetlinewidth{1.505625pt}%
\definecolor{currentstroke}{rgb}{1.000000,0.705882,0.509804}%
\pgfsetstrokecolor{currentstroke}%
\pgfsetstrokeopacity{0.800000}%
\pgfsetdash{}{0pt}%
\pgfpathmoveto{\pgfqpoint{8.131383in}{3.754090in}}%
\pgfpathlineto{\pgfqpoint{5.224831in}{3.978419in}}%
\pgfusepath{stroke}%
\end{pgfscope}%
\begin{pgfscope}%
\pgfpathrectangle{\pgfqpoint{0.570343in}{0.331635in}}{\pgfqpoint{9.300000in}{7.700000in}}%
\pgfusepath{clip}%
\pgfsetrectcap%
\pgfsetroundjoin%
\pgfsetlinewidth{1.505625pt}%
\definecolor{currentstroke}{rgb}{1.000000,0.705882,0.509804}%
\pgfsetstrokecolor{currentstroke}%
\pgfsetstrokeopacity{0.800000}%
\pgfsetdash{}{0pt}%
\pgfpathmoveto{\pgfqpoint{6.702968in}{4.905160in}}%
\pgfpathlineto{\pgfqpoint{5.224831in}{3.978419in}}%
\pgfusepath{stroke}%
\end{pgfscope}%
\begin{pgfscope}%
\pgfpathrectangle{\pgfqpoint{0.570343in}{0.331635in}}{\pgfqpoint{9.300000in}{7.700000in}}%
\pgfusepath{clip}%
\pgfsetrectcap%
\pgfsetroundjoin%
\pgfsetlinewidth{1.505625pt}%
\definecolor{currentstroke}{rgb}{1.000000,0.705882,0.509804}%
\pgfsetstrokecolor{currentstroke}%
\pgfsetstrokeopacity{0.800000}%
\pgfsetdash{}{0pt}%
\pgfpathmoveto{\pgfqpoint{5.355699in}{2.068245in}}%
\pgfpathlineto{\pgfqpoint{5.224831in}{3.978419in}}%
\pgfusepath{stroke}%
\end{pgfscope}%
\begin{pgfscope}%
\pgfpathrectangle{\pgfqpoint{0.570343in}{0.331635in}}{\pgfqpoint{9.300000in}{7.700000in}}%
\pgfusepath{clip}%
\pgfsetrectcap%
\pgfsetroundjoin%
\pgfsetlinewidth{1.505625pt}%
\definecolor{currentstroke}{rgb}{1.000000,0.705882,0.509804}%
\pgfsetstrokecolor{currentstroke}%
\pgfsetstrokeopacity{0.800000}%
\pgfsetdash{}{0pt}%
\pgfpathmoveto{\pgfqpoint{2.219532in}{1.487793in}}%
\pgfpathlineto{\pgfqpoint{5.224831in}{3.978419in}}%
\pgfusepath{stroke}%
\end{pgfscope}%
\begin{pgfscope}%
\pgfpathrectangle{\pgfqpoint{0.570343in}{0.331635in}}{\pgfqpoint{9.300000in}{7.700000in}}%
\pgfusepath{clip}%
\pgfsetrectcap%
\pgfsetroundjoin%
\pgfsetlinewidth{1.505625pt}%
\definecolor{currentstroke}{rgb}{1.000000,0.705882,0.509804}%
\pgfsetstrokecolor{currentstroke}%
\pgfsetstrokeopacity{0.800000}%
\pgfsetdash{}{0pt}%
\pgfpathmoveto{\pgfqpoint{4.514146in}{1.447847in}}%
\pgfpathlineto{\pgfqpoint{5.224831in}{3.978419in}}%
\pgfusepath{stroke}%
\end{pgfscope}%
\begin{pgfscope}%
\pgfpathrectangle{\pgfqpoint{0.570343in}{0.331635in}}{\pgfqpoint{9.300000in}{7.700000in}}%
\pgfusepath{clip}%
\pgfsetrectcap%
\pgfsetroundjoin%
\pgfsetlinewidth{1.505625pt}%
\definecolor{currentstroke}{rgb}{1.000000,0.705882,0.509804}%
\pgfsetstrokecolor{currentstroke}%
\pgfsetstrokeopacity{0.800000}%
\pgfsetdash{}{0pt}%
\pgfpathmoveto{\pgfqpoint{5.290381in}{2.873994in}}%
\pgfpathlineto{\pgfqpoint{5.224831in}{3.978419in}}%
\pgfusepath{stroke}%
\end{pgfscope}%
\begin{pgfscope}%
\pgfpathrectangle{\pgfqpoint{0.570343in}{0.331635in}}{\pgfqpoint{9.300000in}{7.700000in}}%
\pgfusepath{clip}%
\pgfsetrectcap%
\pgfsetroundjoin%
\pgfsetlinewidth{1.505625pt}%
\definecolor{currentstroke}{rgb}{1.000000,0.705882,0.509804}%
\pgfsetstrokecolor{currentstroke}%
\pgfsetstrokeopacity{0.800000}%
\pgfsetdash{}{0pt}%
\pgfpathmoveto{\pgfqpoint{4.901182in}{6.223846in}}%
\pgfpathlineto{\pgfqpoint{5.224831in}{3.978419in}}%
\pgfusepath{stroke}%
\end{pgfscope}%
\begin{pgfscope}%
\pgfpathrectangle{\pgfqpoint{0.570343in}{0.331635in}}{\pgfqpoint{9.300000in}{7.700000in}}%
\pgfusepath{clip}%
\pgfsetrectcap%
\pgfsetroundjoin%
\pgfsetlinewidth{1.505625pt}%
\definecolor{currentstroke}{rgb}{1.000000,0.705882,0.509804}%
\pgfsetstrokecolor{currentstroke}%
\pgfsetstrokeopacity{0.800000}%
\pgfsetdash{}{0pt}%
\pgfpathmoveto{\pgfqpoint{8.177594in}{2.927999in}}%
\pgfpathlineto{\pgfqpoint{5.224831in}{3.978419in}}%
\pgfusepath{stroke}%
\end{pgfscope}%
\begin{pgfscope}%
\pgfpathrectangle{\pgfqpoint{0.570343in}{0.331635in}}{\pgfqpoint{9.300000in}{7.700000in}}%
\pgfusepath{clip}%
\pgfsetrectcap%
\pgfsetroundjoin%
\pgfsetlinewidth{1.505625pt}%
\definecolor{currentstroke}{rgb}{1.000000,0.705882,0.509804}%
\pgfsetstrokecolor{currentstroke}%
\pgfsetstrokeopacity{0.800000}%
\pgfsetdash{}{0pt}%
\pgfpathmoveto{\pgfqpoint{7.239046in}{7.681635in}}%
\pgfpathlineto{\pgfqpoint{5.224831in}{3.978419in}}%
\pgfusepath{stroke}%
\end{pgfscope}%
\begin{pgfscope}%
\pgfpathrectangle{\pgfqpoint{0.570343in}{0.331635in}}{\pgfqpoint{9.300000in}{7.700000in}}%
\pgfusepath{clip}%
\pgfsetrectcap%
\pgfsetroundjoin%
\pgfsetlinewidth{1.505625pt}%
\definecolor{currentstroke}{rgb}{1.000000,0.705882,0.509804}%
\pgfsetstrokecolor{currentstroke}%
\pgfsetstrokeopacity{0.800000}%
\pgfsetdash{}{0pt}%
\pgfpathmoveto{\pgfqpoint{3.182112in}{0.681635in}}%
\pgfpathlineto{\pgfqpoint{5.224831in}{3.978419in}}%
\pgfusepath{stroke}%
\end{pgfscope}%
\begin{pgfscope}%
\pgfpathrectangle{\pgfqpoint{0.570343in}{0.331635in}}{\pgfqpoint{9.300000in}{7.700000in}}%
\pgfusepath{clip}%
\pgfsetrectcap%
\pgfsetroundjoin%
\pgfsetlinewidth{1.505625pt}%
\definecolor{currentstroke}{rgb}{1.000000,0.705882,0.509804}%
\pgfsetstrokecolor{currentstroke}%
\pgfsetstrokeopacity{0.800000}%
\pgfsetdash{}{0pt}%
\pgfpathmoveto{\pgfqpoint{7.238874in}{5.471329in}}%
\pgfpathlineto{\pgfqpoint{5.224831in}{3.978419in}}%
\pgfusepath{stroke}%
\end{pgfscope}%
\begin{pgfscope}%
\pgfpathrectangle{\pgfqpoint{0.570343in}{0.331635in}}{\pgfqpoint{9.300000in}{7.700000in}}%
\pgfusepath{clip}%
\pgfsetrectcap%
\pgfsetroundjoin%
\pgfsetlinewidth{1.505625pt}%
\definecolor{currentstroke}{rgb}{1.000000,0.705882,0.509804}%
\pgfsetstrokecolor{currentstroke}%
\pgfsetstrokeopacity{0.800000}%
\pgfsetdash{}{0pt}%
\pgfpathmoveto{\pgfqpoint{3.375695in}{1.918885in}}%
\pgfpathlineto{\pgfqpoint{5.224831in}{3.978419in}}%
\pgfusepath{stroke}%
\end{pgfscope}%
\begin{pgfscope}%
\pgfpathrectangle{\pgfqpoint{0.570343in}{0.331635in}}{\pgfqpoint{9.300000in}{7.700000in}}%
\pgfusepath{clip}%
\pgfsetrectcap%
\pgfsetroundjoin%
\pgfsetlinewidth{1.505625pt}%
\definecolor{currentstroke}{rgb}{1.000000,0.705882,0.509804}%
\pgfsetstrokecolor{currentstroke}%
\pgfsetstrokeopacity{0.800000}%
\pgfsetdash{}{0pt}%
\pgfpathmoveto{\pgfqpoint{5.763337in}{1.311076in}}%
\pgfpathlineto{\pgfqpoint{5.224831in}{3.978419in}}%
\pgfusepath{stroke}%
\end{pgfscope}%
\begin{pgfscope}%
\pgfpathrectangle{\pgfqpoint{0.570343in}{0.331635in}}{\pgfqpoint{9.300000in}{7.700000in}}%
\pgfusepath{clip}%
\pgfsetrectcap%
\pgfsetroundjoin%
\pgfsetlinewidth{1.505625pt}%
\definecolor{currentstroke}{rgb}{1.000000,0.705882,0.509804}%
\pgfsetstrokecolor{currentstroke}%
\pgfsetstrokeopacity{0.800000}%
\pgfsetdash{}{0pt}%
\pgfpathmoveto{\pgfqpoint{2.235429in}{2.335141in}}%
\pgfpathlineto{\pgfqpoint{5.224831in}{3.978419in}}%
\pgfusepath{stroke}%
\end{pgfscope}%
\begin{pgfscope}%
\pgfpathrectangle{\pgfqpoint{0.570343in}{0.331635in}}{\pgfqpoint{9.300000in}{7.700000in}}%
\pgfusepath{clip}%
\pgfsetrectcap%
\pgfsetroundjoin%
\pgfsetlinewidth{1.505625pt}%
\definecolor{currentstroke}{rgb}{1.000000,0.705882,0.509804}%
\pgfsetstrokecolor{currentstroke}%
\pgfsetstrokeopacity{0.800000}%
\pgfsetdash{}{0pt}%
\pgfpathmoveto{\pgfqpoint{8.228735in}{6.567215in}}%
\pgfpathlineto{\pgfqpoint{5.224831in}{3.978419in}}%
\pgfusepath{stroke}%
\end{pgfscope}%
\begin{pgfscope}%
\pgfpathrectangle{\pgfqpoint{0.570343in}{0.331635in}}{\pgfqpoint{9.300000in}{7.700000in}}%
\pgfusepath{clip}%
\pgfsetrectcap%
\pgfsetroundjoin%
\pgfsetlinewidth{1.505625pt}%
\definecolor{currentstroke}{rgb}{1.000000,0.705882,0.509804}%
\pgfsetstrokecolor{currentstroke}%
\pgfsetstrokeopacity{0.800000}%
\pgfsetdash{}{0pt}%
\pgfpathmoveto{\pgfqpoint{4.520708in}{7.423071in}}%
\pgfpathlineto{\pgfqpoint{5.224831in}{3.978419in}}%
\pgfusepath{stroke}%
\end{pgfscope}%
\begin{pgfscope}%
\pgfpathrectangle{\pgfqpoint{0.570343in}{0.331635in}}{\pgfqpoint{9.300000in}{7.700000in}}%
\pgfusepath{clip}%
\pgfsetrectcap%
\pgfsetroundjoin%
\pgfsetlinewidth{1.505625pt}%
\definecolor{currentstroke}{rgb}{1.000000,0.705882,0.509804}%
\pgfsetstrokecolor{currentstroke}%
\pgfsetstrokeopacity{0.800000}%
\pgfsetdash{}{0pt}%
\pgfpathmoveto{\pgfqpoint{5.816987in}{7.332432in}}%
\pgfpathlineto{\pgfqpoint{5.224831in}{3.978419in}}%
\pgfusepath{stroke}%
\end{pgfscope}%
\begin{pgfscope}%
\pgfpathrectangle{\pgfqpoint{0.570343in}{0.331635in}}{\pgfqpoint{9.300000in}{7.700000in}}%
\pgfusepath{clip}%
\pgfsetrectcap%
\pgfsetroundjoin%
\pgfsetlinewidth{1.505625pt}%
\definecolor{currentstroke}{rgb}{1.000000,0.705882,0.509804}%
\pgfsetstrokecolor{currentstroke}%
\pgfsetstrokeopacity{0.800000}%
\pgfsetdash{}{0pt}%
\pgfpathmoveto{\pgfqpoint{1.860590in}{5.284247in}}%
\pgfpathlineto{\pgfqpoint{5.224831in}{3.978419in}}%
\pgfusepath{stroke}%
\end{pgfscope}%
\begin{pgfscope}%
\pgfpathrectangle{\pgfqpoint{0.570343in}{0.331635in}}{\pgfqpoint{9.300000in}{7.700000in}}%
\pgfusepath{clip}%
\pgfsetrectcap%
\pgfsetroundjoin%
\pgfsetlinewidth{1.505625pt}%
\definecolor{currentstroke}{rgb}{1.000000,0.705882,0.509804}%
\pgfsetstrokecolor{currentstroke}%
\pgfsetstrokeopacity{0.800000}%
\pgfsetdash{}{0pt}%
\pgfpathmoveto{\pgfqpoint{7.102567in}{4.141837in}}%
\pgfpathlineto{\pgfqpoint{5.224831in}{3.978419in}}%
\pgfusepath{stroke}%
\end{pgfscope}%
\begin{pgfscope}%
\pgfpathrectangle{\pgfqpoint{0.570343in}{0.331635in}}{\pgfqpoint{9.300000in}{7.700000in}}%
\pgfusepath{clip}%
\pgfsetrectcap%
\pgfsetroundjoin%
\pgfsetlinewidth{1.505625pt}%
\definecolor{currentstroke}{rgb}{1.000000,0.705882,0.509804}%
\pgfsetstrokecolor{currentstroke}%
\pgfsetstrokeopacity{0.800000}%
\pgfsetdash{}{0pt}%
\pgfpathmoveto{\pgfqpoint{1.082655in}{2.191589in}}%
\pgfpathlineto{\pgfqpoint{5.224831in}{3.978419in}}%
\pgfusepath{stroke}%
\end{pgfscope}%
\begin{pgfscope}%
\pgfpathrectangle{\pgfqpoint{0.570343in}{0.331635in}}{\pgfqpoint{9.300000in}{7.700000in}}%
\pgfusepath{clip}%
\pgfsetrectcap%
\pgfsetroundjoin%
\pgfsetlinewidth{1.505625pt}%
\definecolor{currentstroke}{rgb}{1.000000,0.705882,0.509804}%
\pgfsetstrokecolor{currentstroke}%
\pgfsetstrokeopacity{0.800000}%
\pgfsetdash{}{0pt}%
\pgfpathmoveto{\pgfqpoint{9.447616in}{3.338121in}}%
\pgfpathlineto{\pgfqpoint{5.224831in}{3.978419in}}%
\pgfusepath{stroke}%
\end{pgfscope}%
\begin{pgfscope}%
\pgfpathrectangle{\pgfqpoint{0.570343in}{0.331635in}}{\pgfqpoint{9.300000in}{7.700000in}}%
\pgfusepath{clip}%
\pgfsetrectcap%
\pgfsetroundjoin%
\pgfsetlinewidth{1.505625pt}%
\definecolor{currentstroke}{rgb}{1.000000,0.705882,0.509804}%
\pgfsetstrokecolor{currentstroke}%
\pgfsetstrokeopacity{0.800000}%
\pgfsetdash{}{0pt}%
\pgfpathmoveto{\pgfqpoint{3.829395in}{5.193151in}}%
\pgfpathlineto{\pgfqpoint{5.224831in}{3.978419in}}%
\pgfusepath{stroke}%
\end{pgfscope}%
\begin{pgfscope}%
\pgfpathrectangle{\pgfqpoint{0.570343in}{0.331635in}}{\pgfqpoint{9.300000in}{7.700000in}}%
\pgfusepath{clip}%
\pgfsetrectcap%
\pgfsetroundjoin%
\pgfsetlinewidth{1.505625pt}%
\definecolor{currentstroke}{rgb}{1.000000,0.705882,0.509804}%
\pgfsetstrokecolor{currentstroke}%
\pgfsetstrokeopacity{0.800000}%
\pgfsetdash{}{0pt}%
\pgfpathmoveto{\pgfqpoint{2.176389in}{4.216423in}}%
\pgfpathlineto{\pgfqpoint{5.224831in}{3.978419in}}%
\pgfusepath{stroke}%
\end{pgfscope}%
\begin{pgfscope}%
\pgfpathrectangle{\pgfqpoint{0.570343in}{0.331635in}}{\pgfqpoint{9.300000in}{7.700000in}}%
\pgfusepath{clip}%
\pgfsetrectcap%
\pgfsetroundjoin%
\pgfsetlinewidth{1.505625pt}%
\definecolor{currentstroke}{rgb}{1.000000,0.705882,0.509804}%
\pgfsetstrokecolor{currentstroke}%
\pgfsetstrokeopacity{0.800000}%
\pgfsetdash{}{0pt}%
\pgfpathmoveto{\pgfqpoint{7.119633in}{3.398240in}}%
\pgfpathlineto{\pgfqpoint{5.224831in}{3.978419in}}%
\pgfusepath{stroke}%
\end{pgfscope}%
\begin{pgfscope}%
\pgfpathrectangle{\pgfqpoint{0.570343in}{0.331635in}}{\pgfqpoint{9.300000in}{7.700000in}}%
\pgfusepath{clip}%
\pgfsetrectcap%
\pgfsetroundjoin%
\pgfsetlinewidth{1.505625pt}%
\definecolor{currentstroke}{rgb}{1.000000,0.705882,0.509804}%
\pgfsetstrokecolor{currentstroke}%
\pgfsetstrokeopacity{0.800000}%
\pgfsetdash{}{0pt}%
\pgfpathmoveto{\pgfqpoint{2.977371in}{5.505063in}}%
\pgfpathlineto{\pgfqpoint{5.224831in}{3.978419in}}%
\pgfusepath{stroke}%
\end{pgfscope}%
\begin{pgfscope}%
\pgfsetrectcap%
\pgfsetmiterjoin%
\pgfsetlinewidth{0.803000pt}%
\definecolor{currentstroke}{rgb}{0.000000,0.000000,0.000000}%
\pgfsetstrokecolor{currentstroke}%
\pgfsetdash{}{0pt}%
\pgfpathmoveto{\pgfqpoint{0.570343in}{0.331635in}}%
\pgfpathlineto{\pgfqpoint{0.570343in}{8.031635in}}%
\pgfusepath{stroke}%
\end{pgfscope}%
\begin{pgfscope}%
\pgfsetrectcap%
\pgfsetmiterjoin%
\pgfsetlinewidth{0.803000pt}%
\definecolor{currentstroke}{rgb}{0.000000,0.000000,0.000000}%
\pgfsetstrokecolor{currentstroke}%
\pgfsetdash{}{0pt}%
\pgfpathmoveto{\pgfqpoint{9.870343in}{0.331635in}}%
\pgfpathlineto{\pgfqpoint{9.870343in}{8.031635in}}%
\pgfusepath{stroke}%
\end{pgfscope}%
\begin{pgfscope}%
\pgfsetrectcap%
\pgfsetmiterjoin%
\pgfsetlinewidth{0.803000pt}%
\definecolor{currentstroke}{rgb}{0.000000,0.000000,0.000000}%
\pgfsetstrokecolor{currentstroke}%
\pgfsetdash{}{0pt}%
\pgfpathmoveto{\pgfqpoint{0.570343in}{0.331635in}}%
\pgfpathlineto{\pgfqpoint{9.870343in}{0.331635in}}%
\pgfusepath{stroke}%
\end{pgfscope}%
\begin{pgfscope}%
\pgfsetrectcap%
\pgfsetmiterjoin%
\pgfsetlinewidth{0.803000pt}%
\definecolor{currentstroke}{rgb}{0.000000,0.000000,0.000000}%
\pgfsetstrokecolor{currentstroke}%
\pgfsetdash{}{0pt}%
\pgfpathmoveto{\pgfqpoint{0.570343in}{8.031635in}}%
\pgfpathlineto{\pgfqpoint{9.870343in}{8.031635in}}%
\pgfusepath{stroke}%
\end{pgfscope}%
\begin{pgfscope}%
\definecolor{textcolor}{rgb}{0.000000,0.000000,0.000000}%
\pgfsetstrokecolor{textcolor}%
\pgfsetfillcolor{textcolor}%
\pgftext[x=5.220343in,y=8.114968in,,base]{\color{textcolor}\sffamily\fontsize{12.000000}{14.400000}\selectfont Photo-Realistic Images}%
\end{pgfscope}%
\begin{pgfscope}%
\pgfsetbuttcap%
\pgfsetmiterjoin%
\definecolor{currentfill}{rgb}{1.000000,1.000000,1.000000}%
\pgfsetfillcolor{currentfill}%
\pgfsetfillopacity{0.800000}%
\pgfsetlinewidth{1.003750pt}%
\definecolor{currentstroke}{rgb}{0.800000,0.800000,0.800000}%
\pgfsetstrokecolor{currentstroke}%
\pgfsetstrokeopacity{0.800000}%
\pgfsetdash{}{0pt}%
\pgfpathmoveto{\pgfqpoint{9.967566in}{3.956944in}}%
\pgfpathlineto{\pgfqpoint{11.246496in}{3.956944in}}%
\pgfpathquadraticcurveto{\pgfqpoint{11.274274in}{3.956944in}}{\pgfqpoint{11.274274in}{3.984722in}}%
\pgfpathlineto{\pgfqpoint{11.274274in}{4.378548in}}%
\pgfpathquadraticcurveto{\pgfqpoint{11.274274in}{4.406326in}}{\pgfqpoint{11.246496in}{4.406326in}}%
\pgfpathlineto{\pgfqpoint{9.967566in}{4.406326in}}%
\pgfpathquadraticcurveto{\pgfqpoint{9.939788in}{4.406326in}}{\pgfqpoint{9.939788in}{4.378548in}}%
\pgfpathlineto{\pgfqpoint{9.939788in}{3.984722in}}%
\pgfpathquadraticcurveto{\pgfqpoint{9.939788in}{3.956944in}}{\pgfqpoint{9.967566in}{3.956944in}}%
\pgfpathclose%
\pgfusepath{stroke,fill}%
\end{pgfscope}%
\begin{pgfscope}%
\pgfsetbuttcap%
\pgfsetroundjoin%
\definecolor{currentfill}{rgb}{0.631373,0.788235,0.956863}%
\pgfsetfillcolor{currentfill}%
\pgfsetlinewidth{1.003750pt}%
\definecolor{currentstroke}{rgb}{0.631373,0.788235,0.956863}%
\pgfsetstrokecolor{currentstroke}%
\pgfsetdash{}{0pt}%
\pgfsys@defobject{currentmarker}{\pgfqpoint{-0.041667in}{-0.041667in}}{\pgfqpoint{0.041667in}{0.041667in}}{%
\pgfpathmoveto{\pgfqpoint{0.000000in}{-0.041667in}}%
\pgfpathcurveto{\pgfqpoint{0.011050in}{-0.041667in}}{\pgfqpoint{0.021649in}{-0.037276in}}{\pgfqpoint{0.029463in}{-0.029463in}}%
\pgfpathcurveto{\pgfqpoint{0.037276in}{-0.021649in}}{\pgfqpoint{0.041667in}{-0.011050in}}{\pgfqpoint{0.041667in}{0.000000in}}%
\pgfpathcurveto{\pgfqpoint{0.041667in}{0.011050in}}{\pgfqpoint{0.037276in}{0.021649in}}{\pgfqpoint{0.029463in}{0.029463in}}%
\pgfpathcurveto{\pgfqpoint{0.021649in}{0.037276in}}{\pgfqpoint{0.011050in}{0.041667in}}{\pgfqpoint{0.000000in}{0.041667in}}%
\pgfpathcurveto{\pgfqpoint{-0.011050in}{0.041667in}}{\pgfqpoint{-0.021649in}{0.037276in}}{\pgfqpoint{-0.029463in}{0.029463in}}%
\pgfpathcurveto{\pgfqpoint{-0.037276in}{0.021649in}}{\pgfqpoint{-0.041667in}{0.011050in}}{\pgfqpoint{-0.041667in}{0.000000in}}%
\pgfpathcurveto{\pgfqpoint{-0.041667in}{-0.011050in}}{\pgfqpoint{-0.037276in}{-0.021649in}}{\pgfqpoint{-0.029463in}{-0.029463in}}%
\pgfpathcurveto{\pgfqpoint{-0.021649in}{-0.037276in}}{\pgfqpoint{-0.011050in}{-0.041667in}}{\pgfqpoint{0.000000in}{-0.041667in}}%
\pgfpathclose%
\pgfusepath{stroke,fill}%
}%
\begin{pgfscope}%
\pgfsys@transformshift{10.134232in}{4.281705in}%
\pgfsys@useobject{currentmarker}{}%
\end{pgfscope}%
\end{pgfscope}%
\begin{pgfscope}%
\definecolor{textcolor}{rgb}{0.000000,0.000000,0.000000}%
\pgfsetstrokecolor{textcolor}%
\pgfsetfillcolor{textcolor}%
\pgftext[x=10.384232in,y=4.245247in,left,base]{\color{textcolor}\sffamily\fontsize{10.000000}{12.000000}\selectfont blenderproc}%
\end{pgfscope}%
\begin{pgfscope}%
\pgfsetbuttcap%
\pgfsetroundjoin%
\definecolor{currentfill}{rgb}{1.000000,0.705882,0.509804}%
\pgfsetfillcolor{currentfill}%
\pgfsetlinewidth{1.003750pt}%
\definecolor{currentstroke}{rgb}{1.000000,0.705882,0.509804}%
\pgfsetstrokecolor{currentstroke}%
\pgfsetdash{}{0pt}%
\pgfsys@defobject{currentmarker}{\pgfqpoint{-0.041667in}{-0.041667in}}{\pgfqpoint{0.041667in}{0.041667in}}{%
\pgfpathmoveto{\pgfqpoint{0.000000in}{-0.041667in}}%
\pgfpathcurveto{\pgfqpoint{0.011050in}{-0.041667in}}{\pgfqpoint{0.021649in}{-0.037276in}}{\pgfqpoint{0.029463in}{-0.029463in}}%
\pgfpathcurveto{\pgfqpoint{0.037276in}{-0.021649in}}{\pgfqpoint{0.041667in}{-0.011050in}}{\pgfqpoint{0.041667in}{0.000000in}}%
\pgfpathcurveto{\pgfqpoint{0.041667in}{0.011050in}}{\pgfqpoint{0.037276in}{0.021649in}}{\pgfqpoint{0.029463in}{0.029463in}}%
\pgfpathcurveto{\pgfqpoint{0.021649in}{0.037276in}}{\pgfqpoint{0.011050in}{0.041667in}}{\pgfqpoint{0.000000in}{0.041667in}}%
\pgfpathcurveto{\pgfqpoint{-0.011050in}{0.041667in}}{\pgfqpoint{-0.021649in}{0.037276in}}{\pgfqpoint{-0.029463in}{0.029463in}}%
\pgfpathcurveto{\pgfqpoint{-0.037276in}{0.021649in}}{\pgfqpoint{-0.041667in}{0.011050in}}{\pgfqpoint{-0.041667in}{0.000000in}}%
\pgfpathcurveto{\pgfqpoint{-0.041667in}{-0.011050in}}{\pgfqpoint{-0.037276in}{-0.021649in}}{\pgfqpoint{-0.029463in}{-0.029463in}}%
\pgfpathcurveto{\pgfqpoint{-0.021649in}{-0.037276in}}{\pgfqpoint{-0.011050in}{-0.041667in}}{\pgfqpoint{0.000000in}{-0.041667in}}%
\pgfpathclose%
\pgfusepath{stroke,fill}%
}%
\begin{pgfscope}%
\pgfsys@transformshift{10.134232in}{4.077848in}%
\pgfsys@useobject{currentmarker}{}%
\end{pgfscope}%
\end{pgfscope}%
\begin{pgfscope}%
\definecolor{textcolor}{rgb}{0.000000,0.000000,0.000000}%
\pgfsetstrokecolor{textcolor}%
\pgfsetfillcolor{textcolor}%
\pgftext[x=10.384232in,y=4.041390in,left,base]{\color{textcolor}\sffamily\fontsize{10.000000}{12.000000}\selectfont pix3d}%
\end{pgfscope}%
\end{pgfpicture}%
\makeatother%
\endgroup%
}
    \resizebox{0.49\linewidth}{5cm}{%% Creator: Matplotlib, PGF backend
%%
%% To include the figure in your LaTeX document, write
%%   \input{<filename>.pgf}
%%
%% Make sure the required packages are loaded in your preamble
%%   \usepackage{pgf}
%%
%% Figures using additional raster images can only be included by \input if
%% they are in the same directory as the main LaTeX file. For loading figures
%% from other directories you can use the `import` package
%%   \usepackage{import}
%%
%% and then include the figures with
%%   \import{<path to file>}{<filename>.pgf}
%%
%% Matplotlib used the following preamble
%%   \usepackage{fontspec}
%%   \setmainfont{DejaVuSerif.ttf}[Path=\detokenize{/Users/apple/opt/anaconda3/envs/kaolin/lib/python3.7/site-packages/matplotlib/mpl-data/fonts/ttf/}]
%%   \setsansfont{DejaVuSans.ttf}[Path=\detokenize{/Users/apple/opt/anaconda3/envs/kaolin/lib/python3.7/site-packages/matplotlib/mpl-data/fonts/ttf/}]
%%   \setmonofont{DejaVuSansMono.ttf}[Path=\detokenize{/Users/apple/opt/anaconda3/envs/kaolin/lib/python3.7/site-packages/matplotlib/mpl-data/fonts/ttf/}]
%%
\begingroup%
\makeatletter%
\begin{pgfpicture}%
\pgfpathrectangle{\pgfpointorigin}{\pgfqpoint{11.036411in}{8.341596in}}%
\pgfusepath{use as bounding box, clip}%
\begin{pgfscope}%
\pgfsetbuttcap%
\pgfsetmiterjoin%
\definecolor{currentfill}{rgb}{1.000000,1.000000,1.000000}%
\pgfsetfillcolor{currentfill}%
\pgfsetlinewidth{0.000000pt}%
\definecolor{currentstroke}{rgb}{1.000000,1.000000,1.000000}%
\pgfsetstrokecolor{currentstroke}%
\pgfsetdash{}{0pt}%
\pgfpathmoveto{\pgfqpoint{0.000000in}{0.000000in}}%
\pgfpathlineto{\pgfqpoint{11.036411in}{0.000000in}}%
\pgfpathlineto{\pgfqpoint{11.036411in}{8.341596in}}%
\pgfpathlineto{\pgfqpoint{0.000000in}{8.341596in}}%
\pgfpathclose%
\pgfusepath{fill}%
\end{pgfscope}%
\begin{pgfscope}%
\pgfsetbuttcap%
\pgfsetmiterjoin%
\definecolor{currentfill}{rgb}{1.000000,1.000000,1.000000}%
\pgfsetfillcolor{currentfill}%
\pgfsetlinewidth{0.000000pt}%
\definecolor{currentstroke}{rgb}{0.000000,0.000000,0.000000}%
\pgfsetstrokecolor{currentstroke}%
\pgfsetstrokeopacity{0.000000}%
\pgfsetdash{}{0pt}%
\pgfpathmoveto{\pgfqpoint{0.570343in}{0.331635in}}%
\pgfpathlineto{\pgfqpoint{9.870343in}{0.331635in}}%
\pgfpathlineto{\pgfqpoint{9.870343in}{8.031635in}}%
\pgfpathlineto{\pgfqpoint{0.570343in}{8.031635in}}%
\pgfpathclose%
\pgfusepath{fill}%
\end{pgfscope}%
\begin{pgfscope}%
\pgfpathrectangle{\pgfqpoint{0.570343in}{0.331635in}}{\pgfqpoint{9.300000in}{7.700000in}}%
\pgfusepath{clip}%
\pgfsetbuttcap%
\pgfsetroundjoin%
\definecolor{currentfill}{rgb}{0.631373,0.788235,0.956863}%
\pgfsetfillcolor{currentfill}%
\pgfsetlinewidth{0.481800pt}%
\definecolor{currentstroke}{rgb}{1.000000,1.000000,1.000000}%
\pgfsetstrokecolor{currentstroke}%
\pgfsetdash{}{0pt}%
\pgfpathmoveto{\pgfqpoint{2.375246in}{6.898172in}}%
\pgfpathcurveto{\pgfqpoint{2.386296in}{6.898172in}}{\pgfqpoint{2.396896in}{6.902562in}}{\pgfqpoint{2.404709in}{6.910376in}}%
\pgfpathcurveto{\pgfqpoint{2.412523in}{6.918190in}}{\pgfqpoint{2.416913in}{6.928789in}}{\pgfqpoint{2.416913in}{6.939839in}}%
\pgfpathcurveto{\pgfqpoint{2.416913in}{6.950889in}}{\pgfqpoint{2.412523in}{6.961488in}}{\pgfqpoint{2.404709in}{6.969302in}}%
\pgfpathcurveto{\pgfqpoint{2.396896in}{6.977115in}}{\pgfqpoint{2.386296in}{6.981506in}}{\pgfqpoint{2.375246in}{6.981506in}}%
\pgfpathcurveto{\pgfqpoint{2.364196in}{6.981506in}}{\pgfqpoint{2.353597in}{6.977115in}}{\pgfqpoint{2.345784in}{6.969302in}}%
\pgfpathcurveto{\pgfqpoint{2.337970in}{6.961488in}}{\pgfqpoint{2.333580in}{6.950889in}}{\pgfqpoint{2.333580in}{6.939839in}}%
\pgfpathcurveto{\pgfqpoint{2.333580in}{6.928789in}}{\pgfqpoint{2.337970in}{6.918190in}}{\pgfqpoint{2.345784in}{6.910376in}}%
\pgfpathcurveto{\pgfqpoint{2.353597in}{6.902562in}}{\pgfqpoint{2.364196in}{6.898172in}}{\pgfqpoint{2.375246in}{6.898172in}}%
\pgfpathclose%
\pgfusepath{stroke,fill}%
\end{pgfscope}%
\begin{pgfscope}%
\pgfpathrectangle{\pgfqpoint{0.570343in}{0.331635in}}{\pgfqpoint{9.300000in}{7.700000in}}%
\pgfusepath{clip}%
\pgfsetbuttcap%
\pgfsetroundjoin%
\definecolor{currentfill}{rgb}{0.631373,0.788235,0.956863}%
\pgfsetfillcolor{currentfill}%
\pgfsetlinewidth{0.481800pt}%
\definecolor{currentstroke}{rgb}{1.000000,1.000000,1.000000}%
\pgfsetstrokecolor{currentstroke}%
\pgfsetdash{}{0pt}%
\pgfpathmoveto{\pgfqpoint{3.492960in}{4.419701in}}%
\pgfpathcurveto{\pgfqpoint{3.504010in}{4.419701in}}{\pgfqpoint{3.514609in}{4.424091in}}{\pgfqpoint{3.522422in}{4.431905in}}%
\pgfpathcurveto{\pgfqpoint{3.530236in}{4.439718in}}{\pgfqpoint{3.534626in}{4.450317in}}{\pgfqpoint{3.534626in}{4.461367in}}%
\pgfpathcurveto{\pgfqpoint{3.534626in}{4.472418in}}{\pgfqpoint{3.530236in}{4.483017in}}{\pgfqpoint{3.522422in}{4.490830in}}%
\pgfpathcurveto{\pgfqpoint{3.514609in}{4.498644in}}{\pgfqpoint{3.504010in}{4.503034in}}{\pgfqpoint{3.492960in}{4.503034in}}%
\pgfpathcurveto{\pgfqpoint{3.481909in}{4.503034in}}{\pgfqpoint{3.471310in}{4.498644in}}{\pgfqpoint{3.463497in}{4.490830in}}%
\pgfpathcurveto{\pgfqpoint{3.455683in}{4.483017in}}{\pgfqpoint{3.451293in}{4.472418in}}{\pgfqpoint{3.451293in}{4.461367in}}%
\pgfpathcurveto{\pgfqpoint{3.451293in}{4.450317in}}{\pgfqpoint{3.455683in}{4.439718in}}{\pgfqpoint{3.463497in}{4.431905in}}%
\pgfpathcurveto{\pgfqpoint{3.471310in}{4.424091in}}{\pgfqpoint{3.481909in}{4.419701in}}{\pgfqpoint{3.492960in}{4.419701in}}%
\pgfpathclose%
\pgfusepath{stroke,fill}%
\end{pgfscope}%
\begin{pgfscope}%
\pgfpathrectangle{\pgfqpoint{0.570343in}{0.331635in}}{\pgfqpoint{9.300000in}{7.700000in}}%
\pgfusepath{clip}%
\pgfsetbuttcap%
\pgfsetroundjoin%
\definecolor{currentfill}{rgb}{0.631373,0.788235,0.956863}%
\pgfsetfillcolor{currentfill}%
\pgfsetlinewidth{0.481800pt}%
\definecolor{currentstroke}{rgb}{1.000000,1.000000,1.000000}%
\pgfsetstrokecolor{currentstroke}%
\pgfsetdash{}{0pt}%
\pgfpathmoveto{\pgfqpoint{5.673563in}{5.030838in}}%
\pgfpathcurveto{\pgfqpoint{5.684613in}{5.030838in}}{\pgfqpoint{5.695212in}{5.035228in}}{\pgfqpoint{5.703026in}{5.043042in}}%
\pgfpathcurveto{\pgfqpoint{5.710839in}{5.050856in}}{\pgfqpoint{5.715229in}{5.061455in}}{\pgfqpoint{5.715229in}{5.072505in}}%
\pgfpathcurveto{\pgfqpoint{5.715229in}{5.083555in}}{\pgfqpoint{5.710839in}{5.094154in}}{\pgfqpoint{5.703026in}{5.101967in}}%
\pgfpathcurveto{\pgfqpoint{5.695212in}{5.109781in}}{\pgfqpoint{5.684613in}{5.114171in}}{\pgfqpoint{5.673563in}{5.114171in}}%
\pgfpathcurveto{\pgfqpoint{5.662513in}{5.114171in}}{\pgfqpoint{5.651914in}{5.109781in}}{\pgfqpoint{5.644100in}{5.101967in}}%
\pgfpathcurveto{\pgfqpoint{5.636286in}{5.094154in}}{\pgfqpoint{5.631896in}{5.083555in}}{\pgfqpoint{5.631896in}{5.072505in}}%
\pgfpathcurveto{\pgfqpoint{5.631896in}{5.061455in}}{\pgfqpoint{5.636286in}{5.050856in}}{\pgfqpoint{5.644100in}{5.043042in}}%
\pgfpathcurveto{\pgfqpoint{5.651914in}{5.035228in}}{\pgfqpoint{5.662513in}{5.030838in}}{\pgfqpoint{5.673563in}{5.030838in}}%
\pgfpathclose%
\pgfusepath{stroke,fill}%
\end{pgfscope}%
\begin{pgfscope}%
\pgfpathrectangle{\pgfqpoint{0.570343in}{0.331635in}}{\pgfqpoint{9.300000in}{7.700000in}}%
\pgfusepath{clip}%
\pgfsetbuttcap%
\pgfsetroundjoin%
\definecolor{currentfill}{rgb}{0.631373,0.788235,0.956863}%
\pgfsetfillcolor{currentfill}%
\pgfsetlinewidth{0.481800pt}%
\definecolor{currentstroke}{rgb}{1.000000,1.000000,1.000000}%
\pgfsetstrokecolor{currentstroke}%
\pgfsetdash{}{0pt}%
\pgfpathmoveto{\pgfqpoint{3.420050in}{0.639968in}}%
\pgfpathcurveto{\pgfqpoint{3.431100in}{0.639968in}}{\pgfqpoint{3.441699in}{0.644359in}}{\pgfqpoint{3.449513in}{0.652172in}}%
\pgfpathcurveto{\pgfqpoint{3.457326in}{0.659986in}}{\pgfqpoint{3.461717in}{0.670585in}}{\pgfqpoint{3.461717in}{0.681635in}}%
\pgfpathcurveto{\pgfqpoint{3.461717in}{0.692685in}}{\pgfqpoint{3.457326in}{0.703284in}}{\pgfqpoint{3.449513in}{0.711098in}}%
\pgfpathcurveto{\pgfqpoint{3.441699in}{0.718911in}}{\pgfqpoint{3.431100in}{0.723302in}}{\pgfqpoint{3.420050in}{0.723302in}}%
\pgfpathcurveto{\pgfqpoint{3.409000in}{0.723302in}}{\pgfqpoint{3.398401in}{0.718911in}}{\pgfqpoint{3.390587in}{0.711098in}}%
\pgfpathcurveto{\pgfqpoint{3.382774in}{0.703284in}}{\pgfqpoint{3.378383in}{0.692685in}}{\pgfqpoint{3.378383in}{0.681635in}}%
\pgfpathcurveto{\pgfqpoint{3.378383in}{0.670585in}}{\pgfqpoint{3.382774in}{0.659986in}}{\pgfqpoint{3.390587in}{0.652172in}}%
\pgfpathcurveto{\pgfqpoint{3.398401in}{0.644359in}}{\pgfqpoint{3.409000in}{0.639968in}}{\pgfqpoint{3.420050in}{0.639968in}}%
\pgfpathclose%
\pgfusepath{stroke,fill}%
\end{pgfscope}%
\begin{pgfscope}%
\pgfpathrectangle{\pgfqpoint{0.570343in}{0.331635in}}{\pgfqpoint{9.300000in}{7.700000in}}%
\pgfusepath{clip}%
\pgfsetbuttcap%
\pgfsetroundjoin%
\definecolor{currentfill}{rgb}{0.631373,0.788235,0.956863}%
\pgfsetfillcolor{currentfill}%
\pgfsetlinewidth{0.481800pt}%
\definecolor{currentstroke}{rgb}{1.000000,1.000000,1.000000}%
\pgfsetstrokecolor{currentstroke}%
\pgfsetdash{}{0pt}%
\pgfpathmoveto{\pgfqpoint{4.927031in}{7.639968in}}%
\pgfpathcurveto{\pgfqpoint{4.938081in}{7.639968in}}{\pgfqpoint{4.948680in}{7.644359in}}{\pgfqpoint{4.956494in}{7.652172in}}%
\pgfpathcurveto{\pgfqpoint{4.964307in}{7.659986in}}{\pgfqpoint{4.968698in}{7.670585in}}{\pgfqpoint{4.968698in}{7.681635in}}%
\pgfpathcurveto{\pgfqpoint{4.968698in}{7.692685in}}{\pgfqpoint{4.964307in}{7.703284in}}{\pgfqpoint{4.956494in}{7.711098in}}%
\pgfpathcurveto{\pgfqpoint{4.948680in}{7.718911in}}{\pgfqpoint{4.938081in}{7.723302in}}{\pgfqpoint{4.927031in}{7.723302in}}%
\pgfpathcurveto{\pgfqpoint{4.915981in}{7.723302in}}{\pgfqpoint{4.905382in}{7.718911in}}{\pgfqpoint{4.897568in}{7.711098in}}%
\pgfpathcurveto{\pgfqpoint{4.889755in}{7.703284in}}{\pgfqpoint{4.885364in}{7.692685in}}{\pgfqpoint{4.885364in}{7.681635in}}%
\pgfpathcurveto{\pgfqpoint{4.885364in}{7.670585in}}{\pgfqpoint{4.889755in}{7.659986in}}{\pgfqpoint{4.897568in}{7.652172in}}%
\pgfpathcurveto{\pgfqpoint{4.905382in}{7.644359in}}{\pgfqpoint{4.915981in}{7.639968in}}{\pgfqpoint{4.927031in}{7.639968in}}%
\pgfpathclose%
\pgfusepath{stroke,fill}%
\end{pgfscope}%
\begin{pgfscope}%
\pgfpathrectangle{\pgfqpoint{0.570343in}{0.331635in}}{\pgfqpoint{9.300000in}{7.700000in}}%
\pgfusepath{clip}%
\pgfsetbuttcap%
\pgfsetroundjoin%
\definecolor{currentfill}{rgb}{0.631373,0.788235,0.956863}%
\pgfsetfillcolor{currentfill}%
\pgfsetlinewidth{0.481800pt}%
\definecolor{currentstroke}{rgb}{1.000000,1.000000,1.000000}%
\pgfsetstrokecolor{currentstroke}%
\pgfsetdash{}{0pt}%
\pgfpathmoveto{\pgfqpoint{8.854047in}{5.540076in}}%
\pgfpathcurveto{\pgfqpoint{8.865097in}{5.540076in}}{\pgfqpoint{8.875696in}{5.544466in}}{\pgfqpoint{8.883509in}{5.552280in}}%
\pgfpathcurveto{\pgfqpoint{8.891323in}{5.560093in}}{\pgfqpoint{8.895713in}{5.570692in}}{\pgfqpoint{8.895713in}{5.581742in}}%
\pgfpathcurveto{\pgfqpoint{8.895713in}{5.592793in}}{\pgfqpoint{8.891323in}{5.603392in}}{\pgfqpoint{8.883509in}{5.611205in}}%
\pgfpathcurveto{\pgfqpoint{8.875696in}{5.619019in}}{\pgfqpoint{8.865097in}{5.623409in}}{\pgfqpoint{8.854047in}{5.623409in}}%
\pgfpathcurveto{\pgfqpoint{8.842996in}{5.623409in}}{\pgfqpoint{8.832397in}{5.619019in}}{\pgfqpoint{8.824584in}{5.611205in}}%
\pgfpathcurveto{\pgfqpoint{8.816770in}{5.603392in}}{\pgfqpoint{8.812380in}{5.592793in}}{\pgfqpoint{8.812380in}{5.581742in}}%
\pgfpathcurveto{\pgfqpoint{8.812380in}{5.570692in}}{\pgfqpoint{8.816770in}{5.560093in}}{\pgfqpoint{8.824584in}{5.552280in}}%
\pgfpathcurveto{\pgfqpoint{8.832397in}{5.544466in}}{\pgfqpoint{8.842996in}{5.540076in}}{\pgfqpoint{8.854047in}{5.540076in}}%
\pgfpathclose%
\pgfusepath{stroke,fill}%
\end{pgfscope}%
\begin{pgfscope}%
\pgfpathrectangle{\pgfqpoint{0.570343in}{0.331635in}}{\pgfqpoint{9.300000in}{7.700000in}}%
\pgfusepath{clip}%
\pgfsetbuttcap%
\pgfsetroundjoin%
\definecolor{currentfill}{rgb}{0.631373,0.788235,0.956863}%
\pgfsetfillcolor{currentfill}%
\pgfsetlinewidth{0.481800pt}%
\definecolor{currentstroke}{rgb}{1.000000,1.000000,1.000000}%
\pgfsetstrokecolor{currentstroke}%
\pgfsetdash{}{0pt}%
\pgfpathmoveto{\pgfqpoint{1.547559in}{3.962355in}}%
\pgfpathcurveto{\pgfqpoint{1.558609in}{3.962355in}}{\pgfqpoint{1.569208in}{3.966746in}}{\pgfqpoint{1.577021in}{3.974559in}}%
\pgfpathcurveto{\pgfqpoint{1.584835in}{3.982373in}}{\pgfqpoint{1.589225in}{3.992972in}}{\pgfqpoint{1.589225in}{4.004022in}}%
\pgfpathcurveto{\pgfqpoint{1.589225in}{4.015072in}}{\pgfqpoint{1.584835in}{4.025671in}}{\pgfqpoint{1.577021in}{4.033485in}}%
\pgfpathcurveto{\pgfqpoint{1.569208in}{4.041299in}}{\pgfqpoint{1.558609in}{4.045689in}}{\pgfqpoint{1.547559in}{4.045689in}}%
\pgfpathcurveto{\pgfqpoint{1.536508in}{4.045689in}}{\pgfqpoint{1.525909in}{4.041299in}}{\pgfqpoint{1.518096in}{4.033485in}}%
\pgfpathcurveto{\pgfqpoint{1.510282in}{4.025671in}}{\pgfqpoint{1.505892in}{4.015072in}}{\pgfqpoint{1.505892in}{4.004022in}}%
\pgfpathcurveto{\pgfqpoint{1.505892in}{3.992972in}}{\pgfqpoint{1.510282in}{3.982373in}}{\pgfqpoint{1.518096in}{3.974559in}}%
\pgfpathcurveto{\pgfqpoint{1.525909in}{3.966746in}}{\pgfqpoint{1.536508in}{3.962355in}}{\pgfqpoint{1.547559in}{3.962355in}}%
\pgfpathclose%
\pgfusepath{stroke,fill}%
\end{pgfscope}%
\begin{pgfscope}%
\pgfpathrectangle{\pgfqpoint{0.570343in}{0.331635in}}{\pgfqpoint{9.300000in}{7.700000in}}%
\pgfusepath{clip}%
\pgfsetbuttcap%
\pgfsetroundjoin%
\definecolor{currentfill}{rgb}{0.631373,0.788235,0.956863}%
\pgfsetfillcolor{currentfill}%
\pgfsetlinewidth{0.481800pt}%
\definecolor{currentstroke}{rgb}{1.000000,1.000000,1.000000}%
\pgfsetstrokecolor{currentstroke}%
\pgfsetdash{}{0pt}%
\pgfpathmoveto{\pgfqpoint{2.739938in}{3.672409in}}%
\pgfpathcurveto{\pgfqpoint{2.750988in}{3.672409in}}{\pgfqpoint{2.761587in}{3.676799in}}{\pgfqpoint{2.769401in}{3.684612in}}%
\pgfpathcurveto{\pgfqpoint{2.777215in}{3.692426in}}{\pgfqpoint{2.781605in}{3.703025in}}{\pgfqpoint{2.781605in}{3.714075in}}%
\pgfpathcurveto{\pgfqpoint{2.781605in}{3.725125in}}{\pgfqpoint{2.777215in}{3.735724in}}{\pgfqpoint{2.769401in}{3.743538in}}%
\pgfpathcurveto{\pgfqpoint{2.761587in}{3.751352in}}{\pgfqpoint{2.750988in}{3.755742in}}{\pgfqpoint{2.739938in}{3.755742in}}%
\pgfpathcurveto{\pgfqpoint{2.728888in}{3.755742in}}{\pgfqpoint{2.718289in}{3.751352in}}{\pgfqpoint{2.710475in}{3.743538in}}%
\pgfpathcurveto{\pgfqpoint{2.702662in}{3.735724in}}{\pgfqpoint{2.698272in}{3.725125in}}{\pgfqpoint{2.698272in}{3.714075in}}%
\pgfpathcurveto{\pgfqpoint{2.698272in}{3.703025in}}{\pgfqpoint{2.702662in}{3.692426in}}{\pgfqpoint{2.710475in}{3.684612in}}%
\pgfpathcurveto{\pgfqpoint{2.718289in}{3.676799in}}{\pgfqpoint{2.728888in}{3.672409in}}{\pgfqpoint{2.739938in}{3.672409in}}%
\pgfpathclose%
\pgfusepath{stroke,fill}%
\end{pgfscope}%
\begin{pgfscope}%
\pgfpathrectangle{\pgfqpoint{0.570343in}{0.331635in}}{\pgfqpoint{9.300000in}{7.700000in}}%
\pgfusepath{clip}%
\pgfsetbuttcap%
\pgfsetroundjoin%
\definecolor{currentfill}{rgb}{0.631373,0.788235,0.956863}%
\pgfsetfillcolor{currentfill}%
\pgfsetlinewidth{0.481800pt}%
\definecolor{currentstroke}{rgb}{1.000000,1.000000,1.000000}%
\pgfsetstrokecolor{currentstroke}%
\pgfsetdash{}{0pt}%
\pgfpathmoveto{\pgfqpoint{7.703434in}{5.198382in}}%
\pgfpathcurveto{\pgfqpoint{7.714485in}{5.198382in}}{\pgfqpoint{7.725084in}{5.202772in}}{\pgfqpoint{7.732897in}{5.210586in}}%
\pgfpathcurveto{\pgfqpoint{7.740711in}{5.218399in}}{\pgfqpoint{7.745101in}{5.228998in}}{\pgfqpoint{7.745101in}{5.240049in}}%
\pgfpathcurveto{\pgfqpoint{7.745101in}{5.251099in}}{\pgfqpoint{7.740711in}{5.261698in}}{\pgfqpoint{7.732897in}{5.269511in}}%
\pgfpathcurveto{\pgfqpoint{7.725084in}{5.277325in}}{\pgfqpoint{7.714485in}{5.281715in}}{\pgfqpoint{7.703434in}{5.281715in}}%
\pgfpathcurveto{\pgfqpoint{7.692384in}{5.281715in}}{\pgfqpoint{7.681785in}{5.277325in}}{\pgfqpoint{7.673972in}{5.269511in}}%
\pgfpathcurveto{\pgfqpoint{7.666158in}{5.261698in}}{\pgfqpoint{7.661768in}{5.251099in}}{\pgfqpoint{7.661768in}{5.240049in}}%
\pgfpathcurveto{\pgfqpoint{7.661768in}{5.228998in}}{\pgfqpoint{7.666158in}{5.218399in}}{\pgfqpoint{7.673972in}{5.210586in}}%
\pgfpathcurveto{\pgfqpoint{7.681785in}{5.202772in}}{\pgfqpoint{7.692384in}{5.198382in}}{\pgfqpoint{7.703434in}{5.198382in}}%
\pgfpathclose%
\pgfusepath{stroke,fill}%
\end{pgfscope}%
\begin{pgfscope}%
\pgfpathrectangle{\pgfqpoint{0.570343in}{0.331635in}}{\pgfqpoint{9.300000in}{7.700000in}}%
\pgfusepath{clip}%
\pgfsetbuttcap%
\pgfsetroundjoin%
\definecolor{currentfill}{rgb}{0.631373,0.788235,0.956863}%
\pgfsetfillcolor{currentfill}%
\pgfsetlinewidth{0.481800pt}%
\definecolor{currentstroke}{rgb}{1.000000,1.000000,1.000000}%
\pgfsetstrokecolor{currentstroke}%
\pgfsetdash{}{0pt}%
\pgfpathmoveto{\pgfqpoint{7.113284in}{2.769399in}}%
\pgfpathcurveto{\pgfqpoint{7.124334in}{2.769399in}}{\pgfqpoint{7.134933in}{2.773790in}}{\pgfqpoint{7.142746in}{2.781603in}}%
\pgfpathcurveto{\pgfqpoint{7.150560in}{2.789417in}}{\pgfqpoint{7.154950in}{2.800016in}}{\pgfqpoint{7.154950in}{2.811066in}}%
\pgfpathcurveto{\pgfqpoint{7.154950in}{2.822116in}}{\pgfqpoint{7.150560in}{2.832715in}}{\pgfqpoint{7.142746in}{2.840529in}}%
\pgfpathcurveto{\pgfqpoint{7.134933in}{2.848343in}}{\pgfqpoint{7.124334in}{2.852733in}}{\pgfqpoint{7.113284in}{2.852733in}}%
\pgfpathcurveto{\pgfqpoint{7.102233in}{2.852733in}}{\pgfqpoint{7.091634in}{2.848343in}}{\pgfqpoint{7.083821in}{2.840529in}}%
\pgfpathcurveto{\pgfqpoint{7.076007in}{2.832715in}}{\pgfqpoint{7.071617in}{2.822116in}}{\pgfqpoint{7.071617in}{2.811066in}}%
\pgfpathcurveto{\pgfqpoint{7.071617in}{2.800016in}}{\pgfqpoint{7.076007in}{2.789417in}}{\pgfqpoint{7.083821in}{2.781603in}}%
\pgfpathcurveto{\pgfqpoint{7.091634in}{2.773790in}}{\pgfqpoint{7.102233in}{2.769399in}}{\pgfqpoint{7.113284in}{2.769399in}}%
\pgfpathclose%
\pgfusepath{stroke,fill}%
\end{pgfscope}%
\begin{pgfscope}%
\pgfpathrectangle{\pgfqpoint{0.570343in}{0.331635in}}{\pgfqpoint{9.300000in}{7.700000in}}%
\pgfusepath{clip}%
\pgfsetbuttcap%
\pgfsetroundjoin%
\definecolor{currentfill}{rgb}{0.631373,0.788235,0.956863}%
\pgfsetfillcolor{currentfill}%
\pgfsetlinewidth{0.481800pt}%
\definecolor{currentstroke}{rgb}{1.000000,1.000000,1.000000}%
\pgfsetstrokecolor{currentstroke}%
\pgfsetdash{}{0pt}%
\pgfpathmoveto{\pgfqpoint{2.082625in}{2.090137in}}%
\pgfpathcurveto{\pgfqpoint{2.093675in}{2.090137in}}{\pgfqpoint{2.104275in}{2.094527in}}{\pgfqpoint{2.112088in}{2.102341in}}%
\pgfpathcurveto{\pgfqpoint{2.119902in}{2.110155in}}{\pgfqpoint{2.124292in}{2.120754in}}{\pgfqpoint{2.124292in}{2.131804in}}%
\pgfpathcurveto{\pgfqpoint{2.124292in}{2.142854in}}{\pgfqpoint{2.119902in}{2.153453in}}{\pgfqpoint{2.112088in}{2.161267in}}%
\pgfpathcurveto{\pgfqpoint{2.104275in}{2.169080in}}{\pgfqpoint{2.093675in}{2.173470in}}{\pgfqpoint{2.082625in}{2.173470in}}%
\pgfpathcurveto{\pgfqpoint{2.071575in}{2.173470in}}{\pgfqpoint{2.060976in}{2.169080in}}{\pgfqpoint{2.053163in}{2.161267in}}%
\pgfpathcurveto{\pgfqpoint{2.045349in}{2.153453in}}{\pgfqpoint{2.040959in}{2.142854in}}{\pgfqpoint{2.040959in}{2.131804in}}%
\pgfpathcurveto{\pgfqpoint{2.040959in}{2.120754in}}{\pgfqpoint{2.045349in}{2.110155in}}{\pgfqpoint{2.053163in}{2.102341in}}%
\pgfpathcurveto{\pgfqpoint{2.060976in}{2.094527in}}{\pgfqpoint{2.071575in}{2.090137in}}{\pgfqpoint{2.082625in}{2.090137in}}%
\pgfpathclose%
\pgfusepath{stroke,fill}%
\end{pgfscope}%
\begin{pgfscope}%
\pgfpathrectangle{\pgfqpoint{0.570343in}{0.331635in}}{\pgfqpoint{9.300000in}{7.700000in}}%
\pgfusepath{clip}%
\pgfsetbuttcap%
\pgfsetroundjoin%
\definecolor{currentfill}{rgb}{0.631373,0.788235,0.956863}%
\pgfsetfillcolor{currentfill}%
\pgfsetlinewidth{0.481800pt}%
\definecolor{currentstroke}{rgb}{1.000000,1.000000,1.000000}%
\pgfsetstrokecolor{currentstroke}%
\pgfsetdash{}{0pt}%
\pgfpathmoveto{\pgfqpoint{8.230336in}{6.091749in}}%
\pgfpathcurveto{\pgfqpoint{8.241386in}{6.091749in}}{\pgfqpoint{8.251985in}{6.096139in}}{\pgfqpoint{8.259799in}{6.103953in}}%
\pgfpathcurveto{\pgfqpoint{8.267612in}{6.111767in}}{\pgfqpoint{8.272003in}{6.122366in}}{\pgfqpoint{8.272003in}{6.133416in}}%
\pgfpathcurveto{\pgfqpoint{8.272003in}{6.144466in}}{\pgfqpoint{8.267612in}{6.155065in}}{\pgfqpoint{8.259799in}{6.162879in}}%
\pgfpathcurveto{\pgfqpoint{8.251985in}{6.170692in}}{\pgfqpoint{8.241386in}{6.175082in}}{\pgfqpoint{8.230336in}{6.175082in}}%
\pgfpathcurveto{\pgfqpoint{8.219286in}{6.175082in}}{\pgfqpoint{8.208687in}{6.170692in}}{\pgfqpoint{8.200873in}{6.162879in}}%
\pgfpathcurveto{\pgfqpoint{8.193060in}{6.155065in}}{\pgfqpoint{8.188669in}{6.144466in}}{\pgfqpoint{8.188669in}{6.133416in}}%
\pgfpathcurveto{\pgfqpoint{8.188669in}{6.122366in}}{\pgfqpoint{8.193060in}{6.111767in}}{\pgfqpoint{8.200873in}{6.103953in}}%
\pgfpathcurveto{\pgfqpoint{8.208687in}{6.096139in}}{\pgfqpoint{8.219286in}{6.091749in}}{\pgfqpoint{8.230336in}{6.091749in}}%
\pgfpathclose%
\pgfusepath{stroke,fill}%
\end{pgfscope}%
\begin{pgfscope}%
\pgfpathrectangle{\pgfqpoint{0.570343in}{0.331635in}}{\pgfqpoint{9.300000in}{7.700000in}}%
\pgfusepath{clip}%
\pgfsetbuttcap%
\pgfsetroundjoin%
\definecolor{currentfill}{rgb}{0.631373,0.788235,0.956863}%
\pgfsetfillcolor{currentfill}%
\pgfsetlinewidth{0.481800pt}%
\definecolor{currentstroke}{rgb}{1.000000,1.000000,1.000000}%
\pgfsetstrokecolor{currentstroke}%
\pgfsetdash{}{0pt}%
\pgfpathmoveto{\pgfqpoint{4.884974in}{2.588954in}}%
\pgfpathcurveto{\pgfqpoint{4.896024in}{2.588954in}}{\pgfqpoint{4.906623in}{2.593344in}}{\pgfqpoint{4.914437in}{2.601158in}}%
\pgfpathcurveto{\pgfqpoint{4.922250in}{2.608971in}}{\pgfqpoint{4.926641in}{2.619570in}}{\pgfqpoint{4.926641in}{2.630620in}}%
\pgfpathcurveto{\pgfqpoint{4.926641in}{2.641671in}}{\pgfqpoint{4.922250in}{2.652270in}}{\pgfqpoint{4.914437in}{2.660083in}}%
\pgfpathcurveto{\pgfqpoint{4.906623in}{2.667897in}}{\pgfqpoint{4.896024in}{2.672287in}}{\pgfqpoint{4.884974in}{2.672287in}}%
\pgfpathcurveto{\pgfqpoint{4.873924in}{2.672287in}}{\pgfqpoint{4.863325in}{2.667897in}}{\pgfqpoint{4.855511in}{2.660083in}}%
\pgfpathcurveto{\pgfqpoint{4.847698in}{2.652270in}}{\pgfqpoint{4.843307in}{2.641671in}}{\pgfqpoint{4.843307in}{2.630620in}}%
\pgfpathcurveto{\pgfqpoint{4.843307in}{2.619570in}}{\pgfqpoint{4.847698in}{2.608971in}}{\pgfqpoint{4.855511in}{2.601158in}}%
\pgfpathcurveto{\pgfqpoint{4.863325in}{2.593344in}}{\pgfqpoint{4.873924in}{2.588954in}}{\pgfqpoint{4.884974in}{2.588954in}}%
\pgfpathclose%
\pgfusepath{stroke,fill}%
\end{pgfscope}%
\begin{pgfscope}%
\pgfpathrectangle{\pgfqpoint{0.570343in}{0.331635in}}{\pgfqpoint{9.300000in}{7.700000in}}%
\pgfusepath{clip}%
\pgfsetbuttcap%
\pgfsetroundjoin%
\definecolor{currentfill}{rgb}{0.631373,0.788235,0.956863}%
\pgfsetfillcolor{currentfill}%
\pgfsetlinewidth{0.481800pt}%
\definecolor{currentstroke}{rgb}{1.000000,1.000000,1.000000}%
\pgfsetstrokecolor{currentstroke}%
\pgfsetdash{}{0pt}%
\pgfpathmoveto{\pgfqpoint{1.225108in}{2.741075in}}%
\pgfpathcurveto{\pgfqpoint{1.236158in}{2.741075in}}{\pgfqpoint{1.246757in}{2.745465in}}{\pgfqpoint{1.254570in}{2.753279in}}%
\pgfpathcurveto{\pgfqpoint{1.262384in}{2.761093in}}{\pgfqpoint{1.266774in}{2.771692in}}{\pgfqpoint{1.266774in}{2.782742in}}%
\pgfpathcurveto{\pgfqpoint{1.266774in}{2.793792in}}{\pgfqpoint{1.262384in}{2.804391in}}{\pgfqpoint{1.254570in}{2.812205in}}%
\pgfpathcurveto{\pgfqpoint{1.246757in}{2.820018in}}{\pgfqpoint{1.236158in}{2.824408in}}{\pgfqpoint{1.225108in}{2.824408in}}%
\pgfpathcurveto{\pgfqpoint{1.214057in}{2.824408in}}{\pgfqpoint{1.203458in}{2.820018in}}{\pgfqpoint{1.195645in}{2.812205in}}%
\pgfpathcurveto{\pgfqpoint{1.187831in}{2.804391in}}{\pgfqpoint{1.183441in}{2.793792in}}{\pgfqpoint{1.183441in}{2.782742in}}%
\pgfpathcurveto{\pgfqpoint{1.183441in}{2.771692in}}{\pgfqpoint{1.187831in}{2.761093in}}{\pgfqpoint{1.195645in}{2.753279in}}%
\pgfpathcurveto{\pgfqpoint{1.203458in}{2.745465in}}{\pgfqpoint{1.214057in}{2.741075in}}{\pgfqpoint{1.225108in}{2.741075in}}%
\pgfpathclose%
\pgfusepath{stroke,fill}%
\end{pgfscope}%
\begin{pgfscope}%
\pgfpathrectangle{\pgfqpoint{0.570343in}{0.331635in}}{\pgfqpoint{9.300000in}{7.700000in}}%
\pgfusepath{clip}%
\pgfsetbuttcap%
\pgfsetroundjoin%
\definecolor{currentfill}{rgb}{0.631373,0.788235,0.956863}%
\pgfsetfillcolor{currentfill}%
\pgfsetlinewidth{0.481800pt}%
\definecolor{currentstroke}{rgb}{1.000000,1.000000,1.000000}%
\pgfsetstrokecolor{currentstroke}%
\pgfsetdash{}{0pt}%
\pgfpathmoveto{\pgfqpoint{3.946393in}{3.437064in}}%
\pgfpathcurveto{\pgfqpoint{3.957443in}{3.437064in}}{\pgfqpoint{3.968042in}{3.441454in}}{\pgfqpoint{3.975856in}{3.449267in}}%
\pgfpathcurveto{\pgfqpoint{3.983670in}{3.457081in}}{\pgfqpoint{3.988060in}{3.467680in}}{\pgfqpoint{3.988060in}{3.478730in}}%
\pgfpathcurveto{\pgfqpoint{3.988060in}{3.489780in}}{\pgfqpoint{3.983670in}{3.500379in}}{\pgfqpoint{3.975856in}{3.508193in}}%
\pgfpathcurveto{\pgfqpoint{3.968042in}{3.516007in}}{\pgfqpoint{3.957443in}{3.520397in}}{\pgfqpoint{3.946393in}{3.520397in}}%
\pgfpathcurveto{\pgfqpoint{3.935343in}{3.520397in}}{\pgfqpoint{3.924744in}{3.516007in}}{\pgfqpoint{3.916930in}{3.508193in}}%
\pgfpathcurveto{\pgfqpoint{3.909117in}{3.500379in}}{\pgfqpoint{3.904726in}{3.489780in}}{\pgfqpoint{3.904726in}{3.478730in}}%
\pgfpathcurveto{\pgfqpoint{3.904726in}{3.467680in}}{\pgfqpoint{3.909117in}{3.457081in}}{\pgfqpoint{3.916930in}{3.449267in}}%
\pgfpathcurveto{\pgfqpoint{3.924744in}{3.441454in}}{\pgfqpoint{3.935343in}{3.437064in}}{\pgfqpoint{3.946393in}{3.437064in}}%
\pgfpathclose%
\pgfusepath{stroke,fill}%
\end{pgfscope}%
\begin{pgfscope}%
\pgfpathrectangle{\pgfqpoint{0.570343in}{0.331635in}}{\pgfqpoint{9.300000in}{7.700000in}}%
\pgfusepath{clip}%
\pgfsetbuttcap%
\pgfsetroundjoin%
\definecolor{currentfill}{rgb}{0.631373,0.788235,0.956863}%
\pgfsetfillcolor{currentfill}%
\pgfsetlinewidth{0.481800pt}%
\definecolor{currentstroke}{rgb}{1.000000,1.000000,1.000000}%
\pgfsetstrokecolor{currentstroke}%
\pgfsetdash{}{0pt}%
\pgfpathmoveto{\pgfqpoint{4.886393in}{1.848117in}}%
\pgfpathcurveto{\pgfqpoint{4.897443in}{1.848117in}}{\pgfqpoint{4.908042in}{1.852507in}}{\pgfqpoint{4.915856in}{1.860321in}}%
\pgfpathcurveto{\pgfqpoint{4.923669in}{1.868135in}}{\pgfqpoint{4.928059in}{1.878734in}}{\pgfqpoint{4.928059in}{1.889784in}}%
\pgfpathcurveto{\pgfqpoint{4.928059in}{1.900834in}}{\pgfqpoint{4.923669in}{1.911433in}}{\pgfqpoint{4.915856in}{1.919247in}}%
\pgfpathcurveto{\pgfqpoint{4.908042in}{1.927060in}}{\pgfqpoint{4.897443in}{1.931450in}}{\pgfqpoint{4.886393in}{1.931450in}}%
\pgfpathcurveto{\pgfqpoint{4.875343in}{1.931450in}}{\pgfqpoint{4.864744in}{1.927060in}}{\pgfqpoint{4.856930in}{1.919247in}}%
\pgfpathcurveto{\pgfqpoint{4.849116in}{1.911433in}}{\pgfqpoint{4.844726in}{1.900834in}}{\pgfqpoint{4.844726in}{1.889784in}}%
\pgfpathcurveto{\pgfqpoint{4.844726in}{1.878734in}}{\pgfqpoint{4.849116in}{1.868135in}}{\pgfqpoint{4.856930in}{1.860321in}}%
\pgfpathcurveto{\pgfqpoint{4.864744in}{1.852507in}}{\pgfqpoint{4.875343in}{1.848117in}}{\pgfqpoint{4.886393in}{1.848117in}}%
\pgfpathclose%
\pgfusepath{stroke,fill}%
\end{pgfscope}%
\begin{pgfscope}%
\pgfpathrectangle{\pgfqpoint{0.570343in}{0.331635in}}{\pgfqpoint{9.300000in}{7.700000in}}%
\pgfusepath{clip}%
\pgfsetbuttcap%
\pgfsetroundjoin%
\definecolor{currentfill}{rgb}{0.631373,0.788235,0.956863}%
\pgfsetfillcolor{currentfill}%
\pgfsetlinewidth{0.481800pt}%
\definecolor{currentstroke}{rgb}{1.000000,1.000000,1.000000}%
\pgfsetstrokecolor{currentstroke}%
\pgfsetdash{}{0pt}%
\pgfpathmoveto{\pgfqpoint{3.784493in}{1.682234in}}%
\pgfpathcurveto{\pgfqpoint{3.795543in}{1.682234in}}{\pgfqpoint{3.806142in}{1.686624in}}{\pgfqpoint{3.813956in}{1.694438in}}%
\pgfpathcurveto{\pgfqpoint{3.821769in}{1.702252in}}{\pgfqpoint{3.826160in}{1.712851in}}{\pgfqpoint{3.826160in}{1.723901in}}%
\pgfpathcurveto{\pgfqpoint{3.826160in}{1.734951in}}{\pgfqpoint{3.821769in}{1.745550in}}{\pgfqpoint{3.813956in}{1.753364in}}%
\pgfpathcurveto{\pgfqpoint{3.806142in}{1.761177in}}{\pgfqpoint{3.795543in}{1.765567in}}{\pgfqpoint{3.784493in}{1.765567in}}%
\pgfpathcurveto{\pgfqpoint{3.773443in}{1.765567in}}{\pgfqpoint{3.762844in}{1.761177in}}{\pgfqpoint{3.755030in}{1.753364in}}%
\pgfpathcurveto{\pgfqpoint{3.747217in}{1.745550in}}{\pgfqpoint{3.742826in}{1.734951in}}{\pgfqpoint{3.742826in}{1.723901in}}%
\pgfpathcurveto{\pgfqpoint{3.742826in}{1.712851in}}{\pgfqpoint{3.747217in}{1.702252in}}{\pgfqpoint{3.755030in}{1.694438in}}%
\pgfpathcurveto{\pgfqpoint{3.762844in}{1.686624in}}{\pgfqpoint{3.773443in}{1.682234in}}{\pgfqpoint{3.784493in}{1.682234in}}%
\pgfpathclose%
\pgfusepath{stroke,fill}%
\end{pgfscope}%
\begin{pgfscope}%
\pgfpathrectangle{\pgfqpoint{0.570343in}{0.331635in}}{\pgfqpoint{9.300000in}{7.700000in}}%
\pgfusepath{clip}%
\pgfsetbuttcap%
\pgfsetroundjoin%
\definecolor{currentfill}{rgb}{0.631373,0.788235,0.956863}%
\pgfsetfillcolor{currentfill}%
\pgfsetlinewidth{0.481800pt}%
\definecolor{currentstroke}{rgb}{1.000000,1.000000,1.000000}%
\pgfsetstrokecolor{currentstroke}%
\pgfsetdash{}{0pt}%
\pgfpathmoveto{\pgfqpoint{2.706989in}{2.493185in}}%
\pgfpathcurveto{\pgfqpoint{2.718039in}{2.493185in}}{\pgfqpoint{2.728638in}{2.497575in}}{\pgfqpoint{2.736452in}{2.505389in}}%
\pgfpathcurveto{\pgfqpoint{2.744265in}{2.513202in}}{\pgfqpoint{2.748656in}{2.523802in}}{\pgfqpoint{2.748656in}{2.534852in}}%
\pgfpathcurveto{\pgfqpoint{2.748656in}{2.545902in}}{\pgfqpoint{2.744265in}{2.556501in}}{\pgfqpoint{2.736452in}{2.564314in}}%
\pgfpathcurveto{\pgfqpoint{2.728638in}{2.572128in}}{\pgfqpoint{2.718039in}{2.576518in}}{\pgfqpoint{2.706989in}{2.576518in}}%
\pgfpathcurveto{\pgfqpoint{2.695939in}{2.576518in}}{\pgfqpoint{2.685340in}{2.572128in}}{\pgfqpoint{2.677526in}{2.564314in}}%
\pgfpathcurveto{\pgfqpoint{2.669712in}{2.556501in}}{\pgfqpoint{2.665322in}{2.545902in}}{\pgfqpoint{2.665322in}{2.534852in}}%
\pgfpathcurveto{\pgfqpoint{2.665322in}{2.523802in}}{\pgfqpoint{2.669712in}{2.513202in}}{\pgfqpoint{2.677526in}{2.505389in}}%
\pgfpathcurveto{\pgfqpoint{2.685340in}{2.497575in}}{\pgfqpoint{2.695939in}{2.493185in}}{\pgfqpoint{2.706989in}{2.493185in}}%
\pgfpathclose%
\pgfusepath{stroke,fill}%
\end{pgfscope}%
\begin{pgfscope}%
\pgfpathrectangle{\pgfqpoint{0.570343in}{0.331635in}}{\pgfqpoint{9.300000in}{7.700000in}}%
\pgfusepath{clip}%
\pgfsetbuttcap%
\pgfsetroundjoin%
\definecolor{currentfill}{rgb}{0.631373,0.788235,0.956863}%
\pgfsetfillcolor{currentfill}%
\pgfsetlinewidth{0.481800pt}%
\definecolor{currentstroke}{rgb}{1.000000,1.000000,1.000000}%
\pgfsetstrokecolor{currentstroke}%
\pgfsetdash{}{0pt}%
\pgfpathmoveto{\pgfqpoint{3.495356in}{4.079677in}}%
\pgfpathcurveto{\pgfqpoint{3.506406in}{4.079677in}}{\pgfqpoint{3.517005in}{4.084068in}}{\pgfqpoint{3.524818in}{4.091881in}}%
\pgfpathcurveto{\pgfqpoint{3.532632in}{4.099695in}}{\pgfqpoint{3.537022in}{4.110294in}}{\pgfqpoint{3.537022in}{4.121344in}}%
\pgfpathcurveto{\pgfqpoint{3.537022in}{4.132394in}}{\pgfqpoint{3.532632in}{4.142993in}}{\pgfqpoint{3.524818in}{4.150807in}}%
\pgfpathcurveto{\pgfqpoint{3.517005in}{4.158621in}}{\pgfqpoint{3.506406in}{4.163011in}}{\pgfqpoint{3.495356in}{4.163011in}}%
\pgfpathcurveto{\pgfqpoint{3.484306in}{4.163011in}}{\pgfqpoint{3.473706in}{4.158621in}}{\pgfqpoint{3.465893in}{4.150807in}}%
\pgfpathcurveto{\pgfqpoint{3.458079in}{4.142993in}}{\pgfqpoint{3.453689in}{4.132394in}}{\pgfqpoint{3.453689in}{4.121344in}}%
\pgfpathcurveto{\pgfqpoint{3.453689in}{4.110294in}}{\pgfqpoint{3.458079in}{4.099695in}}{\pgfqpoint{3.465893in}{4.091881in}}%
\pgfpathcurveto{\pgfqpoint{3.473706in}{4.084068in}}{\pgfqpoint{3.484306in}{4.079677in}}{\pgfqpoint{3.495356in}{4.079677in}}%
\pgfpathclose%
\pgfusepath{stroke,fill}%
\end{pgfscope}%
\begin{pgfscope}%
\pgfpathrectangle{\pgfqpoint{0.570343in}{0.331635in}}{\pgfqpoint{9.300000in}{7.700000in}}%
\pgfusepath{clip}%
\pgfsetbuttcap%
\pgfsetroundjoin%
\definecolor{currentfill}{rgb}{0.631373,0.788235,0.956863}%
\pgfsetfillcolor{currentfill}%
\pgfsetlinewidth{0.481800pt}%
\definecolor{currentstroke}{rgb}{1.000000,1.000000,1.000000}%
\pgfsetstrokecolor{currentstroke}%
\pgfsetdash{}{0pt}%
\pgfpathmoveto{\pgfqpoint{0.993071in}{4.504689in}}%
\pgfpathcurveto{\pgfqpoint{1.004121in}{4.504689in}}{\pgfqpoint{1.014720in}{4.509079in}}{\pgfqpoint{1.022533in}{4.516893in}}%
\pgfpathcurveto{\pgfqpoint{1.030347in}{4.524707in}}{\pgfqpoint{1.034737in}{4.535306in}}{\pgfqpoint{1.034737in}{4.546356in}}%
\pgfpathcurveto{\pgfqpoint{1.034737in}{4.557406in}}{\pgfqpoint{1.030347in}{4.568005in}}{\pgfqpoint{1.022533in}{4.575818in}}%
\pgfpathcurveto{\pgfqpoint{1.014720in}{4.583632in}}{\pgfqpoint{1.004121in}{4.588022in}}{\pgfqpoint{0.993071in}{4.588022in}}%
\pgfpathcurveto{\pgfqpoint{0.982020in}{4.588022in}}{\pgfqpoint{0.971421in}{4.583632in}}{\pgfqpoint{0.963608in}{4.575818in}}%
\pgfpathcurveto{\pgfqpoint{0.955794in}{4.568005in}}{\pgfqpoint{0.951404in}{4.557406in}}{\pgfqpoint{0.951404in}{4.546356in}}%
\pgfpathcurveto{\pgfqpoint{0.951404in}{4.535306in}}{\pgfqpoint{0.955794in}{4.524707in}}{\pgfqpoint{0.963608in}{4.516893in}}%
\pgfpathcurveto{\pgfqpoint{0.971421in}{4.509079in}}{\pgfqpoint{0.982020in}{4.504689in}}{\pgfqpoint{0.993071in}{4.504689in}}%
\pgfpathclose%
\pgfusepath{stroke,fill}%
\end{pgfscope}%
\begin{pgfscope}%
\pgfpathrectangle{\pgfqpoint{0.570343in}{0.331635in}}{\pgfqpoint{9.300000in}{7.700000in}}%
\pgfusepath{clip}%
\pgfsetbuttcap%
\pgfsetroundjoin%
\definecolor{currentfill}{rgb}{0.631373,0.788235,0.956863}%
\pgfsetfillcolor{currentfill}%
\pgfsetlinewidth{0.481800pt}%
\definecolor{currentstroke}{rgb}{1.000000,1.000000,1.000000}%
\pgfsetstrokecolor{currentstroke}%
\pgfsetdash{}{0pt}%
\pgfpathmoveto{\pgfqpoint{2.362144in}{4.853320in}}%
\pgfpathcurveto{\pgfqpoint{2.373195in}{4.853320in}}{\pgfqpoint{2.383794in}{4.857710in}}{\pgfqpoint{2.391607in}{4.865524in}}%
\pgfpathcurveto{\pgfqpoint{2.399421in}{4.873338in}}{\pgfqpoint{2.403811in}{4.883937in}}{\pgfqpoint{2.403811in}{4.894987in}}%
\pgfpathcurveto{\pgfqpoint{2.403811in}{4.906037in}}{\pgfqpoint{2.399421in}{4.916636in}}{\pgfqpoint{2.391607in}{4.924450in}}%
\pgfpathcurveto{\pgfqpoint{2.383794in}{4.932263in}}{\pgfqpoint{2.373195in}{4.936654in}}{\pgfqpoint{2.362144in}{4.936654in}}%
\pgfpathcurveto{\pgfqpoint{2.351094in}{4.936654in}}{\pgfqpoint{2.340495in}{4.932263in}}{\pgfqpoint{2.332682in}{4.924450in}}%
\pgfpathcurveto{\pgfqpoint{2.324868in}{4.916636in}}{\pgfqpoint{2.320478in}{4.906037in}}{\pgfqpoint{2.320478in}{4.894987in}}%
\pgfpathcurveto{\pgfqpoint{2.320478in}{4.883937in}}{\pgfqpoint{2.324868in}{4.873338in}}{\pgfqpoint{2.332682in}{4.865524in}}%
\pgfpathcurveto{\pgfqpoint{2.340495in}{4.857710in}}{\pgfqpoint{2.351094in}{4.853320in}}{\pgfqpoint{2.362144in}{4.853320in}}%
\pgfpathclose%
\pgfusepath{stroke,fill}%
\end{pgfscope}%
\begin{pgfscope}%
\pgfpathrectangle{\pgfqpoint{0.570343in}{0.331635in}}{\pgfqpoint{9.300000in}{7.700000in}}%
\pgfusepath{clip}%
\pgfsetbuttcap%
\pgfsetroundjoin%
\definecolor{currentfill}{rgb}{0.631373,0.788235,0.956863}%
\pgfsetfillcolor{currentfill}%
\pgfsetlinewidth{0.481800pt}%
\definecolor{currentstroke}{rgb}{1.000000,1.000000,1.000000}%
\pgfsetstrokecolor{currentstroke}%
\pgfsetdash{}{0pt}%
\pgfpathmoveto{\pgfqpoint{4.437533in}{3.921461in}}%
\pgfpathcurveto{\pgfqpoint{4.448583in}{3.921461in}}{\pgfqpoint{4.459182in}{3.925852in}}{\pgfqpoint{4.466996in}{3.933665in}}%
\pgfpathcurveto{\pgfqpoint{4.474810in}{3.941479in}}{\pgfqpoint{4.479200in}{3.952078in}}{\pgfqpoint{4.479200in}{3.963128in}}%
\pgfpathcurveto{\pgfqpoint{4.479200in}{3.974178in}}{\pgfqpoint{4.474810in}{3.984777in}}{\pgfqpoint{4.466996in}{3.992591in}}%
\pgfpathcurveto{\pgfqpoint{4.459182in}{4.000404in}}{\pgfqpoint{4.448583in}{4.004795in}}{\pgfqpoint{4.437533in}{4.004795in}}%
\pgfpathcurveto{\pgfqpoint{4.426483in}{4.004795in}}{\pgfqpoint{4.415884in}{4.000404in}}{\pgfqpoint{4.408070in}{3.992591in}}%
\pgfpathcurveto{\pgfqpoint{4.400257in}{3.984777in}}{\pgfqpoint{4.395867in}{3.974178in}}{\pgfqpoint{4.395867in}{3.963128in}}%
\pgfpathcurveto{\pgfqpoint{4.395867in}{3.952078in}}{\pgfqpoint{4.400257in}{3.941479in}}{\pgfqpoint{4.408070in}{3.933665in}}%
\pgfpathcurveto{\pgfqpoint{4.415884in}{3.925852in}}{\pgfqpoint{4.426483in}{3.921461in}}{\pgfqpoint{4.437533in}{3.921461in}}%
\pgfpathclose%
\pgfusepath{stroke,fill}%
\end{pgfscope}%
\begin{pgfscope}%
\pgfpathrectangle{\pgfqpoint{0.570343in}{0.331635in}}{\pgfqpoint{9.300000in}{7.700000in}}%
\pgfusepath{clip}%
\pgfsetbuttcap%
\pgfsetroundjoin%
\definecolor{currentfill}{rgb}{0.631373,0.788235,0.956863}%
\pgfsetfillcolor{currentfill}%
\pgfsetlinewidth{0.481800pt}%
\definecolor{currentstroke}{rgb}{1.000000,1.000000,1.000000}%
\pgfsetstrokecolor{currentstroke}%
\pgfsetdash{}{0pt}%
\pgfpathmoveto{\pgfqpoint{3.510625in}{5.131958in}}%
\pgfpathcurveto{\pgfqpoint{3.521675in}{5.131958in}}{\pgfqpoint{3.532274in}{5.136348in}}{\pgfqpoint{3.540088in}{5.144162in}}%
\pgfpathcurveto{\pgfqpoint{3.547901in}{5.151976in}}{\pgfqpoint{3.552291in}{5.162575in}}{\pgfqpoint{3.552291in}{5.173625in}}%
\pgfpathcurveto{\pgfqpoint{3.552291in}{5.184675in}}{\pgfqpoint{3.547901in}{5.195274in}}{\pgfqpoint{3.540088in}{5.203087in}}%
\pgfpathcurveto{\pgfqpoint{3.532274in}{5.210901in}}{\pgfqpoint{3.521675in}{5.215291in}}{\pgfqpoint{3.510625in}{5.215291in}}%
\pgfpathcurveto{\pgfqpoint{3.499575in}{5.215291in}}{\pgfqpoint{3.488976in}{5.210901in}}{\pgfqpoint{3.481162in}{5.203087in}}%
\pgfpathcurveto{\pgfqpoint{3.473348in}{5.195274in}}{\pgfqpoint{3.468958in}{5.184675in}}{\pgfqpoint{3.468958in}{5.173625in}}%
\pgfpathcurveto{\pgfqpoint{3.468958in}{5.162575in}}{\pgfqpoint{3.473348in}{5.151976in}}{\pgfqpoint{3.481162in}{5.144162in}}%
\pgfpathcurveto{\pgfqpoint{3.488976in}{5.136348in}}{\pgfqpoint{3.499575in}{5.131958in}}{\pgfqpoint{3.510625in}{5.131958in}}%
\pgfpathclose%
\pgfusepath{stroke,fill}%
\end{pgfscope}%
\begin{pgfscope}%
\pgfpathrectangle{\pgfqpoint{0.570343in}{0.331635in}}{\pgfqpoint{9.300000in}{7.700000in}}%
\pgfusepath{clip}%
\pgfsetbuttcap%
\pgfsetroundjoin%
\definecolor{currentfill}{rgb}{0.631373,0.788235,0.956863}%
\pgfsetfillcolor{currentfill}%
\pgfsetlinewidth{0.481800pt}%
\definecolor{currentstroke}{rgb}{1.000000,1.000000,1.000000}%
\pgfsetstrokecolor{currentstroke}%
\pgfsetdash{}{0pt}%
\pgfpathmoveto{\pgfqpoint{4.054617in}{2.257712in}}%
\pgfpathcurveto{\pgfqpoint{4.065667in}{2.257712in}}{\pgfqpoint{4.076266in}{2.262103in}}{\pgfqpoint{4.084079in}{2.269916in}}%
\pgfpathcurveto{\pgfqpoint{4.091893in}{2.277730in}}{\pgfqpoint{4.096283in}{2.288329in}}{\pgfqpoint{4.096283in}{2.299379in}}%
\pgfpathcurveto{\pgfqpoint{4.096283in}{2.310429in}}{\pgfqpoint{4.091893in}{2.321028in}}{\pgfqpoint{4.084079in}{2.328842in}}%
\pgfpathcurveto{\pgfqpoint{4.076266in}{2.336655in}}{\pgfqpoint{4.065667in}{2.341046in}}{\pgfqpoint{4.054617in}{2.341046in}}%
\pgfpathcurveto{\pgfqpoint{4.043567in}{2.341046in}}{\pgfqpoint{4.032968in}{2.336655in}}{\pgfqpoint{4.025154in}{2.328842in}}%
\pgfpathcurveto{\pgfqpoint{4.017340in}{2.321028in}}{\pgfqpoint{4.012950in}{2.310429in}}{\pgfqpoint{4.012950in}{2.299379in}}%
\pgfpathcurveto{\pgfqpoint{4.012950in}{2.288329in}}{\pgfqpoint{4.017340in}{2.277730in}}{\pgfqpoint{4.025154in}{2.269916in}}%
\pgfpathcurveto{\pgfqpoint{4.032968in}{2.262103in}}{\pgfqpoint{4.043567in}{2.257712in}}{\pgfqpoint{4.054617in}{2.257712in}}%
\pgfpathclose%
\pgfusepath{stroke,fill}%
\end{pgfscope}%
\begin{pgfscope}%
\pgfpathrectangle{\pgfqpoint{0.570343in}{0.331635in}}{\pgfqpoint{9.300000in}{7.700000in}}%
\pgfusepath{clip}%
\pgfsetbuttcap%
\pgfsetroundjoin%
\definecolor{currentfill}{rgb}{0.631373,0.788235,0.956863}%
\pgfsetfillcolor{currentfill}%
\pgfsetlinewidth{0.481800pt}%
\definecolor{currentstroke}{rgb}{1.000000,1.000000,1.000000}%
\pgfsetstrokecolor{currentstroke}%
\pgfsetdash{}{0pt}%
\pgfpathmoveto{\pgfqpoint{4.954557in}{3.304660in}}%
\pgfpathcurveto{\pgfqpoint{4.965608in}{3.304660in}}{\pgfqpoint{4.976207in}{3.309051in}}{\pgfqpoint{4.984020in}{3.316864in}}%
\pgfpathcurveto{\pgfqpoint{4.991834in}{3.324678in}}{\pgfqpoint{4.996224in}{3.335277in}}{\pgfqpoint{4.996224in}{3.346327in}}%
\pgfpathcurveto{\pgfqpoint{4.996224in}{3.357377in}}{\pgfqpoint{4.991834in}{3.367976in}}{\pgfqpoint{4.984020in}{3.375790in}}%
\pgfpathcurveto{\pgfqpoint{4.976207in}{3.383603in}}{\pgfqpoint{4.965608in}{3.387994in}}{\pgfqpoint{4.954557in}{3.387994in}}%
\pgfpathcurveto{\pgfqpoint{4.943507in}{3.387994in}}{\pgfqpoint{4.932908in}{3.383603in}}{\pgfqpoint{4.925095in}{3.375790in}}%
\pgfpathcurveto{\pgfqpoint{4.917281in}{3.367976in}}{\pgfqpoint{4.912891in}{3.357377in}}{\pgfqpoint{4.912891in}{3.346327in}}%
\pgfpathcurveto{\pgfqpoint{4.912891in}{3.335277in}}{\pgfqpoint{4.917281in}{3.324678in}}{\pgfqpoint{4.925095in}{3.316864in}}%
\pgfpathcurveto{\pgfqpoint{4.932908in}{3.309051in}}{\pgfqpoint{4.943507in}{3.304660in}}{\pgfqpoint{4.954557in}{3.304660in}}%
\pgfpathclose%
\pgfusepath{stroke,fill}%
\end{pgfscope}%
\begin{pgfscope}%
\pgfpathrectangle{\pgfqpoint{0.570343in}{0.331635in}}{\pgfqpoint{9.300000in}{7.700000in}}%
\pgfusepath{clip}%
\pgfsetbuttcap%
\pgfsetroundjoin%
\definecolor{currentfill}{rgb}{0.631373,0.788235,0.956863}%
\pgfsetfillcolor{currentfill}%
\pgfsetlinewidth{0.481800pt}%
\definecolor{currentstroke}{rgb}{1.000000,1.000000,1.000000}%
\pgfsetstrokecolor{currentstroke}%
\pgfsetdash{}{0pt}%
\pgfpathmoveto{\pgfqpoint{2.539065in}{1.355485in}}%
\pgfpathcurveto{\pgfqpoint{2.550115in}{1.355485in}}{\pgfqpoint{2.560714in}{1.359876in}}{\pgfqpoint{2.568527in}{1.367689in}}%
\pgfpathcurveto{\pgfqpoint{2.576341in}{1.375503in}}{\pgfqpoint{2.580731in}{1.386102in}}{\pgfqpoint{2.580731in}{1.397152in}}%
\pgfpathcurveto{\pgfqpoint{2.580731in}{1.408202in}}{\pgfqpoint{2.576341in}{1.418801in}}{\pgfqpoint{2.568527in}{1.426615in}}%
\pgfpathcurveto{\pgfqpoint{2.560714in}{1.434429in}}{\pgfqpoint{2.550115in}{1.438819in}}{\pgfqpoint{2.539065in}{1.438819in}}%
\pgfpathcurveto{\pgfqpoint{2.528015in}{1.438819in}}{\pgfqpoint{2.517415in}{1.434429in}}{\pgfqpoint{2.509602in}{1.426615in}}%
\pgfpathcurveto{\pgfqpoint{2.501788in}{1.418801in}}{\pgfqpoint{2.497398in}{1.408202in}}{\pgfqpoint{2.497398in}{1.397152in}}%
\pgfpathcurveto{\pgfqpoint{2.497398in}{1.386102in}}{\pgfqpoint{2.501788in}{1.375503in}}{\pgfqpoint{2.509602in}{1.367689in}}%
\pgfpathcurveto{\pgfqpoint{2.517415in}{1.359876in}}{\pgfqpoint{2.528015in}{1.355485in}}{\pgfqpoint{2.539065in}{1.355485in}}%
\pgfpathclose%
\pgfusepath{stroke,fill}%
\end{pgfscope}%
\begin{pgfscope}%
\pgfpathrectangle{\pgfqpoint{0.570343in}{0.331635in}}{\pgfqpoint{9.300000in}{7.700000in}}%
\pgfusepath{clip}%
\pgfsetbuttcap%
\pgfsetroundjoin%
\definecolor{currentfill}{rgb}{0.631373,0.788235,0.956863}%
\pgfsetfillcolor{currentfill}%
\pgfsetlinewidth{0.481800pt}%
\definecolor{currentstroke}{rgb}{1.000000,1.000000,1.000000}%
\pgfsetstrokecolor{currentstroke}%
\pgfsetdash{}{0pt}%
\pgfpathmoveto{\pgfqpoint{6.972947in}{3.668887in}}%
\pgfpathcurveto{\pgfqpoint{6.983997in}{3.668887in}}{\pgfqpoint{6.994596in}{3.673277in}}{\pgfqpoint{7.002409in}{3.681091in}}%
\pgfpathcurveto{\pgfqpoint{7.010223in}{3.688904in}}{\pgfqpoint{7.014613in}{3.699503in}}{\pgfqpoint{7.014613in}{3.710554in}}%
\pgfpathcurveto{\pgfqpoint{7.014613in}{3.721604in}}{\pgfqpoint{7.010223in}{3.732203in}}{\pgfqpoint{7.002409in}{3.740016in}}%
\pgfpathcurveto{\pgfqpoint{6.994596in}{3.747830in}}{\pgfqpoint{6.983997in}{3.752220in}}{\pgfqpoint{6.972947in}{3.752220in}}%
\pgfpathcurveto{\pgfqpoint{6.961897in}{3.752220in}}{\pgfqpoint{6.951298in}{3.747830in}}{\pgfqpoint{6.943484in}{3.740016in}}%
\pgfpathcurveto{\pgfqpoint{6.935670in}{3.732203in}}{\pgfqpoint{6.931280in}{3.721604in}}{\pgfqpoint{6.931280in}{3.710554in}}%
\pgfpathcurveto{\pgfqpoint{6.931280in}{3.699503in}}{\pgfqpoint{6.935670in}{3.688904in}}{\pgfqpoint{6.943484in}{3.681091in}}%
\pgfpathcurveto{\pgfqpoint{6.951298in}{3.673277in}}{\pgfqpoint{6.961897in}{3.668887in}}{\pgfqpoint{6.972947in}{3.668887in}}%
\pgfpathclose%
\pgfusepath{stroke,fill}%
\end{pgfscope}%
\begin{pgfscope}%
\pgfpathrectangle{\pgfqpoint{0.570343in}{0.331635in}}{\pgfqpoint{9.300000in}{7.700000in}}%
\pgfusepath{clip}%
\pgfsetbuttcap%
\pgfsetroundjoin%
\definecolor{currentfill}{rgb}{0.631373,0.788235,0.956863}%
\pgfsetfillcolor{currentfill}%
\pgfsetlinewidth{0.481800pt}%
\definecolor{currentstroke}{rgb}{1.000000,1.000000,1.000000}%
\pgfsetstrokecolor{currentstroke}%
\pgfsetdash{}{0pt}%
\pgfpathmoveto{\pgfqpoint{5.923406in}{3.054205in}}%
\pgfpathcurveto{\pgfqpoint{5.934456in}{3.054205in}}{\pgfqpoint{5.945055in}{3.058595in}}{\pgfqpoint{5.952869in}{3.066409in}}%
\pgfpathcurveto{\pgfqpoint{5.960682in}{3.074222in}}{\pgfqpoint{5.965073in}{3.084821in}}{\pgfqpoint{5.965073in}{3.095871in}}%
\pgfpathcurveto{\pgfqpoint{5.965073in}{3.106922in}}{\pgfqpoint{5.960682in}{3.117521in}}{\pgfqpoint{5.952869in}{3.125334in}}%
\pgfpathcurveto{\pgfqpoint{5.945055in}{3.133148in}}{\pgfqpoint{5.934456in}{3.137538in}}{\pgfqpoint{5.923406in}{3.137538in}}%
\pgfpathcurveto{\pgfqpoint{5.912356in}{3.137538in}}{\pgfqpoint{5.901757in}{3.133148in}}{\pgfqpoint{5.893943in}{3.125334in}}%
\pgfpathcurveto{\pgfqpoint{5.886130in}{3.117521in}}{\pgfqpoint{5.881739in}{3.106922in}}{\pgfqpoint{5.881739in}{3.095871in}}%
\pgfpathcurveto{\pgfqpoint{5.881739in}{3.084821in}}{\pgfqpoint{5.886130in}{3.074222in}}{\pgfqpoint{5.893943in}{3.066409in}}%
\pgfpathcurveto{\pgfqpoint{5.901757in}{3.058595in}}{\pgfqpoint{5.912356in}{3.054205in}}{\pgfqpoint{5.923406in}{3.054205in}}%
\pgfpathclose%
\pgfusepath{stroke,fill}%
\end{pgfscope}%
\begin{pgfscope}%
\pgfpathrectangle{\pgfqpoint{0.570343in}{0.331635in}}{\pgfqpoint{9.300000in}{7.700000in}}%
\pgfusepath{clip}%
\pgfsetbuttcap%
\pgfsetroundjoin%
\definecolor{currentfill}{rgb}{1.000000,0.705882,0.509804}%
\pgfsetfillcolor{currentfill}%
\pgfsetlinewidth{0.481800pt}%
\definecolor{currentstroke}{rgb}{1.000000,1.000000,1.000000}%
\pgfsetstrokecolor{currentstroke}%
\pgfsetdash{}{0pt}%
\pgfpathmoveto{\pgfqpoint{6.817529in}{4.793902in}}%
\pgfpathcurveto{\pgfqpoint{6.828579in}{4.793902in}}{\pgfqpoint{6.839178in}{4.798293in}}{\pgfqpoint{6.846992in}{4.806106in}}%
\pgfpathcurveto{\pgfqpoint{6.854805in}{4.813920in}}{\pgfqpoint{6.859196in}{4.824519in}}{\pgfqpoint{6.859196in}{4.835569in}}%
\pgfpathcurveto{\pgfqpoint{6.859196in}{4.846619in}}{\pgfqpoint{6.854805in}{4.857218in}}{\pgfqpoint{6.846992in}{4.865032in}}%
\pgfpathcurveto{\pgfqpoint{6.839178in}{4.872845in}}{\pgfqpoint{6.828579in}{4.877236in}}{\pgfqpoint{6.817529in}{4.877236in}}%
\pgfpathcurveto{\pgfqpoint{6.806479in}{4.877236in}}{\pgfqpoint{6.795880in}{4.872845in}}{\pgfqpoint{6.788066in}{4.865032in}}%
\pgfpathcurveto{\pgfqpoint{6.780253in}{4.857218in}}{\pgfqpoint{6.775862in}{4.846619in}}{\pgfqpoint{6.775862in}{4.835569in}}%
\pgfpathcurveto{\pgfqpoint{6.775862in}{4.824519in}}{\pgfqpoint{6.780253in}{4.813920in}}{\pgfqpoint{6.788066in}{4.806106in}}%
\pgfpathcurveto{\pgfqpoint{6.795880in}{4.798293in}}{\pgfqpoint{6.806479in}{4.793902in}}{\pgfqpoint{6.817529in}{4.793902in}}%
\pgfpathclose%
\pgfusepath{stroke,fill}%
\end{pgfscope}%
\begin{pgfscope}%
\pgfpathrectangle{\pgfqpoint{0.570343in}{0.331635in}}{\pgfqpoint{9.300000in}{7.700000in}}%
\pgfusepath{clip}%
\pgfsetbuttcap%
\pgfsetroundjoin%
\definecolor{currentfill}{rgb}{1.000000,0.705882,0.509804}%
\pgfsetfillcolor{currentfill}%
\pgfsetlinewidth{0.481800pt}%
\definecolor{currentstroke}{rgb}{1.000000,1.000000,1.000000}%
\pgfsetstrokecolor{currentstroke}%
\pgfsetdash{}{0pt}%
\pgfpathmoveto{\pgfqpoint{6.117416in}{7.368199in}}%
\pgfpathcurveto{\pgfqpoint{6.128466in}{7.368199in}}{\pgfqpoint{6.139065in}{7.372590in}}{\pgfqpoint{6.146878in}{7.380403in}}%
\pgfpathcurveto{\pgfqpoint{6.154692in}{7.388217in}}{\pgfqpoint{6.159082in}{7.398816in}}{\pgfqpoint{6.159082in}{7.409866in}}%
\pgfpathcurveto{\pgfqpoint{6.159082in}{7.420916in}}{\pgfqpoint{6.154692in}{7.431515in}}{\pgfqpoint{6.146878in}{7.439329in}}%
\pgfpathcurveto{\pgfqpoint{6.139065in}{7.447142in}}{\pgfqpoint{6.128466in}{7.451533in}}{\pgfqpoint{6.117416in}{7.451533in}}%
\pgfpathcurveto{\pgfqpoint{6.106365in}{7.451533in}}{\pgfqpoint{6.095766in}{7.447142in}}{\pgfqpoint{6.087953in}{7.439329in}}%
\pgfpathcurveto{\pgfqpoint{6.080139in}{7.431515in}}{\pgfqpoint{6.075749in}{7.420916in}}{\pgfqpoint{6.075749in}{7.409866in}}%
\pgfpathcurveto{\pgfqpoint{6.075749in}{7.398816in}}{\pgfqpoint{6.080139in}{7.388217in}}{\pgfqpoint{6.087953in}{7.380403in}}%
\pgfpathcurveto{\pgfqpoint{6.095766in}{7.372590in}}{\pgfqpoint{6.106365in}{7.368199in}}{\pgfqpoint{6.117416in}{7.368199in}}%
\pgfpathclose%
\pgfusepath{stroke,fill}%
\end{pgfscope}%
\begin{pgfscope}%
\pgfpathrectangle{\pgfqpoint{0.570343in}{0.331635in}}{\pgfqpoint{9.300000in}{7.700000in}}%
\pgfusepath{clip}%
\pgfsetbuttcap%
\pgfsetroundjoin%
\definecolor{currentfill}{rgb}{1.000000,0.705882,0.509804}%
\pgfsetfillcolor{currentfill}%
\pgfsetlinewidth{0.481800pt}%
\definecolor{currentstroke}{rgb}{1.000000,1.000000,1.000000}%
\pgfsetstrokecolor{currentstroke}%
\pgfsetdash{}{0pt}%
\pgfpathmoveto{\pgfqpoint{7.933795in}{3.368900in}}%
\pgfpathcurveto{\pgfqpoint{7.944845in}{3.368900in}}{\pgfqpoint{7.955444in}{3.373290in}}{\pgfqpoint{7.963258in}{3.381104in}}%
\pgfpathcurveto{\pgfqpoint{7.971071in}{3.388917in}}{\pgfqpoint{7.975462in}{3.399516in}}{\pgfqpoint{7.975462in}{3.410567in}}%
\pgfpathcurveto{\pgfqpoint{7.975462in}{3.421617in}}{\pgfqpoint{7.971071in}{3.432216in}}{\pgfqpoint{7.963258in}{3.440029in}}%
\pgfpathcurveto{\pgfqpoint{7.955444in}{3.447843in}}{\pgfqpoint{7.944845in}{3.452233in}}{\pgfqpoint{7.933795in}{3.452233in}}%
\pgfpathcurveto{\pgfqpoint{7.922745in}{3.452233in}}{\pgfqpoint{7.912146in}{3.447843in}}{\pgfqpoint{7.904332in}{3.440029in}}%
\pgfpathcurveto{\pgfqpoint{7.896519in}{3.432216in}}{\pgfqpoint{7.892128in}{3.421617in}}{\pgfqpoint{7.892128in}{3.410567in}}%
\pgfpathcurveto{\pgfqpoint{7.892128in}{3.399516in}}{\pgfqpoint{7.896519in}{3.388917in}}{\pgfqpoint{7.904332in}{3.381104in}}%
\pgfpathcurveto{\pgfqpoint{7.912146in}{3.373290in}}{\pgfqpoint{7.922745in}{3.368900in}}{\pgfqpoint{7.933795in}{3.368900in}}%
\pgfpathclose%
\pgfusepath{stroke,fill}%
\end{pgfscope}%
\begin{pgfscope}%
\pgfpathrectangle{\pgfqpoint{0.570343in}{0.331635in}}{\pgfqpoint{9.300000in}{7.700000in}}%
\pgfusepath{clip}%
\pgfsetbuttcap%
\pgfsetroundjoin%
\definecolor{currentfill}{rgb}{1.000000,0.705882,0.509804}%
\pgfsetfillcolor{currentfill}%
\pgfsetlinewidth{0.481800pt}%
\definecolor{currentstroke}{rgb}{1.000000,1.000000,1.000000}%
\pgfsetstrokecolor{currentstroke}%
\pgfsetdash{}{0pt}%
\pgfpathmoveto{\pgfqpoint{7.377901in}{4.220747in}}%
\pgfpathcurveto{\pgfqpoint{7.388951in}{4.220747in}}{\pgfqpoint{7.399550in}{4.225138in}}{\pgfqpoint{7.407364in}{4.232951in}}%
\pgfpathcurveto{\pgfqpoint{7.415178in}{4.240765in}}{\pgfqpoint{7.419568in}{4.251364in}}{\pgfqpoint{7.419568in}{4.262414in}}%
\pgfpathcurveto{\pgfqpoint{7.419568in}{4.273464in}}{\pgfqpoint{7.415178in}{4.284063in}}{\pgfqpoint{7.407364in}{4.291877in}}%
\pgfpathcurveto{\pgfqpoint{7.399550in}{4.299690in}}{\pgfqpoint{7.388951in}{4.304081in}}{\pgfqpoint{7.377901in}{4.304081in}}%
\pgfpathcurveto{\pgfqpoint{7.366851in}{4.304081in}}{\pgfqpoint{7.356252in}{4.299690in}}{\pgfqpoint{7.348438in}{4.291877in}}%
\pgfpathcurveto{\pgfqpoint{7.340625in}{4.284063in}}{\pgfqpoint{7.336235in}{4.273464in}}{\pgfqpoint{7.336235in}{4.262414in}}%
\pgfpathcurveto{\pgfqpoint{7.336235in}{4.251364in}}{\pgfqpoint{7.340625in}{4.240765in}}{\pgfqpoint{7.348438in}{4.232951in}}%
\pgfpathcurveto{\pgfqpoint{7.356252in}{4.225138in}}{\pgfqpoint{7.366851in}{4.220747in}}{\pgfqpoint{7.377901in}{4.220747in}}%
\pgfpathclose%
\pgfusepath{stroke,fill}%
\end{pgfscope}%
\begin{pgfscope}%
\pgfpathrectangle{\pgfqpoint{0.570343in}{0.331635in}}{\pgfqpoint{9.300000in}{7.700000in}}%
\pgfusepath{clip}%
\pgfsetbuttcap%
\pgfsetroundjoin%
\definecolor{currentfill}{rgb}{1.000000,0.705882,0.509804}%
\pgfsetfillcolor{currentfill}%
\pgfsetlinewidth{0.481800pt}%
\definecolor{currentstroke}{rgb}{1.000000,1.000000,1.000000}%
\pgfsetstrokecolor{currentstroke}%
\pgfsetdash{}{0pt}%
\pgfpathmoveto{\pgfqpoint{7.284262in}{2.105459in}}%
\pgfpathcurveto{\pgfqpoint{7.295312in}{2.105459in}}{\pgfqpoint{7.305911in}{2.109849in}}{\pgfqpoint{7.313724in}{2.117662in}}%
\pgfpathcurveto{\pgfqpoint{7.321538in}{2.125476in}}{\pgfqpoint{7.325928in}{2.136075in}}{\pgfqpoint{7.325928in}{2.147125in}}%
\pgfpathcurveto{\pgfqpoint{7.325928in}{2.158175in}}{\pgfqpoint{7.321538in}{2.168774in}}{\pgfqpoint{7.313724in}{2.176588in}}%
\pgfpathcurveto{\pgfqpoint{7.305911in}{2.184402in}}{\pgfqpoint{7.295312in}{2.188792in}}{\pgfqpoint{7.284262in}{2.188792in}}%
\pgfpathcurveto{\pgfqpoint{7.273211in}{2.188792in}}{\pgfqpoint{7.262612in}{2.184402in}}{\pgfqpoint{7.254799in}{2.176588in}}%
\pgfpathcurveto{\pgfqpoint{7.246985in}{2.168774in}}{\pgfqpoint{7.242595in}{2.158175in}}{\pgfqpoint{7.242595in}{2.147125in}}%
\pgfpathcurveto{\pgfqpoint{7.242595in}{2.136075in}}{\pgfqpoint{7.246985in}{2.125476in}}{\pgfqpoint{7.254799in}{2.117662in}}%
\pgfpathcurveto{\pgfqpoint{7.262612in}{2.109849in}}{\pgfqpoint{7.273211in}{2.105459in}}{\pgfqpoint{7.284262in}{2.105459in}}%
\pgfpathclose%
\pgfusepath{stroke,fill}%
\end{pgfscope}%
\begin{pgfscope}%
\pgfpathrectangle{\pgfqpoint{0.570343in}{0.331635in}}{\pgfqpoint{9.300000in}{7.700000in}}%
\pgfusepath{clip}%
\pgfsetbuttcap%
\pgfsetroundjoin%
\definecolor{currentfill}{rgb}{1.000000,0.705882,0.509804}%
\pgfsetfillcolor{currentfill}%
\pgfsetlinewidth{0.481800pt}%
\definecolor{currentstroke}{rgb}{1.000000,1.000000,1.000000}%
\pgfsetstrokecolor{currentstroke}%
\pgfsetdash{}{0pt}%
\pgfpathmoveto{\pgfqpoint{5.804302in}{4.421587in}}%
\pgfpathcurveto{\pgfqpoint{5.815352in}{4.421587in}}{\pgfqpoint{5.825951in}{4.425977in}}{\pgfqpoint{5.833765in}{4.433791in}}%
\pgfpathcurveto{\pgfqpoint{5.841579in}{4.441605in}}{\pgfqpoint{5.845969in}{4.452204in}}{\pgfqpoint{5.845969in}{4.463254in}}%
\pgfpathcurveto{\pgfqpoint{5.845969in}{4.474304in}}{\pgfqpoint{5.841579in}{4.484903in}}{\pgfqpoint{5.833765in}{4.492717in}}%
\pgfpathcurveto{\pgfqpoint{5.825951in}{4.500530in}}{\pgfqpoint{5.815352in}{4.504920in}}{\pgfqpoint{5.804302in}{4.504920in}}%
\pgfpathcurveto{\pgfqpoint{5.793252in}{4.504920in}}{\pgfqpoint{5.782653in}{4.500530in}}{\pgfqpoint{5.774839in}{4.492717in}}%
\pgfpathcurveto{\pgfqpoint{5.767026in}{4.484903in}}{\pgfqpoint{5.762635in}{4.474304in}}{\pgfqpoint{5.762635in}{4.463254in}}%
\pgfpathcurveto{\pgfqpoint{5.762635in}{4.452204in}}{\pgfqpoint{5.767026in}{4.441605in}}{\pgfqpoint{5.774839in}{4.433791in}}%
\pgfpathcurveto{\pgfqpoint{5.782653in}{4.425977in}}{\pgfqpoint{5.793252in}{4.421587in}}{\pgfqpoint{5.804302in}{4.421587in}}%
\pgfpathclose%
\pgfusepath{stroke,fill}%
\end{pgfscope}%
\begin{pgfscope}%
\pgfpathrectangle{\pgfqpoint{0.570343in}{0.331635in}}{\pgfqpoint{9.300000in}{7.700000in}}%
\pgfusepath{clip}%
\pgfsetbuttcap%
\pgfsetroundjoin%
\definecolor{currentfill}{rgb}{1.000000,0.705882,0.509804}%
\pgfsetfillcolor{currentfill}%
\pgfsetlinewidth{0.481800pt}%
\definecolor{currentstroke}{rgb}{1.000000,1.000000,1.000000}%
\pgfsetstrokecolor{currentstroke}%
\pgfsetdash{}{0pt}%
\pgfpathmoveto{\pgfqpoint{9.282814in}{3.497977in}}%
\pgfpathcurveto{\pgfqpoint{9.293864in}{3.497977in}}{\pgfqpoint{9.304463in}{3.502367in}}{\pgfqpoint{9.312277in}{3.510181in}}%
\pgfpathcurveto{\pgfqpoint{9.320091in}{3.517994in}}{\pgfqpoint{9.324481in}{3.528593in}}{\pgfqpoint{9.324481in}{3.539643in}}%
\pgfpathcurveto{\pgfqpoint{9.324481in}{3.550694in}}{\pgfqpoint{9.320091in}{3.561293in}}{\pgfqpoint{9.312277in}{3.569106in}}%
\pgfpathcurveto{\pgfqpoint{9.304463in}{3.576920in}}{\pgfqpoint{9.293864in}{3.581310in}}{\pgfqpoint{9.282814in}{3.581310in}}%
\pgfpathcurveto{\pgfqpoint{9.271764in}{3.581310in}}{\pgfqpoint{9.261165in}{3.576920in}}{\pgfqpoint{9.253351in}{3.569106in}}%
\pgfpathcurveto{\pgfqpoint{9.245538in}{3.561293in}}{\pgfqpoint{9.241148in}{3.550694in}}{\pgfqpoint{9.241148in}{3.539643in}}%
\pgfpathcurveto{\pgfqpoint{9.241148in}{3.528593in}}{\pgfqpoint{9.245538in}{3.517994in}}{\pgfqpoint{9.253351in}{3.510181in}}%
\pgfpathcurveto{\pgfqpoint{9.261165in}{3.502367in}}{\pgfqpoint{9.271764in}{3.497977in}}{\pgfqpoint{9.282814in}{3.497977in}}%
\pgfpathclose%
\pgfusepath{stroke,fill}%
\end{pgfscope}%
\begin{pgfscope}%
\pgfpathrectangle{\pgfqpoint{0.570343in}{0.331635in}}{\pgfqpoint{9.300000in}{7.700000in}}%
\pgfusepath{clip}%
\pgfsetbuttcap%
\pgfsetroundjoin%
\definecolor{currentfill}{rgb}{1.000000,0.705882,0.509804}%
\pgfsetfillcolor{currentfill}%
\pgfsetlinewidth{0.481800pt}%
\definecolor{currentstroke}{rgb}{1.000000,1.000000,1.000000}%
\pgfsetstrokecolor{currentstroke}%
\pgfsetdash{}{0pt}%
\pgfpathmoveto{\pgfqpoint{5.814279in}{3.876417in}}%
\pgfpathcurveto{\pgfqpoint{5.825329in}{3.876417in}}{\pgfqpoint{5.835928in}{3.880807in}}{\pgfqpoint{5.843741in}{3.888620in}}%
\pgfpathcurveto{\pgfqpoint{5.851555in}{3.896434in}}{\pgfqpoint{5.855945in}{3.907033in}}{\pgfqpoint{5.855945in}{3.918083in}}%
\pgfpathcurveto{\pgfqpoint{5.855945in}{3.929133in}}{\pgfqpoint{5.851555in}{3.939732in}}{\pgfqpoint{5.843741in}{3.947546in}}%
\pgfpathcurveto{\pgfqpoint{5.835928in}{3.955360in}}{\pgfqpoint{5.825329in}{3.959750in}}{\pgfqpoint{5.814279in}{3.959750in}}%
\pgfpathcurveto{\pgfqpoint{5.803229in}{3.959750in}}{\pgfqpoint{5.792630in}{3.955360in}}{\pgfqpoint{5.784816in}{3.947546in}}%
\pgfpathcurveto{\pgfqpoint{5.777002in}{3.939732in}}{\pgfqpoint{5.772612in}{3.929133in}}{\pgfqpoint{5.772612in}{3.918083in}}%
\pgfpathcurveto{\pgfqpoint{5.772612in}{3.907033in}}{\pgfqpoint{5.777002in}{3.896434in}}{\pgfqpoint{5.784816in}{3.888620in}}%
\pgfpathcurveto{\pgfqpoint{5.792630in}{3.880807in}}{\pgfqpoint{5.803229in}{3.876417in}}{\pgfqpoint{5.814279in}{3.876417in}}%
\pgfpathclose%
\pgfusepath{stroke,fill}%
\end{pgfscope}%
\begin{pgfscope}%
\pgfpathrectangle{\pgfqpoint{0.570343in}{0.331635in}}{\pgfqpoint{9.300000in}{7.700000in}}%
\pgfusepath{clip}%
\pgfsetbuttcap%
\pgfsetroundjoin%
\definecolor{currentfill}{rgb}{1.000000,0.705882,0.509804}%
\pgfsetfillcolor{currentfill}%
\pgfsetlinewidth{0.481800pt}%
\definecolor{currentstroke}{rgb}{1.000000,1.000000,1.000000}%
\pgfsetstrokecolor{currentstroke}%
\pgfsetdash{}{0pt}%
\pgfpathmoveto{\pgfqpoint{6.347009in}{5.397079in}}%
\pgfpathcurveto{\pgfqpoint{6.358059in}{5.397079in}}{\pgfqpoint{6.368658in}{5.401470in}}{\pgfqpoint{6.376472in}{5.409283in}}%
\pgfpathcurveto{\pgfqpoint{6.384285in}{5.417097in}}{\pgfqpoint{6.388676in}{5.427696in}}{\pgfqpoint{6.388676in}{5.438746in}}%
\pgfpathcurveto{\pgfqpoint{6.388676in}{5.449796in}}{\pgfqpoint{6.384285in}{5.460395in}}{\pgfqpoint{6.376472in}{5.468209in}}%
\pgfpathcurveto{\pgfqpoint{6.368658in}{5.476022in}}{\pgfqpoint{6.358059in}{5.480413in}}{\pgfqpoint{6.347009in}{5.480413in}}%
\pgfpathcurveto{\pgfqpoint{6.335959in}{5.480413in}}{\pgfqpoint{6.325360in}{5.476022in}}{\pgfqpoint{6.317546in}{5.468209in}}%
\pgfpathcurveto{\pgfqpoint{6.309733in}{5.460395in}}{\pgfqpoint{6.305342in}{5.449796in}}{\pgfqpoint{6.305342in}{5.438746in}}%
\pgfpathcurveto{\pgfqpoint{6.305342in}{5.427696in}}{\pgfqpoint{6.309733in}{5.417097in}}{\pgfqpoint{6.317546in}{5.409283in}}%
\pgfpathcurveto{\pgfqpoint{6.325360in}{5.401470in}}{\pgfqpoint{6.335959in}{5.397079in}}{\pgfqpoint{6.347009in}{5.397079in}}%
\pgfpathclose%
\pgfusepath{stroke,fill}%
\end{pgfscope}%
\begin{pgfscope}%
\pgfpathrectangle{\pgfqpoint{0.570343in}{0.331635in}}{\pgfqpoint{9.300000in}{7.700000in}}%
\pgfusepath{clip}%
\pgfsetbuttcap%
\pgfsetroundjoin%
\definecolor{currentfill}{rgb}{1.000000,0.705882,0.509804}%
\pgfsetfillcolor{currentfill}%
\pgfsetlinewidth{0.481800pt}%
\definecolor{currentstroke}{rgb}{1.000000,1.000000,1.000000}%
\pgfsetstrokecolor{currentstroke}%
\pgfsetdash{}{0pt}%
\pgfpathmoveto{\pgfqpoint{8.386537in}{2.506416in}}%
\pgfpathcurveto{\pgfqpoint{8.397587in}{2.506416in}}{\pgfqpoint{8.408186in}{2.510806in}}{\pgfqpoint{8.415999in}{2.518619in}}%
\pgfpathcurveto{\pgfqpoint{8.423813in}{2.526433in}}{\pgfqpoint{8.428203in}{2.537032in}}{\pgfqpoint{8.428203in}{2.548082in}}%
\pgfpathcurveto{\pgfqpoint{8.428203in}{2.559132in}}{\pgfqpoint{8.423813in}{2.569731in}}{\pgfqpoint{8.415999in}{2.577545in}}%
\pgfpathcurveto{\pgfqpoint{8.408186in}{2.585359in}}{\pgfqpoint{8.397587in}{2.589749in}}{\pgfqpoint{8.386537in}{2.589749in}}%
\pgfpathcurveto{\pgfqpoint{8.375486in}{2.589749in}}{\pgfqpoint{8.364887in}{2.585359in}}{\pgfqpoint{8.357074in}{2.577545in}}%
\pgfpathcurveto{\pgfqpoint{8.349260in}{2.569731in}}{\pgfqpoint{8.344870in}{2.559132in}}{\pgfqpoint{8.344870in}{2.548082in}}%
\pgfpathcurveto{\pgfqpoint{8.344870in}{2.537032in}}{\pgfqpoint{8.349260in}{2.526433in}}{\pgfqpoint{8.357074in}{2.518619in}}%
\pgfpathcurveto{\pgfqpoint{8.364887in}{2.510806in}}{\pgfqpoint{8.375486in}{2.506416in}}{\pgfqpoint{8.386537in}{2.506416in}}%
\pgfpathclose%
\pgfusepath{stroke,fill}%
\end{pgfscope}%
\begin{pgfscope}%
\pgfpathrectangle{\pgfqpoint{0.570343in}{0.331635in}}{\pgfqpoint{9.300000in}{7.700000in}}%
\pgfusepath{clip}%
\pgfsetbuttcap%
\pgfsetroundjoin%
\definecolor{currentfill}{rgb}{1.000000,0.705882,0.509804}%
\pgfsetfillcolor{currentfill}%
\pgfsetlinewidth{0.481800pt}%
\definecolor{currentstroke}{rgb}{1.000000,1.000000,1.000000}%
\pgfsetstrokecolor{currentstroke}%
\pgfsetdash{}{0pt}%
\pgfpathmoveto{\pgfqpoint{8.388882in}{3.844007in}}%
\pgfpathcurveto{\pgfqpoint{8.399932in}{3.844007in}}{\pgfqpoint{8.410531in}{3.848398in}}{\pgfqpoint{8.418345in}{3.856211in}}%
\pgfpathcurveto{\pgfqpoint{8.426159in}{3.864025in}}{\pgfqpoint{8.430549in}{3.874624in}}{\pgfqpoint{8.430549in}{3.885674in}}%
\pgfpathcurveto{\pgfqpoint{8.430549in}{3.896724in}}{\pgfqpoint{8.426159in}{3.907323in}}{\pgfqpoint{8.418345in}{3.915137in}}%
\pgfpathcurveto{\pgfqpoint{8.410531in}{3.922950in}}{\pgfqpoint{8.399932in}{3.927341in}}{\pgfqpoint{8.388882in}{3.927341in}}%
\pgfpathcurveto{\pgfqpoint{8.377832in}{3.927341in}}{\pgfqpoint{8.367233in}{3.922950in}}{\pgfqpoint{8.359419in}{3.915137in}}%
\pgfpathcurveto{\pgfqpoint{8.351606in}{3.907323in}}{\pgfqpoint{8.347216in}{3.896724in}}{\pgfqpoint{8.347216in}{3.885674in}}%
\pgfpathcurveto{\pgfqpoint{8.347216in}{3.874624in}}{\pgfqpoint{8.351606in}{3.864025in}}{\pgfqpoint{8.359419in}{3.856211in}}%
\pgfpathcurveto{\pgfqpoint{8.367233in}{3.848398in}}{\pgfqpoint{8.377832in}{3.844007in}}{\pgfqpoint{8.388882in}{3.844007in}}%
\pgfpathclose%
\pgfusepath{stroke,fill}%
\end{pgfscope}%
\begin{pgfscope}%
\pgfpathrectangle{\pgfqpoint{0.570343in}{0.331635in}}{\pgfqpoint{9.300000in}{7.700000in}}%
\pgfusepath{clip}%
\pgfsetbuttcap%
\pgfsetroundjoin%
\definecolor{currentfill}{rgb}{1.000000,0.705882,0.509804}%
\pgfsetfillcolor{currentfill}%
\pgfsetlinewidth{0.481800pt}%
\definecolor{currentstroke}{rgb}{1.000000,1.000000,1.000000}%
\pgfsetstrokecolor{currentstroke}%
\pgfsetdash{}{0pt}%
\pgfpathmoveto{\pgfqpoint{3.136766in}{5.808604in}}%
\pgfpathcurveto{\pgfqpoint{3.147816in}{5.808604in}}{\pgfqpoint{3.158415in}{5.812994in}}{\pgfqpoint{3.166229in}{5.820807in}}%
\pgfpathcurveto{\pgfqpoint{3.174042in}{5.828621in}}{\pgfqpoint{3.178432in}{5.839220in}}{\pgfqpoint{3.178432in}{5.850270in}}%
\pgfpathcurveto{\pgfqpoint{3.178432in}{5.861320in}}{\pgfqpoint{3.174042in}{5.871919in}}{\pgfqpoint{3.166229in}{5.879733in}}%
\pgfpathcurveto{\pgfqpoint{3.158415in}{5.887547in}}{\pgfqpoint{3.147816in}{5.891937in}}{\pgfqpoint{3.136766in}{5.891937in}}%
\pgfpathcurveto{\pgfqpoint{3.125716in}{5.891937in}}{\pgfqpoint{3.115117in}{5.887547in}}{\pgfqpoint{3.107303in}{5.879733in}}%
\pgfpathcurveto{\pgfqpoint{3.099489in}{5.871919in}}{\pgfqpoint{3.095099in}{5.861320in}}{\pgfqpoint{3.095099in}{5.850270in}}%
\pgfpathcurveto{\pgfqpoint{3.095099in}{5.839220in}}{\pgfqpoint{3.099489in}{5.828621in}}{\pgfqpoint{3.107303in}{5.820807in}}%
\pgfpathcurveto{\pgfqpoint{3.115117in}{5.812994in}}{\pgfqpoint{3.125716in}{5.808604in}}{\pgfqpoint{3.136766in}{5.808604in}}%
\pgfpathclose%
\pgfusepath{stroke,fill}%
\end{pgfscope}%
\begin{pgfscope}%
\pgfpathrectangle{\pgfqpoint{0.570343in}{0.331635in}}{\pgfqpoint{9.300000in}{7.700000in}}%
\pgfusepath{clip}%
\pgfsetbuttcap%
\pgfsetroundjoin%
\definecolor{currentfill}{rgb}{1.000000,0.705882,0.509804}%
\pgfsetfillcolor{currentfill}%
\pgfsetlinewidth{0.481800pt}%
\definecolor{currentstroke}{rgb}{1.000000,1.000000,1.000000}%
\pgfsetstrokecolor{currentstroke}%
\pgfsetdash{}{0pt}%
\pgfpathmoveto{\pgfqpoint{5.512434in}{5.962742in}}%
\pgfpathcurveto{\pgfqpoint{5.523484in}{5.962742in}}{\pgfqpoint{5.534083in}{5.967132in}}{\pgfqpoint{5.541896in}{5.974946in}}%
\pgfpathcurveto{\pgfqpoint{5.549710in}{5.982760in}}{\pgfqpoint{5.554100in}{5.993359in}}{\pgfqpoint{5.554100in}{6.004409in}}%
\pgfpathcurveto{\pgfqpoint{5.554100in}{6.015459in}}{\pgfqpoint{5.549710in}{6.026058in}}{\pgfqpoint{5.541896in}{6.033872in}}%
\pgfpathcurveto{\pgfqpoint{5.534083in}{6.041685in}}{\pgfqpoint{5.523484in}{6.046075in}}{\pgfqpoint{5.512434in}{6.046075in}}%
\pgfpathcurveto{\pgfqpoint{5.501384in}{6.046075in}}{\pgfqpoint{5.490785in}{6.041685in}}{\pgfqpoint{5.482971in}{6.033872in}}%
\pgfpathcurveto{\pgfqpoint{5.475157in}{6.026058in}}{\pgfqpoint{5.470767in}{6.015459in}}{\pgfqpoint{5.470767in}{6.004409in}}%
\pgfpathcurveto{\pgfqpoint{5.470767in}{5.993359in}}{\pgfqpoint{5.475157in}{5.982760in}}{\pgfqpoint{5.482971in}{5.974946in}}%
\pgfpathcurveto{\pgfqpoint{5.490785in}{5.967132in}}{\pgfqpoint{5.501384in}{5.962742in}}{\pgfqpoint{5.512434in}{5.962742in}}%
\pgfpathclose%
\pgfusepath{stroke,fill}%
\end{pgfscope}%
\begin{pgfscope}%
\pgfpathrectangle{\pgfqpoint{0.570343in}{0.331635in}}{\pgfqpoint{9.300000in}{7.700000in}}%
\pgfusepath{clip}%
\pgfsetbuttcap%
\pgfsetroundjoin%
\definecolor{currentfill}{rgb}{1.000000,0.705882,0.509804}%
\pgfsetfillcolor{currentfill}%
\pgfsetlinewidth{0.481800pt}%
\definecolor{currentstroke}{rgb}{1.000000,1.000000,1.000000}%
\pgfsetstrokecolor{currentstroke}%
\pgfsetdash{}{0pt}%
\pgfpathmoveto{\pgfqpoint{7.456205in}{1.508541in}}%
\pgfpathcurveto{\pgfqpoint{7.467255in}{1.508541in}}{\pgfqpoint{7.477854in}{1.512931in}}{\pgfqpoint{7.485668in}{1.520745in}}%
\pgfpathcurveto{\pgfqpoint{7.493481in}{1.528558in}}{\pgfqpoint{7.497871in}{1.539157in}}{\pgfqpoint{7.497871in}{1.550207in}}%
\pgfpathcurveto{\pgfqpoint{7.497871in}{1.561258in}}{\pgfqpoint{7.493481in}{1.571857in}}{\pgfqpoint{7.485668in}{1.579670in}}%
\pgfpathcurveto{\pgfqpoint{7.477854in}{1.587484in}}{\pgfqpoint{7.467255in}{1.591874in}}{\pgfqpoint{7.456205in}{1.591874in}}%
\pgfpathcurveto{\pgfqpoint{7.445155in}{1.591874in}}{\pgfqpoint{7.434556in}{1.587484in}}{\pgfqpoint{7.426742in}{1.579670in}}%
\pgfpathcurveto{\pgfqpoint{7.418928in}{1.571857in}}{\pgfqpoint{7.414538in}{1.561258in}}{\pgfqpoint{7.414538in}{1.550207in}}%
\pgfpathcurveto{\pgfqpoint{7.414538in}{1.539157in}}{\pgfqpoint{7.418928in}{1.528558in}}{\pgfqpoint{7.426742in}{1.520745in}}%
\pgfpathcurveto{\pgfqpoint{7.434556in}{1.512931in}}{\pgfqpoint{7.445155in}{1.508541in}}{\pgfqpoint{7.456205in}{1.508541in}}%
\pgfpathclose%
\pgfusepath{stroke,fill}%
\end{pgfscope}%
\begin{pgfscope}%
\pgfpathrectangle{\pgfqpoint{0.570343in}{0.331635in}}{\pgfqpoint{9.300000in}{7.700000in}}%
\pgfusepath{clip}%
\pgfsetbuttcap%
\pgfsetroundjoin%
\definecolor{currentfill}{rgb}{1.000000,0.705882,0.509804}%
\pgfsetfillcolor{currentfill}%
\pgfsetlinewidth{0.481800pt}%
\definecolor{currentstroke}{rgb}{1.000000,1.000000,1.000000}%
\pgfsetstrokecolor{currentstroke}%
\pgfsetdash{}{0pt}%
\pgfpathmoveto{\pgfqpoint{4.644507in}{6.725659in}}%
\pgfpathcurveto{\pgfqpoint{4.655557in}{6.725659in}}{\pgfqpoint{4.666156in}{6.730049in}}{\pgfqpoint{4.673970in}{6.737863in}}%
\pgfpathcurveto{\pgfqpoint{4.681783in}{6.745676in}}{\pgfqpoint{4.686174in}{6.756275in}}{\pgfqpoint{4.686174in}{6.767325in}}%
\pgfpathcurveto{\pgfqpoint{4.686174in}{6.778376in}}{\pgfqpoint{4.681783in}{6.788975in}}{\pgfqpoint{4.673970in}{6.796788in}}%
\pgfpathcurveto{\pgfqpoint{4.666156in}{6.804602in}}{\pgfqpoint{4.655557in}{6.808992in}}{\pgfqpoint{4.644507in}{6.808992in}}%
\pgfpathcurveto{\pgfqpoint{4.633457in}{6.808992in}}{\pgfqpoint{4.622858in}{6.804602in}}{\pgfqpoint{4.615044in}{6.796788in}}%
\pgfpathcurveto{\pgfqpoint{4.607230in}{6.788975in}}{\pgfqpoint{4.602840in}{6.778376in}}{\pgfqpoint{4.602840in}{6.767325in}}%
\pgfpathcurveto{\pgfqpoint{4.602840in}{6.756275in}}{\pgfqpoint{4.607230in}{6.745676in}}{\pgfqpoint{4.615044in}{6.737863in}}%
\pgfpathcurveto{\pgfqpoint{4.622858in}{6.730049in}}{\pgfqpoint{4.633457in}{6.725659in}}{\pgfqpoint{4.644507in}{6.725659in}}%
\pgfpathclose%
\pgfusepath{stroke,fill}%
\end{pgfscope}%
\begin{pgfscope}%
\pgfpathrectangle{\pgfqpoint{0.570343in}{0.331635in}}{\pgfqpoint{9.300000in}{7.700000in}}%
\pgfusepath{clip}%
\pgfsetbuttcap%
\pgfsetroundjoin%
\definecolor{currentfill}{rgb}{1.000000,0.705882,0.509804}%
\pgfsetfillcolor{currentfill}%
\pgfsetlinewidth{0.481800pt}%
\definecolor{currentstroke}{rgb}{1.000000,1.000000,1.000000}%
\pgfsetstrokecolor{currentstroke}%
\pgfsetdash{}{0pt}%
\pgfpathmoveto{\pgfqpoint{4.552181in}{5.485402in}}%
\pgfpathcurveto{\pgfqpoint{4.563231in}{5.485402in}}{\pgfqpoint{4.573831in}{5.489792in}}{\pgfqpoint{4.581644in}{5.497606in}}%
\pgfpathcurveto{\pgfqpoint{4.589458in}{5.505420in}}{\pgfqpoint{4.593848in}{5.516019in}}{\pgfqpoint{4.593848in}{5.527069in}}%
\pgfpathcurveto{\pgfqpoint{4.593848in}{5.538119in}}{\pgfqpoint{4.589458in}{5.548718in}}{\pgfqpoint{4.581644in}{5.556532in}}%
\pgfpathcurveto{\pgfqpoint{4.573831in}{5.564345in}}{\pgfqpoint{4.563231in}{5.568736in}}{\pgfqpoint{4.552181in}{5.568736in}}%
\pgfpathcurveto{\pgfqpoint{4.541131in}{5.568736in}}{\pgfqpoint{4.530532in}{5.564345in}}{\pgfqpoint{4.522719in}{5.556532in}}%
\pgfpathcurveto{\pgfqpoint{4.514905in}{5.548718in}}{\pgfqpoint{4.510515in}{5.538119in}}{\pgfqpoint{4.510515in}{5.527069in}}%
\pgfpathcurveto{\pgfqpoint{4.510515in}{5.516019in}}{\pgfqpoint{4.514905in}{5.505420in}}{\pgfqpoint{4.522719in}{5.497606in}}%
\pgfpathcurveto{\pgfqpoint{4.530532in}{5.489792in}}{\pgfqpoint{4.541131in}{5.485402in}}{\pgfqpoint{4.552181in}{5.485402in}}%
\pgfpathclose%
\pgfusepath{stroke,fill}%
\end{pgfscope}%
\begin{pgfscope}%
\pgfpathrectangle{\pgfqpoint{0.570343in}{0.331635in}}{\pgfqpoint{9.300000in}{7.700000in}}%
\pgfusepath{clip}%
\pgfsetbuttcap%
\pgfsetroundjoin%
\definecolor{currentfill}{rgb}{1.000000,0.705882,0.509804}%
\pgfsetfillcolor{currentfill}%
\pgfsetlinewidth{0.481800pt}%
\definecolor{currentstroke}{rgb}{1.000000,1.000000,1.000000}%
\pgfsetstrokecolor{currentstroke}%
\pgfsetdash{}{0pt}%
\pgfpathmoveto{\pgfqpoint{7.082491in}{7.108311in}}%
\pgfpathcurveto{\pgfqpoint{7.093541in}{7.108311in}}{\pgfqpoint{7.104140in}{7.112701in}}{\pgfqpoint{7.111954in}{7.120515in}}%
\pgfpathcurveto{\pgfqpoint{7.119768in}{7.128329in}}{\pgfqpoint{7.124158in}{7.138928in}}{\pgfqpoint{7.124158in}{7.149978in}}%
\pgfpathcurveto{\pgfqpoint{7.124158in}{7.161028in}}{\pgfqpoint{7.119768in}{7.171627in}}{\pgfqpoint{7.111954in}{7.179441in}}%
\pgfpathcurveto{\pgfqpoint{7.104140in}{7.187254in}}{\pgfqpoint{7.093541in}{7.191645in}}{\pgfqpoint{7.082491in}{7.191645in}}%
\pgfpathcurveto{\pgfqpoint{7.071441in}{7.191645in}}{\pgfqpoint{7.060842in}{7.187254in}}{\pgfqpoint{7.053028in}{7.179441in}}%
\pgfpathcurveto{\pgfqpoint{7.045215in}{7.171627in}}{\pgfqpoint{7.040825in}{7.161028in}}{\pgfqpoint{7.040825in}{7.149978in}}%
\pgfpathcurveto{\pgfqpoint{7.040825in}{7.138928in}}{\pgfqpoint{7.045215in}{7.128329in}}{\pgfqpoint{7.053028in}{7.120515in}}%
\pgfpathcurveto{\pgfqpoint{7.060842in}{7.112701in}}{\pgfqpoint{7.071441in}{7.108311in}}{\pgfqpoint{7.082491in}{7.108311in}}%
\pgfpathclose%
\pgfusepath{stroke,fill}%
\end{pgfscope}%
\begin{pgfscope}%
\pgfpathrectangle{\pgfqpoint{0.570343in}{0.331635in}}{\pgfqpoint{9.300000in}{7.700000in}}%
\pgfusepath{clip}%
\pgfsetbuttcap%
\pgfsetroundjoin%
\definecolor{currentfill}{rgb}{1.000000,0.705882,0.509804}%
\pgfsetfillcolor{currentfill}%
\pgfsetlinewidth{0.481800pt}%
\definecolor{currentstroke}{rgb}{1.000000,1.000000,1.000000}%
\pgfsetstrokecolor{currentstroke}%
\pgfsetdash{}{0pt}%
\pgfpathmoveto{\pgfqpoint{6.077373in}{2.291342in}}%
\pgfpathcurveto{\pgfqpoint{6.088423in}{2.291342in}}{\pgfqpoint{6.099022in}{2.295732in}}{\pgfqpoint{6.106836in}{2.303546in}}%
\pgfpathcurveto{\pgfqpoint{6.114649in}{2.311359in}}{\pgfqpoint{6.119040in}{2.321958in}}{\pgfqpoint{6.119040in}{2.333008in}}%
\pgfpathcurveto{\pgfqpoint{6.119040in}{2.344059in}}{\pgfqpoint{6.114649in}{2.354658in}}{\pgfqpoint{6.106836in}{2.362471in}}%
\pgfpathcurveto{\pgfqpoint{6.099022in}{2.370285in}}{\pgfqpoint{6.088423in}{2.374675in}}{\pgfqpoint{6.077373in}{2.374675in}}%
\pgfpathcurveto{\pgfqpoint{6.066323in}{2.374675in}}{\pgfqpoint{6.055724in}{2.370285in}}{\pgfqpoint{6.047910in}{2.362471in}}%
\pgfpathcurveto{\pgfqpoint{6.040096in}{2.354658in}}{\pgfqpoint{6.035706in}{2.344059in}}{\pgfqpoint{6.035706in}{2.333008in}}%
\pgfpathcurveto{\pgfqpoint{6.035706in}{2.321958in}}{\pgfqpoint{6.040096in}{2.311359in}}{\pgfqpoint{6.047910in}{2.303546in}}%
\pgfpathcurveto{\pgfqpoint{6.055724in}{2.295732in}}{\pgfqpoint{6.066323in}{2.291342in}}{\pgfqpoint{6.077373in}{2.291342in}}%
\pgfpathclose%
\pgfusepath{stroke,fill}%
\end{pgfscope}%
\begin{pgfscope}%
\pgfpathrectangle{\pgfqpoint{0.570343in}{0.331635in}}{\pgfqpoint{9.300000in}{7.700000in}}%
\pgfusepath{clip}%
\pgfsetbuttcap%
\pgfsetroundjoin%
\definecolor{currentfill}{rgb}{1.000000,0.705882,0.509804}%
\pgfsetfillcolor{currentfill}%
\pgfsetlinewidth{0.481800pt}%
\definecolor{currentstroke}{rgb}{1.000000,1.000000,1.000000}%
\pgfsetstrokecolor{currentstroke}%
\pgfsetdash{}{0pt}%
\pgfpathmoveto{\pgfqpoint{4.868813in}{0.938917in}}%
\pgfpathcurveto{\pgfqpoint{4.879863in}{0.938917in}}{\pgfqpoint{4.890462in}{0.943308in}}{\pgfqpoint{4.898276in}{0.951121in}}%
\pgfpathcurveto{\pgfqpoint{4.906089in}{0.958935in}}{\pgfqpoint{4.910480in}{0.969534in}}{\pgfqpoint{4.910480in}{0.980584in}}%
\pgfpathcurveto{\pgfqpoint{4.910480in}{0.991634in}}{\pgfqpoint{4.906089in}{1.002233in}}{\pgfqpoint{4.898276in}{1.010047in}}%
\pgfpathcurveto{\pgfqpoint{4.890462in}{1.017860in}}{\pgfqpoint{4.879863in}{1.022251in}}{\pgfqpoint{4.868813in}{1.022251in}}%
\pgfpathcurveto{\pgfqpoint{4.857763in}{1.022251in}}{\pgfqpoint{4.847164in}{1.017860in}}{\pgfqpoint{4.839350in}{1.010047in}}%
\pgfpathcurveto{\pgfqpoint{4.831537in}{1.002233in}}{\pgfqpoint{4.827146in}{0.991634in}}{\pgfqpoint{4.827146in}{0.980584in}}%
\pgfpathcurveto{\pgfqpoint{4.827146in}{0.969534in}}{\pgfqpoint{4.831537in}{0.958935in}}{\pgfqpoint{4.839350in}{0.951121in}}%
\pgfpathcurveto{\pgfqpoint{4.847164in}{0.943308in}}{\pgfqpoint{4.857763in}{0.938917in}}{\pgfqpoint{4.868813in}{0.938917in}}%
\pgfpathclose%
\pgfusepath{stroke,fill}%
\end{pgfscope}%
\begin{pgfscope}%
\pgfpathrectangle{\pgfqpoint{0.570343in}{0.331635in}}{\pgfqpoint{9.300000in}{7.700000in}}%
\pgfusepath{clip}%
\pgfsetbuttcap%
\pgfsetroundjoin%
\definecolor{currentfill}{rgb}{1.000000,0.705882,0.509804}%
\pgfsetfillcolor{currentfill}%
\pgfsetlinewidth{0.481800pt}%
\definecolor{currentstroke}{rgb}{1.000000,1.000000,1.000000}%
\pgfsetstrokecolor{currentstroke}%
\pgfsetdash{}{0pt}%
\pgfpathmoveto{\pgfqpoint{4.793914in}{4.747558in}}%
\pgfpathcurveto{\pgfqpoint{4.804964in}{4.747558in}}{\pgfqpoint{4.815564in}{4.751948in}}{\pgfqpoint{4.823377in}{4.759762in}}%
\pgfpathcurveto{\pgfqpoint{4.831191in}{4.767576in}}{\pgfqpoint{4.835581in}{4.778175in}}{\pgfqpoint{4.835581in}{4.789225in}}%
\pgfpathcurveto{\pgfqpoint{4.835581in}{4.800275in}}{\pgfqpoint{4.831191in}{4.810874in}}{\pgfqpoint{4.823377in}{4.818687in}}%
\pgfpathcurveto{\pgfqpoint{4.815564in}{4.826501in}}{\pgfqpoint{4.804964in}{4.830891in}}{\pgfqpoint{4.793914in}{4.830891in}}%
\pgfpathcurveto{\pgfqpoint{4.782864in}{4.830891in}}{\pgfqpoint{4.772265in}{4.826501in}}{\pgfqpoint{4.764452in}{4.818687in}}%
\pgfpathcurveto{\pgfqpoint{4.756638in}{4.810874in}}{\pgfqpoint{4.752248in}{4.800275in}}{\pgfqpoint{4.752248in}{4.789225in}}%
\pgfpathcurveto{\pgfqpoint{4.752248in}{4.778175in}}{\pgfqpoint{4.756638in}{4.767576in}}{\pgfqpoint{4.764452in}{4.759762in}}%
\pgfpathcurveto{\pgfqpoint{4.772265in}{4.751948in}}{\pgfqpoint{4.782864in}{4.747558in}}{\pgfqpoint{4.793914in}{4.747558in}}%
\pgfpathclose%
\pgfusepath{stroke,fill}%
\end{pgfscope}%
\begin{pgfscope}%
\pgfpathrectangle{\pgfqpoint{0.570343in}{0.331635in}}{\pgfqpoint{9.300000in}{7.700000in}}%
\pgfusepath{clip}%
\pgfsetbuttcap%
\pgfsetroundjoin%
\definecolor{currentfill}{rgb}{1.000000,0.705882,0.509804}%
\pgfsetfillcolor{currentfill}%
\pgfsetlinewidth{0.481800pt}%
\definecolor{currentstroke}{rgb}{1.000000,1.000000,1.000000}%
\pgfsetstrokecolor{currentstroke}%
\pgfsetdash{}{0pt}%
\pgfpathmoveto{\pgfqpoint{6.545799in}{6.354922in}}%
\pgfpathcurveto{\pgfqpoint{6.556849in}{6.354922in}}{\pgfqpoint{6.567448in}{6.359312in}}{\pgfqpoint{6.575262in}{6.367126in}}%
\pgfpathcurveto{\pgfqpoint{6.583076in}{6.374939in}}{\pgfqpoint{6.587466in}{6.385538in}}{\pgfqpoint{6.587466in}{6.396589in}}%
\pgfpathcurveto{\pgfqpoint{6.587466in}{6.407639in}}{\pgfqpoint{6.583076in}{6.418238in}}{\pgfqpoint{6.575262in}{6.426051in}}%
\pgfpathcurveto{\pgfqpoint{6.567448in}{6.433865in}}{\pgfqpoint{6.556849in}{6.438255in}}{\pgfqpoint{6.545799in}{6.438255in}}%
\pgfpathcurveto{\pgfqpoint{6.534749in}{6.438255in}}{\pgfqpoint{6.524150in}{6.433865in}}{\pgfqpoint{6.516336in}{6.426051in}}%
\pgfpathcurveto{\pgfqpoint{6.508523in}{6.418238in}}{\pgfqpoint{6.504132in}{6.407639in}}{\pgfqpoint{6.504132in}{6.396589in}}%
\pgfpathcurveto{\pgfqpoint{6.504132in}{6.385538in}}{\pgfqpoint{6.508523in}{6.374939in}}{\pgfqpoint{6.516336in}{6.367126in}}%
\pgfpathcurveto{\pgfqpoint{6.524150in}{6.359312in}}{\pgfqpoint{6.534749in}{6.354922in}}{\pgfqpoint{6.545799in}{6.354922in}}%
\pgfpathclose%
\pgfusepath{stroke,fill}%
\end{pgfscope}%
\begin{pgfscope}%
\pgfpathrectangle{\pgfqpoint{0.570343in}{0.331635in}}{\pgfqpoint{9.300000in}{7.700000in}}%
\pgfusepath{clip}%
\pgfsetbuttcap%
\pgfsetroundjoin%
\definecolor{currentfill}{rgb}{1.000000,0.705882,0.509804}%
\pgfsetfillcolor{currentfill}%
\pgfsetlinewidth{0.481800pt}%
\definecolor{currentstroke}{rgb}{1.000000,1.000000,1.000000}%
\pgfsetstrokecolor{currentstroke}%
\pgfsetdash{}{0pt}%
\pgfpathmoveto{\pgfqpoint{3.662589in}{2.790271in}}%
\pgfpathcurveto{\pgfqpoint{3.673640in}{2.790271in}}{\pgfqpoint{3.684239in}{2.794661in}}{\pgfqpoint{3.692052in}{2.802475in}}%
\pgfpathcurveto{\pgfqpoint{3.699866in}{2.810288in}}{\pgfqpoint{3.704256in}{2.820887in}}{\pgfqpoint{3.704256in}{2.831937in}}%
\pgfpathcurveto{\pgfqpoint{3.704256in}{2.842988in}}{\pgfqpoint{3.699866in}{2.853587in}}{\pgfqpoint{3.692052in}{2.861400in}}%
\pgfpathcurveto{\pgfqpoint{3.684239in}{2.869214in}}{\pgfqpoint{3.673640in}{2.873604in}}{\pgfqpoint{3.662589in}{2.873604in}}%
\pgfpathcurveto{\pgfqpoint{3.651539in}{2.873604in}}{\pgfqpoint{3.640940in}{2.869214in}}{\pgfqpoint{3.633127in}{2.861400in}}%
\pgfpathcurveto{\pgfqpoint{3.625313in}{2.853587in}}{\pgfqpoint{3.620923in}{2.842988in}}{\pgfqpoint{3.620923in}{2.831937in}}%
\pgfpathcurveto{\pgfqpoint{3.620923in}{2.820887in}}{\pgfqpoint{3.625313in}{2.810288in}}{\pgfqpoint{3.633127in}{2.802475in}}%
\pgfpathcurveto{\pgfqpoint{3.640940in}{2.794661in}}{\pgfqpoint{3.651539in}{2.790271in}}{\pgfqpoint{3.662589in}{2.790271in}}%
\pgfpathclose%
\pgfusepath{stroke,fill}%
\end{pgfscope}%
\begin{pgfscope}%
\pgfpathrectangle{\pgfqpoint{0.570343in}{0.331635in}}{\pgfqpoint{9.300000in}{7.700000in}}%
\pgfusepath{clip}%
\pgfsetbuttcap%
\pgfsetroundjoin%
\definecolor{currentfill}{rgb}{1.000000,0.705882,0.509804}%
\pgfsetfillcolor{currentfill}%
\pgfsetlinewidth{0.481800pt}%
\definecolor{currentstroke}{rgb}{1.000000,1.000000,1.000000}%
\pgfsetstrokecolor{currentstroke}%
\pgfsetdash{}{0pt}%
\pgfpathmoveto{\pgfqpoint{8.266806in}{7.400156in}}%
\pgfpathcurveto{\pgfqpoint{8.277856in}{7.400156in}}{\pgfqpoint{8.288455in}{7.404546in}}{\pgfqpoint{8.296269in}{7.412360in}}%
\pgfpathcurveto{\pgfqpoint{8.304083in}{7.420174in}}{\pgfqpoint{8.308473in}{7.430773in}}{\pgfqpoint{8.308473in}{7.441823in}}%
\pgfpathcurveto{\pgfqpoint{8.308473in}{7.452873in}}{\pgfqpoint{8.304083in}{7.463472in}}{\pgfqpoint{8.296269in}{7.471286in}}%
\pgfpathcurveto{\pgfqpoint{8.288455in}{7.479099in}}{\pgfqpoint{8.277856in}{7.483489in}}{\pgfqpoint{8.266806in}{7.483489in}}%
\pgfpathcurveto{\pgfqpoint{8.255756in}{7.483489in}}{\pgfqpoint{8.245157in}{7.479099in}}{\pgfqpoint{8.237343in}{7.471286in}}%
\pgfpathcurveto{\pgfqpoint{8.229530in}{7.463472in}}{\pgfqpoint{8.225140in}{7.452873in}}{\pgfqpoint{8.225140in}{7.441823in}}%
\pgfpathcurveto{\pgfqpoint{8.225140in}{7.430773in}}{\pgfqpoint{8.229530in}{7.420174in}}{\pgfqpoint{8.237343in}{7.412360in}}%
\pgfpathcurveto{\pgfqpoint{8.245157in}{7.404546in}}{\pgfqpoint{8.255756in}{7.400156in}}{\pgfqpoint{8.266806in}{7.400156in}}%
\pgfpathclose%
\pgfusepath{stroke,fill}%
\end{pgfscope}%
\begin{pgfscope}%
\pgfpathrectangle{\pgfqpoint{0.570343in}{0.331635in}}{\pgfqpoint{9.300000in}{7.700000in}}%
\pgfusepath{clip}%
\pgfsetbuttcap%
\pgfsetroundjoin%
\definecolor{currentfill}{rgb}{1.000000,0.705882,0.509804}%
\pgfsetfillcolor{currentfill}%
\pgfsetlinewidth{0.481800pt}%
\definecolor{currentstroke}{rgb}{1.000000,1.000000,1.000000}%
\pgfsetstrokecolor{currentstroke}%
\pgfsetdash{}{0pt}%
\pgfpathmoveto{\pgfqpoint{1.678869in}{5.939593in}}%
\pgfpathcurveto{\pgfqpoint{1.689919in}{5.939593in}}{\pgfqpoint{1.700518in}{5.943983in}}{\pgfqpoint{1.708332in}{5.951797in}}%
\pgfpathcurveto{\pgfqpoint{1.716145in}{5.959611in}}{\pgfqpoint{1.720536in}{5.970210in}}{\pgfqpoint{1.720536in}{5.981260in}}%
\pgfpathcurveto{\pgfqpoint{1.720536in}{5.992310in}}{\pgfqpoint{1.716145in}{6.002909in}}{\pgfqpoint{1.708332in}{6.010722in}}%
\pgfpathcurveto{\pgfqpoint{1.700518in}{6.018536in}}{\pgfqpoint{1.689919in}{6.022926in}}{\pgfqpoint{1.678869in}{6.022926in}}%
\pgfpathcurveto{\pgfqpoint{1.667819in}{6.022926in}}{\pgfqpoint{1.657220in}{6.018536in}}{\pgfqpoint{1.649406in}{6.010722in}}%
\pgfpathcurveto{\pgfqpoint{1.641593in}{6.002909in}}{\pgfqpoint{1.637202in}{5.992310in}}{\pgfqpoint{1.637202in}{5.981260in}}%
\pgfpathcurveto{\pgfqpoint{1.637202in}{5.970210in}}{\pgfqpoint{1.641593in}{5.959611in}}{\pgfqpoint{1.649406in}{5.951797in}}%
\pgfpathcurveto{\pgfqpoint{1.657220in}{5.943983in}}{\pgfqpoint{1.667819in}{5.939593in}}{\pgfqpoint{1.678869in}{5.939593in}}%
\pgfpathclose%
\pgfusepath{stroke,fill}%
\end{pgfscope}%
\begin{pgfscope}%
\pgfpathrectangle{\pgfqpoint{0.570343in}{0.331635in}}{\pgfqpoint{9.300000in}{7.700000in}}%
\pgfusepath{clip}%
\pgfsetbuttcap%
\pgfsetroundjoin%
\definecolor{currentfill}{rgb}{1.000000,0.705882,0.509804}%
\pgfsetfillcolor{currentfill}%
\pgfsetlinewidth{0.481800pt}%
\definecolor{currentstroke}{rgb}{1.000000,1.000000,1.000000}%
\pgfsetstrokecolor{currentstroke}%
\pgfsetdash{}{0pt}%
\pgfpathmoveto{\pgfqpoint{8.307790in}{4.537423in}}%
\pgfpathcurveto{\pgfqpoint{8.318840in}{4.537423in}}{\pgfqpoint{8.329439in}{4.541813in}}{\pgfqpoint{8.337253in}{4.549627in}}%
\pgfpathcurveto{\pgfqpoint{8.345066in}{4.557440in}}{\pgfqpoint{8.349457in}{4.568039in}}{\pgfqpoint{8.349457in}{4.579090in}}%
\pgfpathcurveto{\pgfqpoint{8.349457in}{4.590140in}}{\pgfqpoint{8.345066in}{4.600739in}}{\pgfqpoint{8.337253in}{4.608552in}}%
\pgfpathcurveto{\pgfqpoint{8.329439in}{4.616366in}}{\pgfqpoint{8.318840in}{4.620756in}}{\pgfqpoint{8.307790in}{4.620756in}}%
\pgfpathcurveto{\pgfqpoint{8.296740in}{4.620756in}}{\pgfqpoint{8.286141in}{4.616366in}}{\pgfqpoint{8.278327in}{4.608552in}}%
\pgfpathcurveto{\pgfqpoint{8.270514in}{4.600739in}}{\pgfqpoint{8.266123in}{4.590140in}}{\pgfqpoint{8.266123in}{4.579090in}}%
\pgfpathcurveto{\pgfqpoint{8.266123in}{4.568039in}}{\pgfqpoint{8.270514in}{4.557440in}}{\pgfqpoint{8.278327in}{4.549627in}}%
\pgfpathcurveto{\pgfqpoint{8.286141in}{4.541813in}}{\pgfqpoint{8.296740in}{4.537423in}}{\pgfqpoint{8.307790in}{4.537423in}}%
\pgfpathclose%
\pgfusepath{stroke,fill}%
\end{pgfscope}%
\begin{pgfscope}%
\pgfpathrectangle{\pgfqpoint{0.570343in}{0.331635in}}{\pgfqpoint{9.300000in}{7.700000in}}%
\pgfusepath{clip}%
\pgfsetbuttcap%
\pgfsetroundjoin%
\definecolor{currentfill}{rgb}{1.000000,0.705882,0.509804}%
\pgfsetfillcolor{currentfill}%
\pgfsetlinewidth{0.481800pt}%
\definecolor{currentstroke}{rgb}{1.000000,1.000000,1.000000}%
\pgfsetstrokecolor{currentstroke}%
\pgfsetdash{}{0pt}%
\pgfpathmoveto{\pgfqpoint{7.187407in}{5.783280in}}%
\pgfpathcurveto{\pgfqpoint{7.198457in}{5.783280in}}{\pgfqpoint{7.209056in}{5.787670in}}{\pgfqpoint{7.216870in}{5.795483in}}%
\pgfpathcurveto{\pgfqpoint{7.224683in}{5.803297in}}{\pgfqpoint{7.229074in}{5.813896in}}{\pgfqpoint{7.229074in}{5.824946in}}%
\pgfpathcurveto{\pgfqpoint{7.229074in}{5.835996in}}{\pgfqpoint{7.224683in}{5.846595in}}{\pgfqpoint{7.216870in}{5.854409in}}%
\pgfpathcurveto{\pgfqpoint{7.209056in}{5.862223in}}{\pgfqpoint{7.198457in}{5.866613in}}{\pgfqpoint{7.187407in}{5.866613in}}%
\pgfpathcurveto{\pgfqpoint{7.176357in}{5.866613in}}{\pgfqpoint{7.165758in}{5.862223in}}{\pgfqpoint{7.157944in}{5.854409in}}%
\pgfpathcurveto{\pgfqpoint{7.150131in}{5.846595in}}{\pgfqpoint{7.145740in}{5.835996in}}{\pgfqpoint{7.145740in}{5.824946in}}%
\pgfpathcurveto{\pgfqpoint{7.145740in}{5.813896in}}{\pgfqpoint{7.150131in}{5.803297in}}{\pgfqpoint{7.157944in}{5.795483in}}%
\pgfpathcurveto{\pgfqpoint{7.165758in}{5.787670in}}{\pgfqpoint{7.176357in}{5.783280in}}{\pgfqpoint{7.187407in}{5.783280in}}%
\pgfpathclose%
\pgfusepath{stroke,fill}%
\end{pgfscope}%
\begin{pgfscope}%
\pgfpathrectangle{\pgfqpoint{0.570343in}{0.331635in}}{\pgfqpoint{9.300000in}{7.700000in}}%
\pgfusepath{clip}%
\pgfsetbuttcap%
\pgfsetroundjoin%
\definecolor{currentfill}{rgb}{1.000000,0.705882,0.509804}%
\pgfsetfillcolor{currentfill}%
\pgfsetlinewidth{0.481800pt}%
\definecolor{currentstroke}{rgb}{1.000000,1.000000,1.000000}%
\pgfsetstrokecolor{currentstroke}%
\pgfsetdash{}{0pt}%
\pgfpathmoveto{\pgfqpoint{6.326121in}{1.360791in}}%
\pgfpathcurveto{\pgfqpoint{6.337172in}{1.360791in}}{\pgfqpoint{6.347771in}{1.365181in}}{\pgfqpoint{6.355584in}{1.372995in}}%
\pgfpathcurveto{\pgfqpoint{6.363398in}{1.380809in}}{\pgfqpoint{6.367788in}{1.391408in}}{\pgfqpoint{6.367788in}{1.402458in}}%
\pgfpathcurveto{\pgfqpoint{6.367788in}{1.413508in}}{\pgfqpoint{6.363398in}{1.424107in}}{\pgfqpoint{6.355584in}{1.431921in}}%
\pgfpathcurveto{\pgfqpoint{6.347771in}{1.439734in}}{\pgfqpoint{6.337172in}{1.444124in}}{\pgfqpoint{6.326121in}{1.444124in}}%
\pgfpathcurveto{\pgfqpoint{6.315071in}{1.444124in}}{\pgfqpoint{6.304472in}{1.439734in}}{\pgfqpoint{6.296659in}{1.431921in}}%
\pgfpathcurveto{\pgfqpoint{6.288845in}{1.424107in}}{\pgfqpoint{6.284455in}{1.413508in}}{\pgfqpoint{6.284455in}{1.402458in}}%
\pgfpathcurveto{\pgfqpoint{6.284455in}{1.391408in}}{\pgfqpoint{6.288845in}{1.380809in}}{\pgfqpoint{6.296659in}{1.372995in}}%
\pgfpathcurveto{\pgfqpoint{6.304472in}{1.365181in}}{\pgfqpoint{6.315071in}{1.360791in}}{\pgfqpoint{6.326121in}{1.360791in}}%
\pgfpathclose%
\pgfusepath{stroke,fill}%
\end{pgfscope}%
\begin{pgfscope}%
\pgfpathrectangle{\pgfqpoint{0.570343in}{0.331635in}}{\pgfqpoint{9.300000in}{7.700000in}}%
\pgfusepath{clip}%
\pgfsetbuttcap%
\pgfsetroundjoin%
\definecolor{currentfill}{rgb}{1.000000,0.705882,0.509804}%
\pgfsetfillcolor{currentfill}%
\pgfsetlinewidth{0.481800pt}%
\definecolor{currentstroke}{rgb}{1.000000,1.000000,1.000000}%
\pgfsetstrokecolor{currentstroke}%
\pgfsetdash{}{0pt}%
\pgfpathmoveto{\pgfqpoint{9.447616in}{4.615425in}}%
\pgfpathcurveto{\pgfqpoint{9.458666in}{4.615425in}}{\pgfqpoint{9.469265in}{4.619816in}}{\pgfqpoint{9.477079in}{4.627629in}}%
\pgfpathcurveto{\pgfqpoint{9.484892in}{4.635443in}}{\pgfqpoint{9.489283in}{4.646042in}}{\pgfqpoint{9.489283in}{4.657092in}}%
\pgfpathcurveto{\pgfqpoint{9.489283in}{4.668142in}}{\pgfqpoint{9.484892in}{4.678741in}}{\pgfqpoint{9.477079in}{4.686555in}}%
\pgfpathcurveto{\pgfqpoint{9.469265in}{4.694369in}}{\pgfqpoint{9.458666in}{4.698759in}}{\pgfqpoint{9.447616in}{4.698759in}}%
\pgfpathcurveto{\pgfqpoint{9.436566in}{4.698759in}}{\pgfqpoint{9.425967in}{4.694369in}}{\pgfqpoint{9.418153in}{4.686555in}}%
\pgfpathcurveto{\pgfqpoint{9.410340in}{4.678741in}}{\pgfqpoint{9.405949in}{4.668142in}}{\pgfqpoint{9.405949in}{4.657092in}}%
\pgfpathcurveto{\pgfqpoint{9.405949in}{4.646042in}}{\pgfqpoint{9.410340in}{4.635443in}}{\pgfqpoint{9.418153in}{4.627629in}}%
\pgfpathcurveto{\pgfqpoint{9.425967in}{4.619816in}}{\pgfqpoint{9.436566in}{4.615425in}}{\pgfqpoint{9.447616in}{4.615425in}}%
\pgfpathclose%
\pgfusepath{stroke,fill}%
\end{pgfscope}%
\begin{pgfscope}%
\pgfpathrectangle{\pgfqpoint{0.570343in}{0.331635in}}{\pgfqpoint{9.300000in}{7.700000in}}%
\pgfusepath{clip}%
\pgfsetbuttcap%
\pgfsetroundjoin%
\definecolor{currentfill}{rgb}{0.631373,0.788235,0.956863}%
\pgfsetfillcolor{currentfill}%
\pgfsetlinewidth{1.003750pt}%
\definecolor{currentstroke}{rgb}{0.631373,0.788235,0.956863}%
\pgfsetstrokecolor{currentstroke}%
\pgfsetdash{}{0pt}%
\pgfsys@defobject{currentmarker}{\pgfqpoint{-0.041667in}{-0.041667in}}{\pgfqpoint{0.041667in}{0.041667in}}{%
\pgfpathmoveto{\pgfqpoint{0.000000in}{-0.041667in}}%
\pgfpathcurveto{\pgfqpoint{0.011050in}{-0.041667in}}{\pgfqpoint{0.021649in}{-0.037276in}}{\pgfqpoint{0.029463in}{-0.029463in}}%
\pgfpathcurveto{\pgfqpoint{0.037276in}{-0.021649in}}{\pgfqpoint{0.041667in}{-0.011050in}}{\pgfqpoint{0.041667in}{0.000000in}}%
\pgfpathcurveto{\pgfqpoint{0.041667in}{0.011050in}}{\pgfqpoint{0.037276in}{0.021649in}}{\pgfqpoint{0.029463in}{0.029463in}}%
\pgfpathcurveto{\pgfqpoint{0.021649in}{0.037276in}}{\pgfqpoint{0.011050in}{0.041667in}}{\pgfqpoint{0.000000in}{0.041667in}}%
\pgfpathcurveto{\pgfqpoint{-0.011050in}{0.041667in}}{\pgfqpoint{-0.021649in}{0.037276in}}{\pgfqpoint{-0.029463in}{0.029463in}}%
\pgfpathcurveto{\pgfqpoint{-0.037276in}{0.021649in}}{\pgfqpoint{-0.041667in}{0.011050in}}{\pgfqpoint{-0.041667in}{0.000000in}}%
\pgfpathcurveto{\pgfqpoint{-0.041667in}{-0.011050in}}{\pgfqpoint{-0.037276in}{-0.021649in}}{\pgfqpoint{-0.029463in}{-0.029463in}}%
\pgfpathcurveto{\pgfqpoint{-0.021649in}{-0.037276in}}{\pgfqpoint{-0.011050in}{-0.041667in}}{\pgfqpoint{0.000000in}{-0.041667in}}%
\pgfpathclose%
\pgfusepath{stroke,fill}%
}%
\end{pgfscope}%
\begin{pgfscope}%
\pgfpathrectangle{\pgfqpoint{0.570343in}{0.331635in}}{\pgfqpoint{9.300000in}{7.700000in}}%
\pgfusepath{clip}%
\pgfsetbuttcap%
\pgfsetroundjoin%
\definecolor{currentfill}{rgb}{1.000000,0.705882,0.509804}%
\pgfsetfillcolor{currentfill}%
\pgfsetlinewidth{1.003750pt}%
\definecolor{currentstroke}{rgb}{1.000000,0.705882,0.509804}%
\pgfsetstrokecolor{currentstroke}%
\pgfsetdash{}{0pt}%
\pgfsys@defobject{currentmarker}{\pgfqpoint{-0.041667in}{-0.041667in}}{\pgfqpoint{0.041667in}{0.041667in}}{%
\pgfpathmoveto{\pgfqpoint{0.000000in}{-0.041667in}}%
\pgfpathcurveto{\pgfqpoint{0.011050in}{-0.041667in}}{\pgfqpoint{0.021649in}{-0.037276in}}{\pgfqpoint{0.029463in}{-0.029463in}}%
\pgfpathcurveto{\pgfqpoint{0.037276in}{-0.021649in}}{\pgfqpoint{0.041667in}{-0.011050in}}{\pgfqpoint{0.041667in}{0.000000in}}%
\pgfpathcurveto{\pgfqpoint{0.041667in}{0.011050in}}{\pgfqpoint{0.037276in}{0.021649in}}{\pgfqpoint{0.029463in}{0.029463in}}%
\pgfpathcurveto{\pgfqpoint{0.021649in}{0.037276in}}{\pgfqpoint{0.011050in}{0.041667in}}{\pgfqpoint{0.000000in}{0.041667in}}%
\pgfpathcurveto{\pgfqpoint{-0.011050in}{0.041667in}}{\pgfqpoint{-0.021649in}{0.037276in}}{\pgfqpoint{-0.029463in}{0.029463in}}%
\pgfpathcurveto{\pgfqpoint{-0.037276in}{0.021649in}}{\pgfqpoint{-0.041667in}{0.011050in}}{\pgfqpoint{-0.041667in}{0.000000in}}%
\pgfpathcurveto{\pgfqpoint{-0.041667in}{-0.011050in}}{\pgfqpoint{-0.037276in}{-0.021649in}}{\pgfqpoint{-0.029463in}{-0.029463in}}%
\pgfpathcurveto{\pgfqpoint{-0.021649in}{-0.037276in}}{\pgfqpoint{-0.011050in}{-0.041667in}}{\pgfqpoint{0.000000in}{-0.041667in}}%
\pgfpathclose%
\pgfusepath{stroke,fill}%
}%
\end{pgfscope}%
\begin{pgfscope}%
\pgfsetbuttcap%
\pgfsetroundjoin%
\definecolor{currentfill}{rgb}{0.000000,0.000000,0.000000}%
\pgfsetfillcolor{currentfill}%
\pgfsetlinewidth{0.803000pt}%
\definecolor{currentstroke}{rgb}{0.000000,0.000000,0.000000}%
\pgfsetstrokecolor{currentstroke}%
\pgfsetdash{}{0pt}%
\pgfsys@defobject{currentmarker}{\pgfqpoint{0.000000in}{-0.048611in}}{\pgfqpoint{0.000000in}{0.000000in}}{%
\pgfpathmoveto{\pgfqpoint{0.000000in}{0.000000in}}%
\pgfpathlineto{\pgfqpoint{0.000000in}{-0.048611in}}%
\pgfusepath{stroke,fill}%
}%
\begin{pgfscope}%
\pgfsys@transformshift{1.237108in}{0.331635in}%
\pgfsys@useobject{currentmarker}{}%
\end{pgfscope}%
\end{pgfscope}%
\begin{pgfscope}%
\definecolor{textcolor}{rgb}{0.000000,0.000000,0.000000}%
\pgfsetstrokecolor{textcolor}%
\pgfsetfillcolor{textcolor}%
\pgftext[x=1.237108in,y=0.234413in,,top]{\color{textcolor}\sffamily\fontsize{10.000000}{12.000000}\selectfont \ensuremath{-}80}%
\end{pgfscope}%
\begin{pgfscope}%
\pgfsetbuttcap%
\pgfsetroundjoin%
\definecolor{currentfill}{rgb}{0.000000,0.000000,0.000000}%
\pgfsetfillcolor{currentfill}%
\pgfsetlinewidth{0.803000pt}%
\definecolor{currentstroke}{rgb}{0.000000,0.000000,0.000000}%
\pgfsetstrokecolor{currentstroke}%
\pgfsetdash{}{0pt}%
\pgfsys@defobject{currentmarker}{\pgfqpoint{0.000000in}{-0.048611in}}{\pgfqpoint{0.000000in}{0.000000in}}{%
\pgfpathmoveto{\pgfqpoint{0.000000in}{0.000000in}}%
\pgfpathlineto{\pgfqpoint{0.000000in}{-0.048611in}}%
\pgfusepath{stroke,fill}%
}%
\begin{pgfscope}%
\pgfsys@transformshift{2.352555in}{0.331635in}%
\pgfsys@useobject{currentmarker}{}%
\end{pgfscope}%
\end{pgfscope}%
\begin{pgfscope}%
\definecolor{textcolor}{rgb}{0.000000,0.000000,0.000000}%
\pgfsetstrokecolor{textcolor}%
\pgfsetfillcolor{textcolor}%
\pgftext[x=2.352555in,y=0.234413in,,top]{\color{textcolor}\sffamily\fontsize{10.000000}{12.000000}\selectfont \ensuremath{-}60}%
\end{pgfscope}%
\begin{pgfscope}%
\pgfsetbuttcap%
\pgfsetroundjoin%
\definecolor{currentfill}{rgb}{0.000000,0.000000,0.000000}%
\pgfsetfillcolor{currentfill}%
\pgfsetlinewidth{0.803000pt}%
\definecolor{currentstroke}{rgb}{0.000000,0.000000,0.000000}%
\pgfsetstrokecolor{currentstroke}%
\pgfsetdash{}{0pt}%
\pgfsys@defobject{currentmarker}{\pgfqpoint{0.000000in}{-0.048611in}}{\pgfqpoint{0.000000in}{0.000000in}}{%
\pgfpathmoveto{\pgfqpoint{0.000000in}{0.000000in}}%
\pgfpathlineto{\pgfqpoint{0.000000in}{-0.048611in}}%
\pgfusepath{stroke,fill}%
}%
\begin{pgfscope}%
\pgfsys@transformshift{3.468002in}{0.331635in}%
\pgfsys@useobject{currentmarker}{}%
\end{pgfscope}%
\end{pgfscope}%
\begin{pgfscope}%
\definecolor{textcolor}{rgb}{0.000000,0.000000,0.000000}%
\pgfsetstrokecolor{textcolor}%
\pgfsetfillcolor{textcolor}%
\pgftext[x=3.468002in,y=0.234413in,,top]{\color{textcolor}\sffamily\fontsize{10.000000}{12.000000}\selectfont \ensuremath{-}40}%
\end{pgfscope}%
\begin{pgfscope}%
\pgfsetbuttcap%
\pgfsetroundjoin%
\definecolor{currentfill}{rgb}{0.000000,0.000000,0.000000}%
\pgfsetfillcolor{currentfill}%
\pgfsetlinewidth{0.803000pt}%
\definecolor{currentstroke}{rgb}{0.000000,0.000000,0.000000}%
\pgfsetstrokecolor{currentstroke}%
\pgfsetdash{}{0pt}%
\pgfsys@defobject{currentmarker}{\pgfqpoint{0.000000in}{-0.048611in}}{\pgfqpoint{0.000000in}{0.000000in}}{%
\pgfpathmoveto{\pgfqpoint{0.000000in}{0.000000in}}%
\pgfpathlineto{\pgfqpoint{0.000000in}{-0.048611in}}%
\pgfusepath{stroke,fill}%
}%
\begin{pgfscope}%
\pgfsys@transformshift{4.583449in}{0.331635in}%
\pgfsys@useobject{currentmarker}{}%
\end{pgfscope}%
\end{pgfscope}%
\begin{pgfscope}%
\definecolor{textcolor}{rgb}{0.000000,0.000000,0.000000}%
\pgfsetstrokecolor{textcolor}%
\pgfsetfillcolor{textcolor}%
\pgftext[x=4.583449in,y=0.234413in,,top]{\color{textcolor}\sffamily\fontsize{10.000000}{12.000000}\selectfont \ensuremath{-}20}%
\end{pgfscope}%
\begin{pgfscope}%
\pgfsetbuttcap%
\pgfsetroundjoin%
\definecolor{currentfill}{rgb}{0.000000,0.000000,0.000000}%
\pgfsetfillcolor{currentfill}%
\pgfsetlinewidth{0.803000pt}%
\definecolor{currentstroke}{rgb}{0.000000,0.000000,0.000000}%
\pgfsetstrokecolor{currentstroke}%
\pgfsetdash{}{0pt}%
\pgfsys@defobject{currentmarker}{\pgfqpoint{0.000000in}{-0.048611in}}{\pgfqpoint{0.000000in}{0.000000in}}{%
\pgfpathmoveto{\pgfqpoint{0.000000in}{0.000000in}}%
\pgfpathlineto{\pgfqpoint{0.000000in}{-0.048611in}}%
\pgfusepath{stroke,fill}%
}%
\begin{pgfscope}%
\pgfsys@transformshift{5.698896in}{0.331635in}%
\pgfsys@useobject{currentmarker}{}%
\end{pgfscope}%
\end{pgfscope}%
\begin{pgfscope}%
\definecolor{textcolor}{rgb}{0.000000,0.000000,0.000000}%
\pgfsetstrokecolor{textcolor}%
\pgfsetfillcolor{textcolor}%
\pgftext[x=5.698896in,y=0.234413in,,top]{\color{textcolor}\sffamily\fontsize{10.000000}{12.000000}\selectfont 0}%
\end{pgfscope}%
\begin{pgfscope}%
\pgfsetbuttcap%
\pgfsetroundjoin%
\definecolor{currentfill}{rgb}{0.000000,0.000000,0.000000}%
\pgfsetfillcolor{currentfill}%
\pgfsetlinewidth{0.803000pt}%
\definecolor{currentstroke}{rgb}{0.000000,0.000000,0.000000}%
\pgfsetstrokecolor{currentstroke}%
\pgfsetdash{}{0pt}%
\pgfsys@defobject{currentmarker}{\pgfqpoint{0.000000in}{-0.048611in}}{\pgfqpoint{0.000000in}{0.000000in}}{%
\pgfpathmoveto{\pgfqpoint{0.000000in}{0.000000in}}%
\pgfpathlineto{\pgfqpoint{0.000000in}{-0.048611in}}%
\pgfusepath{stroke,fill}%
}%
\begin{pgfscope}%
\pgfsys@transformshift{6.814343in}{0.331635in}%
\pgfsys@useobject{currentmarker}{}%
\end{pgfscope}%
\end{pgfscope}%
\begin{pgfscope}%
\definecolor{textcolor}{rgb}{0.000000,0.000000,0.000000}%
\pgfsetstrokecolor{textcolor}%
\pgfsetfillcolor{textcolor}%
\pgftext[x=6.814343in,y=0.234413in,,top]{\color{textcolor}\sffamily\fontsize{10.000000}{12.000000}\selectfont 20}%
\end{pgfscope}%
\begin{pgfscope}%
\pgfsetbuttcap%
\pgfsetroundjoin%
\definecolor{currentfill}{rgb}{0.000000,0.000000,0.000000}%
\pgfsetfillcolor{currentfill}%
\pgfsetlinewidth{0.803000pt}%
\definecolor{currentstroke}{rgb}{0.000000,0.000000,0.000000}%
\pgfsetstrokecolor{currentstroke}%
\pgfsetdash{}{0pt}%
\pgfsys@defobject{currentmarker}{\pgfqpoint{0.000000in}{-0.048611in}}{\pgfqpoint{0.000000in}{0.000000in}}{%
\pgfpathmoveto{\pgfqpoint{0.000000in}{0.000000in}}%
\pgfpathlineto{\pgfqpoint{0.000000in}{-0.048611in}}%
\pgfusepath{stroke,fill}%
}%
\begin{pgfscope}%
\pgfsys@transformshift{7.929790in}{0.331635in}%
\pgfsys@useobject{currentmarker}{}%
\end{pgfscope}%
\end{pgfscope}%
\begin{pgfscope}%
\definecolor{textcolor}{rgb}{0.000000,0.000000,0.000000}%
\pgfsetstrokecolor{textcolor}%
\pgfsetfillcolor{textcolor}%
\pgftext[x=7.929790in,y=0.234413in,,top]{\color{textcolor}\sffamily\fontsize{10.000000}{12.000000}\selectfont 40}%
\end{pgfscope}%
\begin{pgfscope}%
\pgfsetbuttcap%
\pgfsetroundjoin%
\definecolor{currentfill}{rgb}{0.000000,0.000000,0.000000}%
\pgfsetfillcolor{currentfill}%
\pgfsetlinewidth{0.803000pt}%
\definecolor{currentstroke}{rgb}{0.000000,0.000000,0.000000}%
\pgfsetstrokecolor{currentstroke}%
\pgfsetdash{}{0pt}%
\pgfsys@defobject{currentmarker}{\pgfqpoint{0.000000in}{-0.048611in}}{\pgfqpoint{0.000000in}{0.000000in}}{%
\pgfpathmoveto{\pgfqpoint{0.000000in}{0.000000in}}%
\pgfpathlineto{\pgfqpoint{0.000000in}{-0.048611in}}%
\pgfusepath{stroke,fill}%
}%
\begin{pgfscope}%
\pgfsys@transformshift{9.045237in}{0.331635in}%
\pgfsys@useobject{currentmarker}{}%
\end{pgfscope}%
\end{pgfscope}%
\begin{pgfscope}%
\definecolor{textcolor}{rgb}{0.000000,0.000000,0.000000}%
\pgfsetstrokecolor{textcolor}%
\pgfsetfillcolor{textcolor}%
\pgftext[x=9.045237in,y=0.234413in,,top]{\color{textcolor}\sffamily\fontsize{10.000000}{12.000000}\selectfont 60}%
\end{pgfscope}%
\begin{pgfscope}%
\pgfsetbuttcap%
\pgfsetroundjoin%
\definecolor{currentfill}{rgb}{0.000000,0.000000,0.000000}%
\pgfsetfillcolor{currentfill}%
\pgfsetlinewidth{0.803000pt}%
\definecolor{currentstroke}{rgb}{0.000000,0.000000,0.000000}%
\pgfsetstrokecolor{currentstroke}%
\pgfsetdash{}{0pt}%
\pgfsys@defobject{currentmarker}{\pgfqpoint{-0.048611in}{0.000000in}}{\pgfqpoint{-0.000000in}{0.000000in}}{%
\pgfpathmoveto{\pgfqpoint{-0.000000in}{0.000000in}}%
\pgfpathlineto{\pgfqpoint{-0.048611in}{0.000000in}}%
\pgfusepath{stroke,fill}%
}%
\begin{pgfscope}%
\pgfsys@transformshift{0.570343in}{0.580690in}%
\pgfsys@useobject{currentmarker}{}%
\end{pgfscope}%
\end{pgfscope}%
\begin{pgfscope}%
\definecolor{textcolor}{rgb}{0.000000,0.000000,0.000000}%
\pgfsetstrokecolor{textcolor}%
\pgfsetfillcolor{textcolor}%
\pgftext[x=0.100000in, y=0.527928in, left, base]{\color{textcolor}\sffamily\fontsize{10.000000}{12.000000}\selectfont \ensuremath{-}100}%
\end{pgfscope}%
\begin{pgfscope}%
\pgfsetbuttcap%
\pgfsetroundjoin%
\definecolor{currentfill}{rgb}{0.000000,0.000000,0.000000}%
\pgfsetfillcolor{currentfill}%
\pgfsetlinewidth{0.803000pt}%
\definecolor{currentstroke}{rgb}{0.000000,0.000000,0.000000}%
\pgfsetstrokecolor{currentstroke}%
\pgfsetdash{}{0pt}%
\pgfsys@defobject{currentmarker}{\pgfqpoint{-0.048611in}{0.000000in}}{\pgfqpoint{-0.000000in}{0.000000in}}{%
\pgfpathmoveto{\pgfqpoint{-0.000000in}{0.000000in}}%
\pgfpathlineto{\pgfqpoint{-0.048611in}{0.000000in}}%
\pgfusepath{stroke,fill}%
}%
\begin{pgfscope}%
\pgfsys@transformshift{0.570343in}{1.444910in}%
\pgfsys@useobject{currentmarker}{}%
\end{pgfscope}%
\end{pgfscope}%
\begin{pgfscope}%
\definecolor{textcolor}{rgb}{0.000000,0.000000,0.000000}%
\pgfsetstrokecolor{textcolor}%
\pgfsetfillcolor{textcolor}%
\pgftext[x=0.188365in, y=1.392149in, left, base]{\color{textcolor}\sffamily\fontsize{10.000000}{12.000000}\selectfont \ensuremath{-}75}%
\end{pgfscope}%
\begin{pgfscope}%
\pgfsetbuttcap%
\pgfsetroundjoin%
\definecolor{currentfill}{rgb}{0.000000,0.000000,0.000000}%
\pgfsetfillcolor{currentfill}%
\pgfsetlinewidth{0.803000pt}%
\definecolor{currentstroke}{rgb}{0.000000,0.000000,0.000000}%
\pgfsetstrokecolor{currentstroke}%
\pgfsetdash{}{0pt}%
\pgfsys@defobject{currentmarker}{\pgfqpoint{-0.048611in}{0.000000in}}{\pgfqpoint{-0.000000in}{0.000000in}}{%
\pgfpathmoveto{\pgfqpoint{-0.000000in}{0.000000in}}%
\pgfpathlineto{\pgfqpoint{-0.048611in}{0.000000in}}%
\pgfusepath{stroke,fill}%
}%
\begin{pgfscope}%
\pgfsys@transformshift{0.570343in}{2.309131in}%
\pgfsys@useobject{currentmarker}{}%
\end{pgfscope}%
\end{pgfscope}%
\begin{pgfscope}%
\definecolor{textcolor}{rgb}{0.000000,0.000000,0.000000}%
\pgfsetstrokecolor{textcolor}%
\pgfsetfillcolor{textcolor}%
\pgftext[x=0.188365in, y=2.256369in, left, base]{\color{textcolor}\sffamily\fontsize{10.000000}{12.000000}\selectfont \ensuremath{-}50}%
\end{pgfscope}%
\begin{pgfscope}%
\pgfsetbuttcap%
\pgfsetroundjoin%
\definecolor{currentfill}{rgb}{0.000000,0.000000,0.000000}%
\pgfsetfillcolor{currentfill}%
\pgfsetlinewidth{0.803000pt}%
\definecolor{currentstroke}{rgb}{0.000000,0.000000,0.000000}%
\pgfsetstrokecolor{currentstroke}%
\pgfsetdash{}{0pt}%
\pgfsys@defobject{currentmarker}{\pgfqpoint{-0.048611in}{0.000000in}}{\pgfqpoint{-0.000000in}{0.000000in}}{%
\pgfpathmoveto{\pgfqpoint{-0.000000in}{0.000000in}}%
\pgfpathlineto{\pgfqpoint{-0.048611in}{0.000000in}}%
\pgfusepath{stroke,fill}%
}%
\begin{pgfscope}%
\pgfsys@transformshift{0.570343in}{3.173352in}%
\pgfsys@useobject{currentmarker}{}%
\end{pgfscope}%
\end{pgfscope}%
\begin{pgfscope}%
\definecolor{textcolor}{rgb}{0.000000,0.000000,0.000000}%
\pgfsetstrokecolor{textcolor}%
\pgfsetfillcolor{textcolor}%
\pgftext[x=0.188365in, y=3.120590in, left, base]{\color{textcolor}\sffamily\fontsize{10.000000}{12.000000}\selectfont \ensuremath{-}25}%
\end{pgfscope}%
\begin{pgfscope}%
\pgfsetbuttcap%
\pgfsetroundjoin%
\definecolor{currentfill}{rgb}{0.000000,0.000000,0.000000}%
\pgfsetfillcolor{currentfill}%
\pgfsetlinewidth{0.803000pt}%
\definecolor{currentstroke}{rgb}{0.000000,0.000000,0.000000}%
\pgfsetstrokecolor{currentstroke}%
\pgfsetdash{}{0pt}%
\pgfsys@defobject{currentmarker}{\pgfqpoint{-0.048611in}{0.000000in}}{\pgfqpoint{-0.000000in}{0.000000in}}{%
\pgfpathmoveto{\pgfqpoint{-0.000000in}{0.000000in}}%
\pgfpathlineto{\pgfqpoint{-0.048611in}{0.000000in}}%
\pgfusepath{stroke,fill}%
}%
\begin{pgfscope}%
\pgfsys@transformshift{0.570343in}{4.037572in}%
\pgfsys@useobject{currentmarker}{}%
\end{pgfscope}%
\end{pgfscope}%
\begin{pgfscope}%
\definecolor{textcolor}{rgb}{0.000000,0.000000,0.000000}%
\pgfsetstrokecolor{textcolor}%
\pgfsetfillcolor{textcolor}%
\pgftext[x=0.384756in, y=3.984811in, left, base]{\color{textcolor}\sffamily\fontsize{10.000000}{12.000000}\selectfont 0}%
\end{pgfscope}%
\begin{pgfscope}%
\pgfsetbuttcap%
\pgfsetroundjoin%
\definecolor{currentfill}{rgb}{0.000000,0.000000,0.000000}%
\pgfsetfillcolor{currentfill}%
\pgfsetlinewidth{0.803000pt}%
\definecolor{currentstroke}{rgb}{0.000000,0.000000,0.000000}%
\pgfsetstrokecolor{currentstroke}%
\pgfsetdash{}{0pt}%
\pgfsys@defobject{currentmarker}{\pgfqpoint{-0.048611in}{0.000000in}}{\pgfqpoint{-0.000000in}{0.000000in}}{%
\pgfpathmoveto{\pgfqpoint{-0.000000in}{0.000000in}}%
\pgfpathlineto{\pgfqpoint{-0.048611in}{0.000000in}}%
\pgfusepath{stroke,fill}%
}%
\begin{pgfscope}%
\pgfsys@transformshift{0.570343in}{4.901793in}%
\pgfsys@useobject{currentmarker}{}%
\end{pgfscope}%
\end{pgfscope}%
\begin{pgfscope}%
\definecolor{textcolor}{rgb}{0.000000,0.000000,0.000000}%
\pgfsetstrokecolor{textcolor}%
\pgfsetfillcolor{textcolor}%
\pgftext[x=0.296390in, y=4.849031in, left, base]{\color{textcolor}\sffamily\fontsize{10.000000}{12.000000}\selectfont 25}%
\end{pgfscope}%
\begin{pgfscope}%
\pgfsetbuttcap%
\pgfsetroundjoin%
\definecolor{currentfill}{rgb}{0.000000,0.000000,0.000000}%
\pgfsetfillcolor{currentfill}%
\pgfsetlinewidth{0.803000pt}%
\definecolor{currentstroke}{rgb}{0.000000,0.000000,0.000000}%
\pgfsetstrokecolor{currentstroke}%
\pgfsetdash{}{0pt}%
\pgfsys@defobject{currentmarker}{\pgfqpoint{-0.048611in}{0.000000in}}{\pgfqpoint{-0.000000in}{0.000000in}}{%
\pgfpathmoveto{\pgfqpoint{-0.000000in}{0.000000in}}%
\pgfpathlineto{\pgfqpoint{-0.048611in}{0.000000in}}%
\pgfusepath{stroke,fill}%
}%
\begin{pgfscope}%
\pgfsys@transformshift{0.570343in}{5.766014in}%
\pgfsys@useobject{currentmarker}{}%
\end{pgfscope}%
\end{pgfscope}%
\begin{pgfscope}%
\definecolor{textcolor}{rgb}{0.000000,0.000000,0.000000}%
\pgfsetstrokecolor{textcolor}%
\pgfsetfillcolor{textcolor}%
\pgftext[x=0.296390in, y=5.713252in, left, base]{\color{textcolor}\sffamily\fontsize{10.000000}{12.000000}\selectfont 50}%
\end{pgfscope}%
\begin{pgfscope}%
\pgfsetbuttcap%
\pgfsetroundjoin%
\definecolor{currentfill}{rgb}{0.000000,0.000000,0.000000}%
\pgfsetfillcolor{currentfill}%
\pgfsetlinewidth{0.803000pt}%
\definecolor{currentstroke}{rgb}{0.000000,0.000000,0.000000}%
\pgfsetstrokecolor{currentstroke}%
\pgfsetdash{}{0pt}%
\pgfsys@defobject{currentmarker}{\pgfqpoint{-0.048611in}{0.000000in}}{\pgfqpoint{-0.000000in}{0.000000in}}{%
\pgfpathmoveto{\pgfqpoint{-0.000000in}{0.000000in}}%
\pgfpathlineto{\pgfqpoint{-0.048611in}{0.000000in}}%
\pgfusepath{stroke,fill}%
}%
\begin{pgfscope}%
\pgfsys@transformshift{0.570343in}{6.630234in}%
\pgfsys@useobject{currentmarker}{}%
\end{pgfscope}%
\end{pgfscope}%
\begin{pgfscope}%
\definecolor{textcolor}{rgb}{0.000000,0.000000,0.000000}%
\pgfsetstrokecolor{textcolor}%
\pgfsetfillcolor{textcolor}%
\pgftext[x=0.296390in, y=6.577473in, left, base]{\color{textcolor}\sffamily\fontsize{10.000000}{12.000000}\selectfont 75}%
\end{pgfscope}%
\begin{pgfscope}%
\pgfsetbuttcap%
\pgfsetroundjoin%
\definecolor{currentfill}{rgb}{0.000000,0.000000,0.000000}%
\pgfsetfillcolor{currentfill}%
\pgfsetlinewidth{0.803000pt}%
\definecolor{currentstroke}{rgb}{0.000000,0.000000,0.000000}%
\pgfsetstrokecolor{currentstroke}%
\pgfsetdash{}{0pt}%
\pgfsys@defobject{currentmarker}{\pgfqpoint{-0.048611in}{0.000000in}}{\pgfqpoint{-0.000000in}{0.000000in}}{%
\pgfpathmoveto{\pgfqpoint{-0.000000in}{0.000000in}}%
\pgfpathlineto{\pgfqpoint{-0.048611in}{0.000000in}}%
\pgfusepath{stroke,fill}%
}%
\begin{pgfscope}%
\pgfsys@transformshift{0.570343in}{7.494455in}%
\pgfsys@useobject{currentmarker}{}%
\end{pgfscope}%
\end{pgfscope}%
\begin{pgfscope}%
\definecolor{textcolor}{rgb}{0.000000,0.000000,0.000000}%
\pgfsetstrokecolor{textcolor}%
\pgfsetfillcolor{textcolor}%
\pgftext[x=0.208025in, y=7.441694in, left, base]{\color{textcolor}\sffamily\fontsize{10.000000}{12.000000}\selectfont 100}%
\end{pgfscope}%
\begin{pgfscope}%
\pgfpathrectangle{\pgfqpoint{0.570343in}{0.331635in}}{\pgfqpoint{9.300000in}{7.700000in}}%
\pgfusepath{clip}%
\pgfsetrectcap%
\pgfsetroundjoin%
\pgfsetlinewidth{1.505625pt}%
\definecolor{currentstroke}{rgb}{0.631373,0.788235,0.956863}%
\pgfsetstrokecolor{currentstroke}%
\pgfsetstrokeopacity{0.800000}%
\pgfsetdash{}{0pt}%
\pgfpathmoveto{\pgfqpoint{2.375246in}{6.939839in}}%
\pgfpathlineto{\pgfqpoint{4.244205in}{3.787232in}}%
\pgfusepath{stroke}%
\end{pgfscope}%
\begin{pgfscope}%
\pgfpathrectangle{\pgfqpoint{0.570343in}{0.331635in}}{\pgfqpoint{9.300000in}{7.700000in}}%
\pgfusepath{clip}%
\pgfsetrectcap%
\pgfsetroundjoin%
\pgfsetlinewidth{1.505625pt}%
\definecolor{currentstroke}{rgb}{0.631373,0.788235,0.956863}%
\pgfsetstrokecolor{currentstroke}%
\pgfsetstrokeopacity{0.800000}%
\pgfsetdash{}{0pt}%
\pgfpathmoveto{\pgfqpoint{3.492960in}{4.461367in}}%
\pgfpathlineto{\pgfqpoint{4.244205in}{3.787232in}}%
\pgfusepath{stroke}%
\end{pgfscope}%
\begin{pgfscope}%
\pgfpathrectangle{\pgfqpoint{0.570343in}{0.331635in}}{\pgfqpoint{9.300000in}{7.700000in}}%
\pgfusepath{clip}%
\pgfsetrectcap%
\pgfsetroundjoin%
\pgfsetlinewidth{1.505625pt}%
\definecolor{currentstroke}{rgb}{0.631373,0.788235,0.956863}%
\pgfsetstrokecolor{currentstroke}%
\pgfsetstrokeopacity{0.800000}%
\pgfsetdash{}{0pt}%
\pgfpathmoveto{\pgfqpoint{5.673563in}{5.072505in}}%
\pgfpathlineto{\pgfqpoint{4.244205in}{3.787232in}}%
\pgfusepath{stroke}%
\end{pgfscope}%
\begin{pgfscope}%
\pgfpathrectangle{\pgfqpoint{0.570343in}{0.331635in}}{\pgfqpoint{9.300000in}{7.700000in}}%
\pgfusepath{clip}%
\pgfsetrectcap%
\pgfsetroundjoin%
\pgfsetlinewidth{1.505625pt}%
\definecolor{currentstroke}{rgb}{0.631373,0.788235,0.956863}%
\pgfsetstrokecolor{currentstroke}%
\pgfsetstrokeopacity{0.800000}%
\pgfsetdash{}{0pt}%
\pgfpathmoveto{\pgfqpoint{3.420050in}{0.681635in}}%
\pgfpathlineto{\pgfqpoint{4.244205in}{3.787232in}}%
\pgfusepath{stroke}%
\end{pgfscope}%
\begin{pgfscope}%
\pgfpathrectangle{\pgfqpoint{0.570343in}{0.331635in}}{\pgfqpoint{9.300000in}{7.700000in}}%
\pgfusepath{clip}%
\pgfsetrectcap%
\pgfsetroundjoin%
\pgfsetlinewidth{1.505625pt}%
\definecolor{currentstroke}{rgb}{0.631373,0.788235,0.956863}%
\pgfsetstrokecolor{currentstroke}%
\pgfsetstrokeopacity{0.800000}%
\pgfsetdash{}{0pt}%
\pgfpathmoveto{\pgfqpoint{4.927031in}{7.681635in}}%
\pgfpathlineto{\pgfqpoint{4.244205in}{3.787232in}}%
\pgfusepath{stroke}%
\end{pgfscope}%
\begin{pgfscope}%
\pgfpathrectangle{\pgfqpoint{0.570343in}{0.331635in}}{\pgfqpoint{9.300000in}{7.700000in}}%
\pgfusepath{clip}%
\pgfsetrectcap%
\pgfsetroundjoin%
\pgfsetlinewidth{1.505625pt}%
\definecolor{currentstroke}{rgb}{0.631373,0.788235,0.956863}%
\pgfsetstrokecolor{currentstroke}%
\pgfsetstrokeopacity{0.800000}%
\pgfsetdash{}{0pt}%
\pgfpathmoveto{\pgfqpoint{8.854047in}{5.581742in}}%
\pgfpathlineto{\pgfqpoint{4.244205in}{3.787232in}}%
\pgfusepath{stroke}%
\end{pgfscope}%
\begin{pgfscope}%
\pgfpathrectangle{\pgfqpoint{0.570343in}{0.331635in}}{\pgfqpoint{9.300000in}{7.700000in}}%
\pgfusepath{clip}%
\pgfsetrectcap%
\pgfsetroundjoin%
\pgfsetlinewidth{1.505625pt}%
\definecolor{currentstroke}{rgb}{0.631373,0.788235,0.956863}%
\pgfsetstrokecolor{currentstroke}%
\pgfsetstrokeopacity{0.800000}%
\pgfsetdash{}{0pt}%
\pgfpathmoveto{\pgfqpoint{1.547559in}{4.004022in}}%
\pgfpathlineto{\pgfqpoint{4.244205in}{3.787232in}}%
\pgfusepath{stroke}%
\end{pgfscope}%
\begin{pgfscope}%
\pgfpathrectangle{\pgfqpoint{0.570343in}{0.331635in}}{\pgfqpoint{9.300000in}{7.700000in}}%
\pgfusepath{clip}%
\pgfsetrectcap%
\pgfsetroundjoin%
\pgfsetlinewidth{1.505625pt}%
\definecolor{currentstroke}{rgb}{0.631373,0.788235,0.956863}%
\pgfsetstrokecolor{currentstroke}%
\pgfsetstrokeopacity{0.800000}%
\pgfsetdash{}{0pt}%
\pgfpathmoveto{\pgfqpoint{2.739938in}{3.714075in}}%
\pgfpathlineto{\pgfqpoint{4.244205in}{3.787232in}}%
\pgfusepath{stroke}%
\end{pgfscope}%
\begin{pgfscope}%
\pgfpathrectangle{\pgfqpoint{0.570343in}{0.331635in}}{\pgfqpoint{9.300000in}{7.700000in}}%
\pgfusepath{clip}%
\pgfsetrectcap%
\pgfsetroundjoin%
\pgfsetlinewidth{1.505625pt}%
\definecolor{currentstroke}{rgb}{0.631373,0.788235,0.956863}%
\pgfsetstrokecolor{currentstroke}%
\pgfsetstrokeopacity{0.800000}%
\pgfsetdash{}{0pt}%
\pgfpathmoveto{\pgfqpoint{7.703434in}{5.240049in}}%
\pgfpathlineto{\pgfqpoint{4.244205in}{3.787232in}}%
\pgfusepath{stroke}%
\end{pgfscope}%
\begin{pgfscope}%
\pgfpathrectangle{\pgfqpoint{0.570343in}{0.331635in}}{\pgfqpoint{9.300000in}{7.700000in}}%
\pgfusepath{clip}%
\pgfsetrectcap%
\pgfsetroundjoin%
\pgfsetlinewidth{1.505625pt}%
\definecolor{currentstroke}{rgb}{0.631373,0.788235,0.956863}%
\pgfsetstrokecolor{currentstroke}%
\pgfsetstrokeopacity{0.800000}%
\pgfsetdash{}{0pt}%
\pgfpathmoveto{\pgfqpoint{7.113284in}{2.811066in}}%
\pgfpathlineto{\pgfqpoint{4.244205in}{3.787232in}}%
\pgfusepath{stroke}%
\end{pgfscope}%
\begin{pgfscope}%
\pgfpathrectangle{\pgfqpoint{0.570343in}{0.331635in}}{\pgfqpoint{9.300000in}{7.700000in}}%
\pgfusepath{clip}%
\pgfsetrectcap%
\pgfsetroundjoin%
\pgfsetlinewidth{1.505625pt}%
\definecolor{currentstroke}{rgb}{0.631373,0.788235,0.956863}%
\pgfsetstrokecolor{currentstroke}%
\pgfsetstrokeopacity{0.800000}%
\pgfsetdash{}{0pt}%
\pgfpathmoveto{\pgfqpoint{2.082625in}{2.131804in}}%
\pgfpathlineto{\pgfqpoint{4.244205in}{3.787232in}}%
\pgfusepath{stroke}%
\end{pgfscope}%
\begin{pgfscope}%
\pgfpathrectangle{\pgfqpoint{0.570343in}{0.331635in}}{\pgfqpoint{9.300000in}{7.700000in}}%
\pgfusepath{clip}%
\pgfsetrectcap%
\pgfsetroundjoin%
\pgfsetlinewidth{1.505625pt}%
\definecolor{currentstroke}{rgb}{0.631373,0.788235,0.956863}%
\pgfsetstrokecolor{currentstroke}%
\pgfsetstrokeopacity{0.800000}%
\pgfsetdash{}{0pt}%
\pgfpathmoveto{\pgfqpoint{8.230336in}{6.133416in}}%
\pgfpathlineto{\pgfqpoint{4.244205in}{3.787232in}}%
\pgfusepath{stroke}%
\end{pgfscope}%
\begin{pgfscope}%
\pgfpathrectangle{\pgfqpoint{0.570343in}{0.331635in}}{\pgfqpoint{9.300000in}{7.700000in}}%
\pgfusepath{clip}%
\pgfsetrectcap%
\pgfsetroundjoin%
\pgfsetlinewidth{1.505625pt}%
\definecolor{currentstroke}{rgb}{0.631373,0.788235,0.956863}%
\pgfsetstrokecolor{currentstroke}%
\pgfsetstrokeopacity{0.800000}%
\pgfsetdash{}{0pt}%
\pgfpathmoveto{\pgfqpoint{4.884974in}{2.630620in}}%
\pgfpathlineto{\pgfqpoint{4.244205in}{3.787232in}}%
\pgfusepath{stroke}%
\end{pgfscope}%
\begin{pgfscope}%
\pgfpathrectangle{\pgfqpoint{0.570343in}{0.331635in}}{\pgfqpoint{9.300000in}{7.700000in}}%
\pgfusepath{clip}%
\pgfsetrectcap%
\pgfsetroundjoin%
\pgfsetlinewidth{1.505625pt}%
\definecolor{currentstroke}{rgb}{0.631373,0.788235,0.956863}%
\pgfsetstrokecolor{currentstroke}%
\pgfsetstrokeopacity{0.800000}%
\pgfsetdash{}{0pt}%
\pgfpathmoveto{\pgfqpoint{1.225108in}{2.782742in}}%
\pgfpathlineto{\pgfqpoint{4.244205in}{3.787232in}}%
\pgfusepath{stroke}%
\end{pgfscope}%
\begin{pgfscope}%
\pgfpathrectangle{\pgfqpoint{0.570343in}{0.331635in}}{\pgfqpoint{9.300000in}{7.700000in}}%
\pgfusepath{clip}%
\pgfsetrectcap%
\pgfsetroundjoin%
\pgfsetlinewidth{1.505625pt}%
\definecolor{currentstroke}{rgb}{0.631373,0.788235,0.956863}%
\pgfsetstrokecolor{currentstroke}%
\pgfsetstrokeopacity{0.800000}%
\pgfsetdash{}{0pt}%
\pgfpathmoveto{\pgfqpoint{3.946393in}{3.478730in}}%
\pgfpathlineto{\pgfqpoint{4.244205in}{3.787232in}}%
\pgfusepath{stroke}%
\end{pgfscope}%
\begin{pgfscope}%
\pgfpathrectangle{\pgfqpoint{0.570343in}{0.331635in}}{\pgfqpoint{9.300000in}{7.700000in}}%
\pgfusepath{clip}%
\pgfsetrectcap%
\pgfsetroundjoin%
\pgfsetlinewidth{1.505625pt}%
\definecolor{currentstroke}{rgb}{0.631373,0.788235,0.956863}%
\pgfsetstrokecolor{currentstroke}%
\pgfsetstrokeopacity{0.800000}%
\pgfsetdash{}{0pt}%
\pgfpathmoveto{\pgfqpoint{4.886393in}{1.889784in}}%
\pgfpathlineto{\pgfqpoint{4.244205in}{3.787232in}}%
\pgfusepath{stroke}%
\end{pgfscope}%
\begin{pgfscope}%
\pgfpathrectangle{\pgfqpoint{0.570343in}{0.331635in}}{\pgfqpoint{9.300000in}{7.700000in}}%
\pgfusepath{clip}%
\pgfsetrectcap%
\pgfsetroundjoin%
\pgfsetlinewidth{1.505625pt}%
\definecolor{currentstroke}{rgb}{0.631373,0.788235,0.956863}%
\pgfsetstrokecolor{currentstroke}%
\pgfsetstrokeopacity{0.800000}%
\pgfsetdash{}{0pt}%
\pgfpathmoveto{\pgfqpoint{3.784493in}{1.723901in}}%
\pgfpathlineto{\pgfqpoint{4.244205in}{3.787232in}}%
\pgfusepath{stroke}%
\end{pgfscope}%
\begin{pgfscope}%
\pgfpathrectangle{\pgfqpoint{0.570343in}{0.331635in}}{\pgfqpoint{9.300000in}{7.700000in}}%
\pgfusepath{clip}%
\pgfsetrectcap%
\pgfsetroundjoin%
\pgfsetlinewidth{1.505625pt}%
\definecolor{currentstroke}{rgb}{0.631373,0.788235,0.956863}%
\pgfsetstrokecolor{currentstroke}%
\pgfsetstrokeopacity{0.800000}%
\pgfsetdash{}{0pt}%
\pgfpathmoveto{\pgfqpoint{2.706989in}{2.534852in}}%
\pgfpathlineto{\pgfqpoint{4.244205in}{3.787232in}}%
\pgfusepath{stroke}%
\end{pgfscope}%
\begin{pgfscope}%
\pgfpathrectangle{\pgfqpoint{0.570343in}{0.331635in}}{\pgfqpoint{9.300000in}{7.700000in}}%
\pgfusepath{clip}%
\pgfsetrectcap%
\pgfsetroundjoin%
\pgfsetlinewidth{1.505625pt}%
\definecolor{currentstroke}{rgb}{0.631373,0.788235,0.956863}%
\pgfsetstrokecolor{currentstroke}%
\pgfsetstrokeopacity{0.800000}%
\pgfsetdash{}{0pt}%
\pgfpathmoveto{\pgfqpoint{3.495356in}{4.121344in}}%
\pgfpathlineto{\pgfqpoint{4.244205in}{3.787232in}}%
\pgfusepath{stroke}%
\end{pgfscope}%
\begin{pgfscope}%
\pgfpathrectangle{\pgfqpoint{0.570343in}{0.331635in}}{\pgfqpoint{9.300000in}{7.700000in}}%
\pgfusepath{clip}%
\pgfsetrectcap%
\pgfsetroundjoin%
\pgfsetlinewidth{1.505625pt}%
\definecolor{currentstroke}{rgb}{0.631373,0.788235,0.956863}%
\pgfsetstrokecolor{currentstroke}%
\pgfsetstrokeopacity{0.800000}%
\pgfsetdash{}{0pt}%
\pgfpathmoveto{\pgfqpoint{0.993071in}{4.546356in}}%
\pgfpathlineto{\pgfqpoint{4.244205in}{3.787232in}}%
\pgfusepath{stroke}%
\end{pgfscope}%
\begin{pgfscope}%
\pgfpathrectangle{\pgfqpoint{0.570343in}{0.331635in}}{\pgfqpoint{9.300000in}{7.700000in}}%
\pgfusepath{clip}%
\pgfsetrectcap%
\pgfsetroundjoin%
\pgfsetlinewidth{1.505625pt}%
\definecolor{currentstroke}{rgb}{0.631373,0.788235,0.956863}%
\pgfsetstrokecolor{currentstroke}%
\pgfsetstrokeopacity{0.800000}%
\pgfsetdash{}{0pt}%
\pgfpathmoveto{\pgfqpoint{2.362144in}{4.894987in}}%
\pgfpathlineto{\pgfqpoint{4.244205in}{3.787232in}}%
\pgfusepath{stroke}%
\end{pgfscope}%
\begin{pgfscope}%
\pgfpathrectangle{\pgfqpoint{0.570343in}{0.331635in}}{\pgfqpoint{9.300000in}{7.700000in}}%
\pgfusepath{clip}%
\pgfsetrectcap%
\pgfsetroundjoin%
\pgfsetlinewidth{1.505625pt}%
\definecolor{currentstroke}{rgb}{0.631373,0.788235,0.956863}%
\pgfsetstrokecolor{currentstroke}%
\pgfsetstrokeopacity{0.800000}%
\pgfsetdash{}{0pt}%
\pgfpathmoveto{\pgfqpoint{4.437533in}{3.963128in}}%
\pgfpathlineto{\pgfqpoint{4.244205in}{3.787232in}}%
\pgfusepath{stroke}%
\end{pgfscope}%
\begin{pgfscope}%
\pgfpathrectangle{\pgfqpoint{0.570343in}{0.331635in}}{\pgfqpoint{9.300000in}{7.700000in}}%
\pgfusepath{clip}%
\pgfsetrectcap%
\pgfsetroundjoin%
\pgfsetlinewidth{1.505625pt}%
\definecolor{currentstroke}{rgb}{0.631373,0.788235,0.956863}%
\pgfsetstrokecolor{currentstroke}%
\pgfsetstrokeopacity{0.800000}%
\pgfsetdash{}{0pt}%
\pgfpathmoveto{\pgfqpoint{3.510625in}{5.173625in}}%
\pgfpathlineto{\pgfqpoint{4.244205in}{3.787232in}}%
\pgfusepath{stroke}%
\end{pgfscope}%
\begin{pgfscope}%
\pgfpathrectangle{\pgfqpoint{0.570343in}{0.331635in}}{\pgfqpoint{9.300000in}{7.700000in}}%
\pgfusepath{clip}%
\pgfsetrectcap%
\pgfsetroundjoin%
\pgfsetlinewidth{1.505625pt}%
\definecolor{currentstroke}{rgb}{0.631373,0.788235,0.956863}%
\pgfsetstrokecolor{currentstroke}%
\pgfsetstrokeopacity{0.800000}%
\pgfsetdash{}{0pt}%
\pgfpathmoveto{\pgfqpoint{4.054617in}{2.299379in}}%
\pgfpathlineto{\pgfqpoint{4.244205in}{3.787232in}}%
\pgfusepath{stroke}%
\end{pgfscope}%
\begin{pgfscope}%
\pgfpathrectangle{\pgfqpoint{0.570343in}{0.331635in}}{\pgfqpoint{9.300000in}{7.700000in}}%
\pgfusepath{clip}%
\pgfsetrectcap%
\pgfsetroundjoin%
\pgfsetlinewidth{1.505625pt}%
\definecolor{currentstroke}{rgb}{0.631373,0.788235,0.956863}%
\pgfsetstrokecolor{currentstroke}%
\pgfsetstrokeopacity{0.800000}%
\pgfsetdash{}{0pt}%
\pgfpathmoveto{\pgfqpoint{4.954557in}{3.346327in}}%
\pgfpathlineto{\pgfqpoint{4.244205in}{3.787232in}}%
\pgfusepath{stroke}%
\end{pgfscope}%
\begin{pgfscope}%
\pgfpathrectangle{\pgfqpoint{0.570343in}{0.331635in}}{\pgfqpoint{9.300000in}{7.700000in}}%
\pgfusepath{clip}%
\pgfsetrectcap%
\pgfsetroundjoin%
\pgfsetlinewidth{1.505625pt}%
\definecolor{currentstroke}{rgb}{0.631373,0.788235,0.956863}%
\pgfsetstrokecolor{currentstroke}%
\pgfsetstrokeopacity{0.800000}%
\pgfsetdash{}{0pt}%
\pgfpathmoveto{\pgfqpoint{2.539065in}{1.397152in}}%
\pgfpathlineto{\pgfqpoint{4.244205in}{3.787232in}}%
\pgfusepath{stroke}%
\end{pgfscope}%
\begin{pgfscope}%
\pgfpathrectangle{\pgfqpoint{0.570343in}{0.331635in}}{\pgfqpoint{9.300000in}{7.700000in}}%
\pgfusepath{clip}%
\pgfsetrectcap%
\pgfsetroundjoin%
\pgfsetlinewidth{1.505625pt}%
\definecolor{currentstroke}{rgb}{0.631373,0.788235,0.956863}%
\pgfsetstrokecolor{currentstroke}%
\pgfsetstrokeopacity{0.800000}%
\pgfsetdash{}{0pt}%
\pgfpathmoveto{\pgfqpoint{6.972947in}{3.710554in}}%
\pgfpathlineto{\pgfqpoint{4.244205in}{3.787232in}}%
\pgfusepath{stroke}%
\end{pgfscope}%
\begin{pgfscope}%
\pgfpathrectangle{\pgfqpoint{0.570343in}{0.331635in}}{\pgfqpoint{9.300000in}{7.700000in}}%
\pgfusepath{clip}%
\pgfsetrectcap%
\pgfsetroundjoin%
\pgfsetlinewidth{1.505625pt}%
\definecolor{currentstroke}{rgb}{0.631373,0.788235,0.956863}%
\pgfsetstrokecolor{currentstroke}%
\pgfsetstrokeopacity{0.800000}%
\pgfsetdash{}{0pt}%
\pgfpathmoveto{\pgfqpoint{5.923406in}{3.095871in}}%
\pgfpathlineto{\pgfqpoint{4.244205in}{3.787232in}}%
\pgfusepath{stroke}%
\end{pgfscope}%
\begin{pgfscope}%
\pgfpathrectangle{\pgfqpoint{0.570343in}{0.331635in}}{\pgfqpoint{9.300000in}{7.700000in}}%
\pgfusepath{clip}%
\pgfsetrectcap%
\pgfsetroundjoin%
\pgfsetlinewidth{1.505625pt}%
\definecolor{currentstroke}{rgb}{1.000000,0.705882,0.509804}%
\pgfsetstrokecolor{currentstroke}%
\pgfsetstrokeopacity{0.800000}%
\pgfsetdash{}{0pt}%
\pgfpathmoveto{\pgfqpoint{6.817529in}{4.835569in}}%
\pgfpathlineto{\pgfqpoint{6.396515in}{4.497368in}}%
\pgfusepath{stroke}%
\end{pgfscope}%
\begin{pgfscope}%
\pgfpathrectangle{\pgfqpoint{0.570343in}{0.331635in}}{\pgfqpoint{9.300000in}{7.700000in}}%
\pgfusepath{clip}%
\pgfsetrectcap%
\pgfsetroundjoin%
\pgfsetlinewidth{1.505625pt}%
\definecolor{currentstroke}{rgb}{1.000000,0.705882,0.509804}%
\pgfsetstrokecolor{currentstroke}%
\pgfsetstrokeopacity{0.800000}%
\pgfsetdash{}{0pt}%
\pgfpathmoveto{\pgfqpoint{6.117416in}{7.409866in}}%
\pgfpathlineto{\pgfqpoint{6.396515in}{4.497368in}}%
\pgfusepath{stroke}%
\end{pgfscope}%
\begin{pgfscope}%
\pgfpathrectangle{\pgfqpoint{0.570343in}{0.331635in}}{\pgfqpoint{9.300000in}{7.700000in}}%
\pgfusepath{clip}%
\pgfsetrectcap%
\pgfsetroundjoin%
\pgfsetlinewidth{1.505625pt}%
\definecolor{currentstroke}{rgb}{1.000000,0.705882,0.509804}%
\pgfsetstrokecolor{currentstroke}%
\pgfsetstrokeopacity{0.800000}%
\pgfsetdash{}{0pt}%
\pgfpathmoveto{\pgfqpoint{7.933795in}{3.410567in}}%
\pgfpathlineto{\pgfqpoint{6.396515in}{4.497368in}}%
\pgfusepath{stroke}%
\end{pgfscope}%
\begin{pgfscope}%
\pgfpathrectangle{\pgfqpoint{0.570343in}{0.331635in}}{\pgfqpoint{9.300000in}{7.700000in}}%
\pgfusepath{clip}%
\pgfsetrectcap%
\pgfsetroundjoin%
\pgfsetlinewidth{1.505625pt}%
\definecolor{currentstroke}{rgb}{1.000000,0.705882,0.509804}%
\pgfsetstrokecolor{currentstroke}%
\pgfsetstrokeopacity{0.800000}%
\pgfsetdash{}{0pt}%
\pgfpathmoveto{\pgfqpoint{7.377901in}{4.262414in}}%
\pgfpathlineto{\pgfqpoint{6.396515in}{4.497368in}}%
\pgfusepath{stroke}%
\end{pgfscope}%
\begin{pgfscope}%
\pgfpathrectangle{\pgfqpoint{0.570343in}{0.331635in}}{\pgfqpoint{9.300000in}{7.700000in}}%
\pgfusepath{clip}%
\pgfsetrectcap%
\pgfsetroundjoin%
\pgfsetlinewidth{1.505625pt}%
\definecolor{currentstroke}{rgb}{1.000000,0.705882,0.509804}%
\pgfsetstrokecolor{currentstroke}%
\pgfsetstrokeopacity{0.800000}%
\pgfsetdash{}{0pt}%
\pgfpathmoveto{\pgfqpoint{7.284262in}{2.147125in}}%
\pgfpathlineto{\pgfqpoint{6.396515in}{4.497368in}}%
\pgfusepath{stroke}%
\end{pgfscope}%
\begin{pgfscope}%
\pgfpathrectangle{\pgfqpoint{0.570343in}{0.331635in}}{\pgfqpoint{9.300000in}{7.700000in}}%
\pgfusepath{clip}%
\pgfsetrectcap%
\pgfsetroundjoin%
\pgfsetlinewidth{1.505625pt}%
\definecolor{currentstroke}{rgb}{1.000000,0.705882,0.509804}%
\pgfsetstrokecolor{currentstroke}%
\pgfsetstrokeopacity{0.800000}%
\pgfsetdash{}{0pt}%
\pgfpathmoveto{\pgfqpoint{5.804302in}{4.463254in}}%
\pgfpathlineto{\pgfqpoint{6.396515in}{4.497368in}}%
\pgfusepath{stroke}%
\end{pgfscope}%
\begin{pgfscope}%
\pgfpathrectangle{\pgfqpoint{0.570343in}{0.331635in}}{\pgfqpoint{9.300000in}{7.700000in}}%
\pgfusepath{clip}%
\pgfsetrectcap%
\pgfsetroundjoin%
\pgfsetlinewidth{1.505625pt}%
\definecolor{currentstroke}{rgb}{1.000000,0.705882,0.509804}%
\pgfsetstrokecolor{currentstroke}%
\pgfsetstrokeopacity{0.800000}%
\pgfsetdash{}{0pt}%
\pgfpathmoveto{\pgfqpoint{9.282814in}{3.539643in}}%
\pgfpathlineto{\pgfqpoint{6.396515in}{4.497368in}}%
\pgfusepath{stroke}%
\end{pgfscope}%
\begin{pgfscope}%
\pgfpathrectangle{\pgfqpoint{0.570343in}{0.331635in}}{\pgfqpoint{9.300000in}{7.700000in}}%
\pgfusepath{clip}%
\pgfsetrectcap%
\pgfsetroundjoin%
\pgfsetlinewidth{1.505625pt}%
\definecolor{currentstroke}{rgb}{1.000000,0.705882,0.509804}%
\pgfsetstrokecolor{currentstroke}%
\pgfsetstrokeopacity{0.800000}%
\pgfsetdash{}{0pt}%
\pgfpathmoveto{\pgfqpoint{5.814279in}{3.918083in}}%
\pgfpathlineto{\pgfqpoint{6.396515in}{4.497368in}}%
\pgfusepath{stroke}%
\end{pgfscope}%
\begin{pgfscope}%
\pgfpathrectangle{\pgfqpoint{0.570343in}{0.331635in}}{\pgfqpoint{9.300000in}{7.700000in}}%
\pgfusepath{clip}%
\pgfsetrectcap%
\pgfsetroundjoin%
\pgfsetlinewidth{1.505625pt}%
\definecolor{currentstroke}{rgb}{1.000000,0.705882,0.509804}%
\pgfsetstrokecolor{currentstroke}%
\pgfsetstrokeopacity{0.800000}%
\pgfsetdash{}{0pt}%
\pgfpathmoveto{\pgfqpoint{6.347009in}{5.438746in}}%
\pgfpathlineto{\pgfqpoint{6.396515in}{4.497368in}}%
\pgfusepath{stroke}%
\end{pgfscope}%
\begin{pgfscope}%
\pgfpathrectangle{\pgfqpoint{0.570343in}{0.331635in}}{\pgfqpoint{9.300000in}{7.700000in}}%
\pgfusepath{clip}%
\pgfsetrectcap%
\pgfsetroundjoin%
\pgfsetlinewidth{1.505625pt}%
\definecolor{currentstroke}{rgb}{1.000000,0.705882,0.509804}%
\pgfsetstrokecolor{currentstroke}%
\pgfsetstrokeopacity{0.800000}%
\pgfsetdash{}{0pt}%
\pgfpathmoveto{\pgfqpoint{8.386537in}{2.548082in}}%
\pgfpathlineto{\pgfqpoint{6.396515in}{4.497368in}}%
\pgfusepath{stroke}%
\end{pgfscope}%
\begin{pgfscope}%
\pgfpathrectangle{\pgfqpoint{0.570343in}{0.331635in}}{\pgfqpoint{9.300000in}{7.700000in}}%
\pgfusepath{clip}%
\pgfsetrectcap%
\pgfsetroundjoin%
\pgfsetlinewidth{1.505625pt}%
\definecolor{currentstroke}{rgb}{1.000000,0.705882,0.509804}%
\pgfsetstrokecolor{currentstroke}%
\pgfsetstrokeopacity{0.800000}%
\pgfsetdash{}{0pt}%
\pgfpathmoveto{\pgfqpoint{8.388882in}{3.885674in}}%
\pgfpathlineto{\pgfqpoint{6.396515in}{4.497368in}}%
\pgfusepath{stroke}%
\end{pgfscope}%
\begin{pgfscope}%
\pgfpathrectangle{\pgfqpoint{0.570343in}{0.331635in}}{\pgfqpoint{9.300000in}{7.700000in}}%
\pgfusepath{clip}%
\pgfsetrectcap%
\pgfsetroundjoin%
\pgfsetlinewidth{1.505625pt}%
\definecolor{currentstroke}{rgb}{1.000000,0.705882,0.509804}%
\pgfsetstrokecolor{currentstroke}%
\pgfsetstrokeopacity{0.800000}%
\pgfsetdash{}{0pt}%
\pgfpathmoveto{\pgfqpoint{3.136766in}{5.850270in}}%
\pgfpathlineto{\pgfqpoint{6.396515in}{4.497368in}}%
\pgfusepath{stroke}%
\end{pgfscope}%
\begin{pgfscope}%
\pgfpathrectangle{\pgfqpoint{0.570343in}{0.331635in}}{\pgfqpoint{9.300000in}{7.700000in}}%
\pgfusepath{clip}%
\pgfsetrectcap%
\pgfsetroundjoin%
\pgfsetlinewidth{1.505625pt}%
\definecolor{currentstroke}{rgb}{1.000000,0.705882,0.509804}%
\pgfsetstrokecolor{currentstroke}%
\pgfsetstrokeopacity{0.800000}%
\pgfsetdash{}{0pt}%
\pgfpathmoveto{\pgfqpoint{5.512434in}{6.004409in}}%
\pgfpathlineto{\pgfqpoint{6.396515in}{4.497368in}}%
\pgfusepath{stroke}%
\end{pgfscope}%
\begin{pgfscope}%
\pgfpathrectangle{\pgfqpoint{0.570343in}{0.331635in}}{\pgfqpoint{9.300000in}{7.700000in}}%
\pgfusepath{clip}%
\pgfsetrectcap%
\pgfsetroundjoin%
\pgfsetlinewidth{1.505625pt}%
\definecolor{currentstroke}{rgb}{1.000000,0.705882,0.509804}%
\pgfsetstrokecolor{currentstroke}%
\pgfsetstrokeopacity{0.800000}%
\pgfsetdash{}{0pt}%
\pgfpathmoveto{\pgfqpoint{7.456205in}{1.550207in}}%
\pgfpathlineto{\pgfqpoint{6.396515in}{4.497368in}}%
\pgfusepath{stroke}%
\end{pgfscope}%
\begin{pgfscope}%
\pgfpathrectangle{\pgfqpoint{0.570343in}{0.331635in}}{\pgfqpoint{9.300000in}{7.700000in}}%
\pgfusepath{clip}%
\pgfsetrectcap%
\pgfsetroundjoin%
\pgfsetlinewidth{1.505625pt}%
\definecolor{currentstroke}{rgb}{1.000000,0.705882,0.509804}%
\pgfsetstrokecolor{currentstroke}%
\pgfsetstrokeopacity{0.800000}%
\pgfsetdash{}{0pt}%
\pgfpathmoveto{\pgfqpoint{4.644507in}{6.767325in}}%
\pgfpathlineto{\pgfqpoint{6.396515in}{4.497368in}}%
\pgfusepath{stroke}%
\end{pgfscope}%
\begin{pgfscope}%
\pgfpathrectangle{\pgfqpoint{0.570343in}{0.331635in}}{\pgfqpoint{9.300000in}{7.700000in}}%
\pgfusepath{clip}%
\pgfsetrectcap%
\pgfsetroundjoin%
\pgfsetlinewidth{1.505625pt}%
\definecolor{currentstroke}{rgb}{1.000000,0.705882,0.509804}%
\pgfsetstrokecolor{currentstroke}%
\pgfsetstrokeopacity{0.800000}%
\pgfsetdash{}{0pt}%
\pgfpathmoveto{\pgfqpoint{4.552181in}{5.527069in}}%
\pgfpathlineto{\pgfqpoint{6.396515in}{4.497368in}}%
\pgfusepath{stroke}%
\end{pgfscope}%
\begin{pgfscope}%
\pgfpathrectangle{\pgfqpoint{0.570343in}{0.331635in}}{\pgfqpoint{9.300000in}{7.700000in}}%
\pgfusepath{clip}%
\pgfsetrectcap%
\pgfsetroundjoin%
\pgfsetlinewidth{1.505625pt}%
\definecolor{currentstroke}{rgb}{1.000000,0.705882,0.509804}%
\pgfsetstrokecolor{currentstroke}%
\pgfsetstrokeopacity{0.800000}%
\pgfsetdash{}{0pt}%
\pgfpathmoveto{\pgfqpoint{7.082491in}{7.149978in}}%
\pgfpathlineto{\pgfqpoint{6.396515in}{4.497368in}}%
\pgfusepath{stroke}%
\end{pgfscope}%
\begin{pgfscope}%
\pgfpathrectangle{\pgfqpoint{0.570343in}{0.331635in}}{\pgfqpoint{9.300000in}{7.700000in}}%
\pgfusepath{clip}%
\pgfsetrectcap%
\pgfsetroundjoin%
\pgfsetlinewidth{1.505625pt}%
\definecolor{currentstroke}{rgb}{1.000000,0.705882,0.509804}%
\pgfsetstrokecolor{currentstroke}%
\pgfsetstrokeopacity{0.800000}%
\pgfsetdash{}{0pt}%
\pgfpathmoveto{\pgfqpoint{6.077373in}{2.333008in}}%
\pgfpathlineto{\pgfqpoint{6.396515in}{4.497368in}}%
\pgfusepath{stroke}%
\end{pgfscope}%
\begin{pgfscope}%
\pgfpathrectangle{\pgfqpoint{0.570343in}{0.331635in}}{\pgfqpoint{9.300000in}{7.700000in}}%
\pgfusepath{clip}%
\pgfsetrectcap%
\pgfsetroundjoin%
\pgfsetlinewidth{1.505625pt}%
\definecolor{currentstroke}{rgb}{1.000000,0.705882,0.509804}%
\pgfsetstrokecolor{currentstroke}%
\pgfsetstrokeopacity{0.800000}%
\pgfsetdash{}{0pt}%
\pgfpathmoveto{\pgfqpoint{4.868813in}{0.980584in}}%
\pgfpathlineto{\pgfqpoint{6.396515in}{4.497368in}}%
\pgfusepath{stroke}%
\end{pgfscope}%
\begin{pgfscope}%
\pgfpathrectangle{\pgfqpoint{0.570343in}{0.331635in}}{\pgfqpoint{9.300000in}{7.700000in}}%
\pgfusepath{clip}%
\pgfsetrectcap%
\pgfsetroundjoin%
\pgfsetlinewidth{1.505625pt}%
\definecolor{currentstroke}{rgb}{1.000000,0.705882,0.509804}%
\pgfsetstrokecolor{currentstroke}%
\pgfsetstrokeopacity{0.800000}%
\pgfsetdash{}{0pt}%
\pgfpathmoveto{\pgfqpoint{4.793914in}{4.789225in}}%
\pgfpathlineto{\pgfqpoint{6.396515in}{4.497368in}}%
\pgfusepath{stroke}%
\end{pgfscope}%
\begin{pgfscope}%
\pgfpathrectangle{\pgfqpoint{0.570343in}{0.331635in}}{\pgfqpoint{9.300000in}{7.700000in}}%
\pgfusepath{clip}%
\pgfsetrectcap%
\pgfsetroundjoin%
\pgfsetlinewidth{1.505625pt}%
\definecolor{currentstroke}{rgb}{1.000000,0.705882,0.509804}%
\pgfsetstrokecolor{currentstroke}%
\pgfsetstrokeopacity{0.800000}%
\pgfsetdash{}{0pt}%
\pgfpathmoveto{\pgfqpoint{6.545799in}{6.396589in}}%
\pgfpathlineto{\pgfqpoint{6.396515in}{4.497368in}}%
\pgfusepath{stroke}%
\end{pgfscope}%
\begin{pgfscope}%
\pgfpathrectangle{\pgfqpoint{0.570343in}{0.331635in}}{\pgfqpoint{9.300000in}{7.700000in}}%
\pgfusepath{clip}%
\pgfsetrectcap%
\pgfsetroundjoin%
\pgfsetlinewidth{1.505625pt}%
\definecolor{currentstroke}{rgb}{1.000000,0.705882,0.509804}%
\pgfsetstrokecolor{currentstroke}%
\pgfsetstrokeopacity{0.800000}%
\pgfsetdash{}{0pt}%
\pgfpathmoveto{\pgfqpoint{3.662589in}{2.831937in}}%
\pgfpathlineto{\pgfqpoint{6.396515in}{4.497368in}}%
\pgfusepath{stroke}%
\end{pgfscope}%
\begin{pgfscope}%
\pgfpathrectangle{\pgfqpoint{0.570343in}{0.331635in}}{\pgfqpoint{9.300000in}{7.700000in}}%
\pgfusepath{clip}%
\pgfsetrectcap%
\pgfsetroundjoin%
\pgfsetlinewidth{1.505625pt}%
\definecolor{currentstroke}{rgb}{1.000000,0.705882,0.509804}%
\pgfsetstrokecolor{currentstroke}%
\pgfsetstrokeopacity{0.800000}%
\pgfsetdash{}{0pt}%
\pgfpathmoveto{\pgfqpoint{8.266806in}{7.441823in}}%
\pgfpathlineto{\pgfqpoint{6.396515in}{4.497368in}}%
\pgfusepath{stroke}%
\end{pgfscope}%
\begin{pgfscope}%
\pgfpathrectangle{\pgfqpoint{0.570343in}{0.331635in}}{\pgfqpoint{9.300000in}{7.700000in}}%
\pgfusepath{clip}%
\pgfsetrectcap%
\pgfsetroundjoin%
\pgfsetlinewidth{1.505625pt}%
\definecolor{currentstroke}{rgb}{1.000000,0.705882,0.509804}%
\pgfsetstrokecolor{currentstroke}%
\pgfsetstrokeopacity{0.800000}%
\pgfsetdash{}{0pt}%
\pgfpathmoveto{\pgfqpoint{1.678869in}{5.981260in}}%
\pgfpathlineto{\pgfqpoint{6.396515in}{4.497368in}}%
\pgfusepath{stroke}%
\end{pgfscope}%
\begin{pgfscope}%
\pgfpathrectangle{\pgfqpoint{0.570343in}{0.331635in}}{\pgfqpoint{9.300000in}{7.700000in}}%
\pgfusepath{clip}%
\pgfsetrectcap%
\pgfsetroundjoin%
\pgfsetlinewidth{1.505625pt}%
\definecolor{currentstroke}{rgb}{1.000000,0.705882,0.509804}%
\pgfsetstrokecolor{currentstroke}%
\pgfsetstrokeopacity{0.800000}%
\pgfsetdash{}{0pt}%
\pgfpathmoveto{\pgfqpoint{8.307790in}{4.579090in}}%
\pgfpathlineto{\pgfqpoint{6.396515in}{4.497368in}}%
\pgfusepath{stroke}%
\end{pgfscope}%
\begin{pgfscope}%
\pgfpathrectangle{\pgfqpoint{0.570343in}{0.331635in}}{\pgfqpoint{9.300000in}{7.700000in}}%
\pgfusepath{clip}%
\pgfsetrectcap%
\pgfsetroundjoin%
\pgfsetlinewidth{1.505625pt}%
\definecolor{currentstroke}{rgb}{1.000000,0.705882,0.509804}%
\pgfsetstrokecolor{currentstroke}%
\pgfsetstrokeopacity{0.800000}%
\pgfsetdash{}{0pt}%
\pgfpathmoveto{\pgfqpoint{7.187407in}{5.824946in}}%
\pgfpathlineto{\pgfqpoint{6.396515in}{4.497368in}}%
\pgfusepath{stroke}%
\end{pgfscope}%
\begin{pgfscope}%
\pgfpathrectangle{\pgfqpoint{0.570343in}{0.331635in}}{\pgfqpoint{9.300000in}{7.700000in}}%
\pgfusepath{clip}%
\pgfsetrectcap%
\pgfsetroundjoin%
\pgfsetlinewidth{1.505625pt}%
\definecolor{currentstroke}{rgb}{1.000000,0.705882,0.509804}%
\pgfsetstrokecolor{currentstroke}%
\pgfsetstrokeopacity{0.800000}%
\pgfsetdash{}{0pt}%
\pgfpathmoveto{\pgfqpoint{6.326121in}{1.402458in}}%
\pgfpathlineto{\pgfqpoint{6.396515in}{4.497368in}}%
\pgfusepath{stroke}%
\end{pgfscope}%
\begin{pgfscope}%
\pgfpathrectangle{\pgfqpoint{0.570343in}{0.331635in}}{\pgfqpoint{9.300000in}{7.700000in}}%
\pgfusepath{clip}%
\pgfsetrectcap%
\pgfsetroundjoin%
\pgfsetlinewidth{1.505625pt}%
\definecolor{currentstroke}{rgb}{1.000000,0.705882,0.509804}%
\pgfsetstrokecolor{currentstroke}%
\pgfsetstrokeopacity{0.800000}%
\pgfsetdash{}{0pt}%
\pgfpathmoveto{\pgfqpoint{9.447616in}{4.657092in}}%
\pgfpathlineto{\pgfqpoint{6.396515in}{4.497368in}}%
\pgfusepath{stroke}%
\end{pgfscope}%
\begin{pgfscope}%
\pgfsetrectcap%
\pgfsetmiterjoin%
\pgfsetlinewidth{0.803000pt}%
\definecolor{currentstroke}{rgb}{0.000000,0.000000,0.000000}%
\pgfsetstrokecolor{currentstroke}%
\pgfsetdash{}{0pt}%
\pgfpathmoveto{\pgfqpoint{0.570343in}{0.331635in}}%
\pgfpathlineto{\pgfqpoint{0.570343in}{8.031635in}}%
\pgfusepath{stroke}%
\end{pgfscope}%
\begin{pgfscope}%
\pgfsetrectcap%
\pgfsetmiterjoin%
\pgfsetlinewidth{0.803000pt}%
\definecolor{currentstroke}{rgb}{0.000000,0.000000,0.000000}%
\pgfsetstrokecolor{currentstroke}%
\pgfsetdash{}{0pt}%
\pgfpathmoveto{\pgfqpoint{9.870343in}{0.331635in}}%
\pgfpathlineto{\pgfqpoint{9.870343in}{8.031635in}}%
\pgfusepath{stroke}%
\end{pgfscope}%
\begin{pgfscope}%
\pgfsetrectcap%
\pgfsetmiterjoin%
\pgfsetlinewidth{0.803000pt}%
\definecolor{currentstroke}{rgb}{0.000000,0.000000,0.000000}%
\pgfsetstrokecolor{currentstroke}%
\pgfsetdash{}{0pt}%
\pgfpathmoveto{\pgfqpoint{0.570343in}{0.331635in}}%
\pgfpathlineto{\pgfqpoint{9.870343in}{0.331635in}}%
\pgfusepath{stroke}%
\end{pgfscope}%
\begin{pgfscope}%
\pgfsetrectcap%
\pgfsetmiterjoin%
\pgfsetlinewidth{0.803000pt}%
\definecolor{currentstroke}{rgb}{0.000000,0.000000,0.000000}%
\pgfsetstrokecolor{currentstroke}%
\pgfsetdash{}{0pt}%
\pgfpathmoveto{\pgfqpoint{0.570343in}{8.031635in}}%
\pgfpathlineto{\pgfqpoint{9.870343in}{8.031635in}}%
\pgfusepath{stroke}%
\end{pgfscope}%
\begin{pgfscope}%
\definecolor{textcolor}{rgb}{0.000000,0.000000,0.000000}%
\pgfsetstrokecolor{textcolor}%
\pgfsetfillcolor{textcolor}%
\pgftext[x=5.220343in,y=8.114968in,,base]{\color{textcolor}\sffamily\fontsize{12.000000}{14.400000}\selectfont Photo-Realistic Images}%
\end{pgfscope}%
\begin{pgfscope}%
\pgfsetbuttcap%
\pgfsetmiterjoin%
\definecolor{currentfill}{rgb}{1.000000,1.000000,1.000000}%
\pgfsetfillcolor{currentfill}%
\pgfsetfillopacity{0.800000}%
\pgfsetlinewidth{1.003750pt}%
\definecolor{currentstroke}{rgb}{0.800000,0.800000,0.800000}%
\pgfsetstrokecolor{currentstroke}%
\pgfsetstrokeopacity{0.800000}%
\pgfsetdash{}{0pt}%
\pgfpathmoveto{\pgfqpoint{9.967566in}{3.956944in}}%
\pgfpathlineto{\pgfqpoint{10.908633in}{3.956944in}}%
\pgfpathquadraticcurveto{\pgfqpoint{10.936411in}{3.956944in}}{\pgfqpoint{10.936411in}{3.984722in}}%
\pgfpathlineto{\pgfqpoint{10.936411in}{4.378548in}}%
\pgfpathquadraticcurveto{\pgfqpoint{10.936411in}{4.406326in}}{\pgfqpoint{10.908633in}{4.406326in}}%
\pgfpathlineto{\pgfqpoint{9.967566in}{4.406326in}}%
\pgfpathquadraticcurveto{\pgfqpoint{9.939788in}{4.406326in}}{\pgfqpoint{9.939788in}{4.378548in}}%
\pgfpathlineto{\pgfqpoint{9.939788in}{3.984722in}}%
\pgfpathquadraticcurveto{\pgfqpoint{9.939788in}{3.956944in}}{\pgfqpoint{9.967566in}{3.956944in}}%
\pgfpathclose%
\pgfusepath{stroke,fill}%
\end{pgfscope}%
\begin{pgfscope}%
\pgfsetbuttcap%
\pgfsetroundjoin%
\definecolor{currentfill}{rgb}{0.631373,0.788235,0.956863}%
\pgfsetfillcolor{currentfill}%
\pgfsetlinewidth{1.003750pt}%
\definecolor{currentstroke}{rgb}{0.631373,0.788235,0.956863}%
\pgfsetstrokecolor{currentstroke}%
\pgfsetdash{}{0pt}%
\pgfsys@defobject{currentmarker}{\pgfqpoint{-0.041667in}{-0.041667in}}{\pgfqpoint{0.041667in}{0.041667in}}{%
\pgfpathmoveto{\pgfqpoint{0.000000in}{-0.041667in}}%
\pgfpathcurveto{\pgfqpoint{0.011050in}{-0.041667in}}{\pgfqpoint{0.021649in}{-0.037276in}}{\pgfqpoint{0.029463in}{-0.029463in}}%
\pgfpathcurveto{\pgfqpoint{0.037276in}{-0.021649in}}{\pgfqpoint{0.041667in}{-0.011050in}}{\pgfqpoint{0.041667in}{0.000000in}}%
\pgfpathcurveto{\pgfqpoint{0.041667in}{0.011050in}}{\pgfqpoint{0.037276in}{0.021649in}}{\pgfqpoint{0.029463in}{0.029463in}}%
\pgfpathcurveto{\pgfqpoint{0.021649in}{0.037276in}}{\pgfqpoint{0.011050in}{0.041667in}}{\pgfqpoint{0.000000in}{0.041667in}}%
\pgfpathcurveto{\pgfqpoint{-0.011050in}{0.041667in}}{\pgfqpoint{-0.021649in}{0.037276in}}{\pgfqpoint{-0.029463in}{0.029463in}}%
\pgfpathcurveto{\pgfqpoint{-0.037276in}{0.021649in}}{\pgfqpoint{-0.041667in}{0.011050in}}{\pgfqpoint{-0.041667in}{0.000000in}}%
\pgfpathcurveto{\pgfqpoint{-0.041667in}{-0.011050in}}{\pgfqpoint{-0.037276in}{-0.021649in}}{\pgfqpoint{-0.029463in}{-0.029463in}}%
\pgfpathcurveto{\pgfqpoint{-0.021649in}{-0.037276in}}{\pgfqpoint{-0.011050in}{-0.041667in}}{\pgfqpoint{0.000000in}{-0.041667in}}%
\pgfpathclose%
\pgfusepath{stroke,fill}%
}%
\begin{pgfscope}%
\pgfsys@transformshift{10.134232in}{4.281705in}%
\pgfsys@useobject{currentmarker}{}%
\end{pgfscope}%
\end{pgfscope}%
\begin{pgfscope}%
\definecolor{textcolor}{rgb}{0.000000,0.000000,0.000000}%
\pgfsetstrokecolor{textcolor}%
\pgfsetfillcolor{textcolor}%
\pgftext[x=10.384232in,y=4.245247in,left,base]{\color{textcolor}\sffamily\fontsize{10.000000}{12.000000}\selectfont ai2thor}%
\end{pgfscope}%
\begin{pgfscope}%
\pgfsetbuttcap%
\pgfsetroundjoin%
\definecolor{currentfill}{rgb}{1.000000,0.705882,0.509804}%
\pgfsetfillcolor{currentfill}%
\pgfsetlinewidth{1.003750pt}%
\definecolor{currentstroke}{rgb}{1.000000,0.705882,0.509804}%
\pgfsetstrokecolor{currentstroke}%
\pgfsetdash{}{0pt}%
\pgfsys@defobject{currentmarker}{\pgfqpoint{-0.041667in}{-0.041667in}}{\pgfqpoint{0.041667in}{0.041667in}}{%
\pgfpathmoveto{\pgfqpoint{0.000000in}{-0.041667in}}%
\pgfpathcurveto{\pgfqpoint{0.011050in}{-0.041667in}}{\pgfqpoint{0.021649in}{-0.037276in}}{\pgfqpoint{0.029463in}{-0.029463in}}%
\pgfpathcurveto{\pgfqpoint{0.037276in}{-0.021649in}}{\pgfqpoint{0.041667in}{-0.011050in}}{\pgfqpoint{0.041667in}{0.000000in}}%
\pgfpathcurveto{\pgfqpoint{0.041667in}{0.011050in}}{\pgfqpoint{0.037276in}{0.021649in}}{\pgfqpoint{0.029463in}{0.029463in}}%
\pgfpathcurveto{\pgfqpoint{0.021649in}{0.037276in}}{\pgfqpoint{0.011050in}{0.041667in}}{\pgfqpoint{0.000000in}{0.041667in}}%
\pgfpathcurveto{\pgfqpoint{-0.011050in}{0.041667in}}{\pgfqpoint{-0.021649in}{0.037276in}}{\pgfqpoint{-0.029463in}{0.029463in}}%
\pgfpathcurveto{\pgfqpoint{-0.037276in}{0.021649in}}{\pgfqpoint{-0.041667in}{0.011050in}}{\pgfqpoint{-0.041667in}{0.000000in}}%
\pgfpathcurveto{\pgfqpoint{-0.041667in}{-0.011050in}}{\pgfqpoint{-0.037276in}{-0.021649in}}{\pgfqpoint{-0.029463in}{-0.029463in}}%
\pgfpathcurveto{\pgfqpoint{-0.021649in}{-0.037276in}}{\pgfqpoint{-0.011050in}{-0.041667in}}{\pgfqpoint{0.000000in}{-0.041667in}}%
\pgfpathclose%
\pgfusepath{stroke,fill}%
}%
\begin{pgfscope}%
\pgfsys@transformshift{10.134232in}{4.077848in}%
\pgfsys@useobject{currentmarker}{}%
\end{pgfscope}%
\end{pgfscope}%
\begin{pgfscope}%
\definecolor{textcolor}{rgb}{0.000000,0.000000,0.000000}%
\pgfsetstrokecolor{textcolor}%
\pgfsetfillcolor{textcolor}%
\pgftext[x=10.384232in,y=4.041390in,left,base]{\color{textcolor}\sffamily\fontsize{10.000000}{12.000000}\selectfont pix3d}%
\end{pgfscope}%
\end{pgfpicture}%
\makeatother%
\endgroup%
}\\
    \resizebox{0.49\linewidth}{5cm}{%% Creator: Matplotlib, PGF backend
%%
%% To include the figure in your LaTeX document, write
%%   \input{<filename>.pgf}
%%
%% Make sure the required packages are loaded in your preamble
%%   \usepackage{pgf}
%%
%% Figures using additional raster images can only be included by \input if
%% they are in the same directory as the main LaTeX file. For loading figures
%% from other directories you can use the `import` package
%%   \usepackage{import}
%%
%% and then include the figures with
%%   \import{<path to file>}{<filename>.pgf}
%%
%% Matplotlib used the following preamble
%%   \usepackage{fontspec}
%%   \setmainfont{DejaVuSerif.ttf}[Path=\detokenize{/Users/apple/opt/anaconda3/envs/kaolin/lib/python3.7/site-packages/matplotlib/mpl-data/fonts/ttf/}]
%%   \setsansfont{DejaVuSans.ttf}[Path=\detokenize{/Users/apple/opt/anaconda3/envs/kaolin/lib/python3.7/site-packages/matplotlib/mpl-data/fonts/ttf/}]
%%   \setmonofont{DejaVuSansMono.ttf}[Path=\detokenize{/Users/apple/opt/anaconda3/envs/kaolin/lib/python3.7/site-packages/matplotlib/mpl-data/fonts/ttf/}]
%%
\begingroup%
\makeatletter%
\begin{pgfpicture}%
\pgfpathrectangle{\pgfpointorigin}{\pgfqpoint{11.046719in}{8.341596in}}%
\pgfusepath{use as bounding box, clip}%
\begin{pgfscope}%
\pgfsetbuttcap%
\pgfsetmiterjoin%
\definecolor{currentfill}{rgb}{1.000000,1.000000,1.000000}%
\pgfsetfillcolor{currentfill}%
\pgfsetlinewidth{0.000000pt}%
\definecolor{currentstroke}{rgb}{1.000000,1.000000,1.000000}%
\pgfsetstrokecolor{currentstroke}%
\pgfsetdash{}{0pt}%
\pgfpathmoveto{\pgfqpoint{0.000000in}{0.000000in}}%
\pgfpathlineto{\pgfqpoint{11.046719in}{0.000000in}}%
\pgfpathlineto{\pgfqpoint{11.046719in}{8.341596in}}%
\pgfpathlineto{\pgfqpoint{0.000000in}{8.341596in}}%
\pgfpathclose%
\pgfusepath{fill}%
\end{pgfscope}%
\begin{pgfscope}%
\pgfsetbuttcap%
\pgfsetmiterjoin%
\definecolor{currentfill}{rgb}{1.000000,1.000000,1.000000}%
\pgfsetfillcolor{currentfill}%
\pgfsetlinewidth{0.000000pt}%
\definecolor{currentstroke}{rgb}{0.000000,0.000000,0.000000}%
\pgfsetstrokecolor{currentstroke}%
\pgfsetstrokeopacity{0.000000}%
\pgfsetdash{}{0pt}%
\pgfpathmoveto{\pgfqpoint{0.570343in}{0.331635in}}%
\pgfpathlineto{\pgfqpoint{9.870343in}{0.331635in}}%
\pgfpathlineto{\pgfqpoint{9.870343in}{8.031635in}}%
\pgfpathlineto{\pgfqpoint{0.570343in}{8.031635in}}%
\pgfpathclose%
\pgfusepath{fill}%
\end{pgfscope}%
\begin{pgfscope}%
\pgfpathrectangle{\pgfqpoint{0.570343in}{0.331635in}}{\pgfqpoint{9.300000in}{7.700000in}}%
\pgfusepath{clip}%
\pgfsetbuttcap%
\pgfsetroundjoin%
\definecolor{currentfill}{rgb}{0.631373,0.788235,0.956863}%
\pgfsetfillcolor{currentfill}%
\pgfsetlinewidth{0.481800pt}%
\definecolor{currentstroke}{rgb}{1.000000,1.000000,1.000000}%
\pgfsetstrokecolor{currentstroke}%
\pgfsetdash{}{0pt}%
\pgfpathmoveto{\pgfqpoint{7.239021in}{5.072456in}}%
\pgfpathcurveto{\pgfqpoint{7.250071in}{5.072456in}}{\pgfqpoint{7.260670in}{5.076846in}}{\pgfqpoint{7.268483in}{5.084660in}}%
\pgfpathcurveto{\pgfqpoint{7.276297in}{5.092474in}}{\pgfqpoint{7.280687in}{5.103073in}}{\pgfqpoint{7.280687in}{5.114123in}}%
\pgfpathcurveto{\pgfqpoint{7.280687in}{5.125173in}}{\pgfqpoint{7.276297in}{5.135772in}}{\pgfqpoint{7.268483in}{5.143586in}}%
\pgfpathcurveto{\pgfqpoint{7.260670in}{5.151399in}}{\pgfqpoint{7.250071in}{5.155789in}}{\pgfqpoint{7.239021in}{5.155789in}}%
\pgfpathcurveto{\pgfqpoint{7.227970in}{5.155789in}}{\pgfqpoint{7.217371in}{5.151399in}}{\pgfqpoint{7.209558in}{5.143586in}}%
\pgfpathcurveto{\pgfqpoint{7.201744in}{5.135772in}}{\pgfqpoint{7.197354in}{5.125173in}}{\pgfqpoint{7.197354in}{5.114123in}}%
\pgfpathcurveto{\pgfqpoint{7.197354in}{5.103073in}}{\pgfqpoint{7.201744in}{5.092474in}}{\pgfqpoint{7.209558in}{5.084660in}}%
\pgfpathcurveto{\pgfqpoint{7.217371in}{5.076846in}}{\pgfqpoint{7.227970in}{5.072456in}}{\pgfqpoint{7.239021in}{5.072456in}}%
\pgfpathclose%
\pgfusepath{stroke,fill}%
\end{pgfscope}%
\begin{pgfscope}%
\pgfpathrectangle{\pgfqpoint{0.570343in}{0.331635in}}{\pgfqpoint{9.300000in}{7.700000in}}%
\pgfusepath{clip}%
\pgfsetbuttcap%
\pgfsetroundjoin%
\definecolor{currentfill}{rgb}{0.631373,0.788235,0.956863}%
\pgfsetfillcolor{currentfill}%
\pgfsetlinewidth{0.481800pt}%
\definecolor{currentstroke}{rgb}{1.000000,1.000000,1.000000}%
\pgfsetstrokecolor{currentstroke}%
\pgfsetdash{}{0pt}%
\pgfpathmoveto{\pgfqpoint{6.513392in}{3.968650in}}%
\pgfpathcurveto{\pgfqpoint{6.524442in}{3.968650in}}{\pgfqpoint{6.535041in}{3.973041in}}{\pgfqpoint{6.542855in}{3.980854in}}%
\pgfpathcurveto{\pgfqpoint{6.550668in}{3.988668in}}{\pgfqpoint{6.555059in}{3.999267in}}{\pgfqpoint{6.555059in}{4.010317in}}%
\pgfpathcurveto{\pgfqpoint{6.555059in}{4.021367in}}{\pgfqpoint{6.550668in}{4.031966in}}{\pgfqpoint{6.542855in}{4.039780in}}%
\pgfpathcurveto{\pgfqpoint{6.535041in}{4.047593in}}{\pgfqpoint{6.524442in}{4.051984in}}{\pgfqpoint{6.513392in}{4.051984in}}%
\pgfpathcurveto{\pgfqpoint{6.502342in}{4.051984in}}{\pgfqpoint{6.491743in}{4.047593in}}{\pgfqpoint{6.483929in}{4.039780in}}%
\pgfpathcurveto{\pgfqpoint{6.476116in}{4.031966in}}{\pgfqpoint{6.471725in}{4.021367in}}{\pgfqpoint{6.471725in}{4.010317in}}%
\pgfpathcurveto{\pgfqpoint{6.471725in}{3.999267in}}{\pgfqpoint{6.476116in}{3.988668in}}{\pgfqpoint{6.483929in}{3.980854in}}%
\pgfpathcurveto{\pgfqpoint{6.491743in}{3.973041in}}{\pgfqpoint{6.502342in}{3.968650in}}{\pgfqpoint{6.513392in}{3.968650in}}%
\pgfpathclose%
\pgfusepath{stroke,fill}%
\end{pgfscope}%
\begin{pgfscope}%
\pgfpathrectangle{\pgfqpoint{0.570343in}{0.331635in}}{\pgfqpoint{9.300000in}{7.700000in}}%
\pgfusepath{clip}%
\pgfsetbuttcap%
\pgfsetroundjoin%
\definecolor{currentfill}{rgb}{0.631373,0.788235,0.956863}%
\pgfsetfillcolor{currentfill}%
\pgfsetlinewidth{0.481800pt}%
\definecolor{currentstroke}{rgb}{1.000000,1.000000,1.000000}%
\pgfsetstrokecolor{currentstroke}%
\pgfsetdash{}{0pt}%
\pgfpathmoveto{\pgfqpoint{2.367651in}{5.480048in}}%
\pgfpathcurveto{\pgfqpoint{2.378701in}{5.480048in}}{\pgfqpoint{2.389300in}{5.484438in}}{\pgfqpoint{2.397114in}{5.492252in}}%
\pgfpathcurveto{\pgfqpoint{2.404928in}{5.500066in}}{\pgfqpoint{2.409318in}{5.510665in}}{\pgfqpoint{2.409318in}{5.521715in}}%
\pgfpathcurveto{\pgfqpoint{2.409318in}{5.532765in}}{\pgfqpoint{2.404928in}{5.543364in}}{\pgfqpoint{2.397114in}{5.551177in}}%
\pgfpathcurveto{\pgfqpoint{2.389300in}{5.558991in}}{\pgfqpoint{2.378701in}{5.563381in}}{\pgfqpoint{2.367651in}{5.563381in}}%
\pgfpathcurveto{\pgfqpoint{2.356601in}{5.563381in}}{\pgfqpoint{2.346002in}{5.558991in}}{\pgfqpoint{2.338188in}{5.551177in}}%
\pgfpathcurveto{\pgfqpoint{2.330375in}{5.543364in}}{\pgfqpoint{2.325985in}{5.532765in}}{\pgfqpoint{2.325985in}{5.521715in}}%
\pgfpathcurveto{\pgfqpoint{2.325985in}{5.510665in}}{\pgfqpoint{2.330375in}{5.500066in}}{\pgfqpoint{2.338188in}{5.492252in}}%
\pgfpathcurveto{\pgfqpoint{2.346002in}{5.484438in}}{\pgfqpoint{2.356601in}{5.480048in}}{\pgfqpoint{2.367651in}{5.480048in}}%
\pgfpathclose%
\pgfusepath{stroke,fill}%
\end{pgfscope}%
\begin{pgfscope}%
\pgfpathrectangle{\pgfqpoint{0.570343in}{0.331635in}}{\pgfqpoint{9.300000in}{7.700000in}}%
\pgfusepath{clip}%
\pgfsetbuttcap%
\pgfsetroundjoin%
\definecolor{currentfill}{rgb}{0.631373,0.788235,0.956863}%
\pgfsetfillcolor{currentfill}%
\pgfsetlinewidth{0.481800pt}%
\definecolor{currentstroke}{rgb}{1.000000,1.000000,1.000000}%
\pgfsetstrokecolor{currentstroke}%
\pgfsetdash{}{0pt}%
\pgfpathmoveto{\pgfqpoint{6.809391in}{4.507992in}}%
\pgfpathcurveto{\pgfqpoint{6.820441in}{4.507992in}}{\pgfqpoint{6.831040in}{4.512382in}}{\pgfqpoint{6.838854in}{4.520196in}}%
\pgfpathcurveto{\pgfqpoint{6.846668in}{4.528009in}}{\pgfqpoint{6.851058in}{4.538608in}}{\pgfqpoint{6.851058in}{4.549658in}}%
\pgfpathcurveto{\pgfqpoint{6.851058in}{4.560708in}}{\pgfqpoint{6.846668in}{4.571307in}}{\pgfqpoint{6.838854in}{4.579121in}}%
\pgfpathcurveto{\pgfqpoint{6.831040in}{4.586935in}}{\pgfqpoint{6.820441in}{4.591325in}}{\pgfqpoint{6.809391in}{4.591325in}}%
\pgfpathcurveto{\pgfqpoint{6.798341in}{4.591325in}}{\pgfqpoint{6.787742in}{4.586935in}}{\pgfqpoint{6.779929in}{4.579121in}}%
\pgfpathcurveto{\pgfqpoint{6.772115in}{4.571307in}}{\pgfqpoint{6.767725in}{4.560708in}}{\pgfqpoint{6.767725in}{4.549658in}}%
\pgfpathcurveto{\pgfqpoint{6.767725in}{4.538608in}}{\pgfqpoint{6.772115in}{4.528009in}}{\pgfqpoint{6.779929in}{4.520196in}}%
\pgfpathcurveto{\pgfqpoint{6.787742in}{4.512382in}}{\pgfqpoint{6.798341in}{4.507992in}}{\pgfqpoint{6.809391in}{4.507992in}}%
\pgfpathclose%
\pgfusepath{stroke,fill}%
\end{pgfscope}%
\begin{pgfscope}%
\pgfpathrectangle{\pgfqpoint{0.570343in}{0.331635in}}{\pgfqpoint{9.300000in}{7.700000in}}%
\pgfusepath{clip}%
\pgfsetbuttcap%
\pgfsetroundjoin%
\definecolor{currentfill}{rgb}{0.631373,0.788235,0.956863}%
\pgfsetfillcolor{currentfill}%
\pgfsetlinewidth{0.481800pt}%
\definecolor{currentstroke}{rgb}{1.000000,1.000000,1.000000}%
\pgfsetstrokecolor{currentstroke}%
\pgfsetdash{}{0pt}%
\pgfpathmoveto{\pgfqpoint{4.564595in}{4.371940in}}%
\pgfpathcurveto{\pgfqpoint{4.575645in}{4.371940in}}{\pgfqpoint{4.586244in}{4.376330in}}{\pgfqpoint{4.594058in}{4.384144in}}%
\pgfpathcurveto{\pgfqpoint{4.601872in}{4.391957in}}{\pgfqpoint{4.606262in}{4.402556in}}{\pgfqpoint{4.606262in}{4.413606in}}%
\pgfpathcurveto{\pgfqpoint{4.606262in}{4.424657in}}{\pgfqpoint{4.601872in}{4.435256in}}{\pgfqpoint{4.594058in}{4.443069in}}%
\pgfpathcurveto{\pgfqpoint{4.586244in}{4.450883in}}{\pgfqpoint{4.575645in}{4.455273in}}{\pgfqpoint{4.564595in}{4.455273in}}%
\pgfpathcurveto{\pgfqpoint{4.553545in}{4.455273in}}{\pgfqpoint{4.542946in}{4.450883in}}{\pgfqpoint{4.535132in}{4.443069in}}%
\pgfpathcurveto{\pgfqpoint{4.527319in}{4.435256in}}{\pgfqpoint{4.522928in}{4.424657in}}{\pgfqpoint{4.522928in}{4.413606in}}%
\pgfpathcurveto{\pgfqpoint{4.522928in}{4.402556in}}{\pgfqpoint{4.527319in}{4.391957in}}{\pgfqpoint{4.535132in}{4.384144in}}%
\pgfpathcurveto{\pgfqpoint{4.542946in}{4.376330in}}{\pgfqpoint{4.553545in}{4.371940in}}{\pgfqpoint{4.564595in}{4.371940in}}%
\pgfpathclose%
\pgfusepath{stroke,fill}%
\end{pgfscope}%
\begin{pgfscope}%
\pgfpathrectangle{\pgfqpoint{0.570343in}{0.331635in}}{\pgfqpoint{9.300000in}{7.700000in}}%
\pgfusepath{clip}%
\pgfsetbuttcap%
\pgfsetroundjoin%
\definecolor{currentfill}{rgb}{0.631373,0.788235,0.956863}%
\pgfsetfillcolor{currentfill}%
\pgfsetlinewidth{0.481800pt}%
\definecolor{currentstroke}{rgb}{1.000000,1.000000,1.000000}%
\pgfsetstrokecolor{currentstroke}%
\pgfsetdash{}{0pt}%
\pgfpathmoveto{\pgfqpoint{5.991395in}{5.235590in}}%
\pgfpathcurveto{\pgfqpoint{6.002446in}{5.235590in}}{\pgfqpoint{6.013045in}{5.239980in}}{\pgfqpoint{6.020858in}{5.247794in}}%
\pgfpathcurveto{\pgfqpoint{6.028672in}{5.255608in}}{\pgfqpoint{6.033062in}{5.266207in}}{\pgfqpoint{6.033062in}{5.277257in}}%
\pgfpathcurveto{\pgfqpoint{6.033062in}{5.288307in}}{\pgfqpoint{6.028672in}{5.298906in}}{\pgfqpoint{6.020858in}{5.306720in}}%
\pgfpathcurveto{\pgfqpoint{6.013045in}{5.314533in}}{\pgfqpoint{6.002446in}{5.318924in}}{\pgfqpoint{5.991395in}{5.318924in}}%
\pgfpathcurveto{\pgfqpoint{5.980345in}{5.318924in}}{\pgfqpoint{5.969746in}{5.314533in}}{\pgfqpoint{5.961933in}{5.306720in}}%
\pgfpathcurveto{\pgfqpoint{5.954119in}{5.298906in}}{\pgfqpoint{5.949729in}{5.288307in}}{\pgfqpoint{5.949729in}{5.277257in}}%
\pgfpathcurveto{\pgfqpoint{5.949729in}{5.266207in}}{\pgfqpoint{5.954119in}{5.255608in}}{\pgfqpoint{5.961933in}{5.247794in}}%
\pgfpathcurveto{\pgfqpoint{5.969746in}{5.239980in}}{\pgfqpoint{5.980345in}{5.235590in}}{\pgfqpoint{5.991395in}{5.235590in}}%
\pgfpathclose%
\pgfusepath{stroke,fill}%
\end{pgfscope}%
\begin{pgfscope}%
\pgfpathrectangle{\pgfqpoint{0.570343in}{0.331635in}}{\pgfqpoint{9.300000in}{7.700000in}}%
\pgfusepath{clip}%
\pgfsetbuttcap%
\pgfsetroundjoin%
\definecolor{currentfill}{rgb}{0.631373,0.788235,0.956863}%
\pgfsetfillcolor{currentfill}%
\pgfsetlinewidth{0.481800pt}%
\definecolor{currentstroke}{rgb}{1.000000,1.000000,1.000000}%
\pgfsetstrokecolor{currentstroke}%
\pgfsetdash{}{0pt}%
\pgfpathmoveto{\pgfqpoint{3.383575in}{7.102117in}}%
\pgfpathcurveto{\pgfqpoint{3.394625in}{7.102117in}}{\pgfqpoint{3.405224in}{7.106507in}}{\pgfqpoint{3.413038in}{7.114321in}}%
\pgfpathcurveto{\pgfqpoint{3.420851in}{7.122134in}}{\pgfqpoint{3.425241in}{7.132733in}}{\pgfqpoint{3.425241in}{7.143783in}}%
\pgfpathcurveto{\pgfqpoint{3.425241in}{7.154834in}}{\pgfqpoint{3.420851in}{7.165433in}}{\pgfqpoint{3.413038in}{7.173246in}}%
\pgfpathcurveto{\pgfqpoint{3.405224in}{7.181060in}}{\pgfqpoint{3.394625in}{7.185450in}}{\pgfqpoint{3.383575in}{7.185450in}}%
\pgfpathcurveto{\pgfqpoint{3.372525in}{7.185450in}}{\pgfqpoint{3.361926in}{7.181060in}}{\pgfqpoint{3.354112in}{7.173246in}}%
\pgfpathcurveto{\pgfqpoint{3.346298in}{7.165433in}}{\pgfqpoint{3.341908in}{7.154834in}}{\pgfqpoint{3.341908in}{7.143783in}}%
\pgfpathcurveto{\pgfqpoint{3.341908in}{7.132733in}}{\pgfqpoint{3.346298in}{7.122134in}}{\pgfqpoint{3.354112in}{7.114321in}}%
\pgfpathcurveto{\pgfqpoint{3.361926in}{7.106507in}}{\pgfqpoint{3.372525in}{7.102117in}}{\pgfqpoint{3.383575in}{7.102117in}}%
\pgfpathclose%
\pgfusepath{stroke,fill}%
\end{pgfscope}%
\begin{pgfscope}%
\pgfpathrectangle{\pgfqpoint{0.570343in}{0.331635in}}{\pgfqpoint{9.300000in}{7.700000in}}%
\pgfusepath{clip}%
\pgfsetbuttcap%
\pgfsetroundjoin%
\definecolor{currentfill}{rgb}{0.631373,0.788235,0.956863}%
\pgfsetfillcolor{currentfill}%
\pgfsetlinewidth{0.481800pt}%
\definecolor{currentstroke}{rgb}{1.000000,1.000000,1.000000}%
\pgfsetstrokecolor{currentstroke}%
\pgfsetdash{}{0pt}%
\pgfpathmoveto{\pgfqpoint{3.614236in}{4.220995in}}%
\pgfpathcurveto{\pgfqpoint{3.625286in}{4.220995in}}{\pgfqpoint{3.635885in}{4.225385in}}{\pgfqpoint{3.643698in}{4.233199in}}%
\pgfpathcurveto{\pgfqpoint{3.651512in}{4.241013in}}{\pgfqpoint{3.655902in}{4.251612in}}{\pgfqpoint{3.655902in}{4.262662in}}%
\pgfpathcurveto{\pgfqpoint{3.655902in}{4.273712in}}{\pgfqpoint{3.651512in}{4.284311in}}{\pgfqpoint{3.643698in}{4.292125in}}%
\pgfpathcurveto{\pgfqpoint{3.635885in}{4.299938in}}{\pgfqpoint{3.625286in}{4.304328in}}{\pgfqpoint{3.614236in}{4.304328in}}%
\pgfpathcurveto{\pgfqpoint{3.603185in}{4.304328in}}{\pgfqpoint{3.592586in}{4.299938in}}{\pgfqpoint{3.584773in}{4.292125in}}%
\pgfpathcurveto{\pgfqpoint{3.576959in}{4.284311in}}{\pgfqpoint{3.572569in}{4.273712in}}{\pgfqpoint{3.572569in}{4.262662in}}%
\pgfpathcurveto{\pgfqpoint{3.572569in}{4.251612in}}{\pgfqpoint{3.576959in}{4.241013in}}{\pgfqpoint{3.584773in}{4.233199in}}%
\pgfpathcurveto{\pgfqpoint{3.592586in}{4.225385in}}{\pgfqpoint{3.603185in}{4.220995in}}{\pgfqpoint{3.614236in}{4.220995in}}%
\pgfpathclose%
\pgfusepath{stroke,fill}%
\end{pgfscope}%
\begin{pgfscope}%
\pgfpathrectangle{\pgfqpoint{0.570343in}{0.331635in}}{\pgfqpoint{9.300000in}{7.700000in}}%
\pgfusepath{clip}%
\pgfsetbuttcap%
\pgfsetroundjoin%
\definecolor{currentfill}{rgb}{0.631373,0.788235,0.956863}%
\pgfsetfillcolor{currentfill}%
\pgfsetlinewidth{0.481800pt}%
\definecolor{currentstroke}{rgb}{1.000000,1.000000,1.000000}%
\pgfsetstrokecolor{currentstroke}%
\pgfsetdash{}{0pt}%
\pgfpathmoveto{\pgfqpoint{8.088914in}{1.897247in}}%
\pgfpathcurveto{\pgfqpoint{8.099964in}{1.897247in}}{\pgfqpoint{8.110563in}{1.901637in}}{\pgfqpoint{8.118377in}{1.909451in}}%
\pgfpathcurveto{\pgfqpoint{8.126190in}{1.917265in}}{\pgfqpoint{8.130581in}{1.927864in}}{\pgfqpoint{8.130581in}{1.938914in}}%
\pgfpathcurveto{\pgfqpoint{8.130581in}{1.949964in}}{\pgfqpoint{8.126190in}{1.960563in}}{\pgfqpoint{8.118377in}{1.968377in}}%
\pgfpathcurveto{\pgfqpoint{8.110563in}{1.976190in}}{\pgfqpoint{8.099964in}{1.980580in}}{\pgfqpoint{8.088914in}{1.980580in}}%
\pgfpathcurveto{\pgfqpoint{8.077864in}{1.980580in}}{\pgfqpoint{8.067265in}{1.976190in}}{\pgfqpoint{8.059451in}{1.968377in}}%
\pgfpathcurveto{\pgfqpoint{8.051637in}{1.960563in}}{\pgfqpoint{8.047247in}{1.949964in}}{\pgfqpoint{8.047247in}{1.938914in}}%
\pgfpathcurveto{\pgfqpoint{8.047247in}{1.927864in}}{\pgfqpoint{8.051637in}{1.917265in}}{\pgfqpoint{8.059451in}{1.909451in}}%
\pgfpathcurveto{\pgfqpoint{8.067265in}{1.901637in}}{\pgfqpoint{8.077864in}{1.897247in}}{\pgfqpoint{8.088914in}{1.897247in}}%
\pgfpathclose%
\pgfusepath{stroke,fill}%
\end{pgfscope}%
\begin{pgfscope}%
\pgfpathrectangle{\pgfqpoint{0.570343in}{0.331635in}}{\pgfqpoint{9.300000in}{7.700000in}}%
\pgfusepath{clip}%
\pgfsetbuttcap%
\pgfsetroundjoin%
\definecolor{currentfill}{rgb}{0.631373,0.788235,0.956863}%
\pgfsetfillcolor{currentfill}%
\pgfsetlinewidth{0.481800pt}%
\definecolor{currentstroke}{rgb}{1.000000,1.000000,1.000000}%
\pgfsetstrokecolor{currentstroke}%
\pgfsetdash{}{0pt}%
\pgfpathmoveto{\pgfqpoint{6.893373in}{7.586771in}}%
\pgfpathcurveto{\pgfqpoint{6.904424in}{7.586771in}}{\pgfqpoint{6.915023in}{7.591161in}}{\pgfqpoint{6.922836in}{7.598975in}}%
\pgfpathcurveto{\pgfqpoint{6.930650in}{7.606788in}}{\pgfqpoint{6.935040in}{7.617387in}}{\pgfqpoint{6.935040in}{7.628437in}}%
\pgfpathcurveto{\pgfqpoint{6.935040in}{7.639487in}}{\pgfqpoint{6.930650in}{7.650086in}}{\pgfqpoint{6.922836in}{7.657900in}}%
\pgfpathcurveto{\pgfqpoint{6.915023in}{7.665714in}}{\pgfqpoint{6.904424in}{7.670104in}}{\pgfqpoint{6.893373in}{7.670104in}}%
\pgfpathcurveto{\pgfqpoint{6.882323in}{7.670104in}}{\pgfqpoint{6.871724in}{7.665714in}}{\pgfqpoint{6.863911in}{7.657900in}}%
\pgfpathcurveto{\pgfqpoint{6.856097in}{7.650086in}}{\pgfqpoint{6.851707in}{7.639487in}}{\pgfqpoint{6.851707in}{7.628437in}}%
\pgfpathcurveto{\pgfqpoint{6.851707in}{7.617387in}}{\pgfqpoint{6.856097in}{7.606788in}}{\pgfqpoint{6.863911in}{7.598975in}}%
\pgfpathcurveto{\pgfqpoint{6.871724in}{7.591161in}}{\pgfqpoint{6.882323in}{7.586771in}}{\pgfqpoint{6.893373in}{7.586771in}}%
\pgfpathclose%
\pgfusepath{stroke,fill}%
\end{pgfscope}%
\begin{pgfscope}%
\pgfpathrectangle{\pgfqpoint{0.570343in}{0.331635in}}{\pgfqpoint{9.300000in}{7.700000in}}%
\pgfusepath{clip}%
\pgfsetbuttcap%
\pgfsetroundjoin%
\definecolor{currentfill}{rgb}{0.631373,0.788235,0.956863}%
\pgfsetfillcolor{currentfill}%
\pgfsetlinewidth{0.481800pt}%
\definecolor{currentstroke}{rgb}{1.000000,1.000000,1.000000}%
\pgfsetstrokecolor{currentstroke}%
\pgfsetdash{}{0pt}%
\pgfpathmoveto{\pgfqpoint{8.123953in}{2.895214in}}%
\pgfpathcurveto{\pgfqpoint{8.135003in}{2.895214in}}{\pgfqpoint{8.145602in}{2.899604in}}{\pgfqpoint{8.153416in}{2.907418in}}%
\pgfpathcurveto{\pgfqpoint{8.161229in}{2.915231in}}{\pgfqpoint{8.165620in}{2.925830in}}{\pgfqpoint{8.165620in}{2.936881in}}%
\pgfpathcurveto{\pgfqpoint{8.165620in}{2.947931in}}{\pgfqpoint{8.161229in}{2.958530in}}{\pgfqpoint{8.153416in}{2.966343in}}%
\pgfpathcurveto{\pgfqpoint{8.145602in}{2.974157in}}{\pgfqpoint{8.135003in}{2.978547in}}{\pgfqpoint{8.123953in}{2.978547in}}%
\pgfpathcurveto{\pgfqpoint{8.112903in}{2.978547in}}{\pgfqpoint{8.102304in}{2.974157in}}{\pgfqpoint{8.094490in}{2.966343in}}%
\pgfpathcurveto{\pgfqpoint{8.086676in}{2.958530in}}{\pgfqpoint{8.082286in}{2.947931in}}{\pgfqpoint{8.082286in}{2.936881in}}%
\pgfpathcurveto{\pgfqpoint{8.082286in}{2.925830in}}{\pgfqpoint{8.086676in}{2.915231in}}{\pgfqpoint{8.094490in}{2.907418in}}%
\pgfpathcurveto{\pgfqpoint{8.102304in}{2.899604in}}{\pgfqpoint{8.112903in}{2.895214in}}{\pgfqpoint{8.123953in}{2.895214in}}%
\pgfpathclose%
\pgfusepath{stroke,fill}%
\end{pgfscope}%
\begin{pgfscope}%
\pgfpathrectangle{\pgfqpoint{0.570343in}{0.331635in}}{\pgfqpoint{9.300000in}{7.700000in}}%
\pgfusepath{clip}%
\pgfsetbuttcap%
\pgfsetroundjoin%
\definecolor{currentfill}{rgb}{0.631373,0.788235,0.956863}%
\pgfsetfillcolor{currentfill}%
\pgfsetlinewidth{0.481800pt}%
\definecolor{currentstroke}{rgb}{1.000000,1.000000,1.000000}%
\pgfsetstrokecolor{currentstroke}%
\pgfsetdash{}{0pt}%
\pgfpathmoveto{\pgfqpoint{6.741009in}{5.773028in}}%
\pgfpathcurveto{\pgfqpoint{6.752059in}{5.773028in}}{\pgfqpoint{6.762658in}{5.777418in}}{\pgfqpoint{6.770472in}{5.785231in}}%
\pgfpathcurveto{\pgfqpoint{6.778285in}{5.793045in}}{\pgfqpoint{6.782675in}{5.803644in}}{\pgfqpoint{6.782675in}{5.814694in}}%
\pgfpathcurveto{\pgfqpoint{6.782675in}{5.825744in}}{\pgfqpoint{6.778285in}{5.836343in}}{\pgfqpoint{6.770472in}{5.844157in}}%
\pgfpathcurveto{\pgfqpoint{6.762658in}{5.851971in}}{\pgfqpoint{6.752059in}{5.856361in}}{\pgfqpoint{6.741009in}{5.856361in}}%
\pgfpathcurveto{\pgfqpoint{6.729959in}{5.856361in}}{\pgfqpoint{6.719360in}{5.851971in}}{\pgfqpoint{6.711546in}{5.844157in}}%
\pgfpathcurveto{\pgfqpoint{6.703732in}{5.836343in}}{\pgfqpoint{6.699342in}{5.825744in}}{\pgfqpoint{6.699342in}{5.814694in}}%
\pgfpathcurveto{\pgfqpoint{6.699342in}{5.803644in}}{\pgfqpoint{6.703732in}{5.793045in}}{\pgfqpoint{6.711546in}{5.785231in}}%
\pgfpathcurveto{\pgfqpoint{6.719360in}{5.777418in}}{\pgfqpoint{6.729959in}{5.773028in}}{\pgfqpoint{6.741009in}{5.773028in}}%
\pgfpathclose%
\pgfusepath{stroke,fill}%
\end{pgfscope}%
\begin{pgfscope}%
\pgfpathrectangle{\pgfqpoint{0.570343in}{0.331635in}}{\pgfqpoint{9.300000in}{7.700000in}}%
\pgfusepath{clip}%
\pgfsetbuttcap%
\pgfsetroundjoin%
\definecolor{currentfill}{rgb}{0.631373,0.788235,0.956863}%
\pgfsetfillcolor{currentfill}%
\pgfsetlinewidth{0.481800pt}%
\definecolor{currentstroke}{rgb}{1.000000,1.000000,1.000000}%
\pgfsetstrokecolor{currentstroke}%
\pgfsetdash{}{0pt}%
\pgfpathmoveto{\pgfqpoint{4.238932in}{6.370567in}}%
\pgfpathcurveto{\pgfqpoint{4.249983in}{6.370567in}}{\pgfqpoint{4.260582in}{6.374957in}}{\pgfqpoint{4.268395in}{6.382771in}}%
\pgfpathcurveto{\pgfqpoint{4.276209in}{6.390585in}}{\pgfqpoint{4.280599in}{6.401184in}}{\pgfqpoint{4.280599in}{6.412234in}}%
\pgfpathcurveto{\pgfqpoint{4.280599in}{6.423284in}}{\pgfqpoint{4.276209in}{6.433883in}}{\pgfqpoint{4.268395in}{6.441697in}}%
\pgfpathcurveto{\pgfqpoint{4.260582in}{6.449510in}}{\pgfqpoint{4.249983in}{6.453901in}}{\pgfqpoint{4.238932in}{6.453901in}}%
\pgfpathcurveto{\pgfqpoint{4.227882in}{6.453901in}}{\pgfqpoint{4.217283in}{6.449510in}}{\pgfqpoint{4.209470in}{6.441697in}}%
\pgfpathcurveto{\pgfqpoint{4.201656in}{6.433883in}}{\pgfqpoint{4.197266in}{6.423284in}}{\pgfqpoint{4.197266in}{6.412234in}}%
\pgfpathcurveto{\pgfqpoint{4.197266in}{6.401184in}}{\pgfqpoint{4.201656in}{6.390585in}}{\pgfqpoint{4.209470in}{6.382771in}}%
\pgfpathcurveto{\pgfqpoint{4.217283in}{6.374957in}}{\pgfqpoint{4.227882in}{6.370567in}}{\pgfqpoint{4.238932in}{6.370567in}}%
\pgfpathclose%
\pgfusepath{stroke,fill}%
\end{pgfscope}%
\begin{pgfscope}%
\pgfpathrectangle{\pgfqpoint{0.570343in}{0.331635in}}{\pgfqpoint{9.300000in}{7.700000in}}%
\pgfusepath{clip}%
\pgfsetbuttcap%
\pgfsetroundjoin%
\definecolor{currentfill}{rgb}{0.631373,0.788235,0.956863}%
\pgfsetfillcolor{currentfill}%
\pgfsetlinewidth{0.481800pt}%
\definecolor{currentstroke}{rgb}{1.000000,1.000000,1.000000}%
\pgfsetstrokecolor{currentstroke}%
\pgfsetdash{}{0pt}%
\pgfpathmoveto{\pgfqpoint{3.437634in}{3.457680in}}%
\pgfpathcurveto{\pgfqpoint{3.448684in}{3.457680in}}{\pgfqpoint{3.459283in}{3.462070in}}{\pgfqpoint{3.467097in}{3.469884in}}%
\pgfpathcurveto{\pgfqpoint{3.474911in}{3.477697in}}{\pgfqpoint{3.479301in}{3.488296in}}{\pgfqpoint{3.479301in}{3.499346in}}%
\pgfpathcurveto{\pgfqpoint{3.479301in}{3.510396in}}{\pgfqpoint{3.474911in}{3.520996in}}{\pgfqpoint{3.467097in}{3.528809in}}%
\pgfpathcurveto{\pgfqpoint{3.459283in}{3.536623in}}{\pgfqpoint{3.448684in}{3.541013in}}{\pgfqpoint{3.437634in}{3.541013in}}%
\pgfpathcurveto{\pgfqpoint{3.426584in}{3.541013in}}{\pgfqpoint{3.415985in}{3.536623in}}{\pgfqpoint{3.408171in}{3.528809in}}%
\pgfpathcurveto{\pgfqpoint{3.400358in}{3.520996in}}{\pgfqpoint{3.395968in}{3.510396in}}{\pgfqpoint{3.395968in}{3.499346in}}%
\pgfpathcurveto{\pgfqpoint{3.395968in}{3.488296in}}{\pgfqpoint{3.400358in}{3.477697in}}{\pgfqpoint{3.408171in}{3.469884in}}%
\pgfpathcurveto{\pgfqpoint{3.415985in}{3.462070in}}{\pgfqpoint{3.426584in}{3.457680in}}{\pgfqpoint{3.437634in}{3.457680in}}%
\pgfpathclose%
\pgfusepath{stroke,fill}%
\end{pgfscope}%
\begin{pgfscope}%
\pgfpathrectangle{\pgfqpoint{0.570343in}{0.331635in}}{\pgfqpoint{9.300000in}{7.700000in}}%
\pgfusepath{clip}%
\pgfsetbuttcap%
\pgfsetroundjoin%
\definecolor{currentfill}{rgb}{0.631373,0.788235,0.956863}%
\pgfsetfillcolor{currentfill}%
\pgfsetlinewidth{0.481800pt}%
\definecolor{currentstroke}{rgb}{1.000000,1.000000,1.000000}%
\pgfsetstrokecolor{currentstroke}%
\pgfsetdash{}{0pt}%
\pgfpathmoveto{\pgfqpoint{1.464475in}{6.181012in}}%
\pgfpathcurveto{\pgfqpoint{1.475525in}{6.181012in}}{\pgfqpoint{1.486125in}{6.185403in}}{\pgfqpoint{1.493938in}{6.193216in}}%
\pgfpathcurveto{\pgfqpoint{1.501752in}{6.201030in}}{\pgfqpoint{1.506142in}{6.211629in}}{\pgfqpoint{1.506142in}{6.222679in}}%
\pgfpathcurveto{\pgfqpoint{1.506142in}{6.233729in}}{\pgfqpoint{1.501752in}{6.244328in}}{\pgfqpoint{1.493938in}{6.252142in}}%
\pgfpathcurveto{\pgfqpoint{1.486125in}{6.259955in}}{\pgfqpoint{1.475525in}{6.264346in}}{\pgfqpoint{1.464475in}{6.264346in}}%
\pgfpathcurveto{\pgfqpoint{1.453425in}{6.264346in}}{\pgfqpoint{1.442826in}{6.259955in}}{\pgfqpoint{1.435013in}{6.252142in}}%
\pgfpathcurveto{\pgfqpoint{1.427199in}{6.244328in}}{\pgfqpoint{1.422809in}{6.233729in}}{\pgfqpoint{1.422809in}{6.222679in}}%
\pgfpathcurveto{\pgfqpoint{1.422809in}{6.211629in}}{\pgfqpoint{1.427199in}{6.201030in}}{\pgfqpoint{1.435013in}{6.193216in}}%
\pgfpathcurveto{\pgfqpoint{1.442826in}{6.185403in}}{\pgfqpoint{1.453425in}{6.181012in}}{\pgfqpoint{1.464475in}{6.181012in}}%
\pgfpathclose%
\pgfusepath{stroke,fill}%
\end{pgfscope}%
\begin{pgfscope}%
\pgfpathrectangle{\pgfqpoint{0.570343in}{0.331635in}}{\pgfqpoint{9.300000in}{7.700000in}}%
\pgfusepath{clip}%
\pgfsetbuttcap%
\pgfsetroundjoin%
\definecolor{currentfill}{rgb}{0.631373,0.788235,0.956863}%
\pgfsetfillcolor{currentfill}%
\pgfsetlinewidth{0.481800pt}%
\definecolor{currentstroke}{rgb}{1.000000,1.000000,1.000000}%
\pgfsetstrokecolor{currentstroke}%
\pgfsetdash{}{0pt}%
\pgfpathmoveto{\pgfqpoint{0.993071in}{4.919062in}}%
\pgfpathcurveto{\pgfqpoint{1.004121in}{4.919062in}}{\pgfqpoint{1.014720in}{4.923452in}}{\pgfqpoint{1.022533in}{4.931266in}}%
\pgfpathcurveto{\pgfqpoint{1.030347in}{4.939080in}}{\pgfqpoint{1.034737in}{4.949679in}}{\pgfqpoint{1.034737in}{4.960729in}}%
\pgfpathcurveto{\pgfqpoint{1.034737in}{4.971779in}}{\pgfqpoint{1.030347in}{4.982378in}}{\pgfqpoint{1.022533in}{4.990192in}}%
\pgfpathcurveto{\pgfqpoint{1.014720in}{4.998005in}}{\pgfqpoint{1.004121in}{5.002395in}}{\pgfqpoint{0.993071in}{5.002395in}}%
\pgfpathcurveto{\pgfqpoint{0.982020in}{5.002395in}}{\pgfqpoint{0.971421in}{4.998005in}}{\pgfqpoint{0.963608in}{4.990192in}}%
\pgfpathcurveto{\pgfqpoint{0.955794in}{4.982378in}}{\pgfqpoint{0.951404in}{4.971779in}}{\pgfqpoint{0.951404in}{4.960729in}}%
\pgfpathcurveto{\pgfqpoint{0.951404in}{4.949679in}}{\pgfqpoint{0.955794in}{4.939080in}}{\pgfqpoint{0.963608in}{4.931266in}}%
\pgfpathcurveto{\pgfqpoint{0.971421in}{4.923452in}}{\pgfqpoint{0.982020in}{4.919062in}}{\pgfqpoint{0.993071in}{4.919062in}}%
\pgfpathclose%
\pgfusepath{stroke,fill}%
\end{pgfscope}%
\begin{pgfscope}%
\pgfpathrectangle{\pgfqpoint{0.570343in}{0.331635in}}{\pgfqpoint{9.300000in}{7.700000in}}%
\pgfusepath{clip}%
\pgfsetbuttcap%
\pgfsetroundjoin%
\definecolor{currentfill}{rgb}{0.631373,0.788235,0.956863}%
\pgfsetfillcolor{currentfill}%
\pgfsetlinewidth{0.481800pt}%
\definecolor{currentstroke}{rgb}{1.000000,1.000000,1.000000}%
\pgfsetstrokecolor{currentstroke}%
\pgfsetdash{}{0pt}%
\pgfpathmoveto{\pgfqpoint{5.578651in}{5.921356in}}%
\pgfpathcurveto{\pgfqpoint{5.589701in}{5.921356in}}{\pgfqpoint{5.600300in}{5.925746in}}{\pgfqpoint{5.608114in}{5.933560in}}%
\pgfpathcurveto{\pgfqpoint{5.615927in}{5.941374in}}{\pgfqpoint{5.620318in}{5.951973in}}{\pgfqpoint{5.620318in}{5.963023in}}%
\pgfpathcurveto{\pgfqpoint{5.620318in}{5.974073in}}{\pgfqpoint{5.615927in}{5.984672in}}{\pgfqpoint{5.608114in}{5.992485in}}%
\pgfpathcurveto{\pgfqpoint{5.600300in}{6.000299in}}{\pgfqpoint{5.589701in}{6.004689in}}{\pgfqpoint{5.578651in}{6.004689in}}%
\pgfpathcurveto{\pgfqpoint{5.567601in}{6.004689in}}{\pgfqpoint{5.557002in}{6.000299in}}{\pgfqpoint{5.549188in}{5.992485in}}%
\pgfpathcurveto{\pgfqpoint{5.541375in}{5.984672in}}{\pgfqpoint{5.536984in}{5.974073in}}{\pgfqpoint{5.536984in}{5.963023in}}%
\pgfpathcurveto{\pgfqpoint{5.536984in}{5.951973in}}{\pgfqpoint{5.541375in}{5.941374in}}{\pgfqpoint{5.549188in}{5.933560in}}%
\pgfpathcurveto{\pgfqpoint{5.557002in}{5.925746in}}{\pgfqpoint{5.567601in}{5.921356in}}{\pgfqpoint{5.578651in}{5.921356in}}%
\pgfpathclose%
\pgfusepath{stroke,fill}%
\end{pgfscope}%
\begin{pgfscope}%
\pgfpathrectangle{\pgfqpoint{0.570343in}{0.331635in}}{\pgfqpoint{9.300000in}{7.700000in}}%
\pgfusepath{clip}%
\pgfsetbuttcap%
\pgfsetroundjoin%
\definecolor{currentfill}{rgb}{0.631373,0.788235,0.956863}%
\pgfsetfillcolor{currentfill}%
\pgfsetlinewidth{0.481800pt}%
\definecolor{currentstroke}{rgb}{1.000000,1.000000,1.000000}%
\pgfsetstrokecolor{currentstroke}%
\pgfsetdash{}{0pt}%
\pgfpathmoveto{\pgfqpoint{3.551278in}{5.591021in}}%
\pgfpathcurveto{\pgfqpoint{3.562328in}{5.591021in}}{\pgfqpoint{3.572927in}{5.595411in}}{\pgfqpoint{3.580741in}{5.603225in}}%
\pgfpathcurveto{\pgfqpoint{3.588554in}{5.611038in}}{\pgfqpoint{3.592945in}{5.621637in}}{\pgfqpoint{3.592945in}{5.632687in}}%
\pgfpathcurveto{\pgfqpoint{3.592945in}{5.643738in}}{\pgfqpoint{3.588554in}{5.654337in}}{\pgfqpoint{3.580741in}{5.662150in}}%
\pgfpathcurveto{\pgfqpoint{3.572927in}{5.669964in}}{\pgfqpoint{3.562328in}{5.674354in}}{\pgfqpoint{3.551278in}{5.674354in}}%
\pgfpathcurveto{\pgfqpoint{3.540228in}{5.674354in}}{\pgfqpoint{3.529629in}{5.669964in}}{\pgfqpoint{3.521815in}{5.662150in}}%
\pgfpathcurveto{\pgfqpoint{3.514002in}{5.654337in}}{\pgfqpoint{3.509611in}{5.643738in}}{\pgfqpoint{3.509611in}{5.632687in}}%
\pgfpathcurveto{\pgfqpoint{3.509611in}{5.621637in}}{\pgfqpoint{3.514002in}{5.611038in}}{\pgfqpoint{3.521815in}{5.603225in}}%
\pgfpathcurveto{\pgfqpoint{3.529629in}{5.595411in}}{\pgfqpoint{3.540228in}{5.591021in}}{\pgfqpoint{3.551278in}{5.591021in}}%
\pgfpathclose%
\pgfusepath{stroke,fill}%
\end{pgfscope}%
\begin{pgfscope}%
\pgfpathrectangle{\pgfqpoint{0.570343in}{0.331635in}}{\pgfqpoint{9.300000in}{7.700000in}}%
\pgfusepath{clip}%
\pgfsetbuttcap%
\pgfsetroundjoin%
\definecolor{currentfill}{rgb}{0.631373,0.788235,0.956863}%
\pgfsetfillcolor{currentfill}%
\pgfsetlinewidth{0.481800pt}%
\definecolor{currentstroke}{rgb}{1.000000,1.000000,1.000000}%
\pgfsetstrokecolor{currentstroke}%
\pgfsetdash{}{0pt}%
\pgfpathmoveto{\pgfqpoint{7.539850in}{4.394674in}}%
\pgfpathcurveto{\pgfqpoint{7.550900in}{4.394674in}}{\pgfqpoint{7.561499in}{4.399064in}}{\pgfqpoint{7.569313in}{4.406877in}}%
\pgfpathcurveto{\pgfqpoint{7.577126in}{4.414691in}}{\pgfqpoint{7.581517in}{4.425290in}}{\pgfqpoint{7.581517in}{4.436340in}}%
\pgfpathcurveto{\pgfqpoint{7.581517in}{4.447390in}}{\pgfqpoint{7.577126in}{4.457989in}}{\pgfqpoint{7.569313in}{4.465803in}}%
\pgfpathcurveto{\pgfqpoint{7.561499in}{4.473617in}}{\pgfqpoint{7.550900in}{4.478007in}}{\pgfqpoint{7.539850in}{4.478007in}}%
\pgfpathcurveto{\pgfqpoint{7.528800in}{4.478007in}}{\pgfqpoint{7.518201in}{4.473617in}}{\pgfqpoint{7.510387in}{4.465803in}}%
\pgfpathcurveto{\pgfqpoint{7.502574in}{4.457989in}}{\pgfqpoint{7.498183in}{4.447390in}}{\pgfqpoint{7.498183in}{4.436340in}}%
\pgfpathcurveto{\pgfqpoint{7.498183in}{4.425290in}}{\pgfqpoint{7.502574in}{4.414691in}}{\pgfqpoint{7.510387in}{4.406877in}}%
\pgfpathcurveto{\pgfqpoint{7.518201in}{4.399064in}}{\pgfqpoint{7.528800in}{4.394674in}}{\pgfqpoint{7.539850in}{4.394674in}}%
\pgfpathclose%
\pgfusepath{stroke,fill}%
\end{pgfscope}%
\begin{pgfscope}%
\pgfpathrectangle{\pgfqpoint{0.570343in}{0.331635in}}{\pgfqpoint{9.300000in}{7.700000in}}%
\pgfusepath{clip}%
\pgfsetbuttcap%
\pgfsetroundjoin%
\definecolor{currentfill}{rgb}{0.631373,0.788235,0.956863}%
\pgfsetfillcolor{currentfill}%
\pgfsetlinewidth{0.481800pt}%
\definecolor{currentstroke}{rgb}{1.000000,1.000000,1.000000}%
\pgfsetstrokecolor{currentstroke}%
\pgfsetdash{}{0pt}%
\pgfpathmoveto{\pgfqpoint{6.633358in}{6.377637in}}%
\pgfpathcurveto{\pgfqpoint{6.644408in}{6.377637in}}{\pgfqpoint{6.655007in}{6.382027in}}{\pgfqpoint{6.662820in}{6.389841in}}%
\pgfpathcurveto{\pgfqpoint{6.670634in}{6.397654in}}{\pgfqpoint{6.675024in}{6.408253in}}{\pgfqpoint{6.675024in}{6.419304in}}%
\pgfpathcurveto{\pgfqpoint{6.675024in}{6.430354in}}{\pgfqpoint{6.670634in}{6.440953in}}{\pgfqpoint{6.662820in}{6.448766in}}%
\pgfpathcurveto{\pgfqpoint{6.655007in}{6.456580in}}{\pgfqpoint{6.644408in}{6.460970in}}{\pgfqpoint{6.633358in}{6.460970in}}%
\pgfpathcurveto{\pgfqpoint{6.622307in}{6.460970in}}{\pgfqpoint{6.611708in}{6.456580in}}{\pgfqpoint{6.603895in}{6.448766in}}%
\pgfpathcurveto{\pgfqpoint{6.596081in}{6.440953in}}{\pgfqpoint{6.591691in}{6.430354in}}{\pgfqpoint{6.591691in}{6.419304in}}%
\pgfpathcurveto{\pgfqpoint{6.591691in}{6.408253in}}{\pgfqpoint{6.596081in}{6.397654in}}{\pgfqpoint{6.603895in}{6.389841in}}%
\pgfpathcurveto{\pgfqpoint{6.611708in}{6.382027in}}{\pgfqpoint{6.622307in}{6.377637in}}{\pgfqpoint{6.633358in}{6.377637in}}%
\pgfpathclose%
\pgfusepath{stroke,fill}%
\end{pgfscope}%
\begin{pgfscope}%
\pgfpathrectangle{\pgfqpoint{0.570343in}{0.331635in}}{\pgfqpoint{9.300000in}{7.700000in}}%
\pgfusepath{clip}%
\pgfsetbuttcap%
\pgfsetroundjoin%
\definecolor{currentfill}{rgb}{0.631373,0.788235,0.956863}%
\pgfsetfillcolor{currentfill}%
\pgfsetlinewidth{0.481800pt}%
\definecolor{currentstroke}{rgb}{1.000000,1.000000,1.000000}%
\pgfsetstrokecolor{currentstroke}%
\pgfsetdash{}{0pt}%
\pgfpathmoveto{\pgfqpoint{9.092676in}{7.043292in}}%
\pgfpathcurveto{\pgfqpoint{9.103727in}{7.043292in}}{\pgfqpoint{9.114326in}{7.047682in}}{\pgfqpoint{9.122139in}{7.055496in}}%
\pgfpathcurveto{\pgfqpoint{9.129953in}{7.063310in}}{\pgfqpoint{9.134343in}{7.073909in}}{\pgfqpoint{9.134343in}{7.084959in}}%
\pgfpathcurveto{\pgfqpoint{9.134343in}{7.096009in}}{\pgfqpoint{9.129953in}{7.106608in}}{\pgfqpoint{9.122139in}{7.114422in}}%
\pgfpathcurveto{\pgfqpoint{9.114326in}{7.122235in}}{\pgfqpoint{9.103727in}{7.126625in}}{\pgfqpoint{9.092676in}{7.126625in}}%
\pgfpathcurveto{\pgfqpoint{9.081626in}{7.126625in}}{\pgfqpoint{9.071027in}{7.122235in}}{\pgfqpoint{9.063214in}{7.114422in}}%
\pgfpathcurveto{\pgfqpoint{9.055400in}{7.106608in}}{\pgfqpoint{9.051010in}{7.096009in}}{\pgfqpoint{9.051010in}{7.084959in}}%
\pgfpathcurveto{\pgfqpoint{9.051010in}{7.073909in}}{\pgfqpoint{9.055400in}{7.063310in}}{\pgfqpoint{9.063214in}{7.055496in}}%
\pgfpathcurveto{\pgfqpoint{9.071027in}{7.047682in}}{\pgfqpoint{9.081626in}{7.043292in}}{\pgfqpoint{9.092676in}{7.043292in}}%
\pgfpathclose%
\pgfusepath{stroke,fill}%
\end{pgfscope}%
\begin{pgfscope}%
\pgfpathrectangle{\pgfqpoint{0.570343in}{0.331635in}}{\pgfqpoint{9.300000in}{7.700000in}}%
\pgfusepath{clip}%
\pgfsetbuttcap%
\pgfsetroundjoin%
\definecolor{currentfill}{rgb}{0.631373,0.788235,0.956863}%
\pgfsetfillcolor{currentfill}%
\pgfsetlinewidth{0.481800pt}%
\definecolor{currentstroke}{rgb}{1.000000,1.000000,1.000000}%
\pgfsetstrokecolor{currentstroke}%
\pgfsetdash{}{0pt}%
\pgfpathmoveto{\pgfqpoint{4.705737in}{5.569235in}}%
\pgfpathcurveto{\pgfqpoint{4.716787in}{5.569235in}}{\pgfqpoint{4.727386in}{5.573625in}}{\pgfqpoint{4.735200in}{5.581439in}}%
\pgfpathcurveto{\pgfqpoint{4.743013in}{5.589253in}}{\pgfqpoint{4.747404in}{5.599852in}}{\pgfqpoint{4.747404in}{5.610902in}}%
\pgfpathcurveto{\pgfqpoint{4.747404in}{5.621952in}}{\pgfqpoint{4.743013in}{5.632551in}}{\pgfqpoint{4.735200in}{5.640365in}}%
\pgfpathcurveto{\pgfqpoint{4.727386in}{5.648178in}}{\pgfqpoint{4.716787in}{5.652569in}}{\pgfqpoint{4.705737in}{5.652569in}}%
\pgfpathcurveto{\pgfqpoint{4.694687in}{5.652569in}}{\pgfqpoint{4.684088in}{5.648178in}}{\pgfqpoint{4.676274in}{5.640365in}}%
\pgfpathcurveto{\pgfqpoint{4.668461in}{5.632551in}}{\pgfqpoint{4.664070in}{5.621952in}}{\pgfqpoint{4.664070in}{5.610902in}}%
\pgfpathcurveto{\pgfqpoint{4.664070in}{5.599852in}}{\pgfqpoint{4.668461in}{5.589253in}}{\pgfqpoint{4.676274in}{5.581439in}}%
\pgfpathcurveto{\pgfqpoint{4.684088in}{5.573625in}}{\pgfqpoint{4.694687in}{5.569235in}}{\pgfqpoint{4.705737in}{5.569235in}}%
\pgfpathclose%
\pgfusepath{stroke,fill}%
\end{pgfscope}%
\begin{pgfscope}%
\pgfpathrectangle{\pgfqpoint{0.570343in}{0.331635in}}{\pgfqpoint{9.300000in}{7.700000in}}%
\pgfusepath{clip}%
\pgfsetbuttcap%
\pgfsetroundjoin%
\definecolor{currentfill}{rgb}{0.631373,0.788235,0.956863}%
\pgfsetfillcolor{currentfill}%
\pgfsetlinewidth{0.481800pt}%
\definecolor{currentstroke}{rgb}{1.000000,1.000000,1.000000}%
\pgfsetstrokecolor{currentstroke}%
\pgfsetdash{}{0pt}%
\pgfpathmoveto{\pgfqpoint{3.116901in}{2.790322in}}%
\pgfpathcurveto{\pgfqpoint{3.127951in}{2.790322in}}{\pgfqpoint{3.138550in}{2.794712in}}{\pgfqpoint{3.146364in}{2.802525in}}%
\pgfpathcurveto{\pgfqpoint{3.154177in}{2.810339in}}{\pgfqpoint{3.158568in}{2.820938in}}{\pgfqpoint{3.158568in}{2.831988in}}%
\pgfpathcurveto{\pgfqpoint{3.158568in}{2.843038in}}{\pgfqpoint{3.154177in}{2.853637in}}{\pgfqpoint{3.146364in}{2.861451in}}%
\pgfpathcurveto{\pgfqpoint{3.138550in}{2.869265in}}{\pgfqpoint{3.127951in}{2.873655in}}{\pgfqpoint{3.116901in}{2.873655in}}%
\pgfpathcurveto{\pgfqpoint{3.105851in}{2.873655in}}{\pgfqpoint{3.095252in}{2.869265in}}{\pgfqpoint{3.087438in}{2.861451in}}%
\pgfpathcurveto{\pgfqpoint{3.079625in}{2.853637in}}{\pgfqpoint{3.075234in}{2.843038in}}{\pgfqpoint{3.075234in}{2.831988in}}%
\pgfpathcurveto{\pgfqpoint{3.075234in}{2.820938in}}{\pgfqpoint{3.079625in}{2.810339in}}{\pgfqpoint{3.087438in}{2.802525in}}%
\pgfpathcurveto{\pgfqpoint{3.095252in}{2.794712in}}{\pgfqpoint{3.105851in}{2.790322in}}{\pgfqpoint{3.116901in}{2.790322in}}%
\pgfpathclose%
\pgfusepath{stroke,fill}%
\end{pgfscope}%
\begin{pgfscope}%
\pgfpathrectangle{\pgfqpoint{0.570343in}{0.331635in}}{\pgfqpoint{9.300000in}{7.700000in}}%
\pgfusepath{clip}%
\pgfsetbuttcap%
\pgfsetroundjoin%
\definecolor{currentfill}{rgb}{0.631373,0.788235,0.956863}%
\pgfsetfillcolor{currentfill}%
\pgfsetlinewidth{0.481800pt}%
\definecolor{currentstroke}{rgb}{1.000000,1.000000,1.000000}%
\pgfsetstrokecolor{currentstroke}%
\pgfsetdash{}{0pt}%
\pgfpathmoveto{\pgfqpoint{4.125517in}{2.805137in}}%
\pgfpathcurveto{\pgfqpoint{4.136568in}{2.805137in}}{\pgfqpoint{4.147167in}{2.809527in}}{\pgfqpoint{4.154980in}{2.817340in}}%
\pgfpathcurveto{\pgfqpoint{4.162794in}{2.825154in}}{\pgfqpoint{4.167184in}{2.835753in}}{\pgfqpoint{4.167184in}{2.846803in}}%
\pgfpathcurveto{\pgfqpoint{4.167184in}{2.857853in}}{\pgfqpoint{4.162794in}{2.868452in}}{\pgfqpoint{4.154980in}{2.876266in}}%
\pgfpathcurveto{\pgfqpoint{4.147167in}{2.884080in}}{\pgfqpoint{4.136568in}{2.888470in}}{\pgfqpoint{4.125517in}{2.888470in}}%
\pgfpathcurveto{\pgfqpoint{4.114467in}{2.888470in}}{\pgfqpoint{4.103868in}{2.884080in}}{\pgfqpoint{4.096055in}{2.876266in}}%
\pgfpathcurveto{\pgfqpoint{4.088241in}{2.868452in}}{\pgfqpoint{4.083851in}{2.857853in}}{\pgfqpoint{4.083851in}{2.846803in}}%
\pgfpathcurveto{\pgfqpoint{4.083851in}{2.835753in}}{\pgfqpoint{4.088241in}{2.825154in}}{\pgfqpoint{4.096055in}{2.817340in}}%
\pgfpathcurveto{\pgfqpoint{4.103868in}{2.809527in}}{\pgfqpoint{4.114467in}{2.805137in}}{\pgfqpoint{4.125517in}{2.805137in}}%
\pgfpathclose%
\pgfusepath{stroke,fill}%
\end{pgfscope}%
\begin{pgfscope}%
\pgfpathrectangle{\pgfqpoint{0.570343in}{0.331635in}}{\pgfqpoint{9.300000in}{7.700000in}}%
\pgfusepath{clip}%
\pgfsetbuttcap%
\pgfsetroundjoin%
\definecolor{currentfill}{rgb}{0.631373,0.788235,0.956863}%
\pgfsetfillcolor{currentfill}%
\pgfsetlinewidth{0.481800pt}%
\definecolor{currentstroke}{rgb}{1.000000,1.000000,1.000000}%
\pgfsetstrokecolor{currentstroke}%
\pgfsetdash{}{0pt}%
\pgfpathmoveto{\pgfqpoint{3.950923in}{4.944942in}}%
\pgfpathcurveto{\pgfqpoint{3.961973in}{4.944942in}}{\pgfqpoint{3.972572in}{4.949332in}}{\pgfqpoint{3.980385in}{4.957146in}}%
\pgfpathcurveto{\pgfqpoint{3.988199in}{4.964959in}}{\pgfqpoint{3.992589in}{4.975558in}}{\pgfqpoint{3.992589in}{4.986608in}}%
\pgfpathcurveto{\pgfqpoint{3.992589in}{4.997659in}}{\pgfqpoint{3.988199in}{5.008258in}}{\pgfqpoint{3.980385in}{5.016071in}}%
\pgfpathcurveto{\pgfqpoint{3.972572in}{5.023885in}}{\pgfqpoint{3.961973in}{5.028275in}}{\pgfqpoint{3.950923in}{5.028275in}}%
\pgfpathcurveto{\pgfqpoint{3.939873in}{5.028275in}}{\pgfqpoint{3.929274in}{5.023885in}}{\pgfqpoint{3.921460in}{5.016071in}}%
\pgfpathcurveto{\pgfqpoint{3.913646in}{5.008258in}}{\pgfqpoint{3.909256in}{4.997659in}}{\pgfqpoint{3.909256in}{4.986608in}}%
\pgfpathcurveto{\pgfqpoint{3.909256in}{4.975558in}}{\pgfqpoint{3.913646in}{4.964959in}}{\pgfqpoint{3.921460in}{4.957146in}}%
\pgfpathcurveto{\pgfqpoint{3.929274in}{4.949332in}}{\pgfqpoint{3.939873in}{4.944942in}}{\pgfqpoint{3.950923in}{4.944942in}}%
\pgfpathclose%
\pgfusepath{stroke,fill}%
\end{pgfscope}%
\begin{pgfscope}%
\pgfpathrectangle{\pgfqpoint{0.570343in}{0.331635in}}{\pgfqpoint{9.300000in}{7.700000in}}%
\pgfusepath{clip}%
\pgfsetbuttcap%
\pgfsetroundjoin%
\definecolor{currentfill}{rgb}{0.631373,0.788235,0.956863}%
\pgfsetfillcolor{currentfill}%
\pgfsetlinewidth{0.481800pt}%
\definecolor{currentstroke}{rgb}{1.000000,1.000000,1.000000}%
\pgfsetstrokecolor{currentstroke}%
\pgfsetdash{}{0pt}%
\pgfpathmoveto{\pgfqpoint{2.997649in}{1.895736in}}%
\pgfpathcurveto{\pgfqpoint{3.008699in}{1.895736in}}{\pgfqpoint{3.019298in}{1.900126in}}{\pgfqpoint{3.027112in}{1.907940in}}%
\pgfpathcurveto{\pgfqpoint{3.034926in}{1.915753in}}{\pgfqpoint{3.039316in}{1.926352in}}{\pgfqpoint{3.039316in}{1.937402in}}%
\pgfpathcurveto{\pgfqpoint{3.039316in}{1.948452in}}{\pgfqpoint{3.034926in}{1.959051in}}{\pgfqpoint{3.027112in}{1.966865in}}%
\pgfpathcurveto{\pgfqpoint{3.019298in}{1.974679in}}{\pgfqpoint{3.008699in}{1.979069in}}{\pgfqpoint{2.997649in}{1.979069in}}%
\pgfpathcurveto{\pgfqpoint{2.986599in}{1.979069in}}{\pgfqpoint{2.976000in}{1.974679in}}{\pgfqpoint{2.968186in}{1.966865in}}%
\pgfpathcurveto{\pgfqpoint{2.960373in}{1.959051in}}{\pgfqpoint{2.955983in}{1.948452in}}{\pgfqpoint{2.955983in}{1.937402in}}%
\pgfpathcurveto{\pgfqpoint{2.955983in}{1.926352in}}{\pgfqpoint{2.960373in}{1.915753in}}{\pgfqpoint{2.968186in}{1.907940in}}%
\pgfpathcurveto{\pgfqpoint{2.976000in}{1.900126in}}{\pgfqpoint{2.986599in}{1.895736in}}{\pgfqpoint{2.997649in}{1.895736in}}%
\pgfpathclose%
\pgfusepath{stroke,fill}%
\end{pgfscope}%
\begin{pgfscope}%
\pgfpathrectangle{\pgfqpoint{0.570343in}{0.331635in}}{\pgfqpoint{9.300000in}{7.700000in}}%
\pgfusepath{clip}%
\pgfsetbuttcap%
\pgfsetroundjoin%
\definecolor{currentfill}{rgb}{0.631373,0.788235,0.956863}%
\pgfsetfillcolor{currentfill}%
\pgfsetlinewidth{0.481800pt}%
\definecolor{currentstroke}{rgb}{1.000000,1.000000,1.000000}%
\pgfsetstrokecolor{currentstroke}%
\pgfsetdash{}{0pt}%
\pgfpathmoveto{\pgfqpoint{5.501193in}{6.813359in}}%
\pgfpathcurveto{\pgfqpoint{5.512243in}{6.813359in}}{\pgfqpoint{5.522842in}{6.817749in}}{\pgfqpoint{5.530656in}{6.825563in}}%
\pgfpathcurveto{\pgfqpoint{5.538469in}{6.833377in}}{\pgfqpoint{5.542859in}{6.843976in}}{\pgfqpoint{5.542859in}{6.855026in}}%
\pgfpathcurveto{\pgfqpoint{5.542859in}{6.866076in}}{\pgfqpoint{5.538469in}{6.876675in}}{\pgfqpoint{5.530656in}{6.884489in}}%
\pgfpathcurveto{\pgfqpoint{5.522842in}{6.892302in}}{\pgfqpoint{5.512243in}{6.896692in}}{\pgfqpoint{5.501193in}{6.896692in}}%
\pgfpathcurveto{\pgfqpoint{5.490143in}{6.896692in}}{\pgfqpoint{5.479544in}{6.892302in}}{\pgfqpoint{5.471730in}{6.884489in}}%
\pgfpathcurveto{\pgfqpoint{5.463916in}{6.876675in}}{\pgfqpoint{5.459526in}{6.866076in}}{\pgfqpoint{5.459526in}{6.855026in}}%
\pgfpathcurveto{\pgfqpoint{5.459526in}{6.843976in}}{\pgfqpoint{5.463916in}{6.833377in}}{\pgfqpoint{5.471730in}{6.825563in}}%
\pgfpathcurveto{\pgfqpoint{5.479544in}{6.817749in}}{\pgfqpoint{5.490143in}{6.813359in}}{\pgfqpoint{5.501193in}{6.813359in}}%
\pgfpathclose%
\pgfusepath{stroke,fill}%
\end{pgfscope}%
\begin{pgfscope}%
\pgfpathrectangle{\pgfqpoint{0.570343in}{0.331635in}}{\pgfqpoint{9.300000in}{7.700000in}}%
\pgfusepath{clip}%
\pgfsetbuttcap%
\pgfsetroundjoin%
\definecolor{currentfill}{rgb}{0.631373,0.788235,0.956863}%
\pgfsetfillcolor{currentfill}%
\pgfsetlinewidth{0.481800pt}%
\definecolor{currentstroke}{rgb}{1.000000,1.000000,1.000000}%
\pgfsetstrokecolor{currentstroke}%
\pgfsetdash{}{0pt}%
\pgfpathmoveto{\pgfqpoint{4.900969in}{5.299919in}}%
\pgfpathcurveto{\pgfqpoint{4.912019in}{5.299919in}}{\pgfqpoint{4.922618in}{5.304309in}}{\pgfqpoint{4.930431in}{5.312123in}}%
\pgfpathcurveto{\pgfqpoint{4.938245in}{5.319936in}}{\pgfqpoint{4.942635in}{5.330536in}}{\pgfqpoint{4.942635in}{5.341586in}}%
\pgfpathcurveto{\pgfqpoint{4.942635in}{5.352636in}}{\pgfqpoint{4.938245in}{5.363235in}}{\pgfqpoint{4.930431in}{5.371048in}}%
\pgfpathcurveto{\pgfqpoint{4.922618in}{5.378862in}}{\pgfqpoint{4.912019in}{5.383252in}}{\pgfqpoint{4.900969in}{5.383252in}}%
\pgfpathcurveto{\pgfqpoint{4.889918in}{5.383252in}}{\pgfqpoint{4.879319in}{5.378862in}}{\pgfqpoint{4.871506in}{5.371048in}}%
\pgfpathcurveto{\pgfqpoint{4.863692in}{5.363235in}}{\pgfqpoint{4.859302in}{5.352636in}}{\pgfqpoint{4.859302in}{5.341586in}}%
\pgfpathcurveto{\pgfqpoint{4.859302in}{5.330536in}}{\pgfqpoint{4.863692in}{5.319936in}}{\pgfqpoint{4.871506in}{5.312123in}}%
\pgfpathcurveto{\pgfqpoint{4.879319in}{5.304309in}}{\pgfqpoint{4.889918in}{5.299919in}}{\pgfqpoint{4.900969in}{5.299919in}}%
\pgfpathclose%
\pgfusepath{stroke,fill}%
\end{pgfscope}%
\begin{pgfscope}%
\pgfpathrectangle{\pgfqpoint{0.570343in}{0.331635in}}{\pgfqpoint{9.300000in}{7.700000in}}%
\pgfusepath{clip}%
\pgfsetbuttcap%
\pgfsetroundjoin%
\definecolor{currentfill}{rgb}{1.000000,0.705882,0.509804}%
\pgfsetfillcolor{currentfill}%
\pgfsetlinewidth{0.481800pt}%
\definecolor{currentstroke}{rgb}{1.000000,1.000000,1.000000}%
\pgfsetstrokecolor{currentstroke}%
\pgfsetdash{}{0pt}%
\pgfpathmoveto{\pgfqpoint{5.688827in}{3.673198in}}%
\pgfpathcurveto{\pgfqpoint{5.699877in}{3.673198in}}{\pgfqpoint{5.710476in}{3.677588in}}{\pgfqpoint{5.718290in}{3.685402in}}%
\pgfpathcurveto{\pgfqpoint{5.726104in}{3.693216in}}{\pgfqpoint{5.730494in}{3.703815in}}{\pgfqpoint{5.730494in}{3.714865in}}%
\pgfpathcurveto{\pgfqpoint{5.730494in}{3.725915in}}{\pgfqpoint{5.726104in}{3.736514in}}{\pgfqpoint{5.718290in}{3.744328in}}%
\pgfpathcurveto{\pgfqpoint{5.710476in}{3.752141in}}{\pgfqpoint{5.699877in}{3.756532in}}{\pgfqpoint{5.688827in}{3.756532in}}%
\pgfpathcurveto{\pgfqpoint{5.677777in}{3.756532in}}{\pgfqpoint{5.667178in}{3.752141in}}{\pgfqpoint{5.659364in}{3.744328in}}%
\pgfpathcurveto{\pgfqpoint{5.651551in}{3.736514in}}{\pgfqpoint{5.647161in}{3.725915in}}{\pgfqpoint{5.647161in}{3.714865in}}%
\pgfpathcurveto{\pgfqpoint{5.647161in}{3.703815in}}{\pgfqpoint{5.651551in}{3.693216in}}{\pgfqpoint{5.659364in}{3.685402in}}%
\pgfpathcurveto{\pgfqpoint{5.667178in}{3.677588in}}{\pgfqpoint{5.677777in}{3.673198in}}{\pgfqpoint{5.688827in}{3.673198in}}%
\pgfpathclose%
\pgfusepath{stroke,fill}%
\end{pgfscope}%
\begin{pgfscope}%
\pgfpathrectangle{\pgfqpoint{0.570343in}{0.331635in}}{\pgfqpoint{9.300000in}{7.700000in}}%
\pgfusepath{clip}%
\pgfsetbuttcap%
\pgfsetroundjoin%
\definecolor{currentfill}{rgb}{1.000000,0.705882,0.509804}%
\pgfsetfillcolor{currentfill}%
\pgfsetlinewidth{0.481800pt}%
\definecolor{currentstroke}{rgb}{1.000000,1.000000,1.000000}%
\pgfsetstrokecolor{currentstroke}%
\pgfsetdash{}{0pt}%
\pgfpathmoveto{\pgfqpoint{7.831480in}{6.697908in}}%
\pgfpathcurveto{\pgfqpoint{7.842530in}{6.697908in}}{\pgfqpoint{7.853129in}{6.702298in}}{\pgfqpoint{7.860943in}{6.710112in}}%
\pgfpathcurveto{\pgfqpoint{7.868757in}{6.717925in}}{\pgfqpoint{7.873147in}{6.728524in}}{\pgfqpoint{7.873147in}{6.739574in}}%
\pgfpathcurveto{\pgfqpoint{7.873147in}{6.750625in}}{\pgfqpoint{7.868757in}{6.761224in}}{\pgfqpoint{7.860943in}{6.769037in}}%
\pgfpathcurveto{\pgfqpoint{7.853129in}{6.776851in}}{\pgfqpoint{7.842530in}{6.781241in}}{\pgfqpoint{7.831480in}{6.781241in}}%
\pgfpathcurveto{\pgfqpoint{7.820430in}{6.781241in}}{\pgfqpoint{7.809831in}{6.776851in}}{\pgfqpoint{7.802017in}{6.769037in}}%
\pgfpathcurveto{\pgfqpoint{7.794204in}{6.761224in}}{\pgfqpoint{7.789813in}{6.750625in}}{\pgfqpoint{7.789813in}{6.739574in}}%
\pgfpathcurveto{\pgfqpoint{7.789813in}{6.728524in}}{\pgfqpoint{7.794204in}{6.717925in}}{\pgfqpoint{7.802017in}{6.710112in}}%
\pgfpathcurveto{\pgfqpoint{7.809831in}{6.702298in}}{\pgfqpoint{7.820430in}{6.697908in}}{\pgfqpoint{7.831480in}{6.697908in}}%
\pgfpathclose%
\pgfusepath{stroke,fill}%
\end{pgfscope}%
\begin{pgfscope}%
\pgfpathrectangle{\pgfqpoint{0.570343in}{0.331635in}}{\pgfqpoint{9.300000in}{7.700000in}}%
\pgfusepath{clip}%
\pgfsetbuttcap%
\pgfsetroundjoin%
\definecolor{currentfill}{rgb}{1.000000,0.705882,0.509804}%
\pgfsetfillcolor{currentfill}%
\pgfsetlinewidth{0.481800pt}%
\definecolor{currentstroke}{rgb}{1.000000,1.000000,1.000000}%
\pgfsetstrokecolor{currentstroke}%
\pgfsetdash{}{0pt}%
\pgfpathmoveto{\pgfqpoint{5.338526in}{4.857417in}}%
\pgfpathcurveto{\pgfqpoint{5.349576in}{4.857417in}}{\pgfqpoint{5.360175in}{4.861808in}}{\pgfqpoint{5.367989in}{4.869621in}}%
\pgfpathcurveto{\pgfqpoint{5.375802in}{4.877435in}}{\pgfqpoint{5.380192in}{4.888034in}}{\pgfqpoint{5.380192in}{4.899084in}}%
\pgfpathcurveto{\pgfqpoint{5.380192in}{4.910134in}}{\pgfqpoint{5.375802in}{4.920733in}}{\pgfqpoint{5.367989in}{4.928547in}}%
\pgfpathcurveto{\pgfqpoint{5.360175in}{4.936360in}}{\pgfqpoint{5.349576in}{4.940751in}}{\pgfqpoint{5.338526in}{4.940751in}}%
\pgfpathcurveto{\pgfqpoint{5.327476in}{4.940751in}}{\pgfqpoint{5.316877in}{4.936360in}}{\pgfqpoint{5.309063in}{4.928547in}}%
\pgfpathcurveto{\pgfqpoint{5.301249in}{4.920733in}}{\pgfqpoint{5.296859in}{4.910134in}}{\pgfqpoint{5.296859in}{4.899084in}}%
\pgfpathcurveto{\pgfqpoint{5.296859in}{4.888034in}}{\pgfqpoint{5.301249in}{4.877435in}}{\pgfqpoint{5.309063in}{4.869621in}}%
\pgfpathcurveto{\pgfqpoint{5.316877in}{4.861808in}}{\pgfqpoint{5.327476in}{4.857417in}}{\pgfqpoint{5.338526in}{4.857417in}}%
\pgfpathclose%
\pgfusepath{stroke,fill}%
\end{pgfscope}%
\begin{pgfscope}%
\pgfpathrectangle{\pgfqpoint{0.570343in}{0.331635in}}{\pgfqpoint{9.300000in}{7.700000in}}%
\pgfusepath{clip}%
\pgfsetbuttcap%
\pgfsetroundjoin%
\definecolor{currentfill}{rgb}{1.000000,0.705882,0.509804}%
\pgfsetfillcolor{currentfill}%
\pgfsetlinewidth{0.481800pt}%
\definecolor{currentstroke}{rgb}{1.000000,1.000000,1.000000}%
\pgfsetstrokecolor{currentstroke}%
\pgfsetdash{}{0pt}%
\pgfpathmoveto{\pgfqpoint{6.546058in}{3.193196in}}%
\pgfpathcurveto{\pgfqpoint{6.557108in}{3.193196in}}{\pgfqpoint{6.567707in}{3.197586in}}{\pgfqpoint{6.575520in}{3.205399in}}%
\pgfpathcurveto{\pgfqpoint{6.583334in}{3.213213in}}{\pgfqpoint{6.587724in}{3.223812in}}{\pgfqpoint{6.587724in}{3.234862in}}%
\pgfpathcurveto{\pgfqpoint{6.587724in}{3.245912in}}{\pgfqpoint{6.583334in}{3.256511in}}{\pgfqpoint{6.575520in}{3.264325in}}%
\pgfpathcurveto{\pgfqpoint{6.567707in}{3.272139in}}{\pgfqpoint{6.557108in}{3.276529in}}{\pgfqpoint{6.546058in}{3.276529in}}%
\pgfpathcurveto{\pgfqpoint{6.535008in}{3.276529in}}{\pgfqpoint{6.524408in}{3.272139in}}{\pgfqpoint{6.516595in}{3.264325in}}%
\pgfpathcurveto{\pgfqpoint{6.508781in}{3.256511in}}{\pgfqpoint{6.504391in}{3.245912in}}{\pgfqpoint{6.504391in}{3.234862in}}%
\pgfpathcurveto{\pgfqpoint{6.504391in}{3.223812in}}{\pgfqpoint{6.508781in}{3.213213in}}{\pgfqpoint{6.516595in}{3.205399in}}%
\pgfpathcurveto{\pgfqpoint{6.524408in}{3.197586in}}{\pgfqpoint{6.535008in}{3.193196in}}{\pgfqpoint{6.546058in}{3.193196in}}%
\pgfpathclose%
\pgfusepath{stroke,fill}%
\end{pgfscope}%
\begin{pgfscope}%
\pgfpathrectangle{\pgfqpoint{0.570343in}{0.331635in}}{\pgfqpoint{9.300000in}{7.700000in}}%
\pgfusepath{clip}%
\pgfsetbuttcap%
\pgfsetroundjoin%
\definecolor{currentfill}{rgb}{1.000000,0.705882,0.509804}%
\pgfsetfillcolor{currentfill}%
\pgfsetlinewidth{0.481800pt}%
\definecolor{currentstroke}{rgb}{1.000000,1.000000,1.000000}%
\pgfsetstrokecolor{currentstroke}%
\pgfsetdash{}{0pt}%
\pgfpathmoveto{\pgfqpoint{8.931283in}{5.352643in}}%
\pgfpathcurveto{\pgfqpoint{8.942334in}{5.352643in}}{\pgfqpoint{8.952933in}{5.357033in}}{\pgfqpoint{8.960746in}{5.364847in}}%
\pgfpathcurveto{\pgfqpoint{8.968560in}{5.372660in}}{\pgfqpoint{8.972950in}{5.383259in}}{\pgfqpoint{8.972950in}{5.394309in}}%
\pgfpathcurveto{\pgfqpoint{8.972950in}{5.405360in}}{\pgfqpoint{8.968560in}{5.415959in}}{\pgfqpoint{8.960746in}{5.423772in}}%
\pgfpathcurveto{\pgfqpoint{8.952933in}{5.431586in}}{\pgfqpoint{8.942334in}{5.435976in}}{\pgfqpoint{8.931283in}{5.435976in}}%
\pgfpathcurveto{\pgfqpoint{8.920233in}{5.435976in}}{\pgfqpoint{8.909634in}{5.431586in}}{\pgfqpoint{8.901821in}{5.423772in}}%
\pgfpathcurveto{\pgfqpoint{8.894007in}{5.415959in}}{\pgfqpoint{8.889617in}{5.405360in}}{\pgfqpoint{8.889617in}{5.394309in}}%
\pgfpathcurveto{\pgfqpoint{8.889617in}{5.383259in}}{\pgfqpoint{8.894007in}{5.372660in}}{\pgfqpoint{8.901821in}{5.364847in}}%
\pgfpathcurveto{\pgfqpoint{8.909634in}{5.357033in}}{\pgfqpoint{8.920233in}{5.352643in}}{\pgfqpoint{8.931283in}{5.352643in}}%
\pgfpathclose%
\pgfusepath{stroke,fill}%
\end{pgfscope}%
\begin{pgfscope}%
\pgfpathrectangle{\pgfqpoint{0.570343in}{0.331635in}}{\pgfqpoint{9.300000in}{7.700000in}}%
\pgfusepath{clip}%
\pgfsetbuttcap%
\pgfsetroundjoin%
\definecolor{currentfill}{rgb}{1.000000,0.705882,0.509804}%
\pgfsetfillcolor{currentfill}%
\pgfsetlinewidth{0.481800pt}%
\definecolor{currentstroke}{rgb}{1.000000,1.000000,1.000000}%
\pgfsetstrokecolor{currentstroke}%
\pgfsetdash{}{0pt}%
\pgfpathmoveto{\pgfqpoint{4.854875in}{3.413194in}}%
\pgfpathcurveto{\pgfqpoint{4.865925in}{3.413194in}}{\pgfqpoint{4.876524in}{3.417585in}}{\pgfqpoint{4.884338in}{3.425398in}}%
\pgfpathcurveto{\pgfqpoint{4.892152in}{3.433212in}}{\pgfqpoint{4.896542in}{3.443811in}}{\pgfqpoint{4.896542in}{3.454861in}}%
\pgfpathcurveto{\pgfqpoint{4.896542in}{3.465911in}}{\pgfqpoint{4.892152in}{3.476510in}}{\pgfqpoint{4.884338in}{3.484324in}}%
\pgfpathcurveto{\pgfqpoint{4.876524in}{3.492138in}}{\pgfqpoint{4.865925in}{3.496528in}}{\pgfqpoint{4.854875in}{3.496528in}}%
\pgfpathcurveto{\pgfqpoint{4.843825in}{3.496528in}}{\pgfqpoint{4.833226in}{3.492138in}}{\pgfqpoint{4.825412in}{3.484324in}}%
\pgfpathcurveto{\pgfqpoint{4.817599in}{3.476510in}}{\pgfqpoint{4.813208in}{3.465911in}}{\pgfqpoint{4.813208in}{3.454861in}}%
\pgfpathcurveto{\pgfqpoint{4.813208in}{3.443811in}}{\pgfqpoint{4.817599in}{3.433212in}}{\pgfqpoint{4.825412in}{3.425398in}}%
\pgfpathcurveto{\pgfqpoint{4.833226in}{3.417585in}}{\pgfqpoint{4.843825in}{3.413194in}}{\pgfqpoint{4.854875in}{3.413194in}}%
\pgfpathclose%
\pgfusepath{stroke,fill}%
\end{pgfscope}%
\begin{pgfscope}%
\pgfpathrectangle{\pgfqpoint{0.570343in}{0.331635in}}{\pgfqpoint{9.300000in}{7.700000in}}%
\pgfusepath{clip}%
\pgfsetbuttcap%
\pgfsetroundjoin%
\definecolor{currentfill}{rgb}{1.000000,0.705882,0.509804}%
\pgfsetfillcolor{currentfill}%
\pgfsetlinewidth{0.481800pt}%
\definecolor{currentstroke}{rgb}{1.000000,1.000000,1.000000}%
\pgfsetstrokecolor{currentstroke}%
\pgfsetdash{}{0pt}%
\pgfpathmoveto{\pgfqpoint{7.413796in}{3.602347in}}%
\pgfpathcurveto{\pgfqpoint{7.424846in}{3.602347in}}{\pgfqpoint{7.435445in}{3.606737in}}{\pgfqpoint{7.443258in}{3.614550in}}%
\pgfpathcurveto{\pgfqpoint{7.451072in}{3.622364in}}{\pgfqpoint{7.455462in}{3.632963in}}{\pgfqpoint{7.455462in}{3.644013in}}%
\pgfpathcurveto{\pgfqpoint{7.455462in}{3.655063in}}{\pgfqpoint{7.451072in}{3.665662in}}{\pgfqpoint{7.443258in}{3.673476in}}%
\pgfpathcurveto{\pgfqpoint{7.435445in}{3.681290in}}{\pgfqpoint{7.424846in}{3.685680in}}{\pgfqpoint{7.413796in}{3.685680in}}%
\pgfpathcurveto{\pgfqpoint{7.402745in}{3.685680in}}{\pgfqpoint{7.392146in}{3.681290in}}{\pgfqpoint{7.384333in}{3.673476in}}%
\pgfpathcurveto{\pgfqpoint{7.376519in}{3.665662in}}{\pgfqpoint{7.372129in}{3.655063in}}{\pgfqpoint{7.372129in}{3.644013in}}%
\pgfpathcurveto{\pgfqpoint{7.372129in}{3.632963in}}{\pgfqpoint{7.376519in}{3.622364in}}{\pgfqpoint{7.384333in}{3.614550in}}%
\pgfpathcurveto{\pgfqpoint{7.392146in}{3.606737in}}{\pgfqpoint{7.402745in}{3.602347in}}{\pgfqpoint{7.413796in}{3.602347in}}%
\pgfpathclose%
\pgfusepath{stroke,fill}%
\end{pgfscope}%
\begin{pgfscope}%
\pgfpathrectangle{\pgfqpoint{0.570343in}{0.331635in}}{\pgfqpoint{9.300000in}{7.700000in}}%
\pgfusepath{clip}%
\pgfsetbuttcap%
\pgfsetroundjoin%
\definecolor{currentfill}{rgb}{1.000000,0.705882,0.509804}%
\pgfsetfillcolor{currentfill}%
\pgfsetlinewidth{0.481800pt}%
\definecolor{currentstroke}{rgb}{1.000000,1.000000,1.000000}%
\pgfsetstrokecolor{currentstroke}%
\pgfsetdash{}{0pt}%
\pgfpathmoveto{\pgfqpoint{1.671602in}{3.865489in}}%
\pgfpathcurveto{\pgfqpoint{1.682652in}{3.865489in}}{\pgfqpoint{1.693252in}{3.869879in}}{\pgfqpoint{1.701065in}{3.877693in}}%
\pgfpathcurveto{\pgfqpoint{1.708879in}{3.885507in}}{\pgfqpoint{1.713269in}{3.896106in}}{\pgfqpoint{1.713269in}{3.907156in}}%
\pgfpathcurveto{\pgfqpoint{1.713269in}{3.918206in}}{\pgfqpoint{1.708879in}{3.928805in}}{\pgfqpoint{1.701065in}{3.936619in}}%
\pgfpathcurveto{\pgfqpoint{1.693252in}{3.944432in}}{\pgfqpoint{1.682652in}{3.948823in}}{\pgfqpoint{1.671602in}{3.948823in}}%
\pgfpathcurveto{\pgfqpoint{1.660552in}{3.948823in}}{\pgfqpoint{1.649953in}{3.944432in}}{\pgfqpoint{1.642140in}{3.936619in}}%
\pgfpathcurveto{\pgfqpoint{1.634326in}{3.928805in}}{\pgfqpoint{1.629936in}{3.918206in}}{\pgfqpoint{1.629936in}{3.907156in}}%
\pgfpathcurveto{\pgfqpoint{1.629936in}{3.896106in}}{\pgfqpoint{1.634326in}{3.885507in}}{\pgfqpoint{1.642140in}{3.877693in}}%
\pgfpathcurveto{\pgfqpoint{1.649953in}{3.869879in}}{\pgfqpoint{1.660552in}{3.865489in}}{\pgfqpoint{1.671602in}{3.865489in}}%
\pgfpathclose%
\pgfusepath{stroke,fill}%
\end{pgfscope}%
\begin{pgfscope}%
\pgfpathrectangle{\pgfqpoint{0.570343in}{0.331635in}}{\pgfqpoint{9.300000in}{7.700000in}}%
\pgfusepath{clip}%
\pgfsetbuttcap%
\pgfsetroundjoin%
\definecolor{currentfill}{rgb}{1.000000,0.705882,0.509804}%
\pgfsetfillcolor{currentfill}%
\pgfsetlinewidth{0.481800pt}%
\definecolor{currentstroke}{rgb}{1.000000,1.000000,1.000000}%
\pgfsetstrokecolor{currentstroke}%
\pgfsetdash{}{0pt}%
\pgfpathmoveto{\pgfqpoint{1.885787in}{2.890873in}}%
\pgfpathcurveto{\pgfqpoint{1.896837in}{2.890873in}}{\pgfqpoint{1.907436in}{2.895263in}}{\pgfqpoint{1.915250in}{2.903077in}}%
\pgfpathcurveto{\pgfqpoint{1.923064in}{2.910891in}}{\pgfqpoint{1.927454in}{2.921490in}}{\pgfqpoint{1.927454in}{2.932540in}}%
\pgfpathcurveto{\pgfqpoint{1.927454in}{2.943590in}}{\pgfqpoint{1.923064in}{2.954189in}}{\pgfqpoint{1.915250in}{2.962003in}}%
\pgfpathcurveto{\pgfqpoint{1.907436in}{2.969816in}}{\pgfqpoint{1.896837in}{2.974206in}}{\pgfqpoint{1.885787in}{2.974206in}}%
\pgfpathcurveto{\pgfqpoint{1.874737in}{2.974206in}}{\pgfqpoint{1.864138in}{2.969816in}}{\pgfqpoint{1.856325in}{2.962003in}}%
\pgfpathcurveto{\pgfqpoint{1.848511in}{2.954189in}}{\pgfqpoint{1.844121in}{2.943590in}}{\pgfqpoint{1.844121in}{2.932540in}}%
\pgfpathcurveto{\pgfqpoint{1.844121in}{2.921490in}}{\pgfqpoint{1.848511in}{2.910891in}}{\pgfqpoint{1.856325in}{2.903077in}}%
\pgfpathcurveto{\pgfqpoint{1.864138in}{2.895263in}}{\pgfqpoint{1.874737in}{2.890873in}}{\pgfqpoint{1.885787in}{2.890873in}}%
\pgfpathclose%
\pgfusepath{stroke,fill}%
\end{pgfscope}%
\begin{pgfscope}%
\pgfpathrectangle{\pgfqpoint{0.570343in}{0.331635in}}{\pgfqpoint{9.300000in}{7.700000in}}%
\pgfusepath{clip}%
\pgfsetbuttcap%
\pgfsetroundjoin%
\definecolor{currentfill}{rgb}{1.000000,0.705882,0.509804}%
\pgfsetfillcolor{currentfill}%
\pgfsetlinewidth{0.481800pt}%
\definecolor{currentstroke}{rgb}{1.000000,1.000000,1.000000}%
\pgfsetstrokecolor{currentstroke}%
\pgfsetdash{}{0pt}%
\pgfpathmoveto{\pgfqpoint{5.353276in}{4.246754in}}%
\pgfpathcurveto{\pgfqpoint{5.364327in}{4.246754in}}{\pgfqpoint{5.374926in}{4.251144in}}{\pgfqpoint{5.382739in}{4.258958in}}%
\pgfpathcurveto{\pgfqpoint{5.390553in}{4.266771in}}{\pgfqpoint{5.394943in}{4.277370in}}{\pgfqpoint{5.394943in}{4.288421in}}%
\pgfpathcurveto{\pgfqpoint{5.394943in}{4.299471in}}{\pgfqpoint{5.390553in}{4.310070in}}{\pgfqpoint{5.382739in}{4.317883in}}%
\pgfpathcurveto{\pgfqpoint{5.374926in}{4.325697in}}{\pgfqpoint{5.364327in}{4.330087in}}{\pgfqpoint{5.353276in}{4.330087in}}%
\pgfpathcurveto{\pgfqpoint{5.342226in}{4.330087in}}{\pgfqpoint{5.331627in}{4.325697in}}{\pgfqpoint{5.323814in}{4.317883in}}%
\pgfpathcurveto{\pgfqpoint{5.316000in}{4.310070in}}{\pgfqpoint{5.311610in}{4.299471in}}{\pgfqpoint{5.311610in}{4.288421in}}%
\pgfpathcurveto{\pgfqpoint{5.311610in}{4.277370in}}{\pgfqpoint{5.316000in}{4.266771in}}{\pgfqpoint{5.323814in}{4.258958in}}%
\pgfpathcurveto{\pgfqpoint{5.331627in}{4.251144in}}{\pgfqpoint{5.342226in}{4.246754in}}{\pgfqpoint{5.353276in}{4.246754in}}%
\pgfpathclose%
\pgfusepath{stroke,fill}%
\end{pgfscope}%
\begin{pgfscope}%
\pgfpathrectangle{\pgfqpoint{0.570343in}{0.331635in}}{\pgfqpoint{9.300000in}{7.700000in}}%
\pgfusepath{clip}%
\pgfsetbuttcap%
\pgfsetroundjoin%
\definecolor{currentfill}{rgb}{1.000000,0.705882,0.509804}%
\pgfsetfillcolor{currentfill}%
\pgfsetlinewidth{0.481800pt}%
\definecolor{currentstroke}{rgb}{1.000000,1.000000,1.000000}%
\pgfsetstrokecolor{currentstroke}%
\pgfsetdash{}{0pt}%
\pgfpathmoveto{\pgfqpoint{6.076381in}{4.577443in}}%
\pgfpathcurveto{\pgfqpoint{6.087432in}{4.577443in}}{\pgfqpoint{6.098031in}{4.581833in}}{\pgfqpoint{6.105844in}{4.589647in}}%
\pgfpathcurveto{\pgfqpoint{6.113658in}{4.597461in}}{\pgfqpoint{6.118048in}{4.608060in}}{\pgfqpoint{6.118048in}{4.619110in}}%
\pgfpathcurveto{\pgfqpoint{6.118048in}{4.630160in}}{\pgfqpoint{6.113658in}{4.640759in}}{\pgfqpoint{6.105844in}{4.648572in}}%
\pgfpathcurveto{\pgfqpoint{6.098031in}{4.656386in}}{\pgfqpoint{6.087432in}{4.660776in}}{\pgfqpoint{6.076381in}{4.660776in}}%
\pgfpathcurveto{\pgfqpoint{6.065331in}{4.660776in}}{\pgfqpoint{6.054732in}{4.656386in}}{\pgfqpoint{6.046919in}{4.648572in}}%
\pgfpathcurveto{\pgfqpoint{6.039105in}{4.640759in}}{\pgfqpoint{6.034715in}{4.630160in}}{\pgfqpoint{6.034715in}{4.619110in}}%
\pgfpathcurveto{\pgfqpoint{6.034715in}{4.608060in}}{\pgfqpoint{6.039105in}{4.597461in}}{\pgfqpoint{6.046919in}{4.589647in}}%
\pgfpathcurveto{\pgfqpoint{6.054732in}{4.581833in}}{\pgfqpoint{6.065331in}{4.577443in}}{\pgfqpoint{6.076381in}{4.577443in}}%
\pgfpathclose%
\pgfusepath{stroke,fill}%
\end{pgfscope}%
\begin{pgfscope}%
\pgfpathrectangle{\pgfqpoint{0.570343in}{0.331635in}}{\pgfqpoint{9.300000in}{7.700000in}}%
\pgfusepath{clip}%
\pgfsetbuttcap%
\pgfsetroundjoin%
\definecolor{currentfill}{rgb}{1.000000,0.705882,0.509804}%
\pgfsetfillcolor{currentfill}%
\pgfsetlinewidth{0.481800pt}%
\definecolor{currentstroke}{rgb}{1.000000,1.000000,1.000000}%
\pgfsetstrokecolor{currentstroke}%
\pgfsetdash{}{0pt}%
\pgfpathmoveto{\pgfqpoint{4.577524in}{7.639968in}}%
\pgfpathcurveto{\pgfqpoint{4.588574in}{7.639968in}}{\pgfqpoint{4.599173in}{7.644359in}}{\pgfqpoint{4.606986in}{7.652172in}}%
\pgfpathcurveto{\pgfqpoint{4.614800in}{7.659986in}}{\pgfqpoint{4.619190in}{7.670585in}}{\pgfqpoint{4.619190in}{7.681635in}}%
\pgfpathcurveto{\pgfqpoint{4.619190in}{7.692685in}}{\pgfqpoint{4.614800in}{7.703284in}}{\pgfqpoint{4.606986in}{7.711098in}}%
\pgfpathcurveto{\pgfqpoint{4.599173in}{7.718911in}}{\pgfqpoint{4.588574in}{7.723302in}}{\pgfqpoint{4.577524in}{7.723302in}}%
\pgfpathcurveto{\pgfqpoint{4.566474in}{7.723302in}}{\pgfqpoint{4.555875in}{7.718911in}}{\pgfqpoint{4.548061in}{7.711098in}}%
\pgfpathcurveto{\pgfqpoint{4.540247in}{7.703284in}}{\pgfqpoint{4.535857in}{7.692685in}}{\pgfqpoint{4.535857in}{7.681635in}}%
\pgfpathcurveto{\pgfqpoint{4.535857in}{7.670585in}}{\pgfqpoint{4.540247in}{7.659986in}}{\pgfqpoint{4.548061in}{7.652172in}}%
\pgfpathcurveto{\pgfqpoint{4.555875in}{7.644359in}}{\pgfqpoint{4.566474in}{7.639968in}}{\pgfqpoint{4.577524in}{7.639968in}}%
\pgfpathclose%
\pgfusepath{stroke,fill}%
\end{pgfscope}%
\begin{pgfscope}%
\pgfpathrectangle{\pgfqpoint{0.570343in}{0.331635in}}{\pgfqpoint{9.300000in}{7.700000in}}%
\pgfusepath{clip}%
\pgfsetbuttcap%
\pgfsetroundjoin%
\definecolor{currentfill}{rgb}{1.000000,0.705882,0.509804}%
\pgfsetfillcolor{currentfill}%
\pgfsetlinewidth{0.481800pt}%
\definecolor{currentstroke}{rgb}{1.000000,1.000000,1.000000}%
\pgfsetstrokecolor{currentstroke}%
\pgfsetdash{}{0pt}%
\pgfpathmoveto{\pgfqpoint{2.562027in}{6.241590in}}%
\pgfpathcurveto{\pgfqpoint{2.573077in}{6.241590in}}{\pgfqpoint{2.583676in}{6.245980in}}{\pgfqpoint{2.591490in}{6.253794in}}%
\pgfpathcurveto{\pgfqpoint{2.599303in}{6.261608in}}{\pgfqpoint{2.603694in}{6.272207in}}{\pgfqpoint{2.603694in}{6.283257in}}%
\pgfpathcurveto{\pgfqpoint{2.603694in}{6.294307in}}{\pgfqpoint{2.599303in}{6.304906in}}{\pgfqpoint{2.591490in}{6.312720in}}%
\pgfpathcurveto{\pgfqpoint{2.583676in}{6.320533in}}{\pgfqpoint{2.573077in}{6.324924in}}{\pgfqpoint{2.562027in}{6.324924in}}%
\pgfpathcurveto{\pgfqpoint{2.550977in}{6.324924in}}{\pgfqpoint{2.540378in}{6.320533in}}{\pgfqpoint{2.532564in}{6.312720in}}%
\pgfpathcurveto{\pgfqpoint{2.524751in}{6.304906in}}{\pgfqpoint{2.520360in}{6.294307in}}{\pgfqpoint{2.520360in}{6.283257in}}%
\pgfpathcurveto{\pgfqpoint{2.520360in}{6.272207in}}{\pgfqpoint{2.524751in}{6.261608in}}{\pgfqpoint{2.532564in}{6.253794in}}%
\pgfpathcurveto{\pgfqpoint{2.540378in}{6.245980in}}{\pgfqpoint{2.550977in}{6.241590in}}{\pgfqpoint{2.562027in}{6.241590in}}%
\pgfpathclose%
\pgfusepath{stroke,fill}%
\end{pgfscope}%
\begin{pgfscope}%
\pgfpathrectangle{\pgfqpoint{0.570343in}{0.331635in}}{\pgfqpoint{9.300000in}{7.700000in}}%
\pgfusepath{clip}%
\pgfsetbuttcap%
\pgfsetroundjoin%
\definecolor{currentfill}{rgb}{1.000000,0.705882,0.509804}%
\pgfsetfillcolor{currentfill}%
\pgfsetlinewidth{0.481800pt}%
\definecolor{currentstroke}{rgb}{1.000000,1.000000,1.000000}%
\pgfsetstrokecolor{currentstroke}%
\pgfsetdash{}{0pt}%
\pgfpathmoveto{\pgfqpoint{8.392100in}{4.860635in}}%
\pgfpathcurveto{\pgfqpoint{8.403150in}{4.860635in}}{\pgfqpoint{8.413749in}{4.865025in}}{\pgfqpoint{8.421563in}{4.872839in}}%
\pgfpathcurveto{\pgfqpoint{8.429376in}{4.880652in}}{\pgfqpoint{8.433767in}{4.891251in}}{\pgfqpoint{8.433767in}{4.902301in}}%
\pgfpathcurveto{\pgfqpoint{8.433767in}{4.913352in}}{\pgfqpoint{8.429376in}{4.923951in}}{\pgfqpoint{8.421563in}{4.931764in}}%
\pgfpathcurveto{\pgfqpoint{8.413749in}{4.939578in}}{\pgfqpoint{8.403150in}{4.943968in}}{\pgfqpoint{8.392100in}{4.943968in}}%
\pgfpathcurveto{\pgfqpoint{8.381050in}{4.943968in}}{\pgfqpoint{8.370451in}{4.939578in}}{\pgfqpoint{8.362637in}{4.931764in}}%
\pgfpathcurveto{\pgfqpoint{8.354824in}{4.923951in}}{\pgfqpoint{8.350433in}{4.913352in}}{\pgfqpoint{8.350433in}{4.902301in}}%
\pgfpathcurveto{\pgfqpoint{8.350433in}{4.891251in}}{\pgfqpoint{8.354824in}{4.880652in}}{\pgfqpoint{8.362637in}{4.872839in}}%
\pgfpathcurveto{\pgfqpoint{8.370451in}{4.865025in}}{\pgfqpoint{8.381050in}{4.860635in}}{\pgfqpoint{8.392100in}{4.860635in}}%
\pgfpathclose%
\pgfusepath{stroke,fill}%
\end{pgfscope}%
\begin{pgfscope}%
\pgfpathrectangle{\pgfqpoint{0.570343in}{0.331635in}}{\pgfqpoint{9.300000in}{7.700000in}}%
\pgfusepath{clip}%
\pgfsetbuttcap%
\pgfsetroundjoin%
\definecolor{currentfill}{rgb}{1.000000,0.705882,0.509804}%
\pgfsetfillcolor{currentfill}%
\pgfsetlinewidth{0.481800pt}%
\definecolor{currentstroke}{rgb}{1.000000,1.000000,1.000000}%
\pgfsetstrokecolor{currentstroke}%
\pgfsetdash{}{0pt}%
\pgfpathmoveto{\pgfqpoint{5.806842in}{1.666048in}}%
\pgfpathcurveto{\pgfqpoint{5.817893in}{1.666048in}}{\pgfqpoint{5.828492in}{1.670438in}}{\pgfqpoint{5.836305in}{1.678252in}}%
\pgfpathcurveto{\pgfqpoint{5.844119in}{1.686066in}}{\pgfqpoint{5.848509in}{1.696665in}}{\pgfqpoint{5.848509in}{1.707715in}}%
\pgfpathcurveto{\pgfqpoint{5.848509in}{1.718765in}}{\pgfqpoint{5.844119in}{1.729364in}}{\pgfqpoint{5.836305in}{1.737178in}}%
\pgfpathcurveto{\pgfqpoint{5.828492in}{1.744991in}}{\pgfqpoint{5.817893in}{1.749382in}}{\pgfqpoint{5.806842in}{1.749382in}}%
\pgfpathcurveto{\pgfqpoint{5.795792in}{1.749382in}}{\pgfqpoint{5.785193in}{1.744991in}}{\pgfqpoint{5.777380in}{1.737178in}}%
\pgfpathcurveto{\pgfqpoint{5.769566in}{1.729364in}}{\pgfqpoint{5.765176in}{1.718765in}}{\pgfqpoint{5.765176in}{1.707715in}}%
\pgfpathcurveto{\pgfqpoint{5.765176in}{1.696665in}}{\pgfqpoint{5.769566in}{1.686066in}}{\pgfqpoint{5.777380in}{1.678252in}}%
\pgfpathcurveto{\pgfqpoint{5.785193in}{1.670438in}}{\pgfqpoint{5.795792in}{1.666048in}}{\pgfqpoint{5.806842in}{1.666048in}}%
\pgfpathclose%
\pgfusepath{stroke,fill}%
\end{pgfscope}%
\begin{pgfscope}%
\pgfpathrectangle{\pgfqpoint{0.570343in}{0.331635in}}{\pgfqpoint{9.300000in}{7.700000in}}%
\pgfusepath{clip}%
\pgfsetbuttcap%
\pgfsetroundjoin%
\definecolor{currentfill}{rgb}{1.000000,0.705882,0.509804}%
\pgfsetfillcolor{currentfill}%
\pgfsetlinewidth{0.481800pt}%
\definecolor{currentstroke}{rgb}{1.000000,1.000000,1.000000}%
\pgfsetstrokecolor{currentstroke}%
\pgfsetdash{}{0pt}%
\pgfpathmoveto{\pgfqpoint{4.120169in}{3.694300in}}%
\pgfpathcurveto{\pgfqpoint{4.131219in}{3.694300in}}{\pgfqpoint{4.141818in}{3.698690in}}{\pgfqpoint{4.149632in}{3.706504in}}%
\pgfpathcurveto{\pgfqpoint{4.157446in}{3.714318in}}{\pgfqpoint{4.161836in}{3.724917in}}{\pgfqpoint{4.161836in}{3.735967in}}%
\pgfpathcurveto{\pgfqpoint{4.161836in}{3.747017in}}{\pgfqpoint{4.157446in}{3.757616in}}{\pgfqpoint{4.149632in}{3.765430in}}%
\pgfpathcurveto{\pgfqpoint{4.141818in}{3.773243in}}{\pgfqpoint{4.131219in}{3.777634in}}{\pgfqpoint{4.120169in}{3.777634in}}%
\pgfpathcurveto{\pgfqpoint{4.109119in}{3.777634in}}{\pgfqpoint{4.098520in}{3.773243in}}{\pgfqpoint{4.090707in}{3.765430in}}%
\pgfpathcurveto{\pgfqpoint{4.082893in}{3.757616in}}{\pgfqpoint{4.078503in}{3.747017in}}{\pgfqpoint{4.078503in}{3.735967in}}%
\pgfpathcurveto{\pgfqpoint{4.078503in}{3.724917in}}{\pgfqpoint{4.082893in}{3.714318in}}{\pgfqpoint{4.090707in}{3.706504in}}%
\pgfpathcurveto{\pgfqpoint{4.098520in}{3.698690in}}{\pgfqpoint{4.109119in}{3.694300in}}{\pgfqpoint{4.120169in}{3.694300in}}%
\pgfpathclose%
\pgfusepath{stroke,fill}%
\end{pgfscope}%
\begin{pgfscope}%
\pgfpathrectangle{\pgfqpoint{0.570343in}{0.331635in}}{\pgfqpoint{9.300000in}{7.700000in}}%
\pgfusepath{clip}%
\pgfsetbuttcap%
\pgfsetroundjoin%
\definecolor{currentfill}{rgb}{1.000000,0.705882,0.509804}%
\pgfsetfillcolor{currentfill}%
\pgfsetlinewidth{0.481800pt}%
\definecolor{currentstroke}{rgb}{1.000000,1.000000,1.000000}%
\pgfsetstrokecolor{currentstroke}%
\pgfsetdash{}{0pt}%
\pgfpathmoveto{\pgfqpoint{3.866323in}{1.796214in}}%
\pgfpathcurveto{\pgfqpoint{3.877374in}{1.796214in}}{\pgfqpoint{3.887973in}{1.800604in}}{\pgfqpoint{3.895786in}{1.808417in}}%
\pgfpathcurveto{\pgfqpoint{3.903600in}{1.816231in}}{\pgfqpoint{3.907990in}{1.826830in}}{\pgfqpoint{3.907990in}{1.837880in}}%
\pgfpathcurveto{\pgfqpoint{3.907990in}{1.848930in}}{\pgfqpoint{3.903600in}{1.859529in}}{\pgfqpoint{3.895786in}{1.867343in}}%
\pgfpathcurveto{\pgfqpoint{3.887973in}{1.875157in}}{\pgfqpoint{3.877374in}{1.879547in}}{\pgfqpoint{3.866323in}{1.879547in}}%
\pgfpathcurveto{\pgfqpoint{3.855273in}{1.879547in}}{\pgfqpoint{3.844674in}{1.875157in}}{\pgfqpoint{3.836861in}{1.867343in}}%
\pgfpathcurveto{\pgfqpoint{3.829047in}{1.859529in}}{\pgfqpoint{3.824657in}{1.848930in}}{\pgfqpoint{3.824657in}{1.837880in}}%
\pgfpathcurveto{\pgfqpoint{3.824657in}{1.826830in}}{\pgfqpoint{3.829047in}{1.816231in}}{\pgfqpoint{3.836861in}{1.808417in}}%
\pgfpathcurveto{\pgfqpoint{3.844674in}{1.800604in}}{\pgfqpoint{3.855273in}{1.796214in}}{\pgfqpoint{3.866323in}{1.796214in}}%
\pgfpathclose%
\pgfusepath{stroke,fill}%
\end{pgfscope}%
\begin{pgfscope}%
\pgfpathrectangle{\pgfqpoint{0.570343in}{0.331635in}}{\pgfqpoint{9.300000in}{7.700000in}}%
\pgfusepath{clip}%
\pgfsetbuttcap%
\pgfsetroundjoin%
\definecolor{currentfill}{rgb}{1.000000,0.705882,0.509804}%
\pgfsetfillcolor{currentfill}%
\pgfsetlinewidth{0.481800pt}%
\definecolor{currentstroke}{rgb}{1.000000,1.000000,1.000000}%
\pgfsetstrokecolor{currentstroke}%
\pgfsetdash{}{0pt}%
\pgfpathmoveto{\pgfqpoint{2.788649in}{3.977237in}}%
\pgfpathcurveto{\pgfqpoint{2.799699in}{3.977237in}}{\pgfqpoint{2.810298in}{3.981627in}}{\pgfqpoint{2.818112in}{3.989440in}}%
\pgfpathcurveto{\pgfqpoint{2.825926in}{3.997254in}}{\pgfqpoint{2.830316in}{4.007853in}}{\pgfqpoint{2.830316in}{4.018903in}}%
\pgfpathcurveto{\pgfqpoint{2.830316in}{4.029953in}}{\pgfqpoint{2.825926in}{4.040552in}}{\pgfqpoint{2.818112in}{4.048366in}}%
\pgfpathcurveto{\pgfqpoint{2.810298in}{4.056180in}}{\pgfqpoint{2.799699in}{4.060570in}}{\pgfqpoint{2.788649in}{4.060570in}}%
\pgfpathcurveto{\pgfqpoint{2.777599in}{4.060570in}}{\pgfqpoint{2.767000in}{4.056180in}}{\pgfqpoint{2.759187in}{4.048366in}}%
\pgfpathcurveto{\pgfqpoint{2.751373in}{4.040552in}}{\pgfqpoint{2.746983in}{4.029953in}}{\pgfqpoint{2.746983in}{4.018903in}}%
\pgfpathcurveto{\pgfqpoint{2.746983in}{4.007853in}}{\pgfqpoint{2.751373in}{3.997254in}}{\pgfqpoint{2.759187in}{3.989440in}}%
\pgfpathcurveto{\pgfqpoint{2.767000in}{3.981627in}}{\pgfqpoint{2.777599in}{3.977237in}}{\pgfqpoint{2.788649in}{3.977237in}}%
\pgfpathclose%
\pgfusepath{stroke,fill}%
\end{pgfscope}%
\begin{pgfscope}%
\pgfpathrectangle{\pgfqpoint{0.570343in}{0.331635in}}{\pgfqpoint{9.300000in}{7.700000in}}%
\pgfusepath{clip}%
\pgfsetbuttcap%
\pgfsetroundjoin%
\definecolor{currentfill}{rgb}{1.000000,0.705882,0.509804}%
\pgfsetfillcolor{currentfill}%
\pgfsetlinewidth{0.481800pt}%
\definecolor{currentstroke}{rgb}{1.000000,1.000000,1.000000}%
\pgfsetstrokecolor{currentstroke}%
\pgfsetdash{}{0pt}%
\pgfpathmoveto{\pgfqpoint{9.447616in}{3.622428in}}%
\pgfpathcurveto{\pgfqpoint{9.458666in}{3.622428in}}{\pgfqpoint{9.469265in}{3.626818in}}{\pgfqpoint{9.477079in}{3.634632in}}%
\pgfpathcurveto{\pgfqpoint{9.484892in}{3.642445in}}{\pgfqpoint{9.489283in}{3.653044in}}{\pgfqpoint{9.489283in}{3.664094in}}%
\pgfpathcurveto{\pgfqpoint{9.489283in}{3.675144in}}{\pgfqpoint{9.484892in}{3.685744in}}{\pgfqpoint{9.477079in}{3.693557in}}%
\pgfpathcurveto{\pgfqpoint{9.469265in}{3.701371in}}{\pgfqpoint{9.458666in}{3.705761in}}{\pgfqpoint{9.447616in}{3.705761in}}%
\pgfpathcurveto{\pgfqpoint{9.436566in}{3.705761in}}{\pgfqpoint{9.425967in}{3.701371in}}{\pgfqpoint{9.418153in}{3.693557in}}%
\pgfpathcurveto{\pgfqpoint{9.410340in}{3.685744in}}{\pgfqpoint{9.405949in}{3.675144in}}{\pgfqpoint{9.405949in}{3.664094in}}%
\pgfpathcurveto{\pgfqpoint{9.405949in}{3.653044in}}{\pgfqpoint{9.410340in}{3.642445in}}{\pgfqpoint{9.418153in}{3.634632in}}%
\pgfpathcurveto{\pgfqpoint{9.425967in}{3.626818in}}{\pgfqpoint{9.436566in}{3.622428in}}{\pgfqpoint{9.447616in}{3.622428in}}%
\pgfpathclose%
\pgfusepath{stroke,fill}%
\end{pgfscope}%
\begin{pgfscope}%
\pgfpathrectangle{\pgfqpoint{0.570343in}{0.331635in}}{\pgfqpoint{9.300000in}{7.700000in}}%
\pgfusepath{clip}%
\pgfsetbuttcap%
\pgfsetroundjoin%
\definecolor{currentfill}{rgb}{1.000000,0.705882,0.509804}%
\pgfsetfillcolor{currentfill}%
\pgfsetlinewidth{0.481800pt}%
\definecolor{currentstroke}{rgb}{1.000000,1.000000,1.000000}%
\pgfsetstrokecolor{currentstroke}%
\pgfsetdash{}{0pt}%
\pgfpathmoveto{\pgfqpoint{5.814101in}{0.639968in}}%
\pgfpathcurveto{\pgfqpoint{5.825151in}{0.639968in}}{\pgfqpoint{5.835750in}{0.644359in}}{\pgfqpoint{5.843564in}{0.652172in}}%
\pgfpathcurveto{\pgfqpoint{5.851377in}{0.659986in}}{\pgfqpoint{5.855767in}{0.670585in}}{\pgfqpoint{5.855767in}{0.681635in}}%
\pgfpathcurveto{\pgfqpoint{5.855767in}{0.692685in}}{\pgfqpoint{5.851377in}{0.703284in}}{\pgfqpoint{5.843564in}{0.711098in}}%
\pgfpathcurveto{\pgfqpoint{5.835750in}{0.718911in}}{\pgfqpoint{5.825151in}{0.723302in}}{\pgfqpoint{5.814101in}{0.723302in}}%
\pgfpathcurveto{\pgfqpoint{5.803051in}{0.723302in}}{\pgfqpoint{5.792452in}{0.718911in}}{\pgfqpoint{5.784638in}{0.711098in}}%
\pgfpathcurveto{\pgfqpoint{5.776824in}{0.703284in}}{\pgfqpoint{5.772434in}{0.692685in}}{\pgfqpoint{5.772434in}{0.681635in}}%
\pgfpathcurveto{\pgfqpoint{5.772434in}{0.670585in}}{\pgfqpoint{5.776824in}{0.659986in}}{\pgfqpoint{5.784638in}{0.652172in}}%
\pgfpathcurveto{\pgfqpoint{5.792452in}{0.644359in}}{\pgfqpoint{5.803051in}{0.639968in}}{\pgfqpoint{5.814101in}{0.639968in}}%
\pgfpathclose%
\pgfusepath{stroke,fill}%
\end{pgfscope}%
\begin{pgfscope}%
\pgfpathrectangle{\pgfqpoint{0.570343in}{0.331635in}}{\pgfqpoint{9.300000in}{7.700000in}}%
\pgfusepath{clip}%
\pgfsetbuttcap%
\pgfsetroundjoin%
\definecolor{currentfill}{rgb}{1.000000,0.705882,0.509804}%
\pgfsetfillcolor{currentfill}%
\pgfsetlinewidth{0.481800pt}%
\definecolor{currentstroke}{rgb}{1.000000,1.000000,1.000000}%
\pgfsetstrokecolor{currentstroke}%
\pgfsetdash{}{0pt}%
\pgfpathmoveto{\pgfqpoint{5.056091in}{2.233045in}}%
\pgfpathcurveto{\pgfqpoint{5.067141in}{2.233045in}}{\pgfqpoint{5.077740in}{2.237435in}}{\pgfqpoint{5.085554in}{2.245248in}}%
\pgfpathcurveto{\pgfqpoint{5.093368in}{2.253062in}}{\pgfqpoint{5.097758in}{2.263661in}}{\pgfqpoint{5.097758in}{2.274711in}}%
\pgfpathcurveto{\pgfqpoint{5.097758in}{2.285761in}}{\pgfqpoint{5.093368in}{2.296360in}}{\pgfqpoint{5.085554in}{2.304174in}}%
\pgfpathcurveto{\pgfqpoint{5.077740in}{2.311988in}}{\pgfqpoint{5.067141in}{2.316378in}}{\pgfqpoint{5.056091in}{2.316378in}}%
\pgfpathcurveto{\pgfqpoint{5.045041in}{2.316378in}}{\pgfqpoint{5.034442in}{2.311988in}}{\pgfqpoint{5.026628in}{2.304174in}}%
\pgfpathcurveto{\pgfqpoint{5.018815in}{2.296360in}}{\pgfqpoint{5.014425in}{2.285761in}}{\pgfqpoint{5.014425in}{2.274711in}}%
\pgfpathcurveto{\pgfqpoint{5.014425in}{2.263661in}}{\pgfqpoint{5.018815in}{2.253062in}}{\pgfqpoint{5.026628in}{2.245248in}}%
\pgfpathcurveto{\pgfqpoint{5.034442in}{2.237435in}}{\pgfqpoint{5.045041in}{2.233045in}}{\pgfqpoint{5.056091in}{2.233045in}}%
\pgfpathclose%
\pgfusepath{stroke,fill}%
\end{pgfscope}%
\begin{pgfscope}%
\pgfpathrectangle{\pgfqpoint{0.570343in}{0.331635in}}{\pgfqpoint{9.300000in}{7.700000in}}%
\pgfusepath{clip}%
\pgfsetbuttcap%
\pgfsetroundjoin%
\definecolor{currentfill}{rgb}{1.000000,0.705882,0.509804}%
\pgfsetfillcolor{currentfill}%
\pgfsetlinewidth{0.481800pt}%
\definecolor{currentstroke}{rgb}{1.000000,1.000000,1.000000}%
\pgfsetstrokecolor{currentstroke}%
\pgfsetdash{}{0pt}%
\pgfpathmoveto{\pgfqpoint{8.597790in}{3.777488in}}%
\pgfpathcurveto{\pgfqpoint{8.608840in}{3.777488in}}{\pgfqpoint{8.619439in}{3.781879in}}{\pgfqpoint{8.627252in}{3.789692in}}%
\pgfpathcurveto{\pgfqpoint{8.635066in}{3.797506in}}{\pgfqpoint{8.639456in}{3.808105in}}{\pgfqpoint{8.639456in}{3.819155in}}%
\pgfpathcurveto{\pgfqpoint{8.639456in}{3.830205in}}{\pgfqpoint{8.635066in}{3.840804in}}{\pgfqpoint{8.627252in}{3.848618in}}%
\pgfpathcurveto{\pgfqpoint{8.619439in}{3.856431in}}{\pgfqpoint{8.608840in}{3.860822in}}{\pgfqpoint{8.597790in}{3.860822in}}%
\pgfpathcurveto{\pgfqpoint{8.586740in}{3.860822in}}{\pgfqpoint{8.576141in}{3.856431in}}{\pgfqpoint{8.568327in}{3.848618in}}%
\pgfpathcurveto{\pgfqpoint{8.560513in}{3.840804in}}{\pgfqpoint{8.556123in}{3.830205in}}{\pgfqpoint{8.556123in}{3.819155in}}%
\pgfpathcurveto{\pgfqpoint{8.556123in}{3.808105in}}{\pgfqpoint{8.560513in}{3.797506in}}{\pgfqpoint{8.568327in}{3.789692in}}%
\pgfpathcurveto{\pgfqpoint{8.576141in}{3.781879in}}{\pgfqpoint{8.586740in}{3.777488in}}{\pgfqpoint{8.597790in}{3.777488in}}%
\pgfpathclose%
\pgfusepath{stroke,fill}%
\end{pgfscope}%
\begin{pgfscope}%
\pgfpathrectangle{\pgfqpoint{0.570343in}{0.331635in}}{\pgfqpoint{9.300000in}{7.700000in}}%
\pgfusepath{clip}%
\pgfsetbuttcap%
\pgfsetroundjoin%
\definecolor{currentfill}{rgb}{1.000000,0.705882,0.509804}%
\pgfsetfillcolor{currentfill}%
\pgfsetlinewidth{0.481800pt}%
\definecolor{currentstroke}{rgb}{1.000000,1.000000,1.000000}%
\pgfsetstrokecolor{currentstroke}%
\pgfsetdash{}{0pt}%
\pgfpathmoveto{\pgfqpoint{4.061550in}{0.813270in}}%
\pgfpathcurveto{\pgfqpoint{4.072601in}{0.813270in}}{\pgfqpoint{4.083200in}{0.817660in}}{\pgfqpoint{4.091013in}{0.825474in}}%
\pgfpathcurveto{\pgfqpoint{4.098827in}{0.833287in}}{\pgfqpoint{4.103217in}{0.843886in}}{\pgfqpoint{4.103217in}{0.854936in}}%
\pgfpathcurveto{\pgfqpoint{4.103217in}{0.865987in}}{\pgfqpoint{4.098827in}{0.876586in}}{\pgfqpoint{4.091013in}{0.884399in}}%
\pgfpathcurveto{\pgfqpoint{4.083200in}{0.892213in}}{\pgfqpoint{4.072601in}{0.896603in}}{\pgfqpoint{4.061550in}{0.896603in}}%
\pgfpathcurveto{\pgfqpoint{4.050500in}{0.896603in}}{\pgfqpoint{4.039901in}{0.892213in}}{\pgfqpoint{4.032088in}{0.884399in}}%
\pgfpathcurveto{\pgfqpoint{4.024274in}{0.876586in}}{\pgfqpoint{4.019884in}{0.865987in}}{\pgfqpoint{4.019884in}{0.854936in}}%
\pgfpathcurveto{\pgfqpoint{4.019884in}{0.843886in}}{\pgfqpoint{4.024274in}{0.833287in}}{\pgfqpoint{4.032088in}{0.825474in}}%
\pgfpathcurveto{\pgfqpoint{4.039901in}{0.817660in}}{\pgfqpoint{4.050500in}{0.813270in}}{\pgfqpoint{4.061550in}{0.813270in}}%
\pgfpathclose%
\pgfusepath{stroke,fill}%
\end{pgfscope}%
\begin{pgfscope}%
\pgfpathrectangle{\pgfqpoint{0.570343in}{0.331635in}}{\pgfqpoint{9.300000in}{7.700000in}}%
\pgfusepath{clip}%
\pgfsetbuttcap%
\pgfsetroundjoin%
\definecolor{currentfill}{rgb}{1.000000,0.705882,0.509804}%
\pgfsetfillcolor{currentfill}%
\pgfsetlinewidth{0.481800pt}%
\definecolor{currentstroke}{rgb}{1.000000,1.000000,1.000000}%
\pgfsetstrokecolor{currentstroke}%
\pgfsetdash{}{0pt}%
\pgfpathmoveto{\pgfqpoint{7.863027in}{5.764293in}}%
\pgfpathcurveto{\pgfqpoint{7.874077in}{5.764293in}}{\pgfqpoint{7.884676in}{5.768683in}}{\pgfqpoint{7.892490in}{5.776497in}}%
\pgfpathcurveto{\pgfqpoint{7.900303in}{5.784310in}}{\pgfqpoint{7.904694in}{5.794909in}}{\pgfqpoint{7.904694in}{5.805960in}}%
\pgfpathcurveto{\pgfqpoint{7.904694in}{5.817010in}}{\pgfqpoint{7.900303in}{5.827609in}}{\pgfqpoint{7.892490in}{5.835422in}}%
\pgfpathcurveto{\pgfqpoint{7.884676in}{5.843236in}}{\pgfqpoint{7.874077in}{5.847626in}}{\pgfqpoint{7.863027in}{5.847626in}}%
\pgfpathcurveto{\pgfqpoint{7.851977in}{5.847626in}}{\pgfqpoint{7.841378in}{5.843236in}}{\pgfqpoint{7.833564in}{5.835422in}}%
\pgfpathcurveto{\pgfqpoint{7.825750in}{5.827609in}}{\pgfqpoint{7.821360in}{5.817010in}}{\pgfqpoint{7.821360in}{5.805960in}}%
\pgfpathcurveto{\pgfqpoint{7.821360in}{5.794909in}}{\pgfqpoint{7.825750in}{5.784310in}}{\pgfqpoint{7.833564in}{5.776497in}}%
\pgfpathcurveto{\pgfqpoint{7.841378in}{5.768683in}}{\pgfqpoint{7.851977in}{5.764293in}}{\pgfqpoint{7.863027in}{5.764293in}}%
\pgfpathclose%
\pgfusepath{stroke,fill}%
\end{pgfscope}%
\begin{pgfscope}%
\pgfpathrectangle{\pgfqpoint{0.570343in}{0.331635in}}{\pgfqpoint{9.300000in}{7.700000in}}%
\pgfusepath{clip}%
\pgfsetbuttcap%
\pgfsetroundjoin%
\definecolor{currentfill}{rgb}{1.000000,0.705882,0.509804}%
\pgfsetfillcolor{currentfill}%
\pgfsetlinewidth{0.481800pt}%
\definecolor{currentstroke}{rgb}{1.000000,1.000000,1.000000}%
\pgfsetstrokecolor{currentstroke}%
\pgfsetdash{}{0pt}%
\pgfpathmoveto{\pgfqpoint{6.615719in}{2.465016in}}%
\pgfpathcurveto{\pgfqpoint{6.626770in}{2.465016in}}{\pgfqpoint{6.637369in}{2.469406in}}{\pgfqpoint{6.645182in}{2.477220in}}%
\pgfpathcurveto{\pgfqpoint{6.652996in}{2.485034in}}{\pgfqpoint{6.657386in}{2.495633in}}{\pgfqpoint{6.657386in}{2.506683in}}%
\pgfpathcurveto{\pgfqpoint{6.657386in}{2.517733in}}{\pgfqpoint{6.652996in}{2.528332in}}{\pgfqpoint{6.645182in}{2.536146in}}%
\pgfpathcurveto{\pgfqpoint{6.637369in}{2.543959in}}{\pgfqpoint{6.626770in}{2.548350in}}{\pgfqpoint{6.615719in}{2.548350in}}%
\pgfpathcurveto{\pgfqpoint{6.604669in}{2.548350in}}{\pgfqpoint{6.594070in}{2.543959in}}{\pgfqpoint{6.586257in}{2.536146in}}%
\pgfpathcurveto{\pgfqpoint{6.578443in}{2.528332in}}{\pgfqpoint{6.574053in}{2.517733in}}{\pgfqpoint{6.574053in}{2.506683in}}%
\pgfpathcurveto{\pgfqpoint{6.574053in}{2.495633in}}{\pgfqpoint{6.578443in}{2.485034in}}{\pgfqpoint{6.586257in}{2.477220in}}%
\pgfpathcurveto{\pgfqpoint{6.594070in}{2.469406in}}{\pgfqpoint{6.604669in}{2.465016in}}{\pgfqpoint{6.615719in}{2.465016in}}%
\pgfpathclose%
\pgfusepath{stroke,fill}%
\end{pgfscope}%
\begin{pgfscope}%
\pgfpathrectangle{\pgfqpoint{0.570343in}{0.331635in}}{\pgfqpoint{9.300000in}{7.700000in}}%
\pgfusepath{clip}%
\pgfsetbuttcap%
\pgfsetroundjoin%
\definecolor{currentfill}{rgb}{1.000000,0.705882,0.509804}%
\pgfsetfillcolor{currentfill}%
\pgfsetlinewidth{0.481800pt}%
\definecolor{currentstroke}{rgb}{1.000000,1.000000,1.000000}%
\pgfsetstrokecolor{currentstroke}%
\pgfsetdash{}{0pt}%
\pgfpathmoveto{\pgfqpoint{3.036089in}{4.902839in}}%
\pgfpathcurveto{\pgfqpoint{3.047139in}{4.902839in}}{\pgfqpoint{3.057738in}{4.907229in}}{\pgfqpoint{3.065551in}{4.915043in}}%
\pgfpathcurveto{\pgfqpoint{3.073365in}{4.922856in}}{\pgfqpoint{3.077755in}{4.933455in}}{\pgfqpoint{3.077755in}{4.944505in}}%
\pgfpathcurveto{\pgfqpoint{3.077755in}{4.955555in}}{\pgfqpoint{3.073365in}{4.966155in}}{\pgfqpoint{3.065551in}{4.973968in}}%
\pgfpathcurveto{\pgfqpoint{3.057738in}{4.981782in}}{\pgfqpoint{3.047139in}{4.986172in}}{\pgfqpoint{3.036089in}{4.986172in}}%
\pgfpathcurveto{\pgfqpoint{3.025038in}{4.986172in}}{\pgfqpoint{3.014439in}{4.981782in}}{\pgfqpoint{3.006626in}{4.973968in}}%
\pgfpathcurveto{\pgfqpoint{2.998812in}{4.966155in}}{\pgfqpoint{2.994422in}{4.955555in}}{\pgfqpoint{2.994422in}{4.944505in}}%
\pgfpathcurveto{\pgfqpoint{2.994422in}{4.933455in}}{\pgfqpoint{2.998812in}{4.922856in}}{\pgfqpoint{3.006626in}{4.915043in}}%
\pgfpathcurveto{\pgfqpoint{3.014439in}{4.907229in}}{\pgfqpoint{3.025038in}{4.902839in}}{\pgfqpoint{3.036089in}{4.902839in}}%
\pgfpathclose%
\pgfusepath{stroke,fill}%
\end{pgfscope}%
\begin{pgfscope}%
\pgfpathrectangle{\pgfqpoint{0.570343in}{0.331635in}}{\pgfqpoint{9.300000in}{7.700000in}}%
\pgfusepath{clip}%
\pgfsetbuttcap%
\pgfsetroundjoin%
\definecolor{currentfill}{rgb}{1.000000,0.705882,0.509804}%
\pgfsetfillcolor{currentfill}%
\pgfsetlinewidth{0.481800pt}%
\definecolor{currentstroke}{rgb}{1.000000,1.000000,1.000000}%
\pgfsetstrokecolor{currentstroke}%
\pgfsetdash{}{0pt}%
\pgfpathmoveto{\pgfqpoint{9.333987in}{4.576789in}}%
\pgfpathcurveto{\pgfqpoint{9.345037in}{4.576789in}}{\pgfqpoint{9.355636in}{4.581180in}}{\pgfqpoint{9.363450in}{4.588993in}}%
\pgfpathcurveto{\pgfqpoint{9.371264in}{4.596807in}}{\pgfqpoint{9.375654in}{4.607406in}}{\pgfqpoint{9.375654in}{4.618456in}}%
\pgfpathcurveto{\pgfqpoint{9.375654in}{4.629506in}}{\pgfqpoint{9.371264in}{4.640105in}}{\pgfqpoint{9.363450in}{4.647919in}}%
\pgfpathcurveto{\pgfqpoint{9.355636in}{4.655733in}}{\pgfqpoint{9.345037in}{4.660123in}}{\pgfqpoint{9.333987in}{4.660123in}}%
\pgfpathcurveto{\pgfqpoint{9.322937in}{4.660123in}}{\pgfqpoint{9.312338in}{4.655733in}}{\pgfqpoint{9.304524in}{4.647919in}}%
\pgfpathcurveto{\pgfqpoint{9.296711in}{4.640105in}}{\pgfqpoint{9.292321in}{4.629506in}}{\pgfqpoint{9.292321in}{4.618456in}}%
\pgfpathcurveto{\pgfqpoint{9.292321in}{4.607406in}}{\pgfqpoint{9.296711in}{4.596807in}}{\pgfqpoint{9.304524in}{4.588993in}}%
\pgfpathcurveto{\pgfqpoint{9.312338in}{4.581180in}}{\pgfqpoint{9.322937in}{4.576789in}}{\pgfqpoint{9.333987in}{4.576789in}}%
\pgfpathclose%
\pgfusepath{stroke,fill}%
\end{pgfscope}%
\begin{pgfscope}%
\pgfpathrectangle{\pgfqpoint{0.570343in}{0.331635in}}{\pgfqpoint{9.300000in}{7.700000in}}%
\pgfusepath{clip}%
\pgfsetbuttcap%
\pgfsetroundjoin%
\definecolor{currentfill}{rgb}{1.000000,0.705882,0.509804}%
\pgfsetfillcolor{currentfill}%
\pgfsetlinewidth{0.481800pt}%
\definecolor{currentstroke}{rgb}{1.000000,1.000000,1.000000}%
\pgfsetstrokecolor{currentstroke}%
\pgfsetdash{}{0pt}%
\pgfpathmoveto{\pgfqpoint{5.652131in}{2.830374in}}%
\pgfpathcurveto{\pgfqpoint{5.663181in}{2.830374in}}{\pgfqpoint{5.673780in}{2.834765in}}{\pgfqpoint{5.681594in}{2.842578in}}%
\pgfpathcurveto{\pgfqpoint{5.689407in}{2.850392in}}{\pgfqpoint{5.693798in}{2.860991in}}{\pgfqpoint{5.693798in}{2.872041in}}%
\pgfpathcurveto{\pgfqpoint{5.693798in}{2.883091in}}{\pgfqpoint{5.689407in}{2.893690in}}{\pgfqpoint{5.681594in}{2.901504in}}%
\pgfpathcurveto{\pgfqpoint{5.673780in}{2.909318in}}{\pgfqpoint{5.663181in}{2.913708in}}{\pgfqpoint{5.652131in}{2.913708in}}%
\pgfpathcurveto{\pgfqpoint{5.641081in}{2.913708in}}{\pgfqpoint{5.630482in}{2.909318in}}{\pgfqpoint{5.622668in}{2.901504in}}%
\pgfpathcurveto{\pgfqpoint{5.614855in}{2.893690in}}{\pgfqpoint{5.610464in}{2.883091in}}{\pgfqpoint{5.610464in}{2.872041in}}%
\pgfpathcurveto{\pgfqpoint{5.610464in}{2.860991in}}{\pgfqpoint{5.614855in}{2.850392in}}{\pgfqpoint{5.622668in}{2.842578in}}%
\pgfpathcurveto{\pgfqpoint{5.630482in}{2.834765in}}{\pgfqpoint{5.641081in}{2.830374in}}{\pgfqpoint{5.652131in}{2.830374in}}%
\pgfpathclose%
\pgfusepath{stroke,fill}%
\end{pgfscope}%
\begin{pgfscope}%
\pgfpathrectangle{\pgfqpoint{0.570343in}{0.331635in}}{\pgfqpoint{9.300000in}{7.700000in}}%
\pgfusepath{clip}%
\pgfsetbuttcap%
\pgfsetroundjoin%
\definecolor{currentfill}{rgb}{0.631373,0.788235,0.956863}%
\pgfsetfillcolor{currentfill}%
\pgfsetlinewidth{1.003750pt}%
\definecolor{currentstroke}{rgb}{0.631373,0.788235,0.956863}%
\pgfsetstrokecolor{currentstroke}%
\pgfsetdash{}{0pt}%
\pgfsys@defobject{currentmarker}{\pgfqpoint{-0.041667in}{-0.041667in}}{\pgfqpoint{0.041667in}{0.041667in}}{%
\pgfpathmoveto{\pgfqpoint{0.000000in}{-0.041667in}}%
\pgfpathcurveto{\pgfqpoint{0.011050in}{-0.041667in}}{\pgfqpoint{0.021649in}{-0.037276in}}{\pgfqpoint{0.029463in}{-0.029463in}}%
\pgfpathcurveto{\pgfqpoint{0.037276in}{-0.021649in}}{\pgfqpoint{0.041667in}{-0.011050in}}{\pgfqpoint{0.041667in}{0.000000in}}%
\pgfpathcurveto{\pgfqpoint{0.041667in}{0.011050in}}{\pgfqpoint{0.037276in}{0.021649in}}{\pgfqpoint{0.029463in}{0.029463in}}%
\pgfpathcurveto{\pgfqpoint{0.021649in}{0.037276in}}{\pgfqpoint{0.011050in}{0.041667in}}{\pgfqpoint{0.000000in}{0.041667in}}%
\pgfpathcurveto{\pgfqpoint{-0.011050in}{0.041667in}}{\pgfqpoint{-0.021649in}{0.037276in}}{\pgfqpoint{-0.029463in}{0.029463in}}%
\pgfpathcurveto{\pgfqpoint{-0.037276in}{0.021649in}}{\pgfqpoint{-0.041667in}{0.011050in}}{\pgfqpoint{-0.041667in}{0.000000in}}%
\pgfpathcurveto{\pgfqpoint{-0.041667in}{-0.011050in}}{\pgfqpoint{-0.037276in}{-0.021649in}}{\pgfqpoint{-0.029463in}{-0.029463in}}%
\pgfpathcurveto{\pgfqpoint{-0.021649in}{-0.037276in}}{\pgfqpoint{-0.011050in}{-0.041667in}}{\pgfqpoint{0.000000in}{-0.041667in}}%
\pgfpathclose%
\pgfusepath{stroke,fill}%
}%
\end{pgfscope}%
\begin{pgfscope}%
\pgfpathrectangle{\pgfqpoint{0.570343in}{0.331635in}}{\pgfqpoint{9.300000in}{7.700000in}}%
\pgfusepath{clip}%
\pgfsetbuttcap%
\pgfsetroundjoin%
\definecolor{currentfill}{rgb}{1.000000,0.705882,0.509804}%
\pgfsetfillcolor{currentfill}%
\pgfsetlinewidth{1.003750pt}%
\definecolor{currentstroke}{rgb}{1.000000,0.705882,0.509804}%
\pgfsetstrokecolor{currentstroke}%
\pgfsetdash{}{0pt}%
\pgfsys@defobject{currentmarker}{\pgfqpoint{-0.041667in}{-0.041667in}}{\pgfqpoint{0.041667in}{0.041667in}}{%
\pgfpathmoveto{\pgfqpoint{0.000000in}{-0.041667in}}%
\pgfpathcurveto{\pgfqpoint{0.011050in}{-0.041667in}}{\pgfqpoint{0.021649in}{-0.037276in}}{\pgfqpoint{0.029463in}{-0.029463in}}%
\pgfpathcurveto{\pgfqpoint{0.037276in}{-0.021649in}}{\pgfqpoint{0.041667in}{-0.011050in}}{\pgfqpoint{0.041667in}{0.000000in}}%
\pgfpathcurveto{\pgfqpoint{0.041667in}{0.011050in}}{\pgfqpoint{0.037276in}{0.021649in}}{\pgfqpoint{0.029463in}{0.029463in}}%
\pgfpathcurveto{\pgfqpoint{0.021649in}{0.037276in}}{\pgfqpoint{0.011050in}{0.041667in}}{\pgfqpoint{0.000000in}{0.041667in}}%
\pgfpathcurveto{\pgfqpoint{-0.011050in}{0.041667in}}{\pgfqpoint{-0.021649in}{0.037276in}}{\pgfqpoint{-0.029463in}{0.029463in}}%
\pgfpathcurveto{\pgfqpoint{-0.037276in}{0.021649in}}{\pgfqpoint{-0.041667in}{0.011050in}}{\pgfqpoint{-0.041667in}{0.000000in}}%
\pgfpathcurveto{\pgfqpoint{-0.041667in}{-0.011050in}}{\pgfqpoint{-0.037276in}{-0.021649in}}{\pgfqpoint{-0.029463in}{-0.029463in}}%
\pgfpathcurveto{\pgfqpoint{-0.021649in}{-0.037276in}}{\pgfqpoint{-0.011050in}{-0.041667in}}{\pgfqpoint{0.000000in}{-0.041667in}}%
\pgfpathclose%
\pgfusepath{stroke,fill}%
}%
\end{pgfscope}%
\begin{pgfscope}%
\pgfsetbuttcap%
\pgfsetroundjoin%
\definecolor{currentfill}{rgb}{0.000000,0.000000,0.000000}%
\pgfsetfillcolor{currentfill}%
\pgfsetlinewidth{0.803000pt}%
\definecolor{currentstroke}{rgb}{0.000000,0.000000,0.000000}%
\pgfsetstrokecolor{currentstroke}%
\pgfsetdash{}{0pt}%
\pgfsys@defobject{currentmarker}{\pgfqpoint{0.000000in}{-0.048611in}}{\pgfqpoint{0.000000in}{0.000000in}}{%
\pgfpathmoveto{\pgfqpoint{0.000000in}{0.000000in}}%
\pgfpathlineto{\pgfqpoint{0.000000in}{-0.048611in}}%
\pgfusepath{stroke,fill}%
}%
\begin{pgfscope}%
\pgfsys@transformshift{0.607701in}{0.331635in}%
\pgfsys@useobject{currentmarker}{}%
\end{pgfscope}%
\end{pgfscope}%
\begin{pgfscope}%
\definecolor{textcolor}{rgb}{0.000000,0.000000,0.000000}%
\pgfsetstrokecolor{textcolor}%
\pgfsetfillcolor{textcolor}%
\pgftext[x=0.607701in,y=0.234413in,,top]{\color{textcolor}\sffamily\fontsize{10.000000}{12.000000}\selectfont \ensuremath{-}100}%
\end{pgfscope}%
\begin{pgfscope}%
\pgfsetbuttcap%
\pgfsetroundjoin%
\definecolor{currentfill}{rgb}{0.000000,0.000000,0.000000}%
\pgfsetfillcolor{currentfill}%
\pgfsetlinewidth{0.803000pt}%
\definecolor{currentstroke}{rgb}{0.000000,0.000000,0.000000}%
\pgfsetstrokecolor{currentstroke}%
\pgfsetdash{}{0pt}%
\pgfsys@defobject{currentmarker}{\pgfqpoint{0.000000in}{-0.048611in}}{\pgfqpoint{0.000000in}{0.000000in}}{%
\pgfpathmoveto{\pgfqpoint{0.000000in}{0.000000in}}%
\pgfpathlineto{\pgfqpoint{0.000000in}{-0.048611in}}%
\pgfusepath{stroke,fill}%
}%
\begin{pgfscope}%
\pgfsys@transformshift{1.748621in}{0.331635in}%
\pgfsys@useobject{currentmarker}{}%
\end{pgfscope}%
\end{pgfscope}%
\begin{pgfscope}%
\definecolor{textcolor}{rgb}{0.000000,0.000000,0.000000}%
\pgfsetstrokecolor{textcolor}%
\pgfsetfillcolor{textcolor}%
\pgftext[x=1.748621in,y=0.234413in,,top]{\color{textcolor}\sffamily\fontsize{10.000000}{12.000000}\selectfont \ensuremath{-}75}%
\end{pgfscope}%
\begin{pgfscope}%
\pgfsetbuttcap%
\pgfsetroundjoin%
\definecolor{currentfill}{rgb}{0.000000,0.000000,0.000000}%
\pgfsetfillcolor{currentfill}%
\pgfsetlinewidth{0.803000pt}%
\definecolor{currentstroke}{rgb}{0.000000,0.000000,0.000000}%
\pgfsetstrokecolor{currentstroke}%
\pgfsetdash{}{0pt}%
\pgfsys@defobject{currentmarker}{\pgfqpoint{0.000000in}{-0.048611in}}{\pgfqpoint{0.000000in}{0.000000in}}{%
\pgfpathmoveto{\pgfqpoint{0.000000in}{0.000000in}}%
\pgfpathlineto{\pgfqpoint{0.000000in}{-0.048611in}}%
\pgfusepath{stroke,fill}%
}%
\begin{pgfscope}%
\pgfsys@transformshift{2.889540in}{0.331635in}%
\pgfsys@useobject{currentmarker}{}%
\end{pgfscope}%
\end{pgfscope}%
\begin{pgfscope}%
\definecolor{textcolor}{rgb}{0.000000,0.000000,0.000000}%
\pgfsetstrokecolor{textcolor}%
\pgfsetfillcolor{textcolor}%
\pgftext[x=2.889540in,y=0.234413in,,top]{\color{textcolor}\sffamily\fontsize{10.000000}{12.000000}\selectfont \ensuremath{-}50}%
\end{pgfscope}%
\begin{pgfscope}%
\pgfsetbuttcap%
\pgfsetroundjoin%
\definecolor{currentfill}{rgb}{0.000000,0.000000,0.000000}%
\pgfsetfillcolor{currentfill}%
\pgfsetlinewidth{0.803000pt}%
\definecolor{currentstroke}{rgb}{0.000000,0.000000,0.000000}%
\pgfsetstrokecolor{currentstroke}%
\pgfsetdash{}{0pt}%
\pgfsys@defobject{currentmarker}{\pgfqpoint{0.000000in}{-0.048611in}}{\pgfqpoint{0.000000in}{0.000000in}}{%
\pgfpathmoveto{\pgfqpoint{0.000000in}{0.000000in}}%
\pgfpathlineto{\pgfqpoint{0.000000in}{-0.048611in}}%
\pgfusepath{stroke,fill}%
}%
\begin{pgfscope}%
\pgfsys@transformshift{4.030460in}{0.331635in}%
\pgfsys@useobject{currentmarker}{}%
\end{pgfscope}%
\end{pgfscope}%
\begin{pgfscope}%
\definecolor{textcolor}{rgb}{0.000000,0.000000,0.000000}%
\pgfsetstrokecolor{textcolor}%
\pgfsetfillcolor{textcolor}%
\pgftext[x=4.030460in,y=0.234413in,,top]{\color{textcolor}\sffamily\fontsize{10.000000}{12.000000}\selectfont \ensuremath{-}25}%
\end{pgfscope}%
\begin{pgfscope}%
\pgfsetbuttcap%
\pgfsetroundjoin%
\definecolor{currentfill}{rgb}{0.000000,0.000000,0.000000}%
\pgfsetfillcolor{currentfill}%
\pgfsetlinewidth{0.803000pt}%
\definecolor{currentstroke}{rgb}{0.000000,0.000000,0.000000}%
\pgfsetstrokecolor{currentstroke}%
\pgfsetdash{}{0pt}%
\pgfsys@defobject{currentmarker}{\pgfqpoint{0.000000in}{-0.048611in}}{\pgfqpoint{0.000000in}{0.000000in}}{%
\pgfpathmoveto{\pgfqpoint{0.000000in}{0.000000in}}%
\pgfpathlineto{\pgfqpoint{0.000000in}{-0.048611in}}%
\pgfusepath{stroke,fill}%
}%
\begin{pgfscope}%
\pgfsys@transformshift{5.171380in}{0.331635in}%
\pgfsys@useobject{currentmarker}{}%
\end{pgfscope}%
\end{pgfscope}%
\begin{pgfscope}%
\definecolor{textcolor}{rgb}{0.000000,0.000000,0.000000}%
\pgfsetstrokecolor{textcolor}%
\pgfsetfillcolor{textcolor}%
\pgftext[x=5.171380in,y=0.234413in,,top]{\color{textcolor}\sffamily\fontsize{10.000000}{12.000000}\selectfont 0}%
\end{pgfscope}%
\begin{pgfscope}%
\pgfsetbuttcap%
\pgfsetroundjoin%
\definecolor{currentfill}{rgb}{0.000000,0.000000,0.000000}%
\pgfsetfillcolor{currentfill}%
\pgfsetlinewidth{0.803000pt}%
\definecolor{currentstroke}{rgb}{0.000000,0.000000,0.000000}%
\pgfsetstrokecolor{currentstroke}%
\pgfsetdash{}{0pt}%
\pgfsys@defobject{currentmarker}{\pgfqpoint{0.000000in}{-0.048611in}}{\pgfqpoint{0.000000in}{0.000000in}}{%
\pgfpathmoveto{\pgfqpoint{0.000000in}{0.000000in}}%
\pgfpathlineto{\pgfqpoint{0.000000in}{-0.048611in}}%
\pgfusepath{stroke,fill}%
}%
\begin{pgfscope}%
\pgfsys@transformshift{6.312299in}{0.331635in}%
\pgfsys@useobject{currentmarker}{}%
\end{pgfscope}%
\end{pgfscope}%
\begin{pgfscope}%
\definecolor{textcolor}{rgb}{0.000000,0.000000,0.000000}%
\pgfsetstrokecolor{textcolor}%
\pgfsetfillcolor{textcolor}%
\pgftext[x=6.312299in,y=0.234413in,,top]{\color{textcolor}\sffamily\fontsize{10.000000}{12.000000}\selectfont 25}%
\end{pgfscope}%
\begin{pgfscope}%
\pgfsetbuttcap%
\pgfsetroundjoin%
\definecolor{currentfill}{rgb}{0.000000,0.000000,0.000000}%
\pgfsetfillcolor{currentfill}%
\pgfsetlinewidth{0.803000pt}%
\definecolor{currentstroke}{rgb}{0.000000,0.000000,0.000000}%
\pgfsetstrokecolor{currentstroke}%
\pgfsetdash{}{0pt}%
\pgfsys@defobject{currentmarker}{\pgfqpoint{0.000000in}{-0.048611in}}{\pgfqpoint{0.000000in}{0.000000in}}{%
\pgfpathmoveto{\pgfqpoint{0.000000in}{0.000000in}}%
\pgfpathlineto{\pgfqpoint{0.000000in}{-0.048611in}}%
\pgfusepath{stroke,fill}%
}%
\begin{pgfscope}%
\pgfsys@transformshift{7.453219in}{0.331635in}%
\pgfsys@useobject{currentmarker}{}%
\end{pgfscope}%
\end{pgfscope}%
\begin{pgfscope}%
\definecolor{textcolor}{rgb}{0.000000,0.000000,0.000000}%
\pgfsetstrokecolor{textcolor}%
\pgfsetfillcolor{textcolor}%
\pgftext[x=7.453219in,y=0.234413in,,top]{\color{textcolor}\sffamily\fontsize{10.000000}{12.000000}\selectfont 50}%
\end{pgfscope}%
\begin{pgfscope}%
\pgfsetbuttcap%
\pgfsetroundjoin%
\definecolor{currentfill}{rgb}{0.000000,0.000000,0.000000}%
\pgfsetfillcolor{currentfill}%
\pgfsetlinewidth{0.803000pt}%
\definecolor{currentstroke}{rgb}{0.000000,0.000000,0.000000}%
\pgfsetstrokecolor{currentstroke}%
\pgfsetdash{}{0pt}%
\pgfsys@defobject{currentmarker}{\pgfqpoint{0.000000in}{-0.048611in}}{\pgfqpoint{0.000000in}{0.000000in}}{%
\pgfpathmoveto{\pgfqpoint{0.000000in}{0.000000in}}%
\pgfpathlineto{\pgfqpoint{0.000000in}{-0.048611in}}%
\pgfusepath{stroke,fill}%
}%
\begin{pgfscope}%
\pgfsys@transformshift{8.594139in}{0.331635in}%
\pgfsys@useobject{currentmarker}{}%
\end{pgfscope}%
\end{pgfscope}%
\begin{pgfscope}%
\definecolor{textcolor}{rgb}{0.000000,0.000000,0.000000}%
\pgfsetstrokecolor{textcolor}%
\pgfsetfillcolor{textcolor}%
\pgftext[x=8.594139in,y=0.234413in,,top]{\color{textcolor}\sffamily\fontsize{10.000000}{12.000000}\selectfont 75}%
\end{pgfscope}%
\begin{pgfscope}%
\pgfsetbuttcap%
\pgfsetroundjoin%
\definecolor{currentfill}{rgb}{0.000000,0.000000,0.000000}%
\pgfsetfillcolor{currentfill}%
\pgfsetlinewidth{0.803000pt}%
\definecolor{currentstroke}{rgb}{0.000000,0.000000,0.000000}%
\pgfsetstrokecolor{currentstroke}%
\pgfsetdash{}{0pt}%
\pgfsys@defobject{currentmarker}{\pgfqpoint{0.000000in}{-0.048611in}}{\pgfqpoint{0.000000in}{0.000000in}}{%
\pgfpathmoveto{\pgfqpoint{0.000000in}{0.000000in}}%
\pgfpathlineto{\pgfqpoint{0.000000in}{-0.048611in}}%
\pgfusepath{stroke,fill}%
}%
\begin{pgfscope}%
\pgfsys@transformshift{9.735058in}{0.331635in}%
\pgfsys@useobject{currentmarker}{}%
\end{pgfscope}%
\end{pgfscope}%
\begin{pgfscope}%
\definecolor{textcolor}{rgb}{0.000000,0.000000,0.000000}%
\pgfsetstrokecolor{textcolor}%
\pgfsetfillcolor{textcolor}%
\pgftext[x=9.735058in,y=0.234413in,,top]{\color{textcolor}\sffamily\fontsize{10.000000}{12.000000}\selectfont 100}%
\end{pgfscope}%
\begin{pgfscope}%
\pgfsetbuttcap%
\pgfsetroundjoin%
\definecolor{currentfill}{rgb}{0.000000,0.000000,0.000000}%
\pgfsetfillcolor{currentfill}%
\pgfsetlinewidth{0.803000pt}%
\definecolor{currentstroke}{rgb}{0.000000,0.000000,0.000000}%
\pgfsetstrokecolor{currentstroke}%
\pgfsetdash{}{0pt}%
\pgfsys@defobject{currentmarker}{\pgfqpoint{-0.048611in}{0.000000in}}{\pgfqpoint{-0.000000in}{0.000000in}}{%
\pgfpathmoveto{\pgfqpoint{-0.000000in}{0.000000in}}%
\pgfpathlineto{\pgfqpoint{-0.048611in}{0.000000in}}%
\pgfusepath{stroke,fill}%
}%
\begin{pgfscope}%
\pgfsys@transformshift{0.570343in}{0.590557in}%
\pgfsys@useobject{currentmarker}{}%
\end{pgfscope}%
\end{pgfscope}%
\begin{pgfscope}%
\definecolor{textcolor}{rgb}{0.000000,0.000000,0.000000}%
\pgfsetstrokecolor{textcolor}%
\pgfsetfillcolor{textcolor}%
\pgftext[x=0.100000in, y=0.537796in, left, base]{\color{textcolor}\sffamily\fontsize{10.000000}{12.000000}\selectfont \ensuremath{-}100}%
\end{pgfscope}%
\begin{pgfscope}%
\pgfsetbuttcap%
\pgfsetroundjoin%
\definecolor{currentfill}{rgb}{0.000000,0.000000,0.000000}%
\pgfsetfillcolor{currentfill}%
\pgfsetlinewidth{0.803000pt}%
\definecolor{currentstroke}{rgb}{0.000000,0.000000,0.000000}%
\pgfsetstrokecolor{currentstroke}%
\pgfsetdash{}{0pt}%
\pgfsys@defobject{currentmarker}{\pgfqpoint{-0.048611in}{0.000000in}}{\pgfqpoint{-0.000000in}{0.000000in}}{%
\pgfpathmoveto{\pgfqpoint{-0.000000in}{0.000000in}}%
\pgfpathlineto{\pgfqpoint{-0.048611in}{0.000000in}}%
\pgfusepath{stroke,fill}%
}%
\begin{pgfscope}%
\pgfsys@transformshift{0.570343in}{1.514832in}%
\pgfsys@useobject{currentmarker}{}%
\end{pgfscope}%
\end{pgfscope}%
\begin{pgfscope}%
\definecolor{textcolor}{rgb}{0.000000,0.000000,0.000000}%
\pgfsetstrokecolor{textcolor}%
\pgfsetfillcolor{textcolor}%
\pgftext[x=0.188365in, y=1.462070in, left, base]{\color{textcolor}\sffamily\fontsize{10.000000}{12.000000}\selectfont \ensuremath{-}75}%
\end{pgfscope}%
\begin{pgfscope}%
\pgfsetbuttcap%
\pgfsetroundjoin%
\definecolor{currentfill}{rgb}{0.000000,0.000000,0.000000}%
\pgfsetfillcolor{currentfill}%
\pgfsetlinewidth{0.803000pt}%
\definecolor{currentstroke}{rgb}{0.000000,0.000000,0.000000}%
\pgfsetstrokecolor{currentstroke}%
\pgfsetdash{}{0pt}%
\pgfsys@defobject{currentmarker}{\pgfqpoint{-0.048611in}{0.000000in}}{\pgfqpoint{-0.000000in}{0.000000in}}{%
\pgfpathmoveto{\pgfqpoint{-0.000000in}{0.000000in}}%
\pgfpathlineto{\pgfqpoint{-0.048611in}{0.000000in}}%
\pgfusepath{stroke,fill}%
}%
\begin{pgfscope}%
\pgfsys@transformshift{0.570343in}{2.439106in}%
\pgfsys@useobject{currentmarker}{}%
\end{pgfscope}%
\end{pgfscope}%
\begin{pgfscope}%
\definecolor{textcolor}{rgb}{0.000000,0.000000,0.000000}%
\pgfsetstrokecolor{textcolor}%
\pgfsetfillcolor{textcolor}%
\pgftext[x=0.188365in, y=2.386344in, left, base]{\color{textcolor}\sffamily\fontsize{10.000000}{12.000000}\selectfont \ensuremath{-}50}%
\end{pgfscope}%
\begin{pgfscope}%
\pgfsetbuttcap%
\pgfsetroundjoin%
\definecolor{currentfill}{rgb}{0.000000,0.000000,0.000000}%
\pgfsetfillcolor{currentfill}%
\pgfsetlinewidth{0.803000pt}%
\definecolor{currentstroke}{rgb}{0.000000,0.000000,0.000000}%
\pgfsetstrokecolor{currentstroke}%
\pgfsetdash{}{0pt}%
\pgfsys@defobject{currentmarker}{\pgfqpoint{-0.048611in}{0.000000in}}{\pgfqpoint{-0.000000in}{0.000000in}}{%
\pgfpathmoveto{\pgfqpoint{-0.000000in}{0.000000in}}%
\pgfpathlineto{\pgfqpoint{-0.048611in}{0.000000in}}%
\pgfusepath{stroke,fill}%
}%
\begin{pgfscope}%
\pgfsys@transformshift{0.570343in}{3.363380in}%
\pgfsys@useobject{currentmarker}{}%
\end{pgfscope}%
\end{pgfscope}%
\begin{pgfscope}%
\definecolor{textcolor}{rgb}{0.000000,0.000000,0.000000}%
\pgfsetstrokecolor{textcolor}%
\pgfsetfillcolor{textcolor}%
\pgftext[x=0.188365in, y=3.310619in, left, base]{\color{textcolor}\sffamily\fontsize{10.000000}{12.000000}\selectfont \ensuremath{-}25}%
\end{pgfscope}%
\begin{pgfscope}%
\pgfsetbuttcap%
\pgfsetroundjoin%
\definecolor{currentfill}{rgb}{0.000000,0.000000,0.000000}%
\pgfsetfillcolor{currentfill}%
\pgfsetlinewidth{0.803000pt}%
\definecolor{currentstroke}{rgb}{0.000000,0.000000,0.000000}%
\pgfsetstrokecolor{currentstroke}%
\pgfsetdash{}{0pt}%
\pgfsys@defobject{currentmarker}{\pgfqpoint{-0.048611in}{0.000000in}}{\pgfqpoint{-0.000000in}{0.000000in}}{%
\pgfpathmoveto{\pgfqpoint{-0.000000in}{0.000000in}}%
\pgfpathlineto{\pgfqpoint{-0.048611in}{0.000000in}}%
\pgfusepath{stroke,fill}%
}%
\begin{pgfscope}%
\pgfsys@transformshift{0.570343in}{4.287654in}%
\pgfsys@useobject{currentmarker}{}%
\end{pgfscope}%
\end{pgfscope}%
\begin{pgfscope}%
\definecolor{textcolor}{rgb}{0.000000,0.000000,0.000000}%
\pgfsetstrokecolor{textcolor}%
\pgfsetfillcolor{textcolor}%
\pgftext[x=0.384756in, y=4.234893in, left, base]{\color{textcolor}\sffamily\fontsize{10.000000}{12.000000}\selectfont 0}%
\end{pgfscope}%
\begin{pgfscope}%
\pgfsetbuttcap%
\pgfsetroundjoin%
\definecolor{currentfill}{rgb}{0.000000,0.000000,0.000000}%
\pgfsetfillcolor{currentfill}%
\pgfsetlinewidth{0.803000pt}%
\definecolor{currentstroke}{rgb}{0.000000,0.000000,0.000000}%
\pgfsetstrokecolor{currentstroke}%
\pgfsetdash{}{0pt}%
\pgfsys@defobject{currentmarker}{\pgfqpoint{-0.048611in}{0.000000in}}{\pgfqpoint{-0.000000in}{0.000000in}}{%
\pgfpathmoveto{\pgfqpoint{-0.000000in}{0.000000in}}%
\pgfpathlineto{\pgfqpoint{-0.048611in}{0.000000in}}%
\pgfusepath{stroke,fill}%
}%
\begin{pgfscope}%
\pgfsys@transformshift{0.570343in}{5.211929in}%
\pgfsys@useobject{currentmarker}{}%
\end{pgfscope}%
\end{pgfscope}%
\begin{pgfscope}%
\definecolor{textcolor}{rgb}{0.000000,0.000000,0.000000}%
\pgfsetstrokecolor{textcolor}%
\pgfsetfillcolor{textcolor}%
\pgftext[x=0.296390in, y=5.159167in, left, base]{\color{textcolor}\sffamily\fontsize{10.000000}{12.000000}\selectfont 25}%
\end{pgfscope}%
\begin{pgfscope}%
\pgfsetbuttcap%
\pgfsetroundjoin%
\definecolor{currentfill}{rgb}{0.000000,0.000000,0.000000}%
\pgfsetfillcolor{currentfill}%
\pgfsetlinewidth{0.803000pt}%
\definecolor{currentstroke}{rgb}{0.000000,0.000000,0.000000}%
\pgfsetstrokecolor{currentstroke}%
\pgfsetdash{}{0pt}%
\pgfsys@defobject{currentmarker}{\pgfqpoint{-0.048611in}{0.000000in}}{\pgfqpoint{-0.000000in}{0.000000in}}{%
\pgfpathmoveto{\pgfqpoint{-0.000000in}{0.000000in}}%
\pgfpathlineto{\pgfqpoint{-0.048611in}{0.000000in}}%
\pgfusepath{stroke,fill}%
}%
\begin{pgfscope}%
\pgfsys@transformshift{0.570343in}{6.136203in}%
\pgfsys@useobject{currentmarker}{}%
\end{pgfscope}%
\end{pgfscope}%
\begin{pgfscope}%
\definecolor{textcolor}{rgb}{0.000000,0.000000,0.000000}%
\pgfsetstrokecolor{textcolor}%
\pgfsetfillcolor{textcolor}%
\pgftext[x=0.296390in, y=6.083442in, left, base]{\color{textcolor}\sffamily\fontsize{10.000000}{12.000000}\selectfont 50}%
\end{pgfscope}%
\begin{pgfscope}%
\pgfsetbuttcap%
\pgfsetroundjoin%
\definecolor{currentfill}{rgb}{0.000000,0.000000,0.000000}%
\pgfsetfillcolor{currentfill}%
\pgfsetlinewidth{0.803000pt}%
\definecolor{currentstroke}{rgb}{0.000000,0.000000,0.000000}%
\pgfsetstrokecolor{currentstroke}%
\pgfsetdash{}{0pt}%
\pgfsys@defobject{currentmarker}{\pgfqpoint{-0.048611in}{0.000000in}}{\pgfqpoint{-0.000000in}{0.000000in}}{%
\pgfpathmoveto{\pgfqpoint{-0.000000in}{0.000000in}}%
\pgfpathlineto{\pgfqpoint{-0.048611in}{0.000000in}}%
\pgfusepath{stroke,fill}%
}%
\begin{pgfscope}%
\pgfsys@transformshift{0.570343in}{7.060477in}%
\pgfsys@useobject{currentmarker}{}%
\end{pgfscope}%
\end{pgfscope}%
\begin{pgfscope}%
\definecolor{textcolor}{rgb}{0.000000,0.000000,0.000000}%
\pgfsetstrokecolor{textcolor}%
\pgfsetfillcolor{textcolor}%
\pgftext[x=0.296390in, y=7.007716in, left, base]{\color{textcolor}\sffamily\fontsize{10.000000}{12.000000}\selectfont 75}%
\end{pgfscope}%
\begin{pgfscope}%
\pgfsetbuttcap%
\pgfsetroundjoin%
\definecolor{currentfill}{rgb}{0.000000,0.000000,0.000000}%
\pgfsetfillcolor{currentfill}%
\pgfsetlinewidth{0.803000pt}%
\definecolor{currentstroke}{rgb}{0.000000,0.000000,0.000000}%
\pgfsetstrokecolor{currentstroke}%
\pgfsetdash{}{0pt}%
\pgfsys@defobject{currentmarker}{\pgfqpoint{-0.048611in}{0.000000in}}{\pgfqpoint{-0.000000in}{0.000000in}}{%
\pgfpathmoveto{\pgfqpoint{-0.000000in}{0.000000in}}%
\pgfpathlineto{\pgfqpoint{-0.048611in}{0.000000in}}%
\pgfusepath{stroke,fill}%
}%
\begin{pgfscope}%
\pgfsys@transformshift{0.570343in}{7.984752in}%
\pgfsys@useobject{currentmarker}{}%
\end{pgfscope}%
\end{pgfscope}%
\begin{pgfscope}%
\definecolor{textcolor}{rgb}{0.000000,0.000000,0.000000}%
\pgfsetstrokecolor{textcolor}%
\pgfsetfillcolor{textcolor}%
\pgftext[x=0.208025in, y=7.931990in, left, base]{\color{textcolor}\sffamily\fontsize{10.000000}{12.000000}\selectfont 100}%
\end{pgfscope}%
\begin{pgfscope}%
\pgfpathrectangle{\pgfqpoint{0.570343in}{0.331635in}}{\pgfqpoint{9.300000in}{7.700000in}}%
\pgfusepath{clip}%
\pgfsetrectcap%
\pgfsetroundjoin%
\pgfsetlinewidth{1.505625pt}%
\definecolor{currentstroke}{rgb}{0.631373,0.788235,0.956863}%
\pgfsetstrokecolor{currentstroke}%
\pgfsetstrokeopacity{0.800000}%
\pgfsetdash{}{0pt}%
\pgfpathmoveto{\pgfqpoint{7.239021in}{5.114123in}}%
\pgfpathlineto{\pgfqpoint{5.077119in}{4.987631in}}%
\pgfusepath{stroke}%
\end{pgfscope}%
\begin{pgfscope}%
\pgfpathrectangle{\pgfqpoint{0.570343in}{0.331635in}}{\pgfqpoint{9.300000in}{7.700000in}}%
\pgfusepath{clip}%
\pgfsetrectcap%
\pgfsetroundjoin%
\pgfsetlinewidth{1.505625pt}%
\definecolor{currentstroke}{rgb}{0.631373,0.788235,0.956863}%
\pgfsetstrokecolor{currentstroke}%
\pgfsetstrokeopacity{0.800000}%
\pgfsetdash{}{0pt}%
\pgfpathmoveto{\pgfqpoint{6.513392in}{4.010317in}}%
\pgfpathlineto{\pgfqpoint{5.077119in}{4.987631in}}%
\pgfusepath{stroke}%
\end{pgfscope}%
\begin{pgfscope}%
\pgfpathrectangle{\pgfqpoint{0.570343in}{0.331635in}}{\pgfqpoint{9.300000in}{7.700000in}}%
\pgfusepath{clip}%
\pgfsetrectcap%
\pgfsetroundjoin%
\pgfsetlinewidth{1.505625pt}%
\definecolor{currentstroke}{rgb}{0.631373,0.788235,0.956863}%
\pgfsetstrokecolor{currentstroke}%
\pgfsetstrokeopacity{0.800000}%
\pgfsetdash{}{0pt}%
\pgfpathmoveto{\pgfqpoint{2.367651in}{5.521715in}}%
\pgfpathlineto{\pgfqpoint{5.077119in}{4.987631in}}%
\pgfusepath{stroke}%
\end{pgfscope}%
\begin{pgfscope}%
\pgfpathrectangle{\pgfqpoint{0.570343in}{0.331635in}}{\pgfqpoint{9.300000in}{7.700000in}}%
\pgfusepath{clip}%
\pgfsetrectcap%
\pgfsetroundjoin%
\pgfsetlinewidth{1.505625pt}%
\definecolor{currentstroke}{rgb}{0.631373,0.788235,0.956863}%
\pgfsetstrokecolor{currentstroke}%
\pgfsetstrokeopacity{0.800000}%
\pgfsetdash{}{0pt}%
\pgfpathmoveto{\pgfqpoint{6.809391in}{4.549658in}}%
\pgfpathlineto{\pgfqpoint{5.077119in}{4.987631in}}%
\pgfusepath{stroke}%
\end{pgfscope}%
\begin{pgfscope}%
\pgfpathrectangle{\pgfqpoint{0.570343in}{0.331635in}}{\pgfqpoint{9.300000in}{7.700000in}}%
\pgfusepath{clip}%
\pgfsetrectcap%
\pgfsetroundjoin%
\pgfsetlinewidth{1.505625pt}%
\definecolor{currentstroke}{rgb}{0.631373,0.788235,0.956863}%
\pgfsetstrokecolor{currentstroke}%
\pgfsetstrokeopacity{0.800000}%
\pgfsetdash{}{0pt}%
\pgfpathmoveto{\pgfqpoint{4.564595in}{4.413606in}}%
\pgfpathlineto{\pgfqpoint{5.077119in}{4.987631in}}%
\pgfusepath{stroke}%
\end{pgfscope}%
\begin{pgfscope}%
\pgfpathrectangle{\pgfqpoint{0.570343in}{0.331635in}}{\pgfqpoint{9.300000in}{7.700000in}}%
\pgfusepath{clip}%
\pgfsetrectcap%
\pgfsetroundjoin%
\pgfsetlinewidth{1.505625pt}%
\definecolor{currentstroke}{rgb}{0.631373,0.788235,0.956863}%
\pgfsetstrokecolor{currentstroke}%
\pgfsetstrokeopacity{0.800000}%
\pgfsetdash{}{0pt}%
\pgfpathmoveto{\pgfqpoint{5.991395in}{5.277257in}}%
\pgfpathlineto{\pgfqpoint{5.077119in}{4.987631in}}%
\pgfusepath{stroke}%
\end{pgfscope}%
\begin{pgfscope}%
\pgfpathrectangle{\pgfqpoint{0.570343in}{0.331635in}}{\pgfqpoint{9.300000in}{7.700000in}}%
\pgfusepath{clip}%
\pgfsetrectcap%
\pgfsetroundjoin%
\pgfsetlinewidth{1.505625pt}%
\definecolor{currentstroke}{rgb}{0.631373,0.788235,0.956863}%
\pgfsetstrokecolor{currentstroke}%
\pgfsetstrokeopacity{0.800000}%
\pgfsetdash{}{0pt}%
\pgfpathmoveto{\pgfqpoint{3.383575in}{7.143783in}}%
\pgfpathlineto{\pgfqpoint{5.077119in}{4.987631in}}%
\pgfusepath{stroke}%
\end{pgfscope}%
\begin{pgfscope}%
\pgfpathrectangle{\pgfqpoint{0.570343in}{0.331635in}}{\pgfqpoint{9.300000in}{7.700000in}}%
\pgfusepath{clip}%
\pgfsetrectcap%
\pgfsetroundjoin%
\pgfsetlinewidth{1.505625pt}%
\definecolor{currentstroke}{rgb}{0.631373,0.788235,0.956863}%
\pgfsetstrokecolor{currentstroke}%
\pgfsetstrokeopacity{0.800000}%
\pgfsetdash{}{0pt}%
\pgfpathmoveto{\pgfqpoint{3.614236in}{4.262662in}}%
\pgfpathlineto{\pgfqpoint{5.077119in}{4.987631in}}%
\pgfusepath{stroke}%
\end{pgfscope}%
\begin{pgfscope}%
\pgfpathrectangle{\pgfqpoint{0.570343in}{0.331635in}}{\pgfqpoint{9.300000in}{7.700000in}}%
\pgfusepath{clip}%
\pgfsetrectcap%
\pgfsetroundjoin%
\pgfsetlinewidth{1.505625pt}%
\definecolor{currentstroke}{rgb}{0.631373,0.788235,0.956863}%
\pgfsetstrokecolor{currentstroke}%
\pgfsetstrokeopacity{0.800000}%
\pgfsetdash{}{0pt}%
\pgfpathmoveto{\pgfqpoint{8.088914in}{1.938914in}}%
\pgfpathlineto{\pgfqpoint{5.077119in}{4.987631in}}%
\pgfusepath{stroke}%
\end{pgfscope}%
\begin{pgfscope}%
\pgfpathrectangle{\pgfqpoint{0.570343in}{0.331635in}}{\pgfqpoint{9.300000in}{7.700000in}}%
\pgfusepath{clip}%
\pgfsetrectcap%
\pgfsetroundjoin%
\pgfsetlinewidth{1.505625pt}%
\definecolor{currentstroke}{rgb}{0.631373,0.788235,0.956863}%
\pgfsetstrokecolor{currentstroke}%
\pgfsetstrokeopacity{0.800000}%
\pgfsetdash{}{0pt}%
\pgfpathmoveto{\pgfqpoint{6.893373in}{7.628437in}}%
\pgfpathlineto{\pgfqpoint{5.077119in}{4.987631in}}%
\pgfusepath{stroke}%
\end{pgfscope}%
\begin{pgfscope}%
\pgfpathrectangle{\pgfqpoint{0.570343in}{0.331635in}}{\pgfqpoint{9.300000in}{7.700000in}}%
\pgfusepath{clip}%
\pgfsetrectcap%
\pgfsetroundjoin%
\pgfsetlinewidth{1.505625pt}%
\definecolor{currentstroke}{rgb}{0.631373,0.788235,0.956863}%
\pgfsetstrokecolor{currentstroke}%
\pgfsetstrokeopacity{0.800000}%
\pgfsetdash{}{0pt}%
\pgfpathmoveto{\pgfqpoint{8.123953in}{2.936881in}}%
\pgfpathlineto{\pgfqpoint{5.077119in}{4.987631in}}%
\pgfusepath{stroke}%
\end{pgfscope}%
\begin{pgfscope}%
\pgfpathrectangle{\pgfqpoint{0.570343in}{0.331635in}}{\pgfqpoint{9.300000in}{7.700000in}}%
\pgfusepath{clip}%
\pgfsetrectcap%
\pgfsetroundjoin%
\pgfsetlinewidth{1.505625pt}%
\definecolor{currentstroke}{rgb}{0.631373,0.788235,0.956863}%
\pgfsetstrokecolor{currentstroke}%
\pgfsetstrokeopacity{0.800000}%
\pgfsetdash{}{0pt}%
\pgfpathmoveto{\pgfqpoint{6.741009in}{5.814694in}}%
\pgfpathlineto{\pgfqpoint{5.077119in}{4.987631in}}%
\pgfusepath{stroke}%
\end{pgfscope}%
\begin{pgfscope}%
\pgfpathrectangle{\pgfqpoint{0.570343in}{0.331635in}}{\pgfqpoint{9.300000in}{7.700000in}}%
\pgfusepath{clip}%
\pgfsetrectcap%
\pgfsetroundjoin%
\pgfsetlinewidth{1.505625pt}%
\definecolor{currentstroke}{rgb}{0.631373,0.788235,0.956863}%
\pgfsetstrokecolor{currentstroke}%
\pgfsetstrokeopacity{0.800000}%
\pgfsetdash{}{0pt}%
\pgfpathmoveto{\pgfqpoint{4.238932in}{6.412234in}}%
\pgfpathlineto{\pgfqpoint{5.077119in}{4.987631in}}%
\pgfusepath{stroke}%
\end{pgfscope}%
\begin{pgfscope}%
\pgfpathrectangle{\pgfqpoint{0.570343in}{0.331635in}}{\pgfqpoint{9.300000in}{7.700000in}}%
\pgfusepath{clip}%
\pgfsetrectcap%
\pgfsetroundjoin%
\pgfsetlinewidth{1.505625pt}%
\definecolor{currentstroke}{rgb}{0.631373,0.788235,0.956863}%
\pgfsetstrokecolor{currentstroke}%
\pgfsetstrokeopacity{0.800000}%
\pgfsetdash{}{0pt}%
\pgfpathmoveto{\pgfqpoint{3.437634in}{3.499346in}}%
\pgfpathlineto{\pgfqpoint{5.077119in}{4.987631in}}%
\pgfusepath{stroke}%
\end{pgfscope}%
\begin{pgfscope}%
\pgfpathrectangle{\pgfqpoint{0.570343in}{0.331635in}}{\pgfqpoint{9.300000in}{7.700000in}}%
\pgfusepath{clip}%
\pgfsetrectcap%
\pgfsetroundjoin%
\pgfsetlinewidth{1.505625pt}%
\definecolor{currentstroke}{rgb}{0.631373,0.788235,0.956863}%
\pgfsetstrokecolor{currentstroke}%
\pgfsetstrokeopacity{0.800000}%
\pgfsetdash{}{0pt}%
\pgfpathmoveto{\pgfqpoint{1.464475in}{6.222679in}}%
\pgfpathlineto{\pgfqpoint{5.077119in}{4.987631in}}%
\pgfusepath{stroke}%
\end{pgfscope}%
\begin{pgfscope}%
\pgfpathrectangle{\pgfqpoint{0.570343in}{0.331635in}}{\pgfqpoint{9.300000in}{7.700000in}}%
\pgfusepath{clip}%
\pgfsetrectcap%
\pgfsetroundjoin%
\pgfsetlinewidth{1.505625pt}%
\definecolor{currentstroke}{rgb}{0.631373,0.788235,0.956863}%
\pgfsetstrokecolor{currentstroke}%
\pgfsetstrokeopacity{0.800000}%
\pgfsetdash{}{0pt}%
\pgfpathmoveto{\pgfqpoint{0.993071in}{4.960729in}}%
\pgfpathlineto{\pgfqpoint{5.077119in}{4.987631in}}%
\pgfusepath{stroke}%
\end{pgfscope}%
\begin{pgfscope}%
\pgfpathrectangle{\pgfqpoint{0.570343in}{0.331635in}}{\pgfqpoint{9.300000in}{7.700000in}}%
\pgfusepath{clip}%
\pgfsetrectcap%
\pgfsetroundjoin%
\pgfsetlinewidth{1.505625pt}%
\definecolor{currentstroke}{rgb}{0.631373,0.788235,0.956863}%
\pgfsetstrokecolor{currentstroke}%
\pgfsetstrokeopacity{0.800000}%
\pgfsetdash{}{0pt}%
\pgfpathmoveto{\pgfqpoint{5.578651in}{5.963023in}}%
\pgfpathlineto{\pgfqpoint{5.077119in}{4.987631in}}%
\pgfusepath{stroke}%
\end{pgfscope}%
\begin{pgfscope}%
\pgfpathrectangle{\pgfqpoint{0.570343in}{0.331635in}}{\pgfqpoint{9.300000in}{7.700000in}}%
\pgfusepath{clip}%
\pgfsetrectcap%
\pgfsetroundjoin%
\pgfsetlinewidth{1.505625pt}%
\definecolor{currentstroke}{rgb}{0.631373,0.788235,0.956863}%
\pgfsetstrokecolor{currentstroke}%
\pgfsetstrokeopacity{0.800000}%
\pgfsetdash{}{0pt}%
\pgfpathmoveto{\pgfqpoint{3.551278in}{5.632687in}}%
\pgfpathlineto{\pgfqpoint{5.077119in}{4.987631in}}%
\pgfusepath{stroke}%
\end{pgfscope}%
\begin{pgfscope}%
\pgfpathrectangle{\pgfqpoint{0.570343in}{0.331635in}}{\pgfqpoint{9.300000in}{7.700000in}}%
\pgfusepath{clip}%
\pgfsetrectcap%
\pgfsetroundjoin%
\pgfsetlinewidth{1.505625pt}%
\definecolor{currentstroke}{rgb}{0.631373,0.788235,0.956863}%
\pgfsetstrokecolor{currentstroke}%
\pgfsetstrokeopacity{0.800000}%
\pgfsetdash{}{0pt}%
\pgfpathmoveto{\pgfqpoint{7.539850in}{4.436340in}}%
\pgfpathlineto{\pgfqpoint{5.077119in}{4.987631in}}%
\pgfusepath{stroke}%
\end{pgfscope}%
\begin{pgfscope}%
\pgfpathrectangle{\pgfqpoint{0.570343in}{0.331635in}}{\pgfqpoint{9.300000in}{7.700000in}}%
\pgfusepath{clip}%
\pgfsetrectcap%
\pgfsetroundjoin%
\pgfsetlinewidth{1.505625pt}%
\definecolor{currentstroke}{rgb}{0.631373,0.788235,0.956863}%
\pgfsetstrokecolor{currentstroke}%
\pgfsetstrokeopacity{0.800000}%
\pgfsetdash{}{0pt}%
\pgfpathmoveto{\pgfqpoint{6.633358in}{6.419304in}}%
\pgfpathlineto{\pgfqpoint{5.077119in}{4.987631in}}%
\pgfusepath{stroke}%
\end{pgfscope}%
\begin{pgfscope}%
\pgfpathrectangle{\pgfqpoint{0.570343in}{0.331635in}}{\pgfqpoint{9.300000in}{7.700000in}}%
\pgfusepath{clip}%
\pgfsetrectcap%
\pgfsetroundjoin%
\pgfsetlinewidth{1.505625pt}%
\definecolor{currentstroke}{rgb}{0.631373,0.788235,0.956863}%
\pgfsetstrokecolor{currentstroke}%
\pgfsetstrokeopacity{0.800000}%
\pgfsetdash{}{0pt}%
\pgfpathmoveto{\pgfqpoint{9.092676in}{7.084959in}}%
\pgfpathlineto{\pgfqpoint{5.077119in}{4.987631in}}%
\pgfusepath{stroke}%
\end{pgfscope}%
\begin{pgfscope}%
\pgfpathrectangle{\pgfqpoint{0.570343in}{0.331635in}}{\pgfqpoint{9.300000in}{7.700000in}}%
\pgfusepath{clip}%
\pgfsetrectcap%
\pgfsetroundjoin%
\pgfsetlinewidth{1.505625pt}%
\definecolor{currentstroke}{rgb}{0.631373,0.788235,0.956863}%
\pgfsetstrokecolor{currentstroke}%
\pgfsetstrokeopacity{0.800000}%
\pgfsetdash{}{0pt}%
\pgfpathmoveto{\pgfqpoint{4.705737in}{5.610902in}}%
\pgfpathlineto{\pgfqpoint{5.077119in}{4.987631in}}%
\pgfusepath{stroke}%
\end{pgfscope}%
\begin{pgfscope}%
\pgfpathrectangle{\pgfqpoint{0.570343in}{0.331635in}}{\pgfqpoint{9.300000in}{7.700000in}}%
\pgfusepath{clip}%
\pgfsetrectcap%
\pgfsetroundjoin%
\pgfsetlinewidth{1.505625pt}%
\definecolor{currentstroke}{rgb}{0.631373,0.788235,0.956863}%
\pgfsetstrokecolor{currentstroke}%
\pgfsetstrokeopacity{0.800000}%
\pgfsetdash{}{0pt}%
\pgfpathmoveto{\pgfqpoint{3.116901in}{2.831988in}}%
\pgfpathlineto{\pgfqpoint{5.077119in}{4.987631in}}%
\pgfusepath{stroke}%
\end{pgfscope}%
\begin{pgfscope}%
\pgfpathrectangle{\pgfqpoint{0.570343in}{0.331635in}}{\pgfqpoint{9.300000in}{7.700000in}}%
\pgfusepath{clip}%
\pgfsetrectcap%
\pgfsetroundjoin%
\pgfsetlinewidth{1.505625pt}%
\definecolor{currentstroke}{rgb}{0.631373,0.788235,0.956863}%
\pgfsetstrokecolor{currentstroke}%
\pgfsetstrokeopacity{0.800000}%
\pgfsetdash{}{0pt}%
\pgfpathmoveto{\pgfqpoint{4.125517in}{2.846803in}}%
\pgfpathlineto{\pgfqpoint{5.077119in}{4.987631in}}%
\pgfusepath{stroke}%
\end{pgfscope}%
\begin{pgfscope}%
\pgfpathrectangle{\pgfqpoint{0.570343in}{0.331635in}}{\pgfqpoint{9.300000in}{7.700000in}}%
\pgfusepath{clip}%
\pgfsetrectcap%
\pgfsetroundjoin%
\pgfsetlinewidth{1.505625pt}%
\definecolor{currentstroke}{rgb}{0.631373,0.788235,0.956863}%
\pgfsetstrokecolor{currentstroke}%
\pgfsetstrokeopacity{0.800000}%
\pgfsetdash{}{0pt}%
\pgfpathmoveto{\pgfqpoint{3.950923in}{4.986608in}}%
\pgfpathlineto{\pgfqpoint{5.077119in}{4.987631in}}%
\pgfusepath{stroke}%
\end{pgfscope}%
\begin{pgfscope}%
\pgfpathrectangle{\pgfqpoint{0.570343in}{0.331635in}}{\pgfqpoint{9.300000in}{7.700000in}}%
\pgfusepath{clip}%
\pgfsetrectcap%
\pgfsetroundjoin%
\pgfsetlinewidth{1.505625pt}%
\definecolor{currentstroke}{rgb}{0.631373,0.788235,0.956863}%
\pgfsetstrokecolor{currentstroke}%
\pgfsetstrokeopacity{0.800000}%
\pgfsetdash{}{0pt}%
\pgfpathmoveto{\pgfqpoint{2.997649in}{1.937402in}}%
\pgfpathlineto{\pgfqpoint{5.077119in}{4.987631in}}%
\pgfusepath{stroke}%
\end{pgfscope}%
\begin{pgfscope}%
\pgfpathrectangle{\pgfqpoint{0.570343in}{0.331635in}}{\pgfqpoint{9.300000in}{7.700000in}}%
\pgfusepath{clip}%
\pgfsetrectcap%
\pgfsetroundjoin%
\pgfsetlinewidth{1.505625pt}%
\definecolor{currentstroke}{rgb}{0.631373,0.788235,0.956863}%
\pgfsetstrokecolor{currentstroke}%
\pgfsetstrokeopacity{0.800000}%
\pgfsetdash{}{0pt}%
\pgfpathmoveto{\pgfqpoint{5.501193in}{6.855026in}}%
\pgfpathlineto{\pgfqpoint{5.077119in}{4.987631in}}%
\pgfusepath{stroke}%
\end{pgfscope}%
\begin{pgfscope}%
\pgfpathrectangle{\pgfqpoint{0.570343in}{0.331635in}}{\pgfqpoint{9.300000in}{7.700000in}}%
\pgfusepath{clip}%
\pgfsetrectcap%
\pgfsetroundjoin%
\pgfsetlinewidth{1.505625pt}%
\definecolor{currentstroke}{rgb}{0.631373,0.788235,0.956863}%
\pgfsetstrokecolor{currentstroke}%
\pgfsetstrokeopacity{0.800000}%
\pgfsetdash{}{0pt}%
\pgfpathmoveto{\pgfqpoint{4.900969in}{5.341586in}}%
\pgfpathlineto{\pgfqpoint{5.077119in}{4.987631in}}%
\pgfusepath{stroke}%
\end{pgfscope}%
\begin{pgfscope}%
\pgfpathrectangle{\pgfqpoint{0.570343in}{0.331635in}}{\pgfqpoint{9.300000in}{7.700000in}}%
\pgfusepath{clip}%
\pgfsetrectcap%
\pgfsetroundjoin%
\pgfsetlinewidth{1.505625pt}%
\definecolor{currentstroke}{rgb}{1.000000,0.705882,0.509804}%
\pgfsetstrokecolor{currentstroke}%
\pgfsetstrokeopacity{0.800000}%
\pgfsetdash{}{0pt}%
\pgfpathmoveto{\pgfqpoint{5.688827in}{3.714865in}}%
\pgfpathlineto{\pgfqpoint{5.685130in}{3.894237in}}%
\pgfusepath{stroke}%
\end{pgfscope}%
\begin{pgfscope}%
\pgfpathrectangle{\pgfqpoint{0.570343in}{0.331635in}}{\pgfqpoint{9.300000in}{7.700000in}}%
\pgfusepath{clip}%
\pgfsetrectcap%
\pgfsetroundjoin%
\pgfsetlinewidth{1.505625pt}%
\definecolor{currentstroke}{rgb}{1.000000,0.705882,0.509804}%
\pgfsetstrokecolor{currentstroke}%
\pgfsetstrokeopacity{0.800000}%
\pgfsetdash{}{0pt}%
\pgfpathmoveto{\pgfqpoint{7.831480in}{6.739574in}}%
\pgfpathlineto{\pgfqpoint{5.685130in}{3.894237in}}%
\pgfusepath{stroke}%
\end{pgfscope}%
\begin{pgfscope}%
\pgfpathrectangle{\pgfqpoint{0.570343in}{0.331635in}}{\pgfqpoint{9.300000in}{7.700000in}}%
\pgfusepath{clip}%
\pgfsetrectcap%
\pgfsetroundjoin%
\pgfsetlinewidth{1.505625pt}%
\definecolor{currentstroke}{rgb}{1.000000,0.705882,0.509804}%
\pgfsetstrokecolor{currentstroke}%
\pgfsetstrokeopacity{0.800000}%
\pgfsetdash{}{0pt}%
\pgfpathmoveto{\pgfqpoint{5.338526in}{4.899084in}}%
\pgfpathlineto{\pgfqpoint{5.685130in}{3.894237in}}%
\pgfusepath{stroke}%
\end{pgfscope}%
\begin{pgfscope}%
\pgfpathrectangle{\pgfqpoint{0.570343in}{0.331635in}}{\pgfqpoint{9.300000in}{7.700000in}}%
\pgfusepath{clip}%
\pgfsetrectcap%
\pgfsetroundjoin%
\pgfsetlinewidth{1.505625pt}%
\definecolor{currentstroke}{rgb}{1.000000,0.705882,0.509804}%
\pgfsetstrokecolor{currentstroke}%
\pgfsetstrokeopacity{0.800000}%
\pgfsetdash{}{0pt}%
\pgfpathmoveto{\pgfqpoint{6.546058in}{3.234862in}}%
\pgfpathlineto{\pgfqpoint{5.685130in}{3.894237in}}%
\pgfusepath{stroke}%
\end{pgfscope}%
\begin{pgfscope}%
\pgfpathrectangle{\pgfqpoint{0.570343in}{0.331635in}}{\pgfqpoint{9.300000in}{7.700000in}}%
\pgfusepath{clip}%
\pgfsetrectcap%
\pgfsetroundjoin%
\pgfsetlinewidth{1.505625pt}%
\definecolor{currentstroke}{rgb}{1.000000,0.705882,0.509804}%
\pgfsetstrokecolor{currentstroke}%
\pgfsetstrokeopacity{0.800000}%
\pgfsetdash{}{0pt}%
\pgfpathmoveto{\pgfqpoint{8.931283in}{5.394309in}}%
\pgfpathlineto{\pgfqpoint{5.685130in}{3.894237in}}%
\pgfusepath{stroke}%
\end{pgfscope}%
\begin{pgfscope}%
\pgfpathrectangle{\pgfqpoint{0.570343in}{0.331635in}}{\pgfqpoint{9.300000in}{7.700000in}}%
\pgfusepath{clip}%
\pgfsetrectcap%
\pgfsetroundjoin%
\pgfsetlinewidth{1.505625pt}%
\definecolor{currentstroke}{rgb}{1.000000,0.705882,0.509804}%
\pgfsetstrokecolor{currentstroke}%
\pgfsetstrokeopacity{0.800000}%
\pgfsetdash{}{0pt}%
\pgfpathmoveto{\pgfqpoint{4.854875in}{3.454861in}}%
\pgfpathlineto{\pgfqpoint{5.685130in}{3.894237in}}%
\pgfusepath{stroke}%
\end{pgfscope}%
\begin{pgfscope}%
\pgfpathrectangle{\pgfqpoint{0.570343in}{0.331635in}}{\pgfqpoint{9.300000in}{7.700000in}}%
\pgfusepath{clip}%
\pgfsetrectcap%
\pgfsetroundjoin%
\pgfsetlinewidth{1.505625pt}%
\definecolor{currentstroke}{rgb}{1.000000,0.705882,0.509804}%
\pgfsetstrokecolor{currentstroke}%
\pgfsetstrokeopacity{0.800000}%
\pgfsetdash{}{0pt}%
\pgfpathmoveto{\pgfqpoint{7.413796in}{3.644013in}}%
\pgfpathlineto{\pgfqpoint{5.685130in}{3.894237in}}%
\pgfusepath{stroke}%
\end{pgfscope}%
\begin{pgfscope}%
\pgfpathrectangle{\pgfqpoint{0.570343in}{0.331635in}}{\pgfqpoint{9.300000in}{7.700000in}}%
\pgfusepath{clip}%
\pgfsetrectcap%
\pgfsetroundjoin%
\pgfsetlinewidth{1.505625pt}%
\definecolor{currentstroke}{rgb}{1.000000,0.705882,0.509804}%
\pgfsetstrokecolor{currentstroke}%
\pgfsetstrokeopacity{0.800000}%
\pgfsetdash{}{0pt}%
\pgfpathmoveto{\pgfqpoint{1.671602in}{3.907156in}}%
\pgfpathlineto{\pgfqpoint{5.685130in}{3.894237in}}%
\pgfusepath{stroke}%
\end{pgfscope}%
\begin{pgfscope}%
\pgfpathrectangle{\pgfqpoint{0.570343in}{0.331635in}}{\pgfqpoint{9.300000in}{7.700000in}}%
\pgfusepath{clip}%
\pgfsetrectcap%
\pgfsetroundjoin%
\pgfsetlinewidth{1.505625pt}%
\definecolor{currentstroke}{rgb}{1.000000,0.705882,0.509804}%
\pgfsetstrokecolor{currentstroke}%
\pgfsetstrokeopacity{0.800000}%
\pgfsetdash{}{0pt}%
\pgfpathmoveto{\pgfqpoint{1.885787in}{2.932540in}}%
\pgfpathlineto{\pgfqpoint{5.685130in}{3.894237in}}%
\pgfusepath{stroke}%
\end{pgfscope}%
\begin{pgfscope}%
\pgfpathrectangle{\pgfqpoint{0.570343in}{0.331635in}}{\pgfqpoint{9.300000in}{7.700000in}}%
\pgfusepath{clip}%
\pgfsetrectcap%
\pgfsetroundjoin%
\pgfsetlinewidth{1.505625pt}%
\definecolor{currentstroke}{rgb}{1.000000,0.705882,0.509804}%
\pgfsetstrokecolor{currentstroke}%
\pgfsetstrokeopacity{0.800000}%
\pgfsetdash{}{0pt}%
\pgfpathmoveto{\pgfqpoint{5.353276in}{4.288421in}}%
\pgfpathlineto{\pgfqpoint{5.685130in}{3.894237in}}%
\pgfusepath{stroke}%
\end{pgfscope}%
\begin{pgfscope}%
\pgfpathrectangle{\pgfqpoint{0.570343in}{0.331635in}}{\pgfqpoint{9.300000in}{7.700000in}}%
\pgfusepath{clip}%
\pgfsetrectcap%
\pgfsetroundjoin%
\pgfsetlinewidth{1.505625pt}%
\definecolor{currentstroke}{rgb}{1.000000,0.705882,0.509804}%
\pgfsetstrokecolor{currentstroke}%
\pgfsetstrokeopacity{0.800000}%
\pgfsetdash{}{0pt}%
\pgfpathmoveto{\pgfqpoint{6.076381in}{4.619110in}}%
\pgfpathlineto{\pgfqpoint{5.685130in}{3.894237in}}%
\pgfusepath{stroke}%
\end{pgfscope}%
\begin{pgfscope}%
\pgfpathrectangle{\pgfqpoint{0.570343in}{0.331635in}}{\pgfqpoint{9.300000in}{7.700000in}}%
\pgfusepath{clip}%
\pgfsetrectcap%
\pgfsetroundjoin%
\pgfsetlinewidth{1.505625pt}%
\definecolor{currentstroke}{rgb}{1.000000,0.705882,0.509804}%
\pgfsetstrokecolor{currentstroke}%
\pgfsetstrokeopacity{0.800000}%
\pgfsetdash{}{0pt}%
\pgfpathmoveto{\pgfqpoint{4.577524in}{7.681635in}}%
\pgfpathlineto{\pgfqpoint{5.685130in}{3.894237in}}%
\pgfusepath{stroke}%
\end{pgfscope}%
\begin{pgfscope}%
\pgfpathrectangle{\pgfqpoint{0.570343in}{0.331635in}}{\pgfqpoint{9.300000in}{7.700000in}}%
\pgfusepath{clip}%
\pgfsetrectcap%
\pgfsetroundjoin%
\pgfsetlinewidth{1.505625pt}%
\definecolor{currentstroke}{rgb}{1.000000,0.705882,0.509804}%
\pgfsetstrokecolor{currentstroke}%
\pgfsetstrokeopacity{0.800000}%
\pgfsetdash{}{0pt}%
\pgfpathmoveto{\pgfqpoint{2.562027in}{6.283257in}}%
\pgfpathlineto{\pgfqpoint{5.685130in}{3.894237in}}%
\pgfusepath{stroke}%
\end{pgfscope}%
\begin{pgfscope}%
\pgfpathrectangle{\pgfqpoint{0.570343in}{0.331635in}}{\pgfqpoint{9.300000in}{7.700000in}}%
\pgfusepath{clip}%
\pgfsetrectcap%
\pgfsetroundjoin%
\pgfsetlinewidth{1.505625pt}%
\definecolor{currentstroke}{rgb}{1.000000,0.705882,0.509804}%
\pgfsetstrokecolor{currentstroke}%
\pgfsetstrokeopacity{0.800000}%
\pgfsetdash{}{0pt}%
\pgfpathmoveto{\pgfqpoint{8.392100in}{4.902301in}}%
\pgfpathlineto{\pgfqpoint{5.685130in}{3.894237in}}%
\pgfusepath{stroke}%
\end{pgfscope}%
\begin{pgfscope}%
\pgfpathrectangle{\pgfqpoint{0.570343in}{0.331635in}}{\pgfqpoint{9.300000in}{7.700000in}}%
\pgfusepath{clip}%
\pgfsetrectcap%
\pgfsetroundjoin%
\pgfsetlinewidth{1.505625pt}%
\definecolor{currentstroke}{rgb}{1.000000,0.705882,0.509804}%
\pgfsetstrokecolor{currentstroke}%
\pgfsetstrokeopacity{0.800000}%
\pgfsetdash{}{0pt}%
\pgfpathmoveto{\pgfqpoint{5.806842in}{1.707715in}}%
\pgfpathlineto{\pgfqpoint{5.685130in}{3.894237in}}%
\pgfusepath{stroke}%
\end{pgfscope}%
\begin{pgfscope}%
\pgfpathrectangle{\pgfqpoint{0.570343in}{0.331635in}}{\pgfqpoint{9.300000in}{7.700000in}}%
\pgfusepath{clip}%
\pgfsetrectcap%
\pgfsetroundjoin%
\pgfsetlinewidth{1.505625pt}%
\definecolor{currentstroke}{rgb}{1.000000,0.705882,0.509804}%
\pgfsetstrokecolor{currentstroke}%
\pgfsetstrokeopacity{0.800000}%
\pgfsetdash{}{0pt}%
\pgfpathmoveto{\pgfqpoint{4.120169in}{3.735967in}}%
\pgfpathlineto{\pgfqpoint{5.685130in}{3.894237in}}%
\pgfusepath{stroke}%
\end{pgfscope}%
\begin{pgfscope}%
\pgfpathrectangle{\pgfqpoint{0.570343in}{0.331635in}}{\pgfqpoint{9.300000in}{7.700000in}}%
\pgfusepath{clip}%
\pgfsetrectcap%
\pgfsetroundjoin%
\pgfsetlinewidth{1.505625pt}%
\definecolor{currentstroke}{rgb}{1.000000,0.705882,0.509804}%
\pgfsetstrokecolor{currentstroke}%
\pgfsetstrokeopacity{0.800000}%
\pgfsetdash{}{0pt}%
\pgfpathmoveto{\pgfqpoint{3.866323in}{1.837880in}}%
\pgfpathlineto{\pgfqpoint{5.685130in}{3.894237in}}%
\pgfusepath{stroke}%
\end{pgfscope}%
\begin{pgfscope}%
\pgfpathrectangle{\pgfqpoint{0.570343in}{0.331635in}}{\pgfqpoint{9.300000in}{7.700000in}}%
\pgfusepath{clip}%
\pgfsetrectcap%
\pgfsetroundjoin%
\pgfsetlinewidth{1.505625pt}%
\definecolor{currentstroke}{rgb}{1.000000,0.705882,0.509804}%
\pgfsetstrokecolor{currentstroke}%
\pgfsetstrokeopacity{0.800000}%
\pgfsetdash{}{0pt}%
\pgfpathmoveto{\pgfqpoint{2.788649in}{4.018903in}}%
\pgfpathlineto{\pgfqpoint{5.685130in}{3.894237in}}%
\pgfusepath{stroke}%
\end{pgfscope}%
\begin{pgfscope}%
\pgfpathrectangle{\pgfqpoint{0.570343in}{0.331635in}}{\pgfqpoint{9.300000in}{7.700000in}}%
\pgfusepath{clip}%
\pgfsetrectcap%
\pgfsetroundjoin%
\pgfsetlinewidth{1.505625pt}%
\definecolor{currentstroke}{rgb}{1.000000,0.705882,0.509804}%
\pgfsetstrokecolor{currentstroke}%
\pgfsetstrokeopacity{0.800000}%
\pgfsetdash{}{0pt}%
\pgfpathmoveto{\pgfqpoint{9.447616in}{3.664094in}}%
\pgfpathlineto{\pgfqpoint{5.685130in}{3.894237in}}%
\pgfusepath{stroke}%
\end{pgfscope}%
\begin{pgfscope}%
\pgfpathrectangle{\pgfqpoint{0.570343in}{0.331635in}}{\pgfqpoint{9.300000in}{7.700000in}}%
\pgfusepath{clip}%
\pgfsetrectcap%
\pgfsetroundjoin%
\pgfsetlinewidth{1.505625pt}%
\definecolor{currentstroke}{rgb}{1.000000,0.705882,0.509804}%
\pgfsetstrokecolor{currentstroke}%
\pgfsetstrokeopacity{0.800000}%
\pgfsetdash{}{0pt}%
\pgfpathmoveto{\pgfqpoint{5.814101in}{0.681635in}}%
\pgfpathlineto{\pgfqpoint{5.685130in}{3.894237in}}%
\pgfusepath{stroke}%
\end{pgfscope}%
\begin{pgfscope}%
\pgfpathrectangle{\pgfqpoint{0.570343in}{0.331635in}}{\pgfqpoint{9.300000in}{7.700000in}}%
\pgfusepath{clip}%
\pgfsetrectcap%
\pgfsetroundjoin%
\pgfsetlinewidth{1.505625pt}%
\definecolor{currentstroke}{rgb}{1.000000,0.705882,0.509804}%
\pgfsetstrokecolor{currentstroke}%
\pgfsetstrokeopacity{0.800000}%
\pgfsetdash{}{0pt}%
\pgfpathmoveto{\pgfqpoint{5.056091in}{2.274711in}}%
\pgfpathlineto{\pgfqpoint{5.685130in}{3.894237in}}%
\pgfusepath{stroke}%
\end{pgfscope}%
\begin{pgfscope}%
\pgfpathrectangle{\pgfqpoint{0.570343in}{0.331635in}}{\pgfqpoint{9.300000in}{7.700000in}}%
\pgfusepath{clip}%
\pgfsetrectcap%
\pgfsetroundjoin%
\pgfsetlinewidth{1.505625pt}%
\definecolor{currentstroke}{rgb}{1.000000,0.705882,0.509804}%
\pgfsetstrokecolor{currentstroke}%
\pgfsetstrokeopacity{0.800000}%
\pgfsetdash{}{0pt}%
\pgfpathmoveto{\pgfqpoint{8.597790in}{3.819155in}}%
\pgfpathlineto{\pgfqpoint{5.685130in}{3.894237in}}%
\pgfusepath{stroke}%
\end{pgfscope}%
\begin{pgfscope}%
\pgfpathrectangle{\pgfqpoint{0.570343in}{0.331635in}}{\pgfqpoint{9.300000in}{7.700000in}}%
\pgfusepath{clip}%
\pgfsetrectcap%
\pgfsetroundjoin%
\pgfsetlinewidth{1.505625pt}%
\definecolor{currentstroke}{rgb}{1.000000,0.705882,0.509804}%
\pgfsetstrokecolor{currentstroke}%
\pgfsetstrokeopacity{0.800000}%
\pgfsetdash{}{0pt}%
\pgfpathmoveto{\pgfqpoint{4.061550in}{0.854936in}}%
\pgfpathlineto{\pgfqpoint{5.685130in}{3.894237in}}%
\pgfusepath{stroke}%
\end{pgfscope}%
\begin{pgfscope}%
\pgfpathrectangle{\pgfqpoint{0.570343in}{0.331635in}}{\pgfqpoint{9.300000in}{7.700000in}}%
\pgfusepath{clip}%
\pgfsetrectcap%
\pgfsetroundjoin%
\pgfsetlinewidth{1.505625pt}%
\definecolor{currentstroke}{rgb}{1.000000,0.705882,0.509804}%
\pgfsetstrokecolor{currentstroke}%
\pgfsetstrokeopacity{0.800000}%
\pgfsetdash{}{0pt}%
\pgfpathmoveto{\pgfqpoint{7.863027in}{5.805960in}}%
\pgfpathlineto{\pgfqpoint{5.685130in}{3.894237in}}%
\pgfusepath{stroke}%
\end{pgfscope}%
\begin{pgfscope}%
\pgfpathrectangle{\pgfqpoint{0.570343in}{0.331635in}}{\pgfqpoint{9.300000in}{7.700000in}}%
\pgfusepath{clip}%
\pgfsetrectcap%
\pgfsetroundjoin%
\pgfsetlinewidth{1.505625pt}%
\definecolor{currentstroke}{rgb}{1.000000,0.705882,0.509804}%
\pgfsetstrokecolor{currentstroke}%
\pgfsetstrokeopacity{0.800000}%
\pgfsetdash{}{0pt}%
\pgfpathmoveto{\pgfqpoint{6.615719in}{2.506683in}}%
\pgfpathlineto{\pgfqpoint{5.685130in}{3.894237in}}%
\pgfusepath{stroke}%
\end{pgfscope}%
\begin{pgfscope}%
\pgfpathrectangle{\pgfqpoint{0.570343in}{0.331635in}}{\pgfqpoint{9.300000in}{7.700000in}}%
\pgfusepath{clip}%
\pgfsetrectcap%
\pgfsetroundjoin%
\pgfsetlinewidth{1.505625pt}%
\definecolor{currentstroke}{rgb}{1.000000,0.705882,0.509804}%
\pgfsetstrokecolor{currentstroke}%
\pgfsetstrokeopacity{0.800000}%
\pgfsetdash{}{0pt}%
\pgfpathmoveto{\pgfqpoint{3.036089in}{4.944505in}}%
\pgfpathlineto{\pgfqpoint{5.685130in}{3.894237in}}%
\pgfusepath{stroke}%
\end{pgfscope}%
\begin{pgfscope}%
\pgfpathrectangle{\pgfqpoint{0.570343in}{0.331635in}}{\pgfqpoint{9.300000in}{7.700000in}}%
\pgfusepath{clip}%
\pgfsetrectcap%
\pgfsetroundjoin%
\pgfsetlinewidth{1.505625pt}%
\definecolor{currentstroke}{rgb}{1.000000,0.705882,0.509804}%
\pgfsetstrokecolor{currentstroke}%
\pgfsetstrokeopacity{0.800000}%
\pgfsetdash{}{0pt}%
\pgfpathmoveto{\pgfqpoint{9.333987in}{4.618456in}}%
\pgfpathlineto{\pgfqpoint{5.685130in}{3.894237in}}%
\pgfusepath{stroke}%
\end{pgfscope}%
\begin{pgfscope}%
\pgfpathrectangle{\pgfqpoint{0.570343in}{0.331635in}}{\pgfqpoint{9.300000in}{7.700000in}}%
\pgfusepath{clip}%
\pgfsetrectcap%
\pgfsetroundjoin%
\pgfsetlinewidth{1.505625pt}%
\definecolor{currentstroke}{rgb}{1.000000,0.705882,0.509804}%
\pgfsetstrokecolor{currentstroke}%
\pgfsetstrokeopacity{0.800000}%
\pgfsetdash{}{0pt}%
\pgfpathmoveto{\pgfqpoint{5.652131in}{2.872041in}}%
\pgfpathlineto{\pgfqpoint{5.685130in}{3.894237in}}%
\pgfusepath{stroke}%
\end{pgfscope}%
\begin{pgfscope}%
\pgfsetrectcap%
\pgfsetmiterjoin%
\pgfsetlinewidth{0.803000pt}%
\definecolor{currentstroke}{rgb}{0.000000,0.000000,0.000000}%
\pgfsetstrokecolor{currentstroke}%
\pgfsetdash{}{0pt}%
\pgfpathmoveto{\pgfqpoint{0.570343in}{0.331635in}}%
\pgfpathlineto{\pgfqpoint{0.570343in}{8.031635in}}%
\pgfusepath{stroke}%
\end{pgfscope}%
\begin{pgfscope}%
\pgfsetrectcap%
\pgfsetmiterjoin%
\pgfsetlinewidth{0.803000pt}%
\definecolor{currentstroke}{rgb}{0.000000,0.000000,0.000000}%
\pgfsetstrokecolor{currentstroke}%
\pgfsetdash{}{0pt}%
\pgfpathmoveto{\pgfqpoint{9.870343in}{0.331635in}}%
\pgfpathlineto{\pgfqpoint{9.870343in}{8.031635in}}%
\pgfusepath{stroke}%
\end{pgfscope}%
\begin{pgfscope}%
\pgfsetrectcap%
\pgfsetmiterjoin%
\pgfsetlinewidth{0.803000pt}%
\definecolor{currentstroke}{rgb}{0.000000,0.000000,0.000000}%
\pgfsetstrokecolor{currentstroke}%
\pgfsetdash{}{0pt}%
\pgfpathmoveto{\pgfqpoint{0.570343in}{0.331635in}}%
\pgfpathlineto{\pgfqpoint{9.870343in}{0.331635in}}%
\pgfusepath{stroke}%
\end{pgfscope}%
\begin{pgfscope}%
\pgfsetrectcap%
\pgfsetmiterjoin%
\pgfsetlinewidth{0.803000pt}%
\definecolor{currentstroke}{rgb}{0.000000,0.000000,0.000000}%
\pgfsetstrokecolor{currentstroke}%
\pgfsetdash{}{0pt}%
\pgfpathmoveto{\pgfqpoint{0.570343in}{8.031635in}}%
\pgfpathlineto{\pgfqpoint{9.870343in}{8.031635in}}%
\pgfusepath{stroke}%
\end{pgfscope}%
\begin{pgfscope}%
\definecolor{textcolor}{rgb}{0.000000,0.000000,0.000000}%
\pgfsetstrokecolor{textcolor}%
\pgfsetfillcolor{textcolor}%
\pgftext[x=5.220343in,y=8.114968in,,base]{\color{textcolor}\sffamily\fontsize{12.000000}{14.400000}\selectfont Photo-Realistic Images}%
\end{pgfscope}%
\begin{pgfscope}%
\pgfsetbuttcap%
\pgfsetmiterjoin%
\definecolor{currentfill}{rgb}{1.000000,1.000000,1.000000}%
\pgfsetfillcolor{currentfill}%
\pgfsetfillopacity{0.800000}%
\pgfsetlinewidth{1.003750pt}%
\definecolor{currentstroke}{rgb}{0.800000,0.800000,0.800000}%
\pgfsetstrokecolor{currentstroke}%
\pgfsetstrokeopacity{0.800000}%
\pgfsetdash{}{0pt}%
\pgfpathmoveto{\pgfqpoint{9.967566in}{3.956944in}}%
\pgfpathlineto{\pgfqpoint{10.918941in}{3.956944in}}%
\pgfpathquadraticcurveto{\pgfqpoint{10.946719in}{3.956944in}}{\pgfqpoint{10.946719in}{3.984722in}}%
\pgfpathlineto{\pgfqpoint{10.946719in}{4.378548in}}%
\pgfpathquadraticcurveto{\pgfqpoint{10.946719in}{4.406326in}}{\pgfqpoint{10.918941in}{4.406326in}}%
\pgfpathlineto{\pgfqpoint{9.967566in}{4.406326in}}%
\pgfpathquadraticcurveto{\pgfqpoint{9.939788in}{4.406326in}}{\pgfqpoint{9.939788in}{4.378548in}}%
\pgfpathlineto{\pgfqpoint{9.939788in}{3.984722in}}%
\pgfpathquadraticcurveto{\pgfqpoint{9.939788in}{3.956944in}}{\pgfqpoint{9.967566in}{3.956944in}}%
\pgfpathclose%
\pgfusepath{stroke,fill}%
\end{pgfscope}%
\begin{pgfscope}%
\pgfsetbuttcap%
\pgfsetroundjoin%
\definecolor{currentfill}{rgb}{0.631373,0.788235,0.956863}%
\pgfsetfillcolor{currentfill}%
\pgfsetlinewidth{1.003750pt}%
\definecolor{currentstroke}{rgb}{0.631373,0.788235,0.956863}%
\pgfsetstrokecolor{currentstroke}%
\pgfsetdash{}{0pt}%
\pgfsys@defobject{currentmarker}{\pgfqpoint{-0.041667in}{-0.041667in}}{\pgfqpoint{0.041667in}{0.041667in}}{%
\pgfpathmoveto{\pgfqpoint{0.000000in}{-0.041667in}}%
\pgfpathcurveto{\pgfqpoint{0.011050in}{-0.041667in}}{\pgfqpoint{0.021649in}{-0.037276in}}{\pgfqpoint{0.029463in}{-0.029463in}}%
\pgfpathcurveto{\pgfqpoint{0.037276in}{-0.021649in}}{\pgfqpoint{0.041667in}{-0.011050in}}{\pgfqpoint{0.041667in}{0.000000in}}%
\pgfpathcurveto{\pgfqpoint{0.041667in}{0.011050in}}{\pgfqpoint{0.037276in}{0.021649in}}{\pgfqpoint{0.029463in}{0.029463in}}%
\pgfpathcurveto{\pgfqpoint{0.021649in}{0.037276in}}{\pgfqpoint{0.011050in}{0.041667in}}{\pgfqpoint{0.000000in}{0.041667in}}%
\pgfpathcurveto{\pgfqpoint{-0.011050in}{0.041667in}}{\pgfqpoint{-0.021649in}{0.037276in}}{\pgfqpoint{-0.029463in}{0.029463in}}%
\pgfpathcurveto{\pgfqpoint{-0.037276in}{0.021649in}}{\pgfqpoint{-0.041667in}{0.011050in}}{\pgfqpoint{-0.041667in}{0.000000in}}%
\pgfpathcurveto{\pgfqpoint{-0.041667in}{-0.011050in}}{\pgfqpoint{-0.037276in}{-0.021649in}}{\pgfqpoint{-0.029463in}{-0.029463in}}%
\pgfpathcurveto{\pgfqpoint{-0.021649in}{-0.037276in}}{\pgfqpoint{-0.011050in}{-0.041667in}}{\pgfqpoint{0.000000in}{-0.041667in}}%
\pgfpathclose%
\pgfusepath{stroke,fill}%
}%
\begin{pgfscope}%
\pgfsys@transformshift{10.134232in}{4.281705in}%
\pgfsys@useobject{currentmarker}{}%
\end{pgfscope}%
\end{pgfscope}%
\begin{pgfscope}%
\definecolor{textcolor}{rgb}{0.000000,0.000000,0.000000}%
\pgfsetstrokecolor{textcolor}%
\pgfsetfillcolor{textcolor}%
\pgftext[x=10.384232in,y=4.245247in,left,base]{\color{textcolor}\sffamily\fontsize{10.000000}{12.000000}\selectfont 3dfront}%
\end{pgfscope}%
\begin{pgfscope}%
\pgfsetbuttcap%
\pgfsetroundjoin%
\definecolor{currentfill}{rgb}{1.000000,0.705882,0.509804}%
\pgfsetfillcolor{currentfill}%
\pgfsetlinewidth{1.003750pt}%
\definecolor{currentstroke}{rgb}{1.000000,0.705882,0.509804}%
\pgfsetstrokecolor{currentstroke}%
\pgfsetdash{}{0pt}%
\pgfsys@defobject{currentmarker}{\pgfqpoint{-0.041667in}{-0.041667in}}{\pgfqpoint{0.041667in}{0.041667in}}{%
\pgfpathmoveto{\pgfqpoint{0.000000in}{-0.041667in}}%
\pgfpathcurveto{\pgfqpoint{0.011050in}{-0.041667in}}{\pgfqpoint{0.021649in}{-0.037276in}}{\pgfqpoint{0.029463in}{-0.029463in}}%
\pgfpathcurveto{\pgfqpoint{0.037276in}{-0.021649in}}{\pgfqpoint{0.041667in}{-0.011050in}}{\pgfqpoint{0.041667in}{0.000000in}}%
\pgfpathcurveto{\pgfqpoint{0.041667in}{0.011050in}}{\pgfqpoint{0.037276in}{0.021649in}}{\pgfqpoint{0.029463in}{0.029463in}}%
\pgfpathcurveto{\pgfqpoint{0.021649in}{0.037276in}}{\pgfqpoint{0.011050in}{0.041667in}}{\pgfqpoint{0.000000in}{0.041667in}}%
\pgfpathcurveto{\pgfqpoint{-0.011050in}{0.041667in}}{\pgfqpoint{-0.021649in}{0.037276in}}{\pgfqpoint{-0.029463in}{0.029463in}}%
\pgfpathcurveto{\pgfqpoint{-0.037276in}{0.021649in}}{\pgfqpoint{-0.041667in}{0.011050in}}{\pgfqpoint{-0.041667in}{0.000000in}}%
\pgfpathcurveto{\pgfqpoint{-0.041667in}{-0.011050in}}{\pgfqpoint{-0.037276in}{-0.021649in}}{\pgfqpoint{-0.029463in}{-0.029463in}}%
\pgfpathcurveto{\pgfqpoint{-0.021649in}{-0.037276in}}{\pgfqpoint{-0.011050in}{-0.041667in}}{\pgfqpoint{0.000000in}{-0.041667in}}%
\pgfpathclose%
\pgfusepath{stroke,fill}%
}%
\begin{pgfscope}%
\pgfsys@transformshift{10.134232in}{4.077848in}%
\pgfsys@useobject{currentmarker}{}%
\end{pgfscope}%
\end{pgfscope}%
\begin{pgfscope}%
\definecolor{textcolor}{rgb}{0.000000,0.000000,0.000000}%
\pgfsetstrokecolor{textcolor}%
\pgfsetfillcolor{textcolor}%
\pgftext[x=10.384232in,y=4.041390in,left,base]{\color{textcolor}\sffamily\fontsize{10.000000}{12.000000}\selectfont pix3d}%
\end{pgfscope}%
\end{pgfpicture}%
\makeatother%
\endgroup%
}
    \resizebox{0.49\linewidth}{5cm}{%% Creator: Matplotlib, PGF backend
%%
%% To include the figure in your LaTeX document, write
%%   \input{<filename>.pgf}
%%
%% Make sure the required packages are loaded in your preamble
%%   \usepackage{pgf}
%%
%% Figures using additional raster images can only be included by \input if
%% they are in the same directory as the main LaTeX file. For loading figures
%% from other directories you can use the `import` package
%%   \usepackage{import}
%%
%% and then include the figures with
%%   \import{<path to file>}{<filename>.pgf}
%%
%% Matplotlib used the following preamble
%%   \usepackage{fontspec}
%%   \setmainfont{DejaVuSerif.ttf}[Path=\detokenize{/Users/apple/opt/anaconda3/envs/kaolin/lib/python3.7/site-packages/matplotlib/mpl-data/fonts/ttf/}]
%%   \setsansfont{DejaVuSans.ttf}[Path=\detokenize{/Users/apple/opt/anaconda3/envs/kaolin/lib/python3.7/site-packages/matplotlib/mpl-data/fonts/ttf/}]
%%   \setmonofont{DejaVuSansMono.ttf}[Path=\detokenize{/Users/apple/opt/anaconda3/envs/kaolin/lib/python3.7/site-packages/matplotlib/mpl-data/fonts/ttf/}]
%%
\begingroup%
\makeatletter%
\begin{pgfpicture}%
\pgfpathrectangle{\pgfpointorigin}{\pgfqpoint{11.186964in}{8.341596in}}%
\pgfusepath{use as bounding box, clip}%
\begin{pgfscope}%
\pgfsetbuttcap%
\pgfsetmiterjoin%
\definecolor{currentfill}{rgb}{1.000000,1.000000,1.000000}%
\pgfsetfillcolor{currentfill}%
\pgfsetlinewidth{0.000000pt}%
\definecolor{currentstroke}{rgb}{1.000000,1.000000,1.000000}%
\pgfsetstrokecolor{currentstroke}%
\pgfsetdash{}{0pt}%
\pgfpathmoveto{\pgfqpoint{0.000000in}{0.000000in}}%
\pgfpathlineto{\pgfqpoint{11.186964in}{0.000000in}}%
\pgfpathlineto{\pgfqpoint{11.186964in}{8.341596in}}%
\pgfpathlineto{\pgfqpoint{0.000000in}{8.341596in}}%
\pgfpathclose%
\pgfusepath{fill}%
\end{pgfscope}%
\begin{pgfscope}%
\pgfsetbuttcap%
\pgfsetmiterjoin%
\definecolor{currentfill}{rgb}{1.000000,1.000000,1.000000}%
\pgfsetfillcolor{currentfill}%
\pgfsetlinewidth{0.000000pt}%
\definecolor{currentstroke}{rgb}{0.000000,0.000000,0.000000}%
\pgfsetstrokecolor{currentstroke}%
\pgfsetstrokeopacity{0.000000}%
\pgfsetdash{}{0pt}%
\pgfpathmoveto{\pgfqpoint{0.570343in}{0.331635in}}%
\pgfpathlineto{\pgfqpoint{9.870343in}{0.331635in}}%
\pgfpathlineto{\pgfqpoint{9.870343in}{8.031635in}}%
\pgfpathlineto{\pgfqpoint{0.570343in}{8.031635in}}%
\pgfpathclose%
\pgfusepath{fill}%
\end{pgfscope}%
\begin{pgfscope}%
\pgfpathrectangle{\pgfqpoint{0.570343in}{0.331635in}}{\pgfqpoint{9.300000in}{7.700000in}}%
\pgfusepath{clip}%
\pgfsetbuttcap%
\pgfsetroundjoin%
\definecolor{currentfill}{rgb}{0.631373,0.788235,0.956863}%
\pgfsetfillcolor{currentfill}%
\pgfsetlinewidth{0.481800pt}%
\definecolor{currentstroke}{rgb}{1.000000,1.000000,1.000000}%
\pgfsetstrokecolor{currentstroke}%
\pgfsetdash{}{0pt}%
\pgfpathmoveto{\pgfqpoint{6.993440in}{5.386807in}}%
\pgfpathcurveto{\pgfqpoint{7.004490in}{5.386807in}}{\pgfqpoint{7.015089in}{5.391197in}}{\pgfqpoint{7.022903in}{5.399011in}}%
\pgfpathcurveto{\pgfqpoint{7.030716in}{5.406825in}}{\pgfqpoint{7.035106in}{5.417424in}}{\pgfqpoint{7.035106in}{5.428474in}}%
\pgfpathcurveto{\pgfqpoint{7.035106in}{5.439524in}}{\pgfqpoint{7.030716in}{5.450123in}}{\pgfqpoint{7.022903in}{5.457937in}}%
\pgfpathcurveto{\pgfqpoint{7.015089in}{5.465750in}}{\pgfqpoint{7.004490in}{5.470140in}}{\pgfqpoint{6.993440in}{5.470140in}}%
\pgfpathcurveto{\pgfqpoint{6.982390in}{5.470140in}}{\pgfqpoint{6.971791in}{5.465750in}}{\pgfqpoint{6.963977in}{5.457937in}}%
\pgfpathcurveto{\pgfqpoint{6.956163in}{5.450123in}}{\pgfqpoint{6.951773in}{5.439524in}}{\pgfqpoint{6.951773in}{5.428474in}}%
\pgfpathcurveto{\pgfqpoint{6.951773in}{5.417424in}}{\pgfqpoint{6.956163in}{5.406825in}}{\pgfqpoint{6.963977in}{5.399011in}}%
\pgfpathcurveto{\pgfqpoint{6.971791in}{5.391197in}}{\pgfqpoint{6.982390in}{5.386807in}}{\pgfqpoint{6.993440in}{5.386807in}}%
\pgfpathclose%
\pgfusepath{stroke,fill}%
\end{pgfscope}%
\begin{pgfscope}%
\pgfpathrectangle{\pgfqpoint{0.570343in}{0.331635in}}{\pgfqpoint{9.300000in}{7.700000in}}%
\pgfusepath{clip}%
\pgfsetbuttcap%
\pgfsetroundjoin%
\definecolor{currentfill}{rgb}{0.631373,0.788235,0.956863}%
\pgfsetfillcolor{currentfill}%
\pgfsetlinewidth{0.481800pt}%
\definecolor{currentstroke}{rgb}{1.000000,1.000000,1.000000}%
\pgfsetstrokecolor{currentstroke}%
\pgfsetdash{}{0pt}%
\pgfpathmoveto{\pgfqpoint{7.514448in}{1.636778in}}%
\pgfpathcurveto{\pgfqpoint{7.525498in}{1.636778in}}{\pgfqpoint{7.536097in}{1.641169in}}{\pgfqpoint{7.543910in}{1.648982in}}%
\pgfpathcurveto{\pgfqpoint{7.551724in}{1.656796in}}{\pgfqpoint{7.556114in}{1.667395in}}{\pgfqpoint{7.556114in}{1.678445in}}%
\pgfpathcurveto{\pgfqpoint{7.556114in}{1.689495in}}{\pgfqpoint{7.551724in}{1.700094in}}{\pgfqpoint{7.543910in}{1.707908in}}%
\pgfpathcurveto{\pgfqpoint{7.536097in}{1.715721in}}{\pgfqpoint{7.525498in}{1.720112in}}{\pgfqpoint{7.514448in}{1.720112in}}%
\pgfpathcurveto{\pgfqpoint{7.503397in}{1.720112in}}{\pgfqpoint{7.492798in}{1.715721in}}{\pgfqpoint{7.484985in}{1.707908in}}%
\pgfpathcurveto{\pgfqpoint{7.477171in}{1.700094in}}{\pgfqpoint{7.472781in}{1.689495in}}{\pgfqpoint{7.472781in}{1.678445in}}%
\pgfpathcurveto{\pgfqpoint{7.472781in}{1.667395in}}{\pgfqpoint{7.477171in}{1.656796in}}{\pgfqpoint{7.484985in}{1.648982in}}%
\pgfpathcurveto{\pgfqpoint{7.492798in}{1.641169in}}{\pgfqpoint{7.503397in}{1.636778in}}{\pgfqpoint{7.514448in}{1.636778in}}%
\pgfpathclose%
\pgfusepath{stroke,fill}%
\end{pgfscope}%
\begin{pgfscope}%
\pgfpathrectangle{\pgfqpoint{0.570343in}{0.331635in}}{\pgfqpoint{9.300000in}{7.700000in}}%
\pgfusepath{clip}%
\pgfsetbuttcap%
\pgfsetroundjoin%
\definecolor{currentfill}{rgb}{0.631373,0.788235,0.956863}%
\pgfsetfillcolor{currentfill}%
\pgfsetlinewidth{0.481800pt}%
\definecolor{currentstroke}{rgb}{1.000000,1.000000,1.000000}%
\pgfsetstrokecolor{currentstroke}%
\pgfsetdash{}{0pt}%
\pgfpathmoveto{\pgfqpoint{5.023692in}{7.139318in}}%
\pgfpathcurveto{\pgfqpoint{5.034742in}{7.139318in}}{\pgfqpoint{5.045342in}{7.143708in}}{\pgfqpoint{5.053155in}{7.151522in}}%
\pgfpathcurveto{\pgfqpoint{5.060969in}{7.159335in}}{\pgfqpoint{5.065359in}{7.169934in}}{\pgfqpoint{5.065359in}{7.180984in}}%
\pgfpathcurveto{\pgfqpoint{5.065359in}{7.192035in}}{\pgfqpoint{5.060969in}{7.202634in}}{\pgfqpoint{5.053155in}{7.210447in}}%
\pgfpathcurveto{\pgfqpoint{5.045342in}{7.218261in}}{\pgfqpoint{5.034742in}{7.222651in}}{\pgfqpoint{5.023692in}{7.222651in}}%
\pgfpathcurveto{\pgfqpoint{5.012642in}{7.222651in}}{\pgfqpoint{5.002043in}{7.218261in}}{\pgfqpoint{4.994230in}{7.210447in}}%
\pgfpathcurveto{\pgfqpoint{4.986416in}{7.202634in}}{\pgfqpoint{4.982026in}{7.192035in}}{\pgfqpoint{4.982026in}{7.180984in}}%
\pgfpathcurveto{\pgfqpoint{4.982026in}{7.169934in}}{\pgfqpoint{4.986416in}{7.159335in}}{\pgfqpoint{4.994230in}{7.151522in}}%
\pgfpathcurveto{\pgfqpoint{5.002043in}{7.143708in}}{\pgfqpoint{5.012642in}{7.139318in}}{\pgfqpoint{5.023692in}{7.139318in}}%
\pgfpathclose%
\pgfusepath{stroke,fill}%
\end{pgfscope}%
\begin{pgfscope}%
\pgfpathrectangle{\pgfqpoint{0.570343in}{0.331635in}}{\pgfqpoint{9.300000in}{7.700000in}}%
\pgfusepath{clip}%
\pgfsetbuttcap%
\pgfsetroundjoin%
\definecolor{currentfill}{rgb}{0.631373,0.788235,0.956863}%
\pgfsetfillcolor{currentfill}%
\pgfsetlinewidth{0.481800pt}%
\definecolor{currentstroke}{rgb}{1.000000,1.000000,1.000000}%
\pgfsetstrokecolor{currentstroke}%
\pgfsetdash{}{0pt}%
\pgfpathmoveto{\pgfqpoint{3.832172in}{4.260196in}}%
\pgfpathcurveto{\pgfqpoint{3.843222in}{4.260196in}}{\pgfqpoint{3.853821in}{4.264586in}}{\pgfqpoint{3.861635in}{4.272399in}}%
\pgfpathcurveto{\pgfqpoint{3.869449in}{4.280213in}}{\pgfqpoint{3.873839in}{4.290812in}}{\pgfqpoint{3.873839in}{4.301862in}}%
\pgfpathcurveto{\pgfqpoint{3.873839in}{4.312912in}}{\pgfqpoint{3.869449in}{4.323511in}}{\pgfqpoint{3.861635in}{4.331325in}}%
\pgfpathcurveto{\pgfqpoint{3.853821in}{4.339139in}}{\pgfqpoint{3.843222in}{4.343529in}}{\pgfqpoint{3.832172in}{4.343529in}}%
\pgfpathcurveto{\pgfqpoint{3.821122in}{4.343529in}}{\pgfqpoint{3.810523in}{4.339139in}}{\pgfqpoint{3.802709in}{4.331325in}}%
\pgfpathcurveto{\pgfqpoint{3.794896in}{4.323511in}}{\pgfqpoint{3.790506in}{4.312912in}}{\pgfqpoint{3.790506in}{4.301862in}}%
\pgfpathcurveto{\pgfqpoint{3.790506in}{4.290812in}}{\pgfqpoint{3.794896in}{4.280213in}}{\pgfqpoint{3.802709in}{4.272399in}}%
\pgfpathcurveto{\pgfqpoint{3.810523in}{4.264586in}}{\pgfqpoint{3.821122in}{4.260196in}}{\pgfqpoint{3.832172in}{4.260196in}}%
\pgfpathclose%
\pgfusepath{stroke,fill}%
\end{pgfscope}%
\begin{pgfscope}%
\pgfpathrectangle{\pgfqpoint{0.570343in}{0.331635in}}{\pgfqpoint{9.300000in}{7.700000in}}%
\pgfusepath{clip}%
\pgfsetbuttcap%
\pgfsetroundjoin%
\definecolor{currentfill}{rgb}{0.631373,0.788235,0.956863}%
\pgfsetfillcolor{currentfill}%
\pgfsetlinewidth{0.481800pt}%
\definecolor{currentstroke}{rgb}{1.000000,1.000000,1.000000}%
\pgfsetstrokecolor{currentstroke}%
\pgfsetdash{}{0pt}%
\pgfpathmoveto{\pgfqpoint{5.168249in}{4.633667in}}%
\pgfpathcurveto{\pgfqpoint{5.179299in}{4.633667in}}{\pgfqpoint{5.189898in}{4.638057in}}{\pgfqpoint{5.197712in}{4.645870in}}%
\pgfpathcurveto{\pgfqpoint{5.205525in}{4.653684in}}{\pgfqpoint{5.209915in}{4.664283in}}{\pgfqpoint{5.209915in}{4.675333in}}%
\pgfpathcurveto{\pgfqpoint{5.209915in}{4.686383in}}{\pgfqpoint{5.205525in}{4.696982in}}{\pgfqpoint{5.197712in}{4.704796in}}%
\pgfpathcurveto{\pgfqpoint{5.189898in}{4.712610in}}{\pgfqpoint{5.179299in}{4.717000in}}{\pgfqpoint{5.168249in}{4.717000in}}%
\pgfpathcurveto{\pgfqpoint{5.157199in}{4.717000in}}{\pgfqpoint{5.146600in}{4.712610in}}{\pgfqpoint{5.138786in}{4.704796in}}%
\pgfpathcurveto{\pgfqpoint{5.130972in}{4.696982in}}{\pgfqpoint{5.126582in}{4.686383in}}{\pgfqpoint{5.126582in}{4.675333in}}%
\pgfpathcurveto{\pgfqpoint{5.126582in}{4.664283in}}{\pgfqpoint{5.130972in}{4.653684in}}{\pgfqpoint{5.138786in}{4.645870in}}%
\pgfpathcurveto{\pgfqpoint{5.146600in}{4.638057in}}{\pgfqpoint{5.157199in}{4.633667in}}{\pgfqpoint{5.168249in}{4.633667in}}%
\pgfpathclose%
\pgfusepath{stroke,fill}%
\end{pgfscope}%
\begin{pgfscope}%
\pgfpathrectangle{\pgfqpoint{0.570343in}{0.331635in}}{\pgfqpoint{9.300000in}{7.700000in}}%
\pgfusepath{clip}%
\pgfsetbuttcap%
\pgfsetroundjoin%
\definecolor{currentfill}{rgb}{0.631373,0.788235,0.956863}%
\pgfsetfillcolor{currentfill}%
\pgfsetlinewidth{0.481800pt}%
\definecolor{currentstroke}{rgb}{1.000000,1.000000,1.000000}%
\pgfsetstrokecolor{currentstroke}%
\pgfsetdash{}{0pt}%
\pgfpathmoveto{\pgfqpoint{4.194348in}{1.372368in}}%
\pgfpathcurveto{\pgfqpoint{4.205398in}{1.372368in}}{\pgfqpoint{4.215997in}{1.376758in}}{\pgfqpoint{4.223811in}{1.384572in}}%
\pgfpathcurveto{\pgfqpoint{4.231624in}{1.392385in}}{\pgfqpoint{4.236015in}{1.402984in}}{\pgfqpoint{4.236015in}{1.414034in}}%
\pgfpathcurveto{\pgfqpoint{4.236015in}{1.425084in}}{\pgfqpoint{4.231624in}{1.435684in}}{\pgfqpoint{4.223811in}{1.443497in}}%
\pgfpathcurveto{\pgfqpoint{4.215997in}{1.451311in}}{\pgfqpoint{4.205398in}{1.455701in}}{\pgfqpoint{4.194348in}{1.455701in}}%
\pgfpathcurveto{\pgfqpoint{4.183298in}{1.455701in}}{\pgfqpoint{4.172699in}{1.451311in}}{\pgfqpoint{4.164885in}{1.443497in}}%
\pgfpathcurveto{\pgfqpoint{4.157072in}{1.435684in}}{\pgfqpoint{4.152681in}{1.425084in}}{\pgfqpoint{4.152681in}{1.414034in}}%
\pgfpathcurveto{\pgfqpoint{4.152681in}{1.402984in}}{\pgfqpoint{4.157072in}{1.392385in}}{\pgfqpoint{4.164885in}{1.384572in}}%
\pgfpathcurveto{\pgfqpoint{4.172699in}{1.376758in}}{\pgfqpoint{4.183298in}{1.372368in}}{\pgfqpoint{4.194348in}{1.372368in}}%
\pgfpathclose%
\pgfusepath{stroke,fill}%
\end{pgfscope}%
\begin{pgfscope}%
\pgfpathrectangle{\pgfqpoint{0.570343in}{0.331635in}}{\pgfqpoint{9.300000in}{7.700000in}}%
\pgfusepath{clip}%
\pgfsetbuttcap%
\pgfsetroundjoin%
\definecolor{currentfill}{rgb}{0.631373,0.788235,0.956863}%
\pgfsetfillcolor{currentfill}%
\pgfsetlinewidth{0.481800pt}%
\definecolor{currentstroke}{rgb}{1.000000,1.000000,1.000000}%
\pgfsetstrokecolor{currentstroke}%
\pgfsetdash{}{0pt}%
\pgfpathmoveto{\pgfqpoint{2.750613in}{5.800171in}}%
\pgfpathcurveto{\pgfqpoint{2.761663in}{5.800171in}}{\pgfqpoint{2.772263in}{5.804561in}}{\pgfqpoint{2.780076in}{5.812374in}}%
\pgfpathcurveto{\pgfqpoint{2.787890in}{5.820188in}}{\pgfqpoint{2.792280in}{5.830787in}}{\pgfqpoint{2.792280in}{5.841837in}}%
\pgfpathcurveto{\pgfqpoint{2.792280in}{5.852887in}}{\pgfqpoint{2.787890in}{5.863486in}}{\pgfqpoint{2.780076in}{5.871300in}}%
\pgfpathcurveto{\pgfqpoint{2.772263in}{5.879114in}}{\pgfqpoint{2.761663in}{5.883504in}}{\pgfqpoint{2.750613in}{5.883504in}}%
\pgfpathcurveto{\pgfqpoint{2.739563in}{5.883504in}}{\pgfqpoint{2.728964in}{5.879114in}}{\pgfqpoint{2.721151in}{5.871300in}}%
\pgfpathcurveto{\pgfqpoint{2.713337in}{5.863486in}}{\pgfqpoint{2.708947in}{5.852887in}}{\pgfqpoint{2.708947in}{5.841837in}}%
\pgfpathcurveto{\pgfqpoint{2.708947in}{5.830787in}}{\pgfqpoint{2.713337in}{5.820188in}}{\pgfqpoint{2.721151in}{5.812374in}}%
\pgfpathcurveto{\pgfqpoint{2.728964in}{5.804561in}}{\pgfqpoint{2.739563in}{5.800171in}}{\pgfqpoint{2.750613in}{5.800171in}}%
\pgfpathclose%
\pgfusepath{stroke,fill}%
\end{pgfscope}%
\begin{pgfscope}%
\pgfpathrectangle{\pgfqpoint{0.570343in}{0.331635in}}{\pgfqpoint{9.300000in}{7.700000in}}%
\pgfusepath{clip}%
\pgfsetbuttcap%
\pgfsetroundjoin%
\definecolor{currentfill}{rgb}{0.631373,0.788235,0.956863}%
\pgfsetfillcolor{currentfill}%
\pgfsetlinewidth{0.481800pt}%
\definecolor{currentstroke}{rgb}{1.000000,1.000000,1.000000}%
\pgfsetstrokecolor{currentstroke}%
\pgfsetdash{}{0pt}%
\pgfpathmoveto{\pgfqpoint{3.118617in}{2.356278in}}%
\pgfpathcurveto{\pgfqpoint{3.129667in}{2.356278in}}{\pgfqpoint{3.140266in}{2.360668in}}{\pgfqpoint{3.148080in}{2.368482in}}%
\pgfpathcurveto{\pgfqpoint{3.155893in}{2.376295in}}{\pgfqpoint{3.160284in}{2.386894in}}{\pgfqpoint{3.160284in}{2.397944in}}%
\pgfpathcurveto{\pgfqpoint{3.160284in}{2.408994in}}{\pgfqpoint{3.155893in}{2.419593in}}{\pgfqpoint{3.148080in}{2.427407in}}%
\pgfpathcurveto{\pgfqpoint{3.140266in}{2.435221in}}{\pgfqpoint{3.129667in}{2.439611in}}{\pgfqpoint{3.118617in}{2.439611in}}%
\pgfpathcurveto{\pgfqpoint{3.107567in}{2.439611in}}{\pgfqpoint{3.096968in}{2.435221in}}{\pgfqpoint{3.089154in}{2.427407in}}%
\pgfpathcurveto{\pgfqpoint{3.081341in}{2.419593in}}{\pgfqpoint{3.076950in}{2.408994in}}{\pgfqpoint{3.076950in}{2.397944in}}%
\pgfpathcurveto{\pgfqpoint{3.076950in}{2.386894in}}{\pgfqpoint{3.081341in}{2.376295in}}{\pgfqpoint{3.089154in}{2.368482in}}%
\pgfpathcurveto{\pgfqpoint{3.096968in}{2.360668in}}{\pgfqpoint{3.107567in}{2.356278in}}{\pgfqpoint{3.118617in}{2.356278in}}%
\pgfpathclose%
\pgfusepath{stroke,fill}%
\end{pgfscope}%
\begin{pgfscope}%
\pgfpathrectangle{\pgfqpoint{0.570343in}{0.331635in}}{\pgfqpoint{9.300000in}{7.700000in}}%
\pgfusepath{clip}%
\pgfsetbuttcap%
\pgfsetroundjoin%
\definecolor{currentfill}{rgb}{0.631373,0.788235,0.956863}%
\pgfsetfillcolor{currentfill}%
\pgfsetlinewidth{0.481800pt}%
\definecolor{currentstroke}{rgb}{1.000000,1.000000,1.000000}%
\pgfsetstrokecolor{currentstroke}%
\pgfsetdash{}{0pt}%
\pgfpathmoveto{\pgfqpoint{5.057402in}{1.848505in}}%
\pgfpathcurveto{\pgfqpoint{5.068453in}{1.848505in}}{\pgfqpoint{5.079052in}{1.852895in}}{\pgfqpoint{5.086865in}{1.860709in}}%
\pgfpathcurveto{\pgfqpoint{5.094679in}{1.868522in}}{\pgfqpoint{5.099069in}{1.879121in}}{\pgfqpoint{5.099069in}{1.890172in}}%
\pgfpathcurveto{\pgfqpoint{5.099069in}{1.901222in}}{\pgfqpoint{5.094679in}{1.911821in}}{\pgfqpoint{5.086865in}{1.919634in}}%
\pgfpathcurveto{\pgfqpoint{5.079052in}{1.927448in}}{\pgfqpoint{5.068453in}{1.931838in}}{\pgfqpoint{5.057402in}{1.931838in}}%
\pgfpathcurveto{\pgfqpoint{5.046352in}{1.931838in}}{\pgfqpoint{5.035753in}{1.927448in}}{\pgfqpoint{5.027940in}{1.919634in}}%
\pgfpathcurveto{\pgfqpoint{5.020126in}{1.911821in}}{\pgfqpoint{5.015736in}{1.901222in}}{\pgfqpoint{5.015736in}{1.890172in}}%
\pgfpathcurveto{\pgfqpoint{5.015736in}{1.879121in}}{\pgfqpoint{5.020126in}{1.868522in}}{\pgfqpoint{5.027940in}{1.860709in}}%
\pgfpathcurveto{\pgfqpoint{5.035753in}{1.852895in}}{\pgfqpoint{5.046352in}{1.848505in}}{\pgfqpoint{5.057402in}{1.848505in}}%
\pgfpathclose%
\pgfusepath{stroke,fill}%
\end{pgfscope}%
\begin{pgfscope}%
\pgfpathrectangle{\pgfqpoint{0.570343in}{0.331635in}}{\pgfqpoint{9.300000in}{7.700000in}}%
\pgfusepath{clip}%
\pgfsetbuttcap%
\pgfsetroundjoin%
\definecolor{currentfill}{rgb}{0.631373,0.788235,0.956863}%
\pgfsetfillcolor{currentfill}%
\pgfsetlinewidth{0.481800pt}%
\definecolor{currentstroke}{rgb}{1.000000,1.000000,1.000000}%
\pgfsetstrokecolor{currentstroke}%
\pgfsetdash{}{0pt}%
\pgfpathmoveto{\pgfqpoint{5.910435in}{2.316238in}}%
\pgfpathcurveto{\pgfqpoint{5.921485in}{2.316238in}}{\pgfqpoint{5.932084in}{2.320628in}}{\pgfqpoint{5.939898in}{2.328441in}}%
\pgfpathcurveto{\pgfqpoint{5.947712in}{2.336255in}}{\pgfqpoint{5.952102in}{2.346854in}}{\pgfqpoint{5.952102in}{2.357904in}}%
\pgfpathcurveto{\pgfqpoint{5.952102in}{2.368954in}}{\pgfqpoint{5.947712in}{2.379553in}}{\pgfqpoint{5.939898in}{2.387367in}}%
\pgfpathcurveto{\pgfqpoint{5.932084in}{2.395181in}}{\pgfqpoint{5.921485in}{2.399571in}}{\pgfqpoint{5.910435in}{2.399571in}}%
\pgfpathcurveto{\pgfqpoint{5.899385in}{2.399571in}}{\pgfqpoint{5.888786in}{2.395181in}}{\pgfqpoint{5.880972in}{2.387367in}}%
\pgfpathcurveto{\pgfqpoint{5.873159in}{2.379553in}}{\pgfqpoint{5.868769in}{2.368954in}}{\pgfqpoint{5.868769in}{2.357904in}}%
\pgfpathcurveto{\pgfqpoint{5.868769in}{2.346854in}}{\pgfqpoint{5.873159in}{2.336255in}}{\pgfqpoint{5.880972in}{2.328441in}}%
\pgfpathcurveto{\pgfqpoint{5.888786in}{2.320628in}}{\pgfqpoint{5.899385in}{2.316238in}}{\pgfqpoint{5.910435in}{2.316238in}}%
\pgfpathclose%
\pgfusepath{stroke,fill}%
\end{pgfscope}%
\begin{pgfscope}%
\pgfpathrectangle{\pgfqpoint{0.570343in}{0.331635in}}{\pgfqpoint{9.300000in}{7.700000in}}%
\pgfusepath{clip}%
\pgfsetbuttcap%
\pgfsetroundjoin%
\definecolor{currentfill}{rgb}{0.631373,0.788235,0.956863}%
\pgfsetfillcolor{currentfill}%
\pgfsetlinewidth{0.481800pt}%
\definecolor{currentstroke}{rgb}{1.000000,1.000000,1.000000}%
\pgfsetstrokecolor{currentstroke}%
\pgfsetdash{}{0pt}%
\pgfpathmoveto{\pgfqpoint{5.846427in}{5.254826in}}%
\pgfpathcurveto{\pgfqpoint{5.857477in}{5.254826in}}{\pgfqpoint{5.868076in}{5.259217in}}{\pgfqpoint{5.875890in}{5.267030in}}%
\pgfpathcurveto{\pgfqpoint{5.883704in}{5.274844in}}{\pgfqpoint{5.888094in}{5.285443in}}{\pgfqpoint{5.888094in}{5.296493in}}%
\pgfpathcurveto{\pgfqpoint{5.888094in}{5.307543in}}{\pgfqpoint{5.883704in}{5.318142in}}{\pgfqpoint{5.875890in}{5.325956in}}%
\pgfpathcurveto{\pgfqpoint{5.868076in}{5.333769in}}{\pgfqpoint{5.857477in}{5.338160in}}{\pgfqpoint{5.846427in}{5.338160in}}%
\pgfpathcurveto{\pgfqpoint{5.835377in}{5.338160in}}{\pgfqpoint{5.824778in}{5.333769in}}{\pgfqpoint{5.816964in}{5.325956in}}%
\pgfpathcurveto{\pgfqpoint{5.809151in}{5.318142in}}{\pgfqpoint{5.804761in}{5.307543in}}{\pgfqpoint{5.804761in}{5.296493in}}%
\pgfpathcurveto{\pgfqpoint{5.804761in}{5.285443in}}{\pgfqpoint{5.809151in}{5.274844in}}{\pgfqpoint{5.816964in}{5.267030in}}%
\pgfpathcurveto{\pgfqpoint{5.824778in}{5.259217in}}{\pgfqpoint{5.835377in}{5.254826in}}{\pgfqpoint{5.846427in}{5.254826in}}%
\pgfpathclose%
\pgfusepath{stroke,fill}%
\end{pgfscope}%
\begin{pgfscope}%
\pgfpathrectangle{\pgfqpoint{0.570343in}{0.331635in}}{\pgfqpoint{9.300000in}{7.700000in}}%
\pgfusepath{clip}%
\pgfsetbuttcap%
\pgfsetroundjoin%
\definecolor{currentfill}{rgb}{0.631373,0.788235,0.956863}%
\pgfsetfillcolor{currentfill}%
\pgfsetlinewidth{0.481800pt}%
\definecolor{currentstroke}{rgb}{1.000000,1.000000,1.000000}%
\pgfsetstrokecolor{currentstroke}%
\pgfsetdash{}{0pt}%
\pgfpathmoveto{\pgfqpoint{4.284170in}{4.749655in}}%
\pgfpathcurveto{\pgfqpoint{4.295220in}{4.749655in}}{\pgfqpoint{4.305820in}{4.754045in}}{\pgfqpoint{4.313633in}{4.761859in}}%
\pgfpathcurveto{\pgfqpoint{4.321447in}{4.769672in}}{\pgfqpoint{4.325837in}{4.780271in}}{\pgfqpoint{4.325837in}{4.791321in}}%
\pgfpathcurveto{\pgfqpoint{4.325837in}{4.802372in}}{\pgfqpoint{4.321447in}{4.812971in}}{\pgfqpoint{4.313633in}{4.820784in}}%
\pgfpathcurveto{\pgfqpoint{4.305820in}{4.828598in}}{\pgfqpoint{4.295220in}{4.832988in}}{\pgfqpoint{4.284170in}{4.832988in}}%
\pgfpathcurveto{\pgfqpoint{4.273120in}{4.832988in}}{\pgfqpoint{4.262521in}{4.828598in}}{\pgfqpoint{4.254708in}{4.820784in}}%
\pgfpathcurveto{\pgfqpoint{4.246894in}{4.812971in}}{\pgfqpoint{4.242504in}{4.802372in}}{\pgfqpoint{4.242504in}{4.791321in}}%
\pgfpathcurveto{\pgfqpoint{4.242504in}{4.780271in}}{\pgfqpoint{4.246894in}{4.769672in}}{\pgfqpoint{4.254708in}{4.761859in}}%
\pgfpathcurveto{\pgfqpoint{4.262521in}{4.754045in}}{\pgfqpoint{4.273120in}{4.749655in}}{\pgfqpoint{4.284170in}{4.749655in}}%
\pgfpathclose%
\pgfusepath{stroke,fill}%
\end{pgfscope}%
\begin{pgfscope}%
\pgfpathrectangle{\pgfqpoint{0.570343in}{0.331635in}}{\pgfqpoint{9.300000in}{7.700000in}}%
\pgfusepath{clip}%
\pgfsetbuttcap%
\pgfsetroundjoin%
\definecolor{currentfill}{rgb}{0.631373,0.788235,0.956863}%
\pgfsetfillcolor{currentfill}%
\pgfsetlinewidth{0.481800pt}%
\definecolor{currentstroke}{rgb}{1.000000,1.000000,1.000000}%
\pgfsetstrokecolor{currentstroke}%
\pgfsetdash{}{0pt}%
\pgfpathmoveto{\pgfqpoint{1.673868in}{2.866832in}}%
\pgfpathcurveto{\pgfqpoint{1.684918in}{2.866832in}}{\pgfqpoint{1.695517in}{2.871222in}}{\pgfqpoint{1.703331in}{2.879035in}}%
\pgfpathcurveto{\pgfqpoint{1.711144in}{2.886849in}}{\pgfqpoint{1.715535in}{2.897448in}}{\pgfqpoint{1.715535in}{2.908498in}}%
\pgfpathcurveto{\pgfqpoint{1.715535in}{2.919548in}}{\pgfqpoint{1.711144in}{2.930147in}}{\pgfqpoint{1.703331in}{2.937961in}}%
\pgfpathcurveto{\pgfqpoint{1.695517in}{2.945775in}}{\pgfqpoint{1.684918in}{2.950165in}}{\pgfqpoint{1.673868in}{2.950165in}}%
\pgfpathcurveto{\pgfqpoint{1.662818in}{2.950165in}}{\pgfqpoint{1.652219in}{2.945775in}}{\pgfqpoint{1.644405in}{2.937961in}}%
\pgfpathcurveto{\pgfqpoint{1.636592in}{2.930147in}}{\pgfqpoint{1.632201in}{2.919548in}}{\pgfqpoint{1.632201in}{2.908498in}}%
\pgfpathcurveto{\pgfqpoint{1.632201in}{2.897448in}}{\pgfqpoint{1.636592in}{2.886849in}}{\pgfqpoint{1.644405in}{2.879035in}}%
\pgfpathcurveto{\pgfqpoint{1.652219in}{2.871222in}}{\pgfqpoint{1.662818in}{2.866832in}}{\pgfqpoint{1.673868in}{2.866832in}}%
\pgfpathclose%
\pgfusepath{stroke,fill}%
\end{pgfscope}%
\begin{pgfscope}%
\pgfpathrectangle{\pgfqpoint{0.570343in}{0.331635in}}{\pgfqpoint{9.300000in}{7.700000in}}%
\pgfusepath{clip}%
\pgfsetbuttcap%
\pgfsetroundjoin%
\definecolor{currentfill}{rgb}{0.631373,0.788235,0.956863}%
\pgfsetfillcolor{currentfill}%
\pgfsetlinewidth{0.481800pt}%
\definecolor{currentstroke}{rgb}{1.000000,1.000000,1.000000}%
\pgfsetstrokecolor{currentstroke}%
\pgfsetdash{}{0pt}%
\pgfpathmoveto{\pgfqpoint{4.453376in}{3.392004in}}%
\pgfpathcurveto{\pgfqpoint{4.464426in}{3.392004in}}{\pgfqpoint{4.475025in}{3.396394in}}{\pgfqpoint{4.482839in}{3.404208in}}%
\pgfpathcurveto{\pgfqpoint{4.490652in}{3.412022in}}{\pgfqpoint{4.495042in}{3.422621in}}{\pgfqpoint{4.495042in}{3.433671in}}%
\pgfpathcurveto{\pgfqpoint{4.495042in}{3.444721in}}{\pgfqpoint{4.490652in}{3.455320in}}{\pgfqpoint{4.482839in}{3.463134in}}%
\pgfpathcurveto{\pgfqpoint{4.475025in}{3.470947in}}{\pgfqpoint{4.464426in}{3.475337in}}{\pgfqpoint{4.453376in}{3.475337in}}%
\pgfpathcurveto{\pgfqpoint{4.442326in}{3.475337in}}{\pgfqpoint{4.431727in}{3.470947in}}{\pgfqpoint{4.423913in}{3.463134in}}%
\pgfpathcurveto{\pgfqpoint{4.416099in}{3.455320in}}{\pgfqpoint{4.411709in}{3.444721in}}{\pgfqpoint{4.411709in}{3.433671in}}%
\pgfpathcurveto{\pgfqpoint{4.411709in}{3.422621in}}{\pgfqpoint{4.416099in}{3.412022in}}{\pgfqpoint{4.423913in}{3.404208in}}%
\pgfpathcurveto{\pgfqpoint{4.431727in}{3.396394in}}{\pgfqpoint{4.442326in}{3.392004in}}{\pgfqpoint{4.453376in}{3.392004in}}%
\pgfpathclose%
\pgfusepath{stroke,fill}%
\end{pgfscope}%
\begin{pgfscope}%
\pgfpathrectangle{\pgfqpoint{0.570343in}{0.331635in}}{\pgfqpoint{9.300000in}{7.700000in}}%
\pgfusepath{clip}%
\pgfsetbuttcap%
\pgfsetroundjoin%
\definecolor{currentfill}{rgb}{0.631373,0.788235,0.956863}%
\pgfsetfillcolor{currentfill}%
\pgfsetlinewidth{0.481800pt}%
\definecolor{currentstroke}{rgb}{1.000000,1.000000,1.000000}%
\pgfsetstrokecolor{currentstroke}%
\pgfsetdash{}{0pt}%
\pgfpathmoveto{\pgfqpoint{5.511077in}{3.346155in}}%
\pgfpathcurveto{\pgfqpoint{5.522127in}{3.346155in}}{\pgfqpoint{5.532726in}{3.350545in}}{\pgfqpoint{5.540540in}{3.358359in}}%
\pgfpathcurveto{\pgfqpoint{5.548353in}{3.366173in}}{\pgfqpoint{5.552744in}{3.376772in}}{\pgfqpoint{5.552744in}{3.387822in}}%
\pgfpathcurveto{\pgfqpoint{5.552744in}{3.398872in}}{\pgfqpoint{5.548353in}{3.409471in}}{\pgfqpoint{5.540540in}{3.417284in}}%
\pgfpathcurveto{\pgfqpoint{5.532726in}{3.425098in}}{\pgfqpoint{5.522127in}{3.429488in}}{\pgfqpoint{5.511077in}{3.429488in}}%
\pgfpathcurveto{\pgfqpoint{5.500027in}{3.429488in}}{\pgfqpoint{5.489428in}{3.425098in}}{\pgfqpoint{5.481614in}{3.417284in}}%
\pgfpathcurveto{\pgfqpoint{5.473801in}{3.409471in}}{\pgfqpoint{5.469410in}{3.398872in}}{\pgfqpoint{5.469410in}{3.387822in}}%
\pgfpathcurveto{\pgfqpoint{5.469410in}{3.376772in}}{\pgfqpoint{5.473801in}{3.366173in}}{\pgfqpoint{5.481614in}{3.358359in}}%
\pgfpathcurveto{\pgfqpoint{5.489428in}{3.350545in}}{\pgfqpoint{5.500027in}{3.346155in}}{\pgfqpoint{5.511077in}{3.346155in}}%
\pgfpathclose%
\pgfusepath{stroke,fill}%
\end{pgfscope}%
\begin{pgfscope}%
\pgfpathrectangle{\pgfqpoint{0.570343in}{0.331635in}}{\pgfqpoint{9.300000in}{7.700000in}}%
\pgfusepath{clip}%
\pgfsetbuttcap%
\pgfsetroundjoin%
\definecolor{currentfill}{rgb}{0.631373,0.788235,0.956863}%
\pgfsetfillcolor{currentfill}%
\pgfsetlinewidth{0.481800pt}%
\definecolor{currentstroke}{rgb}{1.000000,1.000000,1.000000}%
\pgfsetstrokecolor{currentstroke}%
\pgfsetdash{}{0pt}%
\pgfpathmoveto{\pgfqpoint{6.465570in}{3.537469in}}%
\pgfpathcurveto{\pgfqpoint{6.476620in}{3.537469in}}{\pgfqpoint{6.487219in}{3.541859in}}{\pgfqpoint{6.495033in}{3.549673in}}%
\pgfpathcurveto{\pgfqpoint{6.502846in}{3.557486in}}{\pgfqpoint{6.507237in}{3.568085in}}{\pgfqpoint{6.507237in}{3.579135in}}%
\pgfpathcurveto{\pgfqpoint{6.507237in}{3.590186in}}{\pgfqpoint{6.502846in}{3.600785in}}{\pgfqpoint{6.495033in}{3.608598in}}%
\pgfpathcurveto{\pgfqpoint{6.487219in}{3.616412in}}{\pgfqpoint{6.476620in}{3.620802in}}{\pgfqpoint{6.465570in}{3.620802in}}%
\pgfpathcurveto{\pgfqpoint{6.454520in}{3.620802in}}{\pgfqpoint{6.443921in}{3.616412in}}{\pgfqpoint{6.436107in}{3.608598in}}%
\pgfpathcurveto{\pgfqpoint{6.428294in}{3.600785in}}{\pgfqpoint{6.423903in}{3.590186in}}{\pgfqpoint{6.423903in}{3.579135in}}%
\pgfpathcurveto{\pgfqpoint{6.423903in}{3.568085in}}{\pgfqpoint{6.428294in}{3.557486in}}{\pgfqpoint{6.436107in}{3.549673in}}%
\pgfpathcurveto{\pgfqpoint{6.443921in}{3.541859in}}{\pgfqpoint{6.454520in}{3.537469in}}{\pgfqpoint{6.465570in}{3.537469in}}%
\pgfpathclose%
\pgfusepath{stroke,fill}%
\end{pgfscope}%
\begin{pgfscope}%
\pgfpathrectangle{\pgfqpoint{0.570343in}{0.331635in}}{\pgfqpoint{9.300000in}{7.700000in}}%
\pgfusepath{clip}%
\pgfsetbuttcap%
\pgfsetroundjoin%
\definecolor{currentfill}{rgb}{0.631373,0.788235,0.956863}%
\pgfsetfillcolor{currentfill}%
\pgfsetlinewidth{0.481800pt}%
\definecolor{currentstroke}{rgb}{1.000000,1.000000,1.000000}%
\pgfsetstrokecolor{currentstroke}%
\pgfsetdash{}{0pt}%
\pgfpathmoveto{\pgfqpoint{5.213120in}{0.639968in}}%
\pgfpathcurveto{\pgfqpoint{5.224170in}{0.639968in}}{\pgfqpoint{5.234769in}{0.644359in}}{\pgfqpoint{5.242583in}{0.652172in}}%
\pgfpathcurveto{\pgfqpoint{5.250396in}{0.659986in}}{\pgfqpoint{5.254787in}{0.670585in}}{\pgfqpoint{5.254787in}{0.681635in}}%
\pgfpathcurveto{\pgfqpoint{5.254787in}{0.692685in}}{\pgfqpoint{5.250396in}{0.703284in}}{\pgfqpoint{5.242583in}{0.711098in}}%
\pgfpathcurveto{\pgfqpoint{5.234769in}{0.718911in}}{\pgfqpoint{5.224170in}{0.723302in}}{\pgfqpoint{5.213120in}{0.723302in}}%
\pgfpathcurveto{\pgfqpoint{5.202070in}{0.723302in}}{\pgfqpoint{5.191471in}{0.718911in}}{\pgfqpoint{5.183657in}{0.711098in}}%
\pgfpathcurveto{\pgfqpoint{5.175844in}{0.703284in}}{\pgfqpoint{5.171453in}{0.692685in}}{\pgfqpoint{5.171453in}{0.681635in}}%
\pgfpathcurveto{\pgfqpoint{5.171453in}{0.670585in}}{\pgfqpoint{5.175844in}{0.659986in}}{\pgfqpoint{5.183657in}{0.652172in}}%
\pgfpathcurveto{\pgfqpoint{5.191471in}{0.644359in}}{\pgfqpoint{5.202070in}{0.639968in}}{\pgfqpoint{5.213120in}{0.639968in}}%
\pgfpathclose%
\pgfusepath{stroke,fill}%
\end{pgfscope}%
\begin{pgfscope}%
\pgfpathrectangle{\pgfqpoint{0.570343in}{0.331635in}}{\pgfqpoint{9.300000in}{7.700000in}}%
\pgfusepath{clip}%
\pgfsetbuttcap%
\pgfsetroundjoin%
\definecolor{currentfill}{rgb}{0.631373,0.788235,0.956863}%
\pgfsetfillcolor{currentfill}%
\pgfsetlinewidth{0.481800pt}%
\definecolor{currentstroke}{rgb}{1.000000,1.000000,1.000000}%
\pgfsetstrokecolor{currentstroke}%
\pgfsetdash{}{0pt}%
\pgfpathmoveto{\pgfqpoint{9.447616in}{4.687116in}}%
\pgfpathcurveto{\pgfqpoint{9.458666in}{4.687116in}}{\pgfqpoint{9.469265in}{4.691506in}}{\pgfqpoint{9.477079in}{4.699320in}}%
\pgfpathcurveto{\pgfqpoint{9.484892in}{4.707133in}}{\pgfqpoint{9.489283in}{4.717732in}}{\pgfqpoint{9.489283in}{4.728782in}}%
\pgfpathcurveto{\pgfqpoint{9.489283in}{4.739833in}}{\pgfqpoint{9.484892in}{4.750432in}}{\pgfqpoint{9.477079in}{4.758245in}}%
\pgfpathcurveto{\pgfqpoint{9.469265in}{4.766059in}}{\pgfqpoint{9.458666in}{4.770449in}}{\pgfqpoint{9.447616in}{4.770449in}}%
\pgfpathcurveto{\pgfqpoint{9.436566in}{4.770449in}}{\pgfqpoint{9.425967in}{4.766059in}}{\pgfqpoint{9.418153in}{4.758245in}}%
\pgfpathcurveto{\pgfqpoint{9.410340in}{4.750432in}}{\pgfqpoint{9.405949in}{4.739833in}}{\pgfqpoint{9.405949in}{4.728782in}}%
\pgfpathcurveto{\pgfqpoint{9.405949in}{4.717732in}}{\pgfqpoint{9.410340in}{4.707133in}}{\pgfqpoint{9.418153in}{4.699320in}}%
\pgfpathcurveto{\pgfqpoint{9.425967in}{4.691506in}}{\pgfqpoint{9.436566in}{4.687116in}}{\pgfqpoint{9.447616in}{4.687116in}}%
\pgfpathclose%
\pgfusepath{stroke,fill}%
\end{pgfscope}%
\begin{pgfscope}%
\pgfpathrectangle{\pgfqpoint{0.570343in}{0.331635in}}{\pgfqpoint{9.300000in}{7.700000in}}%
\pgfusepath{clip}%
\pgfsetbuttcap%
\pgfsetroundjoin%
\definecolor{currentfill}{rgb}{0.631373,0.788235,0.956863}%
\pgfsetfillcolor{currentfill}%
\pgfsetlinewidth{0.481800pt}%
\definecolor{currentstroke}{rgb}{1.000000,1.000000,1.000000}%
\pgfsetstrokecolor{currentstroke}%
\pgfsetdash{}{0pt}%
\pgfpathmoveto{\pgfqpoint{6.337177in}{4.231956in}}%
\pgfpathcurveto{\pgfqpoint{6.348227in}{4.231956in}}{\pgfqpoint{6.358826in}{4.236346in}}{\pgfqpoint{6.366639in}{4.244160in}}%
\pgfpathcurveto{\pgfqpoint{6.374453in}{4.251974in}}{\pgfqpoint{6.378843in}{4.262573in}}{\pgfqpoint{6.378843in}{4.273623in}}%
\pgfpathcurveto{\pgfqpoint{6.378843in}{4.284673in}}{\pgfqpoint{6.374453in}{4.295272in}}{\pgfqpoint{6.366639in}{4.303085in}}%
\pgfpathcurveto{\pgfqpoint{6.358826in}{4.310899in}}{\pgfqpoint{6.348227in}{4.315289in}}{\pgfqpoint{6.337177in}{4.315289in}}%
\pgfpathcurveto{\pgfqpoint{6.326127in}{4.315289in}}{\pgfqpoint{6.315528in}{4.310899in}}{\pgfqpoint{6.307714in}{4.303085in}}%
\pgfpathcurveto{\pgfqpoint{6.299900in}{4.295272in}}{\pgfqpoint{6.295510in}{4.284673in}}{\pgfqpoint{6.295510in}{4.273623in}}%
\pgfpathcurveto{\pgfqpoint{6.295510in}{4.262573in}}{\pgfqpoint{6.299900in}{4.251974in}}{\pgfqpoint{6.307714in}{4.244160in}}%
\pgfpathcurveto{\pgfqpoint{6.315528in}{4.236346in}}{\pgfqpoint{6.326127in}{4.231956in}}{\pgfqpoint{6.337177in}{4.231956in}}%
\pgfpathclose%
\pgfusepath{stroke,fill}%
\end{pgfscope}%
\begin{pgfscope}%
\pgfpathrectangle{\pgfqpoint{0.570343in}{0.331635in}}{\pgfqpoint{9.300000in}{7.700000in}}%
\pgfusepath{clip}%
\pgfsetbuttcap%
\pgfsetroundjoin%
\definecolor{currentfill}{rgb}{0.631373,0.788235,0.956863}%
\pgfsetfillcolor{currentfill}%
\pgfsetlinewidth{0.481800pt}%
\definecolor{currentstroke}{rgb}{1.000000,1.000000,1.000000}%
\pgfsetstrokecolor{currentstroke}%
\pgfsetdash{}{0pt}%
\pgfpathmoveto{\pgfqpoint{3.060240in}{4.505893in}}%
\pgfpathcurveto{\pgfqpoint{3.071291in}{4.505893in}}{\pgfqpoint{3.081890in}{4.510284in}}{\pgfqpoint{3.089703in}{4.518097in}}%
\pgfpathcurveto{\pgfqpoint{3.097517in}{4.525911in}}{\pgfqpoint{3.101907in}{4.536510in}}{\pgfqpoint{3.101907in}{4.547560in}}%
\pgfpathcurveto{\pgfqpoint{3.101907in}{4.558610in}}{\pgfqpoint{3.097517in}{4.569209in}}{\pgfqpoint{3.089703in}{4.577023in}}%
\pgfpathcurveto{\pgfqpoint{3.081890in}{4.584836in}}{\pgfqpoint{3.071291in}{4.589227in}}{\pgfqpoint{3.060240in}{4.589227in}}%
\pgfpathcurveto{\pgfqpoint{3.049190in}{4.589227in}}{\pgfqpoint{3.038591in}{4.584836in}}{\pgfqpoint{3.030778in}{4.577023in}}%
\pgfpathcurveto{\pgfqpoint{3.022964in}{4.569209in}}{\pgfqpoint{3.018574in}{4.558610in}}{\pgfqpoint{3.018574in}{4.547560in}}%
\pgfpathcurveto{\pgfqpoint{3.018574in}{4.536510in}}{\pgfqpoint{3.022964in}{4.525911in}}{\pgfqpoint{3.030778in}{4.518097in}}%
\pgfpathcurveto{\pgfqpoint{3.038591in}{4.510284in}}{\pgfqpoint{3.049190in}{4.505893in}}{\pgfqpoint{3.060240in}{4.505893in}}%
\pgfpathclose%
\pgfusepath{stroke,fill}%
\end{pgfscope}%
\begin{pgfscope}%
\pgfpathrectangle{\pgfqpoint{0.570343in}{0.331635in}}{\pgfqpoint{9.300000in}{7.700000in}}%
\pgfusepath{clip}%
\pgfsetbuttcap%
\pgfsetroundjoin%
\definecolor{currentfill}{rgb}{0.631373,0.788235,0.956863}%
\pgfsetfillcolor{currentfill}%
\pgfsetlinewidth{0.481800pt}%
\definecolor{currentstroke}{rgb}{1.000000,1.000000,1.000000}%
\pgfsetstrokecolor{currentstroke}%
\pgfsetdash{}{0pt}%
\pgfpathmoveto{\pgfqpoint{0.993071in}{5.114428in}}%
\pgfpathcurveto{\pgfqpoint{1.004121in}{5.114428in}}{\pgfqpoint{1.014720in}{5.118818in}}{\pgfqpoint{1.022533in}{5.126632in}}%
\pgfpathcurveto{\pgfqpoint{1.030347in}{5.134446in}}{\pgfqpoint{1.034737in}{5.145045in}}{\pgfqpoint{1.034737in}{5.156095in}}%
\pgfpathcurveto{\pgfqpoint{1.034737in}{5.167145in}}{\pgfqpoint{1.030347in}{5.177744in}}{\pgfqpoint{1.022533in}{5.185558in}}%
\pgfpathcurveto{\pgfqpoint{1.014720in}{5.193371in}}{\pgfqpoint{1.004121in}{5.197762in}}{\pgfqpoint{0.993071in}{5.197762in}}%
\pgfpathcurveto{\pgfqpoint{0.982020in}{5.197762in}}{\pgfqpoint{0.971421in}{5.193371in}}{\pgfqpoint{0.963608in}{5.185558in}}%
\pgfpathcurveto{\pgfqpoint{0.955794in}{5.177744in}}{\pgfqpoint{0.951404in}{5.167145in}}{\pgfqpoint{0.951404in}{5.156095in}}%
\pgfpathcurveto{\pgfqpoint{0.951404in}{5.145045in}}{\pgfqpoint{0.955794in}{5.134446in}}{\pgfqpoint{0.963608in}{5.126632in}}%
\pgfpathcurveto{\pgfqpoint{0.971421in}{5.118818in}}{\pgfqpoint{0.982020in}{5.114428in}}{\pgfqpoint{0.993071in}{5.114428in}}%
\pgfpathclose%
\pgfusepath{stroke,fill}%
\end{pgfscope}%
\begin{pgfscope}%
\pgfpathrectangle{\pgfqpoint{0.570343in}{0.331635in}}{\pgfqpoint{9.300000in}{7.700000in}}%
\pgfusepath{clip}%
\pgfsetbuttcap%
\pgfsetroundjoin%
\definecolor{currentfill}{rgb}{0.631373,0.788235,0.956863}%
\pgfsetfillcolor{currentfill}%
\pgfsetlinewidth{0.481800pt}%
\definecolor{currentstroke}{rgb}{1.000000,1.000000,1.000000}%
\pgfsetstrokecolor{currentstroke}%
\pgfsetdash{}{0pt}%
\pgfpathmoveto{\pgfqpoint{6.861115in}{0.792873in}}%
\pgfpathcurveto{\pgfqpoint{6.872165in}{0.792873in}}{\pgfqpoint{6.882764in}{0.797263in}}{\pgfqpoint{6.890577in}{0.805077in}}%
\pgfpathcurveto{\pgfqpoint{6.898391in}{0.812891in}}{\pgfqpoint{6.902781in}{0.823490in}}{\pgfqpoint{6.902781in}{0.834540in}}%
\pgfpathcurveto{\pgfqpoint{6.902781in}{0.845590in}}{\pgfqpoint{6.898391in}{0.856189in}}{\pgfqpoint{6.890577in}{0.864003in}}%
\pgfpathcurveto{\pgfqpoint{6.882764in}{0.871816in}}{\pgfqpoint{6.872165in}{0.876207in}}{\pgfqpoint{6.861115in}{0.876207in}}%
\pgfpathcurveto{\pgfqpoint{6.850065in}{0.876207in}}{\pgfqpoint{6.839466in}{0.871816in}}{\pgfqpoint{6.831652in}{0.864003in}}%
\pgfpathcurveto{\pgfqpoint{6.823838in}{0.856189in}}{\pgfqpoint{6.819448in}{0.845590in}}{\pgfqpoint{6.819448in}{0.834540in}}%
\pgfpathcurveto{\pgfqpoint{6.819448in}{0.823490in}}{\pgfqpoint{6.823838in}{0.812891in}}{\pgfqpoint{6.831652in}{0.805077in}}%
\pgfpathcurveto{\pgfqpoint{6.839466in}{0.797263in}}{\pgfqpoint{6.850065in}{0.792873in}}{\pgfqpoint{6.861115in}{0.792873in}}%
\pgfpathclose%
\pgfusepath{stroke,fill}%
\end{pgfscope}%
\begin{pgfscope}%
\pgfpathrectangle{\pgfqpoint{0.570343in}{0.331635in}}{\pgfqpoint{9.300000in}{7.700000in}}%
\pgfusepath{clip}%
\pgfsetbuttcap%
\pgfsetroundjoin%
\definecolor{currentfill}{rgb}{0.631373,0.788235,0.956863}%
\pgfsetfillcolor{currentfill}%
\pgfsetlinewidth{0.481800pt}%
\definecolor{currentstroke}{rgb}{1.000000,1.000000,1.000000}%
\pgfsetstrokecolor{currentstroke}%
\pgfsetdash{}{0pt}%
\pgfpathmoveto{\pgfqpoint{4.067058in}{2.345881in}}%
\pgfpathcurveto{\pgfqpoint{4.078108in}{2.345881in}}{\pgfqpoint{4.088707in}{2.350271in}}{\pgfqpoint{4.096521in}{2.358085in}}%
\pgfpathcurveto{\pgfqpoint{4.104335in}{2.365899in}}{\pgfqpoint{4.108725in}{2.376498in}}{\pgfqpoint{4.108725in}{2.387548in}}%
\pgfpathcurveto{\pgfqpoint{4.108725in}{2.398598in}}{\pgfqpoint{4.104335in}{2.409197in}}{\pgfqpoint{4.096521in}{2.417010in}}%
\pgfpathcurveto{\pgfqpoint{4.088707in}{2.424824in}}{\pgfqpoint{4.078108in}{2.429214in}}{\pgfqpoint{4.067058in}{2.429214in}}%
\pgfpathcurveto{\pgfqpoint{4.056008in}{2.429214in}}{\pgfqpoint{4.045409in}{2.424824in}}{\pgfqpoint{4.037595in}{2.417010in}}%
\pgfpathcurveto{\pgfqpoint{4.029782in}{2.409197in}}{\pgfqpoint{4.025391in}{2.398598in}}{\pgfqpoint{4.025391in}{2.387548in}}%
\pgfpathcurveto{\pgfqpoint{4.025391in}{2.376498in}}{\pgfqpoint{4.029782in}{2.365899in}}{\pgfqpoint{4.037595in}{2.358085in}}%
\pgfpathcurveto{\pgfqpoint{4.045409in}{2.350271in}}{\pgfqpoint{4.056008in}{2.345881in}}{\pgfqpoint{4.067058in}{2.345881in}}%
\pgfpathclose%
\pgfusepath{stroke,fill}%
\end{pgfscope}%
\begin{pgfscope}%
\pgfpathrectangle{\pgfqpoint{0.570343in}{0.331635in}}{\pgfqpoint{9.300000in}{7.700000in}}%
\pgfusepath{clip}%
\pgfsetbuttcap%
\pgfsetroundjoin%
\definecolor{currentfill}{rgb}{0.631373,0.788235,0.956863}%
\pgfsetfillcolor{currentfill}%
\pgfsetlinewidth{0.481800pt}%
\definecolor{currentstroke}{rgb}{1.000000,1.000000,1.000000}%
\pgfsetstrokecolor{currentstroke}%
\pgfsetdash{}{0pt}%
\pgfpathmoveto{\pgfqpoint{6.352480in}{5.846096in}}%
\pgfpathcurveto{\pgfqpoint{6.363530in}{5.846096in}}{\pgfqpoint{6.374129in}{5.850486in}}{\pgfqpoint{6.381943in}{5.858300in}}%
\pgfpathcurveto{\pgfqpoint{6.389757in}{5.866113in}}{\pgfqpoint{6.394147in}{5.876713in}}{\pgfqpoint{6.394147in}{5.887763in}}%
\pgfpathcurveto{\pgfqpoint{6.394147in}{5.898813in}}{\pgfqpoint{6.389757in}{5.909412in}}{\pgfqpoint{6.381943in}{5.917225in}}%
\pgfpathcurveto{\pgfqpoint{6.374129in}{5.925039in}}{\pgfqpoint{6.363530in}{5.929429in}}{\pgfqpoint{6.352480in}{5.929429in}}%
\pgfpathcurveto{\pgfqpoint{6.341430in}{5.929429in}}{\pgfqpoint{6.330831in}{5.925039in}}{\pgfqpoint{6.323017in}{5.917225in}}%
\pgfpathcurveto{\pgfqpoint{6.315204in}{5.909412in}}{\pgfqpoint{6.310814in}{5.898813in}}{\pgfqpoint{6.310814in}{5.887763in}}%
\pgfpathcurveto{\pgfqpoint{6.310814in}{5.876713in}}{\pgfqpoint{6.315204in}{5.866113in}}{\pgfqpoint{6.323017in}{5.858300in}}%
\pgfpathcurveto{\pgfqpoint{6.330831in}{5.850486in}}{\pgfqpoint{6.341430in}{5.846096in}}{\pgfqpoint{6.352480in}{5.846096in}}%
\pgfpathclose%
\pgfusepath{stroke,fill}%
\end{pgfscope}%
\begin{pgfscope}%
\pgfpathrectangle{\pgfqpoint{0.570343in}{0.331635in}}{\pgfqpoint{9.300000in}{7.700000in}}%
\pgfusepath{clip}%
\pgfsetbuttcap%
\pgfsetroundjoin%
\definecolor{currentfill}{rgb}{0.631373,0.788235,0.956863}%
\pgfsetfillcolor{currentfill}%
\pgfsetlinewidth{0.481800pt}%
\definecolor{currentstroke}{rgb}{1.000000,1.000000,1.000000}%
\pgfsetstrokecolor{currentstroke}%
\pgfsetdash{}{0pt}%
\pgfpathmoveto{\pgfqpoint{5.146154in}{5.452263in}}%
\pgfpathcurveto{\pgfqpoint{5.157204in}{5.452263in}}{\pgfqpoint{5.167803in}{5.456653in}}{\pgfqpoint{5.175617in}{5.464466in}}%
\pgfpathcurveto{\pgfqpoint{5.183431in}{5.472280in}}{\pgfqpoint{5.187821in}{5.482879in}}{\pgfqpoint{5.187821in}{5.493929in}}%
\pgfpathcurveto{\pgfqpoint{5.187821in}{5.504979in}}{\pgfqpoint{5.183431in}{5.515578in}}{\pgfqpoint{5.175617in}{5.523392in}}%
\pgfpathcurveto{\pgfqpoint{5.167803in}{5.531206in}}{\pgfqpoint{5.157204in}{5.535596in}}{\pgfqpoint{5.146154in}{5.535596in}}%
\pgfpathcurveto{\pgfqpoint{5.135104in}{5.535596in}}{\pgfqpoint{5.124505in}{5.531206in}}{\pgfqpoint{5.116692in}{5.523392in}}%
\pgfpathcurveto{\pgfqpoint{5.108878in}{5.515578in}}{\pgfqpoint{5.104488in}{5.504979in}}{\pgfqpoint{5.104488in}{5.493929in}}%
\pgfpathcurveto{\pgfqpoint{5.104488in}{5.482879in}}{\pgfqpoint{5.108878in}{5.472280in}}{\pgfqpoint{5.116692in}{5.464466in}}%
\pgfpathcurveto{\pgfqpoint{5.124505in}{5.456653in}}{\pgfqpoint{5.135104in}{5.452263in}}{\pgfqpoint{5.146154in}{5.452263in}}%
\pgfpathclose%
\pgfusepath{stroke,fill}%
\end{pgfscope}%
\begin{pgfscope}%
\pgfpathrectangle{\pgfqpoint{0.570343in}{0.331635in}}{\pgfqpoint{9.300000in}{7.700000in}}%
\pgfusepath{clip}%
\pgfsetbuttcap%
\pgfsetroundjoin%
\definecolor{currentfill}{rgb}{0.631373,0.788235,0.956863}%
\pgfsetfillcolor{currentfill}%
\pgfsetlinewidth{0.481800pt}%
\definecolor{currentstroke}{rgb}{1.000000,1.000000,1.000000}%
\pgfsetstrokecolor{currentstroke}%
\pgfsetdash{}{0pt}%
\pgfpathmoveto{\pgfqpoint{7.564757in}{2.443303in}}%
\pgfpathcurveto{\pgfqpoint{7.575807in}{2.443303in}}{\pgfqpoint{7.586406in}{2.447694in}}{\pgfqpoint{7.594220in}{2.455507in}}%
\pgfpathcurveto{\pgfqpoint{7.602034in}{2.463321in}}{\pgfqpoint{7.606424in}{2.473920in}}{\pgfqpoint{7.606424in}{2.484970in}}%
\pgfpathcurveto{\pgfqpoint{7.606424in}{2.496020in}}{\pgfqpoint{7.602034in}{2.506619in}}{\pgfqpoint{7.594220in}{2.514433in}}%
\pgfpathcurveto{\pgfqpoint{7.586406in}{2.522247in}}{\pgfqpoint{7.575807in}{2.526637in}}{\pgfqpoint{7.564757in}{2.526637in}}%
\pgfpathcurveto{\pgfqpoint{7.553707in}{2.526637in}}{\pgfqpoint{7.543108in}{2.522247in}}{\pgfqpoint{7.535294in}{2.514433in}}%
\pgfpathcurveto{\pgfqpoint{7.527481in}{2.506619in}}{\pgfqpoint{7.523091in}{2.496020in}}{\pgfqpoint{7.523091in}{2.484970in}}%
\pgfpathcurveto{\pgfqpoint{7.523091in}{2.473920in}}{\pgfqpoint{7.527481in}{2.463321in}}{\pgfqpoint{7.535294in}{2.455507in}}%
\pgfpathcurveto{\pgfqpoint{7.543108in}{2.447694in}}{\pgfqpoint{7.553707in}{2.443303in}}{\pgfqpoint{7.564757in}{2.443303in}}%
\pgfpathclose%
\pgfusepath{stroke,fill}%
\end{pgfscope}%
\begin{pgfscope}%
\pgfpathrectangle{\pgfqpoint{0.570343in}{0.331635in}}{\pgfqpoint{9.300000in}{7.700000in}}%
\pgfusepath{clip}%
\pgfsetbuttcap%
\pgfsetroundjoin%
\definecolor{currentfill}{rgb}{0.631373,0.788235,0.956863}%
\pgfsetfillcolor{currentfill}%
\pgfsetlinewidth{0.481800pt}%
\definecolor{currentstroke}{rgb}{1.000000,1.000000,1.000000}%
\pgfsetstrokecolor{currentstroke}%
\pgfsetdash{}{0pt}%
\pgfpathmoveto{\pgfqpoint{8.396171in}{4.660144in}}%
\pgfpathcurveto{\pgfqpoint{8.407221in}{4.660144in}}{\pgfqpoint{8.417820in}{4.664534in}}{\pgfqpoint{8.425634in}{4.672348in}}%
\pgfpathcurveto{\pgfqpoint{8.433448in}{4.680162in}}{\pgfqpoint{8.437838in}{4.690761in}}{\pgfqpoint{8.437838in}{4.701811in}}%
\pgfpathcurveto{\pgfqpoint{8.437838in}{4.712861in}}{\pgfqpoint{8.433448in}{4.723460in}}{\pgfqpoint{8.425634in}{4.731274in}}%
\pgfpathcurveto{\pgfqpoint{8.417820in}{4.739087in}}{\pgfqpoint{8.407221in}{4.743477in}}{\pgfqpoint{8.396171in}{4.743477in}}%
\pgfpathcurveto{\pgfqpoint{8.385121in}{4.743477in}}{\pgfqpoint{8.374522in}{4.739087in}}{\pgfqpoint{8.366708in}{4.731274in}}%
\pgfpathcurveto{\pgfqpoint{8.358895in}{4.723460in}}{\pgfqpoint{8.354505in}{4.712861in}}{\pgfqpoint{8.354505in}{4.701811in}}%
\pgfpathcurveto{\pgfqpoint{8.354505in}{4.690761in}}{\pgfqpoint{8.358895in}{4.680162in}}{\pgfqpoint{8.366708in}{4.672348in}}%
\pgfpathcurveto{\pgfqpoint{8.374522in}{4.664534in}}{\pgfqpoint{8.385121in}{4.660144in}}{\pgfqpoint{8.396171in}{4.660144in}}%
\pgfpathclose%
\pgfusepath{stroke,fill}%
\end{pgfscope}%
\begin{pgfscope}%
\pgfpathrectangle{\pgfqpoint{0.570343in}{0.331635in}}{\pgfqpoint{9.300000in}{7.700000in}}%
\pgfusepath{clip}%
\pgfsetbuttcap%
\pgfsetroundjoin%
\definecolor{currentfill}{rgb}{0.631373,0.788235,0.956863}%
\pgfsetfillcolor{currentfill}%
\pgfsetlinewidth{0.481800pt}%
\definecolor{currentstroke}{rgb}{1.000000,1.000000,1.000000}%
\pgfsetstrokecolor{currentstroke}%
\pgfsetdash{}{0pt}%
\pgfpathmoveto{\pgfqpoint{3.531441in}{5.702170in}}%
\pgfpathcurveto{\pgfqpoint{3.542491in}{5.702170in}}{\pgfqpoint{3.553090in}{5.706560in}}{\pgfqpoint{3.560904in}{5.714374in}}%
\pgfpathcurveto{\pgfqpoint{3.568718in}{5.722187in}}{\pgfqpoint{3.573108in}{5.732787in}}{\pgfqpoint{3.573108in}{5.743837in}}%
\pgfpathcurveto{\pgfqpoint{3.573108in}{5.754887in}}{\pgfqpoint{3.568718in}{5.765486in}}{\pgfqpoint{3.560904in}{5.773299in}}%
\pgfpathcurveto{\pgfqpoint{3.553090in}{5.781113in}}{\pgfqpoint{3.542491in}{5.785503in}}{\pgfqpoint{3.531441in}{5.785503in}}%
\pgfpathcurveto{\pgfqpoint{3.520391in}{5.785503in}}{\pgfqpoint{3.509792in}{5.781113in}}{\pgfqpoint{3.501978in}{5.773299in}}%
\pgfpathcurveto{\pgfqpoint{3.494165in}{5.765486in}}{\pgfqpoint{3.489775in}{5.754887in}}{\pgfqpoint{3.489775in}{5.743837in}}%
\pgfpathcurveto{\pgfqpoint{3.489775in}{5.732787in}}{\pgfqpoint{3.494165in}{5.722187in}}{\pgfqpoint{3.501978in}{5.714374in}}%
\pgfpathcurveto{\pgfqpoint{3.509792in}{5.706560in}}{\pgfqpoint{3.520391in}{5.702170in}}{\pgfqpoint{3.531441in}{5.702170in}}%
\pgfpathclose%
\pgfusepath{stroke,fill}%
\end{pgfscope}%
\begin{pgfscope}%
\pgfpathrectangle{\pgfqpoint{0.570343in}{0.331635in}}{\pgfqpoint{9.300000in}{7.700000in}}%
\pgfusepath{clip}%
\pgfsetbuttcap%
\pgfsetroundjoin%
\definecolor{currentfill}{rgb}{1.000000,0.705882,0.509804}%
\pgfsetfillcolor{currentfill}%
\pgfsetlinewidth{0.481800pt}%
\definecolor{currentstroke}{rgb}{1.000000,1.000000,1.000000}%
\pgfsetstrokecolor{currentstroke}%
\pgfsetdash{}{0pt}%
\pgfpathmoveto{\pgfqpoint{5.495895in}{6.139158in}}%
\pgfpathcurveto{\pgfqpoint{5.506945in}{6.139158in}}{\pgfqpoint{5.517544in}{6.143549in}}{\pgfqpoint{5.525358in}{6.151362in}}%
\pgfpathcurveto{\pgfqpoint{5.533172in}{6.159176in}}{\pgfqpoint{5.537562in}{6.169775in}}{\pgfqpoint{5.537562in}{6.180825in}}%
\pgfpathcurveto{\pgfqpoint{5.537562in}{6.191875in}}{\pgfqpoint{5.533172in}{6.202474in}}{\pgfqpoint{5.525358in}{6.210288in}}%
\pgfpathcurveto{\pgfqpoint{5.517544in}{6.218101in}}{\pgfqpoint{5.506945in}{6.222492in}}{\pgfqpoint{5.495895in}{6.222492in}}%
\pgfpathcurveto{\pgfqpoint{5.484845in}{6.222492in}}{\pgfqpoint{5.474246in}{6.218101in}}{\pgfqpoint{5.466432in}{6.210288in}}%
\pgfpathcurveto{\pgfqpoint{5.458619in}{6.202474in}}{\pgfqpoint{5.454229in}{6.191875in}}{\pgfqpoint{5.454229in}{6.180825in}}%
\pgfpathcurveto{\pgfqpoint{5.454229in}{6.169775in}}{\pgfqpoint{5.458619in}{6.159176in}}{\pgfqpoint{5.466432in}{6.151362in}}%
\pgfpathcurveto{\pgfqpoint{5.474246in}{6.143549in}}{\pgfqpoint{5.484845in}{6.139158in}}{\pgfqpoint{5.495895in}{6.139158in}}%
\pgfpathclose%
\pgfusepath{stroke,fill}%
\end{pgfscope}%
\begin{pgfscope}%
\pgfpathrectangle{\pgfqpoint{0.570343in}{0.331635in}}{\pgfqpoint{9.300000in}{7.700000in}}%
\pgfusepath{clip}%
\pgfsetbuttcap%
\pgfsetroundjoin%
\definecolor{currentfill}{rgb}{1.000000,0.705882,0.509804}%
\pgfsetfillcolor{currentfill}%
\pgfsetlinewidth{0.481800pt}%
\definecolor{currentstroke}{rgb}{1.000000,1.000000,1.000000}%
\pgfsetstrokecolor{currentstroke}%
\pgfsetdash{}{0pt}%
\pgfpathmoveto{\pgfqpoint{8.684047in}{3.619507in}}%
\pgfpathcurveto{\pgfqpoint{8.695097in}{3.619507in}}{\pgfqpoint{8.705696in}{3.623897in}}{\pgfqpoint{8.713510in}{3.631711in}}%
\pgfpathcurveto{\pgfqpoint{8.721324in}{3.639525in}}{\pgfqpoint{8.725714in}{3.650124in}}{\pgfqpoint{8.725714in}{3.661174in}}%
\pgfpathcurveto{\pgfqpoint{8.725714in}{3.672224in}}{\pgfqpoint{8.721324in}{3.682823in}}{\pgfqpoint{8.713510in}{3.690637in}}%
\pgfpathcurveto{\pgfqpoint{8.705696in}{3.698450in}}{\pgfqpoint{8.695097in}{3.702840in}}{\pgfqpoint{8.684047in}{3.702840in}}%
\pgfpathcurveto{\pgfqpoint{8.672997in}{3.702840in}}{\pgfqpoint{8.662398in}{3.698450in}}{\pgfqpoint{8.654584in}{3.690637in}}%
\pgfpathcurveto{\pgfqpoint{8.646771in}{3.682823in}}{\pgfqpoint{8.642380in}{3.672224in}}{\pgfqpoint{8.642380in}{3.661174in}}%
\pgfpathcurveto{\pgfqpoint{8.642380in}{3.650124in}}{\pgfqpoint{8.646771in}{3.639525in}}{\pgfqpoint{8.654584in}{3.631711in}}%
\pgfpathcurveto{\pgfqpoint{8.662398in}{3.623897in}}{\pgfqpoint{8.672997in}{3.619507in}}{\pgfqpoint{8.684047in}{3.619507in}}%
\pgfpathclose%
\pgfusepath{stroke,fill}%
\end{pgfscope}%
\begin{pgfscope}%
\pgfpathrectangle{\pgfqpoint{0.570343in}{0.331635in}}{\pgfqpoint{9.300000in}{7.700000in}}%
\pgfusepath{clip}%
\pgfsetbuttcap%
\pgfsetroundjoin%
\definecolor{currentfill}{rgb}{1.000000,0.705882,0.509804}%
\pgfsetfillcolor{currentfill}%
\pgfsetlinewidth{0.481800pt}%
\definecolor{currentstroke}{rgb}{1.000000,1.000000,1.000000}%
\pgfsetstrokecolor{currentstroke}%
\pgfsetdash{}{0pt}%
\pgfpathmoveto{\pgfqpoint{4.658532in}{4.140491in}}%
\pgfpathcurveto{\pgfqpoint{4.669583in}{4.140491in}}{\pgfqpoint{4.680182in}{4.144881in}}{\pgfqpoint{4.687995in}{4.152695in}}%
\pgfpathcurveto{\pgfqpoint{4.695809in}{4.160508in}}{\pgfqpoint{4.700199in}{4.171107in}}{\pgfqpoint{4.700199in}{4.182157in}}%
\pgfpathcurveto{\pgfqpoint{4.700199in}{4.193207in}}{\pgfqpoint{4.695809in}{4.203807in}}{\pgfqpoint{4.687995in}{4.211620in}}%
\pgfpathcurveto{\pgfqpoint{4.680182in}{4.219434in}}{\pgfqpoint{4.669583in}{4.223824in}}{\pgfqpoint{4.658532in}{4.223824in}}%
\pgfpathcurveto{\pgfqpoint{4.647482in}{4.223824in}}{\pgfqpoint{4.636883in}{4.219434in}}{\pgfqpoint{4.629070in}{4.211620in}}%
\pgfpathcurveto{\pgfqpoint{4.621256in}{4.203807in}}{\pgfqpoint{4.616866in}{4.193207in}}{\pgfqpoint{4.616866in}{4.182157in}}%
\pgfpathcurveto{\pgfqpoint{4.616866in}{4.171107in}}{\pgfqpoint{4.621256in}{4.160508in}}{\pgfqpoint{4.629070in}{4.152695in}}%
\pgfpathcurveto{\pgfqpoint{4.636883in}{4.144881in}}{\pgfqpoint{4.647482in}{4.140491in}}{\pgfqpoint{4.658532in}{4.140491in}}%
\pgfpathclose%
\pgfusepath{stroke,fill}%
\end{pgfscope}%
\begin{pgfscope}%
\pgfpathrectangle{\pgfqpoint{0.570343in}{0.331635in}}{\pgfqpoint{9.300000in}{7.700000in}}%
\pgfusepath{clip}%
\pgfsetbuttcap%
\pgfsetroundjoin%
\definecolor{currentfill}{rgb}{1.000000,0.705882,0.509804}%
\pgfsetfillcolor{currentfill}%
\pgfsetlinewidth{0.481800pt}%
\definecolor{currentstroke}{rgb}{1.000000,1.000000,1.000000}%
\pgfsetstrokecolor{currentstroke}%
\pgfsetdash{}{0pt}%
\pgfpathmoveto{\pgfqpoint{5.265486in}{3.699856in}}%
\pgfpathcurveto{\pgfqpoint{5.276536in}{3.699856in}}{\pgfqpoint{5.287135in}{3.704247in}}{\pgfqpoint{5.294948in}{3.712060in}}%
\pgfpathcurveto{\pgfqpoint{5.302762in}{3.719874in}}{\pgfqpoint{5.307152in}{3.730473in}}{\pgfqpoint{5.307152in}{3.741523in}}%
\pgfpathcurveto{\pgfqpoint{5.307152in}{3.752573in}}{\pgfqpoint{5.302762in}{3.763172in}}{\pgfqpoint{5.294948in}{3.770986in}}%
\pgfpathcurveto{\pgfqpoint{5.287135in}{3.778799in}}{\pgfqpoint{5.276536in}{3.783190in}}{\pgfqpoint{5.265486in}{3.783190in}}%
\pgfpathcurveto{\pgfqpoint{5.254436in}{3.783190in}}{\pgfqpoint{5.243837in}{3.778799in}}{\pgfqpoint{5.236023in}{3.770986in}}%
\pgfpathcurveto{\pgfqpoint{5.228209in}{3.763172in}}{\pgfqpoint{5.223819in}{3.752573in}}{\pgfqpoint{5.223819in}{3.741523in}}%
\pgfpathcurveto{\pgfqpoint{5.223819in}{3.730473in}}{\pgfqpoint{5.228209in}{3.719874in}}{\pgfqpoint{5.236023in}{3.712060in}}%
\pgfpathcurveto{\pgfqpoint{5.243837in}{3.704247in}}{\pgfqpoint{5.254436in}{3.699856in}}{\pgfqpoint{5.265486in}{3.699856in}}%
\pgfpathclose%
\pgfusepath{stroke,fill}%
\end{pgfscope}%
\begin{pgfscope}%
\pgfpathrectangle{\pgfqpoint{0.570343in}{0.331635in}}{\pgfqpoint{9.300000in}{7.700000in}}%
\pgfusepath{clip}%
\pgfsetbuttcap%
\pgfsetroundjoin%
\definecolor{currentfill}{rgb}{1.000000,0.705882,0.509804}%
\pgfsetfillcolor{currentfill}%
\pgfsetlinewidth{0.481800pt}%
\definecolor{currentstroke}{rgb}{1.000000,1.000000,1.000000}%
\pgfsetstrokecolor{currentstroke}%
\pgfsetdash{}{0pt}%
\pgfpathmoveto{\pgfqpoint{3.663113in}{6.535316in}}%
\pgfpathcurveto{\pgfqpoint{3.674163in}{6.535316in}}{\pgfqpoint{3.684763in}{6.539706in}}{\pgfqpoint{3.692576in}{6.547520in}}%
\pgfpathcurveto{\pgfqpoint{3.700390in}{6.555334in}}{\pgfqpoint{3.704780in}{6.565933in}}{\pgfqpoint{3.704780in}{6.576983in}}%
\pgfpathcurveto{\pgfqpoint{3.704780in}{6.588033in}}{\pgfqpoint{3.700390in}{6.598632in}}{\pgfqpoint{3.692576in}{6.606446in}}%
\pgfpathcurveto{\pgfqpoint{3.684763in}{6.614259in}}{\pgfqpoint{3.674163in}{6.618649in}}{\pgfqpoint{3.663113in}{6.618649in}}%
\pgfpathcurveto{\pgfqpoint{3.652063in}{6.618649in}}{\pgfqpoint{3.641464in}{6.614259in}}{\pgfqpoint{3.633651in}{6.606446in}}%
\pgfpathcurveto{\pgfqpoint{3.625837in}{6.598632in}}{\pgfqpoint{3.621447in}{6.588033in}}{\pgfqpoint{3.621447in}{6.576983in}}%
\pgfpathcurveto{\pgfqpoint{3.621447in}{6.565933in}}{\pgfqpoint{3.625837in}{6.555334in}}{\pgfqpoint{3.633651in}{6.547520in}}%
\pgfpathcurveto{\pgfqpoint{3.641464in}{6.539706in}}{\pgfqpoint{3.652063in}{6.535316in}}{\pgfqpoint{3.663113in}{6.535316in}}%
\pgfpathclose%
\pgfusepath{stroke,fill}%
\end{pgfscope}%
\begin{pgfscope}%
\pgfpathrectangle{\pgfqpoint{0.570343in}{0.331635in}}{\pgfqpoint{9.300000in}{7.700000in}}%
\pgfusepath{clip}%
\pgfsetbuttcap%
\pgfsetroundjoin%
\definecolor{currentfill}{rgb}{1.000000,0.705882,0.509804}%
\pgfsetfillcolor{currentfill}%
\pgfsetlinewidth{0.481800pt}%
\definecolor{currentstroke}{rgb}{1.000000,1.000000,1.000000}%
\pgfsetstrokecolor{currentstroke}%
\pgfsetdash{}{0pt}%
\pgfpathmoveto{\pgfqpoint{7.006137in}{4.502520in}}%
\pgfpathcurveto{\pgfqpoint{7.017188in}{4.502520in}}{\pgfqpoint{7.027787in}{4.506910in}}{\pgfqpoint{7.035600in}{4.514723in}}%
\pgfpathcurveto{\pgfqpoint{7.043414in}{4.522537in}}{\pgfqpoint{7.047804in}{4.533136in}}{\pgfqpoint{7.047804in}{4.544186in}}%
\pgfpathcurveto{\pgfqpoint{7.047804in}{4.555236in}}{\pgfqpoint{7.043414in}{4.565835in}}{\pgfqpoint{7.035600in}{4.573649in}}%
\pgfpathcurveto{\pgfqpoint{7.027787in}{4.581463in}}{\pgfqpoint{7.017188in}{4.585853in}}{\pgfqpoint{7.006137in}{4.585853in}}%
\pgfpathcurveto{\pgfqpoint{6.995087in}{4.585853in}}{\pgfqpoint{6.984488in}{4.581463in}}{\pgfqpoint{6.976675in}{4.573649in}}%
\pgfpathcurveto{\pgfqpoint{6.968861in}{4.565835in}}{\pgfqpoint{6.964471in}{4.555236in}}{\pgfqpoint{6.964471in}{4.544186in}}%
\pgfpathcurveto{\pgfqpoint{6.964471in}{4.533136in}}{\pgfqpoint{6.968861in}{4.522537in}}{\pgfqpoint{6.976675in}{4.514723in}}%
\pgfpathcurveto{\pgfqpoint{6.984488in}{4.506910in}}{\pgfqpoint{6.995087in}{4.502520in}}{\pgfqpoint{7.006137in}{4.502520in}}%
\pgfpathclose%
\pgfusepath{stroke,fill}%
\end{pgfscope}%
\begin{pgfscope}%
\pgfpathrectangle{\pgfqpoint{0.570343in}{0.331635in}}{\pgfqpoint{9.300000in}{7.700000in}}%
\pgfusepath{clip}%
\pgfsetbuttcap%
\pgfsetroundjoin%
\definecolor{currentfill}{rgb}{1.000000,0.705882,0.509804}%
\pgfsetfillcolor{currentfill}%
\pgfsetlinewidth{0.481800pt}%
\definecolor{currentstroke}{rgb}{1.000000,1.000000,1.000000}%
\pgfsetstrokecolor{currentstroke}%
\pgfsetdash{}{0pt}%
\pgfpathmoveto{\pgfqpoint{7.866554in}{4.490496in}}%
\pgfpathcurveto{\pgfqpoint{7.877604in}{4.490496in}}{\pgfqpoint{7.888203in}{4.494886in}}{\pgfqpoint{7.896017in}{4.502700in}}%
\pgfpathcurveto{\pgfqpoint{7.903830in}{4.510513in}}{\pgfqpoint{7.908220in}{4.521113in}}{\pgfqpoint{7.908220in}{4.532163in}}%
\pgfpathcurveto{\pgfqpoint{7.908220in}{4.543213in}}{\pgfqpoint{7.903830in}{4.553812in}}{\pgfqpoint{7.896017in}{4.561625in}}%
\pgfpathcurveto{\pgfqpoint{7.888203in}{4.569439in}}{\pgfqpoint{7.877604in}{4.573829in}}{\pgfqpoint{7.866554in}{4.573829in}}%
\pgfpathcurveto{\pgfqpoint{7.855504in}{4.573829in}}{\pgfqpoint{7.844905in}{4.569439in}}{\pgfqpoint{7.837091in}{4.561625in}}%
\pgfpathcurveto{\pgfqpoint{7.829277in}{4.553812in}}{\pgfqpoint{7.824887in}{4.543213in}}{\pgfqpoint{7.824887in}{4.532163in}}%
\pgfpathcurveto{\pgfqpoint{7.824887in}{4.521113in}}{\pgfqpoint{7.829277in}{4.510513in}}{\pgfqpoint{7.837091in}{4.502700in}}%
\pgfpathcurveto{\pgfqpoint{7.844905in}{4.494886in}}{\pgfqpoint{7.855504in}{4.490496in}}{\pgfqpoint{7.866554in}{4.490496in}}%
\pgfpathclose%
\pgfusepath{stroke,fill}%
\end{pgfscope}%
\begin{pgfscope}%
\pgfpathrectangle{\pgfqpoint{0.570343in}{0.331635in}}{\pgfqpoint{9.300000in}{7.700000in}}%
\pgfusepath{clip}%
\pgfsetbuttcap%
\pgfsetroundjoin%
\definecolor{currentfill}{rgb}{1.000000,0.705882,0.509804}%
\pgfsetfillcolor{currentfill}%
\pgfsetlinewidth{0.481800pt}%
\definecolor{currentstroke}{rgb}{1.000000,1.000000,1.000000}%
\pgfsetstrokecolor{currentstroke}%
\pgfsetdash{}{0pt}%
\pgfpathmoveto{\pgfqpoint{2.309950in}{4.798666in}}%
\pgfpathcurveto{\pgfqpoint{2.321001in}{4.798666in}}{\pgfqpoint{2.331600in}{4.803056in}}{\pgfqpoint{2.339413in}{4.810870in}}%
\pgfpathcurveto{\pgfqpoint{2.347227in}{4.818684in}}{\pgfqpoint{2.351617in}{4.829283in}}{\pgfqpoint{2.351617in}{4.840333in}}%
\pgfpathcurveto{\pgfqpoint{2.351617in}{4.851383in}}{\pgfqpoint{2.347227in}{4.861982in}}{\pgfqpoint{2.339413in}{4.869796in}}%
\pgfpathcurveto{\pgfqpoint{2.331600in}{4.877609in}}{\pgfqpoint{2.321001in}{4.881999in}}{\pgfqpoint{2.309950in}{4.881999in}}%
\pgfpathcurveto{\pgfqpoint{2.298900in}{4.881999in}}{\pgfqpoint{2.288301in}{4.877609in}}{\pgfqpoint{2.280488in}{4.869796in}}%
\pgfpathcurveto{\pgfqpoint{2.272674in}{4.861982in}}{\pgfqpoint{2.268284in}{4.851383in}}{\pgfqpoint{2.268284in}{4.840333in}}%
\pgfpathcurveto{\pgfqpoint{2.268284in}{4.829283in}}{\pgfqpoint{2.272674in}{4.818684in}}{\pgfqpoint{2.280488in}{4.810870in}}%
\pgfpathcurveto{\pgfqpoint{2.288301in}{4.803056in}}{\pgfqpoint{2.298900in}{4.798666in}}{\pgfqpoint{2.309950in}{4.798666in}}%
\pgfpathclose%
\pgfusepath{stroke,fill}%
\end{pgfscope}%
\begin{pgfscope}%
\pgfpathrectangle{\pgfqpoint{0.570343in}{0.331635in}}{\pgfqpoint{9.300000in}{7.700000in}}%
\pgfusepath{clip}%
\pgfsetbuttcap%
\pgfsetroundjoin%
\definecolor{currentfill}{rgb}{1.000000,0.705882,0.509804}%
\pgfsetfillcolor{currentfill}%
\pgfsetlinewidth{0.481800pt}%
\definecolor{currentstroke}{rgb}{1.000000,1.000000,1.000000}%
\pgfsetstrokecolor{currentstroke}%
\pgfsetdash{}{0pt}%
\pgfpathmoveto{\pgfqpoint{4.597903in}{6.246141in}}%
\pgfpathcurveto{\pgfqpoint{4.608953in}{6.246141in}}{\pgfqpoint{4.619552in}{6.250531in}}{\pgfqpoint{4.627366in}{6.258344in}}%
\pgfpathcurveto{\pgfqpoint{4.635179in}{6.266158in}}{\pgfqpoint{4.639570in}{6.276757in}}{\pgfqpoint{4.639570in}{6.287807in}}%
\pgfpathcurveto{\pgfqpoint{4.639570in}{6.298857in}}{\pgfqpoint{4.635179in}{6.309456in}}{\pgfqpoint{4.627366in}{6.317270in}}%
\pgfpathcurveto{\pgfqpoint{4.619552in}{6.325084in}}{\pgfqpoint{4.608953in}{6.329474in}}{\pgfqpoint{4.597903in}{6.329474in}}%
\pgfpathcurveto{\pgfqpoint{4.586853in}{6.329474in}}{\pgfqpoint{4.576254in}{6.325084in}}{\pgfqpoint{4.568440in}{6.317270in}}%
\pgfpathcurveto{\pgfqpoint{4.560627in}{6.309456in}}{\pgfqpoint{4.556236in}{6.298857in}}{\pgfqpoint{4.556236in}{6.287807in}}%
\pgfpathcurveto{\pgfqpoint{4.556236in}{6.276757in}}{\pgfqpoint{4.560627in}{6.266158in}}{\pgfqpoint{4.568440in}{6.258344in}}%
\pgfpathcurveto{\pgfqpoint{4.576254in}{6.250531in}}{\pgfqpoint{4.586853in}{6.246141in}}{\pgfqpoint{4.597903in}{6.246141in}}%
\pgfpathclose%
\pgfusepath{stroke,fill}%
\end{pgfscope}%
\begin{pgfscope}%
\pgfpathrectangle{\pgfqpoint{0.570343in}{0.331635in}}{\pgfqpoint{9.300000in}{7.700000in}}%
\pgfusepath{clip}%
\pgfsetbuttcap%
\pgfsetroundjoin%
\definecolor{currentfill}{rgb}{1.000000,0.705882,0.509804}%
\pgfsetfillcolor{currentfill}%
\pgfsetlinewidth{0.481800pt}%
\definecolor{currentstroke}{rgb}{1.000000,1.000000,1.000000}%
\pgfsetstrokecolor{currentstroke}%
\pgfsetdash{}{0pt}%
\pgfpathmoveto{\pgfqpoint{2.912749in}{6.686491in}}%
\pgfpathcurveto{\pgfqpoint{2.923799in}{6.686491in}}{\pgfqpoint{2.934398in}{6.690882in}}{\pgfqpoint{2.942212in}{6.698695in}}%
\pgfpathcurveto{\pgfqpoint{2.950026in}{6.706509in}}{\pgfqpoint{2.954416in}{6.717108in}}{\pgfqpoint{2.954416in}{6.728158in}}%
\pgfpathcurveto{\pgfqpoint{2.954416in}{6.739208in}}{\pgfqpoint{2.950026in}{6.749807in}}{\pgfqpoint{2.942212in}{6.757621in}}%
\pgfpathcurveto{\pgfqpoint{2.934398in}{6.765434in}}{\pgfqpoint{2.923799in}{6.769825in}}{\pgfqpoint{2.912749in}{6.769825in}}%
\pgfpathcurveto{\pgfqpoint{2.901699in}{6.769825in}}{\pgfqpoint{2.891100in}{6.765434in}}{\pgfqpoint{2.883287in}{6.757621in}}%
\pgfpathcurveto{\pgfqpoint{2.875473in}{6.749807in}}{\pgfqpoint{2.871083in}{6.739208in}}{\pgfqpoint{2.871083in}{6.728158in}}%
\pgfpathcurveto{\pgfqpoint{2.871083in}{6.717108in}}{\pgfqpoint{2.875473in}{6.706509in}}{\pgfqpoint{2.883287in}{6.698695in}}%
\pgfpathcurveto{\pgfqpoint{2.891100in}{6.690882in}}{\pgfqpoint{2.901699in}{6.686491in}}{\pgfqpoint{2.912749in}{6.686491in}}%
\pgfpathclose%
\pgfusepath{stroke,fill}%
\end{pgfscope}%
\begin{pgfscope}%
\pgfpathrectangle{\pgfqpoint{0.570343in}{0.331635in}}{\pgfqpoint{9.300000in}{7.700000in}}%
\pgfusepath{clip}%
\pgfsetbuttcap%
\pgfsetroundjoin%
\definecolor{currentfill}{rgb}{1.000000,0.705882,0.509804}%
\pgfsetfillcolor{currentfill}%
\pgfsetlinewidth{0.481800pt}%
\definecolor{currentstroke}{rgb}{1.000000,1.000000,1.000000}%
\pgfsetstrokecolor{currentstroke}%
\pgfsetdash{}{0pt}%
\pgfpathmoveto{\pgfqpoint{3.463668in}{5.030027in}}%
\pgfpathcurveto{\pgfqpoint{3.474718in}{5.030027in}}{\pgfqpoint{3.485317in}{5.034417in}}{\pgfqpoint{3.493131in}{5.042231in}}%
\pgfpathcurveto{\pgfqpoint{3.500945in}{5.050044in}}{\pgfqpoint{3.505335in}{5.060643in}}{\pgfqpoint{3.505335in}{5.071694in}}%
\pgfpathcurveto{\pgfqpoint{3.505335in}{5.082744in}}{\pgfqpoint{3.500945in}{5.093343in}}{\pgfqpoint{3.493131in}{5.101156in}}%
\pgfpathcurveto{\pgfqpoint{3.485317in}{5.108970in}}{\pgfqpoint{3.474718in}{5.113360in}}{\pgfqpoint{3.463668in}{5.113360in}}%
\pgfpathcurveto{\pgfqpoint{3.452618in}{5.113360in}}{\pgfqpoint{3.442019in}{5.108970in}}{\pgfqpoint{3.434205in}{5.101156in}}%
\pgfpathcurveto{\pgfqpoint{3.426392in}{5.093343in}}{\pgfqpoint{3.422002in}{5.082744in}}{\pgfqpoint{3.422002in}{5.071694in}}%
\pgfpathcurveto{\pgfqpoint{3.422002in}{5.060643in}}{\pgfqpoint{3.426392in}{5.050044in}}{\pgfqpoint{3.434205in}{5.042231in}}%
\pgfpathcurveto{\pgfqpoint{3.442019in}{5.034417in}}{\pgfqpoint{3.452618in}{5.030027in}}{\pgfqpoint{3.463668in}{5.030027in}}%
\pgfpathclose%
\pgfusepath{stroke,fill}%
\end{pgfscope}%
\begin{pgfscope}%
\pgfpathrectangle{\pgfqpoint{0.570343in}{0.331635in}}{\pgfqpoint{9.300000in}{7.700000in}}%
\pgfusepath{clip}%
\pgfsetbuttcap%
\pgfsetroundjoin%
\definecolor{currentfill}{rgb}{1.000000,0.705882,0.509804}%
\pgfsetfillcolor{currentfill}%
\pgfsetlinewidth{0.481800pt}%
\definecolor{currentstroke}{rgb}{1.000000,1.000000,1.000000}%
\pgfsetstrokecolor{currentstroke}%
\pgfsetdash{}{0pt}%
\pgfpathmoveto{\pgfqpoint{8.862844in}{2.392695in}}%
\pgfpathcurveto{\pgfqpoint{8.873894in}{2.392695in}}{\pgfqpoint{8.884493in}{2.397085in}}{\pgfqpoint{8.892307in}{2.404899in}}%
\pgfpathcurveto{\pgfqpoint{8.900120in}{2.412712in}}{\pgfqpoint{8.904510in}{2.423311in}}{\pgfqpoint{8.904510in}{2.434362in}}%
\pgfpathcurveto{\pgfqpoint{8.904510in}{2.445412in}}{\pgfqpoint{8.900120in}{2.456011in}}{\pgfqpoint{8.892307in}{2.463824in}}%
\pgfpathcurveto{\pgfqpoint{8.884493in}{2.471638in}}{\pgfqpoint{8.873894in}{2.476028in}}{\pgfqpoint{8.862844in}{2.476028in}}%
\pgfpathcurveto{\pgfqpoint{8.851794in}{2.476028in}}{\pgfqpoint{8.841195in}{2.471638in}}{\pgfqpoint{8.833381in}{2.463824in}}%
\pgfpathcurveto{\pgfqpoint{8.825567in}{2.456011in}}{\pgfqpoint{8.821177in}{2.445412in}}{\pgfqpoint{8.821177in}{2.434362in}}%
\pgfpathcurveto{\pgfqpoint{8.821177in}{2.423311in}}{\pgfqpoint{8.825567in}{2.412712in}}{\pgfqpoint{8.833381in}{2.404899in}}%
\pgfpathcurveto{\pgfqpoint{8.841195in}{2.397085in}}{\pgfqpoint{8.851794in}{2.392695in}}{\pgfqpoint{8.862844in}{2.392695in}}%
\pgfpathclose%
\pgfusepath{stroke,fill}%
\end{pgfscope}%
\begin{pgfscope}%
\pgfpathrectangle{\pgfqpoint{0.570343in}{0.331635in}}{\pgfqpoint{9.300000in}{7.700000in}}%
\pgfusepath{clip}%
\pgfsetbuttcap%
\pgfsetroundjoin%
\definecolor{currentfill}{rgb}{1.000000,0.705882,0.509804}%
\pgfsetfillcolor{currentfill}%
\pgfsetlinewidth{0.481800pt}%
\definecolor{currentstroke}{rgb}{1.000000,1.000000,1.000000}%
\pgfsetstrokecolor{currentstroke}%
\pgfsetdash{}{0pt}%
\pgfpathmoveto{\pgfqpoint{7.751788in}{5.266912in}}%
\pgfpathcurveto{\pgfqpoint{7.762838in}{5.266912in}}{\pgfqpoint{7.773437in}{5.271302in}}{\pgfqpoint{7.781251in}{5.279116in}}%
\pgfpathcurveto{\pgfqpoint{7.789065in}{5.286929in}}{\pgfqpoint{7.793455in}{5.297528in}}{\pgfqpoint{7.793455in}{5.308578in}}%
\pgfpathcurveto{\pgfqpoint{7.793455in}{5.319628in}}{\pgfqpoint{7.789065in}{5.330228in}}{\pgfqpoint{7.781251in}{5.338041in}}%
\pgfpathcurveto{\pgfqpoint{7.773437in}{5.345855in}}{\pgfqpoint{7.762838in}{5.350245in}}{\pgfqpoint{7.751788in}{5.350245in}}%
\pgfpathcurveto{\pgfqpoint{7.740738in}{5.350245in}}{\pgfqpoint{7.730139in}{5.345855in}}{\pgfqpoint{7.722326in}{5.338041in}}%
\pgfpathcurveto{\pgfqpoint{7.714512in}{5.330228in}}{\pgfqpoint{7.710122in}{5.319628in}}{\pgfqpoint{7.710122in}{5.308578in}}%
\pgfpathcurveto{\pgfqpoint{7.710122in}{5.297528in}}{\pgfqpoint{7.714512in}{5.286929in}}{\pgfqpoint{7.722326in}{5.279116in}}%
\pgfpathcurveto{\pgfqpoint{7.730139in}{5.271302in}}{\pgfqpoint{7.740738in}{5.266912in}}{\pgfqpoint{7.751788in}{5.266912in}}%
\pgfpathclose%
\pgfusepath{stroke,fill}%
\end{pgfscope}%
\begin{pgfscope}%
\pgfpathrectangle{\pgfqpoint{0.570343in}{0.331635in}}{\pgfqpoint{9.300000in}{7.700000in}}%
\pgfusepath{clip}%
\pgfsetbuttcap%
\pgfsetroundjoin%
\definecolor{currentfill}{rgb}{1.000000,0.705882,0.509804}%
\pgfsetfillcolor{currentfill}%
\pgfsetlinewidth{0.481800pt}%
\definecolor{currentstroke}{rgb}{1.000000,1.000000,1.000000}%
\pgfsetstrokecolor{currentstroke}%
\pgfsetdash{}{0pt}%
\pgfpathmoveto{\pgfqpoint{4.290232in}{5.503628in}}%
\pgfpathcurveto{\pgfqpoint{4.301282in}{5.503628in}}{\pgfqpoint{4.311881in}{5.508018in}}{\pgfqpoint{4.319695in}{5.515832in}}%
\pgfpathcurveto{\pgfqpoint{4.327508in}{5.523646in}}{\pgfqpoint{4.331899in}{5.534245in}}{\pgfqpoint{4.331899in}{5.545295in}}%
\pgfpathcurveto{\pgfqpoint{4.331899in}{5.556345in}}{\pgfqpoint{4.327508in}{5.566944in}}{\pgfqpoint{4.319695in}{5.574758in}}%
\pgfpathcurveto{\pgfqpoint{4.311881in}{5.582571in}}{\pgfqpoint{4.301282in}{5.586962in}}{\pgfqpoint{4.290232in}{5.586962in}}%
\pgfpathcurveto{\pgfqpoint{4.279182in}{5.586962in}}{\pgfqpoint{4.268583in}{5.582571in}}{\pgfqpoint{4.260769in}{5.574758in}}%
\pgfpathcurveto{\pgfqpoint{4.252955in}{5.566944in}}{\pgfqpoint{4.248565in}{5.556345in}}{\pgfqpoint{4.248565in}{5.545295in}}%
\pgfpathcurveto{\pgfqpoint{4.248565in}{5.534245in}}{\pgfqpoint{4.252955in}{5.523646in}}{\pgfqpoint{4.260769in}{5.515832in}}%
\pgfpathcurveto{\pgfqpoint{4.268583in}{5.508018in}}{\pgfqpoint{4.279182in}{5.503628in}}{\pgfqpoint{4.290232in}{5.503628in}}%
\pgfpathclose%
\pgfusepath{stroke,fill}%
\end{pgfscope}%
\begin{pgfscope}%
\pgfpathrectangle{\pgfqpoint{0.570343in}{0.331635in}}{\pgfqpoint{9.300000in}{7.700000in}}%
\pgfusepath{clip}%
\pgfsetbuttcap%
\pgfsetroundjoin%
\definecolor{currentfill}{rgb}{1.000000,0.705882,0.509804}%
\pgfsetfillcolor{currentfill}%
\pgfsetlinewidth{0.481800pt}%
\definecolor{currentstroke}{rgb}{1.000000,1.000000,1.000000}%
\pgfsetstrokecolor{currentstroke}%
\pgfsetdash{}{0pt}%
\pgfpathmoveto{\pgfqpoint{6.599542in}{6.850694in}}%
\pgfpathcurveto{\pgfqpoint{6.610593in}{6.850694in}}{\pgfqpoint{6.621192in}{6.855084in}}{\pgfqpoint{6.629005in}{6.862897in}}%
\pgfpathcurveto{\pgfqpoint{6.636819in}{6.870711in}}{\pgfqpoint{6.641209in}{6.881310in}}{\pgfqpoint{6.641209in}{6.892360in}}%
\pgfpathcurveto{\pgfqpoint{6.641209in}{6.903410in}}{\pgfqpoint{6.636819in}{6.914009in}}{\pgfqpoint{6.629005in}{6.921823in}}%
\pgfpathcurveto{\pgfqpoint{6.621192in}{6.929637in}}{\pgfqpoint{6.610593in}{6.934027in}}{\pgfqpoint{6.599542in}{6.934027in}}%
\pgfpathcurveto{\pgfqpoint{6.588492in}{6.934027in}}{\pgfqpoint{6.577893in}{6.929637in}}{\pgfqpoint{6.570080in}{6.921823in}}%
\pgfpathcurveto{\pgfqpoint{6.562266in}{6.914009in}}{\pgfqpoint{6.557876in}{6.903410in}}{\pgfqpoint{6.557876in}{6.892360in}}%
\pgfpathcurveto{\pgfqpoint{6.557876in}{6.881310in}}{\pgfqpoint{6.562266in}{6.870711in}}{\pgfqpoint{6.570080in}{6.862897in}}%
\pgfpathcurveto{\pgfqpoint{6.577893in}{6.855084in}}{\pgfqpoint{6.588492in}{6.850694in}}{\pgfqpoint{6.599542in}{6.850694in}}%
\pgfpathclose%
\pgfusepath{stroke,fill}%
\end{pgfscope}%
\begin{pgfscope}%
\pgfpathrectangle{\pgfqpoint{0.570343in}{0.331635in}}{\pgfqpoint{9.300000in}{7.700000in}}%
\pgfusepath{clip}%
\pgfsetbuttcap%
\pgfsetroundjoin%
\definecolor{currentfill}{rgb}{1.000000,0.705882,0.509804}%
\pgfsetfillcolor{currentfill}%
\pgfsetlinewidth{0.481800pt}%
\definecolor{currentstroke}{rgb}{1.000000,1.000000,1.000000}%
\pgfsetstrokecolor{currentstroke}%
\pgfsetdash{}{0pt}%
\pgfpathmoveto{\pgfqpoint{5.751027in}{4.391362in}}%
\pgfpathcurveto{\pgfqpoint{5.762077in}{4.391362in}}{\pgfqpoint{5.772676in}{4.395752in}}{\pgfqpoint{5.780490in}{4.403566in}}%
\pgfpathcurveto{\pgfqpoint{5.788304in}{4.411379in}}{\pgfqpoint{5.792694in}{4.421978in}}{\pgfqpoint{5.792694in}{4.433028in}}%
\pgfpathcurveto{\pgfqpoint{5.792694in}{4.444079in}}{\pgfqpoint{5.788304in}{4.454678in}}{\pgfqpoint{5.780490in}{4.462491in}}%
\pgfpathcurveto{\pgfqpoint{5.772676in}{4.470305in}}{\pgfqpoint{5.762077in}{4.474695in}}{\pgfqpoint{5.751027in}{4.474695in}}%
\pgfpathcurveto{\pgfqpoint{5.739977in}{4.474695in}}{\pgfqpoint{5.729378in}{4.470305in}}{\pgfqpoint{5.721564in}{4.462491in}}%
\pgfpathcurveto{\pgfqpoint{5.713751in}{4.454678in}}{\pgfqpoint{5.709361in}{4.444079in}}{\pgfqpoint{5.709361in}{4.433028in}}%
\pgfpathcurveto{\pgfqpoint{5.709361in}{4.421978in}}{\pgfqpoint{5.713751in}{4.411379in}}{\pgfqpoint{5.721564in}{4.403566in}}%
\pgfpathcurveto{\pgfqpoint{5.729378in}{4.395752in}}{\pgfqpoint{5.739977in}{4.391362in}}{\pgfqpoint{5.751027in}{4.391362in}}%
\pgfpathclose%
\pgfusepath{stroke,fill}%
\end{pgfscope}%
\begin{pgfscope}%
\pgfpathrectangle{\pgfqpoint{0.570343in}{0.331635in}}{\pgfqpoint{9.300000in}{7.700000in}}%
\pgfusepath{clip}%
\pgfsetbuttcap%
\pgfsetroundjoin%
\definecolor{currentfill}{rgb}{1.000000,0.705882,0.509804}%
\pgfsetfillcolor{currentfill}%
\pgfsetlinewidth{0.481800pt}%
\definecolor{currentstroke}{rgb}{1.000000,1.000000,1.000000}%
\pgfsetstrokecolor{currentstroke}%
\pgfsetdash{}{0pt}%
\pgfpathmoveto{\pgfqpoint{6.415032in}{4.959397in}}%
\pgfpathcurveto{\pgfqpoint{6.426082in}{4.959397in}}{\pgfqpoint{6.436681in}{4.963788in}}{\pgfqpoint{6.444495in}{4.971601in}}%
\pgfpathcurveto{\pgfqpoint{6.452309in}{4.979415in}}{\pgfqpoint{6.456699in}{4.990014in}}{\pgfqpoint{6.456699in}{5.001064in}}%
\pgfpathcurveto{\pgfqpoint{6.456699in}{5.012114in}}{\pgfqpoint{6.452309in}{5.022713in}}{\pgfqpoint{6.444495in}{5.030527in}}%
\pgfpathcurveto{\pgfqpoint{6.436681in}{5.038341in}}{\pgfqpoint{6.426082in}{5.042731in}}{\pgfqpoint{6.415032in}{5.042731in}}%
\pgfpathcurveto{\pgfqpoint{6.403982in}{5.042731in}}{\pgfqpoint{6.393383in}{5.038341in}}{\pgfqpoint{6.385569in}{5.030527in}}%
\pgfpathcurveto{\pgfqpoint{6.377756in}{5.022713in}}{\pgfqpoint{6.373366in}{5.012114in}}{\pgfqpoint{6.373366in}{5.001064in}}%
\pgfpathcurveto{\pgfqpoint{6.373366in}{4.990014in}}{\pgfqpoint{6.377756in}{4.979415in}}{\pgfqpoint{6.385569in}{4.971601in}}%
\pgfpathcurveto{\pgfqpoint{6.393383in}{4.963788in}}{\pgfqpoint{6.403982in}{4.959397in}}{\pgfqpoint{6.415032in}{4.959397in}}%
\pgfpathclose%
\pgfusepath{stroke,fill}%
\end{pgfscope}%
\begin{pgfscope}%
\pgfpathrectangle{\pgfqpoint{0.570343in}{0.331635in}}{\pgfqpoint{9.300000in}{7.700000in}}%
\pgfusepath{clip}%
\pgfsetbuttcap%
\pgfsetroundjoin%
\definecolor{currentfill}{rgb}{1.000000,0.705882,0.509804}%
\pgfsetfillcolor{currentfill}%
\pgfsetlinewidth{0.481800pt}%
\definecolor{currentstroke}{rgb}{1.000000,1.000000,1.000000}%
\pgfsetstrokecolor{currentstroke}%
\pgfsetdash{}{0pt}%
\pgfpathmoveto{\pgfqpoint{3.801528in}{7.639968in}}%
\pgfpathcurveto{\pgfqpoint{3.812578in}{7.639968in}}{\pgfqpoint{3.823177in}{7.644359in}}{\pgfqpoint{3.830990in}{7.652172in}}%
\pgfpathcurveto{\pgfqpoint{3.838804in}{7.659986in}}{\pgfqpoint{3.843194in}{7.670585in}}{\pgfqpoint{3.843194in}{7.681635in}}%
\pgfpathcurveto{\pgfqpoint{3.843194in}{7.692685in}}{\pgfqpoint{3.838804in}{7.703284in}}{\pgfqpoint{3.830990in}{7.711098in}}%
\pgfpathcurveto{\pgfqpoint{3.823177in}{7.718911in}}{\pgfqpoint{3.812578in}{7.723302in}}{\pgfqpoint{3.801528in}{7.723302in}}%
\pgfpathcurveto{\pgfqpoint{3.790477in}{7.723302in}}{\pgfqpoint{3.779878in}{7.718911in}}{\pgfqpoint{3.772065in}{7.711098in}}%
\pgfpathcurveto{\pgfqpoint{3.764251in}{7.703284in}}{\pgfqpoint{3.759861in}{7.692685in}}{\pgfqpoint{3.759861in}{7.681635in}}%
\pgfpathcurveto{\pgfqpoint{3.759861in}{7.670585in}}{\pgfqpoint{3.764251in}{7.659986in}}{\pgfqpoint{3.772065in}{7.652172in}}%
\pgfpathcurveto{\pgfqpoint{3.779878in}{7.644359in}}{\pgfqpoint{3.790477in}{7.639968in}}{\pgfqpoint{3.801528in}{7.639968in}}%
\pgfpathclose%
\pgfusepath{stroke,fill}%
\end{pgfscope}%
\begin{pgfscope}%
\pgfpathrectangle{\pgfqpoint{0.570343in}{0.331635in}}{\pgfqpoint{9.300000in}{7.700000in}}%
\pgfusepath{clip}%
\pgfsetbuttcap%
\pgfsetroundjoin%
\definecolor{currentfill}{rgb}{1.000000,0.705882,0.509804}%
\pgfsetfillcolor{currentfill}%
\pgfsetlinewidth{0.481800pt}%
\definecolor{currentstroke}{rgb}{1.000000,1.000000,1.000000}%
\pgfsetstrokecolor{currentstroke}%
\pgfsetdash{}{0pt}%
\pgfpathmoveto{\pgfqpoint{1.181459in}{6.288777in}}%
\pgfpathcurveto{\pgfqpoint{1.192509in}{6.288777in}}{\pgfqpoint{1.203108in}{6.293168in}}{\pgfqpoint{1.210922in}{6.300981in}}%
\pgfpathcurveto{\pgfqpoint{1.218735in}{6.308795in}}{\pgfqpoint{1.223125in}{6.319394in}}{\pgfqpoint{1.223125in}{6.330444in}}%
\pgfpathcurveto{\pgfqpoint{1.223125in}{6.341494in}}{\pgfqpoint{1.218735in}{6.352093in}}{\pgfqpoint{1.210922in}{6.359907in}}%
\pgfpathcurveto{\pgfqpoint{1.203108in}{6.367720in}}{\pgfqpoint{1.192509in}{6.372111in}}{\pgfqpoint{1.181459in}{6.372111in}}%
\pgfpathcurveto{\pgfqpoint{1.170409in}{6.372111in}}{\pgfqpoint{1.159810in}{6.367720in}}{\pgfqpoint{1.151996in}{6.359907in}}%
\pgfpathcurveto{\pgfqpoint{1.144182in}{6.352093in}}{\pgfqpoint{1.139792in}{6.341494in}}{\pgfqpoint{1.139792in}{6.330444in}}%
\pgfpathcurveto{\pgfqpoint{1.139792in}{6.319394in}}{\pgfqpoint{1.144182in}{6.308795in}}{\pgfqpoint{1.151996in}{6.300981in}}%
\pgfpathcurveto{\pgfqpoint{1.159810in}{6.293168in}}{\pgfqpoint{1.170409in}{6.288777in}}{\pgfqpoint{1.181459in}{6.288777in}}%
\pgfpathclose%
\pgfusepath{stroke,fill}%
\end{pgfscope}%
\begin{pgfscope}%
\pgfpathrectangle{\pgfqpoint{0.570343in}{0.331635in}}{\pgfqpoint{9.300000in}{7.700000in}}%
\pgfusepath{clip}%
\pgfsetbuttcap%
\pgfsetroundjoin%
\definecolor{currentfill}{rgb}{1.000000,0.705882,0.509804}%
\pgfsetfillcolor{currentfill}%
\pgfsetlinewidth{0.481800pt}%
\definecolor{currentstroke}{rgb}{1.000000,1.000000,1.000000}%
\pgfsetstrokecolor{currentstroke}%
\pgfsetdash{}{0pt}%
\pgfpathmoveto{\pgfqpoint{2.718007in}{3.604351in}}%
\pgfpathcurveto{\pgfqpoint{2.729058in}{3.604351in}}{\pgfqpoint{2.739657in}{3.608742in}}{\pgfqpoint{2.747470in}{3.616555in}}%
\pgfpathcurveto{\pgfqpoint{2.755284in}{3.624369in}}{\pgfqpoint{2.759674in}{3.634968in}}{\pgfqpoint{2.759674in}{3.646018in}}%
\pgfpathcurveto{\pgfqpoint{2.759674in}{3.657068in}}{\pgfqpoint{2.755284in}{3.667667in}}{\pgfqpoint{2.747470in}{3.675481in}}%
\pgfpathcurveto{\pgfqpoint{2.739657in}{3.683295in}}{\pgfqpoint{2.729058in}{3.687685in}}{\pgfqpoint{2.718007in}{3.687685in}}%
\pgfpathcurveto{\pgfqpoint{2.706957in}{3.687685in}}{\pgfqpoint{2.696358in}{3.683295in}}{\pgfqpoint{2.688545in}{3.675481in}}%
\pgfpathcurveto{\pgfqpoint{2.680731in}{3.667667in}}{\pgfqpoint{2.676341in}{3.657068in}}{\pgfqpoint{2.676341in}{3.646018in}}%
\pgfpathcurveto{\pgfqpoint{2.676341in}{3.634968in}}{\pgfqpoint{2.680731in}{3.624369in}}{\pgfqpoint{2.688545in}{3.616555in}}%
\pgfpathcurveto{\pgfqpoint{2.696358in}{3.608742in}}{\pgfqpoint{2.706957in}{3.604351in}}{\pgfqpoint{2.718007in}{3.604351in}}%
\pgfpathclose%
\pgfusepath{stroke,fill}%
\end{pgfscope}%
\begin{pgfscope}%
\pgfpathrectangle{\pgfqpoint{0.570343in}{0.331635in}}{\pgfqpoint{9.300000in}{7.700000in}}%
\pgfusepath{clip}%
\pgfsetbuttcap%
\pgfsetroundjoin%
\definecolor{currentfill}{rgb}{1.000000,0.705882,0.509804}%
\pgfsetfillcolor{currentfill}%
\pgfsetlinewidth{0.481800pt}%
\definecolor{currentstroke}{rgb}{1.000000,1.000000,1.000000}%
\pgfsetstrokecolor{currentstroke}%
\pgfsetdash{}{0pt}%
\pgfpathmoveto{\pgfqpoint{6.602631in}{2.817343in}}%
\pgfpathcurveto{\pgfqpoint{6.613681in}{2.817343in}}{\pgfqpoint{6.624280in}{2.821733in}}{\pgfqpoint{6.632093in}{2.829547in}}%
\pgfpathcurveto{\pgfqpoint{6.639907in}{2.837360in}}{\pgfqpoint{6.644297in}{2.847959in}}{\pgfqpoint{6.644297in}{2.859010in}}%
\pgfpathcurveto{\pgfqpoint{6.644297in}{2.870060in}}{\pgfqpoint{6.639907in}{2.880659in}}{\pgfqpoint{6.632093in}{2.888472in}}%
\pgfpathcurveto{\pgfqpoint{6.624280in}{2.896286in}}{\pgfqpoint{6.613681in}{2.900676in}}{\pgfqpoint{6.602631in}{2.900676in}}%
\pgfpathcurveto{\pgfqpoint{6.591580in}{2.900676in}}{\pgfqpoint{6.580981in}{2.896286in}}{\pgfqpoint{6.573168in}{2.888472in}}%
\pgfpathcurveto{\pgfqpoint{6.565354in}{2.880659in}}{\pgfqpoint{6.560964in}{2.870060in}}{\pgfqpoint{6.560964in}{2.859010in}}%
\pgfpathcurveto{\pgfqpoint{6.560964in}{2.847959in}}{\pgfqpoint{6.565354in}{2.837360in}}{\pgfqpoint{6.573168in}{2.829547in}}%
\pgfpathcurveto{\pgfqpoint{6.580981in}{2.821733in}}{\pgfqpoint{6.591580in}{2.817343in}}{\pgfqpoint{6.602631in}{2.817343in}}%
\pgfpathclose%
\pgfusepath{stroke,fill}%
\end{pgfscope}%
\begin{pgfscope}%
\pgfpathrectangle{\pgfqpoint{0.570343in}{0.331635in}}{\pgfqpoint{9.300000in}{7.700000in}}%
\pgfusepath{clip}%
\pgfsetbuttcap%
\pgfsetroundjoin%
\definecolor{currentfill}{rgb}{1.000000,0.705882,0.509804}%
\pgfsetfillcolor{currentfill}%
\pgfsetlinewidth{0.481800pt}%
\definecolor{currentstroke}{rgb}{1.000000,1.000000,1.000000}%
\pgfsetstrokecolor{currentstroke}%
\pgfsetdash{}{0pt}%
\pgfpathmoveto{\pgfqpoint{1.957036in}{6.560089in}}%
\pgfpathcurveto{\pgfqpoint{1.968087in}{6.560089in}}{\pgfqpoint{1.978686in}{6.564479in}}{\pgfqpoint{1.986499in}{6.572292in}}%
\pgfpathcurveto{\pgfqpoint{1.994313in}{6.580106in}}{\pgfqpoint{1.998703in}{6.590705in}}{\pgfqpoint{1.998703in}{6.601755in}}%
\pgfpathcurveto{\pgfqpoint{1.998703in}{6.612805in}}{\pgfqpoint{1.994313in}{6.623404in}}{\pgfqpoint{1.986499in}{6.631218in}}%
\pgfpathcurveto{\pgfqpoint{1.978686in}{6.639032in}}{\pgfqpoint{1.968087in}{6.643422in}}{\pgfqpoint{1.957036in}{6.643422in}}%
\pgfpathcurveto{\pgfqpoint{1.945986in}{6.643422in}}{\pgfqpoint{1.935387in}{6.639032in}}{\pgfqpoint{1.927574in}{6.631218in}}%
\pgfpathcurveto{\pgfqpoint{1.919760in}{6.623404in}}{\pgfqpoint{1.915370in}{6.612805in}}{\pgfqpoint{1.915370in}{6.601755in}}%
\pgfpathcurveto{\pgfqpoint{1.915370in}{6.590705in}}{\pgfqpoint{1.919760in}{6.580106in}}{\pgfqpoint{1.927574in}{6.572292in}}%
\pgfpathcurveto{\pgfqpoint{1.935387in}{6.564479in}}{\pgfqpoint{1.945986in}{6.560089in}}{\pgfqpoint{1.957036in}{6.560089in}}%
\pgfpathclose%
\pgfusepath{stroke,fill}%
\end{pgfscope}%
\begin{pgfscope}%
\pgfpathrectangle{\pgfqpoint{0.570343in}{0.331635in}}{\pgfqpoint{9.300000in}{7.700000in}}%
\pgfusepath{clip}%
\pgfsetbuttcap%
\pgfsetroundjoin%
\definecolor{currentfill}{rgb}{1.000000,0.705882,0.509804}%
\pgfsetfillcolor{currentfill}%
\pgfsetlinewidth{0.481800pt}%
\definecolor{currentstroke}{rgb}{1.000000,1.000000,1.000000}%
\pgfsetstrokecolor{currentstroke}%
\pgfsetdash{}{0pt}%
\pgfpathmoveto{\pgfqpoint{7.683729in}{6.257111in}}%
\pgfpathcurveto{\pgfqpoint{7.694779in}{6.257111in}}{\pgfqpoint{7.705378in}{6.261501in}}{\pgfqpoint{7.713192in}{6.269315in}}%
\pgfpathcurveto{\pgfqpoint{7.721005in}{6.277128in}}{\pgfqpoint{7.725395in}{6.287727in}}{\pgfqpoint{7.725395in}{6.298777in}}%
\pgfpathcurveto{\pgfqpoint{7.725395in}{6.309827in}}{\pgfqpoint{7.721005in}{6.320427in}}{\pgfqpoint{7.713192in}{6.328240in}}%
\pgfpathcurveto{\pgfqpoint{7.705378in}{6.336054in}}{\pgfqpoint{7.694779in}{6.340444in}}{\pgfqpoint{7.683729in}{6.340444in}}%
\pgfpathcurveto{\pgfqpoint{7.672679in}{6.340444in}}{\pgfqpoint{7.662080in}{6.336054in}}{\pgfqpoint{7.654266in}{6.328240in}}%
\pgfpathcurveto{\pgfqpoint{7.646452in}{6.320427in}}{\pgfqpoint{7.642062in}{6.309827in}}{\pgfqpoint{7.642062in}{6.298777in}}%
\pgfpathcurveto{\pgfqpoint{7.642062in}{6.287727in}}{\pgfqpoint{7.646452in}{6.277128in}}{\pgfqpoint{7.654266in}{6.269315in}}%
\pgfpathcurveto{\pgfqpoint{7.662080in}{6.261501in}}{\pgfqpoint{7.672679in}{6.257111in}}{\pgfqpoint{7.683729in}{6.257111in}}%
\pgfpathclose%
\pgfusepath{stroke,fill}%
\end{pgfscope}%
\begin{pgfscope}%
\pgfpathrectangle{\pgfqpoint{0.570343in}{0.331635in}}{\pgfqpoint{9.300000in}{7.700000in}}%
\pgfusepath{clip}%
\pgfsetbuttcap%
\pgfsetroundjoin%
\definecolor{currentfill}{rgb}{1.000000,0.705882,0.509804}%
\pgfsetfillcolor{currentfill}%
\pgfsetlinewidth{0.481800pt}%
\definecolor{currentstroke}{rgb}{1.000000,1.000000,1.000000}%
\pgfsetstrokecolor{currentstroke}%
\pgfsetdash{}{0pt}%
\pgfpathmoveto{\pgfqpoint{8.418297in}{5.134422in}}%
\pgfpathcurveto{\pgfqpoint{8.429348in}{5.134422in}}{\pgfqpoint{8.439947in}{5.138813in}}{\pgfqpoint{8.447760in}{5.146626in}}%
\pgfpathcurveto{\pgfqpoint{8.455574in}{5.154440in}}{\pgfqpoint{8.459964in}{5.165039in}}{\pgfqpoint{8.459964in}{5.176089in}}%
\pgfpathcurveto{\pgfqpoint{8.459964in}{5.187139in}}{\pgfqpoint{8.455574in}{5.197738in}}{\pgfqpoint{8.447760in}{5.205552in}}%
\pgfpathcurveto{\pgfqpoint{8.439947in}{5.213365in}}{\pgfqpoint{8.429348in}{5.217756in}}{\pgfqpoint{8.418297in}{5.217756in}}%
\pgfpathcurveto{\pgfqpoint{8.407247in}{5.217756in}}{\pgfqpoint{8.396648in}{5.213365in}}{\pgfqpoint{8.388835in}{5.205552in}}%
\pgfpathcurveto{\pgfqpoint{8.381021in}{5.197738in}}{\pgfqpoint{8.376631in}{5.187139in}}{\pgfqpoint{8.376631in}{5.176089in}}%
\pgfpathcurveto{\pgfqpoint{8.376631in}{5.165039in}}{\pgfqpoint{8.381021in}{5.154440in}}{\pgfqpoint{8.388835in}{5.146626in}}%
\pgfpathcurveto{\pgfqpoint{8.396648in}{5.138813in}}{\pgfqpoint{8.407247in}{5.134422in}}{\pgfqpoint{8.418297in}{5.134422in}}%
\pgfpathclose%
\pgfusepath{stroke,fill}%
\end{pgfscope}%
\begin{pgfscope}%
\pgfpathrectangle{\pgfqpoint{0.570343in}{0.331635in}}{\pgfqpoint{9.300000in}{7.700000in}}%
\pgfusepath{clip}%
\pgfsetbuttcap%
\pgfsetroundjoin%
\definecolor{currentfill}{rgb}{1.000000,0.705882,0.509804}%
\pgfsetfillcolor{currentfill}%
\pgfsetlinewidth{0.481800pt}%
\definecolor{currentstroke}{rgb}{1.000000,1.000000,1.000000}%
\pgfsetstrokecolor{currentstroke}%
\pgfsetdash{}{0pt}%
\pgfpathmoveto{\pgfqpoint{1.819365in}{3.967926in}}%
\pgfpathcurveto{\pgfqpoint{1.830415in}{3.967926in}}{\pgfqpoint{1.841014in}{3.972317in}}{\pgfqpoint{1.848828in}{3.980130in}}%
\pgfpathcurveto{\pgfqpoint{1.856642in}{3.987944in}}{\pgfqpoint{1.861032in}{3.998543in}}{\pgfqpoint{1.861032in}{4.009593in}}%
\pgfpathcurveto{\pgfqpoint{1.861032in}{4.020643in}}{\pgfqpoint{1.856642in}{4.031242in}}{\pgfqpoint{1.848828in}{4.039056in}}%
\pgfpathcurveto{\pgfqpoint{1.841014in}{4.046870in}}{\pgfqpoint{1.830415in}{4.051260in}}{\pgfqpoint{1.819365in}{4.051260in}}%
\pgfpathcurveto{\pgfqpoint{1.808315in}{4.051260in}}{\pgfqpoint{1.797716in}{4.046870in}}{\pgfqpoint{1.789902in}{4.039056in}}%
\pgfpathcurveto{\pgfqpoint{1.782089in}{4.031242in}}{\pgfqpoint{1.777699in}{4.020643in}}{\pgfqpoint{1.777699in}{4.009593in}}%
\pgfpathcurveto{\pgfqpoint{1.777699in}{3.998543in}}{\pgfqpoint{1.782089in}{3.987944in}}{\pgfqpoint{1.789902in}{3.980130in}}%
\pgfpathcurveto{\pgfqpoint{1.797716in}{3.972317in}}{\pgfqpoint{1.808315in}{3.967926in}}{\pgfqpoint{1.819365in}{3.967926in}}%
\pgfpathclose%
\pgfusepath{stroke,fill}%
\end{pgfscope}%
\begin{pgfscope}%
\pgfpathrectangle{\pgfqpoint{0.570343in}{0.331635in}}{\pgfqpoint{9.300000in}{7.700000in}}%
\pgfusepath{clip}%
\pgfsetbuttcap%
\pgfsetroundjoin%
\definecolor{currentfill}{rgb}{1.000000,0.705882,0.509804}%
\pgfsetfillcolor{currentfill}%
\pgfsetlinewidth{0.481800pt}%
\definecolor{currentstroke}{rgb}{1.000000,1.000000,1.000000}%
\pgfsetstrokecolor{currentstroke}%
\pgfsetdash{}{0pt}%
\pgfpathmoveto{\pgfqpoint{3.672079in}{3.458964in}}%
\pgfpathcurveto{\pgfqpoint{3.683129in}{3.458964in}}{\pgfqpoint{3.693728in}{3.463354in}}{\pgfqpoint{3.701542in}{3.471168in}}%
\pgfpathcurveto{\pgfqpoint{3.709356in}{3.478982in}}{\pgfqpoint{3.713746in}{3.489581in}}{\pgfqpoint{3.713746in}{3.500631in}}%
\pgfpathcurveto{\pgfqpoint{3.713746in}{3.511681in}}{\pgfqpoint{3.709356in}{3.522280in}}{\pgfqpoint{3.701542in}{3.530094in}}%
\pgfpathcurveto{\pgfqpoint{3.693728in}{3.537907in}}{\pgfqpoint{3.683129in}{3.542297in}}{\pgfqpoint{3.672079in}{3.542297in}}%
\pgfpathcurveto{\pgfqpoint{3.661029in}{3.542297in}}{\pgfqpoint{3.650430in}{3.537907in}}{\pgfqpoint{3.642616in}{3.530094in}}%
\pgfpathcurveto{\pgfqpoint{3.634803in}{3.522280in}}{\pgfqpoint{3.630413in}{3.511681in}}{\pgfqpoint{3.630413in}{3.500631in}}%
\pgfpathcurveto{\pgfqpoint{3.630413in}{3.489581in}}{\pgfqpoint{3.634803in}{3.478982in}}{\pgfqpoint{3.642616in}{3.471168in}}%
\pgfpathcurveto{\pgfqpoint{3.650430in}{3.463354in}}{\pgfqpoint{3.661029in}{3.458964in}}{\pgfqpoint{3.672079in}{3.458964in}}%
\pgfpathclose%
\pgfusepath{stroke,fill}%
\end{pgfscope}%
\begin{pgfscope}%
\pgfpathrectangle{\pgfqpoint{0.570343in}{0.331635in}}{\pgfqpoint{9.300000in}{7.700000in}}%
\pgfusepath{clip}%
\pgfsetbuttcap%
\pgfsetroundjoin%
\definecolor{currentfill}{rgb}{1.000000,0.705882,0.509804}%
\pgfsetfillcolor{currentfill}%
\pgfsetlinewidth{0.481800pt}%
\definecolor{currentstroke}{rgb}{1.000000,1.000000,1.000000}%
\pgfsetstrokecolor{currentstroke}%
\pgfsetdash{}{0pt}%
\pgfpathmoveto{\pgfqpoint{2.042788in}{5.632611in}}%
\pgfpathcurveto{\pgfqpoint{2.053838in}{5.632611in}}{\pgfqpoint{2.064437in}{5.637001in}}{\pgfqpoint{2.072251in}{5.644815in}}%
\pgfpathcurveto{\pgfqpoint{2.080065in}{5.652628in}}{\pgfqpoint{2.084455in}{5.663227in}}{\pgfqpoint{2.084455in}{5.674277in}}%
\pgfpathcurveto{\pgfqpoint{2.084455in}{5.685328in}}{\pgfqpoint{2.080065in}{5.695927in}}{\pgfqpoint{2.072251in}{5.703740in}}%
\pgfpathcurveto{\pgfqpoint{2.064437in}{5.711554in}}{\pgfqpoint{2.053838in}{5.715944in}}{\pgfqpoint{2.042788in}{5.715944in}}%
\pgfpathcurveto{\pgfqpoint{2.031738in}{5.715944in}}{\pgfqpoint{2.021139in}{5.711554in}}{\pgfqpoint{2.013325in}{5.703740in}}%
\pgfpathcurveto{\pgfqpoint{2.005512in}{5.695927in}}{\pgfqpoint{2.001122in}{5.685328in}}{\pgfqpoint{2.001122in}{5.674277in}}%
\pgfpathcurveto{\pgfqpoint{2.001122in}{5.663227in}}{\pgfqpoint{2.005512in}{5.652628in}}{\pgfqpoint{2.013325in}{5.644815in}}%
\pgfpathcurveto{\pgfqpoint{2.021139in}{5.637001in}}{\pgfqpoint{2.031738in}{5.632611in}}{\pgfqpoint{2.042788in}{5.632611in}}%
\pgfpathclose%
\pgfusepath{stroke,fill}%
\end{pgfscope}%
\begin{pgfscope}%
\pgfpathrectangle{\pgfqpoint{0.570343in}{0.331635in}}{\pgfqpoint{9.300000in}{7.700000in}}%
\pgfusepath{clip}%
\pgfsetbuttcap%
\pgfsetroundjoin%
\definecolor{currentfill}{rgb}{1.000000,0.705882,0.509804}%
\pgfsetfillcolor{currentfill}%
\pgfsetlinewidth{0.481800pt}%
\definecolor{currentstroke}{rgb}{1.000000,1.000000,1.000000}%
\pgfsetstrokecolor{currentstroke}%
\pgfsetdash{}{0pt}%
\pgfpathmoveto{\pgfqpoint{4.877879in}{2.790593in}}%
\pgfpathcurveto{\pgfqpoint{4.888930in}{2.790593in}}{\pgfqpoint{4.899529in}{2.794983in}}{\pgfqpoint{4.907342in}{2.802797in}}%
\pgfpathcurveto{\pgfqpoint{4.915156in}{2.810611in}}{\pgfqpoint{4.919546in}{2.821210in}}{\pgfqpoint{4.919546in}{2.832260in}}%
\pgfpathcurveto{\pgfqpoint{4.919546in}{2.843310in}}{\pgfqpoint{4.915156in}{2.853909in}}{\pgfqpoint{4.907342in}{2.861723in}}%
\pgfpathcurveto{\pgfqpoint{4.899529in}{2.869536in}}{\pgfqpoint{4.888930in}{2.873926in}}{\pgfqpoint{4.877879in}{2.873926in}}%
\pgfpathcurveto{\pgfqpoint{4.866829in}{2.873926in}}{\pgfqpoint{4.856230in}{2.869536in}}{\pgfqpoint{4.848417in}{2.861723in}}%
\pgfpathcurveto{\pgfqpoint{4.840603in}{2.853909in}}{\pgfqpoint{4.836213in}{2.843310in}}{\pgfqpoint{4.836213in}{2.832260in}}%
\pgfpathcurveto{\pgfqpoint{4.836213in}{2.821210in}}{\pgfqpoint{4.840603in}{2.810611in}}{\pgfqpoint{4.848417in}{2.802797in}}%
\pgfpathcurveto{\pgfqpoint{4.856230in}{2.794983in}}{\pgfqpoint{4.866829in}{2.790593in}}{\pgfqpoint{4.877879in}{2.790593in}}%
\pgfpathclose%
\pgfusepath{stroke,fill}%
\end{pgfscope}%
\begin{pgfscope}%
\pgfpathrectangle{\pgfqpoint{0.570343in}{0.331635in}}{\pgfqpoint{9.300000in}{7.700000in}}%
\pgfusepath{clip}%
\pgfsetbuttcap%
\pgfsetroundjoin%
\definecolor{currentfill}{rgb}{0.631373,0.788235,0.956863}%
\pgfsetfillcolor{currentfill}%
\pgfsetlinewidth{1.003750pt}%
\definecolor{currentstroke}{rgb}{0.631373,0.788235,0.956863}%
\pgfsetstrokecolor{currentstroke}%
\pgfsetdash{}{0pt}%
\pgfsys@defobject{currentmarker}{\pgfqpoint{-0.041667in}{-0.041667in}}{\pgfqpoint{0.041667in}{0.041667in}}{%
\pgfpathmoveto{\pgfqpoint{0.000000in}{-0.041667in}}%
\pgfpathcurveto{\pgfqpoint{0.011050in}{-0.041667in}}{\pgfqpoint{0.021649in}{-0.037276in}}{\pgfqpoint{0.029463in}{-0.029463in}}%
\pgfpathcurveto{\pgfqpoint{0.037276in}{-0.021649in}}{\pgfqpoint{0.041667in}{-0.011050in}}{\pgfqpoint{0.041667in}{0.000000in}}%
\pgfpathcurveto{\pgfqpoint{0.041667in}{0.011050in}}{\pgfqpoint{0.037276in}{0.021649in}}{\pgfqpoint{0.029463in}{0.029463in}}%
\pgfpathcurveto{\pgfqpoint{0.021649in}{0.037276in}}{\pgfqpoint{0.011050in}{0.041667in}}{\pgfqpoint{0.000000in}{0.041667in}}%
\pgfpathcurveto{\pgfqpoint{-0.011050in}{0.041667in}}{\pgfqpoint{-0.021649in}{0.037276in}}{\pgfqpoint{-0.029463in}{0.029463in}}%
\pgfpathcurveto{\pgfqpoint{-0.037276in}{0.021649in}}{\pgfqpoint{-0.041667in}{0.011050in}}{\pgfqpoint{-0.041667in}{0.000000in}}%
\pgfpathcurveto{\pgfqpoint{-0.041667in}{-0.011050in}}{\pgfqpoint{-0.037276in}{-0.021649in}}{\pgfqpoint{-0.029463in}{-0.029463in}}%
\pgfpathcurveto{\pgfqpoint{-0.021649in}{-0.037276in}}{\pgfqpoint{-0.011050in}{-0.041667in}}{\pgfqpoint{0.000000in}{-0.041667in}}%
\pgfpathclose%
\pgfusepath{stroke,fill}%
}%
\end{pgfscope}%
\begin{pgfscope}%
\pgfpathrectangle{\pgfqpoint{0.570343in}{0.331635in}}{\pgfqpoint{9.300000in}{7.700000in}}%
\pgfusepath{clip}%
\pgfsetbuttcap%
\pgfsetroundjoin%
\definecolor{currentfill}{rgb}{1.000000,0.705882,0.509804}%
\pgfsetfillcolor{currentfill}%
\pgfsetlinewidth{1.003750pt}%
\definecolor{currentstroke}{rgb}{1.000000,0.705882,0.509804}%
\pgfsetstrokecolor{currentstroke}%
\pgfsetdash{}{0pt}%
\pgfsys@defobject{currentmarker}{\pgfqpoint{-0.041667in}{-0.041667in}}{\pgfqpoint{0.041667in}{0.041667in}}{%
\pgfpathmoveto{\pgfqpoint{0.000000in}{-0.041667in}}%
\pgfpathcurveto{\pgfqpoint{0.011050in}{-0.041667in}}{\pgfqpoint{0.021649in}{-0.037276in}}{\pgfqpoint{0.029463in}{-0.029463in}}%
\pgfpathcurveto{\pgfqpoint{0.037276in}{-0.021649in}}{\pgfqpoint{0.041667in}{-0.011050in}}{\pgfqpoint{0.041667in}{0.000000in}}%
\pgfpathcurveto{\pgfqpoint{0.041667in}{0.011050in}}{\pgfqpoint{0.037276in}{0.021649in}}{\pgfqpoint{0.029463in}{0.029463in}}%
\pgfpathcurveto{\pgfqpoint{0.021649in}{0.037276in}}{\pgfqpoint{0.011050in}{0.041667in}}{\pgfqpoint{0.000000in}{0.041667in}}%
\pgfpathcurveto{\pgfqpoint{-0.011050in}{0.041667in}}{\pgfqpoint{-0.021649in}{0.037276in}}{\pgfqpoint{-0.029463in}{0.029463in}}%
\pgfpathcurveto{\pgfqpoint{-0.037276in}{0.021649in}}{\pgfqpoint{-0.041667in}{0.011050in}}{\pgfqpoint{-0.041667in}{0.000000in}}%
\pgfpathcurveto{\pgfqpoint{-0.041667in}{-0.011050in}}{\pgfqpoint{-0.037276in}{-0.021649in}}{\pgfqpoint{-0.029463in}{-0.029463in}}%
\pgfpathcurveto{\pgfqpoint{-0.021649in}{-0.037276in}}{\pgfqpoint{-0.011050in}{-0.041667in}}{\pgfqpoint{0.000000in}{-0.041667in}}%
\pgfpathclose%
\pgfusepath{stroke,fill}%
}%
\end{pgfscope}%
\begin{pgfscope}%
\pgfsetbuttcap%
\pgfsetroundjoin%
\definecolor{currentfill}{rgb}{0.000000,0.000000,0.000000}%
\pgfsetfillcolor{currentfill}%
\pgfsetlinewidth{0.803000pt}%
\definecolor{currentstroke}{rgb}{0.000000,0.000000,0.000000}%
\pgfsetstrokecolor{currentstroke}%
\pgfsetdash{}{0pt}%
\pgfsys@defobject{currentmarker}{\pgfqpoint{0.000000in}{-0.048611in}}{\pgfqpoint{0.000000in}{0.000000in}}{%
\pgfpathmoveto{\pgfqpoint{0.000000in}{0.000000in}}%
\pgfpathlineto{\pgfqpoint{0.000000in}{-0.048611in}}%
\pgfusepath{stroke,fill}%
}%
\begin{pgfscope}%
\pgfsys@transformshift{1.291791in}{0.331635in}%
\pgfsys@useobject{currentmarker}{}%
\end{pgfscope}%
\end{pgfscope}%
\begin{pgfscope}%
\definecolor{textcolor}{rgb}{0.000000,0.000000,0.000000}%
\pgfsetstrokecolor{textcolor}%
\pgfsetfillcolor{textcolor}%
\pgftext[x=1.291791in,y=0.234413in,,top]{\color{textcolor}\sffamily\fontsize{10.000000}{12.000000}\selectfont \ensuremath{-}400}%
\end{pgfscope}%
\begin{pgfscope}%
\pgfsetbuttcap%
\pgfsetroundjoin%
\definecolor{currentfill}{rgb}{0.000000,0.000000,0.000000}%
\pgfsetfillcolor{currentfill}%
\pgfsetlinewidth{0.803000pt}%
\definecolor{currentstroke}{rgb}{0.000000,0.000000,0.000000}%
\pgfsetstrokecolor{currentstroke}%
\pgfsetdash{}{0pt}%
\pgfsys@defobject{currentmarker}{\pgfqpoint{0.000000in}{-0.048611in}}{\pgfqpoint{0.000000in}{0.000000in}}{%
\pgfpathmoveto{\pgfqpoint{0.000000in}{0.000000in}}%
\pgfpathlineto{\pgfqpoint{0.000000in}{-0.048611in}}%
\pgfusepath{stroke,fill}%
}%
\begin{pgfscope}%
\pgfsys@transformshift{3.107960in}{0.331635in}%
\pgfsys@useobject{currentmarker}{}%
\end{pgfscope}%
\end{pgfscope}%
\begin{pgfscope}%
\definecolor{textcolor}{rgb}{0.000000,0.000000,0.000000}%
\pgfsetstrokecolor{textcolor}%
\pgfsetfillcolor{textcolor}%
\pgftext[x=3.107960in,y=0.234413in,,top]{\color{textcolor}\sffamily\fontsize{10.000000}{12.000000}\selectfont \ensuremath{-}200}%
\end{pgfscope}%
\begin{pgfscope}%
\pgfsetbuttcap%
\pgfsetroundjoin%
\definecolor{currentfill}{rgb}{0.000000,0.000000,0.000000}%
\pgfsetfillcolor{currentfill}%
\pgfsetlinewidth{0.803000pt}%
\definecolor{currentstroke}{rgb}{0.000000,0.000000,0.000000}%
\pgfsetstrokecolor{currentstroke}%
\pgfsetdash{}{0pt}%
\pgfsys@defobject{currentmarker}{\pgfqpoint{0.000000in}{-0.048611in}}{\pgfqpoint{0.000000in}{0.000000in}}{%
\pgfpathmoveto{\pgfqpoint{0.000000in}{0.000000in}}%
\pgfpathlineto{\pgfqpoint{0.000000in}{-0.048611in}}%
\pgfusepath{stroke,fill}%
}%
\begin{pgfscope}%
\pgfsys@transformshift{4.924129in}{0.331635in}%
\pgfsys@useobject{currentmarker}{}%
\end{pgfscope}%
\end{pgfscope}%
\begin{pgfscope}%
\definecolor{textcolor}{rgb}{0.000000,0.000000,0.000000}%
\pgfsetstrokecolor{textcolor}%
\pgfsetfillcolor{textcolor}%
\pgftext[x=4.924129in,y=0.234413in,,top]{\color{textcolor}\sffamily\fontsize{10.000000}{12.000000}\selectfont 0}%
\end{pgfscope}%
\begin{pgfscope}%
\pgfsetbuttcap%
\pgfsetroundjoin%
\definecolor{currentfill}{rgb}{0.000000,0.000000,0.000000}%
\pgfsetfillcolor{currentfill}%
\pgfsetlinewidth{0.803000pt}%
\definecolor{currentstroke}{rgb}{0.000000,0.000000,0.000000}%
\pgfsetstrokecolor{currentstroke}%
\pgfsetdash{}{0pt}%
\pgfsys@defobject{currentmarker}{\pgfqpoint{0.000000in}{-0.048611in}}{\pgfqpoint{0.000000in}{0.000000in}}{%
\pgfpathmoveto{\pgfqpoint{0.000000in}{0.000000in}}%
\pgfpathlineto{\pgfqpoint{0.000000in}{-0.048611in}}%
\pgfusepath{stroke,fill}%
}%
\begin{pgfscope}%
\pgfsys@transformshift{6.740298in}{0.331635in}%
\pgfsys@useobject{currentmarker}{}%
\end{pgfscope}%
\end{pgfscope}%
\begin{pgfscope}%
\definecolor{textcolor}{rgb}{0.000000,0.000000,0.000000}%
\pgfsetstrokecolor{textcolor}%
\pgfsetfillcolor{textcolor}%
\pgftext[x=6.740298in,y=0.234413in,,top]{\color{textcolor}\sffamily\fontsize{10.000000}{12.000000}\selectfont 200}%
\end{pgfscope}%
\begin{pgfscope}%
\pgfsetbuttcap%
\pgfsetroundjoin%
\definecolor{currentfill}{rgb}{0.000000,0.000000,0.000000}%
\pgfsetfillcolor{currentfill}%
\pgfsetlinewidth{0.803000pt}%
\definecolor{currentstroke}{rgb}{0.000000,0.000000,0.000000}%
\pgfsetstrokecolor{currentstroke}%
\pgfsetdash{}{0pt}%
\pgfsys@defobject{currentmarker}{\pgfqpoint{0.000000in}{-0.048611in}}{\pgfqpoint{0.000000in}{0.000000in}}{%
\pgfpathmoveto{\pgfqpoint{0.000000in}{0.000000in}}%
\pgfpathlineto{\pgfqpoint{0.000000in}{-0.048611in}}%
\pgfusepath{stroke,fill}%
}%
\begin{pgfscope}%
\pgfsys@transformshift{8.556467in}{0.331635in}%
\pgfsys@useobject{currentmarker}{}%
\end{pgfscope}%
\end{pgfscope}%
\begin{pgfscope}%
\definecolor{textcolor}{rgb}{0.000000,0.000000,0.000000}%
\pgfsetstrokecolor{textcolor}%
\pgfsetfillcolor{textcolor}%
\pgftext[x=8.556467in,y=0.234413in,,top]{\color{textcolor}\sffamily\fontsize{10.000000}{12.000000}\selectfont 400}%
\end{pgfscope}%
\begin{pgfscope}%
\pgfsetbuttcap%
\pgfsetroundjoin%
\definecolor{currentfill}{rgb}{0.000000,0.000000,0.000000}%
\pgfsetfillcolor{currentfill}%
\pgfsetlinewidth{0.803000pt}%
\definecolor{currentstroke}{rgb}{0.000000,0.000000,0.000000}%
\pgfsetstrokecolor{currentstroke}%
\pgfsetdash{}{0pt}%
\pgfsys@defobject{currentmarker}{\pgfqpoint{-0.048611in}{0.000000in}}{\pgfqpoint{-0.000000in}{0.000000in}}{%
\pgfpathmoveto{\pgfqpoint{-0.000000in}{0.000000in}}%
\pgfpathlineto{\pgfqpoint{-0.048611in}{0.000000in}}%
\pgfusepath{stroke,fill}%
}%
\begin{pgfscope}%
\pgfsys@transformshift{0.570343in}{1.104137in}%
\pgfsys@useobject{currentmarker}{}%
\end{pgfscope}%
\end{pgfscope}%
\begin{pgfscope}%
\definecolor{textcolor}{rgb}{0.000000,0.000000,0.000000}%
\pgfsetstrokecolor{textcolor}%
\pgfsetfillcolor{textcolor}%
\pgftext[x=0.100000in, y=1.051375in, left, base]{\color{textcolor}\sffamily\fontsize{10.000000}{12.000000}\selectfont \ensuremath{-}400}%
\end{pgfscope}%
\begin{pgfscope}%
\pgfsetbuttcap%
\pgfsetroundjoin%
\definecolor{currentfill}{rgb}{0.000000,0.000000,0.000000}%
\pgfsetfillcolor{currentfill}%
\pgfsetlinewidth{0.803000pt}%
\definecolor{currentstroke}{rgb}{0.000000,0.000000,0.000000}%
\pgfsetstrokecolor{currentstroke}%
\pgfsetdash{}{0pt}%
\pgfsys@defobject{currentmarker}{\pgfqpoint{-0.048611in}{0.000000in}}{\pgfqpoint{-0.000000in}{0.000000in}}{%
\pgfpathmoveto{\pgfqpoint{-0.000000in}{0.000000in}}%
\pgfpathlineto{\pgfqpoint{-0.048611in}{0.000000in}}%
\pgfusepath{stroke,fill}%
}%
\begin{pgfscope}%
\pgfsys@transformshift{0.570343in}{2.777216in}%
\pgfsys@useobject{currentmarker}{}%
\end{pgfscope}%
\end{pgfscope}%
\begin{pgfscope}%
\definecolor{textcolor}{rgb}{0.000000,0.000000,0.000000}%
\pgfsetstrokecolor{textcolor}%
\pgfsetfillcolor{textcolor}%
\pgftext[x=0.100000in, y=2.724454in, left, base]{\color{textcolor}\sffamily\fontsize{10.000000}{12.000000}\selectfont \ensuremath{-}200}%
\end{pgfscope}%
\begin{pgfscope}%
\pgfsetbuttcap%
\pgfsetroundjoin%
\definecolor{currentfill}{rgb}{0.000000,0.000000,0.000000}%
\pgfsetfillcolor{currentfill}%
\pgfsetlinewidth{0.803000pt}%
\definecolor{currentstroke}{rgb}{0.000000,0.000000,0.000000}%
\pgfsetstrokecolor{currentstroke}%
\pgfsetdash{}{0pt}%
\pgfsys@defobject{currentmarker}{\pgfqpoint{-0.048611in}{0.000000in}}{\pgfqpoint{-0.000000in}{0.000000in}}{%
\pgfpathmoveto{\pgfqpoint{-0.000000in}{0.000000in}}%
\pgfpathlineto{\pgfqpoint{-0.048611in}{0.000000in}}%
\pgfusepath{stroke,fill}%
}%
\begin{pgfscope}%
\pgfsys@transformshift{0.570343in}{4.450295in}%
\pgfsys@useobject{currentmarker}{}%
\end{pgfscope}%
\end{pgfscope}%
\begin{pgfscope}%
\definecolor{textcolor}{rgb}{0.000000,0.000000,0.000000}%
\pgfsetstrokecolor{textcolor}%
\pgfsetfillcolor{textcolor}%
\pgftext[x=0.384756in, y=4.397533in, left, base]{\color{textcolor}\sffamily\fontsize{10.000000}{12.000000}\selectfont 0}%
\end{pgfscope}%
\begin{pgfscope}%
\pgfsetbuttcap%
\pgfsetroundjoin%
\definecolor{currentfill}{rgb}{0.000000,0.000000,0.000000}%
\pgfsetfillcolor{currentfill}%
\pgfsetlinewidth{0.803000pt}%
\definecolor{currentstroke}{rgb}{0.000000,0.000000,0.000000}%
\pgfsetstrokecolor{currentstroke}%
\pgfsetdash{}{0pt}%
\pgfsys@defobject{currentmarker}{\pgfqpoint{-0.048611in}{0.000000in}}{\pgfqpoint{-0.000000in}{0.000000in}}{%
\pgfpathmoveto{\pgfqpoint{-0.000000in}{0.000000in}}%
\pgfpathlineto{\pgfqpoint{-0.048611in}{0.000000in}}%
\pgfusepath{stroke,fill}%
}%
\begin{pgfscope}%
\pgfsys@transformshift{0.570343in}{6.123374in}%
\pgfsys@useobject{currentmarker}{}%
\end{pgfscope}%
\end{pgfscope}%
\begin{pgfscope}%
\definecolor{textcolor}{rgb}{0.000000,0.000000,0.000000}%
\pgfsetstrokecolor{textcolor}%
\pgfsetfillcolor{textcolor}%
\pgftext[x=0.208025in, y=6.070612in, left, base]{\color{textcolor}\sffamily\fontsize{10.000000}{12.000000}\selectfont 200}%
\end{pgfscope}%
\begin{pgfscope}%
\pgfsetbuttcap%
\pgfsetroundjoin%
\definecolor{currentfill}{rgb}{0.000000,0.000000,0.000000}%
\pgfsetfillcolor{currentfill}%
\pgfsetlinewidth{0.803000pt}%
\definecolor{currentstroke}{rgb}{0.000000,0.000000,0.000000}%
\pgfsetstrokecolor{currentstroke}%
\pgfsetdash{}{0pt}%
\pgfsys@defobject{currentmarker}{\pgfqpoint{-0.048611in}{0.000000in}}{\pgfqpoint{-0.000000in}{0.000000in}}{%
\pgfpathmoveto{\pgfqpoint{-0.000000in}{0.000000in}}%
\pgfpathlineto{\pgfqpoint{-0.048611in}{0.000000in}}%
\pgfusepath{stroke,fill}%
}%
\begin{pgfscope}%
\pgfsys@transformshift{0.570343in}{7.796452in}%
\pgfsys@useobject{currentmarker}{}%
\end{pgfscope}%
\end{pgfscope}%
\begin{pgfscope}%
\definecolor{textcolor}{rgb}{0.000000,0.000000,0.000000}%
\pgfsetstrokecolor{textcolor}%
\pgfsetfillcolor{textcolor}%
\pgftext[x=0.208025in, y=7.743691in, left, base]{\color{textcolor}\sffamily\fontsize{10.000000}{12.000000}\selectfont 400}%
\end{pgfscope}%
\begin{pgfscope}%
\pgfpathrectangle{\pgfqpoint{0.570343in}{0.331635in}}{\pgfqpoint{9.300000in}{7.700000in}}%
\pgfusepath{clip}%
\pgfsetrectcap%
\pgfsetroundjoin%
\pgfsetlinewidth{1.505625pt}%
\definecolor{currentstroke}{rgb}{0.631373,0.788235,0.956863}%
\pgfsetstrokecolor{currentstroke}%
\pgfsetstrokeopacity{0.800000}%
\pgfsetdash{}{0pt}%
\pgfpathmoveto{\pgfqpoint{6.993440in}{5.428474in}}%
\pgfpathlineto{\pgfqpoint{5.170297in}{3.838787in}}%
\pgfusepath{stroke}%
\end{pgfscope}%
\begin{pgfscope}%
\pgfpathrectangle{\pgfqpoint{0.570343in}{0.331635in}}{\pgfqpoint{9.300000in}{7.700000in}}%
\pgfusepath{clip}%
\pgfsetrectcap%
\pgfsetroundjoin%
\pgfsetlinewidth{1.505625pt}%
\definecolor{currentstroke}{rgb}{0.631373,0.788235,0.956863}%
\pgfsetstrokecolor{currentstroke}%
\pgfsetstrokeopacity{0.800000}%
\pgfsetdash{}{0pt}%
\pgfpathmoveto{\pgfqpoint{7.514448in}{1.678445in}}%
\pgfpathlineto{\pgfqpoint{5.170297in}{3.838787in}}%
\pgfusepath{stroke}%
\end{pgfscope}%
\begin{pgfscope}%
\pgfpathrectangle{\pgfqpoint{0.570343in}{0.331635in}}{\pgfqpoint{9.300000in}{7.700000in}}%
\pgfusepath{clip}%
\pgfsetrectcap%
\pgfsetroundjoin%
\pgfsetlinewidth{1.505625pt}%
\definecolor{currentstroke}{rgb}{0.631373,0.788235,0.956863}%
\pgfsetstrokecolor{currentstroke}%
\pgfsetstrokeopacity{0.800000}%
\pgfsetdash{}{0pt}%
\pgfpathmoveto{\pgfqpoint{5.023692in}{7.180984in}}%
\pgfpathlineto{\pgfqpoint{5.170297in}{3.838787in}}%
\pgfusepath{stroke}%
\end{pgfscope}%
\begin{pgfscope}%
\pgfpathrectangle{\pgfqpoint{0.570343in}{0.331635in}}{\pgfqpoint{9.300000in}{7.700000in}}%
\pgfusepath{clip}%
\pgfsetrectcap%
\pgfsetroundjoin%
\pgfsetlinewidth{1.505625pt}%
\definecolor{currentstroke}{rgb}{0.631373,0.788235,0.956863}%
\pgfsetstrokecolor{currentstroke}%
\pgfsetstrokeopacity{0.800000}%
\pgfsetdash{}{0pt}%
\pgfpathmoveto{\pgfqpoint{3.832172in}{4.301862in}}%
\pgfpathlineto{\pgfqpoint{5.170297in}{3.838787in}}%
\pgfusepath{stroke}%
\end{pgfscope}%
\begin{pgfscope}%
\pgfpathrectangle{\pgfqpoint{0.570343in}{0.331635in}}{\pgfqpoint{9.300000in}{7.700000in}}%
\pgfusepath{clip}%
\pgfsetrectcap%
\pgfsetroundjoin%
\pgfsetlinewidth{1.505625pt}%
\definecolor{currentstroke}{rgb}{0.631373,0.788235,0.956863}%
\pgfsetstrokecolor{currentstroke}%
\pgfsetstrokeopacity{0.800000}%
\pgfsetdash{}{0pt}%
\pgfpathmoveto{\pgfqpoint{5.168249in}{4.675333in}}%
\pgfpathlineto{\pgfqpoint{5.170297in}{3.838787in}}%
\pgfusepath{stroke}%
\end{pgfscope}%
\begin{pgfscope}%
\pgfpathrectangle{\pgfqpoint{0.570343in}{0.331635in}}{\pgfqpoint{9.300000in}{7.700000in}}%
\pgfusepath{clip}%
\pgfsetrectcap%
\pgfsetroundjoin%
\pgfsetlinewidth{1.505625pt}%
\definecolor{currentstroke}{rgb}{0.631373,0.788235,0.956863}%
\pgfsetstrokecolor{currentstroke}%
\pgfsetstrokeopacity{0.800000}%
\pgfsetdash{}{0pt}%
\pgfpathmoveto{\pgfqpoint{4.194348in}{1.414034in}}%
\pgfpathlineto{\pgfqpoint{5.170297in}{3.838787in}}%
\pgfusepath{stroke}%
\end{pgfscope}%
\begin{pgfscope}%
\pgfpathrectangle{\pgfqpoint{0.570343in}{0.331635in}}{\pgfqpoint{9.300000in}{7.700000in}}%
\pgfusepath{clip}%
\pgfsetrectcap%
\pgfsetroundjoin%
\pgfsetlinewidth{1.505625pt}%
\definecolor{currentstroke}{rgb}{0.631373,0.788235,0.956863}%
\pgfsetstrokecolor{currentstroke}%
\pgfsetstrokeopacity{0.800000}%
\pgfsetdash{}{0pt}%
\pgfpathmoveto{\pgfqpoint{2.750613in}{5.841837in}}%
\pgfpathlineto{\pgfqpoint{5.170297in}{3.838787in}}%
\pgfusepath{stroke}%
\end{pgfscope}%
\begin{pgfscope}%
\pgfpathrectangle{\pgfqpoint{0.570343in}{0.331635in}}{\pgfqpoint{9.300000in}{7.700000in}}%
\pgfusepath{clip}%
\pgfsetrectcap%
\pgfsetroundjoin%
\pgfsetlinewidth{1.505625pt}%
\definecolor{currentstroke}{rgb}{0.631373,0.788235,0.956863}%
\pgfsetstrokecolor{currentstroke}%
\pgfsetstrokeopacity{0.800000}%
\pgfsetdash{}{0pt}%
\pgfpathmoveto{\pgfqpoint{3.118617in}{2.397944in}}%
\pgfpathlineto{\pgfqpoint{5.170297in}{3.838787in}}%
\pgfusepath{stroke}%
\end{pgfscope}%
\begin{pgfscope}%
\pgfpathrectangle{\pgfqpoint{0.570343in}{0.331635in}}{\pgfqpoint{9.300000in}{7.700000in}}%
\pgfusepath{clip}%
\pgfsetrectcap%
\pgfsetroundjoin%
\pgfsetlinewidth{1.505625pt}%
\definecolor{currentstroke}{rgb}{0.631373,0.788235,0.956863}%
\pgfsetstrokecolor{currentstroke}%
\pgfsetstrokeopacity{0.800000}%
\pgfsetdash{}{0pt}%
\pgfpathmoveto{\pgfqpoint{5.057402in}{1.890172in}}%
\pgfpathlineto{\pgfqpoint{5.170297in}{3.838787in}}%
\pgfusepath{stroke}%
\end{pgfscope}%
\begin{pgfscope}%
\pgfpathrectangle{\pgfqpoint{0.570343in}{0.331635in}}{\pgfqpoint{9.300000in}{7.700000in}}%
\pgfusepath{clip}%
\pgfsetrectcap%
\pgfsetroundjoin%
\pgfsetlinewidth{1.505625pt}%
\definecolor{currentstroke}{rgb}{0.631373,0.788235,0.956863}%
\pgfsetstrokecolor{currentstroke}%
\pgfsetstrokeopacity{0.800000}%
\pgfsetdash{}{0pt}%
\pgfpathmoveto{\pgfqpoint{5.910435in}{2.357904in}}%
\pgfpathlineto{\pgfqpoint{5.170297in}{3.838787in}}%
\pgfusepath{stroke}%
\end{pgfscope}%
\begin{pgfscope}%
\pgfpathrectangle{\pgfqpoint{0.570343in}{0.331635in}}{\pgfqpoint{9.300000in}{7.700000in}}%
\pgfusepath{clip}%
\pgfsetrectcap%
\pgfsetroundjoin%
\pgfsetlinewidth{1.505625pt}%
\definecolor{currentstroke}{rgb}{0.631373,0.788235,0.956863}%
\pgfsetstrokecolor{currentstroke}%
\pgfsetstrokeopacity{0.800000}%
\pgfsetdash{}{0pt}%
\pgfpathmoveto{\pgfqpoint{5.846427in}{5.296493in}}%
\pgfpathlineto{\pgfqpoint{5.170297in}{3.838787in}}%
\pgfusepath{stroke}%
\end{pgfscope}%
\begin{pgfscope}%
\pgfpathrectangle{\pgfqpoint{0.570343in}{0.331635in}}{\pgfqpoint{9.300000in}{7.700000in}}%
\pgfusepath{clip}%
\pgfsetrectcap%
\pgfsetroundjoin%
\pgfsetlinewidth{1.505625pt}%
\definecolor{currentstroke}{rgb}{0.631373,0.788235,0.956863}%
\pgfsetstrokecolor{currentstroke}%
\pgfsetstrokeopacity{0.800000}%
\pgfsetdash{}{0pt}%
\pgfpathmoveto{\pgfqpoint{4.284170in}{4.791321in}}%
\pgfpathlineto{\pgfqpoint{5.170297in}{3.838787in}}%
\pgfusepath{stroke}%
\end{pgfscope}%
\begin{pgfscope}%
\pgfpathrectangle{\pgfqpoint{0.570343in}{0.331635in}}{\pgfqpoint{9.300000in}{7.700000in}}%
\pgfusepath{clip}%
\pgfsetrectcap%
\pgfsetroundjoin%
\pgfsetlinewidth{1.505625pt}%
\definecolor{currentstroke}{rgb}{0.631373,0.788235,0.956863}%
\pgfsetstrokecolor{currentstroke}%
\pgfsetstrokeopacity{0.800000}%
\pgfsetdash{}{0pt}%
\pgfpathmoveto{\pgfqpoint{1.673868in}{2.908498in}}%
\pgfpathlineto{\pgfqpoint{5.170297in}{3.838787in}}%
\pgfusepath{stroke}%
\end{pgfscope}%
\begin{pgfscope}%
\pgfpathrectangle{\pgfqpoint{0.570343in}{0.331635in}}{\pgfqpoint{9.300000in}{7.700000in}}%
\pgfusepath{clip}%
\pgfsetrectcap%
\pgfsetroundjoin%
\pgfsetlinewidth{1.505625pt}%
\definecolor{currentstroke}{rgb}{0.631373,0.788235,0.956863}%
\pgfsetstrokecolor{currentstroke}%
\pgfsetstrokeopacity{0.800000}%
\pgfsetdash{}{0pt}%
\pgfpathmoveto{\pgfqpoint{4.453376in}{3.433671in}}%
\pgfpathlineto{\pgfqpoint{5.170297in}{3.838787in}}%
\pgfusepath{stroke}%
\end{pgfscope}%
\begin{pgfscope}%
\pgfpathrectangle{\pgfqpoint{0.570343in}{0.331635in}}{\pgfqpoint{9.300000in}{7.700000in}}%
\pgfusepath{clip}%
\pgfsetrectcap%
\pgfsetroundjoin%
\pgfsetlinewidth{1.505625pt}%
\definecolor{currentstroke}{rgb}{0.631373,0.788235,0.956863}%
\pgfsetstrokecolor{currentstroke}%
\pgfsetstrokeopacity{0.800000}%
\pgfsetdash{}{0pt}%
\pgfpathmoveto{\pgfqpoint{5.511077in}{3.387822in}}%
\pgfpathlineto{\pgfqpoint{5.170297in}{3.838787in}}%
\pgfusepath{stroke}%
\end{pgfscope}%
\begin{pgfscope}%
\pgfpathrectangle{\pgfqpoint{0.570343in}{0.331635in}}{\pgfqpoint{9.300000in}{7.700000in}}%
\pgfusepath{clip}%
\pgfsetrectcap%
\pgfsetroundjoin%
\pgfsetlinewidth{1.505625pt}%
\definecolor{currentstroke}{rgb}{0.631373,0.788235,0.956863}%
\pgfsetstrokecolor{currentstroke}%
\pgfsetstrokeopacity{0.800000}%
\pgfsetdash{}{0pt}%
\pgfpathmoveto{\pgfqpoint{6.465570in}{3.579135in}}%
\pgfpathlineto{\pgfqpoint{5.170297in}{3.838787in}}%
\pgfusepath{stroke}%
\end{pgfscope}%
\begin{pgfscope}%
\pgfpathrectangle{\pgfqpoint{0.570343in}{0.331635in}}{\pgfqpoint{9.300000in}{7.700000in}}%
\pgfusepath{clip}%
\pgfsetrectcap%
\pgfsetroundjoin%
\pgfsetlinewidth{1.505625pt}%
\definecolor{currentstroke}{rgb}{0.631373,0.788235,0.956863}%
\pgfsetstrokecolor{currentstroke}%
\pgfsetstrokeopacity{0.800000}%
\pgfsetdash{}{0pt}%
\pgfpathmoveto{\pgfqpoint{5.213120in}{0.681635in}}%
\pgfpathlineto{\pgfqpoint{5.170297in}{3.838787in}}%
\pgfusepath{stroke}%
\end{pgfscope}%
\begin{pgfscope}%
\pgfpathrectangle{\pgfqpoint{0.570343in}{0.331635in}}{\pgfqpoint{9.300000in}{7.700000in}}%
\pgfusepath{clip}%
\pgfsetrectcap%
\pgfsetroundjoin%
\pgfsetlinewidth{1.505625pt}%
\definecolor{currentstroke}{rgb}{0.631373,0.788235,0.956863}%
\pgfsetstrokecolor{currentstroke}%
\pgfsetstrokeopacity{0.800000}%
\pgfsetdash{}{0pt}%
\pgfpathmoveto{\pgfqpoint{9.447616in}{4.728782in}}%
\pgfpathlineto{\pgfqpoint{5.170297in}{3.838787in}}%
\pgfusepath{stroke}%
\end{pgfscope}%
\begin{pgfscope}%
\pgfpathrectangle{\pgfqpoint{0.570343in}{0.331635in}}{\pgfqpoint{9.300000in}{7.700000in}}%
\pgfusepath{clip}%
\pgfsetrectcap%
\pgfsetroundjoin%
\pgfsetlinewidth{1.505625pt}%
\definecolor{currentstroke}{rgb}{0.631373,0.788235,0.956863}%
\pgfsetstrokecolor{currentstroke}%
\pgfsetstrokeopacity{0.800000}%
\pgfsetdash{}{0pt}%
\pgfpathmoveto{\pgfqpoint{6.337177in}{4.273623in}}%
\pgfpathlineto{\pgfqpoint{5.170297in}{3.838787in}}%
\pgfusepath{stroke}%
\end{pgfscope}%
\begin{pgfscope}%
\pgfpathrectangle{\pgfqpoint{0.570343in}{0.331635in}}{\pgfqpoint{9.300000in}{7.700000in}}%
\pgfusepath{clip}%
\pgfsetrectcap%
\pgfsetroundjoin%
\pgfsetlinewidth{1.505625pt}%
\definecolor{currentstroke}{rgb}{0.631373,0.788235,0.956863}%
\pgfsetstrokecolor{currentstroke}%
\pgfsetstrokeopacity{0.800000}%
\pgfsetdash{}{0pt}%
\pgfpathmoveto{\pgfqpoint{3.060240in}{4.547560in}}%
\pgfpathlineto{\pgfqpoint{5.170297in}{3.838787in}}%
\pgfusepath{stroke}%
\end{pgfscope}%
\begin{pgfscope}%
\pgfpathrectangle{\pgfqpoint{0.570343in}{0.331635in}}{\pgfqpoint{9.300000in}{7.700000in}}%
\pgfusepath{clip}%
\pgfsetrectcap%
\pgfsetroundjoin%
\pgfsetlinewidth{1.505625pt}%
\definecolor{currentstroke}{rgb}{0.631373,0.788235,0.956863}%
\pgfsetstrokecolor{currentstroke}%
\pgfsetstrokeopacity{0.800000}%
\pgfsetdash{}{0pt}%
\pgfpathmoveto{\pgfqpoint{0.993071in}{5.156095in}}%
\pgfpathlineto{\pgfqpoint{5.170297in}{3.838787in}}%
\pgfusepath{stroke}%
\end{pgfscope}%
\begin{pgfscope}%
\pgfpathrectangle{\pgfqpoint{0.570343in}{0.331635in}}{\pgfqpoint{9.300000in}{7.700000in}}%
\pgfusepath{clip}%
\pgfsetrectcap%
\pgfsetroundjoin%
\pgfsetlinewidth{1.505625pt}%
\definecolor{currentstroke}{rgb}{0.631373,0.788235,0.956863}%
\pgfsetstrokecolor{currentstroke}%
\pgfsetstrokeopacity{0.800000}%
\pgfsetdash{}{0pt}%
\pgfpathmoveto{\pgfqpoint{6.861115in}{0.834540in}}%
\pgfpathlineto{\pgfqpoint{5.170297in}{3.838787in}}%
\pgfusepath{stroke}%
\end{pgfscope}%
\begin{pgfscope}%
\pgfpathrectangle{\pgfqpoint{0.570343in}{0.331635in}}{\pgfqpoint{9.300000in}{7.700000in}}%
\pgfusepath{clip}%
\pgfsetrectcap%
\pgfsetroundjoin%
\pgfsetlinewidth{1.505625pt}%
\definecolor{currentstroke}{rgb}{0.631373,0.788235,0.956863}%
\pgfsetstrokecolor{currentstroke}%
\pgfsetstrokeopacity{0.800000}%
\pgfsetdash{}{0pt}%
\pgfpathmoveto{\pgfqpoint{4.067058in}{2.387548in}}%
\pgfpathlineto{\pgfqpoint{5.170297in}{3.838787in}}%
\pgfusepath{stroke}%
\end{pgfscope}%
\begin{pgfscope}%
\pgfpathrectangle{\pgfqpoint{0.570343in}{0.331635in}}{\pgfqpoint{9.300000in}{7.700000in}}%
\pgfusepath{clip}%
\pgfsetrectcap%
\pgfsetroundjoin%
\pgfsetlinewidth{1.505625pt}%
\definecolor{currentstroke}{rgb}{0.631373,0.788235,0.956863}%
\pgfsetstrokecolor{currentstroke}%
\pgfsetstrokeopacity{0.800000}%
\pgfsetdash{}{0pt}%
\pgfpathmoveto{\pgfqpoint{6.352480in}{5.887763in}}%
\pgfpathlineto{\pgfqpoint{5.170297in}{3.838787in}}%
\pgfusepath{stroke}%
\end{pgfscope}%
\begin{pgfscope}%
\pgfpathrectangle{\pgfqpoint{0.570343in}{0.331635in}}{\pgfqpoint{9.300000in}{7.700000in}}%
\pgfusepath{clip}%
\pgfsetrectcap%
\pgfsetroundjoin%
\pgfsetlinewidth{1.505625pt}%
\definecolor{currentstroke}{rgb}{0.631373,0.788235,0.956863}%
\pgfsetstrokecolor{currentstroke}%
\pgfsetstrokeopacity{0.800000}%
\pgfsetdash{}{0pt}%
\pgfpathmoveto{\pgfqpoint{5.146154in}{5.493929in}}%
\pgfpathlineto{\pgfqpoint{5.170297in}{3.838787in}}%
\pgfusepath{stroke}%
\end{pgfscope}%
\begin{pgfscope}%
\pgfpathrectangle{\pgfqpoint{0.570343in}{0.331635in}}{\pgfqpoint{9.300000in}{7.700000in}}%
\pgfusepath{clip}%
\pgfsetrectcap%
\pgfsetroundjoin%
\pgfsetlinewidth{1.505625pt}%
\definecolor{currentstroke}{rgb}{0.631373,0.788235,0.956863}%
\pgfsetstrokecolor{currentstroke}%
\pgfsetstrokeopacity{0.800000}%
\pgfsetdash{}{0pt}%
\pgfpathmoveto{\pgfqpoint{7.564757in}{2.484970in}}%
\pgfpathlineto{\pgfqpoint{5.170297in}{3.838787in}}%
\pgfusepath{stroke}%
\end{pgfscope}%
\begin{pgfscope}%
\pgfpathrectangle{\pgfqpoint{0.570343in}{0.331635in}}{\pgfqpoint{9.300000in}{7.700000in}}%
\pgfusepath{clip}%
\pgfsetrectcap%
\pgfsetroundjoin%
\pgfsetlinewidth{1.505625pt}%
\definecolor{currentstroke}{rgb}{0.631373,0.788235,0.956863}%
\pgfsetstrokecolor{currentstroke}%
\pgfsetstrokeopacity{0.800000}%
\pgfsetdash{}{0pt}%
\pgfpathmoveto{\pgfqpoint{8.396171in}{4.701811in}}%
\pgfpathlineto{\pgfqpoint{5.170297in}{3.838787in}}%
\pgfusepath{stroke}%
\end{pgfscope}%
\begin{pgfscope}%
\pgfpathrectangle{\pgfqpoint{0.570343in}{0.331635in}}{\pgfqpoint{9.300000in}{7.700000in}}%
\pgfusepath{clip}%
\pgfsetrectcap%
\pgfsetroundjoin%
\pgfsetlinewidth{1.505625pt}%
\definecolor{currentstroke}{rgb}{0.631373,0.788235,0.956863}%
\pgfsetstrokecolor{currentstroke}%
\pgfsetstrokeopacity{0.800000}%
\pgfsetdash{}{0pt}%
\pgfpathmoveto{\pgfqpoint{3.531441in}{5.743837in}}%
\pgfpathlineto{\pgfqpoint{5.170297in}{3.838787in}}%
\pgfusepath{stroke}%
\end{pgfscope}%
\begin{pgfscope}%
\pgfpathrectangle{\pgfqpoint{0.570343in}{0.331635in}}{\pgfqpoint{9.300000in}{7.700000in}}%
\pgfusepath{clip}%
\pgfsetrectcap%
\pgfsetroundjoin%
\pgfsetlinewidth{1.505625pt}%
\definecolor{currentstroke}{rgb}{1.000000,0.705882,0.509804}%
\pgfsetstrokecolor{currentstroke}%
\pgfsetstrokeopacity{0.800000}%
\pgfsetdash{}{0pt}%
\pgfpathmoveto{\pgfqpoint{5.495895in}{6.180825in}}%
\pgfpathlineto{\pgfqpoint{5.013189in}{5.020435in}}%
\pgfusepath{stroke}%
\end{pgfscope}%
\begin{pgfscope}%
\pgfpathrectangle{\pgfqpoint{0.570343in}{0.331635in}}{\pgfqpoint{9.300000in}{7.700000in}}%
\pgfusepath{clip}%
\pgfsetrectcap%
\pgfsetroundjoin%
\pgfsetlinewidth{1.505625pt}%
\definecolor{currentstroke}{rgb}{1.000000,0.705882,0.509804}%
\pgfsetstrokecolor{currentstroke}%
\pgfsetstrokeopacity{0.800000}%
\pgfsetdash{}{0pt}%
\pgfpathmoveto{\pgfqpoint{8.684047in}{3.661174in}}%
\pgfpathlineto{\pgfqpoint{5.013189in}{5.020435in}}%
\pgfusepath{stroke}%
\end{pgfscope}%
\begin{pgfscope}%
\pgfpathrectangle{\pgfqpoint{0.570343in}{0.331635in}}{\pgfqpoint{9.300000in}{7.700000in}}%
\pgfusepath{clip}%
\pgfsetrectcap%
\pgfsetroundjoin%
\pgfsetlinewidth{1.505625pt}%
\definecolor{currentstroke}{rgb}{1.000000,0.705882,0.509804}%
\pgfsetstrokecolor{currentstroke}%
\pgfsetstrokeopacity{0.800000}%
\pgfsetdash{}{0pt}%
\pgfpathmoveto{\pgfqpoint{4.658532in}{4.182157in}}%
\pgfpathlineto{\pgfqpoint{5.013189in}{5.020435in}}%
\pgfusepath{stroke}%
\end{pgfscope}%
\begin{pgfscope}%
\pgfpathrectangle{\pgfqpoint{0.570343in}{0.331635in}}{\pgfqpoint{9.300000in}{7.700000in}}%
\pgfusepath{clip}%
\pgfsetrectcap%
\pgfsetroundjoin%
\pgfsetlinewidth{1.505625pt}%
\definecolor{currentstroke}{rgb}{1.000000,0.705882,0.509804}%
\pgfsetstrokecolor{currentstroke}%
\pgfsetstrokeopacity{0.800000}%
\pgfsetdash{}{0pt}%
\pgfpathmoveto{\pgfqpoint{5.265486in}{3.741523in}}%
\pgfpathlineto{\pgfqpoint{5.013189in}{5.020435in}}%
\pgfusepath{stroke}%
\end{pgfscope}%
\begin{pgfscope}%
\pgfpathrectangle{\pgfqpoint{0.570343in}{0.331635in}}{\pgfqpoint{9.300000in}{7.700000in}}%
\pgfusepath{clip}%
\pgfsetrectcap%
\pgfsetroundjoin%
\pgfsetlinewidth{1.505625pt}%
\definecolor{currentstroke}{rgb}{1.000000,0.705882,0.509804}%
\pgfsetstrokecolor{currentstroke}%
\pgfsetstrokeopacity{0.800000}%
\pgfsetdash{}{0pt}%
\pgfpathmoveto{\pgfqpoint{3.663113in}{6.576983in}}%
\pgfpathlineto{\pgfqpoint{5.013189in}{5.020435in}}%
\pgfusepath{stroke}%
\end{pgfscope}%
\begin{pgfscope}%
\pgfpathrectangle{\pgfqpoint{0.570343in}{0.331635in}}{\pgfqpoint{9.300000in}{7.700000in}}%
\pgfusepath{clip}%
\pgfsetrectcap%
\pgfsetroundjoin%
\pgfsetlinewidth{1.505625pt}%
\definecolor{currentstroke}{rgb}{1.000000,0.705882,0.509804}%
\pgfsetstrokecolor{currentstroke}%
\pgfsetstrokeopacity{0.800000}%
\pgfsetdash{}{0pt}%
\pgfpathmoveto{\pgfqpoint{7.006137in}{4.544186in}}%
\pgfpathlineto{\pgfqpoint{5.013189in}{5.020435in}}%
\pgfusepath{stroke}%
\end{pgfscope}%
\begin{pgfscope}%
\pgfpathrectangle{\pgfqpoint{0.570343in}{0.331635in}}{\pgfqpoint{9.300000in}{7.700000in}}%
\pgfusepath{clip}%
\pgfsetrectcap%
\pgfsetroundjoin%
\pgfsetlinewidth{1.505625pt}%
\definecolor{currentstroke}{rgb}{1.000000,0.705882,0.509804}%
\pgfsetstrokecolor{currentstroke}%
\pgfsetstrokeopacity{0.800000}%
\pgfsetdash{}{0pt}%
\pgfpathmoveto{\pgfqpoint{7.866554in}{4.532163in}}%
\pgfpathlineto{\pgfqpoint{5.013189in}{5.020435in}}%
\pgfusepath{stroke}%
\end{pgfscope}%
\begin{pgfscope}%
\pgfpathrectangle{\pgfqpoint{0.570343in}{0.331635in}}{\pgfqpoint{9.300000in}{7.700000in}}%
\pgfusepath{clip}%
\pgfsetrectcap%
\pgfsetroundjoin%
\pgfsetlinewidth{1.505625pt}%
\definecolor{currentstroke}{rgb}{1.000000,0.705882,0.509804}%
\pgfsetstrokecolor{currentstroke}%
\pgfsetstrokeopacity{0.800000}%
\pgfsetdash{}{0pt}%
\pgfpathmoveto{\pgfqpoint{2.309950in}{4.840333in}}%
\pgfpathlineto{\pgfqpoint{5.013189in}{5.020435in}}%
\pgfusepath{stroke}%
\end{pgfscope}%
\begin{pgfscope}%
\pgfpathrectangle{\pgfqpoint{0.570343in}{0.331635in}}{\pgfqpoint{9.300000in}{7.700000in}}%
\pgfusepath{clip}%
\pgfsetrectcap%
\pgfsetroundjoin%
\pgfsetlinewidth{1.505625pt}%
\definecolor{currentstroke}{rgb}{1.000000,0.705882,0.509804}%
\pgfsetstrokecolor{currentstroke}%
\pgfsetstrokeopacity{0.800000}%
\pgfsetdash{}{0pt}%
\pgfpathmoveto{\pgfqpoint{4.597903in}{6.287807in}}%
\pgfpathlineto{\pgfqpoint{5.013189in}{5.020435in}}%
\pgfusepath{stroke}%
\end{pgfscope}%
\begin{pgfscope}%
\pgfpathrectangle{\pgfqpoint{0.570343in}{0.331635in}}{\pgfqpoint{9.300000in}{7.700000in}}%
\pgfusepath{clip}%
\pgfsetrectcap%
\pgfsetroundjoin%
\pgfsetlinewidth{1.505625pt}%
\definecolor{currentstroke}{rgb}{1.000000,0.705882,0.509804}%
\pgfsetstrokecolor{currentstroke}%
\pgfsetstrokeopacity{0.800000}%
\pgfsetdash{}{0pt}%
\pgfpathmoveto{\pgfqpoint{2.912749in}{6.728158in}}%
\pgfpathlineto{\pgfqpoint{5.013189in}{5.020435in}}%
\pgfusepath{stroke}%
\end{pgfscope}%
\begin{pgfscope}%
\pgfpathrectangle{\pgfqpoint{0.570343in}{0.331635in}}{\pgfqpoint{9.300000in}{7.700000in}}%
\pgfusepath{clip}%
\pgfsetrectcap%
\pgfsetroundjoin%
\pgfsetlinewidth{1.505625pt}%
\definecolor{currentstroke}{rgb}{1.000000,0.705882,0.509804}%
\pgfsetstrokecolor{currentstroke}%
\pgfsetstrokeopacity{0.800000}%
\pgfsetdash{}{0pt}%
\pgfpathmoveto{\pgfqpoint{3.463668in}{5.071694in}}%
\pgfpathlineto{\pgfqpoint{5.013189in}{5.020435in}}%
\pgfusepath{stroke}%
\end{pgfscope}%
\begin{pgfscope}%
\pgfpathrectangle{\pgfqpoint{0.570343in}{0.331635in}}{\pgfqpoint{9.300000in}{7.700000in}}%
\pgfusepath{clip}%
\pgfsetrectcap%
\pgfsetroundjoin%
\pgfsetlinewidth{1.505625pt}%
\definecolor{currentstroke}{rgb}{1.000000,0.705882,0.509804}%
\pgfsetstrokecolor{currentstroke}%
\pgfsetstrokeopacity{0.800000}%
\pgfsetdash{}{0pt}%
\pgfpathmoveto{\pgfqpoint{8.862844in}{2.434362in}}%
\pgfpathlineto{\pgfqpoint{5.013189in}{5.020435in}}%
\pgfusepath{stroke}%
\end{pgfscope}%
\begin{pgfscope}%
\pgfpathrectangle{\pgfqpoint{0.570343in}{0.331635in}}{\pgfqpoint{9.300000in}{7.700000in}}%
\pgfusepath{clip}%
\pgfsetrectcap%
\pgfsetroundjoin%
\pgfsetlinewidth{1.505625pt}%
\definecolor{currentstroke}{rgb}{1.000000,0.705882,0.509804}%
\pgfsetstrokecolor{currentstroke}%
\pgfsetstrokeopacity{0.800000}%
\pgfsetdash{}{0pt}%
\pgfpathmoveto{\pgfqpoint{7.751788in}{5.308578in}}%
\pgfpathlineto{\pgfqpoint{5.013189in}{5.020435in}}%
\pgfusepath{stroke}%
\end{pgfscope}%
\begin{pgfscope}%
\pgfpathrectangle{\pgfqpoint{0.570343in}{0.331635in}}{\pgfqpoint{9.300000in}{7.700000in}}%
\pgfusepath{clip}%
\pgfsetrectcap%
\pgfsetroundjoin%
\pgfsetlinewidth{1.505625pt}%
\definecolor{currentstroke}{rgb}{1.000000,0.705882,0.509804}%
\pgfsetstrokecolor{currentstroke}%
\pgfsetstrokeopacity{0.800000}%
\pgfsetdash{}{0pt}%
\pgfpathmoveto{\pgfqpoint{4.290232in}{5.545295in}}%
\pgfpathlineto{\pgfqpoint{5.013189in}{5.020435in}}%
\pgfusepath{stroke}%
\end{pgfscope}%
\begin{pgfscope}%
\pgfpathrectangle{\pgfqpoint{0.570343in}{0.331635in}}{\pgfqpoint{9.300000in}{7.700000in}}%
\pgfusepath{clip}%
\pgfsetrectcap%
\pgfsetroundjoin%
\pgfsetlinewidth{1.505625pt}%
\definecolor{currentstroke}{rgb}{1.000000,0.705882,0.509804}%
\pgfsetstrokecolor{currentstroke}%
\pgfsetstrokeopacity{0.800000}%
\pgfsetdash{}{0pt}%
\pgfpathmoveto{\pgfqpoint{6.599542in}{6.892360in}}%
\pgfpathlineto{\pgfqpoint{5.013189in}{5.020435in}}%
\pgfusepath{stroke}%
\end{pgfscope}%
\begin{pgfscope}%
\pgfpathrectangle{\pgfqpoint{0.570343in}{0.331635in}}{\pgfqpoint{9.300000in}{7.700000in}}%
\pgfusepath{clip}%
\pgfsetrectcap%
\pgfsetroundjoin%
\pgfsetlinewidth{1.505625pt}%
\definecolor{currentstroke}{rgb}{1.000000,0.705882,0.509804}%
\pgfsetstrokecolor{currentstroke}%
\pgfsetstrokeopacity{0.800000}%
\pgfsetdash{}{0pt}%
\pgfpathmoveto{\pgfqpoint{5.751027in}{4.433028in}}%
\pgfpathlineto{\pgfqpoint{5.013189in}{5.020435in}}%
\pgfusepath{stroke}%
\end{pgfscope}%
\begin{pgfscope}%
\pgfpathrectangle{\pgfqpoint{0.570343in}{0.331635in}}{\pgfqpoint{9.300000in}{7.700000in}}%
\pgfusepath{clip}%
\pgfsetrectcap%
\pgfsetroundjoin%
\pgfsetlinewidth{1.505625pt}%
\definecolor{currentstroke}{rgb}{1.000000,0.705882,0.509804}%
\pgfsetstrokecolor{currentstroke}%
\pgfsetstrokeopacity{0.800000}%
\pgfsetdash{}{0pt}%
\pgfpathmoveto{\pgfqpoint{6.415032in}{5.001064in}}%
\pgfpathlineto{\pgfqpoint{5.013189in}{5.020435in}}%
\pgfusepath{stroke}%
\end{pgfscope}%
\begin{pgfscope}%
\pgfpathrectangle{\pgfqpoint{0.570343in}{0.331635in}}{\pgfqpoint{9.300000in}{7.700000in}}%
\pgfusepath{clip}%
\pgfsetrectcap%
\pgfsetroundjoin%
\pgfsetlinewidth{1.505625pt}%
\definecolor{currentstroke}{rgb}{1.000000,0.705882,0.509804}%
\pgfsetstrokecolor{currentstroke}%
\pgfsetstrokeopacity{0.800000}%
\pgfsetdash{}{0pt}%
\pgfpathmoveto{\pgfqpoint{3.801528in}{7.681635in}}%
\pgfpathlineto{\pgfqpoint{5.013189in}{5.020435in}}%
\pgfusepath{stroke}%
\end{pgfscope}%
\begin{pgfscope}%
\pgfpathrectangle{\pgfqpoint{0.570343in}{0.331635in}}{\pgfqpoint{9.300000in}{7.700000in}}%
\pgfusepath{clip}%
\pgfsetrectcap%
\pgfsetroundjoin%
\pgfsetlinewidth{1.505625pt}%
\definecolor{currentstroke}{rgb}{1.000000,0.705882,0.509804}%
\pgfsetstrokecolor{currentstroke}%
\pgfsetstrokeopacity{0.800000}%
\pgfsetdash{}{0pt}%
\pgfpathmoveto{\pgfqpoint{1.181459in}{6.330444in}}%
\pgfpathlineto{\pgfqpoint{5.013189in}{5.020435in}}%
\pgfusepath{stroke}%
\end{pgfscope}%
\begin{pgfscope}%
\pgfpathrectangle{\pgfqpoint{0.570343in}{0.331635in}}{\pgfqpoint{9.300000in}{7.700000in}}%
\pgfusepath{clip}%
\pgfsetrectcap%
\pgfsetroundjoin%
\pgfsetlinewidth{1.505625pt}%
\definecolor{currentstroke}{rgb}{1.000000,0.705882,0.509804}%
\pgfsetstrokecolor{currentstroke}%
\pgfsetstrokeopacity{0.800000}%
\pgfsetdash{}{0pt}%
\pgfpathmoveto{\pgfqpoint{2.718007in}{3.646018in}}%
\pgfpathlineto{\pgfqpoint{5.013189in}{5.020435in}}%
\pgfusepath{stroke}%
\end{pgfscope}%
\begin{pgfscope}%
\pgfpathrectangle{\pgfqpoint{0.570343in}{0.331635in}}{\pgfqpoint{9.300000in}{7.700000in}}%
\pgfusepath{clip}%
\pgfsetrectcap%
\pgfsetroundjoin%
\pgfsetlinewidth{1.505625pt}%
\definecolor{currentstroke}{rgb}{1.000000,0.705882,0.509804}%
\pgfsetstrokecolor{currentstroke}%
\pgfsetstrokeopacity{0.800000}%
\pgfsetdash{}{0pt}%
\pgfpathmoveto{\pgfqpoint{6.602631in}{2.859010in}}%
\pgfpathlineto{\pgfqpoint{5.013189in}{5.020435in}}%
\pgfusepath{stroke}%
\end{pgfscope}%
\begin{pgfscope}%
\pgfpathrectangle{\pgfqpoint{0.570343in}{0.331635in}}{\pgfqpoint{9.300000in}{7.700000in}}%
\pgfusepath{clip}%
\pgfsetrectcap%
\pgfsetroundjoin%
\pgfsetlinewidth{1.505625pt}%
\definecolor{currentstroke}{rgb}{1.000000,0.705882,0.509804}%
\pgfsetstrokecolor{currentstroke}%
\pgfsetstrokeopacity{0.800000}%
\pgfsetdash{}{0pt}%
\pgfpathmoveto{\pgfqpoint{1.957036in}{6.601755in}}%
\pgfpathlineto{\pgfqpoint{5.013189in}{5.020435in}}%
\pgfusepath{stroke}%
\end{pgfscope}%
\begin{pgfscope}%
\pgfpathrectangle{\pgfqpoint{0.570343in}{0.331635in}}{\pgfqpoint{9.300000in}{7.700000in}}%
\pgfusepath{clip}%
\pgfsetrectcap%
\pgfsetroundjoin%
\pgfsetlinewidth{1.505625pt}%
\definecolor{currentstroke}{rgb}{1.000000,0.705882,0.509804}%
\pgfsetstrokecolor{currentstroke}%
\pgfsetstrokeopacity{0.800000}%
\pgfsetdash{}{0pt}%
\pgfpathmoveto{\pgfqpoint{7.683729in}{6.298777in}}%
\pgfpathlineto{\pgfqpoint{5.013189in}{5.020435in}}%
\pgfusepath{stroke}%
\end{pgfscope}%
\begin{pgfscope}%
\pgfpathrectangle{\pgfqpoint{0.570343in}{0.331635in}}{\pgfqpoint{9.300000in}{7.700000in}}%
\pgfusepath{clip}%
\pgfsetrectcap%
\pgfsetroundjoin%
\pgfsetlinewidth{1.505625pt}%
\definecolor{currentstroke}{rgb}{1.000000,0.705882,0.509804}%
\pgfsetstrokecolor{currentstroke}%
\pgfsetstrokeopacity{0.800000}%
\pgfsetdash{}{0pt}%
\pgfpathmoveto{\pgfqpoint{8.418297in}{5.176089in}}%
\pgfpathlineto{\pgfqpoint{5.013189in}{5.020435in}}%
\pgfusepath{stroke}%
\end{pgfscope}%
\begin{pgfscope}%
\pgfpathrectangle{\pgfqpoint{0.570343in}{0.331635in}}{\pgfqpoint{9.300000in}{7.700000in}}%
\pgfusepath{clip}%
\pgfsetrectcap%
\pgfsetroundjoin%
\pgfsetlinewidth{1.505625pt}%
\definecolor{currentstroke}{rgb}{1.000000,0.705882,0.509804}%
\pgfsetstrokecolor{currentstroke}%
\pgfsetstrokeopacity{0.800000}%
\pgfsetdash{}{0pt}%
\pgfpathmoveto{\pgfqpoint{1.819365in}{4.009593in}}%
\pgfpathlineto{\pgfqpoint{5.013189in}{5.020435in}}%
\pgfusepath{stroke}%
\end{pgfscope}%
\begin{pgfscope}%
\pgfpathrectangle{\pgfqpoint{0.570343in}{0.331635in}}{\pgfqpoint{9.300000in}{7.700000in}}%
\pgfusepath{clip}%
\pgfsetrectcap%
\pgfsetroundjoin%
\pgfsetlinewidth{1.505625pt}%
\definecolor{currentstroke}{rgb}{1.000000,0.705882,0.509804}%
\pgfsetstrokecolor{currentstroke}%
\pgfsetstrokeopacity{0.800000}%
\pgfsetdash{}{0pt}%
\pgfpathmoveto{\pgfqpoint{3.672079in}{3.500631in}}%
\pgfpathlineto{\pgfqpoint{5.013189in}{5.020435in}}%
\pgfusepath{stroke}%
\end{pgfscope}%
\begin{pgfscope}%
\pgfpathrectangle{\pgfqpoint{0.570343in}{0.331635in}}{\pgfqpoint{9.300000in}{7.700000in}}%
\pgfusepath{clip}%
\pgfsetrectcap%
\pgfsetroundjoin%
\pgfsetlinewidth{1.505625pt}%
\definecolor{currentstroke}{rgb}{1.000000,0.705882,0.509804}%
\pgfsetstrokecolor{currentstroke}%
\pgfsetstrokeopacity{0.800000}%
\pgfsetdash{}{0pt}%
\pgfpathmoveto{\pgfqpoint{2.042788in}{5.674277in}}%
\pgfpathlineto{\pgfqpoint{5.013189in}{5.020435in}}%
\pgfusepath{stroke}%
\end{pgfscope}%
\begin{pgfscope}%
\pgfpathrectangle{\pgfqpoint{0.570343in}{0.331635in}}{\pgfqpoint{9.300000in}{7.700000in}}%
\pgfusepath{clip}%
\pgfsetrectcap%
\pgfsetroundjoin%
\pgfsetlinewidth{1.505625pt}%
\definecolor{currentstroke}{rgb}{1.000000,0.705882,0.509804}%
\pgfsetstrokecolor{currentstroke}%
\pgfsetstrokeopacity{0.800000}%
\pgfsetdash{}{0pt}%
\pgfpathmoveto{\pgfqpoint{4.877879in}{2.832260in}}%
\pgfpathlineto{\pgfqpoint{5.013189in}{5.020435in}}%
\pgfusepath{stroke}%
\end{pgfscope}%
\begin{pgfscope}%
\pgfsetrectcap%
\pgfsetmiterjoin%
\pgfsetlinewidth{0.803000pt}%
\definecolor{currentstroke}{rgb}{0.000000,0.000000,0.000000}%
\pgfsetstrokecolor{currentstroke}%
\pgfsetdash{}{0pt}%
\pgfpathmoveto{\pgfqpoint{0.570343in}{0.331635in}}%
\pgfpathlineto{\pgfqpoint{0.570343in}{8.031635in}}%
\pgfusepath{stroke}%
\end{pgfscope}%
\begin{pgfscope}%
\pgfsetrectcap%
\pgfsetmiterjoin%
\pgfsetlinewidth{0.803000pt}%
\definecolor{currentstroke}{rgb}{0.000000,0.000000,0.000000}%
\pgfsetstrokecolor{currentstroke}%
\pgfsetdash{}{0pt}%
\pgfpathmoveto{\pgfqpoint{9.870343in}{0.331635in}}%
\pgfpathlineto{\pgfqpoint{9.870343in}{8.031635in}}%
\pgfusepath{stroke}%
\end{pgfscope}%
\begin{pgfscope}%
\pgfsetrectcap%
\pgfsetmiterjoin%
\pgfsetlinewidth{0.803000pt}%
\definecolor{currentstroke}{rgb}{0.000000,0.000000,0.000000}%
\pgfsetstrokecolor{currentstroke}%
\pgfsetdash{}{0pt}%
\pgfpathmoveto{\pgfqpoint{0.570343in}{0.331635in}}%
\pgfpathlineto{\pgfqpoint{9.870343in}{0.331635in}}%
\pgfusepath{stroke}%
\end{pgfscope}%
\begin{pgfscope}%
\pgfsetrectcap%
\pgfsetmiterjoin%
\pgfsetlinewidth{0.803000pt}%
\definecolor{currentstroke}{rgb}{0.000000,0.000000,0.000000}%
\pgfsetstrokecolor{currentstroke}%
\pgfsetdash{}{0pt}%
\pgfpathmoveto{\pgfqpoint{0.570343in}{8.031635in}}%
\pgfpathlineto{\pgfqpoint{9.870343in}{8.031635in}}%
\pgfusepath{stroke}%
\end{pgfscope}%
\begin{pgfscope}%
\definecolor{textcolor}{rgb}{0.000000,0.000000,0.000000}%
\pgfsetstrokecolor{textcolor}%
\pgfsetfillcolor{textcolor}%
\pgftext[x=5.220343in,y=8.114968in,,base]{\color{textcolor}\sffamily\fontsize{12.000000}{14.400000}\selectfont Photo-Realistic Images}%
\end{pgfscope}%
\begin{pgfscope}%
\pgfsetbuttcap%
\pgfsetmiterjoin%
\definecolor{currentfill}{rgb}{1.000000,1.000000,1.000000}%
\pgfsetfillcolor{currentfill}%
\pgfsetfillopacity{0.800000}%
\pgfsetlinewidth{1.003750pt}%
\definecolor{currentstroke}{rgb}{0.800000,0.800000,0.800000}%
\pgfsetstrokecolor{currentstroke}%
\pgfsetstrokeopacity{0.800000}%
\pgfsetdash{}{0pt}%
\pgfpathmoveto{\pgfqpoint{9.967566in}{3.956944in}}%
\pgfpathlineto{\pgfqpoint{11.059186in}{3.956944in}}%
\pgfpathquadraticcurveto{\pgfqpoint{11.086964in}{3.956944in}}{\pgfqpoint{11.086964in}{3.984722in}}%
\pgfpathlineto{\pgfqpoint{11.086964in}{4.378548in}}%
\pgfpathquadraticcurveto{\pgfqpoint{11.086964in}{4.406326in}}{\pgfqpoint{11.059186in}{4.406326in}}%
\pgfpathlineto{\pgfqpoint{9.967566in}{4.406326in}}%
\pgfpathquadraticcurveto{\pgfqpoint{9.939788in}{4.406326in}}{\pgfqpoint{9.939788in}{4.378548in}}%
\pgfpathlineto{\pgfqpoint{9.939788in}{3.984722in}}%
\pgfpathquadraticcurveto{\pgfqpoint{9.939788in}{3.956944in}}{\pgfqpoint{9.967566in}{3.956944in}}%
\pgfpathclose%
\pgfusepath{stroke,fill}%
\end{pgfscope}%
\begin{pgfscope}%
\pgfsetbuttcap%
\pgfsetroundjoin%
\definecolor{currentfill}{rgb}{0.631373,0.788235,0.956863}%
\pgfsetfillcolor{currentfill}%
\pgfsetlinewidth{1.003750pt}%
\definecolor{currentstroke}{rgb}{0.631373,0.788235,0.956863}%
\pgfsetstrokecolor{currentstroke}%
\pgfsetdash{}{0pt}%
\pgfsys@defobject{currentmarker}{\pgfqpoint{-0.041667in}{-0.041667in}}{\pgfqpoint{0.041667in}{0.041667in}}{%
\pgfpathmoveto{\pgfqpoint{0.000000in}{-0.041667in}}%
\pgfpathcurveto{\pgfqpoint{0.011050in}{-0.041667in}}{\pgfqpoint{0.021649in}{-0.037276in}}{\pgfqpoint{0.029463in}{-0.029463in}}%
\pgfpathcurveto{\pgfqpoint{0.037276in}{-0.021649in}}{\pgfqpoint{0.041667in}{-0.011050in}}{\pgfqpoint{0.041667in}{0.000000in}}%
\pgfpathcurveto{\pgfqpoint{0.041667in}{0.011050in}}{\pgfqpoint{0.037276in}{0.021649in}}{\pgfqpoint{0.029463in}{0.029463in}}%
\pgfpathcurveto{\pgfqpoint{0.021649in}{0.037276in}}{\pgfqpoint{0.011050in}{0.041667in}}{\pgfqpoint{0.000000in}{0.041667in}}%
\pgfpathcurveto{\pgfqpoint{-0.011050in}{0.041667in}}{\pgfqpoint{-0.021649in}{0.037276in}}{\pgfqpoint{-0.029463in}{0.029463in}}%
\pgfpathcurveto{\pgfqpoint{-0.037276in}{0.021649in}}{\pgfqpoint{-0.041667in}{0.011050in}}{\pgfqpoint{-0.041667in}{0.000000in}}%
\pgfpathcurveto{\pgfqpoint{-0.041667in}{-0.011050in}}{\pgfqpoint{-0.037276in}{-0.021649in}}{\pgfqpoint{-0.029463in}{-0.029463in}}%
\pgfpathcurveto{\pgfqpoint{-0.021649in}{-0.037276in}}{\pgfqpoint{-0.011050in}{-0.041667in}}{\pgfqpoint{0.000000in}{-0.041667in}}%
\pgfpathclose%
\pgfusepath{stroke,fill}%
}%
\begin{pgfscope}%
\pgfsys@transformshift{10.134232in}{4.281705in}%
\pgfsys@useobject{currentmarker}{}%
\end{pgfscope}%
\end{pgfscope}%
\begin{pgfscope}%
\definecolor{textcolor}{rgb}{0.000000,0.000000,0.000000}%
\pgfsetstrokecolor{textcolor}%
\pgfsetfillcolor{textcolor}%
\pgftext[x=10.384232in,y=4.245247in,left,base]{\color{textcolor}\sffamily\fontsize{10.000000}{12.000000}\selectfont hypersim}%
\end{pgfscope}%
\begin{pgfscope}%
\pgfsetbuttcap%
\pgfsetroundjoin%
\definecolor{currentfill}{rgb}{1.000000,0.705882,0.509804}%
\pgfsetfillcolor{currentfill}%
\pgfsetlinewidth{1.003750pt}%
\definecolor{currentstroke}{rgb}{1.000000,0.705882,0.509804}%
\pgfsetstrokecolor{currentstroke}%
\pgfsetdash{}{0pt}%
\pgfsys@defobject{currentmarker}{\pgfqpoint{-0.041667in}{-0.041667in}}{\pgfqpoint{0.041667in}{0.041667in}}{%
\pgfpathmoveto{\pgfqpoint{0.000000in}{-0.041667in}}%
\pgfpathcurveto{\pgfqpoint{0.011050in}{-0.041667in}}{\pgfqpoint{0.021649in}{-0.037276in}}{\pgfqpoint{0.029463in}{-0.029463in}}%
\pgfpathcurveto{\pgfqpoint{0.037276in}{-0.021649in}}{\pgfqpoint{0.041667in}{-0.011050in}}{\pgfqpoint{0.041667in}{0.000000in}}%
\pgfpathcurveto{\pgfqpoint{0.041667in}{0.011050in}}{\pgfqpoint{0.037276in}{0.021649in}}{\pgfqpoint{0.029463in}{0.029463in}}%
\pgfpathcurveto{\pgfqpoint{0.021649in}{0.037276in}}{\pgfqpoint{0.011050in}{0.041667in}}{\pgfqpoint{0.000000in}{0.041667in}}%
\pgfpathcurveto{\pgfqpoint{-0.011050in}{0.041667in}}{\pgfqpoint{-0.021649in}{0.037276in}}{\pgfqpoint{-0.029463in}{0.029463in}}%
\pgfpathcurveto{\pgfqpoint{-0.037276in}{0.021649in}}{\pgfqpoint{-0.041667in}{0.011050in}}{\pgfqpoint{-0.041667in}{0.000000in}}%
\pgfpathcurveto{\pgfqpoint{-0.041667in}{-0.011050in}}{\pgfqpoint{-0.037276in}{-0.021649in}}{\pgfqpoint{-0.029463in}{-0.029463in}}%
\pgfpathcurveto{\pgfqpoint{-0.021649in}{-0.037276in}}{\pgfqpoint{-0.011050in}{-0.041667in}}{\pgfqpoint{0.000000in}{-0.041667in}}%
\pgfpathclose%
\pgfusepath{stroke,fill}%
}%
\begin{pgfscope}%
\pgfsys@transformshift{10.134232in}{4.077848in}%
\pgfsys@useobject{currentmarker}{}%
\end{pgfscope}%
\end{pgfscope}%
\begin{pgfscope}%
\definecolor{textcolor}{rgb}{0.000000,0.000000,0.000000}%
\pgfsetstrokecolor{textcolor}%
\pgfsetfillcolor{textcolor}%
\pgftext[x=10.384232in,y=4.041390in,left,base]{\color{textcolor}\sffamily\fontsize{10.000000}{12.000000}\selectfont pix3d}%
\end{pgfscope}%
\end{pgfpicture}%
\makeatother%
\endgroup%
}\\
    \resizebox{0.49\linewidth}{5cm}{%% Creator: Matplotlib, PGF backend
%%
%% To include the figure in your LaTeX document, write
%%   \input{<filename>.pgf}
%%
%% Make sure the required packages are loaded in your preamble
%%   \usepackage{pgf}
%%
%% Figures using additional raster images can only be included by \input if
%% they are in the same directory as the main LaTeX file. For loading figures
%% from other directories you can use the `import` package
%%   \usepackage{import}
%%
%% and then include the figures with
%%   \import{<path to file>}{<filename>.pgf}
%%
%% Matplotlib used the following preamble
%%   \usepackage{fontspec}
%%   \setmainfont{DejaVuSerif.ttf}[Path=\detokenize{/Users/apple/opt/anaconda3/envs/kaolin/lib/python3.7/site-packages/matplotlib/mpl-data/fonts/ttf/}]
%%   \setsansfont{DejaVuSans.ttf}[Path=\detokenize{/Users/apple/opt/anaconda3/envs/kaolin/lib/python3.7/site-packages/matplotlib/mpl-data/fonts/ttf/}]
%%   \setmonofont{DejaVuSansMono.ttf}[Path=\detokenize{/Users/apple/opt/anaconda3/envs/kaolin/lib/python3.7/site-packages/matplotlib/mpl-data/fonts/ttf/}]
%%
\begingroup%
\makeatletter%
\begin{pgfpicture}%
\pgfpathrectangle{\pgfpointorigin}{\pgfqpoint{11.249220in}{8.341596in}}%
\pgfusepath{use as bounding box, clip}%
\begin{pgfscope}%
\pgfsetbuttcap%
\pgfsetmiterjoin%
\definecolor{currentfill}{rgb}{1.000000,1.000000,1.000000}%
\pgfsetfillcolor{currentfill}%
\pgfsetlinewidth{0.000000pt}%
\definecolor{currentstroke}{rgb}{1.000000,1.000000,1.000000}%
\pgfsetstrokecolor{currentstroke}%
\pgfsetdash{}{0pt}%
\pgfpathmoveto{\pgfqpoint{0.000000in}{0.000000in}}%
\pgfpathlineto{\pgfqpoint{11.249220in}{0.000000in}}%
\pgfpathlineto{\pgfqpoint{11.249220in}{8.341596in}}%
\pgfpathlineto{\pgfqpoint{0.000000in}{8.341596in}}%
\pgfpathclose%
\pgfusepath{fill}%
\end{pgfscope}%
\begin{pgfscope}%
\pgfsetbuttcap%
\pgfsetmiterjoin%
\definecolor{currentfill}{rgb}{1.000000,1.000000,1.000000}%
\pgfsetfillcolor{currentfill}%
\pgfsetlinewidth{0.000000pt}%
\definecolor{currentstroke}{rgb}{0.000000,0.000000,0.000000}%
\pgfsetstrokecolor{currentstroke}%
\pgfsetstrokeopacity{0.000000}%
\pgfsetdash{}{0pt}%
\pgfpathmoveto{\pgfqpoint{0.570343in}{0.331635in}}%
\pgfpathlineto{\pgfqpoint{9.870343in}{0.331635in}}%
\pgfpathlineto{\pgfqpoint{9.870343in}{8.031635in}}%
\pgfpathlineto{\pgfqpoint{0.570343in}{8.031635in}}%
\pgfpathclose%
\pgfusepath{fill}%
\end{pgfscope}%
\begin{pgfscope}%
\pgfpathrectangle{\pgfqpoint{0.570343in}{0.331635in}}{\pgfqpoint{9.300000in}{7.700000in}}%
\pgfusepath{clip}%
\pgfsetbuttcap%
\pgfsetroundjoin%
\definecolor{currentfill}{rgb}{0.631373,0.788235,0.956863}%
\pgfsetfillcolor{currentfill}%
\pgfsetlinewidth{0.481800pt}%
\definecolor{currentstroke}{rgb}{1.000000,1.000000,1.000000}%
\pgfsetstrokecolor{currentstroke}%
\pgfsetdash{}{0pt}%
\pgfpathmoveto{\pgfqpoint{7.296937in}{5.048531in}}%
\pgfpathcurveto{\pgfqpoint{7.307987in}{5.048531in}}{\pgfqpoint{7.318586in}{5.052921in}}{\pgfqpoint{7.326400in}{5.060735in}}%
\pgfpathcurveto{\pgfqpoint{7.334213in}{5.068549in}}{\pgfqpoint{7.338604in}{5.079148in}}{\pgfqpoint{7.338604in}{5.090198in}}%
\pgfpathcurveto{\pgfqpoint{7.338604in}{5.101248in}}{\pgfqpoint{7.334213in}{5.111847in}}{\pgfqpoint{7.326400in}{5.119660in}}%
\pgfpathcurveto{\pgfqpoint{7.318586in}{5.127474in}}{\pgfqpoint{7.307987in}{5.131864in}}{\pgfqpoint{7.296937in}{5.131864in}}%
\pgfpathcurveto{\pgfqpoint{7.285887in}{5.131864in}}{\pgfqpoint{7.275288in}{5.127474in}}{\pgfqpoint{7.267474in}{5.119660in}}%
\pgfpathcurveto{\pgfqpoint{7.259661in}{5.111847in}}{\pgfqpoint{7.255270in}{5.101248in}}{\pgfqpoint{7.255270in}{5.090198in}}%
\pgfpathcurveto{\pgfqpoint{7.255270in}{5.079148in}}{\pgfqpoint{7.259661in}{5.068549in}}{\pgfqpoint{7.267474in}{5.060735in}}%
\pgfpathcurveto{\pgfqpoint{7.275288in}{5.052921in}}{\pgfqpoint{7.285887in}{5.048531in}}{\pgfqpoint{7.296937in}{5.048531in}}%
\pgfpathclose%
\pgfusepath{stroke,fill}%
\end{pgfscope}%
\begin{pgfscope}%
\pgfpathrectangle{\pgfqpoint{0.570343in}{0.331635in}}{\pgfqpoint{9.300000in}{7.700000in}}%
\pgfusepath{clip}%
\pgfsetbuttcap%
\pgfsetroundjoin%
\definecolor{currentfill}{rgb}{0.631373,0.788235,0.956863}%
\pgfsetfillcolor{currentfill}%
\pgfsetlinewidth{0.481800pt}%
\definecolor{currentstroke}{rgb}{1.000000,1.000000,1.000000}%
\pgfsetstrokecolor{currentstroke}%
\pgfsetdash{}{0pt}%
\pgfpathmoveto{\pgfqpoint{4.019475in}{3.676065in}}%
\pgfpathcurveto{\pgfqpoint{4.030525in}{3.676065in}}{\pgfqpoint{4.041124in}{3.680455in}}{\pgfqpoint{4.048938in}{3.688268in}}%
\pgfpathcurveto{\pgfqpoint{4.056752in}{3.696082in}}{\pgfqpoint{4.061142in}{3.706681in}}{\pgfqpoint{4.061142in}{3.717731in}}%
\pgfpathcurveto{\pgfqpoint{4.061142in}{3.728781in}}{\pgfqpoint{4.056752in}{3.739380in}}{\pgfqpoint{4.048938in}{3.747194in}}%
\pgfpathcurveto{\pgfqpoint{4.041124in}{3.755008in}}{\pgfqpoint{4.030525in}{3.759398in}}{\pgfqpoint{4.019475in}{3.759398in}}%
\pgfpathcurveto{\pgfqpoint{4.008425in}{3.759398in}}{\pgfqpoint{3.997826in}{3.755008in}}{\pgfqpoint{3.990012in}{3.747194in}}%
\pgfpathcurveto{\pgfqpoint{3.982199in}{3.739380in}}{\pgfqpoint{3.977809in}{3.728781in}}{\pgfqpoint{3.977809in}{3.717731in}}%
\pgfpathcurveto{\pgfqpoint{3.977809in}{3.706681in}}{\pgfqpoint{3.982199in}{3.696082in}}{\pgfqpoint{3.990012in}{3.688268in}}%
\pgfpathcurveto{\pgfqpoint{3.997826in}{3.680455in}}{\pgfqpoint{4.008425in}{3.676065in}}{\pgfqpoint{4.019475in}{3.676065in}}%
\pgfpathclose%
\pgfusepath{stroke,fill}%
\end{pgfscope}%
\begin{pgfscope}%
\pgfpathrectangle{\pgfqpoint{0.570343in}{0.331635in}}{\pgfqpoint{9.300000in}{7.700000in}}%
\pgfusepath{clip}%
\pgfsetbuttcap%
\pgfsetroundjoin%
\definecolor{currentfill}{rgb}{0.631373,0.788235,0.956863}%
\pgfsetfillcolor{currentfill}%
\pgfsetlinewidth{0.481800pt}%
\definecolor{currentstroke}{rgb}{1.000000,1.000000,1.000000}%
\pgfsetstrokecolor{currentstroke}%
\pgfsetdash{}{0pt}%
\pgfpathmoveto{\pgfqpoint{4.684473in}{3.314863in}}%
\pgfpathcurveto{\pgfqpoint{4.695523in}{3.314863in}}{\pgfqpoint{4.706122in}{3.319253in}}{\pgfqpoint{4.713936in}{3.327067in}}%
\pgfpathcurveto{\pgfqpoint{4.721749in}{3.334881in}}{\pgfqpoint{4.726140in}{3.345480in}}{\pgfqpoint{4.726140in}{3.356530in}}%
\pgfpathcurveto{\pgfqpoint{4.726140in}{3.367580in}}{\pgfqpoint{4.721749in}{3.378179in}}{\pgfqpoint{4.713936in}{3.385992in}}%
\pgfpathcurveto{\pgfqpoint{4.706122in}{3.393806in}}{\pgfqpoint{4.695523in}{3.398196in}}{\pgfqpoint{4.684473in}{3.398196in}}%
\pgfpathcurveto{\pgfqpoint{4.673423in}{3.398196in}}{\pgfqpoint{4.662824in}{3.393806in}}{\pgfqpoint{4.655010in}{3.385992in}}%
\pgfpathcurveto{\pgfqpoint{4.647196in}{3.378179in}}{\pgfqpoint{4.642806in}{3.367580in}}{\pgfqpoint{4.642806in}{3.356530in}}%
\pgfpathcurveto{\pgfqpoint{4.642806in}{3.345480in}}{\pgfqpoint{4.647196in}{3.334881in}}{\pgfqpoint{4.655010in}{3.327067in}}%
\pgfpathcurveto{\pgfqpoint{4.662824in}{3.319253in}}{\pgfqpoint{4.673423in}{3.314863in}}{\pgfqpoint{4.684473in}{3.314863in}}%
\pgfpathclose%
\pgfusepath{stroke,fill}%
\end{pgfscope}%
\begin{pgfscope}%
\pgfpathrectangle{\pgfqpoint{0.570343in}{0.331635in}}{\pgfqpoint{9.300000in}{7.700000in}}%
\pgfusepath{clip}%
\pgfsetbuttcap%
\pgfsetroundjoin%
\definecolor{currentfill}{rgb}{0.631373,0.788235,0.956863}%
\pgfsetfillcolor{currentfill}%
\pgfsetlinewidth{0.481800pt}%
\definecolor{currentstroke}{rgb}{1.000000,1.000000,1.000000}%
\pgfsetstrokecolor{currentstroke}%
\pgfsetdash{}{0pt}%
\pgfpathmoveto{\pgfqpoint{7.305182in}{2.797483in}}%
\pgfpathcurveto{\pgfqpoint{7.316232in}{2.797483in}}{\pgfqpoint{7.326831in}{2.801874in}}{\pgfqpoint{7.334644in}{2.809687in}}%
\pgfpathcurveto{\pgfqpoint{7.342458in}{2.817501in}}{\pgfqpoint{7.346848in}{2.828100in}}{\pgfqpoint{7.346848in}{2.839150in}}%
\pgfpathcurveto{\pgfqpoint{7.346848in}{2.850200in}}{\pgfqpoint{7.342458in}{2.860799in}}{\pgfqpoint{7.334644in}{2.868613in}}%
\pgfpathcurveto{\pgfqpoint{7.326831in}{2.876426in}}{\pgfqpoint{7.316232in}{2.880817in}}{\pgfqpoint{7.305182in}{2.880817in}}%
\pgfpathcurveto{\pgfqpoint{7.294131in}{2.880817in}}{\pgfqpoint{7.283532in}{2.876426in}}{\pgfqpoint{7.275719in}{2.868613in}}%
\pgfpathcurveto{\pgfqpoint{7.267905in}{2.860799in}}{\pgfqpoint{7.263515in}{2.850200in}}{\pgfqpoint{7.263515in}{2.839150in}}%
\pgfpathcurveto{\pgfqpoint{7.263515in}{2.828100in}}{\pgfqpoint{7.267905in}{2.817501in}}{\pgfqpoint{7.275719in}{2.809687in}}%
\pgfpathcurveto{\pgfqpoint{7.283532in}{2.801874in}}{\pgfqpoint{7.294131in}{2.797483in}}{\pgfqpoint{7.305182in}{2.797483in}}%
\pgfpathclose%
\pgfusepath{stroke,fill}%
\end{pgfscope}%
\begin{pgfscope}%
\pgfpathrectangle{\pgfqpoint{0.570343in}{0.331635in}}{\pgfqpoint{9.300000in}{7.700000in}}%
\pgfusepath{clip}%
\pgfsetbuttcap%
\pgfsetroundjoin%
\definecolor{currentfill}{rgb}{0.631373,0.788235,0.956863}%
\pgfsetfillcolor{currentfill}%
\pgfsetlinewidth{0.481800pt}%
\definecolor{currentstroke}{rgb}{1.000000,1.000000,1.000000}%
\pgfsetstrokecolor{currentstroke}%
\pgfsetdash{}{0pt}%
\pgfpathmoveto{\pgfqpoint{8.044602in}{5.249564in}}%
\pgfpathcurveto{\pgfqpoint{8.055652in}{5.249564in}}{\pgfqpoint{8.066251in}{5.253955in}}{\pgfqpoint{8.074065in}{5.261768in}}%
\pgfpathcurveto{\pgfqpoint{8.081878in}{5.269582in}}{\pgfqpoint{8.086269in}{5.280181in}}{\pgfqpoint{8.086269in}{5.291231in}}%
\pgfpathcurveto{\pgfqpoint{8.086269in}{5.302281in}}{\pgfqpoint{8.081878in}{5.312880in}}{\pgfqpoint{8.074065in}{5.320694in}}%
\pgfpathcurveto{\pgfqpoint{8.066251in}{5.328508in}}{\pgfqpoint{8.055652in}{5.332898in}}{\pgfqpoint{8.044602in}{5.332898in}}%
\pgfpathcurveto{\pgfqpoint{8.033552in}{5.332898in}}{\pgfqpoint{8.022953in}{5.328508in}}{\pgfqpoint{8.015139in}{5.320694in}}%
\pgfpathcurveto{\pgfqpoint{8.007326in}{5.312880in}}{\pgfqpoint{8.002935in}{5.302281in}}{\pgfqpoint{8.002935in}{5.291231in}}%
\pgfpathcurveto{\pgfqpoint{8.002935in}{5.280181in}}{\pgfqpoint{8.007326in}{5.269582in}}{\pgfqpoint{8.015139in}{5.261768in}}%
\pgfpathcurveto{\pgfqpoint{8.022953in}{5.253955in}}{\pgfqpoint{8.033552in}{5.249564in}}{\pgfqpoint{8.044602in}{5.249564in}}%
\pgfpathclose%
\pgfusepath{stroke,fill}%
\end{pgfscope}%
\begin{pgfscope}%
\pgfpathrectangle{\pgfqpoint{0.570343in}{0.331635in}}{\pgfqpoint{9.300000in}{7.700000in}}%
\pgfusepath{clip}%
\pgfsetbuttcap%
\pgfsetroundjoin%
\definecolor{currentfill}{rgb}{0.631373,0.788235,0.956863}%
\pgfsetfillcolor{currentfill}%
\pgfsetlinewidth{0.481800pt}%
\definecolor{currentstroke}{rgb}{1.000000,1.000000,1.000000}%
\pgfsetstrokecolor{currentstroke}%
\pgfsetdash{}{0pt}%
\pgfpathmoveto{\pgfqpoint{6.828282in}{4.387013in}}%
\pgfpathcurveto{\pgfqpoint{6.839332in}{4.387013in}}{\pgfqpoint{6.849931in}{4.391403in}}{\pgfqpoint{6.857745in}{4.399217in}}%
\pgfpathcurveto{\pgfqpoint{6.865558in}{4.407030in}}{\pgfqpoint{6.869948in}{4.417629in}}{\pgfqpoint{6.869948in}{4.428679in}}%
\pgfpathcurveto{\pgfqpoint{6.869948in}{4.439730in}}{\pgfqpoint{6.865558in}{4.450329in}}{\pgfqpoint{6.857745in}{4.458142in}}%
\pgfpathcurveto{\pgfqpoint{6.849931in}{4.465956in}}{\pgfqpoint{6.839332in}{4.470346in}}{\pgfqpoint{6.828282in}{4.470346in}}%
\pgfpathcurveto{\pgfqpoint{6.817232in}{4.470346in}}{\pgfqpoint{6.806633in}{4.465956in}}{\pgfqpoint{6.798819in}{4.458142in}}%
\pgfpathcurveto{\pgfqpoint{6.791005in}{4.450329in}}{\pgfqpoint{6.786615in}{4.439730in}}{\pgfqpoint{6.786615in}{4.428679in}}%
\pgfpathcurveto{\pgfqpoint{6.786615in}{4.417629in}}{\pgfqpoint{6.791005in}{4.407030in}}{\pgfqpoint{6.798819in}{4.399217in}}%
\pgfpathcurveto{\pgfqpoint{6.806633in}{4.391403in}}{\pgfqpoint{6.817232in}{4.387013in}}{\pgfqpoint{6.828282in}{4.387013in}}%
\pgfpathclose%
\pgfusepath{stroke,fill}%
\end{pgfscope}%
\begin{pgfscope}%
\pgfpathrectangle{\pgfqpoint{0.570343in}{0.331635in}}{\pgfqpoint{9.300000in}{7.700000in}}%
\pgfusepath{clip}%
\pgfsetbuttcap%
\pgfsetroundjoin%
\definecolor{currentfill}{rgb}{0.631373,0.788235,0.956863}%
\pgfsetfillcolor{currentfill}%
\pgfsetlinewidth{0.481800pt}%
\definecolor{currentstroke}{rgb}{1.000000,1.000000,1.000000}%
\pgfsetstrokecolor{currentstroke}%
\pgfsetdash{}{0pt}%
\pgfpathmoveto{\pgfqpoint{8.665804in}{4.777059in}}%
\pgfpathcurveto{\pgfqpoint{8.676854in}{4.777059in}}{\pgfqpoint{8.687453in}{4.781449in}}{\pgfqpoint{8.695266in}{4.789263in}}%
\pgfpathcurveto{\pgfqpoint{8.703080in}{4.797076in}}{\pgfqpoint{8.707470in}{4.807675in}}{\pgfqpoint{8.707470in}{4.818725in}}%
\pgfpathcurveto{\pgfqpoint{8.707470in}{4.829776in}}{\pgfqpoint{8.703080in}{4.840375in}}{\pgfqpoint{8.695266in}{4.848188in}}%
\pgfpathcurveto{\pgfqpoint{8.687453in}{4.856002in}}{\pgfqpoint{8.676854in}{4.860392in}}{\pgfqpoint{8.665804in}{4.860392in}}%
\pgfpathcurveto{\pgfqpoint{8.654754in}{4.860392in}}{\pgfqpoint{8.644154in}{4.856002in}}{\pgfqpoint{8.636341in}{4.848188in}}%
\pgfpathcurveto{\pgfqpoint{8.628527in}{4.840375in}}{\pgfqpoint{8.624137in}{4.829776in}}{\pgfqpoint{8.624137in}{4.818725in}}%
\pgfpathcurveto{\pgfqpoint{8.624137in}{4.807675in}}{\pgfqpoint{8.628527in}{4.797076in}}{\pgfqpoint{8.636341in}{4.789263in}}%
\pgfpathcurveto{\pgfqpoint{8.644154in}{4.781449in}}{\pgfqpoint{8.654754in}{4.777059in}}{\pgfqpoint{8.665804in}{4.777059in}}%
\pgfpathclose%
\pgfusepath{stroke,fill}%
\end{pgfscope}%
\begin{pgfscope}%
\pgfpathrectangle{\pgfqpoint{0.570343in}{0.331635in}}{\pgfqpoint{9.300000in}{7.700000in}}%
\pgfusepath{clip}%
\pgfsetbuttcap%
\pgfsetroundjoin%
\definecolor{currentfill}{rgb}{0.631373,0.788235,0.956863}%
\pgfsetfillcolor{currentfill}%
\pgfsetlinewidth{0.481800pt}%
\definecolor{currentstroke}{rgb}{1.000000,1.000000,1.000000}%
\pgfsetstrokecolor{currentstroke}%
\pgfsetdash{}{0pt}%
\pgfpathmoveto{\pgfqpoint{8.976651in}{5.644673in}}%
\pgfpathcurveto{\pgfqpoint{8.987701in}{5.644673in}}{\pgfqpoint{8.998301in}{5.649063in}}{\pgfqpoint{9.006114in}{5.656877in}}%
\pgfpathcurveto{\pgfqpoint{9.013928in}{5.664690in}}{\pgfqpoint{9.018318in}{5.675289in}}{\pgfqpoint{9.018318in}{5.686339in}}%
\pgfpathcurveto{\pgfqpoint{9.018318in}{5.697390in}}{\pgfqpoint{9.013928in}{5.707989in}}{\pgfqpoint{9.006114in}{5.715802in}}%
\pgfpathcurveto{\pgfqpoint{8.998301in}{5.723616in}}{\pgfqpoint{8.987701in}{5.728006in}}{\pgfqpoint{8.976651in}{5.728006in}}%
\pgfpathcurveto{\pgfqpoint{8.965601in}{5.728006in}}{\pgfqpoint{8.955002in}{5.723616in}}{\pgfqpoint{8.947189in}{5.715802in}}%
\pgfpathcurveto{\pgfqpoint{8.939375in}{5.707989in}}{\pgfqpoint{8.934985in}{5.697390in}}{\pgfqpoint{8.934985in}{5.686339in}}%
\pgfpathcurveto{\pgfqpoint{8.934985in}{5.675289in}}{\pgfqpoint{8.939375in}{5.664690in}}{\pgfqpoint{8.947189in}{5.656877in}}%
\pgfpathcurveto{\pgfqpoint{8.955002in}{5.649063in}}{\pgfqpoint{8.965601in}{5.644673in}}{\pgfqpoint{8.976651in}{5.644673in}}%
\pgfpathclose%
\pgfusepath{stroke,fill}%
\end{pgfscope}%
\begin{pgfscope}%
\pgfpathrectangle{\pgfqpoint{0.570343in}{0.331635in}}{\pgfqpoint{9.300000in}{7.700000in}}%
\pgfusepath{clip}%
\pgfsetbuttcap%
\pgfsetroundjoin%
\definecolor{currentfill}{rgb}{0.631373,0.788235,0.956863}%
\pgfsetfillcolor{currentfill}%
\pgfsetlinewidth{0.481800pt}%
\definecolor{currentstroke}{rgb}{1.000000,1.000000,1.000000}%
\pgfsetstrokecolor{currentstroke}%
\pgfsetdash{}{0pt}%
\pgfpathmoveto{\pgfqpoint{5.272172in}{4.776375in}}%
\pgfpathcurveto{\pgfqpoint{5.283222in}{4.776375in}}{\pgfqpoint{5.293821in}{4.780765in}}{\pgfqpoint{5.301634in}{4.788579in}}%
\pgfpathcurveto{\pgfqpoint{5.309448in}{4.796392in}}{\pgfqpoint{5.313838in}{4.806991in}}{\pgfqpoint{5.313838in}{4.818041in}}%
\pgfpathcurveto{\pgfqpoint{5.313838in}{4.829092in}}{\pgfqpoint{5.309448in}{4.839691in}}{\pgfqpoint{5.301634in}{4.847504in}}%
\pgfpathcurveto{\pgfqpoint{5.293821in}{4.855318in}}{\pgfqpoint{5.283222in}{4.859708in}}{\pgfqpoint{5.272172in}{4.859708in}}%
\pgfpathcurveto{\pgfqpoint{5.261121in}{4.859708in}}{\pgfqpoint{5.250522in}{4.855318in}}{\pgfqpoint{5.242709in}{4.847504in}}%
\pgfpathcurveto{\pgfqpoint{5.234895in}{4.839691in}}{\pgfqpoint{5.230505in}{4.829092in}}{\pgfqpoint{5.230505in}{4.818041in}}%
\pgfpathcurveto{\pgfqpoint{5.230505in}{4.806991in}}{\pgfqpoint{5.234895in}{4.796392in}}{\pgfqpoint{5.242709in}{4.788579in}}%
\pgfpathcurveto{\pgfqpoint{5.250522in}{4.780765in}}{\pgfqpoint{5.261121in}{4.776375in}}{\pgfqpoint{5.272172in}{4.776375in}}%
\pgfpathclose%
\pgfusepath{stroke,fill}%
\end{pgfscope}%
\begin{pgfscope}%
\pgfpathrectangle{\pgfqpoint{0.570343in}{0.331635in}}{\pgfqpoint{9.300000in}{7.700000in}}%
\pgfusepath{clip}%
\pgfsetbuttcap%
\pgfsetroundjoin%
\definecolor{currentfill}{rgb}{0.631373,0.788235,0.956863}%
\pgfsetfillcolor{currentfill}%
\pgfsetlinewidth{0.481800pt}%
\definecolor{currentstroke}{rgb}{1.000000,1.000000,1.000000}%
\pgfsetstrokecolor{currentstroke}%
\pgfsetdash{}{0pt}%
\pgfpathmoveto{\pgfqpoint{3.063486in}{6.430430in}}%
\pgfpathcurveto{\pgfqpoint{3.074536in}{6.430430in}}{\pgfqpoint{3.085135in}{6.434820in}}{\pgfqpoint{3.092949in}{6.442634in}}%
\pgfpathcurveto{\pgfqpoint{3.100762in}{6.450447in}}{\pgfqpoint{3.105152in}{6.461046in}}{\pgfqpoint{3.105152in}{6.472096in}}%
\pgfpathcurveto{\pgfqpoint{3.105152in}{6.483146in}}{\pgfqpoint{3.100762in}{6.493745in}}{\pgfqpoint{3.092949in}{6.501559in}}%
\pgfpathcurveto{\pgfqpoint{3.085135in}{6.509373in}}{\pgfqpoint{3.074536in}{6.513763in}}{\pgfqpoint{3.063486in}{6.513763in}}%
\pgfpathcurveto{\pgfqpoint{3.052436in}{6.513763in}}{\pgfqpoint{3.041837in}{6.509373in}}{\pgfqpoint{3.034023in}{6.501559in}}%
\pgfpathcurveto{\pgfqpoint{3.026209in}{6.493745in}}{\pgfqpoint{3.021819in}{6.483146in}}{\pgfqpoint{3.021819in}{6.472096in}}%
\pgfpathcurveto{\pgfqpoint{3.021819in}{6.461046in}}{\pgfqpoint{3.026209in}{6.450447in}}{\pgfqpoint{3.034023in}{6.442634in}}%
\pgfpathcurveto{\pgfqpoint{3.041837in}{6.434820in}}{\pgfqpoint{3.052436in}{6.430430in}}{\pgfqpoint{3.063486in}{6.430430in}}%
\pgfpathclose%
\pgfusepath{stroke,fill}%
\end{pgfscope}%
\begin{pgfscope}%
\pgfpathrectangle{\pgfqpoint{0.570343in}{0.331635in}}{\pgfqpoint{9.300000in}{7.700000in}}%
\pgfusepath{clip}%
\pgfsetbuttcap%
\pgfsetroundjoin%
\definecolor{currentfill}{rgb}{0.631373,0.788235,0.956863}%
\pgfsetfillcolor{currentfill}%
\pgfsetlinewidth{0.481800pt}%
\definecolor{currentstroke}{rgb}{1.000000,1.000000,1.000000}%
\pgfsetstrokecolor{currentstroke}%
\pgfsetdash{}{0pt}%
\pgfpathmoveto{\pgfqpoint{6.063540in}{4.763605in}}%
\pgfpathcurveto{\pgfqpoint{6.074590in}{4.763605in}}{\pgfqpoint{6.085189in}{4.767995in}}{\pgfqpoint{6.093003in}{4.775808in}}%
\pgfpathcurveto{\pgfqpoint{6.100816in}{4.783622in}}{\pgfqpoint{6.105207in}{4.794221in}}{\pgfqpoint{6.105207in}{4.805271in}}%
\pgfpathcurveto{\pgfqpoint{6.105207in}{4.816321in}}{\pgfqpoint{6.100816in}{4.826920in}}{\pgfqpoint{6.093003in}{4.834734in}}%
\pgfpathcurveto{\pgfqpoint{6.085189in}{4.842548in}}{\pgfqpoint{6.074590in}{4.846938in}}{\pgfqpoint{6.063540in}{4.846938in}}%
\pgfpathcurveto{\pgfqpoint{6.052490in}{4.846938in}}{\pgfqpoint{6.041891in}{4.842548in}}{\pgfqpoint{6.034077in}{4.834734in}}%
\pgfpathcurveto{\pgfqpoint{6.026264in}{4.826920in}}{\pgfqpoint{6.021873in}{4.816321in}}{\pgfqpoint{6.021873in}{4.805271in}}%
\pgfpathcurveto{\pgfqpoint{6.021873in}{4.794221in}}{\pgfqpoint{6.026264in}{4.783622in}}{\pgfqpoint{6.034077in}{4.775808in}}%
\pgfpathcurveto{\pgfqpoint{6.041891in}{4.767995in}}{\pgfqpoint{6.052490in}{4.763605in}}{\pgfqpoint{6.063540in}{4.763605in}}%
\pgfpathclose%
\pgfusepath{stroke,fill}%
\end{pgfscope}%
\begin{pgfscope}%
\pgfpathrectangle{\pgfqpoint{0.570343in}{0.331635in}}{\pgfqpoint{9.300000in}{7.700000in}}%
\pgfusepath{clip}%
\pgfsetbuttcap%
\pgfsetroundjoin%
\definecolor{currentfill}{rgb}{0.631373,0.788235,0.956863}%
\pgfsetfillcolor{currentfill}%
\pgfsetlinewidth{0.481800pt}%
\definecolor{currentstroke}{rgb}{1.000000,1.000000,1.000000}%
\pgfsetstrokecolor{currentstroke}%
\pgfsetdash{}{0pt}%
\pgfpathmoveto{\pgfqpoint{1.762911in}{3.854949in}}%
\pgfpathcurveto{\pgfqpoint{1.773961in}{3.854949in}}{\pgfqpoint{1.784560in}{3.859339in}}{\pgfqpoint{1.792374in}{3.867153in}}%
\pgfpathcurveto{\pgfqpoint{1.800187in}{3.874967in}}{\pgfqpoint{1.804577in}{3.885566in}}{\pgfqpoint{1.804577in}{3.896616in}}%
\pgfpathcurveto{\pgfqpoint{1.804577in}{3.907666in}}{\pgfqpoint{1.800187in}{3.918265in}}{\pgfqpoint{1.792374in}{3.926078in}}%
\pgfpathcurveto{\pgfqpoint{1.784560in}{3.933892in}}{\pgfqpoint{1.773961in}{3.938282in}}{\pgfqpoint{1.762911in}{3.938282in}}%
\pgfpathcurveto{\pgfqpoint{1.751861in}{3.938282in}}{\pgfqpoint{1.741262in}{3.933892in}}{\pgfqpoint{1.733448in}{3.926078in}}%
\pgfpathcurveto{\pgfqpoint{1.725634in}{3.918265in}}{\pgfqpoint{1.721244in}{3.907666in}}{\pgfqpoint{1.721244in}{3.896616in}}%
\pgfpathcurveto{\pgfqpoint{1.721244in}{3.885566in}}{\pgfqpoint{1.725634in}{3.874967in}}{\pgfqpoint{1.733448in}{3.867153in}}%
\pgfpathcurveto{\pgfqpoint{1.741262in}{3.859339in}}{\pgfqpoint{1.751861in}{3.854949in}}{\pgfqpoint{1.762911in}{3.854949in}}%
\pgfpathclose%
\pgfusepath{stroke,fill}%
\end{pgfscope}%
\begin{pgfscope}%
\pgfpathrectangle{\pgfqpoint{0.570343in}{0.331635in}}{\pgfqpoint{9.300000in}{7.700000in}}%
\pgfusepath{clip}%
\pgfsetbuttcap%
\pgfsetroundjoin%
\definecolor{currentfill}{rgb}{0.631373,0.788235,0.956863}%
\pgfsetfillcolor{currentfill}%
\pgfsetlinewidth{0.481800pt}%
\definecolor{currentstroke}{rgb}{1.000000,1.000000,1.000000}%
\pgfsetstrokecolor{currentstroke}%
\pgfsetdash{}{0pt}%
\pgfpathmoveto{\pgfqpoint{4.671643in}{5.114087in}}%
\pgfpathcurveto{\pgfqpoint{4.682693in}{5.114087in}}{\pgfqpoint{4.693292in}{5.118477in}}{\pgfqpoint{4.701106in}{5.126291in}}%
\pgfpathcurveto{\pgfqpoint{4.708919in}{5.134105in}}{\pgfqpoint{4.713310in}{5.144704in}}{\pgfqpoint{4.713310in}{5.155754in}}%
\pgfpathcurveto{\pgfqpoint{4.713310in}{5.166804in}}{\pgfqpoint{4.708919in}{5.177403in}}{\pgfqpoint{4.701106in}{5.185216in}}%
\pgfpathcurveto{\pgfqpoint{4.693292in}{5.193030in}}{\pgfqpoint{4.682693in}{5.197420in}}{\pgfqpoint{4.671643in}{5.197420in}}%
\pgfpathcurveto{\pgfqpoint{4.660593in}{5.197420in}}{\pgfqpoint{4.649994in}{5.193030in}}{\pgfqpoint{4.642180in}{5.185216in}}%
\pgfpathcurveto{\pgfqpoint{4.634367in}{5.177403in}}{\pgfqpoint{4.629976in}{5.166804in}}{\pgfqpoint{4.629976in}{5.155754in}}%
\pgfpathcurveto{\pgfqpoint{4.629976in}{5.144704in}}{\pgfqpoint{4.634367in}{5.134105in}}{\pgfqpoint{4.642180in}{5.126291in}}%
\pgfpathcurveto{\pgfqpoint{4.649994in}{5.118477in}}{\pgfqpoint{4.660593in}{5.114087in}}{\pgfqpoint{4.671643in}{5.114087in}}%
\pgfpathclose%
\pgfusepath{stroke,fill}%
\end{pgfscope}%
\begin{pgfscope}%
\pgfpathrectangle{\pgfqpoint{0.570343in}{0.331635in}}{\pgfqpoint{9.300000in}{7.700000in}}%
\pgfusepath{clip}%
\pgfsetbuttcap%
\pgfsetroundjoin%
\definecolor{currentfill}{rgb}{0.631373,0.788235,0.956863}%
\pgfsetfillcolor{currentfill}%
\pgfsetlinewidth{0.481800pt}%
\definecolor{currentstroke}{rgb}{1.000000,1.000000,1.000000}%
\pgfsetstrokecolor{currentstroke}%
\pgfsetdash{}{0pt}%
\pgfpathmoveto{\pgfqpoint{2.082714in}{3.013608in}}%
\pgfpathcurveto{\pgfqpoint{2.093764in}{3.013608in}}{\pgfqpoint{2.104363in}{3.017999in}}{\pgfqpoint{2.112177in}{3.025812in}}%
\pgfpathcurveto{\pgfqpoint{2.119990in}{3.033626in}}{\pgfqpoint{2.124380in}{3.044225in}}{\pgfqpoint{2.124380in}{3.055275in}}%
\pgfpathcurveto{\pgfqpoint{2.124380in}{3.066325in}}{\pgfqpoint{2.119990in}{3.076924in}}{\pgfqpoint{2.112177in}{3.084738in}}%
\pgfpathcurveto{\pgfqpoint{2.104363in}{3.092551in}}{\pgfqpoint{2.093764in}{3.096942in}}{\pgfqpoint{2.082714in}{3.096942in}}%
\pgfpathcurveto{\pgfqpoint{2.071664in}{3.096942in}}{\pgfqpoint{2.061065in}{3.092551in}}{\pgfqpoint{2.053251in}{3.084738in}}%
\pgfpathcurveto{\pgfqpoint{2.045437in}{3.076924in}}{\pgfqpoint{2.041047in}{3.066325in}}{\pgfqpoint{2.041047in}{3.055275in}}%
\pgfpathcurveto{\pgfqpoint{2.041047in}{3.044225in}}{\pgfqpoint{2.045437in}{3.033626in}}{\pgfqpoint{2.053251in}{3.025812in}}%
\pgfpathcurveto{\pgfqpoint{2.061065in}{3.017999in}}{\pgfqpoint{2.071664in}{3.013608in}}{\pgfqpoint{2.082714in}{3.013608in}}%
\pgfpathclose%
\pgfusepath{stroke,fill}%
\end{pgfscope}%
\begin{pgfscope}%
\pgfpathrectangle{\pgfqpoint{0.570343in}{0.331635in}}{\pgfqpoint{9.300000in}{7.700000in}}%
\pgfusepath{clip}%
\pgfsetbuttcap%
\pgfsetroundjoin%
\definecolor{currentfill}{rgb}{0.631373,0.788235,0.956863}%
\pgfsetfillcolor{currentfill}%
\pgfsetlinewidth{0.481800pt}%
\definecolor{currentstroke}{rgb}{1.000000,1.000000,1.000000}%
\pgfsetstrokecolor{currentstroke}%
\pgfsetdash{}{0pt}%
\pgfpathmoveto{\pgfqpoint{2.029248in}{1.536456in}}%
\pgfpathcurveto{\pgfqpoint{2.040298in}{1.536456in}}{\pgfqpoint{2.050897in}{1.540846in}}{\pgfqpoint{2.058711in}{1.548660in}}%
\pgfpathcurveto{\pgfqpoint{2.066524in}{1.556473in}}{\pgfqpoint{2.070915in}{1.567072in}}{\pgfqpoint{2.070915in}{1.578122in}}%
\pgfpathcurveto{\pgfqpoint{2.070915in}{1.589172in}}{\pgfqpoint{2.066524in}{1.599772in}}{\pgfqpoint{2.058711in}{1.607585in}}%
\pgfpathcurveto{\pgfqpoint{2.050897in}{1.615399in}}{\pgfqpoint{2.040298in}{1.619789in}}{\pgfqpoint{2.029248in}{1.619789in}}%
\pgfpathcurveto{\pgfqpoint{2.018198in}{1.619789in}}{\pgfqpoint{2.007599in}{1.615399in}}{\pgfqpoint{1.999785in}{1.607585in}}%
\pgfpathcurveto{\pgfqpoint{1.991972in}{1.599772in}}{\pgfqpoint{1.987581in}{1.589172in}}{\pgfqpoint{1.987581in}{1.578122in}}%
\pgfpathcurveto{\pgfqpoint{1.987581in}{1.567072in}}{\pgfqpoint{1.991972in}{1.556473in}}{\pgfqpoint{1.999785in}{1.548660in}}%
\pgfpathcurveto{\pgfqpoint{2.007599in}{1.540846in}}{\pgfqpoint{2.018198in}{1.536456in}}{\pgfqpoint{2.029248in}{1.536456in}}%
\pgfpathclose%
\pgfusepath{stroke,fill}%
\end{pgfscope}%
\begin{pgfscope}%
\pgfpathrectangle{\pgfqpoint{0.570343in}{0.331635in}}{\pgfqpoint{9.300000in}{7.700000in}}%
\pgfusepath{clip}%
\pgfsetbuttcap%
\pgfsetroundjoin%
\definecolor{currentfill}{rgb}{0.631373,0.788235,0.956863}%
\pgfsetfillcolor{currentfill}%
\pgfsetlinewidth{0.481800pt}%
\definecolor{currentstroke}{rgb}{1.000000,1.000000,1.000000}%
\pgfsetstrokecolor{currentstroke}%
\pgfsetdash{}{0pt}%
\pgfpathmoveto{\pgfqpoint{5.175896in}{3.975564in}}%
\pgfpathcurveto{\pgfqpoint{5.186946in}{3.975564in}}{\pgfqpoint{5.197545in}{3.979954in}}{\pgfqpoint{5.205358in}{3.987768in}}%
\pgfpathcurveto{\pgfqpoint{5.213172in}{3.995581in}}{\pgfqpoint{5.217562in}{4.006181in}}{\pgfqpoint{5.217562in}{4.017231in}}%
\pgfpathcurveto{\pgfqpoint{5.217562in}{4.028281in}}{\pgfqpoint{5.213172in}{4.038880in}}{\pgfqpoint{5.205358in}{4.046693in}}%
\pgfpathcurveto{\pgfqpoint{5.197545in}{4.054507in}}{\pgfqpoint{5.186946in}{4.058897in}}{\pgfqpoint{5.175896in}{4.058897in}}%
\pgfpathcurveto{\pgfqpoint{5.164845in}{4.058897in}}{\pgfqpoint{5.154246in}{4.054507in}}{\pgfqpoint{5.146433in}{4.046693in}}%
\pgfpathcurveto{\pgfqpoint{5.138619in}{4.038880in}}{\pgfqpoint{5.134229in}{4.028281in}}{\pgfqpoint{5.134229in}{4.017231in}}%
\pgfpathcurveto{\pgfqpoint{5.134229in}{4.006181in}}{\pgfqpoint{5.138619in}{3.995581in}}{\pgfqpoint{5.146433in}{3.987768in}}%
\pgfpathcurveto{\pgfqpoint{5.154246in}{3.979954in}}{\pgfqpoint{5.164845in}{3.975564in}}{\pgfqpoint{5.175896in}{3.975564in}}%
\pgfpathclose%
\pgfusepath{stroke,fill}%
\end{pgfscope}%
\begin{pgfscope}%
\pgfpathrectangle{\pgfqpoint{0.570343in}{0.331635in}}{\pgfqpoint{9.300000in}{7.700000in}}%
\pgfusepath{clip}%
\pgfsetbuttcap%
\pgfsetroundjoin%
\definecolor{currentfill}{rgb}{0.631373,0.788235,0.956863}%
\pgfsetfillcolor{currentfill}%
\pgfsetlinewidth{0.481800pt}%
\definecolor{currentstroke}{rgb}{1.000000,1.000000,1.000000}%
\pgfsetstrokecolor{currentstroke}%
\pgfsetdash{}{0pt}%
\pgfpathmoveto{\pgfqpoint{3.674629in}{5.500123in}}%
\pgfpathcurveto{\pgfqpoint{3.685679in}{5.500123in}}{\pgfqpoint{3.696278in}{5.504514in}}{\pgfqpoint{3.704092in}{5.512327in}}%
\pgfpathcurveto{\pgfqpoint{3.711905in}{5.520141in}}{\pgfqpoint{3.716296in}{5.530740in}}{\pgfqpoint{3.716296in}{5.541790in}}%
\pgfpathcurveto{\pgfqpoint{3.716296in}{5.552840in}}{\pgfqpoint{3.711905in}{5.563439in}}{\pgfqpoint{3.704092in}{5.571253in}}%
\pgfpathcurveto{\pgfqpoint{3.696278in}{5.579067in}}{\pgfqpoint{3.685679in}{5.583457in}}{\pgfqpoint{3.674629in}{5.583457in}}%
\pgfpathcurveto{\pgfqpoint{3.663579in}{5.583457in}}{\pgfqpoint{3.652980in}{5.579067in}}{\pgfqpoint{3.645166in}{5.571253in}}%
\pgfpathcurveto{\pgfqpoint{3.637353in}{5.563439in}}{\pgfqpoint{3.632962in}{5.552840in}}{\pgfqpoint{3.632962in}{5.541790in}}%
\pgfpathcurveto{\pgfqpoint{3.632962in}{5.530740in}}{\pgfqpoint{3.637353in}{5.520141in}}{\pgfqpoint{3.645166in}{5.512327in}}%
\pgfpathcurveto{\pgfqpoint{3.652980in}{5.504514in}}{\pgfqpoint{3.663579in}{5.500123in}}{\pgfqpoint{3.674629in}{5.500123in}}%
\pgfpathclose%
\pgfusepath{stroke,fill}%
\end{pgfscope}%
\begin{pgfscope}%
\pgfpathrectangle{\pgfqpoint{0.570343in}{0.331635in}}{\pgfqpoint{9.300000in}{7.700000in}}%
\pgfusepath{clip}%
\pgfsetbuttcap%
\pgfsetroundjoin%
\definecolor{currentfill}{rgb}{0.631373,0.788235,0.956863}%
\pgfsetfillcolor{currentfill}%
\pgfsetlinewidth{0.481800pt}%
\definecolor{currentstroke}{rgb}{1.000000,1.000000,1.000000}%
\pgfsetstrokecolor{currentstroke}%
\pgfsetdash{}{0pt}%
\pgfpathmoveto{\pgfqpoint{1.383929in}{4.998171in}}%
\pgfpathcurveto{\pgfqpoint{1.394980in}{4.998171in}}{\pgfqpoint{1.405579in}{5.002561in}}{\pgfqpoint{1.413392in}{5.010375in}}%
\pgfpathcurveto{\pgfqpoint{1.421206in}{5.018188in}}{\pgfqpoint{1.425596in}{5.028787in}}{\pgfqpoint{1.425596in}{5.039838in}}%
\pgfpathcurveto{\pgfqpoint{1.425596in}{5.050888in}}{\pgfqpoint{1.421206in}{5.061487in}}{\pgfqpoint{1.413392in}{5.069300in}}%
\pgfpathcurveto{\pgfqpoint{1.405579in}{5.077114in}}{\pgfqpoint{1.394980in}{5.081504in}}{\pgfqpoint{1.383929in}{5.081504in}}%
\pgfpathcurveto{\pgfqpoint{1.372879in}{5.081504in}}{\pgfqpoint{1.362280in}{5.077114in}}{\pgfqpoint{1.354467in}{5.069300in}}%
\pgfpathcurveto{\pgfqpoint{1.346653in}{5.061487in}}{\pgfqpoint{1.342263in}{5.050888in}}{\pgfqpoint{1.342263in}{5.039838in}}%
\pgfpathcurveto{\pgfqpoint{1.342263in}{5.028787in}}{\pgfqpoint{1.346653in}{5.018188in}}{\pgfqpoint{1.354467in}{5.010375in}}%
\pgfpathcurveto{\pgfqpoint{1.362280in}{5.002561in}}{\pgfqpoint{1.372879in}{4.998171in}}{\pgfqpoint{1.383929in}{4.998171in}}%
\pgfpathclose%
\pgfusepath{stroke,fill}%
\end{pgfscope}%
\begin{pgfscope}%
\pgfpathrectangle{\pgfqpoint{0.570343in}{0.331635in}}{\pgfqpoint{9.300000in}{7.700000in}}%
\pgfusepath{clip}%
\pgfsetbuttcap%
\pgfsetroundjoin%
\definecolor{currentfill}{rgb}{0.631373,0.788235,0.956863}%
\pgfsetfillcolor{currentfill}%
\pgfsetlinewidth{0.481800pt}%
\definecolor{currentstroke}{rgb}{1.000000,1.000000,1.000000}%
\pgfsetstrokecolor{currentstroke}%
\pgfsetdash{}{0pt}%
\pgfpathmoveto{\pgfqpoint{3.677550in}{4.732443in}}%
\pgfpathcurveto{\pgfqpoint{3.688600in}{4.732443in}}{\pgfqpoint{3.699199in}{4.736833in}}{\pgfqpoint{3.707013in}{4.744646in}}%
\pgfpathcurveto{\pgfqpoint{3.714826in}{4.752460in}}{\pgfqpoint{3.719216in}{4.763059in}}{\pgfqpoint{3.719216in}{4.774109in}}%
\pgfpathcurveto{\pgfqpoint{3.719216in}{4.785159in}}{\pgfqpoint{3.714826in}{4.795758in}}{\pgfqpoint{3.707013in}{4.803572in}}%
\pgfpathcurveto{\pgfqpoint{3.699199in}{4.811386in}}{\pgfqpoint{3.688600in}{4.815776in}}{\pgfqpoint{3.677550in}{4.815776in}}%
\pgfpathcurveto{\pgfqpoint{3.666500in}{4.815776in}}{\pgfqpoint{3.655901in}{4.811386in}}{\pgfqpoint{3.648087in}{4.803572in}}%
\pgfpathcurveto{\pgfqpoint{3.640273in}{4.795758in}}{\pgfqpoint{3.635883in}{4.785159in}}{\pgfqpoint{3.635883in}{4.774109in}}%
\pgfpathcurveto{\pgfqpoint{3.635883in}{4.763059in}}{\pgfqpoint{3.640273in}{4.752460in}}{\pgfqpoint{3.648087in}{4.744646in}}%
\pgfpathcurveto{\pgfqpoint{3.655901in}{4.736833in}}{\pgfqpoint{3.666500in}{4.732443in}}{\pgfqpoint{3.677550in}{4.732443in}}%
\pgfpathclose%
\pgfusepath{stroke,fill}%
\end{pgfscope}%
\begin{pgfscope}%
\pgfpathrectangle{\pgfqpoint{0.570343in}{0.331635in}}{\pgfqpoint{9.300000in}{7.700000in}}%
\pgfusepath{clip}%
\pgfsetbuttcap%
\pgfsetroundjoin%
\definecolor{currentfill}{rgb}{0.631373,0.788235,0.956863}%
\pgfsetfillcolor{currentfill}%
\pgfsetlinewidth{0.481800pt}%
\definecolor{currentstroke}{rgb}{1.000000,1.000000,1.000000}%
\pgfsetstrokecolor{currentstroke}%
\pgfsetdash{}{0pt}%
\pgfpathmoveto{\pgfqpoint{6.563827in}{3.604344in}}%
\pgfpathcurveto{\pgfqpoint{6.574877in}{3.604344in}}{\pgfqpoint{6.585476in}{3.608735in}}{\pgfqpoint{6.593290in}{3.616548in}}%
\pgfpathcurveto{\pgfqpoint{6.601103in}{3.624362in}}{\pgfqpoint{6.605493in}{3.634961in}}{\pgfqpoint{6.605493in}{3.646011in}}%
\pgfpathcurveto{\pgfqpoint{6.605493in}{3.657061in}}{\pgfqpoint{6.601103in}{3.667660in}}{\pgfqpoint{6.593290in}{3.675474in}}%
\pgfpathcurveto{\pgfqpoint{6.585476in}{3.683287in}}{\pgfqpoint{6.574877in}{3.687678in}}{\pgfqpoint{6.563827in}{3.687678in}}%
\pgfpathcurveto{\pgfqpoint{6.552777in}{3.687678in}}{\pgfqpoint{6.542178in}{3.683287in}}{\pgfqpoint{6.534364in}{3.675474in}}%
\pgfpathcurveto{\pgfqpoint{6.526550in}{3.667660in}}{\pgfqpoint{6.522160in}{3.657061in}}{\pgfqpoint{6.522160in}{3.646011in}}%
\pgfpathcurveto{\pgfqpoint{6.522160in}{3.634961in}}{\pgfqpoint{6.526550in}{3.624362in}}{\pgfqpoint{6.534364in}{3.616548in}}%
\pgfpathcurveto{\pgfqpoint{6.542178in}{3.608735in}}{\pgfqpoint{6.552777in}{3.604344in}}{\pgfqpoint{6.563827in}{3.604344in}}%
\pgfpathclose%
\pgfusepath{stroke,fill}%
\end{pgfscope}%
\begin{pgfscope}%
\pgfpathrectangle{\pgfqpoint{0.570343in}{0.331635in}}{\pgfqpoint{9.300000in}{7.700000in}}%
\pgfusepath{clip}%
\pgfsetbuttcap%
\pgfsetroundjoin%
\definecolor{currentfill}{rgb}{0.631373,0.788235,0.956863}%
\pgfsetfillcolor{currentfill}%
\pgfsetlinewidth{0.481800pt}%
\definecolor{currentstroke}{rgb}{1.000000,1.000000,1.000000}%
\pgfsetstrokecolor{currentstroke}%
\pgfsetdash{}{0pt}%
\pgfpathmoveto{\pgfqpoint{4.968812in}{5.867491in}}%
\pgfpathcurveto{\pgfqpoint{4.979862in}{5.867491in}}{\pgfqpoint{4.990461in}{5.871882in}}{\pgfqpoint{4.998275in}{5.879695in}}%
\pgfpathcurveto{\pgfqpoint{5.006088in}{5.887509in}}{\pgfqpoint{5.010479in}{5.898108in}}{\pgfqpoint{5.010479in}{5.909158in}}%
\pgfpathcurveto{\pgfqpoint{5.010479in}{5.920208in}}{\pgfqpoint{5.006088in}{5.930807in}}{\pgfqpoint{4.998275in}{5.938621in}}%
\pgfpathcurveto{\pgfqpoint{4.990461in}{5.946434in}}{\pgfqpoint{4.979862in}{5.950825in}}{\pgfqpoint{4.968812in}{5.950825in}}%
\pgfpathcurveto{\pgfqpoint{4.957762in}{5.950825in}}{\pgfqpoint{4.947163in}{5.946434in}}{\pgfqpoint{4.939349in}{5.938621in}}%
\pgfpathcurveto{\pgfqpoint{4.931535in}{5.930807in}}{\pgfqpoint{4.927145in}{5.920208in}}{\pgfqpoint{4.927145in}{5.909158in}}%
\pgfpathcurveto{\pgfqpoint{4.927145in}{5.898108in}}{\pgfqpoint{4.931535in}{5.887509in}}{\pgfqpoint{4.939349in}{5.879695in}}%
\pgfpathcurveto{\pgfqpoint{4.947163in}{5.871882in}}{\pgfqpoint{4.957762in}{5.867491in}}{\pgfqpoint{4.968812in}{5.867491in}}%
\pgfpathclose%
\pgfusepath{stroke,fill}%
\end{pgfscope}%
\begin{pgfscope}%
\pgfpathrectangle{\pgfqpoint{0.570343in}{0.331635in}}{\pgfqpoint{9.300000in}{7.700000in}}%
\pgfusepath{clip}%
\pgfsetbuttcap%
\pgfsetroundjoin%
\definecolor{currentfill}{rgb}{0.631373,0.788235,0.956863}%
\pgfsetfillcolor{currentfill}%
\pgfsetlinewidth{0.481800pt}%
\definecolor{currentstroke}{rgb}{1.000000,1.000000,1.000000}%
\pgfsetstrokecolor{currentstroke}%
\pgfsetdash{}{0pt}%
\pgfpathmoveto{\pgfqpoint{5.391700in}{1.625125in}}%
\pgfpathcurveto{\pgfqpoint{5.402750in}{1.625125in}}{\pgfqpoint{5.413349in}{1.629515in}}{\pgfqpoint{5.421162in}{1.637329in}}%
\pgfpathcurveto{\pgfqpoint{5.428976in}{1.645143in}}{\pgfqpoint{5.433366in}{1.655742in}}{\pgfqpoint{5.433366in}{1.666792in}}%
\pgfpathcurveto{\pgfqpoint{5.433366in}{1.677842in}}{\pgfqpoint{5.428976in}{1.688441in}}{\pgfqpoint{5.421162in}{1.696254in}}%
\pgfpathcurveto{\pgfqpoint{5.413349in}{1.704068in}}{\pgfqpoint{5.402750in}{1.708458in}}{\pgfqpoint{5.391700in}{1.708458in}}%
\pgfpathcurveto{\pgfqpoint{5.380650in}{1.708458in}}{\pgfqpoint{5.370051in}{1.704068in}}{\pgfqpoint{5.362237in}{1.696254in}}%
\pgfpathcurveto{\pgfqpoint{5.354423in}{1.688441in}}{\pgfqpoint{5.350033in}{1.677842in}}{\pgfqpoint{5.350033in}{1.666792in}}%
\pgfpathcurveto{\pgfqpoint{5.350033in}{1.655742in}}{\pgfqpoint{5.354423in}{1.645143in}}{\pgfqpoint{5.362237in}{1.637329in}}%
\pgfpathcurveto{\pgfqpoint{5.370051in}{1.629515in}}{\pgfqpoint{5.380650in}{1.625125in}}{\pgfqpoint{5.391700in}{1.625125in}}%
\pgfpathclose%
\pgfusepath{stroke,fill}%
\end{pgfscope}%
\begin{pgfscope}%
\pgfpathrectangle{\pgfqpoint{0.570343in}{0.331635in}}{\pgfqpoint{9.300000in}{7.700000in}}%
\pgfusepath{clip}%
\pgfsetbuttcap%
\pgfsetroundjoin%
\definecolor{currentfill}{rgb}{0.631373,0.788235,0.956863}%
\pgfsetfillcolor{currentfill}%
\pgfsetlinewidth{0.481800pt}%
\definecolor{currentstroke}{rgb}{1.000000,1.000000,1.000000}%
\pgfsetstrokecolor{currentstroke}%
\pgfsetdash{}{0pt}%
\pgfpathmoveto{\pgfqpoint{7.110525in}{6.170454in}}%
\pgfpathcurveto{\pgfqpoint{7.121575in}{6.170454in}}{\pgfqpoint{7.132174in}{6.174844in}}{\pgfqpoint{7.139988in}{6.182658in}}%
\pgfpathcurveto{\pgfqpoint{7.147801in}{6.190471in}}{\pgfqpoint{7.152192in}{6.201071in}}{\pgfqpoint{7.152192in}{6.212121in}}%
\pgfpathcurveto{\pgfqpoint{7.152192in}{6.223171in}}{\pgfqpoint{7.147801in}{6.233770in}}{\pgfqpoint{7.139988in}{6.241583in}}%
\pgfpathcurveto{\pgfqpoint{7.132174in}{6.249397in}}{\pgfqpoint{7.121575in}{6.253787in}}{\pgfqpoint{7.110525in}{6.253787in}}%
\pgfpathcurveto{\pgfqpoint{7.099475in}{6.253787in}}{\pgfqpoint{7.088876in}{6.249397in}}{\pgfqpoint{7.081062in}{6.241583in}}%
\pgfpathcurveto{\pgfqpoint{7.073249in}{6.233770in}}{\pgfqpoint{7.068858in}{6.223171in}}{\pgfqpoint{7.068858in}{6.212121in}}%
\pgfpathcurveto{\pgfqpoint{7.068858in}{6.201071in}}{\pgfqpoint{7.073249in}{6.190471in}}{\pgfqpoint{7.081062in}{6.182658in}}%
\pgfpathcurveto{\pgfqpoint{7.088876in}{6.174844in}}{\pgfqpoint{7.099475in}{6.170454in}}{\pgfqpoint{7.110525in}{6.170454in}}%
\pgfpathclose%
\pgfusepath{stroke,fill}%
\end{pgfscope}%
\begin{pgfscope}%
\pgfpathrectangle{\pgfqpoint{0.570343in}{0.331635in}}{\pgfqpoint{9.300000in}{7.700000in}}%
\pgfusepath{clip}%
\pgfsetbuttcap%
\pgfsetroundjoin%
\definecolor{currentfill}{rgb}{0.631373,0.788235,0.956863}%
\pgfsetfillcolor{currentfill}%
\pgfsetlinewidth{0.481800pt}%
\definecolor{currentstroke}{rgb}{1.000000,1.000000,1.000000}%
\pgfsetstrokecolor{currentstroke}%
\pgfsetdash{}{0pt}%
\pgfpathmoveto{\pgfqpoint{3.005187in}{2.523502in}}%
\pgfpathcurveto{\pgfqpoint{3.016237in}{2.523502in}}{\pgfqpoint{3.026836in}{2.527892in}}{\pgfqpoint{3.034650in}{2.535706in}}%
\pgfpathcurveto{\pgfqpoint{3.042463in}{2.543520in}}{\pgfqpoint{3.046854in}{2.554119in}}{\pgfqpoint{3.046854in}{2.565169in}}%
\pgfpathcurveto{\pgfqpoint{3.046854in}{2.576219in}}{\pgfqpoint{3.042463in}{2.586818in}}{\pgfqpoint{3.034650in}{2.594632in}}%
\pgfpathcurveto{\pgfqpoint{3.026836in}{2.602445in}}{\pgfqpoint{3.016237in}{2.606835in}}{\pgfqpoint{3.005187in}{2.606835in}}%
\pgfpathcurveto{\pgfqpoint{2.994137in}{2.606835in}}{\pgfqpoint{2.983538in}{2.602445in}}{\pgfqpoint{2.975724in}{2.594632in}}%
\pgfpathcurveto{\pgfqpoint{2.967911in}{2.586818in}}{\pgfqpoint{2.963520in}{2.576219in}}{\pgfqpoint{2.963520in}{2.565169in}}%
\pgfpathcurveto{\pgfqpoint{2.963520in}{2.554119in}}{\pgfqpoint{2.967911in}{2.543520in}}{\pgfqpoint{2.975724in}{2.535706in}}%
\pgfpathcurveto{\pgfqpoint{2.983538in}{2.527892in}}{\pgfqpoint{2.994137in}{2.523502in}}{\pgfqpoint{3.005187in}{2.523502in}}%
\pgfpathclose%
\pgfusepath{stroke,fill}%
\end{pgfscope}%
\begin{pgfscope}%
\pgfpathrectangle{\pgfqpoint{0.570343in}{0.331635in}}{\pgfqpoint{9.300000in}{7.700000in}}%
\pgfusepath{clip}%
\pgfsetbuttcap%
\pgfsetroundjoin%
\definecolor{currentfill}{rgb}{0.631373,0.788235,0.956863}%
\pgfsetfillcolor{currentfill}%
\pgfsetlinewidth{0.481800pt}%
\definecolor{currentstroke}{rgb}{1.000000,1.000000,1.000000}%
\pgfsetstrokecolor{currentstroke}%
\pgfsetdash{}{0pt}%
\pgfpathmoveto{\pgfqpoint{4.481129in}{4.356585in}}%
\pgfpathcurveto{\pgfqpoint{4.492179in}{4.356585in}}{\pgfqpoint{4.502778in}{4.360975in}}{\pgfqpoint{4.510592in}{4.368789in}}%
\pgfpathcurveto{\pgfqpoint{4.518405in}{4.376603in}}{\pgfqpoint{4.522796in}{4.387202in}}{\pgfqpoint{4.522796in}{4.398252in}}%
\pgfpathcurveto{\pgfqpoint{4.522796in}{4.409302in}}{\pgfqpoint{4.518405in}{4.419901in}}{\pgfqpoint{4.510592in}{4.427715in}}%
\pgfpathcurveto{\pgfqpoint{4.502778in}{4.435528in}}{\pgfqpoint{4.492179in}{4.439919in}}{\pgfqpoint{4.481129in}{4.439919in}}%
\pgfpathcurveto{\pgfqpoint{4.470079in}{4.439919in}}{\pgfqpoint{4.459480in}{4.435528in}}{\pgfqpoint{4.451666in}{4.427715in}}%
\pgfpathcurveto{\pgfqpoint{4.443853in}{4.419901in}}{\pgfqpoint{4.439462in}{4.409302in}}{\pgfqpoint{4.439462in}{4.398252in}}%
\pgfpathcurveto{\pgfqpoint{4.439462in}{4.387202in}}{\pgfqpoint{4.443853in}{4.376603in}}{\pgfqpoint{4.451666in}{4.368789in}}%
\pgfpathcurveto{\pgfqpoint{4.459480in}{4.360975in}}{\pgfqpoint{4.470079in}{4.356585in}}{\pgfqpoint{4.481129in}{4.356585in}}%
\pgfpathclose%
\pgfusepath{stroke,fill}%
\end{pgfscope}%
\begin{pgfscope}%
\pgfpathrectangle{\pgfqpoint{0.570343in}{0.331635in}}{\pgfqpoint{9.300000in}{7.700000in}}%
\pgfusepath{clip}%
\pgfsetbuttcap%
\pgfsetroundjoin%
\definecolor{currentfill}{rgb}{0.631373,0.788235,0.956863}%
\pgfsetfillcolor{currentfill}%
\pgfsetlinewidth{0.481800pt}%
\definecolor{currentstroke}{rgb}{1.000000,1.000000,1.000000}%
\pgfsetstrokecolor{currentstroke}%
\pgfsetdash{}{0pt}%
\pgfpathmoveto{\pgfqpoint{6.045238in}{6.374610in}}%
\pgfpathcurveto{\pgfqpoint{6.056288in}{6.374610in}}{\pgfqpoint{6.066887in}{6.379000in}}{\pgfqpoint{6.074700in}{6.386814in}}%
\pgfpathcurveto{\pgfqpoint{6.082514in}{6.394628in}}{\pgfqpoint{6.086904in}{6.405227in}}{\pgfqpoint{6.086904in}{6.416277in}}%
\pgfpathcurveto{\pgfqpoint{6.086904in}{6.427327in}}{\pgfqpoint{6.082514in}{6.437926in}}{\pgfqpoint{6.074700in}{6.445740in}}%
\pgfpathcurveto{\pgfqpoint{6.066887in}{6.453553in}}{\pgfqpoint{6.056288in}{6.457943in}}{\pgfqpoint{6.045238in}{6.457943in}}%
\pgfpathcurveto{\pgfqpoint{6.034188in}{6.457943in}}{\pgfqpoint{6.023588in}{6.453553in}}{\pgfqpoint{6.015775in}{6.445740in}}%
\pgfpathcurveto{\pgfqpoint{6.007961in}{6.437926in}}{\pgfqpoint{6.003571in}{6.427327in}}{\pgfqpoint{6.003571in}{6.416277in}}%
\pgfpathcurveto{\pgfqpoint{6.003571in}{6.405227in}}{\pgfqpoint{6.007961in}{6.394628in}}{\pgfqpoint{6.015775in}{6.386814in}}%
\pgfpathcurveto{\pgfqpoint{6.023588in}{6.379000in}}{\pgfqpoint{6.034188in}{6.374610in}}{\pgfqpoint{6.045238in}{6.374610in}}%
\pgfpathclose%
\pgfusepath{stroke,fill}%
\end{pgfscope}%
\begin{pgfscope}%
\pgfpathrectangle{\pgfqpoint{0.570343in}{0.331635in}}{\pgfqpoint{9.300000in}{7.700000in}}%
\pgfusepath{clip}%
\pgfsetbuttcap%
\pgfsetroundjoin%
\definecolor{currentfill}{rgb}{0.631373,0.788235,0.956863}%
\pgfsetfillcolor{currentfill}%
\pgfsetlinewidth{0.481800pt}%
\definecolor{currentstroke}{rgb}{1.000000,1.000000,1.000000}%
\pgfsetstrokecolor{currentstroke}%
\pgfsetdash{}{0pt}%
\pgfpathmoveto{\pgfqpoint{9.187938in}{4.053650in}}%
\pgfpathcurveto{\pgfqpoint{9.198989in}{4.053650in}}{\pgfqpoint{9.209588in}{4.058040in}}{\pgfqpoint{9.217401in}{4.065854in}}%
\pgfpathcurveto{\pgfqpoint{9.225215in}{4.073668in}}{\pgfqpoint{9.229605in}{4.084267in}}{\pgfqpoint{9.229605in}{4.095317in}}%
\pgfpathcurveto{\pgfqpoint{9.229605in}{4.106367in}}{\pgfqpoint{9.225215in}{4.116966in}}{\pgfqpoint{9.217401in}{4.124780in}}%
\pgfpathcurveto{\pgfqpoint{9.209588in}{4.132593in}}{\pgfqpoint{9.198989in}{4.136984in}}{\pgfqpoint{9.187938in}{4.136984in}}%
\pgfpathcurveto{\pgfqpoint{9.176888in}{4.136984in}}{\pgfqpoint{9.166289in}{4.132593in}}{\pgfqpoint{9.158476in}{4.124780in}}%
\pgfpathcurveto{\pgfqpoint{9.150662in}{4.116966in}}{\pgfqpoint{9.146272in}{4.106367in}}{\pgfqpoint{9.146272in}{4.095317in}}%
\pgfpathcurveto{\pgfqpoint{9.146272in}{4.084267in}}{\pgfqpoint{9.150662in}{4.073668in}}{\pgfqpoint{9.158476in}{4.065854in}}%
\pgfpathcurveto{\pgfqpoint{9.166289in}{4.058040in}}{\pgfqpoint{9.176888in}{4.053650in}}{\pgfqpoint{9.187938in}{4.053650in}}%
\pgfpathclose%
\pgfusepath{stroke,fill}%
\end{pgfscope}%
\begin{pgfscope}%
\pgfpathrectangle{\pgfqpoint{0.570343in}{0.331635in}}{\pgfqpoint{9.300000in}{7.700000in}}%
\pgfusepath{clip}%
\pgfsetbuttcap%
\pgfsetroundjoin%
\definecolor{currentfill}{rgb}{0.631373,0.788235,0.956863}%
\pgfsetfillcolor{currentfill}%
\pgfsetlinewidth{0.481800pt}%
\definecolor{currentstroke}{rgb}{1.000000,1.000000,1.000000}%
\pgfsetstrokecolor{currentstroke}%
\pgfsetdash{}{0pt}%
\pgfpathmoveto{\pgfqpoint{7.749803in}{4.456360in}}%
\pgfpathcurveto{\pgfqpoint{7.760853in}{4.456360in}}{\pgfqpoint{7.771452in}{4.460750in}}{\pgfqpoint{7.779265in}{4.468563in}}%
\pgfpathcurveto{\pgfqpoint{7.787079in}{4.476377in}}{\pgfqpoint{7.791469in}{4.486976in}}{\pgfqpoint{7.791469in}{4.498026in}}%
\pgfpathcurveto{\pgfqpoint{7.791469in}{4.509076in}}{\pgfqpoint{7.787079in}{4.519675in}}{\pgfqpoint{7.779265in}{4.527489in}}%
\pgfpathcurveto{\pgfqpoint{7.771452in}{4.535303in}}{\pgfqpoint{7.760853in}{4.539693in}}{\pgfqpoint{7.749803in}{4.539693in}}%
\pgfpathcurveto{\pgfqpoint{7.738753in}{4.539693in}}{\pgfqpoint{7.728154in}{4.535303in}}{\pgfqpoint{7.720340in}{4.527489in}}%
\pgfpathcurveto{\pgfqpoint{7.712526in}{4.519675in}}{\pgfqpoint{7.708136in}{4.509076in}}{\pgfqpoint{7.708136in}{4.498026in}}%
\pgfpathcurveto{\pgfqpoint{7.708136in}{4.486976in}}{\pgfqpoint{7.712526in}{4.476377in}}{\pgfqpoint{7.720340in}{4.468563in}}%
\pgfpathcurveto{\pgfqpoint{7.728154in}{4.460750in}}{\pgfqpoint{7.738753in}{4.456360in}}{\pgfqpoint{7.749803in}{4.456360in}}%
\pgfpathclose%
\pgfusepath{stroke,fill}%
\end{pgfscope}%
\begin{pgfscope}%
\pgfpathrectangle{\pgfqpoint{0.570343in}{0.331635in}}{\pgfqpoint{9.300000in}{7.700000in}}%
\pgfusepath{clip}%
\pgfsetbuttcap%
\pgfsetroundjoin%
\definecolor{currentfill}{rgb}{1.000000,0.705882,0.509804}%
\pgfsetfillcolor{currentfill}%
\pgfsetlinewidth{0.481800pt}%
\definecolor{currentstroke}{rgb}{1.000000,1.000000,1.000000}%
\pgfsetstrokecolor{currentstroke}%
\pgfsetdash{}{0pt}%
\pgfpathmoveto{\pgfqpoint{5.715735in}{3.294039in}}%
\pgfpathcurveto{\pgfqpoint{5.726785in}{3.294039in}}{\pgfqpoint{5.737384in}{3.298429in}}{\pgfqpoint{5.745198in}{3.306243in}}%
\pgfpathcurveto{\pgfqpoint{5.753012in}{3.314057in}}{\pgfqpoint{5.757402in}{3.324656in}}{\pgfqpoint{5.757402in}{3.335706in}}%
\pgfpathcurveto{\pgfqpoint{5.757402in}{3.346756in}}{\pgfqpoint{5.753012in}{3.357355in}}{\pgfqpoint{5.745198in}{3.365169in}}%
\pgfpathcurveto{\pgfqpoint{5.737384in}{3.372982in}}{\pgfqpoint{5.726785in}{3.377372in}}{\pgfqpoint{5.715735in}{3.377372in}}%
\pgfpathcurveto{\pgfqpoint{5.704685in}{3.377372in}}{\pgfqpoint{5.694086in}{3.372982in}}{\pgfqpoint{5.686272in}{3.365169in}}%
\pgfpathcurveto{\pgfqpoint{5.678459in}{3.357355in}}{\pgfqpoint{5.674069in}{3.346756in}}{\pgfqpoint{5.674069in}{3.335706in}}%
\pgfpathcurveto{\pgfqpoint{5.674069in}{3.324656in}}{\pgfqpoint{5.678459in}{3.314057in}}{\pgfqpoint{5.686272in}{3.306243in}}%
\pgfpathcurveto{\pgfqpoint{5.694086in}{3.298429in}}{\pgfqpoint{5.704685in}{3.294039in}}{\pgfqpoint{5.715735in}{3.294039in}}%
\pgfpathclose%
\pgfusepath{stroke,fill}%
\end{pgfscope}%
\begin{pgfscope}%
\pgfpathrectangle{\pgfqpoint{0.570343in}{0.331635in}}{\pgfqpoint{9.300000in}{7.700000in}}%
\pgfusepath{clip}%
\pgfsetbuttcap%
\pgfsetroundjoin%
\definecolor{currentfill}{rgb}{1.000000,0.705882,0.509804}%
\pgfsetfillcolor{currentfill}%
\pgfsetlinewidth{0.481800pt}%
\definecolor{currentstroke}{rgb}{1.000000,1.000000,1.000000}%
\pgfsetstrokecolor{currentstroke}%
\pgfsetdash{}{0pt}%
\pgfpathmoveto{\pgfqpoint{7.824153in}{5.921475in}}%
\pgfpathcurveto{\pgfqpoint{7.835204in}{5.921475in}}{\pgfqpoint{7.845803in}{5.925865in}}{\pgfqpoint{7.853616in}{5.933679in}}%
\pgfpathcurveto{\pgfqpoint{7.861430in}{5.941492in}}{\pgfqpoint{7.865820in}{5.952091in}}{\pgfqpoint{7.865820in}{5.963141in}}%
\pgfpathcurveto{\pgfqpoint{7.865820in}{5.974192in}}{\pgfqpoint{7.861430in}{5.984791in}}{\pgfqpoint{7.853616in}{5.992604in}}%
\pgfpathcurveto{\pgfqpoint{7.845803in}{6.000418in}}{\pgfqpoint{7.835204in}{6.004808in}}{\pgfqpoint{7.824153in}{6.004808in}}%
\pgfpathcurveto{\pgfqpoint{7.813103in}{6.004808in}}{\pgfqpoint{7.802504in}{6.000418in}}{\pgfqpoint{7.794691in}{5.992604in}}%
\pgfpathcurveto{\pgfqpoint{7.786877in}{5.984791in}}{\pgfqpoint{7.782487in}{5.974192in}}{\pgfqpoint{7.782487in}{5.963141in}}%
\pgfpathcurveto{\pgfqpoint{7.782487in}{5.952091in}}{\pgfqpoint{7.786877in}{5.941492in}}{\pgfqpoint{7.794691in}{5.933679in}}%
\pgfpathcurveto{\pgfqpoint{7.802504in}{5.925865in}}{\pgfqpoint{7.813103in}{5.921475in}}{\pgfqpoint{7.824153in}{5.921475in}}%
\pgfpathclose%
\pgfusepath{stroke,fill}%
\end{pgfscope}%
\begin{pgfscope}%
\pgfpathrectangle{\pgfqpoint{0.570343in}{0.331635in}}{\pgfqpoint{9.300000in}{7.700000in}}%
\pgfusepath{clip}%
\pgfsetbuttcap%
\pgfsetroundjoin%
\definecolor{currentfill}{rgb}{1.000000,0.705882,0.509804}%
\pgfsetfillcolor{currentfill}%
\pgfsetlinewidth{0.481800pt}%
\definecolor{currentstroke}{rgb}{1.000000,1.000000,1.000000}%
\pgfsetstrokecolor{currentstroke}%
\pgfsetdash{}{0pt}%
\pgfpathmoveto{\pgfqpoint{5.894391in}{4.089252in}}%
\pgfpathcurveto{\pgfqpoint{5.905441in}{4.089252in}}{\pgfqpoint{5.916040in}{4.093643in}}{\pgfqpoint{5.923854in}{4.101456in}}%
\pgfpathcurveto{\pgfqpoint{5.931667in}{4.109270in}}{\pgfqpoint{5.936057in}{4.119869in}}{\pgfqpoint{5.936057in}{4.130919in}}%
\pgfpathcurveto{\pgfqpoint{5.936057in}{4.141969in}}{\pgfqpoint{5.931667in}{4.152568in}}{\pgfqpoint{5.923854in}{4.160382in}}%
\pgfpathcurveto{\pgfqpoint{5.916040in}{4.168195in}}{\pgfqpoint{5.905441in}{4.172586in}}{\pgfqpoint{5.894391in}{4.172586in}}%
\pgfpathcurveto{\pgfqpoint{5.883341in}{4.172586in}}{\pgfqpoint{5.872742in}{4.168195in}}{\pgfqpoint{5.864928in}{4.160382in}}%
\pgfpathcurveto{\pgfqpoint{5.857114in}{4.152568in}}{\pgfqpoint{5.852724in}{4.141969in}}{\pgfqpoint{5.852724in}{4.130919in}}%
\pgfpathcurveto{\pgfqpoint{5.852724in}{4.119869in}}{\pgfqpoint{5.857114in}{4.109270in}}{\pgfqpoint{5.864928in}{4.101456in}}%
\pgfpathcurveto{\pgfqpoint{5.872742in}{4.093643in}}{\pgfqpoint{5.883341in}{4.089252in}}{\pgfqpoint{5.894391in}{4.089252in}}%
\pgfpathclose%
\pgfusepath{stroke,fill}%
\end{pgfscope}%
\begin{pgfscope}%
\pgfpathrectangle{\pgfqpoint{0.570343in}{0.331635in}}{\pgfqpoint{9.300000in}{7.700000in}}%
\pgfusepath{clip}%
\pgfsetbuttcap%
\pgfsetroundjoin%
\definecolor{currentfill}{rgb}{1.000000,0.705882,0.509804}%
\pgfsetfillcolor{currentfill}%
\pgfsetlinewidth{0.481800pt}%
\definecolor{currentstroke}{rgb}{1.000000,1.000000,1.000000}%
\pgfsetstrokecolor{currentstroke}%
\pgfsetdash{}{0pt}%
\pgfpathmoveto{\pgfqpoint{3.273962in}{4.188781in}}%
\pgfpathcurveto{\pgfqpoint{3.285012in}{4.188781in}}{\pgfqpoint{3.295611in}{4.193172in}}{\pgfqpoint{3.303425in}{4.200985in}}%
\pgfpathcurveto{\pgfqpoint{3.311239in}{4.208799in}}{\pgfqpoint{3.315629in}{4.219398in}}{\pgfqpoint{3.315629in}{4.230448in}}%
\pgfpathcurveto{\pgfqpoint{3.315629in}{4.241498in}}{\pgfqpoint{3.311239in}{4.252097in}}{\pgfqpoint{3.303425in}{4.259911in}}%
\pgfpathcurveto{\pgfqpoint{3.295611in}{4.267724in}}{\pgfqpoint{3.285012in}{4.272115in}}{\pgfqpoint{3.273962in}{4.272115in}}%
\pgfpathcurveto{\pgfqpoint{3.262912in}{4.272115in}}{\pgfqpoint{3.252313in}{4.267724in}}{\pgfqpoint{3.244499in}{4.259911in}}%
\pgfpathcurveto{\pgfqpoint{3.236686in}{4.252097in}}{\pgfqpoint{3.232295in}{4.241498in}}{\pgfqpoint{3.232295in}{4.230448in}}%
\pgfpathcurveto{\pgfqpoint{3.232295in}{4.219398in}}{\pgfqpoint{3.236686in}{4.208799in}}{\pgfqpoint{3.244499in}{4.200985in}}%
\pgfpathcurveto{\pgfqpoint{3.252313in}{4.193172in}}{\pgfqpoint{3.262912in}{4.188781in}}{\pgfqpoint{3.273962in}{4.188781in}}%
\pgfpathclose%
\pgfusepath{stroke,fill}%
\end{pgfscope}%
\begin{pgfscope}%
\pgfpathrectangle{\pgfqpoint{0.570343in}{0.331635in}}{\pgfqpoint{9.300000in}{7.700000in}}%
\pgfusepath{clip}%
\pgfsetbuttcap%
\pgfsetroundjoin%
\definecolor{currentfill}{rgb}{1.000000,0.705882,0.509804}%
\pgfsetfillcolor{currentfill}%
\pgfsetlinewidth{0.481800pt}%
\definecolor{currentstroke}{rgb}{1.000000,1.000000,1.000000}%
\pgfsetstrokecolor{currentstroke}%
\pgfsetdash{}{0pt}%
\pgfpathmoveto{\pgfqpoint{4.485603in}{2.482347in}}%
\pgfpathcurveto{\pgfqpoint{4.496653in}{2.482347in}}{\pgfqpoint{4.507252in}{2.486737in}}{\pgfqpoint{4.515066in}{2.494551in}}%
\pgfpathcurveto{\pgfqpoint{4.522879in}{2.502365in}}{\pgfqpoint{4.527269in}{2.512964in}}{\pgfqpoint{4.527269in}{2.524014in}}%
\pgfpathcurveto{\pgfqpoint{4.527269in}{2.535064in}}{\pgfqpoint{4.522879in}{2.545663in}}{\pgfqpoint{4.515066in}{2.553477in}}%
\pgfpathcurveto{\pgfqpoint{4.507252in}{2.561290in}}{\pgfqpoint{4.496653in}{2.565680in}}{\pgfqpoint{4.485603in}{2.565680in}}%
\pgfpathcurveto{\pgfqpoint{4.474553in}{2.565680in}}{\pgfqpoint{4.463954in}{2.561290in}}{\pgfqpoint{4.456140in}{2.553477in}}%
\pgfpathcurveto{\pgfqpoint{4.448326in}{2.545663in}}{\pgfqpoint{4.443936in}{2.535064in}}{\pgfqpoint{4.443936in}{2.524014in}}%
\pgfpathcurveto{\pgfqpoint{4.443936in}{2.512964in}}{\pgfqpoint{4.448326in}{2.502365in}}{\pgfqpoint{4.456140in}{2.494551in}}%
\pgfpathcurveto{\pgfqpoint{4.463954in}{2.486737in}}{\pgfqpoint{4.474553in}{2.482347in}}{\pgfqpoint{4.485603in}{2.482347in}}%
\pgfpathclose%
\pgfusepath{stroke,fill}%
\end{pgfscope}%
\begin{pgfscope}%
\pgfpathrectangle{\pgfqpoint{0.570343in}{0.331635in}}{\pgfqpoint{9.300000in}{7.700000in}}%
\pgfusepath{clip}%
\pgfsetbuttcap%
\pgfsetroundjoin%
\definecolor{currentfill}{rgb}{1.000000,0.705882,0.509804}%
\pgfsetfillcolor{currentfill}%
\pgfsetlinewidth{0.481800pt}%
\definecolor{currentstroke}{rgb}{1.000000,1.000000,1.000000}%
\pgfsetstrokecolor{currentstroke}%
\pgfsetdash{}{0pt}%
\pgfpathmoveto{\pgfqpoint{6.572664in}{5.197620in}}%
\pgfpathcurveto{\pgfqpoint{6.583714in}{5.197620in}}{\pgfqpoint{6.594313in}{5.202010in}}{\pgfqpoint{6.602127in}{5.209824in}}%
\pgfpathcurveto{\pgfqpoint{6.609940in}{5.217637in}}{\pgfqpoint{6.614330in}{5.228236in}}{\pgfqpoint{6.614330in}{5.239286in}}%
\pgfpathcurveto{\pgfqpoint{6.614330in}{5.250336in}}{\pgfqpoint{6.609940in}{5.260936in}}{\pgfqpoint{6.602127in}{5.268749in}}%
\pgfpathcurveto{\pgfqpoint{6.594313in}{5.276563in}}{\pgfqpoint{6.583714in}{5.280953in}}{\pgfqpoint{6.572664in}{5.280953in}}%
\pgfpathcurveto{\pgfqpoint{6.561614in}{5.280953in}}{\pgfqpoint{6.551015in}{5.276563in}}{\pgfqpoint{6.543201in}{5.268749in}}%
\pgfpathcurveto{\pgfqpoint{6.535387in}{5.260936in}}{\pgfqpoint{6.530997in}{5.250336in}}{\pgfqpoint{6.530997in}{5.239286in}}%
\pgfpathcurveto{\pgfqpoint{6.530997in}{5.228236in}}{\pgfqpoint{6.535387in}{5.217637in}}{\pgfqpoint{6.543201in}{5.209824in}}%
\pgfpathcurveto{\pgfqpoint{6.551015in}{5.202010in}}{\pgfqpoint{6.561614in}{5.197620in}}{\pgfqpoint{6.572664in}{5.197620in}}%
\pgfpathclose%
\pgfusepath{stroke,fill}%
\end{pgfscope}%
\begin{pgfscope}%
\pgfpathrectangle{\pgfqpoint{0.570343in}{0.331635in}}{\pgfqpoint{9.300000in}{7.700000in}}%
\pgfusepath{clip}%
\pgfsetbuttcap%
\pgfsetroundjoin%
\definecolor{currentfill}{rgb}{1.000000,0.705882,0.509804}%
\pgfsetfillcolor{currentfill}%
\pgfsetlinewidth{0.481800pt}%
\definecolor{currentstroke}{rgb}{1.000000,1.000000,1.000000}%
\pgfsetstrokecolor{currentstroke}%
\pgfsetdash{}{0pt}%
\pgfpathmoveto{\pgfqpoint{7.581009in}{1.719780in}}%
\pgfpathcurveto{\pgfqpoint{7.592059in}{1.719780in}}{\pgfqpoint{7.602658in}{1.724170in}}{\pgfqpoint{7.610472in}{1.731984in}}%
\pgfpathcurveto{\pgfqpoint{7.618285in}{1.739797in}}{\pgfqpoint{7.622676in}{1.750396in}}{\pgfqpoint{7.622676in}{1.761446in}}%
\pgfpathcurveto{\pgfqpoint{7.622676in}{1.772497in}}{\pgfqpoint{7.618285in}{1.783096in}}{\pgfqpoint{7.610472in}{1.790909in}}%
\pgfpathcurveto{\pgfqpoint{7.602658in}{1.798723in}}{\pgfqpoint{7.592059in}{1.803113in}}{\pgfqpoint{7.581009in}{1.803113in}}%
\pgfpathcurveto{\pgfqpoint{7.569959in}{1.803113in}}{\pgfqpoint{7.559360in}{1.798723in}}{\pgfqpoint{7.551546in}{1.790909in}}%
\pgfpathcurveto{\pgfqpoint{7.543732in}{1.783096in}}{\pgfqpoint{7.539342in}{1.772497in}}{\pgfqpoint{7.539342in}{1.761446in}}%
\pgfpathcurveto{\pgfqpoint{7.539342in}{1.750396in}}{\pgfqpoint{7.543732in}{1.739797in}}{\pgfqpoint{7.551546in}{1.731984in}}%
\pgfpathcurveto{\pgfqpoint{7.559360in}{1.724170in}}{\pgfqpoint{7.569959in}{1.719780in}}{\pgfqpoint{7.581009in}{1.719780in}}%
\pgfpathclose%
\pgfusepath{stroke,fill}%
\end{pgfscope}%
\begin{pgfscope}%
\pgfpathrectangle{\pgfqpoint{0.570343in}{0.331635in}}{\pgfqpoint{9.300000in}{7.700000in}}%
\pgfusepath{clip}%
\pgfsetbuttcap%
\pgfsetroundjoin%
\definecolor{currentfill}{rgb}{1.000000,0.705882,0.509804}%
\pgfsetfillcolor{currentfill}%
\pgfsetlinewidth{0.481800pt}%
\definecolor{currentstroke}{rgb}{1.000000,1.000000,1.000000}%
\pgfsetstrokecolor{currentstroke}%
\pgfsetdash{}{0pt}%
\pgfpathmoveto{\pgfqpoint{1.757394in}{6.484893in}}%
\pgfpathcurveto{\pgfqpoint{1.768445in}{6.484893in}}{\pgfqpoint{1.779044in}{6.489284in}}{\pgfqpoint{1.786857in}{6.497097in}}%
\pgfpathcurveto{\pgfqpoint{1.794671in}{6.504911in}}{\pgfqpoint{1.799061in}{6.515510in}}{\pgfqpoint{1.799061in}{6.526560in}}%
\pgfpathcurveto{\pgfqpoint{1.799061in}{6.537610in}}{\pgfqpoint{1.794671in}{6.548209in}}{\pgfqpoint{1.786857in}{6.556023in}}%
\pgfpathcurveto{\pgfqpoint{1.779044in}{6.563836in}}{\pgfqpoint{1.768445in}{6.568227in}}{\pgfqpoint{1.757394in}{6.568227in}}%
\pgfpathcurveto{\pgfqpoint{1.746344in}{6.568227in}}{\pgfqpoint{1.735745in}{6.563836in}}{\pgfqpoint{1.727932in}{6.556023in}}%
\pgfpathcurveto{\pgfqpoint{1.720118in}{6.548209in}}{\pgfqpoint{1.715728in}{6.537610in}}{\pgfqpoint{1.715728in}{6.526560in}}%
\pgfpathcurveto{\pgfqpoint{1.715728in}{6.515510in}}{\pgfqpoint{1.720118in}{6.504911in}}{\pgfqpoint{1.727932in}{6.497097in}}%
\pgfpathcurveto{\pgfqpoint{1.735745in}{6.489284in}}{\pgfqpoint{1.746344in}{6.484893in}}{\pgfqpoint{1.757394in}{6.484893in}}%
\pgfpathclose%
\pgfusepath{stroke,fill}%
\end{pgfscope}%
\begin{pgfscope}%
\pgfpathrectangle{\pgfqpoint{0.570343in}{0.331635in}}{\pgfqpoint{9.300000in}{7.700000in}}%
\pgfusepath{clip}%
\pgfsetbuttcap%
\pgfsetroundjoin%
\definecolor{currentfill}{rgb}{1.000000,0.705882,0.509804}%
\pgfsetfillcolor{currentfill}%
\pgfsetlinewidth{0.481800pt}%
\definecolor{currentstroke}{rgb}{1.000000,1.000000,1.000000}%
\pgfsetstrokecolor{currentstroke}%
\pgfsetdash{}{0pt}%
\pgfpathmoveto{\pgfqpoint{5.468750in}{7.308381in}}%
\pgfpathcurveto{\pgfqpoint{5.479800in}{7.308381in}}{\pgfqpoint{5.490399in}{7.312772in}}{\pgfqpoint{5.498213in}{7.320585in}}%
\pgfpathcurveto{\pgfqpoint{5.506027in}{7.328399in}}{\pgfqpoint{5.510417in}{7.338998in}}{\pgfqpoint{5.510417in}{7.350048in}}%
\pgfpathcurveto{\pgfqpoint{5.510417in}{7.361098in}}{\pgfqpoint{5.506027in}{7.371697in}}{\pgfqpoint{5.498213in}{7.379511in}}%
\pgfpathcurveto{\pgfqpoint{5.490399in}{7.387324in}}{\pgfqpoint{5.479800in}{7.391715in}}{\pgfqpoint{5.468750in}{7.391715in}}%
\pgfpathcurveto{\pgfqpoint{5.457700in}{7.391715in}}{\pgfqpoint{5.447101in}{7.387324in}}{\pgfqpoint{5.439287in}{7.379511in}}%
\pgfpathcurveto{\pgfqpoint{5.431474in}{7.371697in}}{\pgfqpoint{5.427084in}{7.361098in}}{\pgfqpoint{5.427084in}{7.350048in}}%
\pgfpathcurveto{\pgfqpoint{5.427084in}{7.338998in}}{\pgfqpoint{5.431474in}{7.328399in}}{\pgfqpoint{5.439287in}{7.320585in}}%
\pgfpathcurveto{\pgfqpoint{5.447101in}{7.312772in}}{\pgfqpoint{5.457700in}{7.308381in}}{\pgfqpoint{5.468750in}{7.308381in}}%
\pgfpathclose%
\pgfusepath{stroke,fill}%
\end{pgfscope}%
\begin{pgfscope}%
\pgfpathrectangle{\pgfqpoint{0.570343in}{0.331635in}}{\pgfqpoint{9.300000in}{7.700000in}}%
\pgfusepath{clip}%
\pgfsetbuttcap%
\pgfsetroundjoin%
\definecolor{currentfill}{rgb}{1.000000,0.705882,0.509804}%
\pgfsetfillcolor{currentfill}%
\pgfsetlinewidth{0.481800pt}%
\definecolor{currentstroke}{rgb}{1.000000,1.000000,1.000000}%
\pgfsetstrokecolor{currentstroke}%
\pgfsetdash{}{0pt}%
\pgfpathmoveto{\pgfqpoint{4.136121in}{6.363161in}}%
\pgfpathcurveto{\pgfqpoint{4.147171in}{6.363161in}}{\pgfqpoint{4.157770in}{6.367551in}}{\pgfqpoint{4.165584in}{6.375365in}}%
\pgfpathcurveto{\pgfqpoint{4.173398in}{6.383178in}}{\pgfqpoint{4.177788in}{6.393777in}}{\pgfqpoint{4.177788in}{6.404828in}}%
\pgfpathcurveto{\pgfqpoint{4.177788in}{6.415878in}}{\pgfqpoint{4.173398in}{6.426477in}}{\pgfqpoint{4.165584in}{6.434290in}}%
\pgfpathcurveto{\pgfqpoint{4.157770in}{6.442104in}}{\pgfqpoint{4.147171in}{6.446494in}}{\pgfqpoint{4.136121in}{6.446494in}}%
\pgfpathcurveto{\pgfqpoint{4.125071in}{6.446494in}}{\pgfqpoint{4.114472in}{6.442104in}}{\pgfqpoint{4.106658in}{6.434290in}}%
\pgfpathcurveto{\pgfqpoint{4.098845in}{6.426477in}}{\pgfqpoint{4.094454in}{6.415878in}}{\pgfqpoint{4.094454in}{6.404828in}}%
\pgfpathcurveto{\pgfqpoint{4.094454in}{6.393777in}}{\pgfqpoint{4.098845in}{6.383178in}}{\pgfqpoint{4.106658in}{6.375365in}}%
\pgfpathcurveto{\pgfqpoint{4.114472in}{6.367551in}}{\pgfqpoint{4.125071in}{6.363161in}}{\pgfqpoint{4.136121in}{6.363161in}}%
\pgfpathclose%
\pgfusepath{stroke,fill}%
\end{pgfscope}%
\begin{pgfscope}%
\pgfpathrectangle{\pgfqpoint{0.570343in}{0.331635in}}{\pgfqpoint{9.300000in}{7.700000in}}%
\pgfusepath{clip}%
\pgfsetbuttcap%
\pgfsetroundjoin%
\definecolor{currentfill}{rgb}{1.000000,0.705882,0.509804}%
\pgfsetfillcolor{currentfill}%
\pgfsetlinewidth{0.481800pt}%
\definecolor{currentstroke}{rgb}{1.000000,1.000000,1.000000}%
\pgfsetstrokecolor{currentstroke}%
\pgfsetdash{}{0pt}%
\pgfpathmoveto{\pgfqpoint{3.063964in}{3.535815in}}%
\pgfpathcurveto{\pgfqpoint{3.075014in}{3.535815in}}{\pgfqpoint{3.085613in}{3.540206in}}{\pgfqpoint{3.093427in}{3.548019in}}%
\pgfpathcurveto{\pgfqpoint{3.101241in}{3.555833in}}{\pgfqpoint{3.105631in}{3.566432in}}{\pgfqpoint{3.105631in}{3.577482in}}%
\pgfpathcurveto{\pgfqpoint{3.105631in}{3.588532in}}{\pgfqpoint{3.101241in}{3.599131in}}{\pgfqpoint{3.093427in}{3.606945in}}%
\pgfpathcurveto{\pgfqpoint{3.085613in}{3.614758in}}{\pgfqpoint{3.075014in}{3.619149in}}{\pgfqpoint{3.063964in}{3.619149in}}%
\pgfpathcurveto{\pgfqpoint{3.052914in}{3.619149in}}{\pgfqpoint{3.042315in}{3.614758in}}{\pgfqpoint{3.034501in}{3.606945in}}%
\pgfpathcurveto{\pgfqpoint{3.026688in}{3.599131in}}{\pgfqpoint{3.022297in}{3.588532in}}{\pgfqpoint{3.022297in}{3.577482in}}%
\pgfpathcurveto{\pgfqpoint{3.022297in}{3.566432in}}{\pgfqpoint{3.026688in}{3.555833in}}{\pgfqpoint{3.034501in}{3.548019in}}%
\pgfpathcurveto{\pgfqpoint{3.042315in}{3.540206in}}{\pgfqpoint{3.052914in}{3.535815in}}{\pgfqpoint{3.063964in}{3.535815in}}%
\pgfpathclose%
\pgfusepath{stroke,fill}%
\end{pgfscope}%
\begin{pgfscope}%
\pgfpathrectangle{\pgfqpoint{0.570343in}{0.331635in}}{\pgfqpoint{9.300000in}{7.700000in}}%
\pgfusepath{clip}%
\pgfsetbuttcap%
\pgfsetroundjoin%
\definecolor{currentfill}{rgb}{1.000000,0.705882,0.509804}%
\pgfsetfillcolor{currentfill}%
\pgfsetlinewidth{0.481800pt}%
\definecolor{currentstroke}{rgb}{1.000000,1.000000,1.000000}%
\pgfsetstrokecolor{currentstroke}%
\pgfsetdash{}{0pt}%
\pgfpathmoveto{\pgfqpoint{3.378790in}{1.183747in}}%
\pgfpathcurveto{\pgfqpoint{3.389840in}{1.183747in}}{\pgfqpoint{3.400439in}{1.188137in}}{\pgfqpoint{3.408252in}{1.195951in}}%
\pgfpathcurveto{\pgfqpoint{3.416066in}{1.203765in}}{\pgfqpoint{3.420456in}{1.214364in}}{\pgfqpoint{3.420456in}{1.225414in}}%
\pgfpathcurveto{\pgfqpoint{3.420456in}{1.236464in}}{\pgfqpoint{3.416066in}{1.247063in}}{\pgfqpoint{3.408252in}{1.254877in}}%
\pgfpathcurveto{\pgfqpoint{3.400439in}{1.262690in}}{\pgfqpoint{3.389840in}{1.267080in}}{\pgfqpoint{3.378790in}{1.267080in}}%
\pgfpathcurveto{\pgfqpoint{3.367740in}{1.267080in}}{\pgfqpoint{3.357140in}{1.262690in}}{\pgfqpoint{3.349327in}{1.254877in}}%
\pgfpathcurveto{\pgfqpoint{3.341513in}{1.247063in}}{\pgfqpoint{3.337123in}{1.236464in}}{\pgfqpoint{3.337123in}{1.225414in}}%
\pgfpathcurveto{\pgfqpoint{3.337123in}{1.214364in}}{\pgfqpoint{3.341513in}{1.203765in}}{\pgfqpoint{3.349327in}{1.195951in}}%
\pgfpathcurveto{\pgfqpoint{3.357140in}{1.188137in}}{\pgfqpoint{3.367740in}{1.183747in}}{\pgfqpoint{3.378790in}{1.183747in}}%
\pgfpathclose%
\pgfusepath{stroke,fill}%
\end{pgfscope}%
\begin{pgfscope}%
\pgfpathrectangle{\pgfqpoint{0.570343in}{0.331635in}}{\pgfqpoint{9.300000in}{7.700000in}}%
\pgfusepath{clip}%
\pgfsetbuttcap%
\pgfsetroundjoin%
\definecolor{currentfill}{rgb}{1.000000,0.705882,0.509804}%
\pgfsetfillcolor{currentfill}%
\pgfsetlinewidth{0.481800pt}%
\definecolor{currentstroke}{rgb}{1.000000,1.000000,1.000000}%
\pgfsetstrokecolor{currentstroke}%
\pgfsetdash{}{0pt}%
\pgfpathmoveto{\pgfqpoint{0.993071in}{1.736886in}}%
\pgfpathcurveto{\pgfqpoint{1.004121in}{1.736886in}}{\pgfqpoint{1.014720in}{1.741276in}}{\pgfqpoint{1.022533in}{1.749090in}}%
\pgfpathcurveto{\pgfqpoint{1.030347in}{1.756903in}}{\pgfqpoint{1.034737in}{1.767502in}}{\pgfqpoint{1.034737in}{1.778552in}}%
\pgfpathcurveto{\pgfqpoint{1.034737in}{1.789603in}}{\pgfqpoint{1.030347in}{1.800202in}}{\pgfqpoint{1.022533in}{1.808015in}}%
\pgfpathcurveto{\pgfqpoint{1.014720in}{1.815829in}}{\pgfqpoint{1.004121in}{1.820219in}}{\pgfqpoint{0.993071in}{1.820219in}}%
\pgfpathcurveto{\pgfqpoint{0.982020in}{1.820219in}}{\pgfqpoint{0.971421in}{1.815829in}}{\pgfqpoint{0.963608in}{1.808015in}}%
\pgfpathcurveto{\pgfqpoint{0.955794in}{1.800202in}}{\pgfqpoint{0.951404in}{1.789603in}}{\pgfqpoint{0.951404in}{1.778552in}}%
\pgfpathcurveto{\pgfqpoint{0.951404in}{1.767502in}}{\pgfqpoint{0.955794in}{1.756903in}}{\pgfqpoint{0.963608in}{1.749090in}}%
\pgfpathcurveto{\pgfqpoint{0.971421in}{1.741276in}}{\pgfqpoint{0.982020in}{1.736886in}}{\pgfqpoint{0.993071in}{1.736886in}}%
\pgfpathclose%
\pgfusepath{stroke,fill}%
\end{pgfscope}%
\begin{pgfscope}%
\pgfpathrectangle{\pgfqpoint{0.570343in}{0.331635in}}{\pgfqpoint{9.300000in}{7.700000in}}%
\pgfusepath{clip}%
\pgfsetbuttcap%
\pgfsetroundjoin%
\definecolor{currentfill}{rgb}{1.000000,0.705882,0.509804}%
\pgfsetfillcolor{currentfill}%
\pgfsetlinewidth{0.481800pt}%
\definecolor{currentstroke}{rgb}{1.000000,1.000000,1.000000}%
\pgfsetstrokecolor{currentstroke}%
\pgfsetdash{}{0pt}%
\pgfpathmoveto{\pgfqpoint{5.153229in}{2.615822in}}%
\pgfpathcurveto{\pgfqpoint{5.164279in}{2.615822in}}{\pgfqpoint{5.174878in}{2.620213in}}{\pgfqpoint{5.182691in}{2.628026in}}%
\pgfpathcurveto{\pgfqpoint{5.190505in}{2.635840in}}{\pgfqpoint{5.194895in}{2.646439in}}{\pgfqpoint{5.194895in}{2.657489in}}%
\pgfpathcurveto{\pgfqpoint{5.194895in}{2.668539in}}{\pgfqpoint{5.190505in}{2.679138in}}{\pgfqpoint{5.182691in}{2.686952in}}%
\pgfpathcurveto{\pgfqpoint{5.174878in}{2.694765in}}{\pgfqpoint{5.164279in}{2.699156in}}{\pgfqpoint{5.153229in}{2.699156in}}%
\pgfpathcurveto{\pgfqpoint{5.142178in}{2.699156in}}{\pgfqpoint{5.131579in}{2.694765in}}{\pgfqpoint{5.123766in}{2.686952in}}%
\pgfpathcurveto{\pgfqpoint{5.115952in}{2.679138in}}{\pgfqpoint{5.111562in}{2.668539in}}{\pgfqpoint{5.111562in}{2.657489in}}%
\pgfpathcurveto{\pgfqpoint{5.111562in}{2.646439in}}{\pgfqpoint{5.115952in}{2.635840in}}{\pgfqpoint{5.123766in}{2.628026in}}%
\pgfpathcurveto{\pgfqpoint{5.131579in}{2.620213in}}{\pgfqpoint{5.142178in}{2.615822in}}{\pgfqpoint{5.153229in}{2.615822in}}%
\pgfpathclose%
\pgfusepath{stroke,fill}%
\end{pgfscope}%
\begin{pgfscope}%
\pgfpathrectangle{\pgfqpoint{0.570343in}{0.331635in}}{\pgfqpoint{9.300000in}{7.700000in}}%
\pgfusepath{clip}%
\pgfsetbuttcap%
\pgfsetroundjoin%
\definecolor{currentfill}{rgb}{1.000000,0.705882,0.509804}%
\pgfsetfillcolor{currentfill}%
\pgfsetlinewidth{0.481800pt}%
\definecolor{currentstroke}{rgb}{1.000000,1.000000,1.000000}%
\pgfsetstrokecolor{currentstroke}%
\pgfsetdash{}{0pt}%
\pgfpathmoveto{\pgfqpoint{8.561752in}{3.062614in}}%
\pgfpathcurveto{\pgfqpoint{8.572802in}{3.062614in}}{\pgfqpoint{8.583401in}{3.067004in}}{\pgfqpoint{8.591215in}{3.074818in}}%
\pgfpathcurveto{\pgfqpoint{8.599028in}{3.082632in}}{\pgfqpoint{8.603419in}{3.093231in}}{\pgfqpoint{8.603419in}{3.104281in}}%
\pgfpathcurveto{\pgfqpoint{8.603419in}{3.115331in}}{\pgfqpoint{8.599028in}{3.125930in}}{\pgfqpoint{8.591215in}{3.133744in}}%
\pgfpathcurveto{\pgfqpoint{8.583401in}{3.141557in}}{\pgfqpoint{8.572802in}{3.145947in}}{\pgfqpoint{8.561752in}{3.145947in}}%
\pgfpathcurveto{\pgfqpoint{8.550702in}{3.145947in}}{\pgfqpoint{8.540103in}{3.141557in}}{\pgfqpoint{8.532289in}{3.133744in}}%
\pgfpathcurveto{\pgfqpoint{8.524476in}{3.125930in}}{\pgfqpoint{8.520085in}{3.115331in}}{\pgfqpoint{8.520085in}{3.104281in}}%
\pgfpathcurveto{\pgfqpoint{8.520085in}{3.093231in}}{\pgfqpoint{8.524476in}{3.082632in}}{\pgfqpoint{8.532289in}{3.074818in}}%
\pgfpathcurveto{\pgfqpoint{8.540103in}{3.067004in}}{\pgfqpoint{8.550702in}{3.062614in}}{\pgfqpoint{8.561752in}{3.062614in}}%
\pgfpathclose%
\pgfusepath{stroke,fill}%
\end{pgfscope}%
\begin{pgfscope}%
\pgfpathrectangle{\pgfqpoint{0.570343in}{0.331635in}}{\pgfqpoint{9.300000in}{7.700000in}}%
\pgfusepath{clip}%
\pgfsetbuttcap%
\pgfsetroundjoin%
\definecolor{currentfill}{rgb}{1.000000,0.705882,0.509804}%
\pgfsetfillcolor{currentfill}%
\pgfsetlinewidth{0.481800pt}%
\definecolor{currentstroke}{rgb}{1.000000,1.000000,1.000000}%
\pgfsetstrokecolor{currentstroke}%
\pgfsetdash{}{0pt}%
\pgfpathmoveto{\pgfqpoint{4.047696in}{1.828593in}}%
\pgfpathcurveto{\pgfqpoint{4.058746in}{1.828593in}}{\pgfqpoint{4.069345in}{1.832983in}}{\pgfqpoint{4.077158in}{1.840797in}}%
\pgfpathcurveto{\pgfqpoint{4.084972in}{1.848611in}}{\pgfqpoint{4.089362in}{1.859210in}}{\pgfqpoint{4.089362in}{1.870260in}}%
\pgfpathcurveto{\pgfqpoint{4.089362in}{1.881310in}}{\pgfqpoint{4.084972in}{1.891909in}}{\pgfqpoint{4.077158in}{1.899723in}}%
\pgfpathcurveto{\pgfqpoint{4.069345in}{1.907536in}}{\pgfqpoint{4.058746in}{1.911927in}}{\pgfqpoint{4.047696in}{1.911927in}}%
\pgfpathcurveto{\pgfqpoint{4.036646in}{1.911927in}}{\pgfqpoint{4.026046in}{1.907536in}}{\pgfqpoint{4.018233in}{1.899723in}}%
\pgfpathcurveto{\pgfqpoint{4.010419in}{1.891909in}}{\pgfqpoint{4.006029in}{1.881310in}}{\pgfqpoint{4.006029in}{1.870260in}}%
\pgfpathcurveto{\pgfqpoint{4.006029in}{1.859210in}}{\pgfqpoint{4.010419in}{1.848611in}}{\pgfqpoint{4.018233in}{1.840797in}}%
\pgfpathcurveto{\pgfqpoint{4.026046in}{1.832983in}}{\pgfqpoint{4.036646in}{1.828593in}}{\pgfqpoint{4.047696in}{1.828593in}}%
\pgfpathclose%
\pgfusepath{stroke,fill}%
\end{pgfscope}%
\begin{pgfscope}%
\pgfpathrectangle{\pgfqpoint{0.570343in}{0.331635in}}{\pgfqpoint{9.300000in}{7.700000in}}%
\pgfusepath{clip}%
\pgfsetbuttcap%
\pgfsetroundjoin%
\definecolor{currentfill}{rgb}{1.000000,0.705882,0.509804}%
\pgfsetfillcolor{currentfill}%
\pgfsetlinewidth{0.481800pt}%
\definecolor{currentstroke}{rgb}{1.000000,1.000000,1.000000}%
\pgfsetstrokecolor{currentstroke}%
\pgfsetdash{}{0pt}%
\pgfpathmoveto{\pgfqpoint{8.305186in}{3.919496in}}%
\pgfpathcurveto{\pgfqpoint{8.316236in}{3.919496in}}{\pgfqpoint{8.326835in}{3.923887in}}{\pgfqpoint{8.334648in}{3.931700in}}%
\pgfpathcurveto{\pgfqpoint{8.342462in}{3.939514in}}{\pgfqpoint{8.346852in}{3.950113in}}{\pgfqpoint{8.346852in}{3.961163in}}%
\pgfpathcurveto{\pgfqpoint{8.346852in}{3.972213in}}{\pgfqpoint{8.342462in}{3.982812in}}{\pgfqpoint{8.334648in}{3.990626in}}%
\pgfpathcurveto{\pgfqpoint{8.326835in}{3.998439in}}{\pgfqpoint{8.316236in}{4.002830in}}{\pgfqpoint{8.305186in}{4.002830in}}%
\pgfpathcurveto{\pgfqpoint{8.294136in}{4.002830in}}{\pgfqpoint{8.283537in}{3.998439in}}{\pgfqpoint{8.275723in}{3.990626in}}%
\pgfpathcurveto{\pgfqpoint{8.267909in}{3.982812in}}{\pgfqpoint{8.263519in}{3.972213in}}{\pgfqpoint{8.263519in}{3.961163in}}%
\pgfpathcurveto{\pgfqpoint{8.263519in}{3.950113in}}{\pgfqpoint{8.267909in}{3.939514in}}{\pgfqpoint{8.275723in}{3.931700in}}%
\pgfpathcurveto{\pgfqpoint{8.283537in}{3.923887in}}{\pgfqpoint{8.294136in}{3.919496in}}{\pgfqpoint{8.305186in}{3.919496in}}%
\pgfpathclose%
\pgfusepath{stroke,fill}%
\end{pgfscope}%
\begin{pgfscope}%
\pgfpathrectangle{\pgfqpoint{0.570343in}{0.331635in}}{\pgfqpoint{9.300000in}{7.700000in}}%
\pgfusepath{clip}%
\pgfsetbuttcap%
\pgfsetroundjoin%
\definecolor{currentfill}{rgb}{1.000000,0.705882,0.509804}%
\pgfsetfillcolor{currentfill}%
\pgfsetlinewidth{0.481800pt}%
\definecolor{currentstroke}{rgb}{1.000000,1.000000,1.000000}%
\pgfsetstrokecolor{currentstroke}%
\pgfsetdash{}{0pt}%
\pgfpathmoveto{\pgfqpoint{4.619077in}{0.639968in}}%
\pgfpathcurveto{\pgfqpoint{4.630127in}{0.639968in}}{\pgfqpoint{4.640726in}{0.644359in}}{\pgfqpoint{4.648539in}{0.652172in}}%
\pgfpathcurveto{\pgfqpoint{4.656353in}{0.659986in}}{\pgfqpoint{4.660743in}{0.670585in}}{\pgfqpoint{4.660743in}{0.681635in}}%
\pgfpathcurveto{\pgfqpoint{4.660743in}{0.692685in}}{\pgfqpoint{4.656353in}{0.703284in}}{\pgfqpoint{4.648539in}{0.711098in}}%
\pgfpathcurveto{\pgfqpoint{4.640726in}{0.718911in}}{\pgfqpoint{4.630127in}{0.723302in}}{\pgfqpoint{4.619077in}{0.723302in}}%
\pgfpathcurveto{\pgfqpoint{4.608026in}{0.723302in}}{\pgfqpoint{4.597427in}{0.718911in}}{\pgfqpoint{4.589614in}{0.711098in}}%
\pgfpathcurveto{\pgfqpoint{4.581800in}{0.703284in}}{\pgfqpoint{4.577410in}{0.692685in}}{\pgfqpoint{4.577410in}{0.681635in}}%
\pgfpathcurveto{\pgfqpoint{4.577410in}{0.670585in}}{\pgfqpoint{4.581800in}{0.659986in}}{\pgfqpoint{4.589614in}{0.652172in}}%
\pgfpathcurveto{\pgfqpoint{4.597427in}{0.644359in}}{\pgfqpoint{4.608026in}{0.639968in}}{\pgfqpoint{4.619077in}{0.639968in}}%
\pgfpathclose%
\pgfusepath{stroke,fill}%
\end{pgfscope}%
\begin{pgfscope}%
\pgfpathrectangle{\pgfqpoint{0.570343in}{0.331635in}}{\pgfqpoint{9.300000in}{7.700000in}}%
\pgfusepath{clip}%
\pgfsetbuttcap%
\pgfsetroundjoin%
\definecolor{currentfill}{rgb}{1.000000,0.705882,0.509804}%
\pgfsetfillcolor{currentfill}%
\pgfsetlinewidth{0.481800pt}%
\definecolor{currentstroke}{rgb}{1.000000,1.000000,1.000000}%
\pgfsetstrokecolor{currentstroke}%
\pgfsetdash{}{0pt}%
\pgfpathmoveto{\pgfqpoint{8.308279in}{7.639968in}}%
\pgfpathcurveto{\pgfqpoint{8.319329in}{7.639968in}}{\pgfqpoint{8.329928in}{7.644359in}}{\pgfqpoint{8.337742in}{7.652172in}}%
\pgfpathcurveto{\pgfqpoint{8.345555in}{7.659986in}}{\pgfqpoint{8.349945in}{7.670585in}}{\pgfqpoint{8.349945in}{7.681635in}}%
\pgfpathcurveto{\pgfqpoint{8.349945in}{7.692685in}}{\pgfqpoint{8.345555in}{7.703284in}}{\pgfqpoint{8.337742in}{7.711098in}}%
\pgfpathcurveto{\pgfqpoint{8.329928in}{7.718911in}}{\pgfqpoint{8.319329in}{7.723302in}}{\pgfqpoint{8.308279in}{7.723302in}}%
\pgfpathcurveto{\pgfqpoint{8.297229in}{7.723302in}}{\pgfqpoint{8.286630in}{7.718911in}}{\pgfqpoint{8.278816in}{7.711098in}}%
\pgfpathcurveto{\pgfqpoint{8.271002in}{7.703284in}}{\pgfqpoint{8.266612in}{7.692685in}}{\pgfqpoint{8.266612in}{7.681635in}}%
\pgfpathcurveto{\pgfqpoint{8.266612in}{7.670585in}}{\pgfqpoint{8.271002in}{7.659986in}}{\pgfqpoint{8.278816in}{7.652172in}}%
\pgfpathcurveto{\pgfqpoint{8.286630in}{7.644359in}}{\pgfqpoint{8.297229in}{7.639968in}}{\pgfqpoint{8.308279in}{7.639968in}}%
\pgfpathclose%
\pgfusepath{stroke,fill}%
\end{pgfscope}%
\begin{pgfscope}%
\pgfpathrectangle{\pgfqpoint{0.570343in}{0.331635in}}{\pgfqpoint{9.300000in}{7.700000in}}%
\pgfusepath{clip}%
\pgfsetbuttcap%
\pgfsetroundjoin%
\definecolor{currentfill}{rgb}{1.000000,0.705882,0.509804}%
\pgfsetfillcolor{currentfill}%
\pgfsetlinewidth{0.481800pt}%
\definecolor{currentstroke}{rgb}{1.000000,1.000000,1.000000}%
\pgfsetstrokecolor{currentstroke}%
\pgfsetdash{}{0pt}%
\pgfpathmoveto{\pgfqpoint{6.197001in}{1.217501in}}%
\pgfpathcurveto{\pgfqpoint{6.208051in}{1.217501in}}{\pgfqpoint{6.218650in}{1.221891in}}{\pgfqpoint{6.226463in}{1.229705in}}%
\pgfpathcurveto{\pgfqpoint{6.234277in}{1.237518in}}{\pgfqpoint{6.238667in}{1.248117in}}{\pgfqpoint{6.238667in}{1.259168in}}%
\pgfpathcurveto{\pgfqpoint{6.238667in}{1.270218in}}{\pgfqpoint{6.234277in}{1.280817in}}{\pgfqpoint{6.226463in}{1.288630in}}%
\pgfpathcurveto{\pgfqpoint{6.218650in}{1.296444in}}{\pgfqpoint{6.208051in}{1.300834in}}{\pgfqpoint{6.197001in}{1.300834in}}%
\pgfpathcurveto{\pgfqpoint{6.185951in}{1.300834in}}{\pgfqpoint{6.175352in}{1.296444in}}{\pgfqpoint{6.167538in}{1.288630in}}%
\pgfpathcurveto{\pgfqpoint{6.159724in}{1.280817in}}{\pgfqpoint{6.155334in}{1.270218in}}{\pgfqpoint{6.155334in}{1.259168in}}%
\pgfpathcurveto{\pgfqpoint{6.155334in}{1.248117in}}{\pgfqpoint{6.159724in}{1.237518in}}{\pgfqpoint{6.167538in}{1.229705in}}%
\pgfpathcurveto{\pgfqpoint{6.175352in}{1.221891in}}{\pgfqpoint{6.185951in}{1.217501in}}{\pgfqpoint{6.197001in}{1.217501in}}%
\pgfpathclose%
\pgfusepath{stroke,fill}%
\end{pgfscope}%
\begin{pgfscope}%
\pgfpathrectangle{\pgfqpoint{0.570343in}{0.331635in}}{\pgfqpoint{9.300000in}{7.700000in}}%
\pgfusepath{clip}%
\pgfsetbuttcap%
\pgfsetroundjoin%
\definecolor{currentfill}{rgb}{1.000000,0.705882,0.509804}%
\pgfsetfillcolor{currentfill}%
\pgfsetlinewidth{0.481800pt}%
\definecolor{currentstroke}{rgb}{1.000000,1.000000,1.000000}%
\pgfsetstrokecolor{currentstroke}%
\pgfsetdash{}{0pt}%
\pgfpathmoveto{\pgfqpoint{9.447616in}{5.083458in}}%
\pgfpathcurveto{\pgfqpoint{9.458666in}{5.083458in}}{\pgfqpoint{9.469265in}{5.087849in}}{\pgfqpoint{9.477079in}{5.095662in}}%
\pgfpathcurveto{\pgfqpoint{9.484892in}{5.103476in}}{\pgfqpoint{9.489283in}{5.114075in}}{\pgfqpoint{9.489283in}{5.125125in}}%
\pgfpathcurveto{\pgfqpoint{9.489283in}{5.136175in}}{\pgfqpoint{9.484892in}{5.146774in}}{\pgfqpoint{9.477079in}{5.154588in}}%
\pgfpathcurveto{\pgfqpoint{9.469265in}{5.162402in}}{\pgfqpoint{9.458666in}{5.166792in}}{\pgfqpoint{9.447616in}{5.166792in}}%
\pgfpathcurveto{\pgfqpoint{9.436566in}{5.166792in}}{\pgfqpoint{9.425967in}{5.162402in}}{\pgfqpoint{9.418153in}{5.154588in}}%
\pgfpathcurveto{\pgfqpoint{9.410340in}{5.146774in}}{\pgfqpoint{9.405949in}{5.136175in}}{\pgfqpoint{9.405949in}{5.125125in}}%
\pgfpathcurveto{\pgfqpoint{9.405949in}{5.114075in}}{\pgfqpoint{9.410340in}{5.103476in}}{\pgfqpoint{9.418153in}{5.095662in}}%
\pgfpathcurveto{\pgfqpoint{9.425967in}{5.087849in}}{\pgfqpoint{9.436566in}{5.083458in}}{\pgfqpoint{9.447616in}{5.083458in}}%
\pgfpathclose%
\pgfusepath{stroke,fill}%
\end{pgfscope}%
\begin{pgfscope}%
\pgfpathrectangle{\pgfqpoint{0.570343in}{0.331635in}}{\pgfqpoint{9.300000in}{7.700000in}}%
\pgfusepath{clip}%
\pgfsetbuttcap%
\pgfsetroundjoin%
\definecolor{currentfill}{rgb}{1.000000,0.705882,0.509804}%
\pgfsetfillcolor{currentfill}%
\pgfsetlinewidth{0.481800pt}%
\definecolor{currentstroke}{rgb}{1.000000,1.000000,1.000000}%
\pgfsetstrokecolor{currentstroke}%
\pgfsetdash{}{0pt}%
\pgfpathmoveto{\pgfqpoint{6.034006in}{2.397376in}}%
\pgfpathcurveto{\pgfqpoint{6.045056in}{2.397376in}}{\pgfqpoint{6.055655in}{2.401766in}}{\pgfqpoint{6.063468in}{2.409579in}}%
\pgfpathcurveto{\pgfqpoint{6.071282in}{2.417393in}}{\pgfqpoint{6.075672in}{2.427992in}}{\pgfqpoint{6.075672in}{2.439042in}}%
\pgfpathcurveto{\pgfqpoint{6.075672in}{2.450092in}}{\pgfqpoint{6.071282in}{2.460691in}}{\pgfqpoint{6.063468in}{2.468505in}}%
\pgfpathcurveto{\pgfqpoint{6.055655in}{2.476319in}}{\pgfqpoint{6.045056in}{2.480709in}}{\pgfqpoint{6.034006in}{2.480709in}}%
\pgfpathcurveto{\pgfqpoint{6.022955in}{2.480709in}}{\pgfqpoint{6.012356in}{2.476319in}}{\pgfqpoint{6.004543in}{2.468505in}}%
\pgfpathcurveto{\pgfqpoint{5.996729in}{2.460691in}}{\pgfqpoint{5.992339in}{2.450092in}}{\pgfqpoint{5.992339in}{2.439042in}}%
\pgfpathcurveto{\pgfqpoint{5.992339in}{2.427992in}}{\pgfqpoint{5.996729in}{2.417393in}}{\pgfqpoint{6.004543in}{2.409579in}}%
\pgfpathcurveto{\pgfqpoint{6.012356in}{2.401766in}}{\pgfqpoint{6.022955in}{2.397376in}}{\pgfqpoint{6.034006in}{2.397376in}}%
\pgfpathclose%
\pgfusepath{stroke,fill}%
\end{pgfscope}%
\begin{pgfscope}%
\pgfpathrectangle{\pgfqpoint{0.570343in}{0.331635in}}{\pgfqpoint{9.300000in}{7.700000in}}%
\pgfusepath{clip}%
\pgfsetbuttcap%
\pgfsetroundjoin%
\definecolor{currentfill}{rgb}{1.000000,0.705882,0.509804}%
\pgfsetfillcolor{currentfill}%
\pgfsetlinewidth{0.481800pt}%
\definecolor{currentstroke}{rgb}{1.000000,1.000000,1.000000}%
\pgfsetstrokecolor{currentstroke}%
\pgfsetdash{}{0pt}%
\pgfpathmoveto{\pgfqpoint{8.815623in}{2.157538in}}%
\pgfpathcurveto{\pgfqpoint{8.826674in}{2.157538in}}{\pgfqpoint{8.837273in}{2.161928in}}{\pgfqpoint{8.845086in}{2.169741in}}%
\pgfpathcurveto{\pgfqpoint{8.852900in}{2.177555in}}{\pgfqpoint{8.857290in}{2.188154in}}{\pgfqpoint{8.857290in}{2.199204in}}%
\pgfpathcurveto{\pgfqpoint{8.857290in}{2.210254in}}{\pgfqpoint{8.852900in}{2.220853in}}{\pgfqpoint{8.845086in}{2.228667in}}%
\pgfpathcurveto{\pgfqpoint{8.837273in}{2.236481in}}{\pgfqpoint{8.826674in}{2.240871in}}{\pgfqpoint{8.815623in}{2.240871in}}%
\pgfpathcurveto{\pgfqpoint{8.804573in}{2.240871in}}{\pgfqpoint{8.793974in}{2.236481in}}{\pgfqpoint{8.786161in}{2.228667in}}%
\pgfpathcurveto{\pgfqpoint{8.778347in}{2.220853in}}{\pgfqpoint{8.773957in}{2.210254in}}{\pgfqpoint{8.773957in}{2.199204in}}%
\pgfpathcurveto{\pgfqpoint{8.773957in}{2.188154in}}{\pgfqpoint{8.778347in}{2.177555in}}{\pgfqpoint{8.786161in}{2.169741in}}%
\pgfpathcurveto{\pgfqpoint{8.793974in}{2.161928in}}{\pgfqpoint{8.804573in}{2.157538in}}{\pgfqpoint{8.815623in}{2.157538in}}%
\pgfpathclose%
\pgfusepath{stroke,fill}%
\end{pgfscope}%
\begin{pgfscope}%
\pgfpathrectangle{\pgfqpoint{0.570343in}{0.331635in}}{\pgfqpoint{9.300000in}{7.700000in}}%
\pgfusepath{clip}%
\pgfsetbuttcap%
\pgfsetroundjoin%
\definecolor{currentfill}{rgb}{1.000000,0.705882,0.509804}%
\pgfsetfillcolor{currentfill}%
\pgfsetlinewidth{0.481800pt}%
\definecolor{currentstroke}{rgb}{1.000000,1.000000,1.000000}%
\pgfsetstrokecolor{currentstroke}%
\pgfsetdash{}{0pt}%
\pgfpathmoveto{\pgfqpoint{7.488123in}{3.678053in}}%
\pgfpathcurveto{\pgfqpoint{7.499173in}{3.678053in}}{\pgfqpoint{7.509772in}{3.682443in}}{\pgfqpoint{7.517586in}{3.690257in}}%
\pgfpathcurveto{\pgfqpoint{7.525399in}{3.698070in}}{\pgfqpoint{7.529790in}{3.708669in}}{\pgfqpoint{7.529790in}{3.719720in}}%
\pgfpathcurveto{\pgfqpoint{7.529790in}{3.730770in}}{\pgfqpoint{7.525399in}{3.741369in}}{\pgfqpoint{7.517586in}{3.749182in}}%
\pgfpathcurveto{\pgfqpoint{7.509772in}{3.756996in}}{\pgfqpoint{7.499173in}{3.761386in}}{\pgfqpoint{7.488123in}{3.761386in}}%
\pgfpathcurveto{\pgfqpoint{7.477073in}{3.761386in}}{\pgfqpoint{7.466474in}{3.756996in}}{\pgfqpoint{7.458660in}{3.749182in}}%
\pgfpathcurveto{\pgfqpoint{7.450847in}{3.741369in}}{\pgfqpoint{7.446456in}{3.730770in}}{\pgfqpoint{7.446456in}{3.719720in}}%
\pgfpathcurveto{\pgfqpoint{7.446456in}{3.708669in}}{\pgfqpoint{7.450847in}{3.698070in}}{\pgfqpoint{7.458660in}{3.690257in}}%
\pgfpathcurveto{\pgfqpoint{7.466474in}{3.682443in}}{\pgfqpoint{7.477073in}{3.678053in}}{\pgfqpoint{7.488123in}{3.678053in}}%
\pgfpathclose%
\pgfusepath{stroke,fill}%
\end{pgfscope}%
\begin{pgfscope}%
\pgfpathrectangle{\pgfqpoint{0.570343in}{0.331635in}}{\pgfqpoint{9.300000in}{7.700000in}}%
\pgfusepath{clip}%
\pgfsetbuttcap%
\pgfsetroundjoin%
\definecolor{currentfill}{rgb}{1.000000,0.705882,0.509804}%
\pgfsetfillcolor{currentfill}%
\pgfsetlinewidth{0.481800pt}%
\definecolor{currentstroke}{rgb}{1.000000,1.000000,1.000000}%
\pgfsetstrokecolor{currentstroke}%
\pgfsetdash{}{0pt}%
\pgfpathmoveto{\pgfqpoint{2.671621in}{4.803306in}}%
\pgfpathcurveto{\pgfqpoint{2.682671in}{4.803306in}}{\pgfqpoint{2.693270in}{4.807696in}}{\pgfqpoint{2.701084in}{4.815510in}}%
\pgfpathcurveto{\pgfqpoint{2.708898in}{4.823324in}}{\pgfqpoint{2.713288in}{4.833923in}}{\pgfqpoint{2.713288in}{4.844973in}}%
\pgfpathcurveto{\pgfqpoint{2.713288in}{4.856023in}}{\pgfqpoint{2.708898in}{4.866622in}}{\pgfqpoint{2.701084in}{4.874436in}}%
\pgfpathcurveto{\pgfqpoint{2.693270in}{4.882249in}}{\pgfqpoint{2.682671in}{4.886640in}}{\pgfqpoint{2.671621in}{4.886640in}}%
\pgfpathcurveto{\pgfqpoint{2.660571in}{4.886640in}}{\pgfqpoint{2.649972in}{4.882249in}}{\pgfqpoint{2.642159in}{4.874436in}}%
\pgfpathcurveto{\pgfqpoint{2.634345in}{4.866622in}}{\pgfqpoint{2.629955in}{4.856023in}}{\pgfqpoint{2.629955in}{4.844973in}}%
\pgfpathcurveto{\pgfqpoint{2.629955in}{4.833923in}}{\pgfqpoint{2.634345in}{4.823324in}}{\pgfqpoint{2.642159in}{4.815510in}}%
\pgfpathcurveto{\pgfqpoint{2.649972in}{4.807696in}}{\pgfqpoint{2.660571in}{4.803306in}}{\pgfqpoint{2.671621in}{4.803306in}}%
\pgfpathclose%
\pgfusepath{stroke,fill}%
\end{pgfscope}%
\begin{pgfscope}%
\pgfpathrectangle{\pgfqpoint{0.570343in}{0.331635in}}{\pgfqpoint{9.300000in}{7.700000in}}%
\pgfusepath{clip}%
\pgfsetbuttcap%
\pgfsetroundjoin%
\definecolor{currentfill}{rgb}{1.000000,0.705882,0.509804}%
\pgfsetfillcolor{currentfill}%
\pgfsetlinewidth{0.481800pt}%
\definecolor{currentstroke}{rgb}{1.000000,1.000000,1.000000}%
\pgfsetstrokecolor{currentstroke}%
\pgfsetdash{}{0pt}%
\pgfpathmoveto{\pgfqpoint{7.154996in}{6.987530in}}%
\pgfpathcurveto{\pgfqpoint{7.166046in}{6.987530in}}{\pgfqpoint{7.176645in}{6.991920in}}{\pgfqpoint{7.184459in}{6.999733in}}%
\pgfpathcurveto{\pgfqpoint{7.192272in}{7.007547in}}{\pgfqpoint{7.196663in}{7.018146in}}{\pgfqpoint{7.196663in}{7.029196in}}%
\pgfpathcurveto{\pgfqpoint{7.196663in}{7.040246in}}{\pgfqpoint{7.192272in}{7.050845in}}{\pgfqpoint{7.184459in}{7.058659in}}%
\pgfpathcurveto{\pgfqpoint{7.176645in}{7.066473in}}{\pgfqpoint{7.166046in}{7.070863in}}{\pgfqpoint{7.154996in}{7.070863in}}%
\pgfpathcurveto{\pgfqpoint{7.143946in}{7.070863in}}{\pgfqpoint{7.133347in}{7.066473in}}{\pgfqpoint{7.125533in}{7.058659in}}%
\pgfpathcurveto{\pgfqpoint{7.117720in}{7.050845in}}{\pgfqpoint{7.113329in}{7.040246in}}{\pgfqpoint{7.113329in}{7.029196in}}%
\pgfpathcurveto{\pgfqpoint{7.113329in}{7.018146in}}{\pgfqpoint{7.117720in}{7.007547in}}{\pgfqpoint{7.125533in}{6.999733in}}%
\pgfpathcurveto{\pgfqpoint{7.133347in}{6.991920in}}{\pgfqpoint{7.143946in}{6.987530in}}{\pgfqpoint{7.154996in}{6.987530in}}%
\pgfpathclose%
\pgfusepath{stroke,fill}%
\end{pgfscope}%
\begin{pgfscope}%
\pgfpathrectangle{\pgfqpoint{0.570343in}{0.331635in}}{\pgfqpoint{9.300000in}{7.700000in}}%
\pgfusepath{clip}%
\pgfsetbuttcap%
\pgfsetroundjoin%
\definecolor{currentfill}{rgb}{1.000000,0.705882,0.509804}%
\pgfsetfillcolor{currentfill}%
\pgfsetlinewidth{0.481800pt}%
\definecolor{currentstroke}{rgb}{1.000000,1.000000,1.000000}%
\pgfsetstrokecolor{currentstroke}%
\pgfsetdash{}{0pt}%
\pgfpathmoveto{\pgfqpoint{3.897376in}{2.961996in}}%
\pgfpathcurveto{\pgfqpoint{3.908426in}{2.961996in}}{\pgfqpoint{3.919025in}{2.966387in}}{\pgfqpoint{3.926838in}{2.974200in}}%
\pgfpathcurveto{\pgfqpoint{3.934652in}{2.982014in}}{\pgfqpoint{3.939042in}{2.992613in}}{\pgfqpoint{3.939042in}{3.003663in}}%
\pgfpathcurveto{\pgfqpoint{3.939042in}{3.014713in}}{\pgfqpoint{3.934652in}{3.025312in}}{\pgfqpoint{3.926838in}{3.033126in}}%
\pgfpathcurveto{\pgfqpoint{3.919025in}{3.040939in}}{\pgfqpoint{3.908426in}{3.045330in}}{\pgfqpoint{3.897376in}{3.045330in}}%
\pgfpathcurveto{\pgfqpoint{3.886326in}{3.045330in}}{\pgfqpoint{3.875726in}{3.040939in}}{\pgfqpoint{3.867913in}{3.033126in}}%
\pgfpathcurveto{\pgfqpoint{3.860099in}{3.025312in}}{\pgfqpoint{3.855709in}{3.014713in}}{\pgfqpoint{3.855709in}{3.003663in}}%
\pgfpathcurveto{\pgfqpoint{3.855709in}{2.992613in}}{\pgfqpoint{3.860099in}{2.982014in}}{\pgfqpoint{3.867913in}{2.974200in}}%
\pgfpathcurveto{\pgfqpoint{3.875726in}{2.966387in}}{\pgfqpoint{3.886326in}{2.961996in}}{\pgfqpoint{3.897376in}{2.961996in}}%
\pgfpathclose%
\pgfusepath{stroke,fill}%
\end{pgfscope}%
\begin{pgfscope}%
\pgfpathrectangle{\pgfqpoint{0.570343in}{0.331635in}}{\pgfqpoint{9.300000in}{7.700000in}}%
\pgfusepath{clip}%
\pgfsetbuttcap%
\pgfsetroundjoin%
\definecolor{currentfill}{rgb}{1.000000,0.705882,0.509804}%
\pgfsetfillcolor{currentfill}%
\pgfsetlinewidth{0.481800pt}%
\definecolor{currentstroke}{rgb}{1.000000,1.000000,1.000000}%
\pgfsetstrokecolor{currentstroke}%
\pgfsetdash{}{0pt}%
\pgfpathmoveto{\pgfqpoint{5.935607in}{5.697947in}}%
\pgfpathcurveto{\pgfqpoint{5.946657in}{5.697947in}}{\pgfqpoint{5.957256in}{5.702338in}}{\pgfqpoint{5.965070in}{5.710151in}}%
\pgfpathcurveto{\pgfqpoint{5.972883in}{5.717965in}}{\pgfqpoint{5.977273in}{5.728564in}}{\pgfqpoint{5.977273in}{5.739614in}}%
\pgfpathcurveto{\pgfqpoint{5.977273in}{5.750664in}}{\pgfqpoint{5.972883in}{5.761263in}}{\pgfqpoint{5.965070in}{5.769077in}}%
\pgfpathcurveto{\pgfqpoint{5.957256in}{5.776891in}}{\pgfqpoint{5.946657in}{5.781281in}}{\pgfqpoint{5.935607in}{5.781281in}}%
\pgfpathcurveto{\pgfqpoint{5.924557in}{5.781281in}}{\pgfqpoint{5.913958in}{5.776891in}}{\pgfqpoint{5.906144in}{5.769077in}}%
\pgfpathcurveto{\pgfqpoint{5.898330in}{5.761263in}}{\pgfqpoint{5.893940in}{5.750664in}}{\pgfqpoint{5.893940in}{5.739614in}}%
\pgfpathcurveto{\pgfqpoint{5.893940in}{5.728564in}}{\pgfqpoint{5.898330in}{5.717965in}}{\pgfqpoint{5.906144in}{5.710151in}}%
\pgfpathcurveto{\pgfqpoint{5.913958in}{5.702338in}}{\pgfqpoint{5.924557in}{5.697947in}}{\pgfqpoint{5.935607in}{5.697947in}}%
\pgfpathclose%
\pgfusepath{stroke,fill}%
\end{pgfscope}%
\begin{pgfscope}%
\pgfpathrectangle{\pgfqpoint{0.570343in}{0.331635in}}{\pgfqpoint{9.300000in}{7.700000in}}%
\pgfusepath{clip}%
\pgfsetbuttcap%
\pgfsetroundjoin%
\definecolor{currentfill}{rgb}{0.631373,0.788235,0.956863}%
\pgfsetfillcolor{currentfill}%
\pgfsetlinewidth{1.003750pt}%
\definecolor{currentstroke}{rgb}{0.631373,0.788235,0.956863}%
\pgfsetstrokecolor{currentstroke}%
\pgfsetdash{}{0pt}%
\pgfsys@defobject{currentmarker}{\pgfqpoint{-0.041667in}{-0.041667in}}{\pgfqpoint{0.041667in}{0.041667in}}{%
\pgfpathmoveto{\pgfqpoint{0.000000in}{-0.041667in}}%
\pgfpathcurveto{\pgfqpoint{0.011050in}{-0.041667in}}{\pgfqpoint{0.021649in}{-0.037276in}}{\pgfqpoint{0.029463in}{-0.029463in}}%
\pgfpathcurveto{\pgfqpoint{0.037276in}{-0.021649in}}{\pgfqpoint{0.041667in}{-0.011050in}}{\pgfqpoint{0.041667in}{0.000000in}}%
\pgfpathcurveto{\pgfqpoint{0.041667in}{0.011050in}}{\pgfqpoint{0.037276in}{0.021649in}}{\pgfqpoint{0.029463in}{0.029463in}}%
\pgfpathcurveto{\pgfqpoint{0.021649in}{0.037276in}}{\pgfqpoint{0.011050in}{0.041667in}}{\pgfqpoint{0.000000in}{0.041667in}}%
\pgfpathcurveto{\pgfqpoint{-0.011050in}{0.041667in}}{\pgfqpoint{-0.021649in}{0.037276in}}{\pgfqpoint{-0.029463in}{0.029463in}}%
\pgfpathcurveto{\pgfqpoint{-0.037276in}{0.021649in}}{\pgfqpoint{-0.041667in}{0.011050in}}{\pgfqpoint{-0.041667in}{0.000000in}}%
\pgfpathcurveto{\pgfqpoint{-0.041667in}{-0.011050in}}{\pgfqpoint{-0.037276in}{-0.021649in}}{\pgfqpoint{-0.029463in}{-0.029463in}}%
\pgfpathcurveto{\pgfqpoint{-0.021649in}{-0.037276in}}{\pgfqpoint{-0.011050in}{-0.041667in}}{\pgfqpoint{0.000000in}{-0.041667in}}%
\pgfpathclose%
\pgfusepath{stroke,fill}%
}%
\end{pgfscope}%
\begin{pgfscope}%
\pgfpathrectangle{\pgfqpoint{0.570343in}{0.331635in}}{\pgfqpoint{9.300000in}{7.700000in}}%
\pgfusepath{clip}%
\pgfsetbuttcap%
\pgfsetroundjoin%
\definecolor{currentfill}{rgb}{1.000000,0.705882,0.509804}%
\pgfsetfillcolor{currentfill}%
\pgfsetlinewidth{1.003750pt}%
\definecolor{currentstroke}{rgb}{1.000000,0.705882,0.509804}%
\pgfsetstrokecolor{currentstroke}%
\pgfsetdash{}{0pt}%
\pgfsys@defobject{currentmarker}{\pgfqpoint{-0.041667in}{-0.041667in}}{\pgfqpoint{0.041667in}{0.041667in}}{%
\pgfpathmoveto{\pgfqpoint{0.000000in}{-0.041667in}}%
\pgfpathcurveto{\pgfqpoint{0.011050in}{-0.041667in}}{\pgfqpoint{0.021649in}{-0.037276in}}{\pgfqpoint{0.029463in}{-0.029463in}}%
\pgfpathcurveto{\pgfqpoint{0.037276in}{-0.021649in}}{\pgfqpoint{0.041667in}{-0.011050in}}{\pgfqpoint{0.041667in}{0.000000in}}%
\pgfpathcurveto{\pgfqpoint{0.041667in}{0.011050in}}{\pgfqpoint{0.037276in}{0.021649in}}{\pgfqpoint{0.029463in}{0.029463in}}%
\pgfpathcurveto{\pgfqpoint{0.021649in}{0.037276in}}{\pgfqpoint{0.011050in}{0.041667in}}{\pgfqpoint{0.000000in}{0.041667in}}%
\pgfpathcurveto{\pgfqpoint{-0.011050in}{0.041667in}}{\pgfqpoint{-0.021649in}{0.037276in}}{\pgfqpoint{-0.029463in}{0.029463in}}%
\pgfpathcurveto{\pgfqpoint{-0.037276in}{0.021649in}}{\pgfqpoint{-0.041667in}{0.011050in}}{\pgfqpoint{-0.041667in}{0.000000in}}%
\pgfpathcurveto{\pgfqpoint{-0.041667in}{-0.011050in}}{\pgfqpoint{-0.037276in}{-0.021649in}}{\pgfqpoint{-0.029463in}{-0.029463in}}%
\pgfpathcurveto{\pgfqpoint{-0.021649in}{-0.037276in}}{\pgfqpoint{-0.011050in}{-0.041667in}}{\pgfqpoint{0.000000in}{-0.041667in}}%
\pgfpathclose%
\pgfusepath{stroke,fill}%
}%
\end{pgfscope}%
\begin{pgfscope}%
\pgfsetbuttcap%
\pgfsetroundjoin%
\definecolor{currentfill}{rgb}{0.000000,0.000000,0.000000}%
\pgfsetfillcolor{currentfill}%
\pgfsetlinewidth{0.803000pt}%
\definecolor{currentstroke}{rgb}{0.000000,0.000000,0.000000}%
\pgfsetstrokecolor{currentstroke}%
\pgfsetdash{}{0pt}%
\pgfsys@defobject{currentmarker}{\pgfqpoint{0.000000in}{-0.048611in}}{\pgfqpoint{0.000000in}{0.000000in}}{%
\pgfpathmoveto{\pgfqpoint{0.000000in}{0.000000in}}%
\pgfpathlineto{\pgfqpoint{0.000000in}{-0.048611in}}%
\pgfusepath{stroke,fill}%
}%
\begin{pgfscope}%
\pgfsys@transformshift{2.109765in}{0.331635in}%
\pgfsys@useobject{currentmarker}{}%
\end{pgfscope}%
\end{pgfscope}%
\begin{pgfscope}%
\definecolor{textcolor}{rgb}{0.000000,0.000000,0.000000}%
\pgfsetstrokecolor{textcolor}%
\pgfsetfillcolor{textcolor}%
\pgftext[x=2.109765in,y=0.234413in,,top]{\color{textcolor}\sffamily\fontsize{10.000000}{12.000000}\selectfont \ensuremath{-}100}%
\end{pgfscope}%
\begin{pgfscope}%
\pgfsetbuttcap%
\pgfsetroundjoin%
\definecolor{currentfill}{rgb}{0.000000,0.000000,0.000000}%
\pgfsetfillcolor{currentfill}%
\pgfsetlinewidth{0.803000pt}%
\definecolor{currentstroke}{rgb}{0.000000,0.000000,0.000000}%
\pgfsetstrokecolor{currentstroke}%
\pgfsetdash{}{0pt}%
\pgfsys@defobject{currentmarker}{\pgfqpoint{0.000000in}{-0.048611in}}{\pgfqpoint{0.000000in}{0.000000in}}{%
\pgfpathmoveto{\pgfqpoint{0.000000in}{0.000000in}}%
\pgfpathlineto{\pgfqpoint{0.000000in}{-0.048611in}}%
\pgfusepath{stroke,fill}%
}%
\begin{pgfscope}%
\pgfsys@transformshift{3.797034in}{0.331635in}%
\pgfsys@useobject{currentmarker}{}%
\end{pgfscope}%
\end{pgfscope}%
\begin{pgfscope}%
\definecolor{textcolor}{rgb}{0.000000,0.000000,0.000000}%
\pgfsetstrokecolor{textcolor}%
\pgfsetfillcolor{textcolor}%
\pgftext[x=3.797034in,y=0.234413in,,top]{\color{textcolor}\sffamily\fontsize{10.000000}{12.000000}\selectfont \ensuremath{-}50}%
\end{pgfscope}%
\begin{pgfscope}%
\pgfsetbuttcap%
\pgfsetroundjoin%
\definecolor{currentfill}{rgb}{0.000000,0.000000,0.000000}%
\pgfsetfillcolor{currentfill}%
\pgfsetlinewidth{0.803000pt}%
\definecolor{currentstroke}{rgb}{0.000000,0.000000,0.000000}%
\pgfsetstrokecolor{currentstroke}%
\pgfsetdash{}{0pt}%
\pgfsys@defobject{currentmarker}{\pgfqpoint{0.000000in}{-0.048611in}}{\pgfqpoint{0.000000in}{0.000000in}}{%
\pgfpathmoveto{\pgfqpoint{0.000000in}{0.000000in}}%
\pgfpathlineto{\pgfqpoint{0.000000in}{-0.048611in}}%
\pgfusepath{stroke,fill}%
}%
\begin{pgfscope}%
\pgfsys@transformshift{5.484303in}{0.331635in}%
\pgfsys@useobject{currentmarker}{}%
\end{pgfscope}%
\end{pgfscope}%
\begin{pgfscope}%
\definecolor{textcolor}{rgb}{0.000000,0.000000,0.000000}%
\pgfsetstrokecolor{textcolor}%
\pgfsetfillcolor{textcolor}%
\pgftext[x=5.484303in,y=0.234413in,,top]{\color{textcolor}\sffamily\fontsize{10.000000}{12.000000}\selectfont 0}%
\end{pgfscope}%
\begin{pgfscope}%
\pgfsetbuttcap%
\pgfsetroundjoin%
\definecolor{currentfill}{rgb}{0.000000,0.000000,0.000000}%
\pgfsetfillcolor{currentfill}%
\pgfsetlinewidth{0.803000pt}%
\definecolor{currentstroke}{rgb}{0.000000,0.000000,0.000000}%
\pgfsetstrokecolor{currentstroke}%
\pgfsetdash{}{0pt}%
\pgfsys@defobject{currentmarker}{\pgfqpoint{0.000000in}{-0.048611in}}{\pgfqpoint{0.000000in}{0.000000in}}{%
\pgfpathmoveto{\pgfqpoint{0.000000in}{0.000000in}}%
\pgfpathlineto{\pgfqpoint{0.000000in}{-0.048611in}}%
\pgfusepath{stroke,fill}%
}%
\begin{pgfscope}%
\pgfsys@transformshift{7.171572in}{0.331635in}%
\pgfsys@useobject{currentmarker}{}%
\end{pgfscope}%
\end{pgfscope}%
\begin{pgfscope}%
\definecolor{textcolor}{rgb}{0.000000,0.000000,0.000000}%
\pgfsetstrokecolor{textcolor}%
\pgfsetfillcolor{textcolor}%
\pgftext[x=7.171572in,y=0.234413in,,top]{\color{textcolor}\sffamily\fontsize{10.000000}{12.000000}\selectfont 50}%
\end{pgfscope}%
\begin{pgfscope}%
\pgfsetbuttcap%
\pgfsetroundjoin%
\definecolor{currentfill}{rgb}{0.000000,0.000000,0.000000}%
\pgfsetfillcolor{currentfill}%
\pgfsetlinewidth{0.803000pt}%
\definecolor{currentstroke}{rgb}{0.000000,0.000000,0.000000}%
\pgfsetstrokecolor{currentstroke}%
\pgfsetdash{}{0pt}%
\pgfsys@defobject{currentmarker}{\pgfqpoint{0.000000in}{-0.048611in}}{\pgfqpoint{0.000000in}{0.000000in}}{%
\pgfpathmoveto{\pgfqpoint{0.000000in}{0.000000in}}%
\pgfpathlineto{\pgfqpoint{0.000000in}{-0.048611in}}%
\pgfusepath{stroke,fill}%
}%
\begin{pgfscope}%
\pgfsys@transformshift{8.858841in}{0.331635in}%
\pgfsys@useobject{currentmarker}{}%
\end{pgfscope}%
\end{pgfscope}%
\begin{pgfscope}%
\definecolor{textcolor}{rgb}{0.000000,0.000000,0.000000}%
\pgfsetstrokecolor{textcolor}%
\pgfsetfillcolor{textcolor}%
\pgftext[x=8.858841in,y=0.234413in,,top]{\color{textcolor}\sffamily\fontsize{10.000000}{12.000000}\selectfont 100}%
\end{pgfscope}%
\begin{pgfscope}%
\pgfsetbuttcap%
\pgfsetroundjoin%
\definecolor{currentfill}{rgb}{0.000000,0.000000,0.000000}%
\pgfsetfillcolor{currentfill}%
\pgfsetlinewidth{0.803000pt}%
\definecolor{currentstroke}{rgb}{0.000000,0.000000,0.000000}%
\pgfsetstrokecolor{currentstroke}%
\pgfsetdash{}{0pt}%
\pgfsys@defobject{currentmarker}{\pgfqpoint{-0.048611in}{0.000000in}}{\pgfqpoint{-0.000000in}{0.000000in}}{%
\pgfpathmoveto{\pgfqpoint{-0.000000in}{0.000000in}}%
\pgfpathlineto{\pgfqpoint{-0.048611in}{0.000000in}}%
\pgfusepath{stroke,fill}%
}%
\begin{pgfscope}%
\pgfsys@transformshift{0.570343in}{1.425219in}%
\pgfsys@useobject{currentmarker}{}%
\end{pgfscope}%
\end{pgfscope}%
\begin{pgfscope}%
\definecolor{textcolor}{rgb}{0.000000,0.000000,0.000000}%
\pgfsetstrokecolor{textcolor}%
\pgfsetfillcolor{textcolor}%
\pgftext[x=0.100000in, y=1.372457in, left, base]{\color{textcolor}\sffamily\fontsize{10.000000}{12.000000}\selectfont \ensuremath{-}100}%
\end{pgfscope}%
\begin{pgfscope}%
\pgfsetbuttcap%
\pgfsetroundjoin%
\definecolor{currentfill}{rgb}{0.000000,0.000000,0.000000}%
\pgfsetfillcolor{currentfill}%
\pgfsetlinewidth{0.803000pt}%
\definecolor{currentstroke}{rgb}{0.000000,0.000000,0.000000}%
\pgfsetstrokecolor{currentstroke}%
\pgfsetdash{}{0pt}%
\pgfsys@defobject{currentmarker}{\pgfqpoint{-0.048611in}{0.000000in}}{\pgfqpoint{-0.000000in}{0.000000in}}{%
\pgfpathmoveto{\pgfqpoint{-0.000000in}{0.000000in}}%
\pgfpathlineto{\pgfqpoint{-0.048611in}{0.000000in}}%
\pgfusepath{stroke,fill}%
}%
\begin{pgfscope}%
\pgfsys@transformshift{0.570343in}{2.863412in}%
\pgfsys@useobject{currentmarker}{}%
\end{pgfscope}%
\end{pgfscope}%
\begin{pgfscope}%
\definecolor{textcolor}{rgb}{0.000000,0.000000,0.000000}%
\pgfsetstrokecolor{textcolor}%
\pgfsetfillcolor{textcolor}%
\pgftext[x=0.188365in, y=2.810650in, left, base]{\color{textcolor}\sffamily\fontsize{10.000000}{12.000000}\selectfont \ensuremath{-}50}%
\end{pgfscope}%
\begin{pgfscope}%
\pgfsetbuttcap%
\pgfsetroundjoin%
\definecolor{currentfill}{rgb}{0.000000,0.000000,0.000000}%
\pgfsetfillcolor{currentfill}%
\pgfsetlinewidth{0.803000pt}%
\definecolor{currentstroke}{rgb}{0.000000,0.000000,0.000000}%
\pgfsetstrokecolor{currentstroke}%
\pgfsetdash{}{0pt}%
\pgfsys@defobject{currentmarker}{\pgfqpoint{-0.048611in}{0.000000in}}{\pgfqpoint{-0.000000in}{0.000000in}}{%
\pgfpathmoveto{\pgfqpoint{-0.000000in}{0.000000in}}%
\pgfpathlineto{\pgfqpoint{-0.048611in}{0.000000in}}%
\pgfusepath{stroke,fill}%
}%
\begin{pgfscope}%
\pgfsys@transformshift{0.570343in}{4.301604in}%
\pgfsys@useobject{currentmarker}{}%
\end{pgfscope}%
\end{pgfscope}%
\begin{pgfscope}%
\definecolor{textcolor}{rgb}{0.000000,0.000000,0.000000}%
\pgfsetstrokecolor{textcolor}%
\pgfsetfillcolor{textcolor}%
\pgftext[x=0.384756in, y=4.248843in, left, base]{\color{textcolor}\sffamily\fontsize{10.000000}{12.000000}\selectfont 0}%
\end{pgfscope}%
\begin{pgfscope}%
\pgfsetbuttcap%
\pgfsetroundjoin%
\definecolor{currentfill}{rgb}{0.000000,0.000000,0.000000}%
\pgfsetfillcolor{currentfill}%
\pgfsetlinewidth{0.803000pt}%
\definecolor{currentstroke}{rgb}{0.000000,0.000000,0.000000}%
\pgfsetstrokecolor{currentstroke}%
\pgfsetdash{}{0pt}%
\pgfsys@defobject{currentmarker}{\pgfqpoint{-0.048611in}{0.000000in}}{\pgfqpoint{-0.000000in}{0.000000in}}{%
\pgfpathmoveto{\pgfqpoint{-0.000000in}{0.000000in}}%
\pgfpathlineto{\pgfqpoint{-0.048611in}{0.000000in}}%
\pgfusepath{stroke,fill}%
}%
\begin{pgfscope}%
\pgfsys@transformshift{0.570343in}{5.739797in}%
\pgfsys@useobject{currentmarker}{}%
\end{pgfscope}%
\end{pgfscope}%
\begin{pgfscope}%
\definecolor{textcolor}{rgb}{0.000000,0.000000,0.000000}%
\pgfsetstrokecolor{textcolor}%
\pgfsetfillcolor{textcolor}%
\pgftext[x=0.296390in, y=5.687035in, left, base]{\color{textcolor}\sffamily\fontsize{10.000000}{12.000000}\selectfont 50}%
\end{pgfscope}%
\begin{pgfscope}%
\pgfsetbuttcap%
\pgfsetroundjoin%
\definecolor{currentfill}{rgb}{0.000000,0.000000,0.000000}%
\pgfsetfillcolor{currentfill}%
\pgfsetlinewidth{0.803000pt}%
\definecolor{currentstroke}{rgb}{0.000000,0.000000,0.000000}%
\pgfsetstrokecolor{currentstroke}%
\pgfsetdash{}{0pt}%
\pgfsys@defobject{currentmarker}{\pgfqpoint{-0.048611in}{0.000000in}}{\pgfqpoint{-0.000000in}{0.000000in}}{%
\pgfpathmoveto{\pgfqpoint{-0.000000in}{0.000000in}}%
\pgfpathlineto{\pgfqpoint{-0.048611in}{0.000000in}}%
\pgfusepath{stroke,fill}%
}%
\begin{pgfscope}%
\pgfsys@transformshift{0.570343in}{7.177989in}%
\pgfsys@useobject{currentmarker}{}%
\end{pgfscope}%
\end{pgfscope}%
\begin{pgfscope}%
\definecolor{textcolor}{rgb}{0.000000,0.000000,0.000000}%
\pgfsetstrokecolor{textcolor}%
\pgfsetfillcolor{textcolor}%
\pgftext[x=0.208025in, y=7.125228in, left, base]{\color{textcolor}\sffamily\fontsize{10.000000}{12.000000}\selectfont 100}%
\end{pgfscope}%
\begin{pgfscope}%
\pgfpathrectangle{\pgfqpoint{0.570343in}{0.331635in}}{\pgfqpoint{9.300000in}{7.700000in}}%
\pgfusepath{clip}%
\pgfsetrectcap%
\pgfsetroundjoin%
\pgfsetlinewidth{1.505625pt}%
\definecolor{currentstroke}{rgb}{0.631373,0.788235,0.956863}%
\pgfsetstrokecolor{currentstroke}%
\pgfsetstrokeopacity{0.800000}%
\pgfsetdash{}{0pt}%
\pgfpathmoveto{\pgfqpoint{7.296937in}{5.090198in}}%
\pgfpathlineto{\pgfqpoint{5.327974in}{4.421066in}}%
\pgfusepath{stroke}%
\end{pgfscope}%
\begin{pgfscope}%
\pgfpathrectangle{\pgfqpoint{0.570343in}{0.331635in}}{\pgfqpoint{9.300000in}{7.700000in}}%
\pgfusepath{clip}%
\pgfsetrectcap%
\pgfsetroundjoin%
\pgfsetlinewidth{1.505625pt}%
\definecolor{currentstroke}{rgb}{0.631373,0.788235,0.956863}%
\pgfsetstrokecolor{currentstroke}%
\pgfsetstrokeopacity{0.800000}%
\pgfsetdash{}{0pt}%
\pgfpathmoveto{\pgfqpoint{4.019475in}{3.717731in}}%
\pgfpathlineto{\pgfqpoint{5.327974in}{4.421066in}}%
\pgfusepath{stroke}%
\end{pgfscope}%
\begin{pgfscope}%
\pgfpathrectangle{\pgfqpoint{0.570343in}{0.331635in}}{\pgfqpoint{9.300000in}{7.700000in}}%
\pgfusepath{clip}%
\pgfsetrectcap%
\pgfsetroundjoin%
\pgfsetlinewidth{1.505625pt}%
\definecolor{currentstroke}{rgb}{0.631373,0.788235,0.956863}%
\pgfsetstrokecolor{currentstroke}%
\pgfsetstrokeopacity{0.800000}%
\pgfsetdash{}{0pt}%
\pgfpathmoveto{\pgfqpoint{4.684473in}{3.356530in}}%
\pgfpathlineto{\pgfqpoint{5.327974in}{4.421066in}}%
\pgfusepath{stroke}%
\end{pgfscope}%
\begin{pgfscope}%
\pgfpathrectangle{\pgfqpoint{0.570343in}{0.331635in}}{\pgfqpoint{9.300000in}{7.700000in}}%
\pgfusepath{clip}%
\pgfsetrectcap%
\pgfsetroundjoin%
\pgfsetlinewidth{1.505625pt}%
\definecolor{currentstroke}{rgb}{0.631373,0.788235,0.956863}%
\pgfsetstrokecolor{currentstroke}%
\pgfsetstrokeopacity{0.800000}%
\pgfsetdash{}{0pt}%
\pgfpathmoveto{\pgfqpoint{7.305182in}{2.839150in}}%
\pgfpathlineto{\pgfqpoint{5.327974in}{4.421066in}}%
\pgfusepath{stroke}%
\end{pgfscope}%
\begin{pgfscope}%
\pgfpathrectangle{\pgfqpoint{0.570343in}{0.331635in}}{\pgfqpoint{9.300000in}{7.700000in}}%
\pgfusepath{clip}%
\pgfsetrectcap%
\pgfsetroundjoin%
\pgfsetlinewidth{1.505625pt}%
\definecolor{currentstroke}{rgb}{0.631373,0.788235,0.956863}%
\pgfsetstrokecolor{currentstroke}%
\pgfsetstrokeopacity{0.800000}%
\pgfsetdash{}{0pt}%
\pgfpathmoveto{\pgfqpoint{8.044602in}{5.291231in}}%
\pgfpathlineto{\pgfqpoint{5.327974in}{4.421066in}}%
\pgfusepath{stroke}%
\end{pgfscope}%
\begin{pgfscope}%
\pgfpathrectangle{\pgfqpoint{0.570343in}{0.331635in}}{\pgfqpoint{9.300000in}{7.700000in}}%
\pgfusepath{clip}%
\pgfsetrectcap%
\pgfsetroundjoin%
\pgfsetlinewidth{1.505625pt}%
\definecolor{currentstroke}{rgb}{0.631373,0.788235,0.956863}%
\pgfsetstrokecolor{currentstroke}%
\pgfsetstrokeopacity{0.800000}%
\pgfsetdash{}{0pt}%
\pgfpathmoveto{\pgfqpoint{6.828282in}{4.428679in}}%
\pgfpathlineto{\pgfqpoint{5.327974in}{4.421066in}}%
\pgfusepath{stroke}%
\end{pgfscope}%
\begin{pgfscope}%
\pgfpathrectangle{\pgfqpoint{0.570343in}{0.331635in}}{\pgfqpoint{9.300000in}{7.700000in}}%
\pgfusepath{clip}%
\pgfsetrectcap%
\pgfsetroundjoin%
\pgfsetlinewidth{1.505625pt}%
\definecolor{currentstroke}{rgb}{0.631373,0.788235,0.956863}%
\pgfsetstrokecolor{currentstroke}%
\pgfsetstrokeopacity{0.800000}%
\pgfsetdash{}{0pt}%
\pgfpathmoveto{\pgfqpoint{8.665804in}{4.818725in}}%
\pgfpathlineto{\pgfqpoint{5.327974in}{4.421066in}}%
\pgfusepath{stroke}%
\end{pgfscope}%
\begin{pgfscope}%
\pgfpathrectangle{\pgfqpoint{0.570343in}{0.331635in}}{\pgfqpoint{9.300000in}{7.700000in}}%
\pgfusepath{clip}%
\pgfsetrectcap%
\pgfsetroundjoin%
\pgfsetlinewidth{1.505625pt}%
\definecolor{currentstroke}{rgb}{0.631373,0.788235,0.956863}%
\pgfsetstrokecolor{currentstroke}%
\pgfsetstrokeopacity{0.800000}%
\pgfsetdash{}{0pt}%
\pgfpathmoveto{\pgfqpoint{8.976651in}{5.686339in}}%
\pgfpathlineto{\pgfqpoint{5.327974in}{4.421066in}}%
\pgfusepath{stroke}%
\end{pgfscope}%
\begin{pgfscope}%
\pgfpathrectangle{\pgfqpoint{0.570343in}{0.331635in}}{\pgfqpoint{9.300000in}{7.700000in}}%
\pgfusepath{clip}%
\pgfsetrectcap%
\pgfsetroundjoin%
\pgfsetlinewidth{1.505625pt}%
\definecolor{currentstroke}{rgb}{0.631373,0.788235,0.956863}%
\pgfsetstrokecolor{currentstroke}%
\pgfsetstrokeopacity{0.800000}%
\pgfsetdash{}{0pt}%
\pgfpathmoveto{\pgfqpoint{5.272172in}{4.818041in}}%
\pgfpathlineto{\pgfqpoint{5.327974in}{4.421066in}}%
\pgfusepath{stroke}%
\end{pgfscope}%
\begin{pgfscope}%
\pgfpathrectangle{\pgfqpoint{0.570343in}{0.331635in}}{\pgfqpoint{9.300000in}{7.700000in}}%
\pgfusepath{clip}%
\pgfsetrectcap%
\pgfsetroundjoin%
\pgfsetlinewidth{1.505625pt}%
\definecolor{currentstroke}{rgb}{0.631373,0.788235,0.956863}%
\pgfsetstrokecolor{currentstroke}%
\pgfsetstrokeopacity{0.800000}%
\pgfsetdash{}{0pt}%
\pgfpathmoveto{\pgfqpoint{3.063486in}{6.472096in}}%
\pgfpathlineto{\pgfqpoint{5.327974in}{4.421066in}}%
\pgfusepath{stroke}%
\end{pgfscope}%
\begin{pgfscope}%
\pgfpathrectangle{\pgfqpoint{0.570343in}{0.331635in}}{\pgfqpoint{9.300000in}{7.700000in}}%
\pgfusepath{clip}%
\pgfsetrectcap%
\pgfsetroundjoin%
\pgfsetlinewidth{1.505625pt}%
\definecolor{currentstroke}{rgb}{0.631373,0.788235,0.956863}%
\pgfsetstrokecolor{currentstroke}%
\pgfsetstrokeopacity{0.800000}%
\pgfsetdash{}{0pt}%
\pgfpathmoveto{\pgfqpoint{6.063540in}{4.805271in}}%
\pgfpathlineto{\pgfqpoint{5.327974in}{4.421066in}}%
\pgfusepath{stroke}%
\end{pgfscope}%
\begin{pgfscope}%
\pgfpathrectangle{\pgfqpoint{0.570343in}{0.331635in}}{\pgfqpoint{9.300000in}{7.700000in}}%
\pgfusepath{clip}%
\pgfsetrectcap%
\pgfsetroundjoin%
\pgfsetlinewidth{1.505625pt}%
\definecolor{currentstroke}{rgb}{0.631373,0.788235,0.956863}%
\pgfsetstrokecolor{currentstroke}%
\pgfsetstrokeopacity{0.800000}%
\pgfsetdash{}{0pt}%
\pgfpathmoveto{\pgfqpoint{1.762911in}{3.896616in}}%
\pgfpathlineto{\pgfqpoint{5.327974in}{4.421066in}}%
\pgfusepath{stroke}%
\end{pgfscope}%
\begin{pgfscope}%
\pgfpathrectangle{\pgfqpoint{0.570343in}{0.331635in}}{\pgfqpoint{9.300000in}{7.700000in}}%
\pgfusepath{clip}%
\pgfsetrectcap%
\pgfsetroundjoin%
\pgfsetlinewidth{1.505625pt}%
\definecolor{currentstroke}{rgb}{0.631373,0.788235,0.956863}%
\pgfsetstrokecolor{currentstroke}%
\pgfsetstrokeopacity{0.800000}%
\pgfsetdash{}{0pt}%
\pgfpathmoveto{\pgfqpoint{4.671643in}{5.155754in}}%
\pgfpathlineto{\pgfqpoint{5.327974in}{4.421066in}}%
\pgfusepath{stroke}%
\end{pgfscope}%
\begin{pgfscope}%
\pgfpathrectangle{\pgfqpoint{0.570343in}{0.331635in}}{\pgfqpoint{9.300000in}{7.700000in}}%
\pgfusepath{clip}%
\pgfsetrectcap%
\pgfsetroundjoin%
\pgfsetlinewidth{1.505625pt}%
\definecolor{currentstroke}{rgb}{0.631373,0.788235,0.956863}%
\pgfsetstrokecolor{currentstroke}%
\pgfsetstrokeopacity{0.800000}%
\pgfsetdash{}{0pt}%
\pgfpathmoveto{\pgfqpoint{2.082714in}{3.055275in}}%
\pgfpathlineto{\pgfqpoint{5.327974in}{4.421066in}}%
\pgfusepath{stroke}%
\end{pgfscope}%
\begin{pgfscope}%
\pgfpathrectangle{\pgfqpoint{0.570343in}{0.331635in}}{\pgfqpoint{9.300000in}{7.700000in}}%
\pgfusepath{clip}%
\pgfsetrectcap%
\pgfsetroundjoin%
\pgfsetlinewidth{1.505625pt}%
\definecolor{currentstroke}{rgb}{0.631373,0.788235,0.956863}%
\pgfsetstrokecolor{currentstroke}%
\pgfsetstrokeopacity{0.800000}%
\pgfsetdash{}{0pt}%
\pgfpathmoveto{\pgfqpoint{2.029248in}{1.578122in}}%
\pgfpathlineto{\pgfqpoint{5.327974in}{4.421066in}}%
\pgfusepath{stroke}%
\end{pgfscope}%
\begin{pgfscope}%
\pgfpathrectangle{\pgfqpoint{0.570343in}{0.331635in}}{\pgfqpoint{9.300000in}{7.700000in}}%
\pgfusepath{clip}%
\pgfsetrectcap%
\pgfsetroundjoin%
\pgfsetlinewidth{1.505625pt}%
\definecolor{currentstroke}{rgb}{0.631373,0.788235,0.956863}%
\pgfsetstrokecolor{currentstroke}%
\pgfsetstrokeopacity{0.800000}%
\pgfsetdash{}{0pt}%
\pgfpathmoveto{\pgfqpoint{5.175896in}{4.017231in}}%
\pgfpathlineto{\pgfqpoint{5.327974in}{4.421066in}}%
\pgfusepath{stroke}%
\end{pgfscope}%
\begin{pgfscope}%
\pgfpathrectangle{\pgfqpoint{0.570343in}{0.331635in}}{\pgfqpoint{9.300000in}{7.700000in}}%
\pgfusepath{clip}%
\pgfsetrectcap%
\pgfsetroundjoin%
\pgfsetlinewidth{1.505625pt}%
\definecolor{currentstroke}{rgb}{0.631373,0.788235,0.956863}%
\pgfsetstrokecolor{currentstroke}%
\pgfsetstrokeopacity{0.800000}%
\pgfsetdash{}{0pt}%
\pgfpathmoveto{\pgfqpoint{3.674629in}{5.541790in}}%
\pgfpathlineto{\pgfqpoint{5.327974in}{4.421066in}}%
\pgfusepath{stroke}%
\end{pgfscope}%
\begin{pgfscope}%
\pgfpathrectangle{\pgfqpoint{0.570343in}{0.331635in}}{\pgfqpoint{9.300000in}{7.700000in}}%
\pgfusepath{clip}%
\pgfsetrectcap%
\pgfsetroundjoin%
\pgfsetlinewidth{1.505625pt}%
\definecolor{currentstroke}{rgb}{0.631373,0.788235,0.956863}%
\pgfsetstrokecolor{currentstroke}%
\pgfsetstrokeopacity{0.800000}%
\pgfsetdash{}{0pt}%
\pgfpathmoveto{\pgfqpoint{1.383929in}{5.039838in}}%
\pgfpathlineto{\pgfqpoint{5.327974in}{4.421066in}}%
\pgfusepath{stroke}%
\end{pgfscope}%
\begin{pgfscope}%
\pgfpathrectangle{\pgfqpoint{0.570343in}{0.331635in}}{\pgfqpoint{9.300000in}{7.700000in}}%
\pgfusepath{clip}%
\pgfsetrectcap%
\pgfsetroundjoin%
\pgfsetlinewidth{1.505625pt}%
\definecolor{currentstroke}{rgb}{0.631373,0.788235,0.956863}%
\pgfsetstrokecolor{currentstroke}%
\pgfsetstrokeopacity{0.800000}%
\pgfsetdash{}{0pt}%
\pgfpathmoveto{\pgfqpoint{3.677550in}{4.774109in}}%
\pgfpathlineto{\pgfqpoint{5.327974in}{4.421066in}}%
\pgfusepath{stroke}%
\end{pgfscope}%
\begin{pgfscope}%
\pgfpathrectangle{\pgfqpoint{0.570343in}{0.331635in}}{\pgfqpoint{9.300000in}{7.700000in}}%
\pgfusepath{clip}%
\pgfsetrectcap%
\pgfsetroundjoin%
\pgfsetlinewidth{1.505625pt}%
\definecolor{currentstroke}{rgb}{0.631373,0.788235,0.956863}%
\pgfsetstrokecolor{currentstroke}%
\pgfsetstrokeopacity{0.800000}%
\pgfsetdash{}{0pt}%
\pgfpathmoveto{\pgfqpoint{6.563827in}{3.646011in}}%
\pgfpathlineto{\pgfqpoint{5.327974in}{4.421066in}}%
\pgfusepath{stroke}%
\end{pgfscope}%
\begin{pgfscope}%
\pgfpathrectangle{\pgfqpoint{0.570343in}{0.331635in}}{\pgfqpoint{9.300000in}{7.700000in}}%
\pgfusepath{clip}%
\pgfsetrectcap%
\pgfsetroundjoin%
\pgfsetlinewidth{1.505625pt}%
\definecolor{currentstroke}{rgb}{0.631373,0.788235,0.956863}%
\pgfsetstrokecolor{currentstroke}%
\pgfsetstrokeopacity{0.800000}%
\pgfsetdash{}{0pt}%
\pgfpathmoveto{\pgfqpoint{4.968812in}{5.909158in}}%
\pgfpathlineto{\pgfqpoint{5.327974in}{4.421066in}}%
\pgfusepath{stroke}%
\end{pgfscope}%
\begin{pgfscope}%
\pgfpathrectangle{\pgfqpoint{0.570343in}{0.331635in}}{\pgfqpoint{9.300000in}{7.700000in}}%
\pgfusepath{clip}%
\pgfsetrectcap%
\pgfsetroundjoin%
\pgfsetlinewidth{1.505625pt}%
\definecolor{currentstroke}{rgb}{0.631373,0.788235,0.956863}%
\pgfsetstrokecolor{currentstroke}%
\pgfsetstrokeopacity{0.800000}%
\pgfsetdash{}{0pt}%
\pgfpathmoveto{\pgfqpoint{5.391700in}{1.666792in}}%
\pgfpathlineto{\pgfqpoint{5.327974in}{4.421066in}}%
\pgfusepath{stroke}%
\end{pgfscope}%
\begin{pgfscope}%
\pgfpathrectangle{\pgfqpoint{0.570343in}{0.331635in}}{\pgfqpoint{9.300000in}{7.700000in}}%
\pgfusepath{clip}%
\pgfsetrectcap%
\pgfsetroundjoin%
\pgfsetlinewidth{1.505625pt}%
\definecolor{currentstroke}{rgb}{0.631373,0.788235,0.956863}%
\pgfsetstrokecolor{currentstroke}%
\pgfsetstrokeopacity{0.800000}%
\pgfsetdash{}{0pt}%
\pgfpathmoveto{\pgfqpoint{7.110525in}{6.212121in}}%
\pgfpathlineto{\pgfqpoint{5.327974in}{4.421066in}}%
\pgfusepath{stroke}%
\end{pgfscope}%
\begin{pgfscope}%
\pgfpathrectangle{\pgfqpoint{0.570343in}{0.331635in}}{\pgfqpoint{9.300000in}{7.700000in}}%
\pgfusepath{clip}%
\pgfsetrectcap%
\pgfsetroundjoin%
\pgfsetlinewidth{1.505625pt}%
\definecolor{currentstroke}{rgb}{0.631373,0.788235,0.956863}%
\pgfsetstrokecolor{currentstroke}%
\pgfsetstrokeopacity{0.800000}%
\pgfsetdash{}{0pt}%
\pgfpathmoveto{\pgfqpoint{3.005187in}{2.565169in}}%
\pgfpathlineto{\pgfqpoint{5.327974in}{4.421066in}}%
\pgfusepath{stroke}%
\end{pgfscope}%
\begin{pgfscope}%
\pgfpathrectangle{\pgfqpoint{0.570343in}{0.331635in}}{\pgfqpoint{9.300000in}{7.700000in}}%
\pgfusepath{clip}%
\pgfsetrectcap%
\pgfsetroundjoin%
\pgfsetlinewidth{1.505625pt}%
\definecolor{currentstroke}{rgb}{0.631373,0.788235,0.956863}%
\pgfsetstrokecolor{currentstroke}%
\pgfsetstrokeopacity{0.800000}%
\pgfsetdash{}{0pt}%
\pgfpathmoveto{\pgfqpoint{4.481129in}{4.398252in}}%
\pgfpathlineto{\pgfqpoint{5.327974in}{4.421066in}}%
\pgfusepath{stroke}%
\end{pgfscope}%
\begin{pgfscope}%
\pgfpathrectangle{\pgfqpoint{0.570343in}{0.331635in}}{\pgfqpoint{9.300000in}{7.700000in}}%
\pgfusepath{clip}%
\pgfsetrectcap%
\pgfsetroundjoin%
\pgfsetlinewidth{1.505625pt}%
\definecolor{currentstroke}{rgb}{0.631373,0.788235,0.956863}%
\pgfsetstrokecolor{currentstroke}%
\pgfsetstrokeopacity{0.800000}%
\pgfsetdash{}{0pt}%
\pgfpathmoveto{\pgfqpoint{6.045238in}{6.416277in}}%
\pgfpathlineto{\pgfqpoint{5.327974in}{4.421066in}}%
\pgfusepath{stroke}%
\end{pgfscope}%
\begin{pgfscope}%
\pgfpathrectangle{\pgfqpoint{0.570343in}{0.331635in}}{\pgfqpoint{9.300000in}{7.700000in}}%
\pgfusepath{clip}%
\pgfsetrectcap%
\pgfsetroundjoin%
\pgfsetlinewidth{1.505625pt}%
\definecolor{currentstroke}{rgb}{0.631373,0.788235,0.956863}%
\pgfsetstrokecolor{currentstroke}%
\pgfsetstrokeopacity{0.800000}%
\pgfsetdash{}{0pt}%
\pgfpathmoveto{\pgfqpoint{9.187938in}{4.095317in}}%
\pgfpathlineto{\pgfqpoint{5.327974in}{4.421066in}}%
\pgfusepath{stroke}%
\end{pgfscope}%
\begin{pgfscope}%
\pgfpathrectangle{\pgfqpoint{0.570343in}{0.331635in}}{\pgfqpoint{9.300000in}{7.700000in}}%
\pgfusepath{clip}%
\pgfsetrectcap%
\pgfsetroundjoin%
\pgfsetlinewidth{1.505625pt}%
\definecolor{currentstroke}{rgb}{0.631373,0.788235,0.956863}%
\pgfsetstrokecolor{currentstroke}%
\pgfsetstrokeopacity{0.800000}%
\pgfsetdash{}{0pt}%
\pgfpathmoveto{\pgfqpoint{7.749803in}{4.498026in}}%
\pgfpathlineto{\pgfqpoint{5.327974in}{4.421066in}}%
\pgfusepath{stroke}%
\end{pgfscope}%
\begin{pgfscope}%
\pgfpathrectangle{\pgfqpoint{0.570343in}{0.331635in}}{\pgfqpoint{9.300000in}{7.700000in}}%
\pgfusepath{clip}%
\pgfsetrectcap%
\pgfsetroundjoin%
\pgfsetlinewidth{1.505625pt}%
\definecolor{currentstroke}{rgb}{1.000000,0.705882,0.509804}%
\pgfsetstrokecolor{currentstroke}%
\pgfsetstrokeopacity{0.800000}%
\pgfsetdash{}{0pt}%
\pgfpathmoveto{\pgfqpoint{5.715735in}{3.335706in}}%
\pgfpathlineto{\pgfqpoint{5.599385in}{3.905858in}}%
\pgfusepath{stroke}%
\end{pgfscope}%
\begin{pgfscope}%
\pgfpathrectangle{\pgfqpoint{0.570343in}{0.331635in}}{\pgfqpoint{9.300000in}{7.700000in}}%
\pgfusepath{clip}%
\pgfsetrectcap%
\pgfsetroundjoin%
\pgfsetlinewidth{1.505625pt}%
\definecolor{currentstroke}{rgb}{1.000000,0.705882,0.509804}%
\pgfsetstrokecolor{currentstroke}%
\pgfsetstrokeopacity{0.800000}%
\pgfsetdash{}{0pt}%
\pgfpathmoveto{\pgfqpoint{7.824153in}{5.963141in}}%
\pgfpathlineto{\pgfqpoint{5.599385in}{3.905858in}}%
\pgfusepath{stroke}%
\end{pgfscope}%
\begin{pgfscope}%
\pgfpathrectangle{\pgfqpoint{0.570343in}{0.331635in}}{\pgfqpoint{9.300000in}{7.700000in}}%
\pgfusepath{clip}%
\pgfsetrectcap%
\pgfsetroundjoin%
\pgfsetlinewidth{1.505625pt}%
\definecolor{currentstroke}{rgb}{1.000000,0.705882,0.509804}%
\pgfsetstrokecolor{currentstroke}%
\pgfsetstrokeopacity{0.800000}%
\pgfsetdash{}{0pt}%
\pgfpathmoveto{\pgfqpoint{5.894391in}{4.130919in}}%
\pgfpathlineto{\pgfqpoint{5.599385in}{3.905858in}}%
\pgfusepath{stroke}%
\end{pgfscope}%
\begin{pgfscope}%
\pgfpathrectangle{\pgfqpoint{0.570343in}{0.331635in}}{\pgfqpoint{9.300000in}{7.700000in}}%
\pgfusepath{clip}%
\pgfsetrectcap%
\pgfsetroundjoin%
\pgfsetlinewidth{1.505625pt}%
\definecolor{currentstroke}{rgb}{1.000000,0.705882,0.509804}%
\pgfsetstrokecolor{currentstroke}%
\pgfsetstrokeopacity{0.800000}%
\pgfsetdash{}{0pt}%
\pgfpathmoveto{\pgfqpoint{3.273962in}{4.230448in}}%
\pgfpathlineto{\pgfqpoint{5.599385in}{3.905858in}}%
\pgfusepath{stroke}%
\end{pgfscope}%
\begin{pgfscope}%
\pgfpathrectangle{\pgfqpoint{0.570343in}{0.331635in}}{\pgfqpoint{9.300000in}{7.700000in}}%
\pgfusepath{clip}%
\pgfsetrectcap%
\pgfsetroundjoin%
\pgfsetlinewidth{1.505625pt}%
\definecolor{currentstroke}{rgb}{1.000000,0.705882,0.509804}%
\pgfsetstrokecolor{currentstroke}%
\pgfsetstrokeopacity{0.800000}%
\pgfsetdash{}{0pt}%
\pgfpathmoveto{\pgfqpoint{4.485603in}{2.524014in}}%
\pgfpathlineto{\pgfqpoint{5.599385in}{3.905858in}}%
\pgfusepath{stroke}%
\end{pgfscope}%
\begin{pgfscope}%
\pgfpathrectangle{\pgfqpoint{0.570343in}{0.331635in}}{\pgfqpoint{9.300000in}{7.700000in}}%
\pgfusepath{clip}%
\pgfsetrectcap%
\pgfsetroundjoin%
\pgfsetlinewidth{1.505625pt}%
\definecolor{currentstroke}{rgb}{1.000000,0.705882,0.509804}%
\pgfsetstrokecolor{currentstroke}%
\pgfsetstrokeopacity{0.800000}%
\pgfsetdash{}{0pt}%
\pgfpathmoveto{\pgfqpoint{6.572664in}{5.239286in}}%
\pgfpathlineto{\pgfqpoint{5.599385in}{3.905858in}}%
\pgfusepath{stroke}%
\end{pgfscope}%
\begin{pgfscope}%
\pgfpathrectangle{\pgfqpoint{0.570343in}{0.331635in}}{\pgfqpoint{9.300000in}{7.700000in}}%
\pgfusepath{clip}%
\pgfsetrectcap%
\pgfsetroundjoin%
\pgfsetlinewidth{1.505625pt}%
\definecolor{currentstroke}{rgb}{1.000000,0.705882,0.509804}%
\pgfsetstrokecolor{currentstroke}%
\pgfsetstrokeopacity{0.800000}%
\pgfsetdash{}{0pt}%
\pgfpathmoveto{\pgfqpoint{7.581009in}{1.761446in}}%
\pgfpathlineto{\pgfqpoint{5.599385in}{3.905858in}}%
\pgfusepath{stroke}%
\end{pgfscope}%
\begin{pgfscope}%
\pgfpathrectangle{\pgfqpoint{0.570343in}{0.331635in}}{\pgfqpoint{9.300000in}{7.700000in}}%
\pgfusepath{clip}%
\pgfsetrectcap%
\pgfsetroundjoin%
\pgfsetlinewidth{1.505625pt}%
\definecolor{currentstroke}{rgb}{1.000000,0.705882,0.509804}%
\pgfsetstrokecolor{currentstroke}%
\pgfsetstrokeopacity{0.800000}%
\pgfsetdash{}{0pt}%
\pgfpathmoveto{\pgfqpoint{1.757394in}{6.526560in}}%
\pgfpathlineto{\pgfqpoint{5.599385in}{3.905858in}}%
\pgfusepath{stroke}%
\end{pgfscope}%
\begin{pgfscope}%
\pgfpathrectangle{\pgfqpoint{0.570343in}{0.331635in}}{\pgfqpoint{9.300000in}{7.700000in}}%
\pgfusepath{clip}%
\pgfsetrectcap%
\pgfsetroundjoin%
\pgfsetlinewidth{1.505625pt}%
\definecolor{currentstroke}{rgb}{1.000000,0.705882,0.509804}%
\pgfsetstrokecolor{currentstroke}%
\pgfsetstrokeopacity{0.800000}%
\pgfsetdash{}{0pt}%
\pgfpathmoveto{\pgfqpoint{5.468750in}{7.350048in}}%
\pgfpathlineto{\pgfqpoint{5.599385in}{3.905858in}}%
\pgfusepath{stroke}%
\end{pgfscope}%
\begin{pgfscope}%
\pgfpathrectangle{\pgfqpoint{0.570343in}{0.331635in}}{\pgfqpoint{9.300000in}{7.700000in}}%
\pgfusepath{clip}%
\pgfsetrectcap%
\pgfsetroundjoin%
\pgfsetlinewidth{1.505625pt}%
\definecolor{currentstroke}{rgb}{1.000000,0.705882,0.509804}%
\pgfsetstrokecolor{currentstroke}%
\pgfsetstrokeopacity{0.800000}%
\pgfsetdash{}{0pt}%
\pgfpathmoveto{\pgfqpoint{4.136121in}{6.404828in}}%
\pgfpathlineto{\pgfqpoint{5.599385in}{3.905858in}}%
\pgfusepath{stroke}%
\end{pgfscope}%
\begin{pgfscope}%
\pgfpathrectangle{\pgfqpoint{0.570343in}{0.331635in}}{\pgfqpoint{9.300000in}{7.700000in}}%
\pgfusepath{clip}%
\pgfsetrectcap%
\pgfsetroundjoin%
\pgfsetlinewidth{1.505625pt}%
\definecolor{currentstroke}{rgb}{1.000000,0.705882,0.509804}%
\pgfsetstrokecolor{currentstroke}%
\pgfsetstrokeopacity{0.800000}%
\pgfsetdash{}{0pt}%
\pgfpathmoveto{\pgfqpoint{3.063964in}{3.577482in}}%
\pgfpathlineto{\pgfqpoint{5.599385in}{3.905858in}}%
\pgfusepath{stroke}%
\end{pgfscope}%
\begin{pgfscope}%
\pgfpathrectangle{\pgfqpoint{0.570343in}{0.331635in}}{\pgfqpoint{9.300000in}{7.700000in}}%
\pgfusepath{clip}%
\pgfsetrectcap%
\pgfsetroundjoin%
\pgfsetlinewidth{1.505625pt}%
\definecolor{currentstroke}{rgb}{1.000000,0.705882,0.509804}%
\pgfsetstrokecolor{currentstroke}%
\pgfsetstrokeopacity{0.800000}%
\pgfsetdash{}{0pt}%
\pgfpathmoveto{\pgfqpoint{3.378790in}{1.225414in}}%
\pgfpathlineto{\pgfqpoint{5.599385in}{3.905858in}}%
\pgfusepath{stroke}%
\end{pgfscope}%
\begin{pgfscope}%
\pgfpathrectangle{\pgfqpoint{0.570343in}{0.331635in}}{\pgfqpoint{9.300000in}{7.700000in}}%
\pgfusepath{clip}%
\pgfsetrectcap%
\pgfsetroundjoin%
\pgfsetlinewidth{1.505625pt}%
\definecolor{currentstroke}{rgb}{1.000000,0.705882,0.509804}%
\pgfsetstrokecolor{currentstroke}%
\pgfsetstrokeopacity{0.800000}%
\pgfsetdash{}{0pt}%
\pgfpathmoveto{\pgfqpoint{0.993071in}{1.778552in}}%
\pgfpathlineto{\pgfqpoint{5.599385in}{3.905858in}}%
\pgfusepath{stroke}%
\end{pgfscope}%
\begin{pgfscope}%
\pgfpathrectangle{\pgfqpoint{0.570343in}{0.331635in}}{\pgfqpoint{9.300000in}{7.700000in}}%
\pgfusepath{clip}%
\pgfsetrectcap%
\pgfsetroundjoin%
\pgfsetlinewidth{1.505625pt}%
\definecolor{currentstroke}{rgb}{1.000000,0.705882,0.509804}%
\pgfsetstrokecolor{currentstroke}%
\pgfsetstrokeopacity{0.800000}%
\pgfsetdash{}{0pt}%
\pgfpathmoveto{\pgfqpoint{5.153229in}{2.657489in}}%
\pgfpathlineto{\pgfqpoint{5.599385in}{3.905858in}}%
\pgfusepath{stroke}%
\end{pgfscope}%
\begin{pgfscope}%
\pgfpathrectangle{\pgfqpoint{0.570343in}{0.331635in}}{\pgfqpoint{9.300000in}{7.700000in}}%
\pgfusepath{clip}%
\pgfsetrectcap%
\pgfsetroundjoin%
\pgfsetlinewidth{1.505625pt}%
\definecolor{currentstroke}{rgb}{1.000000,0.705882,0.509804}%
\pgfsetstrokecolor{currentstroke}%
\pgfsetstrokeopacity{0.800000}%
\pgfsetdash{}{0pt}%
\pgfpathmoveto{\pgfqpoint{8.561752in}{3.104281in}}%
\pgfpathlineto{\pgfqpoint{5.599385in}{3.905858in}}%
\pgfusepath{stroke}%
\end{pgfscope}%
\begin{pgfscope}%
\pgfpathrectangle{\pgfqpoint{0.570343in}{0.331635in}}{\pgfqpoint{9.300000in}{7.700000in}}%
\pgfusepath{clip}%
\pgfsetrectcap%
\pgfsetroundjoin%
\pgfsetlinewidth{1.505625pt}%
\definecolor{currentstroke}{rgb}{1.000000,0.705882,0.509804}%
\pgfsetstrokecolor{currentstroke}%
\pgfsetstrokeopacity{0.800000}%
\pgfsetdash{}{0pt}%
\pgfpathmoveto{\pgfqpoint{4.047696in}{1.870260in}}%
\pgfpathlineto{\pgfqpoint{5.599385in}{3.905858in}}%
\pgfusepath{stroke}%
\end{pgfscope}%
\begin{pgfscope}%
\pgfpathrectangle{\pgfqpoint{0.570343in}{0.331635in}}{\pgfqpoint{9.300000in}{7.700000in}}%
\pgfusepath{clip}%
\pgfsetrectcap%
\pgfsetroundjoin%
\pgfsetlinewidth{1.505625pt}%
\definecolor{currentstroke}{rgb}{1.000000,0.705882,0.509804}%
\pgfsetstrokecolor{currentstroke}%
\pgfsetstrokeopacity{0.800000}%
\pgfsetdash{}{0pt}%
\pgfpathmoveto{\pgfqpoint{8.305186in}{3.961163in}}%
\pgfpathlineto{\pgfqpoint{5.599385in}{3.905858in}}%
\pgfusepath{stroke}%
\end{pgfscope}%
\begin{pgfscope}%
\pgfpathrectangle{\pgfqpoint{0.570343in}{0.331635in}}{\pgfqpoint{9.300000in}{7.700000in}}%
\pgfusepath{clip}%
\pgfsetrectcap%
\pgfsetroundjoin%
\pgfsetlinewidth{1.505625pt}%
\definecolor{currentstroke}{rgb}{1.000000,0.705882,0.509804}%
\pgfsetstrokecolor{currentstroke}%
\pgfsetstrokeopacity{0.800000}%
\pgfsetdash{}{0pt}%
\pgfpathmoveto{\pgfqpoint{4.619077in}{0.681635in}}%
\pgfpathlineto{\pgfqpoint{5.599385in}{3.905858in}}%
\pgfusepath{stroke}%
\end{pgfscope}%
\begin{pgfscope}%
\pgfpathrectangle{\pgfqpoint{0.570343in}{0.331635in}}{\pgfqpoint{9.300000in}{7.700000in}}%
\pgfusepath{clip}%
\pgfsetrectcap%
\pgfsetroundjoin%
\pgfsetlinewidth{1.505625pt}%
\definecolor{currentstroke}{rgb}{1.000000,0.705882,0.509804}%
\pgfsetstrokecolor{currentstroke}%
\pgfsetstrokeopacity{0.800000}%
\pgfsetdash{}{0pt}%
\pgfpathmoveto{\pgfqpoint{8.308279in}{7.681635in}}%
\pgfpathlineto{\pgfqpoint{5.599385in}{3.905858in}}%
\pgfusepath{stroke}%
\end{pgfscope}%
\begin{pgfscope}%
\pgfpathrectangle{\pgfqpoint{0.570343in}{0.331635in}}{\pgfqpoint{9.300000in}{7.700000in}}%
\pgfusepath{clip}%
\pgfsetrectcap%
\pgfsetroundjoin%
\pgfsetlinewidth{1.505625pt}%
\definecolor{currentstroke}{rgb}{1.000000,0.705882,0.509804}%
\pgfsetstrokecolor{currentstroke}%
\pgfsetstrokeopacity{0.800000}%
\pgfsetdash{}{0pt}%
\pgfpathmoveto{\pgfqpoint{6.197001in}{1.259168in}}%
\pgfpathlineto{\pgfqpoint{5.599385in}{3.905858in}}%
\pgfusepath{stroke}%
\end{pgfscope}%
\begin{pgfscope}%
\pgfpathrectangle{\pgfqpoint{0.570343in}{0.331635in}}{\pgfqpoint{9.300000in}{7.700000in}}%
\pgfusepath{clip}%
\pgfsetrectcap%
\pgfsetroundjoin%
\pgfsetlinewidth{1.505625pt}%
\definecolor{currentstroke}{rgb}{1.000000,0.705882,0.509804}%
\pgfsetstrokecolor{currentstroke}%
\pgfsetstrokeopacity{0.800000}%
\pgfsetdash{}{0pt}%
\pgfpathmoveto{\pgfqpoint{9.447616in}{5.125125in}}%
\pgfpathlineto{\pgfqpoint{5.599385in}{3.905858in}}%
\pgfusepath{stroke}%
\end{pgfscope}%
\begin{pgfscope}%
\pgfpathrectangle{\pgfqpoint{0.570343in}{0.331635in}}{\pgfqpoint{9.300000in}{7.700000in}}%
\pgfusepath{clip}%
\pgfsetrectcap%
\pgfsetroundjoin%
\pgfsetlinewidth{1.505625pt}%
\definecolor{currentstroke}{rgb}{1.000000,0.705882,0.509804}%
\pgfsetstrokecolor{currentstroke}%
\pgfsetstrokeopacity{0.800000}%
\pgfsetdash{}{0pt}%
\pgfpathmoveto{\pgfqpoint{6.034006in}{2.439042in}}%
\pgfpathlineto{\pgfqpoint{5.599385in}{3.905858in}}%
\pgfusepath{stroke}%
\end{pgfscope}%
\begin{pgfscope}%
\pgfpathrectangle{\pgfqpoint{0.570343in}{0.331635in}}{\pgfqpoint{9.300000in}{7.700000in}}%
\pgfusepath{clip}%
\pgfsetrectcap%
\pgfsetroundjoin%
\pgfsetlinewidth{1.505625pt}%
\definecolor{currentstroke}{rgb}{1.000000,0.705882,0.509804}%
\pgfsetstrokecolor{currentstroke}%
\pgfsetstrokeopacity{0.800000}%
\pgfsetdash{}{0pt}%
\pgfpathmoveto{\pgfqpoint{8.815623in}{2.199204in}}%
\pgfpathlineto{\pgfqpoint{5.599385in}{3.905858in}}%
\pgfusepath{stroke}%
\end{pgfscope}%
\begin{pgfscope}%
\pgfpathrectangle{\pgfqpoint{0.570343in}{0.331635in}}{\pgfqpoint{9.300000in}{7.700000in}}%
\pgfusepath{clip}%
\pgfsetrectcap%
\pgfsetroundjoin%
\pgfsetlinewidth{1.505625pt}%
\definecolor{currentstroke}{rgb}{1.000000,0.705882,0.509804}%
\pgfsetstrokecolor{currentstroke}%
\pgfsetstrokeopacity{0.800000}%
\pgfsetdash{}{0pt}%
\pgfpathmoveto{\pgfqpoint{7.488123in}{3.719720in}}%
\pgfpathlineto{\pgfqpoint{5.599385in}{3.905858in}}%
\pgfusepath{stroke}%
\end{pgfscope}%
\begin{pgfscope}%
\pgfpathrectangle{\pgfqpoint{0.570343in}{0.331635in}}{\pgfqpoint{9.300000in}{7.700000in}}%
\pgfusepath{clip}%
\pgfsetrectcap%
\pgfsetroundjoin%
\pgfsetlinewidth{1.505625pt}%
\definecolor{currentstroke}{rgb}{1.000000,0.705882,0.509804}%
\pgfsetstrokecolor{currentstroke}%
\pgfsetstrokeopacity{0.800000}%
\pgfsetdash{}{0pt}%
\pgfpathmoveto{\pgfqpoint{2.671621in}{4.844973in}}%
\pgfpathlineto{\pgfqpoint{5.599385in}{3.905858in}}%
\pgfusepath{stroke}%
\end{pgfscope}%
\begin{pgfscope}%
\pgfpathrectangle{\pgfqpoint{0.570343in}{0.331635in}}{\pgfqpoint{9.300000in}{7.700000in}}%
\pgfusepath{clip}%
\pgfsetrectcap%
\pgfsetroundjoin%
\pgfsetlinewidth{1.505625pt}%
\definecolor{currentstroke}{rgb}{1.000000,0.705882,0.509804}%
\pgfsetstrokecolor{currentstroke}%
\pgfsetstrokeopacity{0.800000}%
\pgfsetdash{}{0pt}%
\pgfpathmoveto{\pgfqpoint{7.154996in}{7.029196in}}%
\pgfpathlineto{\pgfqpoint{5.599385in}{3.905858in}}%
\pgfusepath{stroke}%
\end{pgfscope}%
\begin{pgfscope}%
\pgfpathrectangle{\pgfqpoint{0.570343in}{0.331635in}}{\pgfqpoint{9.300000in}{7.700000in}}%
\pgfusepath{clip}%
\pgfsetrectcap%
\pgfsetroundjoin%
\pgfsetlinewidth{1.505625pt}%
\definecolor{currentstroke}{rgb}{1.000000,0.705882,0.509804}%
\pgfsetstrokecolor{currentstroke}%
\pgfsetstrokeopacity{0.800000}%
\pgfsetdash{}{0pt}%
\pgfpathmoveto{\pgfqpoint{3.897376in}{3.003663in}}%
\pgfpathlineto{\pgfqpoint{5.599385in}{3.905858in}}%
\pgfusepath{stroke}%
\end{pgfscope}%
\begin{pgfscope}%
\pgfpathrectangle{\pgfqpoint{0.570343in}{0.331635in}}{\pgfqpoint{9.300000in}{7.700000in}}%
\pgfusepath{clip}%
\pgfsetrectcap%
\pgfsetroundjoin%
\pgfsetlinewidth{1.505625pt}%
\definecolor{currentstroke}{rgb}{1.000000,0.705882,0.509804}%
\pgfsetstrokecolor{currentstroke}%
\pgfsetstrokeopacity{0.800000}%
\pgfsetdash{}{0pt}%
\pgfpathmoveto{\pgfqpoint{5.935607in}{5.739614in}}%
\pgfpathlineto{\pgfqpoint{5.599385in}{3.905858in}}%
\pgfusepath{stroke}%
\end{pgfscope}%
\begin{pgfscope}%
\pgfsetrectcap%
\pgfsetmiterjoin%
\pgfsetlinewidth{0.803000pt}%
\definecolor{currentstroke}{rgb}{0.000000,0.000000,0.000000}%
\pgfsetstrokecolor{currentstroke}%
\pgfsetdash{}{0pt}%
\pgfpathmoveto{\pgfqpoint{0.570343in}{0.331635in}}%
\pgfpathlineto{\pgfqpoint{0.570343in}{8.031635in}}%
\pgfusepath{stroke}%
\end{pgfscope}%
\begin{pgfscope}%
\pgfsetrectcap%
\pgfsetmiterjoin%
\pgfsetlinewidth{0.803000pt}%
\definecolor{currentstroke}{rgb}{0.000000,0.000000,0.000000}%
\pgfsetstrokecolor{currentstroke}%
\pgfsetdash{}{0pt}%
\pgfpathmoveto{\pgfqpoint{9.870343in}{0.331635in}}%
\pgfpathlineto{\pgfqpoint{9.870343in}{8.031635in}}%
\pgfusepath{stroke}%
\end{pgfscope}%
\begin{pgfscope}%
\pgfsetrectcap%
\pgfsetmiterjoin%
\pgfsetlinewidth{0.803000pt}%
\definecolor{currentstroke}{rgb}{0.000000,0.000000,0.000000}%
\pgfsetstrokecolor{currentstroke}%
\pgfsetdash{}{0pt}%
\pgfpathmoveto{\pgfqpoint{0.570343in}{0.331635in}}%
\pgfpathlineto{\pgfqpoint{9.870343in}{0.331635in}}%
\pgfusepath{stroke}%
\end{pgfscope}%
\begin{pgfscope}%
\pgfsetrectcap%
\pgfsetmiterjoin%
\pgfsetlinewidth{0.803000pt}%
\definecolor{currentstroke}{rgb}{0.000000,0.000000,0.000000}%
\pgfsetstrokecolor{currentstroke}%
\pgfsetdash{}{0pt}%
\pgfpathmoveto{\pgfqpoint{0.570343in}{8.031635in}}%
\pgfpathlineto{\pgfqpoint{9.870343in}{8.031635in}}%
\pgfusepath{stroke}%
\end{pgfscope}%
\begin{pgfscope}%
\definecolor{textcolor}{rgb}{0.000000,0.000000,0.000000}%
\pgfsetstrokecolor{textcolor}%
\pgfsetfillcolor{textcolor}%
\pgftext[x=5.220343in,y=8.114968in,,base]{\color{textcolor}\sffamily\fontsize{12.000000}{14.400000}\selectfont Photo-Realistic Images}%
\end{pgfscope}%
\begin{pgfscope}%
\pgfsetbuttcap%
\pgfsetmiterjoin%
\definecolor{currentfill}{rgb}{1.000000,1.000000,1.000000}%
\pgfsetfillcolor{currentfill}%
\pgfsetfillopacity{0.800000}%
\pgfsetlinewidth{1.003750pt}%
\definecolor{currentstroke}{rgb}{0.800000,0.800000,0.800000}%
\pgfsetstrokecolor{currentstroke}%
\pgfsetstrokeopacity{0.800000}%
\pgfsetdash{}{0pt}%
\pgfpathmoveto{\pgfqpoint{9.967566in}{3.956944in}}%
\pgfpathlineto{\pgfqpoint{11.121442in}{3.956944in}}%
\pgfpathquadraticcurveto{\pgfqpoint{11.149220in}{3.956944in}}{\pgfqpoint{11.149220in}{3.984722in}}%
\pgfpathlineto{\pgfqpoint{11.149220in}{4.378548in}}%
\pgfpathquadraticcurveto{\pgfqpoint{11.149220in}{4.406326in}}{\pgfqpoint{11.121442in}{4.406326in}}%
\pgfpathlineto{\pgfqpoint{9.967566in}{4.406326in}}%
\pgfpathquadraticcurveto{\pgfqpoint{9.939788in}{4.406326in}}{\pgfqpoint{9.939788in}{4.378548in}}%
\pgfpathlineto{\pgfqpoint{9.939788in}{3.984722in}}%
\pgfpathquadraticcurveto{\pgfqpoint{9.939788in}{3.956944in}}{\pgfqpoint{9.967566in}{3.956944in}}%
\pgfpathclose%
\pgfusepath{stroke,fill}%
\end{pgfscope}%
\begin{pgfscope}%
\pgfsetbuttcap%
\pgfsetroundjoin%
\definecolor{currentfill}{rgb}{0.631373,0.788235,0.956863}%
\pgfsetfillcolor{currentfill}%
\pgfsetlinewidth{1.003750pt}%
\definecolor{currentstroke}{rgb}{0.631373,0.788235,0.956863}%
\pgfsetstrokecolor{currentstroke}%
\pgfsetdash{}{0pt}%
\pgfsys@defobject{currentmarker}{\pgfqpoint{-0.041667in}{-0.041667in}}{\pgfqpoint{0.041667in}{0.041667in}}{%
\pgfpathmoveto{\pgfqpoint{0.000000in}{-0.041667in}}%
\pgfpathcurveto{\pgfqpoint{0.011050in}{-0.041667in}}{\pgfqpoint{0.021649in}{-0.037276in}}{\pgfqpoint{0.029463in}{-0.029463in}}%
\pgfpathcurveto{\pgfqpoint{0.037276in}{-0.021649in}}{\pgfqpoint{0.041667in}{-0.011050in}}{\pgfqpoint{0.041667in}{0.000000in}}%
\pgfpathcurveto{\pgfqpoint{0.041667in}{0.011050in}}{\pgfqpoint{0.037276in}{0.021649in}}{\pgfqpoint{0.029463in}{0.029463in}}%
\pgfpathcurveto{\pgfqpoint{0.021649in}{0.037276in}}{\pgfqpoint{0.011050in}{0.041667in}}{\pgfqpoint{0.000000in}{0.041667in}}%
\pgfpathcurveto{\pgfqpoint{-0.011050in}{0.041667in}}{\pgfqpoint{-0.021649in}{0.037276in}}{\pgfqpoint{-0.029463in}{0.029463in}}%
\pgfpathcurveto{\pgfqpoint{-0.037276in}{0.021649in}}{\pgfqpoint{-0.041667in}{0.011050in}}{\pgfqpoint{-0.041667in}{0.000000in}}%
\pgfpathcurveto{\pgfqpoint{-0.041667in}{-0.011050in}}{\pgfqpoint{-0.037276in}{-0.021649in}}{\pgfqpoint{-0.029463in}{-0.029463in}}%
\pgfpathcurveto{\pgfqpoint{-0.021649in}{-0.037276in}}{\pgfqpoint{-0.011050in}{-0.041667in}}{\pgfqpoint{0.000000in}{-0.041667in}}%
\pgfpathclose%
\pgfusepath{stroke,fill}%
}%
\begin{pgfscope}%
\pgfsys@transformshift{10.134232in}{4.281705in}%
\pgfsys@useobject{currentmarker}{}%
\end{pgfscope}%
\end{pgfscope}%
\begin{pgfscope}%
\definecolor{textcolor}{rgb}{0.000000,0.000000,0.000000}%
\pgfsetstrokecolor{textcolor}%
\pgfsetfillcolor{textcolor}%
\pgftext[x=10.384232in,y=4.245247in,left,base]{\color{textcolor}\sffamily\fontsize{10.000000}{12.000000}\selectfont s2r-3dfree}%
\end{pgfscope}%
\begin{pgfscope}%
\pgfsetbuttcap%
\pgfsetroundjoin%
\definecolor{currentfill}{rgb}{1.000000,0.705882,0.509804}%
\pgfsetfillcolor{currentfill}%
\pgfsetlinewidth{1.003750pt}%
\definecolor{currentstroke}{rgb}{1.000000,0.705882,0.509804}%
\pgfsetstrokecolor{currentstroke}%
\pgfsetdash{}{0pt}%
\pgfsys@defobject{currentmarker}{\pgfqpoint{-0.041667in}{-0.041667in}}{\pgfqpoint{0.041667in}{0.041667in}}{%
\pgfpathmoveto{\pgfqpoint{0.000000in}{-0.041667in}}%
\pgfpathcurveto{\pgfqpoint{0.011050in}{-0.041667in}}{\pgfqpoint{0.021649in}{-0.037276in}}{\pgfqpoint{0.029463in}{-0.029463in}}%
\pgfpathcurveto{\pgfqpoint{0.037276in}{-0.021649in}}{\pgfqpoint{0.041667in}{-0.011050in}}{\pgfqpoint{0.041667in}{0.000000in}}%
\pgfpathcurveto{\pgfqpoint{0.041667in}{0.011050in}}{\pgfqpoint{0.037276in}{0.021649in}}{\pgfqpoint{0.029463in}{0.029463in}}%
\pgfpathcurveto{\pgfqpoint{0.021649in}{0.037276in}}{\pgfqpoint{0.011050in}{0.041667in}}{\pgfqpoint{0.000000in}{0.041667in}}%
\pgfpathcurveto{\pgfqpoint{-0.011050in}{0.041667in}}{\pgfqpoint{-0.021649in}{0.037276in}}{\pgfqpoint{-0.029463in}{0.029463in}}%
\pgfpathcurveto{\pgfqpoint{-0.037276in}{0.021649in}}{\pgfqpoint{-0.041667in}{0.011050in}}{\pgfqpoint{-0.041667in}{0.000000in}}%
\pgfpathcurveto{\pgfqpoint{-0.041667in}{-0.011050in}}{\pgfqpoint{-0.037276in}{-0.021649in}}{\pgfqpoint{-0.029463in}{-0.029463in}}%
\pgfpathcurveto{\pgfqpoint{-0.021649in}{-0.037276in}}{\pgfqpoint{-0.011050in}{-0.041667in}}{\pgfqpoint{0.000000in}{-0.041667in}}%
\pgfpathclose%
\pgfusepath{stroke,fill}%
}%
\begin{pgfscope}%
\pgfsys@transformshift{10.134232in}{4.077848in}%
\pgfsys@useobject{currentmarker}{}%
\end{pgfscope}%
\end{pgfscope}%
\begin{pgfscope}%
\definecolor{textcolor}{rgb}{0.000000,0.000000,0.000000}%
\pgfsetstrokecolor{textcolor}%
\pgfsetfillcolor{textcolor}%
\pgftext[x=10.384232in,y=4.041390in,left,base]{\color{textcolor}\sffamily\fontsize{10.000000}{12.000000}\selectfont pix3d}%
\end{pgfscope}%
\end{pgfpicture}%
\makeatother%
\endgroup%
}\\
%    \resizebox{0.49\linewidth}{5cm}{%% Creator: Matplotlib, PGF backend
%%
%% To include the figure in your LaTeX document, write
%%   \input{<filename>.pgf}
%%
%% Make sure the required packages are loaded in your preamble
%%   \usepackage{pgf}
%%
%% Figures using additional raster images can only be included by \input if
%% they are in the same directory as the main LaTeX file. For loading figures
%% from other directories you can use the `import` package
%%   \usepackage{import}
%%
%% and then include the figures with
%%   \import{<path to file>}{<filename>.pgf}
%%
%% Matplotlib used the following preamble
%%   \usepackage{fontspec}
%%   \setmainfont{DejaVuSerif.ttf}[Path=\detokenize{/Users/apple/opt/anaconda3/envs/kaolin/lib/python3.7/site-packages/matplotlib/mpl-data/fonts/ttf/}]
%%   \setsansfont{DejaVuSans.ttf}[Path=\detokenize{/Users/apple/opt/anaconda3/envs/kaolin/lib/python3.7/site-packages/matplotlib/mpl-data/fonts/ttf/}]
%%   \setmonofont{DejaVuSansMono.ttf}[Path=\detokenize{/Users/apple/opt/anaconda3/envs/kaolin/lib/python3.7/site-packages/matplotlib/mpl-data/fonts/ttf/}]
%%
\begingroup%
\makeatletter%
\begin{pgfpicture}%
\pgfpathrectangle{\pgfpointorigin}{\pgfqpoint{5.630343in}{4.337596in}}%
\pgfusepath{use as bounding box, clip}%
\begin{pgfscope}%
\pgfsetbuttcap%
\pgfsetmiterjoin%
\definecolor{currentfill}{rgb}{1.000000,1.000000,1.000000}%
\pgfsetfillcolor{currentfill}%
\pgfsetlinewidth{0.000000pt}%
\definecolor{currentstroke}{rgb}{1.000000,1.000000,1.000000}%
\pgfsetstrokecolor{currentstroke}%
\pgfsetdash{}{0pt}%
\pgfpathmoveto{\pgfqpoint{0.000000in}{0.000000in}}%
\pgfpathlineto{\pgfqpoint{5.630343in}{0.000000in}}%
\pgfpathlineto{\pgfqpoint{5.630343in}{4.337596in}}%
\pgfpathlineto{\pgfqpoint{0.000000in}{4.337596in}}%
\pgfpathclose%
\pgfusepath{fill}%
\end{pgfscope}%
\begin{pgfscope}%
\pgfsetbuttcap%
\pgfsetmiterjoin%
\definecolor{currentfill}{rgb}{1.000000,1.000000,1.000000}%
\pgfsetfillcolor{currentfill}%
\pgfsetlinewidth{0.000000pt}%
\definecolor{currentstroke}{rgb}{0.000000,0.000000,0.000000}%
\pgfsetstrokecolor{currentstroke}%
\pgfsetstrokeopacity{0.000000}%
\pgfsetdash{}{0pt}%
\pgfpathmoveto{\pgfqpoint{0.570343in}{0.331635in}}%
\pgfpathlineto{\pgfqpoint{5.530343in}{0.331635in}}%
\pgfpathlineto{\pgfqpoint{5.530343in}{4.027635in}}%
\pgfpathlineto{\pgfqpoint{0.570343in}{4.027635in}}%
\pgfpathclose%
\pgfusepath{fill}%
\end{pgfscope}%
\begin{pgfscope}%
\pgfpathrectangle{\pgfqpoint{0.570343in}{0.331635in}}{\pgfqpoint{4.960000in}{3.696000in}}%
\pgfusepath{clip}%
\pgfsetbuttcap%
\pgfsetroundjoin%
\definecolor{currentfill}{rgb}{0.631373,0.788235,0.956863}%
\pgfsetfillcolor{currentfill}%
\pgfsetlinewidth{0.481800pt}%
\definecolor{currentstroke}{rgb}{1.000000,1.000000,1.000000}%
\pgfsetstrokecolor{currentstroke}%
\pgfsetdash{}{0pt}%
\pgfpathmoveto{\pgfqpoint{3.009597in}{1.331461in}}%
\pgfpathcurveto{\pgfqpoint{3.020647in}{1.331461in}}{\pgfqpoint{3.031246in}{1.335851in}}{\pgfqpoint{3.039060in}{1.343665in}}%
\pgfpathcurveto{\pgfqpoint{3.046874in}{1.351479in}}{\pgfqpoint{3.051264in}{1.362078in}}{\pgfqpoint{3.051264in}{1.373128in}}%
\pgfpathcurveto{\pgfqpoint{3.051264in}{1.384178in}}{\pgfqpoint{3.046874in}{1.394777in}}{\pgfqpoint{3.039060in}{1.402590in}}%
\pgfpathcurveto{\pgfqpoint{3.031246in}{1.410404in}}{\pgfqpoint{3.020647in}{1.414794in}}{\pgfqpoint{3.009597in}{1.414794in}}%
\pgfpathcurveto{\pgfqpoint{2.998547in}{1.414794in}}{\pgfqpoint{2.987948in}{1.410404in}}{\pgfqpoint{2.980135in}{1.402590in}}%
\pgfpathcurveto{\pgfqpoint{2.972321in}{1.394777in}}{\pgfqpoint{2.967931in}{1.384178in}}{\pgfqpoint{2.967931in}{1.373128in}}%
\pgfpathcurveto{\pgfqpoint{2.967931in}{1.362078in}}{\pgfqpoint{2.972321in}{1.351479in}}{\pgfqpoint{2.980135in}{1.343665in}}%
\pgfpathcurveto{\pgfqpoint{2.987948in}{1.335851in}}{\pgfqpoint{2.998547in}{1.331461in}}{\pgfqpoint{3.009597in}{1.331461in}}%
\pgfpathclose%
\pgfusepath{stroke,fill}%
\end{pgfscope}%
\begin{pgfscope}%
\pgfpathrectangle{\pgfqpoint{0.570343in}{0.331635in}}{\pgfqpoint{4.960000in}{3.696000in}}%
\pgfusepath{clip}%
\pgfsetbuttcap%
\pgfsetroundjoin%
\definecolor{currentfill}{rgb}{0.631373,0.788235,0.956863}%
\pgfsetfillcolor{currentfill}%
\pgfsetlinewidth{0.481800pt}%
\definecolor{currentstroke}{rgb}{1.000000,1.000000,1.000000}%
\pgfsetstrokecolor{currentstroke}%
\pgfsetdash{}{0pt}%
\pgfpathmoveto{\pgfqpoint{4.747431in}{2.652073in}}%
\pgfpathcurveto{\pgfqpoint{4.758481in}{2.652073in}}{\pgfqpoint{4.769080in}{2.656463in}}{\pgfqpoint{4.776894in}{2.664277in}}%
\pgfpathcurveto{\pgfqpoint{4.784707in}{2.672090in}}{\pgfqpoint{4.789097in}{2.682689in}}{\pgfqpoint{4.789097in}{2.693739in}}%
\pgfpathcurveto{\pgfqpoint{4.789097in}{2.704790in}}{\pgfqpoint{4.784707in}{2.715389in}}{\pgfqpoint{4.776894in}{2.723202in}}%
\pgfpathcurveto{\pgfqpoint{4.769080in}{2.731016in}}{\pgfqpoint{4.758481in}{2.735406in}}{\pgfqpoint{4.747431in}{2.735406in}}%
\pgfpathcurveto{\pgfqpoint{4.736381in}{2.735406in}}{\pgfqpoint{4.725782in}{2.731016in}}{\pgfqpoint{4.717968in}{2.723202in}}%
\pgfpathcurveto{\pgfqpoint{4.710154in}{2.715389in}}{\pgfqpoint{4.705764in}{2.704790in}}{\pgfqpoint{4.705764in}{2.693739in}}%
\pgfpathcurveto{\pgfqpoint{4.705764in}{2.682689in}}{\pgfqpoint{4.710154in}{2.672090in}}{\pgfqpoint{4.717968in}{2.664277in}}%
\pgfpathcurveto{\pgfqpoint{4.725782in}{2.656463in}}{\pgfqpoint{4.736381in}{2.652073in}}{\pgfqpoint{4.747431in}{2.652073in}}%
\pgfpathclose%
\pgfusepath{stroke,fill}%
\end{pgfscope}%
\begin{pgfscope}%
\pgfpathrectangle{\pgfqpoint{0.570343in}{0.331635in}}{\pgfqpoint{4.960000in}{3.696000in}}%
\pgfusepath{clip}%
\pgfsetbuttcap%
\pgfsetroundjoin%
\definecolor{currentfill}{rgb}{0.631373,0.788235,0.956863}%
\pgfsetfillcolor{currentfill}%
\pgfsetlinewidth{0.481800pt}%
\definecolor{currentstroke}{rgb}{1.000000,1.000000,1.000000}%
\pgfsetstrokecolor{currentstroke}%
\pgfsetdash{}{0pt}%
\pgfpathmoveto{\pgfqpoint{3.800801in}{1.122576in}}%
\pgfpathcurveto{\pgfqpoint{3.811851in}{1.122576in}}{\pgfqpoint{3.822450in}{1.126966in}}{\pgfqpoint{3.830264in}{1.134779in}}%
\pgfpathcurveto{\pgfqpoint{3.838077in}{1.142593in}}{\pgfqpoint{3.842468in}{1.153192in}}{\pgfqpoint{3.842468in}{1.164242in}}%
\pgfpathcurveto{\pgfqpoint{3.842468in}{1.175292in}}{\pgfqpoint{3.838077in}{1.185891in}}{\pgfqpoint{3.830264in}{1.193705in}}%
\pgfpathcurveto{\pgfqpoint{3.822450in}{1.201519in}}{\pgfqpoint{3.811851in}{1.205909in}}{\pgfqpoint{3.800801in}{1.205909in}}%
\pgfpathcurveto{\pgfqpoint{3.789751in}{1.205909in}}{\pgfqpoint{3.779152in}{1.201519in}}{\pgfqpoint{3.771338in}{1.193705in}}%
\pgfpathcurveto{\pgfqpoint{3.763525in}{1.185891in}}{\pgfqpoint{3.759134in}{1.175292in}}{\pgfqpoint{3.759134in}{1.164242in}}%
\pgfpathcurveto{\pgfqpoint{3.759134in}{1.153192in}}{\pgfqpoint{3.763525in}{1.142593in}}{\pgfqpoint{3.771338in}{1.134779in}}%
\pgfpathcurveto{\pgfqpoint{3.779152in}{1.126966in}}{\pgfqpoint{3.789751in}{1.122576in}}{\pgfqpoint{3.800801in}{1.122576in}}%
\pgfpathclose%
\pgfusepath{stroke,fill}%
\end{pgfscope}%
\begin{pgfscope}%
\pgfpathrectangle{\pgfqpoint{0.570343in}{0.331635in}}{\pgfqpoint{4.960000in}{3.696000in}}%
\pgfusepath{clip}%
\pgfsetbuttcap%
\pgfsetroundjoin%
\definecolor{currentfill}{rgb}{0.631373,0.788235,0.956863}%
\pgfsetfillcolor{currentfill}%
\pgfsetlinewidth{0.481800pt}%
\definecolor{currentstroke}{rgb}{1.000000,1.000000,1.000000}%
\pgfsetstrokecolor{currentstroke}%
\pgfsetdash{}{0pt}%
\pgfpathmoveto{\pgfqpoint{3.358861in}{1.907180in}}%
\pgfpathcurveto{\pgfqpoint{3.369911in}{1.907180in}}{\pgfqpoint{3.380510in}{1.911570in}}{\pgfqpoint{3.388324in}{1.919384in}}%
\pgfpathcurveto{\pgfqpoint{3.396138in}{1.927198in}}{\pgfqpoint{3.400528in}{1.937797in}}{\pgfqpoint{3.400528in}{1.948847in}}%
\pgfpathcurveto{\pgfqpoint{3.400528in}{1.959897in}}{\pgfqpoint{3.396138in}{1.970496in}}{\pgfqpoint{3.388324in}{1.978310in}}%
\pgfpathcurveto{\pgfqpoint{3.380510in}{1.986123in}}{\pgfqpoint{3.369911in}{1.990513in}}{\pgfqpoint{3.358861in}{1.990513in}}%
\pgfpathcurveto{\pgfqpoint{3.347811in}{1.990513in}}{\pgfqpoint{3.337212in}{1.986123in}}{\pgfqpoint{3.329398in}{1.978310in}}%
\pgfpathcurveto{\pgfqpoint{3.321585in}{1.970496in}}{\pgfqpoint{3.317195in}{1.959897in}}{\pgfqpoint{3.317195in}{1.948847in}}%
\pgfpathcurveto{\pgfqpoint{3.317195in}{1.937797in}}{\pgfqpoint{3.321585in}{1.927198in}}{\pgfqpoint{3.329398in}{1.919384in}}%
\pgfpathcurveto{\pgfqpoint{3.337212in}{1.911570in}}{\pgfqpoint{3.347811in}{1.907180in}}{\pgfqpoint{3.358861in}{1.907180in}}%
\pgfpathclose%
\pgfusepath{stroke,fill}%
\end{pgfscope}%
\begin{pgfscope}%
\pgfpathrectangle{\pgfqpoint{0.570343in}{0.331635in}}{\pgfqpoint{4.960000in}{3.696000in}}%
\pgfusepath{clip}%
\pgfsetbuttcap%
\pgfsetroundjoin%
\definecolor{currentfill}{rgb}{0.631373,0.788235,0.956863}%
\pgfsetfillcolor{currentfill}%
\pgfsetlinewidth{0.481800pt}%
\definecolor{currentstroke}{rgb}{1.000000,1.000000,1.000000}%
\pgfsetstrokecolor{currentstroke}%
\pgfsetdash{}{0pt}%
\pgfpathmoveto{\pgfqpoint{3.351516in}{1.046284in}}%
\pgfpathcurveto{\pgfqpoint{3.362566in}{1.046284in}}{\pgfqpoint{3.373165in}{1.050674in}}{\pgfqpoint{3.380979in}{1.058488in}}%
\pgfpathcurveto{\pgfqpoint{3.388792in}{1.066301in}}{\pgfqpoint{3.393183in}{1.076900in}}{\pgfqpoint{3.393183in}{1.087951in}}%
\pgfpathcurveto{\pgfqpoint{3.393183in}{1.099001in}}{\pgfqpoint{3.388792in}{1.109600in}}{\pgfqpoint{3.380979in}{1.117413in}}%
\pgfpathcurveto{\pgfqpoint{3.373165in}{1.125227in}}{\pgfqpoint{3.362566in}{1.129617in}}{\pgfqpoint{3.351516in}{1.129617in}}%
\pgfpathcurveto{\pgfqpoint{3.340466in}{1.129617in}}{\pgfqpoint{3.329867in}{1.125227in}}{\pgfqpoint{3.322053in}{1.117413in}}%
\pgfpathcurveto{\pgfqpoint{3.314240in}{1.109600in}}{\pgfqpoint{3.309849in}{1.099001in}}{\pgfqpoint{3.309849in}{1.087951in}}%
\pgfpathcurveto{\pgfqpoint{3.309849in}{1.076900in}}{\pgfqpoint{3.314240in}{1.066301in}}{\pgfqpoint{3.322053in}{1.058488in}}%
\pgfpathcurveto{\pgfqpoint{3.329867in}{1.050674in}}{\pgfqpoint{3.340466in}{1.046284in}}{\pgfqpoint{3.351516in}{1.046284in}}%
\pgfpathclose%
\pgfusepath{stroke,fill}%
\end{pgfscope}%
\begin{pgfscope}%
\pgfpathrectangle{\pgfqpoint{0.570343in}{0.331635in}}{\pgfqpoint{4.960000in}{3.696000in}}%
\pgfusepath{clip}%
\pgfsetbuttcap%
\pgfsetroundjoin%
\definecolor{currentfill}{rgb}{0.631373,0.788235,0.956863}%
\pgfsetfillcolor{currentfill}%
\pgfsetlinewidth{0.481800pt}%
\definecolor{currentstroke}{rgb}{1.000000,1.000000,1.000000}%
\pgfsetstrokecolor{currentstroke}%
\pgfsetdash{}{0pt}%
\pgfpathmoveto{\pgfqpoint{2.157332in}{3.115971in}}%
\pgfpathcurveto{\pgfqpoint{2.168382in}{3.115971in}}{\pgfqpoint{2.178981in}{3.120362in}}{\pgfqpoint{2.186795in}{3.128175in}}%
\pgfpathcurveto{\pgfqpoint{2.194609in}{3.135989in}}{\pgfqpoint{2.198999in}{3.146588in}}{\pgfqpoint{2.198999in}{3.157638in}}%
\pgfpathcurveto{\pgfqpoint{2.198999in}{3.168688in}}{\pgfqpoint{2.194609in}{3.179287in}}{\pgfqpoint{2.186795in}{3.187101in}}%
\pgfpathcurveto{\pgfqpoint{2.178981in}{3.194914in}}{\pgfqpoint{2.168382in}{3.199305in}}{\pgfqpoint{2.157332in}{3.199305in}}%
\pgfpathcurveto{\pgfqpoint{2.146282in}{3.199305in}}{\pgfqpoint{2.135683in}{3.194914in}}{\pgfqpoint{2.127870in}{3.187101in}}%
\pgfpathcurveto{\pgfqpoint{2.120056in}{3.179287in}}{\pgfqpoint{2.115666in}{3.168688in}}{\pgfqpoint{2.115666in}{3.157638in}}%
\pgfpathcurveto{\pgfqpoint{2.115666in}{3.146588in}}{\pgfqpoint{2.120056in}{3.135989in}}{\pgfqpoint{2.127870in}{3.128175in}}%
\pgfpathcurveto{\pgfqpoint{2.135683in}{3.120362in}}{\pgfqpoint{2.146282in}{3.115971in}}{\pgfqpoint{2.157332in}{3.115971in}}%
\pgfpathclose%
\pgfusepath{stroke,fill}%
\end{pgfscope}%
\begin{pgfscope}%
\pgfpathrectangle{\pgfqpoint{0.570343in}{0.331635in}}{\pgfqpoint{4.960000in}{3.696000in}}%
\pgfusepath{clip}%
\pgfsetbuttcap%
\pgfsetroundjoin%
\definecolor{currentfill}{rgb}{0.631373,0.788235,0.956863}%
\pgfsetfillcolor{currentfill}%
\pgfsetlinewidth{0.481800pt}%
\definecolor{currentstroke}{rgb}{1.000000,1.000000,1.000000}%
\pgfsetstrokecolor{currentstroke}%
\pgfsetdash{}{0pt}%
\pgfpathmoveto{\pgfqpoint{4.201874in}{1.874651in}}%
\pgfpathcurveto{\pgfqpoint{4.212924in}{1.874651in}}{\pgfqpoint{4.223523in}{1.879042in}}{\pgfqpoint{4.231337in}{1.886855in}}%
\pgfpathcurveto{\pgfqpoint{4.239150in}{1.894669in}}{\pgfqpoint{4.243541in}{1.905268in}}{\pgfqpoint{4.243541in}{1.916318in}}%
\pgfpathcurveto{\pgfqpoint{4.243541in}{1.927368in}}{\pgfqpoint{4.239150in}{1.937967in}}{\pgfqpoint{4.231337in}{1.945781in}}%
\pgfpathcurveto{\pgfqpoint{4.223523in}{1.953594in}}{\pgfqpoint{4.212924in}{1.957985in}}{\pgfqpoint{4.201874in}{1.957985in}}%
\pgfpathcurveto{\pgfqpoint{4.190824in}{1.957985in}}{\pgfqpoint{4.180225in}{1.953594in}}{\pgfqpoint{4.172411in}{1.945781in}}%
\pgfpathcurveto{\pgfqpoint{4.164598in}{1.937967in}}{\pgfqpoint{4.160207in}{1.927368in}}{\pgfqpoint{4.160207in}{1.916318in}}%
\pgfpathcurveto{\pgfqpoint{4.160207in}{1.905268in}}{\pgfqpoint{4.164598in}{1.894669in}}{\pgfqpoint{4.172411in}{1.886855in}}%
\pgfpathcurveto{\pgfqpoint{4.180225in}{1.879042in}}{\pgfqpoint{4.190824in}{1.874651in}}{\pgfqpoint{4.201874in}{1.874651in}}%
\pgfpathclose%
\pgfusepath{stroke,fill}%
\end{pgfscope}%
\begin{pgfscope}%
\pgfpathrectangle{\pgfqpoint{0.570343in}{0.331635in}}{\pgfqpoint{4.960000in}{3.696000in}}%
\pgfusepath{clip}%
\pgfsetbuttcap%
\pgfsetroundjoin%
\definecolor{currentfill}{rgb}{0.631373,0.788235,0.956863}%
\pgfsetfillcolor{currentfill}%
\pgfsetlinewidth{0.481800pt}%
\definecolor{currentstroke}{rgb}{1.000000,1.000000,1.000000}%
\pgfsetstrokecolor{currentstroke}%
\pgfsetdash{}{0pt}%
\pgfpathmoveto{\pgfqpoint{3.667396in}{0.643012in}}%
\pgfpathcurveto{\pgfqpoint{3.678446in}{0.643012in}}{\pgfqpoint{3.689045in}{0.647403in}}{\pgfqpoint{3.696859in}{0.655216in}}%
\pgfpathcurveto{\pgfqpoint{3.704672in}{0.663030in}}{\pgfqpoint{3.709062in}{0.673629in}}{\pgfqpoint{3.709062in}{0.684679in}}%
\pgfpathcurveto{\pgfqpoint{3.709062in}{0.695729in}}{\pgfqpoint{3.704672in}{0.706328in}}{\pgfqpoint{3.696859in}{0.714142in}}%
\pgfpathcurveto{\pgfqpoint{3.689045in}{0.721955in}}{\pgfqpoint{3.678446in}{0.726346in}}{\pgfqpoint{3.667396in}{0.726346in}}%
\pgfpathcurveto{\pgfqpoint{3.656346in}{0.726346in}}{\pgfqpoint{3.645747in}{0.721955in}}{\pgfqpoint{3.637933in}{0.714142in}}%
\pgfpathcurveto{\pgfqpoint{3.630119in}{0.706328in}}{\pgfqpoint{3.625729in}{0.695729in}}{\pgfqpoint{3.625729in}{0.684679in}}%
\pgfpathcurveto{\pgfqpoint{3.625729in}{0.673629in}}{\pgfqpoint{3.630119in}{0.663030in}}{\pgfqpoint{3.637933in}{0.655216in}}%
\pgfpathcurveto{\pgfqpoint{3.645747in}{0.647403in}}{\pgfqpoint{3.656346in}{0.643012in}}{\pgfqpoint{3.667396in}{0.643012in}}%
\pgfpathclose%
\pgfusepath{stroke,fill}%
\end{pgfscope}%
\begin{pgfscope}%
\pgfpathrectangle{\pgfqpoint{0.570343in}{0.331635in}}{\pgfqpoint{4.960000in}{3.696000in}}%
\pgfusepath{clip}%
\pgfsetbuttcap%
\pgfsetroundjoin%
\definecolor{currentfill}{rgb}{0.631373,0.788235,0.956863}%
\pgfsetfillcolor{currentfill}%
\pgfsetlinewidth{0.481800pt}%
\definecolor{currentstroke}{rgb}{1.000000,1.000000,1.000000}%
\pgfsetstrokecolor{currentstroke}%
\pgfsetdash{}{0pt}%
\pgfpathmoveto{\pgfqpoint{3.384863in}{1.519474in}}%
\pgfpathcurveto{\pgfqpoint{3.395913in}{1.519474in}}{\pgfqpoint{3.406512in}{1.523865in}}{\pgfqpoint{3.414326in}{1.531678in}}%
\pgfpathcurveto{\pgfqpoint{3.422140in}{1.539492in}}{\pgfqpoint{3.426530in}{1.550091in}}{\pgfqpoint{3.426530in}{1.561141in}}%
\pgfpathcurveto{\pgfqpoint{3.426530in}{1.572191in}}{\pgfqpoint{3.422140in}{1.582790in}}{\pgfqpoint{3.414326in}{1.590604in}}%
\pgfpathcurveto{\pgfqpoint{3.406512in}{1.598417in}}{\pgfqpoint{3.395913in}{1.602808in}}{\pgfqpoint{3.384863in}{1.602808in}}%
\pgfpathcurveto{\pgfqpoint{3.373813in}{1.602808in}}{\pgfqpoint{3.363214in}{1.598417in}}{\pgfqpoint{3.355400in}{1.590604in}}%
\pgfpathcurveto{\pgfqpoint{3.347587in}{1.582790in}}{\pgfqpoint{3.343196in}{1.572191in}}{\pgfqpoint{3.343196in}{1.561141in}}%
\pgfpathcurveto{\pgfqpoint{3.343196in}{1.550091in}}{\pgfqpoint{3.347587in}{1.539492in}}{\pgfqpoint{3.355400in}{1.531678in}}%
\pgfpathcurveto{\pgfqpoint{3.363214in}{1.523865in}}{\pgfqpoint{3.373813in}{1.519474in}}{\pgfqpoint{3.384863in}{1.519474in}}%
\pgfpathclose%
\pgfusepath{stroke,fill}%
\end{pgfscope}%
\begin{pgfscope}%
\pgfpathrectangle{\pgfqpoint{0.570343in}{0.331635in}}{\pgfqpoint{4.960000in}{3.696000in}}%
\pgfusepath{clip}%
\pgfsetbuttcap%
\pgfsetroundjoin%
\definecolor{currentfill}{rgb}{0.631373,0.788235,0.956863}%
\pgfsetfillcolor{currentfill}%
\pgfsetlinewidth{0.481800pt}%
\definecolor{currentstroke}{rgb}{1.000000,1.000000,1.000000}%
\pgfsetstrokecolor{currentstroke}%
\pgfsetdash{}{0pt}%
\pgfpathmoveto{\pgfqpoint{2.133517in}{2.131617in}}%
\pgfpathcurveto{\pgfqpoint{2.144567in}{2.131617in}}{\pgfqpoint{2.155166in}{2.136007in}}{\pgfqpoint{2.162979in}{2.143821in}}%
\pgfpathcurveto{\pgfqpoint{2.170793in}{2.151635in}}{\pgfqpoint{2.175183in}{2.162234in}}{\pgfqpoint{2.175183in}{2.173284in}}%
\pgfpathcurveto{\pgfqpoint{2.175183in}{2.184334in}}{\pgfqpoint{2.170793in}{2.194933in}}{\pgfqpoint{2.162979in}{2.202747in}}%
\pgfpathcurveto{\pgfqpoint{2.155166in}{2.210560in}}{\pgfqpoint{2.144567in}{2.214950in}}{\pgfqpoint{2.133517in}{2.214950in}}%
\pgfpathcurveto{\pgfqpoint{2.122466in}{2.214950in}}{\pgfqpoint{2.111867in}{2.210560in}}{\pgfqpoint{2.104054in}{2.202747in}}%
\pgfpathcurveto{\pgfqpoint{2.096240in}{2.194933in}}{\pgfqpoint{2.091850in}{2.184334in}}{\pgfqpoint{2.091850in}{2.173284in}}%
\pgfpathcurveto{\pgfqpoint{2.091850in}{2.162234in}}{\pgfqpoint{2.096240in}{2.151635in}}{\pgfqpoint{2.104054in}{2.143821in}}%
\pgfpathcurveto{\pgfqpoint{2.111867in}{2.136007in}}{\pgfqpoint{2.122466in}{2.131617in}}{\pgfqpoint{2.133517in}{2.131617in}}%
\pgfpathclose%
\pgfusepath{stroke,fill}%
\end{pgfscope}%
\begin{pgfscope}%
\pgfpathrectangle{\pgfqpoint{0.570343in}{0.331635in}}{\pgfqpoint{4.960000in}{3.696000in}}%
\pgfusepath{clip}%
\pgfsetbuttcap%
\pgfsetroundjoin%
\definecolor{currentfill}{rgb}{0.631373,0.788235,0.956863}%
\pgfsetfillcolor{currentfill}%
\pgfsetlinewidth{0.481800pt}%
\definecolor{currentstroke}{rgb}{1.000000,1.000000,1.000000}%
\pgfsetstrokecolor{currentstroke}%
\pgfsetdash{}{0pt}%
\pgfpathmoveto{\pgfqpoint{5.304889in}{1.738067in}}%
\pgfpathcurveto{\pgfqpoint{5.315939in}{1.738067in}}{\pgfqpoint{5.326538in}{1.742457in}}{\pgfqpoint{5.334352in}{1.750271in}}%
\pgfpathcurveto{\pgfqpoint{5.342165in}{1.758084in}}{\pgfqpoint{5.346555in}{1.768683in}}{\pgfqpoint{5.346555in}{1.779734in}}%
\pgfpathcurveto{\pgfqpoint{5.346555in}{1.790784in}}{\pgfqpoint{5.342165in}{1.801383in}}{\pgfqpoint{5.334352in}{1.809196in}}%
\pgfpathcurveto{\pgfqpoint{5.326538in}{1.817010in}}{\pgfqpoint{5.315939in}{1.821400in}}{\pgfqpoint{5.304889in}{1.821400in}}%
\pgfpathcurveto{\pgfqpoint{5.293839in}{1.821400in}}{\pgfqpoint{5.283240in}{1.817010in}}{\pgfqpoint{5.275426in}{1.809196in}}%
\pgfpathcurveto{\pgfqpoint{5.267612in}{1.801383in}}{\pgfqpoint{5.263222in}{1.790784in}}{\pgfqpoint{5.263222in}{1.779734in}}%
\pgfpathcurveto{\pgfqpoint{5.263222in}{1.768683in}}{\pgfqpoint{5.267612in}{1.758084in}}{\pgfqpoint{5.275426in}{1.750271in}}%
\pgfpathcurveto{\pgfqpoint{5.283240in}{1.742457in}}{\pgfqpoint{5.293839in}{1.738067in}}{\pgfqpoint{5.304889in}{1.738067in}}%
\pgfpathclose%
\pgfusepath{stroke,fill}%
\end{pgfscope}%
\begin{pgfscope}%
\pgfpathrectangle{\pgfqpoint{0.570343in}{0.331635in}}{\pgfqpoint{4.960000in}{3.696000in}}%
\pgfusepath{clip}%
\pgfsetbuttcap%
\pgfsetroundjoin%
\definecolor{currentfill}{rgb}{0.631373,0.788235,0.956863}%
\pgfsetfillcolor{currentfill}%
\pgfsetlinewidth{0.481800pt}%
\definecolor{currentstroke}{rgb}{1.000000,1.000000,1.000000}%
\pgfsetstrokecolor{currentstroke}%
\pgfsetdash{}{0pt}%
\pgfpathmoveto{\pgfqpoint{3.176792in}{2.210336in}}%
\pgfpathcurveto{\pgfqpoint{3.187842in}{2.210336in}}{\pgfqpoint{3.198441in}{2.214726in}}{\pgfqpoint{3.206255in}{2.222540in}}%
\pgfpathcurveto{\pgfqpoint{3.214068in}{2.230354in}}{\pgfqpoint{3.218459in}{2.240953in}}{\pgfqpoint{3.218459in}{2.252003in}}%
\pgfpathcurveto{\pgfqpoint{3.218459in}{2.263053in}}{\pgfqpoint{3.214068in}{2.273652in}}{\pgfqpoint{3.206255in}{2.281466in}}%
\pgfpathcurveto{\pgfqpoint{3.198441in}{2.289279in}}{\pgfqpoint{3.187842in}{2.293669in}}{\pgfqpoint{3.176792in}{2.293669in}}%
\pgfpathcurveto{\pgfqpoint{3.165742in}{2.293669in}}{\pgfqpoint{3.155143in}{2.289279in}}{\pgfqpoint{3.147329in}{2.281466in}}%
\pgfpathcurveto{\pgfqpoint{3.139516in}{2.273652in}}{\pgfqpoint{3.135125in}{2.263053in}}{\pgfqpoint{3.135125in}{2.252003in}}%
\pgfpathcurveto{\pgfqpoint{3.135125in}{2.240953in}}{\pgfqpoint{3.139516in}{2.230354in}}{\pgfqpoint{3.147329in}{2.222540in}}%
\pgfpathcurveto{\pgfqpoint{3.155143in}{2.214726in}}{\pgfqpoint{3.165742in}{2.210336in}}{\pgfqpoint{3.176792in}{2.210336in}}%
\pgfpathclose%
\pgfusepath{stroke,fill}%
\end{pgfscope}%
\begin{pgfscope}%
\pgfpathrectangle{\pgfqpoint{0.570343in}{0.331635in}}{\pgfqpoint{4.960000in}{3.696000in}}%
\pgfusepath{clip}%
\pgfsetbuttcap%
\pgfsetroundjoin%
\definecolor{currentfill}{rgb}{0.631373,0.788235,0.956863}%
\pgfsetfillcolor{currentfill}%
\pgfsetlinewidth{0.481800pt}%
\definecolor{currentstroke}{rgb}{1.000000,1.000000,1.000000}%
\pgfsetstrokecolor{currentstroke}%
\pgfsetdash{}{0pt}%
\pgfpathmoveto{\pgfqpoint{2.230769in}{1.168284in}}%
\pgfpathcurveto{\pgfqpoint{2.241819in}{1.168284in}}{\pgfqpoint{2.252418in}{1.172674in}}{\pgfqpoint{2.260232in}{1.180488in}}%
\pgfpathcurveto{\pgfqpoint{2.268046in}{1.188301in}}{\pgfqpoint{2.272436in}{1.198900in}}{\pgfqpoint{2.272436in}{1.209950in}}%
\pgfpathcurveto{\pgfqpoint{2.272436in}{1.221000in}}{\pgfqpoint{2.268046in}{1.231599in}}{\pgfqpoint{2.260232in}{1.239413in}}%
\pgfpathcurveto{\pgfqpoint{2.252418in}{1.247227in}}{\pgfqpoint{2.241819in}{1.251617in}}{\pgfqpoint{2.230769in}{1.251617in}}%
\pgfpathcurveto{\pgfqpoint{2.219719in}{1.251617in}}{\pgfqpoint{2.209120in}{1.247227in}}{\pgfqpoint{2.201306in}{1.239413in}}%
\pgfpathcurveto{\pgfqpoint{2.193493in}{1.231599in}}{\pgfqpoint{2.189102in}{1.221000in}}{\pgfqpoint{2.189102in}{1.209950in}}%
\pgfpathcurveto{\pgfqpoint{2.189102in}{1.198900in}}{\pgfqpoint{2.193493in}{1.188301in}}{\pgfqpoint{2.201306in}{1.180488in}}%
\pgfpathcurveto{\pgfqpoint{2.209120in}{1.172674in}}{\pgfqpoint{2.219719in}{1.168284in}}{\pgfqpoint{2.230769in}{1.168284in}}%
\pgfpathclose%
\pgfusepath{stroke,fill}%
\end{pgfscope}%
\begin{pgfscope}%
\pgfpathrectangle{\pgfqpoint{0.570343in}{0.331635in}}{\pgfqpoint{4.960000in}{3.696000in}}%
\pgfusepath{clip}%
\pgfsetbuttcap%
\pgfsetroundjoin%
\definecolor{currentfill}{rgb}{0.631373,0.788235,0.956863}%
\pgfsetfillcolor{currentfill}%
\pgfsetlinewidth{0.481800pt}%
\definecolor{currentstroke}{rgb}{1.000000,1.000000,1.000000}%
\pgfsetstrokecolor{currentstroke}%
\pgfsetdash{}{0pt}%
\pgfpathmoveto{\pgfqpoint{2.677188in}{2.172216in}}%
\pgfpathcurveto{\pgfqpoint{2.688238in}{2.172216in}}{\pgfqpoint{2.698837in}{2.176606in}}{\pgfqpoint{2.706651in}{2.184419in}}%
\pgfpathcurveto{\pgfqpoint{2.714465in}{2.192233in}}{\pgfqpoint{2.718855in}{2.202832in}}{\pgfqpoint{2.718855in}{2.213882in}}%
\pgfpathcurveto{\pgfqpoint{2.718855in}{2.224932in}}{\pgfqpoint{2.714465in}{2.235531in}}{\pgfqpoint{2.706651in}{2.243345in}}%
\pgfpathcurveto{\pgfqpoint{2.698837in}{2.251159in}}{\pgfqpoint{2.688238in}{2.255549in}}{\pgfqpoint{2.677188in}{2.255549in}}%
\pgfpathcurveto{\pgfqpoint{2.666138in}{2.255549in}}{\pgfqpoint{2.655539in}{2.251159in}}{\pgfqpoint{2.647726in}{2.243345in}}%
\pgfpathcurveto{\pgfqpoint{2.639912in}{2.235531in}}{\pgfqpoint{2.635522in}{2.224932in}}{\pgfqpoint{2.635522in}{2.213882in}}%
\pgfpathcurveto{\pgfqpoint{2.635522in}{2.202832in}}{\pgfqpoint{2.639912in}{2.192233in}}{\pgfqpoint{2.647726in}{2.184419in}}%
\pgfpathcurveto{\pgfqpoint{2.655539in}{2.176606in}}{\pgfqpoint{2.666138in}{2.172216in}}{\pgfqpoint{2.677188in}{2.172216in}}%
\pgfpathclose%
\pgfusepath{stroke,fill}%
\end{pgfscope}%
\begin{pgfscope}%
\pgfpathrectangle{\pgfqpoint{0.570343in}{0.331635in}}{\pgfqpoint{4.960000in}{3.696000in}}%
\pgfusepath{clip}%
\pgfsetbuttcap%
\pgfsetroundjoin%
\definecolor{currentfill}{rgb}{0.631373,0.788235,0.956863}%
\pgfsetfillcolor{currentfill}%
\pgfsetlinewidth{0.481800pt}%
\definecolor{currentstroke}{rgb}{1.000000,1.000000,1.000000}%
\pgfsetstrokecolor{currentstroke}%
\pgfsetdash{}{0pt}%
\pgfpathmoveto{\pgfqpoint{3.805258in}{1.743124in}}%
\pgfpathcurveto{\pgfqpoint{3.816308in}{1.743124in}}{\pgfqpoint{3.826907in}{1.747514in}}{\pgfqpoint{3.834721in}{1.755328in}}%
\pgfpathcurveto{\pgfqpoint{3.842535in}{1.763141in}}{\pgfqpoint{3.846925in}{1.773740in}}{\pgfqpoint{3.846925in}{1.784790in}}%
\pgfpathcurveto{\pgfqpoint{3.846925in}{1.795840in}}{\pgfqpoint{3.842535in}{1.806440in}}{\pgfqpoint{3.834721in}{1.814253in}}%
\pgfpathcurveto{\pgfqpoint{3.826907in}{1.822067in}}{\pgfqpoint{3.816308in}{1.826457in}}{\pgfqpoint{3.805258in}{1.826457in}}%
\pgfpathcurveto{\pgfqpoint{3.794208in}{1.826457in}}{\pgfqpoint{3.783609in}{1.822067in}}{\pgfqpoint{3.775795in}{1.814253in}}%
\pgfpathcurveto{\pgfqpoint{3.767982in}{1.806440in}}{\pgfqpoint{3.763591in}{1.795840in}}{\pgfqpoint{3.763591in}{1.784790in}}%
\pgfpathcurveto{\pgfqpoint{3.763591in}{1.773740in}}{\pgfqpoint{3.767982in}{1.763141in}}{\pgfqpoint{3.775795in}{1.755328in}}%
\pgfpathcurveto{\pgfqpoint{3.783609in}{1.747514in}}{\pgfqpoint{3.794208in}{1.743124in}}{\pgfqpoint{3.805258in}{1.743124in}}%
\pgfpathclose%
\pgfusepath{stroke,fill}%
\end{pgfscope}%
\begin{pgfscope}%
\pgfpathrectangle{\pgfqpoint{0.570343in}{0.331635in}}{\pgfqpoint{4.960000in}{3.696000in}}%
\pgfusepath{clip}%
\pgfsetbuttcap%
\pgfsetroundjoin%
\definecolor{currentfill}{rgb}{0.631373,0.788235,0.956863}%
\pgfsetfillcolor{currentfill}%
\pgfsetlinewidth{0.481800pt}%
\definecolor{currentstroke}{rgb}{1.000000,1.000000,1.000000}%
\pgfsetstrokecolor{currentstroke}%
\pgfsetdash{}{0pt}%
\pgfpathmoveto{\pgfqpoint{1.649989in}{2.063703in}}%
\pgfpathcurveto{\pgfqpoint{1.661039in}{2.063703in}}{\pgfqpoint{1.671638in}{2.068094in}}{\pgfqpoint{1.679451in}{2.075907in}}%
\pgfpathcurveto{\pgfqpoint{1.687265in}{2.083721in}}{\pgfqpoint{1.691655in}{2.094320in}}{\pgfqpoint{1.691655in}{2.105370in}}%
\pgfpathcurveto{\pgfqpoint{1.691655in}{2.116420in}}{\pgfqpoint{1.687265in}{2.127019in}}{\pgfqpoint{1.679451in}{2.134833in}}%
\pgfpathcurveto{\pgfqpoint{1.671638in}{2.142646in}}{\pgfqpoint{1.661039in}{2.147037in}}{\pgfqpoint{1.649989in}{2.147037in}}%
\pgfpathcurveto{\pgfqpoint{1.638939in}{2.147037in}}{\pgfqpoint{1.628340in}{2.142646in}}{\pgfqpoint{1.620526in}{2.134833in}}%
\pgfpathcurveto{\pgfqpoint{1.612712in}{2.127019in}}{\pgfqpoint{1.608322in}{2.116420in}}{\pgfqpoint{1.608322in}{2.105370in}}%
\pgfpathcurveto{\pgfqpoint{1.608322in}{2.094320in}}{\pgfqpoint{1.612712in}{2.083721in}}{\pgfqpoint{1.620526in}{2.075907in}}%
\pgfpathcurveto{\pgfqpoint{1.628340in}{2.068094in}}{\pgfqpoint{1.638939in}{2.063703in}}{\pgfqpoint{1.649989in}{2.063703in}}%
\pgfpathclose%
\pgfusepath{stroke,fill}%
\end{pgfscope}%
\begin{pgfscope}%
\pgfpathrectangle{\pgfqpoint{0.570343in}{0.331635in}}{\pgfqpoint{4.960000in}{3.696000in}}%
\pgfusepath{clip}%
\pgfsetbuttcap%
\pgfsetroundjoin%
\definecolor{currentfill}{rgb}{0.631373,0.788235,0.956863}%
\pgfsetfillcolor{currentfill}%
\pgfsetlinewidth{0.481800pt}%
\definecolor{currentstroke}{rgb}{1.000000,1.000000,1.000000}%
\pgfsetstrokecolor{currentstroke}%
\pgfsetdash{}{0pt}%
\pgfpathmoveto{\pgfqpoint{4.280136in}{1.475141in}}%
\pgfpathcurveto{\pgfqpoint{4.291186in}{1.475141in}}{\pgfqpoint{4.301785in}{1.479531in}}{\pgfqpoint{4.309599in}{1.487345in}}%
\pgfpathcurveto{\pgfqpoint{4.317412in}{1.495158in}}{\pgfqpoint{4.321803in}{1.505757in}}{\pgfqpoint{4.321803in}{1.516808in}}%
\pgfpathcurveto{\pgfqpoint{4.321803in}{1.527858in}}{\pgfqpoint{4.317412in}{1.538457in}}{\pgfqpoint{4.309599in}{1.546270in}}%
\pgfpathcurveto{\pgfqpoint{4.301785in}{1.554084in}}{\pgfqpoint{4.291186in}{1.558474in}}{\pgfqpoint{4.280136in}{1.558474in}}%
\pgfpathcurveto{\pgfqpoint{4.269086in}{1.558474in}}{\pgfqpoint{4.258487in}{1.554084in}}{\pgfqpoint{4.250673in}{1.546270in}}%
\pgfpathcurveto{\pgfqpoint{4.242860in}{1.538457in}}{\pgfqpoint{4.238469in}{1.527858in}}{\pgfqpoint{4.238469in}{1.516808in}}%
\pgfpathcurveto{\pgfqpoint{4.238469in}{1.505757in}}{\pgfqpoint{4.242860in}{1.495158in}}{\pgfqpoint{4.250673in}{1.487345in}}%
\pgfpathcurveto{\pgfqpoint{4.258487in}{1.479531in}}{\pgfqpoint{4.269086in}{1.475141in}}{\pgfqpoint{4.280136in}{1.475141in}}%
\pgfpathclose%
\pgfusepath{stroke,fill}%
\end{pgfscope}%
\begin{pgfscope}%
\pgfpathrectangle{\pgfqpoint{0.570343in}{0.331635in}}{\pgfqpoint{4.960000in}{3.696000in}}%
\pgfusepath{clip}%
\pgfsetbuttcap%
\pgfsetroundjoin%
\definecolor{currentfill}{rgb}{0.631373,0.788235,0.956863}%
\pgfsetfillcolor{currentfill}%
\pgfsetlinewidth{0.481800pt}%
\definecolor{currentstroke}{rgb}{1.000000,1.000000,1.000000}%
\pgfsetstrokecolor{currentstroke}%
\pgfsetdash{}{0pt}%
\pgfpathmoveto{\pgfqpoint{2.472600in}{1.865496in}}%
\pgfpathcurveto{\pgfqpoint{2.483650in}{1.865496in}}{\pgfqpoint{2.494249in}{1.869886in}}{\pgfqpoint{2.502062in}{1.877700in}}%
\pgfpathcurveto{\pgfqpoint{2.509876in}{1.885513in}}{\pgfqpoint{2.514266in}{1.896112in}}{\pgfqpoint{2.514266in}{1.907162in}}%
\pgfpathcurveto{\pgfqpoint{2.514266in}{1.918213in}}{\pgfqpoint{2.509876in}{1.928812in}}{\pgfqpoint{2.502062in}{1.936625in}}%
\pgfpathcurveto{\pgfqpoint{2.494249in}{1.944439in}}{\pgfqpoint{2.483650in}{1.948829in}}{\pgfqpoint{2.472600in}{1.948829in}}%
\pgfpathcurveto{\pgfqpoint{2.461549in}{1.948829in}}{\pgfqpoint{2.450950in}{1.944439in}}{\pgfqpoint{2.443137in}{1.936625in}}%
\pgfpathcurveto{\pgfqpoint{2.435323in}{1.928812in}}{\pgfqpoint{2.430933in}{1.918213in}}{\pgfqpoint{2.430933in}{1.907162in}}%
\pgfpathcurveto{\pgfqpoint{2.430933in}{1.896112in}}{\pgfqpoint{2.435323in}{1.885513in}}{\pgfqpoint{2.443137in}{1.877700in}}%
\pgfpathcurveto{\pgfqpoint{2.450950in}{1.869886in}}{\pgfqpoint{2.461549in}{1.865496in}}{\pgfqpoint{2.472600in}{1.865496in}}%
\pgfpathclose%
\pgfusepath{stroke,fill}%
\end{pgfscope}%
\begin{pgfscope}%
\pgfpathrectangle{\pgfqpoint{0.570343in}{0.331635in}}{\pgfqpoint{4.960000in}{3.696000in}}%
\pgfusepath{clip}%
\pgfsetbuttcap%
\pgfsetroundjoin%
\definecolor{currentfill}{rgb}{0.631373,0.788235,0.956863}%
\pgfsetfillcolor{currentfill}%
\pgfsetlinewidth{0.481800pt}%
\definecolor{currentstroke}{rgb}{1.000000,1.000000,1.000000}%
\pgfsetstrokecolor{currentstroke}%
\pgfsetdash{}{0pt}%
\pgfpathmoveto{\pgfqpoint{3.832697in}{2.104501in}}%
\pgfpathcurveto{\pgfqpoint{3.843747in}{2.104501in}}{\pgfqpoint{3.854346in}{2.108891in}}{\pgfqpoint{3.862159in}{2.116704in}}%
\pgfpathcurveto{\pgfqpoint{3.869973in}{2.124518in}}{\pgfqpoint{3.874363in}{2.135117in}}{\pgfqpoint{3.874363in}{2.146167in}}%
\pgfpathcurveto{\pgfqpoint{3.874363in}{2.157217in}}{\pgfqpoint{3.869973in}{2.167816in}}{\pgfqpoint{3.862159in}{2.175630in}}%
\pgfpathcurveto{\pgfqpoint{3.854346in}{2.183444in}}{\pgfqpoint{3.843747in}{2.187834in}}{\pgfqpoint{3.832697in}{2.187834in}}%
\pgfpathcurveto{\pgfqpoint{3.821646in}{2.187834in}}{\pgfqpoint{3.811047in}{2.183444in}}{\pgfqpoint{3.803234in}{2.175630in}}%
\pgfpathcurveto{\pgfqpoint{3.795420in}{2.167816in}}{\pgfqpoint{3.791030in}{2.157217in}}{\pgfqpoint{3.791030in}{2.146167in}}%
\pgfpathcurveto{\pgfqpoint{3.791030in}{2.135117in}}{\pgfqpoint{3.795420in}{2.124518in}}{\pgfqpoint{3.803234in}{2.116704in}}%
\pgfpathcurveto{\pgfqpoint{3.811047in}{2.108891in}}{\pgfqpoint{3.821646in}{2.104501in}}{\pgfqpoint{3.832697in}{2.104501in}}%
\pgfpathclose%
\pgfusepath{stroke,fill}%
\end{pgfscope}%
\begin{pgfscope}%
\pgfpathrectangle{\pgfqpoint{0.570343in}{0.331635in}}{\pgfqpoint{4.960000in}{3.696000in}}%
\pgfusepath{clip}%
\pgfsetbuttcap%
\pgfsetroundjoin%
\definecolor{currentfill}{rgb}{0.631373,0.788235,0.956863}%
\pgfsetfillcolor{currentfill}%
\pgfsetlinewidth{0.481800pt}%
\definecolor{currentstroke}{rgb}{1.000000,1.000000,1.000000}%
\pgfsetstrokecolor{currentstroke}%
\pgfsetdash{}{0pt}%
\pgfpathmoveto{\pgfqpoint{2.958348in}{1.963140in}}%
\pgfpathcurveto{\pgfqpoint{2.969398in}{1.963140in}}{\pgfqpoint{2.979997in}{1.967530in}}{\pgfqpoint{2.987810in}{1.975344in}}%
\pgfpathcurveto{\pgfqpoint{2.995624in}{1.983158in}}{\pgfqpoint{3.000014in}{1.993757in}}{\pgfqpoint{3.000014in}{2.004807in}}%
\pgfpathcurveto{\pgfqpoint{3.000014in}{2.015857in}}{\pgfqpoint{2.995624in}{2.026456in}}{\pgfqpoint{2.987810in}{2.034270in}}%
\pgfpathcurveto{\pgfqpoint{2.979997in}{2.042083in}}{\pgfqpoint{2.969398in}{2.046473in}}{\pgfqpoint{2.958348in}{2.046473in}}%
\pgfpathcurveto{\pgfqpoint{2.947297in}{2.046473in}}{\pgfqpoint{2.936698in}{2.042083in}}{\pgfqpoint{2.928885in}{2.034270in}}%
\pgfpathcurveto{\pgfqpoint{2.921071in}{2.026456in}}{\pgfqpoint{2.916681in}{2.015857in}}{\pgfqpoint{2.916681in}{2.004807in}}%
\pgfpathcurveto{\pgfqpoint{2.916681in}{1.993757in}}{\pgfqpoint{2.921071in}{1.983158in}}{\pgfqpoint{2.928885in}{1.975344in}}%
\pgfpathcurveto{\pgfqpoint{2.936698in}{1.967530in}}{\pgfqpoint{2.947297in}{1.963140in}}{\pgfqpoint{2.958348in}{1.963140in}}%
\pgfpathclose%
\pgfusepath{stroke,fill}%
\end{pgfscope}%
\begin{pgfscope}%
\pgfpathrectangle{\pgfqpoint{0.570343in}{0.331635in}}{\pgfqpoint{4.960000in}{3.696000in}}%
\pgfusepath{clip}%
\pgfsetbuttcap%
\pgfsetroundjoin%
\definecolor{currentfill}{rgb}{0.631373,0.788235,0.956863}%
\pgfsetfillcolor{currentfill}%
\pgfsetlinewidth{0.481800pt}%
\definecolor{currentstroke}{rgb}{1.000000,1.000000,1.000000}%
\pgfsetstrokecolor{currentstroke}%
\pgfsetdash{}{0pt}%
\pgfpathmoveto{\pgfqpoint{1.527720in}{1.130130in}}%
\pgfpathcurveto{\pgfqpoint{1.538771in}{1.130130in}}{\pgfqpoint{1.549370in}{1.134520in}}{\pgfqpoint{1.557183in}{1.142333in}}%
\pgfpathcurveto{\pgfqpoint{1.564997in}{1.150147in}}{\pgfqpoint{1.569387in}{1.160746in}}{\pgfqpoint{1.569387in}{1.171796in}}%
\pgfpathcurveto{\pgfqpoint{1.569387in}{1.182846in}}{\pgfqpoint{1.564997in}{1.193445in}}{\pgfqpoint{1.557183in}{1.201259in}}%
\pgfpathcurveto{\pgfqpoint{1.549370in}{1.209073in}}{\pgfqpoint{1.538771in}{1.213463in}}{\pgfqpoint{1.527720in}{1.213463in}}%
\pgfpathcurveto{\pgfqpoint{1.516670in}{1.213463in}}{\pgfqpoint{1.506071in}{1.209073in}}{\pgfqpoint{1.498258in}{1.201259in}}%
\pgfpathcurveto{\pgfqpoint{1.490444in}{1.193445in}}{\pgfqpoint{1.486054in}{1.182846in}}{\pgfqpoint{1.486054in}{1.171796in}}%
\pgfpathcurveto{\pgfqpoint{1.486054in}{1.160746in}}{\pgfqpoint{1.490444in}{1.150147in}}{\pgfqpoint{1.498258in}{1.142333in}}%
\pgfpathcurveto{\pgfqpoint{1.506071in}{1.134520in}}{\pgfqpoint{1.516670in}{1.130130in}}{\pgfqpoint{1.527720in}{1.130130in}}%
\pgfpathclose%
\pgfusepath{stroke,fill}%
\end{pgfscope}%
\begin{pgfscope}%
\pgfpathrectangle{\pgfqpoint{0.570343in}{0.331635in}}{\pgfqpoint{4.960000in}{3.696000in}}%
\pgfusepath{clip}%
\pgfsetbuttcap%
\pgfsetroundjoin%
\definecolor{currentfill}{rgb}{0.631373,0.788235,0.956863}%
\pgfsetfillcolor{currentfill}%
\pgfsetlinewidth{0.481800pt}%
\definecolor{currentstroke}{rgb}{1.000000,1.000000,1.000000}%
\pgfsetstrokecolor{currentstroke}%
\pgfsetdash{}{0pt}%
\pgfpathmoveto{\pgfqpoint{2.978471in}{0.598498in}}%
\pgfpathcurveto{\pgfqpoint{2.989521in}{0.598498in}}{\pgfqpoint{3.000120in}{0.602888in}}{\pgfqpoint{3.007934in}{0.610702in}}%
\pgfpathcurveto{\pgfqpoint{3.015747in}{0.618515in}}{\pgfqpoint{3.020138in}{0.629114in}}{\pgfqpoint{3.020138in}{0.640164in}}%
\pgfpathcurveto{\pgfqpoint{3.020138in}{0.651215in}}{\pgfqpoint{3.015747in}{0.661814in}}{\pgfqpoint{3.007934in}{0.669627in}}%
\pgfpathcurveto{\pgfqpoint{3.000120in}{0.677441in}}{\pgfqpoint{2.989521in}{0.681831in}}{\pgfqpoint{2.978471in}{0.681831in}}%
\pgfpathcurveto{\pgfqpoint{2.967421in}{0.681831in}}{\pgfqpoint{2.956822in}{0.677441in}}{\pgfqpoint{2.949008in}{0.669627in}}%
\pgfpathcurveto{\pgfqpoint{2.941195in}{0.661814in}}{\pgfqpoint{2.936804in}{0.651215in}}{\pgfqpoint{2.936804in}{0.640164in}}%
\pgfpathcurveto{\pgfqpoint{2.936804in}{0.629114in}}{\pgfqpoint{2.941195in}{0.618515in}}{\pgfqpoint{2.949008in}{0.610702in}}%
\pgfpathcurveto{\pgfqpoint{2.956822in}{0.602888in}}{\pgfqpoint{2.967421in}{0.598498in}}{\pgfqpoint{2.978471in}{0.598498in}}%
\pgfpathclose%
\pgfusepath{stroke,fill}%
\end{pgfscope}%
\begin{pgfscope}%
\pgfpathrectangle{\pgfqpoint{0.570343in}{0.331635in}}{\pgfqpoint{4.960000in}{3.696000in}}%
\pgfusepath{clip}%
\pgfsetbuttcap%
\pgfsetroundjoin%
\definecolor{currentfill}{rgb}{0.631373,0.788235,0.956863}%
\pgfsetfillcolor{currentfill}%
\pgfsetlinewidth{0.481800pt}%
\definecolor{currentstroke}{rgb}{1.000000,1.000000,1.000000}%
\pgfsetstrokecolor{currentstroke}%
\pgfsetdash{}{0pt}%
\pgfpathmoveto{\pgfqpoint{2.744542in}{1.066861in}}%
\pgfpathcurveto{\pgfqpoint{2.755592in}{1.066861in}}{\pgfqpoint{2.766191in}{1.071251in}}{\pgfqpoint{2.774005in}{1.079065in}}%
\pgfpathcurveto{\pgfqpoint{2.781819in}{1.086879in}}{\pgfqpoint{2.786209in}{1.097478in}}{\pgfqpoint{2.786209in}{1.108528in}}%
\pgfpathcurveto{\pgfqpoint{2.786209in}{1.119578in}}{\pgfqpoint{2.781819in}{1.130177in}}{\pgfqpoint{2.774005in}{1.137991in}}%
\pgfpathcurveto{\pgfqpoint{2.766191in}{1.145804in}}{\pgfqpoint{2.755592in}{1.150194in}}{\pgfqpoint{2.744542in}{1.150194in}}%
\pgfpathcurveto{\pgfqpoint{2.733492in}{1.150194in}}{\pgfqpoint{2.722893in}{1.145804in}}{\pgfqpoint{2.715080in}{1.137991in}}%
\pgfpathcurveto{\pgfqpoint{2.707266in}{1.130177in}}{\pgfqpoint{2.702876in}{1.119578in}}{\pgfqpoint{2.702876in}{1.108528in}}%
\pgfpathcurveto{\pgfqpoint{2.702876in}{1.097478in}}{\pgfqpoint{2.707266in}{1.086879in}}{\pgfqpoint{2.715080in}{1.079065in}}%
\pgfpathcurveto{\pgfqpoint{2.722893in}{1.071251in}}{\pgfqpoint{2.733492in}{1.066861in}}{\pgfqpoint{2.744542in}{1.066861in}}%
\pgfpathclose%
\pgfusepath{stroke,fill}%
\end{pgfscope}%
\begin{pgfscope}%
\pgfpathrectangle{\pgfqpoint{0.570343in}{0.331635in}}{\pgfqpoint{4.960000in}{3.696000in}}%
\pgfusepath{clip}%
\pgfsetbuttcap%
\pgfsetroundjoin%
\definecolor{currentfill}{rgb}{0.631373,0.788235,0.956863}%
\pgfsetfillcolor{currentfill}%
\pgfsetlinewidth{0.481800pt}%
\definecolor{currentstroke}{rgb}{1.000000,1.000000,1.000000}%
\pgfsetstrokecolor{currentstroke}%
\pgfsetdash{}{0pt}%
\pgfpathmoveto{\pgfqpoint{1.808176in}{1.610459in}}%
\pgfpathcurveto{\pgfqpoint{1.819226in}{1.610459in}}{\pgfqpoint{1.829825in}{1.614849in}}{\pgfqpoint{1.837639in}{1.622663in}}%
\pgfpathcurveto{\pgfqpoint{1.845452in}{1.630477in}}{\pgfqpoint{1.849843in}{1.641076in}}{\pgfqpoint{1.849843in}{1.652126in}}%
\pgfpathcurveto{\pgfqpoint{1.849843in}{1.663176in}}{\pgfqpoint{1.845452in}{1.673775in}}{\pgfqpoint{1.837639in}{1.681589in}}%
\pgfpathcurveto{\pgfqpoint{1.829825in}{1.689402in}}{\pgfqpoint{1.819226in}{1.693793in}}{\pgfqpoint{1.808176in}{1.693793in}}%
\pgfpathcurveto{\pgfqpoint{1.797126in}{1.693793in}}{\pgfqpoint{1.786527in}{1.689402in}}{\pgfqpoint{1.778713in}{1.681589in}}%
\pgfpathcurveto{\pgfqpoint{1.770900in}{1.673775in}}{\pgfqpoint{1.766509in}{1.663176in}}{\pgfqpoint{1.766509in}{1.652126in}}%
\pgfpathcurveto{\pgfqpoint{1.766509in}{1.641076in}}{\pgfqpoint{1.770900in}{1.630477in}}{\pgfqpoint{1.778713in}{1.622663in}}%
\pgfpathcurveto{\pgfqpoint{1.786527in}{1.614849in}}{\pgfqpoint{1.797126in}{1.610459in}}{\pgfqpoint{1.808176in}{1.610459in}}%
\pgfpathclose%
\pgfusepath{stroke,fill}%
\end{pgfscope}%
\begin{pgfscope}%
\pgfpathrectangle{\pgfqpoint{0.570343in}{0.331635in}}{\pgfqpoint{4.960000in}{3.696000in}}%
\pgfusepath{clip}%
\pgfsetbuttcap%
\pgfsetroundjoin%
\definecolor{currentfill}{rgb}{0.631373,0.788235,0.956863}%
\pgfsetfillcolor{currentfill}%
\pgfsetlinewidth{0.481800pt}%
\definecolor{currentstroke}{rgb}{1.000000,1.000000,1.000000}%
\pgfsetstrokecolor{currentstroke}%
\pgfsetdash{}{0pt}%
\pgfpathmoveto{\pgfqpoint{2.390127in}{1.538733in}}%
\pgfpathcurveto{\pgfqpoint{2.401177in}{1.538733in}}{\pgfqpoint{2.411776in}{1.543123in}}{\pgfqpoint{2.419590in}{1.550937in}}%
\pgfpathcurveto{\pgfqpoint{2.427403in}{1.558750in}}{\pgfqpoint{2.431793in}{1.569349in}}{\pgfqpoint{2.431793in}{1.580399in}}%
\pgfpathcurveto{\pgfqpoint{2.431793in}{1.591450in}}{\pgfqpoint{2.427403in}{1.602049in}}{\pgfqpoint{2.419590in}{1.609862in}}%
\pgfpathcurveto{\pgfqpoint{2.411776in}{1.617676in}}{\pgfqpoint{2.401177in}{1.622066in}}{\pgfqpoint{2.390127in}{1.622066in}}%
\pgfpathcurveto{\pgfqpoint{2.379077in}{1.622066in}}{\pgfqpoint{2.368478in}{1.617676in}}{\pgfqpoint{2.360664in}{1.609862in}}%
\pgfpathcurveto{\pgfqpoint{2.352850in}{1.602049in}}{\pgfqpoint{2.348460in}{1.591450in}}{\pgfqpoint{2.348460in}{1.580399in}}%
\pgfpathcurveto{\pgfqpoint{2.348460in}{1.569349in}}{\pgfqpoint{2.352850in}{1.558750in}}{\pgfqpoint{2.360664in}{1.550937in}}%
\pgfpathcurveto{\pgfqpoint{2.368478in}{1.543123in}}{\pgfqpoint{2.379077in}{1.538733in}}{\pgfqpoint{2.390127in}{1.538733in}}%
\pgfpathclose%
\pgfusepath{stroke,fill}%
\end{pgfscope}%
\begin{pgfscope}%
\pgfpathrectangle{\pgfqpoint{0.570343in}{0.331635in}}{\pgfqpoint{4.960000in}{3.696000in}}%
\pgfusepath{clip}%
\pgfsetbuttcap%
\pgfsetroundjoin%
\definecolor{currentfill}{rgb}{0.631373,0.788235,0.956863}%
\pgfsetfillcolor{currentfill}%
\pgfsetlinewidth{0.481800pt}%
\definecolor{currentstroke}{rgb}{1.000000,1.000000,1.000000}%
\pgfsetstrokecolor{currentstroke}%
\pgfsetdash{}{0pt}%
\pgfpathmoveto{\pgfqpoint{2.744744in}{3.138009in}}%
\pgfpathcurveto{\pgfqpoint{2.755795in}{3.138009in}}{\pgfqpoint{2.766394in}{3.142399in}}{\pgfqpoint{2.774207in}{3.150213in}}%
\pgfpathcurveto{\pgfqpoint{2.782021in}{3.158027in}}{\pgfqpoint{2.786411in}{3.168626in}}{\pgfqpoint{2.786411in}{3.179676in}}%
\pgfpathcurveto{\pgfqpoint{2.786411in}{3.190726in}}{\pgfqpoint{2.782021in}{3.201325in}}{\pgfqpoint{2.774207in}{3.209139in}}%
\pgfpathcurveto{\pgfqpoint{2.766394in}{3.216952in}}{\pgfqpoint{2.755795in}{3.221342in}}{\pgfqpoint{2.744744in}{3.221342in}}%
\pgfpathcurveto{\pgfqpoint{2.733694in}{3.221342in}}{\pgfqpoint{2.723095in}{3.216952in}}{\pgfqpoint{2.715282in}{3.209139in}}%
\pgfpathcurveto{\pgfqpoint{2.707468in}{3.201325in}}{\pgfqpoint{2.703078in}{3.190726in}}{\pgfqpoint{2.703078in}{3.179676in}}%
\pgfpathcurveto{\pgfqpoint{2.703078in}{3.168626in}}{\pgfqpoint{2.707468in}{3.158027in}}{\pgfqpoint{2.715282in}{3.150213in}}%
\pgfpathcurveto{\pgfqpoint{2.723095in}{3.142399in}}{\pgfqpoint{2.733694in}{3.138009in}}{\pgfqpoint{2.744744in}{3.138009in}}%
\pgfpathclose%
\pgfusepath{stroke,fill}%
\end{pgfscope}%
\begin{pgfscope}%
\pgfpathrectangle{\pgfqpoint{0.570343in}{0.331635in}}{\pgfqpoint{4.960000in}{3.696000in}}%
\pgfusepath{clip}%
\pgfsetbuttcap%
\pgfsetroundjoin%
\definecolor{currentfill}{rgb}{0.631373,0.788235,0.956863}%
\pgfsetfillcolor{currentfill}%
\pgfsetlinewidth{0.481800pt}%
\definecolor{currentstroke}{rgb}{1.000000,1.000000,1.000000}%
\pgfsetstrokecolor{currentstroke}%
\pgfsetdash{}{0pt}%
\pgfpathmoveto{\pgfqpoint{2.878184in}{1.662238in}}%
\pgfpathcurveto{\pgfqpoint{2.889234in}{1.662238in}}{\pgfqpoint{2.899833in}{1.666629in}}{\pgfqpoint{2.907647in}{1.674442in}}%
\pgfpathcurveto{\pgfqpoint{2.915460in}{1.682256in}}{\pgfqpoint{2.919851in}{1.692855in}}{\pgfqpoint{2.919851in}{1.703905in}}%
\pgfpathcurveto{\pgfqpoint{2.919851in}{1.714955in}}{\pgfqpoint{2.915460in}{1.725554in}}{\pgfqpoint{2.907647in}{1.733368in}}%
\pgfpathcurveto{\pgfqpoint{2.899833in}{1.741181in}}{\pgfqpoint{2.889234in}{1.745572in}}{\pgfqpoint{2.878184in}{1.745572in}}%
\pgfpathcurveto{\pgfqpoint{2.867134in}{1.745572in}}{\pgfqpoint{2.856535in}{1.741181in}}{\pgfqpoint{2.848721in}{1.733368in}}%
\pgfpathcurveto{\pgfqpoint{2.840908in}{1.725554in}}{\pgfqpoint{2.836517in}{1.714955in}}{\pgfqpoint{2.836517in}{1.703905in}}%
\pgfpathcurveto{\pgfqpoint{2.836517in}{1.692855in}}{\pgfqpoint{2.840908in}{1.682256in}}{\pgfqpoint{2.848721in}{1.674442in}}%
\pgfpathcurveto{\pgfqpoint{2.856535in}{1.666629in}}{\pgfqpoint{2.867134in}{1.662238in}}{\pgfqpoint{2.878184in}{1.662238in}}%
\pgfpathclose%
\pgfusepath{stroke,fill}%
\end{pgfscope}%
\begin{pgfscope}%
\pgfpathrectangle{\pgfqpoint{0.570343in}{0.331635in}}{\pgfqpoint{4.960000in}{3.696000in}}%
\pgfusepath{clip}%
\pgfsetbuttcap%
\pgfsetroundjoin%
\definecolor{currentfill}{rgb}{0.631373,0.788235,0.956863}%
\pgfsetfillcolor{currentfill}%
\pgfsetlinewidth{0.481800pt}%
\definecolor{currentstroke}{rgb}{1.000000,1.000000,1.000000}%
\pgfsetstrokecolor{currentstroke}%
\pgfsetdash{}{0pt}%
\pgfpathmoveto{\pgfqpoint{3.759465in}{1.432960in}}%
\pgfpathcurveto{\pgfqpoint{3.770515in}{1.432960in}}{\pgfqpoint{3.781114in}{1.437350in}}{\pgfqpoint{3.788928in}{1.445164in}}%
\pgfpathcurveto{\pgfqpoint{3.796741in}{1.452977in}}{\pgfqpoint{3.801132in}{1.463576in}}{\pgfqpoint{3.801132in}{1.474627in}}%
\pgfpathcurveto{\pgfqpoint{3.801132in}{1.485677in}}{\pgfqpoint{3.796741in}{1.496276in}}{\pgfqpoint{3.788928in}{1.504089in}}%
\pgfpathcurveto{\pgfqpoint{3.781114in}{1.511903in}}{\pgfqpoint{3.770515in}{1.516293in}}{\pgfqpoint{3.759465in}{1.516293in}}%
\pgfpathcurveto{\pgfqpoint{3.748415in}{1.516293in}}{\pgfqpoint{3.737816in}{1.511903in}}{\pgfqpoint{3.730002in}{1.504089in}}%
\pgfpathcurveto{\pgfqpoint{3.722188in}{1.496276in}}{\pgfqpoint{3.717798in}{1.485677in}}{\pgfqpoint{3.717798in}{1.474627in}}%
\pgfpathcurveto{\pgfqpoint{3.717798in}{1.463576in}}{\pgfqpoint{3.722188in}{1.452977in}}{\pgfqpoint{3.730002in}{1.445164in}}%
\pgfpathcurveto{\pgfqpoint{3.737816in}{1.437350in}}{\pgfqpoint{3.748415in}{1.432960in}}{\pgfqpoint{3.759465in}{1.432960in}}%
\pgfpathclose%
\pgfusepath{stroke,fill}%
\end{pgfscope}%
\begin{pgfscope}%
\pgfpathrectangle{\pgfqpoint{0.570343in}{0.331635in}}{\pgfqpoint{4.960000in}{3.696000in}}%
\pgfusepath{clip}%
\pgfsetbuttcap%
\pgfsetroundjoin%
\definecolor{currentfill}{rgb}{1.000000,0.705882,0.509804}%
\pgfsetfillcolor{currentfill}%
\pgfsetlinewidth{0.481800pt}%
\definecolor{currentstroke}{rgb}{1.000000,1.000000,1.000000}%
\pgfsetstrokecolor{currentstroke}%
\pgfsetdash{}{0pt}%
\pgfpathmoveto{\pgfqpoint{4.010786in}{2.428913in}}%
\pgfpathcurveto{\pgfqpoint{4.021837in}{2.428913in}}{\pgfqpoint{4.032436in}{2.433303in}}{\pgfqpoint{4.040249in}{2.441117in}}%
\pgfpathcurveto{\pgfqpoint{4.048063in}{2.448931in}}{\pgfqpoint{4.052453in}{2.459530in}}{\pgfqpoint{4.052453in}{2.470580in}}%
\pgfpathcurveto{\pgfqpoint{4.052453in}{2.481630in}}{\pgfqpoint{4.048063in}{2.492229in}}{\pgfqpoint{4.040249in}{2.500043in}}%
\pgfpathcurveto{\pgfqpoint{4.032436in}{2.507856in}}{\pgfqpoint{4.021837in}{2.512247in}}{\pgfqpoint{4.010786in}{2.512247in}}%
\pgfpathcurveto{\pgfqpoint{3.999736in}{2.512247in}}{\pgfqpoint{3.989137in}{2.507856in}}{\pgfqpoint{3.981324in}{2.500043in}}%
\pgfpathcurveto{\pgfqpoint{3.973510in}{2.492229in}}{\pgfqpoint{3.969120in}{2.481630in}}{\pgfqpoint{3.969120in}{2.470580in}}%
\pgfpathcurveto{\pgfqpoint{3.969120in}{2.459530in}}{\pgfqpoint{3.973510in}{2.448931in}}{\pgfqpoint{3.981324in}{2.441117in}}%
\pgfpathcurveto{\pgfqpoint{3.989137in}{2.433303in}}{\pgfqpoint{3.999736in}{2.428913in}}{\pgfqpoint{4.010786in}{2.428913in}}%
\pgfpathclose%
\pgfusepath{stroke,fill}%
\end{pgfscope}%
\begin{pgfscope}%
\pgfpathrectangle{\pgfqpoint{0.570343in}{0.331635in}}{\pgfqpoint{4.960000in}{3.696000in}}%
\pgfusepath{clip}%
\pgfsetbuttcap%
\pgfsetroundjoin%
\definecolor{currentfill}{rgb}{1.000000,0.705882,0.509804}%
\pgfsetfillcolor{currentfill}%
\pgfsetlinewidth{0.481800pt}%
\definecolor{currentstroke}{rgb}{1.000000,1.000000,1.000000}%
\pgfsetstrokecolor{currentstroke}%
\pgfsetdash{}{0pt}%
\pgfpathmoveto{\pgfqpoint{3.641382in}{3.235775in}}%
\pgfpathcurveto{\pgfqpoint{3.652432in}{3.235775in}}{\pgfqpoint{3.663031in}{3.240166in}}{\pgfqpoint{3.670845in}{3.247979in}}%
\pgfpathcurveto{\pgfqpoint{3.678659in}{3.255793in}}{\pgfqpoint{3.683049in}{3.266392in}}{\pgfqpoint{3.683049in}{3.277442in}}%
\pgfpathcurveto{\pgfqpoint{3.683049in}{3.288492in}}{\pgfqpoint{3.678659in}{3.299091in}}{\pgfqpoint{3.670845in}{3.306905in}}%
\pgfpathcurveto{\pgfqpoint{3.663031in}{3.314718in}}{\pgfqpoint{3.652432in}{3.319109in}}{\pgfqpoint{3.641382in}{3.319109in}}%
\pgfpathcurveto{\pgfqpoint{3.630332in}{3.319109in}}{\pgfqpoint{3.619733in}{3.314718in}}{\pgfqpoint{3.611919in}{3.306905in}}%
\pgfpathcurveto{\pgfqpoint{3.604106in}{3.299091in}}{\pgfqpoint{3.599716in}{3.288492in}}{\pgfqpoint{3.599716in}{3.277442in}}%
\pgfpathcurveto{\pgfqpoint{3.599716in}{3.266392in}}{\pgfqpoint{3.604106in}{3.255793in}}{\pgfqpoint{3.611919in}{3.247979in}}%
\pgfpathcurveto{\pgfqpoint{3.619733in}{3.240166in}}{\pgfqpoint{3.630332in}{3.235775in}}{\pgfqpoint{3.641382in}{3.235775in}}%
\pgfpathclose%
\pgfusepath{stroke,fill}%
\end{pgfscope}%
\begin{pgfscope}%
\pgfpathrectangle{\pgfqpoint{0.570343in}{0.331635in}}{\pgfqpoint{4.960000in}{3.696000in}}%
\pgfusepath{clip}%
\pgfsetbuttcap%
\pgfsetroundjoin%
\definecolor{currentfill}{rgb}{1.000000,0.705882,0.509804}%
\pgfsetfillcolor{currentfill}%
\pgfsetlinewidth{0.481800pt}%
\definecolor{currentstroke}{rgb}{1.000000,1.000000,1.000000}%
\pgfsetstrokecolor{currentstroke}%
\pgfsetdash{}{0pt}%
\pgfpathmoveto{\pgfqpoint{2.910306in}{2.493024in}}%
\pgfpathcurveto{\pgfqpoint{2.921356in}{2.493024in}}{\pgfqpoint{2.931955in}{2.497414in}}{\pgfqpoint{2.939768in}{2.505228in}}%
\pgfpathcurveto{\pgfqpoint{2.947582in}{2.513041in}}{\pgfqpoint{2.951972in}{2.523640in}}{\pgfqpoint{2.951972in}{2.534690in}}%
\pgfpathcurveto{\pgfqpoint{2.951972in}{2.545740in}}{\pgfqpoint{2.947582in}{2.556340in}}{\pgfqpoint{2.939768in}{2.564153in}}%
\pgfpathcurveto{\pgfqpoint{2.931955in}{2.571967in}}{\pgfqpoint{2.921356in}{2.576357in}}{\pgfqpoint{2.910306in}{2.576357in}}%
\pgfpathcurveto{\pgfqpoint{2.899256in}{2.576357in}}{\pgfqpoint{2.888656in}{2.571967in}}{\pgfqpoint{2.880843in}{2.564153in}}%
\pgfpathcurveto{\pgfqpoint{2.873029in}{2.556340in}}{\pgfqpoint{2.868639in}{2.545740in}}{\pgfqpoint{2.868639in}{2.534690in}}%
\pgfpathcurveto{\pgfqpoint{2.868639in}{2.523640in}}{\pgfqpoint{2.873029in}{2.513041in}}{\pgfqpoint{2.880843in}{2.505228in}}%
\pgfpathcurveto{\pgfqpoint{2.888656in}{2.497414in}}{\pgfqpoint{2.899256in}{2.493024in}}{\pgfqpoint{2.910306in}{2.493024in}}%
\pgfpathclose%
\pgfusepath{stroke,fill}%
\end{pgfscope}%
\begin{pgfscope}%
\pgfpathrectangle{\pgfqpoint{0.570343in}{0.331635in}}{\pgfqpoint{4.960000in}{3.696000in}}%
\pgfusepath{clip}%
\pgfsetbuttcap%
\pgfsetroundjoin%
\definecolor{currentfill}{rgb}{1.000000,0.705882,0.509804}%
\pgfsetfillcolor{currentfill}%
\pgfsetlinewidth{0.481800pt}%
\definecolor{currentstroke}{rgb}{1.000000,1.000000,1.000000}%
\pgfsetstrokecolor{currentstroke}%
\pgfsetdash{}{0pt}%
\pgfpathmoveto{\pgfqpoint{2.662402in}{2.751085in}}%
\pgfpathcurveto{\pgfqpoint{2.673452in}{2.751085in}}{\pgfqpoint{2.684051in}{2.755475in}}{\pgfqpoint{2.691864in}{2.763289in}}%
\pgfpathcurveto{\pgfqpoint{2.699678in}{2.771103in}}{\pgfqpoint{2.704068in}{2.781702in}}{\pgfqpoint{2.704068in}{2.792752in}}%
\pgfpathcurveto{\pgfqpoint{2.704068in}{2.803802in}}{\pgfqpoint{2.699678in}{2.814401in}}{\pgfqpoint{2.691864in}{2.822214in}}%
\pgfpathcurveto{\pgfqpoint{2.684051in}{2.830028in}}{\pgfqpoint{2.673452in}{2.834418in}}{\pgfqpoint{2.662402in}{2.834418in}}%
\pgfpathcurveto{\pgfqpoint{2.651351in}{2.834418in}}{\pgfqpoint{2.640752in}{2.830028in}}{\pgfqpoint{2.632939in}{2.822214in}}%
\pgfpathcurveto{\pgfqpoint{2.625125in}{2.814401in}}{\pgfqpoint{2.620735in}{2.803802in}}{\pgfqpoint{2.620735in}{2.792752in}}%
\pgfpathcurveto{\pgfqpoint{2.620735in}{2.781702in}}{\pgfqpoint{2.625125in}{2.771103in}}{\pgfqpoint{2.632939in}{2.763289in}}%
\pgfpathcurveto{\pgfqpoint{2.640752in}{2.755475in}}{\pgfqpoint{2.651351in}{2.751085in}}{\pgfqpoint{2.662402in}{2.751085in}}%
\pgfpathclose%
\pgfusepath{stroke,fill}%
\end{pgfscope}%
\begin{pgfscope}%
\pgfpathrectangle{\pgfqpoint{0.570343in}{0.331635in}}{\pgfqpoint{4.960000in}{3.696000in}}%
\pgfusepath{clip}%
\pgfsetbuttcap%
\pgfsetroundjoin%
\definecolor{currentfill}{rgb}{1.000000,0.705882,0.509804}%
\pgfsetfillcolor{currentfill}%
\pgfsetlinewidth{0.481800pt}%
\definecolor{currentstroke}{rgb}{1.000000,1.000000,1.000000}%
\pgfsetstrokecolor{currentstroke}%
\pgfsetdash{}{0pt}%
\pgfpathmoveto{\pgfqpoint{1.385602in}{3.454654in}}%
\pgfpathcurveto{\pgfqpoint{1.396652in}{3.454654in}}{\pgfqpoint{1.407251in}{3.459044in}}{\pgfqpoint{1.415065in}{3.466858in}}%
\pgfpathcurveto{\pgfqpoint{1.422879in}{3.474671in}}{\pgfqpoint{1.427269in}{3.485270in}}{\pgfqpoint{1.427269in}{3.496321in}}%
\pgfpathcurveto{\pgfqpoint{1.427269in}{3.507371in}}{\pgfqpoint{1.422879in}{3.517970in}}{\pgfqpoint{1.415065in}{3.525783in}}%
\pgfpathcurveto{\pgfqpoint{1.407251in}{3.533597in}}{\pgfqpoint{1.396652in}{3.537987in}}{\pgfqpoint{1.385602in}{3.537987in}}%
\pgfpathcurveto{\pgfqpoint{1.374552in}{3.537987in}}{\pgfqpoint{1.363953in}{3.533597in}}{\pgfqpoint{1.356139in}{3.525783in}}%
\pgfpathcurveto{\pgfqpoint{1.348326in}{3.517970in}}{\pgfqpoint{1.343936in}{3.507371in}}{\pgfqpoint{1.343936in}{3.496321in}}%
\pgfpathcurveto{\pgfqpoint{1.343936in}{3.485270in}}{\pgfqpoint{1.348326in}{3.474671in}}{\pgfqpoint{1.356139in}{3.466858in}}%
\pgfpathcurveto{\pgfqpoint{1.363953in}{3.459044in}}{\pgfqpoint{1.374552in}{3.454654in}}{\pgfqpoint{1.385602in}{3.454654in}}%
\pgfpathclose%
\pgfusepath{stroke,fill}%
\end{pgfscope}%
\begin{pgfscope}%
\pgfpathrectangle{\pgfqpoint{0.570343in}{0.331635in}}{\pgfqpoint{4.960000in}{3.696000in}}%
\pgfusepath{clip}%
\pgfsetbuttcap%
\pgfsetroundjoin%
\definecolor{currentfill}{rgb}{1.000000,0.705882,0.509804}%
\pgfsetfillcolor{currentfill}%
\pgfsetlinewidth{0.481800pt}%
\definecolor{currentstroke}{rgb}{1.000000,1.000000,1.000000}%
\pgfsetstrokecolor{currentstroke}%
\pgfsetdash{}{0pt}%
\pgfpathmoveto{\pgfqpoint{4.718935in}{1.875375in}}%
\pgfpathcurveto{\pgfqpoint{4.729985in}{1.875375in}}{\pgfqpoint{4.740584in}{1.879765in}}{\pgfqpoint{4.748398in}{1.887579in}}%
\pgfpathcurveto{\pgfqpoint{4.756212in}{1.895392in}}{\pgfqpoint{4.760602in}{1.905991in}}{\pgfqpoint{4.760602in}{1.917042in}}%
\pgfpathcurveto{\pgfqpoint{4.760602in}{1.928092in}}{\pgfqpoint{4.756212in}{1.938691in}}{\pgfqpoint{4.748398in}{1.946504in}}%
\pgfpathcurveto{\pgfqpoint{4.740584in}{1.954318in}}{\pgfqpoint{4.729985in}{1.958708in}}{\pgfqpoint{4.718935in}{1.958708in}}%
\pgfpathcurveto{\pgfqpoint{4.707885in}{1.958708in}}{\pgfqpoint{4.697286in}{1.954318in}}{\pgfqpoint{4.689472in}{1.946504in}}%
\pgfpathcurveto{\pgfqpoint{4.681659in}{1.938691in}}{\pgfqpoint{4.677268in}{1.928092in}}{\pgfqpoint{4.677268in}{1.917042in}}%
\pgfpathcurveto{\pgfqpoint{4.677268in}{1.905991in}}{\pgfqpoint{4.681659in}{1.895392in}}{\pgfqpoint{4.689472in}{1.887579in}}%
\pgfpathcurveto{\pgfqpoint{4.697286in}{1.879765in}}{\pgfqpoint{4.707885in}{1.875375in}}{\pgfqpoint{4.718935in}{1.875375in}}%
\pgfpathclose%
\pgfusepath{stroke,fill}%
\end{pgfscope}%
\begin{pgfscope}%
\pgfpathrectangle{\pgfqpoint{0.570343in}{0.331635in}}{\pgfqpoint{4.960000in}{3.696000in}}%
\pgfusepath{clip}%
\pgfsetbuttcap%
\pgfsetroundjoin%
\definecolor{currentfill}{rgb}{1.000000,0.705882,0.509804}%
\pgfsetfillcolor{currentfill}%
\pgfsetlinewidth{0.481800pt}%
\definecolor{currentstroke}{rgb}{1.000000,1.000000,1.000000}%
\pgfsetstrokecolor{currentstroke}%
\pgfsetdash{}{0pt}%
\pgfpathmoveto{\pgfqpoint{1.526138in}{2.872462in}}%
\pgfpathcurveto{\pgfqpoint{1.537188in}{2.872462in}}{\pgfqpoint{1.547787in}{2.876852in}}{\pgfqpoint{1.555601in}{2.884666in}}%
\pgfpathcurveto{\pgfqpoint{1.563414in}{2.892479in}}{\pgfqpoint{1.567805in}{2.903079in}}{\pgfqpoint{1.567805in}{2.914129in}}%
\pgfpathcurveto{\pgfqpoint{1.567805in}{2.925179in}}{\pgfqpoint{1.563414in}{2.935778in}}{\pgfqpoint{1.555601in}{2.943591in}}%
\pgfpathcurveto{\pgfqpoint{1.547787in}{2.951405in}}{\pgfqpoint{1.537188in}{2.955795in}}{\pgfqpoint{1.526138in}{2.955795in}}%
\pgfpathcurveto{\pgfqpoint{1.515088in}{2.955795in}}{\pgfqpoint{1.504489in}{2.951405in}}{\pgfqpoint{1.496675in}{2.943591in}}%
\pgfpathcurveto{\pgfqpoint{1.488862in}{2.935778in}}{\pgfqpoint{1.484471in}{2.925179in}}{\pgfqpoint{1.484471in}{2.914129in}}%
\pgfpathcurveto{\pgfqpoint{1.484471in}{2.903079in}}{\pgfqpoint{1.488862in}{2.892479in}}{\pgfqpoint{1.496675in}{2.884666in}}%
\pgfpathcurveto{\pgfqpoint{1.504489in}{2.876852in}}{\pgfqpoint{1.515088in}{2.872462in}}{\pgfqpoint{1.526138in}{2.872462in}}%
\pgfpathclose%
\pgfusepath{stroke,fill}%
\end{pgfscope}%
\begin{pgfscope}%
\pgfpathrectangle{\pgfqpoint{0.570343in}{0.331635in}}{\pgfqpoint{4.960000in}{3.696000in}}%
\pgfusepath{clip}%
\pgfsetbuttcap%
\pgfsetroundjoin%
\definecolor{currentfill}{rgb}{1.000000,0.705882,0.509804}%
\pgfsetfillcolor{currentfill}%
\pgfsetlinewidth{0.481800pt}%
\definecolor{currentstroke}{rgb}{1.000000,1.000000,1.000000}%
\pgfsetstrokecolor{currentstroke}%
\pgfsetdash{}{0pt}%
\pgfpathmoveto{\pgfqpoint{5.011810in}{2.229459in}}%
\pgfpathcurveto{\pgfqpoint{5.022860in}{2.229459in}}{\pgfqpoint{5.033459in}{2.233849in}}{\pgfqpoint{5.041273in}{2.241663in}}%
\pgfpathcurveto{\pgfqpoint{5.049086in}{2.249476in}}{\pgfqpoint{5.053477in}{2.260075in}}{\pgfqpoint{5.053477in}{2.271126in}}%
\pgfpathcurveto{\pgfqpoint{5.053477in}{2.282176in}}{\pgfqpoint{5.049086in}{2.292775in}}{\pgfqpoint{5.041273in}{2.300588in}}%
\pgfpathcurveto{\pgfqpoint{5.033459in}{2.308402in}}{\pgfqpoint{5.022860in}{2.312792in}}{\pgfqpoint{5.011810in}{2.312792in}}%
\pgfpathcurveto{\pgfqpoint{5.000760in}{2.312792in}}{\pgfqpoint{4.990161in}{2.308402in}}{\pgfqpoint{4.982347in}{2.300588in}}%
\pgfpathcurveto{\pgfqpoint{4.974533in}{2.292775in}}{\pgfqpoint{4.970143in}{2.282176in}}{\pgfqpoint{4.970143in}{2.271126in}}%
\pgfpathcurveto{\pgfqpoint{4.970143in}{2.260075in}}{\pgfqpoint{4.974533in}{2.249476in}}{\pgfqpoint{4.982347in}{2.241663in}}%
\pgfpathcurveto{\pgfqpoint{4.990161in}{2.233849in}}{\pgfqpoint{5.000760in}{2.229459in}}{\pgfqpoint{5.011810in}{2.229459in}}%
\pgfpathclose%
\pgfusepath{stroke,fill}%
\end{pgfscope}%
\begin{pgfscope}%
\pgfpathrectangle{\pgfqpoint{0.570343in}{0.331635in}}{\pgfqpoint{4.960000in}{3.696000in}}%
\pgfusepath{clip}%
\pgfsetbuttcap%
\pgfsetroundjoin%
\definecolor{currentfill}{rgb}{1.000000,0.705882,0.509804}%
\pgfsetfillcolor{currentfill}%
\pgfsetlinewidth{0.481800pt}%
\definecolor{currentstroke}{rgb}{1.000000,1.000000,1.000000}%
\pgfsetstrokecolor{currentstroke}%
\pgfsetdash{}{0pt}%
\pgfpathmoveto{\pgfqpoint{4.377389in}{1.083755in}}%
\pgfpathcurveto{\pgfqpoint{4.388439in}{1.083755in}}{\pgfqpoint{4.399038in}{1.088145in}}{\pgfqpoint{4.406852in}{1.095959in}}%
\pgfpathcurveto{\pgfqpoint{4.414665in}{1.103772in}}{\pgfqpoint{4.419056in}{1.114371in}}{\pgfqpoint{4.419056in}{1.125422in}}%
\pgfpathcurveto{\pgfqpoint{4.419056in}{1.136472in}}{\pgfqpoint{4.414665in}{1.147071in}}{\pgfqpoint{4.406852in}{1.154884in}}%
\pgfpathcurveto{\pgfqpoint{4.399038in}{1.162698in}}{\pgfqpoint{4.388439in}{1.167088in}}{\pgfqpoint{4.377389in}{1.167088in}}%
\pgfpathcurveto{\pgfqpoint{4.366339in}{1.167088in}}{\pgfqpoint{4.355740in}{1.162698in}}{\pgfqpoint{4.347926in}{1.154884in}}%
\pgfpathcurveto{\pgfqpoint{4.340113in}{1.147071in}}{\pgfqpoint{4.335722in}{1.136472in}}{\pgfqpoint{4.335722in}{1.125422in}}%
\pgfpathcurveto{\pgfqpoint{4.335722in}{1.114371in}}{\pgfqpoint{4.340113in}{1.103772in}}{\pgfqpoint{4.347926in}{1.095959in}}%
\pgfpathcurveto{\pgfqpoint{4.355740in}{1.088145in}}{\pgfqpoint{4.366339in}{1.083755in}}{\pgfqpoint{4.377389in}{1.083755in}}%
\pgfpathclose%
\pgfusepath{stroke,fill}%
\end{pgfscope}%
\begin{pgfscope}%
\pgfpathrectangle{\pgfqpoint{0.570343in}{0.331635in}}{\pgfqpoint{4.960000in}{3.696000in}}%
\pgfusepath{clip}%
\pgfsetbuttcap%
\pgfsetroundjoin%
\definecolor{currentfill}{rgb}{1.000000,0.705882,0.509804}%
\pgfsetfillcolor{currentfill}%
\pgfsetlinewidth{0.481800pt}%
\definecolor{currentstroke}{rgb}{1.000000,1.000000,1.000000}%
\pgfsetstrokecolor{currentstroke}%
\pgfsetdash{}{0pt}%
\pgfpathmoveto{\pgfqpoint{2.162297in}{2.708119in}}%
\pgfpathcurveto{\pgfqpoint{2.173347in}{2.708119in}}{\pgfqpoint{2.183946in}{2.712509in}}{\pgfqpoint{2.191760in}{2.720323in}}%
\pgfpathcurveto{\pgfqpoint{2.199574in}{2.728136in}}{\pgfqpoint{2.203964in}{2.738736in}}{\pgfqpoint{2.203964in}{2.749786in}}%
\pgfpathcurveto{\pgfqpoint{2.203964in}{2.760836in}}{\pgfqpoint{2.199574in}{2.771435in}}{\pgfqpoint{2.191760in}{2.779248in}}%
\pgfpathcurveto{\pgfqpoint{2.183946in}{2.787062in}}{\pgfqpoint{2.173347in}{2.791452in}}{\pgfqpoint{2.162297in}{2.791452in}}%
\pgfpathcurveto{\pgfqpoint{2.151247in}{2.791452in}}{\pgfqpoint{2.140648in}{2.787062in}}{\pgfqpoint{2.132834in}{2.779248in}}%
\pgfpathcurveto{\pgfqpoint{2.125021in}{2.771435in}}{\pgfqpoint{2.120631in}{2.760836in}}{\pgfqpoint{2.120631in}{2.749786in}}%
\pgfpathcurveto{\pgfqpoint{2.120631in}{2.738736in}}{\pgfqpoint{2.125021in}{2.728136in}}{\pgfqpoint{2.132834in}{2.720323in}}%
\pgfpathcurveto{\pgfqpoint{2.140648in}{2.712509in}}{\pgfqpoint{2.151247in}{2.708119in}}{\pgfqpoint{2.162297in}{2.708119in}}%
\pgfpathclose%
\pgfusepath{stroke,fill}%
\end{pgfscope}%
\begin{pgfscope}%
\pgfpathrectangle{\pgfqpoint{0.570343in}{0.331635in}}{\pgfqpoint{4.960000in}{3.696000in}}%
\pgfusepath{clip}%
\pgfsetbuttcap%
\pgfsetroundjoin%
\definecolor{currentfill}{rgb}{1.000000,0.705882,0.509804}%
\pgfsetfillcolor{currentfill}%
\pgfsetlinewidth{0.481800pt}%
\definecolor{currentstroke}{rgb}{1.000000,1.000000,1.000000}%
\pgfsetstrokecolor{currentstroke}%
\pgfsetdash{}{0pt}%
\pgfpathmoveto{\pgfqpoint{2.464678in}{2.431882in}}%
\pgfpathcurveto{\pgfqpoint{2.475729in}{2.431882in}}{\pgfqpoint{2.486328in}{2.436272in}}{\pgfqpoint{2.494141in}{2.444086in}}%
\pgfpathcurveto{\pgfqpoint{2.501955in}{2.451899in}}{\pgfqpoint{2.506345in}{2.462498in}}{\pgfqpoint{2.506345in}{2.473549in}}%
\pgfpathcurveto{\pgfqpoint{2.506345in}{2.484599in}}{\pgfqpoint{2.501955in}{2.495198in}}{\pgfqpoint{2.494141in}{2.503011in}}%
\pgfpathcurveto{\pgfqpoint{2.486328in}{2.510825in}}{\pgfqpoint{2.475729in}{2.515215in}}{\pgfqpoint{2.464678in}{2.515215in}}%
\pgfpathcurveto{\pgfqpoint{2.453628in}{2.515215in}}{\pgfqpoint{2.443029in}{2.510825in}}{\pgfqpoint{2.435216in}{2.503011in}}%
\pgfpathcurveto{\pgfqpoint{2.427402in}{2.495198in}}{\pgfqpoint{2.423012in}{2.484599in}}{\pgfqpoint{2.423012in}{2.473549in}}%
\pgfpathcurveto{\pgfqpoint{2.423012in}{2.462498in}}{\pgfqpoint{2.427402in}{2.451899in}}{\pgfqpoint{2.435216in}{2.444086in}}%
\pgfpathcurveto{\pgfqpoint{2.443029in}{2.436272in}}{\pgfqpoint{2.453628in}{2.431882in}}{\pgfqpoint{2.464678in}{2.431882in}}%
\pgfpathclose%
\pgfusepath{stroke,fill}%
\end{pgfscope}%
\begin{pgfscope}%
\pgfpathrectangle{\pgfqpoint{0.570343in}{0.331635in}}{\pgfqpoint{4.960000in}{3.696000in}}%
\pgfusepath{clip}%
\pgfsetbuttcap%
\pgfsetroundjoin%
\definecolor{currentfill}{rgb}{1.000000,0.705882,0.509804}%
\pgfsetfillcolor{currentfill}%
\pgfsetlinewidth{0.481800pt}%
\definecolor{currentstroke}{rgb}{1.000000,1.000000,1.000000}%
\pgfsetstrokecolor{currentstroke}%
\pgfsetdash{}{0pt}%
\pgfpathmoveto{\pgfqpoint{0.856192in}{1.529585in}}%
\pgfpathcurveto{\pgfqpoint{0.867242in}{1.529585in}}{\pgfqpoint{0.877841in}{1.533975in}}{\pgfqpoint{0.885655in}{1.541789in}}%
\pgfpathcurveto{\pgfqpoint{0.893468in}{1.549603in}}{\pgfqpoint{0.897859in}{1.560202in}}{\pgfqpoint{0.897859in}{1.571252in}}%
\pgfpathcurveto{\pgfqpoint{0.897859in}{1.582302in}}{\pgfqpoint{0.893468in}{1.592901in}}{\pgfqpoint{0.885655in}{1.600715in}}%
\pgfpathcurveto{\pgfqpoint{0.877841in}{1.608528in}}{\pgfqpoint{0.867242in}{1.612918in}}{\pgfqpoint{0.856192in}{1.612918in}}%
\pgfpathcurveto{\pgfqpoint{0.845142in}{1.612918in}}{\pgfqpoint{0.834543in}{1.608528in}}{\pgfqpoint{0.826729in}{1.600715in}}%
\pgfpathcurveto{\pgfqpoint{0.818916in}{1.592901in}}{\pgfqpoint{0.814525in}{1.582302in}}{\pgfqpoint{0.814525in}{1.571252in}}%
\pgfpathcurveto{\pgfqpoint{0.814525in}{1.560202in}}{\pgfqpoint{0.818916in}{1.549603in}}{\pgfqpoint{0.826729in}{1.541789in}}%
\pgfpathcurveto{\pgfqpoint{0.834543in}{1.533975in}}{\pgfqpoint{0.845142in}{1.529585in}}{\pgfqpoint{0.856192in}{1.529585in}}%
\pgfpathclose%
\pgfusepath{stroke,fill}%
\end{pgfscope}%
\begin{pgfscope}%
\pgfpathrectangle{\pgfqpoint{0.570343in}{0.331635in}}{\pgfqpoint{4.960000in}{3.696000in}}%
\pgfusepath{clip}%
\pgfsetbuttcap%
\pgfsetroundjoin%
\definecolor{currentfill}{rgb}{1.000000,0.705882,0.509804}%
\pgfsetfillcolor{currentfill}%
\pgfsetlinewidth{0.481800pt}%
\definecolor{currentstroke}{rgb}{1.000000,1.000000,1.000000}%
\pgfsetstrokecolor{currentstroke}%
\pgfsetdash{}{0pt}%
\pgfpathmoveto{\pgfqpoint{4.626113in}{3.279314in}}%
\pgfpathcurveto{\pgfqpoint{4.637163in}{3.279314in}}{\pgfqpoint{4.647762in}{3.283704in}}{\pgfqpoint{4.655575in}{3.291518in}}%
\pgfpathcurveto{\pgfqpoint{4.663389in}{3.299332in}}{\pgfqpoint{4.667779in}{3.309931in}}{\pgfqpoint{4.667779in}{3.320981in}}%
\pgfpathcurveto{\pgfqpoint{4.667779in}{3.332031in}}{\pgfqpoint{4.663389in}{3.342630in}}{\pgfqpoint{4.655575in}{3.350444in}}%
\pgfpathcurveto{\pgfqpoint{4.647762in}{3.358257in}}{\pgfqpoint{4.637163in}{3.362648in}}{\pgfqpoint{4.626113in}{3.362648in}}%
\pgfpathcurveto{\pgfqpoint{4.615062in}{3.362648in}}{\pgfqpoint{4.604463in}{3.358257in}}{\pgfqpoint{4.596650in}{3.350444in}}%
\pgfpathcurveto{\pgfqpoint{4.588836in}{3.342630in}}{\pgfqpoint{4.584446in}{3.332031in}}{\pgfqpoint{4.584446in}{3.320981in}}%
\pgfpathcurveto{\pgfqpoint{4.584446in}{3.309931in}}{\pgfqpoint{4.588836in}{3.299332in}}{\pgfqpoint{4.596650in}{3.291518in}}%
\pgfpathcurveto{\pgfqpoint{4.604463in}{3.283704in}}{\pgfqpoint{4.615062in}{3.279314in}}{\pgfqpoint{4.626113in}{3.279314in}}%
\pgfpathclose%
\pgfusepath{stroke,fill}%
\end{pgfscope}%
\begin{pgfscope}%
\pgfpathrectangle{\pgfqpoint{0.570343in}{0.331635in}}{\pgfqpoint{4.960000in}{3.696000in}}%
\pgfusepath{clip}%
\pgfsetbuttcap%
\pgfsetroundjoin%
\definecolor{currentfill}{rgb}{1.000000,0.705882,0.509804}%
\pgfsetfillcolor{currentfill}%
\pgfsetlinewidth{0.481800pt}%
\definecolor{currentstroke}{rgb}{1.000000,1.000000,1.000000}%
\pgfsetstrokecolor{currentstroke}%
\pgfsetdash{}{0pt}%
\pgfpathmoveto{\pgfqpoint{1.796127in}{3.311954in}}%
\pgfpathcurveto{\pgfqpoint{1.807177in}{3.311954in}}{\pgfqpoint{1.817776in}{3.316344in}}{\pgfqpoint{1.825589in}{3.324158in}}%
\pgfpathcurveto{\pgfqpoint{1.833403in}{3.331971in}}{\pgfqpoint{1.837793in}{3.342570in}}{\pgfqpoint{1.837793in}{3.353620in}}%
\pgfpathcurveto{\pgfqpoint{1.837793in}{3.364670in}}{\pgfqpoint{1.833403in}{3.375270in}}{\pgfqpoint{1.825589in}{3.383083in}}%
\pgfpathcurveto{\pgfqpoint{1.817776in}{3.390897in}}{\pgfqpoint{1.807177in}{3.395287in}}{\pgfqpoint{1.796127in}{3.395287in}}%
\pgfpathcurveto{\pgfqpoint{1.785076in}{3.395287in}}{\pgfqpoint{1.774477in}{3.390897in}}{\pgfqpoint{1.766664in}{3.383083in}}%
\pgfpathcurveto{\pgfqpoint{1.758850in}{3.375270in}}{\pgfqpoint{1.754460in}{3.364670in}}{\pgfqpoint{1.754460in}{3.353620in}}%
\pgfpathcurveto{\pgfqpoint{1.754460in}{3.342570in}}{\pgfqpoint{1.758850in}{3.331971in}}{\pgfqpoint{1.766664in}{3.324158in}}%
\pgfpathcurveto{\pgfqpoint{1.774477in}{3.316344in}}{\pgfqpoint{1.785076in}{3.311954in}}{\pgfqpoint{1.796127in}{3.311954in}}%
\pgfpathclose%
\pgfusepath{stroke,fill}%
\end{pgfscope}%
\begin{pgfscope}%
\pgfpathrectangle{\pgfqpoint{0.570343in}{0.331635in}}{\pgfqpoint{4.960000in}{3.696000in}}%
\pgfusepath{clip}%
\pgfsetbuttcap%
\pgfsetroundjoin%
\definecolor{currentfill}{rgb}{1.000000,0.705882,0.509804}%
\pgfsetfillcolor{currentfill}%
\pgfsetlinewidth{0.481800pt}%
\definecolor{currentstroke}{rgb}{1.000000,1.000000,1.000000}%
\pgfsetstrokecolor{currentstroke}%
\pgfsetdash{}{0pt}%
\pgfpathmoveto{\pgfqpoint{4.916919in}{1.019231in}}%
\pgfpathcurveto{\pgfqpoint{4.927969in}{1.019231in}}{\pgfqpoint{4.938568in}{1.023621in}}{\pgfqpoint{4.946382in}{1.031435in}}%
\pgfpathcurveto{\pgfqpoint{4.954195in}{1.039248in}}{\pgfqpoint{4.958586in}{1.049847in}}{\pgfqpoint{4.958586in}{1.060897in}}%
\pgfpathcurveto{\pgfqpoint{4.958586in}{1.071948in}}{\pgfqpoint{4.954195in}{1.082547in}}{\pgfqpoint{4.946382in}{1.090360in}}%
\pgfpathcurveto{\pgfqpoint{4.938568in}{1.098174in}}{\pgfqpoint{4.927969in}{1.102564in}}{\pgfqpoint{4.916919in}{1.102564in}}%
\pgfpathcurveto{\pgfqpoint{4.905869in}{1.102564in}}{\pgfqpoint{4.895270in}{1.098174in}}{\pgfqpoint{4.887456in}{1.090360in}}%
\pgfpathcurveto{\pgfqpoint{4.879643in}{1.082547in}}{\pgfqpoint{4.875252in}{1.071948in}}{\pgfqpoint{4.875252in}{1.060897in}}%
\pgfpathcurveto{\pgfqpoint{4.875252in}{1.049847in}}{\pgfqpoint{4.879643in}{1.039248in}}{\pgfqpoint{4.887456in}{1.031435in}}%
\pgfpathcurveto{\pgfqpoint{4.895270in}{1.023621in}}{\pgfqpoint{4.905869in}{1.019231in}}{\pgfqpoint{4.916919in}{1.019231in}}%
\pgfpathclose%
\pgfusepath{stroke,fill}%
\end{pgfscope}%
\begin{pgfscope}%
\pgfpathrectangle{\pgfqpoint{0.570343in}{0.331635in}}{\pgfqpoint{4.960000in}{3.696000in}}%
\pgfusepath{clip}%
\pgfsetbuttcap%
\pgfsetroundjoin%
\definecolor{currentfill}{rgb}{1.000000,0.705882,0.509804}%
\pgfsetfillcolor{currentfill}%
\pgfsetlinewidth{0.481800pt}%
\definecolor{currentstroke}{rgb}{1.000000,1.000000,1.000000}%
\pgfsetstrokecolor{currentstroke}%
\pgfsetdash{}{0pt}%
\pgfpathmoveto{\pgfqpoint{4.374254in}{2.132308in}}%
\pgfpathcurveto{\pgfqpoint{4.385304in}{2.132308in}}{\pgfqpoint{4.395903in}{2.136698in}}{\pgfqpoint{4.403717in}{2.144511in}}%
\pgfpathcurveto{\pgfqpoint{4.411531in}{2.152325in}}{\pgfqpoint{4.415921in}{2.162924in}}{\pgfqpoint{4.415921in}{2.173974in}}%
\pgfpathcurveto{\pgfqpoint{4.415921in}{2.185024in}}{\pgfqpoint{4.411531in}{2.195623in}}{\pgfqpoint{4.403717in}{2.203437in}}%
\pgfpathcurveto{\pgfqpoint{4.395903in}{2.211251in}}{\pgfqpoint{4.385304in}{2.215641in}}{\pgfqpoint{4.374254in}{2.215641in}}%
\pgfpathcurveto{\pgfqpoint{4.363204in}{2.215641in}}{\pgfqpoint{4.352605in}{2.211251in}}{\pgfqpoint{4.344792in}{2.203437in}}%
\pgfpathcurveto{\pgfqpoint{4.336978in}{2.195623in}}{\pgfqpoint{4.332588in}{2.185024in}}{\pgfqpoint{4.332588in}{2.173974in}}%
\pgfpathcurveto{\pgfqpoint{4.332588in}{2.162924in}}{\pgfqpoint{4.336978in}{2.152325in}}{\pgfqpoint{4.344792in}{2.144511in}}%
\pgfpathcurveto{\pgfqpoint{4.352605in}{2.136698in}}{\pgfqpoint{4.363204in}{2.132308in}}{\pgfqpoint{4.374254in}{2.132308in}}%
\pgfpathclose%
\pgfusepath{stroke,fill}%
\end{pgfscope}%
\begin{pgfscope}%
\pgfpathrectangle{\pgfqpoint{0.570343in}{0.331635in}}{\pgfqpoint{4.960000in}{3.696000in}}%
\pgfusepath{clip}%
\pgfsetbuttcap%
\pgfsetroundjoin%
\definecolor{currentfill}{rgb}{1.000000,0.705882,0.509804}%
\pgfsetfillcolor{currentfill}%
\pgfsetlinewidth{0.481800pt}%
\definecolor{currentstroke}{rgb}{1.000000,1.000000,1.000000}%
\pgfsetstrokecolor{currentstroke}%
\pgfsetdash{}{0pt}%
\pgfpathmoveto{\pgfqpoint{3.704677in}{2.903484in}}%
\pgfpathcurveto{\pgfqpoint{3.715727in}{2.903484in}}{\pgfqpoint{3.726326in}{2.907875in}}{\pgfqpoint{3.734140in}{2.915688in}}%
\pgfpathcurveto{\pgfqpoint{3.741953in}{2.923502in}}{\pgfqpoint{3.746344in}{2.934101in}}{\pgfqpoint{3.746344in}{2.945151in}}%
\pgfpathcurveto{\pgfqpoint{3.746344in}{2.956201in}}{\pgfqpoint{3.741953in}{2.966800in}}{\pgfqpoint{3.734140in}{2.974614in}}%
\pgfpathcurveto{\pgfqpoint{3.726326in}{2.982427in}}{\pgfqpoint{3.715727in}{2.986818in}}{\pgfqpoint{3.704677in}{2.986818in}}%
\pgfpathcurveto{\pgfqpoint{3.693627in}{2.986818in}}{\pgfqpoint{3.683028in}{2.982427in}}{\pgfqpoint{3.675214in}{2.974614in}}%
\pgfpathcurveto{\pgfqpoint{3.667401in}{2.966800in}}{\pgfqpoint{3.663010in}{2.956201in}}{\pgfqpoint{3.663010in}{2.945151in}}%
\pgfpathcurveto{\pgfqpoint{3.663010in}{2.934101in}}{\pgfqpoint{3.667401in}{2.923502in}}{\pgfqpoint{3.675214in}{2.915688in}}%
\pgfpathcurveto{\pgfqpoint{3.683028in}{2.907875in}}{\pgfqpoint{3.693627in}{2.903484in}}{\pgfqpoint{3.704677in}{2.903484in}}%
\pgfpathclose%
\pgfusepath{stroke,fill}%
\end{pgfscope}%
\begin{pgfscope}%
\pgfpathrectangle{\pgfqpoint{0.570343in}{0.331635in}}{\pgfqpoint{4.960000in}{3.696000in}}%
\pgfusepath{clip}%
\pgfsetbuttcap%
\pgfsetroundjoin%
\definecolor{currentfill}{rgb}{1.000000,0.705882,0.509804}%
\pgfsetfillcolor{currentfill}%
\pgfsetlinewidth{0.481800pt}%
\definecolor{currentstroke}{rgb}{1.000000,1.000000,1.000000}%
\pgfsetstrokecolor{currentstroke}%
\pgfsetdash{}{0pt}%
\pgfpathmoveto{\pgfqpoint{0.795798in}{2.919565in}}%
\pgfpathcurveto{\pgfqpoint{0.806848in}{2.919565in}}{\pgfqpoint{0.817447in}{2.923955in}}{\pgfqpoint{0.825261in}{2.931769in}}%
\pgfpathcurveto{\pgfqpoint{0.833074in}{2.939582in}}{\pgfqpoint{0.837465in}{2.950181in}}{\pgfqpoint{0.837465in}{2.961231in}}%
\pgfpathcurveto{\pgfqpoint{0.837465in}{2.972282in}}{\pgfqpoint{0.833074in}{2.982881in}}{\pgfqpoint{0.825261in}{2.990694in}}%
\pgfpathcurveto{\pgfqpoint{0.817447in}{2.998508in}}{\pgfqpoint{0.806848in}{3.002898in}}{\pgfqpoint{0.795798in}{3.002898in}}%
\pgfpathcurveto{\pgfqpoint{0.784748in}{3.002898in}}{\pgfqpoint{0.774149in}{2.998508in}}{\pgfqpoint{0.766335in}{2.990694in}}%
\pgfpathcurveto{\pgfqpoint{0.758521in}{2.982881in}}{\pgfqpoint{0.754131in}{2.972282in}}{\pgfqpoint{0.754131in}{2.961231in}}%
\pgfpathcurveto{\pgfqpoint{0.754131in}{2.950181in}}{\pgfqpoint{0.758521in}{2.939582in}}{\pgfqpoint{0.766335in}{2.931769in}}%
\pgfpathcurveto{\pgfqpoint{0.774149in}{2.923955in}}{\pgfqpoint{0.784748in}{2.919565in}}{\pgfqpoint{0.795798in}{2.919565in}}%
\pgfpathclose%
\pgfusepath{stroke,fill}%
\end{pgfscope}%
\begin{pgfscope}%
\pgfpathrectangle{\pgfqpoint{0.570343in}{0.331635in}}{\pgfqpoint{4.960000in}{3.696000in}}%
\pgfusepath{clip}%
\pgfsetbuttcap%
\pgfsetroundjoin%
\definecolor{currentfill}{rgb}{1.000000,0.705882,0.509804}%
\pgfsetfillcolor{currentfill}%
\pgfsetlinewidth{0.481800pt}%
\definecolor{currentstroke}{rgb}{1.000000,1.000000,1.000000}%
\pgfsetstrokecolor{currentstroke}%
\pgfsetdash{}{0pt}%
\pgfpathmoveto{\pgfqpoint{1.844531in}{3.817968in}}%
\pgfpathcurveto{\pgfqpoint{1.855581in}{3.817968in}}{\pgfqpoint{1.866180in}{3.822359in}}{\pgfqpoint{1.873994in}{3.830172in}}%
\pgfpathcurveto{\pgfqpoint{1.881807in}{3.837986in}}{\pgfqpoint{1.886198in}{3.848585in}}{\pgfqpoint{1.886198in}{3.859635in}}%
\pgfpathcurveto{\pgfqpoint{1.886198in}{3.870685in}}{\pgfqpoint{1.881807in}{3.881284in}}{\pgfqpoint{1.873994in}{3.889098in}}%
\pgfpathcurveto{\pgfqpoint{1.866180in}{3.896911in}}{\pgfqpoint{1.855581in}{3.901302in}}{\pgfqpoint{1.844531in}{3.901302in}}%
\pgfpathcurveto{\pgfqpoint{1.833481in}{3.901302in}}{\pgfqpoint{1.822882in}{3.896911in}}{\pgfqpoint{1.815068in}{3.889098in}}%
\pgfpathcurveto{\pgfqpoint{1.807254in}{3.881284in}}{\pgfqpoint{1.802864in}{3.870685in}}{\pgfqpoint{1.802864in}{3.859635in}}%
\pgfpathcurveto{\pgfqpoint{1.802864in}{3.848585in}}{\pgfqpoint{1.807254in}{3.837986in}}{\pgfqpoint{1.815068in}{3.830172in}}%
\pgfpathcurveto{\pgfqpoint{1.822882in}{3.822359in}}{\pgfqpoint{1.833481in}{3.817968in}}{\pgfqpoint{1.844531in}{3.817968in}}%
\pgfpathclose%
\pgfusepath{stroke,fill}%
\end{pgfscope}%
\begin{pgfscope}%
\pgfpathrectangle{\pgfqpoint{0.570343in}{0.331635in}}{\pgfqpoint{4.960000in}{3.696000in}}%
\pgfusepath{clip}%
\pgfsetbuttcap%
\pgfsetroundjoin%
\definecolor{currentfill}{rgb}{1.000000,0.705882,0.509804}%
\pgfsetfillcolor{currentfill}%
\pgfsetlinewidth{0.481800pt}%
\definecolor{currentstroke}{rgb}{1.000000,1.000000,1.000000}%
\pgfsetstrokecolor{currentstroke}%
\pgfsetdash{}{0pt}%
\pgfpathmoveto{\pgfqpoint{4.720753in}{1.443618in}}%
\pgfpathcurveto{\pgfqpoint{4.731803in}{1.443618in}}{\pgfqpoint{4.742402in}{1.448008in}}{\pgfqpoint{4.750216in}{1.455822in}}%
\pgfpathcurveto{\pgfqpoint{4.758030in}{1.463636in}}{\pgfqpoint{4.762420in}{1.474235in}}{\pgfqpoint{4.762420in}{1.485285in}}%
\pgfpathcurveto{\pgfqpoint{4.762420in}{1.496335in}}{\pgfqpoint{4.758030in}{1.506934in}}{\pgfqpoint{4.750216in}{1.514748in}}%
\pgfpathcurveto{\pgfqpoint{4.742402in}{1.522561in}}{\pgfqpoint{4.731803in}{1.526951in}}{\pgfqpoint{4.720753in}{1.526951in}}%
\pgfpathcurveto{\pgfqpoint{4.709703in}{1.526951in}}{\pgfqpoint{4.699104in}{1.522561in}}{\pgfqpoint{4.691290in}{1.514748in}}%
\pgfpathcurveto{\pgfqpoint{4.683477in}{1.506934in}}{\pgfqpoint{4.679086in}{1.496335in}}{\pgfqpoint{4.679086in}{1.485285in}}%
\pgfpathcurveto{\pgfqpoint{4.679086in}{1.474235in}}{\pgfqpoint{4.683477in}{1.463636in}}{\pgfqpoint{4.691290in}{1.455822in}}%
\pgfpathcurveto{\pgfqpoint{4.699104in}{1.448008in}}{\pgfqpoint{4.709703in}{1.443618in}}{\pgfqpoint{4.720753in}{1.443618in}}%
\pgfpathclose%
\pgfusepath{stroke,fill}%
\end{pgfscope}%
\begin{pgfscope}%
\pgfpathrectangle{\pgfqpoint{0.570343in}{0.331635in}}{\pgfqpoint{4.960000in}{3.696000in}}%
\pgfusepath{clip}%
\pgfsetbuttcap%
\pgfsetroundjoin%
\definecolor{currentfill}{rgb}{1.000000,0.705882,0.509804}%
\pgfsetfillcolor{currentfill}%
\pgfsetlinewidth{0.481800pt}%
\definecolor{currentstroke}{rgb}{1.000000,1.000000,1.000000}%
\pgfsetstrokecolor{currentstroke}%
\pgfsetdash{}{0pt}%
\pgfpathmoveto{\pgfqpoint{2.224277in}{0.457968in}}%
\pgfpathcurveto{\pgfqpoint{2.235327in}{0.457968in}}{\pgfqpoint{2.245926in}{0.462359in}}{\pgfqpoint{2.253740in}{0.470172in}}%
\pgfpathcurveto{\pgfqpoint{2.261554in}{0.477986in}}{\pgfqpoint{2.265944in}{0.488585in}}{\pgfqpoint{2.265944in}{0.499635in}}%
\pgfpathcurveto{\pgfqpoint{2.265944in}{0.510685in}}{\pgfqpoint{2.261554in}{0.521284in}}{\pgfqpoint{2.253740in}{0.529098in}}%
\pgfpathcurveto{\pgfqpoint{2.245926in}{0.536911in}}{\pgfqpoint{2.235327in}{0.541302in}}{\pgfqpoint{2.224277in}{0.541302in}}%
\pgfpathcurveto{\pgfqpoint{2.213227in}{0.541302in}}{\pgfqpoint{2.202628in}{0.536911in}}{\pgfqpoint{2.194815in}{0.529098in}}%
\pgfpathcurveto{\pgfqpoint{2.187001in}{0.521284in}}{\pgfqpoint{2.182611in}{0.510685in}}{\pgfqpoint{2.182611in}{0.499635in}}%
\pgfpathcurveto{\pgfqpoint{2.182611in}{0.488585in}}{\pgfqpoint{2.187001in}{0.477986in}}{\pgfqpoint{2.194815in}{0.470172in}}%
\pgfpathcurveto{\pgfqpoint{2.202628in}{0.462359in}}{\pgfqpoint{2.213227in}{0.457968in}}{\pgfqpoint{2.224277in}{0.457968in}}%
\pgfpathclose%
\pgfusepath{stroke,fill}%
\end{pgfscope}%
\begin{pgfscope}%
\pgfpathrectangle{\pgfqpoint{0.570343in}{0.331635in}}{\pgfqpoint{4.960000in}{3.696000in}}%
\pgfusepath{clip}%
\pgfsetbuttcap%
\pgfsetroundjoin%
\definecolor{currentfill}{rgb}{1.000000,0.705882,0.509804}%
\pgfsetfillcolor{currentfill}%
\pgfsetlinewidth{0.481800pt}%
\definecolor{currentstroke}{rgb}{1.000000,1.000000,1.000000}%
\pgfsetstrokecolor{currentstroke}%
\pgfsetdash{}{0pt}%
\pgfpathmoveto{\pgfqpoint{2.748227in}{3.639322in}}%
\pgfpathcurveto{\pgfqpoint{2.759277in}{3.639322in}}{\pgfqpoint{2.769876in}{3.643712in}}{\pgfqpoint{2.777690in}{3.651526in}}%
\pgfpathcurveto{\pgfqpoint{2.785503in}{3.659339in}}{\pgfqpoint{2.789894in}{3.669938in}}{\pgfqpoint{2.789894in}{3.680989in}}%
\pgfpathcurveto{\pgfqpoint{2.789894in}{3.692039in}}{\pgfqpoint{2.785503in}{3.702638in}}{\pgfqpoint{2.777690in}{3.710451in}}%
\pgfpathcurveto{\pgfqpoint{2.769876in}{3.718265in}}{\pgfqpoint{2.759277in}{3.722655in}}{\pgfqpoint{2.748227in}{3.722655in}}%
\pgfpathcurveto{\pgfqpoint{2.737177in}{3.722655in}}{\pgfqpoint{2.726578in}{3.718265in}}{\pgfqpoint{2.718764in}{3.710451in}}%
\pgfpathcurveto{\pgfqpoint{2.710951in}{3.702638in}}{\pgfqpoint{2.706560in}{3.692039in}}{\pgfqpoint{2.706560in}{3.680989in}}%
\pgfpathcurveto{\pgfqpoint{2.706560in}{3.669938in}}{\pgfqpoint{2.710951in}{3.659339in}}{\pgfqpoint{2.718764in}{3.651526in}}%
\pgfpathcurveto{\pgfqpoint{2.726578in}{3.643712in}}{\pgfqpoint{2.737177in}{3.639322in}}{\pgfqpoint{2.748227in}{3.639322in}}%
\pgfpathclose%
\pgfusepath{stroke,fill}%
\end{pgfscope}%
\begin{pgfscope}%
\pgfpathrectangle{\pgfqpoint{0.570343in}{0.331635in}}{\pgfqpoint{4.960000in}{3.696000in}}%
\pgfusepath{clip}%
\pgfsetbuttcap%
\pgfsetroundjoin%
\definecolor{currentfill}{rgb}{1.000000,0.705882,0.509804}%
\pgfsetfillcolor{currentfill}%
\pgfsetlinewidth{0.481800pt}%
\definecolor{currentstroke}{rgb}{1.000000,1.000000,1.000000}%
\pgfsetstrokecolor{currentstroke}%
\pgfsetdash{}{0pt}%
\pgfpathmoveto{\pgfqpoint{1.166400in}{2.379582in}}%
\pgfpathcurveto{\pgfqpoint{1.177450in}{2.379582in}}{\pgfqpoint{1.188049in}{2.383972in}}{\pgfqpoint{1.195863in}{2.391785in}}%
\pgfpathcurveto{\pgfqpoint{1.203676in}{2.399599in}}{\pgfqpoint{1.208067in}{2.410198in}}{\pgfqpoint{1.208067in}{2.421248in}}%
\pgfpathcurveto{\pgfqpoint{1.208067in}{2.432298in}}{\pgfqpoint{1.203676in}{2.442897in}}{\pgfqpoint{1.195863in}{2.450711in}}%
\pgfpathcurveto{\pgfqpoint{1.188049in}{2.458525in}}{\pgfqpoint{1.177450in}{2.462915in}}{\pgfqpoint{1.166400in}{2.462915in}}%
\pgfpathcurveto{\pgfqpoint{1.155350in}{2.462915in}}{\pgfqpoint{1.144751in}{2.458525in}}{\pgfqpoint{1.136937in}{2.450711in}}%
\pgfpathcurveto{\pgfqpoint{1.129123in}{2.442897in}}{\pgfqpoint{1.124733in}{2.432298in}}{\pgfqpoint{1.124733in}{2.421248in}}%
\pgfpathcurveto{\pgfqpoint{1.124733in}{2.410198in}}{\pgfqpoint{1.129123in}{2.399599in}}{\pgfqpoint{1.136937in}{2.391785in}}%
\pgfpathcurveto{\pgfqpoint{1.144751in}{2.383972in}}{\pgfqpoint{1.155350in}{2.379582in}}{\pgfqpoint{1.166400in}{2.379582in}}%
\pgfpathclose%
\pgfusepath{stroke,fill}%
\end{pgfscope}%
\begin{pgfscope}%
\pgfpathrectangle{\pgfqpoint{0.570343in}{0.331635in}}{\pgfqpoint{4.960000in}{3.696000in}}%
\pgfusepath{clip}%
\pgfsetbuttcap%
\pgfsetroundjoin%
\definecolor{currentfill}{rgb}{1.000000,0.705882,0.509804}%
\pgfsetfillcolor{currentfill}%
\pgfsetlinewidth{0.481800pt}%
\definecolor{currentstroke}{rgb}{1.000000,1.000000,1.000000}%
\pgfsetstrokecolor{currentstroke}%
\pgfsetdash{}{0pt}%
\pgfpathmoveto{\pgfqpoint{1.664925in}{2.487393in}}%
\pgfpathcurveto{\pgfqpoint{1.675975in}{2.487393in}}{\pgfqpoint{1.686574in}{2.491784in}}{\pgfqpoint{1.694388in}{2.499597in}}%
\pgfpathcurveto{\pgfqpoint{1.702201in}{2.507411in}}{\pgfqpoint{1.706592in}{2.518010in}}{\pgfqpoint{1.706592in}{2.529060in}}%
\pgfpathcurveto{\pgfqpoint{1.706592in}{2.540110in}}{\pgfqpoint{1.702201in}{2.550709in}}{\pgfqpoint{1.694388in}{2.558523in}}%
\pgfpathcurveto{\pgfqpoint{1.686574in}{2.566336in}}{\pgfqpoint{1.675975in}{2.570727in}}{\pgfqpoint{1.664925in}{2.570727in}}%
\pgfpathcurveto{\pgfqpoint{1.653875in}{2.570727in}}{\pgfqpoint{1.643276in}{2.566336in}}{\pgfqpoint{1.635462in}{2.558523in}}%
\pgfpathcurveto{\pgfqpoint{1.627649in}{2.550709in}}{\pgfqpoint{1.623258in}{2.540110in}}{\pgfqpoint{1.623258in}{2.529060in}}%
\pgfpathcurveto{\pgfqpoint{1.623258in}{2.518010in}}{\pgfqpoint{1.627649in}{2.507411in}}{\pgfqpoint{1.635462in}{2.499597in}}%
\pgfpathcurveto{\pgfqpoint{1.643276in}{2.491784in}}{\pgfqpoint{1.653875in}{2.487393in}}{\pgfqpoint{1.664925in}{2.487393in}}%
\pgfpathclose%
\pgfusepath{stroke,fill}%
\end{pgfscope}%
\begin{pgfscope}%
\pgfpathrectangle{\pgfqpoint{0.570343in}{0.331635in}}{\pgfqpoint{4.960000in}{3.696000in}}%
\pgfusepath{clip}%
\pgfsetbuttcap%
\pgfsetroundjoin%
\definecolor{currentfill}{rgb}{1.000000,0.705882,0.509804}%
\pgfsetfillcolor{currentfill}%
\pgfsetlinewidth{0.481800pt}%
\definecolor{currentstroke}{rgb}{1.000000,1.000000,1.000000}%
\pgfsetstrokecolor{currentstroke}%
\pgfsetdash{}{0pt}%
\pgfpathmoveto{\pgfqpoint{4.177702in}{2.785347in}}%
\pgfpathcurveto{\pgfqpoint{4.188752in}{2.785347in}}{\pgfqpoint{4.199351in}{2.789737in}}{\pgfqpoint{4.207165in}{2.797551in}}%
\pgfpathcurveto{\pgfqpoint{4.214979in}{2.805365in}}{\pgfqpoint{4.219369in}{2.815964in}}{\pgfqpoint{4.219369in}{2.827014in}}%
\pgfpathcurveto{\pgfqpoint{4.219369in}{2.838064in}}{\pgfqpoint{4.214979in}{2.848663in}}{\pgfqpoint{4.207165in}{2.856476in}}%
\pgfpathcurveto{\pgfqpoint{4.199351in}{2.864290in}}{\pgfqpoint{4.188752in}{2.868680in}}{\pgfqpoint{4.177702in}{2.868680in}}%
\pgfpathcurveto{\pgfqpoint{4.166652in}{2.868680in}}{\pgfqpoint{4.156053in}{2.864290in}}{\pgfqpoint{4.148239in}{2.856476in}}%
\pgfpathcurveto{\pgfqpoint{4.140426in}{2.848663in}}{\pgfqpoint{4.136036in}{2.838064in}}{\pgfqpoint{4.136036in}{2.827014in}}%
\pgfpathcurveto{\pgfqpoint{4.136036in}{2.815964in}}{\pgfqpoint{4.140426in}{2.805365in}}{\pgfqpoint{4.148239in}{2.797551in}}%
\pgfpathcurveto{\pgfqpoint{4.156053in}{2.789737in}}{\pgfqpoint{4.166652in}{2.785347in}}{\pgfqpoint{4.177702in}{2.785347in}}%
\pgfpathclose%
\pgfusepath{stroke,fill}%
\end{pgfscope}%
\begin{pgfscope}%
\pgfpathrectangle{\pgfqpoint{0.570343in}{0.331635in}}{\pgfqpoint{4.960000in}{3.696000in}}%
\pgfusepath{clip}%
\pgfsetbuttcap%
\pgfsetroundjoin%
\definecolor{currentfill}{rgb}{1.000000,0.705882,0.509804}%
\pgfsetfillcolor{currentfill}%
\pgfsetlinewidth{0.481800pt}%
\definecolor{currentstroke}{rgb}{1.000000,1.000000,1.000000}%
\pgfsetstrokecolor{currentstroke}%
\pgfsetdash{}{0pt}%
\pgfpathmoveto{\pgfqpoint{3.492167in}{2.375842in}}%
\pgfpathcurveto{\pgfqpoint{3.503218in}{2.375842in}}{\pgfqpoint{3.513817in}{2.380232in}}{\pgfqpoint{3.521630in}{2.388046in}}%
\pgfpathcurveto{\pgfqpoint{3.529444in}{2.395860in}}{\pgfqpoint{3.533834in}{2.406459in}}{\pgfqpoint{3.533834in}{2.417509in}}%
\pgfpathcurveto{\pgfqpoint{3.533834in}{2.428559in}}{\pgfqpoint{3.529444in}{2.439158in}}{\pgfqpoint{3.521630in}{2.446972in}}%
\pgfpathcurveto{\pgfqpoint{3.513817in}{2.454785in}}{\pgfqpoint{3.503218in}{2.459176in}}{\pgfqpoint{3.492167in}{2.459176in}}%
\pgfpathcurveto{\pgfqpoint{3.481117in}{2.459176in}}{\pgfqpoint{3.470518in}{2.454785in}}{\pgfqpoint{3.462705in}{2.446972in}}%
\pgfpathcurveto{\pgfqpoint{3.454891in}{2.439158in}}{\pgfqpoint{3.450501in}{2.428559in}}{\pgfqpoint{3.450501in}{2.417509in}}%
\pgfpathcurveto{\pgfqpoint{3.450501in}{2.406459in}}{\pgfqpoint{3.454891in}{2.395860in}}{\pgfqpoint{3.462705in}{2.388046in}}%
\pgfpathcurveto{\pgfqpoint{3.470518in}{2.380232in}}{\pgfqpoint{3.481117in}{2.375842in}}{\pgfqpoint{3.492167in}{2.375842in}}%
\pgfpathclose%
\pgfusepath{stroke,fill}%
\end{pgfscope}%
\begin{pgfscope}%
\pgfpathrectangle{\pgfqpoint{0.570343in}{0.331635in}}{\pgfqpoint{4.960000in}{3.696000in}}%
\pgfusepath{clip}%
\pgfsetbuttcap%
\pgfsetroundjoin%
\definecolor{currentfill}{rgb}{1.000000,0.705882,0.509804}%
\pgfsetfillcolor{currentfill}%
\pgfsetlinewidth{0.481800pt}%
\definecolor{currentstroke}{rgb}{1.000000,1.000000,1.000000}%
\pgfsetstrokecolor{currentstroke}%
\pgfsetdash{}{0pt}%
\pgfpathmoveto{\pgfqpoint{2.226219in}{3.480126in}}%
\pgfpathcurveto{\pgfqpoint{2.237269in}{3.480126in}}{\pgfqpoint{2.247868in}{3.484517in}}{\pgfqpoint{2.255682in}{3.492330in}}%
\pgfpathcurveto{\pgfqpoint{2.263495in}{3.500144in}}{\pgfqpoint{2.267885in}{3.510743in}}{\pgfqpoint{2.267885in}{3.521793in}}%
\pgfpathcurveto{\pgfqpoint{2.267885in}{3.532843in}}{\pgfqpoint{2.263495in}{3.543442in}}{\pgfqpoint{2.255682in}{3.551256in}}%
\pgfpathcurveto{\pgfqpoint{2.247868in}{3.559070in}}{\pgfqpoint{2.237269in}{3.563460in}}{\pgfqpoint{2.226219in}{3.563460in}}%
\pgfpathcurveto{\pgfqpoint{2.215169in}{3.563460in}}{\pgfqpoint{2.204570in}{3.559070in}}{\pgfqpoint{2.196756in}{3.551256in}}%
\pgfpathcurveto{\pgfqpoint{2.188942in}{3.543442in}}{\pgfqpoint{2.184552in}{3.532843in}}{\pgfqpoint{2.184552in}{3.521793in}}%
\pgfpathcurveto{\pgfqpoint{2.184552in}{3.510743in}}{\pgfqpoint{2.188942in}{3.500144in}}{\pgfqpoint{2.196756in}{3.492330in}}%
\pgfpathcurveto{\pgfqpoint{2.204570in}{3.484517in}}{\pgfqpoint{2.215169in}{3.480126in}}{\pgfqpoint{2.226219in}{3.480126in}}%
\pgfpathclose%
\pgfusepath{stroke,fill}%
\end{pgfscope}%
\begin{pgfscope}%
\pgfpathrectangle{\pgfqpoint{0.570343in}{0.331635in}}{\pgfqpoint{4.960000in}{3.696000in}}%
\pgfusepath{clip}%
\pgfsetbuttcap%
\pgfsetroundjoin%
\definecolor{currentfill}{rgb}{1.000000,0.705882,0.509804}%
\pgfsetfillcolor{currentfill}%
\pgfsetlinewidth{0.481800pt}%
\definecolor{currentstroke}{rgb}{1.000000,1.000000,1.000000}%
\pgfsetstrokecolor{currentstroke}%
\pgfsetdash{}{0pt}%
\pgfpathmoveto{\pgfqpoint{3.231645in}{2.827291in}}%
\pgfpathcurveto{\pgfqpoint{3.242695in}{2.827291in}}{\pgfqpoint{3.253294in}{2.831681in}}{\pgfqpoint{3.261107in}{2.839494in}}%
\pgfpathcurveto{\pgfqpoint{3.268921in}{2.847308in}}{\pgfqpoint{3.273311in}{2.857907in}}{\pgfqpoint{3.273311in}{2.868957in}}%
\pgfpathcurveto{\pgfqpoint{3.273311in}{2.880007in}}{\pgfqpoint{3.268921in}{2.890606in}}{\pgfqpoint{3.261107in}{2.898420in}}%
\pgfpathcurveto{\pgfqpoint{3.253294in}{2.906234in}}{\pgfqpoint{3.242695in}{2.910624in}}{\pgfqpoint{3.231645in}{2.910624in}}%
\pgfpathcurveto{\pgfqpoint{3.220595in}{2.910624in}}{\pgfqpoint{3.209996in}{2.906234in}}{\pgfqpoint{3.202182in}{2.898420in}}%
\pgfpathcurveto{\pgfqpoint{3.194368in}{2.890606in}}{\pgfqpoint{3.189978in}{2.880007in}}{\pgfqpoint{3.189978in}{2.868957in}}%
\pgfpathcurveto{\pgfqpoint{3.189978in}{2.857907in}}{\pgfqpoint{3.194368in}{2.847308in}}{\pgfqpoint{3.202182in}{2.839494in}}%
\pgfpathcurveto{\pgfqpoint{3.209996in}{2.831681in}}{\pgfqpoint{3.220595in}{2.827291in}}{\pgfqpoint{3.231645in}{2.827291in}}%
\pgfpathclose%
\pgfusepath{stroke,fill}%
\end{pgfscope}%
\begin{pgfscope}%
\pgfpathrectangle{\pgfqpoint{0.570343in}{0.331635in}}{\pgfqpoint{4.960000in}{3.696000in}}%
\pgfusepath{clip}%
\pgfsetbuttcap%
\pgfsetroundjoin%
\definecolor{currentfill}{rgb}{0.631373,0.788235,0.956863}%
\pgfsetfillcolor{currentfill}%
\pgfsetlinewidth{1.003750pt}%
\definecolor{currentstroke}{rgb}{0.631373,0.788235,0.956863}%
\pgfsetstrokecolor{currentstroke}%
\pgfsetdash{}{0pt}%
\pgfsys@defobject{currentmarker}{\pgfqpoint{-0.041667in}{-0.041667in}}{\pgfqpoint{0.041667in}{0.041667in}}{%
\pgfpathmoveto{\pgfqpoint{0.000000in}{-0.041667in}}%
\pgfpathcurveto{\pgfqpoint{0.011050in}{-0.041667in}}{\pgfqpoint{0.021649in}{-0.037276in}}{\pgfqpoint{0.029463in}{-0.029463in}}%
\pgfpathcurveto{\pgfqpoint{0.037276in}{-0.021649in}}{\pgfqpoint{0.041667in}{-0.011050in}}{\pgfqpoint{0.041667in}{0.000000in}}%
\pgfpathcurveto{\pgfqpoint{0.041667in}{0.011050in}}{\pgfqpoint{0.037276in}{0.021649in}}{\pgfqpoint{0.029463in}{0.029463in}}%
\pgfpathcurveto{\pgfqpoint{0.021649in}{0.037276in}}{\pgfqpoint{0.011050in}{0.041667in}}{\pgfqpoint{0.000000in}{0.041667in}}%
\pgfpathcurveto{\pgfqpoint{-0.011050in}{0.041667in}}{\pgfqpoint{-0.021649in}{0.037276in}}{\pgfqpoint{-0.029463in}{0.029463in}}%
\pgfpathcurveto{\pgfqpoint{-0.037276in}{0.021649in}}{\pgfqpoint{-0.041667in}{0.011050in}}{\pgfqpoint{-0.041667in}{0.000000in}}%
\pgfpathcurveto{\pgfqpoint{-0.041667in}{-0.011050in}}{\pgfqpoint{-0.037276in}{-0.021649in}}{\pgfqpoint{-0.029463in}{-0.029463in}}%
\pgfpathcurveto{\pgfqpoint{-0.021649in}{-0.037276in}}{\pgfqpoint{-0.011050in}{-0.041667in}}{\pgfqpoint{0.000000in}{-0.041667in}}%
\pgfpathclose%
\pgfusepath{stroke,fill}%
}%
\end{pgfscope}%
\begin{pgfscope}%
\pgfpathrectangle{\pgfqpoint{0.570343in}{0.331635in}}{\pgfqpoint{4.960000in}{3.696000in}}%
\pgfusepath{clip}%
\pgfsetbuttcap%
\pgfsetroundjoin%
\definecolor{currentfill}{rgb}{1.000000,0.705882,0.509804}%
\pgfsetfillcolor{currentfill}%
\pgfsetlinewidth{1.003750pt}%
\definecolor{currentstroke}{rgb}{1.000000,0.705882,0.509804}%
\pgfsetstrokecolor{currentstroke}%
\pgfsetdash{}{0pt}%
\pgfsys@defobject{currentmarker}{\pgfqpoint{-0.041667in}{-0.041667in}}{\pgfqpoint{0.041667in}{0.041667in}}{%
\pgfpathmoveto{\pgfqpoint{0.000000in}{-0.041667in}}%
\pgfpathcurveto{\pgfqpoint{0.011050in}{-0.041667in}}{\pgfqpoint{0.021649in}{-0.037276in}}{\pgfqpoint{0.029463in}{-0.029463in}}%
\pgfpathcurveto{\pgfqpoint{0.037276in}{-0.021649in}}{\pgfqpoint{0.041667in}{-0.011050in}}{\pgfqpoint{0.041667in}{0.000000in}}%
\pgfpathcurveto{\pgfqpoint{0.041667in}{0.011050in}}{\pgfqpoint{0.037276in}{0.021649in}}{\pgfqpoint{0.029463in}{0.029463in}}%
\pgfpathcurveto{\pgfqpoint{0.021649in}{0.037276in}}{\pgfqpoint{0.011050in}{0.041667in}}{\pgfqpoint{0.000000in}{0.041667in}}%
\pgfpathcurveto{\pgfqpoint{-0.011050in}{0.041667in}}{\pgfqpoint{-0.021649in}{0.037276in}}{\pgfqpoint{-0.029463in}{0.029463in}}%
\pgfpathcurveto{\pgfqpoint{-0.037276in}{0.021649in}}{\pgfqpoint{-0.041667in}{0.011050in}}{\pgfqpoint{-0.041667in}{0.000000in}}%
\pgfpathcurveto{\pgfqpoint{-0.041667in}{-0.011050in}}{\pgfqpoint{-0.037276in}{-0.021649in}}{\pgfqpoint{-0.029463in}{-0.029463in}}%
\pgfpathcurveto{\pgfqpoint{-0.021649in}{-0.037276in}}{\pgfqpoint{-0.011050in}{-0.041667in}}{\pgfqpoint{0.000000in}{-0.041667in}}%
\pgfpathclose%
\pgfusepath{stroke,fill}%
}%
\end{pgfscope}%
\begin{pgfscope}%
\pgfsetbuttcap%
\pgfsetroundjoin%
\definecolor{currentfill}{rgb}{0.000000,0.000000,0.000000}%
\pgfsetfillcolor{currentfill}%
\pgfsetlinewidth{0.803000pt}%
\definecolor{currentstroke}{rgb}{0.000000,0.000000,0.000000}%
\pgfsetstrokecolor{currentstroke}%
\pgfsetdash{}{0pt}%
\pgfsys@defobject{currentmarker}{\pgfqpoint{0.000000in}{-0.048611in}}{\pgfqpoint{0.000000in}{0.000000in}}{%
\pgfpathmoveto{\pgfqpoint{0.000000in}{0.000000in}}%
\pgfpathlineto{\pgfqpoint{0.000000in}{-0.048611in}}%
\pgfusepath{stroke,fill}%
}%
\begin{pgfscope}%
\pgfsys@transformshift{0.731994in}{0.331635in}%
\pgfsys@useobject{currentmarker}{}%
\end{pgfscope}%
\end{pgfscope}%
\begin{pgfscope}%
\definecolor{textcolor}{rgb}{0.000000,0.000000,0.000000}%
\pgfsetstrokecolor{textcolor}%
\pgfsetfillcolor{textcolor}%
\pgftext[x=0.731994in,y=0.234413in,,top]{\color{textcolor}\sffamily\fontsize{10.000000}{12.000000}\selectfont \ensuremath{-}150}%
\end{pgfscope}%
\begin{pgfscope}%
\pgfsetbuttcap%
\pgfsetroundjoin%
\definecolor{currentfill}{rgb}{0.000000,0.000000,0.000000}%
\pgfsetfillcolor{currentfill}%
\pgfsetlinewidth{0.803000pt}%
\definecolor{currentstroke}{rgb}{0.000000,0.000000,0.000000}%
\pgfsetstrokecolor{currentstroke}%
\pgfsetdash{}{0pt}%
\pgfsys@defobject{currentmarker}{\pgfqpoint{0.000000in}{-0.048611in}}{\pgfqpoint{0.000000in}{0.000000in}}{%
\pgfpathmoveto{\pgfqpoint{0.000000in}{0.000000in}}%
\pgfpathlineto{\pgfqpoint{0.000000in}{-0.048611in}}%
\pgfusepath{stroke,fill}%
}%
\begin{pgfscope}%
\pgfsys@transformshift{1.499921in}{0.331635in}%
\pgfsys@useobject{currentmarker}{}%
\end{pgfscope}%
\end{pgfscope}%
\begin{pgfscope}%
\definecolor{textcolor}{rgb}{0.000000,0.000000,0.000000}%
\pgfsetstrokecolor{textcolor}%
\pgfsetfillcolor{textcolor}%
\pgftext[x=1.499921in,y=0.234413in,,top]{\color{textcolor}\sffamily\fontsize{10.000000}{12.000000}\selectfont \ensuremath{-}100}%
\end{pgfscope}%
\begin{pgfscope}%
\pgfsetbuttcap%
\pgfsetroundjoin%
\definecolor{currentfill}{rgb}{0.000000,0.000000,0.000000}%
\pgfsetfillcolor{currentfill}%
\pgfsetlinewidth{0.803000pt}%
\definecolor{currentstroke}{rgb}{0.000000,0.000000,0.000000}%
\pgfsetstrokecolor{currentstroke}%
\pgfsetdash{}{0pt}%
\pgfsys@defobject{currentmarker}{\pgfqpoint{0.000000in}{-0.048611in}}{\pgfqpoint{0.000000in}{0.000000in}}{%
\pgfpathmoveto{\pgfqpoint{0.000000in}{0.000000in}}%
\pgfpathlineto{\pgfqpoint{0.000000in}{-0.048611in}}%
\pgfusepath{stroke,fill}%
}%
\begin{pgfscope}%
\pgfsys@transformshift{2.267848in}{0.331635in}%
\pgfsys@useobject{currentmarker}{}%
\end{pgfscope}%
\end{pgfscope}%
\begin{pgfscope}%
\definecolor{textcolor}{rgb}{0.000000,0.000000,0.000000}%
\pgfsetstrokecolor{textcolor}%
\pgfsetfillcolor{textcolor}%
\pgftext[x=2.267848in,y=0.234413in,,top]{\color{textcolor}\sffamily\fontsize{10.000000}{12.000000}\selectfont \ensuremath{-}50}%
\end{pgfscope}%
\begin{pgfscope}%
\pgfsetbuttcap%
\pgfsetroundjoin%
\definecolor{currentfill}{rgb}{0.000000,0.000000,0.000000}%
\pgfsetfillcolor{currentfill}%
\pgfsetlinewidth{0.803000pt}%
\definecolor{currentstroke}{rgb}{0.000000,0.000000,0.000000}%
\pgfsetstrokecolor{currentstroke}%
\pgfsetdash{}{0pt}%
\pgfsys@defobject{currentmarker}{\pgfqpoint{0.000000in}{-0.048611in}}{\pgfqpoint{0.000000in}{0.000000in}}{%
\pgfpathmoveto{\pgfqpoint{0.000000in}{0.000000in}}%
\pgfpathlineto{\pgfqpoint{0.000000in}{-0.048611in}}%
\pgfusepath{stroke,fill}%
}%
\begin{pgfscope}%
\pgfsys@transformshift{3.035775in}{0.331635in}%
\pgfsys@useobject{currentmarker}{}%
\end{pgfscope}%
\end{pgfscope}%
\begin{pgfscope}%
\definecolor{textcolor}{rgb}{0.000000,0.000000,0.000000}%
\pgfsetstrokecolor{textcolor}%
\pgfsetfillcolor{textcolor}%
\pgftext[x=3.035775in,y=0.234413in,,top]{\color{textcolor}\sffamily\fontsize{10.000000}{12.000000}\selectfont 0}%
\end{pgfscope}%
\begin{pgfscope}%
\pgfsetbuttcap%
\pgfsetroundjoin%
\definecolor{currentfill}{rgb}{0.000000,0.000000,0.000000}%
\pgfsetfillcolor{currentfill}%
\pgfsetlinewidth{0.803000pt}%
\definecolor{currentstroke}{rgb}{0.000000,0.000000,0.000000}%
\pgfsetstrokecolor{currentstroke}%
\pgfsetdash{}{0pt}%
\pgfsys@defobject{currentmarker}{\pgfqpoint{0.000000in}{-0.048611in}}{\pgfqpoint{0.000000in}{0.000000in}}{%
\pgfpathmoveto{\pgfqpoint{0.000000in}{0.000000in}}%
\pgfpathlineto{\pgfqpoint{0.000000in}{-0.048611in}}%
\pgfusepath{stroke,fill}%
}%
\begin{pgfscope}%
\pgfsys@transformshift{3.803702in}{0.331635in}%
\pgfsys@useobject{currentmarker}{}%
\end{pgfscope}%
\end{pgfscope}%
\begin{pgfscope}%
\definecolor{textcolor}{rgb}{0.000000,0.000000,0.000000}%
\pgfsetstrokecolor{textcolor}%
\pgfsetfillcolor{textcolor}%
\pgftext[x=3.803702in,y=0.234413in,,top]{\color{textcolor}\sffamily\fontsize{10.000000}{12.000000}\selectfont 50}%
\end{pgfscope}%
\begin{pgfscope}%
\pgfsetbuttcap%
\pgfsetroundjoin%
\definecolor{currentfill}{rgb}{0.000000,0.000000,0.000000}%
\pgfsetfillcolor{currentfill}%
\pgfsetlinewidth{0.803000pt}%
\definecolor{currentstroke}{rgb}{0.000000,0.000000,0.000000}%
\pgfsetstrokecolor{currentstroke}%
\pgfsetdash{}{0pt}%
\pgfsys@defobject{currentmarker}{\pgfqpoint{0.000000in}{-0.048611in}}{\pgfqpoint{0.000000in}{0.000000in}}{%
\pgfpathmoveto{\pgfqpoint{0.000000in}{0.000000in}}%
\pgfpathlineto{\pgfqpoint{0.000000in}{-0.048611in}}%
\pgfusepath{stroke,fill}%
}%
\begin{pgfscope}%
\pgfsys@transformshift{4.571629in}{0.331635in}%
\pgfsys@useobject{currentmarker}{}%
\end{pgfscope}%
\end{pgfscope}%
\begin{pgfscope}%
\definecolor{textcolor}{rgb}{0.000000,0.000000,0.000000}%
\pgfsetstrokecolor{textcolor}%
\pgfsetfillcolor{textcolor}%
\pgftext[x=4.571629in,y=0.234413in,,top]{\color{textcolor}\sffamily\fontsize{10.000000}{12.000000}\selectfont 100}%
\end{pgfscope}%
\begin{pgfscope}%
\pgfsetbuttcap%
\pgfsetroundjoin%
\definecolor{currentfill}{rgb}{0.000000,0.000000,0.000000}%
\pgfsetfillcolor{currentfill}%
\pgfsetlinewidth{0.803000pt}%
\definecolor{currentstroke}{rgb}{0.000000,0.000000,0.000000}%
\pgfsetstrokecolor{currentstroke}%
\pgfsetdash{}{0pt}%
\pgfsys@defobject{currentmarker}{\pgfqpoint{0.000000in}{-0.048611in}}{\pgfqpoint{0.000000in}{0.000000in}}{%
\pgfpathmoveto{\pgfqpoint{0.000000in}{0.000000in}}%
\pgfpathlineto{\pgfqpoint{0.000000in}{-0.048611in}}%
\pgfusepath{stroke,fill}%
}%
\begin{pgfscope}%
\pgfsys@transformshift{5.339556in}{0.331635in}%
\pgfsys@useobject{currentmarker}{}%
\end{pgfscope}%
\end{pgfscope}%
\begin{pgfscope}%
\definecolor{textcolor}{rgb}{0.000000,0.000000,0.000000}%
\pgfsetstrokecolor{textcolor}%
\pgfsetfillcolor{textcolor}%
\pgftext[x=5.339556in,y=0.234413in,,top]{\color{textcolor}\sffamily\fontsize{10.000000}{12.000000}\selectfont 150}%
\end{pgfscope}%
\begin{pgfscope}%
\pgfsetbuttcap%
\pgfsetroundjoin%
\definecolor{currentfill}{rgb}{0.000000,0.000000,0.000000}%
\pgfsetfillcolor{currentfill}%
\pgfsetlinewidth{0.803000pt}%
\definecolor{currentstroke}{rgb}{0.000000,0.000000,0.000000}%
\pgfsetstrokecolor{currentstroke}%
\pgfsetdash{}{0pt}%
\pgfsys@defobject{currentmarker}{\pgfqpoint{-0.048611in}{0.000000in}}{\pgfqpoint{-0.000000in}{0.000000in}}{%
\pgfpathmoveto{\pgfqpoint{-0.000000in}{0.000000in}}%
\pgfpathlineto{\pgfqpoint{-0.048611in}{0.000000in}}%
\pgfusepath{stroke,fill}%
}%
\begin{pgfscope}%
\pgfsys@transformshift{0.570343in}{0.859263in}%
\pgfsys@useobject{currentmarker}{}%
\end{pgfscope}%
\end{pgfscope}%
\begin{pgfscope}%
\definecolor{textcolor}{rgb}{0.000000,0.000000,0.000000}%
\pgfsetstrokecolor{textcolor}%
\pgfsetfillcolor{textcolor}%
\pgftext[x=0.100000in, y=0.806502in, left, base]{\color{textcolor}\sffamily\fontsize{10.000000}{12.000000}\selectfont \ensuremath{-}100}%
\end{pgfscope}%
\begin{pgfscope}%
\pgfsetbuttcap%
\pgfsetroundjoin%
\definecolor{currentfill}{rgb}{0.000000,0.000000,0.000000}%
\pgfsetfillcolor{currentfill}%
\pgfsetlinewidth{0.803000pt}%
\definecolor{currentstroke}{rgb}{0.000000,0.000000,0.000000}%
\pgfsetstrokecolor{currentstroke}%
\pgfsetdash{}{0pt}%
\pgfsys@defobject{currentmarker}{\pgfqpoint{-0.048611in}{0.000000in}}{\pgfqpoint{-0.000000in}{0.000000in}}{%
\pgfpathmoveto{\pgfqpoint{-0.000000in}{0.000000in}}%
\pgfpathlineto{\pgfqpoint{-0.048611in}{0.000000in}}%
\pgfusepath{stroke,fill}%
}%
\begin{pgfscope}%
\pgfsys@transformshift{0.570343in}{1.469591in}%
\pgfsys@useobject{currentmarker}{}%
\end{pgfscope}%
\end{pgfscope}%
\begin{pgfscope}%
\definecolor{textcolor}{rgb}{0.000000,0.000000,0.000000}%
\pgfsetstrokecolor{textcolor}%
\pgfsetfillcolor{textcolor}%
\pgftext[x=0.188365in, y=1.416830in, left, base]{\color{textcolor}\sffamily\fontsize{10.000000}{12.000000}\selectfont \ensuremath{-}50}%
\end{pgfscope}%
\begin{pgfscope}%
\pgfsetbuttcap%
\pgfsetroundjoin%
\definecolor{currentfill}{rgb}{0.000000,0.000000,0.000000}%
\pgfsetfillcolor{currentfill}%
\pgfsetlinewidth{0.803000pt}%
\definecolor{currentstroke}{rgb}{0.000000,0.000000,0.000000}%
\pgfsetstrokecolor{currentstroke}%
\pgfsetdash{}{0pt}%
\pgfsys@defobject{currentmarker}{\pgfqpoint{-0.048611in}{0.000000in}}{\pgfqpoint{-0.000000in}{0.000000in}}{%
\pgfpathmoveto{\pgfqpoint{-0.000000in}{0.000000in}}%
\pgfpathlineto{\pgfqpoint{-0.048611in}{0.000000in}}%
\pgfusepath{stroke,fill}%
}%
\begin{pgfscope}%
\pgfsys@transformshift{0.570343in}{2.079919in}%
\pgfsys@useobject{currentmarker}{}%
\end{pgfscope}%
\end{pgfscope}%
\begin{pgfscope}%
\definecolor{textcolor}{rgb}{0.000000,0.000000,0.000000}%
\pgfsetstrokecolor{textcolor}%
\pgfsetfillcolor{textcolor}%
\pgftext[x=0.384756in, y=2.027158in, left, base]{\color{textcolor}\sffamily\fontsize{10.000000}{12.000000}\selectfont 0}%
\end{pgfscope}%
\begin{pgfscope}%
\pgfsetbuttcap%
\pgfsetroundjoin%
\definecolor{currentfill}{rgb}{0.000000,0.000000,0.000000}%
\pgfsetfillcolor{currentfill}%
\pgfsetlinewidth{0.803000pt}%
\definecolor{currentstroke}{rgb}{0.000000,0.000000,0.000000}%
\pgfsetstrokecolor{currentstroke}%
\pgfsetdash{}{0pt}%
\pgfsys@defobject{currentmarker}{\pgfqpoint{-0.048611in}{0.000000in}}{\pgfqpoint{-0.000000in}{0.000000in}}{%
\pgfpathmoveto{\pgfqpoint{-0.000000in}{0.000000in}}%
\pgfpathlineto{\pgfqpoint{-0.048611in}{0.000000in}}%
\pgfusepath{stroke,fill}%
}%
\begin{pgfscope}%
\pgfsys@transformshift{0.570343in}{2.690247in}%
\pgfsys@useobject{currentmarker}{}%
\end{pgfscope}%
\end{pgfscope}%
\begin{pgfscope}%
\definecolor{textcolor}{rgb}{0.000000,0.000000,0.000000}%
\pgfsetstrokecolor{textcolor}%
\pgfsetfillcolor{textcolor}%
\pgftext[x=0.296390in, y=2.637486in, left, base]{\color{textcolor}\sffamily\fontsize{10.000000}{12.000000}\selectfont 50}%
\end{pgfscope}%
\begin{pgfscope}%
\pgfsetbuttcap%
\pgfsetroundjoin%
\definecolor{currentfill}{rgb}{0.000000,0.000000,0.000000}%
\pgfsetfillcolor{currentfill}%
\pgfsetlinewidth{0.803000pt}%
\definecolor{currentstroke}{rgb}{0.000000,0.000000,0.000000}%
\pgfsetstrokecolor{currentstroke}%
\pgfsetdash{}{0pt}%
\pgfsys@defobject{currentmarker}{\pgfqpoint{-0.048611in}{0.000000in}}{\pgfqpoint{-0.000000in}{0.000000in}}{%
\pgfpathmoveto{\pgfqpoint{-0.000000in}{0.000000in}}%
\pgfpathlineto{\pgfqpoint{-0.048611in}{0.000000in}}%
\pgfusepath{stroke,fill}%
}%
\begin{pgfscope}%
\pgfsys@transformshift{0.570343in}{3.300575in}%
\pgfsys@useobject{currentmarker}{}%
\end{pgfscope}%
\end{pgfscope}%
\begin{pgfscope}%
\definecolor{textcolor}{rgb}{0.000000,0.000000,0.000000}%
\pgfsetstrokecolor{textcolor}%
\pgfsetfillcolor{textcolor}%
\pgftext[x=0.208025in, y=3.247814in, left, base]{\color{textcolor}\sffamily\fontsize{10.000000}{12.000000}\selectfont 100}%
\end{pgfscope}%
\begin{pgfscope}%
\pgfsetbuttcap%
\pgfsetroundjoin%
\definecolor{currentfill}{rgb}{0.000000,0.000000,0.000000}%
\pgfsetfillcolor{currentfill}%
\pgfsetlinewidth{0.803000pt}%
\definecolor{currentstroke}{rgb}{0.000000,0.000000,0.000000}%
\pgfsetstrokecolor{currentstroke}%
\pgfsetdash{}{0pt}%
\pgfsys@defobject{currentmarker}{\pgfqpoint{-0.048611in}{0.000000in}}{\pgfqpoint{-0.000000in}{0.000000in}}{%
\pgfpathmoveto{\pgfqpoint{-0.000000in}{0.000000in}}%
\pgfpathlineto{\pgfqpoint{-0.048611in}{0.000000in}}%
\pgfusepath{stroke,fill}%
}%
\begin{pgfscope}%
\pgfsys@transformshift{0.570343in}{3.910903in}%
\pgfsys@useobject{currentmarker}{}%
\end{pgfscope}%
\end{pgfscope}%
\begin{pgfscope}%
\definecolor{textcolor}{rgb}{0.000000,0.000000,0.000000}%
\pgfsetstrokecolor{textcolor}%
\pgfsetfillcolor{textcolor}%
\pgftext[x=0.208025in, y=3.858142in, left, base]{\color{textcolor}\sffamily\fontsize{10.000000}{12.000000}\selectfont 150}%
\end{pgfscope}%
\begin{pgfscope}%
\pgfpathrectangle{\pgfqpoint{0.570343in}{0.331635in}}{\pgfqpoint{4.960000in}{3.696000in}}%
\pgfusepath{clip}%
\pgfsetrectcap%
\pgfsetroundjoin%
\pgfsetlinewidth{1.505625pt}%
\definecolor{currentstroke}{rgb}{0.631373,0.788235,0.956863}%
\pgfsetstrokecolor{currentstroke}%
\pgfsetstrokeopacity{0.800000}%
\pgfsetdash{}{0pt}%
\pgfpathmoveto{\pgfqpoint{3.009597in}{1.373128in}}%
\pgfpathlineto{\pgfqpoint{3.108332in}{1.756888in}}%
\pgfusepath{stroke}%
\end{pgfscope}%
\begin{pgfscope}%
\pgfpathrectangle{\pgfqpoint{0.570343in}{0.331635in}}{\pgfqpoint{4.960000in}{3.696000in}}%
\pgfusepath{clip}%
\pgfsetrectcap%
\pgfsetroundjoin%
\pgfsetlinewidth{1.505625pt}%
\definecolor{currentstroke}{rgb}{0.631373,0.788235,0.956863}%
\pgfsetstrokecolor{currentstroke}%
\pgfsetstrokeopacity{0.800000}%
\pgfsetdash{}{0pt}%
\pgfpathmoveto{\pgfqpoint{4.747431in}{2.693739in}}%
\pgfpathlineto{\pgfqpoint{3.108332in}{1.756888in}}%
\pgfusepath{stroke}%
\end{pgfscope}%
\begin{pgfscope}%
\pgfpathrectangle{\pgfqpoint{0.570343in}{0.331635in}}{\pgfqpoint{4.960000in}{3.696000in}}%
\pgfusepath{clip}%
\pgfsetrectcap%
\pgfsetroundjoin%
\pgfsetlinewidth{1.505625pt}%
\definecolor{currentstroke}{rgb}{0.631373,0.788235,0.956863}%
\pgfsetstrokecolor{currentstroke}%
\pgfsetstrokeopacity{0.800000}%
\pgfsetdash{}{0pt}%
\pgfpathmoveto{\pgfqpoint{3.800801in}{1.164242in}}%
\pgfpathlineto{\pgfqpoint{3.108332in}{1.756888in}}%
\pgfusepath{stroke}%
\end{pgfscope}%
\begin{pgfscope}%
\pgfpathrectangle{\pgfqpoint{0.570343in}{0.331635in}}{\pgfqpoint{4.960000in}{3.696000in}}%
\pgfusepath{clip}%
\pgfsetrectcap%
\pgfsetroundjoin%
\pgfsetlinewidth{1.505625pt}%
\definecolor{currentstroke}{rgb}{0.631373,0.788235,0.956863}%
\pgfsetstrokecolor{currentstroke}%
\pgfsetstrokeopacity{0.800000}%
\pgfsetdash{}{0pt}%
\pgfpathmoveto{\pgfqpoint{3.358861in}{1.948847in}}%
\pgfpathlineto{\pgfqpoint{3.108332in}{1.756888in}}%
\pgfusepath{stroke}%
\end{pgfscope}%
\begin{pgfscope}%
\pgfpathrectangle{\pgfqpoint{0.570343in}{0.331635in}}{\pgfqpoint{4.960000in}{3.696000in}}%
\pgfusepath{clip}%
\pgfsetrectcap%
\pgfsetroundjoin%
\pgfsetlinewidth{1.505625pt}%
\definecolor{currentstroke}{rgb}{0.631373,0.788235,0.956863}%
\pgfsetstrokecolor{currentstroke}%
\pgfsetstrokeopacity{0.800000}%
\pgfsetdash{}{0pt}%
\pgfpathmoveto{\pgfqpoint{3.351516in}{1.087951in}}%
\pgfpathlineto{\pgfqpoint{3.108332in}{1.756888in}}%
\pgfusepath{stroke}%
\end{pgfscope}%
\begin{pgfscope}%
\pgfpathrectangle{\pgfqpoint{0.570343in}{0.331635in}}{\pgfqpoint{4.960000in}{3.696000in}}%
\pgfusepath{clip}%
\pgfsetrectcap%
\pgfsetroundjoin%
\pgfsetlinewidth{1.505625pt}%
\definecolor{currentstroke}{rgb}{0.631373,0.788235,0.956863}%
\pgfsetstrokecolor{currentstroke}%
\pgfsetstrokeopacity{0.800000}%
\pgfsetdash{}{0pt}%
\pgfpathmoveto{\pgfqpoint{2.157332in}{3.157638in}}%
\pgfpathlineto{\pgfqpoint{3.108332in}{1.756888in}}%
\pgfusepath{stroke}%
\end{pgfscope}%
\begin{pgfscope}%
\pgfpathrectangle{\pgfqpoint{0.570343in}{0.331635in}}{\pgfqpoint{4.960000in}{3.696000in}}%
\pgfusepath{clip}%
\pgfsetrectcap%
\pgfsetroundjoin%
\pgfsetlinewidth{1.505625pt}%
\definecolor{currentstroke}{rgb}{0.631373,0.788235,0.956863}%
\pgfsetstrokecolor{currentstroke}%
\pgfsetstrokeopacity{0.800000}%
\pgfsetdash{}{0pt}%
\pgfpathmoveto{\pgfqpoint{4.201874in}{1.916318in}}%
\pgfpathlineto{\pgfqpoint{3.108332in}{1.756888in}}%
\pgfusepath{stroke}%
\end{pgfscope}%
\begin{pgfscope}%
\pgfpathrectangle{\pgfqpoint{0.570343in}{0.331635in}}{\pgfqpoint{4.960000in}{3.696000in}}%
\pgfusepath{clip}%
\pgfsetrectcap%
\pgfsetroundjoin%
\pgfsetlinewidth{1.505625pt}%
\definecolor{currentstroke}{rgb}{0.631373,0.788235,0.956863}%
\pgfsetstrokecolor{currentstroke}%
\pgfsetstrokeopacity{0.800000}%
\pgfsetdash{}{0pt}%
\pgfpathmoveto{\pgfqpoint{3.667396in}{0.684679in}}%
\pgfpathlineto{\pgfqpoint{3.108332in}{1.756888in}}%
\pgfusepath{stroke}%
\end{pgfscope}%
\begin{pgfscope}%
\pgfpathrectangle{\pgfqpoint{0.570343in}{0.331635in}}{\pgfqpoint{4.960000in}{3.696000in}}%
\pgfusepath{clip}%
\pgfsetrectcap%
\pgfsetroundjoin%
\pgfsetlinewidth{1.505625pt}%
\definecolor{currentstroke}{rgb}{0.631373,0.788235,0.956863}%
\pgfsetstrokecolor{currentstroke}%
\pgfsetstrokeopacity{0.800000}%
\pgfsetdash{}{0pt}%
\pgfpathmoveto{\pgfqpoint{3.384863in}{1.561141in}}%
\pgfpathlineto{\pgfqpoint{3.108332in}{1.756888in}}%
\pgfusepath{stroke}%
\end{pgfscope}%
\begin{pgfscope}%
\pgfpathrectangle{\pgfqpoint{0.570343in}{0.331635in}}{\pgfqpoint{4.960000in}{3.696000in}}%
\pgfusepath{clip}%
\pgfsetrectcap%
\pgfsetroundjoin%
\pgfsetlinewidth{1.505625pt}%
\definecolor{currentstroke}{rgb}{0.631373,0.788235,0.956863}%
\pgfsetstrokecolor{currentstroke}%
\pgfsetstrokeopacity{0.800000}%
\pgfsetdash{}{0pt}%
\pgfpathmoveto{\pgfqpoint{2.133517in}{2.173284in}}%
\pgfpathlineto{\pgfqpoint{3.108332in}{1.756888in}}%
\pgfusepath{stroke}%
\end{pgfscope}%
\begin{pgfscope}%
\pgfpathrectangle{\pgfqpoint{0.570343in}{0.331635in}}{\pgfqpoint{4.960000in}{3.696000in}}%
\pgfusepath{clip}%
\pgfsetrectcap%
\pgfsetroundjoin%
\pgfsetlinewidth{1.505625pt}%
\definecolor{currentstroke}{rgb}{0.631373,0.788235,0.956863}%
\pgfsetstrokecolor{currentstroke}%
\pgfsetstrokeopacity{0.800000}%
\pgfsetdash{}{0pt}%
\pgfpathmoveto{\pgfqpoint{5.304889in}{1.779734in}}%
\pgfpathlineto{\pgfqpoint{3.108332in}{1.756888in}}%
\pgfusepath{stroke}%
\end{pgfscope}%
\begin{pgfscope}%
\pgfpathrectangle{\pgfqpoint{0.570343in}{0.331635in}}{\pgfqpoint{4.960000in}{3.696000in}}%
\pgfusepath{clip}%
\pgfsetrectcap%
\pgfsetroundjoin%
\pgfsetlinewidth{1.505625pt}%
\definecolor{currentstroke}{rgb}{0.631373,0.788235,0.956863}%
\pgfsetstrokecolor{currentstroke}%
\pgfsetstrokeopacity{0.800000}%
\pgfsetdash{}{0pt}%
\pgfpathmoveto{\pgfqpoint{3.176792in}{2.252003in}}%
\pgfpathlineto{\pgfqpoint{3.108332in}{1.756888in}}%
\pgfusepath{stroke}%
\end{pgfscope}%
\begin{pgfscope}%
\pgfpathrectangle{\pgfqpoint{0.570343in}{0.331635in}}{\pgfqpoint{4.960000in}{3.696000in}}%
\pgfusepath{clip}%
\pgfsetrectcap%
\pgfsetroundjoin%
\pgfsetlinewidth{1.505625pt}%
\definecolor{currentstroke}{rgb}{0.631373,0.788235,0.956863}%
\pgfsetstrokecolor{currentstroke}%
\pgfsetstrokeopacity{0.800000}%
\pgfsetdash{}{0pt}%
\pgfpathmoveto{\pgfqpoint{2.230769in}{1.209950in}}%
\pgfpathlineto{\pgfqpoint{3.108332in}{1.756888in}}%
\pgfusepath{stroke}%
\end{pgfscope}%
\begin{pgfscope}%
\pgfpathrectangle{\pgfqpoint{0.570343in}{0.331635in}}{\pgfqpoint{4.960000in}{3.696000in}}%
\pgfusepath{clip}%
\pgfsetrectcap%
\pgfsetroundjoin%
\pgfsetlinewidth{1.505625pt}%
\definecolor{currentstroke}{rgb}{0.631373,0.788235,0.956863}%
\pgfsetstrokecolor{currentstroke}%
\pgfsetstrokeopacity{0.800000}%
\pgfsetdash{}{0pt}%
\pgfpathmoveto{\pgfqpoint{2.677188in}{2.213882in}}%
\pgfpathlineto{\pgfqpoint{3.108332in}{1.756888in}}%
\pgfusepath{stroke}%
\end{pgfscope}%
\begin{pgfscope}%
\pgfpathrectangle{\pgfqpoint{0.570343in}{0.331635in}}{\pgfqpoint{4.960000in}{3.696000in}}%
\pgfusepath{clip}%
\pgfsetrectcap%
\pgfsetroundjoin%
\pgfsetlinewidth{1.505625pt}%
\definecolor{currentstroke}{rgb}{0.631373,0.788235,0.956863}%
\pgfsetstrokecolor{currentstroke}%
\pgfsetstrokeopacity{0.800000}%
\pgfsetdash{}{0pt}%
\pgfpathmoveto{\pgfqpoint{3.805258in}{1.784790in}}%
\pgfpathlineto{\pgfqpoint{3.108332in}{1.756888in}}%
\pgfusepath{stroke}%
\end{pgfscope}%
\begin{pgfscope}%
\pgfpathrectangle{\pgfqpoint{0.570343in}{0.331635in}}{\pgfqpoint{4.960000in}{3.696000in}}%
\pgfusepath{clip}%
\pgfsetrectcap%
\pgfsetroundjoin%
\pgfsetlinewidth{1.505625pt}%
\definecolor{currentstroke}{rgb}{0.631373,0.788235,0.956863}%
\pgfsetstrokecolor{currentstroke}%
\pgfsetstrokeopacity{0.800000}%
\pgfsetdash{}{0pt}%
\pgfpathmoveto{\pgfqpoint{1.649989in}{2.105370in}}%
\pgfpathlineto{\pgfqpoint{3.108332in}{1.756888in}}%
\pgfusepath{stroke}%
\end{pgfscope}%
\begin{pgfscope}%
\pgfpathrectangle{\pgfqpoint{0.570343in}{0.331635in}}{\pgfqpoint{4.960000in}{3.696000in}}%
\pgfusepath{clip}%
\pgfsetrectcap%
\pgfsetroundjoin%
\pgfsetlinewidth{1.505625pt}%
\definecolor{currentstroke}{rgb}{0.631373,0.788235,0.956863}%
\pgfsetstrokecolor{currentstroke}%
\pgfsetstrokeopacity{0.800000}%
\pgfsetdash{}{0pt}%
\pgfpathmoveto{\pgfqpoint{4.280136in}{1.516808in}}%
\pgfpathlineto{\pgfqpoint{3.108332in}{1.756888in}}%
\pgfusepath{stroke}%
\end{pgfscope}%
\begin{pgfscope}%
\pgfpathrectangle{\pgfqpoint{0.570343in}{0.331635in}}{\pgfqpoint{4.960000in}{3.696000in}}%
\pgfusepath{clip}%
\pgfsetrectcap%
\pgfsetroundjoin%
\pgfsetlinewidth{1.505625pt}%
\definecolor{currentstroke}{rgb}{0.631373,0.788235,0.956863}%
\pgfsetstrokecolor{currentstroke}%
\pgfsetstrokeopacity{0.800000}%
\pgfsetdash{}{0pt}%
\pgfpathmoveto{\pgfqpoint{2.472600in}{1.907162in}}%
\pgfpathlineto{\pgfqpoint{3.108332in}{1.756888in}}%
\pgfusepath{stroke}%
\end{pgfscope}%
\begin{pgfscope}%
\pgfpathrectangle{\pgfqpoint{0.570343in}{0.331635in}}{\pgfqpoint{4.960000in}{3.696000in}}%
\pgfusepath{clip}%
\pgfsetrectcap%
\pgfsetroundjoin%
\pgfsetlinewidth{1.505625pt}%
\definecolor{currentstroke}{rgb}{0.631373,0.788235,0.956863}%
\pgfsetstrokecolor{currentstroke}%
\pgfsetstrokeopacity{0.800000}%
\pgfsetdash{}{0pt}%
\pgfpathmoveto{\pgfqpoint{3.832697in}{2.146167in}}%
\pgfpathlineto{\pgfqpoint{3.108332in}{1.756888in}}%
\pgfusepath{stroke}%
\end{pgfscope}%
\begin{pgfscope}%
\pgfpathrectangle{\pgfqpoint{0.570343in}{0.331635in}}{\pgfqpoint{4.960000in}{3.696000in}}%
\pgfusepath{clip}%
\pgfsetrectcap%
\pgfsetroundjoin%
\pgfsetlinewidth{1.505625pt}%
\definecolor{currentstroke}{rgb}{0.631373,0.788235,0.956863}%
\pgfsetstrokecolor{currentstroke}%
\pgfsetstrokeopacity{0.800000}%
\pgfsetdash{}{0pt}%
\pgfpathmoveto{\pgfqpoint{2.958348in}{2.004807in}}%
\pgfpathlineto{\pgfqpoint{3.108332in}{1.756888in}}%
\pgfusepath{stroke}%
\end{pgfscope}%
\begin{pgfscope}%
\pgfpathrectangle{\pgfqpoint{0.570343in}{0.331635in}}{\pgfqpoint{4.960000in}{3.696000in}}%
\pgfusepath{clip}%
\pgfsetrectcap%
\pgfsetroundjoin%
\pgfsetlinewidth{1.505625pt}%
\definecolor{currentstroke}{rgb}{0.631373,0.788235,0.956863}%
\pgfsetstrokecolor{currentstroke}%
\pgfsetstrokeopacity{0.800000}%
\pgfsetdash{}{0pt}%
\pgfpathmoveto{\pgfqpoint{1.527720in}{1.171796in}}%
\pgfpathlineto{\pgfqpoint{3.108332in}{1.756888in}}%
\pgfusepath{stroke}%
\end{pgfscope}%
\begin{pgfscope}%
\pgfpathrectangle{\pgfqpoint{0.570343in}{0.331635in}}{\pgfqpoint{4.960000in}{3.696000in}}%
\pgfusepath{clip}%
\pgfsetrectcap%
\pgfsetroundjoin%
\pgfsetlinewidth{1.505625pt}%
\definecolor{currentstroke}{rgb}{0.631373,0.788235,0.956863}%
\pgfsetstrokecolor{currentstroke}%
\pgfsetstrokeopacity{0.800000}%
\pgfsetdash{}{0pt}%
\pgfpathmoveto{\pgfqpoint{2.978471in}{0.640164in}}%
\pgfpathlineto{\pgfqpoint{3.108332in}{1.756888in}}%
\pgfusepath{stroke}%
\end{pgfscope}%
\begin{pgfscope}%
\pgfpathrectangle{\pgfqpoint{0.570343in}{0.331635in}}{\pgfqpoint{4.960000in}{3.696000in}}%
\pgfusepath{clip}%
\pgfsetrectcap%
\pgfsetroundjoin%
\pgfsetlinewidth{1.505625pt}%
\definecolor{currentstroke}{rgb}{0.631373,0.788235,0.956863}%
\pgfsetstrokecolor{currentstroke}%
\pgfsetstrokeopacity{0.800000}%
\pgfsetdash{}{0pt}%
\pgfpathmoveto{\pgfqpoint{2.744542in}{1.108528in}}%
\pgfpathlineto{\pgfqpoint{3.108332in}{1.756888in}}%
\pgfusepath{stroke}%
\end{pgfscope}%
\begin{pgfscope}%
\pgfpathrectangle{\pgfqpoint{0.570343in}{0.331635in}}{\pgfqpoint{4.960000in}{3.696000in}}%
\pgfusepath{clip}%
\pgfsetrectcap%
\pgfsetroundjoin%
\pgfsetlinewidth{1.505625pt}%
\definecolor{currentstroke}{rgb}{0.631373,0.788235,0.956863}%
\pgfsetstrokecolor{currentstroke}%
\pgfsetstrokeopacity{0.800000}%
\pgfsetdash{}{0pt}%
\pgfpathmoveto{\pgfqpoint{1.808176in}{1.652126in}}%
\pgfpathlineto{\pgfqpoint{3.108332in}{1.756888in}}%
\pgfusepath{stroke}%
\end{pgfscope}%
\begin{pgfscope}%
\pgfpathrectangle{\pgfqpoint{0.570343in}{0.331635in}}{\pgfqpoint{4.960000in}{3.696000in}}%
\pgfusepath{clip}%
\pgfsetrectcap%
\pgfsetroundjoin%
\pgfsetlinewidth{1.505625pt}%
\definecolor{currentstroke}{rgb}{0.631373,0.788235,0.956863}%
\pgfsetstrokecolor{currentstroke}%
\pgfsetstrokeopacity{0.800000}%
\pgfsetdash{}{0pt}%
\pgfpathmoveto{\pgfqpoint{2.390127in}{1.580399in}}%
\pgfpathlineto{\pgfqpoint{3.108332in}{1.756888in}}%
\pgfusepath{stroke}%
\end{pgfscope}%
\begin{pgfscope}%
\pgfpathrectangle{\pgfqpoint{0.570343in}{0.331635in}}{\pgfqpoint{4.960000in}{3.696000in}}%
\pgfusepath{clip}%
\pgfsetrectcap%
\pgfsetroundjoin%
\pgfsetlinewidth{1.505625pt}%
\definecolor{currentstroke}{rgb}{0.631373,0.788235,0.956863}%
\pgfsetstrokecolor{currentstroke}%
\pgfsetstrokeopacity{0.800000}%
\pgfsetdash{}{0pt}%
\pgfpathmoveto{\pgfqpoint{2.744744in}{3.179676in}}%
\pgfpathlineto{\pgfqpoint{3.108332in}{1.756888in}}%
\pgfusepath{stroke}%
\end{pgfscope}%
\begin{pgfscope}%
\pgfpathrectangle{\pgfqpoint{0.570343in}{0.331635in}}{\pgfqpoint{4.960000in}{3.696000in}}%
\pgfusepath{clip}%
\pgfsetrectcap%
\pgfsetroundjoin%
\pgfsetlinewidth{1.505625pt}%
\definecolor{currentstroke}{rgb}{0.631373,0.788235,0.956863}%
\pgfsetstrokecolor{currentstroke}%
\pgfsetstrokeopacity{0.800000}%
\pgfsetdash{}{0pt}%
\pgfpathmoveto{\pgfqpoint{2.878184in}{1.703905in}}%
\pgfpathlineto{\pgfqpoint{3.108332in}{1.756888in}}%
\pgfusepath{stroke}%
\end{pgfscope}%
\begin{pgfscope}%
\pgfpathrectangle{\pgfqpoint{0.570343in}{0.331635in}}{\pgfqpoint{4.960000in}{3.696000in}}%
\pgfusepath{clip}%
\pgfsetrectcap%
\pgfsetroundjoin%
\pgfsetlinewidth{1.505625pt}%
\definecolor{currentstroke}{rgb}{0.631373,0.788235,0.956863}%
\pgfsetstrokecolor{currentstroke}%
\pgfsetstrokeopacity{0.800000}%
\pgfsetdash{}{0pt}%
\pgfpathmoveto{\pgfqpoint{3.759465in}{1.474627in}}%
\pgfpathlineto{\pgfqpoint{3.108332in}{1.756888in}}%
\pgfusepath{stroke}%
\end{pgfscope}%
\begin{pgfscope}%
\pgfpathrectangle{\pgfqpoint{0.570343in}{0.331635in}}{\pgfqpoint{4.960000in}{3.696000in}}%
\pgfusepath{clip}%
\pgfsetrectcap%
\pgfsetroundjoin%
\pgfsetlinewidth{1.505625pt}%
\definecolor{currentstroke}{rgb}{1.000000,0.705882,0.509804}%
\pgfsetstrokecolor{currentstroke}%
\pgfsetstrokeopacity{0.800000}%
\pgfsetdash{}{0pt}%
\pgfpathmoveto{\pgfqpoint{4.010786in}{2.470580in}}%
\pgfpathlineto{\pgfqpoint{2.979952in}{2.554324in}}%
\pgfusepath{stroke}%
\end{pgfscope}%
\begin{pgfscope}%
\pgfpathrectangle{\pgfqpoint{0.570343in}{0.331635in}}{\pgfqpoint{4.960000in}{3.696000in}}%
\pgfusepath{clip}%
\pgfsetrectcap%
\pgfsetroundjoin%
\pgfsetlinewidth{1.505625pt}%
\definecolor{currentstroke}{rgb}{1.000000,0.705882,0.509804}%
\pgfsetstrokecolor{currentstroke}%
\pgfsetstrokeopacity{0.800000}%
\pgfsetdash{}{0pt}%
\pgfpathmoveto{\pgfqpoint{3.641382in}{3.277442in}}%
\pgfpathlineto{\pgfqpoint{2.979952in}{2.554324in}}%
\pgfusepath{stroke}%
\end{pgfscope}%
\begin{pgfscope}%
\pgfpathrectangle{\pgfqpoint{0.570343in}{0.331635in}}{\pgfqpoint{4.960000in}{3.696000in}}%
\pgfusepath{clip}%
\pgfsetrectcap%
\pgfsetroundjoin%
\pgfsetlinewidth{1.505625pt}%
\definecolor{currentstroke}{rgb}{1.000000,0.705882,0.509804}%
\pgfsetstrokecolor{currentstroke}%
\pgfsetstrokeopacity{0.800000}%
\pgfsetdash{}{0pt}%
\pgfpathmoveto{\pgfqpoint{2.910306in}{2.534690in}}%
\pgfpathlineto{\pgfqpoint{2.979952in}{2.554324in}}%
\pgfusepath{stroke}%
\end{pgfscope}%
\begin{pgfscope}%
\pgfpathrectangle{\pgfqpoint{0.570343in}{0.331635in}}{\pgfqpoint{4.960000in}{3.696000in}}%
\pgfusepath{clip}%
\pgfsetrectcap%
\pgfsetroundjoin%
\pgfsetlinewidth{1.505625pt}%
\definecolor{currentstroke}{rgb}{1.000000,0.705882,0.509804}%
\pgfsetstrokecolor{currentstroke}%
\pgfsetstrokeopacity{0.800000}%
\pgfsetdash{}{0pt}%
\pgfpathmoveto{\pgfqpoint{2.662402in}{2.792752in}}%
\pgfpathlineto{\pgfqpoint{2.979952in}{2.554324in}}%
\pgfusepath{stroke}%
\end{pgfscope}%
\begin{pgfscope}%
\pgfpathrectangle{\pgfqpoint{0.570343in}{0.331635in}}{\pgfqpoint{4.960000in}{3.696000in}}%
\pgfusepath{clip}%
\pgfsetrectcap%
\pgfsetroundjoin%
\pgfsetlinewidth{1.505625pt}%
\definecolor{currentstroke}{rgb}{1.000000,0.705882,0.509804}%
\pgfsetstrokecolor{currentstroke}%
\pgfsetstrokeopacity{0.800000}%
\pgfsetdash{}{0pt}%
\pgfpathmoveto{\pgfqpoint{1.385602in}{3.496321in}}%
\pgfpathlineto{\pgfqpoint{2.979952in}{2.554324in}}%
\pgfusepath{stroke}%
\end{pgfscope}%
\begin{pgfscope}%
\pgfpathrectangle{\pgfqpoint{0.570343in}{0.331635in}}{\pgfqpoint{4.960000in}{3.696000in}}%
\pgfusepath{clip}%
\pgfsetrectcap%
\pgfsetroundjoin%
\pgfsetlinewidth{1.505625pt}%
\definecolor{currentstroke}{rgb}{1.000000,0.705882,0.509804}%
\pgfsetstrokecolor{currentstroke}%
\pgfsetstrokeopacity{0.800000}%
\pgfsetdash{}{0pt}%
\pgfpathmoveto{\pgfqpoint{4.718935in}{1.917042in}}%
\pgfpathlineto{\pgfqpoint{2.979952in}{2.554324in}}%
\pgfusepath{stroke}%
\end{pgfscope}%
\begin{pgfscope}%
\pgfpathrectangle{\pgfqpoint{0.570343in}{0.331635in}}{\pgfqpoint{4.960000in}{3.696000in}}%
\pgfusepath{clip}%
\pgfsetrectcap%
\pgfsetroundjoin%
\pgfsetlinewidth{1.505625pt}%
\definecolor{currentstroke}{rgb}{1.000000,0.705882,0.509804}%
\pgfsetstrokecolor{currentstroke}%
\pgfsetstrokeopacity{0.800000}%
\pgfsetdash{}{0pt}%
\pgfpathmoveto{\pgfqpoint{1.526138in}{2.914129in}}%
\pgfpathlineto{\pgfqpoint{2.979952in}{2.554324in}}%
\pgfusepath{stroke}%
\end{pgfscope}%
\begin{pgfscope}%
\pgfpathrectangle{\pgfqpoint{0.570343in}{0.331635in}}{\pgfqpoint{4.960000in}{3.696000in}}%
\pgfusepath{clip}%
\pgfsetrectcap%
\pgfsetroundjoin%
\pgfsetlinewidth{1.505625pt}%
\definecolor{currentstroke}{rgb}{1.000000,0.705882,0.509804}%
\pgfsetstrokecolor{currentstroke}%
\pgfsetstrokeopacity{0.800000}%
\pgfsetdash{}{0pt}%
\pgfpathmoveto{\pgfqpoint{5.011810in}{2.271126in}}%
\pgfpathlineto{\pgfqpoint{2.979952in}{2.554324in}}%
\pgfusepath{stroke}%
\end{pgfscope}%
\begin{pgfscope}%
\pgfpathrectangle{\pgfqpoint{0.570343in}{0.331635in}}{\pgfqpoint{4.960000in}{3.696000in}}%
\pgfusepath{clip}%
\pgfsetrectcap%
\pgfsetroundjoin%
\pgfsetlinewidth{1.505625pt}%
\definecolor{currentstroke}{rgb}{1.000000,0.705882,0.509804}%
\pgfsetstrokecolor{currentstroke}%
\pgfsetstrokeopacity{0.800000}%
\pgfsetdash{}{0pt}%
\pgfpathmoveto{\pgfqpoint{4.377389in}{1.125422in}}%
\pgfpathlineto{\pgfqpoint{2.979952in}{2.554324in}}%
\pgfusepath{stroke}%
\end{pgfscope}%
\begin{pgfscope}%
\pgfpathrectangle{\pgfqpoint{0.570343in}{0.331635in}}{\pgfqpoint{4.960000in}{3.696000in}}%
\pgfusepath{clip}%
\pgfsetrectcap%
\pgfsetroundjoin%
\pgfsetlinewidth{1.505625pt}%
\definecolor{currentstroke}{rgb}{1.000000,0.705882,0.509804}%
\pgfsetstrokecolor{currentstroke}%
\pgfsetstrokeopacity{0.800000}%
\pgfsetdash{}{0pt}%
\pgfpathmoveto{\pgfqpoint{2.162297in}{2.749786in}}%
\pgfpathlineto{\pgfqpoint{2.979952in}{2.554324in}}%
\pgfusepath{stroke}%
\end{pgfscope}%
\begin{pgfscope}%
\pgfpathrectangle{\pgfqpoint{0.570343in}{0.331635in}}{\pgfqpoint{4.960000in}{3.696000in}}%
\pgfusepath{clip}%
\pgfsetrectcap%
\pgfsetroundjoin%
\pgfsetlinewidth{1.505625pt}%
\definecolor{currentstroke}{rgb}{1.000000,0.705882,0.509804}%
\pgfsetstrokecolor{currentstroke}%
\pgfsetstrokeopacity{0.800000}%
\pgfsetdash{}{0pt}%
\pgfpathmoveto{\pgfqpoint{2.464678in}{2.473549in}}%
\pgfpathlineto{\pgfqpoint{2.979952in}{2.554324in}}%
\pgfusepath{stroke}%
\end{pgfscope}%
\begin{pgfscope}%
\pgfpathrectangle{\pgfqpoint{0.570343in}{0.331635in}}{\pgfqpoint{4.960000in}{3.696000in}}%
\pgfusepath{clip}%
\pgfsetrectcap%
\pgfsetroundjoin%
\pgfsetlinewidth{1.505625pt}%
\definecolor{currentstroke}{rgb}{1.000000,0.705882,0.509804}%
\pgfsetstrokecolor{currentstroke}%
\pgfsetstrokeopacity{0.800000}%
\pgfsetdash{}{0pt}%
\pgfpathmoveto{\pgfqpoint{0.856192in}{1.571252in}}%
\pgfpathlineto{\pgfqpoint{2.979952in}{2.554324in}}%
\pgfusepath{stroke}%
\end{pgfscope}%
\begin{pgfscope}%
\pgfpathrectangle{\pgfqpoint{0.570343in}{0.331635in}}{\pgfqpoint{4.960000in}{3.696000in}}%
\pgfusepath{clip}%
\pgfsetrectcap%
\pgfsetroundjoin%
\pgfsetlinewidth{1.505625pt}%
\definecolor{currentstroke}{rgb}{1.000000,0.705882,0.509804}%
\pgfsetstrokecolor{currentstroke}%
\pgfsetstrokeopacity{0.800000}%
\pgfsetdash{}{0pt}%
\pgfpathmoveto{\pgfqpoint{4.626113in}{3.320981in}}%
\pgfpathlineto{\pgfqpoint{2.979952in}{2.554324in}}%
\pgfusepath{stroke}%
\end{pgfscope}%
\begin{pgfscope}%
\pgfpathrectangle{\pgfqpoint{0.570343in}{0.331635in}}{\pgfqpoint{4.960000in}{3.696000in}}%
\pgfusepath{clip}%
\pgfsetrectcap%
\pgfsetroundjoin%
\pgfsetlinewidth{1.505625pt}%
\definecolor{currentstroke}{rgb}{1.000000,0.705882,0.509804}%
\pgfsetstrokecolor{currentstroke}%
\pgfsetstrokeopacity{0.800000}%
\pgfsetdash{}{0pt}%
\pgfpathmoveto{\pgfqpoint{1.796127in}{3.353620in}}%
\pgfpathlineto{\pgfqpoint{2.979952in}{2.554324in}}%
\pgfusepath{stroke}%
\end{pgfscope}%
\begin{pgfscope}%
\pgfpathrectangle{\pgfqpoint{0.570343in}{0.331635in}}{\pgfqpoint{4.960000in}{3.696000in}}%
\pgfusepath{clip}%
\pgfsetrectcap%
\pgfsetroundjoin%
\pgfsetlinewidth{1.505625pt}%
\definecolor{currentstroke}{rgb}{1.000000,0.705882,0.509804}%
\pgfsetstrokecolor{currentstroke}%
\pgfsetstrokeopacity{0.800000}%
\pgfsetdash{}{0pt}%
\pgfpathmoveto{\pgfqpoint{4.916919in}{1.060897in}}%
\pgfpathlineto{\pgfqpoint{2.979952in}{2.554324in}}%
\pgfusepath{stroke}%
\end{pgfscope}%
\begin{pgfscope}%
\pgfpathrectangle{\pgfqpoint{0.570343in}{0.331635in}}{\pgfqpoint{4.960000in}{3.696000in}}%
\pgfusepath{clip}%
\pgfsetrectcap%
\pgfsetroundjoin%
\pgfsetlinewidth{1.505625pt}%
\definecolor{currentstroke}{rgb}{1.000000,0.705882,0.509804}%
\pgfsetstrokecolor{currentstroke}%
\pgfsetstrokeopacity{0.800000}%
\pgfsetdash{}{0pt}%
\pgfpathmoveto{\pgfqpoint{4.374254in}{2.173974in}}%
\pgfpathlineto{\pgfqpoint{2.979952in}{2.554324in}}%
\pgfusepath{stroke}%
\end{pgfscope}%
\begin{pgfscope}%
\pgfpathrectangle{\pgfqpoint{0.570343in}{0.331635in}}{\pgfqpoint{4.960000in}{3.696000in}}%
\pgfusepath{clip}%
\pgfsetrectcap%
\pgfsetroundjoin%
\pgfsetlinewidth{1.505625pt}%
\definecolor{currentstroke}{rgb}{1.000000,0.705882,0.509804}%
\pgfsetstrokecolor{currentstroke}%
\pgfsetstrokeopacity{0.800000}%
\pgfsetdash{}{0pt}%
\pgfpathmoveto{\pgfqpoint{3.704677in}{2.945151in}}%
\pgfpathlineto{\pgfqpoint{2.979952in}{2.554324in}}%
\pgfusepath{stroke}%
\end{pgfscope}%
\begin{pgfscope}%
\pgfpathrectangle{\pgfqpoint{0.570343in}{0.331635in}}{\pgfqpoint{4.960000in}{3.696000in}}%
\pgfusepath{clip}%
\pgfsetrectcap%
\pgfsetroundjoin%
\pgfsetlinewidth{1.505625pt}%
\definecolor{currentstroke}{rgb}{1.000000,0.705882,0.509804}%
\pgfsetstrokecolor{currentstroke}%
\pgfsetstrokeopacity{0.800000}%
\pgfsetdash{}{0pt}%
\pgfpathmoveto{\pgfqpoint{0.795798in}{2.961231in}}%
\pgfpathlineto{\pgfqpoint{2.979952in}{2.554324in}}%
\pgfusepath{stroke}%
\end{pgfscope}%
\begin{pgfscope}%
\pgfpathrectangle{\pgfqpoint{0.570343in}{0.331635in}}{\pgfqpoint{4.960000in}{3.696000in}}%
\pgfusepath{clip}%
\pgfsetrectcap%
\pgfsetroundjoin%
\pgfsetlinewidth{1.505625pt}%
\definecolor{currentstroke}{rgb}{1.000000,0.705882,0.509804}%
\pgfsetstrokecolor{currentstroke}%
\pgfsetstrokeopacity{0.800000}%
\pgfsetdash{}{0pt}%
\pgfpathmoveto{\pgfqpoint{1.844531in}{3.859635in}}%
\pgfpathlineto{\pgfqpoint{2.979952in}{2.554324in}}%
\pgfusepath{stroke}%
\end{pgfscope}%
\begin{pgfscope}%
\pgfpathrectangle{\pgfqpoint{0.570343in}{0.331635in}}{\pgfqpoint{4.960000in}{3.696000in}}%
\pgfusepath{clip}%
\pgfsetrectcap%
\pgfsetroundjoin%
\pgfsetlinewidth{1.505625pt}%
\definecolor{currentstroke}{rgb}{1.000000,0.705882,0.509804}%
\pgfsetstrokecolor{currentstroke}%
\pgfsetstrokeopacity{0.800000}%
\pgfsetdash{}{0pt}%
\pgfpathmoveto{\pgfqpoint{4.720753in}{1.485285in}}%
\pgfpathlineto{\pgfqpoint{2.979952in}{2.554324in}}%
\pgfusepath{stroke}%
\end{pgfscope}%
\begin{pgfscope}%
\pgfpathrectangle{\pgfqpoint{0.570343in}{0.331635in}}{\pgfqpoint{4.960000in}{3.696000in}}%
\pgfusepath{clip}%
\pgfsetrectcap%
\pgfsetroundjoin%
\pgfsetlinewidth{1.505625pt}%
\definecolor{currentstroke}{rgb}{1.000000,0.705882,0.509804}%
\pgfsetstrokecolor{currentstroke}%
\pgfsetstrokeopacity{0.800000}%
\pgfsetdash{}{0pt}%
\pgfpathmoveto{\pgfqpoint{2.224277in}{0.499635in}}%
\pgfpathlineto{\pgfqpoint{2.979952in}{2.554324in}}%
\pgfusepath{stroke}%
\end{pgfscope}%
\begin{pgfscope}%
\pgfpathrectangle{\pgfqpoint{0.570343in}{0.331635in}}{\pgfqpoint{4.960000in}{3.696000in}}%
\pgfusepath{clip}%
\pgfsetrectcap%
\pgfsetroundjoin%
\pgfsetlinewidth{1.505625pt}%
\definecolor{currentstroke}{rgb}{1.000000,0.705882,0.509804}%
\pgfsetstrokecolor{currentstroke}%
\pgfsetstrokeopacity{0.800000}%
\pgfsetdash{}{0pt}%
\pgfpathmoveto{\pgfqpoint{2.748227in}{3.680989in}}%
\pgfpathlineto{\pgfqpoint{2.979952in}{2.554324in}}%
\pgfusepath{stroke}%
\end{pgfscope}%
\begin{pgfscope}%
\pgfpathrectangle{\pgfqpoint{0.570343in}{0.331635in}}{\pgfqpoint{4.960000in}{3.696000in}}%
\pgfusepath{clip}%
\pgfsetrectcap%
\pgfsetroundjoin%
\pgfsetlinewidth{1.505625pt}%
\definecolor{currentstroke}{rgb}{1.000000,0.705882,0.509804}%
\pgfsetstrokecolor{currentstroke}%
\pgfsetstrokeopacity{0.800000}%
\pgfsetdash{}{0pt}%
\pgfpathmoveto{\pgfqpoint{1.166400in}{2.421248in}}%
\pgfpathlineto{\pgfqpoint{2.979952in}{2.554324in}}%
\pgfusepath{stroke}%
\end{pgfscope}%
\begin{pgfscope}%
\pgfpathrectangle{\pgfqpoint{0.570343in}{0.331635in}}{\pgfqpoint{4.960000in}{3.696000in}}%
\pgfusepath{clip}%
\pgfsetrectcap%
\pgfsetroundjoin%
\pgfsetlinewidth{1.505625pt}%
\definecolor{currentstroke}{rgb}{1.000000,0.705882,0.509804}%
\pgfsetstrokecolor{currentstroke}%
\pgfsetstrokeopacity{0.800000}%
\pgfsetdash{}{0pt}%
\pgfpathmoveto{\pgfqpoint{1.664925in}{2.529060in}}%
\pgfpathlineto{\pgfqpoint{2.979952in}{2.554324in}}%
\pgfusepath{stroke}%
\end{pgfscope}%
\begin{pgfscope}%
\pgfpathrectangle{\pgfqpoint{0.570343in}{0.331635in}}{\pgfqpoint{4.960000in}{3.696000in}}%
\pgfusepath{clip}%
\pgfsetrectcap%
\pgfsetroundjoin%
\pgfsetlinewidth{1.505625pt}%
\definecolor{currentstroke}{rgb}{1.000000,0.705882,0.509804}%
\pgfsetstrokecolor{currentstroke}%
\pgfsetstrokeopacity{0.800000}%
\pgfsetdash{}{0pt}%
\pgfpathmoveto{\pgfqpoint{4.177702in}{2.827014in}}%
\pgfpathlineto{\pgfqpoint{2.979952in}{2.554324in}}%
\pgfusepath{stroke}%
\end{pgfscope}%
\begin{pgfscope}%
\pgfpathrectangle{\pgfqpoint{0.570343in}{0.331635in}}{\pgfqpoint{4.960000in}{3.696000in}}%
\pgfusepath{clip}%
\pgfsetrectcap%
\pgfsetroundjoin%
\pgfsetlinewidth{1.505625pt}%
\definecolor{currentstroke}{rgb}{1.000000,0.705882,0.509804}%
\pgfsetstrokecolor{currentstroke}%
\pgfsetstrokeopacity{0.800000}%
\pgfsetdash{}{0pt}%
\pgfpathmoveto{\pgfqpoint{3.492167in}{2.417509in}}%
\pgfpathlineto{\pgfqpoint{2.979952in}{2.554324in}}%
\pgfusepath{stroke}%
\end{pgfscope}%
\begin{pgfscope}%
\pgfpathrectangle{\pgfqpoint{0.570343in}{0.331635in}}{\pgfqpoint{4.960000in}{3.696000in}}%
\pgfusepath{clip}%
\pgfsetrectcap%
\pgfsetroundjoin%
\pgfsetlinewidth{1.505625pt}%
\definecolor{currentstroke}{rgb}{1.000000,0.705882,0.509804}%
\pgfsetstrokecolor{currentstroke}%
\pgfsetstrokeopacity{0.800000}%
\pgfsetdash{}{0pt}%
\pgfpathmoveto{\pgfqpoint{2.226219in}{3.521793in}}%
\pgfpathlineto{\pgfqpoint{2.979952in}{2.554324in}}%
\pgfusepath{stroke}%
\end{pgfscope}%
\begin{pgfscope}%
\pgfpathrectangle{\pgfqpoint{0.570343in}{0.331635in}}{\pgfqpoint{4.960000in}{3.696000in}}%
\pgfusepath{clip}%
\pgfsetrectcap%
\pgfsetroundjoin%
\pgfsetlinewidth{1.505625pt}%
\definecolor{currentstroke}{rgb}{1.000000,0.705882,0.509804}%
\pgfsetstrokecolor{currentstroke}%
\pgfsetstrokeopacity{0.800000}%
\pgfsetdash{}{0pt}%
\pgfpathmoveto{\pgfqpoint{3.231645in}{2.868957in}}%
\pgfpathlineto{\pgfqpoint{2.979952in}{2.554324in}}%
\pgfusepath{stroke}%
\end{pgfscope}%
\begin{pgfscope}%
\pgfsetrectcap%
\pgfsetmiterjoin%
\pgfsetlinewidth{0.803000pt}%
\definecolor{currentstroke}{rgb}{0.000000,0.000000,0.000000}%
\pgfsetstrokecolor{currentstroke}%
\pgfsetdash{}{0pt}%
\pgfpathmoveto{\pgfqpoint{0.570343in}{0.331635in}}%
\pgfpathlineto{\pgfqpoint{0.570343in}{4.027635in}}%
\pgfusepath{stroke}%
\end{pgfscope}%
\begin{pgfscope}%
\pgfsetrectcap%
\pgfsetmiterjoin%
\pgfsetlinewidth{0.803000pt}%
\definecolor{currentstroke}{rgb}{0.000000,0.000000,0.000000}%
\pgfsetstrokecolor{currentstroke}%
\pgfsetdash{}{0pt}%
\pgfpathmoveto{\pgfqpoint{5.530343in}{0.331635in}}%
\pgfpathlineto{\pgfqpoint{5.530343in}{4.027635in}}%
\pgfusepath{stroke}%
\end{pgfscope}%
\begin{pgfscope}%
\pgfsetrectcap%
\pgfsetmiterjoin%
\pgfsetlinewidth{0.803000pt}%
\definecolor{currentstroke}{rgb}{0.000000,0.000000,0.000000}%
\pgfsetstrokecolor{currentstroke}%
\pgfsetdash{}{0pt}%
\pgfpathmoveto{\pgfqpoint{0.570343in}{0.331635in}}%
\pgfpathlineto{\pgfqpoint{5.530343in}{0.331635in}}%
\pgfusepath{stroke}%
\end{pgfscope}%
\begin{pgfscope}%
\pgfsetrectcap%
\pgfsetmiterjoin%
\pgfsetlinewidth{0.803000pt}%
\definecolor{currentstroke}{rgb}{0.000000,0.000000,0.000000}%
\pgfsetstrokecolor{currentstroke}%
\pgfsetdash{}{0pt}%
\pgfpathmoveto{\pgfqpoint{0.570343in}{4.027635in}}%
\pgfpathlineto{\pgfqpoint{5.530343in}{4.027635in}}%
\pgfusepath{stroke}%
\end{pgfscope}%
\begin{pgfscope}%
\definecolor{textcolor}{rgb}{0.000000,0.000000,0.000000}%
\pgfsetstrokecolor{textcolor}%
\pgfsetfillcolor{textcolor}%
\pgftext[x=3.050343in,y=4.110968in,,base]{\color{textcolor}\sffamily\fontsize{12.000000}{14.400000}\selectfont t-SNE for pix3d and interiornet}%
\end{pgfscope}%
\begin{pgfscope}%
\pgfsetbuttcap%
\pgfsetmiterjoin%
\definecolor{currentfill}{rgb}{1.000000,1.000000,1.000000}%
\pgfsetfillcolor{currentfill}%
\pgfsetfillopacity{0.800000}%
\pgfsetlinewidth{1.003750pt}%
\definecolor{currentstroke}{rgb}{0.800000,0.800000,0.800000}%
\pgfsetstrokecolor{currentstroke}%
\pgfsetstrokeopacity{0.800000}%
\pgfsetdash{}{0pt}%
\pgfpathmoveto{\pgfqpoint{4.258900in}{3.508809in}}%
\pgfpathlineto{\pgfqpoint{5.433121in}{3.508809in}}%
\pgfpathquadraticcurveto{\pgfqpoint{5.460899in}{3.508809in}}{\pgfqpoint{5.460899in}{3.536587in}}%
\pgfpathlineto{\pgfqpoint{5.460899in}{3.930413in}}%
\pgfpathquadraticcurveto{\pgfqpoint{5.460899in}{3.958191in}}{\pgfqpoint{5.433121in}{3.958191in}}%
\pgfpathlineto{\pgfqpoint{4.258900in}{3.958191in}}%
\pgfpathquadraticcurveto{\pgfqpoint{4.231122in}{3.958191in}}{\pgfqpoint{4.231122in}{3.930413in}}%
\pgfpathlineto{\pgfqpoint{4.231122in}{3.536587in}}%
\pgfpathquadraticcurveto{\pgfqpoint{4.231122in}{3.508809in}}{\pgfqpoint{4.258900in}{3.508809in}}%
\pgfpathclose%
\pgfusepath{stroke,fill}%
\end{pgfscope}%
\begin{pgfscope}%
\pgfsetbuttcap%
\pgfsetroundjoin%
\definecolor{currentfill}{rgb}{0.631373,0.788235,0.956863}%
\pgfsetfillcolor{currentfill}%
\pgfsetlinewidth{1.003750pt}%
\definecolor{currentstroke}{rgb}{0.631373,0.788235,0.956863}%
\pgfsetstrokecolor{currentstroke}%
\pgfsetdash{}{0pt}%
\pgfsys@defobject{currentmarker}{\pgfqpoint{-0.041667in}{-0.041667in}}{\pgfqpoint{0.041667in}{0.041667in}}{%
\pgfpathmoveto{\pgfqpoint{0.000000in}{-0.041667in}}%
\pgfpathcurveto{\pgfqpoint{0.011050in}{-0.041667in}}{\pgfqpoint{0.021649in}{-0.037276in}}{\pgfqpoint{0.029463in}{-0.029463in}}%
\pgfpathcurveto{\pgfqpoint{0.037276in}{-0.021649in}}{\pgfqpoint{0.041667in}{-0.011050in}}{\pgfqpoint{0.041667in}{0.000000in}}%
\pgfpathcurveto{\pgfqpoint{0.041667in}{0.011050in}}{\pgfqpoint{0.037276in}{0.021649in}}{\pgfqpoint{0.029463in}{0.029463in}}%
\pgfpathcurveto{\pgfqpoint{0.021649in}{0.037276in}}{\pgfqpoint{0.011050in}{0.041667in}}{\pgfqpoint{0.000000in}{0.041667in}}%
\pgfpathcurveto{\pgfqpoint{-0.011050in}{0.041667in}}{\pgfqpoint{-0.021649in}{0.037276in}}{\pgfqpoint{-0.029463in}{0.029463in}}%
\pgfpathcurveto{\pgfqpoint{-0.037276in}{0.021649in}}{\pgfqpoint{-0.041667in}{0.011050in}}{\pgfqpoint{-0.041667in}{0.000000in}}%
\pgfpathcurveto{\pgfqpoint{-0.041667in}{-0.011050in}}{\pgfqpoint{-0.037276in}{-0.021649in}}{\pgfqpoint{-0.029463in}{-0.029463in}}%
\pgfpathcurveto{\pgfqpoint{-0.021649in}{-0.037276in}}{\pgfqpoint{-0.011050in}{-0.041667in}}{\pgfqpoint{0.000000in}{-0.041667in}}%
\pgfpathclose%
\pgfusepath{stroke,fill}%
}%
\begin{pgfscope}%
\pgfsys@transformshift{4.425566in}{3.833570in}%
\pgfsys@useobject{currentmarker}{}%
\end{pgfscope}%
\end{pgfscope}%
\begin{pgfscope}%
\definecolor{textcolor}{rgb}{0.000000,0.000000,0.000000}%
\pgfsetstrokecolor{textcolor}%
\pgfsetfillcolor{textcolor}%
\pgftext[x=4.675566in,y=3.797112in,left,base]{\color{textcolor}\sffamily\fontsize{10.000000}{12.000000}\selectfont interiornet}%
\end{pgfscope}%
\begin{pgfscope}%
\pgfsetbuttcap%
\pgfsetroundjoin%
\definecolor{currentfill}{rgb}{1.000000,0.705882,0.509804}%
\pgfsetfillcolor{currentfill}%
\pgfsetlinewidth{1.003750pt}%
\definecolor{currentstroke}{rgb}{1.000000,0.705882,0.509804}%
\pgfsetstrokecolor{currentstroke}%
\pgfsetdash{}{0pt}%
\pgfsys@defobject{currentmarker}{\pgfqpoint{-0.041667in}{-0.041667in}}{\pgfqpoint{0.041667in}{0.041667in}}{%
\pgfpathmoveto{\pgfqpoint{0.000000in}{-0.041667in}}%
\pgfpathcurveto{\pgfqpoint{0.011050in}{-0.041667in}}{\pgfqpoint{0.021649in}{-0.037276in}}{\pgfqpoint{0.029463in}{-0.029463in}}%
\pgfpathcurveto{\pgfqpoint{0.037276in}{-0.021649in}}{\pgfqpoint{0.041667in}{-0.011050in}}{\pgfqpoint{0.041667in}{0.000000in}}%
\pgfpathcurveto{\pgfqpoint{0.041667in}{0.011050in}}{\pgfqpoint{0.037276in}{0.021649in}}{\pgfqpoint{0.029463in}{0.029463in}}%
\pgfpathcurveto{\pgfqpoint{0.021649in}{0.037276in}}{\pgfqpoint{0.011050in}{0.041667in}}{\pgfqpoint{0.000000in}{0.041667in}}%
\pgfpathcurveto{\pgfqpoint{-0.011050in}{0.041667in}}{\pgfqpoint{-0.021649in}{0.037276in}}{\pgfqpoint{-0.029463in}{0.029463in}}%
\pgfpathcurveto{\pgfqpoint{-0.037276in}{0.021649in}}{\pgfqpoint{-0.041667in}{0.011050in}}{\pgfqpoint{-0.041667in}{0.000000in}}%
\pgfpathcurveto{\pgfqpoint{-0.041667in}{-0.011050in}}{\pgfqpoint{-0.037276in}{-0.021649in}}{\pgfqpoint{-0.029463in}{-0.029463in}}%
\pgfpathcurveto{\pgfqpoint{-0.021649in}{-0.037276in}}{\pgfqpoint{-0.011050in}{-0.041667in}}{\pgfqpoint{0.000000in}{-0.041667in}}%
\pgfpathclose%
\pgfusepath{stroke,fill}%
}%
\begin{pgfscope}%
\pgfsys@transformshift{4.425566in}{3.629713in}%
\pgfsys@useobject{currentmarker}{}%
\end{pgfscope}%
\end{pgfscope}%
\begin{pgfscope}%
\definecolor{textcolor}{rgb}{0.000000,0.000000,0.000000}%
\pgfsetstrokecolor{textcolor}%
\pgfsetfillcolor{textcolor}%
\pgftext[x=4.675566in,y=3.593255in,left,base]{\color{textcolor}\sffamily\fontsize{10.000000}{12.000000}\selectfont pix3d}%
\end{pgfscope}%
\end{pgfpicture}%
\makeatother%
\endgroup%
}

    \caption{T-SNE visualisation for images from individual photo-realistic synthetic dataset compared with Pix3D latent space.
        (Left to right, top to bottom) Openrooms, SceneNet, Blenderproc, \gls{ai2thor}, \gls{front}, Hyperism, \gls{free} in blue;
        compared with Pix3D in orange.}
    \label{fig:tsne per dataset}
\end{figure}


\subsection{Quantitative}\label{subsec:quantitative}

We visualised the embedding space using T-SNE in the above subsection~\ref{subsec:qualitative}.
All the datasets seem to have a very close relation with the real dataset as we see atleast some points being in intersection with the latent space of real dataset.
In this section,we compare the datasets with quantitative assessment using Mean Squared Error(MSE) and Fr\'echet Inception Distance(FID)
As seen in the table~\ref{tab:quantitative-dataset-comparison}, we see that BlenderProc has least MSE of 5.53 to Pix3D,
while openrooms has highest MES of 7.63.
\gls{free} has a MSE of 7.075 which is below Openrooms and Hyperism.
Interestingly, \gls{ai2thor} seems to perform better in quantitative assessment of photorealism,
where both MSE and \gls{fid} are considerably lesser than other dataset,
proving that \gls{ai2thor} is closer to real dataset than what was visualised in figures~\ref{fig:photorealistic tsne} and~\ref{fig:tsne per dataset}.
\gls{free} has a \gls{fid} of 178.83 which is lesser than Openrooms, Hyperism and SceneNet.

\begin{table}[ht]
    \centering
    \begin{tabular}{|c |c |c |c|}
        \hline
        Dataset & \gls{mse} & \gls{fid} \\ [0.5ex]
        \hline\hline
        Openrooms & 7.63 & 189.43 \\
        \hline
        \Gls{ai2thor} & 6.94 & 164.61 \\
        \hline
        BlenderProc & 5.53 & 173.37 \\
        \hline
        Hyperism & 7.12 & 186.57 \\
        \hline
        \Gls{front} & 6.93 & 167.65 \\
        \hline
        InteriorNet & xx & xx \\
        \hline
        SceneNet & 6.7553 & 185.49 \\
        \hline
        \Gls{free} & 7.0750 & 178.83 \\[1ex]
        \hline
    \end{tabular}
    \caption{Table represents quantitative measure to compare synthetic dataset distribution with the real dataset(Pix3D)}
    \label{tab:quantitative-dataset-comparison}
\end{table}

\section{Datasets}\label{sec:datasets}
In this section, the datasets used for the following evaluations will be described.
The datasets are intended to have variations in domain randomisation to check its performance on 3D reconstruction task.

\subsection{Pix3D}
As mentioned in~\ref{subsec:why-pix3d?}, we use a real dataset from~\cite{pix3d} which is a collection of indoor scenes.
The 2 classes 'misc' and 'tools' are eliminated so that we focus only on furnitures.
The total images after the reduction is 9954 with 354 unique models.
The train and validation dataset is divided in the ratio of 70:30 giving us 6814 images from training and 3140 images for validation/test.
We do not have a test set only for this dataset since it is already limited, and the validation set is used as test set while testing with synthetic data.
Samples are as shown in~\ref{fig:samples for synthetic and real comparison}.

\subsection{\Gls{free} Version 1}
Version 1 of \gls{free} was created by keeping the models in the center of a default 3D room.
The camera distance was randomised between 1 and 2.5 meters from the model.
The camera view points and textures were randomised.
A total of 70000 images were synthetically generated using the \gls{free} 'Single Room pipeline' with 10000 images per category.
Samples are as shown in~\ref{fig:samples for synthetic and real comparison}.

\subsection{\Gls{free} Version 2}
Version 1 of \gls{free} was created by keeping the models in the center of a defualt 3D room.
The camera distance was randomised between 1 and 2.5 meters from the model.
The camera view points and textures were randomised.
A total of 21000 images were synthetically generated using the \gls{free} 'Multi Object pipeline' with 3000 images per category.
Samples are as shown in~\ref{fig:samples for synthetic and real comparison}.

\subsection{\Gls{free} Ablation}\label{subsec:s2r:3dfree-ablation}
To study the affects of parameters of domain randomisation, a study was conducted on chair dataset by omitting few factors one at a time.
This dataset were divided into 5 categories with different randomisation parameters.
The sample images with different randomisation is as shown in figure~\ref{fig:domain randomisation for ablation study}.


\begin{figure}
    \centering
    \resizebox{\textwidth}{!}{%% Creator: Matplotlib, PGF backend
%%
%% To include the figure in your LaTeX document, write
%%   \input{<filename>.pgf}
%%
%% Make sure the required packages are loaded in your preamble
%%   \usepackage{pgf}
%%
%% Figures using additional raster images can only be included by \input if
%% they are in the same directory as the main LaTeX file. For loading figures
%% from other directories you can use the `import` package
%%   \usepackage{import}
%%
%% and then include the figures with
%%   \import{<path to file>}{<filename>.pgf}
%%
%% Matplotlib used the following preamble
%%   \usepackage{fontspec}
%%   \setmainfont{DejaVuSerif.ttf}[Path=\detokenize{/Users/apple/opt/anaconda3/envs/kaolin/lib/python3.7/site-packages/matplotlib/mpl-data/fonts/ttf/}]
%%   \setsansfont{DejaVuSans.ttf}[Path=\detokenize{/Users/apple/opt/anaconda3/envs/kaolin/lib/python3.7/site-packages/matplotlib/mpl-data/fonts/ttf/}]
%%   \setmonofont{DejaVuSansMono.ttf}[Path=\detokenize{/Users/apple/opt/anaconda3/envs/kaolin/lib/python3.7/site-packages/matplotlib/mpl-data/fonts/ttf/}]
%%
\begingroup%
\makeatletter%
\begin{pgfpicture}%
\pgfpathrectangle{\pgfpointorigin}{\pgfqpoint{12.476162in}{8.341596in}}%
\pgfusepath{use as bounding box, clip}%
\begin{pgfscope}%
\pgfsetbuttcap%
\pgfsetmiterjoin%
\definecolor{currentfill}{rgb}{1.000000,1.000000,1.000000}%
\pgfsetfillcolor{currentfill}%
\pgfsetlinewidth{0.000000pt}%
\definecolor{currentstroke}{rgb}{1.000000,1.000000,1.000000}%
\pgfsetstrokecolor{currentstroke}%
\pgfsetdash{}{0pt}%
\pgfpathmoveto{\pgfqpoint{0.000000in}{0.000000in}}%
\pgfpathlineto{\pgfqpoint{12.476162in}{0.000000in}}%
\pgfpathlineto{\pgfqpoint{12.476162in}{8.341596in}}%
\pgfpathlineto{\pgfqpoint{0.000000in}{8.341596in}}%
\pgfpathclose%
\pgfusepath{fill}%
\end{pgfscope}%
\begin{pgfscope}%
\pgfsetbuttcap%
\pgfsetmiterjoin%
\definecolor{currentfill}{rgb}{1.000000,1.000000,1.000000}%
\pgfsetfillcolor{currentfill}%
\pgfsetlinewidth{0.000000pt}%
\definecolor{currentstroke}{rgb}{0.000000,0.000000,0.000000}%
\pgfsetstrokecolor{currentstroke}%
\pgfsetstrokeopacity{0.000000}%
\pgfsetdash{}{0pt}%
\pgfpathmoveto{\pgfqpoint{0.481978in}{0.331635in}}%
\pgfpathlineto{\pgfqpoint{9.781978in}{0.331635in}}%
\pgfpathlineto{\pgfqpoint{9.781978in}{8.031635in}}%
\pgfpathlineto{\pgfqpoint{0.481978in}{8.031635in}}%
\pgfpathclose%
\pgfusepath{fill}%
\end{pgfscope}%
\begin{pgfscope}%
\pgfpathrectangle{\pgfqpoint{0.481978in}{0.331635in}}{\pgfqpoint{9.300000in}{7.700000in}}%
\pgfusepath{clip}%
\pgfsetbuttcap%
\pgfsetroundjoin%
\definecolor{currentfill}{rgb}{0.631373,0.788235,0.956863}%
\pgfsetfillcolor{currentfill}%
\pgfsetlinewidth{0.481800pt}%
\definecolor{currentstroke}{rgb}{1.000000,1.000000,1.000000}%
\pgfsetstrokecolor{currentstroke}%
\pgfsetdash{}{0pt}%
\pgfpathmoveto{\pgfqpoint{3.350250in}{0.795870in}}%
\pgfpathcurveto{\pgfqpoint{3.361300in}{0.795870in}}{\pgfqpoint{3.371899in}{0.800261in}}{\pgfqpoint{3.379713in}{0.808074in}}%
\pgfpathcurveto{\pgfqpoint{3.387527in}{0.815888in}}{\pgfqpoint{3.391917in}{0.826487in}}{\pgfqpoint{3.391917in}{0.837537in}}%
\pgfpathcurveto{\pgfqpoint{3.391917in}{0.848587in}}{\pgfqpoint{3.387527in}{0.859186in}}{\pgfqpoint{3.379713in}{0.867000in}}%
\pgfpathcurveto{\pgfqpoint{3.371899in}{0.874813in}}{\pgfqpoint{3.361300in}{0.879204in}}{\pgfqpoint{3.350250in}{0.879204in}}%
\pgfpathcurveto{\pgfqpoint{3.339200in}{0.879204in}}{\pgfqpoint{3.328601in}{0.874813in}}{\pgfqpoint{3.320788in}{0.867000in}}%
\pgfpathcurveto{\pgfqpoint{3.312974in}{0.859186in}}{\pgfqpoint{3.308584in}{0.848587in}}{\pgfqpoint{3.308584in}{0.837537in}}%
\pgfpathcurveto{\pgfqpoint{3.308584in}{0.826487in}}{\pgfqpoint{3.312974in}{0.815888in}}{\pgfqpoint{3.320788in}{0.808074in}}%
\pgfpathcurveto{\pgfqpoint{3.328601in}{0.800261in}}{\pgfqpoint{3.339200in}{0.795870in}}{\pgfqpoint{3.350250in}{0.795870in}}%
\pgfpathclose%
\pgfusepath{stroke,fill}%
\end{pgfscope}%
\begin{pgfscope}%
\pgfpathrectangle{\pgfqpoint{0.481978in}{0.331635in}}{\pgfqpoint{9.300000in}{7.700000in}}%
\pgfusepath{clip}%
\pgfsetbuttcap%
\pgfsetroundjoin%
\definecolor{currentfill}{rgb}{0.631373,0.788235,0.956863}%
\pgfsetfillcolor{currentfill}%
\pgfsetlinewidth{0.481800pt}%
\definecolor{currentstroke}{rgb}{1.000000,1.000000,1.000000}%
\pgfsetstrokecolor{currentstroke}%
\pgfsetdash{}{0pt}%
\pgfpathmoveto{\pgfqpoint{3.059115in}{5.220279in}}%
\pgfpathcurveto{\pgfqpoint{3.070165in}{5.220279in}}{\pgfqpoint{3.080764in}{5.224669in}}{\pgfqpoint{3.088578in}{5.232482in}}%
\pgfpathcurveto{\pgfqpoint{3.096392in}{5.240296in}}{\pgfqpoint{3.100782in}{5.250895in}}{\pgfqpoint{3.100782in}{5.261945in}}%
\pgfpathcurveto{\pgfqpoint{3.100782in}{5.272995in}}{\pgfqpoint{3.096392in}{5.283594in}}{\pgfqpoint{3.088578in}{5.291408in}}%
\pgfpathcurveto{\pgfqpoint{3.080764in}{5.299222in}}{\pgfqpoint{3.070165in}{5.303612in}}{\pgfqpoint{3.059115in}{5.303612in}}%
\pgfpathcurveto{\pgfqpoint{3.048065in}{5.303612in}}{\pgfqpoint{3.037466in}{5.299222in}}{\pgfqpoint{3.029652in}{5.291408in}}%
\pgfpathcurveto{\pgfqpoint{3.021839in}{5.283594in}}{\pgfqpoint{3.017448in}{5.272995in}}{\pgfqpoint{3.017448in}{5.261945in}}%
\pgfpathcurveto{\pgfqpoint{3.017448in}{5.250895in}}{\pgfqpoint{3.021839in}{5.240296in}}{\pgfqpoint{3.029652in}{5.232482in}}%
\pgfpathcurveto{\pgfqpoint{3.037466in}{5.224669in}}{\pgfqpoint{3.048065in}{5.220279in}}{\pgfqpoint{3.059115in}{5.220279in}}%
\pgfpathclose%
\pgfusepath{stroke,fill}%
\end{pgfscope}%
\begin{pgfscope}%
\pgfpathrectangle{\pgfqpoint{0.481978in}{0.331635in}}{\pgfqpoint{9.300000in}{7.700000in}}%
\pgfusepath{clip}%
\pgfsetbuttcap%
\pgfsetroundjoin%
\definecolor{currentfill}{rgb}{0.631373,0.788235,0.956863}%
\pgfsetfillcolor{currentfill}%
\pgfsetlinewidth{0.481800pt}%
\definecolor{currentstroke}{rgb}{1.000000,1.000000,1.000000}%
\pgfsetstrokecolor{currentstroke}%
\pgfsetdash{}{0pt}%
\pgfpathmoveto{\pgfqpoint{4.845549in}{3.728687in}}%
\pgfpathcurveto{\pgfqpoint{4.856599in}{3.728687in}}{\pgfqpoint{4.867198in}{3.733077in}}{\pgfqpoint{4.875011in}{3.740891in}}%
\pgfpathcurveto{\pgfqpoint{4.882825in}{3.748705in}}{\pgfqpoint{4.887215in}{3.759304in}}{\pgfqpoint{4.887215in}{3.770354in}}%
\pgfpathcurveto{\pgfqpoint{4.887215in}{3.781404in}}{\pgfqpoint{4.882825in}{3.792003in}}{\pgfqpoint{4.875011in}{3.799817in}}%
\pgfpathcurveto{\pgfqpoint{4.867198in}{3.807630in}}{\pgfqpoint{4.856599in}{3.812021in}}{\pgfqpoint{4.845549in}{3.812021in}}%
\pgfpathcurveto{\pgfqpoint{4.834499in}{3.812021in}}{\pgfqpoint{4.823900in}{3.807630in}}{\pgfqpoint{4.816086in}{3.799817in}}%
\pgfpathcurveto{\pgfqpoint{4.808272in}{3.792003in}}{\pgfqpoint{4.803882in}{3.781404in}}{\pgfqpoint{4.803882in}{3.770354in}}%
\pgfpathcurveto{\pgfqpoint{4.803882in}{3.759304in}}{\pgfqpoint{4.808272in}{3.748705in}}{\pgfqpoint{4.816086in}{3.740891in}}%
\pgfpathcurveto{\pgfqpoint{4.823900in}{3.733077in}}{\pgfqpoint{4.834499in}{3.728687in}}{\pgfqpoint{4.845549in}{3.728687in}}%
\pgfpathclose%
\pgfusepath{stroke,fill}%
\end{pgfscope}%
\begin{pgfscope}%
\pgfpathrectangle{\pgfqpoint{0.481978in}{0.331635in}}{\pgfqpoint{9.300000in}{7.700000in}}%
\pgfusepath{clip}%
\pgfsetbuttcap%
\pgfsetroundjoin%
\definecolor{currentfill}{rgb}{0.631373,0.788235,0.956863}%
\pgfsetfillcolor{currentfill}%
\pgfsetlinewidth{0.481800pt}%
\definecolor{currentstroke}{rgb}{1.000000,1.000000,1.000000}%
\pgfsetstrokecolor{currentstroke}%
\pgfsetdash{}{0pt}%
\pgfpathmoveto{\pgfqpoint{1.304480in}{3.048221in}}%
\pgfpathcurveto{\pgfqpoint{1.315530in}{3.048221in}}{\pgfqpoint{1.326129in}{3.052611in}}{\pgfqpoint{1.333942in}{3.060424in}}%
\pgfpathcurveto{\pgfqpoint{1.341756in}{3.068238in}}{\pgfqpoint{1.346146in}{3.078837in}}{\pgfqpoint{1.346146in}{3.089887in}}%
\pgfpathcurveto{\pgfqpoint{1.346146in}{3.100937in}}{\pgfqpoint{1.341756in}{3.111536in}}{\pgfqpoint{1.333942in}{3.119350in}}%
\pgfpathcurveto{\pgfqpoint{1.326129in}{3.127164in}}{\pgfqpoint{1.315530in}{3.131554in}}{\pgfqpoint{1.304480in}{3.131554in}}%
\pgfpathcurveto{\pgfqpoint{1.293429in}{3.131554in}}{\pgfqpoint{1.282830in}{3.127164in}}{\pgfqpoint{1.275017in}{3.119350in}}%
\pgfpathcurveto{\pgfqpoint{1.267203in}{3.111536in}}{\pgfqpoint{1.262813in}{3.100937in}}{\pgfqpoint{1.262813in}{3.089887in}}%
\pgfpathcurveto{\pgfqpoint{1.262813in}{3.078837in}}{\pgfqpoint{1.267203in}{3.068238in}}{\pgfqpoint{1.275017in}{3.060424in}}%
\pgfpathcurveto{\pgfqpoint{1.282830in}{3.052611in}}{\pgfqpoint{1.293429in}{3.048221in}}{\pgfqpoint{1.304480in}{3.048221in}}%
\pgfpathclose%
\pgfusepath{stroke,fill}%
\end{pgfscope}%
\begin{pgfscope}%
\pgfpathrectangle{\pgfqpoint{0.481978in}{0.331635in}}{\pgfqpoint{9.300000in}{7.700000in}}%
\pgfusepath{clip}%
\pgfsetbuttcap%
\pgfsetroundjoin%
\definecolor{currentfill}{rgb}{0.631373,0.788235,0.956863}%
\pgfsetfillcolor{currentfill}%
\pgfsetlinewidth{0.481800pt}%
\definecolor{currentstroke}{rgb}{1.000000,1.000000,1.000000}%
\pgfsetstrokecolor{currentstroke}%
\pgfsetdash{}{0pt}%
\pgfpathmoveto{\pgfqpoint{4.583472in}{2.392940in}}%
\pgfpathcurveto{\pgfqpoint{4.594522in}{2.392940in}}{\pgfqpoint{4.605121in}{2.397330in}}{\pgfqpoint{4.612935in}{2.405144in}}%
\pgfpathcurveto{\pgfqpoint{4.620748in}{2.412957in}}{\pgfqpoint{4.625139in}{2.423556in}}{\pgfqpoint{4.625139in}{2.434606in}}%
\pgfpathcurveto{\pgfqpoint{4.625139in}{2.445657in}}{\pgfqpoint{4.620748in}{2.456256in}}{\pgfqpoint{4.612935in}{2.464069in}}%
\pgfpathcurveto{\pgfqpoint{4.605121in}{2.471883in}}{\pgfqpoint{4.594522in}{2.476273in}}{\pgfqpoint{4.583472in}{2.476273in}}%
\pgfpathcurveto{\pgfqpoint{4.572422in}{2.476273in}}{\pgfqpoint{4.561823in}{2.471883in}}{\pgfqpoint{4.554009in}{2.464069in}}%
\pgfpathcurveto{\pgfqpoint{4.546196in}{2.456256in}}{\pgfqpoint{4.541805in}{2.445657in}}{\pgfqpoint{4.541805in}{2.434606in}}%
\pgfpathcurveto{\pgfqpoint{4.541805in}{2.423556in}}{\pgfqpoint{4.546196in}{2.412957in}}{\pgfqpoint{4.554009in}{2.405144in}}%
\pgfpathcurveto{\pgfqpoint{4.561823in}{2.397330in}}{\pgfqpoint{4.572422in}{2.392940in}}{\pgfqpoint{4.583472in}{2.392940in}}%
\pgfpathclose%
\pgfusepath{stroke,fill}%
\end{pgfscope}%
\begin{pgfscope}%
\pgfpathrectangle{\pgfqpoint{0.481978in}{0.331635in}}{\pgfqpoint{9.300000in}{7.700000in}}%
\pgfusepath{clip}%
\pgfsetbuttcap%
\pgfsetroundjoin%
\definecolor{currentfill}{rgb}{0.631373,0.788235,0.956863}%
\pgfsetfillcolor{currentfill}%
\pgfsetlinewidth{0.481800pt}%
\definecolor{currentstroke}{rgb}{1.000000,1.000000,1.000000}%
\pgfsetstrokecolor{currentstroke}%
\pgfsetdash{}{0pt}%
\pgfpathmoveto{\pgfqpoint{5.509106in}{5.260891in}}%
\pgfpathcurveto{\pgfqpoint{5.520156in}{5.260891in}}{\pgfqpoint{5.530755in}{5.265281in}}{\pgfqpoint{5.538568in}{5.273094in}}%
\pgfpathcurveto{\pgfqpoint{5.546382in}{5.280908in}}{\pgfqpoint{5.550772in}{5.291507in}}{\pgfqpoint{5.550772in}{5.302557in}}%
\pgfpathcurveto{\pgfqpoint{5.550772in}{5.313607in}}{\pgfqpoint{5.546382in}{5.324206in}}{\pgfqpoint{5.538568in}{5.332020in}}%
\pgfpathcurveto{\pgfqpoint{5.530755in}{5.339834in}}{\pgfqpoint{5.520156in}{5.344224in}}{\pgfqpoint{5.509106in}{5.344224in}}%
\pgfpathcurveto{\pgfqpoint{5.498055in}{5.344224in}}{\pgfqpoint{5.487456in}{5.339834in}}{\pgfqpoint{5.479643in}{5.332020in}}%
\pgfpathcurveto{\pgfqpoint{5.471829in}{5.324206in}}{\pgfqpoint{5.467439in}{5.313607in}}{\pgfqpoint{5.467439in}{5.302557in}}%
\pgfpathcurveto{\pgfqpoint{5.467439in}{5.291507in}}{\pgfqpoint{5.471829in}{5.280908in}}{\pgfqpoint{5.479643in}{5.273094in}}%
\pgfpathcurveto{\pgfqpoint{5.487456in}{5.265281in}}{\pgfqpoint{5.498055in}{5.260891in}}{\pgfqpoint{5.509106in}{5.260891in}}%
\pgfpathclose%
\pgfusepath{stroke,fill}%
\end{pgfscope}%
\begin{pgfscope}%
\pgfpathrectangle{\pgfqpoint{0.481978in}{0.331635in}}{\pgfqpoint{9.300000in}{7.700000in}}%
\pgfusepath{clip}%
\pgfsetbuttcap%
\pgfsetroundjoin%
\definecolor{currentfill}{rgb}{0.631373,0.788235,0.956863}%
\pgfsetfillcolor{currentfill}%
\pgfsetlinewidth{0.481800pt}%
\definecolor{currentstroke}{rgb}{1.000000,1.000000,1.000000}%
\pgfsetstrokecolor{currentstroke}%
\pgfsetdash{}{0pt}%
\pgfpathmoveto{\pgfqpoint{1.270538in}{3.021766in}}%
\pgfpathcurveto{\pgfqpoint{1.281588in}{3.021766in}}{\pgfqpoint{1.292187in}{3.026156in}}{\pgfqpoint{1.300001in}{3.033970in}}%
\pgfpathcurveto{\pgfqpoint{1.307815in}{3.041783in}}{\pgfqpoint{1.312205in}{3.052382in}}{\pgfqpoint{1.312205in}{3.063432in}}%
\pgfpathcurveto{\pgfqpoint{1.312205in}{3.074483in}}{\pgfqpoint{1.307815in}{3.085082in}}{\pgfqpoint{1.300001in}{3.092895in}}%
\pgfpathcurveto{\pgfqpoint{1.292187in}{3.100709in}}{\pgfqpoint{1.281588in}{3.105099in}}{\pgfqpoint{1.270538in}{3.105099in}}%
\pgfpathcurveto{\pgfqpoint{1.259488in}{3.105099in}}{\pgfqpoint{1.248889in}{3.100709in}}{\pgfqpoint{1.241075in}{3.092895in}}%
\pgfpathcurveto{\pgfqpoint{1.233262in}{3.085082in}}{\pgfqpoint{1.228872in}{3.074483in}}{\pgfqpoint{1.228872in}{3.063432in}}%
\pgfpathcurveto{\pgfqpoint{1.228872in}{3.052382in}}{\pgfqpoint{1.233262in}{3.041783in}}{\pgfqpoint{1.241075in}{3.033970in}}%
\pgfpathcurveto{\pgfqpoint{1.248889in}{3.026156in}}{\pgfqpoint{1.259488in}{3.021766in}}{\pgfqpoint{1.270538in}{3.021766in}}%
\pgfpathclose%
\pgfusepath{stroke,fill}%
\end{pgfscope}%
\begin{pgfscope}%
\pgfpathrectangle{\pgfqpoint{0.481978in}{0.331635in}}{\pgfqpoint{9.300000in}{7.700000in}}%
\pgfusepath{clip}%
\pgfsetbuttcap%
\pgfsetroundjoin%
\definecolor{currentfill}{rgb}{0.631373,0.788235,0.956863}%
\pgfsetfillcolor{currentfill}%
\pgfsetlinewidth{0.481800pt}%
\definecolor{currentstroke}{rgb}{1.000000,1.000000,1.000000}%
\pgfsetstrokecolor{currentstroke}%
\pgfsetdash{}{0pt}%
\pgfpathmoveto{\pgfqpoint{4.233419in}{2.502329in}}%
\pgfpathcurveto{\pgfqpoint{4.244470in}{2.502329in}}{\pgfqpoint{4.255069in}{2.506720in}}{\pgfqpoint{4.262882in}{2.514533in}}%
\pgfpathcurveto{\pgfqpoint{4.270696in}{2.522347in}}{\pgfqpoint{4.275086in}{2.532946in}}{\pgfqpoint{4.275086in}{2.543996in}}%
\pgfpathcurveto{\pgfqpoint{4.275086in}{2.555046in}}{\pgfqpoint{4.270696in}{2.565645in}}{\pgfqpoint{4.262882in}{2.573459in}}%
\pgfpathcurveto{\pgfqpoint{4.255069in}{2.581272in}}{\pgfqpoint{4.244470in}{2.585663in}}{\pgfqpoint{4.233419in}{2.585663in}}%
\pgfpathcurveto{\pgfqpoint{4.222369in}{2.585663in}}{\pgfqpoint{4.211770in}{2.581272in}}{\pgfqpoint{4.203957in}{2.573459in}}%
\pgfpathcurveto{\pgfqpoint{4.196143in}{2.565645in}}{\pgfqpoint{4.191753in}{2.555046in}}{\pgfqpoint{4.191753in}{2.543996in}}%
\pgfpathcurveto{\pgfqpoint{4.191753in}{2.532946in}}{\pgfqpoint{4.196143in}{2.522347in}}{\pgfqpoint{4.203957in}{2.514533in}}%
\pgfpathcurveto{\pgfqpoint{4.211770in}{2.506720in}}{\pgfqpoint{4.222369in}{2.502329in}}{\pgfqpoint{4.233419in}{2.502329in}}%
\pgfpathclose%
\pgfusepath{stroke,fill}%
\end{pgfscope}%
\begin{pgfscope}%
\pgfpathrectangle{\pgfqpoint{0.481978in}{0.331635in}}{\pgfqpoint{9.300000in}{7.700000in}}%
\pgfusepath{clip}%
\pgfsetbuttcap%
\pgfsetroundjoin%
\definecolor{currentfill}{rgb}{0.631373,0.788235,0.956863}%
\pgfsetfillcolor{currentfill}%
\pgfsetlinewidth{0.481800pt}%
\definecolor{currentstroke}{rgb}{1.000000,1.000000,1.000000}%
\pgfsetstrokecolor{currentstroke}%
\pgfsetdash{}{0pt}%
\pgfpathmoveto{\pgfqpoint{2.873447in}{4.546373in}}%
\pgfpathcurveto{\pgfqpoint{2.884497in}{4.546373in}}{\pgfqpoint{2.895096in}{4.550764in}}{\pgfqpoint{2.902910in}{4.558577in}}%
\pgfpathcurveto{\pgfqpoint{2.910723in}{4.566391in}}{\pgfqpoint{2.915114in}{4.576990in}}{\pgfqpoint{2.915114in}{4.588040in}}%
\pgfpathcurveto{\pgfqpoint{2.915114in}{4.599090in}}{\pgfqpoint{2.910723in}{4.609689in}}{\pgfqpoint{2.902910in}{4.617503in}}%
\pgfpathcurveto{\pgfqpoint{2.895096in}{4.625317in}}{\pgfqpoint{2.884497in}{4.629707in}}{\pgfqpoint{2.873447in}{4.629707in}}%
\pgfpathcurveto{\pgfqpoint{2.862397in}{4.629707in}}{\pgfqpoint{2.851798in}{4.625317in}}{\pgfqpoint{2.843984in}{4.617503in}}%
\pgfpathcurveto{\pgfqpoint{2.836171in}{4.609689in}}{\pgfqpoint{2.831780in}{4.599090in}}{\pgfqpoint{2.831780in}{4.588040in}}%
\pgfpathcurveto{\pgfqpoint{2.831780in}{4.576990in}}{\pgfqpoint{2.836171in}{4.566391in}}{\pgfqpoint{2.843984in}{4.558577in}}%
\pgfpathcurveto{\pgfqpoint{2.851798in}{4.550764in}}{\pgfqpoint{2.862397in}{4.546373in}}{\pgfqpoint{2.873447in}{4.546373in}}%
\pgfpathclose%
\pgfusepath{stroke,fill}%
\end{pgfscope}%
\begin{pgfscope}%
\pgfpathrectangle{\pgfqpoint{0.481978in}{0.331635in}}{\pgfqpoint{9.300000in}{7.700000in}}%
\pgfusepath{clip}%
\pgfsetbuttcap%
\pgfsetroundjoin%
\definecolor{currentfill}{rgb}{0.631373,0.788235,0.956863}%
\pgfsetfillcolor{currentfill}%
\pgfsetlinewidth{0.481800pt}%
\definecolor{currentstroke}{rgb}{1.000000,1.000000,1.000000}%
\pgfsetstrokecolor{currentstroke}%
\pgfsetdash{}{0pt}%
\pgfpathmoveto{\pgfqpoint{6.038124in}{7.639968in}}%
\pgfpathcurveto{\pgfqpoint{6.049174in}{7.639968in}}{\pgfqpoint{6.059773in}{7.644359in}}{\pgfqpoint{6.067587in}{7.652172in}}%
\pgfpathcurveto{\pgfqpoint{6.075401in}{7.659986in}}{\pgfqpoint{6.079791in}{7.670585in}}{\pgfqpoint{6.079791in}{7.681635in}}%
\pgfpathcurveto{\pgfqpoint{6.079791in}{7.692685in}}{\pgfqpoint{6.075401in}{7.703284in}}{\pgfqpoint{6.067587in}{7.711098in}}%
\pgfpathcurveto{\pgfqpoint{6.059773in}{7.718911in}}{\pgfqpoint{6.049174in}{7.723302in}}{\pgfqpoint{6.038124in}{7.723302in}}%
\pgfpathcurveto{\pgfqpoint{6.027074in}{7.723302in}}{\pgfqpoint{6.016475in}{7.718911in}}{\pgfqpoint{6.008661in}{7.711098in}}%
\pgfpathcurveto{\pgfqpoint{6.000848in}{7.703284in}}{\pgfqpoint{5.996458in}{7.692685in}}{\pgfqpoint{5.996458in}{7.681635in}}%
\pgfpathcurveto{\pgfqpoint{5.996458in}{7.670585in}}{\pgfqpoint{6.000848in}{7.659986in}}{\pgfqpoint{6.008661in}{7.652172in}}%
\pgfpathcurveto{\pgfqpoint{6.016475in}{7.644359in}}{\pgfqpoint{6.027074in}{7.639968in}}{\pgfqpoint{6.038124in}{7.639968in}}%
\pgfpathclose%
\pgfusepath{stroke,fill}%
\end{pgfscope}%
\begin{pgfscope}%
\pgfpathrectangle{\pgfqpoint{0.481978in}{0.331635in}}{\pgfqpoint{9.300000in}{7.700000in}}%
\pgfusepath{clip}%
\pgfsetbuttcap%
\pgfsetroundjoin%
\definecolor{currentfill}{rgb}{0.631373,0.788235,0.956863}%
\pgfsetfillcolor{currentfill}%
\pgfsetlinewidth{0.481800pt}%
\definecolor{currentstroke}{rgb}{1.000000,1.000000,1.000000}%
\pgfsetstrokecolor{currentstroke}%
\pgfsetdash{}{0pt}%
\pgfpathmoveto{\pgfqpoint{6.068535in}{7.007729in}}%
\pgfpathcurveto{\pgfqpoint{6.079585in}{7.007729in}}{\pgfqpoint{6.090184in}{7.012120in}}{\pgfqpoint{6.097998in}{7.019933in}}%
\pgfpathcurveto{\pgfqpoint{6.105812in}{7.027747in}}{\pgfqpoint{6.110202in}{7.038346in}}{\pgfqpoint{6.110202in}{7.049396in}}%
\pgfpathcurveto{\pgfqpoint{6.110202in}{7.060446in}}{\pgfqpoint{6.105812in}{7.071045in}}{\pgfqpoint{6.097998in}{7.078859in}}%
\pgfpathcurveto{\pgfqpoint{6.090184in}{7.086672in}}{\pgfqpoint{6.079585in}{7.091063in}}{\pgfqpoint{6.068535in}{7.091063in}}%
\pgfpathcurveto{\pgfqpoint{6.057485in}{7.091063in}}{\pgfqpoint{6.046886in}{7.086672in}}{\pgfqpoint{6.039072in}{7.078859in}}%
\pgfpathcurveto{\pgfqpoint{6.031259in}{7.071045in}}{\pgfqpoint{6.026869in}{7.060446in}}{\pgfqpoint{6.026869in}{7.049396in}}%
\pgfpathcurveto{\pgfqpoint{6.026869in}{7.038346in}}{\pgfqpoint{6.031259in}{7.027747in}}{\pgfqpoint{6.039072in}{7.019933in}}%
\pgfpathcurveto{\pgfqpoint{6.046886in}{7.012120in}}{\pgfqpoint{6.057485in}{7.007729in}}{\pgfqpoint{6.068535in}{7.007729in}}%
\pgfpathclose%
\pgfusepath{stroke,fill}%
\end{pgfscope}%
\begin{pgfscope}%
\pgfpathrectangle{\pgfqpoint{0.481978in}{0.331635in}}{\pgfqpoint{9.300000in}{7.700000in}}%
\pgfusepath{clip}%
\pgfsetbuttcap%
\pgfsetroundjoin%
\definecolor{currentfill}{rgb}{0.631373,0.788235,0.956863}%
\pgfsetfillcolor{currentfill}%
\pgfsetlinewidth{0.481800pt}%
\definecolor{currentstroke}{rgb}{1.000000,1.000000,1.000000}%
\pgfsetstrokecolor{currentstroke}%
\pgfsetdash{}{0pt}%
\pgfpathmoveto{\pgfqpoint{0.904705in}{2.664956in}}%
\pgfpathcurveto{\pgfqpoint{0.915755in}{2.664956in}}{\pgfqpoint{0.926354in}{2.669346in}}{\pgfqpoint{0.934168in}{2.677160in}}%
\pgfpathcurveto{\pgfqpoint{0.941982in}{2.684973in}}{\pgfqpoint{0.946372in}{2.695572in}}{\pgfqpoint{0.946372in}{2.706623in}}%
\pgfpathcurveto{\pgfqpoint{0.946372in}{2.717673in}}{\pgfqpoint{0.941982in}{2.728272in}}{\pgfqpoint{0.934168in}{2.736085in}}%
\pgfpathcurveto{\pgfqpoint{0.926354in}{2.743899in}}{\pgfqpoint{0.915755in}{2.748289in}}{\pgfqpoint{0.904705in}{2.748289in}}%
\pgfpathcurveto{\pgfqpoint{0.893655in}{2.748289in}}{\pgfqpoint{0.883056in}{2.743899in}}{\pgfqpoint{0.875242in}{2.736085in}}%
\pgfpathcurveto{\pgfqpoint{0.867429in}{2.728272in}}{\pgfqpoint{0.863039in}{2.717673in}}{\pgfqpoint{0.863039in}{2.706623in}}%
\pgfpathcurveto{\pgfqpoint{0.863039in}{2.695572in}}{\pgfqpoint{0.867429in}{2.684973in}}{\pgfqpoint{0.875242in}{2.677160in}}%
\pgfpathcurveto{\pgfqpoint{0.883056in}{2.669346in}}{\pgfqpoint{0.893655in}{2.664956in}}{\pgfqpoint{0.904705in}{2.664956in}}%
\pgfpathclose%
\pgfusepath{stroke,fill}%
\end{pgfscope}%
\begin{pgfscope}%
\pgfpathrectangle{\pgfqpoint{0.481978in}{0.331635in}}{\pgfqpoint{9.300000in}{7.700000in}}%
\pgfusepath{clip}%
\pgfsetbuttcap%
\pgfsetroundjoin%
\definecolor{currentfill}{rgb}{0.631373,0.788235,0.956863}%
\pgfsetfillcolor{currentfill}%
\pgfsetlinewidth{0.481800pt}%
\definecolor{currentstroke}{rgb}{1.000000,1.000000,1.000000}%
\pgfsetstrokecolor{currentstroke}%
\pgfsetdash{}{0pt}%
\pgfpathmoveto{\pgfqpoint{3.355697in}{0.882879in}}%
\pgfpathcurveto{\pgfqpoint{3.366747in}{0.882879in}}{\pgfqpoint{3.377346in}{0.887270in}}{\pgfqpoint{3.385160in}{0.895083in}}%
\pgfpathcurveto{\pgfqpoint{3.392973in}{0.902897in}}{\pgfqpoint{3.397364in}{0.913496in}}{\pgfqpoint{3.397364in}{0.924546in}}%
\pgfpathcurveto{\pgfqpoint{3.397364in}{0.935596in}}{\pgfqpoint{3.392973in}{0.946195in}}{\pgfqpoint{3.385160in}{0.954009in}}%
\pgfpathcurveto{\pgfqpoint{3.377346in}{0.961823in}}{\pgfqpoint{3.366747in}{0.966213in}}{\pgfqpoint{3.355697in}{0.966213in}}%
\pgfpathcurveto{\pgfqpoint{3.344647in}{0.966213in}}{\pgfqpoint{3.334048in}{0.961823in}}{\pgfqpoint{3.326234in}{0.954009in}}%
\pgfpathcurveto{\pgfqpoint{3.318420in}{0.946195in}}{\pgfqpoint{3.314030in}{0.935596in}}{\pgfqpoint{3.314030in}{0.924546in}}%
\pgfpathcurveto{\pgfqpoint{3.314030in}{0.913496in}}{\pgfqpoint{3.318420in}{0.902897in}}{\pgfqpoint{3.326234in}{0.895083in}}%
\pgfpathcurveto{\pgfqpoint{3.334048in}{0.887270in}}{\pgfqpoint{3.344647in}{0.882879in}}{\pgfqpoint{3.355697in}{0.882879in}}%
\pgfpathclose%
\pgfusepath{stroke,fill}%
\end{pgfscope}%
\begin{pgfscope}%
\pgfpathrectangle{\pgfqpoint{0.481978in}{0.331635in}}{\pgfqpoint{9.300000in}{7.700000in}}%
\pgfusepath{clip}%
\pgfsetbuttcap%
\pgfsetroundjoin%
\definecolor{currentfill}{rgb}{0.631373,0.788235,0.956863}%
\pgfsetfillcolor{currentfill}%
\pgfsetlinewidth{0.481800pt}%
\definecolor{currentstroke}{rgb}{1.000000,1.000000,1.000000}%
\pgfsetstrokecolor{currentstroke}%
\pgfsetdash{}{0pt}%
\pgfpathmoveto{\pgfqpoint{1.822383in}{3.039349in}}%
\pgfpathcurveto{\pgfqpoint{1.833433in}{3.039349in}}{\pgfqpoint{1.844032in}{3.043739in}}{\pgfqpoint{1.851846in}{3.051553in}}%
\pgfpathcurveto{\pgfqpoint{1.859659in}{3.059367in}}{\pgfqpoint{1.864050in}{3.069966in}}{\pgfqpoint{1.864050in}{3.081016in}}%
\pgfpathcurveto{\pgfqpoint{1.864050in}{3.092066in}}{\pgfqpoint{1.859659in}{3.102665in}}{\pgfqpoint{1.851846in}{3.110479in}}%
\pgfpathcurveto{\pgfqpoint{1.844032in}{3.118292in}}{\pgfqpoint{1.833433in}{3.122682in}}{\pgfqpoint{1.822383in}{3.122682in}}%
\pgfpathcurveto{\pgfqpoint{1.811333in}{3.122682in}}{\pgfqpoint{1.800734in}{3.118292in}}{\pgfqpoint{1.792920in}{3.110479in}}%
\pgfpathcurveto{\pgfqpoint{1.785107in}{3.102665in}}{\pgfqpoint{1.780716in}{3.092066in}}{\pgfqpoint{1.780716in}{3.081016in}}%
\pgfpathcurveto{\pgfqpoint{1.780716in}{3.069966in}}{\pgfqpoint{1.785107in}{3.059367in}}{\pgfqpoint{1.792920in}{3.051553in}}%
\pgfpathcurveto{\pgfqpoint{1.800734in}{3.043739in}}{\pgfqpoint{1.811333in}{3.039349in}}{\pgfqpoint{1.822383in}{3.039349in}}%
\pgfpathclose%
\pgfusepath{stroke,fill}%
\end{pgfscope}%
\begin{pgfscope}%
\pgfpathrectangle{\pgfqpoint{0.481978in}{0.331635in}}{\pgfqpoint{9.300000in}{7.700000in}}%
\pgfusepath{clip}%
\pgfsetbuttcap%
\pgfsetroundjoin%
\definecolor{currentfill}{rgb}{0.631373,0.788235,0.956863}%
\pgfsetfillcolor{currentfill}%
\pgfsetlinewidth{0.481800pt}%
\definecolor{currentstroke}{rgb}{1.000000,1.000000,1.000000}%
\pgfsetstrokecolor{currentstroke}%
\pgfsetdash{}{0pt}%
\pgfpathmoveto{\pgfqpoint{6.431702in}{7.269912in}}%
\pgfpathcurveto{\pgfqpoint{6.442753in}{7.269912in}}{\pgfqpoint{6.453352in}{7.274302in}}{\pgfqpoint{6.461165in}{7.282116in}}%
\pgfpathcurveto{\pgfqpoint{6.468979in}{7.289929in}}{\pgfqpoint{6.473369in}{7.300528in}}{\pgfqpoint{6.473369in}{7.311578in}}%
\pgfpathcurveto{\pgfqpoint{6.473369in}{7.322629in}}{\pgfqpoint{6.468979in}{7.333228in}}{\pgfqpoint{6.461165in}{7.341041in}}%
\pgfpathcurveto{\pgfqpoint{6.453352in}{7.348855in}}{\pgfqpoint{6.442753in}{7.353245in}}{\pgfqpoint{6.431702in}{7.353245in}}%
\pgfpathcurveto{\pgfqpoint{6.420652in}{7.353245in}}{\pgfqpoint{6.410053in}{7.348855in}}{\pgfqpoint{6.402240in}{7.341041in}}%
\pgfpathcurveto{\pgfqpoint{6.394426in}{7.333228in}}{\pgfqpoint{6.390036in}{7.322629in}}{\pgfqpoint{6.390036in}{7.311578in}}%
\pgfpathcurveto{\pgfqpoint{6.390036in}{7.300528in}}{\pgfqpoint{6.394426in}{7.289929in}}{\pgfqpoint{6.402240in}{7.282116in}}%
\pgfpathcurveto{\pgfqpoint{6.410053in}{7.274302in}}{\pgfqpoint{6.420652in}{7.269912in}}{\pgfqpoint{6.431702in}{7.269912in}}%
\pgfpathclose%
\pgfusepath{stroke,fill}%
\end{pgfscope}%
\begin{pgfscope}%
\pgfpathrectangle{\pgfqpoint{0.481978in}{0.331635in}}{\pgfqpoint{9.300000in}{7.700000in}}%
\pgfusepath{clip}%
\pgfsetbuttcap%
\pgfsetroundjoin%
\definecolor{currentfill}{rgb}{0.631373,0.788235,0.956863}%
\pgfsetfillcolor{currentfill}%
\pgfsetlinewidth{0.481800pt}%
\definecolor{currentstroke}{rgb}{1.000000,1.000000,1.000000}%
\pgfsetstrokecolor{currentstroke}%
\pgfsetdash{}{0pt}%
\pgfpathmoveto{\pgfqpoint{1.601212in}{5.246053in}}%
\pgfpathcurveto{\pgfqpoint{1.612262in}{5.246053in}}{\pgfqpoint{1.622861in}{5.250443in}}{\pgfqpoint{1.630675in}{5.258257in}}%
\pgfpathcurveto{\pgfqpoint{1.638488in}{5.266070in}}{\pgfqpoint{1.642878in}{5.276669in}}{\pgfqpoint{1.642878in}{5.287720in}}%
\pgfpathcurveto{\pgfqpoint{1.642878in}{5.298770in}}{\pgfqpoint{1.638488in}{5.309369in}}{\pgfqpoint{1.630675in}{5.317182in}}%
\pgfpathcurveto{\pgfqpoint{1.622861in}{5.324996in}}{\pgfqpoint{1.612262in}{5.329386in}}{\pgfqpoint{1.601212in}{5.329386in}}%
\pgfpathcurveto{\pgfqpoint{1.590162in}{5.329386in}}{\pgfqpoint{1.579563in}{5.324996in}}{\pgfqpoint{1.571749in}{5.317182in}}%
\pgfpathcurveto{\pgfqpoint{1.563935in}{5.309369in}}{\pgfqpoint{1.559545in}{5.298770in}}{\pgfqpoint{1.559545in}{5.287720in}}%
\pgfpathcurveto{\pgfqpoint{1.559545in}{5.276669in}}{\pgfqpoint{1.563935in}{5.266070in}}{\pgfqpoint{1.571749in}{5.258257in}}%
\pgfpathcurveto{\pgfqpoint{1.579563in}{5.250443in}}{\pgfqpoint{1.590162in}{5.246053in}}{\pgfqpoint{1.601212in}{5.246053in}}%
\pgfpathclose%
\pgfusepath{stroke,fill}%
\end{pgfscope}%
\begin{pgfscope}%
\pgfpathrectangle{\pgfqpoint{0.481978in}{0.331635in}}{\pgfqpoint{9.300000in}{7.700000in}}%
\pgfusepath{clip}%
\pgfsetbuttcap%
\pgfsetroundjoin%
\definecolor{currentfill}{rgb}{0.631373,0.788235,0.956863}%
\pgfsetfillcolor{currentfill}%
\pgfsetlinewidth{0.481800pt}%
\definecolor{currentstroke}{rgb}{1.000000,1.000000,1.000000}%
\pgfsetstrokecolor{currentstroke}%
\pgfsetdash{}{0pt}%
\pgfpathmoveto{\pgfqpoint{2.280847in}{6.304822in}}%
\pgfpathcurveto{\pgfqpoint{2.291897in}{6.304822in}}{\pgfqpoint{2.302496in}{6.309212in}}{\pgfqpoint{2.310310in}{6.317026in}}%
\pgfpathcurveto{\pgfqpoint{2.318123in}{6.324840in}}{\pgfqpoint{2.322514in}{6.335439in}}{\pgfqpoint{2.322514in}{6.346489in}}%
\pgfpathcurveto{\pgfqpoint{2.322514in}{6.357539in}}{\pgfqpoint{2.318123in}{6.368138in}}{\pgfqpoint{2.310310in}{6.375951in}}%
\pgfpathcurveto{\pgfqpoint{2.302496in}{6.383765in}}{\pgfqpoint{2.291897in}{6.388155in}}{\pgfqpoint{2.280847in}{6.388155in}}%
\pgfpathcurveto{\pgfqpoint{2.269797in}{6.388155in}}{\pgfqpoint{2.259198in}{6.383765in}}{\pgfqpoint{2.251384in}{6.375951in}}%
\pgfpathcurveto{\pgfqpoint{2.243570in}{6.368138in}}{\pgfqpoint{2.239180in}{6.357539in}}{\pgfqpoint{2.239180in}{6.346489in}}%
\pgfpathcurveto{\pgfqpoint{2.239180in}{6.335439in}}{\pgfqpoint{2.243570in}{6.324840in}}{\pgfqpoint{2.251384in}{6.317026in}}%
\pgfpathcurveto{\pgfqpoint{2.259198in}{6.309212in}}{\pgfqpoint{2.269797in}{6.304822in}}{\pgfqpoint{2.280847in}{6.304822in}}%
\pgfpathclose%
\pgfusepath{stroke,fill}%
\end{pgfscope}%
\begin{pgfscope}%
\pgfpathrectangle{\pgfqpoint{0.481978in}{0.331635in}}{\pgfqpoint{9.300000in}{7.700000in}}%
\pgfusepath{clip}%
\pgfsetbuttcap%
\pgfsetroundjoin%
\definecolor{currentfill}{rgb}{0.631373,0.788235,0.956863}%
\pgfsetfillcolor{currentfill}%
\pgfsetlinewidth{0.481800pt}%
\definecolor{currentstroke}{rgb}{1.000000,1.000000,1.000000}%
\pgfsetstrokecolor{currentstroke}%
\pgfsetdash{}{0pt}%
\pgfpathmoveto{\pgfqpoint{1.914988in}{4.693993in}}%
\pgfpathcurveto{\pgfqpoint{1.926039in}{4.693993in}}{\pgfqpoint{1.936638in}{4.698383in}}{\pgfqpoint{1.944451in}{4.706197in}}%
\pgfpathcurveto{\pgfqpoint{1.952265in}{4.714010in}}{\pgfqpoint{1.956655in}{4.724609in}}{\pgfqpoint{1.956655in}{4.735660in}}%
\pgfpathcurveto{\pgfqpoint{1.956655in}{4.746710in}}{\pgfqpoint{1.952265in}{4.757309in}}{\pgfqpoint{1.944451in}{4.765122in}}%
\pgfpathcurveto{\pgfqpoint{1.936638in}{4.772936in}}{\pgfqpoint{1.926039in}{4.777326in}}{\pgfqpoint{1.914988in}{4.777326in}}%
\pgfpathcurveto{\pgfqpoint{1.903938in}{4.777326in}}{\pgfqpoint{1.893339in}{4.772936in}}{\pgfqpoint{1.885526in}{4.765122in}}%
\pgfpathcurveto{\pgfqpoint{1.877712in}{4.757309in}}{\pgfqpoint{1.873322in}{4.746710in}}{\pgfqpoint{1.873322in}{4.735660in}}%
\pgfpathcurveto{\pgfqpoint{1.873322in}{4.724609in}}{\pgfqpoint{1.877712in}{4.714010in}}{\pgfqpoint{1.885526in}{4.706197in}}%
\pgfpathcurveto{\pgfqpoint{1.893339in}{4.698383in}}{\pgfqpoint{1.903938in}{4.693993in}}{\pgfqpoint{1.914988in}{4.693993in}}%
\pgfpathclose%
\pgfusepath{stroke,fill}%
\end{pgfscope}%
\begin{pgfscope}%
\pgfpathrectangle{\pgfqpoint{0.481978in}{0.331635in}}{\pgfqpoint{9.300000in}{7.700000in}}%
\pgfusepath{clip}%
\pgfsetbuttcap%
\pgfsetroundjoin%
\definecolor{currentfill}{rgb}{0.631373,0.788235,0.956863}%
\pgfsetfillcolor{currentfill}%
\pgfsetlinewidth{0.481800pt}%
\definecolor{currentstroke}{rgb}{1.000000,1.000000,1.000000}%
\pgfsetstrokecolor{currentstroke}%
\pgfsetdash{}{0pt}%
\pgfpathmoveto{\pgfqpoint{3.751753in}{1.411582in}}%
\pgfpathcurveto{\pgfqpoint{3.762803in}{1.411582in}}{\pgfqpoint{3.773402in}{1.415972in}}{\pgfqpoint{3.781215in}{1.423785in}}%
\pgfpathcurveto{\pgfqpoint{3.789029in}{1.431599in}}{\pgfqpoint{3.793419in}{1.442198in}}{\pgfqpoint{3.793419in}{1.453248in}}%
\pgfpathcurveto{\pgfqpoint{3.793419in}{1.464298in}}{\pgfqpoint{3.789029in}{1.474897in}}{\pgfqpoint{3.781215in}{1.482711in}}%
\pgfpathcurveto{\pgfqpoint{3.773402in}{1.490525in}}{\pgfqpoint{3.762803in}{1.494915in}}{\pgfqpoint{3.751753in}{1.494915in}}%
\pgfpathcurveto{\pgfqpoint{3.740702in}{1.494915in}}{\pgfqpoint{3.730103in}{1.490525in}}{\pgfqpoint{3.722290in}{1.482711in}}%
\pgfpathcurveto{\pgfqpoint{3.714476in}{1.474897in}}{\pgfqpoint{3.710086in}{1.464298in}}{\pgfqpoint{3.710086in}{1.453248in}}%
\pgfpathcurveto{\pgfqpoint{3.710086in}{1.442198in}}{\pgfqpoint{3.714476in}{1.431599in}}{\pgfqpoint{3.722290in}{1.423785in}}%
\pgfpathcurveto{\pgfqpoint{3.730103in}{1.415972in}}{\pgfqpoint{3.740702in}{1.411582in}}{\pgfqpoint{3.751753in}{1.411582in}}%
\pgfpathclose%
\pgfusepath{stroke,fill}%
\end{pgfscope}%
\begin{pgfscope}%
\pgfpathrectangle{\pgfqpoint{0.481978in}{0.331635in}}{\pgfqpoint{9.300000in}{7.700000in}}%
\pgfusepath{clip}%
\pgfsetbuttcap%
\pgfsetroundjoin%
\definecolor{currentfill}{rgb}{0.631373,0.788235,0.956863}%
\pgfsetfillcolor{currentfill}%
\pgfsetlinewidth{0.481800pt}%
\definecolor{currentstroke}{rgb}{1.000000,1.000000,1.000000}%
\pgfsetstrokecolor{currentstroke}%
\pgfsetdash{}{0pt}%
\pgfpathmoveto{\pgfqpoint{5.753665in}{7.213922in}}%
\pgfpathcurveto{\pgfqpoint{5.764715in}{7.213922in}}{\pgfqpoint{5.775314in}{7.218313in}}{\pgfqpoint{5.783128in}{7.226126in}}%
\pgfpathcurveto{\pgfqpoint{5.790942in}{7.233940in}}{\pgfqpoint{5.795332in}{7.244539in}}{\pgfqpoint{5.795332in}{7.255589in}}%
\pgfpathcurveto{\pgfqpoint{5.795332in}{7.266639in}}{\pgfqpoint{5.790942in}{7.277238in}}{\pgfqpoint{5.783128in}{7.285052in}}%
\pgfpathcurveto{\pgfqpoint{5.775314in}{7.292865in}}{\pgfqpoint{5.764715in}{7.297256in}}{\pgfqpoint{5.753665in}{7.297256in}}%
\pgfpathcurveto{\pgfqpoint{5.742615in}{7.297256in}}{\pgfqpoint{5.732016in}{7.292865in}}{\pgfqpoint{5.724203in}{7.285052in}}%
\pgfpathcurveto{\pgfqpoint{5.716389in}{7.277238in}}{\pgfqpoint{5.711999in}{7.266639in}}{\pgfqpoint{5.711999in}{7.255589in}}%
\pgfpathcurveto{\pgfqpoint{5.711999in}{7.244539in}}{\pgfqpoint{5.716389in}{7.233940in}}{\pgfqpoint{5.724203in}{7.226126in}}%
\pgfpathcurveto{\pgfqpoint{5.732016in}{7.218313in}}{\pgfqpoint{5.742615in}{7.213922in}}{\pgfqpoint{5.753665in}{7.213922in}}%
\pgfpathclose%
\pgfusepath{stroke,fill}%
\end{pgfscope}%
\begin{pgfscope}%
\pgfpathrectangle{\pgfqpoint{0.481978in}{0.331635in}}{\pgfqpoint{9.300000in}{7.700000in}}%
\pgfusepath{clip}%
\pgfsetbuttcap%
\pgfsetroundjoin%
\definecolor{currentfill}{rgb}{0.631373,0.788235,0.956863}%
\pgfsetfillcolor{currentfill}%
\pgfsetlinewidth{0.481800pt}%
\definecolor{currentstroke}{rgb}{1.000000,1.000000,1.000000}%
\pgfsetstrokecolor{currentstroke}%
\pgfsetdash{}{0pt}%
\pgfpathmoveto{\pgfqpoint{5.495955in}{5.191634in}}%
\pgfpathcurveto{\pgfqpoint{5.507005in}{5.191634in}}{\pgfqpoint{5.517604in}{5.196024in}}{\pgfqpoint{5.525417in}{5.203837in}}%
\pgfpathcurveto{\pgfqpoint{5.533231in}{5.211651in}}{\pgfqpoint{5.537621in}{5.222250in}}{\pgfqpoint{5.537621in}{5.233300in}}%
\pgfpathcurveto{\pgfqpoint{5.537621in}{5.244350in}}{\pgfqpoint{5.533231in}{5.254949in}}{\pgfqpoint{5.525417in}{5.262763in}}%
\pgfpathcurveto{\pgfqpoint{5.517604in}{5.270577in}}{\pgfqpoint{5.507005in}{5.274967in}}{\pgfqpoint{5.495955in}{5.274967in}}%
\pgfpathcurveto{\pgfqpoint{5.484904in}{5.274967in}}{\pgfqpoint{5.474305in}{5.270577in}}{\pgfqpoint{5.466492in}{5.262763in}}%
\pgfpathcurveto{\pgfqpoint{5.458678in}{5.254949in}}{\pgfqpoint{5.454288in}{5.244350in}}{\pgfqpoint{5.454288in}{5.233300in}}%
\pgfpathcurveto{\pgfqpoint{5.454288in}{5.222250in}}{\pgfqpoint{5.458678in}{5.211651in}}{\pgfqpoint{5.466492in}{5.203837in}}%
\pgfpathcurveto{\pgfqpoint{5.474305in}{5.196024in}}{\pgfqpoint{5.484904in}{5.191634in}}{\pgfqpoint{5.495955in}{5.191634in}}%
\pgfpathclose%
\pgfusepath{stroke,fill}%
\end{pgfscope}%
\begin{pgfscope}%
\pgfpathrectangle{\pgfqpoint{0.481978in}{0.331635in}}{\pgfqpoint{9.300000in}{7.700000in}}%
\pgfusepath{clip}%
\pgfsetbuttcap%
\pgfsetroundjoin%
\definecolor{currentfill}{rgb}{0.631373,0.788235,0.956863}%
\pgfsetfillcolor{currentfill}%
\pgfsetlinewidth{0.481800pt}%
\definecolor{currentstroke}{rgb}{1.000000,1.000000,1.000000}%
\pgfsetstrokecolor{currentstroke}%
\pgfsetdash{}{0pt}%
\pgfpathmoveto{\pgfqpoint{4.238672in}{4.793073in}}%
\pgfpathcurveto{\pgfqpoint{4.249723in}{4.793073in}}{\pgfqpoint{4.260322in}{4.797463in}}{\pgfqpoint{4.268135in}{4.805277in}}%
\pgfpathcurveto{\pgfqpoint{4.275949in}{4.813090in}}{\pgfqpoint{4.280339in}{4.823689in}}{\pgfqpoint{4.280339in}{4.834740in}}%
\pgfpathcurveto{\pgfqpoint{4.280339in}{4.845790in}}{\pgfqpoint{4.275949in}{4.856389in}}{\pgfqpoint{4.268135in}{4.864202in}}%
\pgfpathcurveto{\pgfqpoint{4.260322in}{4.872016in}}{\pgfqpoint{4.249723in}{4.876406in}}{\pgfqpoint{4.238672in}{4.876406in}}%
\pgfpathcurveto{\pgfqpoint{4.227622in}{4.876406in}}{\pgfqpoint{4.217023in}{4.872016in}}{\pgfqpoint{4.209210in}{4.864202in}}%
\pgfpathcurveto{\pgfqpoint{4.201396in}{4.856389in}}{\pgfqpoint{4.197006in}{4.845790in}}{\pgfqpoint{4.197006in}{4.834740in}}%
\pgfpathcurveto{\pgfqpoint{4.197006in}{4.823689in}}{\pgfqpoint{4.201396in}{4.813090in}}{\pgfqpoint{4.209210in}{4.805277in}}%
\pgfpathcurveto{\pgfqpoint{4.217023in}{4.797463in}}{\pgfqpoint{4.227622in}{4.793073in}}{\pgfqpoint{4.238672in}{4.793073in}}%
\pgfpathclose%
\pgfusepath{stroke,fill}%
\end{pgfscope}%
\begin{pgfscope}%
\pgfpathrectangle{\pgfqpoint{0.481978in}{0.331635in}}{\pgfqpoint{9.300000in}{7.700000in}}%
\pgfusepath{clip}%
\pgfsetbuttcap%
\pgfsetroundjoin%
\definecolor{currentfill}{rgb}{0.631373,0.788235,0.956863}%
\pgfsetfillcolor{currentfill}%
\pgfsetlinewidth{0.481800pt}%
\definecolor{currentstroke}{rgb}{1.000000,1.000000,1.000000}%
\pgfsetstrokecolor{currentstroke}%
\pgfsetdash{}{0pt}%
\pgfpathmoveto{\pgfqpoint{6.175837in}{7.474137in}}%
\pgfpathcurveto{\pgfqpoint{6.186887in}{7.474137in}}{\pgfqpoint{6.197486in}{7.478527in}}{\pgfqpoint{6.205300in}{7.486341in}}%
\pgfpathcurveto{\pgfqpoint{6.213114in}{7.494154in}}{\pgfqpoint{6.217504in}{7.504753in}}{\pgfqpoint{6.217504in}{7.515803in}}%
\pgfpathcurveto{\pgfqpoint{6.217504in}{7.526854in}}{\pgfqpoint{6.213114in}{7.537453in}}{\pgfqpoint{6.205300in}{7.545266in}}%
\pgfpathcurveto{\pgfqpoint{6.197486in}{7.553080in}}{\pgfqpoint{6.186887in}{7.557470in}}{\pgfqpoint{6.175837in}{7.557470in}}%
\pgfpathcurveto{\pgfqpoint{6.164787in}{7.557470in}}{\pgfqpoint{6.154188in}{7.553080in}}{\pgfqpoint{6.146374in}{7.545266in}}%
\pgfpathcurveto{\pgfqpoint{6.138561in}{7.537453in}}{\pgfqpoint{6.134170in}{7.526854in}}{\pgfqpoint{6.134170in}{7.515803in}}%
\pgfpathcurveto{\pgfqpoint{6.134170in}{7.504753in}}{\pgfqpoint{6.138561in}{7.494154in}}{\pgfqpoint{6.146374in}{7.486341in}}%
\pgfpathcurveto{\pgfqpoint{6.154188in}{7.478527in}}{\pgfqpoint{6.164787in}{7.474137in}}{\pgfqpoint{6.175837in}{7.474137in}}%
\pgfpathclose%
\pgfusepath{stroke,fill}%
\end{pgfscope}%
\begin{pgfscope}%
\pgfpathrectangle{\pgfqpoint{0.481978in}{0.331635in}}{\pgfqpoint{9.300000in}{7.700000in}}%
\pgfusepath{clip}%
\pgfsetbuttcap%
\pgfsetroundjoin%
\definecolor{currentfill}{rgb}{0.631373,0.788235,0.956863}%
\pgfsetfillcolor{currentfill}%
\pgfsetlinewidth{0.481800pt}%
\definecolor{currentstroke}{rgb}{1.000000,1.000000,1.000000}%
\pgfsetstrokecolor{currentstroke}%
\pgfsetdash{}{0pt}%
\pgfpathmoveto{\pgfqpoint{4.277708in}{2.774798in}}%
\pgfpathcurveto{\pgfqpoint{4.288759in}{2.774798in}}{\pgfqpoint{4.299358in}{2.779188in}}{\pgfqpoint{4.307171in}{2.787002in}}%
\pgfpathcurveto{\pgfqpoint{4.314985in}{2.794816in}}{\pgfqpoint{4.319375in}{2.805415in}}{\pgfqpoint{4.319375in}{2.816465in}}%
\pgfpathcurveto{\pgfqpoint{4.319375in}{2.827515in}}{\pgfqpoint{4.314985in}{2.838114in}}{\pgfqpoint{4.307171in}{2.845928in}}%
\pgfpathcurveto{\pgfqpoint{4.299358in}{2.853741in}}{\pgfqpoint{4.288759in}{2.858131in}}{\pgfqpoint{4.277708in}{2.858131in}}%
\pgfpathcurveto{\pgfqpoint{4.266658in}{2.858131in}}{\pgfqpoint{4.256059in}{2.853741in}}{\pgfqpoint{4.248246in}{2.845928in}}%
\pgfpathcurveto{\pgfqpoint{4.240432in}{2.838114in}}{\pgfqpoint{4.236042in}{2.827515in}}{\pgfqpoint{4.236042in}{2.816465in}}%
\pgfpathcurveto{\pgfqpoint{4.236042in}{2.805415in}}{\pgfqpoint{4.240432in}{2.794816in}}{\pgfqpoint{4.248246in}{2.787002in}}%
\pgfpathcurveto{\pgfqpoint{4.256059in}{2.779188in}}{\pgfqpoint{4.266658in}{2.774798in}}{\pgfqpoint{4.277708in}{2.774798in}}%
\pgfpathclose%
\pgfusepath{stroke,fill}%
\end{pgfscope}%
\begin{pgfscope}%
\pgfpathrectangle{\pgfqpoint{0.481978in}{0.331635in}}{\pgfqpoint{9.300000in}{7.700000in}}%
\pgfusepath{clip}%
\pgfsetbuttcap%
\pgfsetroundjoin%
\definecolor{currentfill}{rgb}{0.631373,0.788235,0.956863}%
\pgfsetfillcolor{currentfill}%
\pgfsetlinewidth{0.481800pt}%
\definecolor{currentstroke}{rgb}{1.000000,1.000000,1.000000}%
\pgfsetstrokecolor{currentstroke}%
\pgfsetdash{}{0pt}%
\pgfpathmoveto{\pgfqpoint{1.180315in}{4.828736in}}%
\pgfpathcurveto{\pgfqpoint{1.191365in}{4.828736in}}{\pgfqpoint{1.201964in}{4.833126in}}{\pgfqpoint{1.209778in}{4.840939in}}%
\pgfpathcurveto{\pgfqpoint{1.217592in}{4.848753in}}{\pgfqpoint{1.221982in}{4.859352in}}{\pgfqpoint{1.221982in}{4.870402in}}%
\pgfpathcurveto{\pgfqpoint{1.221982in}{4.881452in}}{\pgfqpoint{1.217592in}{4.892051in}}{\pgfqpoint{1.209778in}{4.899865in}}%
\pgfpathcurveto{\pgfqpoint{1.201964in}{4.907679in}}{\pgfqpoint{1.191365in}{4.912069in}}{\pgfqpoint{1.180315in}{4.912069in}}%
\pgfpathcurveto{\pgfqpoint{1.169265in}{4.912069in}}{\pgfqpoint{1.158666in}{4.907679in}}{\pgfqpoint{1.150852in}{4.899865in}}%
\pgfpathcurveto{\pgfqpoint{1.143039in}{4.892051in}}{\pgfqpoint{1.138649in}{4.881452in}}{\pgfqpoint{1.138649in}{4.870402in}}%
\pgfpathcurveto{\pgfqpoint{1.138649in}{4.859352in}}{\pgfqpoint{1.143039in}{4.848753in}}{\pgfqpoint{1.150852in}{4.840939in}}%
\pgfpathcurveto{\pgfqpoint{1.158666in}{4.833126in}}{\pgfqpoint{1.169265in}{4.828736in}}{\pgfqpoint{1.180315in}{4.828736in}}%
\pgfpathclose%
\pgfusepath{stroke,fill}%
\end{pgfscope}%
\begin{pgfscope}%
\pgfpathrectangle{\pgfqpoint{0.481978in}{0.331635in}}{\pgfqpoint{9.300000in}{7.700000in}}%
\pgfusepath{clip}%
\pgfsetbuttcap%
\pgfsetroundjoin%
\definecolor{currentfill}{rgb}{0.631373,0.788235,0.956863}%
\pgfsetfillcolor{currentfill}%
\pgfsetlinewidth{0.481800pt}%
\definecolor{currentstroke}{rgb}{1.000000,1.000000,1.000000}%
\pgfsetstrokecolor{currentstroke}%
\pgfsetdash{}{0pt}%
\pgfpathmoveto{\pgfqpoint{4.575953in}{4.795945in}}%
\pgfpathcurveto{\pgfqpoint{4.587003in}{4.795945in}}{\pgfqpoint{4.597602in}{4.800336in}}{\pgfqpoint{4.605416in}{4.808149in}}%
\pgfpathcurveto{\pgfqpoint{4.613229in}{4.815963in}}{\pgfqpoint{4.617619in}{4.826562in}}{\pgfqpoint{4.617619in}{4.837612in}}%
\pgfpathcurveto{\pgfqpoint{4.617619in}{4.848662in}}{\pgfqpoint{4.613229in}{4.859261in}}{\pgfqpoint{4.605416in}{4.867075in}}%
\pgfpathcurveto{\pgfqpoint{4.597602in}{4.874888in}}{\pgfqpoint{4.587003in}{4.879279in}}{\pgfqpoint{4.575953in}{4.879279in}}%
\pgfpathcurveto{\pgfqpoint{4.564903in}{4.879279in}}{\pgfqpoint{4.554304in}{4.874888in}}{\pgfqpoint{4.546490in}{4.867075in}}%
\pgfpathcurveto{\pgfqpoint{4.538676in}{4.859261in}}{\pgfqpoint{4.534286in}{4.848662in}}{\pgfqpoint{4.534286in}{4.837612in}}%
\pgfpathcurveto{\pgfqpoint{4.534286in}{4.826562in}}{\pgfqpoint{4.538676in}{4.815963in}}{\pgfqpoint{4.546490in}{4.808149in}}%
\pgfpathcurveto{\pgfqpoint{4.554304in}{4.800336in}}{\pgfqpoint{4.564903in}{4.795945in}}{\pgfqpoint{4.575953in}{4.795945in}}%
\pgfpathclose%
\pgfusepath{stroke,fill}%
\end{pgfscope}%
\begin{pgfscope}%
\pgfpathrectangle{\pgfqpoint{0.481978in}{0.331635in}}{\pgfqpoint{9.300000in}{7.700000in}}%
\pgfusepath{clip}%
\pgfsetbuttcap%
\pgfsetroundjoin%
\definecolor{currentfill}{rgb}{0.631373,0.788235,0.956863}%
\pgfsetfillcolor{currentfill}%
\pgfsetlinewidth{0.481800pt}%
\definecolor{currentstroke}{rgb}{1.000000,1.000000,1.000000}%
\pgfsetstrokecolor{currentstroke}%
\pgfsetdash{}{0pt}%
\pgfpathmoveto{\pgfqpoint{7.415126in}{4.531322in}}%
\pgfpathcurveto{\pgfqpoint{7.426176in}{4.531322in}}{\pgfqpoint{7.436775in}{4.535712in}}{\pgfqpoint{7.444589in}{4.543526in}}%
\pgfpathcurveto{\pgfqpoint{7.452402in}{4.551340in}}{\pgfqpoint{7.456793in}{4.561939in}}{\pgfqpoint{7.456793in}{4.572989in}}%
\pgfpathcurveto{\pgfqpoint{7.456793in}{4.584039in}}{\pgfqpoint{7.452402in}{4.594638in}}{\pgfqpoint{7.444589in}{4.602452in}}%
\pgfpathcurveto{\pgfqpoint{7.436775in}{4.610265in}}{\pgfqpoint{7.426176in}{4.614655in}}{\pgfqpoint{7.415126in}{4.614655in}}%
\pgfpathcurveto{\pgfqpoint{7.404076in}{4.614655in}}{\pgfqpoint{7.393477in}{4.610265in}}{\pgfqpoint{7.385663in}{4.602452in}}%
\pgfpathcurveto{\pgfqpoint{7.377850in}{4.594638in}}{\pgfqpoint{7.373459in}{4.584039in}}{\pgfqpoint{7.373459in}{4.572989in}}%
\pgfpathcurveto{\pgfqpoint{7.373459in}{4.561939in}}{\pgfqpoint{7.377850in}{4.551340in}}{\pgfqpoint{7.385663in}{4.543526in}}%
\pgfpathcurveto{\pgfqpoint{7.393477in}{4.535712in}}{\pgfqpoint{7.404076in}{4.531322in}}{\pgfqpoint{7.415126in}{4.531322in}}%
\pgfpathclose%
\pgfusepath{stroke,fill}%
\end{pgfscope}%
\begin{pgfscope}%
\pgfpathrectangle{\pgfqpoint{0.481978in}{0.331635in}}{\pgfqpoint{9.300000in}{7.700000in}}%
\pgfusepath{clip}%
\pgfsetbuttcap%
\pgfsetroundjoin%
\definecolor{currentfill}{rgb}{0.631373,0.788235,0.956863}%
\pgfsetfillcolor{currentfill}%
\pgfsetlinewidth{0.481800pt}%
\definecolor{currentstroke}{rgb}{1.000000,1.000000,1.000000}%
\pgfsetstrokecolor{currentstroke}%
\pgfsetdash{}{0pt}%
\pgfpathmoveto{\pgfqpoint{5.743162in}{7.207909in}}%
\pgfpathcurveto{\pgfqpoint{5.754212in}{7.207909in}}{\pgfqpoint{5.764811in}{7.212299in}}{\pgfqpoint{5.772625in}{7.220113in}}%
\pgfpathcurveto{\pgfqpoint{5.780439in}{7.227926in}}{\pgfqpoint{5.784829in}{7.238525in}}{\pgfqpoint{5.784829in}{7.249576in}}%
\pgfpathcurveto{\pgfqpoint{5.784829in}{7.260626in}}{\pgfqpoint{5.780439in}{7.271225in}}{\pgfqpoint{5.772625in}{7.279038in}}%
\pgfpathcurveto{\pgfqpoint{5.764811in}{7.286852in}}{\pgfqpoint{5.754212in}{7.291242in}}{\pgfqpoint{5.743162in}{7.291242in}}%
\pgfpathcurveto{\pgfqpoint{5.732112in}{7.291242in}}{\pgfqpoint{5.721513in}{7.286852in}}{\pgfqpoint{5.713699in}{7.279038in}}%
\pgfpathcurveto{\pgfqpoint{5.705886in}{7.271225in}}{\pgfqpoint{5.701495in}{7.260626in}}{\pgfqpoint{5.701495in}{7.249576in}}%
\pgfpathcurveto{\pgfqpoint{5.701495in}{7.238525in}}{\pgfqpoint{5.705886in}{7.227926in}}{\pgfqpoint{5.713699in}{7.220113in}}%
\pgfpathcurveto{\pgfqpoint{5.721513in}{7.212299in}}{\pgfqpoint{5.732112in}{7.207909in}}{\pgfqpoint{5.743162in}{7.207909in}}%
\pgfpathclose%
\pgfusepath{stroke,fill}%
\end{pgfscope}%
\begin{pgfscope}%
\pgfpathrectangle{\pgfqpoint{0.481978in}{0.331635in}}{\pgfqpoint{9.300000in}{7.700000in}}%
\pgfusepath{clip}%
\pgfsetbuttcap%
\pgfsetroundjoin%
\definecolor{currentfill}{rgb}{0.631373,0.788235,0.956863}%
\pgfsetfillcolor{currentfill}%
\pgfsetlinewidth{0.481800pt}%
\definecolor{currentstroke}{rgb}{1.000000,1.000000,1.000000}%
\pgfsetstrokecolor{currentstroke}%
\pgfsetdash{}{0pt}%
\pgfpathmoveto{\pgfqpoint{3.731632in}{6.321424in}}%
\pgfpathcurveto{\pgfqpoint{3.742682in}{6.321424in}}{\pgfqpoint{3.753281in}{6.325815in}}{\pgfqpoint{3.761095in}{6.333628in}}%
\pgfpathcurveto{\pgfqpoint{3.768908in}{6.341442in}}{\pgfqpoint{3.773299in}{6.352041in}}{\pgfqpoint{3.773299in}{6.363091in}}%
\pgfpathcurveto{\pgfqpoint{3.773299in}{6.374141in}}{\pgfqpoint{3.768908in}{6.384740in}}{\pgfqpoint{3.761095in}{6.392554in}}%
\pgfpathcurveto{\pgfqpoint{3.753281in}{6.400367in}}{\pgfqpoint{3.742682in}{6.404758in}}{\pgfqpoint{3.731632in}{6.404758in}}%
\pgfpathcurveto{\pgfqpoint{3.720582in}{6.404758in}}{\pgfqpoint{3.709983in}{6.400367in}}{\pgfqpoint{3.702169in}{6.392554in}}%
\pgfpathcurveto{\pgfqpoint{3.694356in}{6.384740in}}{\pgfqpoint{3.689965in}{6.374141in}}{\pgfqpoint{3.689965in}{6.363091in}}%
\pgfpathcurveto{\pgfqpoint{3.689965in}{6.352041in}}{\pgfqpoint{3.694356in}{6.341442in}}{\pgfqpoint{3.702169in}{6.333628in}}%
\pgfpathcurveto{\pgfqpoint{3.709983in}{6.325815in}}{\pgfqpoint{3.720582in}{6.321424in}}{\pgfqpoint{3.731632in}{6.321424in}}%
\pgfpathclose%
\pgfusepath{stroke,fill}%
\end{pgfscope}%
\begin{pgfscope}%
\pgfpathrectangle{\pgfqpoint{0.481978in}{0.331635in}}{\pgfqpoint{9.300000in}{7.700000in}}%
\pgfusepath{clip}%
\pgfsetbuttcap%
\pgfsetroundjoin%
\definecolor{currentfill}{rgb}{0.631373,0.788235,0.956863}%
\pgfsetfillcolor{currentfill}%
\pgfsetlinewidth{0.481800pt}%
\definecolor{currentstroke}{rgb}{1.000000,1.000000,1.000000}%
\pgfsetstrokecolor{currentstroke}%
\pgfsetdash{}{0pt}%
\pgfpathmoveto{\pgfqpoint{3.354108in}{0.639968in}}%
\pgfpathcurveto{\pgfqpoint{3.365158in}{0.639968in}}{\pgfqpoint{3.375757in}{0.644359in}}{\pgfqpoint{3.383571in}{0.652172in}}%
\pgfpathcurveto{\pgfqpoint{3.391384in}{0.659986in}}{\pgfqpoint{3.395775in}{0.670585in}}{\pgfqpoint{3.395775in}{0.681635in}}%
\pgfpathcurveto{\pgfqpoint{3.395775in}{0.692685in}}{\pgfqpoint{3.391384in}{0.703284in}}{\pgfqpoint{3.383571in}{0.711098in}}%
\pgfpathcurveto{\pgfqpoint{3.375757in}{0.718911in}}{\pgfqpoint{3.365158in}{0.723302in}}{\pgfqpoint{3.354108in}{0.723302in}}%
\pgfpathcurveto{\pgfqpoint{3.343058in}{0.723302in}}{\pgfqpoint{3.332459in}{0.718911in}}{\pgfqpoint{3.324645in}{0.711098in}}%
\pgfpathcurveto{\pgfqpoint{3.316832in}{0.703284in}}{\pgfqpoint{3.312441in}{0.692685in}}{\pgfqpoint{3.312441in}{0.681635in}}%
\pgfpathcurveto{\pgfqpoint{3.312441in}{0.670585in}}{\pgfqpoint{3.316832in}{0.659986in}}{\pgfqpoint{3.324645in}{0.652172in}}%
\pgfpathcurveto{\pgfqpoint{3.332459in}{0.644359in}}{\pgfqpoint{3.343058in}{0.639968in}}{\pgfqpoint{3.354108in}{0.639968in}}%
\pgfpathclose%
\pgfusepath{stroke,fill}%
\end{pgfscope}%
\begin{pgfscope}%
\pgfpathrectangle{\pgfqpoint{0.481978in}{0.331635in}}{\pgfqpoint{9.300000in}{7.700000in}}%
\pgfusepath{clip}%
\pgfsetbuttcap%
\pgfsetroundjoin%
\definecolor{currentfill}{rgb}{0.631373,0.788235,0.956863}%
\pgfsetfillcolor{currentfill}%
\pgfsetlinewidth{0.481800pt}%
\definecolor{currentstroke}{rgb}{1.000000,1.000000,1.000000}%
\pgfsetstrokecolor{currentstroke}%
\pgfsetdash{}{0pt}%
\pgfpathmoveto{\pgfqpoint{2.507727in}{5.202892in}}%
\pgfpathcurveto{\pgfqpoint{2.518777in}{5.202892in}}{\pgfqpoint{2.529376in}{5.207282in}}{\pgfqpoint{2.537190in}{5.215096in}}%
\pgfpathcurveto{\pgfqpoint{2.545003in}{5.222910in}}{\pgfqpoint{2.549394in}{5.233509in}}{\pgfqpoint{2.549394in}{5.244559in}}%
\pgfpathcurveto{\pgfqpoint{2.549394in}{5.255609in}}{\pgfqpoint{2.545003in}{5.266208in}}{\pgfqpoint{2.537190in}{5.274022in}}%
\pgfpathcurveto{\pgfqpoint{2.529376in}{5.281835in}}{\pgfqpoint{2.518777in}{5.286225in}}{\pgfqpoint{2.507727in}{5.286225in}}%
\pgfpathcurveto{\pgfqpoint{2.496677in}{5.286225in}}{\pgfqpoint{2.486078in}{5.281835in}}{\pgfqpoint{2.478264in}{5.274022in}}%
\pgfpathcurveto{\pgfqpoint{2.470451in}{5.266208in}}{\pgfqpoint{2.466060in}{5.255609in}}{\pgfqpoint{2.466060in}{5.244559in}}%
\pgfpathcurveto{\pgfqpoint{2.466060in}{5.233509in}}{\pgfqpoint{2.470451in}{5.222910in}}{\pgfqpoint{2.478264in}{5.215096in}}%
\pgfpathcurveto{\pgfqpoint{2.486078in}{5.207282in}}{\pgfqpoint{2.496677in}{5.202892in}}{\pgfqpoint{2.507727in}{5.202892in}}%
\pgfpathclose%
\pgfusepath{stroke,fill}%
\end{pgfscope}%
\begin{pgfscope}%
\pgfpathrectangle{\pgfqpoint{0.481978in}{0.331635in}}{\pgfqpoint{9.300000in}{7.700000in}}%
\pgfusepath{clip}%
\pgfsetbuttcap%
\pgfsetroundjoin%
\definecolor{currentfill}{rgb}{0.631373,0.788235,0.956863}%
\pgfsetfillcolor{currentfill}%
\pgfsetlinewidth{0.481800pt}%
\definecolor{currentstroke}{rgb}{1.000000,1.000000,1.000000}%
\pgfsetstrokecolor{currentstroke}%
\pgfsetdash{}{0pt}%
\pgfpathmoveto{\pgfqpoint{3.345563in}{4.675887in}}%
\pgfpathcurveto{\pgfqpoint{3.356613in}{4.675887in}}{\pgfqpoint{3.367212in}{4.680278in}}{\pgfqpoint{3.375026in}{4.688091in}}%
\pgfpathcurveto{\pgfqpoint{3.382839in}{4.695905in}}{\pgfqpoint{3.387230in}{4.706504in}}{\pgfqpoint{3.387230in}{4.717554in}}%
\pgfpathcurveto{\pgfqpoint{3.387230in}{4.728604in}}{\pgfqpoint{3.382839in}{4.739203in}}{\pgfqpoint{3.375026in}{4.747017in}}%
\pgfpathcurveto{\pgfqpoint{3.367212in}{4.754830in}}{\pgfqpoint{3.356613in}{4.759221in}}{\pgfqpoint{3.345563in}{4.759221in}}%
\pgfpathcurveto{\pgfqpoint{3.334513in}{4.759221in}}{\pgfqpoint{3.323914in}{4.754830in}}{\pgfqpoint{3.316100in}{4.747017in}}%
\pgfpathcurveto{\pgfqpoint{3.308287in}{4.739203in}}{\pgfqpoint{3.303896in}{4.728604in}}{\pgfqpoint{3.303896in}{4.717554in}}%
\pgfpathcurveto{\pgfqpoint{3.303896in}{4.706504in}}{\pgfqpoint{3.308287in}{4.695905in}}{\pgfqpoint{3.316100in}{4.688091in}}%
\pgfpathcurveto{\pgfqpoint{3.323914in}{4.680278in}}{\pgfqpoint{3.334513in}{4.675887in}}{\pgfqpoint{3.345563in}{4.675887in}}%
\pgfpathclose%
\pgfusepath{stroke,fill}%
\end{pgfscope}%
\begin{pgfscope}%
\pgfpathrectangle{\pgfqpoint{0.481978in}{0.331635in}}{\pgfqpoint{9.300000in}{7.700000in}}%
\pgfusepath{clip}%
\pgfsetbuttcap%
\pgfsetroundjoin%
\definecolor{currentfill}{rgb}{0.631373,0.788235,0.956863}%
\pgfsetfillcolor{currentfill}%
\pgfsetlinewidth{0.481800pt}%
\definecolor{currentstroke}{rgb}{1.000000,1.000000,1.000000}%
\pgfsetstrokecolor{currentstroke}%
\pgfsetdash{}{0pt}%
\pgfpathmoveto{\pgfqpoint{5.829195in}{5.212750in}}%
\pgfpathcurveto{\pgfqpoint{5.840246in}{5.212750in}}{\pgfqpoint{5.850845in}{5.217140in}}{\pgfqpoint{5.858658in}{5.224954in}}%
\pgfpathcurveto{\pgfqpoint{5.866472in}{5.232767in}}{\pgfqpoint{5.870862in}{5.243367in}}{\pgfqpoint{5.870862in}{5.254417in}}%
\pgfpathcurveto{\pgfqpoint{5.870862in}{5.265467in}}{\pgfqpoint{5.866472in}{5.276066in}}{\pgfqpoint{5.858658in}{5.283879in}}%
\pgfpathcurveto{\pgfqpoint{5.850845in}{5.291693in}}{\pgfqpoint{5.840246in}{5.296083in}}{\pgfqpoint{5.829195in}{5.296083in}}%
\pgfpathcurveto{\pgfqpoint{5.818145in}{5.296083in}}{\pgfqpoint{5.807546in}{5.291693in}}{\pgfqpoint{5.799733in}{5.283879in}}%
\pgfpathcurveto{\pgfqpoint{5.791919in}{5.276066in}}{\pgfqpoint{5.787529in}{5.265467in}}{\pgfqpoint{5.787529in}{5.254417in}}%
\pgfpathcurveto{\pgfqpoint{5.787529in}{5.243367in}}{\pgfqpoint{5.791919in}{5.232767in}}{\pgfqpoint{5.799733in}{5.224954in}}%
\pgfpathcurveto{\pgfqpoint{5.807546in}{5.217140in}}{\pgfqpoint{5.818145in}{5.212750in}}{\pgfqpoint{5.829195in}{5.212750in}}%
\pgfpathclose%
\pgfusepath{stroke,fill}%
\end{pgfscope}%
\begin{pgfscope}%
\pgfpathrectangle{\pgfqpoint{0.481978in}{0.331635in}}{\pgfqpoint{9.300000in}{7.700000in}}%
\pgfusepath{clip}%
\pgfsetbuttcap%
\pgfsetroundjoin%
\definecolor{currentfill}{rgb}{0.631373,0.788235,0.956863}%
\pgfsetfillcolor{currentfill}%
\pgfsetlinewidth{0.481800pt}%
\definecolor{currentstroke}{rgb}{1.000000,1.000000,1.000000}%
\pgfsetstrokecolor{currentstroke}%
\pgfsetdash{}{0pt}%
\pgfpathmoveto{\pgfqpoint{1.266407in}{2.502917in}}%
\pgfpathcurveto{\pgfqpoint{1.277457in}{2.502917in}}{\pgfqpoint{1.288056in}{2.507308in}}{\pgfqpoint{1.295870in}{2.515121in}}%
\pgfpathcurveto{\pgfqpoint{1.303683in}{2.522935in}}{\pgfqpoint{1.308074in}{2.533534in}}{\pgfqpoint{1.308074in}{2.544584in}}%
\pgfpathcurveto{\pgfqpoint{1.308074in}{2.555634in}}{\pgfqpoint{1.303683in}{2.566233in}}{\pgfqpoint{1.295870in}{2.574047in}}%
\pgfpathcurveto{\pgfqpoint{1.288056in}{2.581861in}}{\pgfqpoint{1.277457in}{2.586251in}}{\pgfqpoint{1.266407in}{2.586251in}}%
\pgfpathcurveto{\pgfqpoint{1.255357in}{2.586251in}}{\pgfqpoint{1.244758in}{2.581861in}}{\pgfqpoint{1.236944in}{2.574047in}}%
\pgfpathcurveto{\pgfqpoint{1.229131in}{2.566233in}}{\pgfqpoint{1.224740in}{2.555634in}}{\pgfqpoint{1.224740in}{2.544584in}}%
\pgfpathcurveto{\pgfqpoint{1.224740in}{2.533534in}}{\pgfqpoint{1.229131in}{2.522935in}}{\pgfqpoint{1.236944in}{2.515121in}}%
\pgfpathcurveto{\pgfqpoint{1.244758in}{2.507308in}}{\pgfqpoint{1.255357in}{2.502917in}}{\pgfqpoint{1.266407in}{2.502917in}}%
\pgfpathclose%
\pgfusepath{stroke,fill}%
\end{pgfscope}%
\begin{pgfscope}%
\pgfpathrectangle{\pgfqpoint{0.481978in}{0.331635in}}{\pgfqpoint{9.300000in}{7.700000in}}%
\pgfusepath{clip}%
\pgfsetbuttcap%
\pgfsetroundjoin%
\definecolor{currentfill}{rgb}{0.631373,0.788235,0.956863}%
\pgfsetfillcolor{currentfill}%
\pgfsetlinewidth{0.481800pt}%
\definecolor{currentstroke}{rgb}{1.000000,1.000000,1.000000}%
\pgfsetstrokecolor{currentstroke}%
\pgfsetdash{}{0pt}%
\pgfpathmoveto{\pgfqpoint{1.280257in}{3.702244in}}%
\pgfpathcurveto{\pgfqpoint{1.291307in}{3.702244in}}{\pgfqpoint{1.301906in}{3.706634in}}{\pgfqpoint{1.309720in}{3.714448in}}%
\pgfpathcurveto{\pgfqpoint{1.317533in}{3.722261in}}{\pgfqpoint{1.321923in}{3.732860in}}{\pgfqpoint{1.321923in}{3.743911in}}%
\pgfpathcurveto{\pgfqpoint{1.321923in}{3.754961in}}{\pgfqpoint{1.317533in}{3.765560in}}{\pgfqpoint{1.309720in}{3.773373in}}%
\pgfpathcurveto{\pgfqpoint{1.301906in}{3.781187in}}{\pgfqpoint{1.291307in}{3.785577in}}{\pgfqpoint{1.280257in}{3.785577in}}%
\pgfpathcurveto{\pgfqpoint{1.269207in}{3.785577in}}{\pgfqpoint{1.258608in}{3.781187in}}{\pgfqpoint{1.250794in}{3.773373in}}%
\pgfpathcurveto{\pgfqpoint{1.242980in}{3.765560in}}{\pgfqpoint{1.238590in}{3.754961in}}{\pgfqpoint{1.238590in}{3.743911in}}%
\pgfpathcurveto{\pgfqpoint{1.238590in}{3.732860in}}{\pgfqpoint{1.242980in}{3.722261in}}{\pgfqpoint{1.250794in}{3.714448in}}%
\pgfpathcurveto{\pgfqpoint{1.258608in}{3.706634in}}{\pgfqpoint{1.269207in}{3.702244in}}{\pgfqpoint{1.280257in}{3.702244in}}%
\pgfpathclose%
\pgfusepath{stroke,fill}%
\end{pgfscope}%
\begin{pgfscope}%
\pgfpathrectangle{\pgfqpoint{0.481978in}{0.331635in}}{\pgfqpoint{9.300000in}{7.700000in}}%
\pgfusepath{clip}%
\pgfsetbuttcap%
\pgfsetroundjoin%
\definecolor{currentfill}{rgb}{0.631373,0.788235,0.956863}%
\pgfsetfillcolor{currentfill}%
\pgfsetlinewidth{0.481800pt}%
\definecolor{currentstroke}{rgb}{1.000000,1.000000,1.000000}%
\pgfsetstrokecolor{currentstroke}%
\pgfsetdash{}{0pt}%
\pgfpathmoveto{\pgfqpoint{1.366790in}{2.436232in}}%
\pgfpathcurveto{\pgfqpoint{1.377840in}{2.436232in}}{\pgfqpoint{1.388439in}{2.440622in}}{\pgfqpoint{1.396252in}{2.448436in}}%
\pgfpathcurveto{\pgfqpoint{1.404066in}{2.456249in}}{\pgfqpoint{1.408456in}{2.466848in}}{\pgfqpoint{1.408456in}{2.477899in}}%
\pgfpathcurveto{\pgfqpoint{1.408456in}{2.488949in}}{\pgfqpoint{1.404066in}{2.499548in}}{\pgfqpoint{1.396252in}{2.507361in}}%
\pgfpathcurveto{\pgfqpoint{1.388439in}{2.515175in}}{\pgfqpoint{1.377840in}{2.519565in}}{\pgfqpoint{1.366790in}{2.519565in}}%
\pgfpathcurveto{\pgfqpoint{1.355740in}{2.519565in}}{\pgfqpoint{1.345141in}{2.515175in}}{\pgfqpoint{1.337327in}{2.507361in}}%
\pgfpathcurveto{\pgfqpoint{1.329513in}{2.499548in}}{\pgfqpoint{1.325123in}{2.488949in}}{\pgfqpoint{1.325123in}{2.477899in}}%
\pgfpathcurveto{\pgfqpoint{1.325123in}{2.466848in}}{\pgfqpoint{1.329513in}{2.456249in}}{\pgfqpoint{1.337327in}{2.448436in}}%
\pgfpathcurveto{\pgfqpoint{1.345141in}{2.440622in}}{\pgfqpoint{1.355740in}{2.436232in}}{\pgfqpoint{1.366790in}{2.436232in}}%
\pgfpathclose%
\pgfusepath{stroke,fill}%
\end{pgfscope}%
\begin{pgfscope}%
\pgfpathrectangle{\pgfqpoint{0.481978in}{0.331635in}}{\pgfqpoint{9.300000in}{7.700000in}}%
\pgfusepath{clip}%
\pgfsetbuttcap%
\pgfsetroundjoin%
\definecolor{currentfill}{rgb}{0.631373,0.788235,0.956863}%
\pgfsetfillcolor{currentfill}%
\pgfsetlinewidth{0.481800pt}%
\definecolor{currentstroke}{rgb}{1.000000,1.000000,1.000000}%
\pgfsetstrokecolor{currentstroke}%
\pgfsetdash{}{0pt}%
\pgfpathmoveto{\pgfqpoint{5.131403in}{3.775556in}}%
\pgfpathcurveto{\pgfqpoint{5.142453in}{3.775556in}}{\pgfqpoint{5.153052in}{3.779946in}}{\pgfqpoint{5.160866in}{3.787760in}}%
\pgfpathcurveto{\pgfqpoint{5.168679in}{3.795573in}}{\pgfqpoint{5.173070in}{3.806172in}}{\pgfqpoint{5.173070in}{3.817222in}}%
\pgfpathcurveto{\pgfqpoint{5.173070in}{3.828273in}}{\pgfqpoint{5.168679in}{3.838872in}}{\pgfqpoint{5.160866in}{3.846685in}}%
\pgfpathcurveto{\pgfqpoint{5.153052in}{3.854499in}}{\pgfqpoint{5.142453in}{3.858889in}}{\pgfqpoint{5.131403in}{3.858889in}}%
\pgfpathcurveto{\pgfqpoint{5.120353in}{3.858889in}}{\pgfqpoint{5.109754in}{3.854499in}}{\pgfqpoint{5.101940in}{3.846685in}}%
\pgfpathcurveto{\pgfqpoint{5.094127in}{3.838872in}}{\pgfqpoint{5.089736in}{3.828273in}}{\pgfqpoint{5.089736in}{3.817222in}}%
\pgfpathcurveto{\pgfqpoint{5.089736in}{3.806172in}}{\pgfqpoint{5.094127in}{3.795573in}}{\pgfqpoint{5.101940in}{3.787760in}}%
\pgfpathcurveto{\pgfqpoint{5.109754in}{3.779946in}}{\pgfqpoint{5.120353in}{3.775556in}}{\pgfqpoint{5.131403in}{3.775556in}}%
\pgfpathclose%
\pgfusepath{stroke,fill}%
\end{pgfscope}%
\begin{pgfscope}%
\pgfpathrectangle{\pgfqpoint{0.481978in}{0.331635in}}{\pgfqpoint{9.300000in}{7.700000in}}%
\pgfusepath{clip}%
\pgfsetbuttcap%
\pgfsetroundjoin%
\definecolor{currentfill}{rgb}{0.631373,0.788235,0.956863}%
\pgfsetfillcolor{currentfill}%
\pgfsetlinewidth{0.481800pt}%
\definecolor{currentstroke}{rgb}{1.000000,1.000000,1.000000}%
\pgfsetstrokecolor{currentstroke}%
\pgfsetdash{}{0pt}%
\pgfpathmoveto{\pgfqpoint{6.077197in}{6.974685in}}%
\pgfpathcurveto{\pgfqpoint{6.088248in}{6.974685in}}{\pgfqpoint{6.098847in}{6.979075in}}{\pgfqpoint{6.106660in}{6.986889in}}%
\pgfpathcurveto{\pgfqpoint{6.114474in}{6.994702in}}{\pgfqpoint{6.118864in}{7.005301in}}{\pgfqpoint{6.118864in}{7.016351in}}%
\pgfpathcurveto{\pgfqpoint{6.118864in}{7.027402in}}{\pgfqpoint{6.114474in}{7.038001in}}{\pgfqpoint{6.106660in}{7.045814in}}%
\pgfpathcurveto{\pgfqpoint{6.098847in}{7.053628in}}{\pgfqpoint{6.088248in}{7.058018in}}{\pgfqpoint{6.077197in}{7.058018in}}%
\pgfpathcurveto{\pgfqpoint{6.066147in}{7.058018in}}{\pgfqpoint{6.055548in}{7.053628in}}{\pgfqpoint{6.047735in}{7.045814in}}%
\pgfpathcurveto{\pgfqpoint{6.039921in}{7.038001in}}{\pgfqpoint{6.035531in}{7.027402in}}{\pgfqpoint{6.035531in}{7.016351in}}%
\pgfpathcurveto{\pgfqpoint{6.035531in}{7.005301in}}{\pgfqpoint{6.039921in}{6.994702in}}{\pgfqpoint{6.047735in}{6.986889in}}%
\pgfpathcurveto{\pgfqpoint{6.055548in}{6.979075in}}{\pgfqpoint{6.066147in}{6.974685in}}{\pgfqpoint{6.077197in}{6.974685in}}%
\pgfpathclose%
\pgfusepath{stroke,fill}%
\end{pgfscope}%
\begin{pgfscope}%
\pgfpathrectangle{\pgfqpoint{0.481978in}{0.331635in}}{\pgfqpoint{9.300000in}{7.700000in}}%
\pgfusepath{clip}%
\pgfsetbuttcap%
\pgfsetroundjoin%
\definecolor{currentfill}{rgb}{0.631373,0.788235,0.956863}%
\pgfsetfillcolor{currentfill}%
\pgfsetlinewidth{0.481800pt}%
\definecolor{currentstroke}{rgb}{1.000000,1.000000,1.000000}%
\pgfsetstrokecolor{currentstroke}%
\pgfsetdash{}{0pt}%
\pgfpathmoveto{\pgfqpoint{5.267507in}{3.437471in}}%
\pgfpathcurveto{\pgfqpoint{5.278557in}{3.437471in}}{\pgfqpoint{5.289156in}{3.441861in}}{\pgfqpoint{5.296970in}{3.449674in}}%
\pgfpathcurveto{\pgfqpoint{5.304783in}{3.457488in}}{\pgfqpoint{5.309173in}{3.468087in}}{\pgfqpoint{5.309173in}{3.479137in}}%
\pgfpathcurveto{\pgfqpoint{5.309173in}{3.490187in}}{\pgfqpoint{5.304783in}{3.500786in}}{\pgfqpoint{5.296970in}{3.508600in}}%
\pgfpathcurveto{\pgfqpoint{5.289156in}{3.516414in}}{\pgfqpoint{5.278557in}{3.520804in}}{\pgfqpoint{5.267507in}{3.520804in}}%
\pgfpathcurveto{\pgfqpoint{5.256457in}{3.520804in}}{\pgfqpoint{5.245858in}{3.516414in}}{\pgfqpoint{5.238044in}{3.508600in}}%
\pgfpathcurveto{\pgfqpoint{5.230230in}{3.500786in}}{\pgfqpoint{5.225840in}{3.490187in}}{\pgfqpoint{5.225840in}{3.479137in}}%
\pgfpathcurveto{\pgfqpoint{5.225840in}{3.468087in}}{\pgfqpoint{5.230230in}{3.457488in}}{\pgfqpoint{5.238044in}{3.449674in}}%
\pgfpathcurveto{\pgfqpoint{5.245858in}{3.441861in}}{\pgfqpoint{5.256457in}{3.437471in}}{\pgfqpoint{5.267507in}{3.437471in}}%
\pgfpathclose%
\pgfusepath{stroke,fill}%
\end{pgfscope}%
\begin{pgfscope}%
\pgfpathrectangle{\pgfqpoint{0.481978in}{0.331635in}}{\pgfqpoint{9.300000in}{7.700000in}}%
\pgfusepath{clip}%
\pgfsetbuttcap%
\pgfsetroundjoin%
\definecolor{currentfill}{rgb}{0.631373,0.788235,0.956863}%
\pgfsetfillcolor{currentfill}%
\pgfsetlinewidth{0.481800pt}%
\definecolor{currentstroke}{rgb}{1.000000,1.000000,1.000000}%
\pgfsetstrokecolor{currentstroke}%
\pgfsetdash{}{0pt}%
\pgfpathmoveto{\pgfqpoint{5.422432in}{4.371502in}}%
\pgfpathcurveto{\pgfqpoint{5.433483in}{4.371502in}}{\pgfqpoint{5.444082in}{4.375892in}}{\pgfqpoint{5.451895in}{4.383705in}}%
\pgfpathcurveto{\pgfqpoint{5.459709in}{4.391519in}}{\pgfqpoint{5.464099in}{4.402118in}}{\pgfqpoint{5.464099in}{4.413168in}}%
\pgfpathcurveto{\pgfqpoint{5.464099in}{4.424218in}}{\pgfqpoint{5.459709in}{4.434817in}}{\pgfqpoint{5.451895in}{4.442631in}}%
\pgfpathcurveto{\pgfqpoint{5.444082in}{4.450445in}}{\pgfqpoint{5.433483in}{4.454835in}}{\pgfqpoint{5.422432in}{4.454835in}}%
\pgfpathcurveto{\pgfqpoint{5.411382in}{4.454835in}}{\pgfqpoint{5.400783in}{4.450445in}}{\pgfqpoint{5.392970in}{4.442631in}}%
\pgfpathcurveto{\pgfqpoint{5.385156in}{4.434817in}}{\pgfqpoint{5.380766in}{4.424218in}}{\pgfqpoint{5.380766in}{4.413168in}}%
\pgfpathcurveto{\pgfqpoint{5.380766in}{4.402118in}}{\pgfqpoint{5.385156in}{4.391519in}}{\pgfqpoint{5.392970in}{4.383705in}}%
\pgfpathcurveto{\pgfqpoint{5.400783in}{4.375892in}}{\pgfqpoint{5.411382in}{4.371502in}}{\pgfqpoint{5.422432in}{4.371502in}}%
\pgfpathclose%
\pgfusepath{stroke,fill}%
\end{pgfscope}%
\begin{pgfscope}%
\pgfpathrectangle{\pgfqpoint{0.481978in}{0.331635in}}{\pgfqpoint{9.300000in}{7.700000in}}%
\pgfusepath{clip}%
\pgfsetbuttcap%
\pgfsetroundjoin%
\definecolor{currentfill}{rgb}{0.631373,0.788235,0.956863}%
\pgfsetfillcolor{currentfill}%
\pgfsetlinewidth{0.481800pt}%
\definecolor{currentstroke}{rgb}{1.000000,1.000000,1.000000}%
\pgfsetstrokecolor{currentstroke}%
\pgfsetdash{}{0pt}%
\pgfpathmoveto{\pgfqpoint{4.576144in}{3.852406in}}%
\pgfpathcurveto{\pgfqpoint{4.587194in}{3.852406in}}{\pgfqpoint{4.597793in}{3.856796in}}{\pgfqpoint{4.605606in}{3.864609in}}%
\pgfpathcurveto{\pgfqpoint{4.613420in}{3.872423in}}{\pgfqpoint{4.617810in}{3.883022in}}{\pgfqpoint{4.617810in}{3.894072in}}%
\pgfpathcurveto{\pgfqpoint{4.617810in}{3.905122in}}{\pgfqpoint{4.613420in}{3.915721in}}{\pgfqpoint{4.605606in}{3.923535in}}%
\pgfpathcurveto{\pgfqpoint{4.597793in}{3.931349in}}{\pgfqpoint{4.587194in}{3.935739in}}{\pgfqpoint{4.576144in}{3.935739in}}%
\pgfpathcurveto{\pgfqpoint{4.565093in}{3.935739in}}{\pgfqpoint{4.554494in}{3.931349in}}{\pgfqpoint{4.546681in}{3.923535in}}%
\pgfpathcurveto{\pgfqpoint{4.538867in}{3.915721in}}{\pgfqpoint{4.534477in}{3.905122in}}{\pgfqpoint{4.534477in}{3.894072in}}%
\pgfpathcurveto{\pgfqpoint{4.534477in}{3.883022in}}{\pgfqpoint{4.538867in}{3.872423in}}{\pgfqpoint{4.546681in}{3.864609in}}%
\pgfpathcurveto{\pgfqpoint{4.554494in}{3.856796in}}{\pgfqpoint{4.565093in}{3.852406in}}{\pgfqpoint{4.576144in}{3.852406in}}%
\pgfpathclose%
\pgfusepath{stroke,fill}%
\end{pgfscope}%
\begin{pgfscope}%
\pgfpathrectangle{\pgfqpoint{0.481978in}{0.331635in}}{\pgfqpoint{9.300000in}{7.700000in}}%
\pgfusepath{clip}%
\pgfsetbuttcap%
\pgfsetroundjoin%
\definecolor{currentfill}{rgb}{0.631373,0.788235,0.956863}%
\pgfsetfillcolor{currentfill}%
\pgfsetlinewidth{0.481800pt}%
\definecolor{currentstroke}{rgb}{1.000000,1.000000,1.000000}%
\pgfsetstrokecolor{currentstroke}%
\pgfsetdash{}{0pt}%
\pgfpathmoveto{\pgfqpoint{2.109185in}{3.342395in}}%
\pgfpathcurveto{\pgfqpoint{2.120235in}{3.342395in}}{\pgfqpoint{2.130834in}{3.346786in}}{\pgfqpoint{2.138648in}{3.354599in}}%
\pgfpathcurveto{\pgfqpoint{2.146461in}{3.362413in}}{\pgfqpoint{2.150852in}{3.373012in}}{\pgfqpoint{2.150852in}{3.384062in}}%
\pgfpathcurveto{\pgfqpoint{2.150852in}{3.395112in}}{\pgfqpoint{2.146461in}{3.405711in}}{\pgfqpoint{2.138648in}{3.413525in}}%
\pgfpathcurveto{\pgfqpoint{2.130834in}{3.421338in}}{\pgfqpoint{2.120235in}{3.425729in}}{\pgfqpoint{2.109185in}{3.425729in}}%
\pgfpathcurveto{\pgfqpoint{2.098135in}{3.425729in}}{\pgfqpoint{2.087536in}{3.421338in}}{\pgfqpoint{2.079722in}{3.413525in}}%
\pgfpathcurveto{\pgfqpoint{2.071908in}{3.405711in}}{\pgfqpoint{2.067518in}{3.395112in}}{\pgfqpoint{2.067518in}{3.384062in}}%
\pgfpathcurveto{\pgfqpoint{2.067518in}{3.373012in}}{\pgfqpoint{2.071908in}{3.362413in}}{\pgfqpoint{2.079722in}{3.354599in}}%
\pgfpathcurveto{\pgfqpoint{2.087536in}{3.346786in}}{\pgfqpoint{2.098135in}{3.342395in}}{\pgfqpoint{2.109185in}{3.342395in}}%
\pgfpathclose%
\pgfusepath{stroke,fill}%
\end{pgfscope}%
\begin{pgfscope}%
\pgfpathrectangle{\pgfqpoint{0.481978in}{0.331635in}}{\pgfqpoint{9.300000in}{7.700000in}}%
\pgfusepath{clip}%
\pgfsetbuttcap%
\pgfsetroundjoin%
\definecolor{currentfill}{rgb}{0.631373,0.788235,0.956863}%
\pgfsetfillcolor{currentfill}%
\pgfsetlinewidth{0.481800pt}%
\definecolor{currentstroke}{rgb}{1.000000,1.000000,1.000000}%
\pgfsetstrokecolor{currentstroke}%
\pgfsetdash{}{0pt}%
\pgfpathmoveto{\pgfqpoint{3.045397in}{4.041120in}}%
\pgfpathcurveto{\pgfqpoint{3.056447in}{4.041120in}}{\pgfqpoint{3.067046in}{4.045510in}}{\pgfqpoint{3.074860in}{4.053324in}}%
\pgfpathcurveto{\pgfqpoint{3.082673in}{4.061137in}}{\pgfqpoint{3.087063in}{4.071736in}}{\pgfqpoint{3.087063in}{4.082787in}}%
\pgfpathcurveto{\pgfqpoint{3.087063in}{4.093837in}}{\pgfqpoint{3.082673in}{4.104436in}}{\pgfqpoint{3.074860in}{4.112249in}}%
\pgfpathcurveto{\pgfqpoint{3.067046in}{4.120063in}}{\pgfqpoint{3.056447in}{4.124453in}}{\pgfqpoint{3.045397in}{4.124453in}}%
\pgfpathcurveto{\pgfqpoint{3.034347in}{4.124453in}}{\pgfqpoint{3.023748in}{4.120063in}}{\pgfqpoint{3.015934in}{4.112249in}}%
\pgfpathcurveto{\pgfqpoint{3.008120in}{4.104436in}}{\pgfqpoint{3.003730in}{4.093837in}}{\pgfqpoint{3.003730in}{4.082787in}}%
\pgfpathcurveto{\pgfqpoint{3.003730in}{4.071736in}}{\pgfqpoint{3.008120in}{4.061137in}}{\pgfqpoint{3.015934in}{4.053324in}}%
\pgfpathcurveto{\pgfqpoint{3.023748in}{4.045510in}}{\pgfqpoint{3.034347in}{4.041120in}}{\pgfqpoint{3.045397in}{4.041120in}}%
\pgfpathclose%
\pgfusepath{stroke,fill}%
\end{pgfscope}%
\begin{pgfscope}%
\pgfpathrectangle{\pgfqpoint{0.481978in}{0.331635in}}{\pgfqpoint{9.300000in}{7.700000in}}%
\pgfusepath{clip}%
\pgfsetbuttcap%
\pgfsetroundjoin%
\definecolor{currentfill}{rgb}{0.631373,0.788235,0.956863}%
\pgfsetfillcolor{currentfill}%
\pgfsetlinewidth{0.481800pt}%
\definecolor{currentstroke}{rgb}{1.000000,1.000000,1.000000}%
\pgfsetstrokecolor{currentstroke}%
\pgfsetdash{}{0pt}%
\pgfpathmoveto{\pgfqpoint{2.937522in}{5.069906in}}%
\pgfpathcurveto{\pgfqpoint{2.948572in}{5.069906in}}{\pgfqpoint{2.959171in}{5.074297in}}{\pgfqpoint{2.966985in}{5.082110in}}%
\pgfpathcurveto{\pgfqpoint{2.974799in}{5.089924in}}{\pgfqpoint{2.979189in}{5.100523in}}{\pgfqpoint{2.979189in}{5.111573in}}%
\pgfpathcurveto{\pgfqpoint{2.979189in}{5.122623in}}{\pgfqpoint{2.974799in}{5.133222in}}{\pgfqpoint{2.966985in}{5.141036in}}%
\pgfpathcurveto{\pgfqpoint{2.959171in}{5.148849in}}{\pgfqpoint{2.948572in}{5.153240in}}{\pgfqpoint{2.937522in}{5.153240in}}%
\pgfpathcurveto{\pgfqpoint{2.926472in}{5.153240in}}{\pgfqpoint{2.915873in}{5.148849in}}{\pgfqpoint{2.908059in}{5.141036in}}%
\pgfpathcurveto{\pgfqpoint{2.900246in}{5.133222in}}{\pgfqpoint{2.895856in}{5.122623in}}{\pgfqpoint{2.895856in}{5.111573in}}%
\pgfpathcurveto{\pgfqpoint{2.895856in}{5.100523in}}{\pgfqpoint{2.900246in}{5.089924in}}{\pgfqpoint{2.908059in}{5.082110in}}%
\pgfpathcurveto{\pgfqpoint{2.915873in}{5.074297in}}{\pgfqpoint{2.926472in}{5.069906in}}{\pgfqpoint{2.937522in}{5.069906in}}%
\pgfpathclose%
\pgfusepath{stroke,fill}%
\end{pgfscope}%
\begin{pgfscope}%
\pgfpathrectangle{\pgfqpoint{0.481978in}{0.331635in}}{\pgfqpoint{9.300000in}{7.700000in}}%
\pgfusepath{clip}%
\pgfsetbuttcap%
\pgfsetroundjoin%
\definecolor{currentfill}{rgb}{0.631373,0.788235,0.956863}%
\pgfsetfillcolor{currentfill}%
\pgfsetlinewidth{0.481800pt}%
\definecolor{currentstroke}{rgb}{1.000000,1.000000,1.000000}%
\pgfsetstrokecolor{currentstroke}%
\pgfsetdash{}{0pt}%
\pgfpathmoveto{\pgfqpoint{2.204208in}{6.210084in}}%
\pgfpathcurveto{\pgfqpoint{2.215259in}{6.210084in}}{\pgfqpoint{2.225858in}{6.214475in}}{\pgfqpoint{2.233671in}{6.222288in}}%
\pgfpathcurveto{\pgfqpoint{2.241485in}{6.230102in}}{\pgfqpoint{2.245875in}{6.240701in}}{\pgfqpoint{2.245875in}{6.251751in}}%
\pgfpathcurveto{\pgfqpoint{2.245875in}{6.262801in}}{\pgfqpoint{2.241485in}{6.273400in}}{\pgfqpoint{2.233671in}{6.281214in}}%
\pgfpathcurveto{\pgfqpoint{2.225858in}{6.289027in}}{\pgfqpoint{2.215259in}{6.293418in}}{\pgfqpoint{2.204208in}{6.293418in}}%
\pgfpathcurveto{\pgfqpoint{2.193158in}{6.293418in}}{\pgfqpoint{2.182559in}{6.289027in}}{\pgfqpoint{2.174746in}{6.281214in}}%
\pgfpathcurveto{\pgfqpoint{2.166932in}{6.273400in}}{\pgfqpoint{2.162542in}{6.262801in}}{\pgfqpoint{2.162542in}{6.251751in}}%
\pgfpathcurveto{\pgfqpoint{2.162542in}{6.240701in}}{\pgfqpoint{2.166932in}{6.230102in}}{\pgfqpoint{2.174746in}{6.222288in}}%
\pgfpathcurveto{\pgfqpoint{2.182559in}{6.214475in}}{\pgfqpoint{2.193158in}{6.210084in}}{\pgfqpoint{2.204208in}{6.210084in}}%
\pgfpathclose%
\pgfusepath{stroke,fill}%
\end{pgfscope}%
\begin{pgfscope}%
\pgfpathrectangle{\pgfqpoint{0.481978in}{0.331635in}}{\pgfqpoint{9.300000in}{7.700000in}}%
\pgfusepath{clip}%
\pgfsetbuttcap%
\pgfsetroundjoin%
\definecolor{currentfill}{rgb}{0.631373,0.788235,0.956863}%
\pgfsetfillcolor{currentfill}%
\pgfsetlinewidth{0.481800pt}%
\definecolor{currentstroke}{rgb}{1.000000,1.000000,1.000000}%
\pgfsetstrokecolor{currentstroke}%
\pgfsetdash{}{0pt}%
\pgfpathmoveto{\pgfqpoint{3.780219in}{6.346116in}}%
\pgfpathcurveto{\pgfqpoint{3.791269in}{6.346116in}}{\pgfqpoint{3.801868in}{6.350506in}}{\pgfqpoint{3.809681in}{6.358320in}}%
\pgfpathcurveto{\pgfqpoint{3.817495in}{6.366133in}}{\pgfqpoint{3.821885in}{6.376732in}}{\pgfqpoint{3.821885in}{6.387783in}}%
\pgfpathcurveto{\pgfqpoint{3.821885in}{6.398833in}}{\pgfqpoint{3.817495in}{6.409432in}}{\pgfqpoint{3.809681in}{6.417245in}}%
\pgfpathcurveto{\pgfqpoint{3.801868in}{6.425059in}}{\pgfqpoint{3.791269in}{6.429449in}}{\pgfqpoint{3.780219in}{6.429449in}}%
\pgfpathcurveto{\pgfqpoint{3.769168in}{6.429449in}}{\pgfqpoint{3.758569in}{6.425059in}}{\pgfqpoint{3.750756in}{6.417245in}}%
\pgfpathcurveto{\pgfqpoint{3.742942in}{6.409432in}}{\pgfqpoint{3.738552in}{6.398833in}}{\pgfqpoint{3.738552in}{6.387783in}}%
\pgfpathcurveto{\pgfqpoint{3.738552in}{6.376732in}}{\pgfqpoint{3.742942in}{6.366133in}}{\pgfqpoint{3.750756in}{6.358320in}}%
\pgfpathcurveto{\pgfqpoint{3.758569in}{6.350506in}}{\pgfqpoint{3.769168in}{6.346116in}}{\pgfqpoint{3.780219in}{6.346116in}}%
\pgfpathclose%
\pgfusepath{stroke,fill}%
\end{pgfscope}%
\begin{pgfscope}%
\pgfpathrectangle{\pgfqpoint{0.481978in}{0.331635in}}{\pgfqpoint{9.300000in}{7.700000in}}%
\pgfusepath{clip}%
\pgfsetbuttcap%
\pgfsetroundjoin%
\definecolor{currentfill}{rgb}{0.631373,0.788235,0.956863}%
\pgfsetfillcolor{currentfill}%
\pgfsetlinewidth{0.481800pt}%
\definecolor{currentstroke}{rgb}{1.000000,1.000000,1.000000}%
\pgfsetstrokecolor{currentstroke}%
\pgfsetdash{}{0pt}%
\pgfpathmoveto{\pgfqpoint{4.814839in}{2.753029in}}%
\pgfpathcurveto{\pgfqpoint{4.825890in}{2.753029in}}{\pgfqpoint{4.836489in}{2.757419in}}{\pgfqpoint{4.844302in}{2.765233in}}%
\pgfpathcurveto{\pgfqpoint{4.852116in}{2.773046in}}{\pgfqpoint{4.856506in}{2.783646in}}{\pgfqpoint{4.856506in}{2.794696in}}%
\pgfpathcurveto{\pgfqpoint{4.856506in}{2.805746in}}{\pgfqpoint{4.852116in}{2.816345in}}{\pgfqpoint{4.844302in}{2.824158in}}%
\pgfpathcurveto{\pgfqpoint{4.836489in}{2.831972in}}{\pgfqpoint{4.825890in}{2.836362in}}{\pgfqpoint{4.814839in}{2.836362in}}%
\pgfpathcurveto{\pgfqpoint{4.803789in}{2.836362in}}{\pgfqpoint{4.793190in}{2.831972in}}{\pgfqpoint{4.785377in}{2.824158in}}%
\pgfpathcurveto{\pgfqpoint{4.777563in}{2.816345in}}{\pgfqpoint{4.773173in}{2.805746in}}{\pgfqpoint{4.773173in}{2.794696in}}%
\pgfpathcurveto{\pgfqpoint{4.773173in}{2.783646in}}{\pgfqpoint{4.777563in}{2.773046in}}{\pgfqpoint{4.785377in}{2.765233in}}%
\pgfpathcurveto{\pgfqpoint{4.793190in}{2.757419in}}{\pgfqpoint{4.803789in}{2.753029in}}{\pgfqpoint{4.814839in}{2.753029in}}%
\pgfpathclose%
\pgfusepath{stroke,fill}%
\end{pgfscope}%
\begin{pgfscope}%
\pgfpathrectangle{\pgfqpoint{0.481978in}{0.331635in}}{\pgfqpoint{9.300000in}{7.700000in}}%
\pgfusepath{clip}%
\pgfsetbuttcap%
\pgfsetroundjoin%
\definecolor{currentfill}{rgb}{0.631373,0.788235,0.956863}%
\pgfsetfillcolor{currentfill}%
\pgfsetlinewidth{0.481800pt}%
\definecolor{currentstroke}{rgb}{1.000000,1.000000,1.000000}%
\pgfsetstrokecolor{currentstroke}%
\pgfsetdash{}{0pt}%
\pgfpathmoveto{\pgfqpoint{2.753007in}{3.827926in}}%
\pgfpathcurveto{\pgfqpoint{2.764057in}{3.827926in}}{\pgfqpoint{2.774656in}{3.832316in}}{\pgfqpoint{2.782470in}{3.840130in}}%
\pgfpathcurveto{\pgfqpoint{2.790284in}{3.847944in}}{\pgfqpoint{2.794674in}{3.858543in}}{\pgfqpoint{2.794674in}{3.869593in}}%
\pgfpathcurveto{\pgfqpoint{2.794674in}{3.880643in}}{\pgfqpoint{2.790284in}{3.891242in}}{\pgfqpoint{2.782470in}{3.899056in}}%
\pgfpathcurveto{\pgfqpoint{2.774656in}{3.906869in}}{\pgfqpoint{2.764057in}{3.911259in}}{\pgfqpoint{2.753007in}{3.911259in}}%
\pgfpathcurveto{\pgfqpoint{2.741957in}{3.911259in}}{\pgfqpoint{2.731358in}{3.906869in}}{\pgfqpoint{2.723544in}{3.899056in}}%
\pgfpathcurveto{\pgfqpoint{2.715731in}{3.891242in}}{\pgfqpoint{2.711341in}{3.880643in}}{\pgfqpoint{2.711341in}{3.869593in}}%
\pgfpathcurveto{\pgfqpoint{2.711341in}{3.858543in}}{\pgfqpoint{2.715731in}{3.847944in}}{\pgfqpoint{2.723544in}{3.840130in}}%
\pgfpathcurveto{\pgfqpoint{2.731358in}{3.832316in}}{\pgfqpoint{2.741957in}{3.827926in}}{\pgfqpoint{2.753007in}{3.827926in}}%
\pgfpathclose%
\pgfusepath{stroke,fill}%
\end{pgfscope}%
\begin{pgfscope}%
\pgfpathrectangle{\pgfqpoint{0.481978in}{0.331635in}}{\pgfqpoint{9.300000in}{7.700000in}}%
\pgfusepath{clip}%
\pgfsetbuttcap%
\pgfsetroundjoin%
\definecolor{currentfill}{rgb}{0.631373,0.788235,0.956863}%
\pgfsetfillcolor{currentfill}%
\pgfsetlinewidth{0.481800pt}%
\definecolor{currentstroke}{rgb}{1.000000,1.000000,1.000000}%
\pgfsetstrokecolor{currentstroke}%
\pgfsetdash{}{0pt}%
\pgfpathmoveto{\pgfqpoint{2.663463in}{5.659614in}}%
\pgfpathcurveto{\pgfqpoint{2.674513in}{5.659614in}}{\pgfqpoint{2.685112in}{5.664004in}}{\pgfqpoint{2.692925in}{5.671818in}}%
\pgfpathcurveto{\pgfqpoint{2.700739in}{5.679631in}}{\pgfqpoint{2.705129in}{5.690230in}}{\pgfqpoint{2.705129in}{5.701280in}}%
\pgfpathcurveto{\pgfqpoint{2.705129in}{5.712330in}}{\pgfqpoint{2.700739in}{5.722929in}}{\pgfqpoint{2.692925in}{5.730743in}}%
\pgfpathcurveto{\pgfqpoint{2.685112in}{5.738557in}}{\pgfqpoint{2.674513in}{5.742947in}}{\pgfqpoint{2.663463in}{5.742947in}}%
\pgfpathcurveto{\pgfqpoint{2.652412in}{5.742947in}}{\pgfqpoint{2.641813in}{5.738557in}}{\pgfqpoint{2.634000in}{5.730743in}}%
\pgfpathcurveto{\pgfqpoint{2.626186in}{5.722929in}}{\pgfqpoint{2.621796in}{5.712330in}}{\pgfqpoint{2.621796in}{5.701280in}}%
\pgfpathcurveto{\pgfqpoint{2.621796in}{5.690230in}}{\pgfqpoint{2.626186in}{5.679631in}}{\pgfqpoint{2.634000in}{5.671818in}}%
\pgfpathcurveto{\pgfqpoint{2.641813in}{5.664004in}}{\pgfqpoint{2.652412in}{5.659614in}}{\pgfqpoint{2.663463in}{5.659614in}}%
\pgfpathclose%
\pgfusepath{stroke,fill}%
\end{pgfscope}%
\begin{pgfscope}%
\pgfpathrectangle{\pgfqpoint{0.481978in}{0.331635in}}{\pgfqpoint{9.300000in}{7.700000in}}%
\pgfusepath{clip}%
\pgfsetbuttcap%
\pgfsetroundjoin%
\definecolor{currentfill}{rgb}{0.631373,0.788235,0.956863}%
\pgfsetfillcolor{currentfill}%
\pgfsetlinewidth{0.481800pt}%
\definecolor{currentstroke}{rgb}{1.000000,1.000000,1.000000}%
\pgfsetstrokecolor{currentstroke}%
\pgfsetdash{}{0pt}%
\pgfpathmoveto{\pgfqpoint{6.114065in}{7.438633in}}%
\pgfpathcurveto{\pgfqpoint{6.125115in}{7.438633in}}{\pgfqpoint{6.135714in}{7.443023in}}{\pgfqpoint{6.143528in}{7.450837in}}%
\pgfpathcurveto{\pgfqpoint{6.151342in}{7.458650in}}{\pgfqpoint{6.155732in}{7.469249in}}{\pgfqpoint{6.155732in}{7.480299in}}%
\pgfpathcurveto{\pgfqpoint{6.155732in}{7.491349in}}{\pgfqpoint{6.151342in}{7.501949in}}{\pgfqpoint{6.143528in}{7.509762in}}%
\pgfpathcurveto{\pgfqpoint{6.135714in}{7.517576in}}{\pgfqpoint{6.125115in}{7.521966in}}{\pgfqpoint{6.114065in}{7.521966in}}%
\pgfpathcurveto{\pgfqpoint{6.103015in}{7.521966in}}{\pgfqpoint{6.092416in}{7.517576in}}{\pgfqpoint{6.084602in}{7.509762in}}%
\pgfpathcurveto{\pgfqpoint{6.076789in}{7.501949in}}{\pgfqpoint{6.072398in}{7.491349in}}{\pgfqpoint{6.072398in}{7.480299in}}%
\pgfpathcurveto{\pgfqpoint{6.072398in}{7.469249in}}{\pgfqpoint{6.076789in}{7.458650in}}{\pgfqpoint{6.084602in}{7.450837in}}%
\pgfpathcurveto{\pgfqpoint{6.092416in}{7.443023in}}{\pgfqpoint{6.103015in}{7.438633in}}{\pgfqpoint{6.114065in}{7.438633in}}%
\pgfpathclose%
\pgfusepath{stroke,fill}%
\end{pgfscope}%
\begin{pgfscope}%
\pgfpathrectangle{\pgfqpoint{0.481978in}{0.331635in}}{\pgfqpoint{9.300000in}{7.700000in}}%
\pgfusepath{clip}%
\pgfsetbuttcap%
\pgfsetroundjoin%
\definecolor{currentfill}{rgb}{1.000000,0.705882,0.509804}%
\pgfsetfillcolor{currentfill}%
\pgfsetlinewidth{0.481800pt}%
\definecolor{currentstroke}{rgb}{1.000000,1.000000,1.000000}%
\pgfsetstrokecolor{currentstroke}%
\pgfsetdash{}{0pt}%
\pgfpathmoveto{\pgfqpoint{3.154468in}{2.446661in}}%
\pgfpathcurveto{\pgfqpoint{3.165519in}{2.446661in}}{\pgfqpoint{3.176118in}{2.451052in}}{\pgfqpoint{3.183931in}{2.458865in}}%
\pgfpathcurveto{\pgfqpoint{3.191745in}{2.466679in}}{\pgfqpoint{3.196135in}{2.477278in}}{\pgfqpoint{3.196135in}{2.488328in}}%
\pgfpathcurveto{\pgfqpoint{3.196135in}{2.499378in}}{\pgfqpoint{3.191745in}{2.509977in}}{\pgfqpoint{3.183931in}{2.517791in}}%
\pgfpathcurveto{\pgfqpoint{3.176118in}{2.525604in}}{\pgfqpoint{3.165519in}{2.529995in}}{\pgfqpoint{3.154468in}{2.529995in}}%
\pgfpathcurveto{\pgfqpoint{3.143418in}{2.529995in}}{\pgfqpoint{3.132819in}{2.525604in}}{\pgfqpoint{3.125006in}{2.517791in}}%
\pgfpathcurveto{\pgfqpoint{3.117192in}{2.509977in}}{\pgfqpoint{3.112802in}{2.499378in}}{\pgfqpoint{3.112802in}{2.488328in}}%
\pgfpathcurveto{\pgfqpoint{3.112802in}{2.477278in}}{\pgfqpoint{3.117192in}{2.466679in}}{\pgfqpoint{3.125006in}{2.458865in}}%
\pgfpathcurveto{\pgfqpoint{3.132819in}{2.451052in}}{\pgfqpoint{3.143418in}{2.446661in}}{\pgfqpoint{3.154468in}{2.446661in}}%
\pgfpathclose%
\pgfusepath{stroke,fill}%
\end{pgfscope}%
\begin{pgfscope}%
\pgfpathrectangle{\pgfqpoint{0.481978in}{0.331635in}}{\pgfqpoint{9.300000in}{7.700000in}}%
\pgfusepath{clip}%
\pgfsetbuttcap%
\pgfsetroundjoin%
\definecolor{currentfill}{rgb}{1.000000,0.705882,0.509804}%
\pgfsetfillcolor{currentfill}%
\pgfsetlinewidth{0.481800pt}%
\definecolor{currentstroke}{rgb}{1.000000,1.000000,1.000000}%
\pgfsetstrokecolor{currentstroke}%
\pgfsetdash{}{0pt}%
\pgfpathmoveto{\pgfqpoint{1.545207in}{2.058337in}}%
\pgfpathcurveto{\pgfqpoint{1.556257in}{2.058337in}}{\pgfqpoint{1.566856in}{2.062727in}}{\pgfqpoint{1.574670in}{2.070541in}}%
\pgfpathcurveto{\pgfqpoint{1.582484in}{2.078355in}}{\pgfqpoint{1.586874in}{2.088954in}}{\pgfqpoint{1.586874in}{2.100004in}}%
\pgfpathcurveto{\pgfqpoint{1.586874in}{2.111054in}}{\pgfqpoint{1.582484in}{2.121653in}}{\pgfqpoint{1.574670in}{2.129467in}}%
\pgfpathcurveto{\pgfqpoint{1.566856in}{2.137280in}}{\pgfqpoint{1.556257in}{2.141671in}}{\pgfqpoint{1.545207in}{2.141671in}}%
\pgfpathcurveto{\pgfqpoint{1.534157in}{2.141671in}}{\pgfqpoint{1.523558in}{2.137280in}}{\pgfqpoint{1.515744in}{2.129467in}}%
\pgfpathcurveto{\pgfqpoint{1.507931in}{2.121653in}}{\pgfqpoint{1.503541in}{2.111054in}}{\pgfqpoint{1.503541in}{2.100004in}}%
\pgfpathcurveto{\pgfqpoint{1.503541in}{2.088954in}}{\pgfqpoint{1.507931in}{2.078355in}}{\pgfqpoint{1.515744in}{2.070541in}}%
\pgfpathcurveto{\pgfqpoint{1.523558in}{2.062727in}}{\pgfqpoint{1.534157in}{2.058337in}}{\pgfqpoint{1.545207in}{2.058337in}}%
\pgfpathclose%
\pgfusepath{stroke,fill}%
\end{pgfscope}%
\begin{pgfscope}%
\pgfpathrectangle{\pgfqpoint{0.481978in}{0.331635in}}{\pgfqpoint{9.300000in}{7.700000in}}%
\pgfusepath{clip}%
\pgfsetbuttcap%
\pgfsetroundjoin%
\definecolor{currentfill}{rgb}{1.000000,0.705882,0.509804}%
\pgfsetfillcolor{currentfill}%
\pgfsetlinewidth{0.481800pt}%
\definecolor{currentstroke}{rgb}{1.000000,1.000000,1.000000}%
\pgfsetstrokecolor{currentstroke}%
\pgfsetdash{}{0pt}%
\pgfpathmoveto{\pgfqpoint{3.307338in}{3.365821in}}%
\pgfpathcurveto{\pgfqpoint{3.318388in}{3.365821in}}{\pgfqpoint{3.328987in}{3.370211in}}{\pgfqpoint{3.336801in}{3.378025in}}%
\pgfpathcurveto{\pgfqpoint{3.344614in}{3.385838in}}{\pgfqpoint{3.349005in}{3.396437in}}{\pgfqpoint{3.349005in}{3.407488in}}%
\pgfpathcurveto{\pgfqpoint{3.349005in}{3.418538in}}{\pgfqpoint{3.344614in}{3.429137in}}{\pgfqpoint{3.336801in}{3.436950in}}%
\pgfpathcurveto{\pgfqpoint{3.328987in}{3.444764in}}{\pgfqpoint{3.318388in}{3.449154in}}{\pgfqpoint{3.307338in}{3.449154in}}%
\pgfpathcurveto{\pgfqpoint{3.296288in}{3.449154in}}{\pgfqpoint{3.285689in}{3.444764in}}{\pgfqpoint{3.277875in}{3.436950in}}%
\pgfpathcurveto{\pgfqpoint{3.270062in}{3.429137in}}{\pgfqpoint{3.265671in}{3.418538in}}{\pgfqpoint{3.265671in}{3.407488in}}%
\pgfpathcurveto{\pgfqpoint{3.265671in}{3.396437in}}{\pgfqpoint{3.270062in}{3.385838in}}{\pgfqpoint{3.277875in}{3.378025in}}%
\pgfpathcurveto{\pgfqpoint{3.285689in}{3.370211in}}{\pgfqpoint{3.296288in}{3.365821in}}{\pgfqpoint{3.307338in}{3.365821in}}%
\pgfpathclose%
\pgfusepath{stroke,fill}%
\end{pgfscope}%
\begin{pgfscope}%
\pgfpathrectangle{\pgfqpoint{0.481978in}{0.331635in}}{\pgfqpoint{9.300000in}{7.700000in}}%
\pgfusepath{clip}%
\pgfsetbuttcap%
\pgfsetroundjoin%
\definecolor{currentfill}{rgb}{1.000000,0.705882,0.509804}%
\pgfsetfillcolor{currentfill}%
\pgfsetlinewidth{0.481800pt}%
\definecolor{currentstroke}{rgb}{1.000000,1.000000,1.000000}%
\pgfsetstrokecolor{currentstroke}%
\pgfsetdash{}{0pt}%
\pgfpathmoveto{\pgfqpoint{5.143128in}{7.146530in}}%
\pgfpathcurveto{\pgfqpoint{5.154178in}{7.146530in}}{\pgfqpoint{5.164777in}{7.150920in}}{\pgfqpoint{5.172591in}{7.158734in}}%
\pgfpathcurveto{\pgfqpoint{5.180405in}{7.166547in}}{\pgfqpoint{5.184795in}{7.177146in}}{\pgfqpoint{5.184795in}{7.188196in}}%
\pgfpathcurveto{\pgfqpoint{5.184795in}{7.199247in}}{\pgfqpoint{5.180405in}{7.209846in}}{\pgfqpoint{5.172591in}{7.217659in}}%
\pgfpathcurveto{\pgfqpoint{5.164777in}{7.225473in}}{\pgfqpoint{5.154178in}{7.229863in}}{\pgfqpoint{5.143128in}{7.229863in}}%
\pgfpathcurveto{\pgfqpoint{5.132078in}{7.229863in}}{\pgfqpoint{5.121479in}{7.225473in}}{\pgfqpoint{5.113665in}{7.217659in}}%
\pgfpathcurveto{\pgfqpoint{5.105852in}{7.209846in}}{\pgfqpoint{5.101461in}{7.199247in}}{\pgfqpoint{5.101461in}{7.188196in}}%
\pgfpathcurveto{\pgfqpoint{5.101461in}{7.177146in}}{\pgfqpoint{5.105852in}{7.166547in}}{\pgfqpoint{5.113665in}{7.158734in}}%
\pgfpathcurveto{\pgfqpoint{5.121479in}{7.150920in}}{\pgfqpoint{5.132078in}{7.146530in}}{\pgfqpoint{5.143128in}{7.146530in}}%
\pgfpathclose%
\pgfusepath{stroke,fill}%
\end{pgfscope}%
\begin{pgfscope}%
\pgfpathrectangle{\pgfqpoint{0.481978in}{0.331635in}}{\pgfqpoint{9.300000in}{7.700000in}}%
\pgfusepath{clip}%
\pgfsetbuttcap%
\pgfsetroundjoin%
\definecolor{currentfill}{rgb}{1.000000,0.705882,0.509804}%
\pgfsetfillcolor{currentfill}%
\pgfsetlinewidth{0.481800pt}%
\definecolor{currentstroke}{rgb}{1.000000,1.000000,1.000000}%
\pgfsetstrokecolor{currentstroke}%
\pgfsetdash{}{0pt}%
\pgfpathmoveto{\pgfqpoint{3.238886in}{1.903284in}}%
\pgfpathcurveto{\pgfqpoint{3.249936in}{1.903284in}}{\pgfqpoint{3.260535in}{1.907675in}}{\pgfqpoint{3.268349in}{1.915488in}}%
\pgfpathcurveto{\pgfqpoint{3.276162in}{1.923302in}}{\pgfqpoint{3.280553in}{1.933901in}}{\pgfqpoint{3.280553in}{1.944951in}}%
\pgfpathcurveto{\pgfqpoint{3.280553in}{1.956001in}}{\pgfqpoint{3.276162in}{1.966600in}}{\pgfqpoint{3.268349in}{1.974414in}}%
\pgfpathcurveto{\pgfqpoint{3.260535in}{1.982227in}}{\pgfqpoint{3.249936in}{1.986618in}}{\pgfqpoint{3.238886in}{1.986618in}}%
\pgfpathcurveto{\pgfqpoint{3.227836in}{1.986618in}}{\pgfqpoint{3.217237in}{1.982227in}}{\pgfqpoint{3.209423in}{1.974414in}}%
\pgfpathcurveto{\pgfqpoint{3.201610in}{1.966600in}}{\pgfqpoint{3.197219in}{1.956001in}}{\pgfqpoint{3.197219in}{1.944951in}}%
\pgfpathcurveto{\pgfqpoint{3.197219in}{1.933901in}}{\pgfqpoint{3.201610in}{1.923302in}}{\pgfqpoint{3.209423in}{1.915488in}}%
\pgfpathcurveto{\pgfqpoint{3.217237in}{1.907675in}}{\pgfqpoint{3.227836in}{1.903284in}}{\pgfqpoint{3.238886in}{1.903284in}}%
\pgfpathclose%
\pgfusepath{stroke,fill}%
\end{pgfscope}%
\begin{pgfscope}%
\pgfpathrectangle{\pgfqpoint{0.481978in}{0.331635in}}{\pgfqpoint{9.300000in}{7.700000in}}%
\pgfusepath{clip}%
\pgfsetbuttcap%
\pgfsetroundjoin%
\definecolor{currentfill}{rgb}{1.000000,0.705882,0.509804}%
\pgfsetfillcolor{currentfill}%
\pgfsetlinewidth{0.481800pt}%
\definecolor{currentstroke}{rgb}{1.000000,1.000000,1.000000}%
\pgfsetstrokecolor{currentstroke}%
\pgfsetdash{}{0pt}%
\pgfpathmoveto{\pgfqpoint{6.096137in}{1.233341in}}%
\pgfpathcurveto{\pgfqpoint{6.107187in}{1.233341in}}{\pgfqpoint{6.117786in}{1.237731in}}{\pgfqpoint{6.125600in}{1.245545in}}%
\pgfpathcurveto{\pgfqpoint{6.133413in}{1.253359in}}{\pgfqpoint{6.137804in}{1.263958in}}{\pgfqpoint{6.137804in}{1.275008in}}%
\pgfpathcurveto{\pgfqpoint{6.137804in}{1.286058in}}{\pgfqpoint{6.133413in}{1.296657in}}{\pgfqpoint{6.125600in}{1.304471in}}%
\pgfpathcurveto{\pgfqpoint{6.117786in}{1.312284in}}{\pgfqpoint{6.107187in}{1.316675in}}{\pgfqpoint{6.096137in}{1.316675in}}%
\pgfpathcurveto{\pgfqpoint{6.085087in}{1.316675in}}{\pgfqpoint{6.074488in}{1.312284in}}{\pgfqpoint{6.066674in}{1.304471in}}%
\pgfpathcurveto{\pgfqpoint{6.058861in}{1.296657in}}{\pgfqpoint{6.054470in}{1.286058in}}{\pgfqpoint{6.054470in}{1.275008in}}%
\pgfpathcurveto{\pgfqpoint{6.054470in}{1.263958in}}{\pgfqpoint{6.058861in}{1.253359in}}{\pgfqpoint{6.066674in}{1.245545in}}%
\pgfpathcurveto{\pgfqpoint{6.074488in}{1.237731in}}{\pgfqpoint{6.085087in}{1.233341in}}{\pgfqpoint{6.096137in}{1.233341in}}%
\pgfpathclose%
\pgfusepath{stroke,fill}%
\end{pgfscope}%
\begin{pgfscope}%
\pgfpathrectangle{\pgfqpoint{0.481978in}{0.331635in}}{\pgfqpoint{9.300000in}{7.700000in}}%
\pgfusepath{clip}%
\pgfsetbuttcap%
\pgfsetroundjoin%
\definecolor{currentfill}{rgb}{1.000000,0.705882,0.509804}%
\pgfsetfillcolor{currentfill}%
\pgfsetlinewidth{0.481800pt}%
\definecolor{currentstroke}{rgb}{1.000000,1.000000,1.000000}%
\pgfsetstrokecolor{currentstroke}%
\pgfsetdash{}{0pt}%
\pgfpathmoveto{\pgfqpoint{8.892538in}{6.000024in}}%
\pgfpathcurveto{\pgfqpoint{8.903589in}{6.000024in}}{\pgfqpoint{8.914188in}{6.004414in}}{\pgfqpoint{8.922001in}{6.012228in}}%
\pgfpathcurveto{\pgfqpoint{8.929815in}{6.020041in}}{\pgfqpoint{8.934205in}{6.030640in}}{\pgfqpoint{8.934205in}{6.041691in}}%
\pgfpathcurveto{\pgfqpoint{8.934205in}{6.052741in}}{\pgfqpoint{8.929815in}{6.063340in}}{\pgfqpoint{8.922001in}{6.071153in}}%
\pgfpathcurveto{\pgfqpoint{8.914188in}{6.078967in}}{\pgfqpoint{8.903589in}{6.083357in}}{\pgfqpoint{8.892538in}{6.083357in}}%
\pgfpathcurveto{\pgfqpoint{8.881488in}{6.083357in}}{\pgfqpoint{8.870889in}{6.078967in}}{\pgfqpoint{8.863076in}{6.071153in}}%
\pgfpathcurveto{\pgfqpoint{8.855262in}{6.063340in}}{\pgfqpoint{8.850872in}{6.052741in}}{\pgfqpoint{8.850872in}{6.041691in}}%
\pgfpathcurveto{\pgfqpoint{8.850872in}{6.030640in}}{\pgfqpoint{8.855262in}{6.020041in}}{\pgfqpoint{8.863076in}{6.012228in}}%
\pgfpathcurveto{\pgfqpoint{8.870889in}{6.004414in}}{\pgfqpoint{8.881488in}{6.000024in}}{\pgfqpoint{8.892538in}{6.000024in}}%
\pgfpathclose%
\pgfusepath{stroke,fill}%
\end{pgfscope}%
\begin{pgfscope}%
\pgfpathrectangle{\pgfqpoint{0.481978in}{0.331635in}}{\pgfqpoint{9.300000in}{7.700000in}}%
\pgfusepath{clip}%
\pgfsetbuttcap%
\pgfsetroundjoin%
\definecolor{currentfill}{rgb}{1.000000,0.705882,0.509804}%
\pgfsetfillcolor{currentfill}%
\pgfsetlinewidth{0.481800pt}%
\definecolor{currentstroke}{rgb}{1.000000,1.000000,1.000000}%
\pgfsetstrokecolor{currentstroke}%
\pgfsetdash{}{0pt}%
\pgfpathmoveto{\pgfqpoint{8.792161in}{5.432877in}}%
\pgfpathcurveto{\pgfqpoint{8.803212in}{5.432877in}}{\pgfqpoint{8.813811in}{5.437267in}}{\pgfqpoint{8.821624in}{5.445081in}}%
\pgfpathcurveto{\pgfqpoint{8.829438in}{5.452895in}}{\pgfqpoint{8.833828in}{5.463494in}}{\pgfqpoint{8.833828in}{5.474544in}}%
\pgfpathcurveto{\pgfqpoint{8.833828in}{5.485594in}}{\pgfqpoint{8.829438in}{5.496193in}}{\pgfqpoint{8.821624in}{5.504006in}}%
\pgfpathcurveto{\pgfqpoint{8.813811in}{5.511820in}}{\pgfqpoint{8.803212in}{5.516210in}}{\pgfqpoint{8.792161in}{5.516210in}}%
\pgfpathcurveto{\pgfqpoint{8.781111in}{5.516210in}}{\pgfqpoint{8.770512in}{5.511820in}}{\pgfqpoint{8.762699in}{5.504006in}}%
\pgfpathcurveto{\pgfqpoint{8.754885in}{5.496193in}}{\pgfqpoint{8.750495in}{5.485594in}}{\pgfqpoint{8.750495in}{5.474544in}}%
\pgfpathcurveto{\pgfqpoint{8.750495in}{5.463494in}}{\pgfqpoint{8.754885in}{5.452895in}}{\pgfqpoint{8.762699in}{5.445081in}}%
\pgfpathcurveto{\pgfqpoint{8.770512in}{5.437267in}}{\pgfqpoint{8.781111in}{5.432877in}}{\pgfqpoint{8.792161in}{5.432877in}}%
\pgfpathclose%
\pgfusepath{stroke,fill}%
\end{pgfscope}%
\begin{pgfscope}%
\pgfpathrectangle{\pgfqpoint{0.481978in}{0.331635in}}{\pgfqpoint{9.300000in}{7.700000in}}%
\pgfusepath{clip}%
\pgfsetbuttcap%
\pgfsetroundjoin%
\definecolor{currentfill}{rgb}{1.000000,0.705882,0.509804}%
\pgfsetfillcolor{currentfill}%
\pgfsetlinewidth{0.481800pt}%
\definecolor{currentstroke}{rgb}{1.000000,1.000000,1.000000}%
\pgfsetstrokecolor{currentstroke}%
\pgfsetdash{}{0pt}%
\pgfpathmoveto{\pgfqpoint{4.884502in}{4.767853in}}%
\pgfpathcurveto{\pgfqpoint{4.895552in}{4.767853in}}{\pgfqpoint{4.906151in}{4.772243in}}{\pgfqpoint{4.913965in}{4.780057in}}%
\pgfpathcurveto{\pgfqpoint{4.921778in}{4.787870in}}{\pgfqpoint{4.926169in}{4.798469in}}{\pgfqpoint{4.926169in}{4.809519in}}%
\pgfpathcurveto{\pgfqpoint{4.926169in}{4.820570in}}{\pgfqpoint{4.921778in}{4.831169in}}{\pgfqpoint{4.913965in}{4.838982in}}%
\pgfpathcurveto{\pgfqpoint{4.906151in}{4.846796in}}{\pgfqpoint{4.895552in}{4.851186in}}{\pgfqpoint{4.884502in}{4.851186in}}%
\pgfpathcurveto{\pgfqpoint{4.873452in}{4.851186in}}{\pgfqpoint{4.862853in}{4.846796in}}{\pgfqpoint{4.855039in}{4.838982in}}%
\pgfpathcurveto{\pgfqpoint{4.847226in}{4.831169in}}{\pgfqpoint{4.842835in}{4.820570in}}{\pgfqpoint{4.842835in}{4.809519in}}%
\pgfpathcurveto{\pgfqpoint{4.842835in}{4.798469in}}{\pgfqpoint{4.847226in}{4.787870in}}{\pgfqpoint{4.855039in}{4.780057in}}%
\pgfpathcurveto{\pgfqpoint{4.862853in}{4.772243in}}{\pgfqpoint{4.873452in}{4.767853in}}{\pgfqpoint{4.884502in}{4.767853in}}%
\pgfpathclose%
\pgfusepath{stroke,fill}%
\end{pgfscope}%
\begin{pgfscope}%
\pgfpathrectangle{\pgfqpoint{0.481978in}{0.331635in}}{\pgfqpoint{9.300000in}{7.700000in}}%
\pgfusepath{clip}%
\pgfsetbuttcap%
\pgfsetroundjoin%
\definecolor{currentfill}{rgb}{1.000000,0.705882,0.509804}%
\pgfsetfillcolor{currentfill}%
\pgfsetlinewidth{0.481800pt}%
\definecolor{currentstroke}{rgb}{1.000000,1.000000,1.000000}%
\pgfsetstrokecolor{currentstroke}%
\pgfsetdash{}{0pt}%
\pgfpathmoveto{\pgfqpoint{8.839374in}{4.799422in}}%
\pgfpathcurveto{\pgfqpoint{8.850424in}{4.799422in}}{\pgfqpoint{8.861023in}{4.803812in}}{\pgfqpoint{8.868836in}{4.811626in}}%
\pgfpathcurveto{\pgfqpoint{8.876650in}{4.819439in}}{\pgfqpoint{8.881040in}{4.830038in}}{\pgfqpoint{8.881040in}{4.841088in}}%
\pgfpathcurveto{\pgfqpoint{8.881040in}{4.852138in}}{\pgfqpoint{8.876650in}{4.862737in}}{\pgfqpoint{8.868836in}{4.870551in}}%
\pgfpathcurveto{\pgfqpoint{8.861023in}{4.878365in}}{\pgfqpoint{8.850424in}{4.882755in}}{\pgfqpoint{8.839374in}{4.882755in}}%
\pgfpathcurveto{\pgfqpoint{8.828324in}{4.882755in}}{\pgfqpoint{8.817724in}{4.878365in}}{\pgfqpoint{8.809911in}{4.870551in}}%
\pgfpathcurveto{\pgfqpoint{8.802097in}{4.862737in}}{\pgfqpoint{8.797707in}{4.852138in}}{\pgfqpoint{8.797707in}{4.841088in}}%
\pgfpathcurveto{\pgfqpoint{8.797707in}{4.830038in}}{\pgfqpoint{8.802097in}{4.819439in}}{\pgfqpoint{8.809911in}{4.811626in}}%
\pgfpathcurveto{\pgfqpoint{8.817724in}{4.803812in}}{\pgfqpoint{8.828324in}{4.799422in}}{\pgfqpoint{8.839374in}{4.799422in}}%
\pgfpathclose%
\pgfusepath{stroke,fill}%
\end{pgfscope}%
\begin{pgfscope}%
\pgfpathrectangle{\pgfqpoint{0.481978in}{0.331635in}}{\pgfqpoint{9.300000in}{7.700000in}}%
\pgfusepath{clip}%
\pgfsetbuttcap%
\pgfsetroundjoin%
\definecolor{currentfill}{rgb}{1.000000,0.705882,0.509804}%
\pgfsetfillcolor{currentfill}%
\pgfsetlinewidth{0.481800pt}%
\definecolor{currentstroke}{rgb}{1.000000,1.000000,1.000000}%
\pgfsetstrokecolor{currentstroke}%
\pgfsetdash{}{0pt}%
\pgfpathmoveto{\pgfqpoint{8.645625in}{5.963537in}}%
\pgfpathcurveto{\pgfqpoint{8.656675in}{5.963537in}}{\pgfqpoint{8.667274in}{5.967928in}}{\pgfqpoint{8.675088in}{5.975741in}}%
\pgfpathcurveto{\pgfqpoint{8.682902in}{5.983555in}}{\pgfqpoint{8.687292in}{5.994154in}}{\pgfqpoint{8.687292in}{6.005204in}}%
\pgfpathcurveto{\pgfqpoint{8.687292in}{6.016254in}}{\pgfqpoint{8.682902in}{6.026853in}}{\pgfqpoint{8.675088in}{6.034667in}}%
\pgfpathcurveto{\pgfqpoint{8.667274in}{6.042480in}}{\pgfqpoint{8.656675in}{6.046871in}}{\pgfqpoint{8.645625in}{6.046871in}}%
\pgfpathcurveto{\pgfqpoint{8.634575in}{6.046871in}}{\pgfqpoint{8.623976in}{6.042480in}}{\pgfqpoint{8.616162in}{6.034667in}}%
\pgfpathcurveto{\pgfqpoint{8.608349in}{6.026853in}}{\pgfqpoint{8.603959in}{6.016254in}}{\pgfqpoint{8.603959in}{6.005204in}}%
\pgfpathcurveto{\pgfqpoint{8.603959in}{5.994154in}}{\pgfqpoint{8.608349in}{5.983555in}}{\pgfqpoint{8.616162in}{5.975741in}}%
\pgfpathcurveto{\pgfqpoint{8.623976in}{5.967928in}}{\pgfqpoint{8.634575in}{5.963537in}}{\pgfqpoint{8.645625in}{5.963537in}}%
\pgfpathclose%
\pgfusepath{stroke,fill}%
\end{pgfscope}%
\begin{pgfscope}%
\pgfpathrectangle{\pgfqpoint{0.481978in}{0.331635in}}{\pgfqpoint{9.300000in}{7.700000in}}%
\pgfusepath{clip}%
\pgfsetbuttcap%
\pgfsetroundjoin%
\definecolor{currentfill}{rgb}{1.000000,0.705882,0.509804}%
\pgfsetfillcolor{currentfill}%
\pgfsetlinewidth{0.481800pt}%
\definecolor{currentstroke}{rgb}{1.000000,1.000000,1.000000}%
\pgfsetstrokecolor{currentstroke}%
\pgfsetdash{}{0pt}%
\pgfpathmoveto{\pgfqpoint{1.541644in}{2.029501in}}%
\pgfpathcurveto{\pgfqpoint{1.552694in}{2.029501in}}{\pgfqpoint{1.563293in}{2.033891in}}{\pgfqpoint{1.571107in}{2.041705in}}%
\pgfpathcurveto{\pgfqpoint{1.578921in}{2.049518in}}{\pgfqpoint{1.583311in}{2.060117in}}{\pgfqpoint{1.583311in}{2.071167in}}%
\pgfpathcurveto{\pgfqpoint{1.583311in}{2.082218in}}{\pgfqpoint{1.578921in}{2.092817in}}{\pgfqpoint{1.571107in}{2.100630in}}%
\pgfpathcurveto{\pgfqpoint{1.563293in}{2.108444in}}{\pgfqpoint{1.552694in}{2.112834in}}{\pgfqpoint{1.541644in}{2.112834in}}%
\pgfpathcurveto{\pgfqpoint{1.530594in}{2.112834in}}{\pgfqpoint{1.519995in}{2.108444in}}{\pgfqpoint{1.512181in}{2.100630in}}%
\pgfpathcurveto{\pgfqpoint{1.504368in}{2.092817in}}{\pgfqpoint{1.499978in}{2.082218in}}{\pgfqpoint{1.499978in}{2.071167in}}%
\pgfpathcurveto{\pgfqpoint{1.499978in}{2.060117in}}{\pgfqpoint{1.504368in}{2.049518in}}{\pgfqpoint{1.512181in}{2.041705in}}%
\pgfpathcurveto{\pgfqpoint{1.519995in}{2.033891in}}{\pgfqpoint{1.530594in}{2.029501in}}{\pgfqpoint{1.541644in}{2.029501in}}%
\pgfpathclose%
\pgfusepath{stroke,fill}%
\end{pgfscope}%
\begin{pgfscope}%
\pgfpathrectangle{\pgfqpoint{0.481978in}{0.331635in}}{\pgfqpoint{9.300000in}{7.700000in}}%
\pgfusepath{clip}%
\pgfsetbuttcap%
\pgfsetroundjoin%
\definecolor{currentfill}{rgb}{1.000000,0.705882,0.509804}%
\pgfsetfillcolor{currentfill}%
\pgfsetlinewidth{0.481800pt}%
\definecolor{currentstroke}{rgb}{1.000000,1.000000,1.000000}%
\pgfsetstrokecolor{currentstroke}%
\pgfsetdash{}{0pt}%
\pgfpathmoveto{\pgfqpoint{8.938916in}{5.368764in}}%
\pgfpathcurveto{\pgfqpoint{8.949966in}{5.368764in}}{\pgfqpoint{8.960565in}{5.373154in}}{\pgfqpoint{8.968379in}{5.380967in}}%
\pgfpathcurveto{\pgfqpoint{8.976193in}{5.388781in}}{\pgfqpoint{8.980583in}{5.399380in}}{\pgfqpoint{8.980583in}{5.410430in}}%
\pgfpathcurveto{\pgfqpoint{8.980583in}{5.421480in}}{\pgfqpoint{8.976193in}{5.432079in}}{\pgfqpoint{8.968379in}{5.439893in}}%
\pgfpathcurveto{\pgfqpoint{8.960565in}{5.447707in}}{\pgfqpoint{8.949966in}{5.452097in}}{\pgfqpoint{8.938916in}{5.452097in}}%
\pgfpathcurveto{\pgfqpoint{8.927866in}{5.452097in}}{\pgfqpoint{8.917267in}{5.447707in}}{\pgfqpoint{8.909453in}{5.439893in}}%
\pgfpathcurveto{\pgfqpoint{8.901640in}{5.432079in}}{\pgfqpoint{8.897250in}{5.421480in}}{\pgfqpoint{8.897250in}{5.410430in}}%
\pgfpathcurveto{\pgfqpoint{8.897250in}{5.399380in}}{\pgfqpoint{8.901640in}{5.388781in}}{\pgfqpoint{8.909453in}{5.380967in}}%
\pgfpathcurveto{\pgfqpoint{8.917267in}{5.373154in}}{\pgfqpoint{8.927866in}{5.368764in}}{\pgfqpoint{8.938916in}{5.368764in}}%
\pgfpathclose%
\pgfusepath{stroke,fill}%
\end{pgfscope}%
\begin{pgfscope}%
\pgfpathrectangle{\pgfqpoint{0.481978in}{0.331635in}}{\pgfqpoint{9.300000in}{7.700000in}}%
\pgfusepath{clip}%
\pgfsetbuttcap%
\pgfsetroundjoin%
\definecolor{currentfill}{rgb}{1.000000,0.705882,0.509804}%
\pgfsetfillcolor{currentfill}%
\pgfsetlinewidth{0.481800pt}%
\definecolor{currentstroke}{rgb}{1.000000,1.000000,1.000000}%
\pgfsetstrokecolor{currentstroke}%
\pgfsetdash{}{0pt}%
\pgfpathmoveto{\pgfqpoint{3.682482in}{4.956310in}}%
\pgfpathcurveto{\pgfqpoint{3.693532in}{4.956310in}}{\pgfqpoint{3.704131in}{4.960700in}}{\pgfqpoint{3.711945in}{4.968514in}}%
\pgfpathcurveto{\pgfqpoint{3.719759in}{4.976328in}}{\pgfqpoint{3.724149in}{4.986927in}}{\pgfqpoint{3.724149in}{4.997977in}}%
\pgfpathcurveto{\pgfqpoint{3.724149in}{5.009027in}}{\pgfqpoint{3.719759in}{5.019626in}}{\pgfqpoint{3.711945in}{5.027440in}}%
\pgfpathcurveto{\pgfqpoint{3.704131in}{5.035253in}}{\pgfqpoint{3.693532in}{5.039644in}}{\pgfqpoint{3.682482in}{5.039644in}}%
\pgfpathcurveto{\pgfqpoint{3.671432in}{5.039644in}}{\pgfqpoint{3.660833in}{5.035253in}}{\pgfqpoint{3.653019in}{5.027440in}}%
\pgfpathcurveto{\pgfqpoint{3.645206in}{5.019626in}}{\pgfqpoint{3.640816in}{5.009027in}}{\pgfqpoint{3.640816in}{4.997977in}}%
\pgfpathcurveto{\pgfqpoint{3.640816in}{4.986927in}}{\pgfqpoint{3.645206in}{4.976328in}}{\pgfqpoint{3.653019in}{4.968514in}}%
\pgfpathcurveto{\pgfqpoint{3.660833in}{4.960700in}}{\pgfqpoint{3.671432in}{4.956310in}}{\pgfqpoint{3.682482in}{4.956310in}}%
\pgfpathclose%
\pgfusepath{stroke,fill}%
\end{pgfscope}%
\begin{pgfscope}%
\pgfpathrectangle{\pgfqpoint{0.481978in}{0.331635in}}{\pgfqpoint{9.300000in}{7.700000in}}%
\pgfusepath{clip}%
\pgfsetbuttcap%
\pgfsetroundjoin%
\definecolor{currentfill}{rgb}{1.000000,0.705882,0.509804}%
\pgfsetfillcolor{currentfill}%
\pgfsetlinewidth{0.481800pt}%
\definecolor{currentstroke}{rgb}{1.000000,1.000000,1.000000}%
\pgfsetstrokecolor{currentstroke}%
\pgfsetdash{}{0pt}%
\pgfpathmoveto{\pgfqpoint{7.449113in}{2.447310in}}%
\pgfpathcurveto{\pgfqpoint{7.460163in}{2.447310in}}{\pgfqpoint{7.470762in}{2.451700in}}{\pgfqpoint{7.478576in}{2.459514in}}%
\pgfpathcurveto{\pgfqpoint{7.486390in}{2.467327in}}{\pgfqpoint{7.490780in}{2.477927in}}{\pgfqpoint{7.490780in}{2.488977in}}%
\pgfpathcurveto{\pgfqpoint{7.490780in}{2.500027in}}{\pgfqpoint{7.486390in}{2.510626in}}{\pgfqpoint{7.478576in}{2.518439in}}%
\pgfpathcurveto{\pgfqpoint{7.470762in}{2.526253in}}{\pgfqpoint{7.460163in}{2.530643in}}{\pgfqpoint{7.449113in}{2.530643in}}%
\pgfpathcurveto{\pgfqpoint{7.438063in}{2.530643in}}{\pgfqpoint{7.427464in}{2.526253in}}{\pgfqpoint{7.419650in}{2.518439in}}%
\pgfpathcurveto{\pgfqpoint{7.411837in}{2.510626in}}{\pgfqpoint{7.407447in}{2.500027in}}{\pgfqpoint{7.407447in}{2.488977in}}%
\pgfpathcurveto{\pgfqpoint{7.407447in}{2.477927in}}{\pgfqpoint{7.411837in}{2.467327in}}{\pgfqpoint{7.419650in}{2.459514in}}%
\pgfpathcurveto{\pgfqpoint{7.427464in}{2.451700in}}{\pgfqpoint{7.438063in}{2.447310in}}{\pgfqpoint{7.449113in}{2.447310in}}%
\pgfpathclose%
\pgfusepath{stroke,fill}%
\end{pgfscope}%
\begin{pgfscope}%
\pgfpathrectangle{\pgfqpoint{0.481978in}{0.331635in}}{\pgfqpoint{9.300000in}{7.700000in}}%
\pgfusepath{clip}%
\pgfsetbuttcap%
\pgfsetroundjoin%
\definecolor{currentfill}{rgb}{1.000000,0.705882,0.509804}%
\pgfsetfillcolor{currentfill}%
\pgfsetlinewidth{0.481800pt}%
\definecolor{currentstroke}{rgb}{1.000000,1.000000,1.000000}%
\pgfsetstrokecolor{currentstroke}%
\pgfsetdash{}{0pt}%
\pgfpathmoveto{\pgfqpoint{8.325442in}{5.151331in}}%
\pgfpathcurveto{\pgfqpoint{8.336492in}{5.151331in}}{\pgfqpoint{8.347091in}{5.155721in}}{\pgfqpoint{8.354904in}{5.163535in}}%
\pgfpathcurveto{\pgfqpoint{8.362718in}{5.171348in}}{\pgfqpoint{8.367108in}{5.181947in}}{\pgfqpoint{8.367108in}{5.192998in}}%
\pgfpathcurveto{\pgfqpoint{8.367108in}{5.204048in}}{\pgfqpoint{8.362718in}{5.214647in}}{\pgfqpoint{8.354904in}{5.222460in}}%
\pgfpathcurveto{\pgfqpoint{8.347091in}{5.230274in}}{\pgfqpoint{8.336492in}{5.234664in}}{\pgfqpoint{8.325442in}{5.234664in}}%
\pgfpathcurveto{\pgfqpoint{8.314392in}{5.234664in}}{\pgfqpoint{8.303793in}{5.230274in}}{\pgfqpoint{8.295979in}{5.222460in}}%
\pgfpathcurveto{\pgfqpoint{8.288165in}{5.214647in}}{\pgfqpoint{8.283775in}{5.204048in}}{\pgfqpoint{8.283775in}{5.192998in}}%
\pgfpathcurveto{\pgfqpoint{8.283775in}{5.181947in}}{\pgfqpoint{8.288165in}{5.171348in}}{\pgfqpoint{8.295979in}{5.163535in}}%
\pgfpathcurveto{\pgfqpoint{8.303793in}{5.155721in}}{\pgfqpoint{8.314392in}{5.151331in}}{\pgfqpoint{8.325442in}{5.151331in}}%
\pgfpathclose%
\pgfusepath{stroke,fill}%
\end{pgfscope}%
\begin{pgfscope}%
\pgfpathrectangle{\pgfqpoint{0.481978in}{0.331635in}}{\pgfqpoint{9.300000in}{7.700000in}}%
\pgfusepath{clip}%
\pgfsetbuttcap%
\pgfsetroundjoin%
\definecolor{currentfill}{rgb}{1.000000,0.705882,0.509804}%
\pgfsetfillcolor{currentfill}%
\pgfsetlinewidth{0.481800pt}%
\definecolor{currentstroke}{rgb}{1.000000,1.000000,1.000000}%
\pgfsetstrokecolor{currentstroke}%
\pgfsetdash{}{0pt}%
\pgfpathmoveto{\pgfqpoint{7.578449in}{4.350693in}}%
\pgfpathcurveto{\pgfqpoint{7.589499in}{4.350693in}}{\pgfqpoint{7.600098in}{4.355084in}}{\pgfqpoint{7.607912in}{4.362897in}}%
\pgfpathcurveto{\pgfqpoint{7.615726in}{4.370711in}}{\pgfqpoint{7.620116in}{4.381310in}}{\pgfqpoint{7.620116in}{4.392360in}}%
\pgfpathcurveto{\pgfqpoint{7.620116in}{4.403410in}}{\pgfqpoint{7.615726in}{4.414009in}}{\pgfqpoint{7.607912in}{4.421823in}}%
\pgfpathcurveto{\pgfqpoint{7.600098in}{4.429636in}}{\pgfqpoint{7.589499in}{4.434027in}}{\pgfqpoint{7.578449in}{4.434027in}}%
\pgfpathcurveto{\pgfqpoint{7.567399in}{4.434027in}}{\pgfqpoint{7.556800in}{4.429636in}}{\pgfqpoint{7.548986in}{4.421823in}}%
\pgfpathcurveto{\pgfqpoint{7.541173in}{4.414009in}}{\pgfqpoint{7.536783in}{4.403410in}}{\pgfqpoint{7.536783in}{4.392360in}}%
\pgfpathcurveto{\pgfqpoint{7.536783in}{4.381310in}}{\pgfqpoint{7.541173in}{4.370711in}}{\pgfqpoint{7.548986in}{4.362897in}}%
\pgfpathcurveto{\pgfqpoint{7.556800in}{4.355084in}}{\pgfqpoint{7.567399in}{4.350693in}}{\pgfqpoint{7.578449in}{4.350693in}}%
\pgfpathclose%
\pgfusepath{stroke,fill}%
\end{pgfscope}%
\begin{pgfscope}%
\pgfpathrectangle{\pgfqpoint{0.481978in}{0.331635in}}{\pgfqpoint{9.300000in}{7.700000in}}%
\pgfusepath{clip}%
\pgfsetbuttcap%
\pgfsetroundjoin%
\definecolor{currentfill}{rgb}{1.000000,0.705882,0.509804}%
\pgfsetfillcolor{currentfill}%
\pgfsetlinewidth{0.481800pt}%
\definecolor{currentstroke}{rgb}{1.000000,1.000000,1.000000}%
\pgfsetstrokecolor{currentstroke}%
\pgfsetdash{}{0pt}%
\pgfpathmoveto{\pgfqpoint{6.440915in}{3.359023in}}%
\pgfpathcurveto{\pgfqpoint{6.451966in}{3.359023in}}{\pgfqpoint{6.462565in}{3.363413in}}{\pgfqpoint{6.470378in}{3.371227in}}%
\pgfpathcurveto{\pgfqpoint{6.478192in}{3.379041in}}{\pgfqpoint{6.482582in}{3.389640in}}{\pgfqpoint{6.482582in}{3.400690in}}%
\pgfpathcurveto{\pgfqpoint{6.482582in}{3.411740in}}{\pgfqpoint{6.478192in}{3.422339in}}{\pgfqpoint{6.470378in}{3.430152in}}%
\pgfpathcurveto{\pgfqpoint{6.462565in}{3.437966in}}{\pgfqpoint{6.451966in}{3.442356in}}{\pgfqpoint{6.440915in}{3.442356in}}%
\pgfpathcurveto{\pgfqpoint{6.429865in}{3.442356in}}{\pgfqpoint{6.419266in}{3.437966in}}{\pgfqpoint{6.411453in}{3.430152in}}%
\pgfpathcurveto{\pgfqpoint{6.403639in}{3.422339in}}{\pgfqpoint{6.399249in}{3.411740in}}{\pgfqpoint{6.399249in}{3.400690in}}%
\pgfpathcurveto{\pgfqpoint{6.399249in}{3.389640in}}{\pgfqpoint{6.403639in}{3.379041in}}{\pgfqpoint{6.411453in}{3.371227in}}%
\pgfpathcurveto{\pgfqpoint{6.419266in}{3.363413in}}{\pgfqpoint{6.429865in}{3.359023in}}{\pgfqpoint{6.440915in}{3.359023in}}%
\pgfpathclose%
\pgfusepath{stroke,fill}%
\end{pgfscope}%
\begin{pgfscope}%
\pgfpathrectangle{\pgfqpoint{0.481978in}{0.331635in}}{\pgfqpoint{9.300000in}{7.700000in}}%
\pgfusepath{clip}%
\pgfsetbuttcap%
\pgfsetroundjoin%
\definecolor{currentfill}{rgb}{1.000000,0.705882,0.509804}%
\pgfsetfillcolor{currentfill}%
\pgfsetlinewidth{0.481800pt}%
\definecolor{currentstroke}{rgb}{1.000000,1.000000,1.000000}%
\pgfsetstrokecolor{currentstroke}%
\pgfsetdash{}{0pt}%
\pgfpathmoveto{\pgfqpoint{3.469279in}{3.827559in}}%
\pgfpathcurveto{\pgfqpoint{3.480329in}{3.827559in}}{\pgfqpoint{3.490928in}{3.831949in}}{\pgfqpoint{3.498742in}{3.839763in}}%
\pgfpathcurveto{\pgfqpoint{3.506555in}{3.847576in}}{\pgfqpoint{3.510946in}{3.858175in}}{\pgfqpoint{3.510946in}{3.869225in}}%
\pgfpathcurveto{\pgfqpoint{3.510946in}{3.880275in}}{\pgfqpoint{3.506555in}{3.890874in}}{\pgfqpoint{3.498742in}{3.898688in}}%
\pgfpathcurveto{\pgfqpoint{3.490928in}{3.906502in}}{\pgfqpoint{3.480329in}{3.910892in}}{\pgfqpoint{3.469279in}{3.910892in}}%
\pgfpathcurveto{\pgfqpoint{3.458229in}{3.910892in}}{\pgfqpoint{3.447630in}{3.906502in}}{\pgfqpoint{3.439816in}{3.898688in}}%
\pgfpathcurveto{\pgfqpoint{3.432003in}{3.890874in}}{\pgfqpoint{3.427612in}{3.880275in}}{\pgfqpoint{3.427612in}{3.869225in}}%
\pgfpathcurveto{\pgfqpoint{3.427612in}{3.858175in}}{\pgfqpoint{3.432003in}{3.847576in}}{\pgfqpoint{3.439816in}{3.839763in}}%
\pgfpathcurveto{\pgfqpoint{3.447630in}{3.831949in}}{\pgfqpoint{3.458229in}{3.827559in}}{\pgfqpoint{3.469279in}{3.827559in}}%
\pgfpathclose%
\pgfusepath{stroke,fill}%
\end{pgfscope}%
\begin{pgfscope}%
\pgfpathrectangle{\pgfqpoint{0.481978in}{0.331635in}}{\pgfqpoint{9.300000in}{7.700000in}}%
\pgfusepath{clip}%
\pgfsetbuttcap%
\pgfsetroundjoin%
\definecolor{currentfill}{rgb}{1.000000,0.705882,0.509804}%
\pgfsetfillcolor{currentfill}%
\pgfsetlinewidth{0.481800pt}%
\definecolor{currentstroke}{rgb}{1.000000,1.000000,1.000000}%
\pgfsetstrokecolor{currentstroke}%
\pgfsetdash{}{0pt}%
\pgfpathmoveto{\pgfqpoint{6.295661in}{1.566834in}}%
\pgfpathcurveto{\pgfqpoint{6.306711in}{1.566834in}}{\pgfqpoint{6.317310in}{1.571225in}}{\pgfqpoint{6.325124in}{1.579038in}}%
\pgfpathcurveto{\pgfqpoint{6.332938in}{1.586852in}}{\pgfqpoint{6.337328in}{1.597451in}}{\pgfqpoint{6.337328in}{1.608501in}}%
\pgfpathcurveto{\pgfqpoint{6.337328in}{1.619551in}}{\pgfqpoint{6.332938in}{1.630150in}}{\pgfqpoint{6.325124in}{1.637964in}}%
\pgfpathcurveto{\pgfqpoint{6.317310in}{1.645777in}}{\pgfqpoint{6.306711in}{1.650168in}}{\pgfqpoint{6.295661in}{1.650168in}}%
\pgfpathcurveto{\pgfqpoint{6.284611in}{1.650168in}}{\pgfqpoint{6.274012in}{1.645777in}}{\pgfqpoint{6.266198in}{1.637964in}}%
\pgfpathcurveto{\pgfqpoint{6.258385in}{1.630150in}}{\pgfqpoint{6.253994in}{1.619551in}}{\pgfqpoint{6.253994in}{1.608501in}}%
\pgfpathcurveto{\pgfqpoint{6.253994in}{1.597451in}}{\pgfqpoint{6.258385in}{1.586852in}}{\pgfqpoint{6.266198in}{1.579038in}}%
\pgfpathcurveto{\pgfqpoint{6.274012in}{1.571225in}}{\pgfqpoint{6.284611in}{1.566834in}}{\pgfqpoint{6.295661in}{1.566834in}}%
\pgfpathclose%
\pgfusepath{stroke,fill}%
\end{pgfscope}%
\begin{pgfscope}%
\pgfpathrectangle{\pgfqpoint{0.481978in}{0.331635in}}{\pgfqpoint{9.300000in}{7.700000in}}%
\pgfusepath{clip}%
\pgfsetbuttcap%
\pgfsetroundjoin%
\definecolor{currentfill}{rgb}{1.000000,0.705882,0.509804}%
\pgfsetfillcolor{currentfill}%
\pgfsetlinewidth{0.481800pt}%
\definecolor{currentstroke}{rgb}{1.000000,1.000000,1.000000}%
\pgfsetstrokecolor{currentstroke}%
\pgfsetdash{}{0pt}%
\pgfpathmoveto{\pgfqpoint{3.458025in}{3.938973in}}%
\pgfpathcurveto{\pgfqpoint{3.469075in}{3.938973in}}{\pgfqpoint{3.479674in}{3.943364in}}{\pgfqpoint{3.487487in}{3.951177in}}%
\pgfpathcurveto{\pgfqpoint{3.495301in}{3.958991in}}{\pgfqpoint{3.499691in}{3.969590in}}{\pgfqpoint{3.499691in}{3.980640in}}%
\pgfpathcurveto{\pgfqpoint{3.499691in}{3.991690in}}{\pgfqpoint{3.495301in}{4.002289in}}{\pgfqpoint{3.487487in}{4.010103in}}%
\pgfpathcurveto{\pgfqpoint{3.479674in}{4.017916in}}{\pgfqpoint{3.469075in}{4.022307in}}{\pgfqpoint{3.458025in}{4.022307in}}%
\pgfpathcurveto{\pgfqpoint{3.446974in}{4.022307in}}{\pgfqpoint{3.436375in}{4.017916in}}{\pgfqpoint{3.428562in}{4.010103in}}%
\pgfpathcurveto{\pgfqpoint{3.420748in}{4.002289in}}{\pgfqpoint{3.416358in}{3.991690in}}{\pgfqpoint{3.416358in}{3.980640in}}%
\pgfpathcurveto{\pgfqpoint{3.416358in}{3.969590in}}{\pgfqpoint{3.420748in}{3.958991in}}{\pgfqpoint{3.428562in}{3.951177in}}%
\pgfpathcurveto{\pgfqpoint{3.436375in}{3.943364in}}{\pgfqpoint{3.446974in}{3.938973in}}{\pgfqpoint{3.458025in}{3.938973in}}%
\pgfpathclose%
\pgfusepath{stroke,fill}%
\end{pgfscope}%
\begin{pgfscope}%
\pgfpathrectangle{\pgfqpoint{0.481978in}{0.331635in}}{\pgfqpoint{9.300000in}{7.700000in}}%
\pgfusepath{clip}%
\pgfsetbuttcap%
\pgfsetroundjoin%
\definecolor{currentfill}{rgb}{1.000000,0.705882,0.509804}%
\pgfsetfillcolor{currentfill}%
\pgfsetlinewidth{0.481800pt}%
\definecolor{currentstroke}{rgb}{1.000000,1.000000,1.000000}%
\pgfsetstrokecolor{currentstroke}%
\pgfsetdash{}{0pt}%
\pgfpathmoveto{\pgfqpoint{4.270877in}{5.382117in}}%
\pgfpathcurveto{\pgfqpoint{4.281927in}{5.382117in}}{\pgfqpoint{4.292526in}{5.386508in}}{\pgfqpoint{4.300340in}{5.394321in}}%
\pgfpathcurveto{\pgfqpoint{4.308153in}{5.402135in}}{\pgfqpoint{4.312543in}{5.412734in}}{\pgfqpoint{4.312543in}{5.423784in}}%
\pgfpathcurveto{\pgfqpoint{4.312543in}{5.434834in}}{\pgfqpoint{4.308153in}{5.445433in}}{\pgfqpoint{4.300340in}{5.453247in}}%
\pgfpathcurveto{\pgfqpoint{4.292526in}{5.461060in}}{\pgfqpoint{4.281927in}{5.465451in}}{\pgfqpoint{4.270877in}{5.465451in}}%
\pgfpathcurveto{\pgfqpoint{4.259827in}{5.465451in}}{\pgfqpoint{4.249228in}{5.461060in}}{\pgfqpoint{4.241414in}{5.453247in}}%
\pgfpathcurveto{\pgfqpoint{4.233600in}{5.445433in}}{\pgfqpoint{4.229210in}{5.434834in}}{\pgfqpoint{4.229210in}{5.423784in}}%
\pgfpathcurveto{\pgfqpoint{4.229210in}{5.412734in}}{\pgfqpoint{4.233600in}{5.402135in}}{\pgfqpoint{4.241414in}{5.394321in}}%
\pgfpathcurveto{\pgfqpoint{4.249228in}{5.386508in}}{\pgfqpoint{4.259827in}{5.382117in}}{\pgfqpoint{4.270877in}{5.382117in}}%
\pgfpathclose%
\pgfusepath{stroke,fill}%
\end{pgfscope}%
\begin{pgfscope}%
\pgfpathrectangle{\pgfqpoint{0.481978in}{0.331635in}}{\pgfqpoint{9.300000in}{7.700000in}}%
\pgfusepath{clip}%
\pgfsetbuttcap%
\pgfsetroundjoin%
\definecolor{currentfill}{rgb}{1.000000,0.705882,0.509804}%
\pgfsetfillcolor{currentfill}%
\pgfsetlinewidth{0.481800pt}%
\definecolor{currentstroke}{rgb}{1.000000,1.000000,1.000000}%
\pgfsetstrokecolor{currentstroke}%
\pgfsetdash{}{0pt}%
\pgfpathmoveto{\pgfqpoint{5.749725in}{1.916657in}}%
\pgfpathcurveto{\pgfqpoint{5.760775in}{1.916657in}}{\pgfqpoint{5.771374in}{1.921047in}}{\pgfqpoint{5.779188in}{1.928861in}}%
\pgfpathcurveto{\pgfqpoint{5.787002in}{1.936675in}}{\pgfqpoint{5.791392in}{1.947274in}}{\pgfqpoint{5.791392in}{1.958324in}}%
\pgfpathcurveto{\pgfqpoint{5.791392in}{1.969374in}}{\pgfqpoint{5.787002in}{1.979973in}}{\pgfqpoint{5.779188in}{1.987787in}}%
\pgfpathcurveto{\pgfqpoint{5.771374in}{1.995600in}}{\pgfqpoint{5.760775in}{1.999990in}}{\pgfqpoint{5.749725in}{1.999990in}}%
\pgfpathcurveto{\pgfqpoint{5.738675in}{1.999990in}}{\pgfqpoint{5.728076in}{1.995600in}}{\pgfqpoint{5.720262in}{1.987787in}}%
\pgfpathcurveto{\pgfqpoint{5.712449in}{1.979973in}}{\pgfqpoint{5.708059in}{1.969374in}}{\pgfqpoint{5.708059in}{1.958324in}}%
\pgfpathcurveto{\pgfqpoint{5.708059in}{1.947274in}}{\pgfqpoint{5.712449in}{1.936675in}}{\pgfqpoint{5.720262in}{1.928861in}}%
\pgfpathcurveto{\pgfqpoint{5.728076in}{1.921047in}}{\pgfqpoint{5.738675in}{1.916657in}}{\pgfqpoint{5.749725in}{1.916657in}}%
\pgfpathclose%
\pgfusepath{stroke,fill}%
\end{pgfscope}%
\begin{pgfscope}%
\pgfpathrectangle{\pgfqpoint{0.481978in}{0.331635in}}{\pgfqpoint{9.300000in}{7.700000in}}%
\pgfusepath{clip}%
\pgfsetbuttcap%
\pgfsetroundjoin%
\definecolor{currentfill}{rgb}{1.000000,0.705882,0.509804}%
\pgfsetfillcolor{currentfill}%
\pgfsetlinewidth{0.481800pt}%
\definecolor{currentstroke}{rgb}{1.000000,1.000000,1.000000}%
\pgfsetstrokecolor{currentstroke}%
\pgfsetdash{}{0pt}%
\pgfpathmoveto{\pgfqpoint{2.458775in}{1.687705in}}%
\pgfpathcurveto{\pgfqpoint{2.469825in}{1.687705in}}{\pgfqpoint{2.480424in}{1.692095in}}{\pgfqpoint{2.488238in}{1.699909in}}%
\pgfpathcurveto{\pgfqpoint{2.496051in}{1.707723in}}{\pgfqpoint{2.500442in}{1.718322in}}{\pgfqpoint{2.500442in}{1.729372in}}%
\pgfpathcurveto{\pgfqpoint{2.500442in}{1.740422in}}{\pgfqpoint{2.496051in}{1.751021in}}{\pgfqpoint{2.488238in}{1.758835in}}%
\pgfpathcurveto{\pgfqpoint{2.480424in}{1.766648in}}{\pgfqpoint{2.469825in}{1.771038in}}{\pgfqpoint{2.458775in}{1.771038in}}%
\pgfpathcurveto{\pgfqpoint{2.447725in}{1.771038in}}{\pgfqpoint{2.437126in}{1.766648in}}{\pgfqpoint{2.429312in}{1.758835in}}%
\pgfpathcurveto{\pgfqpoint{2.421499in}{1.751021in}}{\pgfqpoint{2.417108in}{1.740422in}}{\pgfqpoint{2.417108in}{1.729372in}}%
\pgfpathcurveto{\pgfqpoint{2.417108in}{1.718322in}}{\pgfqpoint{2.421499in}{1.707723in}}{\pgfqpoint{2.429312in}{1.699909in}}%
\pgfpathcurveto{\pgfqpoint{2.437126in}{1.692095in}}{\pgfqpoint{2.447725in}{1.687705in}}{\pgfqpoint{2.458775in}{1.687705in}}%
\pgfpathclose%
\pgfusepath{stroke,fill}%
\end{pgfscope}%
\begin{pgfscope}%
\pgfpathrectangle{\pgfqpoint{0.481978in}{0.331635in}}{\pgfqpoint{9.300000in}{7.700000in}}%
\pgfusepath{clip}%
\pgfsetbuttcap%
\pgfsetroundjoin%
\definecolor{currentfill}{rgb}{1.000000,0.705882,0.509804}%
\pgfsetfillcolor{currentfill}%
\pgfsetlinewidth{0.481800pt}%
\definecolor{currentstroke}{rgb}{1.000000,1.000000,1.000000}%
\pgfsetstrokecolor{currentstroke}%
\pgfsetdash{}{0pt}%
\pgfpathmoveto{\pgfqpoint{4.692250in}{5.263851in}}%
\pgfpathcurveto{\pgfqpoint{4.703300in}{5.263851in}}{\pgfqpoint{4.713899in}{5.268242in}}{\pgfqpoint{4.721713in}{5.276055in}}%
\pgfpathcurveto{\pgfqpoint{4.729527in}{5.283869in}}{\pgfqpoint{4.733917in}{5.294468in}}{\pgfqpoint{4.733917in}{5.305518in}}%
\pgfpathcurveto{\pgfqpoint{4.733917in}{5.316568in}}{\pgfqpoint{4.729527in}{5.327167in}}{\pgfqpoint{4.721713in}{5.334981in}}%
\pgfpathcurveto{\pgfqpoint{4.713899in}{5.342795in}}{\pgfqpoint{4.703300in}{5.347185in}}{\pgfqpoint{4.692250in}{5.347185in}}%
\pgfpathcurveto{\pgfqpoint{4.681200in}{5.347185in}}{\pgfqpoint{4.670601in}{5.342795in}}{\pgfqpoint{4.662787in}{5.334981in}}%
\pgfpathcurveto{\pgfqpoint{4.654974in}{5.327167in}}{\pgfqpoint{4.650584in}{5.316568in}}{\pgfqpoint{4.650584in}{5.305518in}}%
\pgfpathcurveto{\pgfqpoint{4.650584in}{5.294468in}}{\pgfqpoint{4.654974in}{5.283869in}}{\pgfqpoint{4.662787in}{5.276055in}}%
\pgfpathcurveto{\pgfqpoint{4.670601in}{5.268242in}}{\pgfqpoint{4.681200in}{5.263851in}}{\pgfqpoint{4.692250in}{5.263851in}}%
\pgfpathclose%
\pgfusepath{stroke,fill}%
\end{pgfscope}%
\begin{pgfscope}%
\pgfpathrectangle{\pgfqpoint{0.481978in}{0.331635in}}{\pgfqpoint{9.300000in}{7.700000in}}%
\pgfusepath{clip}%
\pgfsetbuttcap%
\pgfsetroundjoin%
\definecolor{currentfill}{rgb}{1.000000,0.705882,0.509804}%
\pgfsetfillcolor{currentfill}%
\pgfsetlinewidth{0.481800pt}%
\definecolor{currentstroke}{rgb}{1.000000,1.000000,1.000000}%
\pgfsetstrokecolor{currentstroke}%
\pgfsetdash{}{0pt}%
\pgfpathmoveto{\pgfqpoint{7.916388in}{4.181299in}}%
\pgfpathcurveto{\pgfqpoint{7.927438in}{4.181299in}}{\pgfqpoint{7.938037in}{4.185689in}}{\pgfqpoint{7.945851in}{4.193503in}}%
\pgfpathcurveto{\pgfqpoint{7.953665in}{4.201317in}}{\pgfqpoint{7.958055in}{4.211916in}}{\pgfqpoint{7.958055in}{4.222966in}}%
\pgfpathcurveto{\pgfqpoint{7.958055in}{4.234016in}}{\pgfqpoint{7.953665in}{4.244615in}}{\pgfqpoint{7.945851in}{4.252429in}}%
\pgfpathcurveto{\pgfqpoint{7.938037in}{4.260242in}}{\pgfqpoint{7.927438in}{4.264632in}}{\pgfqpoint{7.916388in}{4.264632in}}%
\pgfpathcurveto{\pgfqpoint{7.905338in}{4.264632in}}{\pgfqpoint{7.894739in}{4.260242in}}{\pgfqpoint{7.886926in}{4.252429in}}%
\pgfpathcurveto{\pgfqpoint{7.879112in}{4.244615in}}{\pgfqpoint{7.874722in}{4.234016in}}{\pgfqpoint{7.874722in}{4.222966in}}%
\pgfpathcurveto{\pgfqpoint{7.874722in}{4.211916in}}{\pgfqpoint{7.879112in}{4.201317in}}{\pgfqpoint{7.886926in}{4.193503in}}%
\pgfpathcurveto{\pgfqpoint{7.894739in}{4.185689in}}{\pgfqpoint{7.905338in}{4.181299in}}{\pgfqpoint{7.916388in}{4.181299in}}%
\pgfpathclose%
\pgfusepath{stroke,fill}%
\end{pgfscope}%
\begin{pgfscope}%
\pgfpathrectangle{\pgfqpoint{0.481978in}{0.331635in}}{\pgfqpoint{9.300000in}{7.700000in}}%
\pgfusepath{clip}%
\pgfsetbuttcap%
\pgfsetroundjoin%
\definecolor{currentfill}{rgb}{1.000000,0.705882,0.509804}%
\pgfsetfillcolor{currentfill}%
\pgfsetlinewidth{0.481800pt}%
\definecolor{currentstroke}{rgb}{1.000000,1.000000,1.000000}%
\pgfsetstrokecolor{currentstroke}%
\pgfsetdash{}{0pt}%
\pgfpathmoveto{\pgfqpoint{4.752028in}{3.418430in}}%
\pgfpathcurveto{\pgfqpoint{4.763078in}{3.418430in}}{\pgfqpoint{4.773677in}{3.422821in}}{\pgfqpoint{4.781491in}{3.430634in}}%
\pgfpathcurveto{\pgfqpoint{4.789304in}{3.438448in}}{\pgfqpoint{4.793694in}{3.449047in}}{\pgfqpoint{4.793694in}{3.460097in}}%
\pgfpathcurveto{\pgfqpoint{4.793694in}{3.471147in}}{\pgfqpoint{4.789304in}{3.481746in}}{\pgfqpoint{4.781491in}{3.489560in}}%
\pgfpathcurveto{\pgfqpoint{4.773677in}{3.497374in}}{\pgfqpoint{4.763078in}{3.501764in}}{\pgfqpoint{4.752028in}{3.501764in}}%
\pgfpathcurveto{\pgfqpoint{4.740978in}{3.501764in}}{\pgfqpoint{4.730379in}{3.497374in}}{\pgfqpoint{4.722565in}{3.489560in}}%
\pgfpathcurveto{\pgfqpoint{4.714751in}{3.481746in}}{\pgfqpoint{4.710361in}{3.471147in}}{\pgfqpoint{4.710361in}{3.460097in}}%
\pgfpathcurveto{\pgfqpoint{4.710361in}{3.449047in}}{\pgfqpoint{4.714751in}{3.438448in}}{\pgfqpoint{4.722565in}{3.430634in}}%
\pgfpathcurveto{\pgfqpoint{4.730379in}{3.422821in}}{\pgfqpoint{4.740978in}{3.418430in}}{\pgfqpoint{4.752028in}{3.418430in}}%
\pgfpathclose%
\pgfusepath{stroke,fill}%
\end{pgfscope}%
\begin{pgfscope}%
\pgfpathrectangle{\pgfqpoint{0.481978in}{0.331635in}}{\pgfqpoint{9.300000in}{7.700000in}}%
\pgfusepath{clip}%
\pgfsetbuttcap%
\pgfsetroundjoin%
\definecolor{currentfill}{rgb}{1.000000,0.705882,0.509804}%
\pgfsetfillcolor{currentfill}%
\pgfsetlinewidth{0.481800pt}%
\definecolor{currentstroke}{rgb}{1.000000,1.000000,1.000000}%
\pgfsetstrokecolor{currentstroke}%
\pgfsetdash{}{0pt}%
\pgfpathmoveto{\pgfqpoint{8.514292in}{5.224514in}}%
\pgfpathcurveto{\pgfqpoint{8.525342in}{5.224514in}}{\pgfqpoint{8.535941in}{5.228904in}}{\pgfqpoint{8.543755in}{5.236717in}}%
\pgfpathcurveto{\pgfqpoint{8.551569in}{5.244531in}}{\pgfqpoint{8.555959in}{5.255130in}}{\pgfqpoint{8.555959in}{5.266180in}}%
\pgfpathcurveto{\pgfqpoint{8.555959in}{5.277230in}}{\pgfqpoint{8.551569in}{5.287829in}}{\pgfqpoint{8.543755in}{5.295643in}}%
\pgfpathcurveto{\pgfqpoint{8.535941in}{5.303457in}}{\pgfqpoint{8.525342in}{5.307847in}}{\pgfqpoint{8.514292in}{5.307847in}}%
\pgfpathcurveto{\pgfqpoint{8.503242in}{5.307847in}}{\pgfqpoint{8.492643in}{5.303457in}}{\pgfqpoint{8.484829in}{5.295643in}}%
\pgfpathcurveto{\pgfqpoint{8.477016in}{5.287829in}}{\pgfqpoint{8.472626in}{5.277230in}}{\pgfqpoint{8.472626in}{5.266180in}}%
\pgfpathcurveto{\pgfqpoint{8.472626in}{5.255130in}}{\pgfqpoint{8.477016in}{5.244531in}}{\pgfqpoint{8.484829in}{5.236717in}}%
\pgfpathcurveto{\pgfqpoint{8.492643in}{5.228904in}}{\pgfqpoint{8.503242in}{5.224514in}}{\pgfqpoint{8.514292in}{5.224514in}}%
\pgfpathclose%
\pgfusepath{stroke,fill}%
\end{pgfscope}%
\begin{pgfscope}%
\pgfpathrectangle{\pgfqpoint{0.481978in}{0.331635in}}{\pgfqpoint{9.300000in}{7.700000in}}%
\pgfusepath{clip}%
\pgfsetbuttcap%
\pgfsetroundjoin%
\definecolor{currentfill}{rgb}{1.000000,0.705882,0.509804}%
\pgfsetfillcolor{currentfill}%
\pgfsetlinewidth{0.481800pt}%
\definecolor{currentstroke}{rgb}{1.000000,1.000000,1.000000}%
\pgfsetstrokecolor{currentstroke}%
\pgfsetdash{}{0pt}%
\pgfpathmoveto{\pgfqpoint{3.988115in}{1.904248in}}%
\pgfpathcurveto{\pgfqpoint{3.999165in}{1.904248in}}{\pgfqpoint{4.009764in}{1.908639in}}{\pgfqpoint{4.017578in}{1.916452in}}%
\pgfpathcurveto{\pgfqpoint{4.025392in}{1.924266in}}{\pgfqpoint{4.029782in}{1.934865in}}{\pgfqpoint{4.029782in}{1.945915in}}%
\pgfpathcurveto{\pgfqpoint{4.029782in}{1.956965in}}{\pgfqpoint{4.025392in}{1.967564in}}{\pgfqpoint{4.017578in}{1.975378in}}%
\pgfpathcurveto{\pgfqpoint{4.009764in}{1.983191in}}{\pgfqpoint{3.999165in}{1.987582in}}{\pgfqpoint{3.988115in}{1.987582in}}%
\pgfpathcurveto{\pgfqpoint{3.977065in}{1.987582in}}{\pgfqpoint{3.966466in}{1.983191in}}{\pgfqpoint{3.958652in}{1.975378in}}%
\pgfpathcurveto{\pgfqpoint{3.950839in}{1.967564in}}{\pgfqpoint{3.946448in}{1.956965in}}{\pgfqpoint{3.946448in}{1.945915in}}%
\pgfpathcurveto{\pgfqpoint{3.946448in}{1.934865in}}{\pgfqpoint{3.950839in}{1.924266in}}{\pgfqpoint{3.958652in}{1.916452in}}%
\pgfpathcurveto{\pgfqpoint{3.966466in}{1.908639in}}{\pgfqpoint{3.977065in}{1.904248in}}{\pgfqpoint{3.988115in}{1.904248in}}%
\pgfpathclose%
\pgfusepath{stroke,fill}%
\end{pgfscope}%
\begin{pgfscope}%
\pgfpathrectangle{\pgfqpoint{0.481978in}{0.331635in}}{\pgfqpoint{9.300000in}{7.700000in}}%
\pgfusepath{clip}%
\pgfsetbuttcap%
\pgfsetroundjoin%
\definecolor{currentfill}{rgb}{1.000000,0.705882,0.509804}%
\pgfsetfillcolor{currentfill}%
\pgfsetlinewidth{0.481800pt}%
\definecolor{currentstroke}{rgb}{1.000000,1.000000,1.000000}%
\pgfsetstrokecolor{currentstroke}%
\pgfsetdash{}{0pt}%
\pgfpathmoveto{\pgfqpoint{3.637460in}{4.439078in}}%
\pgfpathcurveto{\pgfqpoint{3.648510in}{4.439078in}}{\pgfqpoint{3.659109in}{4.443469in}}{\pgfqpoint{3.666923in}{4.451282in}}%
\pgfpathcurveto{\pgfqpoint{3.674737in}{4.459096in}}{\pgfqpoint{3.679127in}{4.469695in}}{\pgfqpoint{3.679127in}{4.480745in}}%
\pgfpathcurveto{\pgfqpoint{3.679127in}{4.491795in}}{\pgfqpoint{3.674737in}{4.502394in}}{\pgfqpoint{3.666923in}{4.510208in}}%
\pgfpathcurveto{\pgfqpoint{3.659109in}{4.518022in}}{\pgfqpoint{3.648510in}{4.522412in}}{\pgfqpoint{3.637460in}{4.522412in}}%
\pgfpathcurveto{\pgfqpoint{3.626410in}{4.522412in}}{\pgfqpoint{3.615811in}{4.518022in}}{\pgfqpoint{3.607997in}{4.510208in}}%
\pgfpathcurveto{\pgfqpoint{3.600184in}{4.502394in}}{\pgfqpoint{3.595794in}{4.491795in}}{\pgfqpoint{3.595794in}{4.480745in}}%
\pgfpathcurveto{\pgfqpoint{3.595794in}{4.469695in}}{\pgfqpoint{3.600184in}{4.459096in}}{\pgfqpoint{3.607997in}{4.451282in}}%
\pgfpathcurveto{\pgfqpoint{3.615811in}{4.443469in}}{\pgfqpoint{3.626410in}{4.439078in}}{\pgfqpoint{3.637460in}{4.439078in}}%
\pgfpathclose%
\pgfusepath{stroke,fill}%
\end{pgfscope}%
\begin{pgfscope}%
\pgfpathrectangle{\pgfqpoint{0.481978in}{0.331635in}}{\pgfqpoint{9.300000in}{7.700000in}}%
\pgfusepath{clip}%
\pgfsetbuttcap%
\pgfsetroundjoin%
\definecolor{currentfill}{rgb}{1.000000,0.705882,0.509804}%
\pgfsetfillcolor{currentfill}%
\pgfsetlinewidth{0.481800pt}%
\definecolor{currentstroke}{rgb}{1.000000,1.000000,1.000000}%
\pgfsetstrokecolor{currentstroke}%
\pgfsetdash{}{0pt}%
\pgfpathmoveto{\pgfqpoint{3.666577in}{4.453388in}}%
\pgfpathcurveto{\pgfqpoint{3.677627in}{4.453388in}}{\pgfqpoint{3.688226in}{4.457779in}}{\pgfqpoint{3.696040in}{4.465592in}}%
\pgfpathcurveto{\pgfqpoint{3.703854in}{4.473406in}}{\pgfqpoint{3.708244in}{4.484005in}}{\pgfqpoint{3.708244in}{4.495055in}}%
\pgfpathcurveto{\pgfqpoint{3.708244in}{4.506105in}}{\pgfqpoint{3.703854in}{4.516704in}}{\pgfqpoint{3.696040in}{4.524518in}}%
\pgfpathcurveto{\pgfqpoint{3.688226in}{4.532331in}}{\pgfqpoint{3.677627in}{4.536722in}}{\pgfqpoint{3.666577in}{4.536722in}}%
\pgfpathcurveto{\pgfqpoint{3.655527in}{4.536722in}}{\pgfqpoint{3.644928in}{4.532331in}}{\pgfqpoint{3.637114in}{4.524518in}}%
\pgfpathcurveto{\pgfqpoint{3.629301in}{4.516704in}}{\pgfqpoint{3.624910in}{4.506105in}}{\pgfqpoint{3.624910in}{4.495055in}}%
\pgfpathcurveto{\pgfqpoint{3.624910in}{4.484005in}}{\pgfqpoint{3.629301in}{4.473406in}}{\pgfqpoint{3.637114in}{4.465592in}}%
\pgfpathcurveto{\pgfqpoint{3.644928in}{4.457779in}}{\pgfqpoint{3.655527in}{4.453388in}}{\pgfqpoint{3.666577in}{4.453388in}}%
\pgfpathclose%
\pgfusepath{stroke,fill}%
\end{pgfscope}%
\begin{pgfscope}%
\pgfpathrectangle{\pgfqpoint{0.481978in}{0.331635in}}{\pgfqpoint{9.300000in}{7.700000in}}%
\pgfusepath{clip}%
\pgfsetbuttcap%
\pgfsetroundjoin%
\definecolor{currentfill}{rgb}{1.000000,0.705882,0.509804}%
\pgfsetfillcolor{currentfill}%
\pgfsetlinewidth{0.481800pt}%
\definecolor{currentstroke}{rgb}{1.000000,1.000000,1.000000}%
\pgfsetstrokecolor{currentstroke}%
\pgfsetdash{}{0pt}%
\pgfpathmoveto{\pgfqpoint{3.344754in}{3.212776in}}%
\pgfpathcurveto{\pgfqpoint{3.355805in}{3.212776in}}{\pgfqpoint{3.366404in}{3.217166in}}{\pgfqpoint{3.374217in}{3.224980in}}%
\pgfpathcurveto{\pgfqpoint{3.382031in}{3.232794in}}{\pgfqpoint{3.386421in}{3.243393in}}{\pgfqpoint{3.386421in}{3.254443in}}%
\pgfpathcurveto{\pgfqpoint{3.386421in}{3.265493in}}{\pgfqpoint{3.382031in}{3.276092in}}{\pgfqpoint{3.374217in}{3.283905in}}%
\pgfpathcurveto{\pgfqpoint{3.366404in}{3.291719in}}{\pgfqpoint{3.355805in}{3.296109in}}{\pgfqpoint{3.344754in}{3.296109in}}%
\pgfpathcurveto{\pgfqpoint{3.333704in}{3.296109in}}{\pgfqpoint{3.323105in}{3.291719in}}{\pgfqpoint{3.315292in}{3.283905in}}%
\pgfpathcurveto{\pgfqpoint{3.307478in}{3.276092in}}{\pgfqpoint{3.303088in}{3.265493in}}{\pgfqpoint{3.303088in}{3.254443in}}%
\pgfpathcurveto{\pgfqpoint{3.303088in}{3.243393in}}{\pgfqpoint{3.307478in}{3.232794in}}{\pgfqpoint{3.315292in}{3.224980in}}%
\pgfpathcurveto{\pgfqpoint{3.323105in}{3.217166in}}{\pgfqpoint{3.333704in}{3.212776in}}{\pgfqpoint{3.344754in}{3.212776in}}%
\pgfpathclose%
\pgfusepath{stroke,fill}%
\end{pgfscope}%
\begin{pgfscope}%
\pgfpathrectangle{\pgfqpoint{0.481978in}{0.331635in}}{\pgfqpoint{9.300000in}{7.700000in}}%
\pgfusepath{clip}%
\pgfsetbuttcap%
\pgfsetroundjoin%
\definecolor{currentfill}{rgb}{1.000000,0.705882,0.509804}%
\pgfsetfillcolor{currentfill}%
\pgfsetlinewidth{0.481800pt}%
\definecolor{currentstroke}{rgb}{1.000000,1.000000,1.000000}%
\pgfsetstrokecolor{currentstroke}%
\pgfsetdash{}{0pt}%
\pgfpathmoveto{\pgfqpoint{7.394228in}{2.316863in}}%
\pgfpathcurveto{\pgfqpoint{7.405278in}{2.316863in}}{\pgfqpoint{7.415877in}{2.321253in}}{\pgfqpoint{7.423691in}{2.329066in}}%
\pgfpathcurveto{\pgfqpoint{7.431504in}{2.336880in}}{\pgfqpoint{7.435894in}{2.347479in}}{\pgfqpoint{7.435894in}{2.358529in}}%
\pgfpathcurveto{\pgfqpoint{7.435894in}{2.369579in}}{\pgfqpoint{7.431504in}{2.380178in}}{\pgfqpoint{7.423691in}{2.387992in}}%
\pgfpathcurveto{\pgfqpoint{7.415877in}{2.395806in}}{\pgfqpoint{7.405278in}{2.400196in}}{\pgfqpoint{7.394228in}{2.400196in}}%
\pgfpathcurveto{\pgfqpoint{7.383178in}{2.400196in}}{\pgfqpoint{7.372579in}{2.395806in}}{\pgfqpoint{7.364765in}{2.387992in}}%
\pgfpathcurveto{\pgfqpoint{7.356951in}{2.380178in}}{\pgfqpoint{7.352561in}{2.369579in}}{\pgfqpoint{7.352561in}{2.358529in}}%
\pgfpathcurveto{\pgfqpoint{7.352561in}{2.347479in}}{\pgfqpoint{7.356951in}{2.336880in}}{\pgfqpoint{7.364765in}{2.329066in}}%
\pgfpathcurveto{\pgfqpoint{7.372579in}{2.321253in}}{\pgfqpoint{7.383178in}{2.316863in}}{\pgfqpoint{7.394228in}{2.316863in}}%
\pgfpathclose%
\pgfusepath{stroke,fill}%
\end{pgfscope}%
\begin{pgfscope}%
\pgfpathrectangle{\pgfqpoint{0.481978in}{0.331635in}}{\pgfqpoint{9.300000in}{7.700000in}}%
\pgfusepath{clip}%
\pgfsetbuttcap%
\pgfsetroundjoin%
\definecolor{currentfill}{rgb}{1.000000,0.705882,0.509804}%
\pgfsetfillcolor{currentfill}%
\pgfsetlinewidth{0.481800pt}%
\definecolor{currentstroke}{rgb}{1.000000,1.000000,1.000000}%
\pgfsetstrokecolor{currentstroke}%
\pgfsetdash{}{0pt}%
\pgfpathmoveto{\pgfqpoint{2.694400in}{2.093625in}}%
\pgfpathcurveto{\pgfqpoint{2.705451in}{2.093625in}}{\pgfqpoint{2.716050in}{2.098015in}}{\pgfqpoint{2.723863in}{2.105829in}}%
\pgfpathcurveto{\pgfqpoint{2.731677in}{2.113642in}}{\pgfqpoint{2.736067in}{2.124241in}}{\pgfqpoint{2.736067in}{2.135292in}}%
\pgfpathcurveto{\pgfqpoint{2.736067in}{2.146342in}}{\pgfqpoint{2.731677in}{2.156941in}}{\pgfqpoint{2.723863in}{2.164754in}}%
\pgfpathcurveto{\pgfqpoint{2.716050in}{2.172568in}}{\pgfqpoint{2.705451in}{2.176958in}}{\pgfqpoint{2.694400in}{2.176958in}}%
\pgfpathcurveto{\pgfqpoint{2.683350in}{2.176958in}}{\pgfqpoint{2.672751in}{2.172568in}}{\pgfqpoint{2.664938in}{2.164754in}}%
\pgfpathcurveto{\pgfqpoint{2.657124in}{2.156941in}}{\pgfqpoint{2.652734in}{2.146342in}}{\pgfqpoint{2.652734in}{2.135292in}}%
\pgfpathcurveto{\pgfqpoint{2.652734in}{2.124241in}}{\pgfqpoint{2.657124in}{2.113642in}}{\pgfqpoint{2.664938in}{2.105829in}}%
\pgfpathcurveto{\pgfqpoint{2.672751in}{2.098015in}}{\pgfqpoint{2.683350in}{2.093625in}}{\pgfqpoint{2.694400in}{2.093625in}}%
\pgfpathclose%
\pgfusepath{stroke,fill}%
\end{pgfscope}%
\begin{pgfscope}%
\pgfpathrectangle{\pgfqpoint{0.481978in}{0.331635in}}{\pgfqpoint{9.300000in}{7.700000in}}%
\pgfusepath{clip}%
\pgfsetbuttcap%
\pgfsetroundjoin%
\definecolor{currentfill}{rgb}{1.000000,0.705882,0.509804}%
\pgfsetfillcolor{currentfill}%
\pgfsetlinewidth{0.481800pt}%
\definecolor{currentstroke}{rgb}{1.000000,1.000000,1.000000}%
\pgfsetstrokecolor{currentstroke}%
\pgfsetdash{}{0pt}%
\pgfpathmoveto{\pgfqpoint{5.916550in}{1.501443in}}%
\pgfpathcurveto{\pgfqpoint{5.927600in}{1.501443in}}{\pgfqpoint{5.938199in}{1.505833in}}{\pgfqpoint{5.946013in}{1.513647in}}%
\pgfpathcurveto{\pgfqpoint{5.953826in}{1.521460in}}{\pgfqpoint{5.958217in}{1.532059in}}{\pgfqpoint{5.958217in}{1.543110in}}%
\pgfpathcurveto{\pgfqpoint{5.958217in}{1.554160in}}{\pgfqpoint{5.953826in}{1.564759in}}{\pgfqpoint{5.946013in}{1.572572in}}%
\pgfpathcurveto{\pgfqpoint{5.938199in}{1.580386in}}{\pgfqpoint{5.927600in}{1.584776in}}{\pgfqpoint{5.916550in}{1.584776in}}%
\pgfpathcurveto{\pgfqpoint{5.905500in}{1.584776in}}{\pgfqpoint{5.894901in}{1.580386in}}{\pgfqpoint{5.887087in}{1.572572in}}%
\pgfpathcurveto{\pgfqpoint{5.879274in}{1.564759in}}{\pgfqpoint{5.874883in}{1.554160in}}{\pgfqpoint{5.874883in}{1.543110in}}%
\pgfpathcurveto{\pgfqpoint{5.874883in}{1.532059in}}{\pgfqpoint{5.879274in}{1.521460in}}{\pgfqpoint{5.887087in}{1.513647in}}%
\pgfpathcurveto{\pgfqpoint{5.894901in}{1.505833in}}{\pgfqpoint{5.905500in}{1.501443in}}{\pgfqpoint{5.916550in}{1.501443in}}%
\pgfpathclose%
\pgfusepath{stroke,fill}%
\end{pgfscope}%
\begin{pgfscope}%
\pgfpathrectangle{\pgfqpoint{0.481978in}{0.331635in}}{\pgfqpoint{9.300000in}{7.700000in}}%
\pgfusepath{clip}%
\pgfsetbuttcap%
\pgfsetroundjoin%
\definecolor{currentfill}{rgb}{1.000000,0.705882,0.509804}%
\pgfsetfillcolor{currentfill}%
\pgfsetlinewidth{0.481800pt}%
\definecolor{currentstroke}{rgb}{1.000000,1.000000,1.000000}%
\pgfsetstrokecolor{currentstroke}%
\pgfsetdash{}{0pt}%
\pgfpathmoveto{\pgfqpoint{6.083860in}{2.927174in}}%
\pgfpathcurveto{\pgfqpoint{6.094910in}{2.927174in}}{\pgfqpoint{6.105509in}{2.931564in}}{\pgfqpoint{6.113322in}{2.939378in}}%
\pgfpathcurveto{\pgfqpoint{6.121136in}{2.947192in}}{\pgfqpoint{6.125526in}{2.957791in}}{\pgfqpoint{6.125526in}{2.968841in}}%
\pgfpathcurveto{\pgfqpoint{6.125526in}{2.979891in}}{\pgfqpoint{6.121136in}{2.990490in}}{\pgfqpoint{6.113322in}{2.998304in}}%
\pgfpathcurveto{\pgfqpoint{6.105509in}{3.006117in}}{\pgfqpoint{6.094910in}{3.010507in}}{\pgfqpoint{6.083860in}{3.010507in}}%
\pgfpathcurveto{\pgfqpoint{6.072810in}{3.010507in}}{\pgfqpoint{6.062211in}{3.006117in}}{\pgfqpoint{6.054397in}{2.998304in}}%
\pgfpathcurveto{\pgfqpoint{6.046583in}{2.990490in}}{\pgfqpoint{6.042193in}{2.979891in}}{\pgfqpoint{6.042193in}{2.968841in}}%
\pgfpathcurveto{\pgfqpoint{6.042193in}{2.957791in}}{\pgfqpoint{6.046583in}{2.947192in}}{\pgfqpoint{6.054397in}{2.939378in}}%
\pgfpathcurveto{\pgfqpoint{6.062211in}{2.931564in}}{\pgfqpoint{6.072810in}{2.927174in}}{\pgfqpoint{6.083860in}{2.927174in}}%
\pgfpathclose%
\pgfusepath{stroke,fill}%
\end{pgfscope}%
\begin{pgfscope}%
\pgfpathrectangle{\pgfqpoint{0.481978in}{0.331635in}}{\pgfqpoint{9.300000in}{7.700000in}}%
\pgfusepath{clip}%
\pgfsetbuttcap%
\pgfsetroundjoin%
\definecolor{currentfill}{rgb}{1.000000,0.705882,0.509804}%
\pgfsetfillcolor{currentfill}%
\pgfsetlinewidth{0.481800pt}%
\definecolor{currentstroke}{rgb}{1.000000,1.000000,1.000000}%
\pgfsetstrokecolor{currentstroke}%
\pgfsetdash{}{0pt}%
\pgfpathmoveto{\pgfqpoint{4.639314in}{3.250673in}}%
\pgfpathcurveto{\pgfqpoint{4.650364in}{3.250673in}}{\pgfqpoint{4.660963in}{3.255063in}}{\pgfqpoint{4.668776in}{3.262877in}}%
\pgfpathcurveto{\pgfqpoint{4.676590in}{3.270690in}}{\pgfqpoint{4.680980in}{3.281289in}}{\pgfqpoint{4.680980in}{3.292339in}}%
\pgfpathcurveto{\pgfqpoint{4.680980in}{3.303390in}}{\pgfqpoint{4.676590in}{3.313989in}}{\pgfqpoint{4.668776in}{3.321802in}}%
\pgfpathcurveto{\pgfqpoint{4.660963in}{3.329616in}}{\pgfqpoint{4.650364in}{3.334006in}}{\pgfqpoint{4.639314in}{3.334006in}}%
\pgfpathcurveto{\pgfqpoint{4.628264in}{3.334006in}}{\pgfqpoint{4.617665in}{3.329616in}}{\pgfqpoint{4.609851in}{3.321802in}}%
\pgfpathcurveto{\pgfqpoint{4.602037in}{3.313989in}}{\pgfqpoint{4.597647in}{3.303390in}}{\pgfqpoint{4.597647in}{3.292339in}}%
\pgfpathcurveto{\pgfqpoint{4.597647in}{3.281289in}}{\pgfqpoint{4.602037in}{3.270690in}}{\pgfqpoint{4.609851in}{3.262877in}}%
\pgfpathcurveto{\pgfqpoint{4.617665in}{3.255063in}}{\pgfqpoint{4.628264in}{3.250673in}}{\pgfqpoint{4.639314in}{3.250673in}}%
\pgfpathclose%
\pgfusepath{stroke,fill}%
\end{pgfscope}%
\begin{pgfscope}%
\pgfpathrectangle{\pgfqpoint{0.481978in}{0.331635in}}{\pgfqpoint{9.300000in}{7.700000in}}%
\pgfusepath{clip}%
\pgfsetbuttcap%
\pgfsetroundjoin%
\definecolor{currentfill}{rgb}{1.000000,0.705882,0.509804}%
\pgfsetfillcolor{currentfill}%
\pgfsetlinewidth{0.481800pt}%
\definecolor{currentstroke}{rgb}{1.000000,1.000000,1.000000}%
\pgfsetstrokecolor{currentstroke}%
\pgfsetdash{}{0pt}%
\pgfpathmoveto{\pgfqpoint{8.886711in}{5.704878in}}%
\pgfpathcurveto{\pgfqpoint{8.897762in}{5.704878in}}{\pgfqpoint{8.908361in}{5.709268in}}{\pgfqpoint{8.916174in}{5.717082in}}%
\pgfpathcurveto{\pgfqpoint{8.923988in}{5.724895in}}{\pgfqpoint{8.928378in}{5.735494in}}{\pgfqpoint{8.928378in}{5.746544in}}%
\pgfpathcurveto{\pgfqpoint{8.928378in}{5.757595in}}{\pgfqpoint{8.923988in}{5.768194in}}{\pgfqpoint{8.916174in}{5.776007in}}%
\pgfpathcurveto{\pgfqpoint{8.908361in}{5.783821in}}{\pgfqpoint{8.897762in}{5.788211in}}{\pgfqpoint{8.886711in}{5.788211in}}%
\pgfpathcurveto{\pgfqpoint{8.875661in}{5.788211in}}{\pgfqpoint{8.865062in}{5.783821in}}{\pgfqpoint{8.857249in}{5.776007in}}%
\pgfpathcurveto{\pgfqpoint{8.849435in}{5.768194in}}{\pgfqpoint{8.845045in}{5.757595in}}{\pgfqpoint{8.845045in}{5.746544in}}%
\pgfpathcurveto{\pgfqpoint{8.845045in}{5.735494in}}{\pgfqpoint{8.849435in}{5.724895in}}{\pgfqpoint{8.857249in}{5.717082in}}%
\pgfpathcurveto{\pgfqpoint{8.865062in}{5.709268in}}{\pgfqpoint{8.875661in}{5.704878in}}{\pgfqpoint{8.886711in}{5.704878in}}%
\pgfpathclose%
\pgfusepath{stroke,fill}%
\end{pgfscope}%
\begin{pgfscope}%
\pgfpathrectangle{\pgfqpoint{0.481978in}{0.331635in}}{\pgfqpoint{9.300000in}{7.700000in}}%
\pgfusepath{clip}%
\pgfsetbuttcap%
\pgfsetroundjoin%
\definecolor{currentfill}{rgb}{1.000000,0.705882,0.509804}%
\pgfsetfillcolor{currentfill}%
\pgfsetlinewidth{0.481800pt}%
\definecolor{currentstroke}{rgb}{1.000000,1.000000,1.000000}%
\pgfsetstrokecolor{currentstroke}%
\pgfsetdash{}{0pt}%
\pgfpathmoveto{\pgfqpoint{3.960687in}{2.268367in}}%
\pgfpathcurveto{\pgfqpoint{3.971737in}{2.268367in}}{\pgfqpoint{3.982336in}{2.272757in}}{\pgfqpoint{3.990150in}{2.280571in}}%
\pgfpathcurveto{\pgfqpoint{3.997963in}{2.288384in}}{\pgfqpoint{4.002353in}{2.298983in}}{\pgfqpoint{4.002353in}{2.310033in}}%
\pgfpathcurveto{\pgfqpoint{4.002353in}{2.321084in}}{\pgfqpoint{3.997963in}{2.331683in}}{\pgfqpoint{3.990150in}{2.339496in}}%
\pgfpathcurveto{\pgfqpoint{3.982336in}{2.347310in}}{\pgfqpoint{3.971737in}{2.351700in}}{\pgfqpoint{3.960687in}{2.351700in}}%
\pgfpathcurveto{\pgfqpoint{3.949637in}{2.351700in}}{\pgfqpoint{3.939038in}{2.347310in}}{\pgfqpoint{3.931224in}{2.339496in}}%
\pgfpathcurveto{\pgfqpoint{3.923410in}{2.331683in}}{\pgfqpoint{3.919020in}{2.321084in}}{\pgfqpoint{3.919020in}{2.310033in}}%
\pgfpathcurveto{\pgfqpoint{3.919020in}{2.298983in}}{\pgfqpoint{3.923410in}{2.288384in}}{\pgfqpoint{3.931224in}{2.280571in}}%
\pgfpathcurveto{\pgfqpoint{3.939038in}{2.272757in}}{\pgfqpoint{3.949637in}{2.268367in}}{\pgfqpoint{3.960687in}{2.268367in}}%
\pgfpathclose%
\pgfusepath{stroke,fill}%
\end{pgfscope}%
\begin{pgfscope}%
\pgfpathrectangle{\pgfqpoint{0.481978in}{0.331635in}}{\pgfqpoint{9.300000in}{7.700000in}}%
\pgfusepath{clip}%
\pgfsetbuttcap%
\pgfsetroundjoin%
\definecolor{currentfill}{rgb}{1.000000,0.705882,0.509804}%
\pgfsetfillcolor{currentfill}%
\pgfsetlinewidth{0.481800pt}%
\definecolor{currentstroke}{rgb}{1.000000,1.000000,1.000000}%
\pgfsetstrokecolor{currentstroke}%
\pgfsetdash{}{0pt}%
\pgfpathmoveto{\pgfqpoint{3.907332in}{1.719292in}}%
\pgfpathcurveto{\pgfqpoint{3.918382in}{1.719292in}}{\pgfqpoint{3.928981in}{1.723682in}}{\pgfqpoint{3.936795in}{1.731496in}}%
\pgfpathcurveto{\pgfqpoint{3.944608in}{1.739309in}}{\pgfqpoint{3.948999in}{1.749908in}}{\pgfqpoint{3.948999in}{1.760958in}}%
\pgfpathcurveto{\pgfqpoint{3.948999in}{1.772009in}}{\pgfqpoint{3.944608in}{1.782608in}}{\pgfqpoint{3.936795in}{1.790421in}}%
\pgfpathcurveto{\pgfqpoint{3.928981in}{1.798235in}}{\pgfqpoint{3.918382in}{1.802625in}}{\pgfqpoint{3.907332in}{1.802625in}}%
\pgfpathcurveto{\pgfqpoint{3.896282in}{1.802625in}}{\pgfqpoint{3.885683in}{1.798235in}}{\pgfqpoint{3.877869in}{1.790421in}}%
\pgfpathcurveto{\pgfqpoint{3.870055in}{1.782608in}}{\pgfqpoint{3.865665in}{1.772009in}}{\pgfqpoint{3.865665in}{1.760958in}}%
\pgfpathcurveto{\pgfqpoint{3.865665in}{1.749908in}}{\pgfqpoint{3.870055in}{1.739309in}}{\pgfqpoint{3.877869in}{1.731496in}}%
\pgfpathcurveto{\pgfqpoint{3.885683in}{1.723682in}}{\pgfqpoint{3.896282in}{1.719292in}}{\pgfqpoint{3.907332in}{1.719292in}}%
\pgfpathclose%
\pgfusepath{stroke,fill}%
\end{pgfscope}%
\begin{pgfscope}%
\pgfpathrectangle{\pgfqpoint{0.481978in}{0.331635in}}{\pgfqpoint{9.300000in}{7.700000in}}%
\pgfusepath{clip}%
\pgfsetbuttcap%
\pgfsetroundjoin%
\definecolor{currentfill}{rgb}{1.000000,0.705882,0.509804}%
\pgfsetfillcolor{currentfill}%
\pgfsetlinewidth{0.481800pt}%
\definecolor{currentstroke}{rgb}{1.000000,1.000000,1.000000}%
\pgfsetstrokecolor{currentstroke}%
\pgfsetdash{}{0pt}%
\pgfpathmoveto{\pgfqpoint{4.745564in}{5.335530in}}%
\pgfpathcurveto{\pgfqpoint{4.756614in}{5.335530in}}{\pgfqpoint{4.767213in}{5.339920in}}{\pgfqpoint{4.775026in}{5.347734in}}%
\pgfpathcurveto{\pgfqpoint{4.782840in}{5.355548in}}{\pgfqpoint{4.787230in}{5.366147in}}{\pgfqpoint{4.787230in}{5.377197in}}%
\pgfpathcurveto{\pgfqpoint{4.787230in}{5.388247in}}{\pgfqpoint{4.782840in}{5.398846in}}{\pgfqpoint{4.775026in}{5.406660in}}%
\pgfpathcurveto{\pgfqpoint{4.767213in}{5.414473in}}{\pgfqpoint{4.756614in}{5.418864in}}{\pgfqpoint{4.745564in}{5.418864in}}%
\pgfpathcurveto{\pgfqpoint{4.734513in}{5.418864in}}{\pgfqpoint{4.723914in}{5.414473in}}{\pgfqpoint{4.716101in}{5.406660in}}%
\pgfpathcurveto{\pgfqpoint{4.708287in}{5.398846in}}{\pgfqpoint{4.703897in}{5.388247in}}{\pgfqpoint{4.703897in}{5.377197in}}%
\pgfpathcurveto{\pgfqpoint{4.703897in}{5.366147in}}{\pgfqpoint{4.708287in}{5.355548in}}{\pgfqpoint{4.716101in}{5.347734in}}%
\pgfpathcurveto{\pgfqpoint{4.723914in}{5.339920in}}{\pgfqpoint{4.734513in}{5.335530in}}{\pgfqpoint{4.745564in}{5.335530in}}%
\pgfpathclose%
\pgfusepath{stroke,fill}%
\end{pgfscope}%
\begin{pgfscope}%
\pgfpathrectangle{\pgfqpoint{0.481978in}{0.331635in}}{\pgfqpoint{9.300000in}{7.700000in}}%
\pgfusepath{clip}%
\pgfsetbuttcap%
\pgfsetroundjoin%
\definecolor{currentfill}{rgb}{1.000000,0.705882,0.509804}%
\pgfsetfillcolor{currentfill}%
\pgfsetlinewidth{0.481800pt}%
\definecolor{currentstroke}{rgb}{1.000000,1.000000,1.000000}%
\pgfsetstrokecolor{currentstroke}%
\pgfsetdash{}{0pt}%
\pgfpathmoveto{\pgfqpoint{4.104700in}{5.418936in}}%
\pgfpathcurveto{\pgfqpoint{4.115750in}{5.418936in}}{\pgfqpoint{4.126350in}{5.423326in}}{\pgfqpoint{4.134163in}{5.431139in}}%
\pgfpathcurveto{\pgfqpoint{4.141977in}{5.438953in}}{\pgfqpoint{4.146367in}{5.449552in}}{\pgfqpoint{4.146367in}{5.460602in}}%
\pgfpathcurveto{\pgfqpoint{4.146367in}{5.471652in}}{\pgfqpoint{4.141977in}{5.482251in}}{\pgfqpoint{4.134163in}{5.490065in}}%
\pgfpathcurveto{\pgfqpoint{4.126350in}{5.497879in}}{\pgfqpoint{4.115750in}{5.502269in}}{\pgfqpoint{4.104700in}{5.502269in}}%
\pgfpathcurveto{\pgfqpoint{4.093650in}{5.502269in}}{\pgfqpoint{4.083051in}{5.497879in}}{\pgfqpoint{4.075238in}{5.490065in}}%
\pgfpathcurveto{\pgfqpoint{4.067424in}{5.482251in}}{\pgfqpoint{4.063034in}{5.471652in}}{\pgfqpoint{4.063034in}{5.460602in}}%
\pgfpathcurveto{\pgfqpoint{4.063034in}{5.449552in}}{\pgfqpoint{4.067424in}{5.438953in}}{\pgfqpoint{4.075238in}{5.431139in}}%
\pgfpathcurveto{\pgfqpoint{4.083051in}{5.423326in}}{\pgfqpoint{4.093650in}{5.418936in}}{\pgfqpoint{4.104700in}{5.418936in}}%
\pgfpathclose%
\pgfusepath{stroke,fill}%
\end{pgfscope}%
\begin{pgfscope}%
\pgfpathrectangle{\pgfqpoint{0.481978in}{0.331635in}}{\pgfqpoint{9.300000in}{7.700000in}}%
\pgfusepath{clip}%
\pgfsetbuttcap%
\pgfsetroundjoin%
\definecolor{currentfill}{rgb}{1.000000,0.705882,0.509804}%
\pgfsetfillcolor{currentfill}%
\pgfsetlinewidth{0.481800pt}%
\definecolor{currentstroke}{rgb}{1.000000,1.000000,1.000000}%
\pgfsetstrokecolor{currentstroke}%
\pgfsetdash{}{0pt}%
\pgfpathmoveto{\pgfqpoint{8.529282in}{4.532862in}}%
\pgfpathcurveto{\pgfqpoint{8.540332in}{4.532862in}}{\pgfqpoint{8.550931in}{4.537253in}}{\pgfqpoint{8.558745in}{4.545066in}}%
\pgfpathcurveto{\pgfqpoint{8.566558in}{4.552880in}}{\pgfqpoint{8.570948in}{4.563479in}}{\pgfqpoint{8.570948in}{4.574529in}}%
\pgfpathcurveto{\pgfqpoint{8.570948in}{4.585579in}}{\pgfqpoint{8.566558in}{4.596178in}}{\pgfqpoint{8.558745in}{4.603992in}}%
\pgfpathcurveto{\pgfqpoint{8.550931in}{4.611806in}}{\pgfqpoint{8.540332in}{4.616196in}}{\pgfqpoint{8.529282in}{4.616196in}}%
\pgfpathcurveto{\pgfqpoint{8.518232in}{4.616196in}}{\pgfqpoint{8.507633in}{4.611806in}}{\pgfqpoint{8.499819in}{4.603992in}}%
\pgfpathcurveto{\pgfqpoint{8.492005in}{4.596178in}}{\pgfqpoint{8.487615in}{4.585579in}}{\pgfqpoint{8.487615in}{4.574529in}}%
\pgfpathcurveto{\pgfqpoint{8.487615in}{4.563479in}}{\pgfqpoint{8.492005in}{4.552880in}}{\pgfqpoint{8.499819in}{4.545066in}}%
\pgfpathcurveto{\pgfqpoint{8.507633in}{4.537253in}}{\pgfqpoint{8.518232in}{4.532862in}}{\pgfqpoint{8.529282in}{4.532862in}}%
\pgfpathclose%
\pgfusepath{stroke,fill}%
\end{pgfscope}%
\begin{pgfscope}%
\pgfpathrectangle{\pgfqpoint{0.481978in}{0.331635in}}{\pgfqpoint{9.300000in}{7.700000in}}%
\pgfusepath{clip}%
\pgfsetbuttcap%
\pgfsetroundjoin%
\definecolor{currentfill}{rgb}{1.000000,0.705882,0.509804}%
\pgfsetfillcolor{currentfill}%
\pgfsetlinewidth{0.481800pt}%
\definecolor{currentstroke}{rgb}{1.000000,1.000000,1.000000}%
\pgfsetstrokecolor{currentstroke}%
\pgfsetdash{}{0pt}%
\pgfpathmoveto{\pgfqpoint{7.907970in}{3.973013in}}%
\pgfpathcurveto{\pgfqpoint{7.919021in}{3.973013in}}{\pgfqpoint{7.929620in}{3.977403in}}{\pgfqpoint{7.937433in}{3.985217in}}%
\pgfpathcurveto{\pgfqpoint{7.945247in}{3.993030in}}{\pgfqpoint{7.949637in}{4.003629in}}{\pgfqpoint{7.949637in}{4.014679in}}%
\pgfpathcurveto{\pgfqpoint{7.949637in}{4.025730in}}{\pgfqpoint{7.945247in}{4.036329in}}{\pgfqpoint{7.937433in}{4.044142in}}%
\pgfpathcurveto{\pgfqpoint{7.929620in}{4.051956in}}{\pgfqpoint{7.919021in}{4.056346in}}{\pgfqpoint{7.907970in}{4.056346in}}%
\pgfpathcurveto{\pgfqpoint{7.896920in}{4.056346in}}{\pgfqpoint{7.886321in}{4.051956in}}{\pgfqpoint{7.878508in}{4.044142in}}%
\pgfpathcurveto{\pgfqpoint{7.870694in}{4.036329in}}{\pgfqpoint{7.866304in}{4.025730in}}{\pgfqpoint{7.866304in}{4.014679in}}%
\pgfpathcurveto{\pgfqpoint{7.866304in}{4.003629in}}{\pgfqpoint{7.870694in}{3.993030in}}{\pgfqpoint{7.878508in}{3.985217in}}%
\pgfpathcurveto{\pgfqpoint{7.886321in}{3.977403in}}{\pgfqpoint{7.896920in}{3.973013in}}{\pgfqpoint{7.907970in}{3.973013in}}%
\pgfpathclose%
\pgfusepath{stroke,fill}%
\end{pgfscope}%
\begin{pgfscope}%
\pgfpathrectangle{\pgfqpoint{0.481978in}{0.331635in}}{\pgfqpoint{9.300000in}{7.700000in}}%
\pgfusepath{clip}%
\pgfsetbuttcap%
\pgfsetroundjoin%
\definecolor{currentfill}{rgb}{1.000000,0.705882,0.509804}%
\pgfsetfillcolor{currentfill}%
\pgfsetlinewidth{0.481800pt}%
\definecolor{currentstroke}{rgb}{1.000000,1.000000,1.000000}%
\pgfsetstrokecolor{currentstroke}%
\pgfsetdash{}{0pt}%
\pgfpathmoveto{\pgfqpoint{8.433076in}{4.518523in}}%
\pgfpathcurveto{\pgfqpoint{8.444126in}{4.518523in}}{\pgfqpoint{8.454725in}{4.522914in}}{\pgfqpoint{8.462538in}{4.530727in}}%
\pgfpathcurveto{\pgfqpoint{8.470352in}{4.538541in}}{\pgfqpoint{8.474742in}{4.549140in}}{\pgfqpoint{8.474742in}{4.560190in}}%
\pgfpathcurveto{\pgfqpoint{8.474742in}{4.571240in}}{\pgfqpoint{8.470352in}{4.581839in}}{\pgfqpoint{8.462538in}{4.589653in}}%
\pgfpathcurveto{\pgfqpoint{8.454725in}{4.597467in}}{\pgfqpoint{8.444126in}{4.601857in}}{\pgfqpoint{8.433076in}{4.601857in}}%
\pgfpathcurveto{\pgfqpoint{8.422025in}{4.601857in}}{\pgfqpoint{8.411426in}{4.597467in}}{\pgfqpoint{8.403613in}{4.589653in}}%
\pgfpathcurveto{\pgfqpoint{8.395799in}{4.581839in}}{\pgfqpoint{8.391409in}{4.571240in}}{\pgfqpoint{8.391409in}{4.560190in}}%
\pgfpathcurveto{\pgfqpoint{8.391409in}{4.549140in}}{\pgfqpoint{8.395799in}{4.538541in}}{\pgfqpoint{8.403613in}{4.530727in}}%
\pgfpathcurveto{\pgfqpoint{8.411426in}{4.522914in}}{\pgfqpoint{8.422025in}{4.518523in}}{\pgfqpoint{8.433076in}{4.518523in}}%
\pgfpathclose%
\pgfusepath{stroke,fill}%
\end{pgfscope}%
\begin{pgfscope}%
\pgfpathrectangle{\pgfqpoint{0.481978in}{0.331635in}}{\pgfqpoint{9.300000in}{7.700000in}}%
\pgfusepath{clip}%
\pgfsetbuttcap%
\pgfsetroundjoin%
\definecolor{currentfill}{rgb}{1.000000,0.705882,0.509804}%
\pgfsetfillcolor{currentfill}%
\pgfsetlinewidth{0.481800pt}%
\definecolor{currentstroke}{rgb}{1.000000,1.000000,1.000000}%
\pgfsetstrokecolor{currentstroke}%
\pgfsetdash{}{0pt}%
\pgfpathmoveto{\pgfqpoint{3.128738in}{2.053690in}}%
\pgfpathcurveto{\pgfqpoint{3.139789in}{2.053690in}}{\pgfqpoint{3.150388in}{2.058080in}}{\pgfqpoint{3.158201in}{2.065894in}}%
\pgfpathcurveto{\pgfqpoint{3.166015in}{2.073708in}}{\pgfqpoint{3.170405in}{2.084307in}}{\pgfqpoint{3.170405in}{2.095357in}}%
\pgfpathcurveto{\pgfqpoint{3.170405in}{2.106407in}}{\pgfqpoint{3.166015in}{2.117006in}}{\pgfqpoint{3.158201in}{2.124820in}}%
\pgfpathcurveto{\pgfqpoint{3.150388in}{2.132633in}}{\pgfqpoint{3.139789in}{2.137024in}}{\pgfqpoint{3.128738in}{2.137024in}}%
\pgfpathcurveto{\pgfqpoint{3.117688in}{2.137024in}}{\pgfqpoint{3.107089in}{2.132633in}}{\pgfqpoint{3.099276in}{2.124820in}}%
\pgfpathcurveto{\pgfqpoint{3.091462in}{2.117006in}}{\pgfqpoint{3.087072in}{2.106407in}}{\pgfqpoint{3.087072in}{2.095357in}}%
\pgfpathcurveto{\pgfqpoint{3.087072in}{2.084307in}}{\pgfqpoint{3.091462in}{2.073708in}}{\pgfqpoint{3.099276in}{2.065894in}}%
\pgfpathcurveto{\pgfqpoint{3.107089in}{2.058080in}}{\pgfqpoint{3.117688in}{2.053690in}}{\pgfqpoint{3.128738in}{2.053690in}}%
\pgfpathclose%
\pgfusepath{stroke,fill}%
\end{pgfscope}%
\begin{pgfscope}%
\pgfpathrectangle{\pgfqpoint{0.481978in}{0.331635in}}{\pgfqpoint{9.300000in}{7.700000in}}%
\pgfusepath{clip}%
\pgfsetbuttcap%
\pgfsetroundjoin%
\definecolor{currentfill}{rgb}{1.000000,0.705882,0.509804}%
\pgfsetfillcolor{currentfill}%
\pgfsetlinewidth{0.481800pt}%
\definecolor{currentstroke}{rgb}{1.000000,1.000000,1.000000}%
\pgfsetstrokecolor{currentstroke}%
\pgfsetdash{}{0pt}%
\pgfpathmoveto{\pgfqpoint{6.750608in}{3.030787in}}%
\pgfpathcurveto{\pgfqpoint{6.761658in}{3.030787in}}{\pgfqpoint{6.772257in}{3.035178in}}{\pgfqpoint{6.780070in}{3.042991in}}%
\pgfpathcurveto{\pgfqpoint{6.787884in}{3.050805in}}{\pgfqpoint{6.792274in}{3.061404in}}{\pgfqpoint{6.792274in}{3.072454in}}%
\pgfpathcurveto{\pgfqpoint{6.792274in}{3.083504in}}{\pgfqpoint{6.787884in}{3.094103in}}{\pgfqpoint{6.780070in}{3.101917in}}%
\pgfpathcurveto{\pgfqpoint{6.772257in}{3.109730in}}{\pgfqpoint{6.761658in}{3.114121in}}{\pgfqpoint{6.750608in}{3.114121in}}%
\pgfpathcurveto{\pgfqpoint{6.739557in}{3.114121in}}{\pgfqpoint{6.728958in}{3.109730in}}{\pgfqpoint{6.721145in}{3.101917in}}%
\pgfpathcurveto{\pgfqpoint{6.713331in}{3.094103in}}{\pgfqpoint{6.708941in}{3.083504in}}{\pgfqpoint{6.708941in}{3.072454in}}%
\pgfpathcurveto{\pgfqpoint{6.708941in}{3.061404in}}{\pgfqpoint{6.713331in}{3.050805in}}{\pgfqpoint{6.721145in}{3.042991in}}%
\pgfpathcurveto{\pgfqpoint{6.728958in}{3.035178in}}{\pgfqpoint{6.739557in}{3.030787in}}{\pgfqpoint{6.750608in}{3.030787in}}%
\pgfpathclose%
\pgfusepath{stroke,fill}%
\end{pgfscope}%
\begin{pgfscope}%
\pgfpathrectangle{\pgfqpoint{0.481978in}{0.331635in}}{\pgfqpoint{9.300000in}{7.700000in}}%
\pgfusepath{clip}%
\pgfsetbuttcap%
\pgfsetroundjoin%
\definecolor{currentfill}{rgb}{1.000000,0.705882,0.509804}%
\pgfsetfillcolor{currentfill}%
\pgfsetlinewidth{0.481800pt}%
\definecolor{currentstroke}{rgb}{1.000000,1.000000,1.000000}%
\pgfsetstrokecolor{currentstroke}%
\pgfsetdash{}{0pt}%
\pgfpathmoveto{\pgfqpoint{3.434837in}{4.030228in}}%
\pgfpathcurveto{\pgfqpoint{3.445887in}{4.030228in}}{\pgfqpoint{3.456486in}{4.034618in}}{\pgfqpoint{3.464300in}{4.042432in}}%
\pgfpathcurveto{\pgfqpoint{3.472114in}{4.050245in}}{\pgfqpoint{3.476504in}{4.060844in}}{\pgfqpoint{3.476504in}{4.071894in}}%
\pgfpathcurveto{\pgfqpoint{3.476504in}{4.082945in}}{\pgfqpoint{3.472114in}{4.093544in}}{\pgfqpoint{3.464300in}{4.101357in}}%
\pgfpathcurveto{\pgfqpoint{3.456486in}{4.109171in}}{\pgfqpoint{3.445887in}{4.113561in}}{\pgfqpoint{3.434837in}{4.113561in}}%
\pgfpathcurveto{\pgfqpoint{3.423787in}{4.113561in}}{\pgfqpoint{3.413188in}{4.109171in}}{\pgfqpoint{3.405375in}{4.101357in}}%
\pgfpathcurveto{\pgfqpoint{3.397561in}{4.093544in}}{\pgfqpoint{3.393171in}{4.082945in}}{\pgfqpoint{3.393171in}{4.071894in}}%
\pgfpathcurveto{\pgfqpoint{3.393171in}{4.060844in}}{\pgfqpoint{3.397561in}{4.050245in}}{\pgfqpoint{3.405375in}{4.042432in}}%
\pgfpathcurveto{\pgfqpoint{3.413188in}{4.034618in}}{\pgfqpoint{3.423787in}{4.030228in}}{\pgfqpoint{3.434837in}{4.030228in}}%
\pgfpathclose%
\pgfusepath{stroke,fill}%
\end{pgfscope}%
\begin{pgfscope}%
\pgfpathrectangle{\pgfqpoint{0.481978in}{0.331635in}}{\pgfqpoint{9.300000in}{7.700000in}}%
\pgfusepath{clip}%
\pgfsetbuttcap%
\pgfsetroundjoin%
\definecolor{currentfill}{rgb}{1.000000,0.705882,0.509804}%
\pgfsetfillcolor{currentfill}%
\pgfsetlinewidth{0.481800pt}%
\definecolor{currentstroke}{rgb}{1.000000,1.000000,1.000000}%
\pgfsetstrokecolor{currentstroke}%
\pgfsetdash{}{0pt}%
\pgfpathmoveto{\pgfqpoint{2.745959in}{2.280244in}}%
\pgfpathcurveto{\pgfqpoint{2.757010in}{2.280244in}}{\pgfqpoint{2.767609in}{2.284634in}}{\pgfqpoint{2.775422in}{2.292448in}}%
\pgfpathcurveto{\pgfqpoint{2.783236in}{2.300262in}}{\pgfqpoint{2.787626in}{2.310861in}}{\pgfqpoint{2.787626in}{2.321911in}}%
\pgfpathcurveto{\pgfqpoint{2.787626in}{2.332961in}}{\pgfqpoint{2.783236in}{2.343560in}}{\pgfqpoint{2.775422in}{2.351374in}}%
\pgfpathcurveto{\pgfqpoint{2.767609in}{2.359187in}}{\pgfqpoint{2.757010in}{2.363577in}}{\pgfqpoint{2.745959in}{2.363577in}}%
\pgfpathcurveto{\pgfqpoint{2.734909in}{2.363577in}}{\pgfqpoint{2.724310in}{2.359187in}}{\pgfqpoint{2.716497in}{2.351374in}}%
\pgfpathcurveto{\pgfqpoint{2.708683in}{2.343560in}}{\pgfqpoint{2.704293in}{2.332961in}}{\pgfqpoint{2.704293in}{2.321911in}}%
\pgfpathcurveto{\pgfqpoint{2.704293in}{2.310861in}}{\pgfqpoint{2.708683in}{2.300262in}}{\pgfqpoint{2.716497in}{2.292448in}}%
\pgfpathcurveto{\pgfqpoint{2.724310in}{2.284634in}}{\pgfqpoint{2.734909in}{2.280244in}}{\pgfqpoint{2.745959in}{2.280244in}}%
\pgfpathclose%
\pgfusepath{stroke,fill}%
\end{pgfscope}%
\begin{pgfscope}%
\pgfpathrectangle{\pgfqpoint{0.481978in}{0.331635in}}{\pgfqpoint{9.300000in}{7.700000in}}%
\pgfusepath{clip}%
\pgfsetbuttcap%
\pgfsetroundjoin%
\definecolor{currentfill}{rgb}{1.000000,0.705882,0.509804}%
\pgfsetfillcolor{currentfill}%
\pgfsetlinewidth{0.481800pt}%
\definecolor{currentstroke}{rgb}{1.000000,1.000000,1.000000}%
\pgfsetstrokecolor{currentstroke}%
\pgfsetdash{}{0pt}%
\pgfpathmoveto{\pgfqpoint{8.714562in}{5.295792in}}%
\pgfpathcurveto{\pgfqpoint{8.725612in}{5.295792in}}{\pgfqpoint{8.736211in}{5.300182in}}{\pgfqpoint{8.744024in}{5.307996in}}%
\pgfpathcurveto{\pgfqpoint{8.751838in}{5.315809in}}{\pgfqpoint{8.756228in}{5.326408in}}{\pgfqpoint{8.756228in}{5.337458in}}%
\pgfpathcurveto{\pgfqpoint{8.756228in}{5.348509in}}{\pgfqpoint{8.751838in}{5.359108in}}{\pgfqpoint{8.744024in}{5.366921in}}%
\pgfpathcurveto{\pgfqpoint{8.736211in}{5.374735in}}{\pgfqpoint{8.725612in}{5.379125in}}{\pgfqpoint{8.714562in}{5.379125in}}%
\pgfpathcurveto{\pgfqpoint{8.703512in}{5.379125in}}{\pgfqpoint{8.692913in}{5.374735in}}{\pgfqpoint{8.685099in}{5.366921in}}%
\pgfpathcurveto{\pgfqpoint{8.677285in}{5.359108in}}{\pgfqpoint{8.672895in}{5.348509in}}{\pgfqpoint{8.672895in}{5.337458in}}%
\pgfpathcurveto{\pgfqpoint{8.672895in}{5.326408in}}{\pgfqpoint{8.677285in}{5.315809in}}{\pgfqpoint{8.685099in}{5.307996in}}%
\pgfpathcurveto{\pgfqpoint{8.692913in}{5.300182in}}{\pgfqpoint{8.703512in}{5.295792in}}{\pgfqpoint{8.714562in}{5.295792in}}%
\pgfpathclose%
\pgfusepath{stroke,fill}%
\end{pgfscope}%
\begin{pgfscope}%
\pgfpathrectangle{\pgfqpoint{0.481978in}{0.331635in}}{\pgfqpoint{9.300000in}{7.700000in}}%
\pgfusepath{clip}%
\pgfsetbuttcap%
\pgfsetroundjoin%
\definecolor{currentfill}{rgb}{0.552941,0.898039,0.631373}%
\pgfsetfillcolor{currentfill}%
\pgfsetlinewidth{0.481800pt}%
\definecolor{currentstroke}{rgb}{1.000000,1.000000,1.000000}%
\pgfsetstrokecolor{currentstroke}%
\pgfsetdash{}{0pt}%
\pgfpathmoveto{\pgfqpoint{5.163299in}{7.301501in}}%
\pgfpathcurveto{\pgfqpoint{5.174349in}{7.301501in}}{\pgfqpoint{5.184948in}{7.305892in}}{\pgfqpoint{5.192762in}{7.313705in}}%
\pgfpathcurveto{\pgfqpoint{5.200575in}{7.321519in}}{\pgfqpoint{5.204966in}{7.332118in}}{\pgfqpoint{5.204966in}{7.343168in}}%
\pgfpathcurveto{\pgfqpoint{5.204966in}{7.354218in}}{\pgfqpoint{5.200575in}{7.364817in}}{\pgfqpoint{5.192762in}{7.372631in}}%
\pgfpathcurveto{\pgfqpoint{5.184948in}{7.380445in}}{\pgfqpoint{5.174349in}{7.384835in}}{\pgfqpoint{5.163299in}{7.384835in}}%
\pgfpathcurveto{\pgfqpoint{5.152249in}{7.384835in}}{\pgfqpoint{5.141650in}{7.380445in}}{\pgfqpoint{5.133836in}{7.372631in}}%
\pgfpathcurveto{\pgfqpoint{5.126023in}{7.364817in}}{\pgfqpoint{5.121632in}{7.354218in}}{\pgfqpoint{5.121632in}{7.343168in}}%
\pgfpathcurveto{\pgfqpoint{5.121632in}{7.332118in}}{\pgfqpoint{5.126023in}{7.321519in}}{\pgfqpoint{5.133836in}{7.313705in}}%
\pgfpathcurveto{\pgfqpoint{5.141650in}{7.305892in}}{\pgfqpoint{5.152249in}{7.301501in}}{\pgfqpoint{5.163299in}{7.301501in}}%
\pgfpathclose%
\pgfusepath{stroke,fill}%
\end{pgfscope}%
\begin{pgfscope}%
\pgfpathrectangle{\pgfqpoint{0.481978in}{0.331635in}}{\pgfqpoint{9.300000in}{7.700000in}}%
\pgfusepath{clip}%
\pgfsetbuttcap%
\pgfsetroundjoin%
\definecolor{currentfill}{rgb}{0.552941,0.898039,0.631373}%
\pgfsetfillcolor{currentfill}%
\pgfsetlinewidth{0.481800pt}%
\definecolor{currentstroke}{rgb}{1.000000,1.000000,1.000000}%
\pgfsetstrokecolor{currentstroke}%
\pgfsetdash{}{0pt}%
\pgfpathmoveto{\pgfqpoint{3.761360in}{5.722980in}}%
\pgfpathcurveto{\pgfqpoint{3.772410in}{5.722980in}}{\pgfqpoint{3.783009in}{5.727370in}}{\pgfqpoint{3.790823in}{5.735183in}}%
\pgfpathcurveto{\pgfqpoint{3.798637in}{5.742997in}}{\pgfqpoint{3.803027in}{5.753596in}}{\pgfqpoint{3.803027in}{5.764646in}}%
\pgfpathcurveto{\pgfqpoint{3.803027in}{5.775696in}}{\pgfqpoint{3.798637in}{5.786295in}}{\pgfqpoint{3.790823in}{5.794109in}}%
\pgfpathcurveto{\pgfqpoint{3.783009in}{5.801923in}}{\pgfqpoint{3.772410in}{5.806313in}}{\pgfqpoint{3.761360in}{5.806313in}}%
\pgfpathcurveto{\pgfqpoint{3.750310in}{5.806313in}}{\pgfqpoint{3.739711in}{5.801923in}}{\pgfqpoint{3.731897in}{5.794109in}}%
\pgfpathcurveto{\pgfqpoint{3.724084in}{5.786295in}}{\pgfqpoint{3.719693in}{5.775696in}}{\pgfqpoint{3.719693in}{5.764646in}}%
\pgfpathcurveto{\pgfqpoint{3.719693in}{5.753596in}}{\pgfqpoint{3.724084in}{5.742997in}}{\pgfqpoint{3.731897in}{5.735183in}}%
\pgfpathcurveto{\pgfqpoint{3.739711in}{5.727370in}}{\pgfqpoint{3.750310in}{5.722980in}}{\pgfqpoint{3.761360in}{5.722980in}}%
\pgfpathclose%
\pgfusepath{stroke,fill}%
\end{pgfscope}%
\begin{pgfscope}%
\pgfpathrectangle{\pgfqpoint{0.481978in}{0.331635in}}{\pgfqpoint{9.300000in}{7.700000in}}%
\pgfusepath{clip}%
\pgfsetbuttcap%
\pgfsetroundjoin%
\definecolor{currentfill}{rgb}{0.552941,0.898039,0.631373}%
\pgfsetfillcolor{currentfill}%
\pgfsetlinewidth{0.481800pt}%
\definecolor{currentstroke}{rgb}{1.000000,1.000000,1.000000}%
\pgfsetstrokecolor{currentstroke}%
\pgfsetdash{}{0pt}%
\pgfpathmoveto{\pgfqpoint{3.580350in}{3.046636in}}%
\pgfpathcurveto{\pgfqpoint{3.591400in}{3.046636in}}{\pgfqpoint{3.601999in}{3.051026in}}{\pgfqpoint{3.609813in}{3.058840in}}%
\pgfpathcurveto{\pgfqpoint{3.617626in}{3.066654in}}{\pgfqpoint{3.622017in}{3.077253in}}{\pgfqpoint{3.622017in}{3.088303in}}%
\pgfpathcurveto{\pgfqpoint{3.622017in}{3.099353in}}{\pgfqpoint{3.617626in}{3.109952in}}{\pgfqpoint{3.609813in}{3.117765in}}%
\pgfpathcurveto{\pgfqpoint{3.601999in}{3.125579in}}{\pgfqpoint{3.591400in}{3.129969in}}{\pgfqpoint{3.580350in}{3.129969in}}%
\pgfpathcurveto{\pgfqpoint{3.569300in}{3.129969in}}{\pgfqpoint{3.558701in}{3.125579in}}{\pgfqpoint{3.550887in}{3.117765in}}%
\pgfpathcurveto{\pgfqpoint{3.543073in}{3.109952in}}{\pgfqpoint{3.538683in}{3.099353in}}{\pgfqpoint{3.538683in}{3.088303in}}%
\pgfpathcurveto{\pgfqpoint{3.538683in}{3.077253in}}{\pgfqpoint{3.543073in}{3.066654in}}{\pgfqpoint{3.550887in}{3.058840in}}%
\pgfpathcurveto{\pgfqpoint{3.558701in}{3.051026in}}{\pgfqpoint{3.569300in}{3.046636in}}{\pgfqpoint{3.580350in}{3.046636in}}%
\pgfpathclose%
\pgfusepath{stroke,fill}%
\end{pgfscope}%
\begin{pgfscope}%
\pgfpathrectangle{\pgfqpoint{0.481978in}{0.331635in}}{\pgfqpoint{9.300000in}{7.700000in}}%
\pgfusepath{clip}%
\pgfsetbuttcap%
\pgfsetroundjoin%
\definecolor{currentfill}{rgb}{0.552941,0.898039,0.631373}%
\pgfsetfillcolor{currentfill}%
\pgfsetlinewidth{0.481800pt}%
\definecolor{currentstroke}{rgb}{1.000000,1.000000,1.000000}%
\pgfsetstrokecolor{currentstroke}%
\pgfsetdash{}{0pt}%
\pgfpathmoveto{\pgfqpoint{8.242673in}{4.179758in}}%
\pgfpathcurveto{\pgfqpoint{8.253723in}{4.179758in}}{\pgfqpoint{8.264322in}{4.184148in}}{\pgfqpoint{8.272136in}{4.191962in}}%
\pgfpathcurveto{\pgfqpoint{8.279949in}{4.199775in}}{\pgfqpoint{8.284340in}{4.210374in}}{\pgfqpoint{8.284340in}{4.221424in}}%
\pgfpathcurveto{\pgfqpoint{8.284340in}{4.232475in}}{\pgfqpoint{8.279949in}{4.243074in}}{\pgfqpoint{8.272136in}{4.250887in}}%
\pgfpathcurveto{\pgfqpoint{8.264322in}{4.258701in}}{\pgfqpoint{8.253723in}{4.263091in}}{\pgfqpoint{8.242673in}{4.263091in}}%
\pgfpathcurveto{\pgfqpoint{8.231623in}{4.263091in}}{\pgfqpoint{8.221024in}{4.258701in}}{\pgfqpoint{8.213210in}{4.250887in}}%
\pgfpathcurveto{\pgfqpoint{8.205396in}{4.243074in}}{\pgfqpoint{8.201006in}{4.232475in}}{\pgfqpoint{8.201006in}{4.221424in}}%
\pgfpathcurveto{\pgfqpoint{8.201006in}{4.210374in}}{\pgfqpoint{8.205396in}{4.199775in}}{\pgfqpoint{8.213210in}{4.191962in}}%
\pgfpathcurveto{\pgfqpoint{8.221024in}{4.184148in}}{\pgfqpoint{8.231623in}{4.179758in}}{\pgfqpoint{8.242673in}{4.179758in}}%
\pgfpathclose%
\pgfusepath{stroke,fill}%
\end{pgfscope}%
\begin{pgfscope}%
\pgfpathrectangle{\pgfqpoint{0.481978in}{0.331635in}}{\pgfqpoint{9.300000in}{7.700000in}}%
\pgfusepath{clip}%
\pgfsetbuttcap%
\pgfsetroundjoin%
\definecolor{currentfill}{rgb}{0.552941,0.898039,0.631373}%
\pgfsetfillcolor{currentfill}%
\pgfsetlinewidth{0.481800pt}%
\definecolor{currentstroke}{rgb}{1.000000,1.000000,1.000000}%
\pgfsetstrokecolor{currentstroke}%
\pgfsetdash{}{0pt}%
\pgfpathmoveto{\pgfqpoint{4.433234in}{5.029314in}}%
\pgfpathcurveto{\pgfqpoint{4.444284in}{5.029314in}}{\pgfqpoint{4.454883in}{5.033704in}}{\pgfqpoint{4.462696in}{5.041518in}}%
\pgfpathcurveto{\pgfqpoint{4.470510in}{5.049332in}}{\pgfqpoint{4.474900in}{5.059931in}}{\pgfqpoint{4.474900in}{5.070981in}}%
\pgfpathcurveto{\pgfqpoint{4.474900in}{5.082031in}}{\pgfqpoint{4.470510in}{5.092630in}}{\pgfqpoint{4.462696in}{5.100444in}}%
\pgfpathcurveto{\pgfqpoint{4.454883in}{5.108257in}}{\pgfqpoint{4.444284in}{5.112648in}}{\pgfqpoint{4.433234in}{5.112648in}}%
\pgfpathcurveto{\pgfqpoint{4.422183in}{5.112648in}}{\pgfqpoint{4.411584in}{5.108257in}}{\pgfqpoint{4.403771in}{5.100444in}}%
\pgfpathcurveto{\pgfqpoint{4.395957in}{5.092630in}}{\pgfqpoint{4.391567in}{5.082031in}}{\pgfqpoint{4.391567in}{5.070981in}}%
\pgfpathcurveto{\pgfqpoint{4.391567in}{5.059931in}}{\pgfqpoint{4.395957in}{5.049332in}}{\pgfqpoint{4.403771in}{5.041518in}}%
\pgfpathcurveto{\pgfqpoint{4.411584in}{5.033704in}}{\pgfqpoint{4.422183in}{5.029314in}}{\pgfqpoint{4.433234in}{5.029314in}}%
\pgfpathclose%
\pgfusepath{stroke,fill}%
\end{pgfscope}%
\begin{pgfscope}%
\pgfpathrectangle{\pgfqpoint{0.481978in}{0.331635in}}{\pgfqpoint{9.300000in}{7.700000in}}%
\pgfusepath{clip}%
\pgfsetbuttcap%
\pgfsetroundjoin%
\definecolor{currentfill}{rgb}{0.552941,0.898039,0.631373}%
\pgfsetfillcolor{currentfill}%
\pgfsetlinewidth{0.481800pt}%
\definecolor{currentstroke}{rgb}{1.000000,1.000000,1.000000}%
\pgfsetstrokecolor{currentstroke}%
\pgfsetdash{}{0pt}%
\pgfpathmoveto{\pgfqpoint{9.194655in}{5.622269in}}%
\pgfpathcurveto{\pgfqpoint{9.205705in}{5.622269in}}{\pgfqpoint{9.216304in}{5.626659in}}{\pgfqpoint{9.224117in}{5.634473in}}%
\pgfpathcurveto{\pgfqpoint{9.231931in}{5.642286in}}{\pgfqpoint{9.236321in}{5.652885in}}{\pgfqpoint{9.236321in}{5.663935in}}%
\pgfpathcurveto{\pgfqpoint{9.236321in}{5.674985in}}{\pgfqpoint{9.231931in}{5.685584in}}{\pgfqpoint{9.224117in}{5.693398in}}%
\pgfpathcurveto{\pgfqpoint{9.216304in}{5.701212in}}{\pgfqpoint{9.205705in}{5.705602in}}{\pgfqpoint{9.194655in}{5.705602in}}%
\pgfpathcurveto{\pgfqpoint{9.183604in}{5.705602in}}{\pgfqpoint{9.173005in}{5.701212in}}{\pgfqpoint{9.165192in}{5.693398in}}%
\pgfpathcurveto{\pgfqpoint{9.157378in}{5.685584in}}{\pgfqpoint{9.152988in}{5.674985in}}{\pgfqpoint{9.152988in}{5.663935in}}%
\pgfpathcurveto{\pgfqpoint{9.152988in}{5.652885in}}{\pgfqpoint{9.157378in}{5.642286in}}{\pgfqpoint{9.165192in}{5.634473in}}%
\pgfpathcurveto{\pgfqpoint{9.173005in}{5.626659in}}{\pgfqpoint{9.183604in}{5.622269in}}{\pgfqpoint{9.194655in}{5.622269in}}%
\pgfpathclose%
\pgfusepath{stroke,fill}%
\end{pgfscope}%
\begin{pgfscope}%
\pgfpathrectangle{\pgfqpoint{0.481978in}{0.331635in}}{\pgfqpoint{9.300000in}{7.700000in}}%
\pgfusepath{clip}%
\pgfsetbuttcap%
\pgfsetroundjoin%
\definecolor{currentfill}{rgb}{0.552941,0.898039,0.631373}%
\pgfsetfillcolor{currentfill}%
\pgfsetlinewidth{0.481800pt}%
\definecolor{currentstroke}{rgb}{1.000000,1.000000,1.000000}%
\pgfsetstrokecolor{currentstroke}%
\pgfsetdash{}{0pt}%
\pgfpathmoveto{\pgfqpoint{9.032589in}{5.038453in}}%
\pgfpathcurveto{\pgfqpoint{9.043639in}{5.038453in}}{\pgfqpoint{9.054238in}{5.042843in}}{\pgfqpoint{9.062052in}{5.050657in}}%
\pgfpathcurveto{\pgfqpoint{9.069866in}{5.058470in}}{\pgfqpoint{9.074256in}{5.069069in}}{\pgfqpoint{9.074256in}{5.080119in}}%
\pgfpathcurveto{\pgfqpoint{9.074256in}{5.091169in}}{\pgfqpoint{9.069866in}{5.101769in}}{\pgfqpoint{9.062052in}{5.109582in}}%
\pgfpathcurveto{\pgfqpoint{9.054238in}{5.117396in}}{\pgfqpoint{9.043639in}{5.121786in}}{\pgfqpoint{9.032589in}{5.121786in}}%
\pgfpathcurveto{\pgfqpoint{9.021539in}{5.121786in}}{\pgfqpoint{9.010940in}{5.117396in}}{\pgfqpoint{9.003127in}{5.109582in}}%
\pgfpathcurveto{\pgfqpoint{8.995313in}{5.101769in}}{\pgfqpoint{8.990923in}{5.091169in}}{\pgfqpoint{8.990923in}{5.080119in}}%
\pgfpathcurveto{\pgfqpoint{8.990923in}{5.069069in}}{\pgfqpoint{8.995313in}{5.058470in}}{\pgfqpoint{9.003127in}{5.050657in}}%
\pgfpathcurveto{\pgfqpoint{9.010940in}{5.042843in}}{\pgfqpoint{9.021539in}{5.038453in}}{\pgfqpoint{9.032589in}{5.038453in}}%
\pgfpathclose%
\pgfusepath{stroke,fill}%
\end{pgfscope}%
\begin{pgfscope}%
\pgfpathrectangle{\pgfqpoint{0.481978in}{0.331635in}}{\pgfqpoint{9.300000in}{7.700000in}}%
\pgfusepath{clip}%
\pgfsetbuttcap%
\pgfsetroundjoin%
\definecolor{currentfill}{rgb}{0.552941,0.898039,0.631373}%
\pgfsetfillcolor{currentfill}%
\pgfsetlinewidth{0.481800pt}%
\definecolor{currentstroke}{rgb}{1.000000,1.000000,1.000000}%
\pgfsetstrokecolor{currentstroke}%
\pgfsetdash{}{0pt}%
\pgfpathmoveto{\pgfqpoint{3.942708in}{5.631876in}}%
\pgfpathcurveto{\pgfqpoint{3.953758in}{5.631876in}}{\pgfqpoint{3.964357in}{5.636267in}}{\pgfqpoint{3.972171in}{5.644080in}}%
\pgfpathcurveto{\pgfqpoint{3.979985in}{5.651894in}}{\pgfqpoint{3.984375in}{5.662493in}}{\pgfqpoint{3.984375in}{5.673543in}}%
\pgfpathcurveto{\pgfqpoint{3.984375in}{5.684593in}}{\pgfqpoint{3.979985in}{5.695192in}}{\pgfqpoint{3.972171in}{5.703006in}}%
\pgfpathcurveto{\pgfqpoint{3.964357in}{5.710819in}}{\pgfqpoint{3.953758in}{5.715210in}}{\pgfqpoint{3.942708in}{5.715210in}}%
\pgfpathcurveto{\pgfqpoint{3.931658in}{5.715210in}}{\pgfqpoint{3.921059in}{5.710819in}}{\pgfqpoint{3.913245in}{5.703006in}}%
\pgfpathcurveto{\pgfqpoint{3.905432in}{5.695192in}}{\pgfqpoint{3.901042in}{5.684593in}}{\pgfqpoint{3.901042in}{5.673543in}}%
\pgfpathcurveto{\pgfqpoint{3.901042in}{5.662493in}}{\pgfqpoint{3.905432in}{5.651894in}}{\pgfqpoint{3.913245in}{5.644080in}}%
\pgfpathcurveto{\pgfqpoint{3.921059in}{5.636267in}}{\pgfqpoint{3.931658in}{5.631876in}}{\pgfqpoint{3.942708in}{5.631876in}}%
\pgfpathclose%
\pgfusepath{stroke,fill}%
\end{pgfscope}%
\begin{pgfscope}%
\pgfpathrectangle{\pgfqpoint{0.481978in}{0.331635in}}{\pgfqpoint{9.300000in}{7.700000in}}%
\pgfusepath{clip}%
\pgfsetbuttcap%
\pgfsetroundjoin%
\definecolor{currentfill}{rgb}{0.552941,0.898039,0.631373}%
\pgfsetfillcolor{currentfill}%
\pgfsetlinewidth{0.481800pt}%
\definecolor{currentstroke}{rgb}{1.000000,1.000000,1.000000}%
\pgfsetstrokecolor{currentstroke}%
\pgfsetdash{}{0pt}%
\pgfpathmoveto{\pgfqpoint{5.874679in}{2.681112in}}%
\pgfpathcurveto{\pgfqpoint{5.885729in}{2.681112in}}{\pgfqpoint{5.896328in}{2.685502in}}{\pgfqpoint{5.904141in}{2.693316in}}%
\pgfpathcurveto{\pgfqpoint{5.911955in}{2.701129in}}{\pgfqpoint{5.916345in}{2.711728in}}{\pgfqpoint{5.916345in}{2.722779in}}%
\pgfpathcurveto{\pgfqpoint{5.916345in}{2.733829in}}{\pgfqpoint{5.911955in}{2.744428in}}{\pgfqpoint{5.904141in}{2.752241in}}%
\pgfpathcurveto{\pgfqpoint{5.896328in}{2.760055in}}{\pgfqpoint{5.885729in}{2.764445in}}{\pgfqpoint{5.874679in}{2.764445in}}%
\pgfpathcurveto{\pgfqpoint{5.863628in}{2.764445in}}{\pgfqpoint{5.853029in}{2.760055in}}{\pgfqpoint{5.845216in}{2.752241in}}%
\pgfpathcurveto{\pgfqpoint{5.837402in}{2.744428in}}{\pgfqpoint{5.833012in}{2.733829in}}{\pgfqpoint{5.833012in}{2.722779in}}%
\pgfpathcurveto{\pgfqpoint{5.833012in}{2.711728in}}{\pgfqpoint{5.837402in}{2.701129in}}{\pgfqpoint{5.845216in}{2.693316in}}%
\pgfpathcurveto{\pgfqpoint{5.853029in}{2.685502in}}{\pgfqpoint{5.863628in}{2.681112in}}{\pgfqpoint{5.874679in}{2.681112in}}%
\pgfpathclose%
\pgfusepath{stroke,fill}%
\end{pgfscope}%
\begin{pgfscope}%
\pgfpathrectangle{\pgfqpoint{0.481978in}{0.331635in}}{\pgfqpoint{9.300000in}{7.700000in}}%
\pgfusepath{clip}%
\pgfsetbuttcap%
\pgfsetroundjoin%
\definecolor{currentfill}{rgb}{0.552941,0.898039,0.631373}%
\pgfsetfillcolor{currentfill}%
\pgfsetlinewidth{0.481800pt}%
\definecolor{currentstroke}{rgb}{1.000000,1.000000,1.000000}%
\pgfsetstrokecolor{currentstroke}%
\pgfsetdash{}{0pt}%
\pgfpathmoveto{\pgfqpoint{6.449863in}{3.139229in}}%
\pgfpathcurveto{\pgfqpoint{6.460914in}{3.139229in}}{\pgfqpoint{6.471513in}{3.143620in}}{\pgfqpoint{6.479326in}{3.151433in}}%
\pgfpathcurveto{\pgfqpoint{6.487140in}{3.159247in}}{\pgfqpoint{6.491530in}{3.169846in}}{\pgfqpoint{6.491530in}{3.180896in}}%
\pgfpathcurveto{\pgfqpoint{6.491530in}{3.191946in}}{\pgfqpoint{6.487140in}{3.202545in}}{\pgfqpoint{6.479326in}{3.210359in}}%
\pgfpathcurveto{\pgfqpoint{6.471513in}{3.218173in}}{\pgfqpoint{6.460914in}{3.222563in}}{\pgfqpoint{6.449863in}{3.222563in}}%
\pgfpathcurveto{\pgfqpoint{6.438813in}{3.222563in}}{\pgfqpoint{6.428214in}{3.218173in}}{\pgfqpoint{6.420401in}{3.210359in}}%
\pgfpathcurveto{\pgfqpoint{6.412587in}{3.202545in}}{\pgfqpoint{6.408197in}{3.191946in}}{\pgfqpoint{6.408197in}{3.180896in}}%
\pgfpathcurveto{\pgfqpoint{6.408197in}{3.169846in}}{\pgfqpoint{6.412587in}{3.159247in}}{\pgfqpoint{6.420401in}{3.151433in}}%
\pgfpathcurveto{\pgfqpoint{6.428214in}{3.143620in}}{\pgfqpoint{6.438813in}{3.139229in}}{\pgfqpoint{6.449863in}{3.139229in}}%
\pgfpathclose%
\pgfusepath{stroke,fill}%
\end{pgfscope}%
\begin{pgfscope}%
\pgfpathrectangle{\pgfqpoint{0.481978in}{0.331635in}}{\pgfqpoint{9.300000in}{7.700000in}}%
\pgfusepath{clip}%
\pgfsetbuttcap%
\pgfsetroundjoin%
\definecolor{currentfill}{rgb}{0.552941,0.898039,0.631373}%
\pgfsetfillcolor{currentfill}%
\pgfsetlinewidth{0.481800pt}%
\definecolor{currentstroke}{rgb}{1.000000,1.000000,1.000000}%
\pgfsetstrokecolor{currentstroke}%
\pgfsetdash{}{0pt}%
\pgfpathmoveto{\pgfqpoint{2.941597in}{6.325321in}}%
\pgfpathcurveto{\pgfqpoint{2.952647in}{6.325321in}}{\pgfqpoint{2.963246in}{6.329712in}}{\pgfqpoint{2.971060in}{6.337525in}}%
\pgfpathcurveto{\pgfqpoint{2.978874in}{6.345339in}}{\pgfqpoint{2.983264in}{6.355938in}}{\pgfqpoint{2.983264in}{6.366988in}}%
\pgfpathcurveto{\pgfqpoint{2.983264in}{6.378038in}}{\pgfqpoint{2.978874in}{6.388637in}}{\pgfqpoint{2.971060in}{6.396451in}}%
\pgfpathcurveto{\pgfqpoint{2.963246in}{6.404264in}}{\pgfqpoint{2.952647in}{6.408655in}}{\pgfqpoint{2.941597in}{6.408655in}}%
\pgfpathcurveto{\pgfqpoint{2.930547in}{6.408655in}}{\pgfqpoint{2.919948in}{6.404264in}}{\pgfqpoint{2.912135in}{6.396451in}}%
\pgfpathcurveto{\pgfqpoint{2.904321in}{6.388637in}}{\pgfqpoint{2.899931in}{6.378038in}}{\pgfqpoint{2.899931in}{6.366988in}}%
\pgfpathcurveto{\pgfqpoint{2.899931in}{6.355938in}}{\pgfqpoint{2.904321in}{6.345339in}}{\pgfqpoint{2.912135in}{6.337525in}}%
\pgfpathcurveto{\pgfqpoint{2.919948in}{6.329712in}}{\pgfqpoint{2.930547in}{6.325321in}}{\pgfqpoint{2.941597in}{6.325321in}}%
\pgfpathclose%
\pgfusepath{stroke,fill}%
\end{pgfscope}%
\begin{pgfscope}%
\pgfpathrectangle{\pgfqpoint{0.481978in}{0.331635in}}{\pgfqpoint{9.300000in}{7.700000in}}%
\pgfusepath{clip}%
\pgfsetbuttcap%
\pgfsetroundjoin%
\definecolor{currentfill}{rgb}{0.552941,0.898039,0.631373}%
\pgfsetfillcolor{currentfill}%
\pgfsetlinewidth{0.481800pt}%
\definecolor{currentstroke}{rgb}{1.000000,1.000000,1.000000}%
\pgfsetstrokecolor{currentstroke}%
\pgfsetdash{}{0pt}%
\pgfpathmoveto{\pgfqpoint{4.671581in}{6.073104in}}%
\pgfpathcurveto{\pgfqpoint{4.682631in}{6.073104in}}{\pgfqpoint{4.693230in}{6.077495in}}{\pgfqpoint{4.701044in}{6.085308in}}%
\pgfpathcurveto{\pgfqpoint{4.708857in}{6.093122in}}{\pgfqpoint{4.713248in}{6.103721in}}{\pgfqpoint{4.713248in}{6.114771in}}%
\pgfpathcurveto{\pgfqpoint{4.713248in}{6.125821in}}{\pgfqpoint{4.708857in}{6.136420in}}{\pgfqpoint{4.701044in}{6.144234in}}%
\pgfpathcurveto{\pgfqpoint{4.693230in}{6.152047in}}{\pgfqpoint{4.682631in}{6.156438in}}{\pgfqpoint{4.671581in}{6.156438in}}%
\pgfpathcurveto{\pgfqpoint{4.660531in}{6.156438in}}{\pgfqpoint{4.649932in}{6.152047in}}{\pgfqpoint{4.642118in}{6.144234in}}%
\pgfpathcurveto{\pgfqpoint{4.634304in}{6.136420in}}{\pgfqpoint{4.629914in}{6.125821in}}{\pgfqpoint{4.629914in}{6.114771in}}%
\pgfpathcurveto{\pgfqpoint{4.629914in}{6.103721in}}{\pgfqpoint{4.634304in}{6.093122in}}{\pgfqpoint{4.642118in}{6.085308in}}%
\pgfpathcurveto{\pgfqpoint{4.649932in}{6.077495in}}{\pgfqpoint{4.660531in}{6.073104in}}{\pgfqpoint{4.671581in}{6.073104in}}%
\pgfpathclose%
\pgfusepath{stroke,fill}%
\end{pgfscope}%
\begin{pgfscope}%
\pgfpathrectangle{\pgfqpoint{0.481978in}{0.331635in}}{\pgfqpoint{9.300000in}{7.700000in}}%
\pgfusepath{clip}%
\pgfsetbuttcap%
\pgfsetroundjoin%
\definecolor{currentfill}{rgb}{0.552941,0.898039,0.631373}%
\pgfsetfillcolor{currentfill}%
\pgfsetlinewidth{0.481800pt}%
\definecolor{currentstroke}{rgb}{1.000000,1.000000,1.000000}%
\pgfsetstrokecolor{currentstroke}%
\pgfsetdash{}{0pt}%
\pgfpathmoveto{\pgfqpoint{3.259508in}{2.916456in}}%
\pgfpathcurveto{\pgfqpoint{3.270559in}{2.916456in}}{\pgfqpoint{3.281158in}{2.920846in}}{\pgfqpoint{3.288971in}{2.928660in}}%
\pgfpathcurveto{\pgfqpoint{3.296785in}{2.936473in}}{\pgfqpoint{3.301175in}{2.947072in}}{\pgfqpoint{3.301175in}{2.958122in}}%
\pgfpathcurveto{\pgfqpoint{3.301175in}{2.969172in}}{\pgfqpoint{3.296785in}{2.979771in}}{\pgfqpoint{3.288971in}{2.987585in}}%
\pgfpathcurveto{\pgfqpoint{3.281158in}{2.995399in}}{\pgfqpoint{3.270559in}{2.999789in}}{\pgfqpoint{3.259508in}{2.999789in}}%
\pgfpathcurveto{\pgfqpoint{3.248458in}{2.999789in}}{\pgfqpoint{3.237859in}{2.995399in}}{\pgfqpoint{3.230046in}{2.987585in}}%
\pgfpathcurveto{\pgfqpoint{3.222232in}{2.979771in}}{\pgfqpoint{3.217842in}{2.969172in}}{\pgfqpoint{3.217842in}{2.958122in}}%
\pgfpathcurveto{\pgfqpoint{3.217842in}{2.947072in}}{\pgfqpoint{3.222232in}{2.936473in}}{\pgfqpoint{3.230046in}{2.928660in}}%
\pgfpathcurveto{\pgfqpoint{3.237859in}{2.920846in}}{\pgfqpoint{3.248458in}{2.916456in}}{\pgfqpoint{3.259508in}{2.916456in}}%
\pgfpathclose%
\pgfusepath{stroke,fill}%
\end{pgfscope}%
\begin{pgfscope}%
\pgfpathrectangle{\pgfqpoint{0.481978in}{0.331635in}}{\pgfqpoint{9.300000in}{7.700000in}}%
\pgfusepath{clip}%
\pgfsetbuttcap%
\pgfsetroundjoin%
\definecolor{currentfill}{rgb}{0.552941,0.898039,0.631373}%
\pgfsetfillcolor{currentfill}%
\pgfsetlinewidth{0.481800pt}%
\definecolor{currentstroke}{rgb}{1.000000,1.000000,1.000000}%
\pgfsetstrokecolor{currentstroke}%
\pgfsetdash{}{0pt}%
\pgfpathmoveto{\pgfqpoint{7.805062in}{3.402898in}}%
\pgfpathcurveto{\pgfqpoint{7.816112in}{3.402898in}}{\pgfqpoint{7.826711in}{3.407288in}}{\pgfqpoint{7.834525in}{3.415101in}}%
\pgfpathcurveto{\pgfqpoint{7.842338in}{3.422915in}}{\pgfqpoint{7.846728in}{3.433514in}}{\pgfqpoint{7.846728in}{3.444564in}}%
\pgfpathcurveto{\pgfqpoint{7.846728in}{3.455614in}}{\pgfqpoint{7.842338in}{3.466213in}}{\pgfqpoint{7.834525in}{3.474027in}}%
\pgfpathcurveto{\pgfqpoint{7.826711in}{3.481841in}}{\pgfqpoint{7.816112in}{3.486231in}}{\pgfqpoint{7.805062in}{3.486231in}}%
\pgfpathcurveto{\pgfqpoint{7.794012in}{3.486231in}}{\pgfqpoint{7.783413in}{3.481841in}}{\pgfqpoint{7.775599in}{3.474027in}}%
\pgfpathcurveto{\pgfqpoint{7.767785in}{3.466213in}}{\pgfqpoint{7.763395in}{3.455614in}}{\pgfqpoint{7.763395in}{3.444564in}}%
\pgfpathcurveto{\pgfqpoint{7.763395in}{3.433514in}}{\pgfqpoint{7.767785in}{3.422915in}}{\pgfqpoint{7.775599in}{3.415101in}}%
\pgfpathcurveto{\pgfqpoint{7.783413in}{3.407288in}}{\pgfqpoint{7.794012in}{3.402898in}}{\pgfqpoint{7.805062in}{3.402898in}}%
\pgfpathclose%
\pgfusepath{stroke,fill}%
\end{pgfscope}%
\begin{pgfscope}%
\pgfpathrectangle{\pgfqpoint{0.481978in}{0.331635in}}{\pgfqpoint{9.300000in}{7.700000in}}%
\pgfusepath{clip}%
\pgfsetbuttcap%
\pgfsetroundjoin%
\definecolor{currentfill}{rgb}{0.552941,0.898039,0.631373}%
\pgfsetfillcolor{currentfill}%
\pgfsetlinewidth{0.481800pt}%
\definecolor{currentstroke}{rgb}{1.000000,1.000000,1.000000}%
\pgfsetstrokecolor{currentstroke}%
\pgfsetdash{}{0pt}%
\pgfpathmoveto{\pgfqpoint{4.750388in}{7.153024in}}%
\pgfpathcurveto{\pgfqpoint{4.761438in}{7.153024in}}{\pgfqpoint{4.772037in}{7.157414in}}{\pgfqpoint{4.779850in}{7.165228in}}%
\pgfpathcurveto{\pgfqpoint{4.787664in}{7.173042in}}{\pgfqpoint{4.792054in}{7.183641in}}{\pgfqpoint{4.792054in}{7.194691in}}%
\pgfpathcurveto{\pgfqpoint{4.792054in}{7.205741in}}{\pgfqpoint{4.787664in}{7.216340in}}{\pgfqpoint{4.779850in}{7.224154in}}%
\pgfpathcurveto{\pgfqpoint{4.772037in}{7.231967in}}{\pgfqpoint{4.761438in}{7.236357in}}{\pgfqpoint{4.750388in}{7.236357in}}%
\pgfpathcurveto{\pgfqpoint{4.739338in}{7.236357in}}{\pgfqpoint{4.728739in}{7.231967in}}{\pgfqpoint{4.720925in}{7.224154in}}%
\pgfpathcurveto{\pgfqpoint{4.713111in}{7.216340in}}{\pgfqpoint{4.708721in}{7.205741in}}{\pgfqpoint{4.708721in}{7.194691in}}%
\pgfpathcurveto{\pgfqpoint{4.708721in}{7.183641in}}{\pgfqpoint{4.713111in}{7.173042in}}{\pgfqpoint{4.720925in}{7.165228in}}%
\pgfpathcurveto{\pgfqpoint{4.728739in}{7.157414in}}{\pgfqpoint{4.739338in}{7.153024in}}{\pgfqpoint{4.750388in}{7.153024in}}%
\pgfpathclose%
\pgfusepath{stroke,fill}%
\end{pgfscope}%
\begin{pgfscope}%
\pgfpathrectangle{\pgfqpoint{0.481978in}{0.331635in}}{\pgfqpoint{9.300000in}{7.700000in}}%
\pgfusepath{clip}%
\pgfsetbuttcap%
\pgfsetroundjoin%
\definecolor{currentfill}{rgb}{0.552941,0.898039,0.631373}%
\pgfsetfillcolor{currentfill}%
\pgfsetlinewidth{0.481800pt}%
\definecolor{currentstroke}{rgb}{1.000000,1.000000,1.000000}%
\pgfsetstrokecolor{currentstroke}%
\pgfsetdash{}{0pt}%
\pgfpathmoveto{\pgfqpoint{4.386424in}{5.529843in}}%
\pgfpathcurveto{\pgfqpoint{4.397474in}{5.529843in}}{\pgfqpoint{4.408073in}{5.534233in}}{\pgfqpoint{4.415887in}{5.542047in}}%
\pgfpathcurveto{\pgfqpoint{4.423700in}{5.549861in}}{\pgfqpoint{4.428091in}{5.560460in}}{\pgfqpoint{4.428091in}{5.571510in}}%
\pgfpathcurveto{\pgfqpoint{4.428091in}{5.582560in}}{\pgfqpoint{4.423700in}{5.593159in}}{\pgfqpoint{4.415887in}{5.600972in}}%
\pgfpathcurveto{\pgfqpoint{4.408073in}{5.608786in}}{\pgfqpoint{4.397474in}{5.613176in}}{\pgfqpoint{4.386424in}{5.613176in}}%
\pgfpathcurveto{\pgfqpoint{4.375374in}{5.613176in}}{\pgfqpoint{4.364775in}{5.608786in}}{\pgfqpoint{4.356961in}{5.600972in}}%
\pgfpathcurveto{\pgfqpoint{4.349148in}{5.593159in}}{\pgfqpoint{4.344757in}{5.582560in}}{\pgfqpoint{4.344757in}{5.571510in}}%
\pgfpathcurveto{\pgfqpoint{4.344757in}{5.560460in}}{\pgfqpoint{4.349148in}{5.549861in}}{\pgfqpoint{4.356961in}{5.542047in}}%
\pgfpathcurveto{\pgfqpoint{4.364775in}{5.534233in}}{\pgfqpoint{4.375374in}{5.529843in}}{\pgfqpoint{4.386424in}{5.529843in}}%
\pgfpathclose%
\pgfusepath{stroke,fill}%
\end{pgfscope}%
\begin{pgfscope}%
\pgfpathrectangle{\pgfqpoint{0.481978in}{0.331635in}}{\pgfqpoint{9.300000in}{7.700000in}}%
\pgfusepath{clip}%
\pgfsetbuttcap%
\pgfsetroundjoin%
\definecolor{currentfill}{rgb}{0.552941,0.898039,0.631373}%
\pgfsetfillcolor{currentfill}%
\pgfsetlinewidth{0.481800pt}%
\definecolor{currentstroke}{rgb}{1.000000,1.000000,1.000000}%
\pgfsetstrokecolor{currentstroke}%
\pgfsetdash{}{0pt}%
\pgfpathmoveto{\pgfqpoint{4.878045in}{6.061659in}}%
\pgfpathcurveto{\pgfqpoint{4.889095in}{6.061659in}}{\pgfqpoint{4.899694in}{6.066049in}}{\pgfqpoint{4.907508in}{6.073863in}}%
\pgfpathcurveto{\pgfqpoint{4.915321in}{6.081677in}}{\pgfqpoint{4.919712in}{6.092276in}}{\pgfqpoint{4.919712in}{6.103326in}}%
\pgfpathcurveto{\pgfqpoint{4.919712in}{6.114376in}}{\pgfqpoint{4.915321in}{6.124975in}}{\pgfqpoint{4.907508in}{6.132789in}}%
\pgfpathcurveto{\pgfqpoint{4.899694in}{6.140602in}}{\pgfqpoint{4.889095in}{6.144993in}}{\pgfqpoint{4.878045in}{6.144993in}}%
\pgfpathcurveto{\pgfqpoint{4.866995in}{6.144993in}}{\pgfqpoint{4.856396in}{6.140602in}}{\pgfqpoint{4.848582in}{6.132789in}}%
\pgfpathcurveto{\pgfqpoint{4.840769in}{6.124975in}}{\pgfqpoint{4.836378in}{6.114376in}}{\pgfqpoint{4.836378in}{6.103326in}}%
\pgfpathcurveto{\pgfqpoint{4.836378in}{6.092276in}}{\pgfqpoint{4.840769in}{6.081677in}}{\pgfqpoint{4.848582in}{6.073863in}}%
\pgfpathcurveto{\pgfqpoint{4.856396in}{6.066049in}}{\pgfqpoint{4.866995in}{6.061659in}}{\pgfqpoint{4.878045in}{6.061659in}}%
\pgfpathclose%
\pgfusepath{stroke,fill}%
\end{pgfscope}%
\begin{pgfscope}%
\pgfpathrectangle{\pgfqpoint{0.481978in}{0.331635in}}{\pgfqpoint{9.300000in}{7.700000in}}%
\pgfusepath{clip}%
\pgfsetbuttcap%
\pgfsetroundjoin%
\definecolor{currentfill}{rgb}{0.552941,0.898039,0.631373}%
\pgfsetfillcolor{currentfill}%
\pgfsetlinewidth{0.481800pt}%
\definecolor{currentstroke}{rgb}{1.000000,1.000000,1.000000}%
\pgfsetstrokecolor{currentstroke}%
\pgfsetdash{}{0pt}%
\pgfpathmoveto{\pgfqpoint{9.259474in}{5.157624in}}%
\pgfpathcurveto{\pgfqpoint{9.270524in}{5.157624in}}{\pgfqpoint{9.281123in}{5.162014in}}{\pgfqpoint{9.288937in}{5.169827in}}%
\pgfpathcurveto{\pgfqpoint{9.296750in}{5.177641in}}{\pgfqpoint{9.301140in}{5.188240in}}{\pgfqpoint{9.301140in}{5.199290in}}%
\pgfpathcurveto{\pgfqpoint{9.301140in}{5.210340in}}{\pgfqpoint{9.296750in}{5.220939in}}{\pgfqpoint{9.288937in}{5.228753in}}%
\pgfpathcurveto{\pgfqpoint{9.281123in}{5.236567in}}{\pgfqpoint{9.270524in}{5.240957in}}{\pgfqpoint{9.259474in}{5.240957in}}%
\pgfpathcurveto{\pgfqpoint{9.248424in}{5.240957in}}{\pgfqpoint{9.237825in}{5.236567in}}{\pgfqpoint{9.230011in}{5.228753in}}%
\pgfpathcurveto{\pgfqpoint{9.222197in}{5.220939in}}{\pgfqpoint{9.217807in}{5.210340in}}{\pgfqpoint{9.217807in}{5.199290in}}%
\pgfpathcurveto{\pgfqpoint{9.217807in}{5.188240in}}{\pgfqpoint{9.222197in}{5.177641in}}{\pgfqpoint{9.230011in}{5.169827in}}%
\pgfpathcurveto{\pgfqpoint{9.237825in}{5.162014in}}{\pgfqpoint{9.248424in}{5.157624in}}{\pgfqpoint{9.259474in}{5.157624in}}%
\pgfpathclose%
\pgfusepath{stroke,fill}%
\end{pgfscope}%
\begin{pgfscope}%
\pgfpathrectangle{\pgfqpoint{0.481978in}{0.331635in}}{\pgfqpoint{9.300000in}{7.700000in}}%
\pgfusepath{clip}%
\pgfsetbuttcap%
\pgfsetroundjoin%
\definecolor{currentfill}{rgb}{0.552941,0.898039,0.631373}%
\pgfsetfillcolor{currentfill}%
\pgfsetlinewidth{0.481800pt}%
\definecolor{currentstroke}{rgb}{1.000000,1.000000,1.000000}%
\pgfsetstrokecolor{currentstroke}%
\pgfsetdash{}{0pt}%
\pgfpathmoveto{\pgfqpoint{8.569592in}{4.750652in}}%
\pgfpathcurveto{\pgfqpoint{8.580643in}{4.750652in}}{\pgfqpoint{8.591242in}{4.755042in}}{\pgfqpoint{8.599055in}{4.762856in}}%
\pgfpathcurveto{\pgfqpoint{8.606869in}{4.770670in}}{\pgfqpoint{8.611259in}{4.781269in}}{\pgfqpoint{8.611259in}{4.792319in}}%
\pgfpathcurveto{\pgfqpoint{8.611259in}{4.803369in}}{\pgfqpoint{8.606869in}{4.813968in}}{\pgfqpoint{8.599055in}{4.821782in}}%
\pgfpathcurveto{\pgfqpoint{8.591242in}{4.829595in}}{\pgfqpoint{8.580643in}{4.833985in}}{\pgfqpoint{8.569592in}{4.833985in}}%
\pgfpathcurveto{\pgfqpoint{8.558542in}{4.833985in}}{\pgfqpoint{8.547943in}{4.829595in}}{\pgfqpoint{8.540130in}{4.821782in}}%
\pgfpathcurveto{\pgfqpoint{8.532316in}{4.813968in}}{\pgfqpoint{8.527926in}{4.803369in}}{\pgfqpoint{8.527926in}{4.792319in}}%
\pgfpathcurveto{\pgfqpoint{8.527926in}{4.781269in}}{\pgfqpoint{8.532316in}{4.770670in}}{\pgfqpoint{8.540130in}{4.762856in}}%
\pgfpathcurveto{\pgfqpoint{8.547943in}{4.755042in}}{\pgfqpoint{8.558542in}{4.750652in}}{\pgfqpoint{8.569592in}{4.750652in}}%
\pgfpathclose%
\pgfusepath{stroke,fill}%
\end{pgfscope}%
\begin{pgfscope}%
\pgfpathrectangle{\pgfqpoint{0.481978in}{0.331635in}}{\pgfqpoint{9.300000in}{7.700000in}}%
\pgfusepath{clip}%
\pgfsetbuttcap%
\pgfsetroundjoin%
\definecolor{currentfill}{rgb}{0.552941,0.898039,0.631373}%
\pgfsetfillcolor{currentfill}%
\pgfsetlinewidth{0.481800pt}%
\definecolor{currentstroke}{rgb}{1.000000,1.000000,1.000000}%
\pgfsetstrokecolor{currentstroke}%
\pgfsetdash{}{0pt}%
\pgfpathmoveto{\pgfqpoint{8.786772in}{4.529927in}}%
\pgfpathcurveto{\pgfqpoint{8.797823in}{4.529927in}}{\pgfqpoint{8.808422in}{4.534317in}}{\pgfqpoint{8.816235in}{4.542131in}}%
\pgfpathcurveto{\pgfqpoint{8.824049in}{4.549944in}}{\pgfqpoint{8.828439in}{4.560543in}}{\pgfqpoint{8.828439in}{4.571593in}}%
\pgfpathcurveto{\pgfqpoint{8.828439in}{4.582644in}}{\pgfqpoint{8.824049in}{4.593243in}}{\pgfqpoint{8.816235in}{4.601056in}}%
\pgfpathcurveto{\pgfqpoint{8.808422in}{4.608870in}}{\pgfqpoint{8.797823in}{4.613260in}}{\pgfqpoint{8.786772in}{4.613260in}}%
\pgfpathcurveto{\pgfqpoint{8.775722in}{4.613260in}}{\pgfqpoint{8.765123in}{4.608870in}}{\pgfqpoint{8.757310in}{4.601056in}}%
\pgfpathcurveto{\pgfqpoint{8.749496in}{4.593243in}}{\pgfqpoint{8.745106in}{4.582644in}}{\pgfqpoint{8.745106in}{4.571593in}}%
\pgfpathcurveto{\pgfqpoint{8.745106in}{4.560543in}}{\pgfqpoint{8.749496in}{4.549944in}}{\pgfqpoint{8.757310in}{4.542131in}}%
\pgfpathcurveto{\pgfqpoint{8.765123in}{4.534317in}}{\pgfqpoint{8.775722in}{4.529927in}}{\pgfqpoint{8.786772in}{4.529927in}}%
\pgfpathclose%
\pgfusepath{stroke,fill}%
\end{pgfscope}%
\begin{pgfscope}%
\pgfpathrectangle{\pgfqpoint{0.481978in}{0.331635in}}{\pgfqpoint{9.300000in}{7.700000in}}%
\pgfusepath{clip}%
\pgfsetbuttcap%
\pgfsetroundjoin%
\definecolor{currentfill}{rgb}{0.552941,0.898039,0.631373}%
\pgfsetfillcolor{currentfill}%
\pgfsetlinewidth{0.481800pt}%
\definecolor{currentstroke}{rgb}{1.000000,1.000000,1.000000}%
\pgfsetstrokecolor{currentstroke}%
\pgfsetdash{}{0pt}%
\pgfpathmoveto{\pgfqpoint{7.733235in}{4.864578in}}%
\pgfpathcurveto{\pgfqpoint{7.744285in}{4.864578in}}{\pgfqpoint{7.754884in}{4.868968in}}{\pgfqpoint{7.762698in}{4.876782in}}%
\pgfpathcurveto{\pgfqpoint{7.770512in}{4.884595in}}{\pgfqpoint{7.774902in}{4.895194in}}{\pgfqpoint{7.774902in}{4.906244in}}%
\pgfpathcurveto{\pgfqpoint{7.774902in}{4.917294in}}{\pgfqpoint{7.770512in}{4.927893in}}{\pgfqpoint{7.762698in}{4.935707in}}%
\pgfpathcurveto{\pgfqpoint{7.754884in}{4.943521in}}{\pgfqpoint{7.744285in}{4.947911in}}{\pgfqpoint{7.733235in}{4.947911in}}%
\pgfpathcurveto{\pgfqpoint{7.722185in}{4.947911in}}{\pgfqpoint{7.711586in}{4.943521in}}{\pgfqpoint{7.703772in}{4.935707in}}%
\pgfpathcurveto{\pgfqpoint{7.695959in}{4.927893in}}{\pgfqpoint{7.691568in}{4.917294in}}{\pgfqpoint{7.691568in}{4.906244in}}%
\pgfpathcurveto{\pgfqpoint{7.691568in}{4.895194in}}{\pgfqpoint{7.695959in}{4.884595in}}{\pgfqpoint{7.703772in}{4.876782in}}%
\pgfpathcurveto{\pgfqpoint{7.711586in}{4.868968in}}{\pgfqpoint{7.722185in}{4.864578in}}{\pgfqpoint{7.733235in}{4.864578in}}%
\pgfpathclose%
\pgfusepath{stroke,fill}%
\end{pgfscope}%
\begin{pgfscope}%
\pgfpathrectangle{\pgfqpoint{0.481978in}{0.331635in}}{\pgfqpoint{9.300000in}{7.700000in}}%
\pgfusepath{clip}%
\pgfsetbuttcap%
\pgfsetroundjoin%
\definecolor{currentfill}{rgb}{0.552941,0.898039,0.631373}%
\pgfsetfillcolor{currentfill}%
\pgfsetlinewidth{0.481800pt}%
\definecolor{currentstroke}{rgb}{1.000000,1.000000,1.000000}%
\pgfsetstrokecolor{currentstroke}%
\pgfsetdash{}{0pt}%
\pgfpathmoveto{\pgfqpoint{5.023248in}{5.790519in}}%
\pgfpathcurveto{\pgfqpoint{5.034298in}{5.790519in}}{\pgfqpoint{5.044897in}{5.794910in}}{\pgfqpoint{5.052710in}{5.802723in}}%
\pgfpathcurveto{\pgfqpoint{5.060524in}{5.810537in}}{\pgfqpoint{5.064914in}{5.821136in}}{\pgfqpoint{5.064914in}{5.832186in}}%
\pgfpathcurveto{\pgfqpoint{5.064914in}{5.843236in}}{\pgfqpoint{5.060524in}{5.853835in}}{\pgfqpoint{5.052710in}{5.861649in}}%
\pgfpathcurveto{\pgfqpoint{5.044897in}{5.869462in}}{\pgfqpoint{5.034298in}{5.873853in}}{\pgfqpoint{5.023248in}{5.873853in}}%
\pgfpathcurveto{\pgfqpoint{5.012198in}{5.873853in}}{\pgfqpoint{5.001599in}{5.869462in}}{\pgfqpoint{4.993785in}{5.861649in}}%
\pgfpathcurveto{\pgfqpoint{4.985971in}{5.853835in}}{\pgfqpoint{4.981581in}{5.843236in}}{\pgfqpoint{4.981581in}{5.832186in}}%
\pgfpathcurveto{\pgfqpoint{4.981581in}{5.821136in}}{\pgfqpoint{4.985971in}{5.810537in}}{\pgfqpoint{4.993785in}{5.802723in}}%
\pgfpathcurveto{\pgfqpoint{5.001599in}{5.794910in}}{\pgfqpoint{5.012198in}{5.790519in}}{\pgfqpoint{5.023248in}{5.790519in}}%
\pgfpathclose%
\pgfusepath{stroke,fill}%
\end{pgfscope}%
\begin{pgfscope}%
\pgfpathrectangle{\pgfqpoint{0.481978in}{0.331635in}}{\pgfqpoint{9.300000in}{7.700000in}}%
\pgfusepath{clip}%
\pgfsetbuttcap%
\pgfsetroundjoin%
\definecolor{currentfill}{rgb}{0.552941,0.898039,0.631373}%
\pgfsetfillcolor{currentfill}%
\pgfsetlinewidth{0.481800pt}%
\definecolor{currentstroke}{rgb}{1.000000,1.000000,1.000000}%
\pgfsetstrokecolor{currentstroke}%
\pgfsetdash{}{0pt}%
\pgfpathmoveto{\pgfqpoint{4.955252in}{6.232287in}}%
\pgfpathcurveto{\pgfqpoint{4.966302in}{6.232287in}}{\pgfqpoint{4.976901in}{6.236677in}}{\pgfqpoint{4.984715in}{6.244490in}}%
\pgfpathcurveto{\pgfqpoint{4.992528in}{6.252304in}}{\pgfqpoint{4.996918in}{6.262903in}}{\pgfqpoint{4.996918in}{6.273953in}}%
\pgfpathcurveto{\pgfqpoint{4.996918in}{6.285003in}}{\pgfqpoint{4.992528in}{6.295602in}}{\pgfqpoint{4.984715in}{6.303416in}}%
\pgfpathcurveto{\pgfqpoint{4.976901in}{6.311230in}}{\pgfqpoint{4.966302in}{6.315620in}}{\pgfqpoint{4.955252in}{6.315620in}}%
\pgfpathcurveto{\pgfqpoint{4.944202in}{6.315620in}}{\pgfqpoint{4.933603in}{6.311230in}}{\pgfqpoint{4.925789in}{6.303416in}}%
\pgfpathcurveto{\pgfqpoint{4.917975in}{6.295602in}}{\pgfqpoint{4.913585in}{6.285003in}}{\pgfqpoint{4.913585in}{6.273953in}}%
\pgfpathcurveto{\pgfqpoint{4.913585in}{6.262903in}}{\pgfqpoint{4.917975in}{6.252304in}}{\pgfqpoint{4.925789in}{6.244490in}}%
\pgfpathcurveto{\pgfqpoint{4.933603in}{6.236677in}}{\pgfqpoint{4.944202in}{6.232287in}}{\pgfqpoint{4.955252in}{6.232287in}}%
\pgfpathclose%
\pgfusepath{stroke,fill}%
\end{pgfscope}%
\begin{pgfscope}%
\pgfpathrectangle{\pgfqpoint{0.481978in}{0.331635in}}{\pgfqpoint{9.300000in}{7.700000in}}%
\pgfusepath{clip}%
\pgfsetbuttcap%
\pgfsetroundjoin%
\definecolor{currentfill}{rgb}{0.552941,0.898039,0.631373}%
\pgfsetfillcolor{currentfill}%
\pgfsetlinewidth{0.481800pt}%
\definecolor{currentstroke}{rgb}{1.000000,1.000000,1.000000}%
\pgfsetstrokecolor{currentstroke}%
\pgfsetdash{}{0pt}%
\pgfpathmoveto{\pgfqpoint{7.433098in}{2.223941in}}%
\pgfpathcurveto{\pgfqpoint{7.444148in}{2.223941in}}{\pgfqpoint{7.454747in}{2.228332in}}{\pgfqpoint{7.462561in}{2.236145in}}%
\pgfpathcurveto{\pgfqpoint{7.470375in}{2.243959in}}{\pgfqpoint{7.474765in}{2.254558in}}{\pgfqpoint{7.474765in}{2.265608in}}%
\pgfpathcurveto{\pgfqpoint{7.474765in}{2.276658in}}{\pgfqpoint{7.470375in}{2.287257in}}{\pgfqpoint{7.462561in}{2.295071in}}%
\pgfpathcurveto{\pgfqpoint{7.454747in}{2.302884in}}{\pgfqpoint{7.444148in}{2.307275in}}{\pgfqpoint{7.433098in}{2.307275in}}%
\pgfpathcurveto{\pgfqpoint{7.422048in}{2.307275in}}{\pgfqpoint{7.411449in}{2.302884in}}{\pgfqpoint{7.403635in}{2.295071in}}%
\pgfpathcurveto{\pgfqpoint{7.395822in}{2.287257in}}{\pgfqpoint{7.391431in}{2.276658in}}{\pgfqpoint{7.391431in}{2.265608in}}%
\pgfpathcurveto{\pgfqpoint{7.391431in}{2.254558in}}{\pgfqpoint{7.395822in}{2.243959in}}{\pgfqpoint{7.403635in}{2.236145in}}%
\pgfpathcurveto{\pgfqpoint{7.411449in}{2.228332in}}{\pgfqpoint{7.422048in}{2.223941in}}{\pgfqpoint{7.433098in}{2.223941in}}%
\pgfpathclose%
\pgfusepath{stroke,fill}%
\end{pgfscope}%
\begin{pgfscope}%
\pgfpathrectangle{\pgfqpoint{0.481978in}{0.331635in}}{\pgfqpoint{9.300000in}{7.700000in}}%
\pgfusepath{clip}%
\pgfsetbuttcap%
\pgfsetroundjoin%
\definecolor{currentfill}{rgb}{0.552941,0.898039,0.631373}%
\pgfsetfillcolor{currentfill}%
\pgfsetlinewidth{0.481800pt}%
\definecolor{currentstroke}{rgb}{1.000000,1.000000,1.000000}%
\pgfsetstrokecolor{currentstroke}%
\pgfsetdash{}{0pt}%
\pgfpathmoveto{\pgfqpoint{8.232070in}{5.773420in}}%
\pgfpathcurveto{\pgfqpoint{8.243120in}{5.773420in}}{\pgfqpoint{8.253719in}{5.777810in}}{\pgfqpoint{8.261532in}{5.785624in}}%
\pgfpathcurveto{\pgfqpoint{8.269346in}{5.793438in}}{\pgfqpoint{8.273736in}{5.804037in}}{\pgfqpoint{8.273736in}{5.815087in}}%
\pgfpathcurveto{\pgfqpoint{8.273736in}{5.826137in}}{\pgfqpoint{8.269346in}{5.836736in}}{\pgfqpoint{8.261532in}{5.844550in}}%
\pgfpathcurveto{\pgfqpoint{8.253719in}{5.852363in}}{\pgfqpoint{8.243120in}{5.856753in}}{\pgfqpoint{8.232070in}{5.856753in}}%
\pgfpathcurveto{\pgfqpoint{8.221019in}{5.856753in}}{\pgfqpoint{8.210420in}{5.852363in}}{\pgfqpoint{8.202607in}{5.844550in}}%
\pgfpathcurveto{\pgfqpoint{8.194793in}{5.836736in}}{\pgfqpoint{8.190403in}{5.826137in}}{\pgfqpoint{8.190403in}{5.815087in}}%
\pgfpathcurveto{\pgfqpoint{8.190403in}{5.804037in}}{\pgfqpoint{8.194793in}{5.793438in}}{\pgfqpoint{8.202607in}{5.785624in}}%
\pgfpathcurveto{\pgfqpoint{8.210420in}{5.777810in}}{\pgfqpoint{8.221019in}{5.773420in}}{\pgfqpoint{8.232070in}{5.773420in}}%
\pgfpathclose%
\pgfusepath{stroke,fill}%
\end{pgfscope}%
\begin{pgfscope}%
\pgfpathrectangle{\pgfqpoint{0.481978in}{0.331635in}}{\pgfqpoint{9.300000in}{7.700000in}}%
\pgfusepath{clip}%
\pgfsetbuttcap%
\pgfsetroundjoin%
\definecolor{currentfill}{rgb}{0.552941,0.898039,0.631373}%
\pgfsetfillcolor{currentfill}%
\pgfsetlinewidth{0.481800pt}%
\definecolor{currentstroke}{rgb}{1.000000,1.000000,1.000000}%
\pgfsetstrokecolor{currentstroke}%
\pgfsetdash{}{0pt}%
\pgfpathmoveto{\pgfqpoint{3.011222in}{1.643530in}}%
\pgfpathcurveto{\pgfqpoint{3.022272in}{1.643530in}}{\pgfqpoint{3.032871in}{1.647920in}}{\pgfqpoint{3.040685in}{1.655734in}}%
\pgfpathcurveto{\pgfqpoint{3.048498in}{1.663547in}}{\pgfqpoint{3.052889in}{1.674146in}}{\pgfqpoint{3.052889in}{1.685197in}}%
\pgfpathcurveto{\pgfqpoint{3.052889in}{1.696247in}}{\pgfqpoint{3.048498in}{1.706846in}}{\pgfqpoint{3.040685in}{1.714659in}}%
\pgfpathcurveto{\pgfqpoint{3.032871in}{1.722473in}}{\pgfqpoint{3.022272in}{1.726863in}}{\pgfqpoint{3.011222in}{1.726863in}}%
\pgfpathcurveto{\pgfqpoint{3.000172in}{1.726863in}}{\pgfqpoint{2.989573in}{1.722473in}}{\pgfqpoint{2.981759in}{1.714659in}}%
\pgfpathcurveto{\pgfqpoint{2.973946in}{1.706846in}}{\pgfqpoint{2.969555in}{1.696247in}}{\pgfqpoint{2.969555in}{1.685197in}}%
\pgfpathcurveto{\pgfqpoint{2.969555in}{1.674146in}}{\pgfqpoint{2.973946in}{1.663547in}}{\pgfqpoint{2.981759in}{1.655734in}}%
\pgfpathcurveto{\pgfqpoint{2.989573in}{1.647920in}}{\pgfqpoint{3.000172in}{1.643530in}}{\pgfqpoint{3.011222in}{1.643530in}}%
\pgfpathclose%
\pgfusepath{stroke,fill}%
\end{pgfscope}%
\begin{pgfscope}%
\pgfpathrectangle{\pgfqpoint{0.481978in}{0.331635in}}{\pgfqpoint{9.300000in}{7.700000in}}%
\pgfusepath{clip}%
\pgfsetbuttcap%
\pgfsetroundjoin%
\definecolor{currentfill}{rgb}{0.552941,0.898039,0.631373}%
\pgfsetfillcolor{currentfill}%
\pgfsetlinewidth{0.481800pt}%
\definecolor{currentstroke}{rgb}{1.000000,1.000000,1.000000}%
\pgfsetstrokecolor{currentstroke}%
\pgfsetdash{}{0pt}%
\pgfpathmoveto{\pgfqpoint{4.782892in}{6.107152in}}%
\pgfpathcurveto{\pgfqpoint{4.793942in}{6.107152in}}{\pgfqpoint{4.804541in}{6.111542in}}{\pgfqpoint{4.812355in}{6.119355in}}%
\pgfpathcurveto{\pgfqpoint{4.820169in}{6.127169in}}{\pgfqpoint{4.824559in}{6.137768in}}{\pgfqpoint{4.824559in}{6.148818in}}%
\pgfpathcurveto{\pgfqpoint{4.824559in}{6.159868in}}{\pgfqpoint{4.820169in}{6.170467in}}{\pgfqpoint{4.812355in}{6.178281in}}%
\pgfpathcurveto{\pgfqpoint{4.804541in}{6.186095in}}{\pgfqpoint{4.793942in}{6.190485in}}{\pgfqpoint{4.782892in}{6.190485in}}%
\pgfpathcurveto{\pgfqpoint{4.771842in}{6.190485in}}{\pgfqpoint{4.761243in}{6.186095in}}{\pgfqpoint{4.753429in}{6.178281in}}%
\pgfpathcurveto{\pgfqpoint{4.745616in}{6.170467in}}{\pgfqpoint{4.741225in}{6.159868in}}{\pgfqpoint{4.741225in}{6.148818in}}%
\pgfpathcurveto{\pgfqpoint{4.741225in}{6.137768in}}{\pgfqpoint{4.745616in}{6.127169in}}{\pgfqpoint{4.753429in}{6.119355in}}%
\pgfpathcurveto{\pgfqpoint{4.761243in}{6.111542in}}{\pgfqpoint{4.771842in}{6.107152in}}{\pgfqpoint{4.782892in}{6.107152in}}%
\pgfpathclose%
\pgfusepath{stroke,fill}%
\end{pgfscope}%
\begin{pgfscope}%
\pgfpathrectangle{\pgfqpoint{0.481978in}{0.331635in}}{\pgfqpoint{9.300000in}{7.700000in}}%
\pgfusepath{clip}%
\pgfsetbuttcap%
\pgfsetroundjoin%
\definecolor{currentfill}{rgb}{0.552941,0.898039,0.631373}%
\pgfsetfillcolor{currentfill}%
\pgfsetlinewidth{0.481800pt}%
\definecolor{currentstroke}{rgb}{1.000000,1.000000,1.000000}%
\pgfsetstrokecolor{currentstroke}%
\pgfsetdash{}{0pt}%
\pgfpathmoveto{\pgfqpoint{3.628307in}{5.255411in}}%
\pgfpathcurveto{\pgfqpoint{3.639357in}{5.255411in}}{\pgfqpoint{3.649956in}{5.259801in}}{\pgfqpoint{3.657770in}{5.267615in}}%
\pgfpathcurveto{\pgfqpoint{3.665584in}{5.275428in}}{\pgfqpoint{3.669974in}{5.286028in}}{\pgfqpoint{3.669974in}{5.297078in}}%
\pgfpathcurveto{\pgfqpoint{3.669974in}{5.308128in}}{\pgfqpoint{3.665584in}{5.318727in}}{\pgfqpoint{3.657770in}{5.326540in}}%
\pgfpathcurveto{\pgfqpoint{3.649956in}{5.334354in}}{\pgfqpoint{3.639357in}{5.338744in}}{\pgfqpoint{3.628307in}{5.338744in}}%
\pgfpathcurveto{\pgfqpoint{3.617257in}{5.338744in}}{\pgfqpoint{3.606658in}{5.334354in}}{\pgfqpoint{3.598845in}{5.326540in}}%
\pgfpathcurveto{\pgfqpoint{3.591031in}{5.318727in}}{\pgfqpoint{3.586641in}{5.308128in}}{\pgfqpoint{3.586641in}{5.297078in}}%
\pgfpathcurveto{\pgfqpoint{3.586641in}{5.286028in}}{\pgfqpoint{3.591031in}{5.275428in}}{\pgfqpoint{3.598845in}{5.267615in}}%
\pgfpathcurveto{\pgfqpoint{3.606658in}{5.259801in}}{\pgfqpoint{3.617257in}{5.255411in}}{\pgfqpoint{3.628307in}{5.255411in}}%
\pgfpathclose%
\pgfusepath{stroke,fill}%
\end{pgfscope}%
\begin{pgfscope}%
\pgfpathrectangle{\pgfqpoint{0.481978in}{0.331635in}}{\pgfqpoint{9.300000in}{7.700000in}}%
\pgfusepath{clip}%
\pgfsetbuttcap%
\pgfsetroundjoin%
\definecolor{currentfill}{rgb}{0.552941,0.898039,0.631373}%
\pgfsetfillcolor{currentfill}%
\pgfsetlinewidth{0.481800pt}%
\definecolor{currentstroke}{rgb}{1.000000,1.000000,1.000000}%
\pgfsetstrokecolor{currentstroke}%
\pgfsetdash{}{0pt}%
\pgfpathmoveto{\pgfqpoint{3.687739in}{5.199691in}}%
\pgfpathcurveto{\pgfqpoint{3.698789in}{5.199691in}}{\pgfqpoint{3.709388in}{5.204081in}}{\pgfqpoint{3.717201in}{5.211895in}}%
\pgfpathcurveto{\pgfqpoint{3.725015in}{5.219708in}}{\pgfqpoint{3.729405in}{5.230307in}}{\pgfqpoint{3.729405in}{5.241357in}}%
\pgfpathcurveto{\pgfqpoint{3.729405in}{5.252408in}}{\pgfqpoint{3.725015in}{5.263007in}}{\pgfqpoint{3.717201in}{5.270820in}}%
\pgfpathcurveto{\pgfqpoint{3.709388in}{5.278634in}}{\pgfqpoint{3.698789in}{5.283024in}}{\pgfqpoint{3.687739in}{5.283024in}}%
\pgfpathcurveto{\pgfqpoint{3.676689in}{5.283024in}}{\pgfqpoint{3.666090in}{5.278634in}}{\pgfqpoint{3.658276in}{5.270820in}}%
\pgfpathcurveto{\pgfqpoint{3.650462in}{5.263007in}}{\pgfqpoint{3.646072in}{5.252408in}}{\pgfqpoint{3.646072in}{5.241357in}}%
\pgfpathcurveto{\pgfqpoint{3.646072in}{5.230307in}}{\pgfqpoint{3.650462in}{5.219708in}}{\pgfqpoint{3.658276in}{5.211895in}}%
\pgfpathcurveto{\pgfqpoint{3.666090in}{5.204081in}}{\pgfqpoint{3.676689in}{5.199691in}}{\pgfqpoint{3.687739in}{5.199691in}}%
\pgfpathclose%
\pgfusepath{stroke,fill}%
\end{pgfscope}%
\begin{pgfscope}%
\pgfpathrectangle{\pgfqpoint{0.481978in}{0.331635in}}{\pgfqpoint{9.300000in}{7.700000in}}%
\pgfusepath{clip}%
\pgfsetbuttcap%
\pgfsetroundjoin%
\definecolor{currentfill}{rgb}{0.552941,0.898039,0.631373}%
\pgfsetfillcolor{currentfill}%
\pgfsetlinewidth{0.481800pt}%
\definecolor{currentstroke}{rgb}{1.000000,1.000000,1.000000}%
\pgfsetstrokecolor{currentstroke}%
\pgfsetdash{}{0pt}%
\pgfpathmoveto{\pgfqpoint{7.445589in}{2.785107in}}%
\pgfpathcurveto{\pgfqpoint{7.456639in}{2.785107in}}{\pgfqpoint{7.467238in}{2.789497in}}{\pgfqpoint{7.475051in}{2.797311in}}%
\pgfpathcurveto{\pgfqpoint{7.482865in}{2.805124in}}{\pgfqpoint{7.487255in}{2.815723in}}{\pgfqpoint{7.487255in}{2.826774in}}%
\pgfpathcurveto{\pgfqpoint{7.487255in}{2.837824in}}{\pgfqpoint{7.482865in}{2.848423in}}{\pgfqpoint{7.475051in}{2.856236in}}%
\pgfpathcurveto{\pgfqpoint{7.467238in}{2.864050in}}{\pgfqpoint{7.456639in}{2.868440in}}{\pgfqpoint{7.445589in}{2.868440in}}%
\pgfpathcurveto{\pgfqpoint{7.434539in}{2.868440in}}{\pgfqpoint{7.423940in}{2.864050in}}{\pgfqpoint{7.416126in}{2.856236in}}%
\pgfpathcurveto{\pgfqpoint{7.408312in}{2.848423in}}{\pgfqpoint{7.403922in}{2.837824in}}{\pgfqpoint{7.403922in}{2.826774in}}%
\pgfpathcurveto{\pgfqpoint{7.403922in}{2.815723in}}{\pgfqpoint{7.408312in}{2.805124in}}{\pgfqpoint{7.416126in}{2.797311in}}%
\pgfpathcurveto{\pgfqpoint{7.423940in}{2.789497in}}{\pgfqpoint{7.434539in}{2.785107in}}{\pgfqpoint{7.445589in}{2.785107in}}%
\pgfpathclose%
\pgfusepath{stroke,fill}%
\end{pgfscope}%
\begin{pgfscope}%
\pgfpathrectangle{\pgfqpoint{0.481978in}{0.331635in}}{\pgfqpoint{9.300000in}{7.700000in}}%
\pgfusepath{clip}%
\pgfsetbuttcap%
\pgfsetroundjoin%
\definecolor{currentfill}{rgb}{0.552941,0.898039,0.631373}%
\pgfsetfillcolor{currentfill}%
\pgfsetlinewidth{0.481800pt}%
\definecolor{currentstroke}{rgb}{1.000000,1.000000,1.000000}%
\pgfsetstrokecolor{currentstroke}%
\pgfsetdash{}{0pt}%
\pgfpathmoveto{\pgfqpoint{8.934791in}{5.101472in}}%
\pgfpathcurveto{\pgfqpoint{8.945841in}{5.101472in}}{\pgfqpoint{8.956440in}{5.105862in}}{\pgfqpoint{8.964253in}{5.113676in}}%
\pgfpathcurveto{\pgfqpoint{8.972067in}{5.121489in}}{\pgfqpoint{8.976457in}{5.132088in}}{\pgfqpoint{8.976457in}{5.143139in}}%
\pgfpathcurveto{\pgfqpoint{8.976457in}{5.154189in}}{\pgfqpoint{8.972067in}{5.164788in}}{\pgfqpoint{8.964253in}{5.172601in}}%
\pgfpathcurveto{\pgfqpoint{8.956440in}{5.180415in}}{\pgfqpoint{8.945841in}{5.184805in}}{\pgfqpoint{8.934791in}{5.184805in}}%
\pgfpathcurveto{\pgfqpoint{8.923741in}{5.184805in}}{\pgfqpoint{8.913142in}{5.180415in}}{\pgfqpoint{8.905328in}{5.172601in}}%
\pgfpathcurveto{\pgfqpoint{8.897514in}{5.164788in}}{\pgfqpoint{8.893124in}{5.154189in}}{\pgfqpoint{8.893124in}{5.143139in}}%
\pgfpathcurveto{\pgfqpoint{8.893124in}{5.132088in}}{\pgfqpoint{8.897514in}{5.121489in}}{\pgfqpoint{8.905328in}{5.113676in}}%
\pgfpathcurveto{\pgfqpoint{8.913142in}{5.105862in}}{\pgfqpoint{8.923741in}{5.101472in}}{\pgfqpoint{8.934791in}{5.101472in}}%
\pgfpathclose%
\pgfusepath{stroke,fill}%
\end{pgfscope}%
\begin{pgfscope}%
\pgfpathrectangle{\pgfqpoint{0.481978in}{0.331635in}}{\pgfqpoint{9.300000in}{7.700000in}}%
\pgfusepath{clip}%
\pgfsetbuttcap%
\pgfsetroundjoin%
\definecolor{currentfill}{rgb}{0.552941,0.898039,0.631373}%
\pgfsetfillcolor{currentfill}%
\pgfsetlinewidth{0.481800pt}%
\definecolor{currentstroke}{rgb}{1.000000,1.000000,1.000000}%
\pgfsetstrokecolor{currentstroke}%
\pgfsetdash{}{0pt}%
\pgfpathmoveto{\pgfqpoint{3.508572in}{3.680026in}}%
\pgfpathcurveto{\pgfqpoint{3.519622in}{3.680026in}}{\pgfqpoint{3.530221in}{3.684417in}}{\pgfqpoint{3.538035in}{3.692230in}}%
\pgfpathcurveto{\pgfqpoint{3.545848in}{3.700044in}}{\pgfqpoint{3.550239in}{3.710643in}}{\pgfqpoint{3.550239in}{3.721693in}}%
\pgfpathcurveto{\pgfqpoint{3.550239in}{3.732743in}}{\pgfqpoint{3.545848in}{3.743342in}}{\pgfqpoint{3.538035in}{3.751156in}}%
\pgfpathcurveto{\pgfqpoint{3.530221in}{3.758969in}}{\pgfqpoint{3.519622in}{3.763360in}}{\pgfqpoint{3.508572in}{3.763360in}}%
\pgfpathcurveto{\pgfqpoint{3.497522in}{3.763360in}}{\pgfqpoint{3.486923in}{3.758969in}}{\pgfqpoint{3.479109in}{3.751156in}}%
\pgfpathcurveto{\pgfqpoint{3.471295in}{3.743342in}}{\pgfqpoint{3.466905in}{3.732743in}}{\pgfqpoint{3.466905in}{3.721693in}}%
\pgfpathcurveto{\pgfqpoint{3.466905in}{3.710643in}}{\pgfqpoint{3.471295in}{3.700044in}}{\pgfqpoint{3.479109in}{3.692230in}}%
\pgfpathcurveto{\pgfqpoint{3.486923in}{3.684417in}}{\pgfqpoint{3.497522in}{3.680026in}}{\pgfqpoint{3.508572in}{3.680026in}}%
\pgfpathclose%
\pgfusepath{stroke,fill}%
\end{pgfscope}%
\begin{pgfscope}%
\pgfpathrectangle{\pgfqpoint{0.481978in}{0.331635in}}{\pgfqpoint{9.300000in}{7.700000in}}%
\pgfusepath{clip}%
\pgfsetbuttcap%
\pgfsetroundjoin%
\definecolor{currentfill}{rgb}{0.552941,0.898039,0.631373}%
\pgfsetfillcolor{currentfill}%
\pgfsetlinewidth{0.481800pt}%
\definecolor{currentstroke}{rgb}{1.000000,1.000000,1.000000}%
\pgfsetstrokecolor{currentstroke}%
\pgfsetdash{}{0pt}%
\pgfpathmoveto{\pgfqpoint{4.192874in}{1.517858in}}%
\pgfpathcurveto{\pgfqpoint{4.203924in}{1.517858in}}{\pgfqpoint{4.214523in}{1.522248in}}{\pgfqpoint{4.222337in}{1.530062in}}%
\pgfpathcurveto{\pgfqpoint{4.230151in}{1.537875in}}{\pgfqpoint{4.234541in}{1.548474in}}{\pgfqpoint{4.234541in}{1.559524in}}%
\pgfpathcurveto{\pgfqpoint{4.234541in}{1.570575in}}{\pgfqpoint{4.230151in}{1.581174in}}{\pgfqpoint{4.222337in}{1.588987in}}%
\pgfpathcurveto{\pgfqpoint{4.214523in}{1.596801in}}{\pgfqpoint{4.203924in}{1.601191in}}{\pgfqpoint{4.192874in}{1.601191in}}%
\pgfpathcurveto{\pgfqpoint{4.181824in}{1.601191in}}{\pgfqpoint{4.171225in}{1.596801in}}{\pgfqpoint{4.163411in}{1.588987in}}%
\pgfpathcurveto{\pgfqpoint{4.155598in}{1.581174in}}{\pgfqpoint{4.151207in}{1.570575in}}{\pgfqpoint{4.151207in}{1.559524in}}%
\pgfpathcurveto{\pgfqpoint{4.151207in}{1.548474in}}{\pgfqpoint{4.155598in}{1.537875in}}{\pgfqpoint{4.163411in}{1.530062in}}%
\pgfpathcurveto{\pgfqpoint{4.171225in}{1.522248in}}{\pgfqpoint{4.181824in}{1.517858in}}{\pgfqpoint{4.192874in}{1.517858in}}%
\pgfpathclose%
\pgfusepath{stroke,fill}%
\end{pgfscope}%
\begin{pgfscope}%
\pgfpathrectangle{\pgfqpoint{0.481978in}{0.331635in}}{\pgfqpoint{9.300000in}{7.700000in}}%
\pgfusepath{clip}%
\pgfsetbuttcap%
\pgfsetroundjoin%
\definecolor{currentfill}{rgb}{0.552941,0.898039,0.631373}%
\pgfsetfillcolor{currentfill}%
\pgfsetlinewidth{0.481800pt}%
\definecolor{currentstroke}{rgb}{1.000000,1.000000,1.000000}%
\pgfsetstrokecolor{currentstroke}%
\pgfsetdash{}{0pt}%
\pgfpathmoveto{\pgfqpoint{4.001930in}{4.990969in}}%
\pgfpathcurveto{\pgfqpoint{4.012980in}{4.990969in}}{\pgfqpoint{4.023579in}{4.995360in}}{\pgfqpoint{4.031393in}{5.003173in}}%
\pgfpathcurveto{\pgfqpoint{4.039206in}{5.010987in}}{\pgfqpoint{4.043597in}{5.021586in}}{\pgfqpoint{4.043597in}{5.032636in}}%
\pgfpathcurveto{\pgfqpoint{4.043597in}{5.043686in}}{\pgfqpoint{4.039206in}{5.054285in}}{\pgfqpoint{4.031393in}{5.062099in}}%
\pgfpathcurveto{\pgfqpoint{4.023579in}{5.069912in}}{\pgfqpoint{4.012980in}{5.074303in}}{\pgfqpoint{4.001930in}{5.074303in}}%
\pgfpathcurveto{\pgfqpoint{3.990880in}{5.074303in}}{\pgfqpoint{3.980281in}{5.069912in}}{\pgfqpoint{3.972467in}{5.062099in}}%
\pgfpathcurveto{\pgfqpoint{3.964654in}{5.054285in}}{\pgfqpoint{3.960263in}{5.043686in}}{\pgfqpoint{3.960263in}{5.032636in}}%
\pgfpathcurveto{\pgfqpoint{3.960263in}{5.021586in}}{\pgfqpoint{3.964654in}{5.010987in}}{\pgfqpoint{3.972467in}{5.003173in}}%
\pgfpathcurveto{\pgfqpoint{3.980281in}{4.995360in}}{\pgfqpoint{3.990880in}{4.990969in}}{\pgfqpoint{4.001930in}{4.990969in}}%
\pgfpathclose%
\pgfusepath{stroke,fill}%
\end{pgfscope}%
\begin{pgfscope}%
\pgfpathrectangle{\pgfqpoint{0.481978in}{0.331635in}}{\pgfqpoint{9.300000in}{7.700000in}}%
\pgfusepath{clip}%
\pgfsetbuttcap%
\pgfsetroundjoin%
\definecolor{currentfill}{rgb}{0.552941,0.898039,0.631373}%
\pgfsetfillcolor{currentfill}%
\pgfsetlinewidth{0.481800pt}%
\definecolor{currentstroke}{rgb}{1.000000,1.000000,1.000000}%
\pgfsetstrokecolor{currentstroke}%
\pgfsetdash{}{0pt}%
\pgfpathmoveto{\pgfqpoint{6.060999in}{1.751246in}}%
\pgfpathcurveto{\pgfqpoint{6.072049in}{1.751246in}}{\pgfqpoint{6.082648in}{1.755636in}}{\pgfqpoint{6.090462in}{1.763449in}}%
\pgfpathcurveto{\pgfqpoint{6.098275in}{1.771263in}}{\pgfqpoint{6.102666in}{1.781862in}}{\pgfqpoint{6.102666in}{1.792912in}}%
\pgfpathcurveto{\pgfqpoint{6.102666in}{1.803962in}}{\pgfqpoint{6.098275in}{1.814561in}}{\pgfqpoint{6.090462in}{1.822375in}}%
\pgfpathcurveto{\pgfqpoint{6.082648in}{1.830189in}}{\pgfqpoint{6.072049in}{1.834579in}}{\pgfqpoint{6.060999in}{1.834579in}}%
\pgfpathcurveto{\pgfqpoint{6.049949in}{1.834579in}}{\pgfqpoint{6.039350in}{1.830189in}}{\pgfqpoint{6.031536in}{1.822375in}}%
\pgfpathcurveto{\pgfqpoint{6.023723in}{1.814561in}}{\pgfqpoint{6.019332in}{1.803962in}}{\pgfqpoint{6.019332in}{1.792912in}}%
\pgfpathcurveto{\pgfqpoint{6.019332in}{1.781862in}}{\pgfqpoint{6.023723in}{1.771263in}}{\pgfqpoint{6.031536in}{1.763449in}}%
\pgfpathcurveto{\pgfqpoint{6.039350in}{1.755636in}}{\pgfqpoint{6.049949in}{1.751246in}}{\pgfqpoint{6.060999in}{1.751246in}}%
\pgfpathclose%
\pgfusepath{stroke,fill}%
\end{pgfscope}%
\begin{pgfscope}%
\pgfpathrectangle{\pgfqpoint{0.481978in}{0.331635in}}{\pgfqpoint{9.300000in}{7.700000in}}%
\pgfusepath{clip}%
\pgfsetbuttcap%
\pgfsetroundjoin%
\definecolor{currentfill}{rgb}{0.552941,0.898039,0.631373}%
\pgfsetfillcolor{currentfill}%
\pgfsetlinewidth{0.481800pt}%
\definecolor{currentstroke}{rgb}{1.000000,1.000000,1.000000}%
\pgfsetstrokecolor{currentstroke}%
\pgfsetdash{}{0pt}%
\pgfpathmoveto{\pgfqpoint{2.449894in}{1.656799in}}%
\pgfpathcurveto{\pgfqpoint{2.460944in}{1.656799in}}{\pgfqpoint{2.471543in}{1.661190in}}{\pgfqpoint{2.479357in}{1.669003in}}%
\pgfpathcurveto{\pgfqpoint{2.487170in}{1.676817in}}{\pgfqpoint{2.491561in}{1.687416in}}{\pgfqpoint{2.491561in}{1.698466in}}%
\pgfpathcurveto{\pgfqpoint{2.491561in}{1.709516in}}{\pgfqpoint{2.487170in}{1.720115in}}{\pgfqpoint{2.479357in}{1.727929in}}%
\pgfpathcurveto{\pgfqpoint{2.471543in}{1.735742in}}{\pgfqpoint{2.460944in}{1.740133in}}{\pgfqpoint{2.449894in}{1.740133in}}%
\pgfpathcurveto{\pgfqpoint{2.438844in}{1.740133in}}{\pgfqpoint{2.428245in}{1.735742in}}{\pgfqpoint{2.420431in}{1.727929in}}%
\pgfpathcurveto{\pgfqpoint{2.412617in}{1.720115in}}{\pgfqpoint{2.408227in}{1.709516in}}{\pgfqpoint{2.408227in}{1.698466in}}%
\pgfpathcurveto{\pgfqpoint{2.408227in}{1.687416in}}{\pgfqpoint{2.412617in}{1.676817in}}{\pgfqpoint{2.420431in}{1.669003in}}%
\pgfpathcurveto{\pgfqpoint{2.428245in}{1.661190in}}{\pgfqpoint{2.438844in}{1.656799in}}{\pgfqpoint{2.449894in}{1.656799in}}%
\pgfpathclose%
\pgfusepath{stroke,fill}%
\end{pgfscope}%
\begin{pgfscope}%
\pgfpathrectangle{\pgfqpoint{0.481978in}{0.331635in}}{\pgfqpoint{9.300000in}{7.700000in}}%
\pgfusepath{clip}%
\pgfsetbuttcap%
\pgfsetroundjoin%
\definecolor{currentfill}{rgb}{0.552941,0.898039,0.631373}%
\pgfsetfillcolor{currentfill}%
\pgfsetlinewidth{0.481800pt}%
\definecolor{currentstroke}{rgb}{1.000000,1.000000,1.000000}%
\pgfsetstrokecolor{currentstroke}%
\pgfsetdash{}{0pt}%
\pgfpathmoveto{\pgfqpoint{7.028940in}{4.709424in}}%
\pgfpathcurveto{\pgfqpoint{7.039990in}{4.709424in}}{\pgfqpoint{7.050589in}{4.713814in}}{\pgfqpoint{7.058402in}{4.721628in}}%
\pgfpathcurveto{\pgfqpoint{7.066216in}{4.729441in}}{\pgfqpoint{7.070606in}{4.740040in}}{\pgfqpoint{7.070606in}{4.751090in}}%
\pgfpathcurveto{\pgfqpoint{7.070606in}{4.762141in}}{\pgfqpoint{7.066216in}{4.772740in}}{\pgfqpoint{7.058402in}{4.780553in}}%
\pgfpathcurveto{\pgfqpoint{7.050589in}{4.788367in}}{\pgfqpoint{7.039990in}{4.792757in}}{\pgfqpoint{7.028940in}{4.792757in}}%
\pgfpathcurveto{\pgfqpoint{7.017890in}{4.792757in}}{\pgfqpoint{7.007290in}{4.788367in}}{\pgfqpoint{6.999477in}{4.780553in}}%
\pgfpathcurveto{\pgfqpoint{6.991663in}{4.772740in}}{\pgfqpoint{6.987273in}{4.762141in}}{\pgfqpoint{6.987273in}{4.751090in}}%
\pgfpathcurveto{\pgfqpoint{6.987273in}{4.740040in}}{\pgfqpoint{6.991663in}{4.729441in}}{\pgfqpoint{6.999477in}{4.721628in}}%
\pgfpathcurveto{\pgfqpoint{7.007290in}{4.713814in}}{\pgfqpoint{7.017890in}{4.709424in}}{\pgfqpoint{7.028940in}{4.709424in}}%
\pgfpathclose%
\pgfusepath{stroke,fill}%
\end{pgfscope}%
\begin{pgfscope}%
\pgfpathrectangle{\pgfqpoint{0.481978in}{0.331635in}}{\pgfqpoint{9.300000in}{7.700000in}}%
\pgfusepath{clip}%
\pgfsetbuttcap%
\pgfsetroundjoin%
\definecolor{currentfill}{rgb}{0.552941,0.898039,0.631373}%
\pgfsetfillcolor{currentfill}%
\pgfsetlinewidth{0.481800pt}%
\definecolor{currentstroke}{rgb}{1.000000,1.000000,1.000000}%
\pgfsetstrokecolor{currentstroke}%
\pgfsetdash{}{0pt}%
\pgfpathmoveto{\pgfqpoint{5.259017in}{7.554084in}}%
\pgfpathcurveto{\pgfqpoint{5.270067in}{7.554084in}}{\pgfqpoint{5.280666in}{7.558475in}}{\pgfqpoint{5.288480in}{7.566288in}}%
\pgfpathcurveto{\pgfqpoint{5.296294in}{7.574102in}}{\pgfqpoint{5.300684in}{7.584701in}}{\pgfqpoint{5.300684in}{7.595751in}}%
\pgfpathcurveto{\pgfqpoint{5.300684in}{7.606801in}}{\pgfqpoint{5.296294in}{7.617400in}}{\pgfqpoint{5.288480in}{7.625214in}}%
\pgfpathcurveto{\pgfqpoint{5.280666in}{7.633027in}}{\pgfqpoint{5.270067in}{7.637418in}}{\pgfqpoint{5.259017in}{7.637418in}}%
\pgfpathcurveto{\pgfqpoint{5.247967in}{7.637418in}}{\pgfqpoint{5.237368in}{7.633027in}}{\pgfqpoint{5.229554in}{7.625214in}}%
\pgfpathcurveto{\pgfqpoint{5.221741in}{7.617400in}}{\pgfqpoint{5.217350in}{7.606801in}}{\pgfqpoint{5.217350in}{7.595751in}}%
\pgfpathcurveto{\pgfqpoint{5.217350in}{7.584701in}}{\pgfqpoint{5.221741in}{7.574102in}}{\pgfqpoint{5.229554in}{7.566288in}}%
\pgfpathcurveto{\pgfqpoint{5.237368in}{7.558475in}}{\pgfqpoint{5.247967in}{7.554084in}}{\pgfqpoint{5.259017in}{7.554084in}}%
\pgfpathclose%
\pgfusepath{stroke,fill}%
\end{pgfscope}%
\begin{pgfscope}%
\pgfpathrectangle{\pgfqpoint{0.481978in}{0.331635in}}{\pgfqpoint{9.300000in}{7.700000in}}%
\pgfusepath{clip}%
\pgfsetbuttcap%
\pgfsetroundjoin%
\definecolor{currentfill}{rgb}{0.552941,0.898039,0.631373}%
\pgfsetfillcolor{currentfill}%
\pgfsetlinewidth{0.481800pt}%
\definecolor{currentstroke}{rgb}{1.000000,1.000000,1.000000}%
\pgfsetstrokecolor{currentstroke}%
\pgfsetdash{}{0pt}%
\pgfpathmoveto{\pgfqpoint{9.184137in}{5.064662in}}%
\pgfpathcurveto{\pgfqpoint{9.195187in}{5.064662in}}{\pgfqpoint{9.205786in}{5.069052in}}{\pgfqpoint{9.213600in}{5.076866in}}%
\pgfpathcurveto{\pgfqpoint{9.221414in}{5.084680in}}{\pgfqpoint{9.225804in}{5.095279in}}{\pgfqpoint{9.225804in}{5.106329in}}%
\pgfpathcurveto{\pgfqpoint{9.225804in}{5.117379in}}{\pgfqpoint{9.221414in}{5.127978in}}{\pgfqpoint{9.213600in}{5.135792in}}%
\pgfpathcurveto{\pgfqpoint{9.205786in}{5.143605in}}{\pgfqpoint{9.195187in}{5.147995in}}{\pgfqpoint{9.184137in}{5.147995in}}%
\pgfpathcurveto{\pgfqpoint{9.173087in}{5.147995in}}{\pgfqpoint{9.162488in}{5.143605in}}{\pgfqpoint{9.154675in}{5.135792in}}%
\pgfpathcurveto{\pgfqpoint{9.146861in}{5.127978in}}{\pgfqpoint{9.142471in}{5.117379in}}{\pgfqpoint{9.142471in}{5.106329in}}%
\pgfpathcurveto{\pgfqpoint{9.142471in}{5.095279in}}{\pgfqpoint{9.146861in}{5.084680in}}{\pgfqpoint{9.154675in}{5.076866in}}%
\pgfpathcurveto{\pgfqpoint{9.162488in}{5.069052in}}{\pgfqpoint{9.173087in}{5.064662in}}{\pgfqpoint{9.184137in}{5.064662in}}%
\pgfpathclose%
\pgfusepath{stroke,fill}%
\end{pgfscope}%
\begin{pgfscope}%
\pgfpathrectangle{\pgfqpoint{0.481978in}{0.331635in}}{\pgfqpoint{9.300000in}{7.700000in}}%
\pgfusepath{clip}%
\pgfsetbuttcap%
\pgfsetroundjoin%
\definecolor{currentfill}{rgb}{0.552941,0.898039,0.631373}%
\pgfsetfillcolor{currentfill}%
\pgfsetlinewidth{0.481800pt}%
\definecolor{currentstroke}{rgb}{1.000000,1.000000,1.000000}%
\pgfsetstrokecolor{currentstroke}%
\pgfsetdash{}{0pt}%
\pgfpathmoveto{\pgfqpoint{8.944873in}{6.252081in}}%
\pgfpathcurveto{\pgfqpoint{8.955923in}{6.252081in}}{\pgfqpoint{8.966522in}{6.256471in}}{\pgfqpoint{8.974336in}{6.264285in}}%
\pgfpathcurveto{\pgfqpoint{8.982149in}{6.272098in}}{\pgfqpoint{8.986540in}{6.282697in}}{\pgfqpoint{8.986540in}{6.293748in}}%
\pgfpathcurveto{\pgfqpoint{8.986540in}{6.304798in}}{\pgfqpoint{8.982149in}{6.315397in}}{\pgfqpoint{8.974336in}{6.323210in}}%
\pgfpathcurveto{\pgfqpoint{8.966522in}{6.331024in}}{\pgfqpoint{8.955923in}{6.335414in}}{\pgfqpoint{8.944873in}{6.335414in}}%
\pgfpathcurveto{\pgfqpoint{8.933823in}{6.335414in}}{\pgfqpoint{8.923224in}{6.331024in}}{\pgfqpoint{8.915410in}{6.323210in}}%
\pgfpathcurveto{\pgfqpoint{8.907597in}{6.315397in}}{\pgfqpoint{8.903206in}{6.304798in}}{\pgfqpoint{8.903206in}{6.293748in}}%
\pgfpathcurveto{\pgfqpoint{8.903206in}{6.282697in}}{\pgfqpoint{8.907597in}{6.272098in}}{\pgfqpoint{8.915410in}{6.264285in}}%
\pgfpathcurveto{\pgfqpoint{8.923224in}{6.256471in}}{\pgfqpoint{8.933823in}{6.252081in}}{\pgfqpoint{8.944873in}{6.252081in}}%
\pgfpathclose%
\pgfusepath{stroke,fill}%
\end{pgfscope}%
\begin{pgfscope}%
\pgfpathrectangle{\pgfqpoint{0.481978in}{0.331635in}}{\pgfqpoint{9.300000in}{7.700000in}}%
\pgfusepath{clip}%
\pgfsetbuttcap%
\pgfsetroundjoin%
\definecolor{currentfill}{rgb}{0.552941,0.898039,0.631373}%
\pgfsetfillcolor{currentfill}%
\pgfsetlinewidth{0.481800pt}%
\definecolor{currentstroke}{rgb}{1.000000,1.000000,1.000000}%
\pgfsetstrokecolor{currentstroke}%
\pgfsetdash{}{0pt}%
\pgfpathmoveto{\pgfqpoint{7.733013in}{3.449529in}}%
\pgfpathcurveto{\pgfqpoint{7.744063in}{3.449529in}}{\pgfqpoint{7.754662in}{3.453919in}}{\pgfqpoint{7.762476in}{3.461733in}}%
\pgfpathcurveto{\pgfqpoint{7.770289in}{3.469547in}}{\pgfqpoint{7.774679in}{3.480146in}}{\pgfqpoint{7.774679in}{3.491196in}}%
\pgfpathcurveto{\pgfqpoint{7.774679in}{3.502246in}}{\pgfqpoint{7.770289in}{3.512845in}}{\pgfqpoint{7.762476in}{3.520659in}}%
\pgfpathcurveto{\pgfqpoint{7.754662in}{3.528472in}}{\pgfqpoint{7.744063in}{3.532862in}}{\pgfqpoint{7.733013in}{3.532862in}}%
\pgfpathcurveto{\pgfqpoint{7.721963in}{3.532862in}}{\pgfqpoint{7.711364in}{3.528472in}}{\pgfqpoint{7.703550in}{3.520659in}}%
\pgfpathcurveto{\pgfqpoint{7.695736in}{3.512845in}}{\pgfqpoint{7.691346in}{3.502246in}}{\pgfqpoint{7.691346in}{3.491196in}}%
\pgfpathcurveto{\pgfqpoint{7.691346in}{3.480146in}}{\pgfqpoint{7.695736in}{3.469547in}}{\pgfqpoint{7.703550in}{3.461733in}}%
\pgfpathcurveto{\pgfqpoint{7.711364in}{3.453919in}}{\pgfqpoint{7.721963in}{3.449529in}}{\pgfqpoint{7.733013in}{3.449529in}}%
\pgfpathclose%
\pgfusepath{stroke,fill}%
\end{pgfscope}%
\begin{pgfscope}%
\pgfpathrectangle{\pgfqpoint{0.481978in}{0.331635in}}{\pgfqpoint{9.300000in}{7.700000in}}%
\pgfusepath{clip}%
\pgfsetbuttcap%
\pgfsetroundjoin%
\definecolor{currentfill}{rgb}{0.552941,0.898039,0.631373}%
\pgfsetfillcolor{currentfill}%
\pgfsetlinewidth{0.481800pt}%
\definecolor{currentstroke}{rgb}{1.000000,1.000000,1.000000}%
\pgfsetstrokecolor{currentstroke}%
\pgfsetdash{}{0pt}%
\pgfpathmoveto{\pgfqpoint{5.907061in}{2.059171in}}%
\pgfpathcurveto{\pgfqpoint{5.918111in}{2.059171in}}{\pgfqpoint{5.928710in}{2.063562in}}{\pgfqpoint{5.936524in}{2.071375in}}%
\pgfpathcurveto{\pgfqpoint{5.944338in}{2.079189in}}{\pgfqpoint{5.948728in}{2.089788in}}{\pgfqpoint{5.948728in}{2.100838in}}%
\pgfpathcurveto{\pgfqpoint{5.948728in}{2.111888in}}{\pgfqpoint{5.944338in}{2.122487in}}{\pgfqpoint{5.936524in}{2.130301in}}%
\pgfpathcurveto{\pgfqpoint{5.928710in}{2.138114in}}{\pgfqpoint{5.918111in}{2.142505in}}{\pgfqpoint{5.907061in}{2.142505in}}%
\pgfpathcurveto{\pgfqpoint{5.896011in}{2.142505in}}{\pgfqpoint{5.885412in}{2.138114in}}{\pgfqpoint{5.877598in}{2.130301in}}%
\pgfpathcurveto{\pgfqpoint{5.869785in}{2.122487in}}{\pgfqpoint{5.865395in}{2.111888in}}{\pgfqpoint{5.865395in}{2.100838in}}%
\pgfpathcurveto{\pgfqpoint{5.865395in}{2.089788in}}{\pgfqpoint{5.869785in}{2.079189in}}{\pgfqpoint{5.877598in}{2.071375in}}%
\pgfpathcurveto{\pgfqpoint{5.885412in}{2.063562in}}{\pgfqpoint{5.896011in}{2.059171in}}{\pgfqpoint{5.907061in}{2.059171in}}%
\pgfpathclose%
\pgfusepath{stroke,fill}%
\end{pgfscope}%
\begin{pgfscope}%
\pgfpathrectangle{\pgfqpoint{0.481978in}{0.331635in}}{\pgfqpoint{9.300000in}{7.700000in}}%
\pgfusepath{clip}%
\pgfsetbuttcap%
\pgfsetroundjoin%
\definecolor{currentfill}{rgb}{0.552941,0.898039,0.631373}%
\pgfsetfillcolor{currentfill}%
\pgfsetlinewidth{0.481800pt}%
\definecolor{currentstroke}{rgb}{1.000000,1.000000,1.000000}%
\pgfsetstrokecolor{currentstroke}%
\pgfsetdash{}{0pt}%
\pgfpathmoveto{\pgfqpoint{4.839240in}{5.869672in}}%
\pgfpathcurveto{\pgfqpoint{4.850291in}{5.869672in}}{\pgfqpoint{4.860890in}{5.874062in}}{\pgfqpoint{4.868703in}{5.881876in}}%
\pgfpathcurveto{\pgfqpoint{4.876517in}{5.889689in}}{\pgfqpoint{4.880907in}{5.900288in}}{\pgfqpoint{4.880907in}{5.911338in}}%
\pgfpathcurveto{\pgfqpoint{4.880907in}{5.922389in}}{\pgfqpoint{4.876517in}{5.932988in}}{\pgfqpoint{4.868703in}{5.940801in}}%
\pgfpathcurveto{\pgfqpoint{4.860890in}{5.948615in}}{\pgfqpoint{4.850291in}{5.953005in}}{\pgfqpoint{4.839240in}{5.953005in}}%
\pgfpathcurveto{\pgfqpoint{4.828190in}{5.953005in}}{\pgfqpoint{4.817591in}{5.948615in}}{\pgfqpoint{4.809778in}{5.940801in}}%
\pgfpathcurveto{\pgfqpoint{4.801964in}{5.932988in}}{\pgfqpoint{4.797574in}{5.922389in}}{\pgfqpoint{4.797574in}{5.911338in}}%
\pgfpathcurveto{\pgfqpoint{4.797574in}{5.900288in}}{\pgfqpoint{4.801964in}{5.889689in}}{\pgfqpoint{4.809778in}{5.881876in}}%
\pgfpathcurveto{\pgfqpoint{4.817591in}{5.874062in}}{\pgfqpoint{4.828190in}{5.869672in}}{\pgfqpoint{4.839240in}{5.869672in}}%
\pgfpathclose%
\pgfusepath{stroke,fill}%
\end{pgfscope}%
\begin{pgfscope}%
\pgfpathrectangle{\pgfqpoint{0.481978in}{0.331635in}}{\pgfqpoint{9.300000in}{7.700000in}}%
\pgfusepath{clip}%
\pgfsetbuttcap%
\pgfsetroundjoin%
\definecolor{currentfill}{rgb}{0.552941,0.898039,0.631373}%
\pgfsetfillcolor{currentfill}%
\pgfsetlinewidth{0.481800pt}%
\definecolor{currentstroke}{rgb}{1.000000,1.000000,1.000000}%
\pgfsetstrokecolor{currentstroke}%
\pgfsetdash{}{0pt}%
\pgfpathmoveto{\pgfqpoint{2.381245in}{2.262941in}}%
\pgfpathcurveto{\pgfqpoint{2.392295in}{2.262941in}}{\pgfqpoint{2.402894in}{2.267331in}}{\pgfqpoint{2.410708in}{2.275144in}}%
\pgfpathcurveto{\pgfqpoint{2.418522in}{2.282958in}}{\pgfqpoint{2.422912in}{2.293557in}}{\pgfqpoint{2.422912in}{2.304607in}}%
\pgfpathcurveto{\pgfqpoint{2.422912in}{2.315657in}}{\pgfqpoint{2.418522in}{2.326256in}}{\pgfqpoint{2.410708in}{2.334070in}}%
\pgfpathcurveto{\pgfqpoint{2.402894in}{2.341884in}}{\pgfqpoint{2.392295in}{2.346274in}}{\pgfqpoint{2.381245in}{2.346274in}}%
\pgfpathcurveto{\pgfqpoint{2.370195in}{2.346274in}}{\pgfqpoint{2.359596in}{2.341884in}}{\pgfqpoint{2.351782in}{2.334070in}}%
\pgfpathcurveto{\pgfqpoint{2.343969in}{2.326256in}}{\pgfqpoint{2.339579in}{2.315657in}}{\pgfqpoint{2.339579in}{2.304607in}}%
\pgfpathcurveto{\pgfqpoint{2.339579in}{2.293557in}}{\pgfqpoint{2.343969in}{2.282958in}}{\pgfqpoint{2.351782in}{2.275144in}}%
\pgfpathcurveto{\pgfqpoint{2.359596in}{2.267331in}}{\pgfqpoint{2.370195in}{2.262941in}}{\pgfqpoint{2.381245in}{2.262941in}}%
\pgfpathclose%
\pgfusepath{stroke,fill}%
\end{pgfscope}%
\begin{pgfscope}%
\pgfpathrectangle{\pgfqpoint{0.481978in}{0.331635in}}{\pgfqpoint{9.300000in}{7.700000in}}%
\pgfusepath{clip}%
\pgfsetbuttcap%
\pgfsetroundjoin%
\definecolor{currentfill}{rgb}{0.552941,0.898039,0.631373}%
\pgfsetfillcolor{currentfill}%
\pgfsetlinewidth{0.481800pt}%
\definecolor{currentstroke}{rgb}{1.000000,1.000000,1.000000}%
\pgfsetstrokecolor{currentstroke}%
\pgfsetdash{}{0pt}%
\pgfpathmoveto{\pgfqpoint{7.266400in}{2.360768in}}%
\pgfpathcurveto{\pgfqpoint{7.277450in}{2.360768in}}{\pgfqpoint{7.288049in}{2.365158in}}{\pgfqpoint{7.295863in}{2.372972in}}%
\pgfpathcurveto{\pgfqpoint{7.303677in}{2.380785in}}{\pgfqpoint{7.308067in}{2.391384in}}{\pgfqpoint{7.308067in}{2.402434in}}%
\pgfpathcurveto{\pgfqpoint{7.308067in}{2.413484in}}{\pgfqpoint{7.303677in}{2.424083in}}{\pgfqpoint{7.295863in}{2.431897in}}%
\pgfpathcurveto{\pgfqpoint{7.288049in}{2.439711in}}{\pgfqpoint{7.277450in}{2.444101in}}{\pgfqpoint{7.266400in}{2.444101in}}%
\pgfpathcurveto{\pgfqpoint{7.255350in}{2.444101in}}{\pgfqpoint{7.244751in}{2.439711in}}{\pgfqpoint{7.236937in}{2.431897in}}%
\pgfpathcurveto{\pgfqpoint{7.229124in}{2.424083in}}{\pgfqpoint{7.224734in}{2.413484in}}{\pgfqpoint{7.224734in}{2.402434in}}%
\pgfpathcurveto{\pgfqpoint{7.224734in}{2.391384in}}{\pgfqpoint{7.229124in}{2.380785in}}{\pgfqpoint{7.236937in}{2.372972in}}%
\pgfpathcurveto{\pgfqpoint{7.244751in}{2.365158in}}{\pgfqpoint{7.255350in}{2.360768in}}{\pgfqpoint{7.266400in}{2.360768in}}%
\pgfpathclose%
\pgfusepath{stroke,fill}%
\end{pgfscope}%
\begin{pgfscope}%
\pgfpathrectangle{\pgfqpoint{0.481978in}{0.331635in}}{\pgfqpoint{9.300000in}{7.700000in}}%
\pgfusepath{clip}%
\pgfsetbuttcap%
\pgfsetroundjoin%
\definecolor{currentfill}{rgb}{0.552941,0.898039,0.631373}%
\pgfsetfillcolor{currentfill}%
\pgfsetlinewidth{0.481800pt}%
\definecolor{currentstroke}{rgb}{1.000000,1.000000,1.000000}%
\pgfsetstrokecolor{currentstroke}%
\pgfsetdash{}{0pt}%
\pgfpathmoveto{\pgfqpoint{3.410609in}{3.339609in}}%
\pgfpathcurveto{\pgfqpoint{3.421659in}{3.339609in}}{\pgfqpoint{3.432258in}{3.344000in}}{\pgfqpoint{3.440072in}{3.351813in}}%
\pgfpathcurveto{\pgfqpoint{3.447885in}{3.359627in}}{\pgfqpoint{3.452276in}{3.370226in}}{\pgfqpoint{3.452276in}{3.381276in}}%
\pgfpathcurveto{\pgfqpoint{3.452276in}{3.392326in}}{\pgfqpoint{3.447885in}{3.402925in}}{\pgfqpoint{3.440072in}{3.410739in}}%
\pgfpathcurveto{\pgfqpoint{3.432258in}{3.418553in}}{\pgfqpoint{3.421659in}{3.422943in}}{\pgfqpoint{3.410609in}{3.422943in}}%
\pgfpathcurveto{\pgfqpoint{3.399559in}{3.422943in}}{\pgfqpoint{3.388960in}{3.418553in}}{\pgfqpoint{3.381146in}{3.410739in}}%
\pgfpathcurveto{\pgfqpoint{3.373333in}{3.402925in}}{\pgfqpoint{3.368942in}{3.392326in}}{\pgfqpoint{3.368942in}{3.381276in}}%
\pgfpathcurveto{\pgfqpoint{3.368942in}{3.370226in}}{\pgfqpoint{3.373333in}{3.359627in}}{\pgfqpoint{3.381146in}{3.351813in}}%
\pgfpathcurveto{\pgfqpoint{3.388960in}{3.344000in}}{\pgfqpoint{3.399559in}{3.339609in}}{\pgfqpoint{3.410609in}{3.339609in}}%
\pgfpathclose%
\pgfusepath{stroke,fill}%
\end{pgfscope}%
\begin{pgfscope}%
\pgfpathrectangle{\pgfqpoint{0.481978in}{0.331635in}}{\pgfqpoint{9.300000in}{7.700000in}}%
\pgfusepath{clip}%
\pgfsetbuttcap%
\pgfsetroundjoin%
\definecolor{currentfill}{rgb}{0.552941,0.898039,0.631373}%
\pgfsetfillcolor{currentfill}%
\pgfsetlinewidth{0.481800pt}%
\definecolor{currentstroke}{rgb}{1.000000,1.000000,1.000000}%
\pgfsetstrokecolor{currentstroke}%
\pgfsetdash{}{0pt}%
\pgfpathmoveto{\pgfqpoint{9.044900in}{5.269063in}}%
\pgfpathcurveto{\pgfqpoint{9.055950in}{5.269063in}}{\pgfqpoint{9.066549in}{5.273453in}}{\pgfqpoint{9.074363in}{5.281267in}}%
\pgfpathcurveto{\pgfqpoint{9.082176in}{5.289081in}}{\pgfqpoint{9.086566in}{5.299680in}}{\pgfqpoint{9.086566in}{5.310730in}}%
\pgfpathcurveto{\pgfqpoint{9.086566in}{5.321780in}}{\pgfqpoint{9.082176in}{5.332379in}}{\pgfqpoint{9.074363in}{5.340192in}}%
\pgfpathcurveto{\pgfqpoint{9.066549in}{5.348006in}}{\pgfqpoint{9.055950in}{5.352396in}}{\pgfqpoint{9.044900in}{5.352396in}}%
\pgfpathcurveto{\pgfqpoint{9.033850in}{5.352396in}}{\pgfqpoint{9.023251in}{5.348006in}}{\pgfqpoint{9.015437in}{5.340192in}}%
\pgfpathcurveto{\pgfqpoint{9.007623in}{5.332379in}}{\pgfqpoint{9.003233in}{5.321780in}}{\pgfqpoint{9.003233in}{5.310730in}}%
\pgfpathcurveto{\pgfqpoint{9.003233in}{5.299680in}}{\pgfqpoint{9.007623in}{5.289081in}}{\pgfqpoint{9.015437in}{5.281267in}}%
\pgfpathcurveto{\pgfqpoint{9.023251in}{5.273453in}}{\pgfqpoint{9.033850in}{5.269063in}}{\pgfqpoint{9.044900in}{5.269063in}}%
\pgfpathclose%
\pgfusepath{stroke,fill}%
\end{pgfscope}%
\begin{pgfscope}%
\pgfpathrectangle{\pgfqpoint{0.481978in}{0.331635in}}{\pgfqpoint{9.300000in}{7.700000in}}%
\pgfusepath{clip}%
\pgfsetbuttcap%
\pgfsetroundjoin%
\definecolor{currentfill}{rgb}{0.552941,0.898039,0.631373}%
\pgfsetfillcolor{currentfill}%
\pgfsetlinewidth{0.481800pt}%
\definecolor{currentstroke}{rgb}{1.000000,1.000000,1.000000}%
\pgfsetstrokecolor{currentstroke}%
\pgfsetdash{}{0pt}%
\pgfpathmoveto{\pgfqpoint{8.024517in}{4.648323in}}%
\pgfpathcurveto{\pgfqpoint{8.035567in}{4.648323in}}{\pgfqpoint{8.046166in}{4.652713in}}{\pgfqpoint{8.053980in}{4.660527in}}%
\pgfpathcurveto{\pgfqpoint{8.061793in}{4.668341in}}{\pgfqpoint{8.066184in}{4.678940in}}{\pgfqpoint{8.066184in}{4.689990in}}%
\pgfpathcurveto{\pgfqpoint{8.066184in}{4.701040in}}{\pgfqpoint{8.061793in}{4.711639in}}{\pgfqpoint{8.053980in}{4.719453in}}%
\pgfpathcurveto{\pgfqpoint{8.046166in}{4.727266in}}{\pgfqpoint{8.035567in}{4.731656in}}{\pgfqpoint{8.024517in}{4.731656in}}%
\pgfpathcurveto{\pgfqpoint{8.013467in}{4.731656in}}{\pgfqpoint{8.002868in}{4.727266in}}{\pgfqpoint{7.995054in}{4.719453in}}%
\pgfpathcurveto{\pgfqpoint{7.987240in}{4.711639in}}{\pgfqpoint{7.982850in}{4.701040in}}{\pgfqpoint{7.982850in}{4.689990in}}%
\pgfpathcurveto{\pgfqpoint{7.982850in}{4.678940in}}{\pgfqpoint{7.987240in}{4.668341in}}{\pgfqpoint{7.995054in}{4.660527in}}%
\pgfpathcurveto{\pgfqpoint{8.002868in}{4.652713in}}{\pgfqpoint{8.013467in}{4.648323in}}{\pgfqpoint{8.024517in}{4.648323in}}%
\pgfpathclose%
\pgfusepath{stroke,fill}%
\end{pgfscope}%
\begin{pgfscope}%
\pgfpathrectangle{\pgfqpoint{0.481978in}{0.331635in}}{\pgfqpoint{9.300000in}{7.700000in}}%
\pgfusepath{clip}%
\pgfsetbuttcap%
\pgfsetroundjoin%
\definecolor{currentfill}{rgb}{0.552941,0.898039,0.631373}%
\pgfsetfillcolor{currentfill}%
\pgfsetlinewidth{0.481800pt}%
\definecolor{currentstroke}{rgb}{1.000000,1.000000,1.000000}%
\pgfsetstrokecolor{currentstroke}%
\pgfsetdash{}{0pt}%
\pgfpathmoveto{\pgfqpoint{3.710429in}{4.713864in}}%
\pgfpathcurveto{\pgfqpoint{3.721480in}{4.713864in}}{\pgfqpoint{3.732079in}{4.718254in}}{\pgfqpoint{3.739892in}{4.726068in}}%
\pgfpathcurveto{\pgfqpoint{3.747706in}{4.733881in}}{\pgfqpoint{3.752096in}{4.744480in}}{\pgfqpoint{3.752096in}{4.755530in}}%
\pgfpathcurveto{\pgfqpoint{3.752096in}{4.766580in}}{\pgfqpoint{3.747706in}{4.777179in}}{\pgfqpoint{3.739892in}{4.784993in}}%
\pgfpathcurveto{\pgfqpoint{3.732079in}{4.792807in}}{\pgfqpoint{3.721480in}{4.797197in}}{\pgfqpoint{3.710429in}{4.797197in}}%
\pgfpathcurveto{\pgfqpoint{3.699379in}{4.797197in}}{\pgfqpoint{3.688780in}{4.792807in}}{\pgfqpoint{3.680967in}{4.784993in}}%
\pgfpathcurveto{\pgfqpoint{3.673153in}{4.777179in}}{\pgfqpoint{3.668763in}{4.766580in}}{\pgfqpoint{3.668763in}{4.755530in}}%
\pgfpathcurveto{\pgfqpoint{3.668763in}{4.744480in}}{\pgfqpoint{3.673153in}{4.733881in}}{\pgfqpoint{3.680967in}{4.726068in}}%
\pgfpathcurveto{\pgfqpoint{3.688780in}{4.718254in}}{\pgfqpoint{3.699379in}{4.713864in}}{\pgfqpoint{3.710429in}{4.713864in}}%
\pgfpathclose%
\pgfusepath{stroke,fill}%
\end{pgfscope}%
\begin{pgfscope}%
\pgfpathrectangle{\pgfqpoint{0.481978in}{0.331635in}}{\pgfqpoint{9.300000in}{7.700000in}}%
\pgfusepath{clip}%
\pgfsetbuttcap%
\pgfsetroundjoin%
\definecolor{currentfill}{rgb}{0.552941,0.898039,0.631373}%
\pgfsetfillcolor{currentfill}%
\pgfsetlinewidth{0.481800pt}%
\definecolor{currentstroke}{rgb}{1.000000,1.000000,1.000000}%
\pgfsetstrokecolor{currentstroke}%
\pgfsetdash{}{0pt}%
\pgfpathmoveto{\pgfqpoint{5.690812in}{2.637048in}}%
\pgfpathcurveto{\pgfqpoint{5.701862in}{2.637048in}}{\pgfqpoint{5.712461in}{2.641438in}}{\pgfqpoint{5.720275in}{2.649252in}}%
\pgfpathcurveto{\pgfqpoint{5.728088in}{2.657065in}}{\pgfqpoint{5.732479in}{2.667664in}}{\pgfqpoint{5.732479in}{2.678714in}}%
\pgfpathcurveto{\pgfqpoint{5.732479in}{2.689765in}}{\pgfqpoint{5.728088in}{2.700364in}}{\pgfqpoint{5.720275in}{2.708177in}}%
\pgfpathcurveto{\pgfqpoint{5.712461in}{2.715991in}}{\pgfqpoint{5.701862in}{2.720381in}}{\pgfqpoint{5.690812in}{2.720381in}}%
\pgfpathcurveto{\pgfqpoint{5.679762in}{2.720381in}}{\pgfqpoint{5.669163in}{2.715991in}}{\pgfqpoint{5.661349in}{2.708177in}}%
\pgfpathcurveto{\pgfqpoint{5.653536in}{2.700364in}}{\pgfqpoint{5.649145in}{2.689765in}}{\pgfqpoint{5.649145in}{2.678714in}}%
\pgfpathcurveto{\pgfqpoint{5.649145in}{2.667664in}}{\pgfqpoint{5.653536in}{2.657065in}}{\pgfqpoint{5.661349in}{2.649252in}}%
\pgfpathcurveto{\pgfqpoint{5.669163in}{2.641438in}}{\pgfqpoint{5.679762in}{2.637048in}}{\pgfqpoint{5.690812in}{2.637048in}}%
\pgfpathclose%
\pgfusepath{stroke,fill}%
\end{pgfscope}%
\begin{pgfscope}%
\pgfpathrectangle{\pgfqpoint{0.481978in}{0.331635in}}{\pgfqpoint{9.300000in}{7.700000in}}%
\pgfusepath{clip}%
\pgfsetbuttcap%
\pgfsetroundjoin%
\definecolor{currentfill}{rgb}{1.000000,0.623529,0.607843}%
\pgfsetfillcolor{currentfill}%
\pgfsetlinewidth{0.481800pt}%
\definecolor{currentstroke}{rgb}{1.000000,1.000000,1.000000}%
\pgfsetstrokecolor{currentstroke}%
\pgfsetdash{}{0pt}%
\pgfpathmoveto{\pgfqpoint{7.737423in}{5.840523in}}%
\pgfpathcurveto{\pgfqpoint{7.748473in}{5.840523in}}{\pgfqpoint{7.759072in}{5.844913in}}{\pgfqpoint{7.766885in}{5.852727in}}%
\pgfpathcurveto{\pgfqpoint{7.774699in}{5.860541in}}{\pgfqpoint{7.779089in}{5.871140in}}{\pgfqpoint{7.779089in}{5.882190in}}%
\pgfpathcurveto{\pgfqpoint{7.779089in}{5.893240in}}{\pgfqpoint{7.774699in}{5.903839in}}{\pgfqpoint{7.766885in}{5.911653in}}%
\pgfpathcurveto{\pgfqpoint{7.759072in}{5.919466in}}{\pgfqpoint{7.748473in}{5.923856in}}{\pgfqpoint{7.737423in}{5.923856in}}%
\pgfpathcurveto{\pgfqpoint{7.726372in}{5.923856in}}{\pgfqpoint{7.715773in}{5.919466in}}{\pgfqpoint{7.707960in}{5.911653in}}%
\pgfpathcurveto{\pgfqpoint{7.700146in}{5.903839in}}{\pgfqpoint{7.695756in}{5.893240in}}{\pgfqpoint{7.695756in}{5.882190in}}%
\pgfpathcurveto{\pgfqpoint{7.695756in}{5.871140in}}{\pgfqpoint{7.700146in}{5.860541in}}{\pgfqpoint{7.707960in}{5.852727in}}%
\pgfpathcurveto{\pgfqpoint{7.715773in}{5.844913in}}{\pgfqpoint{7.726372in}{5.840523in}}{\pgfqpoint{7.737423in}{5.840523in}}%
\pgfpathclose%
\pgfusepath{stroke,fill}%
\end{pgfscope}%
\begin{pgfscope}%
\pgfpathrectangle{\pgfqpoint{0.481978in}{0.331635in}}{\pgfqpoint{9.300000in}{7.700000in}}%
\pgfusepath{clip}%
\pgfsetbuttcap%
\pgfsetroundjoin%
\definecolor{currentfill}{rgb}{1.000000,0.623529,0.607843}%
\pgfsetfillcolor{currentfill}%
\pgfsetlinewidth{0.481800pt}%
\definecolor{currentstroke}{rgb}{1.000000,1.000000,1.000000}%
\pgfsetstrokecolor{currentstroke}%
\pgfsetdash{}{0pt}%
\pgfpathmoveto{\pgfqpoint{8.478500in}{5.560595in}}%
\pgfpathcurveto{\pgfqpoint{8.489550in}{5.560595in}}{\pgfqpoint{8.500149in}{5.564985in}}{\pgfqpoint{8.507962in}{5.572799in}}%
\pgfpathcurveto{\pgfqpoint{8.515776in}{5.580613in}}{\pgfqpoint{8.520166in}{5.591212in}}{\pgfqpoint{8.520166in}{5.602262in}}%
\pgfpathcurveto{\pgfqpoint{8.520166in}{5.613312in}}{\pgfqpoint{8.515776in}{5.623911in}}{\pgfqpoint{8.507962in}{5.631724in}}%
\pgfpathcurveto{\pgfqpoint{8.500149in}{5.639538in}}{\pgfqpoint{8.489550in}{5.643928in}}{\pgfqpoint{8.478500in}{5.643928in}}%
\pgfpathcurveto{\pgfqpoint{8.467450in}{5.643928in}}{\pgfqpoint{8.456850in}{5.639538in}}{\pgfqpoint{8.449037in}{5.631724in}}%
\pgfpathcurveto{\pgfqpoint{8.441223in}{5.623911in}}{\pgfqpoint{8.436833in}{5.613312in}}{\pgfqpoint{8.436833in}{5.602262in}}%
\pgfpathcurveto{\pgfqpoint{8.436833in}{5.591212in}}{\pgfqpoint{8.441223in}{5.580613in}}{\pgfqpoint{8.449037in}{5.572799in}}%
\pgfpathcurveto{\pgfqpoint{8.456850in}{5.564985in}}{\pgfqpoint{8.467450in}{5.560595in}}{\pgfqpoint{8.478500in}{5.560595in}}%
\pgfpathclose%
\pgfusepath{stroke,fill}%
\end{pgfscope}%
\begin{pgfscope}%
\pgfpathrectangle{\pgfqpoint{0.481978in}{0.331635in}}{\pgfqpoint{9.300000in}{7.700000in}}%
\pgfusepath{clip}%
\pgfsetbuttcap%
\pgfsetroundjoin%
\definecolor{currentfill}{rgb}{1.000000,0.623529,0.607843}%
\pgfsetfillcolor{currentfill}%
\pgfsetlinewidth{0.481800pt}%
\definecolor{currentstroke}{rgb}{1.000000,1.000000,1.000000}%
\pgfsetstrokecolor{currentstroke}%
\pgfsetdash{}{0pt}%
\pgfpathmoveto{\pgfqpoint{5.299357in}{2.415253in}}%
\pgfpathcurveto{\pgfqpoint{5.310407in}{2.415253in}}{\pgfqpoint{5.321006in}{2.419644in}}{\pgfqpoint{5.328820in}{2.427457in}}%
\pgfpathcurveto{\pgfqpoint{5.336633in}{2.435271in}}{\pgfqpoint{5.341024in}{2.445870in}}{\pgfqpoint{5.341024in}{2.456920in}}%
\pgfpathcurveto{\pgfqpoint{5.341024in}{2.467970in}}{\pgfqpoint{5.336633in}{2.478569in}}{\pgfqpoint{5.328820in}{2.486383in}}%
\pgfpathcurveto{\pgfqpoint{5.321006in}{2.494196in}}{\pgfqpoint{5.310407in}{2.498587in}}{\pgfqpoint{5.299357in}{2.498587in}}%
\pgfpathcurveto{\pgfqpoint{5.288307in}{2.498587in}}{\pgfqpoint{5.277708in}{2.494196in}}{\pgfqpoint{5.269894in}{2.486383in}}%
\pgfpathcurveto{\pgfqpoint{5.262081in}{2.478569in}}{\pgfqpoint{5.257690in}{2.467970in}}{\pgfqpoint{5.257690in}{2.456920in}}%
\pgfpathcurveto{\pgfqpoint{5.257690in}{2.445870in}}{\pgfqpoint{5.262081in}{2.435271in}}{\pgfqpoint{5.269894in}{2.427457in}}%
\pgfpathcurveto{\pgfqpoint{5.277708in}{2.419644in}}{\pgfqpoint{5.288307in}{2.415253in}}{\pgfqpoint{5.299357in}{2.415253in}}%
\pgfpathclose%
\pgfusepath{stroke,fill}%
\end{pgfscope}%
\begin{pgfscope}%
\pgfpathrectangle{\pgfqpoint{0.481978in}{0.331635in}}{\pgfqpoint{9.300000in}{7.700000in}}%
\pgfusepath{clip}%
\pgfsetbuttcap%
\pgfsetroundjoin%
\definecolor{currentfill}{rgb}{1.000000,0.623529,0.607843}%
\pgfsetfillcolor{currentfill}%
\pgfsetlinewidth{0.481800pt}%
\definecolor{currentstroke}{rgb}{1.000000,1.000000,1.000000}%
\pgfsetstrokecolor{currentstroke}%
\pgfsetdash{}{0pt}%
\pgfpathmoveto{\pgfqpoint{5.050465in}{2.386257in}}%
\pgfpathcurveto{\pgfqpoint{5.061515in}{2.386257in}}{\pgfqpoint{5.072114in}{2.390648in}}{\pgfqpoint{5.079928in}{2.398461in}}%
\pgfpathcurveto{\pgfqpoint{5.087741in}{2.406275in}}{\pgfqpoint{5.092132in}{2.416874in}}{\pgfqpoint{5.092132in}{2.427924in}}%
\pgfpathcurveto{\pgfqpoint{5.092132in}{2.438974in}}{\pgfqpoint{5.087741in}{2.449573in}}{\pgfqpoint{5.079928in}{2.457387in}}%
\pgfpathcurveto{\pgfqpoint{5.072114in}{2.465200in}}{\pgfqpoint{5.061515in}{2.469591in}}{\pgfqpoint{5.050465in}{2.469591in}}%
\pgfpathcurveto{\pgfqpoint{5.039415in}{2.469591in}}{\pgfqpoint{5.028816in}{2.465200in}}{\pgfqpoint{5.021002in}{2.457387in}}%
\pgfpathcurveto{\pgfqpoint{5.013189in}{2.449573in}}{\pgfqpoint{5.008798in}{2.438974in}}{\pgfqpoint{5.008798in}{2.427924in}}%
\pgfpathcurveto{\pgfqpoint{5.008798in}{2.416874in}}{\pgfqpoint{5.013189in}{2.406275in}}{\pgfqpoint{5.021002in}{2.398461in}}%
\pgfpathcurveto{\pgfqpoint{5.028816in}{2.390648in}}{\pgfqpoint{5.039415in}{2.386257in}}{\pgfqpoint{5.050465in}{2.386257in}}%
\pgfpathclose%
\pgfusepath{stroke,fill}%
\end{pgfscope}%
\begin{pgfscope}%
\pgfpathrectangle{\pgfqpoint{0.481978in}{0.331635in}}{\pgfqpoint{9.300000in}{7.700000in}}%
\pgfusepath{clip}%
\pgfsetbuttcap%
\pgfsetroundjoin%
\definecolor{currentfill}{rgb}{1.000000,0.623529,0.607843}%
\pgfsetfillcolor{currentfill}%
\pgfsetlinewidth{0.481800pt}%
\definecolor{currentstroke}{rgb}{1.000000,1.000000,1.000000}%
\pgfsetstrokecolor{currentstroke}%
\pgfsetdash{}{0pt}%
\pgfpathmoveto{\pgfqpoint{4.424855in}{3.449766in}}%
\pgfpathcurveto{\pgfqpoint{4.435905in}{3.449766in}}{\pgfqpoint{4.446504in}{3.454156in}}{\pgfqpoint{4.454318in}{3.461970in}}%
\pgfpathcurveto{\pgfqpoint{4.462131in}{3.469783in}}{\pgfqpoint{4.466522in}{3.480382in}}{\pgfqpoint{4.466522in}{3.491433in}}%
\pgfpathcurveto{\pgfqpoint{4.466522in}{3.502483in}}{\pgfqpoint{4.462131in}{3.513082in}}{\pgfqpoint{4.454318in}{3.520895in}}%
\pgfpathcurveto{\pgfqpoint{4.446504in}{3.528709in}}{\pgfqpoint{4.435905in}{3.533099in}}{\pgfqpoint{4.424855in}{3.533099in}}%
\pgfpathcurveto{\pgfqpoint{4.413805in}{3.533099in}}{\pgfqpoint{4.403206in}{3.528709in}}{\pgfqpoint{4.395392in}{3.520895in}}%
\pgfpathcurveto{\pgfqpoint{4.387579in}{3.513082in}}{\pgfqpoint{4.383188in}{3.502483in}}{\pgfqpoint{4.383188in}{3.491433in}}%
\pgfpathcurveto{\pgfqpoint{4.383188in}{3.480382in}}{\pgfqpoint{4.387579in}{3.469783in}}{\pgfqpoint{4.395392in}{3.461970in}}%
\pgfpathcurveto{\pgfqpoint{4.403206in}{3.454156in}}{\pgfqpoint{4.413805in}{3.449766in}}{\pgfqpoint{4.424855in}{3.449766in}}%
\pgfpathclose%
\pgfusepath{stroke,fill}%
\end{pgfscope}%
\begin{pgfscope}%
\pgfpathrectangle{\pgfqpoint{0.481978in}{0.331635in}}{\pgfqpoint{9.300000in}{7.700000in}}%
\pgfusepath{clip}%
\pgfsetbuttcap%
\pgfsetroundjoin%
\definecolor{currentfill}{rgb}{1.000000,0.623529,0.607843}%
\pgfsetfillcolor{currentfill}%
\pgfsetlinewidth{0.481800pt}%
\definecolor{currentstroke}{rgb}{1.000000,1.000000,1.000000}%
\pgfsetstrokecolor{currentstroke}%
\pgfsetdash{}{0pt}%
\pgfpathmoveto{\pgfqpoint{4.703040in}{4.110492in}}%
\pgfpathcurveto{\pgfqpoint{4.714090in}{4.110492in}}{\pgfqpoint{4.724689in}{4.114882in}}{\pgfqpoint{4.732503in}{4.122696in}}%
\pgfpathcurveto{\pgfqpoint{4.740316in}{4.130509in}}{\pgfqpoint{4.744706in}{4.141108in}}{\pgfqpoint{4.744706in}{4.152158in}}%
\pgfpathcurveto{\pgfqpoint{4.744706in}{4.163209in}}{\pgfqpoint{4.740316in}{4.173808in}}{\pgfqpoint{4.732503in}{4.181621in}}%
\pgfpathcurveto{\pgfqpoint{4.724689in}{4.189435in}}{\pgfqpoint{4.714090in}{4.193825in}}{\pgfqpoint{4.703040in}{4.193825in}}%
\pgfpathcurveto{\pgfqpoint{4.691990in}{4.193825in}}{\pgfqpoint{4.681391in}{4.189435in}}{\pgfqpoint{4.673577in}{4.181621in}}%
\pgfpathcurveto{\pgfqpoint{4.665763in}{4.173808in}}{\pgfqpoint{4.661373in}{4.163209in}}{\pgfqpoint{4.661373in}{4.152158in}}%
\pgfpathcurveto{\pgfqpoint{4.661373in}{4.141108in}}{\pgfqpoint{4.665763in}{4.130509in}}{\pgfqpoint{4.673577in}{4.122696in}}%
\pgfpathcurveto{\pgfqpoint{4.681391in}{4.114882in}}{\pgfqpoint{4.691990in}{4.110492in}}{\pgfqpoint{4.703040in}{4.110492in}}%
\pgfpathclose%
\pgfusepath{stroke,fill}%
\end{pgfscope}%
\begin{pgfscope}%
\pgfpathrectangle{\pgfqpoint{0.481978in}{0.331635in}}{\pgfqpoint{9.300000in}{7.700000in}}%
\pgfusepath{clip}%
\pgfsetbuttcap%
\pgfsetroundjoin%
\definecolor{currentfill}{rgb}{1.000000,0.623529,0.607843}%
\pgfsetfillcolor{currentfill}%
\pgfsetlinewidth{0.481800pt}%
\definecolor{currentstroke}{rgb}{1.000000,1.000000,1.000000}%
\pgfsetstrokecolor{currentstroke}%
\pgfsetdash{}{0pt}%
\pgfpathmoveto{\pgfqpoint{2.809853in}{3.839407in}}%
\pgfpathcurveto{\pgfqpoint{2.820903in}{3.839407in}}{\pgfqpoint{2.831502in}{3.843797in}}{\pgfqpoint{2.839316in}{3.851610in}}%
\pgfpathcurveto{\pgfqpoint{2.847129in}{3.859424in}}{\pgfqpoint{2.851520in}{3.870023in}}{\pgfqpoint{2.851520in}{3.881073in}}%
\pgfpathcurveto{\pgfqpoint{2.851520in}{3.892123in}}{\pgfqpoint{2.847129in}{3.902722in}}{\pgfqpoint{2.839316in}{3.910536in}}%
\pgfpathcurveto{\pgfqpoint{2.831502in}{3.918350in}}{\pgfqpoint{2.820903in}{3.922740in}}{\pgfqpoint{2.809853in}{3.922740in}}%
\pgfpathcurveto{\pgfqpoint{2.798803in}{3.922740in}}{\pgfqpoint{2.788204in}{3.918350in}}{\pgfqpoint{2.780390in}{3.910536in}}%
\pgfpathcurveto{\pgfqpoint{2.772577in}{3.902722in}}{\pgfqpoint{2.768186in}{3.892123in}}{\pgfqpoint{2.768186in}{3.881073in}}%
\pgfpathcurveto{\pgfqpoint{2.768186in}{3.870023in}}{\pgfqpoint{2.772577in}{3.859424in}}{\pgfqpoint{2.780390in}{3.851610in}}%
\pgfpathcurveto{\pgfqpoint{2.788204in}{3.843797in}}{\pgfqpoint{2.798803in}{3.839407in}}{\pgfqpoint{2.809853in}{3.839407in}}%
\pgfpathclose%
\pgfusepath{stroke,fill}%
\end{pgfscope}%
\begin{pgfscope}%
\pgfpathrectangle{\pgfqpoint{0.481978in}{0.331635in}}{\pgfqpoint{9.300000in}{7.700000in}}%
\pgfusepath{clip}%
\pgfsetbuttcap%
\pgfsetroundjoin%
\definecolor{currentfill}{rgb}{1.000000,0.623529,0.607843}%
\pgfsetfillcolor{currentfill}%
\pgfsetlinewidth{0.481800pt}%
\definecolor{currentstroke}{rgb}{1.000000,1.000000,1.000000}%
\pgfsetstrokecolor{currentstroke}%
\pgfsetdash{}{0pt}%
\pgfpathmoveto{\pgfqpoint{3.971771in}{3.008119in}}%
\pgfpathcurveto{\pgfqpoint{3.982821in}{3.008119in}}{\pgfqpoint{3.993421in}{3.012509in}}{\pgfqpoint{4.001234in}{3.020323in}}%
\pgfpathcurveto{\pgfqpoint{4.009048in}{3.028136in}}{\pgfqpoint{4.013438in}{3.038735in}}{\pgfqpoint{4.013438in}{3.049785in}}%
\pgfpathcurveto{\pgfqpoint{4.013438in}{3.060835in}}{\pgfqpoint{4.009048in}{3.071435in}}{\pgfqpoint{4.001234in}{3.079248in}}%
\pgfpathcurveto{\pgfqpoint{3.993421in}{3.087062in}}{\pgfqpoint{3.982821in}{3.091452in}}{\pgfqpoint{3.971771in}{3.091452in}}%
\pgfpathcurveto{\pgfqpoint{3.960721in}{3.091452in}}{\pgfqpoint{3.950122in}{3.087062in}}{\pgfqpoint{3.942309in}{3.079248in}}%
\pgfpathcurveto{\pgfqpoint{3.934495in}{3.071435in}}{\pgfqpoint{3.930105in}{3.060835in}}{\pgfqpoint{3.930105in}{3.049785in}}%
\pgfpathcurveto{\pgfqpoint{3.930105in}{3.038735in}}{\pgfqpoint{3.934495in}{3.028136in}}{\pgfqpoint{3.942309in}{3.020323in}}%
\pgfpathcurveto{\pgfqpoint{3.950122in}{3.012509in}}{\pgfqpoint{3.960721in}{3.008119in}}{\pgfqpoint{3.971771in}{3.008119in}}%
\pgfpathclose%
\pgfusepath{stroke,fill}%
\end{pgfscope}%
\begin{pgfscope}%
\pgfpathrectangle{\pgfqpoint{0.481978in}{0.331635in}}{\pgfqpoint{9.300000in}{7.700000in}}%
\pgfusepath{clip}%
\pgfsetbuttcap%
\pgfsetroundjoin%
\definecolor{currentfill}{rgb}{1.000000,0.623529,0.607843}%
\pgfsetfillcolor{currentfill}%
\pgfsetlinewidth{0.481800pt}%
\definecolor{currentstroke}{rgb}{1.000000,1.000000,1.000000}%
\pgfsetstrokecolor{currentstroke}%
\pgfsetdash{}{0pt}%
\pgfpathmoveto{\pgfqpoint{7.681317in}{5.060668in}}%
\pgfpathcurveto{\pgfqpoint{7.692368in}{5.060668in}}{\pgfqpoint{7.702967in}{5.065058in}}{\pgfqpoint{7.710780in}{5.072872in}}%
\pgfpathcurveto{\pgfqpoint{7.718594in}{5.080685in}}{\pgfqpoint{7.722984in}{5.091284in}}{\pgfqpoint{7.722984in}{5.102335in}}%
\pgfpathcurveto{\pgfqpoint{7.722984in}{5.113385in}}{\pgfqpoint{7.718594in}{5.123984in}}{\pgfqpoint{7.710780in}{5.131797in}}%
\pgfpathcurveto{\pgfqpoint{7.702967in}{5.139611in}}{\pgfqpoint{7.692368in}{5.144001in}}{\pgfqpoint{7.681317in}{5.144001in}}%
\pgfpathcurveto{\pgfqpoint{7.670267in}{5.144001in}}{\pgfqpoint{7.659668in}{5.139611in}}{\pgfqpoint{7.651855in}{5.131797in}}%
\pgfpathcurveto{\pgfqpoint{7.644041in}{5.123984in}}{\pgfqpoint{7.639651in}{5.113385in}}{\pgfqpoint{7.639651in}{5.102335in}}%
\pgfpathcurveto{\pgfqpoint{7.639651in}{5.091284in}}{\pgfqpoint{7.644041in}{5.080685in}}{\pgfqpoint{7.651855in}{5.072872in}}%
\pgfpathcurveto{\pgfqpoint{7.659668in}{5.065058in}}{\pgfqpoint{7.670267in}{5.060668in}}{\pgfqpoint{7.681317in}{5.060668in}}%
\pgfpathclose%
\pgfusepath{stroke,fill}%
\end{pgfscope}%
\begin{pgfscope}%
\pgfpathrectangle{\pgfqpoint{0.481978in}{0.331635in}}{\pgfqpoint{9.300000in}{7.700000in}}%
\pgfusepath{clip}%
\pgfsetbuttcap%
\pgfsetroundjoin%
\definecolor{currentfill}{rgb}{1.000000,0.623529,0.607843}%
\pgfsetfillcolor{currentfill}%
\pgfsetlinewidth{0.481800pt}%
\definecolor{currentstroke}{rgb}{1.000000,1.000000,1.000000}%
\pgfsetstrokecolor{currentstroke}%
\pgfsetdash{}{0pt}%
\pgfpathmoveto{\pgfqpoint{4.922074in}{3.099400in}}%
\pgfpathcurveto{\pgfqpoint{4.933124in}{3.099400in}}{\pgfqpoint{4.943723in}{3.103790in}}{\pgfqpoint{4.951537in}{3.111604in}}%
\pgfpathcurveto{\pgfqpoint{4.959350in}{3.119417in}}{\pgfqpoint{4.963741in}{3.130016in}}{\pgfqpoint{4.963741in}{3.141067in}}%
\pgfpathcurveto{\pgfqpoint{4.963741in}{3.152117in}}{\pgfqpoint{4.959350in}{3.162716in}}{\pgfqpoint{4.951537in}{3.170529in}}%
\pgfpathcurveto{\pgfqpoint{4.943723in}{3.178343in}}{\pgfqpoint{4.933124in}{3.182733in}}{\pgfqpoint{4.922074in}{3.182733in}}%
\pgfpathcurveto{\pgfqpoint{4.911024in}{3.182733in}}{\pgfqpoint{4.900425in}{3.178343in}}{\pgfqpoint{4.892611in}{3.170529in}}%
\pgfpathcurveto{\pgfqpoint{4.884798in}{3.162716in}}{\pgfqpoint{4.880407in}{3.152117in}}{\pgfqpoint{4.880407in}{3.141067in}}%
\pgfpathcurveto{\pgfqpoint{4.880407in}{3.130016in}}{\pgfqpoint{4.884798in}{3.119417in}}{\pgfqpoint{4.892611in}{3.111604in}}%
\pgfpathcurveto{\pgfqpoint{4.900425in}{3.103790in}}{\pgfqpoint{4.911024in}{3.099400in}}{\pgfqpoint{4.922074in}{3.099400in}}%
\pgfpathclose%
\pgfusepath{stroke,fill}%
\end{pgfscope}%
\begin{pgfscope}%
\pgfpathrectangle{\pgfqpoint{0.481978in}{0.331635in}}{\pgfqpoint{9.300000in}{7.700000in}}%
\pgfusepath{clip}%
\pgfsetbuttcap%
\pgfsetroundjoin%
\definecolor{currentfill}{rgb}{1.000000,0.623529,0.607843}%
\pgfsetfillcolor{currentfill}%
\pgfsetlinewidth{0.481800pt}%
\definecolor{currentstroke}{rgb}{1.000000,1.000000,1.000000}%
\pgfsetstrokecolor{currentstroke}%
\pgfsetdash{}{0pt}%
\pgfpathmoveto{\pgfqpoint{4.349051in}{3.104492in}}%
\pgfpathcurveto{\pgfqpoint{4.360101in}{3.104492in}}{\pgfqpoint{4.370700in}{3.108882in}}{\pgfqpoint{4.378514in}{3.116696in}}%
\pgfpathcurveto{\pgfqpoint{4.386327in}{3.124510in}}{\pgfqpoint{4.390718in}{3.135109in}}{\pgfqpoint{4.390718in}{3.146159in}}%
\pgfpathcurveto{\pgfqpoint{4.390718in}{3.157209in}}{\pgfqpoint{4.386327in}{3.167808in}}{\pgfqpoint{4.378514in}{3.175621in}}%
\pgfpathcurveto{\pgfqpoint{4.370700in}{3.183435in}}{\pgfqpoint{4.360101in}{3.187825in}}{\pgfqpoint{4.349051in}{3.187825in}}%
\pgfpathcurveto{\pgfqpoint{4.338001in}{3.187825in}}{\pgfqpoint{4.327402in}{3.183435in}}{\pgfqpoint{4.319588in}{3.175621in}}%
\pgfpathcurveto{\pgfqpoint{4.311775in}{3.167808in}}{\pgfqpoint{4.307384in}{3.157209in}}{\pgfqpoint{4.307384in}{3.146159in}}%
\pgfpathcurveto{\pgfqpoint{4.307384in}{3.135109in}}{\pgfqpoint{4.311775in}{3.124510in}}{\pgfqpoint{4.319588in}{3.116696in}}%
\pgfpathcurveto{\pgfqpoint{4.327402in}{3.108882in}}{\pgfqpoint{4.338001in}{3.104492in}}{\pgfqpoint{4.349051in}{3.104492in}}%
\pgfpathclose%
\pgfusepath{stroke,fill}%
\end{pgfscope}%
\begin{pgfscope}%
\pgfpathrectangle{\pgfqpoint{0.481978in}{0.331635in}}{\pgfqpoint{9.300000in}{7.700000in}}%
\pgfusepath{clip}%
\pgfsetbuttcap%
\pgfsetroundjoin%
\definecolor{currentfill}{rgb}{1.000000,0.623529,0.607843}%
\pgfsetfillcolor{currentfill}%
\pgfsetlinewidth{0.481800pt}%
\definecolor{currentstroke}{rgb}{1.000000,1.000000,1.000000}%
\pgfsetstrokecolor{currentstroke}%
\pgfsetdash{}{0pt}%
\pgfpathmoveto{\pgfqpoint{4.285306in}{3.855591in}}%
\pgfpathcurveto{\pgfqpoint{4.296356in}{3.855591in}}{\pgfqpoint{4.306955in}{3.859981in}}{\pgfqpoint{4.314769in}{3.867794in}}%
\pgfpathcurveto{\pgfqpoint{4.322583in}{3.875608in}}{\pgfqpoint{4.326973in}{3.886207in}}{\pgfqpoint{4.326973in}{3.897257in}}%
\pgfpathcurveto{\pgfqpoint{4.326973in}{3.908307in}}{\pgfqpoint{4.322583in}{3.918906in}}{\pgfqpoint{4.314769in}{3.926720in}}%
\pgfpathcurveto{\pgfqpoint{4.306955in}{3.934534in}}{\pgfqpoint{4.296356in}{3.938924in}}{\pgfqpoint{4.285306in}{3.938924in}}%
\pgfpathcurveto{\pgfqpoint{4.274256in}{3.938924in}}{\pgfqpoint{4.263657in}{3.934534in}}{\pgfqpoint{4.255843in}{3.926720in}}%
\pgfpathcurveto{\pgfqpoint{4.248030in}{3.918906in}}{\pgfqpoint{4.243639in}{3.908307in}}{\pgfqpoint{4.243639in}{3.897257in}}%
\pgfpathcurveto{\pgfqpoint{4.243639in}{3.886207in}}{\pgfqpoint{4.248030in}{3.875608in}}{\pgfqpoint{4.255843in}{3.867794in}}%
\pgfpathcurveto{\pgfqpoint{4.263657in}{3.859981in}}{\pgfqpoint{4.274256in}{3.855591in}}{\pgfqpoint{4.285306in}{3.855591in}}%
\pgfpathclose%
\pgfusepath{stroke,fill}%
\end{pgfscope}%
\begin{pgfscope}%
\pgfpathrectangle{\pgfqpoint{0.481978in}{0.331635in}}{\pgfqpoint{9.300000in}{7.700000in}}%
\pgfusepath{clip}%
\pgfsetbuttcap%
\pgfsetroundjoin%
\definecolor{currentfill}{rgb}{1.000000,0.623529,0.607843}%
\pgfsetfillcolor{currentfill}%
\pgfsetlinewidth{0.481800pt}%
\definecolor{currentstroke}{rgb}{1.000000,1.000000,1.000000}%
\pgfsetstrokecolor{currentstroke}%
\pgfsetdash{}{0pt}%
\pgfpathmoveto{\pgfqpoint{5.628759in}{1.318174in}}%
\pgfpathcurveto{\pgfqpoint{5.639809in}{1.318174in}}{\pgfqpoint{5.650408in}{1.322564in}}{\pgfqpoint{5.658221in}{1.330378in}}%
\pgfpathcurveto{\pgfqpoint{5.666035in}{1.338191in}}{\pgfqpoint{5.670425in}{1.348790in}}{\pgfqpoint{5.670425in}{1.359840in}}%
\pgfpathcurveto{\pgfqpoint{5.670425in}{1.370890in}}{\pgfqpoint{5.666035in}{1.381489in}}{\pgfqpoint{5.658221in}{1.389303in}}%
\pgfpathcurveto{\pgfqpoint{5.650408in}{1.397117in}}{\pgfqpoint{5.639809in}{1.401507in}}{\pgfqpoint{5.628759in}{1.401507in}}%
\pgfpathcurveto{\pgfqpoint{5.617708in}{1.401507in}}{\pgfqpoint{5.607109in}{1.397117in}}{\pgfqpoint{5.599296in}{1.389303in}}%
\pgfpathcurveto{\pgfqpoint{5.591482in}{1.381489in}}{\pgfqpoint{5.587092in}{1.370890in}}{\pgfqpoint{5.587092in}{1.359840in}}%
\pgfpathcurveto{\pgfqpoint{5.587092in}{1.348790in}}{\pgfqpoint{5.591482in}{1.338191in}}{\pgfqpoint{5.599296in}{1.330378in}}%
\pgfpathcurveto{\pgfqpoint{5.607109in}{1.322564in}}{\pgfqpoint{5.617708in}{1.318174in}}{\pgfqpoint{5.628759in}{1.318174in}}%
\pgfpathclose%
\pgfusepath{stroke,fill}%
\end{pgfscope}%
\begin{pgfscope}%
\pgfpathrectangle{\pgfqpoint{0.481978in}{0.331635in}}{\pgfqpoint{9.300000in}{7.700000in}}%
\pgfusepath{clip}%
\pgfsetbuttcap%
\pgfsetroundjoin%
\definecolor{currentfill}{rgb}{1.000000,0.623529,0.607843}%
\pgfsetfillcolor{currentfill}%
\pgfsetlinewidth{0.481800pt}%
\definecolor{currentstroke}{rgb}{1.000000,1.000000,1.000000}%
\pgfsetstrokecolor{currentstroke}%
\pgfsetdash{}{0pt}%
\pgfpathmoveto{\pgfqpoint{2.208081in}{3.690280in}}%
\pgfpathcurveto{\pgfqpoint{2.219131in}{3.690280in}}{\pgfqpoint{2.229730in}{3.694670in}}{\pgfqpoint{2.237544in}{3.702484in}}%
\pgfpathcurveto{\pgfqpoint{2.245357in}{3.710298in}}{\pgfqpoint{2.249748in}{3.720897in}}{\pgfqpoint{2.249748in}{3.731947in}}%
\pgfpathcurveto{\pgfqpoint{2.249748in}{3.742997in}}{\pgfqpoint{2.245357in}{3.753596in}}{\pgfqpoint{2.237544in}{3.761410in}}%
\pgfpathcurveto{\pgfqpoint{2.229730in}{3.769223in}}{\pgfqpoint{2.219131in}{3.773614in}}{\pgfqpoint{2.208081in}{3.773614in}}%
\pgfpathcurveto{\pgfqpoint{2.197031in}{3.773614in}}{\pgfqpoint{2.186432in}{3.769223in}}{\pgfqpoint{2.178618in}{3.761410in}}%
\pgfpathcurveto{\pgfqpoint{2.170805in}{3.753596in}}{\pgfqpoint{2.166414in}{3.742997in}}{\pgfqpoint{2.166414in}{3.731947in}}%
\pgfpathcurveto{\pgfqpoint{2.166414in}{3.720897in}}{\pgfqpoint{2.170805in}{3.710298in}}{\pgfqpoint{2.178618in}{3.702484in}}%
\pgfpathcurveto{\pgfqpoint{2.186432in}{3.694670in}}{\pgfqpoint{2.197031in}{3.690280in}}{\pgfqpoint{2.208081in}{3.690280in}}%
\pgfpathclose%
\pgfusepath{stroke,fill}%
\end{pgfscope}%
\begin{pgfscope}%
\pgfpathrectangle{\pgfqpoint{0.481978in}{0.331635in}}{\pgfqpoint{9.300000in}{7.700000in}}%
\pgfusepath{clip}%
\pgfsetbuttcap%
\pgfsetroundjoin%
\definecolor{currentfill}{rgb}{1.000000,0.623529,0.607843}%
\pgfsetfillcolor{currentfill}%
\pgfsetlinewidth{0.481800pt}%
\definecolor{currentstroke}{rgb}{1.000000,1.000000,1.000000}%
\pgfsetstrokecolor{currentstroke}%
\pgfsetdash{}{0pt}%
\pgfpathmoveto{\pgfqpoint{5.627987in}{1.316150in}}%
\pgfpathcurveto{\pgfqpoint{5.639037in}{1.316150in}}{\pgfqpoint{5.649636in}{1.320540in}}{\pgfqpoint{5.657450in}{1.328354in}}%
\pgfpathcurveto{\pgfqpoint{5.665263in}{1.336167in}}{\pgfqpoint{5.669654in}{1.346766in}}{\pgfqpoint{5.669654in}{1.357817in}}%
\pgfpathcurveto{\pgfqpoint{5.669654in}{1.368867in}}{\pgfqpoint{5.665263in}{1.379466in}}{\pgfqpoint{5.657450in}{1.387279in}}%
\pgfpathcurveto{\pgfqpoint{5.649636in}{1.395093in}}{\pgfqpoint{5.639037in}{1.399483in}}{\pgfqpoint{5.627987in}{1.399483in}}%
\pgfpathcurveto{\pgfqpoint{5.616937in}{1.399483in}}{\pgfqpoint{5.606338in}{1.395093in}}{\pgfqpoint{5.598524in}{1.387279in}}%
\pgfpathcurveto{\pgfqpoint{5.590711in}{1.379466in}}{\pgfqpoint{5.586320in}{1.368867in}}{\pgfqpoint{5.586320in}{1.357817in}}%
\pgfpathcurveto{\pgfqpoint{5.586320in}{1.346766in}}{\pgfqpoint{5.590711in}{1.336167in}}{\pgfqpoint{5.598524in}{1.328354in}}%
\pgfpathcurveto{\pgfqpoint{5.606338in}{1.320540in}}{\pgfqpoint{5.616937in}{1.316150in}}{\pgfqpoint{5.627987in}{1.316150in}}%
\pgfpathclose%
\pgfusepath{stroke,fill}%
\end{pgfscope}%
\begin{pgfscope}%
\pgfpathrectangle{\pgfqpoint{0.481978in}{0.331635in}}{\pgfqpoint{9.300000in}{7.700000in}}%
\pgfusepath{clip}%
\pgfsetbuttcap%
\pgfsetroundjoin%
\definecolor{currentfill}{rgb}{1.000000,0.623529,0.607843}%
\pgfsetfillcolor{currentfill}%
\pgfsetlinewidth{0.481800pt}%
\definecolor{currentstroke}{rgb}{1.000000,1.000000,1.000000}%
\pgfsetstrokecolor{currentstroke}%
\pgfsetdash{}{0pt}%
\pgfpathmoveto{\pgfqpoint{7.678869in}{4.472682in}}%
\pgfpathcurveto{\pgfqpoint{7.689919in}{4.472682in}}{\pgfqpoint{7.700518in}{4.477072in}}{\pgfqpoint{7.708332in}{4.484886in}}%
\pgfpathcurveto{\pgfqpoint{7.716145in}{4.492700in}}{\pgfqpoint{7.720536in}{4.503299in}}{\pgfqpoint{7.720536in}{4.514349in}}%
\pgfpathcurveto{\pgfqpoint{7.720536in}{4.525399in}}{\pgfqpoint{7.716145in}{4.535998in}}{\pgfqpoint{7.708332in}{4.543812in}}%
\pgfpathcurveto{\pgfqpoint{7.700518in}{4.551625in}}{\pgfqpoint{7.689919in}{4.556015in}}{\pgfqpoint{7.678869in}{4.556015in}}%
\pgfpathcurveto{\pgfqpoint{7.667819in}{4.556015in}}{\pgfqpoint{7.657220in}{4.551625in}}{\pgfqpoint{7.649406in}{4.543812in}}%
\pgfpathcurveto{\pgfqpoint{7.641593in}{4.535998in}}{\pgfqpoint{7.637202in}{4.525399in}}{\pgfqpoint{7.637202in}{4.514349in}}%
\pgfpathcurveto{\pgfqpoint{7.637202in}{4.503299in}}{\pgfqpoint{7.641593in}{4.492700in}}{\pgfqpoint{7.649406in}{4.484886in}}%
\pgfpathcurveto{\pgfqpoint{7.657220in}{4.477072in}}{\pgfqpoint{7.667819in}{4.472682in}}{\pgfqpoint{7.678869in}{4.472682in}}%
\pgfpathclose%
\pgfusepath{stroke,fill}%
\end{pgfscope}%
\begin{pgfscope}%
\pgfpathrectangle{\pgfqpoint{0.481978in}{0.331635in}}{\pgfqpoint{9.300000in}{7.700000in}}%
\pgfusepath{clip}%
\pgfsetbuttcap%
\pgfsetroundjoin%
\definecolor{currentfill}{rgb}{1.000000,0.623529,0.607843}%
\pgfsetfillcolor{currentfill}%
\pgfsetlinewidth{0.481800pt}%
\definecolor{currentstroke}{rgb}{1.000000,1.000000,1.000000}%
\pgfsetstrokecolor{currentstroke}%
\pgfsetdash{}{0pt}%
\pgfpathmoveto{\pgfqpoint{3.663088in}{1.980706in}}%
\pgfpathcurveto{\pgfqpoint{3.674139in}{1.980706in}}{\pgfqpoint{3.684738in}{1.985096in}}{\pgfqpoint{3.692551in}{1.992910in}}%
\pgfpathcurveto{\pgfqpoint{3.700365in}{2.000723in}}{\pgfqpoint{3.704755in}{2.011322in}}{\pgfqpoint{3.704755in}{2.022372in}}%
\pgfpathcurveto{\pgfqpoint{3.704755in}{2.033423in}}{\pgfqpoint{3.700365in}{2.044022in}}{\pgfqpoint{3.692551in}{2.051835in}}%
\pgfpathcurveto{\pgfqpoint{3.684738in}{2.059649in}}{\pgfqpoint{3.674139in}{2.064039in}}{\pgfqpoint{3.663088in}{2.064039in}}%
\pgfpathcurveto{\pgfqpoint{3.652038in}{2.064039in}}{\pgfqpoint{3.641439in}{2.059649in}}{\pgfqpoint{3.633626in}{2.051835in}}%
\pgfpathcurveto{\pgfqpoint{3.625812in}{2.044022in}}{\pgfqpoint{3.621422in}{2.033423in}}{\pgfqpoint{3.621422in}{2.022372in}}%
\pgfpathcurveto{\pgfqpoint{3.621422in}{2.011322in}}{\pgfqpoint{3.625812in}{2.000723in}}{\pgfqpoint{3.633626in}{1.992910in}}%
\pgfpathcurveto{\pgfqpoint{3.641439in}{1.985096in}}{\pgfqpoint{3.652038in}{1.980706in}}{\pgfqpoint{3.663088in}{1.980706in}}%
\pgfpathclose%
\pgfusepath{stroke,fill}%
\end{pgfscope}%
\begin{pgfscope}%
\pgfpathrectangle{\pgfqpoint{0.481978in}{0.331635in}}{\pgfqpoint{9.300000in}{7.700000in}}%
\pgfusepath{clip}%
\pgfsetbuttcap%
\pgfsetroundjoin%
\definecolor{currentfill}{rgb}{1.000000,0.623529,0.607843}%
\pgfsetfillcolor{currentfill}%
\pgfsetlinewidth{0.481800pt}%
\definecolor{currentstroke}{rgb}{1.000000,1.000000,1.000000}%
\pgfsetstrokecolor{currentstroke}%
\pgfsetdash{}{0pt}%
\pgfpathmoveto{\pgfqpoint{2.974676in}{3.669425in}}%
\pgfpathcurveto{\pgfqpoint{2.985726in}{3.669425in}}{\pgfqpoint{2.996325in}{3.673815in}}{\pgfqpoint{3.004138in}{3.681629in}}%
\pgfpathcurveto{\pgfqpoint{3.011952in}{3.689442in}}{\pgfqpoint{3.016342in}{3.700041in}}{\pgfqpoint{3.016342in}{3.711091in}}%
\pgfpathcurveto{\pgfqpoint{3.016342in}{3.722141in}}{\pgfqpoint{3.011952in}{3.732741in}}{\pgfqpoint{3.004138in}{3.740554in}}%
\pgfpathcurveto{\pgfqpoint{2.996325in}{3.748368in}}{\pgfqpoint{2.985726in}{3.752758in}}{\pgfqpoint{2.974676in}{3.752758in}}%
\pgfpathcurveto{\pgfqpoint{2.963625in}{3.752758in}}{\pgfqpoint{2.953026in}{3.748368in}}{\pgfqpoint{2.945213in}{3.740554in}}%
\pgfpathcurveto{\pgfqpoint{2.937399in}{3.732741in}}{\pgfqpoint{2.933009in}{3.722141in}}{\pgfqpoint{2.933009in}{3.711091in}}%
\pgfpathcurveto{\pgfqpoint{2.933009in}{3.700041in}}{\pgfqpoint{2.937399in}{3.689442in}}{\pgfqpoint{2.945213in}{3.681629in}}%
\pgfpathcurveto{\pgfqpoint{2.953026in}{3.673815in}}{\pgfqpoint{2.963625in}{3.669425in}}{\pgfqpoint{2.974676in}{3.669425in}}%
\pgfpathclose%
\pgfusepath{stroke,fill}%
\end{pgfscope}%
\begin{pgfscope}%
\pgfpathrectangle{\pgfqpoint{0.481978in}{0.331635in}}{\pgfqpoint{9.300000in}{7.700000in}}%
\pgfusepath{clip}%
\pgfsetbuttcap%
\pgfsetroundjoin%
\definecolor{currentfill}{rgb}{1.000000,0.623529,0.607843}%
\pgfsetfillcolor{currentfill}%
\pgfsetlinewidth{0.481800pt}%
\definecolor{currentstroke}{rgb}{1.000000,1.000000,1.000000}%
\pgfsetstrokecolor{currentstroke}%
\pgfsetdash{}{0pt}%
\pgfpathmoveto{\pgfqpoint{4.287456in}{3.550314in}}%
\pgfpathcurveto{\pgfqpoint{4.298506in}{3.550314in}}{\pgfqpoint{4.309105in}{3.554704in}}{\pgfqpoint{4.316919in}{3.562518in}}%
\pgfpathcurveto{\pgfqpoint{4.324732in}{3.570331in}}{\pgfqpoint{4.329122in}{3.580930in}}{\pgfqpoint{4.329122in}{3.591981in}}%
\pgfpathcurveto{\pgfqpoint{4.329122in}{3.603031in}}{\pgfqpoint{4.324732in}{3.613630in}}{\pgfqpoint{4.316919in}{3.621443in}}%
\pgfpathcurveto{\pgfqpoint{4.309105in}{3.629257in}}{\pgfqpoint{4.298506in}{3.633647in}}{\pgfqpoint{4.287456in}{3.633647in}}%
\pgfpathcurveto{\pgfqpoint{4.276406in}{3.633647in}}{\pgfqpoint{4.265807in}{3.629257in}}{\pgfqpoint{4.257993in}{3.621443in}}%
\pgfpathcurveto{\pgfqpoint{4.250179in}{3.613630in}}{\pgfqpoint{4.245789in}{3.603031in}}{\pgfqpoint{4.245789in}{3.591981in}}%
\pgfpathcurveto{\pgfqpoint{4.245789in}{3.580930in}}{\pgfqpoint{4.250179in}{3.570331in}}{\pgfqpoint{4.257993in}{3.562518in}}%
\pgfpathcurveto{\pgfqpoint{4.265807in}{3.554704in}}{\pgfqpoint{4.276406in}{3.550314in}}{\pgfqpoint{4.287456in}{3.550314in}}%
\pgfpathclose%
\pgfusepath{stroke,fill}%
\end{pgfscope}%
\begin{pgfscope}%
\pgfpathrectangle{\pgfqpoint{0.481978in}{0.331635in}}{\pgfqpoint{9.300000in}{7.700000in}}%
\pgfusepath{clip}%
\pgfsetbuttcap%
\pgfsetroundjoin%
\definecolor{currentfill}{rgb}{1.000000,0.623529,0.607843}%
\pgfsetfillcolor{currentfill}%
\pgfsetlinewidth{0.481800pt}%
\definecolor{currentstroke}{rgb}{1.000000,1.000000,1.000000}%
\pgfsetstrokecolor{currentstroke}%
\pgfsetdash{}{0pt}%
\pgfpathmoveto{\pgfqpoint{1.967877in}{5.548168in}}%
\pgfpathcurveto{\pgfqpoint{1.978927in}{5.548168in}}{\pgfqpoint{1.989526in}{5.552558in}}{\pgfqpoint{1.997339in}{5.560372in}}%
\pgfpathcurveto{\pgfqpoint{2.005153in}{5.568185in}}{\pgfqpoint{2.009543in}{5.578784in}}{\pgfqpoint{2.009543in}{5.589834in}}%
\pgfpathcurveto{\pgfqpoint{2.009543in}{5.600885in}}{\pgfqpoint{2.005153in}{5.611484in}}{\pgfqpoint{1.997339in}{5.619297in}}%
\pgfpathcurveto{\pgfqpoint{1.989526in}{5.627111in}}{\pgfqpoint{1.978927in}{5.631501in}}{\pgfqpoint{1.967877in}{5.631501in}}%
\pgfpathcurveto{\pgfqpoint{1.956827in}{5.631501in}}{\pgfqpoint{1.946228in}{5.627111in}}{\pgfqpoint{1.938414in}{5.619297in}}%
\pgfpathcurveto{\pgfqpoint{1.930600in}{5.611484in}}{\pgfqpoint{1.926210in}{5.600885in}}{\pgfqpoint{1.926210in}{5.589834in}}%
\pgfpathcurveto{\pgfqpoint{1.926210in}{5.578784in}}{\pgfqpoint{1.930600in}{5.568185in}}{\pgfqpoint{1.938414in}{5.560372in}}%
\pgfpathcurveto{\pgfqpoint{1.946228in}{5.552558in}}{\pgfqpoint{1.956827in}{5.548168in}}{\pgfqpoint{1.967877in}{5.548168in}}%
\pgfpathclose%
\pgfusepath{stroke,fill}%
\end{pgfscope}%
\begin{pgfscope}%
\pgfpathrectangle{\pgfqpoint{0.481978in}{0.331635in}}{\pgfqpoint{9.300000in}{7.700000in}}%
\pgfusepath{clip}%
\pgfsetbuttcap%
\pgfsetroundjoin%
\definecolor{currentfill}{rgb}{1.000000,0.623529,0.607843}%
\pgfsetfillcolor{currentfill}%
\pgfsetlinewidth{0.481800pt}%
\definecolor{currentstroke}{rgb}{1.000000,1.000000,1.000000}%
\pgfsetstrokecolor{currentstroke}%
\pgfsetdash{}{0pt}%
\pgfpathmoveto{\pgfqpoint{3.570734in}{2.892749in}}%
\pgfpathcurveto{\pgfqpoint{3.581784in}{2.892749in}}{\pgfqpoint{3.592383in}{2.897139in}}{\pgfqpoint{3.600197in}{2.904953in}}%
\pgfpathcurveto{\pgfqpoint{3.608010in}{2.912766in}}{\pgfqpoint{3.612401in}{2.923365in}}{\pgfqpoint{3.612401in}{2.934416in}}%
\pgfpathcurveto{\pgfqpoint{3.612401in}{2.945466in}}{\pgfqpoint{3.608010in}{2.956065in}}{\pgfqpoint{3.600197in}{2.963878in}}%
\pgfpathcurveto{\pgfqpoint{3.592383in}{2.971692in}}{\pgfqpoint{3.581784in}{2.976082in}}{\pgfqpoint{3.570734in}{2.976082in}}%
\pgfpathcurveto{\pgfqpoint{3.559684in}{2.976082in}}{\pgfqpoint{3.549085in}{2.971692in}}{\pgfqpoint{3.541271in}{2.963878in}}%
\pgfpathcurveto{\pgfqpoint{3.533458in}{2.956065in}}{\pgfqpoint{3.529067in}{2.945466in}}{\pgfqpoint{3.529067in}{2.934416in}}%
\pgfpathcurveto{\pgfqpoint{3.529067in}{2.923365in}}{\pgfqpoint{3.533458in}{2.912766in}}{\pgfqpoint{3.541271in}{2.904953in}}%
\pgfpathcurveto{\pgfqpoint{3.549085in}{2.897139in}}{\pgfqpoint{3.559684in}{2.892749in}}{\pgfqpoint{3.570734in}{2.892749in}}%
\pgfpathclose%
\pgfusepath{stroke,fill}%
\end{pgfscope}%
\begin{pgfscope}%
\pgfpathrectangle{\pgfqpoint{0.481978in}{0.331635in}}{\pgfqpoint{9.300000in}{7.700000in}}%
\pgfusepath{clip}%
\pgfsetbuttcap%
\pgfsetroundjoin%
\definecolor{currentfill}{rgb}{1.000000,0.623529,0.607843}%
\pgfsetfillcolor{currentfill}%
\pgfsetlinewidth{0.481800pt}%
\definecolor{currentstroke}{rgb}{1.000000,1.000000,1.000000}%
\pgfsetstrokecolor{currentstroke}%
\pgfsetdash{}{0pt}%
\pgfpathmoveto{\pgfqpoint{5.661130in}{4.631812in}}%
\pgfpathcurveto{\pgfqpoint{5.672180in}{4.631812in}}{\pgfqpoint{5.682779in}{4.636202in}}{\pgfqpoint{5.690593in}{4.644015in}}%
\pgfpathcurveto{\pgfqpoint{5.698407in}{4.651829in}}{\pgfqpoint{5.702797in}{4.662428in}}{\pgfqpoint{5.702797in}{4.673478in}}%
\pgfpathcurveto{\pgfqpoint{5.702797in}{4.684528in}}{\pgfqpoint{5.698407in}{4.695127in}}{\pgfqpoint{5.690593in}{4.702941in}}%
\pgfpathcurveto{\pgfqpoint{5.682779in}{4.710755in}}{\pgfqpoint{5.672180in}{4.715145in}}{\pgfqpoint{5.661130in}{4.715145in}}%
\pgfpathcurveto{\pgfqpoint{5.650080in}{4.715145in}}{\pgfqpoint{5.639481in}{4.710755in}}{\pgfqpoint{5.631667in}{4.702941in}}%
\pgfpathcurveto{\pgfqpoint{5.623854in}{4.695127in}}{\pgfqpoint{5.619463in}{4.684528in}}{\pgfqpoint{5.619463in}{4.673478in}}%
\pgfpathcurveto{\pgfqpoint{5.619463in}{4.662428in}}{\pgfqpoint{5.623854in}{4.651829in}}{\pgfqpoint{5.631667in}{4.644015in}}%
\pgfpathcurveto{\pgfqpoint{5.639481in}{4.636202in}}{\pgfqpoint{5.650080in}{4.631812in}}{\pgfqpoint{5.661130in}{4.631812in}}%
\pgfpathclose%
\pgfusepath{stroke,fill}%
\end{pgfscope}%
\begin{pgfscope}%
\pgfpathrectangle{\pgfqpoint{0.481978in}{0.331635in}}{\pgfqpoint{9.300000in}{7.700000in}}%
\pgfusepath{clip}%
\pgfsetbuttcap%
\pgfsetroundjoin%
\definecolor{currentfill}{rgb}{1.000000,0.623529,0.607843}%
\pgfsetfillcolor{currentfill}%
\pgfsetlinewidth{0.481800pt}%
\definecolor{currentstroke}{rgb}{1.000000,1.000000,1.000000}%
\pgfsetstrokecolor{currentstroke}%
\pgfsetdash{}{0pt}%
\pgfpathmoveto{\pgfqpoint{4.720327in}{4.321572in}}%
\pgfpathcurveto{\pgfqpoint{4.731377in}{4.321572in}}{\pgfqpoint{4.741976in}{4.325963in}}{\pgfqpoint{4.749790in}{4.333776in}}%
\pgfpathcurveto{\pgfqpoint{4.757604in}{4.341590in}}{\pgfqpoint{4.761994in}{4.352189in}}{\pgfqpoint{4.761994in}{4.363239in}}%
\pgfpathcurveto{\pgfqpoint{4.761994in}{4.374289in}}{\pgfqpoint{4.757604in}{4.384888in}}{\pgfqpoint{4.749790in}{4.392702in}}%
\pgfpathcurveto{\pgfqpoint{4.741976in}{4.400515in}}{\pgfqpoint{4.731377in}{4.404906in}}{\pgfqpoint{4.720327in}{4.404906in}}%
\pgfpathcurveto{\pgfqpoint{4.709277in}{4.404906in}}{\pgfqpoint{4.698678in}{4.400515in}}{\pgfqpoint{4.690864in}{4.392702in}}%
\pgfpathcurveto{\pgfqpoint{4.683051in}{4.384888in}}{\pgfqpoint{4.678660in}{4.374289in}}{\pgfqpoint{4.678660in}{4.363239in}}%
\pgfpathcurveto{\pgfqpoint{4.678660in}{4.352189in}}{\pgfqpoint{4.683051in}{4.341590in}}{\pgfqpoint{4.690864in}{4.333776in}}%
\pgfpathcurveto{\pgfqpoint{4.698678in}{4.325963in}}{\pgfqpoint{4.709277in}{4.321572in}}{\pgfqpoint{4.720327in}{4.321572in}}%
\pgfpathclose%
\pgfusepath{stroke,fill}%
\end{pgfscope}%
\begin{pgfscope}%
\pgfpathrectangle{\pgfqpoint{0.481978in}{0.331635in}}{\pgfqpoint{9.300000in}{7.700000in}}%
\pgfusepath{clip}%
\pgfsetbuttcap%
\pgfsetroundjoin%
\definecolor{currentfill}{rgb}{1.000000,0.623529,0.607843}%
\pgfsetfillcolor{currentfill}%
\pgfsetlinewidth{0.481800pt}%
\definecolor{currentstroke}{rgb}{1.000000,1.000000,1.000000}%
\pgfsetstrokecolor{currentstroke}%
\pgfsetdash{}{0pt}%
\pgfpathmoveto{\pgfqpoint{2.339602in}{4.088730in}}%
\pgfpathcurveto{\pgfqpoint{2.350652in}{4.088730in}}{\pgfqpoint{2.361251in}{4.093120in}}{\pgfqpoint{2.369065in}{4.100934in}}%
\pgfpathcurveto{\pgfqpoint{2.376879in}{4.108747in}}{\pgfqpoint{2.381269in}{4.119346in}}{\pgfqpoint{2.381269in}{4.130396in}}%
\pgfpathcurveto{\pgfqpoint{2.381269in}{4.141446in}}{\pgfqpoint{2.376879in}{4.152045in}}{\pgfqpoint{2.369065in}{4.159859in}}%
\pgfpathcurveto{\pgfqpoint{2.361251in}{4.167673in}}{\pgfqpoint{2.350652in}{4.172063in}}{\pgfqpoint{2.339602in}{4.172063in}}%
\pgfpathcurveto{\pgfqpoint{2.328552in}{4.172063in}}{\pgfqpoint{2.317953in}{4.167673in}}{\pgfqpoint{2.310139in}{4.159859in}}%
\pgfpathcurveto{\pgfqpoint{2.302326in}{4.152045in}}{\pgfqpoint{2.297936in}{4.141446in}}{\pgfqpoint{2.297936in}{4.130396in}}%
\pgfpathcurveto{\pgfqpoint{2.297936in}{4.119346in}}{\pgfqpoint{2.302326in}{4.108747in}}{\pgfqpoint{2.310139in}{4.100934in}}%
\pgfpathcurveto{\pgfqpoint{2.317953in}{4.093120in}}{\pgfqpoint{2.328552in}{4.088730in}}{\pgfqpoint{2.339602in}{4.088730in}}%
\pgfpathclose%
\pgfusepath{stroke,fill}%
\end{pgfscope}%
\begin{pgfscope}%
\pgfpathrectangle{\pgfqpoint{0.481978in}{0.331635in}}{\pgfqpoint{9.300000in}{7.700000in}}%
\pgfusepath{clip}%
\pgfsetbuttcap%
\pgfsetroundjoin%
\definecolor{currentfill}{rgb}{1.000000,0.623529,0.607843}%
\pgfsetfillcolor{currentfill}%
\pgfsetlinewidth{0.481800pt}%
\definecolor{currentstroke}{rgb}{1.000000,1.000000,1.000000}%
\pgfsetstrokecolor{currentstroke}%
\pgfsetdash{}{0pt}%
\pgfpathmoveto{\pgfqpoint{7.776459in}{5.338891in}}%
\pgfpathcurveto{\pgfqpoint{7.787510in}{5.338891in}}{\pgfqpoint{7.798109in}{5.343282in}}{\pgfqpoint{7.805922in}{5.351095in}}%
\pgfpathcurveto{\pgfqpoint{7.813736in}{5.358909in}}{\pgfqpoint{7.818126in}{5.369508in}}{\pgfqpoint{7.818126in}{5.380558in}}%
\pgfpathcurveto{\pgfqpoint{7.818126in}{5.391608in}}{\pgfqpoint{7.813736in}{5.402207in}}{\pgfqpoint{7.805922in}{5.410021in}}%
\pgfpathcurveto{\pgfqpoint{7.798109in}{5.417834in}}{\pgfqpoint{7.787510in}{5.422225in}}{\pgfqpoint{7.776459in}{5.422225in}}%
\pgfpathcurveto{\pgfqpoint{7.765409in}{5.422225in}}{\pgfqpoint{7.754810in}{5.417834in}}{\pgfqpoint{7.746997in}{5.410021in}}%
\pgfpathcurveto{\pgfqpoint{7.739183in}{5.402207in}}{\pgfqpoint{7.734793in}{5.391608in}}{\pgfqpoint{7.734793in}{5.380558in}}%
\pgfpathcurveto{\pgfqpoint{7.734793in}{5.369508in}}{\pgfqpoint{7.739183in}{5.358909in}}{\pgfqpoint{7.746997in}{5.351095in}}%
\pgfpathcurveto{\pgfqpoint{7.754810in}{5.343282in}}{\pgfqpoint{7.765409in}{5.338891in}}{\pgfqpoint{7.776459in}{5.338891in}}%
\pgfpathclose%
\pgfusepath{stroke,fill}%
\end{pgfscope}%
\begin{pgfscope}%
\pgfpathrectangle{\pgfqpoint{0.481978in}{0.331635in}}{\pgfqpoint{9.300000in}{7.700000in}}%
\pgfusepath{clip}%
\pgfsetbuttcap%
\pgfsetroundjoin%
\definecolor{currentfill}{rgb}{1.000000,0.623529,0.607843}%
\pgfsetfillcolor{currentfill}%
\pgfsetlinewidth{0.481800pt}%
\definecolor{currentstroke}{rgb}{1.000000,1.000000,1.000000}%
\pgfsetstrokecolor{currentstroke}%
\pgfsetdash{}{0pt}%
\pgfpathmoveto{\pgfqpoint{5.832034in}{1.465477in}}%
\pgfpathcurveto{\pgfqpoint{5.843084in}{1.465477in}}{\pgfqpoint{5.853683in}{1.469868in}}{\pgfqpoint{5.861497in}{1.477681in}}%
\pgfpathcurveto{\pgfqpoint{5.869310in}{1.485495in}}{\pgfqpoint{5.873701in}{1.496094in}}{\pgfqpoint{5.873701in}{1.507144in}}%
\pgfpathcurveto{\pgfqpoint{5.873701in}{1.518194in}}{\pgfqpoint{5.869310in}{1.528793in}}{\pgfqpoint{5.861497in}{1.536607in}}%
\pgfpathcurveto{\pgfqpoint{5.853683in}{1.544421in}}{\pgfqpoint{5.843084in}{1.548811in}}{\pgfqpoint{5.832034in}{1.548811in}}%
\pgfpathcurveto{\pgfqpoint{5.820984in}{1.548811in}}{\pgfqpoint{5.810385in}{1.544421in}}{\pgfqpoint{5.802571in}{1.536607in}}%
\pgfpathcurveto{\pgfqpoint{5.794757in}{1.528793in}}{\pgfqpoint{5.790367in}{1.518194in}}{\pgfqpoint{5.790367in}{1.507144in}}%
\pgfpathcurveto{\pgfqpoint{5.790367in}{1.496094in}}{\pgfqpoint{5.794757in}{1.485495in}}{\pgfqpoint{5.802571in}{1.477681in}}%
\pgfpathcurveto{\pgfqpoint{5.810385in}{1.469868in}}{\pgfqpoint{5.820984in}{1.465477in}}{\pgfqpoint{5.832034in}{1.465477in}}%
\pgfpathclose%
\pgfusepath{stroke,fill}%
\end{pgfscope}%
\begin{pgfscope}%
\pgfpathrectangle{\pgfqpoint{0.481978in}{0.331635in}}{\pgfqpoint{9.300000in}{7.700000in}}%
\pgfusepath{clip}%
\pgfsetbuttcap%
\pgfsetroundjoin%
\definecolor{currentfill}{rgb}{1.000000,0.623529,0.607843}%
\pgfsetfillcolor{currentfill}%
\pgfsetlinewidth{0.481800pt}%
\definecolor{currentstroke}{rgb}{1.000000,1.000000,1.000000}%
\pgfsetstrokecolor{currentstroke}%
\pgfsetdash{}{0pt}%
\pgfpathmoveto{\pgfqpoint{8.011004in}{6.234740in}}%
\pgfpathcurveto{\pgfqpoint{8.022054in}{6.234740in}}{\pgfqpoint{8.032653in}{6.239130in}}{\pgfqpoint{8.040467in}{6.246944in}}%
\pgfpathcurveto{\pgfqpoint{8.048281in}{6.254757in}}{\pgfqpoint{8.052671in}{6.265356in}}{\pgfqpoint{8.052671in}{6.276407in}}%
\pgfpathcurveto{\pgfqpoint{8.052671in}{6.287457in}}{\pgfqpoint{8.048281in}{6.298056in}}{\pgfqpoint{8.040467in}{6.305869in}}%
\pgfpathcurveto{\pgfqpoint{8.032653in}{6.313683in}}{\pgfqpoint{8.022054in}{6.318073in}}{\pgfqpoint{8.011004in}{6.318073in}}%
\pgfpathcurveto{\pgfqpoint{7.999954in}{6.318073in}}{\pgfqpoint{7.989355in}{6.313683in}}{\pgfqpoint{7.981541in}{6.305869in}}%
\pgfpathcurveto{\pgfqpoint{7.973728in}{6.298056in}}{\pgfqpoint{7.969337in}{6.287457in}}{\pgfqpoint{7.969337in}{6.276407in}}%
\pgfpathcurveto{\pgfqpoint{7.969337in}{6.265356in}}{\pgfqpoint{7.973728in}{6.254757in}}{\pgfqpoint{7.981541in}{6.246944in}}%
\pgfpathcurveto{\pgfqpoint{7.989355in}{6.239130in}}{\pgfqpoint{7.999954in}{6.234740in}}{\pgfqpoint{8.011004in}{6.234740in}}%
\pgfpathclose%
\pgfusepath{stroke,fill}%
\end{pgfscope}%
\begin{pgfscope}%
\pgfpathrectangle{\pgfqpoint{0.481978in}{0.331635in}}{\pgfqpoint{9.300000in}{7.700000in}}%
\pgfusepath{clip}%
\pgfsetbuttcap%
\pgfsetroundjoin%
\definecolor{currentfill}{rgb}{1.000000,0.623529,0.607843}%
\pgfsetfillcolor{currentfill}%
\pgfsetlinewidth{0.481800pt}%
\definecolor{currentstroke}{rgb}{1.000000,1.000000,1.000000}%
\pgfsetstrokecolor{currentstroke}%
\pgfsetdash{}{0pt}%
\pgfpathmoveto{\pgfqpoint{2.908514in}{3.287743in}}%
\pgfpathcurveto{\pgfqpoint{2.919564in}{3.287743in}}{\pgfqpoint{2.930163in}{3.292133in}}{\pgfqpoint{2.937976in}{3.299947in}}%
\pgfpathcurveto{\pgfqpoint{2.945790in}{3.307760in}}{\pgfqpoint{2.950180in}{3.318359in}}{\pgfqpoint{2.950180in}{3.329410in}}%
\pgfpathcurveto{\pgfqpoint{2.950180in}{3.340460in}}{\pgfqpoint{2.945790in}{3.351059in}}{\pgfqpoint{2.937976in}{3.358872in}}%
\pgfpathcurveto{\pgfqpoint{2.930163in}{3.366686in}}{\pgfqpoint{2.919564in}{3.371076in}}{\pgfqpoint{2.908514in}{3.371076in}}%
\pgfpathcurveto{\pgfqpoint{2.897463in}{3.371076in}}{\pgfqpoint{2.886864in}{3.366686in}}{\pgfqpoint{2.879051in}{3.358872in}}%
\pgfpathcurveto{\pgfqpoint{2.871237in}{3.351059in}}{\pgfqpoint{2.866847in}{3.340460in}}{\pgfqpoint{2.866847in}{3.329410in}}%
\pgfpathcurveto{\pgfqpoint{2.866847in}{3.318359in}}{\pgfqpoint{2.871237in}{3.307760in}}{\pgfqpoint{2.879051in}{3.299947in}}%
\pgfpathcurveto{\pgfqpoint{2.886864in}{3.292133in}}{\pgfqpoint{2.897463in}{3.287743in}}{\pgfqpoint{2.908514in}{3.287743in}}%
\pgfpathclose%
\pgfusepath{stroke,fill}%
\end{pgfscope}%
\begin{pgfscope}%
\pgfpathrectangle{\pgfqpoint{0.481978in}{0.331635in}}{\pgfqpoint{9.300000in}{7.700000in}}%
\pgfusepath{clip}%
\pgfsetbuttcap%
\pgfsetroundjoin%
\definecolor{currentfill}{rgb}{1.000000,0.623529,0.607843}%
\pgfsetfillcolor{currentfill}%
\pgfsetlinewidth{0.481800pt}%
\definecolor{currentstroke}{rgb}{1.000000,1.000000,1.000000}%
\pgfsetstrokecolor{currentstroke}%
\pgfsetdash{}{0pt}%
\pgfpathmoveto{\pgfqpoint{6.465338in}{2.324944in}}%
\pgfpathcurveto{\pgfqpoint{6.476388in}{2.324944in}}{\pgfqpoint{6.486987in}{2.329335in}}{\pgfqpoint{6.494801in}{2.337148in}}%
\pgfpathcurveto{\pgfqpoint{6.502614in}{2.344962in}}{\pgfqpoint{6.507004in}{2.355561in}}{\pgfqpoint{6.507004in}{2.366611in}}%
\pgfpathcurveto{\pgfqpoint{6.507004in}{2.377661in}}{\pgfqpoint{6.502614in}{2.388260in}}{\pgfqpoint{6.494801in}{2.396074in}}%
\pgfpathcurveto{\pgfqpoint{6.486987in}{2.403887in}}{\pgfqpoint{6.476388in}{2.408278in}}{\pgfqpoint{6.465338in}{2.408278in}}%
\pgfpathcurveto{\pgfqpoint{6.454288in}{2.408278in}}{\pgfqpoint{6.443689in}{2.403887in}}{\pgfqpoint{6.435875in}{2.396074in}}%
\pgfpathcurveto{\pgfqpoint{6.428061in}{2.388260in}}{\pgfqpoint{6.423671in}{2.377661in}}{\pgfqpoint{6.423671in}{2.366611in}}%
\pgfpathcurveto{\pgfqpoint{6.423671in}{2.355561in}}{\pgfqpoint{6.428061in}{2.344962in}}{\pgfqpoint{6.435875in}{2.337148in}}%
\pgfpathcurveto{\pgfqpoint{6.443689in}{2.329335in}}{\pgfqpoint{6.454288in}{2.324944in}}{\pgfqpoint{6.465338in}{2.324944in}}%
\pgfpathclose%
\pgfusepath{stroke,fill}%
\end{pgfscope}%
\begin{pgfscope}%
\pgfpathrectangle{\pgfqpoint{0.481978in}{0.331635in}}{\pgfqpoint{9.300000in}{7.700000in}}%
\pgfusepath{clip}%
\pgfsetbuttcap%
\pgfsetroundjoin%
\definecolor{currentfill}{rgb}{1.000000,0.623529,0.607843}%
\pgfsetfillcolor{currentfill}%
\pgfsetlinewidth{0.481800pt}%
\definecolor{currentstroke}{rgb}{1.000000,1.000000,1.000000}%
\pgfsetstrokecolor{currentstroke}%
\pgfsetdash{}{0pt}%
\pgfpathmoveto{\pgfqpoint{8.106589in}{6.301680in}}%
\pgfpathcurveto{\pgfqpoint{8.117639in}{6.301680in}}{\pgfqpoint{8.128238in}{6.306070in}}{\pgfqpoint{8.136051in}{6.313883in}}%
\pgfpathcurveto{\pgfqpoint{8.143865in}{6.321697in}}{\pgfqpoint{8.148255in}{6.332296in}}{\pgfqpoint{8.148255in}{6.343346in}}%
\pgfpathcurveto{\pgfqpoint{8.148255in}{6.354396in}}{\pgfqpoint{8.143865in}{6.364995in}}{\pgfqpoint{8.136051in}{6.372809in}}%
\pgfpathcurveto{\pgfqpoint{8.128238in}{6.380623in}}{\pgfqpoint{8.117639in}{6.385013in}}{\pgfqpoint{8.106589in}{6.385013in}}%
\pgfpathcurveto{\pgfqpoint{8.095538in}{6.385013in}}{\pgfqpoint{8.084939in}{6.380623in}}{\pgfqpoint{8.077126in}{6.372809in}}%
\pgfpathcurveto{\pgfqpoint{8.069312in}{6.364995in}}{\pgfqpoint{8.064922in}{6.354396in}}{\pgfqpoint{8.064922in}{6.343346in}}%
\pgfpathcurveto{\pgfqpoint{8.064922in}{6.332296in}}{\pgfqpoint{8.069312in}{6.321697in}}{\pgfqpoint{8.077126in}{6.313883in}}%
\pgfpathcurveto{\pgfqpoint{8.084939in}{6.306070in}}{\pgfqpoint{8.095538in}{6.301680in}}{\pgfqpoint{8.106589in}{6.301680in}}%
\pgfpathclose%
\pgfusepath{stroke,fill}%
\end{pgfscope}%
\begin{pgfscope}%
\pgfpathrectangle{\pgfqpoint{0.481978in}{0.331635in}}{\pgfqpoint{9.300000in}{7.700000in}}%
\pgfusepath{clip}%
\pgfsetbuttcap%
\pgfsetroundjoin%
\definecolor{currentfill}{rgb}{1.000000,0.623529,0.607843}%
\pgfsetfillcolor{currentfill}%
\pgfsetlinewidth{0.481800pt}%
\definecolor{currentstroke}{rgb}{1.000000,1.000000,1.000000}%
\pgfsetstrokecolor{currentstroke}%
\pgfsetdash{}{0pt}%
\pgfpathmoveto{\pgfqpoint{7.991875in}{5.222385in}}%
\pgfpathcurveto{\pgfqpoint{8.002925in}{5.222385in}}{\pgfqpoint{8.013524in}{5.226776in}}{\pgfqpoint{8.021338in}{5.234589in}}%
\pgfpathcurveto{\pgfqpoint{8.029151in}{5.242403in}}{\pgfqpoint{8.033541in}{5.253002in}}{\pgfqpoint{8.033541in}{5.264052in}}%
\pgfpathcurveto{\pgfqpoint{8.033541in}{5.275102in}}{\pgfqpoint{8.029151in}{5.285701in}}{\pgfqpoint{8.021338in}{5.293515in}}%
\pgfpathcurveto{\pgfqpoint{8.013524in}{5.301328in}}{\pgfqpoint{8.002925in}{5.305719in}}{\pgfqpoint{7.991875in}{5.305719in}}%
\pgfpathcurveto{\pgfqpoint{7.980825in}{5.305719in}}{\pgfqpoint{7.970226in}{5.301328in}}{\pgfqpoint{7.962412in}{5.293515in}}%
\pgfpathcurveto{\pgfqpoint{7.954598in}{5.285701in}}{\pgfqpoint{7.950208in}{5.275102in}}{\pgfqpoint{7.950208in}{5.264052in}}%
\pgfpathcurveto{\pgfqpoint{7.950208in}{5.253002in}}{\pgfqpoint{7.954598in}{5.242403in}}{\pgfqpoint{7.962412in}{5.234589in}}%
\pgfpathcurveto{\pgfqpoint{7.970226in}{5.226776in}}{\pgfqpoint{7.980825in}{5.222385in}}{\pgfqpoint{7.991875in}{5.222385in}}%
\pgfpathclose%
\pgfusepath{stroke,fill}%
\end{pgfscope}%
\begin{pgfscope}%
\pgfpathrectangle{\pgfqpoint{0.481978in}{0.331635in}}{\pgfqpoint{9.300000in}{7.700000in}}%
\pgfusepath{clip}%
\pgfsetbuttcap%
\pgfsetroundjoin%
\definecolor{currentfill}{rgb}{1.000000,0.623529,0.607843}%
\pgfsetfillcolor{currentfill}%
\pgfsetlinewidth{0.481800pt}%
\definecolor{currentstroke}{rgb}{1.000000,1.000000,1.000000}%
\pgfsetstrokecolor{currentstroke}%
\pgfsetdash{}{0pt}%
\pgfpathmoveto{\pgfqpoint{7.423833in}{5.532170in}}%
\pgfpathcurveto{\pgfqpoint{7.434883in}{5.532170in}}{\pgfqpoint{7.445482in}{5.536561in}}{\pgfqpoint{7.453296in}{5.544374in}}%
\pgfpathcurveto{\pgfqpoint{7.461110in}{5.552188in}}{\pgfqpoint{7.465500in}{5.562787in}}{\pgfqpoint{7.465500in}{5.573837in}}%
\pgfpathcurveto{\pgfqpoint{7.465500in}{5.584887in}}{\pgfqpoint{7.461110in}{5.595486in}}{\pgfqpoint{7.453296in}{5.603300in}}%
\pgfpathcurveto{\pgfqpoint{7.445482in}{5.611113in}}{\pgfqpoint{7.434883in}{5.615504in}}{\pgfqpoint{7.423833in}{5.615504in}}%
\pgfpathcurveto{\pgfqpoint{7.412783in}{5.615504in}}{\pgfqpoint{7.402184in}{5.611113in}}{\pgfqpoint{7.394371in}{5.603300in}}%
\pgfpathcurveto{\pgfqpoint{7.386557in}{5.595486in}}{\pgfqpoint{7.382167in}{5.584887in}}{\pgfqpoint{7.382167in}{5.573837in}}%
\pgfpathcurveto{\pgfqpoint{7.382167in}{5.562787in}}{\pgfqpoint{7.386557in}{5.552188in}}{\pgfqpoint{7.394371in}{5.544374in}}%
\pgfpathcurveto{\pgfqpoint{7.402184in}{5.536561in}}{\pgfqpoint{7.412783in}{5.532170in}}{\pgfqpoint{7.423833in}{5.532170in}}%
\pgfpathclose%
\pgfusepath{stroke,fill}%
\end{pgfscope}%
\begin{pgfscope}%
\pgfpathrectangle{\pgfqpoint{0.481978in}{0.331635in}}{\pgfqpoint{9.300000in}{7.700000in}}%
\pgfusepath{clip}%
\pgfsetbuttcap%
\pgfsetroundjoin%
\definecolor{currentfill}{rgb}{1.000000,0.623529,0.607843}%
\pgfsetfillcolor{currentfill}%
\pgfsetlinewidth{0.481800pt}%
\definecolor{currentstroke}{rgb}{1.000000,1.000000,1.000000}%
\pgfsetstrokecolor{currentstroke}%
\pgfsetdash{}{0pt}%
\pgfpathmoveto{\pgfqpoint{4.192970in}{5.909242in}}%
\pgfpathcurveto{\pgfqpoint{4.204020in}{5.909242in}}{\pgfqpoint{4.214619in}{5.913632in}}{\pgfqpoint{4.222433in}{5.921445in}}%
\pgfpathcurveto{\pgfqpoint{4.230247in}{5.929259in}}{\pgfqpoint{4.234637in}{5.939858in}}{\pgfqpoint{4.234637in}{5.950908in}}%
\pgfpathcurveto{\pgfqpoint{4.234637in}{5.961958in}}{\pgfqpoint{4.230247in}{5.972557in}}{\pgfqpoint{4.222433in}{5.980371in}}%
\pgfpathcurveto{\pgfqpoint{4.214619in}{5.988185in}}{\pgfqpoint{4.204020in}{5.992575in}}{\pgfqpoint{4.192970in}{5.992575in}}%
\pgfpathcurveto{\pgfqpoint{4.181920in}{5.992575in}}{\pgfqpoint{4.171321in}{5.988185in}}{\pgfqpoint{4.163508in}{5.980371in}}%
\pgfpathcurveto{\pgfqpoint{4.155694in}{5.972557in}}{\pgfqpoint{4.151304in}{5.961958in}}{\pgfqpoint{4.151304in}{5.950908in}}%
\pgfpathcurveto{\pgfqpoint{4.151304in}{5.939858in}}{\pgfqpoint{4.155694in}{5.929259in}}{\pgfqpoint{4.163508in}{5.921445in}}%
\pgfpathcurveto{\pgfqpoint{4.171321in}{5.913632in}}{\pgfqpoint{4.181920in}{5.909242in}}{\pgfqpoint{4.192970in}{5.909242in}}%
\pgfpathclose%
\pgfusepath{stroke,fill}%
\end{pgfscope}%
\begin{pgfscope}%
\pgfpathrectangle{\pgfqpoint{0.481978in}{0.331635in}}{\pgfqpoint{9.300000in}{7.700000in}}%
\pgfusepath{clip}%
\pgfsetbuttcap%
\pgfsetroundjoin%
\definecolor{currentfill}{rgb}{1.000000,0.623529,0.607843}%
\pgfsetfillcolor{currentfill}%
\pgfsetlinewidth{0.481800pt}%
\definecolor{currentstroke}{rgb}{1.000000,1.000000,1.000000}%
\pgfsetstrokecolor{currentstroke}%
\pgfsetdash{}{0pt}%
\pgfpathmoveto{\pgfqpoint{6.770367in}{5.447632in}}%
\pgfpathcurveto{\pgfqpoint{6.781417in}{5.447632in}}{\pgfqpoint{6.792016in}{5.452022in}}{\pgfqpoint{6.799829in}{5.459836in}}%
\pgfpathcurveto{\pgfqpoint{6.807643in}{5.467649in}}{\pgfqpoint{6.812033in}{5.478248in}}{\pgfqpoint{6.812033in}{5.489298in}}%
\pgfpathcurveto{\pgfqpoint{6.812033in}{5.500349in}}{\pgfqpoint{6.807643in}{5.510948in}}{\pgfqpoint{6.799829in}{5.518761in}}%
\pgfpathcurveto{\pgfqpoint{6.792016in}{5.526575in}}{\pgfqpoint{6.781417in}{5.530965in}}{\pgfqpoint{6.770367in}{5.530965in}}%
\pgfpathcurveto{\pgfqpoint{6.759317in}{5.530965in}}{\pgfqpoint{6.748718in}{5.526575in}}{\pgfqpoint{6.740904in}{5.518761in}}%
\pgfpathcurveto{\pgfqpoint{6.733090in}{5.510948in}}{\pgfqpoint{6.728700in}{5.500349in}}{\pgfqpoint{6.728700in}{5.489298in}}%
\pgfpathcurveto{\pgfqpoint{6.728700in}{5.478248in}}{\pgfqpoint{6.733090in}{5.467649in}}{\pgfqpoint{6.740904in}{5.459836in}}%
\pgfpathcurveto{\pgfqpoint{6.748718in}{5.452022in}}{\pgfqpoint{6.759317in}{5.447632in}}{\pgfqpoint{6.770367in}{5.447632in}}%
\pgfpathclose%
\pgfusepath{stroke,fill}%
\end{pgfscope}%
\begin{pgfscope}%
\pgfpathrectangle{\pgfqpoint{0.481978in}{0.331635in}}{\pgfqpoint{9.300000in}{7.700000in}}%
\pgfusepath{clip}%
\pgfsetbuttcap%
\pgfsetroundjoin%
\definecolor{currentfill}{rgb}{1.000000,0.623529,0.607843}%
\pgfsetfillcolor{currentfill}%
\pgfsetlinewidth{0.481800pt}%
\definecolor{currentstroke}{rgb}{1.000000,1.000000,1.000000}%
\pgfsetstrokecolor{currentstroke}%
\pgfsetdash{}{0pt}%
\pgfpathmoveto{\pgfqpoint{7.446866in}{5.657018in}}%
\pgfpathcurveto{\pgfqpoint{7.457917in}{5.657018in}}{\pgfqpoint{7.468516in}{5.661409in}}{\pgfqpoint{7.476329in}{5.669222in}}%
\pgfpathcurveto{\pgfqpoint{7.484143in}{5.677036in}}{\pgfqpoint{7.488533in}{5.687635in}}{\pgfqpoint{7.488533in}{5.698685in}}%
\pgfpathcurveto{\pgfqpoint{7.488533in}{5.709735in}}{\pgfqpoint{7.484143in}{5.720334in}}{\pgfqpoint{7.476329in}{5.728148in}}%
\pgfpathcurveto{\pgfqpoint{7.468516in}{5.735961in}}{\pgfqpoint{7.457917in}{5.740352in}}{\pgfqpoint{7.446866in}{5.740352in}}%
\pgfpathcurveto{\pgfqpoint{7.435816in}{5.740352in}}{\pgfqpoint{7.425217in}{5.735961in}}{\pgfqpoint{7.417404in}{5.728148in}}%
\pgfpathcurveto{\pgfqpoint{7.409590in}{5.720334in}}{\pgfqpoint{7.405200in}{5.709735in}}{\pgfqpoint{7.405200in}{5.698685in}}%
\pgfpathcurveto{\pgfqpoint{7.405200in}{5.687635in}}{\pgfqpoint{7.409590in}{5.677036in}}{\pgfqpoint{7.417404in}{5.669222in}}%
\pgfpathcurveto{\pgfqpoint{7.425217in}{5.661409in}}{\pgfqpoint{7.435816in}{5.657018in}}{\pgfqpoint{7.446866in}{5.657018in}}%
\pgfpathclose%
\pgfusepath{stroke,fill}%
\end{pgfscope}%
\begin{pgfscope}%
\pgfpathrectangle{\pgfqpoint{0.481978in}{0.331635in}}{\pgfqpoint{9.300000in}{7.700000in}}%
\pgfusepath{clip}%
\pgfsetbuttcap%
\pgfsetroundjoin%
\definecolor{currentfill}{rgb}{1.000000,0.623529,0.607843}%
\pgfsetfillcolor{currentfill}%
\pgfsetlinewidth{0.481800pt}%
\definecolor{currentstroke}{rgb}{1.000000,1.000000,1.000000}%
\pgfsetstrokecolor{currentstroke}%
\pgfsetdash{}{0pt}%
\pgfpathmoveto{\pgfqpoint{4.422769in}{4.131207in}}%
\pgfpathcurveto{\pgfqpoint{4.433819in}{4.131207in}}{\pgfqpoint{4.444418in}{4.135598in}}{\pgfqpoint{4.452231in}{4.143411in}}%
\pgfpathcurveto{\pgfqpoint{4.460045in}{4.151225in}}{\pgfqpoint{4.464435in}{4.161824in}}{\pgfqpoint{4.464435in}{4.172874in}}%
\pgfpathcurveto{\pgfqpoint{4.464435in}{4.183924in}}{\pgfqpoint{4.460045in}{4.194523in}}{\pgfqpoint{4.452231in}{4.202337in}}%
\pgfpathcurveto{\pgfqpoint{4.444418in}{4.210150in}}{\pgfqpoint{4.433819in}{4.214541in}}{\pgfqpoint{4.422769in}{4.214541in}}%
\pgfpathcurveto{\pgfqpoint{4.411719in}{4.214541in}}{\pgfqpoint{4.401120in}{4.210150in}}{\pgfqpoint{4.393306in}{4.202337in}}%
\pgfpathcurveto{\pgfqpoint{4.385492in}{4.194523in}}{\pgfqpoint{4.381102in}{4.183924in}}{\pgfqpoint{4.381102in}{4.172874in}}%
\pgfpathcurveto{\pgfqpoint{4.381102in}{4.161824in}}{\pgfqpoint{4.385492in}{4.151225in}}{\pgfqpoint{4.393306in}{4.143411in}}%
\pgfpathcurveto{\pgfqpoint{4.401120in}{4.135598in}}{\pgfqpoint{4.411719in}{4.131207in}}{\pgfqpoint{4.422769in}{4.131207in}}%
\pgfpathclose%
\pgfusepath{stroke,fill}%
\end{pgfscope}%
\begin{pgfscope}%
\pgfpathrectangle{\pgfqpoint{0.481978in}{0.331635in}}{\pgfqpoint{9.300000in}{7.700000in}}%
\pgfusepath{clip}%
\pgfsetbuttcap%
\pgfsetroundjoin%
\definecolor{currentfill}{rgb}{1.000000,0.623529,0.607843}%
\pgfsetfillcolor{currentfill}%
\pgfsetlinewidth{0.481800pt}%
\definecolor{currentstroke}{rgb}{1.000000,1.000000,1.000000}%
\pgfsetstrokecolor{currentstroke}%
\pgfsetdash{}{0pt}%
\pgfpathmoveto{\pgfqpoint{4.632997in}{1.454063in}}%
\pgfpathcurveto{\pgfqpoint{4.644047in}{1.454063in}}{\pgfqpoint{4.654646in}{1.458453in}}{\pgfqpoint{4.662460in}{1.466266in}}%
\pgfpathcurveto{\pgfqpoint{4.670273in}{1.474080in}}{\pgfqpoint{4.674663in}{1.484679in}}{\pgfqpoint{4.674663in}{1.495729in}}%
\pgfpathcurveto{\pgfqpoint{4.674663in}{1.506779in}}{\pgfqpoint{4.670273in}{1.517378in}}{\pgfqpoint{4.662460in}{1.525192in}}%
\pgfpathcurveto{\pgfqpoint{4.654646in}{1.533006in}}{\pgfqpoint{4.644047in}{1.537396in}}{\pgfqpoint{4.632997in}{1.537396in}}%
\pgfpathcurveto{\pgfqpoint{4.621947in}{1.537396in}}{\pgfqpoint{4.611348in}{1.533006in}}{\pgfqpoint{4.603534in}{1.525192in}}%
\pgfpathcurveto{\pgfqpoint{4.595720in}{1.517378in}}{\pgfqpoint{4.591330in}{1.506779in}}{\pgfqpoint{4.591330in}{1.495729in}}%
\pgfpathcurveto{\pgfqpoint{4.591330in}{1.484679in}}{\pgfqpoint{4.595720in}{1.474080in}}{\pgfqpoint{4.603534in}{1.466266in}}%
\pgfpathcurveto{\pgfqpoint{4.611348in}{1.458453in}}{\pgfqpoint{4.621947in}{1.454063in}}{\pgfqpoint{4.632997in}{1.454063in}}%
\pgfpathclose%
\pgfusepath{stroke,fill}%
\end{pgfscope}%
\begin{pgfscope}%
\pgfpathrectangle{\pgfqpoint{0.481978in}{0.331635in}}{\pgfqpoint{9.300000in}{7.700000in}}%
\pgfusepath{clip}%
\pgfsetbuttcap%
\pgfsetroundjoin%
\definecolor{currentfill}{rgb}{1.000000,0.623529,0.607843}%
\pgfsetfillcolor{currentfill}%
\pgfsetlinewidth{0.481800pt}%
\definecolor{currentstroke}{rgb}{1.000000,1.000000,1.000000}%
\pgfsetstrokecolor{currentstroke}%
\pgfsetdash{}{0pt}%
\pgfpathmoveto{\pgfqpoint{3.319473in}{2.303496in}}%
\pgfpathcurveto{\pgfqpoint{3.330523in}{2.303496in}}{\pgfqpoint{3.341122in}{2.307886in}}{\pgfqpoint{3.348936in}{2.315699in}}%
\pgfpathcurveto{\pgfqpoint{3.356750in}{2.323513in}}{\pgfqpoint{3.361140in}{2.334112in}}{\pgfqpoint{3.361140in}{2.345162in}}%
\pgfpathcurveto{\pgfqpoint{3.361140in}{2.356212in}}{\pgfqpoint{3.356750in}{2.366811in}}{\pgfqpoint{3.348936in}{2.374625in}}%
\pgfpathcurveto{\pgfqpoint{3.341122in}{2.382439in}}{\pgfqpoint{3.330523in}{2.386829in}}{\pgfqpoint{3.319473in}{2.386829in}}%
\pgfpathcurveto{\pgfqpoint{3.308423in}{2.386829in}}{\pgfqpoint{3.297824in}{2.382439in}}{\pgfqpoint{3.290010in}{2.374625in}}%
\pgfpathcurveto{\pgfqpoint{3.282197in}{2.366811in}}{\pgfqpoint{3.277807in}{2.356212in}}{\pgfqpoint{3.277807in}{2.345162in}}%
\pgfpathcurveto{\pgfqpoint{3.277807in}{2.334112in}}{\pgfqpoint{3.282197in}{2.323513in}}{\pgfqpoint{3.290010in}{2.315699in}}%
\pgfpathcurveto{\pgfqpoint{3.297824in}{2.307886in}}{\pgfqpoint{3.308423in}{2.303496in}}{\pgfqpoint{3.319473in}{2.303496in}}%
\pgfpathclose%
\pgfusepath{stroke,fill}%
\end{pgfscope}%
\begin{pgfscope}%
\pgfpathrectangle{\pgfqpoint{0.481978in}{0.331635in}}{\pgfqpoint{9.300000in}{7.700000in}}%
\pgfusepath{clip}%
\pgfsetbuttcap%
\pgfsetroundjoin%
\definecolor{currentfill}{rgb}{1.000000,0.623529,0.607843}%
\pgfsetfillcolor{currentfill}%
\pgfsetlinewidth{0.481800pt}%
\definecolor{currentstroke}{rgb}{1.000000,1.000000,1.000000}%
\pgfsetstrokecolor{currentstroke}%
\pgfsetdash{}{0pt}%
\pgfpathmoveto{\pgfqpoint{3.661029in}{2.484883in}}%
\pgfpathcurveto{\pgfqpoint{3.672079in}{2.484883in}}{\pgfqpoint{3.682678in}{2.489274in}}{\pgfqpoint{3.690491in}{2.497087in}}%
\pgfpathcurveto{\pgfqpoint{3.698305in}{2.504901in}}{\pgfqpoint{3.702695in}{2.515500in}}{\pgfqpoint{3.702695in}{2.526550in}}%
\pgfpathcurveto{\pgfqpoint{3.702695in}{2.537600in}}{\pgfqpoint{3.698305in}{2.548199in}}{\pgfqpoint{3.690491in}{2.556013in}}%
\pgfpathcurveto{\pgfqpoint{3.682678in}{2.563826in}}{\pgfqpoint{3.672079in}{2.568217in}}{\pgfqpoint{3.661029in}{2.568217in}}%
\pgfpathcurveto{\pgfqpoint{3.649978in}{2.568217in}}{\pgfqpoint{3.639379in}{2.563826in}}{\pgfqpoint{3.631566in}{2.556013in}}%
\pgfpathcurveto{\pgfqpoint{3.623752in}{2.548199in}}{\pgfqpoint{3.619362in}{2.537600in}}{\pgfqpoint{3.619362in}{2.526550in}}%
\pgfpathcurveto{\pgfqpoint{3.619362in}{2.515500in}}{\pgfqpoint{3.623752in}{2.504901in}}{\pgfqpoint{3.631566in}{2.497087in}}%
\pgfpathcurveto{\pgfqpoint{3.639379in}{2.489274in}}{\pgfqpoint{3.649978in}{2.484883in}}{\pgfqpoint{3.661029in}{2.484883in}}%
\pgfpathclose%
\pgfusepath{stroke,fill}%
\end{pgfscope}%
\begin{pgfscope}%
\pgfpathrectangle{\pgfqpoint{0.481978in}{0.331635in}}{\pgfqpoint{9.300000in}{7.700000in}}%
\pgfusepath{clip}%
\pgfsetbuttcap%
\pgfsetroundjoin%
\definecolor{currentfill}{rgb}{1.000000,0.623529,0.607843}%
\pgfsetfillcolor{currentfill}%
\pgfsetlinewidth{0.481800pt}%
\definecolor{currentstroke}{rgb}{1.000000,1.000000,1.000000}%
\pgfsetstrokecolor{currentstroke}%
\pgfsetdash{}{0pt}%
\pgfpathmoveto{\pgfqpoint{3.845381in}{3.202561in}}%
\pgfpathcurveto{\pgfqpoint{3.856431in}{3.202561in}}{\pgfqpoint{3.867030in}{3.206951in}}{\pgfqpoint{3.874844in}{3.214765in}}%
\pgfpathcurveto{\pgfqpoint{3.882657in}{3.222578in}}{\pgfqpoint{3.887048in}{3.233177in}}{\pgfqpoint{3.887048in}{3.244228in}}%
\pgfpathcurveto{\pgfqpoint{3.887048in}{3.255278in}}{\pgfqpoint{3.882657in}{3.265877in}}{\pgfqpoint{3.874844in}{3.273690in}}%
\pgfpathcurveto{\pgfqpoint{3.867030in}{3.281504in}}{\pgfqpoint{3.856431in}{3.285894in}}{\pgfqpoint{3.845381in}{3.285894in}}%
\pgfpathcurveto{\pgfqpoint{3.834331in}{3.285894in}}{\pgfqpoint{3.823732in}{3.281504in}}{\pgfqpoint{3.815918in}{3.273690in}}%
\pgfpathcurveto{\pgfqpoint{3.808105in}{3.265877in}}{\pgfqpoint{3.803714in}{3.255278in}}{\pgfqpoint{3.803714in}{3.244228in}}%
\pgfpathcurveto{\pgfqpoint{3.803714in}{3.233177in}}{\pgfqpoint{3.808105in}{3.222578in}}{\pgfqpoint{3.815918in}{3.214765in}}%
\pgfpathcurveto{\pgfqpoint{3.823732in}{3.206951in}}{\pgfqpoint{3.834331in}{3.202561in}}{\pgfqpoint{3.845381in}{3.202561in}}%
\pgfpathclose%
\pgfusepath{stroke,fill}%
\end{pgfscope}%
\begin{pgfscope}%
\pgfpathrectangle{\pgfqpoint{0.481978in}{0.331635in}}{\pgfqpoint{9.300000in}{7.700000in}}%
\pgfusepath{clip}%
\pgfsetbuttcap%
\pgfsetroundjoin%
\definecolor{currentfill}{rgb}{1.000000,0.623529,0.607843}%
\pgfsetfillcolor{currentfill}%
\pgfsetlinewidth{0.481800pt}%
\definecolor{currentstroke}{rgb}{1.000000,1.000000,1.000000}%
\pgfsetstrokecolor{currentstroke}%
\pgfsetdash{}{0pt}%
\pgfpathmoveto{\pgfqpoint{4.050059in}{3.494463in}}%
\pgfpathcurveto{\pgfqpoint{4.061109in}{3.494463in}}{\pgfqpoint{4.071708in}{3.498853in}}{\pgfqpoint{4.079522in}{3.506667in}}%
\pgfpathcurveto{\pgfqpoint{4.087336in}{3.514480in}}{\pgfqpoint{4.091726in}{3.525079in}}{\pgfqpoint{4.091726in}{3.536130in}}%
\pgfpathcurveto{\pgfqpoint{4.091726in}{3.547180in}}{\pgfqpoint{4.087336in}{3.557779in}}{\pgfqpoint{4.079522in}{3.565592in}}%
\pgfpathcurveto{\pgfqpoint{4.071708in}{3.573406in}}{\pgfqpoint{4.061109in}{3.577796in}}{\pgfqpoint{4.050059in}{3.577796in}}%
\pgfpathcurveto{\pgfqpoint{4.039009in}{3.577796in}}{\pgfqpoint{4.028410in}{3.573406in}}{\pgfqpoint{4.020596in}{3.565592in}}%
\pgfpathcurveto{\pgfqpoint{4.012783in}{3.557779in}}{\pgfqpoint{4.008393in}{3.547180in}}{\pgfqpoint{4.008393in}{3.536130in}}%
\pgfpathcurveto{\pgfqpoint{4.008393in}{3.525079in}}{\pgfqpoint{4.012783in}{3.514480in}}{\pgfqpoint{4.020596in}{3.506667in}}%
\pgfpathcurveto{\pgfqpoint{4.028410in}{3.498853in}}{\pgfqpoint{4.039009in}{3.494463in}}{\pgfqpoint{4.050059in}{3.494463in}}%
\pgfpathclose%
\pgfusepath{stroke,fill}%
\end{pgfscope}%
\begin{pgfscope}%
\pgfpathrectangle{\pgfqpoint{0.481978in}{0.331635in}}{\pgfqpoint{9.300000in}{7.700000in}}%
\pgfusepath{clip}%
\pgfsetbuttcap%
\pgfsetroundjoin%
\definecolor{currentfill}{rgb}{1.000000,0.623529,0.607843}%
\pgfsetfillcolor{currentfill}%
\pgfsetlinewidth{0.481800pt}%
\definecolor{currentstroke}{rgb}{1.000000,1.000000,1.000000}%
\pgfsetstrokecolor{currentstroke}%
\pgfsetdash{}{0pt}%
\pgfpathmoveto{\pgfqpoint{4.925237in}{7.423806in}}%
\pgfpathcurveto{\pgfqpoint{4.936287in}{7.423806in}}{\pgfqpoint{4.946887in}{7.428197in}}{\pgfqpoint{4.954700in}{7.436010in}}%
\pgfpathcurveto{\pgfqpoint{4.962514in}{7.443824in}}{\pgfqpoint{4.966904in}{7.454423in}}{\pgfqpoint{4.966904in}{7.465473in}}%
\pgfpathcurveto{\pgfqpoint{4.966904in}{7.476523in}}{\pgfqpoint{4.962514in}{7.487122in}}{\pgfqpoint{4.954700in}{7.494936in}}%
\pgfpathcurveto{\pgfqpoint{4.946887in}{7.502749in}}{\pgfqpoint{4.936287in}{7.507140in}}{\pgfqpoint{4.925237in}{7.507140in}}%
\pgfpathcurveto{\pgfqpoint{4.914187in}{7.507140in}}{\pgfqpoint{4.903588in}{7.502749in}}{\pgfqpoint{4.895775in}{7.494936in}}%
\pgfpathcurveto{\pgfqpoint{4.887961in}{7.487122in}}{\pgfqpoint{4.883571in}{7.476523in}}{\pgfqpoint{4.883571in}{7.465473in}}%
\pgfpathcurveto{\pgfqpoint{4.883571in}{7.454423in}}{\pgfqpoint{4.887961in}{7.443824in}}{\pgfqpoint{4.895775in}{7.436010in}}%
\pgfpathcurveto{\pgfqpoint{4.903588in}{7.428197in}}{\pgfqpoint{4.914187in}{7.423806in}}{\pgfqpoint{4.925237in}{7.423806in}}%
\pgfpathclose%
\pgfusepath{stroke,fill}%
\end{pgfscope}%
\begin{pgfscope}%
\pgfpathrectangle{\pgfqpoint{0.481978in}{0.331635in}}{\pgfqpoint{9.300000in}{7.700000in}}%
\pgfusepath{clip}%
\pgfsetbuttcap%
\pgfsetroundjoin%
\definecolor{currentfill}{rgb}{1.000000,0.623529,0.607843}%
\pgfsetfillcolor{currentfill}%
\pgfsetlinewidth{0.481800pt}%
\definecolor{currentstroke}{rgb}{1.000000,1.000000,1.000000}%
\pgfsetstrokecolor{currentstroke}%
\pgfsetdash{}{0pt}%
\pgfpathmoveto{\pgfqpoint{6.991308in}{3.546324in}}%
\pgfpathcurveto{\pgfqpoint{7.002358in}{3.546324in}}{\pgfqpoint{7.012957in}{3.550714in}}{\pgfqpoint{7.020771in}{3.558528in}}%
\pgfpathcurveto{\pgfqpoint{7.028584in}{3.566341in}}{\pgfqpoint{7.032975in}{3.576940in}}{\pgfqpoint{7.032975in}{3.587991in}}%
\pgfpathcurveto{\pgfqpoint{7.032975in}{3.599041in}}{\pgfqpoint{7.028584in}{3.609640in}}{\pgfqpoint{7.020771in}{3.617453in}}%
\pgfpathcurveto{\pgfqpoint{7.012957in}{3.625267in}}{\pgfqpoint{7.002358in}{3.629657in}}{\pgfqpoint{6.991308in}{3.629657in}}%
\pgfpathcurveto{\pgfqpoint{6.980258in}{3.629657in}}{\pgfqpoint{6.969659in}{3.625267in}}{\pgfqpoint{6.961845in}{3.617453in}}%
\pgfpathcurveto{\pgfqpoint{6.954032in}{3.609640in}}{\pgfqpoint{6.949641in}{3.599041in}}{\pgfqpoint{6.949641in}{3.587991in}}%
\pgfpathcurveto{\pgfqpoint{6.949641in}{3.576940in}}{\pgfqpoint{6.954032in}{3.566341in}}{\pgfqpoint{6.961845in}{3.558528in}}%
\pgfpathcurveto{\pgfqpoint{6.969659in}{3.550714in}}{\pgfqpoint{6.980258in}{3.546324in}}{\pgfqpoint{6.991308in}{3.546324in}}%
\pgfpathclose%
\pgfusepath{stroke,fill}%
\end{pgfscope}%
\begin{pgfscope}%
\pgfpathrectangle{\pgfqpoint{0.481978in}{0.331635in}}{\pgfqpoint{9.300000in}{7.700000in}}%
\pgfusepath{clip}%
\pgfsetbuttcap%
\pgfsetroundjoin%
\definecolor{currentfill}{rgb}{1.000000,0.623529,0.607843}%
\pgfsetfillcolor{currentfill}%
\pgfsetlinewidth{0.481800pt}%
\definecolor{currentstroke}{rgb}{1.000000,1.000000,1.000000}%
\pgfsetstrokecolor{currentstroke}%
\pgfsetdash{}{0pt}%
\pgfpathmoveto{\pgfqpoint{8.618470in}{6.020902in}}%
\pgfpathcurveto{\pgfqpoint{8.629520in}{6.020902in}}{\pgfqpoint{8.640119in}{6.025293in}}{\pgfqpoint{8.647933in}{6.033106in}}%
\pgfpathcurveto{\pgfqpoint{8.655747in}{6.040920in}}{\pgfqpoint{8.660137in}{6.051519in}}{\pgfqpoint{8.660137in}{6.062569in}}%
\pgfpathcurveto{\pgfqpoint{8.660137in}{6.073619in}}{\pgfqpoint{8.655747in}{6.084218in}}{\pgfqpoint{8.647933in}{6.092032in}}%
\pgfpathcurveto{\pgfqpoint{8.640119in}{6.099845in}}{\pgfqpoint{8.629520in}{6.104236in}}{\pgfqpoint{8.618470in}{6.104236in}}%
\pgfpathcurveto{\pgfqpoint{8.607420in}{6.104236in}}{\pgfqpoint{8.596821in}{6.099845in}}{\pgfqpoint{8.589007in}{6.092032in}}%
\pgfpathcurveto{\pgfqpoint{8.581194in}{6.084218in}}{\pgfqpoint{8.576804in}{6.073619in}}{\pgfqpoint{8.576804in}{6.062569in}}%
\pgfpathcurveto{\pgfqpoint{8.576804in}{6.051519in}}{\pgfqpoint{8.581194in}{6.040920in}}{\pgfqpoint{8.589007in}{6.033106in}}%
\pgfpathcurveto{\pgfqpoint{8.596821in}{6.025293in}}{\pgfqpoint{8.607420in}{6.020902in}}{\pgfqpoint{8.618470in}{6.020902in}}%
\pgfpathclose%
\pgfusepath{stroke,fill}%
\end{pgfscope}%
\begin{pgfscope}%
\pgfpathrectangle{\pgfqpoint{0.481978in}{0.331635in}}{\pgfqpoint{9.300000in}{7.700000in}}%
\pgfusepath{clip}%
\pgfsetbuttcap%
\pgfsetroundjoin%
\definecolor{currentfill}{rgb}{1.000000,0.623529,0.607843}%
\pgfsetfillcolor{currentfill}%
\pgfsetlinewidth{0.481800pt}%
\definecolor{currentstroke}{rgb}{1.000000,1.000000,1.000000}%
\pgfsetstrokecolor{currentstroke}%
\pgfsetdash{}{0pt}%
\pgfpathmoveto{\pgfqpoint{7.337079in}{5.217206in}}%
\pgfpathcurveto{\pgfqpoint{7.348130in}{5.217206in}}{\pgfqpoint{7.358729in}{5.221596in}}{\pgfqpoint{7.366542in}{5.229410in}}%
\pgfpathcurveto{\pgfqpoint{7.374356in}{5.237223in}}{\pgfqpoint{7.378746in}{5.247822in}}{\pgfqpoint{7.378746in}{5.258872in}}%
\pgfpathcurveto{\pgfqpoint{7.378746in}{5.269922in}}{\pgfqpoint{7.374356in}{5.280521in}}{\pgfqpoint{7.366542in}{5.288335in}}%
\pgfpathcurveto{\pgfqpoint{7.358729in}{5.296149in}}{\pgfqpoint{7.348130in}{5.300539in}}{\pgfqpoint{7.337079in}{5.300539in}}%
\pgfpathcurveto{\pgfqpoint{7.326029in}{5.300539in}}{\pgfqpoint{7.315430in}{5.296149in}}{\pgfqpoint{7.307617in}{5.288335in}}%
\pgfpathcurveto{\pgfqpoint{7.299803in}{5.280521in}}{\pgfqpoint{7.295413in}{5.269922in}}{\pgfqpoint{7.295413in}{5.258872in}}%
\pgfpathcurveto{\pgfqpoint{7.295413in}{5.247822in}}{\pgfqpoint{7.299803in}{5.237223in}}{\pgfqpoint{7.307617in}{5.229410in}}%
\pgfpathcurveto{\pgfqpoint{7.315430in}{5.221596in}}{\pgfqpoint{7.326029in}{5.217206in}}{\pgfqpoint{7.337079in}{5.217206in}}%
\pgfpathclose%
\pgfusepath{stroke,fill}%
\end{pgfscope}%
\begin{pgfscope}%
\pgfpathrectangle{\pgfqpoint{0.481978in}{0.331635in}}{\pgfqpoint{9.300000in}{7.700000in}}%
\pgfusepath{clip}%
\pgfsetbuttcap%
\pgfsetroundjoin%
\definecolor{currentfill}{rgb}{1.000000,0.623529,0.607843}%
\pgfsetfillcolor{currentfill}%
\pgfsetlinewidth{0.481800pt}%
\definecolor{currentstroke}{rgb}{1.000000,1.000000,1.000000}%
\pgfsetstrokecolor{currentstroke}%
\pgfsetdash{}{0pt}%
\pgfpathmoveto{\pgfqpoint{4.553085in}{2.981640in}}%
\pgfpathcurveto{\pgfqpoint{4.564135in}{2.981640in}}{\pgfqpoint{4.574734in}{2.986030in}}{\pgfqpoint{4.582547in}{2.993844in}}%
\pgfpathcurveto{\pgfqpoint{4.590361in}{3.001658in}}{\pgfqpoint{4.594751in}{3.012257in}}{\pgfqpoint{4.594751in}{3.023307in}}%
\pgfpathcurveto{\pgfqpoint{4.594751in}{3.034357in}}{\pgfqpoint{4.590361in}{3.044956in}}{\pgfqpoint{4.582547in}{3.052769in}}%
\pgfpathcurveto{\pgfqpoint{4.574734in}{3.060583in}}{\pgfqpoint{4.564135in}{3.064973in}}{\pgfqpoint{4.553085in}{3.064973in}}%
\pgfpathcurveto{\pgfqpoint{4.542035in}{3.064973in}}{\pgfqpoint{4.531436in}{3.060583in}}{\pgfqpoint{4.523622in}{3.052769in}}%
\pgfpathcurveto{\pgfqpoint{4.515808in}{3.044956in}}{\pgfqpoint{4.511418in}{3.034357in}}{\pgfqpoint{4.511418in}{3.023307in}}%
\pgfpathcurveto{\pgfqpoint{4.511418in}{3.012257in}}{\pgfqpoint{4.515808in}{3.001658in}}{\pgfqpoint{4.523622in}{2.993844in}}%
\pgfpathcurveto{\pgfqpoint{4.531436in}{2.986030in}}{\pgfqpoint{4.542035in}{2.981640in}}{\pgfqpoint{4.553085in}{2.981640in}}%
\pgfpathclose%
\pgfusepath{stroke,fill}%
\end{pgfscope}%
\begin{pgfscope}%
\pgfpathrectangle{\pgfqpoint{0.481978in}{0.331635in}}{\pgfqpoint{9.300000in}{7.700000in}}%
\pgfusepath{clip}%
\pgfsetbuttcap%
\pgfsetroundjoin%
\definecolor{currentfill}{rgb}{1.000000,0.623529,0.607843}%
\pgfsetfillcolor{currentfill}%
\pgfsetlinewidth{0.481800pt}%
\definecolor{currentstroke}{rgb}{1.000000,1.000000,1.000000}%
\pgfsetstrokecolor{currentstroke}%
\pgfsetdash{}{0pt}%
\pgfpathmoveto{\pgfqpoint{9.095716in}{5.848566in}}%
\pgfpathcurveto{\pgfqpoint{9.106766in}{5.848566in}}{\pgfqpoint{9.117365in}{5.852956in}}{\pgfqpoint{9.125179in}{5.860770in}}%
\pgfpathcurveto{\pgfqpoint{9.132993in}{5.868584in}}{\pgfqpoint{9.137383in}{5.879183in}}{\pgfqpoint{9.137383in}{5.890233in}}%
\pgfpathcurveto{\pgfqpoint{9.137383in}{5.901283in}}{\pgfqpoint{9.132993in}{5.911882in}}{\pgfqpoint{9.125179in}{5.919696in}}%
\pgfpathcurveto{\pgfqpoint{9.117365in}{5.927509in}}{\pgfqpoint{9.106766in}{5.931899in}}{\pgfqpoint{9.095716in}{5.931899in}}%
\pgfpathcurveto{\pgfqpoint{9.084666in}{5.931899in}}{\pgfqpoint{9.074067in}{5.927509in}}{\pgfqpoint{9.066253in}{5.919696in}}%
\pgfpathcurveto{\pgfqpoint{9.058440in}{5.911882in}}{\pgfqpoint{9.054050in}{5.901283in}}{\pgfqpoint{9.054050in}{5.890233in}}%
\pgfpathcurveto{\pgfqpoint{9.054050in}{5.879183in}}{\pgfqpoint{9.058440in}{5.868584in}}{\pgfqpoint{9.066253in}{5.860770in}}%
\pgfpathcurveto{\pgfqpoint{9.074067in}{5.852956in}}{\pgfqpoint{9.084666in}{5.848566in}}{\pgfqpoint{9.095716in}{5.848566in}}%
\pgfpathclose%
\pgfusepath{stroke,fill}%
\end{pgfscope}%
\begin{pgfscope}%
\pgfpathrectangle{\pgfqpoint{0.481978in}{0.331635in}}{\pgfqpoint{9.300000in}{7.700000in}}%
\pgfusepath{clip}%
\pgfsetbuttcap%
\pgfsetroundjoin%
\definecolor{currentfill}{rgb}{1.000000,0.623529,0.607843}%
\pgfsetfillcolor{currentfill}%
\pgfsetlinewidth{0.481800pt}%
\definecolor{currentstroke}{rgb}{1.000000,1.000000,1.000000}%
\pgfsetstrokecolor{currentstroke}%
\pgfsetdash{}{0pt}%
\pgfpathmoveto{\pgfqpoint{2.403697in}{3.430362in}}%
\pgfpathcurveto{\pgfqpoint{2.414747in}{3.430362in}}{\pgfqpoint{2.425346in}{3.434752in}}{\pgfqpoint{2.433160in}{3.442566in}}%
\pgfpathcurveto{\pgfqpoint{2.440973in}{3.450380in}}{\pgfqpoint{2.445364in}{3.460979in}}{\pgfqpoint{2.445364in}{3.472029in}}%
\pgfpathcurveto{\pgfqpoint{2.445364in}{3.483079in}}{\pgfqpoint{2.440973in}{3.493678in}}{\pgfqpoint{2.433160in}{3.501491in}}%
\pgfpathcurveto{\pgfqpoint{2.425346in}{3.509305in}}{\pgfqpoint{2.414747in}{3.513695in}}{\pgfqpoint{2.403697in}{3.513695in}}%
\pgfpathcurveto{\pgfqpoint{2.392647in}{3.513695in}}{\pgfqpoint{2.382048in}{3.509305in}}{\pgfqpoint{2.374234in}{3.501491in}}%
\pgfpathcurveto{\pgfqpoint{2.366420in}{3.493678in}}{\pgfqpoint{2.362030in}{3.483079in}}{\pgfqpoint{2.362030in}{3.472029in}}%
\pgfpathcurveto{\pgfqpoint{2.362030in}{3.460979in}}{\pgfqpoint{2.366420in}{3.450380in}}{\pgfqpoint{2.374234in}{3.442566in}}%
\pgfpathcurveto{\pgfqpoint{2.382048in}{3.434752in}}{\pgfqpoint{2.392647in}{3.430362in}}{\pgfqpoint{2.403697in}{3.430362in}}%
\pgfpathclose%
\pgfusepath{stroke,fill}%
\end{pgfscope}%
\begin{pgfscope}%
\pgfpathrectangle{\pgfqpoint{0.481978in}{0.331635in}}{\pgfqpoint{9.300000in}{7.700000in}}%
\pgfusepath{clip}%
\pgfsetbuttcap%
\pgfsetroundjoin%
\definecolor{currentfill}{rgb}{1.000000,0.623529,0.607843}%
\pgfsetfillcolor{currentfill}%
\pgfsetlinewidth{0.481800pt}%
\definecolor{currentstroke}{rgb}{1.000000,1.000000,1.000000}%
\pgfsetstrokecolor{currentstroke}%
\pgfsetdash{}{0pt}%
\pgfpathmoveto{\pgfqpoint{7.815154in}{4.410412in}}%
\pgfpathcurveto{\pgfqpoint{7.826205in}{4.410412in}}{\pgfqpoint{7.836804in}{4.414802in}}{\pgfqpoint{7.844617in}{4.422616in}}%
\pgfpathcurveto{\pgfqpoint{7.852431in}{4.430429in}}{\pgfqpoint{7.856821in}{4.441028in}}{\pgfqpoint{7.856821in}{4.452079in}}%
\pgfpathcurveto{\pgfqpoint{7.856821in}{4.463129in}}{\pgfqpoint{7.852431in}{4.473728in}}{\pgfqpoint{7.844617in}{4.481541in}}%
\pgfpathcurveto{\pgfqpoint{7.836804in}{4.489355in}}{\pgfqpoint{7.826205in}{4.493745in}}{\pgfqpoint{7.815154in}{4.493745in}}%
\pgfpathcurveto{\pgfqpoint{7.804104in}{4.493745in}}{\pgfqpoint{7.793505in}{4.489355in}}{\pgfqpoint{7.785692in}{4.481541in}}%
\pgfpathcurveto{\pgfqpoint{7.777878in}{4.473728in}}{\pgfqpoint{7.773488in}{4.463129in}}{\pgfqpoint{7.773488in}{4.452079in}}%
\pgfpathcurveto{\pgfqpoint{7.773488in}{4.441028in}}{\pgfqpoint{7.777878in}{4.430429in}}{\pgfqpoint{7.785692in}{4.422616in}}%
\pgfpathcurveto{\pgfqpoint{7.793505in}{4.414802in}}{\pgfqpoint{7.804104in}{4.410412in}}{\pgfqpoint{7.815154in}{4.410412in}}%
\pgfpathclose%
\pgfusepath{stroke,fill}%
\end{pgfscope}%
\begin{pgfscope}%
\pgfpathrectangle{\pgfqpoint{0.481978in}{0.331635in}}{\pgfqpoint{9.300000in}{7.700000in}}%
\pgfusepath{clip}%
\pgfsetbuttcap%
\pgfsetroundjoin%
\definecolor{currentfill}{rgb}{1.000000,0.623529,0.607843}%
\pgfsetfillcolor{currentfill}%
\pgfsetlinewidth{0.481800pt}%
\definecolor{currentstroke}{rgb}{1.000000,1.000000,1.000000}%
\pgfsetstrokecolor{currentstroke}%
\pgfsetdash{}{0pt}%
\pgfpathmoveto{\pgfqpoint{4.473626in}{1.982566in}}%
\pgfpathcurveto{\pgfqpoint{4.484676in}{1.982566in}}{\pgfqpoint{4.495275in}{1.986956in}}{\pgfqpoint{4.503089in}{1.994770in}}%
\pgfpathcurveto{\pgfqpoint{4.510902in}{2.002583in}}{\pgfqpoint{4.515293in}{2.013182in}}{\pgfqpoint{4.515293in}{2.024232in}}%
\pgfpathcurveto{\pgfqpoint{4.515293in}{2.035282in}}{\pgfqpoint{4.510902in}{2.045881in}}{\pgfqpoint{4.503089in}{2.053695in}}%
\pgfpathcurveto{\pgfqpoint{4.495275in}{2.061509in}}{\pgfqpoint{4.484676in}{2.065899in}}{\pgfqpoint{4.473626in}{2.065899in}}%
\pgfpathcurveto{\pgfqpoint{4.462576in}{2.065899in}}{\pgfqpoint{4.451977in}{2.061509in}}{\pgfqpoint{4.444163in}{2.053695in}}%
\pgfpathcurveto{\pgfqpoint{4.436350in}{2.045881in}}{\pgfqpoint{4.431959in}{2.035282in}}{\pgfqpoint{4.431959in}{2.024232in}}%
\pgfpathcurveto{\pgfqpoint{4.431959in}{2.013182in}}{\pgfqpoint{4.436350in}{2.002583in}}{\pgfqpoint{4.444163in}{1.994770in}}%
\pgfpathcurveto{\pgfqpoint{4.451977in}{1.986956in}}{\pgfqpoint{4.462576in}{1.982566in}}{\pgfqpoint{4.473626in}{1.982566in}}%
\pgfpathclose%
\pgfusepath{stroke,fill}%
\end{pgfscope}%
\begin{pgfscope}%
\pgfpathrectangle{\pgfqpoint{0.481978in}{0.331635in}}{\pgfqpoint{9.300000in}{7.700000in}}%
\pgfusepath{clip}%
\pgfsetbuttcap%
\pgfsetroundjoin%
\definecolor{currentfill}{rgb}{0.815686,0.733333,1.000000}%
\pgfsetfillcolor{currentfill}%
\pgfsetlinewidth{0.481800pt}%
\definecolor{currentstroke}{rgb}{1.000000,1.000000,1.000000}%
\pgfsetstrokecolor{currentstroke}%
\pgfsetdash{}{0pt}%
\pgfpathmoveto{\pgfqpoint{7.152848in}{5.351280in}}%
\pgfpathcurveto{\pgfqpoint{7.163898in}{5.351280in}}{\pgfqpoint{7.174497in}{5.355670in}}{\pgfqpoint{7.182311in}{5.363484in}}%
\pgfpathcurveto{\pgfqpoint{7.190125in}{5.371297in}}{\pgfqpoint{7.194515in}{5.381896in}}{\pgfqpoint{7.194515in}{5.392946in}}%
\pgfpathcurveto{\pgfqpoint{7.194515in}{5.403996in}}{\pgfqpoint{7.190125in}{5.414596in}}{\pgfqpoint{7.182311in}{5.422409in}}%
\pgfpathcurveto{\pgfqpoint{7.174497in}{5.430223in}}{\pgfqpoint{7.163898in}{5.434613in}}{\pgfqpoint{7.152848in}{5.434613in}}%
\pgfpathcurveto{\pgfqpoint{7.141798in}{5.434613in}}{\pgfqpoint{7.131199in}{5.430223in}}{\pgfqpoint{7.123385in}{5.422409in}}%
\pgfpathcurveto{\pgfqpoint{7.115572in}{5.414596in}}{\pgfqpoint{7.111181in}{5.403996in}}{\pgfqpoint{7.111181in}{5.392946in}}%
\pgfpathcurveto{\pgfqpoint{7.111181in}{5.381896in}}{\pgfqpoint{7.115572in}{5.371297in}}{\pgfqpoint{7.123385in}{5.363484in}}%
\pgfpathcurveto{\pgfqpoint{7.131199in}{5.355670in}}{\pgfqpoint{7.141798in}{5.351280in}}{\pgfqpoint{7.152848in}{5.351280in}}%
\pgfpathclose%
\pgfusepath{stroke,fill}%
\end{pgfscope}%
\begin{pgfscope}%
\pgfpathrectangle{\pgfqpoint{0.481978in}{0.331635in}}{\pgfqpoint{9.300000in}{7.700000in}}%
\pgfusepath{clip}%
\pgfsetbuttcap%
\pgfsetroundjoin%
\definecolor{currentfill}{rgb}{0.815686,0.733333,1.000000}%
\pgfsetfillcolor{currentfill}%
\pgfsetlinewidth{0.481800pt}%
\definecolor{currentstroke}{rgb}{1.000000,1.000000,1.000000}%
\pgfsetstrokecolor{currentstroke}%
\pgfsetdash{}{0pt}%
\pgfpathmoveto{\pgfqpoint{7.702166in}{5.065536in}}%
\pgfpathcurveto{\pgfqpoint{7.713216in}{5.065536in}}{\pgfqpoint{7.723815in}{5.069927in}}{\pgfqpoint{7.731629in}{5.077740in}}%
\pgfpathcurveto{\pgfqpoint{7.739442in}{5.085554in}}{\pgfqpoint{7.743832in}{5.096153in}}{\pgfqpoint{7.743832in}{5.107203in}}%
\pgfpathcurveto{\pgfqpoint{7.743832in}{5.118253in}}{\pgfqpoint{7.739442in}{5.128852in}}{\pgfqpoint{7.731629in}{5.136666in}}%
\pgfpathcurveto{\pgfqpoint{7.723815in}{5.144480in}}{\pgfqpoint{7.713216in}{5.148870in}}{\pgfqpoint{7.702166in}{5.148870in}}%
\pgfpathcurveto{\pgfqpoint{7.691116in}{5.148870in}}{\pgfqpoint{7.680517in}{5.144480in}}{\pgfqpoint{7.672703in}{5.136666in}}%
\pgfpathcurveto{\pgfqpoint{7.664889in}{5.128852in}}{\pgfqpoint{7.660499in}{5.118253in}}{\pgfqpoint{7.660499in}{5.107203in}}%
\pgfpathcurveto{\pgfqpoint{7.660499in}{5.096153in}}{\pgfqpoint{7.664889in}{5.085554in}}{\pgfqpoint{7.672703in}{5.077740in}}%
\pgfpathcurveto{\pgfqpoint{7.680517in}{5.069927in}}{\pgfqpoint{7.691116in}{5.065536in}}{\pgfqpoint{7.702166in}{5.065536in}}%
\pgfpathclose%
\pgfusepath{stroke,fill}%
\end{pgfscope}%
\begin{pgfscope}%
\pgfpathrectangle{\pgfqpoint{0.481978in}{0.331635in}}{\pgfqpoint{9.300000in}{7.700000in}}%
\pgfusepath{clip}%
\pgfsetbuttcap%
\pgfsetroundjoin%
\definecolor{currentfill}{rgb}{0.815686,0.733333,1.000000}%
\pgfsetfillcolor{currentfill}%
\pgfsetlinewidth{0.481800pt}%
\definecolor{currentstroke}{rgb}{1.000000,1.000000,1.000000}%
\pgfsetstrokecolor{currentstroke}%
\pgfsetdash{}{0pt}%
\pgfpathmoveto{\pgfqpoint{2.872669in}{3.175352in}}%
\pgfpathcurveto{\pgfqpoint{2.883719in}{3.175352in}}{\pgfqpoint{2.894318in}{3.179742in}}{\pgfqpoint{2.902132in}{3.187556in}}%
\pgfpathcurveto{\pgfqpoint{2.909945in}{3.195370in}}{\pgfqpoint{2.914336in}{3.205969in}}{\pgfqpoint{2.914336in}{3.217019in}}%
\pgfpathcurveto{\pgfqpoint{2.914336in}{3.228069in}}{\pgfqpoint{2.909945in}{3.238668in}}{\pgfqpoint{2.902132in}{3.246482in}}%
\pgfpathcurveto{\pgfqpoint{2.894318in}{3.254295in}}{\pgfqpoint{2.883719in}{3.258686in}}{\pgfqpoint{2.872669in}{3.258686in}}%
\pgfpathcurveto{\pgfqpoint{2.861619in}{3.258686in}}{\pgfqpoint{2.851020in}{3.254295in}}{\pgfqpoint{2.843206in}{3.246482in}}%
\pgfpathcurveto{\pgfqpoint{2.835392in}{3.238668in}}{\pgfqpoint{2.831002in}{3.228069in}}{\pgfqpoint{2.831002in}{3.217019in}}%
\pgfpathcurveto{\pgfqpoint{2.831002in}{3.205969in}}{\pgfqpoint{2.835392in}{3.195370in}}{\pgfqpoint{2.843206in}{3.187556in}}%
\pgfpathcurveto{\pgfqpoint{2.851020in}{3.179742in}}{\pgfqpoint{2.861619in}{3.175352in}}{\pgfqpoint{2.872669in}{3.175352in}}%
\pgfpathclose%
\pgfusepath{stroke,fill}%
\end{pgfscope}%
\begin{pgfscope}%
\pgfpathrectangle{\pgfqpoint{0.481978in}{0.331635in}}{\pgfqpoint{9.300000in}{7.700000in}}%
\pgfusepath{clip}%
\pgfsetbuttcap%
\pgfsetroundjoin%
\definecolor{currentfill}{rgb}{0.815686,0.733333,1.000000}%
\pgfsetfillcolor{currentfill}%
\pgfsetlinewidth{0.481800pt}%
\definecolor{currentstroke}{rgb}{1.000000,1.000000,1.000000}%
\pgfsetstrokecolor{currentstroke}%
\pgfsetdash{}{0pt}%
\pgfpathmoveto{\pgfqpoint{3.619377in}{2.321220in}}%
\pgfpathcurveto{\pgfqpoint{3.630427in}{2.321220in}}{\pgfqpoint{3.641026in}{2.325610in}}{\pgfqpoint{3.648840in}{2.333423in}}%
\pgfpathcurveto{\pgfqpoint{3.656654in}{2.341237in}}{\pgfqpoint{3.661044in}{2.351836in}}{\pgfqpoint{3.661044in}{2.362886in}}%
\pgfpathcurveto{\pgfqpoint{3.661044in}{2.373936in}}{\pgfqpoint{3.656654in}{2.384535in}}{\pgfqpoint{3.648840in}{2.392349in}}%
\pgfpathcurveto{\pgfqpoint{3.641026in}{2.400163in}}{\pgfqpoint{3.630427in}{2.404553in}}{\pgfqpoint{3.619377in}{2.404553in}}%
\pgfpathcurveto{\pgfqpoint{3.608327in}{2.404553in}}{\pgfqpoint{3.597728in}{2.400163in}}{\pgfqpoint{3.589914in}{2.392349in}}%
\pgfpathcurveto{\pgfqpoint{3.582101in}{2.384535in}}{\pgfqpoint{3.577710in}{2.373936in}}{\pgfqpoint{3.577710in}{2.362886in}}%
\pgfpathcurveto{\pgfqpoint{3.577710in}{2.351836in}}{\pgfqpoint{3.582101in}{2.341237in}}{\pgfqpoint{3.589914in}{2.333423in}}%
\pgfpathcurveto{\pgfqpoint{3.597728in}{2.325610in}}{\pgfqpoint{3.608327in}{2.321220in}}{\pgfqpoint{3.619377in}{2.321220in}}%
\pgfpathclose%
\pgfusepath{stroke,fill}%
\end{pgfscope}%
\begin{pgfscope}%
\pgfpathrectangle{\pgfqpoint{0.481978in}{0.331635in}}{\pgfqpoint{9.300000in}{7.700000in}}%
\pgfusepath{clip}%
\pgfsetbuttcap%
\pgfsetroundjoin%
\definecolor{currentfill}{rgb}{0.815686,0.733333,1.000000}%
\pgfsetfillcolor{currentfill}%
\pgfsetlinewidth{0.481800pt}%
\definecolor{currentstroke}{rgb}{1.000000,1.000000,1.000000}%
\pgfsetstrokecolor{currentstroke}%
\pgfsetdash{}{0pt}%
\pgfpathmoveto{\pgfqpoint{3.448106in}{2.470564in}}%
\pgfpathcurveto{\pgfqpoint{3.459156in}{2.470564in}}{\pgfqpoint{3.469755in}{2.474954in}}{\pgfqpoint{3.477568in}{2.482767in}}%
\pgfpathcurveto{\pgfqpoint{3.485382in}{2.490581in}}{\pgfqpoint{3.489772in}{2.501180in}}{\pgfqpoint{3.489772in}{2.512230in}}%
\pgfpathcurveto{\pgfqpoint{3.489772in}{2.523280in}}{\pgfqpoint{3.485382in}{2.533879in}}{\pgfqpoint{3.477568in}{2.541693in}}%
\pgfpathcurveto{\pgfqpoint{3.469755in}{2.549507in}}{\pgfqpoint{3.459156in}{2.553897in}}{\pgfqpoint{3.448106in}{2.553897in}}%
\pgfpathcurveto{\pgfqpoint{3.437055in}{2.553897in}}{\pgfqpoint{3.426456in}{2.549507in}}{\pgfqpoint{3.418643in}{2.541693in}}%
\pgfpathcurveto{\pgfqpoint{3.410829in}{2.533879in}}{\pgfqpoint{3.406439in}{2.523280in}}{\pgfqpoint{3.406439in}{2.512230in}}%
\pgfpathcurveto{\pgfqpoint{3.406439in}{2.501180in}}{\pgfqpoint{3.410829in}{2.490581in}}{\pgfqpoint{3.418643in}{2.482767in}}%
\pgfpathcurveto{\pgfqpoint{3.426456in}{2.474954in}}{\pgfqpoint{3.437055in}{2.470564in}}{\pgfqpoint{3.448106in}{2.470564in}}%
\pgfpathclose%
\pgfusepath{stroke,fill}%
\end{pgfscope}%
\begin{pgfscope}%
\pgfpathrectangle{\pgfqpoint{0.481978in}{0.331635in}}{\pgfqpoint{9.300000in}{7.700000in}}%
\pgfusepath{clip}%
\pgfsetbuttcap%
\pgfsetroundjoin%
\definecolor{currentfill}{rgb}{0.815686,0.733333,1.000000}%
\pgfsetfillcolor{currentfill}%
\pgfsetlinewidth{0.481800pt}%
\definecolor{currentstroke}{rgb}{1.000000,1.000000,1.000000}%
\pgfsetstrokecolor{currentstroke}%
\pgfsetdash{}{0pt}%
\pgfpathmoveto{\pgfqpoint{6.968468in}{3.304397in}}%
\pgfpathcurveto{\pgfqpoint{6.979518in}{3.304397in}}{\pgfqpoint{6.990117in}{3.308788in}}{\pgfqpoint{6.997931in}{3.316601in}}%
\pgfpathcurveto{\pgfqpoint{7.005744in}{3.324415in}}{\pgfqpoint{7.010135in}{3.335014in}}{\pgfqpoint{7.010135in}{3.346064in}}%
\pgfpathcurveto{\pgfqpoint{7.010135in}{3.357114in}}{\pgfqpoint{7.005744in}{3.367713in}}{\pgfqpoint{6.997931in}{3.375527in}}%
\pgfpathcurveto{\pgfqpoint{6.990117in}{3.383340in}}{\pgfqpoint{6.979518in}{3.387731in}}{\pgfqpoint{6.968468in}{3.387731in}}%
\pgfpathcurveto{\pgfqpoint{6.957418in}{3.387731in}}{\pgfqpoint{6.946819in}{3.383340in}}{\pgfqpoint{6.939005in}{3.375527in}}%
\pgfpathcurveto{\pgfqpoint{6.931192in}{3.367713in}}{\pgfqpoint{6.926801in}{3.357114in}}{\pgfqpoint{6.926801in}{3.346064in}}%
\pgfpathcurveto{\pgfqpoint{6.926801in}{3.335014in}}{\pgfqpoint{6.931192in}{3.324415in}}{\pgfqpoint{6.939005in}{3.316601in}}%
\pgfpathcurveto{\pgfqpoint{6.946819in}{3.308788in}}{\pgfqpoint{6.957418in}{3.304397in}}{\pgfqpoint{6.968468in}{3.304397in}}%
\pgfpathclose%
\pgfusepath{stroke,fill}%
\end{pgfscope}%
\begin{pgfscope}%
\pgfpathrectangle{\pgfqpoint{0.481978in}{0.331635in}}{\pgfqpoint{9.300000in}{7.700000in}}%
\pgfusepath{clip}%
\pgfsetbuttcap%
\pgfsetroundjoin%
\definecolor{currentfill}{rgb}{0.815686,0.733333,1.000000}%
\pgfsetfillcolor{currentfill}%
\pgfsetlinewidth{0.481800pt}%
\definecolor{currentstroke}{rgb}{1.000000,1.000000,1.000000}%
\pgfsetstrokecolor{currentstroke}%
\pgfsetdash{}{0pt}%
\pgfpathmoveto{\pgfqpoint{3.260742in}{6.989602in}}%
\pgfpathcurveto{\pgfqpoint{3.271792in}{6.989602in}}{\pgfqpoint{3.282391in}{6.993992in}}{\pgfqpoint{3.290204in}{7.001806in}}%
\pgfpathcurveto{\pgfqpoint{3.298018in}{7.009620in}}{\pgfqpoint{3.302408in}{7.020219in}}{\pgfqpoint{3.302408in}{7.031269in}}%
\pgfpathcurveto{\pgfqpoint{3.302408in}{7.042319in}}{\pgfqpoint{3.298018in}{7.052918in}}{\pgfqpoint{3.290204in}{7.060732in}}%
\pgfpathcurveto{\pgfqpoint{3.282391in}{7.068545in}}{\pgfqpoint{3.271792in}{7.072935in}}{\pgfqpoint{3.260742in}{7.072935in}}%
\pgfpathcurveto{\pgfqpoint{3.249692in}{7.072935in}}{\pgfqpoint{3.239092in}{7.068545in}}{\pgfqpoint{3.231279in}{7.060732in}}%
\pgfpathcurveto{\pgfqpoint{3.223465in}{7.052918in}}{\pgfqpoint{3.219075in}{7.042319in}}{\pgfqpoint{3.219075in}{7.031269in}}%
\pgfpathcurveto{\pgfqpoint{3.219075in}{7.020219in}}{\pgfqpoint{3.223465in}{7.009620in}}{\pgfqpoint{3.231279in}{7.001806in}}%
\pgfpathcurveto{\pgfqpoint{3.239092in}{6.993992in}}{\pgfqpoint{3.249692in}{6.989602in}}{\pgfqpoint{3.260742in}{6.989602in}}%
\pgfpathclose%
\pgfusepath{stroke,fill}%
\end{pgfscope}%
\begin{pgfscope}%
\pgfpathrectangle{\pgfqpoint{0.481978in}{0.331635in}}{\pgfqpoint{9.300000in}{7.700000in}}%
\pgfusepath{clip}%
\pgfsetbuttcap%
\pgfsetroundjoin%
\definecolor{currentfill}{rgb}{0.815686,0.733333,1.000000}%
\pgfsetfillcolor{currentfill}%
\pgfsetlinewidth{0.481800pt}%
\definecolor{currentstroke}{rgb}{1.000000,1.000000,1.000000}%
\pgfsetstrokecolor{currentstroke}%
\pgfsetdash{}{0pt}%
\pgfpathmoveto{\pgfqpoint{3.172404in}{5.733347in}}%
\pgfpathcurveto{\pgfqpoint{3.183454in}{5.733347in}}{\pgfqpoint{3.194053in}{5.737738in}}{\pgfqpoint{3.201867in}{5.745551in}}%
\pgfpathcurveto{\pgfqpoint{3.209680in}{5.753365in}}{\pgfqpoint{3.214071in}{5.763964in}}{\pgfqpoint{3.214071in}{5.775014in}}%
\pgfpathcurveto{\pgfqpoint{3.214071in}{5.786064in}}{\pgfqpoint{3.209680in}{5.796663in}}{\pgfqpoint{3.201867in}{5.804477in}}%
\pgfpathcurveto{\pgfqpoint{3.194053in}{5.812290in}}{\pgfqpoint{3.183454in}{5.816681in}}{\pgfqpoint{3.172404in}{5.816681in}}%
\pgfpathcurveto{\pgfqpoint{3.161354in}{5.816681in}}{\pgfqpoint{3.150755in}{5.812290in}}{\pgfqpoint{3.142941in}{5.804477in}}%
\pgfpathcurveto{\pgfqpoint{3.135127in}{5.796663in}}{\pgfqpoint{3.130737in}{5.786064in}}{\pgfqpoint{3.130737in}{5.775014in}}%
\pgfpathcurveto{\pgfqpoint{3.130737in}{5.763964in}}{\pgfqpoint{3.135127in}{5.753365in}}{\pgfqpoint{3.142941in}{5.745551in}}%
\pgfpathcurveto{\pgfqpoint{3.150755in}{5.737738in}}{\pgfqpoint{3.161354in}{5.733347in}}{\pgfqpoint{3.172404in}{5.733347in}}%
\pgfpathclose%
\pgfusepath{stroke,fill}%
\end{pgfscope}%
\begin{pgfscope}%
\pgfpathrectangle{\pgfqpoint{0.481978in}{0.331635in}}{\pgfqpoint{9.300000in}{7.700000in}}%
\pgfusepath{clip}%
\pgfsetbuttcap%
\pgfsetroundjoin%
\definecolor{currentfill}{rgb}{0.815686,0.733333,1.000000}%
\pgfsetfillcolor{currentfill}%
\pgfsetlinewidth{0.481800pt}%
\definecolor{currentstroke}{rgb}{1.000000,1.000000,1.000000}%
\pgfsetstrokecolor{currentstroke}%
\pgfsetdash{}{0pt}%
\pgfpathmoveto{\pgfqpoint{2.953247in}{2.659230in}}%
\pgfpathcurveto{\pgfqpoint{2.964297in}{2.659230in}}{\pgfqpoint{2.974896in}{2.663621in}}{\pgfqpoint{2.982710in}{2.671434in}}%
\pgfpathcurveto{\pgfqpoint{2.990523in}{2.679248in}}{\pgfqpoint{2.994914in}{2.689847in}}{\pgfqpoint{2.994914in}{2.700897in}}%
\pgfpathcurveto{\pgfqpoint{2.994914in}{2.711947in}}{\pgfqpoint{2.990523in}{2.722546in}}{\pgfqpoint{2.982710in}{2.730360in}}%
\pgfpathcurveto{\pgfqpoint{2.974896in}{2.738173in}}{\pgfqpoint{2.964297in}{2.742564in}}{\pgfqpoint{2.953247in}{2.742564in}}%
\pgfpathcurveto{\pgfqpoint{2.942197in}{2.742564in}}{\pgfqpoint{2.931598in}{2.738173in}}{\pgfqpoint{2.923784in}{2.730360in}}%
\pgfpathcurveto{\pgfqpoint{2.915970in}{2.722546in}}{\pgfqpoint{2.911580in}{2.711947in}}{\pgfqpoint{2.911580in}{2.700897in}}%
\pgfpathcurveto{\pgfqpoint{2.911580in}{2.689847in}}{\pgfqpoint{2.915970in}{2.679248in}}{\pgfqpoint{2.923784in}{2.671434in}}%
\pgfpathcurveto{\pgfqpoint{2.931598in}{2.663621in}}{\pgfqpoint{2.942197in}{2.659230in}}{\pgfqpoint{2.953247in}{2.659230in}}%
\pgfpathclose%
\pgfusepath{stroke,fill}%
\end{pgfscope}%
\begin{pgfscope}%
\pgfpathrectangle{\pgfqpoint{0.481978in}{0.331635in}}{\pgfqpoint{9.300000in}{7.700000in}}%
\pgfusepath{clip}%
\pgfsetbuttcap%
\pgfsetroundjoin%
\definecolor{currentfill}{rgb}{0.815686,0.733333,1.000000}%
\pgfsetfillcolor{currentfill}%
\pgfsetlinewidth{0.481800pt}%
\definecolor{currentstroke}{rgb}{1.000000,1.000000,1.000000}%
\pgfsetstrokecolor{currentstroke}%
\pgfsetdash{}{0pt}%
\pgfpathmoveto{\pgfqpoint{8.116420in}{5.574617in}}%
\pgfpathcurveto{\pgfqpoint{8.127470in}{5.574617in}}{\pgfqpoint{8.138069in}{5.579007in}}{\pgfqpoint{8.145883in}{5.586821in}}%
\pgfpathcurveto{\pgfqpoint{8.153697in}{5.594635in}}{\pgfqpoint{8.158087in}{5.605234in}}{\pgfqpoint{8.158087in}{5.616284in}}%
\pgfpathcurveto{\pgfqpoint{8.158087in}{5.627334in}}{\pgfqpoint{8.153697in}{5.637933in}}{\pgfqpoint{8.145883in}{5.645747in}}%
\pgfpathcurveto{\pgfqpoint{8.138069in}{5.653560in}}{\pgfqpoint{8.127470in}{5.657950in}}{\pgfqpoint{8.116420in}{5.657950in}}%
\pgfpathcurveto{\pgfqpoint{8.105370in}{5.657950in}}{\pgfqpoint{8.094771in}{5.653560in}}{\pgfqpoint{8.086957in}{5.645747in}}%
\pgfpathcurveto{\pgfqpoint{8.079144in}{5.637933in}}{\pgfqpoint{8.074754in}{5.627334in}}{\pgfqpoint{8.074754in}{5.616284in}}%
\pgfpathcurveto{\pgfqpoint{8.074754in}{5.605234in}}{\pgfqpoint{8.079144in}{5.594635in}}{\pgfqpoint{8.086957in}{5.586821in}}%
\pgfpathcurveto{\pgfqpoint{8.094771in}{5.579007in}}{\pgfqpoint{8.105370in}{5.574617in}}{\pgfqpoint{8.116420in}{5.574617in}}%
\pgfpathclose%
\pgfusepath{stroke,fill}%
\end{pgfscope}%
\begin{pgfscope}%
\pgfpathrectangle{\pgfqpoint{0.481978in}{0.331635in}}{\pgfqpoint{9.300000in}{7.700000in}}%
\pgfusepath{clip}%
\pgfsetbuttcap%
\pgfsetroundjoin%
\definecolor{currentfill}{rgb}{0.815686,0.733333,1.000000}%
\pgfsetfillcolor{currentfill}%
\pgfsetlinewidth{0.481800pt}%
\definecolor{currentstroke}{rgb}{1.000000,1.000000,1.000000}%
\pgfsetstrokecolor{currentstroke}%
\pgfsetdash{}{0pt}%
\pgfpathmoveto{\pgfqpoint{6.906230in}{4.688829in}}%
\pgfpathcurveto{\pgfqpoint{6.917280in}{4.688829in}}{\pgfqpoint{6.927879in}{4.693220in}}{\pgfqpoint{6.935692in}{4.701033in}}%
\pgfpathcurveto{\pgfqpoint{6.943506in}{4.708847in}}{\pgfqpoint{6.947896in}{4.719446in}}{\pgfqpoint{6.947896in}{4.730496in}}%
\pgfpathcurveto{\pgfqpoint{6.947896in}{4.741546in}}{\pgfqpoint{6.943506in}{4.752145in}}{\pgfqpoint{6.935692in}{4.759959in}}%
\pgfpathcurveto{\pgfqpoint{6.927879in}{4.767772in}}{\pgfqpoint{6.917280in}{4.772163in}}{\pgfqpoint{6.906230in}{4.772163in}}%
\pgfpathcurveto{\pgfqpoint{6.895179in}{4.772163in}}{\pgfqpoint{6.884580in}{4.767772in}}{\pgfqpoint{6.876767in}{4.759959in}}%
\pgfpathcurveto{\pgfqpoint{6.868953in}{4.752145in}}{\pgfqpoint{6.864563in}{4.741546in}}{\pgfqpoint{6.864563in}{4.730496in}}%
\pgfpathcurveto{\pgfqpoint{6.864563in}{4.719446in}}{\pgfqpoint{6.868953in}{4.708847in}}{\pgfqpoint{6.876767in}{4.701033in}}%
\pgfpathcurveto{\pgfqpoint{6.884580in}{4.693220in}}{\pgfqpoint{6.895179in}{4.688829in}}{\pgfqpoint{6.906230in}{4.688829in}}%
\pgfpathclose%
\pgfusepath{stroke,fill}%
\end{pgfscope}%
\begin{pgfscope}%
\pgfpathrectangle{\pgfqpoint{0.481978in}{0.331635in}}{\pgfqpoint{9.300000in}{7.700000in}}%
\pgfusepath{clip}%
\pgfsetbuttcap%
\pgfsetroundjoin%
\definecolor{currentfill}{rgb}{0.815686,0.733333,1.000000}%
\pgfsetfillcolor{currentfill}%
\pgfsetlinewidth{0.481800pt}%
\definecolor{currentstroke}{rgb}{1.000000,1.000000,1.000000}%
\pgfsetstrokecolor{currentstroke}%
\pgfsetdash{}{0pt}%
\pgfpathmoveto{\pgfqpoint{3.815826in}{4.045962in}}%
\pgfpathcurveto{\pgfqpoint{3.826876in}{4.045962in}}{\pgfqpoint{3.837475in}{4.050352in}}{\pgfqpoint{3.845289in}{4.058166in}}%
\pgfpathcurveto{\pgfqpoint{3.853103in}{4.065979in}}{\pgfqpoint{3.857493in}{4.076578in}}{\pgfqpoint{3.857493in}{4.087628in}}%
\pgfpathcurveto{\pgfqpoint{3.857493in}{4.098678in}}{\pgfqpoint{3.853103in}{4.109278in}}{\pgfqpoint{3.845289in}{4.117091in}}%
\pgfpathcurveto{\pgfqpoint{3.837475in}{4.124905in}}{\pgfqpoint{3.826876in}{4.129295in}}{\pgfqpoint{3.815826in}{4.129295in}}%
\pgfpathcurveto{\pgfqpoint{3.804776in}{4.129295in}}{\pgfqpoint{3.794177in}{4.124905in}}{\pgfqpoint{3.786364in}{4.117091in}}%
\pgfpathcurveto{\pgfqpoint{3.778550in}{4.109278in}}{\pgfqpoint{3.774160in}{4.098678in}}{\pgfqpoint{3.774160in}{4.087628in}}%
\pgfpathcurveto{\pgfqpoint{3.774160in}{4.076578in}}{\pgfqpoint{3.778550in}{4.065979in}}{\pgfqpoint{3.786364in}{4.058166in}}%
\pgfpathcurveto{\pgfqpoint{3.794177in}{4.050352in}}{\pgfqpoint{3.804776in}{4.045962in}}{\pgfqpoint{3.815826in}{4.045962in}}%
\pgfpathclose%
\pgfusepath{stroke,fill}%
\end{pgfscope}%
\begin{pgfscope}%
\pgfpathrectangle{\pgfqpoint{0.481978in}{0.331635in}}{\pgfqpoint{9.300000in}{7.700000in}}%
\pgfusepath{clip}%
\pgfsetbuttcap%
\pgfsetroundjoin%
\definecolor{currentfill}{rgb}{0.815686,0.733333,1.000000}%
\pgfsetfillcolor{currentfill}%
\pgfsetlinewidth{0.481800pt}%
\definecolor{currentstroke}{rgb}{1.000000,1.000000,1.000000}%
\pgfsetstrokecolor{currentstroke}%
\pgfsetdash{}{0pt}%
\pgfpathmoveto{\pgfqpoint{6.962413in}{5.760379in}}%
\pgfpathcurveto{\pgfqpoint{6.973463in}{5.760379in}}{\pgfqpoint{6.984062in}{5.764769in}}{\pgfqpoint{6.991875in}{5.772582in}}%
\pgfpathcurveto{\pgfqpoint{6.999689in}{5.780396in}}{\pgfqpoint{7.004079in}{5.790995in}}{\pgfqpoint{7.004079in}{5.802045in}}%
\pgfpathcurveto{\pgfqpoint{7.004079in}{5.813095in}}{\pgfqpoint{6.999689in}{5.823694in}}{\pgfqpoint{6.991875in}{5.831508in}}%
\pgfpathcurveto{\pgfqpoint{6.984062in}{5.839322in}}{\pgfqpoint{6.973463in}{5.843712in}}{\pgfqpoint{6.962413in}{5.843712in}}%
\pgfpathcurveto{\pgfqpoint{6.951362in}{5.843712in}}{\pgfqpoint{6.940763in}{5.839322in}}{\pgfqpoint{6.932950in}{5.831508in}}%
\pgfpathcurveto{\pgfqpoint{6.925136in}{5.823694in}}{\pgfqpoint{6.920746in}{5.813095in}}{\pgfqpoint{6.920746in}{5.802045in}}%
\pgfpathcurveto{\pgfqpoint{6.920746in}{5.790995in}}{\pgfqpoint{6.925136in}{5.780396in}}{\pgfqpoint{6.932950in}{5.772582in}}%
\pgfpathcurveto{\pgfqpoint{6.940763in}{5.764769in}}{\pgfqpoint{6.951362in}{5.760379in}}{\pgfqpoint{6.962413in}{5.760379in}}%
\pgfpathclose%
\pgfusepath{stroke,fill}%
\end{pgfscope}%
\begin{pgfscope}%
\pgfpathrectangle{\pgfqpoint{0.481978in}{0.331635in}}{\pgfqpoint{9.300000in}{7.700000in}}%
\pgfusepath{clip}%
\pgfsetbuttcap%
\pgfsetroundjoin%
\definecolor{currentfill}{rgb}{0.815686,0.733333,1.000000}%
\pgfsetfillcolor{currentfill}%
\pgfsetlinewidth{0.481800pt}%
\definecolor{currentstroke}{rgb}{1.000000,1.000000,1.000000}%
\pgfsetstrokecolor{currentstroke}%
\pgfsetdash{}{0pt}%
\pgfpathmoveto{\pgfqpoint{7.236127in}{5.351377in}}%
\pgfpathcurveto{\pgfqpoint{7.247177in}{5.351377in}}{\pgfqpoint{7.257776in}{5.355768in}}{\pgfqpoint{7.265589in}{5.363581in}}%
\pgfpathcurveto{\pgfqpoint{7.273403in}{5.371395in}}{\pgfqpoint{7.277793in}{5.381994in}}{\pgfqpoint{7.277793in}{5.393044in}}%
\pgfpathcurveto{\pgfqpoint{7.277793in}{5.404094in}}{\pgfqpoint{7.273403in}{5.414693in}}{\pgfqpoint{7.265589in}{5.422507in}}%
\pgfpathcurveto{\pgfqpoint{7.257776in}{5.430321in}}{\pgfqpoint{7.247177in}{5.434711in}}{\pgfqpoint{7.236127in}{5.434711in}}%
\pgfpathcurveto{\pgfqpoint{7.225076in}{5.434711in}}{\pgfqpoint{7.214477in}{5.430321in}}{\pgfqpoint{7.206664in}{5.422507in}}%
\pgfpathcurveto{\pgfqpoint{7.198850in}{5.414693in}}{\pgfqpoint{7.194460in}{5.404094in}}{\pgfqpoint{7.194460in}{5.393044in}}%
\pgfpathcurveto{\pgfqpoint{7.194460in}{5.381994in}}{\pgfqpoint{7.198850in}{5.371395in}}{\pgfqpoint{7.206664in}{5.363581in}}%
\pgfpathcurveto{\pgfqpoint{7.214477in}{5.355768in}}{\pgfqpoint{7.225076in}{5.351377in}}{\pgfqpoint{7.236127in}{5.351377in}}%
\pgfpathclose%
\pgfusepath{stroke,fill}%
\end{pgfscope}%
\begin{pgfscope}%
\pgfpathrectangle{\pgfqpoint{0.481978in}{0.331635in}}{\pgfqpoint{9.300000in}{7.700000in}}%
\pgfusepath{clip}%
\pgfsetbuttcap%
\pgfsetroundjoin%
\definecolor{currentfill}{rgb}{0.815686,0.733333,1.000000}%
\pgfsetfillcolor{currentfill}%
\pgfsetlinewidth{0.481800pt}%
\definecolor{currentstroke}{rgb}{1.000000,1.000000,1.000000}%
\pgfsetstrokecolor{currentstroke}%
\pgfsetdash{}{0pt}%
\pgfpathmoveto{\pgfqpoint{8.116846in}{6.482719in}}%
\pgfpathcurveto{\pgfqpoint{8.127896in}{6.482719in}}{\pgfqpoint{8.138495in}{6.487109in}}{\pgfqpoint{8.146309in}{6.494923in}}%
\pgfpathcurveto{\pgfqpoint{8.154122in}{6.502737in}}{\pgfqpoint{8.158512in}{6.513336in}}{\pgfqpoint{8.158512in}{6.524386in}}%
\pgfpathcurveto{\pgfqpoint{8.158512in}{6.535436in}}{\pgfqpoint{8.154122in}{6.546035in}}{\pgfqpoint{8.146309in}{6.553849in}}%
\pgfpathcurveto{\pgfqpoint{8.138495in}{6.561662in}}{\pgfqpoint{8.127896in}{6.566052in}}{\pgfqpoint{8.116846in}{6.566052in}}%
\pgfpathcurveto{\pgfqpoint{8.105796in}{6.566052in}}{\pgfqpoint{8.095197in}{6.561662in}}{\pgfqpoint{8.087383in}{6.553849in}}%
\pgfpathcurveto{\pgfqpoint{8.079569in}{6.546035in}}{\pgfqpoint{8.075179in}{6.535436in}}{\pgfqpoint{8.075179in}{6.524386in}}%
\pgfpathcurveto{\pgfqpoint{8.075179in}{6.513336in}}{\pgfqpoint{8.079569in}{6.502737in}}{\pgfqpoint{8.087383in}{6.494923in}}%
\pgfpathcurveto{\pgfqpoint{8.095197in}{6.487109in}}{\pgfqpoint{8.105796in}{6.482719in}}{\pgfqpoint{8.116846in}{6.482719in}}%
\pgfpathclose%
\pgfusepath{stroke,fill}%
\end{pgfscope}%
\begin{pgfscope}%
\pgfpathrectangle{\pgfqpoint{0.481978in}{0.331635in}}{\pgfqpoint{9.300000in}{7.700000in}}%
\pgfusepath{clip}%
\pgfsetbuttcap%
\pgfsetroundjoin%
\definecolor{currentfill}{rgb}{0.815686,0.733333,1.000000}%
\pgfsetfillcolor{currentfill}%
\pgfsetlinewidth{0.481800pt}%
\definecolor{currentstroke}{rgb}{1.000000,1.000000,1.000000}%
\pgfsetstrokecolor{currentstroke}%
\pgfsetdash{}{0pt}%
\pgfpathmoveto{\pgfqpoint{3.449504in}{2.725034in}}%
\pgfpathcurveto{\pgfqpoint{3.460554in}{2.725034in}}{\pgfqpoint{3.471153in}{2.729424in}}{\pgfqpoint{3.478967in}{2.737238in}}%
\pgfpathcurveto{\pgfqpoint{3.486780in}{2.745051in}}{\pgfqpoint{3.491171in}{2.755651in}}{\pgfqpoint{3.491171in}{2.766701in}}%
\pgfpathcurveto{\pgfqpoint{3.491171in}{2.777751in}}{\pgfqpoint{3.486780in}{2.788350in}}{\pgfqpoint{3.478967in}{2.796163in}}%
\pgfpathcurveto{\pgfqpoint{3.471153in}{2.803977in}}{\pgfqpoint{3.460554in}{2.808367in}}{\pgfqpoint{3.449504in}{2.808367in}}%
\pgfpathcurveto{\pgfqpoint{3.438454in}{2.808367in}}{\pgfqpoint{3.427855in}{2.803977in}}{\pgfqpoint{3.420041in}{2.796163in}}%
\pgfpathcurveto{\pgfqpoint{3.412228in}{2.788350in}}{\pgfqpoint{3.407837in}{2.777751in}}{\pgfqpoint{3.407837in}{2.766701in}}%
\pgfpathcurveto{\pgfqpoint{3.407837in}{2.755651in}}{\pgfqpoint{3.412228in}{2.745051in}}{\pgfqpoint{3.420041in}{2.737238in}}%
\pgfpathcurveto{\pgfqpoint{3.427855in}{2.729424in}}{\pgfqpoint{3.438454in}{2.725034in}}{\pgfqpoint{3.449504in}{2.725034in}}%
\pgfpathclose%
\pgfusepath{stroke,fill}%
\end{pgfscope}%
\begin{pgfscope}%
\pgfpathrectangle{\pgfqpoint{0.481978in}{0.331635in}}{\pgfqpoint{9.300000in}{7.700000in}}%
\pgfusepath{clip}%
\pgfsetbuttcap%
\pgfsetroundjoin%
\definecolor{currentfill}{rgb}{0.815686,0.733333,1.000000}%
\pgfsetfillcolor{currentfill}%
\pgfsetlinewidth{0.481800pt}%
\definecolor{currentstroke}{rgb}{1.000000,1.000000,1.000000}%
\pgfsetstrokecolor{currentstroke}%
\pgfsetdash{}{0pt}%
\pgfpathmoveto{\pgfqpoint{4.947086in}{5.379971in}}%
\pgfpathcurveto{\pgfqpoint{4.958136in}{5.379971in}}{\pgfqpoint{4.968735in}{5.384361in}}{\pgfqpoint{4.976549in}{5.392175in}}%
\pgfpathcurveto{\pgfqpoint{4.984363in}{5.399988in}}{\pgfqpoint{4.988753in}{5.410587in}}{\pgfqpoint{4.988753in}{5.421637in}}%
\pgfpathcurveto{\pgfqpoint{4.988753in}{5.432688in}}{\pgfqpoint{4.984363in}{5.443287in}}{\pgfqpoint{4.976549in}{5.451100in}}%
\pgfpathcurveto{\pgfqpoint{4.968735in}{5.458914in}}{\pgfqpoint{4.958136in}{5.463304in}}{\pgfqpoint{4.947086in}{5.463304in}}%
\pgfpathcurveto{\pgfqpoint{4.936036in}{5.463304in}}{\pgfqpoint{4.925437in}{5.458914in}}{\pgfqpoint{4.917623in}{5.451100in}}%
\pgfpathcurveto{\pgfqpoint{4.909810in}{5.443287in}}{\pgfqpoint{4.905420in}{5.432688in}}{\pgfqpoint{4.905420in}{5.421637in}}%
\pgfpathcurveto{\pgfqpoint{4.905420in}{5.410587in}}{\pgfqpoint{4.909810in}{5.399988in}}{\pgfqpoint{4.917623in}{5.392175in}}%
\pgfpathcurveto{\pgfqpoint{4.925437in}{5.384361in}}{\pgfqpoint{4.936036in}{5.379971in}}{\pgfqpoint{4.947086in}{5.379971in}}%
\pgfpathclose%
\pgfusepath{stroke,fill}%
\end{pgfscope}%
\begin{pgfscope}%
\pgfpathrectangle{\pgfqpoint{0.481978in}{0.331635in}}{\pgfqpoint{9.300000in}{7.700000in}}%
\pgfusepath{clip}%
\pgfsetbuttcap%
\pgfsetroundjoin%
\definecolor{currentfill}{rgb}{0.815686,0.733333,1.000000}%
\pgfsetfillcolor{currentfill}%
\pgfsetlinewidth{0.481800pt}%
\definecolor{currentstroke}{rgb}{1.000000,1.000000,1.000000}%
\pgfsetstrokecolor{currentstroke}%
\pgfsetdash{}{0pt}%
\pgfpathmoveto{\pgfqpoint{6.925838in}{2.873806in}}%
\pgfpathcurveto{\pgfqpoint{6.936889in}{2.873806in}}{\pgfqpoint{6.947488in}{2.878197in}}{\pgfqpoint{6.955301in}{2.886010in}}%
\pgfpathcurveto{\pgfqpoint{6.963115in}{2.893824in}}{\pgfqpoint{6.967505in}{2.904423in}}{\pgfqpoint{6.967505in}{2.915473in}}%
\pgfpathcurveto{\pgfqpoint{6.967505in}{2.926523in}}{\pgfqpoint{6.963115in}{2.937122in}}{\pgfqpoint{6.955301in}{2.944936in}}%
\pgfpathcurveto{\pgfqpoint{6.947488in}{2.952749in}}{\pgfqpoint{6.936889in}{2.957140in}}{\pgfqpoint{6.925838in}{2.957140in}}%
\pgfpathcurveto{\pgfqpoint{6.914788in}{2.957140in}}{\pgfqpoint{6.904189in}{2.952749in}}{\pgfqpoint{6.896376in}{2.944936in}}%
\pgfpathcurveto{\pgfqpoint{6.888562in}{2.937122in}}{\pgfqpoint{6.884172in}{2.926523in}}{\pgfqpoint{6.884172in}{2.915473in}}%
\pgfpathcurveto{\pgfqpoint{6.884172in}{2.904423in}}{\pgfqpoint{6.888562in}{2.893824in}}{\pgfqpoint{6.896376in}{2.886010in}}%
\pgfpathcurveto{\pgfqpoint{6.904189in}{2.878197in}}{\pgfqpoint{6.914788in}{2.873806in}}{\pgfqpoint{6.925838in}{2.873806in}}%
\pgfpathclose%
\pgfusepath{stroke,fill}%
\end{pgfscope}%
\begin{pgfscope}%
\pgfpathrectangle{\pgfqpoint{0.481978in}{0.331635in}}{\pgfqpoint{9.300000in}{7.700000in}}%
\pgfusepath{clip}%
\pgfsetbuttcap%
\pgfsetroundjoin%
\definecolor{currentfill}{rgb}{0.815686,0.733333,1.000000}%
\pgfsetfillcolor{currentfill}%
\pgfsetlinewidth{0.481800pt}%
\definecolor{currentstroke}{rgb}{1.000000,1.000000,1.000000}%
\pgfsetstrokecolor{currentstroke}%
\pgfsetdash{}{0pt}%
\pgfpathmoveto{\pgfqpoint{7.338785in}{3.293368in}}%
\pgfpathcurveto{\pgfqpoint{7.349835in}{3.293368in}}{\pgfqpoint{7.360434in}{3.297758in}}{\pgfqpoint{7.368247in}{3.305572in}}%
\pgfpathcurveto{\pgfqpoint{7.376061in}{3.313385in}}{\pgfqpoint{7.380451in}{3.323984in}}{\pgfqpoint{7.380451in}{3.335035in}}%
\pgfpathcurveto{\pgfqpoint{7.380451in}{3.346085in}}{\pgfqpoint{7.376061in}{3.356684in}}{\pgfqpoint{7.368247in}{3.364497in}}%
\pgfpathcurveto{\pgfqpoint{7.360434in}{3.372311in}}{\pgfqpoint{7.349835in}{3.376701in}}{\pgfqpoint{7.338785in}{3.376701in}}%
\pgfpathcurveto{\pgfqpoint{7.327734in}{3.376701in}}{\pgfqpoint{7.317135in}{3.372311in}}{\pgfqpoint{7.309322in}{3.364497in}}%
\pgfpathcurveto{\pgfqpoint{7.301508in}{3.356684in}}{\pgfqpoint{7.297118in}{3.346085in}}{\pgfqpoint{7.297118in}{3.335035in}}%
\pgfpathcurveto{\pgfqpoint{7.297118in}{3.323984in}}{\pgfqpoint{7.301508in}{3.313385in}}{\pgfqpoint{7.309322in}{3.305572in}}%
\pgfpathcurveto{\pgfqpoint{7.317135in}{3.297758in}}{\pgfqpoint{7.327734in}{3.293368in}}{\pgfqpoint{7.338785in}{3.293368in}}%
\pgfpathclose%
\pgfusepath{stroke,fill}%
\end{pgfscope}%
\begin{pgfscope}%
\pgfpathrectangle{\pgfqpoint{0.481978in}{0.331635in}}{\pgfqpoint{9.300000in}{7.700000in}}%
\pgfusepath{clip}%
\pgfsetbuttcap%
\pgfsetroundjoin%
\definecolor{currentfill}{rgb}{0.815686,0.733333,1.000000}%
\pgfsetfillcolor{currentfill}%
\pgfsetlinewidth{0.481800pt}%
\definecolor{currentstroke}{rgb}{1.000000,1.000000,1.000000}%
\pgfsetstrokecolor{currentstroke}%
\pgfsetdash{}{0pt}%
\pgfpathmoveto{\pgfqpoint{3.955717in}{4.093278in}}%
\pgfpathcurveto{\pgfqpoint{3.966767in}{4.093278in}}{\pgfqpoint{3.977366in}{4.097669in}}{\pgfqpoint{3.985180in}{4.105482in}}%
\pgfpathcurveto{\pgfqpoint{3.992994in}{4.113296in}}{\pgfqpoint{3.997384in}{4.123895in}}{\pgfqpoint{3.997384in}{4.134945in}}%
\pgfpathcurveto{\pgfqpoint{3.997384in}{4.145995in}}{\pgfqpoint{3.992994in}{4.156594in}}{\pgfqpoint{3.985180in}{4.164408in}}%
\pgfpathcurveto{\pgfqpoint{3.977366in}{4.172221in}}{\pgfqpoint{3.966767in}{4.176612in}}{\pgfqpoint{3.955717in}{4.176612in}}%
\pgfpathcurveto{\pgfqpoint{3.944667in}{4.176612in}}{\pgfqpoint{3.934068in}{4.172221in}}{\pgfqpoint{3.926254in}{4.164408in}}%
\pgfpathcurveto{\pgfqpoint{3.918441in}{4.156594in}}{\pgfqpoint{3.914050in}{4.145995in}}{\pgfqpoint{3.914050in}{4.134945in}}%
\pgfpathcurveto{\pgfqpoint{3.914050in}{4.123895in}}{\pgfqpoint{3.918441in}{4.113296in}}{\pgfqpoint{3.926254in}{4.105482in}}%
\pgfpathcurveto{\pgfqpoint{3.934068in}{4.097669in}}{\pgfqpoint{3.944667in}{4.093278in}}{\pgfqpoint{3.955717in}{4.093278in}}%
\pgfpathclose%
\pgfusepath{stroke,fill}%
\end{pgfscope}%
\begin{pgfscope}%
\pgfpathrectangle{\pgfqpoint{0.481978in}{0.331635in}}{\pgfqpoint{9.300000in}{7.700000in}}%
\pgfusepath{clip}%
\pgfsetbuttcap%
\pgfsetroundjoin%
\definecolor{currentfill}{rgb}{0.815686,0.733333,1.000000}%
\pgfsetfillcolor{currentfill}%
\pgfsetlinewidth{0.481800pt}%
\definecolor{currentstroke}{rgb}{1.000000,1.000000,1.000000}%
\pgfsetstrokecolor{currentstroke}%
\pgfsetdash{}{0pt}%
\pgfpathmoveto{\pgfqpoint{2.160067in}{1.882672in}}%
\pgfpathcurveto{\pgfqpoint{2.171117in}{1.882672in}}{\pgfqpoint{2.181717in}{1.887062in}}{\pgfqpoint{2.189530in}{1.894876in}}%
\pgfpathcurveto{\pgfqpoint{2.197344in}{1.902689in}}{\pgfqpoint{2.201734in}{1.913288in}}{\pgfqpoint{2.201734in}{1.924338in}}%
\pgfpathcurveto{\pgfqpoint{2.201734in}{1.935389in}}{\pgfqpoint{2.197344in}{1.945988in}}{\pgfqpoint{2.189530in}{1.953801in}}%
\pgfpathcurveto{\pgfqpoint{2.181717in}{1.961615in}}{\pgfqpoint{2.171117in}{1.966005in}}{\pgfqpoint{2.160067in}{1.966005in}}%
\pgfpathcurveto{\pgfqpoint{2.149017in}{1.966005in}}{\pgfqpoint{2.138418in}{1.961615in}}{\pgfqpoint{2.130605in}{1.953801in}}%
\pgfpathcurveto{\pgfqpoint{2.122791in}{1.945988in}}{\pgfqpoint{2.118401in}{1.935389in}}{\pgfqpoint{2.118401in}{1.924338in}}%
\pgfpathcurveto{\pgfqpoint{2.118401in}{1.913288in}}{\pgfqpoint{2.122791in}{1.902689in}}{\pgfqpoint{2.130605in}{1.894876in}}%
\pgfpathcurveto{\pgfqpoint{2.138418in}{1.887062in}}{\pgfqpoint{2.149017in}{1.882672in}}{\pgfqpoint{2.160067in}{1.882672in}}%
\pgfpathclose%
\pgfusepath{stroke,fill}%
\end{pgfscope}%
\begin{pgfscope}%
\pgfpathrectangle{\pgfqpoint{0.481978in}{0.331635in}}{\pgfqpoint{9.300000in}{7.700000in}}%
\pgfusepath{clip}%
\pgfsetbuttcap%
\pgfsetroundjoin%
\definecolor{currentfill}{rgb}{0.815686,0.733333,1.000000}%
\pgfsetfillcolor{currentfill}%
\pgfsetlinewidth{0.481800pt}%
\definecolor{currentstroke}{rgb}{1.000000,1.000000,1.000000}%
\pgfsetstrokecolor{currentstroke}%
\pgfsetdash{}{0pt}%
\pgfpathmoveto{\pgfqpoint{4.198073in}{4.221155in}}%
\pgfpathcurveto{\pgfqpoint{4.209123in}{4.221155in}}{\pgfqpoint{4.219722in}{4.225546in}}{\pgfqpoint{4.227535in}{4.233359in}}%
\pgfpathcurveto{\pgfqpoint{4.235349in}{4.241173in}}{\pgfqpoint{4.239739in}{4.251772in}}{\pgfqpoint{4.239739in}{4.262822in}}%
\pgfpathcurveto{\pgfqpoint{4.239739in}{4.273872in}}{\pgfqpoint{4.235349in}{4.284471in}}{\pgfqpoint{4.227535in}{4.292285in}}%
\pgfpathcurveto{\pgfqpoint{4.219722in}{4.300098in}}{\pgfqpoint{4.209123in}{4.304489in}}{\pgfqpoint{4.198073in}{4.304489in}}%
\pgfpathcurveto{\pgfqpoint{4.187023in}{4.304489in}}{\pgfqpoint{4.176424in}{4.300098in}}{\pgfqpoint{4.168610in}{4.292285in}}%
\pgfpathcurveto{\pgfqpoint{4.160796in}{4.284471in}}{\pgfqpoint{4.156406in}{4.273872in}}{\pgfqpoint{4.156406in}{4.262822in}}%
\pgfpathcurveto{\pgfqpoint{4.156406in}{4.251772in}}{\pgfqpoint{4.160796in}{4.241173in}}{\pgfqpoint{4.168610in}{4.233359in}}%
\pgfpathcurveto{\pgfqpoint{4.176424in}{4.225546in}}{\pgfqpoint{4.187023in}{4.221155in}}{\pgfqpoint{4.198073in}{4.221155in}}%
\pgfpathclose%
\pgfusepath{stroke,fill}%
\end{pgfscope}%
\begin{pgfscope}%
\pgfpathrectangle{\pgfqpoint{0.481978in}{0.331635in}}{\pgfqpoint{9.300000in}{7.700000in}}%
\pgfusepath{clip}%
\pgfsetbuttcap%
\pgfsetroundjoin%
\definecolor{currentfill}{rgb}{0.815686,0.733333,1.000000}%
\pgfsetfillcolor{currentfill}%
\pgfsetlinewidth{0.481800pt}%
\definecolor{currentstroke}{rgb}{1.000000,1.000000,1.000000}%
\pgfsetstrokecolor{currentstroke}%
\pgfsetdash{}{0pt}%
\pgfpathmoveto{\pgfqpoint{3.029876in}{4.890266in}}%
\pgfpathcurveto{\pgfqpoint{3.040926in}{4.890266in}}{\pgfqpoint{3.051525in}{4.894657in}}{\pgfqpoint{3.059339in}{4.902470in}}%
\pgfpathcurveto{\pgfqpoint{3.067152in}{4.910284in}}{\pgfqpoint{3.071542in}{4.920883in}}{\pgfqpoint{3.071542in}{4.931933in}}%
\pgfpathcurveto{\pgfqpoint{3.071542in}{4.942983in}}{\pgfqpoint{3.067152in}{4.953582in}}{\pgfqpoint{3.059339in}{4.961396in}}%
\pgfpathcurveto{\pgfqpoint{3.051525in}{4.969209in}}{\pgfqpoint{3.040926in}{4.973600in}}{\pgfqpoint{3.029876in}{4.973600in}}%
\pgfpathcurveto{\pgfqpoint{3.018826in}{4.973600in}}{\pgfqpoint{3.008227in}{4.969209in}}{\pgfqpoint{3.000413in}{4.961396in}}%
\pgfpathcurveto{\pgfqpoint{2.992599in}{4.953582in}}{\pgfqpoint{2.988209in}{4.942983in}}{\pgfqpoint{2.988209in}{4.931933in}}%
\pgfpathcurveto{\pgfqpoint{2.988209in}{4.920883in}}{\pgfqpoint{2.992599in}{4.910284in}}{\pgfqpoint{3.000413in}{4.902470in}}%
\pgfpathcurveto{\pgfqpoint{3.008227in}{4.894657in}}{\pgfqpoint{3.018826in}{4.890266in}}{\pgfqpoint{3.029876in}{4.890266in}}%
\pgfpathclose%
\pgfusepath{stroke,fill}%
\end{pgfscope}%
\begin{pgfscope}%
\pgfpathrectangle{\pgfqpoint{0.481978in}{0.331635in}}{\pgfqpoint{9.300000in}{7.700000in}}%
\pgfusepath{clip}%
\pgfsetbuttcap%
\pgfsetroundjoin%
\definecolor{currentfill}{rgb}{0.815686,0.733333,1.000000}%
\pgfsetfillcolor{currentfill}%
\pgfsetlinewidth{0.481800pt}%
\definecolor{currentstroke}{rgb}{1.000000,1.000000,1.000000}%
\pgfsetstrokecolor{currentstroke}%
\pgfsetdash{}{0pt}%
\pgfpathmoveto{\pgfqpoint{1.310658in}{4.217106in}}%
\pgfpathcurveto{\pgfqpoint{1.321708in}{4.217106in}}{\pgfqpoint{1.332307in}{4.221497in}}{\pgfqpoint{1.340121in}{4.229310in}}%
\pgfpathcurveto{\pgfqpoint{1.347934in}{4.237124in}}{\pgfqpoint{1.352325in}{4.247723in}}{\pgfqpoint{1.352325in}{4.258773in}}%
\pgfpathcurveto{\pgfqpoint{1.352325in}{4.269823in}}{\pgfqpoint{1.347934in}{4.280422in}}{\pgfqpoint{1.340121in}{4.288236in}}%
\pgfpathcurveto{\pgfqpoint{1.332307in}{4.296050in}}{\pgfqpoint{1.321708in}{4.300440in}}{\pgfqpoint{1.310658in}{4.300440in}}%
\pgfpathcurveto{\pgfqpoint{1.299608in}{4.300440in}}{\pgfqpoint{1.289009in}{4.296050in}}{\pgfqpoint{1.281195in}{4.288236in}}%
\pgfpathcurveto{\pgfqpoint{1.273382in}{4.280422in}}{\pgfqpoint{1.268991in}{4.269823in}}{\pgfqpoint{1.268991in}{4.258773in}}%
\pgfpathcurveto{\pgfqpoint{1.268991in}{4.247723in}}{\pgfqpoint{1.273382in}{4.237124in}}{\pgfqpoint{1.281195in}{4.229310in}}%
\pgfpathcurveto{\pgfqpoint{1.289009in}{4.221497in}}{\pgfqpoint{1.299608in}{4.217106in}}{\pgfqpoint{1.310658in}{4.217106in}}%
\pgfpathclose%
\pgfusepath{stroke,fill}%
\end{pgfscope}%
\begin{pgfscope}%
\pgfpathrectangle{\pgfqpoint{0.481978in}{0.331635in}}{\pgfqpoint{9.300000in}{7.700000in}}%
\pgfusepath{clip}%
\pgfsetbuttcap%
\pgfsetroundjoin%
\definecolor{currentfill}{rgb}{0.815686,0.733333,1.000000}%
\pgfsetfillcolor{currentfill}%
\pgfsetlinewidth{0.481800pt}%
\definecolor{currentstroke}{rgb}{1.000000,1.000000,1.000000}%
\pgfsetstrokecolor{currentstroke}%
\pgfsetdash{}{0pt}%
\pgfpathmoveto{\pgfqpoint{5.061821in}{4.668455in}}%
\pgfpathcurveto{\pgfqpoint{5.072872in}{4.668455in}}{\pgfqpoint{5.083471in}{4.672845in}}{\pgfqpoint{5.091284in}{4.680659in}}%
\pgfpathcurveto{\pgfqpoint{5.099098in}{4.688473in}}{\pgfqpoint{5.103488in}{4.699072in}}{\pgfqpoint{5.103488in}{4.710122in}}%
\pgfpathcurveto{\pgfqpoint{5.103488in}{4.721172in}}{\pgfqpoint{5.099098in}{4.731771in}}{\pgfqpoint{5.091284in}{4.739585in}}%
\pgfpathcurveto{\pgfqpoint{5.083471in}{4.747398in}}{\pgfqpoint{5.072872in}{4.751788in}}{\pgfqpoint{5.061821in}{4.751788in}}%
\pgfpathcurveto{\pgfqpoint{5.050771in}{4.751788in}}{\pgfqpoint{5.040172in}{4.747398in}}{\pgfqpoint{5.032359in}{4.739585in}}%
\pgfpathcurveto{\pgfqpoint{5.024545in}{4.731771in}}{\pgfqpoint{5.020155in}{4.721172in}}{\pgfqpoint{5.020155in}{4.710122in}}%
\pgfpathcurveto{\pgfqpoint{5.020155in}{4.699072in}}{\pgfqpoint{5.024545in}{4.688473in}}{\pgfqpoint{5.032359in}{4.680659in}}%
\pgfpathcurveto{\pgfqpoint{5.040172in}{4.672845in}}{\pgfqpoint{5.050771in}{4.668455in}}{\pgfqpoint{5.061821in}{4.668455in}}%
\pgfpathclose%
\pgfusepath{stroke,fill}%
\end{pgfscope}%
\begin{pgfscope}%
\pgfpathrectangle{\pgfqpoint{0.481978in}{0.331635in}}{\pgfqpoint{9.300000in}{7.700000in}}%
\pgfusepath{clip}%
\pgfsetbuttcap%
\pgfsetroundjoin%
\definecolor{currentfill}{rgb}{0.815686,0.733333,1.000000}%
\pgfsetfillcolor{currentfill}%
\pgfsetlinewidth{0.481800pt}%
\definecolor{currentstroke}{rgb}{1.000000,1.000000,1.000000}%
\pgfsetstrokecolor{currentstroke}%
\pgfsetdash{}{0pt}%
\pgfpathmoveto{\pgfqpoint{5.046020in}{5.012374in}}%
\pgfpathcurveto{\pgfqpoint{5.057070in}{5.012374in}}{\pgfqpoint{5.067669in}{5.016764in}}{\pgfqpoint{5.075482in}{5.024578in}}%
\pgfpathcurveto{\pgfqpoint{5.083296in}{5.032391in}}{\pgfqpoint{5.087686in}{5.042990in}}{\pgfqpoint{5.087686in}{5.054040in}}%
\pgfpathcurveto{\pgfqpoint{5.087686in}{5.065091in}}{\pgfqpoint{5.083296in}{5.075690in}}{\pgfqpoint{5.075482in}{5.083503in}}%
\pgfpathcurveto{\pgfqpoint{5.067669in}{5.091317in}}{\pgfqpoint{5.057070in}{5.095707in}}{\pgfqpoint{5.046020in}{5.095707in}}%
\pgfpathcurveto{\pgfqpoint{5.034970in}{5.095707in}}{\pgfqpoint{5.024371in}{5.091317in}}{\pgfqpoint{5.016557in}{5.083503in}}%
\pgfpathcurveto{\pgfqpoint{5.008743in}{5.075690in}}{\pgfqpoint{5.004353in}{5.065091in}}{\pgfqpoint{5.004353in}{5.054040in}}%
\pgfpathcurveto{\pgfqpoint{5.004353in}{5.042990in}}{\pgfqpoint{5.008743in}{5.032391in}}{\pgfqpoint{5.016557in}{5.024578in}}%
\pgfpathcurveto{\pgfqpoint{5.024371in}{5.016764in}}{\pgfqpoint{5.034970in}{5.012374in}}{\pgfqpoint{5.046020in}{5.012374in}}%
\pgfpathclose%
\pgfusepath{stroke,fill}%
\end{pgfscope}%
\begin{pgfscope}%
\pgfpathrectangle{\pgfqpoint{0.481978in}{0.331635in}}{\pgfqpoint{9.300000in}{7.700000in}}%
\pgfusepath{clip}%
\pgfsetbuttcap%
\pgfsetroundjoin%
\definecolor{currentfill}{rgb}{0.815686,0.733333,1.000000}%
\pgfsetfillcolor{currentfill}%
\pgfsetlinewidth{0.481800pt}%
\definecolor{currentstroke}{rgb}{1.000000,1.000000,1.000000}%
\pgfsetstrokecolor{currentstroke}%
\pgfsetdash{}{0pt}%
\pgfpathmoveto{\pgfqpoint{7.569943in}{3.637722in}}%
\pgfpathcurveto{\pgfqpoint{7.580993in}{3.637722in}}{\pgfqpoint{7.591592in}{3.642112in}}{\pgfqpoint{7.599406in}{3.649926in}}%
\pgfpathcurveto{\pgfqpoint{7.607219in}{3.657740in}}{\pgfqpoint{7.611610in}{3.668339in}}{\pgfqpoint{7.611610in}{3.679389in}}%
\pgfpathcurveto{\pgfqpoint{7.611610in}{3.690439in}}{\pgfqpoint{7.607219in}{3.701038in}}{\pgfqpoint{7.599406in}{3.708851in}}%
\pgfpathcurveto{\pgfqpoint{7.591592in}{3.716665in}}{\pgfqpoint{7.580993in}{3.721055in}}{\pgfqpoint{7.569943in}{3.721055in}}%
\pgfpathcurveto{\pgfqpoint{7.558893in}{3.721055in}}{\pgfqpoint{7.548294in}{3.716665in}}{\pgfqpoint{7.540480in}{3.708851in}}%
\pgfpathcurveto{\pgfqpoint{7.532667in}{3.701038in}}{\pgfqpoint{7.528276in}{3.690439in}}{\pgfqpoint{7.528276in}{3.679389in}}%
\pgfpathcurveto{\pgfqpoint{7.528276in}{3.668339in}}{\pgfqpoint{7.532667in}{3.657740in}}{\pgfqpoint{7.540480in}{3.649926in}}%
\pgfpathcurveto{\pgfqpoint{7.548294in}{3.642112in}}{\pgfqpoint{7.558893in}{3.637722in}}{\pgfqpoint{7.569943in}{3.637722in}}%
\pgfpathclose%
\pgfusepath{stroke,fill}%
\end{pgfscope}%
\begin{pgfscope}%
\pgfpathrectangle{\pgfqpoint{0.481978in}{0.331635in}}{\pgfqpoint{9.300000in}{7.700000in}}%
\pgfusepath{clip}%
\pgfsetbuttcap%
\pgfsetroundjoin%
\definecolor{currentfill}{rgb}{0.815686,0.733333,1.000000}%
\pgfsetfillcolor{currentfill}%
\pgfsetlinewidth{0.481800pt}%
\definecolor{currentstroke}{rgb}{1.000000,1.000000,1.000000}%
\pgfsetstrokecolor{currentstroke}%
\pgfsetdash{}{0pt}%
\pgfpathmoveto{\pgfqpoint{7.674879in}{3.033872in}}%
\pgfpathcurveto{\pgfqpoint{7.685929in}{3.033872in}}{\pgfqpoint{7.696528in}{3.038262in}}{\pgfqpoint{7.704342in}{3.046076in}}%
\pgfpathcurveto{\pgfqpoint{7.712155in}{3.053889in}}{\pgfqpoint{7.716546in}{3.064489in}}{\pgfqpoint{7.716546in}{3.075539in}}%
\pgfpathcurveto{\pgfqpoint{7.716546in}{3.086589in}}{\pgfqpoint{7.712155in}{3.097188in}}{\pgfqpoint{7.704342in}{3.105001in}}%
\pgfpathcurveto{\pgfqpoint{7.696528in}{3.112815in}}{\pgfqpoint{7.685929in}{3.117205in}}{\pgfqpoint{7.674879in}{3.117205in}}%
\pgfpathcurveto{\pgfqpoint{7.663829in}{3.117205in}}{\pgfqpoint{7.653230in}{3.112815in}}{\pgfqpoint{7.645416in}{3.105001in}}%
\pgfpathcurveto{\pgfqpoint{7.637603in}{3.097188in}}{\pgfqpoint{7.633212in}{3.086589in}}{\pgfqpoint{7.633212in}{3.075539in}}%
\pgfpathcurveto{\pgfqpoint{7.633212in}{3.064489in}}{\pgfqpoint{7.637603in}{3.053889in}}{\pgfqpoint{7.645416in}{3.046076in}}%
\pgfpathcurveto{\pgfqpoint{7.653230in}{3.038262in}}{\pgfqpoint{7.663829in}{3.033872in}}{\pgfqpoint{7.674879in}{3.033872in}}%
\pgfpathclose%
\pgfusepath{stroke,fill}%
\end{pgfscope}%
\begin{pgfscope}%
\pgfpathrectangle{\pgfqpoint{0.481978in}{0.331635in}}{\pgfqpoint{9.300000in}{7.700000in}}%
\pgfusepath{clip}%
\pgfsetbuttcap%
\pgfsetroundjoin%
\definecolor{currentfill}{rgb}{0.815686,0.733333,1.000000}%
\pgfsetfillcolor{currentfill}%
\pgfsetlinewidth{0.481800pt}%
\definecolor{currentstroke}{rgb}{1.000000,1.000000,1.000000}%
\pgfsetstrokecolor{currentstroke}%
\pgfsetdash{}{0pt}%
\pgfpathmoveto{\pgfqpoint{5.428071in}{3.072967in}}%
\pgfpathcurveto{\pgfqpoint{5.439121in}{3.072967in}}{\pgfqpoint{5.449720in}{3.077357in}}{\pgfqpoint{5.457534in}{3.085171in}}%
\pgfpathcurveto{\pgfqpoint{5.465347in}{3.092985in}}{\pgfqpoint{5.469737in}{3.103584in}}{\pgfqpoint{5.469737in}{3.114634in}}%
\pgfpathcurveto{\pgfqpoint{5.469737in}{3.125684in}}{\pgfqpoint{5.465347in}{3.136283in}}{\pgfqpoint{5.457534in}{3.144097in}}%
\pgfpathcurveto{\pgfqpoint{5.449720in}{3.151910in}}{\pgfqpoint{5.439121in}{3.156300in}}{\pgfqpoint{5.428071in}{3.156300in}}%
\pgfpathcurveto{\pgfqpoint{5.417021in}{3.156300in}}{\pgfqpoint{5.406422in}{3.151910in}}{\pgfqpoint{5.398608in}{3.144097in}}%
\pgfpathcurveto{\pgfqpoint{5.390794in}{3.136283in}}{\pgfqpoint{5.386404in}{3.125684in}}{\pgfqpoint{5.386404in}{3.114634in}}%
\pgfpathcurveto{\pgfqpoint{5.386404in}{3.103584in}}{\pgfqpoint{5.390794in}{3.092985in}}{\pgfqpoint{5.398608in}{3.085171in}}%
\pgfpathcurveto{\pgfqpoint{5.406422in}{3.077357in}}{\pgfqpoint{5.417021in}{3.072967in}}{\pgfqpoint{5.428071in}{3.072967in}}%
\pgfpathclose%
\pgfusepath{stroke,fill}%
\end{pgfscope}%
\begin{pgfscope}%
\pgfpathrectangle{\pgfqpoint{0.481978in}{0.331635in}}{\pgfqpoint{9.300000in}{7.700000in}}%
\pgfusepath{clip}%
\pgfsetbuttcap%
\pgfsetroundjoin%
\definecolor{currentfill}{rgb}{0.815686,0.733333,1.000000}%
\pgfsetfillcolor{currentfill}%
\pgfsetlinewidth{0.481800pt}%
\definecolor{currentstroke}{rgb}{1.000000,1.000000,1.000000}%
\pgfsetstrokecolor{currentstroke}%
\pgfsetdash{}{0pt}%
\pgfpathmoveto{\pgfqpoint{2.403350in}{3.601610in}}%
\pgfpathcurveto{\pgfqpoint{2.414400in}{3.601610in}}{\pgfqpoint{2.425000in}{3.606000in}}{\pgfqpoint{2.432813in}{3.613814in}}%
\pgfpathcurveto{\pgfqpoint{2.440627in}{3.621627in}}{\pgfqpoint{2.445017in}{3.632226in}}{\pgfqpoint{2.445017in}{3.643277in}}%
\pgfpathcurveto{\pgfqpoint{2.445017in}{3.654327in}}{\pgfqpoint{2.440627in}{3.664926in}}{\pgfqpoint{2.432813in}{3.672739in}}%
\pgfpathcurveto{\pgfqpoint{2.425000in}{3.680553in}}{\pgfqpoint{2.414400in}{3.684943in}}{\pgfqpoint{2.403350in}{3.684943in}}%
\pgfpathcurveto{\pgfqpoint{2.392300in}{3.684943in}}{\pgfqpoint{2.381701in}{3.680553in}}{\pgfqpoint{2.373888in}{3.672739in}}%
\pgfpathcurveto{\pgfqpoint{2.366074in}{3.664926in}}{\pgfqpoint{2.361684in}{3.654327in}}{\pgfqpoint{2.361684in}{3.643277in}}%
\pgfpathcurveto{\pgfqpoint{2.361684in}{3.632226in}}{\pgfqpoint{2.366074in}{3.621627in}}{\pgfqpoint{2.373888in}{3.613814in}}%
\pgfpathcurveto{\pgfqpoint{2.381701in}{3.606000in}}{\pgfqpoint{2.392300in}{3.601610in}}{\pgfqpoint{2.403350in}{3.601610in}}%
\pgfpathclose%
\pgfusepath{stroke,fill}%
\end{pgfscope}%
\begin{pgfscope}%
\pgfpathrectangle{\pgfqpoint{0.481978in}{0.331635in}}{\pgfqpoint{9.300000in}{7.700000in}}%
\pgfusepath{clip}%
\pgfsetbuttcap%
\pgfsetroundjoin%
\definecolor{currentfill}{rgb}{0.815686,0.733333,1.000000}%
\pgfsetfillcolor{currentfill}%
\pgfsetlinewidth{0.481800pt}%
\definecolor{currentstroke}{rgb}{1.000000,1.000000,1.000000}%
\pgfsetstrokecolor{currentstroke}%
\pgfsetdash{}{0pt}%
\pgfpathmoveto{\pgfqpoint{6.041612in}{5.592475in}}%
\pgfpathcurveto{\pgfqpoint{6.052662in}{5.592475in}}{\pgfqpoint{6.063261in}{5.596865in}}{\pgfqpoint{6.071075in}{5.604679in}}%
\pgfpathcurveto{\pgfqpoint{6.078889in}{5.612492in}}{\pgfqpoint{6.083279in}{5.623091in}}{\pgfqpoint{6.083279in}{5.634142in}}%
\pgfpathcurveto{\pgfqpoint{6.083279in}{5.645192in}}{\pgfqpoint{6.078889in}{5.655791in}}{\pgfqpoint{6.071075in}{5.663604in}}%
\pgfpathcurveto{\pgfqpoint{6.063261in}{5.671418in}}{\pgfqpoint{6.052662in}{5.675808in}}{\pgfqpoint{6.041612in}{5.675808in}}%
\pgfpathcurveto{\pgfqpoint{6.030562in}{5.675808in}}{\pgfqpoint{6.019963in}{5.671418in}}{\pgfqpoint{6.012150in}{5.663604in}}%
\pgfpathcurveto{\pgfqpoint{6.004336in}{5.655791in}}{\pgfqpoint{5.999946in}{5.645192in}}{\pgfqpoint{5.999946in}{5.634142in}}%
\pgfpathcurveto{\pgfqpoint{5.999946in}{5.623091in}}{\pgfqpoint{6.004336in}{5.612492in}}{\pgfqpoint{6.012150in}{5.604679in}}%
\pgfpathcurveto{\pgfqpoint{6.019963in}{5.596865in}}{\pgfqpoint{6.030562in}{5.592475in}}{\pgfqpoint{6.041612in}{5.592475in}}%
\pgfpathclose%
\pgfusepath{stroke,fill}%
\end{pgfscope}%
\begin{pgfscope}%
\pgfpathrectangle{\pgfqpoint{0.481978in}{0.331635in}}{\pgfqpoint{9.300000in}{7.700000in}}%
\pgfusepath{clip}%
\pgfsetbuttcap%
\pgfsetroundjoin%
\definecolor{currentfill}{rgb}{0.815686,0.733333,1.000000}%
\pgfsetfillcolor{currentfill}%
\pgfsetlinewidth{0.481800pt}%
\definecolor{currentstroke}{rgb}{1.000000,1.000000,1.000000}%
\pgfsetstrokecolor{currentstroke}%
\pgfsetdash{}{0pt}%
\pgfpathmoveto{\pgfqpoint{8.383007in}{6.162090in}}%
\pgfpathcurveto{\pgfqpoint{8.394058in}{6.162090in}}{\pgfqpoint{8.404657in}{6.166480in}}{\pgfqpoint{8.412470in}{6.174294in}}%
\pgfpathcurveto{\pgfqpoint{8.420284in}{6.182107in}}{\pgfqpoint{8.424674in}{6.192706in}}{\pgfqpoint{8.424674in}{6.203756in}}%
\pgfpathcurveto{\pgfqpoint{8.424674in}{6.214807in}}{\pgfqpoint{8.420284in}{6.225406in}}{\pgfqpoint{8.412470in}{6.233219in}}%
\pgfpathcurveto{\pgfqpoint{8.404657in}{6.241033in}}{\pgfqpoint{8.394058in}{6.245423in}}{\pgfqpoint{8.383007in}{6.245423in}}%
\pgfpathcurveto{\pgfqpoint{8.371957in}{6.245423in}}{\pgfqpoint{8.361358in}{6.241033in}}{\pgfqpoint{8.353545in}{6.233219in}}%
\pgfpathcurveto{\pgfqpoint{8.345731in}{6.225406in}}{\pgfqpoint{8.341341in}{6.214807in}}{\pgfqpoint{8.341341in}{6.203756in}}%
\pgfpathcurveto{\pgfqpoint{8.341341in}{6.192706in}}{\pgfqpoint{8.345731in}{6.182107in}}{\pgfqpoint{8.353545in}{6.174294in}}%
\pgfpathcurveto{\pgfqpoint{8.361358in}{6.166480in}}{\pgfqpoint{8.371957in}{6.162090in}}{\pgfqpoint{8.383007in}{6.162090in}}%
\pgfpathclose%
\pgfusepath{stroke,fill}%
\end{pgfscope}%
\begin{pgfscope}%
\pgfpathrectangle{\pgfqpoint{0.481978in}{0.331635in}}{\pgfqpoint{9.300000in}{7.700000in}}%
\pgfusepath{clip}%
\pgfsetbuttcap%
\pgfsetroundjoin%
\definecolor{currentfill}{rgb}{0.815686,0.733333,1.000000}%
\pgfsetfillcolor{currentfill}%
\pgfsetlinewidth{0.481800pt}%
\definecolor{currentstroke}{rgb}{1.000000,1.000000,1.000000}%
\pgfsetstrokecolor{currentstroke}%
\pgfsetdash{}{0pt}%
\pgfpathmoveto{\pgfqpoint{4.398769in}{4.525825in}}%
\pgfpathcurveto{\pgfqpoint{4.409819in}{4.525825in}}{\pgfqpoint{4.420418in}{4.530215in}}{\pgfqpoint{4.428231in}{4.538029in}}%
\pgfpathcurveto{\pgfqpoint{4.436045in}{4.545842in}}{\pgfqpoint{4.440435in}{4.556441in}}{\pgfqpoint{4.440435in}{4.567491in}}%
\pgfpathcurveto{\pgfqpoint{4.440435in}{4.578542in}}{\pgfqpoint{4.436045in}{4.589141in}}{\pgfqpoint{4.428231in}{4.596954in}}%
\pgfpathcurveto{\pgfqpoint{4.420418in}{4.604768in}}{\pgfqpoint{4.409819in}{4.609158in}}{\pgfqpoint{4.398769in}{4.609158in}}%
\pgfpathcurveto{\pgfqpoint{4.387718in}{4.609158in}}{\pgfqpoint{4.377119in}{4.604768in}}{\pgfqpoint{4.369306in}{4.596954in}}%
\pgfpathcurveto{\pgfqpoint{4.361492in}{4.589141in}}{\pgfqpoint{4.357102in}{4.578542in}}{\pgfqpoint{4.357102in}{4.567491in}}%
\pgfpathcurveto{\pgfqpoint{4.357102in}{4.556441in}}{\pgfqpoint{4.361492in}{4.545842in}}{\pgfqpoint{4.369306in}{4.538029in}}%
\pgfpathcurveto{\pgfqpoint{4.377119in}{4.530215in}}{\pgfqpoint{4.387718in}{4.525825in}}{\pgfqpoint{4.398769in}{4.525825in}}%
\pgfpathclose%
\pgfusepath{stroke,fill}%
\end{pgfscope}%
\begin{pgfscope}%
\pgfpathrectangle{\pgfqpoint{0.481978in}{0.331635in}}{\pgfqpoint{9.300000in}{7.700000in}}%
\pgfusepath{clip}%
\pgfsetbuttcap%
\pgfsetroundjoin%
\definecolor{currentfill}{rgb}{0.815686,0.733333,1.000000}%
\pgfsetfillcolor{currentfill}%
\pgfsetlinewidth{0.481800pt}%
\definecolor{currentstroke}{rgb}{1.000000,1.000000,1.000000}%
\pgfsetstrokecolor{currentstroke}%
\pgfsetdash{}{0pt}%
\pgfpathmoveto{\pgfqpoint{8.682490in}{5.777703in}}%
\pgfpathcurveto{\pgfqpoint{8.693540in}{5.777703in}}{\pgfqpoint{8.704139in}{5.782093in}}{\pgfqpoint{8.711953in}{5.789906in}}%
\pgfpathcurveto{\pgfqpoint{8.719766in}{5.797720in}}{\pgfqpoint{8.724157in}{5.808319in}}{\pgfqpoint{8.724157in}{5.819369in}}%
\pgfpathcurveto{\pgfqpoint{8.724157in}{5.830419in}}{\pgfqpoint{8.719766in}{5.841018in}}{\pgfqpoint{8.711953in}{5.848832in}}%
\pgfpathcurveto{\pgfqpoint{8.704139in}{5.856646in}}{\pgfqpoint{8.693540in}{5.861036in}}{\pgfqpoint{8.682490in}{5.861036in}}%
\pgfpathcurveto{\pgfqpoint{8.671440in}{5.861036in}}{\pgfqpoint{8.660841in}{5.856646in}}{\pgfqpoint{8.653027in}{5.848832in}}%
\pgfpathcurveto{\pgfqpoint{8.645213in}{5.841018in}}{\pgfqpoint{8.640823in}{5.830419in}}{\pgfqpoint{8.640823in}{5.819369in}}%
\pgfpathcurveto{\pgfqpoint{8.640823in}{5.808319in}}{\pgfqpoint{8.645213in}{5.797720in}}{\pgfqpoint{8.653027in}{5.789906in}}%
\pgfpathcurveto{\pgfqpoint{8.660841in}{5.782093in}}{\pgfqpoint{8.671440in}{5.777703in}}{\pgfqpoint{8.682490in}{5.777703in}}%
\pgfpathclose%
\pgfusepath{stroke,fill}%
\end{pgfscope}%
\begin{pgfscope}%
\pgfpathrectangle{\pgfqpoint{0.481978in}{0.331635in}}{\pgfqpoint{9.300000in}{7.700000in}}%
\pgfusepath{clip}%
\pgfsetbuttcap%
\pgfsetroundjoin%
\definecolor{currentfill}{rgb}{0.815686,0.733333,1.000000}%
\pgfsetfillcolor{currentfill}%
\pgfsetlinewidth{0.481800pt}%
\definecolor{currentstroke}{rgb}{1.000000,1.000000,1.000000}%
\pgfsetstrokecolor{currentstroke}%
\pgfsetdash{}{0pt}%
\pgfpathmoveto{\pgfqpoint{5.664598in}{1.113364in}}%
\pgfpathcurveto{\pgfqpoint{5.675648in}{1.113364in}}{\pgfqpoint{5.686248in}{1.117754in}}{\pgfqpoint{5.694061in}{1.125568in}}%
\pgfpathcurveto{\pgfqpoint{5.701875in}{1.133382in}}{\pgfqpoint{5.706265in}{1.143981in}}{\pgfqpoint{5.706265in}{1.155031in}}%
\pgfpathcurveto{\pgfqpoint{5.706265in}{1.166081in}}{\pgfqpoint{5.701875in}{1.176680in}}{\pgfqpoint{5.694061in}{1.184494in}}%
\pgfpathcurveto{\pgfqpoint{5.686248in}{1.192307in}}{\pgfqpoint{5.675648in}{1.196698in}}{\pgfqpoint{5.664598in}{1.196698in}}%
\pgfpathcurveto{\pgfqpoint{5.653548in}{1.196698in}}{\pgfqpoint{5.642949in}{1.192307in}}{\pgfqpoint{5.635136in}{1.184494in}}%
\pgfpathcurveto{\pgfqpoint{5.627322in}{1.176680in}}{\pgfqpoint{5.622932in}{1.166081in}}{\pgfqpoint{5.622932in}{1.155031in}}%
\pgfpathcurveto{\pgfqpoint{5.622932in}{1.143981in}}{\pgfqpoint{5.627322in}{1.133382in}}{\pgfqpoint{5.635136in}{1.125568in}}%
\pgfpathcurveto{\pgfqpoint{5.642949in}{1.117754in}}{\pgfqpoint{5.653548in}{1.113364in}}{\pgfqpoint{5.664598in}{1.113364in}}%
\pgfpathclose%
\pgfusepath{stroke,fill}%
\end{pgfscope}%
\begin{pgfscope}%
\pgfpathrectangle{\pgfqpoint{0.481978in}{0.331635in}}{\pgfqpoint{9.300000in}{7.700000in}}%
\pgfusepath{clip}%
\pgfsetbuttcap%
\pgfsetroundjoin%
\definecolor{currentfill}{rgb}{0.815686,0.733333,1.000000}%
\pgfsetfillcolor{currentfill}%
\pgfsetlinewidth{0.481800pt}%
\definecolor{currentstroke}{rgb}{1.000000,1.000000,1.000000}%
\pgfsetstrokecolor{currentstroke}%
\pgfsetdash{}{0pt}%
\pgfpathmoveto{\pgfqpoint{1.939227in}{4.311857in}}%
\pgfpathcurveto{\pgfqpoint{1.950277in}{4.311857in}}{\pgfqpoint{1.960876in}{4.316247in}}{\pgfqpoint{1.968690in}{4.324061in}}%
\pgfpathcurveto{\pgfqpoint{1.976504in}{4.331874in}}{\pgfqpoint{1.980894in}{4.342473in}}{\pgfqpoint{1.980894in}{4.353523in}}%
\pgfpathcurveto{\pgfqpoint{1.980894in}{4.364574in}}{\pgfqpoint{1.976504in}{4.375173in}}{\pgfqpoint{1.968690in}{4.382986in}}%
\pgfpathcurveto{\pgfqpoint{1.960876in}{4.390800in}}{\pgfqpoint{1.950277in}{4.395190in}}{\pgfqpoint{1.939227in}{4.395190in}}%
\pgfpathcurveto{\pgfqpoint{1.928177in}{4.395190in}}{\pgfqpoint{1.917578in}{4.390800in}}{\pgfqpoint{1.909765in}{4.382986in}}%
\pgfpathcurveto{\pgfqpoint{1.901951in}{4.375173in}}{\pgfqpoint{1.897561in}{4.364574in}}{\pgfqpoint{1.897561in}{4.353523in}}%
\pgfpathcurveto{\pgfqpoint{1.897561in}{4.342473in}}{\pgfqpoint{1.901951in}{4.331874in}}{\pgfqpoint{1.909765in}{4.324061in}}%
\pgfpathcurveto{\pgfqpoint{1.917578in}{4.316247in}}{\pgfqpoint{1.928177in}{4.311857in}}{\pgfqpoint{1.939227in}{4.311857in}}%
\pgfpathclose%
\pgfusepath{stroke,fill}%
\end{pgfscope}%
\begin{pgfscope}%
\pgfpathrectangle{\pgfqpoint{0.481978in}{0.331635in}}{\pgfqpoint{9.300000in}{7.700000in}}%
\pgfusepath{clip}%
\pgfsetbuttcap%
\pgfsetroundjoin%
\definecolor{currentfill}{rgb}{0.815686,0.733333,1.000000}%
\pgfsetfillcolor{currentfill}%
\pgfsetlinewidth{0.481800pt}%
\definecolor{currentstroke}{rgb}{1.000000,1.000000,1.000000}%
\pgfsetstrokecolor{currentstroke}%
\pgfsetdash{}{0pt}%
\pgfpathmoveto{\pgfqpoint{6.777031in}{4.407810in}}%
\pgfpathcurveto{\pgfqpoint{6.788081in}{4.407810in}}{\pgfqpoint{6.798680in}{4.412200in}}{\pgfqpoint{6.806494in}{4.420014in}}%
\pgfpathcurveto{\pgfqpoint{6.814308in}{4.427828in}}{\pgfqpoint{6.818698in}{4.438427in}}{\pgfqpoint{6.818698in}{4.449477in}}%
\pgfpathcurveto{\pgfqpoint{6.818698in}{4.460527in}}{\pgfqpoint{6.814308in}{4.471126in}}{\pgfqpoint{6.806494in}{4.478939in}}%
\pgfpathcurveto{\pgfqpoint{6.798680in}{4.486753in}}{\pgfqpoint{6.788081in}{4.491143in}}{\pgfqpoint{6.777031in}{4.491143in}}%
\pgfpathcurveto{\pgfqpoint{6.765981in}{4.491143in}}{\pgfqpoint{6.755382in}{4.486753in}}{\pgfqpoint{6.747568in}{4.478939in}}%
\pgfpathcurveto{\pgfqpoint{6.739755in}{4.471126in}}{\pgfqpoint{6.735365in}{4.460527in}}{\pgfqpoint{6.735365in}{4.449477in}}%
\pgfpathcurveto{\pgfqpoint{6.735365in}{4.438427in}}{\pgfqpoint{6.739755in}{4.427828in}}{\pgfqpoint{6.747568in}{4.420014in}}%
\pgfpathcurveto{\pgfqpoint{6.755382in}{4.412200in}}{\pgfqpoint{6.765981in}{4.407810in}}{\pgfqpoint{6.777031in}{4.407810in}}%
\pgfpathclose%
\pgfusepath{stroke,fill}%
\end{pgfscope}%
\begin{pgfscope}%
\pgfpathrectangle{\pgfqpoint{0.481978in}{0.331635in}}{\pgfqpoint{9.300000in}{7.700000in}}%
\pgfusepath{clip}%
\pgfsetbuttcap%
\pgfsetroundjoin%
\definecolor{currentfill}{rgb}{0.815686,0.733333,1.000000}%
\pgfsetfillcolor{currentfill}%
\pgfsetlinewidth{0.481800pt}%
\definecolor{currentstroke}{rgb}{1.000000,1.000000,1.000000}%
\pgfsetstrokecolor{currentstroke}%
\pgfsetdash{}{0pt}%
\pgfpathmoveto{\pgfqpoint{3.798724in}{2.123955in}}%
\pgfpathcurveto{\pgfqpoint{3.809775in}{2.123955in}}{\pgfqpoint{3.820374in}{2.128345in}}{\pgfqpoint{3.828187in}{2.136158in}}%
\pgfpathcurveto{\pgfqpoint{3.836001in}{2.143972in}}{\pgfqpoint{3.840391in}{2.154571in}}{\pgfqpoint{3.840391in}{2.165621in}}%
\pgfpathcurveto{\pgfqpoint{3.840391in}{2.176671in}}{\pgfqpoint{3.836001in}{2.187270in}}{\pgfqpoint{3.828187in}{2.195084in}}%
\pgfpathcurveto{\pgfqpoint{3.820374in}{2.202898in}}{\pgfqpoint{3.809775in}{2.207288in}}{\pgfqpoint{3.798724in}{2.207288in}}%
\pgfpathcurveto{\pgfqpoint{3.787674in}{2.207288in}}{\pgfqpoint{3.777075in}{2.202898in}}{\pgfqpoint{3.769262in}{2.195084in}}%
\pgfpathcurveto{\pgfqpoint{3.761448in}{2.187270in}}{\pgfqpoint{3.757058in}{2.176671in}}{\pgfqpoint{3.757058in}{2.165621in}}%
\pgfpathcurveto{\pgfqpoint{3.757058in}{2.154571in}}{\pgfqpoint{3.761448in}{2.143972in}}{\pgfqpoint{3.769262in}{2.136158in}}%
\pgfpathcurveto{\pgfqpoint{3.777075in}{2.128345in}}{\pgfqpoint{3.787674in}{2.123955in}}{\pgfqpoint{3.798724in}{2.123955in}}%
\pgfpathclose%
\pgfusepath{stroke,fill}%
\end{pgfscope}%
\begin{pgfscope}%
\pgfpathrectangle{\pgfqpoint{0.481978in}{0.331635in}}{\pgfqpoint{9.300000in}{7.700000in}}%
\pgfusepath{clip}%
\pgfsetbuttcap%
\pgfsetroundjoin%
\definecolor{currentfill}{rgb}{0.815686,0.733333,1.000000}%
\pgfsetfillcolor{currentfill}%
\pgfsetlinewidth{0.481800pt}%
\definecolor{currentstroke}{rgb}{1.000000,1.000000,1.000000}%
\pgfsetstrokecolor{currentstroke}%
\pgfsetdash{}{0pt}%
\pgfpathmoveto{\pgfqpoint{1.738451in}{3.862058in}}%
\pgfpathcurveto{\pgfqpoint{1.749501in}{3.862058in}}{\pgfqpoint{1.760100in}{3.866448in}}{\pgfqpoint{1.767914in}{3.874262in}}%
\pgfpathcurveto{\pgfqpoint{1.775728in}{3.882075in}}{\pgfqpoint{1.780118in}{3.892674in}}{\pgfqpoint{1.780118in}{3.903724in}}%
\pgfpathcurveto{\pgfqpoint{1.780118in}{3.914774in}}{\pgfqpoint{1.775728in}{3.925373in}}{\pgfqpoint{1.767914in}{3.933187in}}%
\pgfpathcurveto{\pgfqpoint{1.760100in}{3.941001in}}{\pgfqpoint{1.749501in}{3.945391in}}{\pgfqpoint{1.738451in}{3.945391in}}%
\pgfpathcurveto{\pgfqpoint{1.727401in}{3.945391in}}{\pgfqpoint{1.716802in}{3.941001in}}{\pgfqpoint{1.708989in}{3.933187in}}%
\pgfpathcurveto{\pgfqpoint{1.701175in}{3.925373in}}{\pgfqpoint{1.696785in}{3.914774in}}{\pgfqpoint{1.696785in}{3.903724in}}%
\pgfpathcurveto{\pgfqpoint{1.696785in}{3.892674in}}{\pgfqpoint{1.701175in}{3.882075in}}{\pgfqpoint{1.708989in}{3.874262in}}%
\pgfpathcurveto{\pgfqpoint{1.716802in}{3.866448in}}{\pgfqpoint{1.727401in}{3.862058in}}{\pgfqpoint{1.738451in}{3.862058in}}%
\pgfpathclose%
\pgfusepath{stroke,fill}%
\end{pgfscope}%
\begin{pgfscope}%
\pgfpathrectangle{\pgfqpoint{0.481978in}{0.331635in}}{\pgfqpoint{9.300000in}{7.700000in}}%
\pgfusepath{clip}%
\pgfsetbuttcap%
\pgfsetroundjoin%
\definecolor{currentfill}{rgb}{0.815686,0.733333,1.000000}%
\pgfsetfillcolor{currentfill}%
\pgfsetlinewidth{0.481800pt}%
\definecolor{currentstroke}{rgb}{1.000000,1.000000,1.000000}%
\pgfsetstrokecolor{currentstroke}%
\pgfsetdash{}{0pt}%
\pgfpathmoveto{\pgfqpoint{8.508088in}{3.964215in}}%
\pgfpathcurveto{\pgfqpoint{8.519138in}{3.964215in}}{\pgfqpoint{8.529737in}{3.968605in}}{\pgfqpoint{8.537550in}{3.976419in}}%
\pgfpathcurveto{\pgfqpoint{8.545364in}{3.984232in}}{\pgfqpoint{8.549754in}{3.994831in}}{\pgfqpoint{8.549754in}{4.005881in}}%
\pgfpathcurveto{\pgfqpoint{8.549754in}{4.016932in}}{\pgfqpoint{8.545364in}{4.027531in}}{\pgfqpoint{8.537550in}{4.035344in}}%
\pgfpathcurveto{\pgfqpoint{8.529737in}{4.043158in}}{\pgfqpoint{8.519138in}{4.047548in}}{\pgfqpoint{8.508088in}{4.047548in}}%
\pgfpathcurveto{\pgfqpoint{8.497037in}{4.047548in}}{\pgfqpoint{8.486438in}{4.043158in}}{\pgfqpoint{8.478625in}{4.035344in}}%
\pgfpathcurveto{\pgfqpoint{8.470811in}{4.027531in}}{\pgfqpoint{8.466421in}{4.016932in}}{\pgfqpoint{8.466421in}{4.005881in}}%
\pgfpathcurveto{\pgfqpoint{8.466421in}{3.994831in}}{\pgfqpoint{8.470811in}{3.984232in}}{\pgfqpoint{8.478625in}{3.976419in}}%
\pgfpathcurveto{\pgfqpoint{8.486438in}{3.968605in}}{\pgfqpoint{8.497037in}{3.964215in}}{\pgfqpoint{8.508088in}{3.964215in}}%
\pgfpathclose%
\pgfusepath{stroke,fill}%
\end{pgfscope}%
\begin{pgfscope}%
\pgfpathrectangle{\pgfqpoint{0.481978in}{0.331635in}}{\pgfqpoint{9.300000in}{7.700000in}}%
\pgfusepath{clip}%
\pgfsetbuttcap%
\pgfsetroundjoin%
\definecolor{currentfill}{rgb}{0.815686,0.733333,1.000000}%
\pgfsetfillcolor{currentfill}%
\pgfsetlinewidth{0.481800pt}%
\definecolor{currentstroke}{rgb}{1.000000,1.000000,1.000000}%
\pgfsetstrokecolor{currentstroke}%
\pgfsetdash{}{0pt}%
\pgfpathmoveto{\pgfqpoint{6.837191in}{4.849315in}}%
\pgfpathcurveto{\pgfqpoint{6.848241in}{4.849315in}}{\pgfqpoint{6.858840in}{4.853705in}}{\pgfqpoint{6.866654in}{4.861519in}}%
\pgfpathcurveto{\pgfqpoint{6.874468in}{4.869333in}}{\pgfqpoint{6.878858in}{4.879932in}}{\pgfqpoint{6.878858in}{4.890982in}}%
\pgfpathcurveto{\pgfqpoint{6.878858in}{4.902032in}}{\pgfqpoint{6.874468in}{4.912631in}}{\pgfqpoint{6.866654in}{4.920444in}}%
\pgfpathcurveto{\pgfqpoint{6.858840in}{4.928258in}}{\pgfqpoint{6.848241in}{4.932648in}}{\pgfqpoint{6.837191in}{4.932648in}}%
\pgfpathcurveto{\pgfqpoint{6.826141in}{4.932648in}}{\pgfqpoint{6.815542in}{4.928258in}}{\pgfqpoint{6.807729in}{4.920444in}}%
\pgfpathcurveto{\pgfqpoint{6.799915in}{4.912631in}}{\pgfqpoint{6.795525in}{4.902032in}}{\pgfqpoint{6.795525in}{4.890982in}}%
\pgfpathcurveto{\pgfqpoint{6.795525in}{4.879932in}}{\pgfqpoint{6.799915in}{4.869333in}}{\pgfqpoint{6.807729in}{4.861519in}}%
\pgfpathcurveto{\pgfqpoint{6.815542in}{4.853705in}}{\pgfqpoint{6.826141in}{4.849315in}}{\pgfqpoint{6.837191in}{4.849315in}}%
\pgfpathclose%
\pgfusepath{stroke,fill}%
\end{pgfscope}%
\begin{pgfscope}%
\pgfpathrectangle{\pgfqpoint{0.481978in}{0.331635in}}{\pgfqpoint{9.300000in}{7.700000in}}%
\pgfusepath{clip}%
\pgfsetbuttcap%
\pgfsetroundjoin%
\definecolor{currentfill}{rgb}{0.815686,0.733333,1.000000}%
\pgfsetfillcolor{currentfill}%
\pgfsetlinewidth{0.481800pt}%
\definecolor{currentstroke}{rgb}{1.000000,1.000000,1.000000}%
\pgfsetstrokecolor{currentstroke}%
\pgfsetdash{}{0pt}%
\pgfpathmoveto{\pgfqpoint{4.101155in}{3.717957in}}%
\pgfpathcurveto{\pgfqpoint{4.112206in}{3.717957in}}{\pgfqpoint{4.122805in}{3.722347in}}{\pgfqpoint{4.130618in}{3.730161in}}%
\pgfpathcurveto{\pgfqpoint{4.138432in}{3.737975in}}{\pgfqpoint{4.142822in}{3.748574in}}{\pgfqpoint{4.142822in}{3.759624in}}%
\pgfpathcurveto{\pgfqpoint{4.142822in}{3.770674in}}{\pgfqpoint{4.138432in}{3.781273in}}{\pgfqpoint{4.130618in}{3.789086in}}%
\pgfpathcurveto{\pgfqpoint{4.122805in}{3.796900in}}{\pgfqpoint{4.112206in}{3.801290in}}{\pgfqpoint{4.101155in}{3.801290in}}%
\pgfpathcurveto{\pgfqpoint{4.090105in}{3.801290in}}{\pgfqpoint{4.079506in}{3.796900in}}{\pgfqpoint{4.071693in}{3.789086in}}%
\pgfpathcurveto{\pgfqpoint{4.063879in}{3.781273in}}{\pgfqpoint{4.059489in}{3.770674in}}{\pgfqpoint{4.059489in}{3.759624in}}%
\pgfpathcurveto{\pgfqpoint{4.059489in}{3.748574in}}{\pgfqpoint{4.063879in}{3.737975in}}{\pgfqpoint{4.071693in}{3.730161in}}%
\pgfpathcurveto{\pgfqpoint{4.079506in}{3.722347in}}{\pgfqpoint{4.090105in}{3.717957in}}{\pgfqpoint{4.101155in}{3.717957in}}%
\pgfpathclose%
\pgfusepath{stroke,fill}%
\end{pgfscope}%
\begin{pgfscope}%
\pgfpathrectangle{\pgfqpoint{0.481978in}{0.331635in}}{\pgfqpoint{9.300000in}{7.700000in}}%
\pgfusepath{clip}%
\pgfsetbuttcap%
\pgfsetroundjoin%
\definecolor{currentfill}{rgb}{0.815686,0.733333,1.000000}%
\pgfsetfillcolor{currentfill}%
\pgfsetlinewidth{0.481800pt}%
\definecolor{currentstroke}{rgb}{1.000000,1.000000,1.000000}%
\pgfsetstrokecolor{currentstroke}%
\pgfsetdash{}{0pt}%
\pgfpathmoveto{\pgfqpoint{7.340850in}{3.253024in}}%
\pgfpathcurveto{\pgfqpoint{7.351900in}{3.253024in}}{\pgfqpoint{7.362499in}{3.257414in}}{\pgfqpoint{7.370313in}{3.265227in}}%
\pgfpathcurveto{\pgfqpoint{7.378127in}{3.273041in}}{\pgfqpoint{7.382517in}{3.283640in}}{\pgfqpoint{7.382517in}{3.294690in}}%
\pgfpathcurveto{\pgfqpoint{7.382517in}{3.305740in}}{\pgfqpoint{7.378127in}{3.316339in}}{\pgfqpoint{7.370313in}{3.324153in}}%
\pgfpathcurveto{\pgfqpoint{7.362499in}{3.331967in}}{\pgfqpoint{7.351900in}{3.336357in}}{\pgfqpoint{7.340850in}{3.336357in}}%
\pgfpathcurveto{\pgfqpoint{7.329800in}{3.336357in}}{\pgfqpoint{7.319201in}{3.331967in}}{\pgfqpoint{7.311387in}{3.324153in}}%
\pgfpathcurveto{\pgfqpoint{7.303574in}{3.316339in}}{\pgfqpoint{7.299184in}{3.305740in}}{\pgfqpoint{7.299184in}{3.294690in}}%
\pgfpathcurveto{\pgfqpoint{7.299184in}{3.283640in}}{\pgfqpoint{7.303574in}{3.273041in}}{\pgfqpoint{7.311387in}{3.265227in}}%
\pgfpathcurveto{\pgfqpoint{7.319201in}{3.257414in}}{\pgfqpoint{7.329800in}{3.253024in}}{\pgfqpoint{7.340850in}{3.253024in}}%
\pgfpathclose%
\pgfusepath{stroke,fill}%
\end{pgfscope}%
\begin{pgfscope}%
\pgfpathrectangle{\pgfqpoint{0.481978in}{0.331635in}}{\pgfqpoint{9.300000in}{7.700000in}}%
\pgfusepath{clip}%
\pgfsetbuttcap%
\pgfsetroundjoin%
\definecolor{currentfill}{rgb}{0.815686,0.733333,1.000000}%
\pgfsetfillcolor{currentfill}%
\pgfsetlinewidth{0.481800pt}%
\definecolor{currentstroke}{rgb}{1.000000,1.000000,1.000000}%
\pgfsetstrokecolor{currentstroke}%
\pgfsetdash{}{0pt}%
\pgfpathmoveto{\pgfqpoint{4.878178in}{5.062333in}}%
\pgfpathcurveto{\pgfqpoint{4.889228in}{5.062333in}}{\pgfqpoint{4.899827in}{5.066724in}}{\pgfqpoint{4.907641in}{5.074537in}}%
\pgfpathcurveto{\pgfqpoint{4.915454in}{5.082351in}}{\pgfqpoint{4.919845in}{5.092950in}}{\pgfqpoint{4.919845in}{5.104000in}}%
\pgfpathcurveto{\pgfqpoint{4.919845in}{5.115050in}}{\pgfqpoint{4.915454in}{5.125649in}}{\pgfqpoint{4.907641in}{5.133463in}}%
\pgfpathcurveto{\pgfqpoint{4.899827in}{5.141277in}}{\pgfqpoint{4.889228in}{5.145667in}}{\pgfqpoint{4.878178in}{5.145667in}}%
\pgfpathcurveto{\pgfqpoint{4.867128in}{5.145667in}}{\pgfqpoint{4.856529in}{5.141277in}}{\pgfqpoint{4.848715in}{5.133463in}}%
\pgfpathcurveto{\pgfqpoint{4.840902in}{5.125649in}}{\pgfqpoint{4.836511in}{5.115050in}}{\pgfqpoint{4.836511in}{5.104000in}}%
\pgfpathcurveto{\pgfqpoint{4.836511in}{5.092950in}}{\pgfqpoint{4.840902in}{5.082351in}}{\pgfqpoint{4.848715in}{5.074537in}}%
\pgfpathcurveto{\pgfqpoint{4.856529in}{5.066724in}}{\pgfqpoint{4.867128in}{5.062333in}}{\pgfqpoint{4.878178in}{5.062333in}}%
\pgfpathclose%
\pgfusepath{stroke,fill}%
\end{pgfscope}%
\begin{pgfscope}%
\pgfpathrectangle{\pgfqpoint{0.481978in}{0.331635in}}{\pgfqpoint{9.300000in}{7.700000in}}%
\pgfusepath{clip}%
\pgfsetbuttcap%
\pgfsetroundjoin%
\definecolor{currentfill}{rgb}{0.815686,0.733333,1.000000}%
\pgfsetfillcolor{currentfill}%
\pgfsetlinewidth{0.481800pt}%
\definecolor{currentstroke}{rgb}{1.000000,1.000000,1.000000}%
\pgfsetstrokecolor{currentstroke}%
\pgfsetdash{}{0pt}%
\pgfpathmoveto{\pgfqpoint{3.093874in}{4.847172in}}%
\pgfpathcurveto{\pgfqpoint{3.104924in}{4.847172in}}{\pgfqpoint{3.115523in}{4.851563in}}{\pgfqpoint{3.123337in}{4.859376in}}%
\pgfpathcurveto{\pgfqpoint{3.131150in}{4.867190in}}{\pgfqpoint{3.135541in}{4.877789in}}{\pgfqpoint{3.135541in}{4.888839in}}%
\pgfpathcurveto{\pgfqpoint{3.135541in}{4.899889in}}{\pgfqpoint{3.131150in}{4.910488in}}{\pgfqpoint{3.123337in}{4.918302in}}%
\pgfpathcurveto{\pgfqpoint{3.115523in}{4.926115in}}{\pgfqpoint{3.104924in}{4.930506in}}{\pgfqpoint{3.093874in}{4.930506in}}%
\pgfpathcurveto{\pgfqpoint{3.082824in}{4.930506in}}{\pgfqpoint{3.072225in}{4.926115in}}{\pgfqpoint{3.064411in}{4.918302in}}%
\pgfpathcurveto{\pgfqpoint{3.056598in}{4.910488in}}{\pgfqpoint{3.052207in}{4.899889in}}{\pgfqpoint{3.052207in}{4.888839in}}%
\pgfpathcurveto{\pgfqpoint{3.052207in}{4.877789in}}{\pgfqpoint{3.056598in}{4.867190in}}{\pgfqpoint{3.064411in}{4.859376in}}%
\pgfpathcurveto{\pgfqpoint{3.072225in}{4.851563in}}{\pgfqpoint{3.082824in}{4.847172in}}{\pgfqpoint{3.093874in}{4.847172in}}%
\pgfpathclose%
\pgfusepath{stroke,fill}%
\end{pgfscope}%
\begin{pgfscope}%
\pgfpathrectangle{\pgfqpoint{0.481978in}{0.331635in}}{\pgfqpoint{9.300000in}{7.700000in}}%
\pgfusepath{clip}%
\pgfsetbuttcap%
\pgfsetroundjoin%
\definecolor{currentfill}{rgb}{0.815686,0.733333,1.000000}%
\pgfsetfillcolor{currentfill}%
\pgfsetlinewidth{0.481800pt}%
\definecolor{currentstroke}{rgb}{1.000000,1.000000,1.000000}%
\pgfsetstrokecolor{currentstroke}%
\pgfsetdash{}{0pt}%
\pgfpathmoveto{\pgfqpoint{5.273114in}{1.865984in}}%
\pgfpathcurveto{\pgfqpoint{5.284164in}{1.865984in}}{\pgfqpoint{5.294763in}{1.870375in}}{\pgfqpoint{5.302577in}{1.878188in}}%
\pgfpathcurveto{\pgfqpoint{5.310390in}{1.886002in}}{\pgfqpoint{5.314781in}{1.896601in}}{\pgfqpoint{5.314781in}{1.907651in}}%
\pgfpathcurveto{\pgfqpoint{5.314781in}{1.918701in}}{\pgfqpoint{5.310390in}{1.929300in}}{\pgfqpoint{5.302577in}{1.937114in}}%
\pgfpathcurveto{\pgfqpoint{5.294763in}{1.944927in}}{\pgfqpoint{5.284164in}{1.949318in}}{\pgfqpoint{5.273114in}{1.949318in}}%
\pgfpathcurveto{\pgfqpoint{5.262064in}{1.949318in}}{\pgfqpoint{5.251465in}{1.944927in}}{\pgfqpoint{5.243651in}{1.937114in}}%
\pgfpathcurveto{\pgfqpoint{5.235838in}{1.929300in}}{\pgfqpoint{5.231447in}{1.918701in}}{\pgfqpoint{5.231447in}{1.907651in}}%
\pgfpathcurveto{\pgfqpoint{5.231447in}{1.896601in}}{\pgfqpoint{5.235838in}{1.886002in}}{\pgfqpoint{5.243651in}{1.878188in}}%
\pgfpathcurveto{\pgfqpoint{5.251465in}{1.870375in}}{\pgfqpoint{5.262064in}{1.865984in}}{\pgfqpoint{5.273114in}{1.865984in}}%
\pgfpathclose%
\pgfusepath{stroke,fill}%
\end{pgfscope}%
\begin{pgfscope}%
\pgfpathrectangle{\pgfqpoint{0.481978in}{0.331635in}}{\pgfqpoint{9.300000in}{7.700000in}}%
\pgfusepath{clip}%
\pgfsetbuttcap%
\pgfsetroundjoin%
\definecolor{currentfill}{rgb}{0.815686,0.733333,1.000000}%
\pgfsetfillcolor{currentfill}%
\pgfsetlinewidth{0.481800pt}%
\definecolor{currentstroke}{rgb}{1.000000,1.000000,1.000000}%
\pgfsetstrokecolor{currentstroke}%
\pgfsetdash{}{0pt}%
\pgfpathmoveto{\pgfqpoint{8.330701in}{5.607165in}}%
\pgfpathcurveto{\pgfqpoint{8.341751in}{5.607165in}}{\pgfqpoint{8.352350in}{5.611556in}}{\pgfqpoint{8.360163in}{5.619369in}}%
\pgfpathcurveto{\pgfqpoint{8.367977in}{5.627183in}}{\pgfqpoint{8.372367in}{5.637782in}}{\pgfqpoint{8.372367in}{5.648832in}}%
\pgfpathcurveto{\pgfqpoint{8.372367in}{5.659882in}}{\pgfqpoint{8.367977in}{5.670481in}}{\pgfqpoint{8.360163in}{5.678295in}}%
\pgfpathcurveto{\pgfqpoint{8.352350in}{5.686109in}}{\pgfqpoint{8.341751in}{5.690499in}}{\pgfqpoint{8.330701in}{5.690499in}}%
\pgfpathcurveto{\pgfqpoint{8.319651in}{5.690499in}}{\pgfqpoint{8.309052in}{5.686109in}}{\pgfqpoint{8.301238in}{5.678295in}}%
\pgfpathcurveto{\pgfqpoint{8.293424in}{5.670481in}}{\pgfqpoint{8.289034in}{5.659882in}}{\pgfqpoint{8.289034in}{5.648832in}}%
\pgfpathcurveto{\pgfqpoint{8.289034in}{5.637782in}}{\pgfqpoint{8.293424in}{5.627183in}}{\pgfqpoint{8.301238in}{5.619369in}}%
\pgfpathcurveto{\pgfqpoint{8.309052in}{5.611556in}}{\pgfqpoint{8.319651in}{5.607165in}}{\pgfqpoint{8.330701in}{5.607165in}}%
\pgfpathclose%
\pgfusepath{stroke,fill}%
\end{pgfscope}%
\begin{pgfscope}%
\pgfpathrectangle{\pgfqpoint{0.481978in}{0.331635in}}{\pgfqpoint{9.300000in}{7.700000in}}%
\pgfusepath{clip}%
\pgfsetbuttcap%
\pgfsetroundjoin%
\definecolor{currentfill}{rgb}{0.815686,0.733333,1.000000}%
\pgfsetfillcolor{currentfill}%
\pgfsetlinewidth{0.481800pt}%
\definecolor{currentstroke}{rgb}{1.000000,1.000000,1.000000}%
\pgfsetstrokecolor{currentstroke}%
\pgfsetdash{}{0pt}%
\pgfpathmoveto{\pgfqpoint{7.364685in}{2.898601in}}%
\pgfpathcurveto{\pgfqpoint{7.375735in}{2.898601in}}{\pgfqpoint{7.386334in}{2.902991in}}{\pgfqpoint{7.394148in}{2.910805in}}%
\pgfpathcurveto{\pgfqpoint{7.401961in}{2.918618in}}{\pgfqpoint{7.406351in}{2.929217in}}{\pgfqpoint{7.406351in}{2.940268in}}%
\pgfpathcurveto{\pgfqpoint{7.406351in}{2.951318in}}{\pgfqpoint{7.401961in}{2.961917in}}{\pgfqpoint{7.394148in}{2.969730in}}%
\pgfpathcurveto{\pgfqpoint{7.386334in}{2.977544in}}{\pgfqpoint{7.375735in}{2.981934in}}{\pgfqpoint{7.364685in}{2.981934in}}%
\pgfpathcurveto{\pgfqpoint{7.353635in}{2.981934in}}{\pgfqpoint{7.343036in}{2.977544in}}{\pgfqpoint{7.335222in}{2.969730in}}%
\pgfpathcurveto{\pgfqpoint{7.327408in}{2.961917in}}{\pgfqpoint{7.323018in}{2.951318in}}{\pgfqpoint{7.323018in}{2.940268in}}%
\pgfpathcurveto{\pgfqpoint{7.323018in}{2.929217in}}{\pgfqpoint{7.327408in}{2.918618in}}{\pgfqpoint{7.335222in}{2.910805in}}%
\pgfpathcurveto{\pgfqpoint{7.343036in}{2.902991in}}{\pgfqpoint{7.353635in}{2.898601in}}{\pgfqpoint{7.364685in}{2.898601in}}%
\pgfpathclose%
\pgfusepath{stroke,fill}%
\end{pgfscope}%
\begin{pgfscope}%
\pgfpathrectangle{\pgfqpoint{0.481978in}{0.331635in}}{\pgfqpoint{9.300000in}{7.700000in}}%
\pgfusepath{clip}%
\pgfsetbuttcap%
\pgfsetroundjoin%
\definecolor{currentfill}{rgb}{0.815686,0.733333,1.000000}%
\pgfsetfillcolor{currentfill}%
\pgfsetlinewidth{0.481800pt}%
\definecolor{currentstroke}{rgb}{1.000000,1.000000,1.000000}%
\pgfsetstrokecolor{currentstroke}%
\pgfsetdash{}{0pt}%
\pgfpathmoveto{\pgfqpoint{2.332471in}{2.474830in}}%
\pgfpathcurveto{\pgfqpoint{2.343521in}{2.474830in}}{\pgfqpoint{2.354120in}{2.479221in}}{\pgfqpoint{2.361934in}{2.487034in}}%
\pgfpathcurveto{\pgfqpoint{2.369747in}{2.494848in}}{\pgfqpoint{2.374138in}{2.505447in}}{\pgfqpoint{2.374138in}{2.516497in}}%
\pgfpathcurveto{\pgfqpoint{2.374138in}{2.527547in}}{\pgfqpoint{2.369747in}{2.538146in}}{\pgfqpoint{2.361934in}{2.545960in}}%
\pgfpathcurveto{\pgfqpoint{2.354120in}{2.553773in}}{\pgfqpoint{2.343521in}{2.558164in}}{\pgfqpoint{2.332471in}{2.558164in}}%
\pgfpathcurveto{\pgfqpoint{2.321421in}{2.558164in}}{\pgfqpoint{2.310822in}{2.553773in}}{\pgfqpoint{2.303008in}{2.545960in}}%
\pgfpathcurveto{\pgfqpoint{2.295195in}{2.538146in}}{\pgfqpoint{2.290804in}{2.527547in}}{\pgfqpoint{2.290804in}{2.516497in}}%
\pgfpathcurveto{\pgfqpoint{2.290804in}{2.505447in}}{\pgfqpoint{2.295195in}{2.494848in}}{\pgfqpoint{2.303008in}{2.487034in}}%
\pgfpathcurveto{\pgfqpoint{2.310822in}{2.479221in}}{\pgfqpoint{2.321421in}{2.474830in}}{\pgfqpoint{2.332471in}{2.474830in}}%
\pgfpathclose%
\pgfusepath{stroke,fill}%
\end{pgfscope}%
\begin{pgfscope}%
\pgfpathrectangle{\pgfqpoint{0.481978in}{0.331635in}}{\pgfqpoint{9.300000in}{7.700000in}}%
\pgfusepath{clip}%
\pgfsetbuttcap%
\pgfsetroundjoin%
\definecolor{currentfill}{rgb}{0.815686,0.733333,1.000000}%
\pgfsetfillcolor{currentfill}%
\pgfsetlinewidth{0.481800pt}%
\definecolor{currentstroke}{rgb}{1.000000,1.000000,1.000000}%
\pgfsetstrokecolor{currentstroke}%
\pgfsetdash{}{0pt}%
\pgfpathmoveto{\pgfqpoint{6.392422in}{4.189674in}}%
\pgfpathcurveto{\pgfqpoint{6.403472in}{4.189674in}}{\pgfqpoint{6.414071in}{4.194065in}}{\pgfqpoint{6.421884in}{4.201878in}}%
\pgfpathcurveto{\pgfqpoint{6.429698in}{4.209692in}}{\pgfqpoint{6.434088in}{4.220291in}}{\pgfqpoint{6.434088in}{4.231341in}}%
\pgfpathcurveto{\pgfqpoint{6.434088in}{4.242391in}}{\pgfqpoint{6.429698in}{4.252990in}}{\pgfqpoint{6.421884in}{4.260804in}}%
\pgfpathcurveto{\pgfqpoint{6.414071in}{4.268617in}}{\pgfqpoint{6.403472in}{4.273008in}}{\pgfqpoint{6.392422in}{4.273008in}}%
\pgfpathcurveto{\pgfqpoint{6.381371in}{4.273008in}}{\pgfqpoint{6.370772in}{4.268617in}}{\pgfqpoint{6.362959in}{4.260804in}}%
\pgfpathcurveto{\pgfqpoint{6.355145in}{4.252990in}}{\pgfqpoint{6.350755in}{4.242391in}}{\pgfqpoint{6.350755in}{4.231341in}}%
\pgfpathcurveto{\pgfqpoint{6.350755in}{4.220291in}}{\pgfqpoint{6.355145in}{4.209692in}}{\pgfqpoint{6.362959in}{4.201878in}}%
\pgfpathcurveto{\pgfqpoint{6.370772in}{4.194065in}}{\pgfqpoint{6.381371in}{4.189674in}}{\pgfqpoint{6.392422in}{4.189674in}}%
\pgfpathclose%
\pgfusepath{stroke,fill}%
\end{pgfscope}%
\begin{pgfscope}%
\pgfpathrectangle{\pgfqpoint{0.481978in}{0.331635in}}{\pgfqpoint{9.300000in}{7.700000in}}%
\pgfusepath{clip}%
\pgfsetbuttcap%
\pgfsetroundjoin%
\definecolor{currentfill}{rgb}{0.870588,0.733333,0.607843}%
\pgfsetfillcolor{currentfill}%
\pgfsetlinewidth{0.481800pt}%
\definecolor{currentstroke}{rgb}{1.000000,1.000000,1.000000}%
\pgfsetstrokecolor{currentstroke}%
\pgfsetdash{}{0pt}%
\pgfpathmoveto{\pgfqpoint{3.278906in}{4.335174in}}%
\pgfpathcurveto{\pgfqpoint{3.289956in}{4.335174in}}{\pgfqpoint{3.300555in}{4.339564in}}{\pgfqpoint{3.308368in}{4.347377in}}%
\pgfpathcurveto{\pgfqpoint{3.316182in}{4.355191in}}{\pgfqpoint{3.320572in}{4.365790in}}{\pgfqpoint{3.320572in}{4.376840in}}%
\pgfpathcurveto{\pgfqpoint{3.320572in}{4.387890in}}{\pgfqpoint{3.316182in}{4.398489in}}{\pgfqpoint{3.308368in}{4.406303in}}%
\pgfpathcurveto{\pgfqpoint{3.300555in}{4.414117in}}{\pgfqpoint{3.289956in}{4.418507in}}{\pgfqpoint{3.278906in}{4.418507in}}%
\pgfpathcurveto{\pgfqpoint{3.267856in}{4.418507in}}{\pgfqpoint{3.257257in}{4.414117in}}{\pgfqpoint{3.249443in}{4.406303in}}%
\pgfpathcurveto{\pgfqpoint{3.241629in}{4.398489in}}{\pgfqpoint{3.237239in}{4.387890in}}{\pgfqpoint{3.237239in}{4.376840in}}%
\pgfpathcurveto{\pgfqpoint{3.237239in}{4.365790in}}{\pgfqpoint{3.241629in}{4.355191in}}{\pgfqpoint{3.249443in}{4.347377in}}%
\pgfpathcurveto{\pgfqpoint{3.257257in}{4.339564in}}{\pgfqpoint{3.267856in}{4.335174in}}{\pgfqpoint{3.278906in}{4.335174in}}%
\pgfpathclose%
\pgfusepath{stroke,fill}%
\end{pgfscope}%
\begin{pgfscope}%
\pgfpathrectangle{\pgfqpoint{0.481978in}{0.331635in}}{\pgfqpoint{9.300000in}{7.700000in}}%
\pgfusepath{clip}%
\pgfsetbuttcap%
\pgfsetroundjoin%
\definecolor{currentfill}{rgb}{0.870588,0.733333,0.607843}%
\pgfsetfillcolor{currentfill}%
\pgfsetlinewidth{0.481800pt}%
\definecolor{currentstroke}{rgb}{1.000000,1.000000,1.000000}%
\pgfsetstrokecolor{currentstroke}%
\pgfsetdash{}{0pt}%
\pgfpathmoveto{\pgfqpoint{1.806406in}{3.472401in}}%
\pgfpathcurveto{\pgfqpoint{1.817456in}{3.472401in}}{\pgfqpoint{1.828055in}{3.476791in}}{\pgfqpoint{1.835868in}{3.484605in}}%
\pgfpathcurveto{\pgfqpoint{1.843682in}{3.492418in}}{\pgfqpoint{1.848072in}{3.503017in}}{\pgfqpoint{1.848072in}{3.514068in}}%
\pgfpathcurveto{\pgfqpoint{1.848072in}{3.525118in}}{\pgfqpoint{1.843682in}{3.535717in}}{\pgfqpoint{1.835868in}{3.543530in}}%
\pgfpathcurveto{\pgfqpoint{1.828055in}{3.551344in}}{\pgfqpoint{1.817456in}{3.555734in}}{\pgfqpoint{1.806406in}{3.555734in}}%
\pgfpathcurveto{\pgfqpoint{1.795355in}{3.555734in}}{\pgfqpoint{1.784756in}{3.551344in}}{\pgfqpoint{1.776943in}{3.543530in}}%
\pgfpathcurveto{\pgfqpoint{1.769129in}{3.535717in}}{\pgfqpoint{1.764739in}{3.525118in}}{\pgfqpoint{1.764739in}{3.514068in}}%
\pgfpathcurveto{\pgfqpoint{1.764739in}{3.503017in}}{\pgfqpoint{1.769129in}{3.492418in}}{\pgfqpoint{1.776943in}{3.484605in}}%
\pgfpathcurveto{\pgfqpoint{1.784756in}{3.476791in}}{\pgfqpoint{1.795355in}{3.472401in}}{\pgfqpoint{1.806406in}{3.472401in}}%
\pgfpathclose%
\pgfusepath{stroke,fill}%
\end{pgfscope}%
\begin{pgfscope}%
\pgfpathrectangle{\pgfqpoint{0.481978in}{0.331635in}}{\pgfqpoint{9.300000in}{7.700000in}}%
\pgfusepath{clip}%
\pgfsetbuttcap%
\pgfsetroundjoin%
\definecolor{currentfill}{rgb}{0.870588,0.733333,0.607843}%
\pgfsetfillcolor{currentfill}%
\pgfsetlinewidth{0.481800pt}%
\definecolor{currentstroke}{rgb}{1.000000,1.000000,1.000000}%
\pgfsetstrokecolor{currentstroke}%
\pgfsetdash{}{0pt}%
\pgfpathmoveto{\pgfqpoint{3.901522in}{2.710673in}}%
\pgfpathcurveto{\pgfqpoint{3.912572in}{2.710673in}}{\pgfqpoint{3.923171in}{2.715064in}}{\pgfqpoint{3.930984in}{2.722877in}}%
\pgfpathcurveto{\pgfqpoint{3.938798in}{2.730691in}}{\pgfqpoint{3.943188in}{2.741290in}}{\pgfqpoint{3.943188in}{2.752340in}}%
\pgfpathcurveto{\pgfqpoint{3.943188in}{2.763390in}}{\pgfqpoint{3.938798in}{2.773989in}}{\pgfqpoint{3.930984in}{2.781803in}}%
\pgfpathcurveto{\pgfqpoint{3.923171in}{2.789616in}}{\pgfqpoint{3.912572in}{2.794007in}}{\pgfqpoint{3.901522in}{2.794007in}}%
\pgfpathcurveto{\pgfqpoint{3.890472in}{2.794007in}}{\pgfqpoint{3.879873in}{2.789616in}}{\pgfqpoint{3.872059in}{2.781803in}}%
\pgfpathcurveto{\pgfqpoint{3.864245in}{2.773989in}}{\pgfqpoint{3.859855in}{2.763390in}}{\pgfqpoint{3.859855in}{2.752340in}}%
\pgfpathcurveto{\pgfqpoint{3.859855in}{2.741290in}}{\pgfqpoint{3.864245in}{2.730691in}}{\pgfqpoint{3.872059in}{2.722877in}}%
\pgfpathcurveto{\pgfqpoint{3.879873in}{2.715064in}}{\pgfqpoint{3.890472in}{2.710673in}}{\pgfqpoint{3.901522in}{2.710673in}}%
\pgfpathclose%
\pgfusepath{stroke,fill}%
\end{pgfscope}%
\begin{pgfscope}%
\pgfpathrectangle{\pgfqpoint{0.481978in}{0.331635in}}{\pgfqpoint{9.300000in}{7.700000in}}%
\pgfusepath{clip}%
\pgfsetbuttcap%
\pgfsetroundjoin%
\definecolor{currentfill}{rgb}{0.870588,0.733333,0.607843}%
\pgfsetfillcolor{currentfill}%
\pgfsetlinewidth{0.481800pt}%
\definecolor{currentstroke}{rgb}{1.000000,1.000000,1.000000}%
\pgfsetstrokecolor{currentstroke}%
\pgfsetdash{}{0pt}%
\pgfpathmoveto{\pgfqpoint{2.577647in}{1.919824in}}%
\pgfpathcurveto{\pgfqpoint{2.588697in}{1.919824in}}{\pgfqpoint{2.599296in}{1.924214in}}{\pgfqpoint{2.607110in}{1.932027in}}%
\pgfpathcurveto{\pgfqpoint{2.614923in}{1.939841in}}{\pgfqpoint{2.619314in}{1.950440in}}{\pgfqpoint{2.619314in}{1.961490in}}%
\pgfpathcurveto{\pgfqpoint{2.619314in}{1.972540in}}{\pgfqpoint{2.614923in}{1.983139in}}{\pgfqpoint{2.607110in}{1.990953in}}%
\pgfpathcurveto{\pgfqpoint{2.599296in}{1.998767in}}{\pgfqpoint{2.588697in}{2.003157in}}{\pgfqpoint{2.577647in}{2.003157in}}%
\pgfpathcurveto{\pgfqpoint{2.566597in}{2.003157in}}{\pgfqpoint{2.555998in}{1.998767in}}{\pgfqpoint{2.548184in}{1.990953in}}%
\pgfpathcurveto{\pgfqpoint{2.540371in}{1.983139in}}{\pgfqpoint{2.535980in}{1.972540in}}{\pgfqpoint{2.535980in}{1.961490in}}%
\pgfpathcurveto{\pgfqpoint{2.535980in}{1.950440in}}{\pgfqpoint{2.540371in}{1.939841in}}{\pgfqpoint{2.548184in}{1.932027in}}%
\pgfpathcurveto{\pgfqpoint{2.555998in}{1.924214in}}{\pgfqpoint{2.566597in}{1.919824in}}{\pgfqpoint{2.577647in}{1.919824in}}%
\pgfpathclose%
\pgfusepath{stroke,fill}%
\end{pgfscope}%
\begin{pgfscope}%
\pgfpathrectangle{\pgfqpoint{0.481978in}{0.331635in}}{\pgfqpoint{9.300000in}{7.700000in}}%
\pgfusepath{clip}%
\pgfsetbuttcap%
\pgfsetroundjoin%
\definecolor{currentfill}{rgb}{0.870588,0.733333,0.607843}%
\pgfsetfillcolor{currentfill}%
\pgfsetlinewidth{0.481800pt}%
\definecolor{currentstroke}{rgb}{1.000000,1.000000,1.000000}%
\pgfsetstrokecolor{currentstroke}%
\pgfsetdash{}{0pt}%
\pgfpathmoveto{\pgfqpoint{2.542882in}{4.387959in}}%
\pgfpathcurveto{\pgfqpoint{2.553932in}{4.387959in}}{\pgfqpoint{2.564531in}{4.392349in}}{\pgfqpoint{2.572344in}{4.400162in}}%
\pgfpathcurveto{\pgfqpoint{2.580158in}{4.407976in}}{\pgfqpoint{2.584548in}{4.418575in}}{\pgfqpoint{2.584548in}{4.429625in}}%
\pgfpathcurveto{\pgfqpoint{2.584548in}{4.440675in}}{\pgfqpoint{2.580158in}{4.451274in}}{\pgfqpoint{2.572344in}{4.459088in}}%
\pgfpathcurveto{\pgfqpoint{2.564531in}{4.466902in}}{\pgfqpoint{2.553932in}{4.471292in}}{\pgfqpoint{2.542882in}{4.471292in}}%
\pgfpathcurveto{\pgfqpoint{2.531831in}{4.471292in}}{\pgfqpoint{2.521232in}{4.466902in}}{\pgfqpoint{2.513419in}{4.459088in}}%
\pgfpathcurveto{\pgfqpoint{2.505605in}{4.451274in}}{\pgfqpoint{2.501215in}{4.440675in}}{\pgfqpoint{2.501215in}{4.429625in}}%
\pgfpathcurveto{\pgfqpoint{2.501215in}{4.418575in}}{\pgfqpoint{2.505605in}{4.407976in}}{\pgfqpoint{2.513419in}{4.400162in}}%
\pgfpathcurveto{\pgfqpoint{2.521232in}{4.392349in}}{\pgfqpoint{2.531831in}{4.387959in}}{\pgfqpoint{2.542882in}{4.387959in}}%
\pgfpathclose%
\pgfusepath{stroke,fill}%
\end{pgfscope}%
\begin{pgfscope}%
\pgfpathrectangle{\pgfqpoint{0.481978in}{0.331635in}}{\pgfqpoint{9.300000in}{7.700000in}}%
\pgfusepath{clip}%
\pgfsetbuttcap%
\pgfsetroundjoin%
\definecolor{currentfill}{rgb}{0.870588,0.733333,0.607843}%
\pgfsetfillcolor{currentfill}%
\pgfsetlinewidth{0.481800pt}%
\definecolor{currentstroke}{rgb}{1.000000,1.000000,1.000000}%
\pgfsetstrokecolor{currentstroke}%
\pgfsetdash{}{0pt}%
\pgfpathmoveto{\pgfqpoint{6.100140in}{5.036268in}}%
\pgfpathcurveto{\pgfqpoint{6.111190in}{5.036268in}}{\pgfqpoint{6.121789in}{5.040658in}}{\pgfqpoint{6.129603in}{5.048472in}}%
\pgfpathcurveto{\pgfqpoint{6.137416in}{5.056285in}}{\pgfqpoint{6.141806in}{5.066884in}}{\pgfqpoint{6.141806in}{5.077934in}}%
\pgfpathcurveto{\pgfqpoint{6.141806in}{5.088984in}}{\pgfqpoint{6.137416in}{5.099583in}}{\pgfqpoint{6.129603in}{5.107397in}}%
\pgfpathcurveto{\pgfqpoint{6.121789in}{5.115211in}}{\pgfqpoint{6.111190in}{5.119601in}}{\pgfqpoint{6.100140in}{5.119601in}}%
\pgfpathcurveto{\pgfqpoint{6.089090in}{5.119601in}}{\pgfqpoint{6.078491in}{5.115211in}}{\pgfqpoint{6.070677in}{5.107397in}}%
\pgfpathcurveto{\pgfqpoint{6.062863in}{5.099583in}}{\pgfqpoint{6.058473in}{5.088984in}}{\pgfqpoint{6.058473in}{5.077934in}}%
\pgfpathcurveto{\pgfqpoint{6.058473in}{5.066884in}}{\pgfqpoint{6.062863in}{5.056285in}}{\pgfqpoint{6.070677in}{5.048472in}}%
\pgfpathcurveto{\pgfqpoint{6.078491in}{5.040658in}}{\pgfqpoint{6.089090in}{5.036268in}}{\pgfqpoint{6.100140in}{5.036268in}}%
\pgfpathclose%
\pgfusepath{stroke,fill}%
\end{pgfscope}%
\begin{pgfscope}%
\pgfpathrectangle{\pgfqpoint{0.481978in}{0.331635in}}{\pgfqpoint{9.300000in}{7.700000in}}%
\pgfusepath{clip}%
\pgfsetbuttcap%
\pgfsetroundjoin%
\definecolor{currentfill}{rgb}{0.870588,0.733333,0.607843}%
\pgfsetfillcolor{currentfill}%
\pgfsetlinewidth{0.481800pt}%
\definecolor{currentstroke}{rgb}{1.000000,1.000000,1.000000}%
\pgfsetstrokecolor{currentstroke}%
\pgfsetdash{}{0pt}%
\pgfpathmoveto{\pgfqpoint{4.111074in}{4.530151in}}%
\pgfpathcurveto{\pgfqpoint{4.122124in}{4.530151in}}{\pgfqpoint{4.132723in}{4.534541in}}{\pgfqpoint{4.140537in}{4.542355in}}%
\pgfpathcurveto{\pgfqpoint{4.148351in}{4.550168in}}{\pgfqpoint{4.152741in}{4.560767in}}{\pgfqpoint{4.152741in}{4.571818in}}%
\pgfpathcurveto{\pgfqpoint{4.152741in}{4.582868in}}{\pgfqpoint{4.148351in}{4.593467in}}{\pgfqpoint{4.140537in}{4.601280in}}%
\pgfpathcurveto{\pgfqpoint{4.132723in}{4.609094in}}{\pgfqpoint{4.122124in}{4.613484in}}{\pgfqpoint{4.111074in}{4.613484in}}%
\pgfpathcurveto{\pgfqpoint{4.100024in}{4.613484in}}{\pgfqpoint{4.089425in}{4.609094in}}{\pgfqpoint{4.081611in}{4.601280in}}%
\pgfpathcurveto{\pgfqpoint{4.073798in}{4.593467in}}{\pgfqpoint{4.069408in}{4.582868in}}{\pgfqpoint{4.069408in}{4.571818in}}%
\pgfpathcurveto{\pgfqpoint{4.069408in}{4.560767in}}{\pgfqpoint{4.073798in}{4.550168in}}{\pgfqpoint{4.081611in}{4.542355in}}%
\pgfpathcurveto{\pgfqpoint{4.089425in}{4.534541in}}{\pgfqpoint{4.100024in}{4.530151in}}{\pgfqpoint{4.111074in}{4.530151in}}%
\pgfpathclose%
\pgfusepath{stroke,fill}%
\end{pgfscope}%
\begin{pgfscope}%
\pgfpathrectangle{\pgfqpoint{0.481978in}{0.331635in}}{\pgfqpoint{9.300000in}{7.700000in}}%
\pgfusepath{clip}%
\pgfsetbuttcap%
\pgfsetroundjoin%
\definecolor{currentfill}{rgb}{0.870588,0.733333,0.607843}%
\pgfsetfillcolor{currentfill}%
\pgfsetlinewidth{0.481800pt}%
\definecolor{currentstroke}{rgb}{1.000000,1.000000,1.000000}%
\pgfsetstrokecolor{currentstroke}%
\pgfsetdash{}{0pt}%
\pgfpathmoveto{\pgfqpoint{2.567413in}{4.041877in}}%
\pgfpathcurveto{\pgfqpoint{2.578463in}{4.041877in}}{\pgfqpoint{2.589062in}{4.046267in}}{\pgfqpoint{2.596875in}{4.054081in}}%
\pgfpathcurveto{\pgfqpoint{2.604689in}{4.061894in}}{\pgfqpoint{2.609079in}{4.072493in}}{\pgfqpoint{2.609079in}{4.083544in}}%
\pgfpathcurveto{\pgfqpoint{2.609079in}{4.094594in}}{\pgfqpoint{2.604689in}{4.105193in}}{\pgfqpoint{2.596875in}{4.113006in}}%
\pgfpathcurveto{\pgfqpoint{2.589062in}{4.120820in}}{\pgfqpoint{2.578463in}{4.125210in}}{\pgfqpoint{2.567413in}{4.125210in}}%
\pgfpathcurveto{\pgfqpoint{2.556363in}{4.125210in}}{\pgfqpoint{2.545763in}{4.120820in}}{\pgfqpoint{2.537950in}{4.113006in}}%
\pgfpathcurveto{\pgfqpoint{2.530136in}{4.105193in}}{\pgfqpoint{2.525746in}{4.094594in}}{\pgfqpoint{2.525746in}{4.083544in}}%
\pgfpathcurveto{\pgfqpoint{2.525746in}{4.072493in}}{\pgfqpoint{2.530136in}{4.061894in}}{\pgfqpoint{2.537950in}{4.054081in}}%
\pgfpathcurveto{\pgfqpoint{2.545763in}{4.046267in}}{\pgfqpoint{2.556363in}{4.041877in}}{\pgfqpoint{2.567413in}{4.041877in}}%
\pgfpathclose%
\pgfusepath{stroke,fill}%
\end{pgfscope}%
\begin{pgfscope}%
\pgfpathrectangle{\pgfqpoint{0.481978in}{0.331635in}}{\pgfqpoint{9.300000in}{7.700000in}}%
\pgfusepath{clip}%
\pgfsetbuttcap%
\pgfsetroundjoin%
\definecolor{currentfill}{rgb}{0.870588,0.733333,0.607843}%
\pgfsetfillcolor{currentfill}%
\pgfsetlinewidth{0.481800pt}%
\definecolor{currentstroke}{rgb}{1.000000,1.000000,1.000000}%
\pgfsetstrokecolor{currentstroke}%
\pgfsetdash{}{0pt}%
\pgfpathmoveto{\pgfqpoint{2.475746in}{3.962700in}}%
\pgfpathcurveto{\pgfqpoint{2.486796in}{3.962700in}}{\pgfqpoint{2.497395in}{3.967090in}}{\pgfqpoint{2.505209in}{3.974904in}}%
\pgfpathcurveto{\pgfqpoint{2.513022in}{3.982717in}}{\pgfqpoint{2.517413in}{3.993316in}}{\pgfqpoint{2.517413in}{4.004366in}}%
\pgfpathcurveto{\pgfqpoint{2.517413in}{4.015417in}}{\pgfqpoint{2.513022in}{4.026016in}}{\pgfqpoint{2.505209in}{4.033829in}}%
\pgfpathcurveto{\pgfqpoint{2.497395in}{4.041643in}}{\pgfqpoint{2.486796in}{4.046033in}}{\pgfqpoint{2.475746in}{4.046033in}}%
\pgfpathcurveto{\pgfqpoint{2.464696in}{4.046033in}}{\pgfqpoint{2.454097in}{4.041643in}}{\pgfqpoint{2.446283in}{4.033829in}}%
\pgfpathcurveto{\pgfqpoint{2.438470in}{4.026016in}}{\pgfqpoint{2.434079in}{4.015417in}}{\pgfqpoint{2.434079in}{4.004366in}}%
\pgfpathcurveto{\pgfqpoint{2.434079in}{3.993316in}}{\pgfqpoint{2.438470in}{3.982717in}}{\pgfqpoint{2.446283in}{3.974904in}}%
\pgfpathcurveto{\pgfqpoint{2.454097in}{3.967090in}}{\pgfqpoint{2.464696in}{3.962700in}}{\pgfqpoint{2.475746in}{3.962700in}}%
\pgfpathclose%
\pgfusepath{stroke,fill}%
\end{pgfscope}%
\begin{pgfscope}%
\pgfpathrectangle{\pgfqpoint{0.481978in}{0.331635in}}{\pgfqpoint{9.300000in}{7.700000in}}%
\pgfusepath{clip}%
\pgfsetbuttcap%
\pgfsetroundjoin%
\definecolor{currentfill}{rgb}{0.870588,0.733333,0.607843}%
\pgfsetfillcolor{currentfill}%
\pgfsetlinewidth{0.481800pt}%
\definecolor{currentstroke}{rgb}{1.000000,1.000000,1.000000}%
\pgfsetstrokecolor{currentstroke}%
\pgfsetdash{}{0pt}%
\pgfpathmoveto{\pgfqpoint{7.321265in}{5.989130in}}%
\pgfpathcurveto{\pgfqpoint{7.332315in}{5.989130in}}{\pgfqpoint{7.342914in}{5.993521in}}{\pgfqpoint{7.350728in}{6.001334in}}%
\pgfpathcurveto{\pgfqpoint{7.358541in}{6.009148in}}{\pgfqpoint{7.362932in}{6.019747in}}{\pgfqpoint{7.362932in}{6.030797in}}%
\pgfpathcurveto{\pgfqpoint{7.362932in}{6.041847in}}{\pgfqpoint{7.358541in}{6.052446in}}{\pgfqpoint{7.350728in}{6.060260in}}%
\pgfpathcurveto{\pgfqpoint{7.342914in}{6.068073in}}{\pgfqpoint{7.332315in}{6.072464in}}{\pgfqpoint{7.321265in}{6.072464in}}%
\pgfpathcurveto{\pgfqpoint{7.310215in}{6.072464in}}{\pgfqpoint{7.299616in}{6.068073in}}{\pgfqpoint{7.291802in}{6.060260in}}%
\pgfpathcurveto{\pgfqpoint{7.283988in}{6.052446in}}{\pgfqpoint{7.279598in}{6.041847in}}{\pgfqpoint{7.279598in}{6.030797in}}%
\pgfpathcurveto{\pgfqpoint{7.279598in}{6.019747in}}{\pgfqpoint{7.283988in}{6.009148in}}{\pgfqpoint{7.291802in}{6.001334in}}%
\pgfpathcurveto{\pgfqpoint{7.299616in}{5.993521in}}{\pgfqpoint{7.310215in}{5.989130in}}{\pgfqpoint{7.321265in}{5.989130in}}%
\pgfpathclose%
\pgfusepath{stroke,fill}%
\end{pgfscope}%
\begin{pgfscope}%
\pgfpathrectangle{\pgfqpoint{0.481978in}{0.331635in}}{\pgfqpoint{9.300000in}{7.700000in}}%
\pgfusepath{clip}%
\pgfsetbuttcap%
\pgfsetroundjoin%
\definecolor{currentfill}{rgb}{0.870588,0.733333,0.607843}%
\pgfsetfillcolor{currentfill}%
\pgfsetlinewidth{0.481800pt}%
\definecolor{currentstroke}{rgb}{1.000000,1.000000,1.000000}%
\pgfsetstrokecolor{currentstroke}%
\pgfsetdash{}{0pt}%
\pgfpathmoveto{\pgfqpoint{1.982524in}{5.164596in}}%
\pgfpathcurveto{\pgfqpoint{1.993574in}{5.164596in}}{\pgfqpoint{2.004173in}{5.168986in}}{\pgfqpoint{2.011987in}{5.176800in}}%
\pgfpathcurveto{\pgfqpoint{2.019800in}{5.184613in}}{\pgfqpoint{2.024191in}{5.195212in}}{\pgfqpoint{2.024191in}{5.206262in}}%
\pgfpathcurveto{\pgfqpoint{2.024191in}{5.217313in}}{\pgfqpoint{2.019800in}{5.227912in}}{\pgfqpoint{2.011987in}{5.235725in}}%
\pgfpathcurveto{\pgfqpoint{2.004173in}{5.243539in}}{\pgfqpoint{1.993574in}{5.247929in}}{\pgfqpoint{1.982524in}{5.247929in}}%
\pgfpathcurveto{\pgfqpoint{1.971474in}{5.247929in}}{\pgfqpoint{1.960875in}{5.243539in}}{\pgfqpoint{1.953061in}{5.235725in}}%
\pgfpathcurveto{\pgfqpoint{1.945248in}{5.227912in}}{\pgfqpoint{1.940857in}{5.217313in}}{\pgfqpoint{1.940857in}{5.206262in}}%
\pgfpathcurveto{\pgfqpoint{1.940857in}{5.195212in}}{\pgfqpoint{1.945248in}{5.184613in}}{\pgfqpoint{1.953061in}{5.176800in}}%
\pgfpathcurveto{\pgfqpoint{1.960875in}{5.168986in}}{\pgfqpoint{1.971474in}{5.164596in}}{\pgfqpoint{1.982524in}{5.164596in}}%
\pgfpathclose%
\pgfusepath{stroke,fill}%
\end{pgfscope}%
\begin{pgfscope}%
\pgfpathrectangle{\pgfqpoint{0.481978in}{0.331635in}}{\pgfqpoint{9.300000in}{7.700000in}}%
\pgfusepath{clip}%
\pgfsetbuttcap%
\pgfsetroundjoin%
\definecolor{currentfill}{rgb}{0.870588,0.733333,0.607843}%
\pgfsetfillcolor{currentfill}%
\pgfsetlinewidth{0.481800pt}%
\definecolor{currentstroke}{rgb}{1.000000,1.000000,1.000000}%
\pgfsetstrokecolor{currentstroke}%
\pgfsetdash{}{0pt}%
\pgfpathmoveto{\pgfqpoint{2.801903in}{4.359870in}}%
\pgfpathcurveto{\pgfqpoint{2.812953in}{4.359870in}}{\pgfqpoint{2.823552in}{4.364261in}}{\pgfqpoint{2.831365in}{4.372074in}}%
\pgfpathcurveto{\pgfqpoint{2.839179in}{4.379888in}}{\pgfqpoint{2.843569in}{4.390487in}}{\pgfqpoint{2.843569in}{4.401537in}}%
\pgfpathcurveto{\pgfqpoint{2.843569in}{4.412587in}}{\pgfqpoint{2.839179in}{4.423186in}}{\pgfqpoint{2.831365in}{4.431000in}}%
\pgfpathcurveto{\pgfqpoint{2.823552in}{4.438813in}}{\pgfqpoint{2.812953in}{4.443204in}}{\pgfqpoint{2.801903in}{4.443204in}}%
\pgfpathcurveto{\pgfqpoint{2.790852in}{4.443204in}}{\pgfqpoint{2.780253in}{4.438813in}}{\pgfqpoint{2.772440in}{4.431000in}}%
\pgfpathcurveto{\pgfqpoint{2.764626in}{4.423186in}}{\pgfqpoint{2.760236in}{4.412587in}}{\pgfqpoint{2.760236in}{4.401537in}}%
\pgfpathcurveto{\pgfqpoint{2.760236in}{4.390487in}}{\pgfqpoint{2.764626in}{4.379888in}}{\pgfqpoint{2.772440in}{4.372074in}}%
\pgfpathcurveto{\pgfqpoint{2.780253in}{4.364261in}}{\pgfqpoint{2.790852in}{4.359870in}}{\pgfqpoint{2.801903in}{4.359870in}}%
\pgfpathclose%
\pgfusepath{stroke,fill}%
\end{pgfscope}%
\begin{pgfscope}%
\pgfpathrectangle{\pgfqpoint{0.481978in}{0.331635in}}{\pgfqpoint{9.300000in}{7.700000in}}%
\pgfusepath{clip}%
\pgfsetbuttcap%
\pgfsetroundjoin%
\definecolor{currentfill}{rgb}{0.870588,0.733333,0.607843}%
\pgfsetfillcolor{currentfill}%
\pgfsetlinewidth{0.481800pt}%
\definecolor{currentstroke}{rgb}{1.000000,1.000000,1.000000}%
\pgfsetstrokecolor{currentstroke}%
\pgfsetdash{}{0pt}%
\pgfpathmoveto{\pgfqpoint{3.503709in}{4.271868in}}%
\pgfpathcurveto{\pgfqpoint{3.514759in}{4.271868in}}{\pgfqpoint{3.525358in}{4.276258in}}{\pgfqpoint{3.533171in}{4.284072in}}%
\pgfpathcurveto{\pgfqpoint{3.540985in}{4.291886in}}{\pgfqpoint{3.545375in}{4.302485in}}{\pgfqpoint{3.545375in}{4.313535in}}%
\pgfpathcurveto{\pgfqpoint{3.545375in}{4.324585in}}{\pgfqpoint{3.540985in}{4.335184in}}{\pgfqpoint{3.533171in}{4.342998in}}%
\pgfpathcurveto{\pgfqpoint{3.525358in}{4.350811in}}{\pgfqpoint{3.514759in}{4.355201in}}{\pgfqpoint{3.503709in}{4.355201in}}%
\pgfpathcurveto{\pgfqpoint{3.492659in}{4.355201in}}{\pgfqpoint{3.482059in}{4.350811in}}{\pgfqpoint{3.474246in}{4.342998in}}%
\pgfpathcurveto{\pgfqpoint{3.466432in}{4.335184in}}{\pgfqpoint{3.462042in}{4.324585in}}{\pgfqpoint{3.462042in}{4.313535in}}%
\pgfpathcurveto{\pgfqpoint{3.462042in}{4.302485in}}{\pgfqpoint{3.466432in}{4.291886in}}{\pgfqpoint{3.474246in}{4.284072in}}%
\pgfpathcurveto{\pgfqpoint{3.482059in}{4.276258in}}{\pgfqpoint{3.492659in}{4.271868in}}{\pgfqpoint{3.503709in}{4.271868in}}%
\pgfpathclose%
\pgfusepath{stroke,fill}%
\end{pgfscope}%
\begin{pgfscope}%
\pgfpathrectangle{\pgfqpoint{0.481978in}{0.331635in}}{\pgfqpoint{9.300000in}{7.700000in}}%
\pgfusepath{clip}%
\pgfsetbuttcap%
\pgfsetroundjoin%
\definecolor{currentfill}{rgb}{0.870588,0.733333,0.607843}%
\pgfsetfillcolor{currentfill}%
\pgfsetlinewidth{0.481800pt}%
\definecolor{currentstroke}{rgb}{1.000000,1.000000,1.000000}%
\pgfsetstrokecolor{currentstroke}%
\pgfsetdash{}{0pt}%
\pgfpathmoveto{\pgfqpoint{7.049429in}{6.014836in}}%
\pgfpathcurveto{\pgfqpoint{7.060479in}{6.014836in}}{\pgfqpoint{7.071078in}{6.019226in}}{\pgfqpoint{7.078892in}{6.027039in}}%
\pgfpathcurveto{\pgfqpoint{7.086705in}{6.034853in}}{\pgfqpoint{7.091096in}{6.045452in}}{\pgfqpoint{7.091096in}{6.056502in}}%
\pgfpathcurveto{\pgfqpoint{7.091096in}{6.067552in}}{\pgfqpoint{7.086705in}{6.078151in}}{\pgfqpoint{7.078892in}{6.085965in}}%
\pgfpathcurveto{\pgfqpoint{7.071078in}{6.093779in}}{\pgfqpoint{7.060479in}{6.098169in}}{\pgfqpoint{7.049429in}{6.098169in}}%
\pgfpathcurveto{\pgfqpoint{7.038379in}{6.098169in}}{\pgfqpoint{7.027780in}{6.093779in}}{\pgfqpoint{7.019966in}{6.085965in}}%
\pgfpathcurveto{\pgfqpoint{7.012153in}{6.078151in}}{\pgfqpoint{7.007762in}{6.067552in}}{\pgfqpoint{7.007762in}{6.056502in}}%
\pgfpathcurveto{\pgfqpoint{7.007762in}{6.045452in}}{\pgfqpoint{7.012153in}{6.034853in}}{\pgfqpoint{7.019966in}{6.027039in}}%
\pgfpathcurveto{\pgfqpoint{7.027780in}{6.019226in}}{\pgfqpoint{7.038379in}{6.014836in}}{\pgfqpoint{7.049429in}{6.014836in}}%
\pgfpathclose%
\pgfusepath{stroke,fill}%
\end{pgfscope}%
\begin{pgfscope}%
\pgfpathrectangle{\pgfqpoint{0.481978in}{0.331635in}}{\pgfqpoint{9.300000in}{7.700000in}}%
\pgfusepath{clip}%
\pgfsetbuttcap%
\pgfsetroundjoin%
\definecolor{currentfill}{rgb}{0.870588,0.733333,0.607843}%
\pgfsetfillcolor{currentfill}%
\pgfsetlinewidth{0.481800pt}%
\definecolor{currentstroke}{rgb}{1.000000,1.000000,1.000000}%
\pgfsetstrokecolor{currentstroke}%
\pgfsetdash{}{0pt}%
\pgfpathmoveto{\pgfqpoint{6.348280in}{4.805197in}}%
\pgfpathcurveto{\pgfqpoint{6.359330in}{4.805197in}}{\pgfqpoint{6.369929in}{4.809587in}}{\pgfqpoint{6.377743in}{4.817401in}}%
\pgfpathcurveto{\pgfqpoint{6.385557in}{4.825215in}}{\pgfqpoint{6.389947in}{4.835814in}}{\pgfqpoint{6.389947in}{4.846864in}}%
\pgfpathcurveto{\pgfqpoint{6.389947in}{4.857914in}}{\pgfqpoint{6.385557in}{4.868513in}}{\pgfqpoint{6.377743in}{4.876327in}}%
\pgfpathcurveto{\pgfqpoint{6.369929in}{4.884140in}}{\pgfqpoint{6.359330in}{4.888530in}}{\pgfqpoint{6.348280in}{4.888530in}}%
\pgfpathcurveto{\pgfqpoint{6.337230in}{4.888530in}}{\pgfqpoint{6.326631in}{4.884140in}}{\pgfqpoint{6.318818in}{4.876327in}}%
\pgfpathcurveto{\pgfqpoint{6.311004in}{4.868513in}}{\pgfqpoint{6.306614in}{4.857914in}}{\pgfqpoint{6.306614in}{4.846864in}}%
\pgfpathcurveto{\pgfqpoint{6.306614in}{4.835814in}}{\pgfqpoint{6.311004in}{4.825215in}}{\pgfqpoint{6.318818in}{4.817401in}}%
\pgfpathcurveto{\pgfqpoint{6.326631in}{4.809587in}}{\pgfqpoint{6.337230in}{4.805197in}}{\pgfqpoint{6.348280in}{4.805197in}}%
\pgfpathclose%
\pgfusepath{stroke,fill}%
\end{pgfscope}%
\begin{pgfscope}%
\pgfpathrectangle{\pgfqpoint{0.481978in}{0.331635in}}{\pgfqpoint{9.300000in}{7.700000in}}%
\pgfusepath{clip}%
\pgfsetbuttcap%
\pgfsetroundjoin%
\definecolor{currentfill}{rgb}{0.870588,0.733333,0.607843}%
\pgfsetfillcolor{currentfill}%
\pgfsetlinewidth{0.481800pt}%
\definecolor{currentstroke}{rgb}{1.000000,1.000000,1.000000}%
\pgfsetstrokecolor{currentstroke}%
\pgfsetdash{}{0pt}%
\pgfpathmoveto{\pgfqpoint{5.923605in}{3.640188in}}%
\pgfpathcurveto{\pgfqpoint{5.934656in}{3.640188in}}{\pgfqpoint{5.945255in}{3.644578in}}{\pgfqpoint{5.953068in}{3.652392in}}%
\pgfpathcurveto{\pgfqpoint{5.960882in}{3.660205in}}{\pgfqpoint{5.965272in}{3.670804in}}{\pgfqpoint{5.965272in}{3.681854in}}%
\pgfpathcurveto{\pgfqpoint{5.965272in}{3.692905in}}{\pgfqpoint{5.960882in}{3.703504in}}{\pgfqpoint{5.953068in}{3.711317in}}%
\pgfpathcurveto{\pgfqpoint{5.945255in}{3.719131in}}{\pgfqpoint{5.934656in}{3.723521in}}{\pgfqpoint{5.923605in}{3.723521in}}%
\pgfpathcurveto{\pgfqpoint{5.912555in}{3.723521in}}{\pgfqpoint{5.901956in}{3.719131in}}{\pgfqpoint{5.894143in}{3.711317in}}%
\pgfpathcurveto{\pgfqpoint{5.886329in}{3.703504in}}{\pgfqpoint{5.881939in}{3.692905in}}{\pgfqpoint{5.881939in}{3.681854in}}%
\pgfpathcurveto{\pgfqpoint{5.881939in}{3.670804in}}{\pgfqpoint{5.886329in}{3.660205in}}{\pgfqpoint{5.894143in}{3.652392in}}%
\pgfpathcurveto{\pgfqpoint{5.901956in}{3.644578in}}{\pgfqpoint{5.912555in}{3.640188in}}{\pgfqpoint{5.923605in}{3.640188in}}%
\pgfpathclose%
\pgfusepath{stroke,fill}%
\end{pgfscope}%
\begin{pgfscope}%
\pgfpathrectangle{\pgfqpoint{0.481978in}{0.331635in}}{\pgfqpoint{9.300000in}{7.700000in}}%
\pgfusepath{clip}%
\pgfsetbuttcap%
\pgfsetroundjoin%
\definecolor{currentfill}{rgb}{0.870588,0.733333,0.607843}%
\pgfsetfillcolor{currentfill}%
\pgfsetlinewidth{0.481800pt}%
\definecolor{currentstroke}{rgb}{1.000000,1.000000,1.000000}%
\pgfsetstrokecolor{currentstroke}%
\pgfsetdash{}{0pt}%
\pgfpathmoveto{\pgfqpoint{5.344914in}{3.994410in}}%
\pgfpathcurveto{\pgfqpoint{5.355965in}{3.994410in}}{\pgfqpoint{5.366564in}{3.998800in}}{\pgfqpoint{5.374377in}{4.006614in}}%
\pgfpathcurveto{\pgfqpoint{5.382191in}{4.014427in}}{\pgfqpoint{5.386581in}{4.025027in}}{\pgfqpoint{5.386581in}{4.036077in}}%
\pgfpathcurveto{\pgfqpoint{5.386581in}{4.047127in}}{\pgfqpoint{5.382191in}{4.057726in}}{\pgfqpoint{5.374377in}{4.065539in}}%
\pgfpathcurveto{\pgfqpoint{5.366564in}{4.073353in}}{\pgfqpoint{5.355965in}{4.077743in}}{\pgfqpoint{5.344914in}{4.077743in}}%
\pgfpathcurveto{\pgfqpoint{5.333864in}{4.077743in}}{\pgfqpoint{5.323265in}{4.073353in}}{\pgfqpoint{5.315452in}{4.065539in}}%
\pgfpathcurveto{\pgfqpoint{5.307638in}{4.057726in}}{\pgfqpoint{5.303248in}{4.047127in}}{\pgfqpoint{5.303248in}{4.036077in}}%
\pgfpathcurveto{\pgfqpoint{5.303248in}{4.025027in}}{\pgfqpoint{5.307638in}{4.014427in}}{\pgfqpoint{5.315452in}{4.006614in}}%
\pgfpathcurveto{\pgfqpoint{5.323265in}{3.998800in}}{\pgfqpoint{5.333864in}{3.994410in}}{\pgfqpoint{5.344914in}{3.994410in}}%
\pgfpathclose%
\pgfusepath{stroke,fill}%
\end{pgfscope}%
\begin{pgfscope}%
\pgfpathrectangle{\pgfqpoint{0.481978in}{0.331635in}}{\pgfqpoint{9.300000in}{7.700000in}}%
\pgfusepath{clip}%
\pgfsetbuttcap%
\pgfsetroundjoin%
\definecolor{currentfill}{rgb}{0.870588,0.733333,0.607843}%
\pgfsetfillcolor{currentfill}%
\pgfsetlinewidth{0.481800pt}%
\definecolor{currentstroke}{rgb}{1.000000,1.000000,1.000000}%
\pgfsetstrokecolor{currentstroke}%
\pgfsetdash{}{0pt}%
\pgfpathmoveto{\pgfqpoint{4.527667in}{3.702609in}}%
\pgfpathcurveto{\pgfqpoint{4.538717in}{3.702609in}}{\pgfqpoint{4.549316in}{3.706999in}}{\pgfqpoint{4.557129in}{3.714813in}}%
\pgfpathcurveto{\pgfqpoint{4.564943in}{3.722627in}}{\pgfqpoint{4.569333in}{3.733226in}}{\pgfqpoint{4.569333in}{3.744276in}}%
\pgfpathcurveto{\pgfqpoint{4.569333in}{3.755326in}}{\pgfqpoint{4.564943in}{3.765925in}}{\pgfqpoint{4.557129in}{3.773739in}}%
\pgfpathcurveto{\pgfqpoint{4.549316in}{3.781552in}}{\pgfqpoint{4.538717in}{3.785943in}}{\pgfqpoint{4.527667in}{3.785943in}}%
\pgfpathcurveto{\pgfqpoint{4.516616in}{3.785943in}}{\pgfqpoint{4.506017in}{3.781552in}}{\pgfqpoint{4.498204in}{3.773739in}}%
\pgfpathcurveto{\pgfqpoint{4.490390in}{3.765925in}}{\pgfqpoint{4.486000in}{3.755326in}}{\pgfqpoint{4.486000in}{3.744276in}}%
\pgfpathcurveto{\pgfqpoint{4.486000in}{3.733226in}}{\pgfqpoint{4.490390in}{3.722627in}}{\pgfqpoint{4.498204in}{3.714813in}}%
\pgfpathcurveto{\pgfqpoint{4.506017in}{3.706999in}}{\pgfqpoint{4.516616in}{3.702609in}}{\pgfqpoint{4.527667in}{3.702609in}}%
\pgfpathclose%
\pgfusepath{stroke,fill}%
\end{pgfscope}%
\begin{pgfscope}%
\pgfpathrectangle{\pgfqpoint{0.481978in}{0.331635in}}{\pgfqpoint{9.300000in}{7.700000in}}%
\pgfusepath{clip}%
\pgfsetbuttcap%
\pgfsetroundjoin%
\definecolor{currentfill}{rgb}{0.870588,0.733333,0.607843}%
\pgfsetfillcolor{currentfill}%
\pgfsetlinewidth{0.481800pt}%
\definecolor{currentstroke}{rgb}{1.000000,1.000000,1.000000}%
\pgfsetstrokecolor{currentstroke}%
\pgfsetdash{}{0pt}%
\pgfpathmoveto{\pgfqpoint{2.059850in}{3.994794in}}%
\pgfpathcurveto{\pgfqpoint{2.070900in}{3.994794in}}{\pgfqpoint{2.081499in}{3.999184in}}{\pgfqpoint{2.089313in}{4.006997in}}%
\pgfpathcurveto{\pgfqpoint{2.097126in}{4.014811in}}{\pgfqpoint{2.101516in}{4.025410in}}{\pgfqpoint{2.101516in}{4.036460in}}%
\pgfpathcurveto{\pgfqpoint{2.101516in}{4.047510in}}{\pgfqpoint{2.097126in}{4.058109in}}{\pgfqpoint{2.089313in}{4.065923in}}%
\pgfpathcurveto{\pgfqpoint{2.081499in}{4.073737in}}{\pgfqpoint{2.070900in}{4.078127in}}{\pgfqpoint{2.059850in}{4.078127in}}%
\pgfpathcurveto{\pgfqpoint{2.048800in}{4.078127in}}{\pgfqpoint{2.038201in}{4.073737in}}{\pgfqpoint{2.030387in}{4.065923in}}%
\pgfpathcurveto{\pgfqpoint{2.022573in}{4.058109in}}{\pgfqpoint{2.018183in}{4.047510in}}{\pgfqpoint{2.018183in}{4.036460in}}%
\pgfpathcurveto{\pgfqpoint{2.018183in}{4.025410in}}{\pgfqpoint{2.022573in}{4.014811in}}{\pgfqpoint{2.030387in}{4.006997in}}%
\pgfpathcurveto{\pgfqpoint{2.038201in}{3.999184in}}{\pgfqpoint{2.048800in}{3.994794in}}{\pgfqpoint{2.059850in}{3.994794in}}%
\pgfpathclose%
\pgfusepath{stroke,fill}%
\end{pgfscope}%
\begin{pgfscope}%
\pgfpathrectangle{\pgfqpoint{0.481978in}{0.331635in}}{\pgfqpoint{9.300000in}{7.700000in}}%
\pgfusepath{clip}%
\pgfsetbuttcap%
\pgfsetroundjoin%
\definecolor{currentfill}{rgb}{0.870588,0.733333,0.607843}%
\pgfsetfillcolor{currentfill}%
\pgfsetlinewidth{0.481800pt}%
\definecolor{currentstroke}{rgb}{1.000000,1.000000,1.000000}%
\pgfsetstrokecolor{currentstroke}%
\pgfsetdash{}{0pt}%
\pgfpathmoveto{\pgfqpoint{5.974496in}{4.908445in}}%
\pgfpathcurveto{\pgfqpoint{5.985546in}{4.908445in}}{\pgfqpoint{5.996145in}{4.912835in}}{\pgfqpoint{6.003958in}{4.920649in}}%
\pgfpathcurveto{\pgfqpoint{6.011772in}{4.928462in}}{\pgfqpoint{6.016162in}{4.939061in}}{\pgfqpoint{6.016162in}{4.950112in}}%
\pgfpathcurveto{\pgfqpoint{6.016162in}{4.961162in}}{\pgfqpoint{6.011772in}{4.971761in}}{\pgfqpoint{6.003958in}{4.979574in}}%
\pgfpathcurveto{\pgfqpoint{5.996145in}{4.987388in}}{\pgfqpoint{5.985546in}{4.991778in}}{\pgfqpoint{5.974496in}{4.991778in}}%
\pgfpathcurveto{\pgfqpoint{5.963445in}{4.991778in}}{\pgfqpoint{5.952846in}{4.987388in}}{\pgfqpoint{5.945033in}{4.979574in}}%
\pgfpathcurveto{\pgfqpoint{5.937219in}{4.971761in}}{\pgfqpoint{5.932829in}{4.961162in}}{\pgfqpoint{5.932829in}{4.950112in}}%
\pgfpathcurveto{\pgfqpoint{5.932829in}{4.939061in}}{\pgfqpoint{5.937219in}{4.928462in}}{\pgfqpoint{5.945033in}{4.920649in}}%
\pgfpathcurveto{\pgfqpoint{5.952846in}{4.912835in}}{\pgfqpoint{5.963445in}{4.908445in}}{\pgfqpoint{5.974496in}{4.908445in}}%
\pgfpathclose%
\pgfusepath{stroke,fill}%
\end{pgfscope}%
\begin{pgfscope}%
\pgfpathrectangle{\pgfqpoint{0.481978in}{0.331635in}}{\pgfqpoint{9.300000in}{7.700000in}}%
\pgfusepath{clip}%
\pgfsetbuttcap%
\pgfsetroundjoin%
\definecolor{currentfill}{rgb}{0.870588,0.733333,0.607843}%
\pgfsetfillcolor{currentfill}%
\pgfsetlinewidth{0.481800pt}%
\definecolor{currentstroke}{rgb}{1.000000,1.000000,1.000000}%
\pgfsetstrokecolor{currentstroke}%
\pgfsetdash{}{0pt}%
\pgfpathmoveto{\pgfqpoint{2.319348in}{4.753136in}}%
\pgfpathcurveto{\pgfqpoint{2.330398in}{4.753136in}}{\pgfqpoint{2.340997in}{4.757526in}}{\pgfqpoint{2.348811in}{4.765339in}}%
\pgfpathcurveto{\pgfqpoint{2.356624in}{4.773153in}}{\pgfqpoint{2.361015in}{4.783752in}}{\pgfqpoint{2.361015in}{4.794802in}}%
\pgfpathcurveto{\pgfqpoint{2.361015in}{4.805852in}}{\pgfqpoint{2.356624in}{4.816451in}}{\pgfqpoint{2.348811in}{4.824265in}}%
\pgfpathcurveto{\pgfqpoint{2.340997in}{4.832079in}}{\pgfqpoint{2.330398in}{4.836469in}}{\pgfqpoint{2.319348in}{4.836469in}}%
\pgfpathcurveto{\pgfqpoint{2.308298in}{4.836469in}}{\pgfqpoint{2.297699in}{4.832079in}}{\pgfqpoint{2.289885in}{4.824265in}}%
\pgfpathcurveto{\pgfqpoint{2.282071in}{4.816451in}}{\pgfqpoint{2.277681in}{4.805852in}}{\pgfqpoint{2.277681in}{4.794802in}}%
\pgfpathcurveto{\pgfqpoint{2.277681in}{4.783752in}}{\pgfqpoint{2.282071in}{4.773153in}}{\pgfqpoint{2.289885in}{4.765339in}}%
\pgfpathcurveto{\pgfqpoint{2.297699in}{4.757526in}}{\pgfqpoint{2.308298in}{4.753136in}}{\pgfqpoint{2.319348in}{4.753136in}}%
\pgfpathclose%
\pgfusepath{stroke,fill}%
\end{pgfscope}%
\begin{pgfscope}%
\pgfpathrectangle{\pgfqpoint{0.481978in}{0.331635in}}{\pgfqpoint{9.300000in}{7.700000in}}%
\pgfusepath{clip}%
\pgfsetbuttcap%
\pgfsetroundjoin%
\definecolor{currentfill}{rgb}{0.870588,0.733333,0.607843}%
\pgfsetfillcolor{currentfill}%
\pgfsetlinewidth{0.481800pt}%
\definecolor{currentstroke}{rgb}{1.000000,1.000000,1.000000}%
\pgfsetstrokecolor{currentstroke}%
\pgfsetdash{}{0pt}%
\pgfpathmoveto{\pgfqpoint{6.238443in}{6.200764in}}%
\pgfpathcurveto{\pgfqpoint{6.249493in}{6.200764in}}{\pgfqpoint{6.260092in}{6.205154in}}{\pgfqpoint{6.267905in}{6.212968in}}%
\pgfpathcurveto{\pgfqpoint{6.275719in}{6.220782in}}{\pgfqpoint{6.280109in}{6.231381in}}{\pgfqpoint{6.280109in}{6.242431in}}%
\pgfpathcurveto{\pgfqpoint{6.280109in}{6.253481in}}{\pgfqpoint{6.275719in}{6.264080in}}{\pgfqpoint{6.267905in}{6.271894in}}%
\pgfpathcurveto{\pgfqpoint{6.260092in}{6.279707in}}{\pgfqpoint{6.249493in}{6.284097in}}{\pgfqpoint{6.238443in}{6.284097in}}%
\pgfpathcurveto{\pgfqpoint{6.227392in}{6.284097in}}{\pgfqpoint{6.216793in}{6.279707in}}{\pgfqpoint{6.208980in}{6.271894in}}%
\pgfpathcurveto{\pgfqpoint{6.201166in}{6.264080in}}{\pgfqpoint{6.196776in}{6.253481in}}{\pgfqpoint{6.196776in}{6.242431in}}%
\pgfpathcurveto{\pgfqpoint{6.196776in}{6.231381in}}{\pgfqpoint{6.201166in}{6.220782in}}{\pgfqpoint{6.208980in}{6.212968in}}%
\pgfpathcurveto{\pgfqpoint{6.216793in}{6.205154in}}{\pgfqpoint{6.227392in}{6.200764in}}{\pgfqpoint{6.238443in}{6.200764in}}%
\pgfpathclose%
\pgfusepath{stroke,fill}%
\end{pgfscope}%
\begin{pgfscope}%
\pgfpathrectangle{\pgfqpoint{0.481978in}{0.331635in}}{\pgfqpoint{9.300000in}{7.700000in}}%
\pgfusepath{clip}%
\pgfsetbuttcap%
\pgfsetroundjoin%
\definecolor{currentfill}{rgb}{0.870588,0.733333,0.607843}%
\pgfsetfillcolor{currentfill}%
\pgfsetlinewidth{0.481800pt}%
\definecolor{currentstroke}{rgb}{1.000000,1.000000,1.000000}%
\pgfsetstrokecolor{currentstroke}%
\pgfsetdash{}{0pt}%
\pgfpathmoveto{\pgfqpoint{4.167128in}{3.855525in}}%
\pgfpathcurveto{\pgfqpoint{4.178178in}{3.855525in}}{\pgfqpoint{4.188778in}{3.859915in}}{\pgfqpoint{4.196591in}{3.867729in}}%
\pgfpathcurveto{\pgfqpoint{4.204405in}{3.875542in}}{\pgfqpoint{4.208795in}{3.886141in}}{\pgfqpoint{4.208795in}{3.897191in}}%
\pgfpathcurveto{\pgfqpoint{4.208795in}{3.908241in}}{\pgfqpoint{4.204405in}{3.918841in}}{\pgfqpoint{4.196591in}{3.926654in}}%
\pgfpathcurveto{\pgfqpoint{4.188778in}{3.934468in}}{\pgfqpoint{4.178178in}{3.938858in}}{\pgfqpoint{4.167128in}{3.938858in}}%
\pgfpathcurveto{\pgfqpoint{4.156078in}{3.938858in}}{\pgfqpoint{4.145479in}{3.934468in}}{\pgfqpoint{4.137666in}{3.926654in}}%
\pgfpathcurveto{\pgfqpoint{4.129852in}{3.918841in}}{\pgfqpoint{4.125462in}{3.908241in}}{\pgfqpoint{4.125462in}{3.897191in}}%
\pgfpathcurveto{\pgfqpoint{4.125462in}{3.886141in}}{\pgfqpoint{4.129852in}{3.875542in}}{\pgfqpoint{4.137666in}{3.867729in}}%
\pgfpathcurveto{\pgfqpoint{4.145479in}{3.859915in}}{\pgfqpoint{4.156078in}{3.855525in}}{\pgfqpoint{4.167128in}{3.855525in}}%
\pgfpathclose%
\pgfusepath{stroke,fill}%
\end{pgfscope}%
\begin{pgfscope}%
\pgfpathrectangle{\pgfqpoint{0.481978in}{0.331635in}}{\pgfqpoint{9.300000in}{7.700000in}}%
\pgfusepath{clip}%
\pgfsetbuttcap%
\pgfsetroundjoin%
\definecolor{currentfill}{rgb}{0.870588,0.733333,0.607843}%
\pgfsetfillcolor{currentfill}%
\pgfsetlinewidth{0.481800pt}%
\definecolor{currentstroke}{rgb}{1.000000,1.000000,1.000000}%
\pgfsetstrokecolor{currentstroke}%
\pgfsetdash{}{0pt}%
\pgfpathmoveto{\pgfqpoint{4.383560in}{3.556162in}}%
\pgfpathcurveto{\pgfqpoint{4.394610in}{3.556162in}}{\pgfqpoint{4.405209in}{3.560553in}}{\pgfqpoint{4.413023in}{3.568366in}}%
\pgfpathcurveto{\pgfqpoint{4.420836in}{3.576180in}}{\pgfqpoint{4.425227in}{3.586779in}}{\pgfqpoint{4.425227in}{3.597829in}}%
\pgfpathcurveto{\pgfqpoint{4.425227in}{3.608879in}}{\pgfqpoint{4.420836in}{3.619478in}}{\pgfqpoint{4.413023in}{3.627292in}}%
\pgfpathcurveto{\pgfqpoint{4.405209in}{3.635105in}}{\pgfqpoint{4.394610in}{3.639496in}}{\pgfqpoint{4.383560in}{3.639496in}}%
\pgfpathcurveto{\pgfqpoint{4.372510in}{3.639496in}}{\pgfqpoint{4.361911in}{3.635105in}}{\pgfqpoint{4.354097in}{3.627292in}}%
\pgfpathcurveto{\pgfqpoint{4.346284in}{3.619478in}}{\pgfqpoint{4.341893in}{3.608879in}}{\pgfqpoint{4.341893in}{3.597829in}}%
\pgfpathcurveto{\pgfqpoint{4.341893in}{3.586779in}}{\pgfqpoint{4.346284in}{3.576180in}}{\pgfqpoint{4.354097in}{3.568366in}}%
\pgfpathcurveto{\pgfqpoint{4.361911in}{3.560553in}}{\pgfqpoint{4.372510in}{3.556162in}}{\pgfqpoint{4.383560in}{3.556162in}}%
\pgfpathclose%
\pgfusepath{stroke,fill}%
\end{pgfscope}%
\begin{pgfscope}%
\pgfpathrectangle{\pgfqpoint{0.481978in}{0.331635in}}{\pgfqpoint{9.300000in}{7.700000in}}%
\pgfusepath{clip}%
\pgfsetbuttcap%
\pgfsetroundjoin%
\definecolor{currentfill}{rgb}{0.870588,0.733333,0.607843}%
\pgfsetfillcolor{currentfill}%
\pgfsetlinewidth{0.481800pt}%
\definecolor{currentstroke}{rgb}{1.000000,1.000000,1.000000}%
\pgfsetstrokecolor{currentstroke}%
\pgfsetdash{}{0pt}%
\pgfpathmoveto{\pgfqpoint{5.769577in}{3.960324in}}%
\pgfpathcurveto{\pgfqpoint{5.780628in}{3.960324in}}{\pgfqpoint{5.791227in}{3.964714in}}{\pgfqpoint{5.799040in}{3.972527in}}%
\pgfpathcurveto{\pgfqpoint{5.806854in}{3.980341in}}{\pgfqpoint{5.811244in}{3.990940in}}{\pgfqpoint{5.811244in}{4.001990in}}%
\pgfpathcurveto{\pgfqpoint{5.811244in}{4.013040in}}{\pgfqpoint{5.806854in}{4.023639in}}{\pgfqpoint{5.799040in}{4.031453in}}%
\pgfpathcurveto{\pgfqpoint{5.791227in}{4.039267in}}{\pgfqpoint{5.780628in}{4.043657in}}{\pgfqpoint{5.769577in}{4.043657in}}%
\pgfpathcurveto{\pgfqpoint{5.758527in}{4.043657in}}{\pgfqpoint{5.747928in}{4.039267in}}{\pgfqpoint{5.740115in}{4.031453in}}%
\pgfpathcurveto{\pgfqpoint{5.732301in}{4.023639in}}{\pgfqpoint{5.727911in}{4.013040in}}{\pgfqpoint{5.727911in}{4.001990in}}%
\pgfpathcurveto{\pgfqpoint{5.727911in}{3.990940in}}{\pgfqpoint{5.732301in}{3.980341in}}{\pgfqpoint{5.740115in}{3.972527in}}%
\pgfpathcurveto{\pgfqpoint{5.747928in}{3.964714in}}{\pgfqpoint{5.758527in}{3.960324in}}{\pgfqpoint{5.769577in}{3.960324in}}%
\pgfpathclose%
\pgfusepath{stroke,fill}%
\end{pgfscope}%
\begin{pgfscope}%
\pgfpathrectangle{\pgfqpoint{0.481978in}{0.331635in}}{\pgfqpoint{9.300000in}{7.700000in}}%
\pgfusepath{clip}%
\pgfsetbuttcap%
\pgfsetroundjoin%
\definecolor{currentfill}{rgb}{0.870588,0.733333,0.607843}%
\pgfsetfillcolor{currentfill}%
\pgfsetlinewidth{0.481800pt}%
\definecolor{currentstroke}{rgb}{1.000000,1.000000,1.000000}%
\pgfsetstrokecolor{currentstroke}%
\pgfsetdash{}{0pt}%
\pgfpathmoveto{\pgfqpoint{9.359251in}{4.528409in}}%
\pgfpathcurveto{\pgfqpoint{9.370301in}{4.528409in}}{\pgfqpoint{9.380900in}{4.532799in}}{\pgfqpoint{9.388713in}{4.540613in}}%
\pgfpathcurveto{\pgfqpoint{9.396527in}{4.548427in}}{\pgfqpoint{9.400917in}{4.559026in}}{\pgfqpoint{9.400917in}{4.570076in}}%
\pgfpathcurveto{\pgfqpoint{9.400917in}{4.581126in}}{\pgfqpoint{9.396527in}{4.591725in}}{\pgfqpoint{9.388713in}{4.599539in}}%
\pgfpathcurveto{\pgfqpoint{9.380900in}{4.607352in}}{\pgfqpoint{9.370301in}{4.611743in}}{\pgfqpoint{9.359251in}{4.611743in}}%
\pgfpathcurveto{\pgfqpoint{9.348201in}{4.611743in}}{\pgfqpoint{9.337601in}{4.607352in}}{\pgfqpoint{9.329788in}{4.599539in}}%
\pgfpathcurveto{\pgfqpoint{9.321974in}{4.591725in}}{\pgfqpoint{9.317584in}{4.581126in}}{\pgfqpoint{9.317584in}{4.570076in}}%
\pgfpathcurveto{\pgfqpoint{9.317584in}{4.559026in}}{\pgfqpoint{9.321974in}{4.548427in}}{\pgfqpoint{9.329788in}{4.540613in}}%
\pgfpathcurveto{\pgfqpoint{9.337601in}{4.532799in}}{\pgfqpoint{9.348201in}{4.528409in}}{\pgfqpoint{9.359251in}{4.528409in}}%
\pgfpathclose%
\pgfusepath{stroke,fill}%
\end{pgfscope}%
\begin{pgfscope}%
\pgfpathrectangle{\pgfqpoint{0.481978in}{0.331635in}}{\pgfqpoint{9.300000in}{7.700000in}}%
\pgfusepath{clip}%
\pgfsetbuttcap%
\pgfsetroundjoin%
\definecolor{currentfill}{rgb}{0.870588,0.733333,0.607843}%
\pgfsetfillcolor{currentfill}%
\pgfsetlinewidth{0.481800pt}%
\definecolor{currentstroke}{rgb}{1.000000,1.000000,1.000000}%
\pgfsetstrokecolor{currentstroke}%
\pgfsetdash{}{0pt}%
\pgfpathmoveto{\pgfqpoint{5.675309in}{2.630964in}}%
\pgfpathcurveto{\pgfqpoint{5.686359in}{2.630964in}}{\pgfqpoint{5.696958in}{2.635354in}}{\pgfqpoint{5.704772in}{2.643168in}}%
\pgfpathcurveto{\pgfqpoint{5.712586in}{2.650981in}}{\pgfqpoint{5.716976in}{2.661581in}}{\pgfqpoint{5.716976in}{2.672631in}}%
\pgfpathcurveto{\pgfqpoint{5.716976in}{2.683681in}}{\pgfqpoint{5.712586in}{2.694280in}}{\pgfqpoint{5.704772in}{2.702093in}}%
\pgfpathcurveto{\pgfqpoint{5.696958in}{2.709907in}}{\pgfqpoint{5.686359in}{2.714297in}}{\pgfqpoint{5.675309in}{2.714297in}}%
\pgfpathcurveto{\pgfqpoint{5.664259in}{2.714297in}}{\pgfqpoint{5.653660in}{2.709907in}}{\pgfqpoint{5.645846in}{2.702093in}}%
\pgfpathcurveto{\pgfqpoint{5.638033in}{2.694280in}}{\pgfqpoint{5.633643in}{2.683681in}}{\pgfqpoint{5.633643in}{2.672631in}}%
\pgfpathcurveto{\pgfqpoint{5.633643in}{2.661581in}}{\pgfqpoint{5.638033in}{2.650981in}}{\pgfqpoint{5.645846in}{2.643168in}}%
\pgfpathcurveto{\pgfqpoint{5.653660in}{2.635354in}}{\pgfqpoint{5.664259in}{2.630964in}}{\pgfqpoint{5.675309in}{2.630964in}}%
\pgfpathclose%
\pgfusepath{stroke,fill}%
\end{pgfscope}%
\begin{pgfscope}%
\pgfpathrectangle{\pgfqpoint{0.481978in}{0.331635in}}{\pgfqpoint{9.300000in}{7.700000in}}%
\pgfusepath{clip}%
\pgfsetbuttcap%
\pgfsetroundjoin%
\definecolor{currentfill}{rgb}{0.870588,0.733333,0.607843}%
\pgfsetfillcolor{currentfill}%
\pgfsetlinewidth{0.481800pt}%
\definecolor{currentstroke}{rgb}{1.000000,1.000000,1.000000}%
\pgfsetstrokecolor{currentstroke}%
\pgfsetdash{}{0pt}%
\pgfpathmoveto{\pgfqpoint{3.026449in}{4.306297in}}%
\pgfpathcurveto{\pgfqpoint{3.037499in}{4.306297in}}{\pgfqpoint{3.048098in}{4.310688in}}{\pgfqpoint{3.055912in}{4.318501in}}%
\pgfpathcurveto{\pgfqpoint{3.063726in}{4.326315in}}{\pgfqpoint{3.068116in}{4.336914in}}{\pgfqpoint{3.068116in}{4.347964in}}%
\pgfpathcurveto{\pgfqpoint{3.068116in}{4.359014in}}{\pgfqpoint{3.063726in}{4.369613in}}{\pgfqpoint{3.055912in}{4.377427in}}%
\pgfpathcurveto{\pgfqpoint{3.048098in}{4.385240in}}{\pgfqpoint{3.037499in}{4.389631in}}{\pgfqpoint{3.026449in}{4.389631in}}%
\pgfpathcurveto{\pgfqpoint{3.015399in}{4.389631in}}{\pgfqpoint{3.004800in}{4.385240in}}{\pgfqpoint{2.996986in}{4.377427in}}%
\pgfpathcurveto{\pgfqpoint{2.989173in}{4.369613in}}{\pgfqpoint{2.984783in}{4.359014in}}{\pgfqpoint{2.984783in}{4.347964in}}%
\pgfpathcurveto{\pgfqpoint{2.984783in}{4.336914in}}{\pgfqpoint{2.989173in}{4.326315in}}{\pgfqpoint{2.996986in}{4.318501in}}%
\pgfpathcurveto{\pgfqpoint{3.004800in}{4.310688in}}{\pgfqpoint{3.015399in}{4.306297in}}{\pgfqpoint{3.026449in}{4.306297in}}%
\pgfpathclose%
\pgfusepath{stroke,fill}%
\end{pgfscope}%
\begin{pgfscope}%
\pgfpathrectangle{\pgfqpoint{0.481978in}{0.331635in}}{\pgfqpoint{9.300000in}{7.700000in}}%
\pgfusepath{clip}%
\pgfsetbuttcap%
\pgfsetroundjoin%
\definecolor{currentfill}{rgb}{0.870588,0.733333,0.607843}%
\pgfsetfillcolor{currentfill}%
\pgfsetlinewidth{0.481800pt}%
\definecolor{currentstroke}{rgb}{1.000000,1.000000,1.000000}%
\pgfsetstrokecolor{currentstroke}%
\pgfsetdash{}{0pt}%
\pgfpathmoveto{\pgfqpoint{6.761280in}{5.824363in}}%
\pgfpathcurveto{\pgfqpoint{6.772330in}{5.824363in}}{\pgfqpoint{6.782929in}{5.828753in}}{\pgfqpoint{6.790742in}{5.836567in}}%
\pgfpathcurveto{\pgfqpoint{6.798556in}{5.844381in}}{\pgfqpoint{6.802946in}{5.854980in}}{\pgfqpoint{6.802946in}{5.866030in}}%
\pgfpathcurveto{\pgfqpoint{6.802946in}{5.877080in}}{\pgfqpoint{6.798556in}{5.887679in}}{\pgfqpoint{6.790742in}{5.895492in}}%
\pgfpathcurveto{\pgfqpoint{6.782929in}{5.903306in}}{\pgfqpoint{6.772330in}{5.907696in}}{\pgfqpoint{6.761280in}{5.907696in}}%
\pgfpathcurveto{\pgfqpoint{6.750229in}{5.907696in}}{\pgfqpoint{6.739630in}{5.903306in}}{\pgfqpoint{6.731817in}{5.895492in}}%
\pgfpathcurveto{\pgfqpoint{6.724003in}{5.887679in}}{\pgfqpoint{6.719613in}{5.877080in}}{\pgfqpoint{6.719613in}{5.866030in}}%
\pgfpathcurveto{\pgfqpoint{6.719613in}{5.854980in}}{\pgfqpoint{6.724003in}{5.844381in}}{\pgfqpoint{6.731817in}{5.836567in}}%
\pgfpathcurveto{\pgfqpoint{6.739630in}{5.828753in}}{\pgfqpoint{6.750229in}{5.824363in}}{\pgfqpoint{6.761280in}{5.824363in}}%
\pgfpathclose%
\pgfusepath{stroke,fill}%
\end{pgfscope}%
\begin{pgfscope}%
\pgfpathrectangle{\pgfqpoint{0.481978in}{0.331635in}}{\pgfqpoint{9.300000in}{7.700000in}}%
\pgfusepath{clip}%
\pgfsetbuttcap%
\pgfsetroundjoin%
\definecolor{currentfill}{rgb}{0.870588,0.733333,0.607843}%
\pgfsetfillcolor{currentfill}%
\pgfsetlinewidth{0.481800pt}%
\definecolor{currentstroke}{rgb}{1.000000,1.000000,1.000000}%
\pgfsetstrokecolor{currentstroke}%
\pgfsetdash{}{0pt}%
\pgfpathmoveto{\pgfqpoint{2.445109in}{3.067913in}}%
\pgfpathcurveto{\pgfqpoint{2.456159in}{3.067913in}}{\pgfqpoint{2.466758in}{3.072303in}}{\pgfqpoint{2.474571in}{3.080117in}}%
\pgfpathcurveto{\pgfqpoint{2.482385in}{3.087930in}}{\pgfqpoint{2.486775in}{3.098529in}}{\pgfqpoint{2.486775in}{3.109579in}}%
\pgfpathcurveto{\pgfqpoint{2.486775in}{3.120630in}}{\pgfqpoint{2.482385in}{3.131229in}}{\pgfqpoint{2.474571in}{3.139042in}}%
\pgfpathcurveto{\pgfqpoint{2.466758in}{3.146856in}}{\pgfqpoint{2.456159in}{3.151246in}}{\pgfqpoint{2.445109in}{3.151246in}}%
\pgfpathcurveto{\pgfqpoint{2.434059in}{3.151246in}}{\pgfqpoint{2.423459in}{3.146856in}}{\pgfqpoint{2.415646in}{3.139042in}}%
\pgfpathcurveto{\pgfqpoint{2.407832in}{3.131229in}}{\pgfqpoint{2.403442in}{3.120630in}}{\pgfqpoint{2.403442in}{3.109579in}}%
\pgfpathcurveto{\pgfqpoint{2.403442in}{3.098529in}}{\pgfqpoint{2.407832in}{3.087930in}}{\pgfqpoint{2.415646in}{3.080117in}}%
\pgfpathcurveto{\pgfqpoint{2.423459in}{3.072303in}}{\pgfqpoint{2.434059in}{3.067913in}}{\pgfqpoint{2.445109in}{3.067913in}}%
\pgfpathclose%
\pgfusepath{stroke,fill}%
\end{pgfscope}%
\begin{pgfscope}%
\pgfpathrectangle{\pgfqpoint{0.481978in}{0.331635in}}{\pgfqpoint{9.300000in}{7.700000in}}%
\pgfusepath{clip}%
\pgfsetbuttcap%
\pgfsetroundjoin%
\definecolor{currentfill}{rgb}{0.870588,0.733333,0.607843}%
\pgfsetfillcolor{currentfill}%
\pgfsetlinewidth{0.481800pt}%
\definecolor{currentstroke}{rgb}{1.000000,1.000000,1.000000}%
\pgfsetstrokecolor{currentstroke}%
\pgfsetdash{}{0pt}%
\pgfpathmoveto{\pgfqpoint{2.556572in}{4.445719in}}%
\pgfpathcurveto{\pgfqpoint{2.567622in}{4.445719in}}{\pgfqpoint{2.578221in}{4.450109in}}{\pgfqpoint{2.586035in}{4.457922in}}%
\pgfpathcurveto{\pgfqpoint{2.593848in}{4.465736in}}{\pgfqpoint{2.598239in}{4.476335in}}{\pgfqpoint{2.598239in}{4.487385in}}%
\pgfpathcurveto{\pgfqpoint{2.598239in}{4.498435in}}{\pgfqpoint{2.593848in}{4.509034in}}{\pgfqpoint{2.586035in}{4.516848in}}%
\pgfpathcurveto{\pgfqpoint{2.578221in}{4.524662in}}{\pgfqpoint{2.567622in}{4.529052in}}{\pgfqpoint{2.556572in}{4.529052in}}%
\pgfpathcurveto{\pgfqpoint{2.545522in}{4.529052in}}{\pgfqpoint{2.534923in}{4.524662in}}{\pgfqpoint{2.527109in}{4.516848in}}%
\pgfpathcurveto{\pgfqpoint{2.519296in}{4.509034in}}{\pgfqpoint{2.514905in}{4.498435in}}{\pgfqpoint{2.514905in}{4.487385in}}%
\pgfpathcurveto{\pgfqpoint{2.514905in}{4.476335in}}{\pgfqpoint{2.519296in}{4.465736in}}{\pgfqpoint{2.527109in}{4.457922in}}%
\pgfpathcurveto{\pgfqpoint{2.534923in}{4.450109in}}{\pgfqpoint{2.545522in}{4.445719in}}{\pgfqpoint{2.556572in}{4.445719in}}%
\pgfpathclose%
\pgfusepath{stroke,fill}%
\end{pgfscope}%
\begin{pgfscope}%
\pgfpathrectangle{\pgfqpoint{0.481978in}{0.331635in}}{\pgfqpoint{9.300000in}{7.700000in}}%
\pgfusepath{clip}%
\pgfsetbuttcap%
\pgfsetroundjoin%
\definecolor{currentfill}{rgb}{0.870588,0.733333,0.607843}%
\pgfsetfillcolor{currentfill}%
\pgfsetlinewidth{0.481800pt}%
\definecolor{currentstroke}{rgb}{1.000000,1.000000,1.000000}%
\pgfsetstrokecolor{currentstroke}%
\pgfsetdash{}{0pt}%
\pgfpathmoveto{\pgfqpoint{3.956527in}{3.265885in}}%
\pgfpathcurveto{\pgfqpoint{3.967578in}{3.265885in}}{\pgfqpoint{3.978177in}{3.270275in}}{\pgfqpoint{3.985990in}{3.278089in}}%
\pgfpathcurveto{\pgfqpoint{3.993804in}{3.285902in}}{\pgfqpoint{3.998194in}{3.296501in}}{\pgfqpoint{3.998194in}{3.307551in}}%
\pgfpathcurveto{\pgfqpoint{3.998194in}{3.318602in}}{\pgfqpoint{3.993804in}{3.329201in}}{\pgfqpoint{3.985990in}{3.337014in}}%
\pgfpathcurveto{\pgfqpoint{3.978177in}{3.344828in}}{\pgfqpoint{3.967578in}{3.349218in}}{\pgfqpoint{3.956527in}{3.349218in}}%
\pgfpathcurveto{\pgfqpoint{3.945477in}{3.349218in}}{\pgfqpoint{3.934878in}{3.344828in}}{\pgfqpoint{3.927065in}{3.337014in}}%
\pgfpathcurveto{\pgfqpoint{3.919251in}{3.329201in}}{\pgfqpoint{3.914861in}{3.318602in}}{\pgfqpoint{3.914861in}{3.307551in}}%
\pgfpathcurveto{\pgfqpoint{3.914861in}{3.296501in}}{\pgfqpoint{3.919251in}{3.285902in}}{\pgfqpoint{3.927065in}{3.278089in}}%
\pgfpathcurveto{\pgfqpoint{3.934878in}{3.270275in}}{\pgfqpoint{3.945477in}{3.265885in}}{\pgfqpoint{3.956527in}{3.265885in}}%
\pgfpathclose%
\pgfusepath{stroke,fill}%
\end{pgfscope}%
\begin{pgfscope}%
\pgfpathrectangle{\pgfqpoint{0.481978in}{0.331635in}}{\pgfqpoint{9.300000in}{7.700000in}}%
\pgfusepath{clip}%
\pgfsetbuttcap%
\pgfsetroundjoin%
\definecolor{currentfill}{rgb}{0.870588,0.733333,0.607843}%
\pgfsetfillcolor{currentfill}%
\pgfsetlinewidth{0.481800pt}%
\definecolor{currentstroke}{rgb}{1.000000,1.000000,1.000000}%
\pgfsetstrokecolor{currentstroke}%
\pgfsetdash{}{0pt}%
\pgfpathmoveto{\pgfqpoint{2.593480in}{4.701301in}}%
\pgfpathcurveto{\pgfqpoint{2.604530in}{4.701301in}}{\pgfqpoint{2.615129in}{4.705692in}}{\pgfqpoint{2.622942in}{4.713505in}}%
\pgfpathcurveto{\pgfqpoint{2.630756in}{4.721319in}}{\pgfqpoint{2.635146in}{4.731918in}}{\pgfqpoint{2.635146in}{4.742968in}}%
\pgfpathcurveto{\pgfqpoint{2.635146in}{4.754018in}}{\pgfqpoint{2.630756in}{4.764617in}}{\pgfqpoint{2.622942in}{4.772431in}}%
\pgfpathcurveto{\pgfqpoint{2.615129in}{4.780244in}}{\pgfqpoint{2.604530in}{4.784635in}}{\pgfqpoint{2.593480in}{4.784635in}}%
\pgfpathcurveto{\pgfqpoint{2.582429in}{4.784635in}}{\pgfqpoint{2.571830in}{4.780244in}}{\pgfqpoint{2.564017in}{4.772431in}}%
\pgfpathcurveto{\pgfqpoint{2.556203in}{4.764617in}}{\pgfqpoint{2.551813in}{4.754018in}}{\pgfqpoint{2.551813in}{4.742968in}}%
\pgfpathcurveto{\pgfqpoint{2.551813in}{4.731918in}}{\pgfqpoint{2.556203in}{4.721319in}}{\pgfqpoint{2.564017in}{4.713505in}}%
\pgfpathcurveto{\pgfqpoint{2.571830in}{4.705692in}}{\pgfqpoint{2.582429in}{4.701301in}}{\pgfqpoint{2.593480in}{4.701301in}}%
\pgfpathclose%
\pgfusepath{stroke,fill}%
\end{pgfscope}%
\begin{pgfscope}%
\pgfpathrectangle{\pgfqpoint{0.481978in}{0.331635in}}{\pgfqpoint{9.300000in}{7.700000in}}%
\pgfusepath{clip}%
\pgfsetbuttcap%
\pgfsetroundjoin%
\definecolor{currentfill}{rgb}{0.870588,0.733333,0.607843}%
\pgfsetfillcolor{currentfill}%
\pgfsetlinewidth{0.481800pt}%
\definecolor{currentstroke}{rgb}{1.000000,1.000000,1.000000}%
\pgfsetstrokecolor{currentstroke}%
\pgfsetdash{}{0pt}%
\pgfpathmoveto{\pgfqpoint{1.766805in}{3.461988in}}%
\pgfpathcurveto{\pgfqpoint{1.777855in}{3.461988in}}{\pgfqpoint{1.788454in}{3.466378in}}{\pgfqpoint{1.796268in}{3.474192in}}%
\pgfpathcurveto{\pgfqpoint{1.804082in}{3.482006in}}{\pgfqpoint{1.808472in}{3.492605in}}{\pgfqpoint{1.808472in}{3.503655in}}%
\pgfpathcurveto{\pgfqpoint{1.808472in}{3.514705in}}{\pgfqpoint{1.804082in}{3.525304in}}{\pgfqpoint{1.796268in}{3.533118in}}%
\pgfpathcurveto{\pgfqpoint{1.788454in}{3.540931in}}{\pgfqpoint{1.777855in}{3.545321in}}{\pgfqpoint{1.766805in}{3.545321in}}%
\pgfpathcurveto{\pgfqpoint{1.755755in}{3.545321in}}{\pgfqpoint{1.745156in}{3.540931in}}{\pgfqpoint{1.737343in}{3.533118in}}%
\pgfpathcurveto{\pgfqpoint{1.729529in}{3.525304in}}{\pgfqpoint{1.725139in}{3.514705in}}{\pgfqpoint{1.725139in}{3.503655in}}%
\pgfpathcurveto{\pgfqpoint{1.725139in}{3.492605in}}{\pgfqpoint{1.729529in}{3.482006in}}{\pgfqpoint{1.737343in}{3.474192in}}%
\pgfpathcurveto{\pgfqpoint{1.745156in}{3.466378in}}{\pgfqpoint{1.755755in}{3.461988in}}{\pgfqpoint{1.766805in}{3.461988in}}%
\pgfpathclose%
\pgfusepath{stroke,fill}%
\end{pgfscope}%
\begin{pgfscope}%
\pgfpathrectangle{\pgfqpoint{0.481978in}{0.331635in}}{\pgfqpoint{9.300000in}{7.700000in}}%
\pgfusepath{clip}%
\pgfsetbuttcap%
\pgfsetroundjoin%
\definecolor{currentfill}{rgb}{0.870588,0.733333,0.607843}%
\pgfsetfillcolor{currentfill}%
\pgfsetlinewidth{0.481800pt}%
\definecolor{currentstroke}{rgb}{1.000000,1.000000,1.000000}%
\pgfsetstrokecolor{currentstroke}%
\pgfsetdash{}{0pt}%
\pgfpathmoveto{\pgfqpoint{4.977562in}{4.240982in}}%
\pgfpathcurveto{\pgfqpoint{4.988613in}{4.240982in}}{\pgfqpoint{4.999212in}{4.245373in}}{\pgfqpoint{5.007025in}{4.253186in}}%
\pgfpathcurveto{\pgfqpoint{5.014839in}{4.261000in}}{\pgfqpoint{5.019229in}{4.271599in}}{\pgfqpoint{5.019229in}{4.282649in}}%
\pgfpathcurveto{\pgfqpoint{5.019229in}{4.293699in}}{\pgfqpoint{5.014839in}{4.304298in}}{\pgfqpoint{5.007025in}{4.312112in}}%
\pgfpathcurveto{\pgfqpoint{4.999212in}{4.319926in}}{\pgfqpoint{4.988613in}{4.324316in}}{\pgfqpoint{4.977562in}{4.324316in}}%
\pgfpathcurveto{\pgfqpoint{4.966512in}{4.324316in}}{\pgfqpoint{4.955913in}{4.319926in}}{\pgfqpoint{4.948100in}{4.312112in}}%
\pgfpathcurveto{\pgfqpoint{4.940286in}{4.304298in}}{\pgfqpoint{4.935896in}{4.293699in}}{\pgfqpoint{4.935896in}{4.282649in}}%
\pgfpathcurveto{\pgfqpoint{4.935896in}{4.271599in}}{\pgfqpoint{4.940286in}{4.261000in}}{\pgfqpoint{4.948100in}{4.253186in}}%
\pgfpathcurveto{\pgfqpoint{4.955913in}{4.245373in}}{\pgfqpoint{4.966512in}{4.240982in}}{\pgfqpoint{4.977562in}{4.240982in}}%
\pgfpathclose%
\pgfusepath{stroke,fill}%
\end{pgfscope}%
\begin{pgfscope}%
\pgfpathrectangle{\pgfqpoint{0.481978in}{0.331635in}}{\pgfqpoint{9.300000in}{7.700000in}}%
\pgfusepath{clip}%
\pgfsetbuttcap%
\pgfsetroundjoin%
\definecolor{currentfill}{rgb}{0.870588,0.733333,0.607843}%
\pgfsetfillcolor{currentfill}%
\pgfsetlinewidth{0.481800pt}%
\definecolor{currentstroke}{rgb}{1.000000,1.000000,1.000000}%
\pgfsetstrokecolor{currentstroke}%
\pgfsetdash{}{0pt}%
\pgfpathmoveto{\pgfqpoint{5.616567in}{3.948372in}}%
\pgfpathcurveto{\pgfqpoint{5.627617in}{3.948372in}}{\pgfqpoint{5.638216in}{3.952762in}}{\pgfqpoint{5.646030in}{3.960576in}}%
\pgfpathcurveto{\pgfqpoint{5.653844in}{3.968390in}}{\pgfqpoint{5.658234in}{3.978989in}}{\pgfqpoint{5.658234in}{3.990039in}}%
\pgfpathcurveto{\pgfqpoint{5.658234in}{4.001089in}}{\pgfqpoint{5.653844in}{4.011688in}}{\pgfqpoint{5.646030in}{4.019502in}}%
\pgfpathcurveto{\pgfqpoint{5.638216in}{4.027315in}}{\pgfqpoint{5.627617in}{4.031705in}}{\pgfqpoint{5.616567in}{4.031705in}}%
\pgfpathcurveto{\pgfqpoint{5.605517in}{4.031705in}}{\pgfqpoint{5.594918in}{4.027315in}}{\pgfqpoint{5.587105in}{4.019502in}}%
\pgfpathcurveto{\pgfqpoint{5.579291in}{4.011688in}}{\pgfqpoint{5.574901in}{4.001089in}}{\pgfqpoint{5.574901in}{3.990039in}}%
\pgfpathcurveto{\pgfqpoint{5.574901in}{3.978989in}}{\pgfqpoint{5.579291in}{3.968390in}}{\pgfqpoint{5.587105in}{3.960576in}}%
\pgfpathcurveto{\pgfqpoint{5.594918in}{3.952762in}}{\pgfqpoint{5.605517in}{3.948372in}}{\pgfqpoint{5.616567in}{3.948372in}}%
\pgfpathclose%
\pgfusepath{stroke,fill}%
\end{pgfscope}%
\begin{pgfscope}%
\pgfpathrectangle{\pgfqpoint{0.481978in}{0.331635in}}{\pgfqpoint{9.300000in}{7.700000in}}%
\pgfusepath{clip}%
\pgfsetbuttcap%
\pgfsetroundjoin%
\definecolor{currentfill}{rgb}{0.870588,0.733333,0.607843}%
\pgfsetfillcolor{currentfill}%
\pgfsetlinewidth{0.481800pt}%
\definecolor{currentstroke}{rgb}{1.000000,1.000000,1.000000}%
\pgfsetstrokecolor{currentstroke}%
\pgfsetdash{}{0pt}%
\pgfpathmoveto{\pgfqpoint{2.747420in}{2.538850in}}%
\pgfpathcurveto{\pgfqpoint{2.758470in}{2.538850in}}{\pgfqpoint{2.769069in}{2.543241in}}{\pgfqpoint{2.776883in}{2.551054in}}%
\pgfpathcurveto{\pgfqpoint{2.784696in}{2.558868in}}{\pgfqpoint{2.789087in}{2.569467in}}{\pgfqpoint{2.789087in}{2.580517in}}%
\pgfpathcurveto{\pgfqpoint{2.789087in}{2.591567in}}{\pgfqpoint{2.784696in}{2.602166in}}{\pgfqpoint{2.776883in}{2.609980in}}%
\pgfpathcurveto{\pgfqpoint{2.769069in}{2.617793in}}{\pgfqpoint{2.758470in}{2.622184in}}{\pgfqpoint{2.747420in}{2.622184in}}%
\pgfpathcurveto{\pgfqpoint{2.736370in}{2.622184in}}{\pgfqpoint{2.725771in}{2.617793in}}{\pgfqpoint{2.717957in}{2.609980in}}%
\pgfpathcurveto{\pgfqpoint{2.710143in}{2.602166in}}{\pgfqpoint{2.705753in}{2.591567in}}{\pgfqpoint{2.705753in}{2.580517in}}%
\pgfpathcurveto{\pgfqpoint{2.705753in}{2.569467in}}{\pgfqpoint{2.710143in}{2.558868in}}{\pgfqpoint{2.717957in}{2.551054in}}%
\pgfpathcurveto{\pgfqpoint{2.725771in}{2.543241in}}{\pgfqpoint{2.736370in}{2.538850in}}{\pgfqpoint{2.747420in}{2.538850in}}%
\pgfpathclose%
\pgfusepath{stroke,fill}%
\end{pgfscope}%
\begin{pgfscope}%
\pgfpathrectangle{\pgfqpoint{0.481978in}{0.331635in}}{\pgfqpoint{9.300000in}{7.700000in}}%
\pgfusepath{clip}%
\pgfsetbuttcap%
\pgfsetroundjoin%
\definecolor{currentfill}{rgb}{0.870588,0.733333,0.607843}%
\pgfsetfillcolor{currentfill}%
\pgfsetlinewidth{0.481800pt}%
\definecolor{currentstroke}{rgb}{1.000000,1.000000,1.000000}%
\pgfsetstrokecolor{currentstroke}%
\pgfsetdash{}{0pt}%
\pgfpathmoveto{\pgfqpoint{3.710302in}{3.401243in}}%
\pgfpathcurveto{\pgfqpoint{3.721352in}{3.401243in}}{\pgfqpoint{3.731951in}{3.405633in}}{\pgfqpoint{3.739765in}{3.413447in}}%
\pgfpathcurveto{\pgfqpoint{3.747578in}{3.421260in}}{\pgfqpoint{3.751969in}{3.431859in}}{\pgfqpoint{3.751969in}{3.442909in}}%
\pgfpathcurveto{\pgfqpoint{3.751969in}{3.453960in}}{\pgfqpoint{3.747578in}{3.464559in}}{\pgfqpoint{3.739765in}{3.472372in}}%
\pgfpathcurveto{\pgfqpoint{3.731951in}{3.480186in}}{\pgfqpoint{3.721352in}{3.484576in}}{\pgfqpoint{3.710302in}{3.484576in}}%
\pgfpathcurveto{\pgfqpoint{3.699252in}{3.484576in}}{\pgfqpoint{3.688653in}{3.480186in}}{\pgfqpoint{3.680839in}{3.472372in}}%
\pgfpathcurveto{\pgfqpoint{3.673026in}{3.464559in}}{\pgfqpoint{3.668635in}{3.453960in}}{\pgfqpoint{3.668635in}{3.442909in}}%
\pgfpathcurveto{\pgfqpoint{3.668635in}{3.431859in}}{\pgfqpoint{3.673026in}{3.421260in}}{\pgfqpoint{3.680839in}{3.413447in}}%
\pgfpathcurveto{\pgfqpoint{3.688653in}{3.405633in}}{\pgfqpoint{3.699252in}{3.401243in}}{\pgfqpoint{3.710302in}{3.401243in}}%
\pgfpathclose%
\pgfusepath{stroke,fill}%
\end{pgfscope}%
\begin{pgfscope}%
\pgfpathrectangle{\pgfqpoint{0.481978in}{0.331635in}}{\pgfqpoint{9.300000in}{7.700000in}}%
\pgfusepath{clip}%
\pgfsetbuttcap%
\pgfsetroundjoin%
\definecolor{currentfill}{rgb}{0.870588,0.733333,0.607843}%
\pgfsetfillcolor{currentfill}%
\pgfsetlinewidth{0.481800pt}%
\definecolor{currentstroke}{rgb}{1.000000,1.000000,1.000000}%
\pgfsetstrokecolor{currentstroke}%
\pgfsetdash{}{0pt}%
\pgfpathmoveto{\pgfqpoint{2.149626in}{4.841912in}}%
\pgfpathcurveto{\pgfqpoint{2.160677in}{4.841912in}}{\pgfqpoint{2.171276in}{4.846302in}}{\pgfqpoint{2.179089in}{4.854116in}}%
\pgfpathcurveto{\pgfqpoint{2.186903in}{4.861929in}}{\pgfqpoint{2.191293in}{4.872528in}}{\pgfqpoint{2.191293in}{4.883578in}}%
\pgfpathcurveto{\pgfqpoint{2.191293in}{4.894629in}}{\pgfqpoint{2.186903in}{4.905228in}}{\pgfqpoint{2.179089in}{4.913041in}}%
\pgfpathcurveto{\pgfqpoint{2.171276in}{4.920855in}}{\pgfqpoint{2.160677in}{4.925245in}}{\pgfqpoint{2.149626in}{4.925245in}}%
\pgfpathcurveto{\pgfqpoint{2.138576in}{4.925245in}}{\pgfqpoint{2.127977in}{4.920855in}}{\pgfqpoint{2.120164in}{4.913041in}}%
\pgfpathcurveto{\pgfqpoint{2.112350in}{4.905228in}}{\pgfqpoint{2.107960in}{4.894629in}}{\pgfqpoint{2.107960in}{4.883578in}}%
\pgfpathcurveto{\pgfqpoint{2.107960in}{4.872528in}}{\pgfqpoint{2.112350in}{4.861929in}}{\pgfqpoint{2.120164in}{4.854116in}}%
\pgfpathcurveto{\pgfqpoint{2.127977in}{4.846302in}}{\pgfqpoint{2.138576in}{4.841912in}}{\pgfqpoint{2.149626in}{4.841912in}}%
\pgfpathclose%
\pgfusepath{stroke,fill}%
\end{pgfscope}%
\begin{pgfscope}%
\pgfpathrectangle{\pgfqpoint{0.481978in}{0.331635in}}{\pgfqpoint{9.300000in}{7.700000in}}%
\pgfusepath{clip}%
\pgfsetbuttcap%
\pgfsetroundjoin%
\definecolor{currentfill}{rgb}{0.870588,0.733333,0.607843}%
\pgfsetfillcolor{currentfill}%
\pgfsetlinewidth{0.481800pt}%
\definecolor{currentstroke}{rgb}{1.000000,1.000000,1.000000}%
\pgfsetstrokecolor{currentstroke}%
\pgfsetdash{}{0pt}%
\pgfpathmoveto{\pgfqpoint{2.077561in}{3.061756in}}%
\pgfpathcurveto{\pgfqpoint{2.088611in}{3.061756in}}{\pgfqpoint{2.099210in}{3.066146in}}{\pgfqpoint{2.107024in}{3.073960in}}%
\pgfpathcurveto{\pgfqpoint{2.114838in}{3.081773in}}{\pgfqpoint{2.119228in}{3.092372in}}{\pgfqpoint{2.119228in}{3.103422in}}%
\pgfpathcurveto{\pgfqpoint{2.119228in}{3.114473in}}{\pgfqpoint{2.114838in}{3.125072in}}{\pgfqpoint{2.107024in}{3.132885in}}%
\pgfpathcurveto{\pgfqpoint{2.099210in}{3.140699in}}{\pgfqpoint{2.088611in}{3.145089in}}{\pgfqpoint{2.077561in}{3.145089in}}%
\pgfpathcurveto{\pgfqpoint{2.066511in}{3.145089in}}{\pgfqpoint{2.055912in}{3.140699in}}{\pgfqpoint{2.048099in}{3.132885in}}%
\pgfpathcurveto{\pgfqpoint{2.040285in}{3.125072in}}{\pgfqpoint{2.035895in}{3.114473in}}{\pgfqpoint{2.035895in}{3.103422in}}%
\pgfpathcurveto{\pgfqpoint{2.035895in}{3.092372in}}{\pgfqpoint{2.040285in}{3.081773in}}{\pgfqpoint{2.048099in}{3.073960in}}%
\pgfpathcurveto{\pgfqpoint{2.055912in}{3.066146in}}{\pgfqpoint{2.066511in}{3.061756in}}{\pgfqpoint{2.077561in}{3.061756in}}%
\pgfpathclose%
\pgfusepath{stroke,fill}%
\end{pgfscope}%
\begin{pgfscope}%
\pgfpathrectangle{\pgfqpoint{0.481978in}{0.331635in}}{\pgfqpoint{9.300000in}{7.700000in}}%
\pgfusepath{clip}%
\pgfsetbuttcap%
\pgfsetroundjoin%
\definecolor{currentfill}{rgb}{0.870588,0.733333,0.607843}%
\pgfsetfillcolor{currentfill}%
\pgfsetlinewidth{0.481800pt}%
\definecolor{currentstroke}{rgb}{1.000000,1.000000,1.000000}%
\pgfsetstrokecolor{currentstroke}%
\pgfsetdash{}{0pt}%
\pgfpathmoveto{\pgfqpoint{1.957320in}{5.188017in}}%
\pgfpathcurveto{\pgfqpoint{1.968370in}{5.188017in}}{\pgfqpoint{1.978969in}{5.192407in}}{\pgfqpoint{1.986783in}{5.200221in}}%
\pgfpathcurveto{\pgfqpoint{1.994596in}{5.208035in}}{\pgfqpoint{1.998987in}{5.218634in}}{\pgfqpoint{1.998987in}{5.229684in}}%
\pgfpathcurveto{\pgfqpoint{1.998987in}{5.240734in}}{\pgfqpoint{1.994596in}{5.251333in}}{\pgfqpoint{1.986783in}{5.259147in}}%
\pgfpathcurveto{\pgfqpoint{1.978969in}{5.266960in}}{\pgfqpoint{1.968370in}{5.271351in}}{\pgfqpoint{1.957320in}{5.271351in}}%
\pgfpathcurveto{\pgfqpoint{1.946270in}{5.271351in}}{\pgfqpoint{1.935671in}{5.266960in}}{\pgfqpoint{1.927857in}{5.259147in}}%
\pgfpathcurveto{\pgfqpoint{1.920044in}{5.251333in}}{\pgfqpoint{1.915653in}{5.240734in}}{\pgfqpoint{1.915653in}{5.229684in}}%
\pgfpathcurveto{\pgfqpoint{1.915653in}{5.218634in}}{\pgfqpoint{1.920044in}{5.208035in}}{\pgfqpoint{1.927857in}{5.200221in}}%
\pgfpathcurveto{\pgfqpoint{1.935671in}{5.192407in}}{\pgfqpoint{1.946270in}{5.188017in}}{\pgfqpoint{1.957320in}{5.188017in}}%
\pgfpathclose%
\pgfusepath{stroke,fill}%
\end{pgfscope}%
\begin{pgfscope}%
\pgfpathrectangle{\pgfqpoint{0.481978in}{0.331635in}}{\pgfqpoint{9.300000in}{7.700000in}}%
\pgfusepath{clip}%
\pgfsetbuttcap%
\pgfsetroundjoin%
\definecolor{currentfill}{rgb}{0.870588,0.733333,0.607843}%
\pgfsetfillcolor{currentfill}%
\pgfsetlinewidth{0.481800pt}%
\definecolor{currentstroke}{rgb}{1.000000,1.000000,1.000000}%
\pgfsetstrokecolor{currentstroke}%
\pgfsetdash{}{0pt}%
\pgfpathmoveto{\pgfqpoint{2.660918in}{3.508920in}}%
\pgfpathcurveto{\pgfqpoint{2.671968in}{3.508920in}}{\pgfqpoint{2.682567in}{3.513311in}}{\pgfqpoint{2.690380in}{3.521124in}}%
\pgfpathcurveto{\pgfqpoint{2.698194in}{3.528938in}}{\pgfqpoint{2.702584in}{3.539537in}}{\pgfqpoint{2.702584in}{3.550587in}}%
\pgfpathcurveto{\pgfqpoint{2.702584in}{3.561637in}}{\pgfqpoint{2.698194in}{3.572236in}}{\pgfqpoint{2.690380in}{3.580050in}}%
\pgfpathcurveto{\pgfqpoint{2.682567in}{3.587864in}}{\pgfqpoint{2.671968in}{3.592254in}}{\pgfqpoint{2.660918in}{3.592254in}}%
\pgfpathcurveto{\pgfqpoint{2.649868in}{3.592254in}}{\pgfqpoint{2.639269in}{3.587864in}}{\pgfqpoint{2.631455in}{3.580050in}}%
\pgfpathcurveto{\pgfqpoint{2.623641in}{3.572236in}}{\pgfqpoint{2.619251in}{3.561637in}}{\pgfqpoint{2.619251in}{3.550587in}}%
\pgfpathcurveto{\pgfqpoint{2.619251in}{3.539537in}}{\pgfqpoint{2.623641in}{3.528938in}}{\pgfqpoint{2.631455in}{3.521124in}}%
\pgfpathcurveto{\pgfqpoint{2.639269in}{3.513311in}}{\pgfqpoint{2.649868in}{3.508920in}}{\pgfqpoint{2.660918in}{3.508920in}}%
\pgfpathclose%
\pgfusepath{stroke,fill}%
\end{pgfscope}%
\begin{pgfscope}%
\pgfpathrectangle{\pgfqpoint{0.481978in}{0.331635in}}{\pgfqpoint{9.300000in}{7.700000in}}%
\pgfusepath{clip}%
\pgfsetbuttcap%
\pgfsetroundjoin%
\definecolor{currentfill}{rgb}{0.870588,0.733333,0.607843}%
\pgfsetfillcolor{currentfill}%
\pgfsetlinewidth{0.481800pt}%
\definecolor{currentstroke}{rgb}{1.000000,1.000000,1.000000}%
\pgfsetstrokecolor{currentstroke}%
\pgfsetdash{}{0pt}%
\pgfpathmoveto{\pgfqpoint{7.871019in}{5.508865in}}%
\pgfpathcurveto{\pgfqpoint{7.882069in}{5.508865in}}{\pgfqpoint{7.892668in}{5.513255in}}{\pgfqpoint{7.900481in}{5.521069in}}%
\pgfpathcurveto{\pgfqpoint{7.908295in}{5.528882in}}{\pgfqpoint{7.912685in}{5.539481in}}{\pgfqpoint{7.912685in}{5.550531in}}%
\pgfpathcurveto{\pgfqpoint{7.912685in}{5.561582in}}{\pgfqpoint{7.908295in}{5.572181in}}{\pgfqpoint{7.900481in}{5.579994in}}%
\pgfpathcurveto{\pgfqpoint{7.892668in}{5.587808in}}{\pgfqpoint{7.882069in}{5.592198in}}{\pgfqpoint{7.871019in}{5.592198in}}%
\pgfpathcurveto{\pgfqpoint{7.859968in}{5.592198in}}{\pgfqpoint{7.849369in}{5.587808in}}{\pgfqpoint{7.841556in}{5.579994in}}%
\pgfpathcurveto{\pgfqpoint{7.833742in}{5.572181in}}{\pgfqpoint{7.829352in}{5.561582in}}{\pgfqpoint{7.829352in}{5.550531in}}%
\pgfpathcurveto{\pgfqpoint{7.829352in}{5.539481in}}{\pgfqpoint{7.833742in}{5.528882in}}{\pgfqpoint{7.841556in}{5.521069in}}%
\pgfpathcurveto{\pgfqpoint{7.849369in}{5.513255in}}{\pgfqpoint{7.859968in}{5.508865in}}{\pgfqpoint{7.871019in}{5.508865in}}%
\pgfpathclose%
\pgfusepath{stroke,fill}%
\end{pgfscope}%
\begin{pgfscope}%
\pgfpathrectangle{\pgfqpoint{0.481978in}{0.331635in}}{\pgfqpoint{9.300000in}{7.700000in}}%
\pgfusepath{clip}%
\pgfsetbuttcap%
\pgfsetroundjoin%
\definecolor{currentfill}{rgb}{0.870588,0.733333,0.607843}%
\pgfsetfillcolor{currentfill}%
\pgfsetlinewidth{0.481800pt}%
\definecolor{currentstroke}{rgb}{1.000000,1.000000,1.000000}%
\pgfsetstrokecolor{currentstroke}%
\pgfsetdash{}{0pt}%
\pgfpathmoveto{\pgfqpoint{5.367711in}{4.135078in}}%
\pgfpathcurveto{\pgfqpoint{5.378761in}{4.135078in}}{\pgfqpoint{5.389360in}{4.139468in}}{\pgfqpoint{5.397173in}{4.147281in}}%
\pgfpathcurveto{\pgfqpoint{5.404987in}{4.155095in}}{\pgfqpoint{5.409377in}{4.165694in}}{\pgfqpoint{5.409377in}{4.176744in}}%
\pgfpathcurveto{\pgfqpoint{5.409377in}{4.187794in}}{\pgfqpoint{5.404987in}{4.198393in}}{\pgfqpoint{5.397173in}{4.206207in}}%
\pgfpathcurveto{\pgfqpoint{5.389360in}{4.214021in}}{\pgfqpoint{5.378761in}{4.218411in}}{\pgfqpoint{5.367711in}{4.218411in}}%
\pgfpathcurveto{\pgfqpoint{5.356660in}{4.218411in}}{\pgfqpoint{5.346061in}{4.214021in}}{\pgfqpoint{5.338248in}{4.206207in}}%
\pgfpathcurveto{\pgfqpoint{5.330434in}{4.198393in}}{\pgfqpoint{5.326044in}{4.187794in}}{\pgfqpoint{5.326044in}{4.176744in}}%
\pgfpathcurveto{\pgfqpoint{5.326044in}{4.165694in}}{\pgfqpoint{5.330434in}{4.155095in}}{\pgfqpoint{5.338248in}{4.147281in}}%
\pgfpathcurveto{\pgfqpoint{5.346061in}{4.139468in}}{\pgfqpoint{5.356660in}{4.135078in}}{\pgfqpoint{5.367711in}{4.135078in}}%
\pgfpathclose%
\pgfusepath{stroke,fill}%
\end{pgfscope}%
\begin{pgfscope}%
\pgfpathrectangle{\pgfqpoint{0.481978in}{0.331635in}}{\pgfqpoint{9.300000in}{7.700000in}}%
\pgfusepath{clip}%
\pgfsetbuttcap%
\pgfsetroundjoin%
\definecolor{currentfill}{rgb}{0.870588,0.733333,0.607843}%
\pgfsetfillcolor{currentfill}%
\pgfsetlinewidth{0.481800pt}%
\definecolor{currentstroke}{rgb}{1.000000,1.000000,1.000000}%
\pgfsetstrokecolor{currentstroke}%
\pgfsetdash{}{0pt}%
\pgfpathmoveto{\pgfqpoint{2.321412in}{4.202976in}}%
\pgfpathcurveto{\pgfqpoint{2.332462in}{4.202976in}}{\pgfqpoint{2.343061in}{4.207366in}}{\pgfqpoint{2.350875in}{4.215179in}}%
\pgfpathcurveto{\pgfqpoint{2.358689in}{4.222993in}}{\pgfqpoint{2.363079in}{4.233592in}}{\pgfqpoint{2.363079in}{4.244642in}}%
\pgfpathcurveto{\pgfqpoint{2.363079in}{4.255692in}}{\pgfqpoint{2.358689in}{4.266291in}}{\pgfqpoint{2.350875in}{4.274105in}}%
\pgfpathcurveto{\pgfqpoint{2.343061in}{4.281919in}}{\pgfqpoint{2.332462in}{4.286309in}}{\pgfqpoint{2.321412in}{4.286309in}}%
\pgfpathcurveto{\pgfqpoint{2.310362in}{4.286309in}}{\pgfqpoint{2.299763in}{4.281919in}}{\pgfqpoint{2.291949in}{4.274105in}}%
\pgfpathcurveto{\pgfqpoint{2.284136in}{4.266291in}}{\pgfqpoint{2.279745in}{4.255692in}}{\pgfqpoint{2.279745in}{4.244642in}}%
\pgfpathcurveto{\pgfqpoint{2.279745in}{4.233592in}}{\pgfqpoint{2.284136in}{4.222993in}}{\pgfqpoint{2.291949in}{4.215179in}}%
\pgfpathcurveto{\pgfqpoint{2.299763in}{4.207366in}}{\pgfqpoint{2.310362in}{4.202976in}}{\pgfqpoint{2.321412in}{4.202976in}}%
\pgfpathclose%
\pgfusepath{stroke,fill}%
\end{pgfscope}%
\begin{pgfscope}%
\pgfpathrectangle{\pgfqpoint{0.481978in}{0.331635in}}{\pgfqpoint{9.300000in}{7.700000in}}%
\pgfusepath{clip}%
\pgfsetbuttcap%
\pgfsetroundjoin%
\definecolor{currentfill}{rgb}{0.870588,0.733333,0.607843}%
\pgfsetfillcolor{currentfill}%
\pgfsetlinewidth{0.481800pt}%
\definecolor{currentstroke}{rgb}{1.000000,1.000000,1.000000}%
\pgfsetstrokecolor{currentstroke}%
\pgfsetdash{}{0pt}%
\pgfpathmoveto{\pgfqpoint{7.119603in}{6.261982in}}%
\pgfpathcurveto{\pgfqpoint{7.130653in}{6.261982in}}{\pgfqpoint{7.141252in}{6.266372in}}{\pgfqpoint{7.149065in}{6.274186in}}%
\pgfpathcurveto{\pgfqpoint{7.156879in}{6.282000in}}{\pgfqpoint{7.161269in}{6.292599in}}{\pgfqpoint{7.161269in}{6.303649in}}%
\pgfpathcurveto{\pgfqpoint{7.161269in}{6.314699in}}{\pgfqpoint{7.156879in}{6.325298in}}{\pgfqpoint{7.149065in}{6.333111in}}%
\pgfpathcurveto{\pgfqpoint{7.141252in}{6.340925in}}{\pgfqpoint{7.130653in}{6.345315in}}{\pgfqpoint{7.119603in}{6.345315in}}%
\pgfpathcurveto{\pgfqpoint{7.108553in}{6.345315in}}{\pgfqpoint{7.097954in}{6.340925in}}{\pgfqpoint{7.090140in}{6.333111in}}%
\pgfpathcurveto{\pgfqpoint{7.082326in}{6.325298in}}{\pgfqpoint{7.077936in}{6.314699in}}{\pgfqpoint{7.077936in}{6.303649in}}%
\pgfpathcurveto{\pgfqpoint{7.077936in}{6.292599in}}{\pgfqpoint{7.082326in}{6.282000in}}{\pgfqpoint{7.090140in}{6.274186in}}%
\pgfpathcurveto{\pgfqpoint{7.097954in}{6.266372in}}{\pgfqpoint{7.108553in}{6.261982in}}{\pgfqpoint{7.119603in}{6.261982in}}%
\pgfpathclose%
\pgfusepath{stroke,fill}%
\end{pgfscope}%
\begin{pgfscope}%
\pgfpathrectangle{\pgfqpoint{0.481978in}{0.331635in}}{\pgfqpoint{9.300000in}{7.700000in}}%
\pgfusepath{clip}%
\pgfsetbuttcap%
\pgfsetroundjoin%
\definecolor{currentfill}{rgb}{0.870588,0.733333,0.607843}%
\pgfsetfillcolor{currentfill}%
\pgfsetlinewidth{0.481800pt}%
\definecolor{currentstroke}{rgb}{1.000000,1.000000,1.000000}%
\pgfsetstrokecolor{currentstroke}%
\pgfsetdash{}{0pt}%
\pgfpathmoveto{\pgfqpoint{6.487284in}{5.462464in}}%
\pgfpathcurveto{\pgfqpoint{6.498334in}{5.462464in}}{\pgfqpoint{6.508933in}{5.466854in}}{\pgfqpoint{6.516747in}{5.474668in}}%
\pgfpathcurveto{\pgfqpoint{6.524560in}{5.482482in}}{\pgfqpoint{6.528951in}{5.493081in}}{\pgfqpoint{6.528951in}{5.504131in}}%
\pgfpathcurveto{\pgfqpoint{6.528951in}{5.515181in}}{\pgfqpoint{6.524560in}{5.525780in}}{\pgfqpoint{6.516747in}{5.533594in}}%
\pgfpathcurveto{\pgfqpoint{6.508933in}{5.541407in}}{\pgfqpoint{6.498334in}{5.545798in}}{\pgfqpoint{6.487284in}{5.545798in}}%
\pgfpathcurveto{\pgfqpoint{6.476234in}{5.545798in}}{\pgfqpoint{6.465635in}{5.541407in}}{\pgfqpoint{6.457821in}{5.533594in}}%
\pgfpathcurveto{\pgfqpoint{6.450008in}{5.525780in}}{\pgfqpoint{6.445617in}{5.515181in}}{\pgfqpoint{6.445617in}{5.504131in}}%
\pgfpathcurveto{\pgfqpoint{6.445617in}{5.493081in}}{\pgfqpoint{6.450008in}{5.482482in}}{\pgfqpoint{6.457821in}{5.474668in}}%
\pgfpathcurveto{\pgfqpoint{6.465635in}{5.466854in}}{\pgfqpoint{6.476234in}{5.462464in}}{\pgfqpoint{6.487284in}{5.462464in}}%
\pgfpathclose%
\pgfusepath{stroke,fill}%
\end{pgfscope}%
\begin{pgfscope}%
\pgfpathrectangle{\pgfqpoint{0.481978in}{0.331635in}}{\pgfqpoint{9.300000in}{7.700000in}}%
\pgfusepath{clip}%
\pgfsetbuttcap%
\pgfsetroundjoin%
\definecolor{currentfill}{rgb}{0.870588,0.733333,0.607843}%
\pgfsetfillcolor{currentfill}%
\pgfsetlinewidth{0.481800pt}%
\definecolor{currentstroke}{rgb}{1.000000,1.000000,1.000000}%
\pgfsetstrokecolor{currentstroke}%
\pgfsetdash{}{0pt}%
\pgfpathmoveto{\pgfqpoint{8.920274in}{3.513533in}}%
\pgfpathcurveto{\pgfqpoint{8.931324in}{3.513533in}}{\pgfqpoint{8.941923in}{3.517923in}}{\pgfqpoint{8.949737in}{3.525737in}}%
\pgfpathcurveto{\pgfqpoint{8.957550in}{3.533551in}}{\pgfqpoint{8.961941in}{3.544150in}}{\pgfqpoint{8.961941in}{3.555200in}}%
\pgfpathcurveto{\pgfqpoint{8.961941in}{3.566250in}}{\pgfqpoint{8.957550in}{3.576849in}}{\pgfqpoint{8.949737in}{3.584663in}}%
\pgfpathcurveto{\pgfqpoint{8.941923in}{3.592476in}}{\pgfqpoint{8.931324in}{3.596867in}}{\pgfqpoint{8.920274in}{3.596867in}}%
\pgfpathcurveto{\pgfqpoint{8.909224in}{3.596867in}}{\pgfqpoint{8.898625in}{3.592476in}}{\pgfqpoint{8.890811in}{3.584663in}}%
\pgfpathcurveto{\pgfqpoint{8.882998in}{3.576849in}}{\pgfqpoint{8.878607in}{3.566250in}}{\pgfqpoint{8.878607in}{3.555200in}}%
\pgfpathcurveto{\pgfqpoint{8.878607in}{3.544150in}}{\pgfqpoint{8.882998in}{3.533551in}}{\pgfqpoint{8.890811in}{3.525737in}}%
\pgfpathcurveto{\pgfqpoint{8.898625in}{3.517923in}}{\pgfqpoint{8.909224in}{3.513533in}}{\pgfqpoint{8.920274in}{3.513533in}}%
\pgfpathclose%
\pgfusepath{stroke,fill}%
\end{pgfscope}%
\begin{pgfscope}%
\pgfpathrectangle{\pgfqpoint{0.481978in}{0.331635in}}{\pgfqpoint{9.300000in}{7.700000in}}%
\pgfusepath{clip}%
\pgfsetbuttcap%
\pgfsetroundjoin%
\definecolor{currentfill}{rgb}{0.870588,0.733333,0.607843}%
\pgfsetfillcolor{currentfill}%
\pgfsetlinewidth{0.481800pt}%
\definecolor{currentstroke}{rgb}{1.000000,1.000000,1.000000}%
\pgfsetstrokecolor{currentstroke}%
\pgfsetdash{}{0pt}%
\pgfpathmoveto{\pgfqpoint{7.303090in}{6.103993in}}%
\pgfpathcurveto{\pgfqpoint{7.314140in}{6.103993in}}{\pgfqpoint{7.324739in}{6.108384in}}{\pgfqpoint{7.332553in}{6.116197in}}%
\pgfpathcurveto{\pgfqpoint{7.340366in}{6.124011in}}{\pgfqpoint{7.344757in}{6.134610in}}{\pgfqpoint{7.344757in}{6.145660in}}%
\pgfpathcurveto{\pgfqpoint{7.344757in}{6.156710in}}{\pgfqpoint{7.340366in}{6.167309in}}{\pgfqpoint{7.332553in}{6.175123in}}%
\pgfpathcurveto{\pgfqpoint{7.324739in}{6.182936in}}{\pgfqpoint{7.314140in}{6.187327in}}{\pgfqpoint{7.303090in}{6.187327in}}%
\pgfpathcurveto{\pgfqpoint{7.292040in}{6.187327in}}{\pgfqpoint{7.281441in}{6.182936in}}{\pgfqpoint{7.273627in}{6.175123in}}%
\pgfpathcurveto{\pgfqpoint{7.265814in}{6.167309in}}{\pgfqpoint{7.261423in}{6.156710in}}{\pgfqpoint{7.261423in}{6.145660in}}%
\pgfpathcurveto{\pgfqpoint{7.261423in}{6.134610in}}{\pgfqpoint{7.265814in}{6.124011in}}{\pgfqpoint{7.273627in}{6.116197in}}%
\pgfpathcurveto{\pgfqpoint{7.281441in}{6.108384in}}{\pgfqpoint{7.292040in}{6.103993in}}{\pgfqpoint{7.303090in}{6.103993in}}%
\pgfpathclose%
\pgfusepath{stroke,fill}%
\end{pgfscope}%
\begin{pgfscope}%
\pgfpathrectangle{\pgfqpoint{0.481978in}{0.331635in}}{\pgfqpoint{9.300000in}{7.700000in}}%
\pgfusepath{clip}%
\pgfsetbuttcap%
\pgfsetroundjoin%
\definecolor{currentfill}{rgb}{0.870588,0.733333,0.607843}%
\pgfsetfillcolor{currentfill}%
\pgfsetlinewidth{0.481800pt}%
\definecolor{currentstroke}{rgb}{1.000000,1.000000,1.000000}%
\pgfsetstrokecolor{currentstroke}%
\pgfsetdash{}{0pt}%
\pgfpathmoveto{\pgfqpoint{9.189203in}{4.278681in}}%
\pgfpathcurveto{\pgfqpoint{9.200253in}{4.278681in}}{\pgfqpoint{9.210852in}{4.283071in}}{\pgfqpoint{9.218666in}{4.290885in}}%
\pgfpathcurveto{\pgfqpoint{9.226480in}{4.298699in}}{\pgfqpoint{9.230870in}{4.309298in}}{\pgfqpoint{9.230870in}{4.320348in}}%
\pgfpathcurveto{\pgfqpoint{9.230870in}{4.331398in}}{\pgfqpoint{9.226480in}{4.341997in}}{\pgfqpoint{9.218666in}{4.349811in}}%
\pgfpathcurveto{\pgfqpoint{9.210852in}{4.357624in}}{\pgfqpoint{9.200253in}{4.362014in}}{\pgfqpoint{9.189203in}{4.362014in}}%
\pgfpathcurveto{\pgfqpoint{9.178153in}{4.362014in}}{\pgfqpoint{9.167554in}{4.357624in}}{\pgfqpoint{9.159740in}{4.349811in}}%
\pgfpathcurveto{\pgfqpoint{9.151927in}{4.341997in}}{\pgfqpoint{9.147537in}{4.331398in}}{\pgfqpoint{9.147537in}{4.320348in}}%
\pgfpathcurveto{\pgfqpoint{9.147537in}{4.309298in}}{\pgfqpoint{9.151927in}{4.298699in}}{\pgfqpoint{9.159740in}{4.290885in}}%
\pgfpathcurveto{\pgfqpoint{9.167554in}{4.283071in}}{\pgfqpoint{9.178153in}{4.278681in}}{\pgfqpoint{9.189203in}{4.278681in}}%
\pgfpathclose%
\pgfusepath{stroke,fill}%
\end{pgfscope}%
\begin{pgfscope}%
\pgfpathrectangle{\pgfqpoint{0.481978in}{0.331635in}}{\pgfqpoint{9.300000in}{7.700000in}}%
\pgfusepath{clip}%
\pgfsetbuttcap%
\pgfsetroundjoin%
\definecolor{currentfill}{rgb}{0.631373,0.788235,0.956863}%
\pgfsetfillcolor{currentfill}%
\pgfsetlinewidth{1.003750pt}%
\definecolor{currentstroke}{rgb}{0.631373,0.788235,0.956863}%
\pgfsetstrokecolor{currentstroke}%
\pgfsetdash{}{0pt}%
\pgfsys@defobject{currentmarker}{\pgfqpoint{-0.041667in}{-0.041667in}}{\pgfqpoint{0.041667in}{0.041667in}}{%
\pgfpathmoveto{\pgfqpoint{0.000000in}{-0.041667in}}%
\pgfpathcurveto{\pgfqpoint{0.011050in}{-0.041667in}}{\pgfqpoint{0.021649in}{-0.037276in}}{\pgfqpoint{0.029463in}{-0.029463in}}%
\pgfpathcurveto{\pgfqpoint{0.037276in}{-0.021649in}}{\pgfqpoint{0.041667in}{-0.011050in}}{\pgfqpoint{0.041667in}{0.000000in}}%
\pgfpathcurveto{\pgfqpoint{0.041667in}{0.011050in}}{\pgfqpoint{0.037276in}{0.021649in}}{\pgfqpoint{0.029463in}{0.029463in}}%
\pgfpathcurveto{\pgfqpoint{0.021649in}{0.037276in}}{\pgfqpoint{0.011050in}{0.041667in}}{\pgfqpoint{0.000000in}{0.041667in}}%
\pgfpathcurveto{\pgfqpoint{-0.011050in}{0.041667in}}{\pgfqpoint{-0.021649in}{0.037276in}}{\pgfqpoint{-0.029463in}{0.029463in}}%
\pgfpathcurveto{\pgfqpoint{-0.037276in}{0.021649in}}{\pgfqpoint{-0.041667in}{0.011050in}}{\pgfqpoint{-0.041667in}{0.000000in}}%
\pgfpathcurveto{\pgfqpoint{-0.041667in}{-0.011050in}}{\pgfqpoint{-0.037276in}{-0.021649in}}{\pgfqpoint{-0.029463in}{-0.029463in}}%
\pgfpathcurveto{\pgfqpoint{-0.021649in}{-0.037276in}}{\pgfqpoint{-0.011050in}{-0.041667in}}{\pgfqpoint{0.000000in}{-0.041667in}}%
\pgfpathclose%
\pgfusepath{stroke,fill}%
}%
\end{pgfscope}%
\begin{pgfscope}%
\pgfpathrectangle{\pgfqpoint{0.481978in}{0.331635in}}{\pgfqpoint{9.300000in}{7.700000in}}%
\pgfusepath{clip}%
\pgfsetbuttcap%
\pgfsetroundjoin%
\definecolor{currentfill}{rgb}{1.000000,0.705882,0.509804}%
\pgfsetfillcolor{currentfill}%
\pgfsetlinewidth{1.003750pt}%
\definecolor{currentstroke}{rgb}{1.000000,0.705882,0.509804}%
\pgfsetstrokecolor{currentstroke}%
\pgfsetdash{}{0pt}%
\pgfsys@defobject{currentmarker}{\pgfqpoint{-0.041667in}{-0.041667in}}{\pgfqpoint{0.041667in}{0.041667in}}{%
\pgfpathmoveto{\pgfqpoint{0.000000in}{-0.041667in}}%
\pgfpathcurveto{\pgfqpoint{0.011050in}{-0.041667in}}{\pgfqpoint{0.021649in}{-0.037276in}}{\pgfqpoint{0.029463in}{-0.029463in}}%
\pgfpathcurveto{\pgfqpoint{0.037276in}{-0.021649in}}{\pgfqpoint{0.041667in}{-0.011050in}}{\pgfqpoint{0.041667in}{0.000000in}}%
\pgfpathcurveto{\pgfqpoint{0.041667in}{0.011050in}}{\pgfqpoint{0.037276in}{0.021649in}}{\pgfqpoint{0.029463in}{0.029463in}}%
\pgfpathcurveto{\pgfqpoint{0.021649in}{0.037276in}}{\pgfqpoint{0.011050in}{0.041667in}}{\pgfqpoint{0.000000in}{0.041667in}}%
\pgfpathcurveto{\pgfqpoint{-0.011050in}{0.041667in}}{\pgfqpoint{-0.021649in}{0.037276in}}{\pgfqpoint{-0.029463in}{0.029463in}}%
\pgfpathcurveto{\pgfqpoint{-0.037276in}{0.021649in}}{\pgfqpoint{-0.041667in}{0.011050in}}{\pgfqpoint{-0.041667in}{0.000000in}}%
\pgfpathcurveto{\pgfqpoint{-0.041667in}{-0.011050in}}{\pgfqpoint{-0.037276in}{-0.021649in}}{\pgfqpoint{-0.029463in}{-0.029463in}}%
\pgfpathcurveto{\pgfqpoint{-0.021649in}{-0.037276in}}{\pgfqpoint{-0.011050in}{-0.041667in}}{\pgfqpoint{0.000000in}{-0.041667in}}%
\pgfpathclose%
\pgfusepath{stroke,fill}%
}%
\end{pgfscope}%
\begin{pgfscope}%
\pgfpathrectangle{\pgfqpoint{0.481978in}{0.331635in}}{\pgfqpoint{9.300000in}{7.700000in}}%
\pgfusepath{clip}%
\pgfsetbuttcap%
\pgfsetroundjoin%
\definecolor{currentfill}{rgb}{0.552941,0.898039,0.631373}%
\pgfsetfillcolor{currentfill}%
\pgfsetlinewidth{1.003750pt}%
\definecolor{currentstroke}{rgb}{0.552941,0.898039,0.631373}%
\pgfsetstrokecolor{currentstroke}%
\pgfsetdash{}{0pt}%
\pgfsys@defobject{currentmarker}{\pgfqpoint{-0.041667in}{-0.041667in}}{\pgfqpoint{0.041667in}{0.041667in}}{%
\pgfpathmoveto{\pgfqpoint{0.000000in}{-0.041667in}}%
\pgfpathcurveto{\pgfqpoint{0.011050in}{-0.041667in}}{\pgfqpoint{0.021649in}{-0.037276in}}{\pgfqpoint{0.029463in}{-0.029463in}}%
\pgfpathcurveto{\pgfqpoint{0.037276in}{-0.021649in}}{\pgfqpoint{0.041667in}{-0.011050in}}{\pgfqpoint{0.041667in}{0.000000in}}%
\pgfpathcurveto{\pgfqpoint{0.041667in}{0.011050in}}{\pgfqpoint{0.037276in}{0.021649in}}{\pgfqpoint{0.029463in}{0.029463in}}%
\pgfpathcurveto{\pgfqpoint{0.021649in}{0.037276in}}{\pgfqpoint{0.011050in}{0.041667in}}{\pgfqpoint{0.000000in}{0.041667in}}%
\pgfpathcurveto{\pgfqpoint{-0.011050in}{0.041667in}}{\pgfqpoint{-0.021649in}{0.037276in}}{\pgfqpoint{-0.029463in}{0.029463in}}%
\pgfpathcurveto{\pgfqpoint{-0.037276in}{0.021649in}}{\pgfqpoint{-0.041667in}{0.011050in}}{\pgfqpoint{-0.041667in}{0.000000in}}%
\pgfpathcurveto{\pgfqpoint{-0.041667in}{-0.011050in}}{\pgfqpoint{-0.037276in}{-0.021649in}}{\pgfqpoint{-0.029463in}{-0.029463in}}%
\pgfpathcurveto{\pgfqpoint{-0.021649in}{-0.037276in}}{\pgfqpoint{-0.011050in}{-0.041667in}}{\pgfqpoint{0.000000in}{-0.041667in}}%
\pgfpathclose%
\pgfusepath{stroke,fill}%
}%
\end{pgfscope}%
\begin{pgfscope}%
\pgfpathrectangle{\pgfqpoint{0.481978in}{0.331635in}}{\pgfqpoint{9.300000in}{7.700000in}}%
\pgfusepath{clip}%
\pgfsetbuttcap%
\pgfsetroundjoin%
\definecolor{currentfill}{rgb}{1.000000,0.623529,0.607843}%
\pgfsetfillcolor{currentfill}%
\pgfsetlinewidth{1.003750pt}%
\definecolor{currentstroke}{rgb}{1.000000,0.623529,0.607843}%
\pgfsetstrokecolor{currentstroke}%
\pgfsetdash{}{0pt}%
\pgfsys@defobject{currentmarker}{\pgfqpoint{-0.041667in}{-0.041667in}}{\pgfqpoint{0.041667in}{0.041667in}}{%
\pgfpathmoveto{\pgfqpoint{0.000000in}{-0.041667in}}%
\pgfpathcurveto{\pgfqpoint{0.011050in}{-0.041667in}}{\pgfqpoint{0.021649in}{-0.037276in}}{\pgfqpoint{0.029463in}{-0.029463in}}%
\pgfpathcurveto{\pgfqpoint{0.037276in}{-0.021649in}}{\pgfqpoint{0.041667in}{-0.011050in}}{\pgfqpoint{0.041667in}{0.000000in}}%
\pgfpathcurveto{\pgfqpoint{0.041667in}{0.011050in}}{\pgfqpoint{0.037276in}{0.021649in}}{\pgfqpoint{0.029463in}{0.029463in}}%
\pgfpathcurveto{\pgfqpoint{0.021649in}{0.037276in}}{\pgfqpoint{0.011050in}{0.041667in}}{\pgfqpoint{0.000000in}{0.041667in}}%
\pgfpathcurveto{\pgfqpoint{-0.011050in}{0.041667in}}{\pgfqpoint{-0.021649in}{0.037276in}}{\pgfqpoint{-0.029463in}{0.029463in}}%
\pgfpathcurveto{\pgfqpoint{-0.037276in}{0.021649in}}{\pgfqpoint{-0.041667in}{0.011050in}}{\pgfqpoint{-0.041667in}{0.000000in}}%
\pgfpathcurveto{\pgfqpoint{-0.041667in}{-0.011050in}}{\pgfqpoint{-0.037276in}{-0.021649in}}{\pgfqpoint{-0.029463in}{-0.029463in}}%
\pgfpathcurveto{\pgfqpoint{-0.021649in}{-0.037276in}}{\pgfqpoint{-0.011050in}{-0.041667in}}{\pgfqpoint{0.000000in}{-0.041667in}}%
\pgfpathclose%
\pgfusepath{stroke,fill}%
}%
\end{pgfscope}%
\begin{pgfscope}%
\pgfpathrectangle{\pgfqpoint{0.481978in}{0.331635in}}{\pgfqpoint{9.300000in}{7.700000in}}%
\pgfusepath{clip}%
\pgfsetbuttcap%
\pgfsetroundjoin%
\definecolor{currentfill}{rgb}{0.815686,0.733333,1.000000}%
\pgfsetfillcolor{currentfill}%
\pgfsetlinewidth{1.003750pt}%
\definecolor{currentstroke}{rgb}{0.815686,0.733333,1.000000}%
\pgfsetstrokecolor{currentstroke}%
\pgfsetdash{}{0pt}%
\pgfsys@defobject{currentmarker}{\pgfqpoint{-0.041667in}{-0.041667in}}{\pgfqpoint{0.041667in}{0.041667in}}{%
\pgfpathmoveto{\pgfqpoint{0.000000in}{-0.041667in}}%
\pgfpathcurveto{\pgfqpoint{0.011050in}{-0.041667in}}{\pgfqpoint{0.021649in}{-0.037276in}}{\pgfqpoint{0.029463in}{-0.029463in}}%
\pgfpathcurveto{\pgfqpoint{0.037276in}{-0.021649in}}{\pgfqpoint{0.041667in}{-0.011050in}}{\pgfqpoint{0.041667in}{0.000000in}}%
\pgfpathcurveto{\pgfqpoint{0.041667in}{0.011050in}}{\pgfqpoint{0.037276in}{0.021649in}}{\pgfqpoint{0.029463in}{0.029463in}}%
\pgfpathcurveto{\pgfqpoint{0.021649in}{0.037276in}}{\pgfqpoint{0.011050in}{0.041667in}}{\pgfqpoint{0.000000in}{0.041667in}}%
\pgfpathcurveto{\pgfqpoint{-0.011050in}{0.041667in}}{\pgfqpoint{-0.021649in}{0.037276in}}{\pgfqpoint{-0.029463in}{0.029463in}}%
\pgfpathcurveto{\pgfqpoint{-0.037276in}{0.021649in}}{\pgfqpoint{-0.041667in}{0.011050in}}{\pgfqpoint{-0.041667in}{0.000000in}}%
\pgfpathcurveto{\pgfqpoint{-0.041667in}{-0.011050in}}{\pgfqpoint{-0.037276in}{-0.021649in}}{\pgfqpoint{-0.029463in}{-0.029463in}}%
\pgfpathcurveto{\pgfqpoint{-0.021649in}{-0.037276in}}{\pgfqpoint{-0.011050in}{-0.041667in}}{\pgfqpoint{0.000000in}{-0.041667in}}%
\pgfpathclose%
\pgfusepath{stroke,fill}%
}%
\end{pgfscope}%
\begin{pgfscope}%
\pgfpathrectangle{\pgfqpoint{0.481978in}{0.331635in}}{\pgfqpoint{9.300000in}{7.700000in}}%
\pgfusepath{clip}%
\pgfsetbuttcap%
\pgfsetroundjoin%
\definecolor{currentfill}{rgb}{0.870588,0.733333,0.607843}%
\pgfsetfillcolor{currentfill}%
\pgfsetlinewidth{1.003750pt}%
\definecolor{currentstroke}{rgb}{0.870588,0.733333,0.607843}%
\pgfsetstrokecolor{currentstroke}%
\pgfsetdash{}{0pt}%
\pgfsys@defobject{currentmarker}{\pgfqpoint{-0.041667in}{-0.041667in}}{\pgfqpoint{0.041667in}{0.041667in}}{%
\pgfpathmoveto{\pgfqpoint{0.000000in}{-0.041667in}}%
\pgfpathcurveto{\pgfqpoint{0.011050in}{-0.041667in}}{\pgfqpoint{0.021649in}{-0.037276in}}{\pgfqpoint{0.029463in}{-0.029463in}}%
\pgfpathcurveto{\pgfqpoint{0.037276in}{-0.021649in}}{\pgfqpoint{0.041667in}{-0.011050in}}{\pgfqpoint{0.041667in}{0.000000in}}%
\pgfpathcurveto{\pgfqpoint{0.041667in}{0.011050in}}{\pgfqpoint{0.037276in}{0.021649in}}{\pgfqpoint{0.029463in}{0.029463in}}%
\pgfpathcurveto{\pgfqpoint{0.021649in}{0.037276in}}{\pgfqpoint{0.011050in}{0.041667in}}{\pgfqpoint{0.000000in}{0.041667in}}%
\pgfpathcurveto{\pgfqpoint{-0.011050in}{0.041667in}}{\pgfqpoint{-0.021649in}{0.037276in}}{\pgfqpoint{-0.029463in}{0.029463in}}%
\pgfpathcurveto{\pgfqpoint{-0.037276in}{0.021649in}}{\pgfqpoint{-0.041667in}{0.011050in}}{\pgfqpoint{-0.041667in}{0.000000in}}%
\pgfpathcurveto{\pgfqpoint{-0.041667in}{-0.011050in}}{\pgfqpoint{-0.037276in}{-0.021649in}}{\pgfqpoint{-0.029463in}{-0.029463in}}%
\pgfpathcurveto{\pgfqpoint{-0.021649in}{-0.037276in}}{\pgfqpoint{-0.011050in}{-0.041667in}}{\pgfqpoint{0.000000in}{-0.041667in}}%
\pgfpathclose%
\pgfusepath{stroke,fill}%
}%
\end{pgfscope}%
\begin{pgfscope}%
\pgfsetbuttcap%
\pgfsetroundjoin%
\definecolor{currentfill}{rgb}{0.000000,0.000000,0.000000}%
\pgfsetfillcolor{currentfill}%
\pgfsetlinewidth{0.803000pt}%
\definecolor{currentstroke}{rgb}{0.000000,0.000000,0.000000}%
\pgfsetstrokecolor{currentstroke}%
\pgfsetdash{}{0pt}%
\pgfsys@defobject{currentmarker}{\pgfqpoint{0.000000in}{-0.048611in}}{\pgfqpoint{0.000000in}{0.000000in}}{%
\pgfpathmoveto{\pgfqpoint{0.000000in}{0.000000in}}%
\pgfpathlineto{\pgfqpoint{0.000000in}{-0.048611in}}%
\pgfusepath{stroke,fill}%
}%
\begin{pgfscope}%
\pgfsys@transformshift{0.980433in}{0.331635in}%
\pgfsys@useobject{currentmarker}{}%
\end{pgfscope}%
\end{pgfscope}%
\begin{pgfscope}%
\definecolor{textcolor}{rgb}{0.000000,0.000000,0.000000}%
\pgfsetstrokecolor{textcolor}%
\pgfsetfillcolor{textcolor}%
\pgftext[x=0.980433in,y=0.234413in,,top]{\color{textcolor}\sffamily\fontsize{10.000000}{12.000000}\selectfont \ensuremath{-}20}%
\end{pgfscope}%
\begin{pgfscope}%
\pgfsetbuttcap%
\pgfsetroundjoin%
\definecolor{currentfill}{rgb}{0.000000,0.000000,0.000000}%
\pgfsetfillcolor{currentfill}%
\pgfsetlinewidth{0.803000pt}%
\definecolor{currentstroke}{rgb}{0.000000,0.000000,0.000000}%
\pgfsetstrokecolor{currentstroke}%
\pgfsetdash{}{0pt}%
\pgfsys@defobject{currentmarker}{\pgfqpoint{0.000000in}{-0.048611in}}{\pgfqpoint{0.000000in}{0.000000in}}{%
\pgfpathmoveto{\pgfqpoint{0.000000in}{0.000000in}}%
\pgfpathlineto{\pgfqpoint{0.000000in}{-0.048611in}}%
\pgfusepath{stroke,fill}%
}%
\begin{pgfscope}%
\pgfsys@transformshift{2.182672in}{0.331635in}%
\pgfsys@useobject{currentmarker}{}%
\end{pgfscope}%
\end{pgfscope}%
\begin{pgfscope}%
\definecolor{textcolor}{rgb}{0.000000,0.000000,0.000000}%
\pgfsetstrokecolor{textcolor}%
\pgfsetfillcolor{textcolor}%
\pgftext[x=2.182672in,y=0.234413in,,top]{\color{textcolor}\sffamily\fontsize{10.000000}{12.000000}\selectfont \ensuremath{-}15}%
\end{pgfscope}%
\begin{pgfscope}%
\pgfsetbuttcap%
\pgfsetroundjoin%
\definecolor{currentfill}{rgb}{0.000000,0.000000,0.000000}%
\pgfsetfillcolor{currentfill}%
\pgfsetlinewidth{0.803000pt}%
\definecolor{currentstroke}{rgb}{0.000000,0.000000,0.000000}%
\pgfsetstrokecolor{currentstroke}%
\pgfsetdash{}{0pt}%
\pgfsys@defobject{currentmarker}{\pgfqpoint{0.000000in}{-0.048611in}}{\pgfqpoint{0.000000in}{0.000000in}}{%
\pgfpathmoveto{\pgfqpoint{0.000000in}{0.000000in}}%
\pgfpathlineto{\pgfqpoint{0.000000in}{-0.048611in}}%
\pgfusepath{stroke,fill}%
}%
\begin{pgfscope}%
\pgfsys@transformshift{3.384911in}{0.331635in}%
\pgfsys@useobject{currentmarker}{}%
\end{pgfscope}%
\end{pgfscope}%
\begin{pgfscope}%
\definecolor{textcolor}{rgb}{0.000000,0.000000,0.000000}%
\pgfsetstrokecolor{textcolor}%
\pgfsetfillcolor{textcolor}%
\pgftext[x=3.384911in,y=0.234413in,,top]{\color{textcolor}\sffamily\fontsize{10.000000}{12.000000}\selectfont \ensuremath{-}10}%
\end{pgfscope}%
\begin{pgfscope}%
\pgfsetbuttcap%
\pgfsetroundjoin%
\definecolor{currentfill}{rgb}{0.000000,0.000000,0.000000}%
\pgfsetfillcolor{currentfill}%
\pgfsetlinewidth{0.803000pt}%
\definecolor{currentstroke}{rgb}{0.000000,0.000000,0.000000}%
\pgfsetstrokecolor{currentstroke}%
\pgfsetdash{}{0pt}%
\pgfsys@defobject{currentmarker}{\pgfqpoint{0.000000in}{-0.048611in}}{\pgfqpoint{0.000000in}{0.000000in}}{%
\pgfpathmoveto{\pgfqpoint{0.000000in}{0.000000in}}%
\pgfpathlineto{\pgfqpoint{0.000000in}{-0.048611in}}%
\pgfusepath{stroke,fill}%
}%
\begin{pgfscope}%
\pgfsys@transformshift{4.587150in}{0.331635in}%
\pgfsys@useobject{currentmarker}{}%
\end{pgfscope}%
\end{pgfscope}%
\begin{pgfscope}%
\definecolor{textcolor}{rgb}{0.000000,0.000000,0.000000}%
\pgfsetstrokecolor{textcolor}%
\pgfsetfillcolor{textcolor}%
\pgftext[x=4.587150in,y=0.234413in,,top]{\color{textcolor}\sffamily\fontsize{10.000000}{12.000000}\selectfont \ensuremath{-}5}%
\end{pgfscope}%
\begin{pgfscope}%
\pgfsetbuttcap%
\pgfsetroundjoin%
\definecolor{currentfill}{rgb}{0.000000,0.000000,0.000000}%
\pgfsetfillcolor{currentfill}%
\pgfsetlinewidth{0.803000pt}%
\definecolor{currentstroke}{rgb}{0.000000,0.000000,0.000000}%
\pgfsetstrokecolor{currentstroke}%
\pgfsetdash{}{0pt}%
\pgfsys@defobject{currentmarker}{\pgfqpoint{0.000000in}{-0.048611in}}{\pgfqpoint{0.000000in}{0.000000in}}{%
\pgfpathmoveto{\pgfqpoint{0.000000in}{0.000000in}}%
\pgfpathlineto{\pgfqpoint{0.000000in}{-0.048611in}}%
\pgfusepath{stroke,fill}%
}%
\begin{pgfscope}%
\pgfsys@transformshift{5.789389in}{0.331635in}%
\pgfsys@useobject{currentmarker}{}%
\end{pgfscope}%
\end{pgfscope}%
\begin{pgfscope}%
\definecolor{textcolor}{rgb}{0.000000,0.000000,0.000000}%
\pgfsetstrokecolor{textcolor}%
\pgfsetfillcolor{textcolor}%
\pgftext[x=5.789389in,y=0.234413in,,top]{\color{textcolor}\sffamily\fontsize{10.000000}{12.000000}\selectfont 0}%
\end{pgfscope}%
\begin{pgfscope}%
\pgfsetbuttcap%
\pgfsetroundjoin%
\definecolor{currentfill}{rgb}{0.000000,0.000000,0.000000}%
\pgfsetfillcolor{currentfill}%
\pgfsetlinewidth{0.803000pt}%
\definecolor{currentstroke}{rgb}{0.000000,0.000000,0.000000}%
\pgfsetstrokecolor{currentstroke}%
\pgfsetdash{}{0pt}%
\pgfsys@defobject{currentmarker}{\pgfqpoint{0.000000in}{-0.048611in}}{\pgfqpoint{0.000000in}{0.000000in}}{%
\pgfpathmoveto{\pgfqpoint{0.000000in}{0.000000in}}%
\pgfpathlineto{\pgfqpoint{0.000000in}{-0.048611in}}%
\pgfusepath{stroke,fill}%
}%
\begin{pgfscope}%
\pgfsys@transformshift{6.991629in}{0.331635in}%
\pgfsys@useobject{currentmarker}{}%
\end{pgfscope}%
\end{pgfscope}%
\begin{pgfscope}%
\definecolor{textcolor}{rgb}{0.000000,0.000000,0.000000}%
\pgfsetstrokecolor{textcolor}%
\pgfsetfillcolor{textcolor}%
\pgftext[x=6.991629in,y=0.234413in,,top]{\color{textcolor}\sffamily\fontsize{10.000000}{12.000000}\selectfont 5}%
\end{pgfscope}%
\begin{pgfscope}%
\pgfsetbuttcap%
\pgfsetroundjoin%
\definecolor{currentfill}{rgb}{0.000000,0.000000,0.000000}%
\pgfsetfillcolor{currentfill}%
\pgfsetlinewidth{0.803000pt}%
\definecolor{currentstroke}{rgb}{0.000000,0.000000,0.000000}%
\pgfsetstrokecolor{currentstroke}%
\pgfsetdash{}{0pt}%
\pgfsys@defobject{currentmarker}{\pgfqpoint{0.000000in}{-0.048611in}}{\pgfqpoint{0.000000in}{0.000000in}}{%
\pgfpathmoveto{\pgfqpoint{0.000000in}{0.000000in}}%
\pgfpathlineto{\pgfqpoint{0.000000in}{-0.048611in}}%
\pgfusepath{stroke,fill}%
}%
\begin{pgfscope}%
\pgfsys@transformshift{8.193868in}{0.331635in}%
\pgfsys@useobject{currentmarker}{}%
\end{pgfscope}%
\end{pgfscope}%
\begin{pgfscope}%
\definecolor{textcolor}{rgb}{0.000000,0.000000,0.000000}%
\pgfsetstrokecolor{textcolor}%
\pgfsetfillcolor{textcolor}%
\pgftext[x=8.193868in,y=0.234413in,,top]{\color{textcolor}\sffamily\fontsize{10.000000}{12.000000}\selectfont 10}%
\end{pgfscope}%
\begin{pgfscope}%
\pgfsetbuttcap%
\pgfsetroundjoin%
\definecolor{currentfill}{rgb}{0.000000,0.000000,0.000000}%
\pgfsetfillcolor{currentfill}%
\pgfsetlinewidth{0.803000pt}%
\definecolor{currentstroke}{rgb}{0.000000,0.000000,0.000000}%
\pgfsetstrokecolor{currentstroke}%
\pgfsetdash{}{0pt}%
\pgfsys@defobject{currentmarker}{\pgfqpoint{0.000000in}{-0.048611in}}{\pgfqpoint{0.000000in}{0.000000in}}{%
\pgfpathmoveto{\pgfqpoint{0.000000in}{0.000000in}}%
\pgfpathlineto{\pgfqpoint{0.000000in}{-0.048611in}}%
\pgfusepath{stroke,fill}%
}%
\begin{pgfscope}%
\pgfsys@transformshift{9.396107in}{0.331635in}%
\pgfsys@useobject{currentmarker}{}%
\end{pgfscope}%
\end{pgfscope}%
\begin{pgfscope}%
\definecolor{textcolor}{rgb}{0.000000,0.000000,0.000000}%
\pgfsetstrokecolor{textcolor}%
\pgfsetfillcolor{textcolor}%
\pgftext[x=9.396107in,y=0.234413in,,top]{\color{textcolor}\sffamily\fontsize{10.000000}{12.000000}\selectfont 15}%
\end{pgfscope}%
\begin{pgfscope}%
\pgfsetbuttcap%
\pgfsetroundjoin%
\definecolor{currentfill}{rgb}{0.000000,0.000000,0.000000}%
\pgfsetfillcolor{currentfill}%
\pgfsetlinewidth{0.803000pt}%
\definecolor{currentstroke}{rgb}{0.000000,0.000000,0.000000}%
\pgfsetstrokecolor{currentstroke}%
\pgfsetdash{}{0pt}%
\pgfsys@defobject{currentmarker}{\pgfqpoint{-0.048611in}{0.000000in}}{\pgfqpoint{-0.000000in}{0.000000in}}{%
\pgfpathmoveto{\pgfqpoint{-0.000000in}{0.000000in}}%
\pgfpathlineto{\pgfqpoint{-0.048611in}{0.000000in}}%
\pgfusepath{stroke,fill}%
}%
\begin{pgfscope}%
\pgfsys@transformshift{0.481978in}{0.607744in}%
\pgfsys@useobject{currentmarker}{}%
\end{pgfscope}%
\end{pgfscope}%
\begin{pgfscope}%
\definecolor{textcolor}{rgb}{0.000000,0.000000,0.000000}%
\pgfsetstrokecolor{textcolor}%
\pgfsetfillcolor{textcolor}%
\pgftext[x=0.100000in, y=0.554982in, left, base]{\color{textcolor}\sffamily\fontsize{10.000000}{12.000000}\selectfont \ensuremath{-}15}%
\end{pgfscope}%
\begin{pgfscope}%
\pgfsetbuttcap%
\pgfsetroundjoin%
\definecolor{currentfill}{rgb}{0.000000,0.000000,0.000000}%
\pgfsetfillcolor{currentfill}%
\pgfsetlinewidth{0.803000pt}%
\definecolor{currentstroke}{rgb}{0.000000,0.000000,0.000000}%
\pgfsetstrokecolor{currentstroke}%
\pgfsetdash{}{0pt}%
\pgfsys@defobject{currentmarker}{\pgfqpoint{-0.048611in}{0.000000in}}{\pgfqpoint{-0.000000in}{0.000000in}}{%
\pgfpathmoveto{\pgfqpoint{-0.000000in}{0.000000in}}%
\pgfpathlineto{\pgfqpoint{-0.048611in}{0.000000in}}%
\pgfusepath{stroke,fill}%
}%
\begin{pgfscope}%
\pgfsys@transformshift{0.481978in}{1.846943in}%
\pgfsys@useobject{currentmarker}{}%
\end{pgfscope}%
\end{pgfscope}%
\begin{pgfscope}%
\definecolor{textcolor}{rgb}{0.000000,0.000000,0.000000}%
\pgfsetstrokecolor{textcolor}%
\pgfsetfillcolor{textcolor}%
\pgftext[x=0.100000in, y=1.794182in, left, base]{\color{textcolor}\sffamily\fontsize{10.000000}{12.000000}\selectfont \ensuremath{-}10}%
\end{pgfscope}%
\begin{pgfscope}%
\pgfsetbuttcap%
\pgfsetroundjoin%
\definecolor{currentfill}{rgb}{0.000000,0.000000,0.000000}%
\pgfsetfillcolor{currentfill}%
\pgfsetlinewidth{0.803000pt}%
\definecolor{currentstroke}{rgb}{0.000000,0.000000,0.000000}%
\pgfsetstrokecolor{currentstroke}%
\pgfsetdash{}{0pt}%
\pgfsys@defobject{currentmarker}{\pgfqpoint{-0.048611in}{0.000000in}}{\pgfqpoint{-0.000000in}{0.000000in}}{%
\pgfpathmoveto{\pgfqpoint{-0.000000in}{0.000000in}}%
\pgfpathlineto{\pgfqpoint{-0.048611in}{0.000000in}}%
\pgfusepath{stroke,fill}%
}%
\begin{pgfscope}%
\pgfsys@transformshift{0.481978in}{3.086143in}%
\pgfsys@useobject{currentmarker}{}%
\end{pgfscope}%
\end{pgfscope}%
\begin{pgfscope}%
\definecolor{textcolor}{rgb}{0.000000,0.000000,0.000000}%
\pgfsetstrokecolor{textcolor}%
\pgfsetfillcolor{textcolor}%
\pgftext[x=0.188365in, y=3.033381in, left, base]{\color{textcolor}\sffamily\fontsize{10.000000}{12.000000}\selectfont \ensuremath{-}5}%
\end{pgfscope}%
\begin{pgfscope}%
\pgfsetbuttcap%
\pgfsetroundjoin%
\definecolor{currentfill}{rgb}{0.000000,0.000000,0.000000}%
\pgfsetfillcolor{currentfill}%
\pgfsetlinewidth{0.803000pt}%
\definecolor{currentstroke}{rgb}{0.000000,0.000000,0.000000}%
\pgfsetstrokecolor{currentstroke}%
\pgfsetdash{}{0pt}%
\pgfsys@defobject{currentmarker}{\pgfqpoint{-0.048611in}{0.000000in}}{\pgfqpoint{-0.000000in}{0.000000in}}{%
\pgfpathmoveto{\pgfqpoint{-0.000000in}{0.000000in}}%
\pgfpathlineto{\pgfqpoint{-0.048611in}{0.000000in}}%
\pgfusepath{stroke,fill}%
}%
\begin{pgfscope}%
\pgfsys@transformshift{0.481978in}{4.325342in}%
\pgfsys@useobject{currentmarker}{}%
\end{pgfscope}%
\end{pgfscope}%
\begin{pgfscope}%
\definecolor{textcolor}{rgb}{0.000000,0.000000,0.000000}%
\pgfsetstrokecolor{textcolor}%
\pgfsetfillcolor{textcolor}%
\pgftext[x=0.296390in, y=4.272581in, left, base]{\color{textcolor}\sffamily\fontsize{10.000000}{12.000000}\selectfont 0}%
\end{pgfscope}%
\begin{pgfscope}%
\pgfsetbuttcap%
\pgfsetroundjoin%
\definecolor{currentfill}{rgb}{0.000000,0.000000,0.000000}%
\pgfsetfillcolor{currentfill}%
\pgfsetlinewidth{0.803000pt}%
\definecolor{currentstroke}{rgb}{0.000000,0.000000,0.000000}%
\pgfsetstrokecolor{currentstroke}%
\pgfsetdash{}{0pt}%
\pgfsys@defobject{currentmarker}{\pgfqpoint{-0.048611in}{0.000000in}}{\pgfqpoint{-0.000000in}{0.000000in}}{%
\pgfpathmoveto{\pgfqpoint{-0.000000in}{0.000000in}}%
\pgfpathlineto{\pgfqpoint{-0.048611in}{0.000000in}}%
\pgfusepath{stroke,fill}%
}%
\begin{pgfscope}%
\pgfsys@transformshift{0.481978in}{5.564542in}%
\pgfsys@useobject{currentmarker}{}%
\end{pgfscope}%
\end{pgfscope}%
\begin{pgfscope}%
\definecolor{textcolor}{rgb}{0.000000,0.000000,0.000000}%
\pgfsetstrokecolor{textcolor}%
\pgfsetfillcolor{textcolor}%
\pgftext[x=0.296390in, y=5.511780in, left, base]{\color{textcolor}\sffamily\fontsize{10.000000}{12.000000}\selectfont 5}%
\end{pgfscope}%
\begin{pgfscope}%
\pgfsetbuttcap%
\pgfsetroundjoin%
\definecolor{currentfill}{rgb}{0.000000,0.000000,0.000000}%
\pgfsetfillcolor{currentfill}%
\pgfsetlinewidth{0.803000pt}%
\definecolor{currentstroke}{rgb}{0.000000,0.000000,0.000000}%
\pgfsetstrokecolor{currentstroke}%
\pgfsetdash{}{0pt}%
\pgfsys@defobject{currentmarker}{\pgfqpoint{-0.048611in}{0.000000in}}{\pgfqpoint{-0.000000in}{0.000000in}}{%
\pgfpathmoveto{\pgfqpoint{-0.000000in}{0.000000in}}%
\pgfpathlineto{\pgfqpoint{-0.048611in}{0.000000in}}%
\pgfusepath{stroke,fill}%
}%
\begin{pgfscope}%
\pgfsys@transformshift{0.481978in}{6.803741in}%
\pgfsys@useobject{currentmarker}{}%
\end{pgfscope}%
\end{pgfscope}%
\begin{pgfscope}%
\definecolor{textcolor}{rgb}{0.000000,0.000000,0.000000}%
\pgfsetstrokecolor{textcolor}%
\pgfsetfillcolor{textcolor}%
\pgftext[x=0.208025in, y=6.750979in, left, base]{\color{textcolor}\sffamily\fontsize{10.000000}{12.000000}\selectfont 10}%
\end{pgfscope}%
\begin{pgfscope}%
\pgfpathrectangle{\pgfqpoint{0.481978in}{0.331635in}}{\pgfqpoint{9.300000in}{7.700000in}}%
\pgfusepath{clip}%
\pgfsetrectcap%
\pgfsetroundjoin%
\pgfsetlinewidth{1.505625pt}%
\definecolor{currentstroke}{rgb}{0.631373,0.788235,0.956863}%
\pgfsetstrokecolor{currentstroke}%
\pgfsetstrokeopacity{0.200000}%
\pgfsetdash{}{0pt}%
\pgfpathmoveto{\pgfqpoint{3.350250in}{0.837537in}}%
\pgfpathlineto{\pgfqpoint{3.792680in}{4.507363in}}%
\pgfusepath{stroke}%
\end{pgfscope}%
\begin{pgfscope}%
\pgfpathrectangle{\pgfqpoint{0.481978in}{0.331635in}}{\pgfqpoint{9.300000in}{7.700000in}}%
\pgfusepath{clip}%
\pgfsetrectcap%
\pgfsetroundjoin%
\pgfsetlinewidth{1.505625pt}%
\definecolor{currentstroke}{rgb}{0.631373,0.788235,0.956863}%
\pgfsetstrokecolor{currentstroke}%
\pgfsetstrokeopacity{0.200000}%
\pgfsetdash{}{0pt}%
\pgfpathmoveto{\pgfqpoint{3.059115in}{5.261945in}}%
\pgfpathlineto{\pgfqpoint{3.792680in}{4.507363in}}%
\pgfusepath{stroke}%
\end{pgfscope}%
\begin{pgfscope}%
\pgfpathrectangle{\pgfqpoint{0.481978in}{0.331635in}}{\pgfqpoint{9.300000in}{7.700000in}}%
\pgfusepath{clip}%
\pgfsetrectcap%
\pgfsetroundjoin%
\pgfsetlinewidth{1.505625pt}%
\definecolor{currentstroke}{rgb}{0.631373,0.788235,0.956863}%
\pgfsetstrokecolor{currentstroke}%
\pgfsetstrokeopacity{0.200000}%
\pgfsetdash{}{0pt}%
\pgfpathmoveto{\pgfqpoint{4.845549in}{3.770354in}}%
\pgfpathlineto{\pgfqpoint{3.792680in}{4.507363in}}%
\pgfusepath{stroke}%
\end{pgfscope}%
\begin{pgfscope}%
\pgfpathrectangle{\pgfqpoint{0.481978in}{0.331635in}}{\pgfqpoint{9.300000in}{7.700000in}}%
\pgfusepath{clip}%
\pgfsetrectcap%
\pgfsetroundjoin%
\pgfsetlinewidth{1.505625pt}%
\definecolor{currentstroke}{rgb}{0.631373,0.788235,0.956863}%
\pgfsetstrokecolor{currentstroke}%
\pgfsetstrokeopacity{0.200000}%
\pgfsetdash{}{0pt}%
\pgfpathmoveto{\pgfqpoint{1.304480in}{3.089887in}}%
\pgfpathlineto{\pgfqpoint{3.792680in}{4.507363in}}%
\pgfusepath{stroke}%
\end{pgfscope}%
\begin{pgfscope}%
\pgfpathrectangle{\pgfqpoint{0.481978in}{0.331635in}}{\pgfqpoint{9.300000in}{7.700000in}}%
\pgfusepath{clip}%
\pgfsetrectcap%
\pgfsetroundjoin%
\pgfsetlinewidth{1.505625pt}%
\definecolor{currentstroke}{rgb}{0.631373,0.788235,0.956863}%
\pgfsetstrokecolor{currentstroke}%
\pgfsetstrokeopacity{0.200000}%
\pgfsetdash{}{0pt}%
\pgfpathmoveto{\pgfqpoint{4.583472in}{2.434606in}}%
\pgfpathlineto{\pgfqpoint{3.792680in}{4.507363in}}%
\pgfusepath{stroke}%
\end{pgfscope}%
\begin{pgfscope}%
\pgfpathrectangle{\pgfqpoint{0.481978in}{0.331635in}}{\pgfqpoint{9.300000in}{7.700000in}}%
\pgfusepath{clip}%
\pgfsetrectcap%
\pgfsetroundjoin%
\pgfsetlinewidth{1.505625pt}%
\definecolor{currentstroke}{rgb}{0.631373,0.788235,0.956863}%
\pgfsetstrokecolor{currentstroke}%
\pgfsetstrokeopacity{0.200000}%
\pgfsetdash{}{0pt}%
\pgfpathmoveto{\pgfqpoint{5.509106in}{5.302557in}}%
\pgfpathlineto{\pgfqpoint{3.792680in}{4.507363in}}%
\pgfusepath{stroke}%
\end{pgfscope}%
\begin{pgfscope}%
\pgfpathrectangle{\pgfqpoint{0.481978in}{0.331635in}}{\pgfqpoint{9.300000in}{7.700000in}}%
\pgfusepath{clip}%
\pgfsetrectcap%
\pgfsetroundjoin%
\pgfsetlinewidth{1.505625pt}%
\definecolor{currentstroke}{rgb}{0.631373,0.788235,0.956863}%
\pgfsetstrokecolor{currentstroke}%
\pgfsetstrokeopacity{0.200000}%
\pgfsetdash{}{0pt}%
\pgfpathmoveto{\pgfqpoint{1.270538in}{3.063432in}}%
\pgfpathlineto{\pgfqpoint{3.792680in}{4.507363in}}%
\pgfusepath{stroke}%
\end{pgfscope}%
\begin{pgfscope}%
\pgfpathrectangle{\pgfqpoint{0.481978in}{0.331635in}}{\pgfqpoint{9.300000in}{7.700000in}}%
\pgfusepath{clip}%
\pgfsetrectcap%
\pgfsetroundjoin%
\pgfsetlinewidth{1.505625pt}%
\definecolor{currentstroke}{rgb}{0.631373,0.788235,0.956863}%
\pgfsetstrokecolor{currentstroke}%
\pgfsetstrokeopacity{0.200000}%
\pgfsetdash{}{0pt}%
\pgfpathmoveto{\pgfqpoint{4.233419in}{2.543996in}}%
\pgfpathlineto{\pgfqpoint{3.792680in}{4.507363in}}%
\pgfusepath{stroke}%
\end{pgfscope}%
\begin{pgfscope}%
\pgfpathrectangle{\pgfqpoint{0.481978in}{0.331635in}}{\pgfqpoint{9.300000in}{7.700000in}}%
\pgfusepath{clip}%
\pgfsetrectcap%
\pgfsetroundjoin%
\pgfsetlinewidth{1.505625pt}%
\definecolor{currentstroke}{rgb}{0.631373,0.788235,0.956863}%
\pgfsetstrokecolor{currentstroke}%
\pgfsetstrokeopacity{0.200000}%
\pgfsetdash{}{0pt}%
\pgfpathmoveto{\pgfqpoint{2.873447in}{4.588040in}}%
\pgfpathlineto{\pgfqpoint{3.792680in}{4.507363in}}%
\pgfusepath{stroke}%
\end{pgfscope}%
\begin{pgfscope}%
\pgfpathrectangle{\pgfqpoint{0.481978in}{0.331635in}}{\pgfqpoint{9.300000in}{7.700000in}}%
\pgfusepath{clip}%
\pgfsetrectcap%
\pgfsetroundjoin%
\pgfsetlinewidth{1.505625pt}%
\definecolor{currentstroke}{rgb}{0.631373,0.788235,0.956863}%
\pgfsetstrokecolor{currentstroke}%
\pgfsetstrokeopacity{0.200000}%
\pgfsetdash{}{0pt}%
\pgfpathmoveto{\pgfqpoint{6.038124in}{7.681635in}}%
\pgfpathlineto{\pgfqpoint{3.792680in}{4.507363in}}%
\pgfusepath{stroke}%
\end{pgfscope}%
\begin{pgfscope}%
\pgfpathrectangle{\pgfqpoint{0.481978in}{0.331635in}}{\pgfqpoint{9.300000in}{7.700000in}}%
\pgfusepath{clip}%
\pgfsetrectcap%
\pgfsetroundjoin%
\pgfsetlinewidth{1.505625pt}%
\definecolor{currentstroke}{rgb}{0.631373,0.788235,0.956863}%
\pgfsetstrokecolor{currentstroke}%
\pgfsetstrokeopacity{0.200000}%
\pgfsetdash{}{0pt}%
\pgfpathmoveto{\pgfqpoint{6.068535in}{7.049396in}}%
\pgfpathlineto{\pgfqpoint{3.792680in}{4.507363in}}%
\pgfusepath{stroke}%
\end{pgfscope}%
\begin{pgfscope}%
\pgfpathrectangle{\pgfqpoint{0.481978in}{0.331635in}}{\pgfqpoint{9.300000in}{7.700000in}}%
\pgfusepath{clip}%
\pgfsetrectcap%
\pgfsetroundjoin%
\pgfsetlinewidth{1.505625pt}%
\definecolor{currentstroke}{rgb}{0.631373,0.788235,0.956863}%
\pgfsetstrokecolor{currentstroke}%
\pgfsetstrokeopacity{0.200000}%
\pgfsetdash{}{0pt}%
\pgfpathmoveto{\pgfqpoint{0.904705in}{2.706623in}}%
\pgfpathlineto{\pgfqpoint{3.792680in}{4.507363in}}%
\pgfusepath{stroke}%
\end{pgfscope}%
\begin{pgfscope}%
\pgfpathrectangle{\pgfqpoint{0.481978in}{0.331635in}}{\pgfqpoint{9.300000in}{7.700000in}}%
\pgfusepath{clip}%
\pgfsetrectcap%
\pgfsetroundjoin%
\pgfsetlinewidth{1.505625pt}%
\definecolor{currentstroke}{rgb}{0.631373,0.788235,0.956863}%
\pgfsetstrokecolor{currentstroke}%
\pgfsetstrokeopacity{0.200000}%
\pgfsetdash{}{0pt}%
\pgfpathmoveto{\pgfqpoint{3.355697in}{0.924546in}}%
\pgfpathlineto{\pgfqpoint{3.792680in}{4.507363in}}%
\pgfusepath{stroke}%
\end{pgfscope}%
\begin{pgfscope}%
\pgfpathrectangle{\pgfqpoint{0.481978in}{0.331635in}}{\pgfqpoint{9.300000in}{7.700000in}}%
\pgfusepath{clip}%
\pgfsetrectcap%
\pgfsetroundjoin%
\pgfsetlinewidth{1.505625pt}%
\definecolor{currentstroke}{rgb}{0.631373,0.788235,0.956863}%
\pgfsetstrokecolor{currentstroke}%
\pgfsetstrokeopacity{0.200000}%
\pgfsetdash{}{0pt}%
\pgfpathmoveto{\pgfqpoint{1.822383in}{3.081016in}}%
\pgfpathlineto{\pgfqpoint{3.792680in}{4.507363in}}%
\pgfusepath{stroke}%
\end{pgfscope}%
\begin{pgfscope}%
\pgfpathrectangle{\pgfqpoint{0.481978in}{0.331635in}}{\pgfqpoint{9.300000in}{7.700000in}}%
\pgfusepath{clip}%
\pgfsetrectcap%
\pgfsetroundjoin%
\pgfsetlinewidth{1.505625pt}%
\definecolor{currentstroke}{rgb}{0.631373,0.788235,0.956863}%
\pgfsetstrokecolor{currentstroke}%
\pgfsetstrokeopacity{0.200000}%
\pgfsetdash{}{0pt}%
\pgfpathmoveto{\pgfqpoint{6.431702in}{7.311578in}}%
\pgfpathlineto{\pgfqpoint{3.792680in}{4.507363in}}%
\pgfusepath{stroke}%
\end{pgfscope}%
\begin{pgfscope}%
\pgfpathrectangle{\pgfqpoint{0.481978in}{0.331635in}}{\pgfqpoint{9.300000in}{7.700000in}}%
\pgfusepath{clip}%
\pgfsetrectcap%
\pgfsetroundjoin%
\pgfsetlinewidth{1.505625pt}%
\definecolor{currentstroke}{rgb}{0.631373,0.788235,0.956863}%
\pgfsetstrokecolor{currentstroke}%
\pgfsetstrokeopacity{0.200000}%
\pgfsetdash{}{0pt}%
\pgfpathmoveto{\pgfqpoint{1.601212in}{5.287720in}}%
\pgfpathlineto{\pgfqpoint{3.792680in}{4.507363in}}%
\pgfusepath{stroke}%
\end{pgfscope}%
\begin{pgfscope}%
\pgfpathrectangle{\pgfqpoint{0.481978in}{0.331635in}}{\pgfqpoint{9.300000in}{7.700000in}}%
\pgfusepath{clip}%
\pgfsetrectcap%
\pgfsetroundjoin%
\pgfsetlinewidth{1.505625pt}%
\definecolor{currentstroke}{rgb}{0.631373,0.788235,0.956863}%
\pgfsetstrokecolor{currentstroke}%
\pgfsetstrokeopacity{0.200000}%
\pgfsetdash{}{0pt}%
\pgfpathmoveto{\pgfqpoint{2.280847in}{6.346489in}}%
\pgfpathlineto{\pgfqpoint{3.792680in}{4.507363in}}%
\pgfusepath{stroke}%
\end{pgfscope}%
\begin{pgfscope}%
\pgfpathrectangle{\pgfqpoint{0.481978in}{0.331635in}}{\pgfqpoint{9.300000in}{7.700000in}}%
\pgfusepath{clip}%
\pgfsetrectcap%
\pgfsetroundjoin%
\pgfsetlinewidth{1.505625pt}%
\definecolor{currentstroke}{rgb}{0.631373,0.788235,0.956863}%
\pgfsetstrokecolor{currentstroke}%
\pgfsetstrokeopacity{0.200000}%
\pgfsetdash{}{0pt}%
\pgfpathmoveto{\pgfqpoint{1.914988in}{4.735660in}}%
\pgfpathlineto{\pgfqpoint{3.792680in}{4.507363in}}%
\pgfusepath{stroke}%
\end{pgfscope}%
\begin{pgfscope}%
\pgfpathrectangle{\pgfqpoint{0.481978in}{0.331635in}}{\pgfqpoint{9.300000in}{7.700000in}}%
\pgfusepath{clip}%
\pgfsetrectcap%
\pgfsetroundjoin%
\pgfsetlinewidth{1.505625pt}%
\definecolor{currentstroke}{rgb}{0.631373,0.788235,0.956863}%
\pgfsetstrokecolor{currentstroke}%
\pgfsetstrokeopacity{0.200000}%
\pgfsetdash{}{0pt}%
\pgfpathmoveto{\pgfqpoint{3.751753in}{1.453248in}}%
\pgfpathlineto{\pgfqpoint{3.792680in}{4.507363in}}%
\pgfusepath{stroke}%
\end{pgfscope}%
\begin{pgfscope}%
\pgfpathrectangle{\pgfqpoint{0.481978in}{0.331635in}}{\pgfqpoint{9.300000in}{7.700000in}}%
\pgfusepath{clip}%
\pgfsetrectcap%
\pgfsetroundjoin%
\pgfsetlinewidth{1.505625pt}%
\definecolor{currentstroke}{rgb}{0.631373,0.788235,0.956863}%
\pgfsetstrokecolor{currentstroke}%
\pgfsetstrokeopacity{0.200000}%
\pgfsetdash{}{0pt}%
\pgfpathmoveto{\pgfqpoint{5.753665in}{7.255589in}}%
\pgfpathlineto{\pgfqpoint{3.792680in}{4.507363in}}%
\pgfusepath{stroke}%
\end{pgfscope}%
\begin{pgfscope}%
\pgfpathrectangle{\pgfqpoint{0.481978in}{0.331635in}}{\pgfqpoint{9.300000in}{7.700000in}}%
\pgfusepath{clip}%
\pgfsetrectcap%
\pgfsetroundjoin%
\pgfsetlinewidth{1.505625pt}%
\definecolor{currentstroke}{rgb}{0.631373,0.788235,0.956863}%
\pgfsetstrokecolor{currentstroke}%
\pgfsetstrokeopacity{0.200000}%
\pgfsetdash{}{0pt}%
\pgfpathmoveto{\pgfqpoint{5.495955in}{5.233300in}}%
\pgfpathlineto{\pgfqpoint{3.792680in}{4.507363in}}%
\pgfusepath{stroke}%
\end{pgfscope}%
\begin{pgfscope}%
\pgfpathrectangle{\pgfqpoint{0.481978in}{0.331635in}}{\pgfqpoint{9.300000in}{7.700000in}}%
\pgfusepath{clip}%
\pgfsetrectcap%
\pgfsetroundjoin%
\pgfsetlinewidth{1.505625pt}%
\definecolor{currentstroke}{rgb}{0.631373,0.788235,0.956863}%
\pgfsetstrokecolor{currentstroke}%
\pgfsetstrokeopacity{0.200000}%
\pgfsetdash{}{0pt}%
\pgfpathmoveto{\pgfqpoint{4.238672in}{4.834740in}}%
\pgfpathlineto{\pgfqpoint{3.792680in}{4.507363in}}%
\pgfusepath{stroke}%
\end{pgfscope}%
\begin{pgfscope}%
\pgfpathrectangle{\pgfqpoint{0.481978in}{0.331635in}}{\pgfqpoint{9.300000in}{7.700000in}}%
\pgfusepath{clip}%
\pgfsetrectcap%
\pgfsetroundjoin%
\pgfsetlinewidth{1.505625pt}%
\definecolor{currentstroke}{rgb}{0.631373,0.788235,0.956863}%
\pgfsetstrokecolor{currentstroke}%
\pgfsetstrokeopacity{0.200000}%
\pgfsetdash{}{0pt}%
\pgfpathmoveto{\pgfqpoint{6.175837in}{7.515803in}}%
\pgfpathlineto{\pgfqpoint{3.792680in}{4.507363in}}%
\pgfusepath{stroke}%
\end{pgfscope}%
\begin{pgfscope}%
\pgfpathrectangle{\pgfqpoint{0.481978in}{0.331635in}}{\pgfqpoint{9.300000in}{7.700000in}}%
\pgfusepath{clip}%
\pgfsetrectcap%
\pgfsetroundjoin%
\pgfsetlinewidth{1.505625pt}%
\definecolor{currentstroke}{rgb}{0.631373,0.788235,0.956863}%
\pgfsetstrokecolor{currentstroke}%
\pgfsetstrokeopacity{0.200000}%
\pgfsetdash{}{0pt}%
\pgfpathmoveto{\pgfqpoint{4.277708in}{2.816465in}}%
\pgfpathlineto{\pgfqpoint{3.792680in}{4.507363in}}%
\pgfusepath{stroke}%
\end{pgfscope}%
\begin{pgfscope}%
\pgfpathrectangle{\pgfqpoint{0.481978in}{0.331635in}}{\pgfqpoint{9.300000in}{7.700000in}}%
\pgfusepath{clip}%
\pgfsetrectcap%
\pgfsetroundjoin%
\pgfsetlinewidth{1.505625pt}%
\definecolor{currentstroke}{rgb}{0.631373,0.788235,0.956863}%
\pgfsetstrokecolor{currentstroke}%
\pgfsetstrokeopacity{0.200000}%
\pgfsetdash{}{0pt}%
\pgfpathmoveto{\pgfqpoint{1.180315in}{4.870402in}}%
\pgfpathlineto{\pgfqpoint{3.792680in}{4.507363in}}%
\pgfusepath{stroke}%
\end{pgfscope}%
\begin{pgfscope}%
\pgfpathrectangle{\pgfqpoint{0.481978in}{0.331635in}}{\pgfqpoint{9.300000in}{7.700000in}}%
\pgfusepath{clip}%
\pgfsetrectcap%
\pgfsetroundjoin%
\pgfsetlinewidth{1.505625pt}%
\definecolor{currentstroke}{rgb}{0.631373,0.788235,0.956863}%
\pgfsetstrokecolor{currentstroke}%
\pgfsetstrokeopacity{0.200000}%
\pgfsetdash{}{0pt}%
\pgfpathmoveto{\pgfqpoint{4.575953in}{4.837612in}}%
\pgfpathlineto{\pgfqpoint{3.792680in}{4.507363in}}%
\pgfusepath{stroke}%
\end{pgfscope}%
\begin{pgfscope}%
\pgfpathrectangle{\pgfqpoint{0.481978in}{0.331635in}}{\pgfqpoint{9.300000in}{7.700000in}}%
\pgfusepath{clip}%
\pgfsetrectcap%
\pgfsetroundjoin%
\pgfsetlinewidth{1.505625pt}%
\definecolor{currentstroke}{rgb}{0.631373,0.788235,0.956863}%
\pgfsetstrokecolor{currentstroke}%
\pgfsetstrokeopacity{0.200000}%
\pgfsetdash{}{0pt}%
\pgfpathmoveto{\pgfqpoint{7.415126in}{4.572989in}}%
\pgfpathlineto{\pgfqpoint{3.792680in}{4.507363in}}%
\pgfusepath{stroke}%
\end{pgfscope}%
\begin{pgfscope}%
\pgfpathrectangle{\pgfqpoint{0.481978in}{0.331635in}}{\pgfqpoint{9.300000in}{7.700000in}}%
\pgfusepath{clip}%
\pgfsetrectcap%
\pgfsetroundjoin%
\pgfsetlinewidth{1.505625pt}%
\definecolor{currentstroke}{rgb}{0.631373,0.788235,0.956863}%
\pgfsetstrokecolor{currentstroke}%
\pgfsetstrokeopacity{0.200000}%
\pgfsetdash{}{0pt}%
\pgfpathmoveto{\pgfqpoint{5.743162in}{7.249576in}}%
\pgfpathlineto{\pgfqpoint{3.792680in}{4.507363in}}%
\pgfusepath{stroke}%
\end{pgfscope}%
\begin{pgfscope}%
\pgfpathrectangle{\pgfqpoint{0.481978in}{0.331635in}}{\pgfqpoint{9.300000in}{7.700000in}}%
\pgfusepath{clip}%
\pgfsetrectcap%
\pgfsetroundjoin%
\pgfsetlinewidth{1.505625pt}%
\definecolor{currentstroke}{rgb}{0.631373,0.788235,0.956863}%
\pgfsetstrokecolor{currentstroke}%
\pgfsetstrokeopacity{0.200000}%
\pgfsetdash{}{0pt}%
\pgfpathmoveto{\pgfqpoint{3.731632in}{6.363091in}}%
\pgfpathlineto{\pgfqpoint{3.792680in}{4.507363in}}%
\pgfusepath{stroke}%
\end{pgfscope}%
\begin{pgfscope}%
\pgfpathrectangle{\pgfqpoint{0.481978in}{0.331635in}}{\pgfqpoint{9.300000in}{7.700000in}}%
\pgfusepath{clip}%
\pgfsetrectcap%
\pgfsetroundjoin%
\pgfsetlinewidth{1.505625pt}%
\definecolor{currentstroke}{rgb}{0.631373,0.788235,0.956863}%
\pgfsetstrokecolor{currentstroke}%
\pgfsetstrokeopacity{0.200000}%
\pgfsetdash{}{0pt}%
\pgfpathmoveto{\pgfqpoint{3.354108in}{0.681635in}}%
\pgfpathlineto{\pgfqpoint{3.792680in}{4.507363in}}%
\pgfusepath{stroke}%
\end{pgfscope}%
\begin{pgfscope}%
\pgfpathrectangle{\pgfqpoint{0.481978in}{0.331635in}}{\pgfqpoint{9.300000in}{7.700000in}}%
\pgfusepath{clip}%
\pgfsetrectcap%
\pgfsetroundjoin%
\pgfsetlinewidth{1.505625pt}%
\definecolor{currentstroke}{rgb}{0.631373,0.788235,0.956863}%
\pgfsetstrokecolor{currentstroke}%
\pgfsetstrokeopacity{0.200000}%
\pgfsetdash{}{0pt}%
\pgfpathmoveto{\pgfqpoint{2.507727in}{5.244559in}}%
\pgfpathlineto{\pgfqpoint{3.792680in}{4.507363in}}%
\pgfusepath{stroke}%
\end{pgfscope}%
\begin{pgfscope}%
\pgfpathrectangle{\pgfqpoint{0.481978in}{0.331635in}}{\pgfqpoint{9.300000in}{7.700000in}}%
\pgfusepath{clip}%
\pgfsetrectcap%
\pgfsetroundjoin%
\pgfsetlinewidth{1.505625pt}%
\definecolor{currentstroke}{rgb}{0.631373,0.788235,0.956863}%
\pgfsetstrokecolor{currentstroke}%
\pgfsetstrokeopacity{0.200000}%
\pgfsetdash{}{0pt}%
\pgfpathmoveto{\pgfqpoint{3.345563in}{4.717554in}}%
\pgfpathlineto{\pgfqpoint{3.792680in}{4.507363in}}%
\pgfusepath{stroke}%
\end{pgfscope}%
\begin{pgfscope}%
\pgfpathrectangle{\pgfqpoint{0.481978in}{0.331635in}}{\pgfqpoint{9.300000in}{7.700000in}}%
\pgfusepath{clip}%
\pgfsetrectcap%
\pgfsetroundjoin%
\pgfsetlinewidth{1.505625pt}%
\definecolor{currentstroke}{rgb}{0.631373,0.788235,0.956863}%
\pgfsetstrokecolor{currentstroke}%
\pgfsetstrokeopacity{0.200000}%
\pgfsetdash{}{0pt}%
\pgfpathmoveto{\pgfqpoint{5.829195in}{5.254417in}}%
\pgfpathlineto{\pgfqpoint{3.792680in}{4.507363in}}%
\pgfusepath{stroke}%
\end{pgfscope}%
\begin{pgfscope}%
\pgfpathrectangle{\pgfqpoint{0.481978in}{0.331635in}}{\pgfqpoint{9.300000in}{7.700000in}}%
\pgfusepath{clip}%
\pgfsetrectcap%
\pgfsetroundjoin%
\pgfsetlinewidth{1.505625pt}%
\definecolor{currentstroke}{rgb}{0.631373,0.788235,0.956863}%
\pgfsetstrokecolor{currentstroke}%
\pgfsetstrokeopacity{0.200000}%
\pgfsetdash{}{0pt}%
\pgfpathmoveto{\pgfqpoint{1.266407in}{2.544584in}}%
\pgfpathlineto{\pgfqpoint{3.792680in}{4.507363in}}%
\pgfusepath{stroke}%
\end{pgfscope}%
\begin{pgfscope}%
\pgfpathrectangle{\pgfqpoint{0.481978in}{0.331635in}}{\pgfqpoint{9.300000in}{7.700000in}}%
\pgfusepath{clip}%
\pgfsetrectcap%
\pgfsetroundjoin%
\pgfsetlinewidth{1.505625pt}%
\definecolor{currentstroke}{rgb}{0.631373,0.788235,0.956863}%
\pgfsetstrokecolor{currentstroke}%
\pgfsetstrokeopacity{0.200000}%
\pgfsetdash{}{0pt}%
\pgfpathmoveto{\pgfqpoint{1.280257in}{3.743911in}}%
\pgfpathlineto{\pgfqpoint{3.792680in}{4.507363in}}%
\pgfusepath{stroke}%
\end{pgfscope}%
\begin{pgfscope}%
\pgfpathrectangle{\pgfqpoint{0.481978in}{0.331635in}}{\pgfqpoint{9.300000in}{7.700000in}}%
\pgfusepath{clip}%
\pgfsetrectcap%
\pgfsetroundjoin%
\pgfsetlinewidth{1.505625pt}%
\definecolor{currentstroke}{rgb}{0.631373,0.788235,0.956863}%
\pgfsetstrokecolor{currentstroke}%
\pgfsetstrokeopacity{0.200000}%
\pgfsetdash{}{0pt}%
\pgfpathmoveto{\pgfqpoint{1.366790in}{2.477899in}}%
\pgfpathlineto{\pgfqpoint{3.792680in}{4.507363in}}%
\pgfusepath{stroke}%
\end{pgfscope}%
\begin{pgfscope}%
\pgfpathrectangle{\pgfqpoint{0.481978in}{0.331635in}}{\pgfqpoint{9.300000in}{7.700000in}}%
\pgfusepath{clip}%
\pgfsetrectcap%
\pgfsetroundjoin%
\pgfsetlinewidth{1.505625pt}%
\definecolor{currentstroke}{rgb}{0.631373,0.788235,0.956863}%
\pgfsetstrokecolor{currentstroke}%
\pgfsetstrokeopacity{0.200000}%
\pgfsetdash{}{0pt}%
\pgfpathmoveto{\pgfqpoint{5.131403in}{3.817222in}}%
\pgfpathlineto{\pgfqpoint{3.792680in}{4.507363in}}%
\pgfusepath{stroke}%
\end{pgfscope}%
\begin{pgfscope}%
\pgfpathrectangle{\pgfqpoint{0.481978in}{0.331635in}}{\pgfqpoint{9.300000in}{7.700000in}}%
\pgfusepath{clip}%
\pgfsetrectcap%
\pgfsetroundjoin%
\pgfsetlinewidth{1.505625pt}%
\definecolor{currentstroke}{rgb}{0.631373,0.788235,0.956863}%
\pgfsetstrokecolor{currentstroke}%
\pgfsetstrokeopacity{0.200000}%
\pgfsetdash{}{0pt}%
\pgfpathmoveto{\pgfqpoint{6.077197in}{7.016351in}}%
\pgfpathlineto{\pgfqpoint{3.792680in}{4.507363in}}%
\pgfusepath{stroke}%
\end{pgfscope}%
\begin{pgfscope}%
\pgfpathrectangle{\pgfqpoint{0.481978in}{0.331635in}}{\pgfqpoint{9.300000in}{7.700000in}}%
\pgfusepath{clip}%
\pgfsetrectcap%
\pgfsetroundjoin%
\pgfsetlinewidth{1.505625pt}%
\definecolor{currentstroke}{rgb}{0.631373,0.788235,0.956863}%
\pgfsetstrokecolor{currentstroke}%
\pgfsetstrokeopacity{0.200000}%
\pgfsetdash{}{0pt}%
\pgfpathmoveto{\pgfqpoint{5.267507in}{3.479137in}}%
\pgfpathlineto{\pgfqpoint{3.792680in}{4.507363in}}%
\pgfusepath{stroke}%
\end{pgfscope}%
\begin{pgfscope}%
\pgfpathrectangle{\pgfqpoint{0.481978in}{0.331635in}}{\pgfqpoint{9.300000in}{7.700000in}}%
\pgfusepath{clip}%
\pgfsetrectcap%
\pgfsetroundjoin%
\pgfsetlinewidth{1.505625pt}%
\definecolor{currentstroke}{rgb}{0.631373,0.788235,0.956863}%
\pgfsetstrokecolor{currentstroke}%
\pgfsetstrokeopacity{0.200000}%
\pgfsetdash{}{0pt}%
\pgfpathmoveto{\pgfqpoint{5.422432in}{4.413168in}}%
\pgfpathlineto{\pgfqpoint{3.792680in}{4.507363in}}%
\pgfusepath{stroke}%
\end{pgfscope}%
\begin{pgfscope}%
\pgfpathrectangle{\pgfqpoint{0.481978in}{0.331635in}}{\pgfqpoint{9.300000in}{7.700000in}}%
\pgfusepath{clip}%
\pgfsetrectcap%
\pgfsetroundjoin%
\pgfsetlinewidth{1.505625pt}%
\definecolor{currentstroke}{rgb}{0.631373,0.788235,0.956863}%
\pgfsetstrokecolor{currentstroke}%
\pgfsetstrokeopacity{0.200000}%
\pgfsetdash{}{0pt}%
\pgfpathmoveto{\pgfqpoint{4.576144in}{3.894072in}}%
\pgfpathlineto{\pgfqpoint{3.792680in}{4.507363in}}%
\pgfusepath{stroke}%
\end{pgfscope}%
\begin{pgfscope}%
\pgfpathrectangle{\pgfqpoint{0.481978in}{0.331635in}}{\pgfqpoint{9.300000in}{7.700000in}}%
\pgfusepath{clip}%
\pgfsetrectcap%
\pgfsetroundjoin%
\pgfsetlinewidth{1.505625pt}%
\definecolor{currentstroke}{rgb}{0.631373,0.788235,0.956863}%
\pgfsetstrokecolor{currentstroke}%
\pgfsetstrokeopacity{0.200000}%
\pgfsetdash{}{0pt}%
\pgfpathmoveto{\pgfqpoint{2.109185in}{3.384062in}}%
\pgfpathlineto{\pgfqpoint{3.792680in}{4.507363in}}%
\pgfusepath{stroke}%
\end{pgfscope}%
\begin{pgfscope}%
\pgfpathrectangle{\pgfqpoint{0.481978in}{0.331635in}}{\pgfqpoint{9.300000in}{7.700000in}}%
\pgfusepath{clip}%
\pgfsetrectcap%
\pgfsetroundjoin%
\pgfsetlinewidth{1.505625pt}%
\definecolor{currentstroke}{rgb}{0.631373,0.788235,0.956863}%
\pgfsetstrokecolor{currentstroke}%
\pgfsetstrokeopacity{0.200000}%
\pgfsetdash{}{0pt}%
\pgfpathmoveto{\pgfqpoint{3.045397in}{4.082787in}}%
\pgfpathlineto{\pgfqpoint{3.792680in}{4.507363in}}%
\pgfusepath{stroke}%
\end{pgfscope}%
\begin{pgfscope}%
\pgfpathrectangle{\pgfqpoint{0.481978in}{0.331635in}}{\pgfqpoint{9.300000in}{7.700000in}}%
\pgfusepath{clip}%
\pgfsetrectcap%
\pgfsetroundjoin%
\pgfsetlinewidth{1.505625pt}%
\definecolor{currentstroke}{rgb}{0.631373,0.788235,0.956863}%
\pgfsetstrokecolor{currentstroke}%
\pgfsetstrokeopacity{0.200000}%
\pgfsetdash{}{0pt}%
\pgfpathmoveto{\pgfqpoint{2.937522in}{5.111573in}}%
\pgfpathlineto{\pgfqpoint{3.792680in}{4.507363in}}%
\pgfusepath{stroke}%
\end{pgfscope}%
\begin{pgfscope}%
\pgfpathrectangle{\pgfqpoint{0.481978in}{0.331635in}}{\pgfqpoint{9.300000in}{7.700000in}}%
\pgfusepath{clip}%
\pgfsetrectcap%
\pgfsetroundjoin%
\pgfsetlinewidth{1.505625pt}%
\definecolor{currentstroke}{rgb}{0.631373,0.788235,0.956863}%
\pgfsetstrokecolor{currentstroke}%
\pgfsetstrokeopacity{0.200000}%
\pgfsetdash{}{0pt}%
\pgfpathmoveto{\pgfqpoint{2.204208in}{6.251751in}}%
\pgfpathlineto{\pgfqpoint{3.792680in}{4.507363in}}%
\pgfusepath{stroke}%
\end{pgfscope}%
\begin{pgfscope}%
\pgfpathrectangle{\pgfqpoint{0.481978in}{0.331635in}}{\pgfqpoint{9.300000in}{7.700000in}}%
\pgfusepath{clip}%
\pgfsetrectcap%
\pgfsetroundjoin%
\pgfsetlinewidth{1.505625pt}%
\definecolor{currentstroke}{rgb}{0.631373,0.788235,0.956863}%
\pgfsetstrokecolor{currentstroke}%
\pgfsetstrokeopacity{0.200000}%
\pgfsetdash{}{0pt}%
\pgfpathmoveto{\pgfqpoint{3.780219in}{6.387783in}}%
\pgfpathlineto{\pgfqpoint{3.792680in}{4.507363in}}%
\pgfusepath{stroke}%
\end{pgfscope}%
\begin{pgfscope}%
\pgfpathrectangle{\pgfqpoint{0.481978in}{0.331635in}}{\pgfqpoint{9.300000in}{7.700000in}}%
\pgfusepath{clip}%
\pgfsetrectcap%
\pgfsetroundjoin%
\pgfsetlinewidth{1.505625pt}%
\definecolor{currentstroke}{rgb}{0.631373,0.788235,0.956863}%
\pgfsetstrokecolor{currentstroke}%
\pgfsetstrokeopacity{0.200000}%
\pgfsetdash{}{0pt}%
\pgfpathmoveto{\pgfqpoint{4.814839in}{2.794696in}}%
\pgfpathlineto{\pgfqpoint{3.792680in}{4.507363in}}%
\pgfusepath{stroke}%
\end{pgfscope}%
\begin{pgfscope}%
\pgfpathrectangle{\pgfqpoint{0.481978in}{0.331635in}}{\pgfqpoint{9.300000in}{7.700000in}}%
\pgfusepath{clip}%
\pgfsetrectcap%
\pgfsetroundjoin%
\pgfsetlinewidth{1.505625pt}%
\definecolor{currentstroke}{rgb}{0.631373,0.788235,0.956863}%
\pgfsetstrokecolor{currentstroke}%
\pgfsetstrokeopacity{0.200000}%
\pgfsetdash{}{0pt}%
\pgfpathmoveto{\pgfqpoint{2.753007in}{3.869593in}}%
\pgfpathlineto{\pgfqpoint{3.792680in}{4.507363in}}%
\pgfusepath{stroke}%
\end{pgfscope}%
\begin{pgfscope}%
\pgfpathrectangle{\pgfqpoint{0.481978in}{0.331635in}}{\pgfqpoint{9.300000in}{7.700000in}}%
\pgfusepath{clip}%
\pgfsetrectcap%
\pgfsetroundjoin%
\pgfsetlinewidth{1.505625pt}%
\definecolor{currentstroke}{rgb}{0.631373,0.788235,0.956863}%
\pgfsetstrokecolor{currentstroke}%
\pgfsetstrokeopacity{0.200000}%
\pgfsetdash{}{0pt}%
\pgfpathmoveto{\pgfqpoint{2.663463in}{5.701280in}}%
\pgfpathlineto{\pgfqpoint{3.792680in}{4.507363in}}%
\pgfusepath{stroke}%
\end{pgfscope}%
\begin{pgfscope}%
\pgfpathrectangle{\pgfqpoint{0.481978in}{0.331635in}}{\pgfqpoint{9.300000in}{7.700000in}}%
\pgfusepath{clip}%
\pgfsetrectcap%
\pgfsetroundjoin%
\pgfsetlinewidth{1.505625pt}%
\definecolor{currentstroke}{rgb}{0.631373,0.788235,0.956863}%
\pgfsetstrokecolor{currentstroke}%
\pgfsetstrokeopacity{0.200000}%
\pgfsetdash{}{0pt}%
\pgfpathmoveto{\pgfqpoint{6.114065in}{7.480299in}}%
\pgfpathlineto{\pgfqpoint{3.792680in}{4.507363in}}%
\pgfusepath{stroke}%
\end{pgfscope}%
\begin{pgfscope}%
\pgfpathrectangle{\pgfqpoint{0.481978in}{0.331635in}}{\pgfqpoint{9.300000in}{7.700000in}}%
\pgfusepath{clip}%
\pgfsetrectcap%
\pgfsetroundjoin%
\pgfsetlinewidth{1.505625pt}%
\definecolor{currentstroke}{rgb}{1.000000,0.705882,0.509804}%
\pgfsetstrokecolor{currentstroke}%
\pgfsetstrokeopacity{0.200000}%
\pgfsetdash{}{0pt}%
\pgfpathmoveto{\pgfqpoint{3.154468in}{2.488328in}}%
\pgfpathlineto{\pgfqpoint{5.493778in}{3.750666in}}%
\pgfusepath{stroke}%
\end{pgfscope}%
\begin{pgfscope}%
\pgfpathrectangle{\pgfqpoint{0.481978in}{0.331635in}}{\pgfqpoint{9.300000in}{7.700000in}}%
\pgfusepath{clip}%
\pgfsetrectcap%
\pgfsetroundjoin%
\pgfsetlinewidth{1.505625pt}%
\definecolor{currentstroke}{rgb}{1.000000,0.705882,0.509804}%
\pgfsetstrokecolor{currentstroke}%
\pgfsetstrokeopacity{0.200000}%
\pgfsetdash{}{0pt}%
\pgfpathmoveto{\pgfqpoint{1.545207in}{2.100004in}}%
\pgfpathlineto{\pgfqpoint{5.493778in}{3.750666in}}%
\pgfusepath{stroke}%
\end{pgfscope}%
\begin{pgfscope}%
\pgfpathrectangle{\pgfqpoint{0.481978in}{0.331635in}}{\pgfqpoint{9.300000in}{7.700000in}}%
\pgfusepath{clip}%
\pgfsetrectcap%
\pgfsetroundjoin%
\pgfsetlinewidth{1.505625pt}%
\definecolor{currentstroke}{rgb}{1.000000,0.705882,0.509804}%
\pgfsetstrokecolor{currentstroke}%
\pgfsetstrokeopacity{0.200000}%
\pgfsetdash{}{0pt}%
\pgfpathmoveto{\pgfqpoint{3.307338in}{3.407488in}}%
\pgfpathlineto{\pgfqpoint{5.493778in}{3.750666in}}%
\pgfusepath{stroke}%
\end{pgfscope}%
\begin{pgfscope}%
\pgfpathrectangle{\pgfqpoint{0.481978in}{0.331635in}}{\pgfqpoint{9.300000in}{7.700000in}}%
\pgfusepath{clip}%
\pgfsetrectcap%
\pgfsetroundjoin%
\pgfsetlinewidth{1.505625pt}%
\definecolor{currentstroke}{rgb}{1.000000,0.705882,0.509804}%
\pgfsetstrokecolor{currentstroke}%
\pgfsetstrokeopacity{0.200000}%
\pgfsetdash{}{0pt}%
\pgfpathmoveto{\pgfqpoint{5.143128in}{7.188196in}}%
\pgfpathlineto{\pgfqpoint{5.493778in}{3.750666in}}%
\pgfusepath{stroke}%
\end{pgfscope}%
\begin{pgfscope}%
\pgfpathrectangle{\pgfqpoint{0.481978in}{0.331635in}}{\pgfqpoint{9.300000in}{7.700000in}}%
\pgfusepath{clip}%
\pgfsetrectcap%
\pgfsetroundjoin%
\pgfsetlinewidth{1.505625pt}%
\definecolor{currentstroke}{rgb}{1.000000,0.705882,0.509804}%
\pgfsetstrokecolor{currentstroke}%
\pgfsetstrokeopacity{0.200000}%
\pgfsetdash{}{0pt}%
\pgfpathmoveto{\pgfqpoint{3.238886in}{1.944951in}}%
\pgfpathlineto{\pgfqpoint{5.493778in}{3.750666in}}%
\pgfusepath{stroke}%
\end{pgfscope}%
\begin{pgfscope}%
\pgfpathrectangle{\pgfqpoint{0.481978in}{0.331635in}}{\pgfqpoint{9.300000in}{7.700000in}}%
\pgfusepath{clip}%
\pgfsetrectcap%
\pgfsetroundjoin%
\pgfsetlinewidth{1.505625pt}%
\definecolor{currentstroke}{rgb}{1.000000,0.705882,0.509804}%
\pgfsetstrokecolor{currentstroke}%
\pgfsetstrokeopacity{0.200000}%
\pgfsetdash{}{0pt}%
\pgfpathmoveto{\pgfqpoint{6.096137in}{1.275008in}}%
\pgfpathlineto{\pgfqpoint{5.493778in}{3.750666in}}%
\pgfusepath{stroke}%
\end{pgfscope}%
\begin{pgfscope}%
\pgfpathrectangle{\pgfqpoint{0.481978in}{0.331635in}}{\pgfqpoint{9.300000in}{7.700000in}}%
\pgfusepath{clip}%
\pgfsetrectcap%
\pgfsetroundjoin%
\pgfsetlinewidth{1.505625pt}%
\definecolor{currentstroke}{rgb}{1.000000,0.705882,0.509804}%
\pgfsetstrokecolor{currentstroke}%
\pgfsetstrokeopacity{0.200000}%
\pgfsetdash{}{0pt}%
\pgfpathmoveto{\pgfqpoint{8.892538in}{6.041691in}}%
\pgfpathlineto{\pgfqpoint{5.493778in}{3.750666in}}%
\pgfusepath{stroke}%
\end{pgfscope}%
\begin{pgfscope}%
\pgfpathrectangle{\pgfqpoint{0.481978in}{0.331635in}}{\pgfqpoint{9.300000in}{7.700000in}}%
\pgfusepath{clip}%
\pgfsetrectcap%
\pgfsetroundjoin%
\pgfsetlinewidth{1.505625pt}%
\definecolor{currentstroke}{rgb}{1.000000,0.705882,0.509804}%
\pgfsetstrokecolor{currentstroke}%
\pgfsetstrokeopacity{0.200000}%
\pgfsetdash{}{0pt}%
\pgfpathmoveto{\pgfqpoint{8.792161in}{5.474544in}}%
\pgfpathlineto{\pgfqpoint{5.493778in}{3.750666in}}%
\pgfusepath{stroke}%
\end{pgfscope}%
\begin{pgfscope}%
\pgfpathrectangle{\pgfqpoint{0.481978in}{0.331635in}}{\pgfqpoint{9.300000in}{7.700000in}}%
\pgfusepath{clip}%
\pgfsetrectcap%
\pgfsetroundjoin%
\pgfsetlinewidth{1.505625pt}%
\definecolor{currentstroke}{rgb}{1.000000,0.705882,0.509804}%
\pgfsetstrokecolor{currentstroke}%
\pgfsetstrokeopacity{0.200000}%
\pgfsetdash{}{0pt}%
\pgfpathmoveto{\pgfqpoint{4.884502in}{4.809519in}}%
\pgfpathlineto{\pgfqpoint{5.493778in}{3.750666in}}%
\pgfusepath{stroke}%
\end{pgfscope}%
\begin{pgfscope}%
\pgfpathrectangle{\pgfqpoint{0.481978in}{0.331635in}}{\pgfqpoint{9.300000in}{7.700000in}}%
\pgfusepath{clip}%
\pgfsetrectcap%
\pgfsetroundjoin%
\pgfsetlinewidth{1.505625pt}%
\definecolor{currentstroke}{rgb}{1.000000,0.705882,0.509804}%
\pgfsetstrokecolor{currentstroke}%
\pgfsetstrokeopacity{0.200000}%
\pgfsetdash{}{0pt}%
\pgfpathmoveto{\pgfqpoint{8.839374in}{4.841088in}}%
\pgfpathlineto{\pgfqpoint{5.493778in}{3.750666in}}%
\pgfusepath{stroke}%
\end{pgfscope}%
\begin{pgfscope}%
\pgfpathrectangle{\pgfqpoint{0.481978in}{0.331635in}}{\pgfqpoint{9.300000in}{7.700000in}}%
\pgfusepath{clip}%
\pgfsetrectcap%
\pgfsetroundjoin%
\pgfsetlinewidth{1.505625pt}%
\definecolor{currentstroke}{rgb}{1.000000,0.705882,0.509804}%
\pgfsetstrokecolor{currentstroke}%
\pgfsetstrokeopacity{0.200000}%
\pgfsetdash{}{0pt}%
\pgfpathmoveto{\pgfqpoint{8.645625in}{6.005204in}}%
\pgfpathlineto{\pgfqpoint{5.493778in}{3.750666in}}%
\pgfusepath{stroke}%
\end{pgfscope}%
\begin{pgfscope}%
\pgfpathrectangle{\pgfqpoint{0.481978in}{0.331635in}}{\pgfqpoint{9.300000in}{7.700000in}}%
\pgfusepath{clip}%
\pgfsetrectcap%
\pgfsetroundjoin%
\pgfsetlinewidth{1.505625pt}%
\definecolor{currentstroke}{rgb}{1.000000,0.705882,0.509804}%
\pgfsetstrokecolor{currentstroke}%
\pgfsetstrokeopacity{0.200000}%
\pgfsetdash{}{0pt}%
\pgfpathmoveto{\pgfqpoint{1.541644in}{2.071167in}}%
\pgfpathlineto{\pgfqpoint{5.493778in}{3.750666in}}%
\pgfusepath{stroke}%
\end{pgfscope}%
\begin{pgfscope}%
\pgfpathrectangle{\pgfqpoint{0.481978in}{0.331635in}}{\pgfqpoint{9.300000in}{7.700000in}}%
\pgfusepath{clip}%
\pgfsetrectcap%
\pgfsetroundjoin%
\pgfsetlinewidth{1.505625pt}%
\definecolor{currentstroke}{rgb}{1.000000,0.705882,0.509804}%
\pgfsetstrokecolor{currentstroke}%
\pgfsetstrokeopacity{0.200000}%
\pgfsetdash{}{0pt}%
\pgfpathmoveto{\pgfqpoint{8.938916in}{5.410430in}}%
\pgfpathlineto{\pgfqpoint{5.493778in}{3.750666in}}%
\pgfusepath{stroke}%
\end{pgfscope}%
\begin{pgfscope}%
\pgfpathrectangle{\pgfqpoint{0.481978in}{0.331635in}}{\pgfqpoint{9.300000in}{7.700000in}}%
\pgfusepath{clip}%
\pgfsetrectcap%
\pgfsetroundjoin%
\pgfsetlinewidth{1.505625pt}%
\definecolor{currentstroke}{rgb}{1.000000,0.705882,0.509804}%
\pgfsetstrokecolor{currentstroke}%
\pgfsetstrokeopacity{0.200000}%
\pgfsetdash{}{0pt}%
\pgfpathmoveto{\pgfqpoint{3.682482in}{4.997977in}}%
\pgfpathlineto{\pgfqpoint{5.493778in}{3.750666in}}%
\pgfusepath{stroke}%
\end{pgfscope}%
\begin{pgfscope}%
\pgfpathrectangle{\pgfqpoint{0.481978in}{0.331635in}}{\pgfqpoint{9.300000in}{7.700000in}}%
\pgfusepath{clip}%
\pgfsetrectcap%
\pgfsetroundjoin%
\pgfsetlinewidth{1.505625pt}%
\definecolor{currentstroke}{rgb}{1.000000,0.705882,0.509804}%
\pgfsetstrokecolor{currentstroke}%
\pgfsetstrokeopacity{0.200000}%
\pgfsetdash{}{0pt}%
\pgfpathmoveto{\pgfqpoint{7.449113in}{2.488977in}}%
\pgfpathlineto{\pgfqpoint{5.493778in}{3.750666in}}%
\pgfusepath{stroke}%
\end{pgfscope}%
\begin{pgfscope}%
\pgfpathrectangle{\pgfqpoint{0.481978in}{0.331635in}}{\pgfqpoint{9.300000in}{7.700000in}}%
\pgfusepath{clip}%
\pgfsetrectcap%
\pgfsetroundjoin%
\pgfsetlinewidth{1.505625pt}%
\definecolor{currentstroke}{rgb}{1.000000,0.705882,0.509804}%
\pgfsetstrokecolor{currentstroke}%
\pgfsetstrokeopacity{0.200000}%
\pgfsetdash{}{0pt}%
\pgfpathmoveto{\pgfqpoint{8.325442in}{5.192998in}}%
\pgfpathlineto{\pgfqpoint{5.493778in}{3.750666in}}%
\pgfusepath{stroke}%
\end{pgfscope}%
\begin{pgfscope}%
\pgfpathrectangle{\pgfqpoint{0.481978in}{0.331635in}}{\pgfqpoint{9.300000in}{7.700000in}}%
\pgfusepath{clip}%
\pgfsetrectcap%
\pgfsetroundjoin%
\pgfsetlinewidth{1.505625pt}%
\definecolor{currentstroke}{rgb}{1.000000,0.705882,0.509804}%
\pgfsetstrokecolor{currentstroke}%
\pgfsetstrokeopacity{0.200000}%
\pgfsetdash{}{0pt}%
\pgfpathmoveto{\pgfqpoint{7.578449in}{4.392360in}}%
\pgfpathlineto{\pgfqpoint{5.493778in}{3.750666in}}%
\pgfusepath{stroke}%
\end{pgfscope}%
\begin{pgfscope}%
\pgfpathrectangle{\pgfqpoint{0.481978in}{0.331635in}}{\pgfqpoint{9.300000in}{7.700000in}}%
\pgfusepath{clip}%
\pgfsetrectcap%
\pgfsetroundjoin%
\pgfsetlinewidth{1.505625pt}%
\definecolor{currentstroke}{rgb}{1.000000,0.705882,0.509804}%
\pgfsetstrokecolor{currentstroke}%
\pgfsetstrokeopacity{0.200000}%
\pgfsetdash{}{0pt}%
\pgfpathmoveto{\pgfqpoint{6.440915in}{3.400690in}}%
\pgfpathlineto{\pgfqpoint{5.493778in}{3.750666in}}%
\pgfusepath{stroke}%
\end{pgfscope}%
\begin{pgfscope}%
\pgfpathrectangle{\pgfqpoint{0.481978in}{0.331635in}}{\pgfqpoint{9.300000in}{7.700000in}}%
\pgfusepath{clip}%
\pgfsetrectcap%
\pgfsetroundjoin%
\pgfsetlinewidth{1.505625pt}%
\definecolor{currentstroke}{rgb}{1.000000,0.705882,0.509804}%
\pgfsetstrokecolor{currentstroke}%
\pgfsetstrokeopacity{0.200000}%
\pgfsetdash{}{0pt}%
\pgfpathmoveto{\pgfqpoint{3.469279in}{3.869225in}}%
\pgfpathlineto{\pgfqpoint{5.493778in}{3.750666in}}%
\pgfusepath{stroke}%
\end{pgfscope}%
\begin{pgfscope}%
\pgfpathrectangle{\pgfqpoint{0.481978in}{0.331635in}}{\pgfqpoint{9.300000in}{7.700000in}}%
\pgfusepath{clip}%
\pgfsetrectcap%
\pgfsetroundjoin%
\pgfsetlinewidth{1.505625pt}%
\definecolor{currentstroke}{rgb}{1.000000,0.705882,0.509804}%
\pgfsetstrokecolor{currentstroke}%
\pgfsetstrokeopacity{0.200000}%
\pgfsetdash{}{0pt}%
\pgfpathmoveto{\pgfqpoint{6.295661in}{1.608501in}}%
\pgfpathlineto{\pgfqpoint{5.493778in}{3.750666in}}%
\pgfusepath{stroke}%
\end{pgfscope}%
\begin{pgfscope}%
\pgfpathrectangle{\pgfqpoint{0.481978in}{0.331635in}}{\pgfqpoint{9.300000in}{7.700000in}}%
\pgfusepath{clip}%
\pgfsetrectcap%
\pgfsetroundjoin%
\pgfsetlinewidth{1.505625pt}%
\definecolor{currentstroke}{rgb}{1.000000,0.705882,0.509804}%
\pgfsetstrokecolor{currentstroke}%
\pgfsetstrokeopacity{0.200000}%
\pgfsetdash{}{0pt}%
\pgfpathmoveto{\pgfqpoint{3.458025in}{3.980640in}}%
\pgfpathlineto{\pgfqpoint{5.493778in}{3.750666in}}%
\pgfusepath{stroke}%
\end{pgfscope}%
\begin{pgfscope}%
\pgfpathrectangle{\pgfqpoint{0.481978in}{0.331635in}}{\pgfqpoint{9.300000in}{7.700000in}}%
\pgfusepath{clip}%
\pgfsetrectcap%
\pgfsetroundjoin%
\pgfsetlinewidth{1.505625pt}%
\definecolor{currentstroke}{rgb}{1.000000,0.705882,0.509804}%
\pgfsetstrokecolor{currentstroke}%
\pgfsetstrokeopacity{0.200000}%
\pgfsetdash{}{0pt}%
\pgfpathmoveto{\pgfqpoint{4.270877in}{5.423784in}}%
\pgfpathlineto{\pgfqpoint{5.493778in}{3.750666in}}%
\pgfusepath{stroke}%
\end{pgfscope}%
\begin{pgfscope}%
\pgfpathrectangle{\pgfqpoint{0.481978in}{0.331635in}}{\pgfqpoint{9.300000in}{7.700000in}}%
\pgfusepath{clip}%
\pgfsetrectcap%
\pgfsetroundjoin%
\pgfsetlinewidth{1.505625pt}%
\definecolor{currentstroke}{rgb}{1.000000,0.705882,0.509804}%
\pgfsetstrokecolor{currentstroke}%
\pgfsetstrokeopacity{0.200000}%
\pgfsetdash{}{0pt}%
\pgfpathmoveto{\pgfqpoint{5.749725in}{1.958324in}}%
\pgfpathlineto{\pgfqpoint{5.493778in}{3.750666in}}%
\pgfusepath{stroke}%
\end{pgfscope}%
\begin{pgfscope}%
\pgfpathrectangle{\pgfqpoint{0.481978in}{0.331635in}}{\pgfqpoint{9.300000in}{7.700000in}}%
\pgfusepath{clip}%
\pgfsetrectcap%
\pgfsetroundjoin%
\pgfsetlinewidth{1.505625pt}%
\definecolor{currentstroke}{rgb}{1.000000,0.705882,0.509804}%
\pgfsetstrokecolor{currentstroke}%
\pgfsetstrokeopacity{0.200000}%
\pgfsetdash{}{0pt}%
\pgfpathmoveto{\pgfqpoint{2.458775in}{1.729372in}}%
\pgfpathlineto{\pgfqpoint{5.493778in}{3.750666in}}%
\pgfusepath{stroke}%
\end{pgfscope}%
\begin{pgfscope}%
\pgfpathrectangle{\pgfqpoint{0.481978in}{0.331635in}}{\pgfqpoint{9.300000in}{7.700000in}}%
\pgfusepath{clip}%
\pgfsetrectcap%
\pgfsetroundjoin%
\pgfsetlinewidth{1.505625pt}%
\definecolor{currentstroke}{rgb}{1.000000,0.705882,0.509804}%
\pgfsetstrokecolor{currentstroke}%
\pgfsetstrokeopacity{0.200000}%
\pgfsetdash{}{0pt}%
\pgfpathmoveto{\pgfqpoint{4.692250in}{5.305518in}}%
\pgfpathlineto{\pgfqpoint{5.493778in}{3.750666in}}%
\pgfusepath{stroke}%
\end{pgfscope}%
\begin{pgfscope}%
\pgfpathrectangle{\pgfqpoint{0.481978in}{0.331635in}}{\pgfqpoint{9.300000in}{7.700000in}}%
\pgfusepath{clip}%
\pgfsetrectcap%
\pgfsetroundjoin%
\pgfsetlinewidth{1.505625pt}%
\definecolor{currentstroke}{rgb}{1.000000,0.705882,0.509804}%
\pgfsetstrokecolor{currentstroke}%
\pgfsetstrokeopacity{0.200000}%
\pgfsetdash{}{0pt}%
\pgfpathmoveto{\pgfqpoint{7.916388in}{4.222966in}}%
\pgfpathlineto{\pgfqpoint{5.493778in}{3.750666in}}%
\pgfusepath{stroke}%
\end{pgfscope}%
\begin{pgfscope}%
\pgfpathrectangle{\pgfqpoint{0.481978in}{0.331635in}}{\pgfqpoint{9.300000in}{7.700000in}}%
\pgfusepath{clip}%
\pgfsetrectcap%
\pgfsetroundjoin%
\pgfsetlinewidth{1.505625pt}%
\definecolor{currentstroke}{rgb}{1.000000,0.705882,0.509804}%
\pgfsetstrokecolor{currentstroke}%
\pgfsetstrokeopacity{0.200000}%
\pgfsetdash{}{0pt}%
\pgfpathmoveto{\pgfqpoint{4.752028in}{3.460097in}}%
\pgfpathlineto{\pgfqpoint{5.493778in}{3.750666in}}%
\pgfusepath{stroke}%
\end{pgfscope}%
\begin{pgfscope}%
\pgfpathrectangle{\pgfqpoint{0.481978in}{0.331635in}}{\pgfqpoint{9.300000in}{7.700000in}}%
\pgfusepath{clip}%
\pgfsetrectcap%
\pgfsetroundjoin%
\pgfsetlinewidth{1.505625pt}%
\definecolor{currentstroke}{rgb}{1.000000,0.705882,0.509804}%
\pgfsetstrokecolor{currentstroke}%
\pgfsetstrokeopacity{0.200000}%
\pgfsetdash{}{0pt}%
\pgfpathmoveto{\pgfqpoint{8.514292in}{5.266180in}}%
\pgfpathlineto{\pgfqpoint{5.493778in}{3.750666in}}%
\pgfusepath{stroke}%
\end{pgfscope}%
\begin{pgfscope}%
\pgfpathrectangle{\pgfqpoint{0.481978in}{0.331635in}}{\pgfqpoint{9.300000in}{7.700000in}}%
\pgfusepath{clip}%
\pgfsetrectcap%
\pgfsetroundjoin%
\pgfsetlinewidth{1.505625pt}%
\definecolor{currentstroke}{rgb}{1.000000,0.705882,0.509804}%
\pgfsetstrokecolor{currentstroke}%
\pgfsetstrokeopacity{0.200000}%
\pgfsetdash{}{0pt}%
\pgfpathmoveto{\pgfqpoint{3.988115in}{1.945915in}}%
\pgfpathlineto{\pgfqpoint{5.493778in}{3.750666in}}%
\pgfusepath{stroke}%
\end{pgfscope}%
\begin{pgfscope}%
\pgfpathrectangle{\pgfqpoint{0.481978in}{0.331635in}}{\pgfqpoint{9.300000in}{7.700000in}}%
\pgfusepath{clip}%
\pgfsetrectcap%
\pgfsetroundjoin%
\pgfsetlinewidth{1.505625pt}%
\definecolor{currentstroke}{rgb}{1.000000,0.705882,0.509804}%
\pgfsetstrokecolor{currentstroke}%
\pgfsetstrokeopacity{0.200000}%
\pgfsetdash{}{0pt}%
\pgfpathmoveto{\pgfqpoint{3.637460in}{4.480745in}}%
\pgfpathlineto{\pgfqpoint{5.493778in}{3.750666in}}%
\pgfusepath{stroke}%
\end{pgfscope}%
\begin{pgfscope}%
\pgfpathrectangle{\pgfqpoint{0.481978in}{0.331635in}}{\pgfqpoint{9.300000in}{7.700000in}}%
\pgfusepath{clip}%
\pgfsetrectcap%
\pgfsetroundjoin%
\pgfsetlinewidth{1.505625pt}%
\definecolor{currentstroke}{rgb}{1.000000,0.705882,0.509804}%
\pgfsetstrokecolor{currentstroke}%
\pgfsetstrokeopacity{0.200000}%
\pgfsetdash{}{0pt}%
\pgfpathmoveto{\pgfqpoint{3.666577in}{4.495055in}}%
\pgfpathlineto{\pgfqpoint{5.493778in}{3.750666in}}%
\pgfusepath{stroke}%
\end{pgfscope}%
\begin{pgfscope}%
\pgfpathrectangle{\pgfqpoint{0.481978in}{0.331635in}}{\pgfqpoint{9.300000in}{7.700000in}}%
\pgfusepath{clip}%
\pgfsetrectcap%
\pgfsetroundjoin%
\pgfsetlinewidth{1.505625pt}%
\definecolor{currentstroke}{rgb}{1.000000,0.705882,0.509804}%
\pgfsetstrokecolor{currentstroke}%
\pgfsetstrokeopacity{0.200000}%
\pgfsetdash{}{0pt}%
\pgfpathmoveto{\pgfqpoint{3.344754in}{3.254443in}}%
\pgfpathlineto{\pgfqpoint{5.493778in}{3.750666in}}%
\pgfusepath{stroke}%
\end{pgfscope}%
\begin{pgfscope}%
\pgfpathrectangle{\pgfqpoint{0.481978in}{0.331635in}}{\pgfqpoint{9.300000in}{7.700000in}}%
\pgfusepath{clip}%
\pgfsetrectcap%
\pgfsetroundjoin%
\pgfsetlinewidth{1.505625pt}%
\definecolor{currentstroke}{rgb}{1.000000,0.705882,0.509804}%
\pgfsetstrokecolor{currentstroke}%
\pgfsetstrokeopacity{0.200000}%
\pgfsetdash{}{0pt}%
\pgfpathmoveto{\pgfqpoint{7.394228in}{2.358529in}}%
\pgfpathlineto{\pgfqpoint{5.493778in}{3.750666in}}%
\pgfusepath{stroke}%
\end{pgfscope}%
\begin{pgfscope}%
\pgfpathrectangle{\pgfqpoint{0.481978in}{0.331635in}}{\pgfqpoint{9.300000in}{7.700000in}}%
\pgfusepath{clip}%
\pgfsetrectcap%
\pgfsetroundjoin%
\pgfsetlinewidth{1.505625pt}%
\definecolor{currentstroke}{rgb}{1.000000,0.705882,0.509804}%
\pgfsetstrokecolor{currentstroke}%
\pgfsetstrokeopacity{0.200000}%
\pgfsetdash{}{0pt}%
\pgfpathmoveto{\pgfqpoint{2.694400in}{2.135292in}}%
\pgfpathlineto{\pgfqpoint{5.493778in}{3.750666in}}%
\pgfusepath{stroke}%
\end{pgfscope}%
\begin{pgfscope}%
\pgfpathrectangle{\pgfqpoint{0.481978in}{0.331635in}}{\pgfqpoint{9.300000in}{7.700000in}}%
\pgfusepath{clip}%
\pgfsetrectcap%
\pgfsetroundjoin%
\pgfsetlinewidth{1.505625pt}%
\definecolor{currentstroke}{rgb}{1.000000,0.705882,0.509804}%
\pgfsetstrokecolor{currentstroke}%
\pgfsetstrokeopacity{0.200000}%
\pgfsetdash{}{0pt}%
\pgfpathmoveto{\pgfqpoint{5.916550in}{1.543110in}}%
\pgfpathlineto{\pgfqpoint{5.493778in}{3.750666in}}%
\pgfusepath{stroke}%
\end{pgfscope}%
\begin{pgfscope}%
\pgfpathrectangle{\pgfqpoint{0.481978in}{0.331635in}}{\pgfqpoint{9.300000in}{7.700000in}}%
\pgfusepath{clip}%
\pgfsetrectcap%
\pgfsetroundjoin%
\pgfsetlinewidth{1.505625pt}%
\definecolor{currentstroke}{rgb}{1.000000,0.705882,0.509804}%
\pgfsetstrokecolor{currentstroke}%
\pgfsetstrokeopacity{0.200000}%
\pgfsetdash{}{0pt}%
\pgfpathmoveto{\pgfqpoint{6.083860in}{2.968841in}}%
\pgfpathlineto{\pgfqpoint{5.493778in}{3.750666in}}%
\pgfusepath{stroke}%
\end{pgfscope}%
\begin{pgfscope}%
\pgfpathrectangle{\pgfqpoint{0.481978in}{0.331635in}}{\pgfqpoint{9.300000in}{7.700000in}}%
\pgfusepath{clip}%
\pgfsetrectcap%
\pgfsetroundjoin%
\pgfsetlinewidth{1.505625pt}%
\definecolor{currentstroke}{rgb}{1.000000,0.705882,0.509804}%
\pgfsetstrokecolor{currentstroke}%
\pgfsetstrokeopacity{0.200000}%
\pgfsetdash{}{0pt}%
\pgfpathmoveto{\pgfqpoint{4.639314in}{3.292339in}}%
\pgfpathlineto{\pgfqpoint{5.493778in}{3.750666in}}%
\pgfusepath{stroke}%
\end{pgfscope}%
\begin{pgfscope}%
\pgfpathrectangle{\pgfqpoint{0.481978in}{0.331635in}}{\pgfqpoint{9.300000in}{7.700000in}}%
\pgfusepath{clip}%
\pgfsetrectcap%
\pgfsetroundjoin%
\pgfsetlinewidth{1.505625pt}%
\definecolor{currentstroke}{rgb}{1.000000,0.705882,0.509804}%
\pgfsetstrokecolor{currentstroke}%
\pgfsetstrokeopacity{0.200000}%
\pgfsetdash{}{0pt}%
\pgfpathmoveto{\pgfqpoint{8.886711in}{5.746544in}}%
\pgfpathlineto{\pgfqpoint{5.493778in}{3.750666in}}%
\pgfusepath{stroke}%
\end{pgfscope}%
\begin{pgfscope}%
\pgfpathrectangle{\pgfqpoint{0.481978in}{0.331635in}}{\pgfqpoint{9.300000in}{7.700000in}}%
\pgfusepath{clip}%
\pgfsetrectcap%
\pgfsetroundjoin%
\pgfsetlinewidth{1.505625pt}%
\definecolor{currentstroke}{rgb}{1.000000,0.705882,0.509804}%
\pgfsetstrokecolor{currentstroke}%
\pgfsetstrokeopacity{0.200000}%
\pgfsetdash{}{0pt}%
\pgfpathmoveto{\pgfqpoint{3.960687in}{2.310033in}}%
\pgfpathlineto{\pgfqpoint{5.493778in}{3.750666in}}%
\pgfusepath{stroke}%
\end{pgfscope}%
\begin{pgfscope}%
\pgfpathrectangle{\pgfqpoint{0.481978in}{0.331635in}}{\pgfqpoint{9.300000in}{7.700000in}}%
\pgfusepath{clip}%
\pgfsetrectcap%
\pgfsetroundjoin%
\pgfsetlinewidth{1.505625pt}%
\definecolor{currentstroke}{rgb}{1.000000,0.705882,0.509804}%
\pgfsetstrokecolor{currentstroke}%
\pgfsetstrokeopacity{0.200000}%
\pgfsetdash{}{0pt}%
\pgfpathmoveto{\pgfqpoint{3.907332in}{1.760958in}}%
\pgfpathlineto{\pgfqpoint{5.493778in}{3.750666in}}%
\pgfusepath{stroke}%
\end{pgfscope}%
\begin{pgfscope}%
\pgfpathrectangle{\pgfqpoint{0.481978in}{0.331635in}}{\pgfqpoint{9.300000in}{7.700000in}}%
\pgfusepath{clip}%
\pgfsetrectcap%
\pgfsetroundjoin%
\pgfsetlinewidth{1.505625pt}%
\definecolor{currentstroke}{rgb}{1.000000,0.705882,0.509804}%
\pgfsetstrokecolor{currentstroke}%
\pgfsetstrokeopacity{0.200000}%
\pgfsetdash{}{0pt}%
\pgfpathmoveto{\pgfqpoint{4.745564in}{5.377197in}}%
\pgfpathlineto{\pgfqpoint{5.493778in}{3.750666in}}%
\pgfusepath{stroke}%
\end{pgfscope}%
\begin{pgfscope}%
\pgfpathrectangle{\pgfqpoint{0.481978in}{0.331635in}}{\pgfqpoint{9.300000in}{7.700000in}}%
\pgfusepath{clip}%
\pgfsetrectcap%
\pgfsetroundjoin%
\pgfsetlinewidth{1.505625pt}%
\definecolor{currentstroke}{rgb}{1.000000,0.705882,0.509804}%
\pgfsetstrokecolor{currentstroke}%
\pgfsetstrokeopacity{0.200000}%
\pgfsetdash{}{0pt}%
\pgfpathmoveto{\pgfqpoint{4.104700in}{5.460602in}}%
\pgfpathlineto{\pgfqpoint{5.493778in}{3.750666in}}%
\pgfusepath{stroke}%
\end{pgfscope}%
\begin{pgfscope}%
\pgfpathrectangle{\pgfqpoint{0.481978in}{0.331635in}}{\pgfqpoint{9.300000in}{7.700000in}}%
\pgfusepath{clip}%
\pgfsetrectcap%
\pgfsetroundjoin%
\pgfsetlinewidth{1.505625pt}%
\definecolor{currentstroke}{rgb}{1.000000,0.705882,0.509804}%
\pgfsetstrokecolor{currentstroke}%
\pgfsetstrokeopacity{0.200000}%
\pgfsetdash{}{0pt}%
\pgfpathmoveto{\pgfqpoint{8.529282in}{4.574529in}}%
\pgfpathlineto{\pgfqpoint{5.493778in}{3.750666in}}%
\pgfusepath{stroke}%
\end{pgfscope}%
\begin{pgfscope}%
\pgfpathrectangle{\pgfqpoint{0.481978in}{0.331635in}}{\pgfqpoint{9.300000in}{7.700000in}}%
\pgfusepath{clip}%
\pgfsetrectcap%
\pgfsetroundjoin%
\pgfsetlinewidth{1.505625pt}%
\definecolor{currentstroke}{rgb}{1.000000,0.705882,0.509804}%
\pgfsetstrokecolor{currentstroke}%
\pgfsetstrokeopacity{0.200000}%
\pgfsetdash{}{0pt}%
\pgfpathmoveto{\pgfqpoint{7.907970in}{4.014679in}}%
\pgfpathlineto{\pgfqpoint{5.493778in}{3.750666in}}%
\pgfusepath{stroke}%
\end{pgfscope}%
\begin{pgfscope}%
\pgfpathrectangle{\pgfqpoint{0.481978in}{0.331635in}}{\pgfqpoint{9.300000in}{7.700000in}}%
\pgfusepath{clip}%
\pgfsetrectcap%
\pgfsetroundjoin%
\pgfsetlinewidth{1.505625pt}%
\definecolor{currentstroke}{rgb}{1.000000,0.705882,0.509804}%
\pgfsetstrokecolor{currentstroke}%
\pgfsetstrokeopacity{0.200000}%
\pgfsetdash{}{0pt}%
\pgfpathmoveto{\pgfqpoint{8.433076in}{4.560190in}}%
\pgfpathlineto{\pgfqpoint{5.493778in}{3.750666in}}%
\pgfusepath{stroke}%
\end{pgfscope}%
\begin{pgfscope}%
\pgfpathrectangle{\pgfqpoint{0.481978in}{0.331635in}}{\pgfqpoint{9.300000in}{7.700000in}}%
\pgfusepath{clip}%
\pgfsetrectcap%
\pgfsetroundjoin%
\pgfsetlinewidth{1.505625pt}%
\definecolor{currentstroke}{rgb}{1.000000,0.705882,0.509804}%
\pgfsetstrokecolor{currentstroke}%
\pgfsetstrokeopacity{0.200000}%
\pgfsetdash{}{0pt}%
\pgfpathmoveto{\pgfqpoint{3.128738in}{2.095357in}}%
\pgfpathlineto{\pgfqpoint{5.493778in}{3.750666in}}%
\pgfusepath{stroke}%
\end{pgfscope}%
\begin{pgfscope}%
\pgfpathrectangle{\pgfqpoint{0.481978in}{0.331635in}}{\pgfqpoint{9.300000in}{7.700000in}}%
\pgfusepath{clip}%
\pgfsetrectcap%
\pgfsetroundjoin%
\pgfsetlinewidth{1.505625pt}%
\definecolor{currentstroke}{rgb}{1.000000,0.705882,0.509804}%
\pgfsetstrokecolor{currentstroke}%
\pgfsetstrokeopacity{0.200000}%
\pgfsetdash{}{0pt}%
\pgfpathmoveto{\pgfqpoint{6.750608in}{3.072454in}}%
\pgfpathlineto{\pgfqpoint{5.493778in}{3.750666in}}%
\pgfusepath{stroke}%
\end{pgfscope}%
\begin{pgfscope}%
\pgfpathrectangle{\pgfqpoint{0.481978in}{0.331635in}}{\pgfqpoint{9.300000in}{7.700000in}}%
\pgfusepath{clip}%
\pgfsetrectcap%
\pgfsetroundjoin%
\pgfsetlinewidth{1.505625pt}%
\definecolor{currentstroke}{rgb}{1.000000,0.705882,0.509804}%
\pgfsetstrokecolor{currentstroke}%
\pgfsetstrokeopacity{0.200000}%
\pgfsetdash{}{0pt}%
\pgfpathmoveto{\pgfqpoint{3.434837in}{4.071894in}}%
\pgfpathlineto{\pgfqpoint{5.493778in}{3.750666in}}%
\pgfusepath{stroke}%
\end{pgfscope}%
\begin{pgfscope}%
\pgfpathrectangle{\pgfqpoint{0.481978in}{0.331635in}}{\pgfqpoint{9.300000in}{7.700000in}}%
\pgfusepath{clip}%
\pgfsetrectcap%
\pgfsetroundjoin%
\pgfsetlinewidth{1.505625pt}%
\definecolor{currentstroke}{rgb}{1.000000,0.705882,0.509804}%
\pgfsetstrokecolor{currentstroke}%
\pgfsetstrokeopacity{0.200000}%
\pgfsetdash{}{0pt}%
\pgfpathmoveto{\pgfqpoint{2.745959in}{2.321911in}}%
\pgfpathlineto{\pgfqpoint{5.493778in}{3.750666in}}%
\pgfusepath{stroke}%
\end{pgfscope}%
\begin{pgfscope}%
\pgfpathrectangle{\pgfqpoint{0.481978in}{0.331635in}}{\pgfqpoint{9.300000in}{7.700000in}}%
\pgfusepath{clip}%
\pgfsetrectcap%
\pgfsetroundjoin%
\pgfsetlinewidth{1.505625pt}%
\definecolor{currentstroke}{rgb}{1.000000,0.705882,0.509804}%
\pgfsetstrokecolor{currentstroke}%
\pgfsetstrokeopacity{0.200000}%
\pgfsetdash{}{0pt}%
\pgfpathmoveto{\pgfqpoint{8.714562in}{5.337458in}}%
\pgfpathlineto{\pgfqpoint{5.493778in}{3.750666in}}%
\pgfusepath{stroke}%
\end{pgfscope}%
\begin{pgfscope}%
\pgfpathrectangle{\pgfqpoint{0.481978in}{0.331635in}}{\pgfqpoint{9.300000in}{7.700000in}}%
\pgfusepath{clip}%
\pgfsetrectcap%
\pgfsetroundjoin%
\pgfsetlinewidth{1.505625pt}%
\definecolor{currentstroke}{rgb}{0.552941,0.898039,0.631373}%
\pgfsetstrokecolor{currentstroke}%
\pgfsetstrokeopacity{0.200000}%
\pgfsetdash{}{0pt}%
\pgfpathmoveto{\pgfqpoint{5.163299in}{7.343168in}}%
\pgfpathlineto{\pgfqpoint{5.849815in}{4.522824in}}%
\pgfusepath{stroke}%
\end{pgfscope}%
\begin{pgfscope}%
\pgfpathrectangle{\pgfqpoint{0.481978in}{0.331635in}}{\pgfqpoint{9.300000in}{7.700000in}}%
\pgfusepath{clip}%
\pgfsetrectcap%
\pgfsetroundjoin%
\pgfsetlinewidth{1.505625pt}%
\definecolor{currentstroke}{rgb}{0.552941,0.898039,0.631373}%
\pgfsetstrokecolor{currentstroke}%
\pgfsetstrokeopacity{0.200000}%
\pgfsetdash{}{0pt}%
\pgfpathmoveto{\pgfqpoint{3.761360in}{5.764646in}}%
\pgfpathlineto{\pgfqpoint{5.849815in}{4.522824in}}%
\pgfusepath{stroke}%
\end{pgfscope}%
\begin{pgfscope}%
\pgfpathrectangle{\pgfqpoint{0.481978in}{0.331635in}}{\pgfqpoint{9.300000in}{7.700000in}}%
\pgfusepath{clip}%
\pgfsetrectcap%
\pgfsetroundjoin%
\pgfsetlinewidth{1.505625pt}%
\definecolor{currentstroke}{rgb}{0.552941,0.898039,0.631373}%
\pgfsetstrokecolor{currentstroke}%
\pgfsetstrokeopacity{0.200000}%
\pgfsetdash{}{0pt}%
\pgfpathmoveto{\pgfqpoint{3.580350in}{3.088303in}}%
\pgfpathlineto{\pgfqpoint{5.849815in}{4.522824in}}%
\pgfusepath{stroke}%
\end{pgfscope}%
\begin{pgfscope}%
\pgfpathrectangle{\pgfqpoint{0.481978in}{0.331635in}}{\pgfqpoint{9.300000in}{7.700000in}}%
\pgfusepath{clip}%
\pgfsetrectcap%
\pgfsetroundjoin%
\pgfsetlinewidth{1.505625pt}%
\definecolor{currentstroke}{rgb}{0.552941,0.898039,0.631373}%
\pgfsetstrokecolor{currentstroke}%
\pgfsetstrokeopacity{0.200000}%
\pgfsetdash{}{0pt}%
\pgfpathmoveto{\pgfqpoint{8.242673in}{4.221424in}}%
\pgfpathlineto{\pgfqpoint{5.849815in}{4.522824in}}%
\pgfusepath{stroke}%
\end{pgfscope}%
\begin{pgfscope}%
\pgfpathrectangle{\pgfqpoint{0.481978in}{0.331635in}}{\pgfqpoint{9.300000in}{7.700000in}}%
\pgfusepath{clip}%
\pgfsetrectcap%
\pgfsetroundjoin%
\pgfsetlinewidth{1.505625pt}%
\definecolor{currentstroke}{rgb}{0.552941,0.898039,0.631373}%
\pgfsetstrokecolor{currentstroke}%
\pgfsetstrokeopacity{0.200000}%
\pgfsetdash{}{0pt}%
\pgfpathmoveto{\pgfqpoint{4.433234in}{5.070981in}}%
\pgfpathlineto{\pgfqpoint{5.849815in}{4.522824in}}%
\pgfusepath{stroke}%
\end{pgfscope}%
\begin{pgfscope}%
\pgfpathrectangle{\pgfqpoint{0.481978in}{0.331635in}}{\pgfqpoint{9.300000in}{7.700000in}}%
\pgfusepath{clip}%
\pgfsetrectcap%
\pgfsetroundjoin%
\pgfsetlinewidth{1.505625pt}%
\definecolor{currentstroke}{rgb}{0.552941,0.898039,0.631373}%
\pgfsetstrokecolor{currentstroke}%
\pgfsetstrokeopacity{0.200000}%
\pgfsetdash{}{0pt}%
\pgfpathmoveto{\pgfqpoint{9.194655in}{5.663935in}}%
\pgfpathlineto{\pgfqpoint{5.849815in}{4.522824in}}%
\pgfusepath{stroke}%
\end{pgfscope}%
\begin{pgfscope}%
\pgfpathrectangle{\pgfqpoint{0.481978in}{0.331635in}}{\pgfqpoint{9.300000in}{7.700000in}}%
\pgfusepath{clip}%
\pgfsetrectcap%
\pgfsetroundjoin%
\pgfsetlinewidth{1.505625pt}%
\definecolor{currentstroke}{rgb}{0.552941,0.898039,0.631373}%
\pgfsetstrokecolor{currentstroke}%
\pgfsetstrokeopacity{0.200000}%
\pgfsetdash{}{0pt}%
\pgfpathmoveto{\pgfqpoint{9.032589in}{5.080119in}}%
\pgfpathlineto{\pgfqpoint{5.849815in}{4.522824in}}%
\pgfusepath{stroke}%
\end{pgfscope}%
\begin{pgfscope}%
\pgfpathrectangle{\pgfqpoint{0.481978in}{0.331635in}}{\pgfqpoint{9.300000in}{7.700000in}}%
\pgfusepath{clip}%
\pgfsetrectcap%
\pgfsetroundjoin%
\pgfsetlinewidth{1.505625pt}%
\definecolor{currentstroke}{rgb}{0.552941,0.898039,0.631373}%
\pgfsetstrokecolor{currentstroke}%
\pgfsetstrokeopacity{0.200000}%
\pgfsetdash{}{0pt}%
\pgfpathmoveto{\pgfqpoint{3.942708in}{5.673543in}}%
\pgfpathlineto{\pgfqpoint{5.849815in}{4.522824in}}%
\pgfusepath{stroke}%
\end{pgfscope}%
\begin{pgfscope}%
\pgfpathrectangle{\pgfqpoint{0.481978in}{0.331635in}}{\pgfqpoint{9.300000in}{7.700000in}}%
\pgfusepath{clip}%
\pgfsetrectcap%
\pgfsetroundjoin%
\pgfsetlinewidth{1.505625pt}%
\definecolor{currentstroke}{rgb}{0.552941,0.898039,0.631373}%
\pgfsetstrokecolor{currentstroke}%
\pgfsetstrokeopacity{0.200000}%
\pgfsetdash{}{0pt}%
\pgfpathmoveto{\pgfqpoint{5.874679in}{2.722779in}}%
\pgfpathlineto{\pgfqpoint{5.849815in}{4.522824in}}%
\pgfusepath{stroke}%
\end{pgfscope}%
\begin{pgfscope}%
\pgfpathrectangle{\pgfqpoint{0.481978in}{0.331635in}}{\pgfqpoint{9.300000in}{7.700000in}}%
\pgfusepath{clip}%
\pgfsetrectcap%
\pgfsetroundjoin%
\pgfsetlinewidth{1.505625pt}%
\definecolor{currentstroke}{rgb}{0.552941,0.898039,0.631373}%
\pgfsetstrokecolor{currentstroke}%
\pgfsetstrokeopacity{0.200000}%
\pgfsetdash{}{0pt}%
\pgfpathmoveto{\pgfqpoint{6.449863in}{3.180896in}}%
\pgfpathlineto{\pgfqpoint{5.849815in}{4.522824in}}%
\pgfusepath{stroke}%
\end{pgfscope}%
\begin{pgfscope}%
\pgfpathrectangle{\pgfqpoint{0.481978in}{0.331635in}}{\pgfqpoint{9.300000in}{7.700000in}}%
\pgfusepath{clip}%
\pgfsetrectcap%
\pgfsetroundjoin%
\pgfsetlinewidth{1.505625pt}%
\definecolor{currentstroke}{rgb}{0.552941,0.898039,0.631373}%
\pgfsetstrokecolor{currentstroke}%
\pgfsetstrokeopacity{0.200000}%
\pgfsetdash{}{0pt}%
\pgfpathmoveto{\pgfqpoint{2.941597in}{6.366988in}}%
\pgfpathlineto{\pgfqpoint{5.849815in}{4.522824in}}%
\pgfusepath{stroke}%
\end{pgfscope}%
\begin{pgfscope}%
\pgfpathrectangle{\pgfqpoint{0.481978in}{0.331635in}}{\pgfqpoint{9.300000in}{7.700000in}}%
\pgfusepath{clip}%
\pgfsetrectcap%
\pgfsetroundjoin%
\pgfsetlinewidth{1.505625pt}%
\definecolor{currentstroke}{rgb}{0.552941,0.898039,0.631373}%
\pgfsetstrokecolor{currentstroke}%
\pgfsetstrokeopacity{0.200000}%
\pgfsetdash{}{0pt}%
\pgfpathmoveto{\pgfqpoint{4.671581in}{6.114771in}}%
\pgfpathlineto{\pgfqpoint{5.849815in}{4.522824in}}%
\pgfusepath{stroke}%
\end{pgfscope}%
\begin{pgfscope}%
\pgfpathrectangle{\pgfqpoint{0.481978in}{0.331635in}}{\pgfqpoint{9.300000in}{7.700000in}}%
\pgfusepath{clip}%
\pgfsetrectcap%
\pgfsetroundjoin%
\pgfsetlinewidth{1.505625pt}%
\definecolor{currentstroke}{rgb}{0.552941,0.898039,0.631373}%
\pgfsetstrokecolor{currentstroke}%
\pgfsetstrokeopacity{0.200000}%
\pgfsetdash{}{0pt}%
\pgfpathmoveto{\pgfqpoint{3.259508in}{2.958122in}}%
\pgfpathlineto{\pgfqpoint{5.849815in}{4.522824in}}%
\pgfusepath{stroke}%
\end{pgfscope}%
\begin{pgfscope}%
\pgfpathrectangle{\pgfqpoint{0.481978in}{0.331635in}}{\pgfqpoint{9.300000in}{7.700000in}}%
\pgfusepath{clip}%
\pgfsetrectcap%
\pgfsetroundjoin%
\pgfsetlinewidth{1.505625pt}%
\definecolor{currentstroke}{rgb}{0.552941,0.898039,0.631373}%
\pgfsetstrokecolor{currentstroke}%
\pgfsetstrokeopacity{0.200000}%
\pgfsetdash{}{0pt}%
\pgfpathmoveto{\pgfqpoint{7.805062in}{3.444564in}}%
\pgfpathlineto{\pgfqpoint{5.849815in}{4.522824in}}%
\pgfusepath{stroke}%
\end{pgfscope}%
\begin{pgfscope}%
\pgfpathrectangle{\pgfqpoint{0.481978in}{0.331635in}}{\pgfqpoint{9.300000in}{7.700000in}}%
\pgfusepath{clip}%
\pgfsetrectcap%
\pgfsetroundjoin%
\pgfsetlinewidth{1.505625pt}%
\definecolor{currentstroke}{rgb}{0.552941,0.898039,0.631373}%
\pgfsetstrokecolor{currentstroke}%
\pgfsetstrokeopacity{0.200000}%
\pgfsetdash{}{0pt}%
\pgfpathmoveto{\pgfqpoint{4.750388in}{7.194691in}}%
\pgfpathlineto{\pgfqpoint{5.849815in}{4.522824in}}%
\pgfusepath{stroke}%
\end{pgfscope}%
\begin{pgfscope}%
\pgfpathrectangle{\pgfqpoint{0.481978in}{0.331635in}}{\pgfqpoint{9.300000in}{7.700000in}}%
\pgfusepath{clip}%
\pgfsetrectcap%
\pgfsetroundjoin%
\pgfsetlinewidth{1.505625pt}%
\definecolor{currentstroke}{rgb}{0.552941,0.898039,0.631373}%
\pgfsetstrokecolor{currentstroke}%
\pgfsetstrokeopacity{0.200000}%
\pgfsetdash{}{0pt}%
\pgfpathmoveto{\pgfqpoint{4.386424in}{5.571510in}}%
\pgfpathlineto{\pgfqpoint{5.849815in}{4.522824in}}%
\pgfusepath{stroke}%
\end{pgfscope}%
\begin{pgfscope}%
\pgfpathrectangle{\pgfqpoint{0.481978in}{0.331635in}}{\pgfqpoint{9.300000in}{7.700000in}}%
\pgfusepath{clip}%
\pgfsetrectcap%
\pgfsetroundjoin%
\pgfsetlinewidth{1.505625pt}%
\definecolor{currentstroke}{rgb}{0.552941,0.898039,0.631373}%
\pgfsetstrokecolor{currentstroke}%
\pgfsetstrokeopacity{0.200000}%
\pgfsetdash{}{0pt}%
\pgfpathmoveto{\pgfqpoint{4.878045in}{6.103326in}}%
\pgfpathlineto{\pgfqpoint{5.849815in}{4.522824in}}%
\pgfusepath{stroke}%
\end{pgfscope}%
\begin{pgfscope}%
\pgfpathrectangle{\pgfqpoint{0.481978in}{0.331635in}}{\pgfqpoint{9.300000in}{7.700000in}}%
\pgfusepath{clip}%
\pgfsetrectcap%
\pgfsetroundjoin%
\pgfsetlinewidth{1.505625pt}%
\definecolor{currentstroke}{rgb}{0.552941,0.898039,0.631373}%
\pgfsetstrokecolor{currentstroke}%
\pgfsetstrokeopacity{0.200000}%
\pgfsetdash{}{0pt}%
\pgfpathmoveto{\pgfqpoint{9.259474in}{5.199290in}}%
\pgfpathlineto{\pgfqpoint{5.849815in}{4.522824in}}%
\pgfusepath{stroke}%
\end{pgfscope}%
\begin{pgfscope}%
\pgfpathrectangle{\pgfqpoint{0.481978in}{0.331635in}}{\pgfqpoint{9.300000in}{7.700000in}}%
\pgfusepath{clip}%
\pgfsetrectcap%
\pgfsetroundjoin%
\pgfsetlinewidth{1.505625pt}%
\definecolor{currentstroke}{rgb}{0.552941,0.898039,0.631373}%
\pgfsetstrokecolor{currentstroke}%
\pgfsetstrokeopacity{0.200000}%
\pgfsetdash{}{0pt}%
\pgfpathmoveto{\pgfqpoint{8.569592in}{4.792319in}}%
\pgfpathlineto{\pgfqpoint{5.849815in}{4.522824in}}%
\pgfusepath{stroke}%
\end{pgfscope}%
\begin{pgfscope}%
\pgfpathrectangle{\pgfqpoint{0.481978in}{0.331635in}}{\pgfqpoint{9.300000in}{7.700000in}}%
\pgfusepath{clip}%
\pgfsetrectcap%
\pgfsetroundjoin%
\pgfsetlinewidth{1.505625pt}%
\definecolor{currentstroke}{rgb}{0.552941,0.898039,0.631373}%
\pgfsetstrokecolor{currentstroke}%
\pgfsetstrokeopacity{0.200000}%
\pgfsetdash{}{0pt}%
\pgfpathmoveto{\pgfqpoint{8.786772in}{4.571593in}}%
\pgfpathlineto{\pgfqpoint{5.849815in}{4.522824in}}%
\pgfusepath{stroke}%
\end{pgfscope}%
\begin{pgfscope}%
\pgfpathrectangle{\pgfqpoint{0.481978in}{0.331635in}}{\pgfqpoint{9.300000in}{7.700000in}}%
\pgfusepath{clip}%
\pgfsetrectcap%
\pgfsetroundjoin%
\pgfsetlinewidth{1.505625pt}%
\definecolor{currentstroke}{rgb}{0.552941,0.898039,0.631373}%
\pgfsetstrokecolor{currentstroke}%
\pgfsetstrokeopacity{0.200000}%
\pgfsetdash{}{0pt}%
\pgfpathmoveto{\pgfqpoint{7.733235in}{4.906244in}}%
\pgfpathlineto{\pgfqpoint{5.849815in}{4.522824in}}%
\pgfusepath{stroke}%
\end{pgfscope}%
\begin{pgfscope}%
\pgfpathrectangle{\pgfqpoint{0.481978in}{0.331635in}}{\pgfqpoint{9.300000in}{7.700000in}}%
\pgfusepath{clip}%
\pgfsetrectcap%
\pgfsetroundjoin%
\pgfsetlinewidth{1.505625pt}%
\definecolor{currentstroke}{rgb}{0.552941,0.898039,0.631373}%
\pgfsetstrokecolor{currentstroke}%
\pgfsetstrokeopacity{0.200000}%
\pgfsetdash{}{0pt}%
\pgfpathmoveto{\pgfqpoint{5.023248in}{5.832186in}}%
\pgfpathlineto{\pgfqpoint{5.849815in}{4.522824in}}%
\pgfusepath{stroke}%
\end{pgfscope}%
\begin{pgfscope}%
\pgfpathrectangle{\pgfqpoint{0.481978in}{0.331635in}}{\pgfqpoint{9.300000in}{7.700000in}}%
\pgfusepath{clip}%
\pgfsetrectcap%
\pgfsetroundjoin%
\pgfsetlinewidth{1.505625pt}%
\definecolor{currentstroke}{rgb}{0.552941,0.898039,0.631373}%
\pgfsetstrokecolor{currentstroke}%
\pgfsetstrokeopacity{0.200000}%
\pgfsetdash{}{0pt}%
\pgfpathmoveto{\pgfqpoint{4.955252in}{6.273953in}}%
\pgfpathlineto{\pgfqpoint{5.849815in}{4.522824in}}%
\pgfusepath{stroke}%
\end{pgfscope}%
\begin{pgfscope}%
\pgfpathrectangle{\pgfqpoint{0.481978in}{0.331635in}}{\pgfqpoint{9.300000in}{7.700000in}}%
\pgfusepath{clip}%
\pgfsetrectcap%
\pgfsetroundjoin%
\pgfsetlinewidth{1.505625pt}%
\definecolor{currentstroke}{rgb}{0.552941,0.898039,0.631373}%
\pgfsetstrokecolor{currentstroke}%
\pgfsetstrokeopacity{0.200000}%
\pgfsetdash{}{0pt}%
\pgfpathmoveto{\pgfqpoint{7.433098in}{2.265608in}}%
\pgfpathlineto{\pgfqpoint{5.849815in}{4.522824in}}%
\pgfusepath{stroke}%
\end{pgfscope}%
\begin{pgfscope}%
\pgfpathrectangle{\pgfqpoint{0.481978in}{0.331635in}}{\pgfqpoint{9.300000in}{7.700000in}}%
\pgfusepath{clip}%
\pgfsetrectcap%
\pgfsetroundjoin%
\pgfsetlinewidth{1.505625pt}%
\definecolor{currentstroke}{rgb}{0.552941,0.898039,0.631373}%
\pgfsetstrokecolor{currentstroke}%
\pgfsetstrokeopacity{0.200000}%
\pgfsetdash{}{0pt}%
\pgfpathmoveto{\pgfqpoint{8.232070in}{5.815087in}}%
\pgfpathlineto{\pgfqpoint{5.849815in}{4.522824in}}%
\pgfusepath{stroke}%
\end{pgfscope}%
\begin{pgfscope}%
\pgfpathrectangle{\pgfqpoint{0.481978in}{0.331635in}}{\pgfqpoint{9.300000in}{7.700000in}}%
\pgfusepath{clip}%
\pgfsetrectcap%
\pgfsetroundjoin%
\pgfsetlinewidth{1.505625pt}%
\definecolor{currentstroke}{rgb}{0.552941,0.898039,0.631373}%
\pgfsetstrokecolor{currentstroke}%
\pgfsetstrokeopacity{0.200000}%
\pgfsetdash{}{0pt}%
\pgfpathmoveto{\pgfqpoint{3.011222in}{1.685197in}}%
\pgfpathlineto{\pgfqpoint{5.849815in}{4.522824in}}%
\pgfusepath{stroke}%
\end{pgfscope}%
\begin{pgfscope}%
\pgfpathrectangle{\pgfqpoint{0.481978in}{0.331635in}}{\pgfqpoint{9.300000in}{7.700000in}}%
\pgfusepath{clip}%
\pgfsetrectcap%
\pgfsetroundjoin%
\pgfsetlinewidth{1.505625pt}%
\definecolor{currentstroke}{rgb}{0.552941,0.898039,0.631373}%
\pgfsetstrokecolor{currentstroke}%
\pgfsetstrokeopacity{0.200000}%
\pgfsetdash{}{0pt}%
\pgfpathmoveto{\pgfqpoint{4.782892in}{6.148818in}}%
\pgfpathlineto{\pgfqpoint{5.849815in}{4.522824in}}%
\pgfusepath{stroke}%
\end{pgfscope}%
\begin{pgfscope}%
\pgfpathrectangle{\pgfqpoint{0.481978in}{0.331635in}}{\pgfqpoint{9.300000in}{7.700000in}}%
\pgfusepath{clip}%
\pgfsetrectcap%
\pgfsetroundjoin%
\pgfsetlinewidth{1.505625pt}%
\definecolor{currentstroke}{rgb}{0.552941,0.898039,0.631373}%
\pgfsetstrokecolor{currentstroke}%
\pgfsetstrokeopacity{0.200000}%
\pgfsetdash{}{0pt}%
\pgfpathmoveto{\pgfqpoint{3.628307in}{5.297078in}}%
\pgfpathlineto{\pgfqpoint{5.849815in}{4.522824in}}%
\pgfusepath{stroke}%
\end{pgfscope}%
\begin{pgfscope}%
\pgfpathrectangle{\pgfqpoint{0.481978in}{0.331635in}}{\pgfqpoint{9.300000in}{7.700000in}}%
\pgfusepath{clip}%
\pgfsetrectcap%
\pgfsetroundjoin%
\pgfsetlinewidth{1.505625pt}%
\definecolor{currentstroke}{rgb}{0.552941,0.898039,0.631373}%
\pgfsetstrokecolor{currentstroke}%
\pgfsetstrokeopacity{0.200000}%
\pgfsetdash{}{0pt}%
\pgfpathmoveto{\pgfqpoint{3.687739in}{5.241357in}}%
\pgfpathlineto{\pgfqpoint{5.849815in}{4.522824in}}%
\pgfusepath{stroke}%
\end{pgfscope}%
\begin{pgfscope}%
\pgfpathrectangle{\pgfqpoint{0.481978in}{0.331635in}}{\pgfqpoint{9.300000in}{7.700000in}}%
\pgfusepath{clip}%
\pgfsetrectcap%
\pgfsetroundjoin%
\pgfsetlinewidth{1.505625pt}%
\definecolor{currentstroke}{rgb}{0.552941,0.898039,0.631373}%
\pgfsetstrokecolor{currentstroke}%
\pgfsetstrokeopacity{0.200000}%
\pgfsetdash{}{0pt}%
\pgfpathmoveto{\pgfqpoint{7.445589in}{2.826774in}}%
\pgfpathlineto{\pgfqpoint{5.849815in}{4.522824in}}%
\pgfusepath{stroke}%
\end{pgfscope}%
\begin{pgfscope}%
\pgfpathrectangle{\pgfqpoint{0.481978in}{0.331635in}}{\pgfqpoint{9.300000in}{7.700000in}}%
\pgfusepath{clip}%
\pgfsetrectcap%
\pgfsetroundjoin%
\pgfsetlinewidth{1.505625pt}%
\definecolor{currentstroke}{rgb}{0.552941,0.898039,0.631373}%
\pgfsetstrokecolor{currentstroke}%
\pgfsetstrokeopacity{0.200000}%
\pgfsetdash{}{0pt}%
\pgfpathmoveto{\pgfqpoint{8.934791in}{5.143139in}}%
\pgfpathlineto{\pgfqpoint{5.849815in}{4.522824in}}%
\pgfusepath{stroke}%
\end{pgfscope}%
\begin{pgfscope}%
\pgfpathrectangle{\pgfqpoint{0.481978in}{0.331635in}}{\pgfqpoint{9.300000in}{7.700000in}}%
\pgfusepath{clip}%
\pgfsetrectcap%
\pgfsetroundjoin%
\pgfsetlinewidth{1.505625pt}%
\definecolor{currentstroke}{rgb}{0.552941,0.898039,0.631373}%
\pgfsetstrokecolor{currentstroke}%
\pgfsetstrokeopacity{0.200000}%
\pgfsetdash{}{0pt}%
\pgfpathmoveto{\pgfqpoint{3.508572in}{3.721693in}}%
\pgfpathlineto{\pgfqpoint{5.849815in}{4.522824in}}%
\pgfusepath{stroke}%
\end{pgfscope}%
\begin{pgfscope}%
\pgfpathrectangle{\pgfqpoint{0.481978in}{0.331635in}}{\pgfqpoint{9.300000in}{7.700000in}}%
\pgfusepath{clip}%
\pgfsetrectcap%
\pgfsetroundjoin%
\pgfsetlinewidth{1.505625pt}%
\definecolor{currentstroke}{rgb}{0.552941,0.898039,0.631373}%
\pgfsetstrokecolor{currentstroke}%
\pgfsetstrokeopacity{0.200000}%
\pgfsetdash{}{0pt}%
\pgfpathmoveto{\pgfqpoint{4.192874in}{1.559524in}}%
\pgfpathlineto{\pgfqpoint{5.849815in}{4.522824in}}%
\pgfusepath{stroke}%
\end{pgfscope}%
\begin{pgfscope}%
\pgfpathrectangle{\pgfqpoint{0.481978in}{0.331635in}}{\pgfqpoint{9.300000in}{7.700000in}}%
\pgfusepath{clip}%
\pgfsetrectcap%
\pgfsetroundjoin%
\pgfsetlinewidth{1.505625pt}%
\definecolor{currentstroke}{rgb}{0.552941,0.898039,0.631373}%
\pgfsetstrokecolor{currentstroke}%
\pgfsetstrokeopacity{0.200000}%
\pgfsetdash{}{0pt}%
\pgfpathmoveto{\pgfqpoint{4.001930in}{5.032636in}}%
\pgfpathlineto{\pgfqpoint{5.849815in}{4.522824in}}%
\pgfusepath{stroke}%
\end{pgfscope}%
\begin{pgfscope}%
\pgfpathrectangle{\pgfqpoint{0.481978in}{0.331635in}}{\pgfqpoint{9.300000in}{7.700000in}}%
\pgfusepath{clip}%
\pgfsetrectcap%
\pgfsetroundjoin%
\pgfsetlinewidth{1.505625pt}%
\definecolor{currentstroke}{rgb}{0.552941,0.898039,0.631373}%
\pgfsetstrokecolor{currentstroke}%
\pgfsetstrokeopacity{0.200000}%
\pgfsetdash{}{0pt}%
\pgfpathmoveto{\pgfqpoint{6.060999in}{1.792912in}}%
\pgfpathlineto{\pgfqpoint{5.849815in}{4.522824in}}%
\pgfusepath{stroke}%
\end{pgfscope}%
\begin{pgfscope}%
\pgfpathrectangle{\pgfqpoint{0.481978in}{0.331635in}}{\pgfqpoint{9.300000in}{7.700000in}}%
\pgfusepath{clip}%
\pgfsetrectcap%
\pgfsetroundjoin%
\pgfsetlinewidth{1.505625pt}%
\definecolor{currentstroke}{rgb}{0.552941,0.898039,0.631373}%
\pgfsetstrokecolor{currentstroke}%
\pgfsetstrokeopacity{0.200000}%
\pgfsetdash{}{0pt}%
\pgfpathmoveto{\pgfqpoint{2.449894in}{1.698466in}}%
\pgfpathlineto{\pgfqpoint{5.849815in}{4.522824in}}%
\pgfusepath{stroke}%
\end{pgfscope}%
\begin{pgfscope}%
\pgfpathrectangle{\pgfqpoint{0.481978in}{0.331635in}}{\pgfqpoint{9.300000in}{7.700000in}}%
\pgfusepath{clip}%
\pgfsetrectcap%
\pgfsetroundjoin%
\pgfsetlinewidth{1.505625pt}%
\definecolor{currentstroke}{rgb}{0.552941,0.898039,0.631373}%
\pgfsetstrokecolor{currentstroke}%
\pgfsetstrokeopacity{0.200000}%
\pgfsetdash{}{0pt}%
\pgfpathmoveto{\pgfqpoint{7.028940in}{4.751090in}}%
\pgfpathlineto{\pgfqpoint{5.849815in}{4.522824in}}%
\pgfusepath{stroke}%
\end{pgfscope}%
\begin{pgfscope}%
\pgfpathrectangle{\pgfqpoint{0.481978in}{0.331635in}}{\pgfqpoint{9.300000in}{7.700000in}}%
\pgfusepath{clip}%
\pgfsetrectcap%
\pgfsetroundjoin%
\pgfsetlinewidth{1.505625pt}%
\definecolor{currentstroke}{rgb}{0.552941,0.898039,0.631373}%
\pgfsetstrokecolor{currentstroke}%
\pgfsetstrokeopacity{0.200000}%
\pgfsetdash{}{0pt}%
\pgfpathmoveto{\pgfqpoint{5.259017in}{7.595751in}}%
\pgfpathlineto{\pgfqpoint{5.849815in}{4.522824in}}%
\pgfusepath{stroke}%
\end{pgfscope}%
\begin{pgfscope}%
\pgfpathrectangle{\pgfqpoint{0.481978in}{0.331635in}}{\pgfqpoint{9.300000in}{7.700000in}}%
\pgfusepath{clip}%
\pgfsetrectcap%
\pgfsetroundjoin%
\pgfsetlinewidth{1.505625pt}%
\definecolor{currentstroke}{rgb}{0.552941,0.898039,0.631373}%
\pgfsetstrokecolor{currentstroke}%
\pgfsetstrokeopacity{0.200000}%
\pgfsetdash{}{0pt}%
\pgfpathmoveto{\pgfqpoint{9.184137in}{5.106329in}}%
\pgfpathlineto{\pgfqpoint{5.849815in}{4.522824in}}%
\pgfusepath{stroke}%
\end{pgfscope}%
\begin{pgfscope}%
\pgfpathrectangle{\pgfqpoint{0.481978in}{0.331635in}}{\pgfqpoint{9.300000in}{7.700000in}}%
\pgfusepath{clip}%
\pgfsetrectcap%
\pgfsetroundjoin%
\pgfsetlinewidth{1.505625pt}%
\definecolor{currentstroke}{rgb}{0.552941,0.898039,0.631373}%
\pgfsetstrokecolor{currentstroke}%
\pgfsetstrokeopacity{0.200000}%
\pgfsetdash{}{0pt}%
\pgfpathmoveto{\pgfqpoint{8.944873in}{6.293748in}}%
\pgfpathlineto{\pgfqpoint{5.849815in}{4.522824in}}%
\pgfusepath{stroke}%
\end{pgfscope}%
\begin{pgfscope}%
\pgfpathrectangle{\pgfqpoint{0.481978in}{0.331635in}}{\pgfqpoint{9.300000in}{7.700000in}}%
\pgfusepath{clip}%
\pgfsetrectcap%
\pgfsetroundjoin%
\pgfsetlinewidth{1.505625pt}%
\definecolor{currentstroke}{rgb}{0.552941,0.898039,0.631373}%
\pgfsetstrokecolor{currentstroke}%
\pgfsetstrokeopacity{0.200000}%
\pgfsetdash{}{0pt}%
\pgfpathmoveto{\pgfqpoint{7.733013in}{3.491196in}}%
\pgfpathlineto{\pgfqpoint{5.849815in}{4.522824in}}%
\pgfusepath{stroke}%
\end{pgfscope}%
\begin{pgfscope}%
\pgfpathrectangle{\pgfqpoint{0.481978in}{0.331635in}}{\pgfqpoint{9.300000in}{7.700000in}}%
\pgfusepath{clip}%
\pgfsetrectcap%
\pgfsetroundjoin%
\pgfsetlinewidth{1.505625pt}%
\definecolor{currentstroke}{rgb}{0.552941,0.898039,0.631373}%
\pgfsetstrokecolor{currentstroke}%
\pgfsetstrokeopacity{0.200000}%
\pgfsetdash{}{0pt}%
\pgfpathmoveto{\pgfqpoint{5.907061in}{2.100838in}}%
\pgfpathlineto{\pgfqpoint{5.849815in}{4.522824in}}%
\pgfusepath{stroke}%
\end{pgfscope}%
\begin{pgfscope}%
\pgfpathrectangle{\pgfqpoint{0.481978in}{0.331635in}}{\pgfqpoint{9.300000in}{7.700000in}}%
\pgfusepath{clip}%
\pgfsetrectcap%
\pgfsetroundjoin%
\pgfsetlinewidth{1.505625pt}%
\definecolor{currentstroke}{rgb}{0.552941,0.898039,0.631373}%
\pgfsetstrokecolor{currentstroke}%
\pgfsetstrokeopacity{0.200000}%
\pgfsetdash{}{0pt}%
\pgfpathmoveto{\pgfqpoint{4.839240in}{5.911338in}}%
\pgfpathlineto{\pgfqpoint{5.849815in}{4.522824in}}%
\pgfusepath{stroke}%
\end{pgfscope}%
\begin{pgfscope}%
\pgfpathrectangle{\pgfqpoint{0.481978in}{0.331635in}}{\pgfqpoint{9.300000in}{7.700000in}}%
\pgfusepath{clip}%
\pgfsetrectcap%
\pgfsetroundjoin%
\pgfsetlinewidth{1.505625pt}%
\definecolor{currentstroke}{rgb}{0.552941,0.898039,0.631373}%
\pgfsetstrokecolor{currentstroke}%
\pgfsetstrokeopacity{0.200000}%
\pgfsetdash{}{0pt}%
\pgfpathmoveto{\pgfqpoint{2.381245in}{2.304607in}}%
\pgfpathlineto{\pgfqpoint{5.849815in}{4.522824in}}%
\pgfusepath{stroke}%
\end{pgfscope}%
\begin{pgfscope}%
\pgfpathrectangle{\pgfqpoint{0.481978in}{0.331635in}}{\pgfqpoint{9.300000in}{7.700000in}}%
\pgfusepath{clip}%
\pgfsetrectcap%
\pgfsetroundjoin%
\pgfsetlinewidth{1.505625pt}%
\definecolor{currentstroke}{rgb}{0.552941,0.898039,0.631373}%
\pgfsetstrokecolor{currentstroke}%
\pgfsetstrokeopacity{0.200000}%
\pgfsetdash{}{0pt}%
\pgfpathmoveto{\pgfqpoint{7.266400in}{2.402434in}}%
\pgfpathlineto{\pgfqpoint{5.849815in}{4.522824in}}%
\pgfusepath{stroke}%
\end{pgfscope}%
\begin{pgfscope}%
\pgfpathrectangle{\pgfqpoint{0.481978in}{0.331635in}}{\pgfqpoint{9.300000in}{7.700000in}}%
\pgfusepath{clip}%
\pgfsetrectcap%
\pgfsetroundjoin%
\pgfsetlinewidth{1.505625pt}%
\definecolor{currentstroke}{rgb}{0.552941,0.898039,0.631373}%
\pgfsetstrokecolor{currentstroke}%
\pgfsetstrokeopacity{0.200000}%
\pgfsetdash{}{0pt}%
\pgfpathmoveto{\pgfqpoint{3.410609in}{3.381276in}}%
\pgfpathlineto{\pgfqpoint{5.849815in}{4.522824in}}%
\pgfusepath{stroke}%
\end{pgfscope}%
\begin{pgfscope}%
\pgfpathrectangle{\pgfqpoint{0.481978in}{0.331635in}}{\pgfqpoint{9.300000in}{7.700000in}}%
\pgfusepath{clip}%
\pgfsetrectcap%
\pgfsetroundjoin%
\pgfsetlinewidth{1.505625pt}%
\definecolor{currentstroke}{rgb}{0.552941,0.898039,0.631373}%
\pgfsetstrokecolor{currentstroke}%
\pgfsetstrokeopacity{0.200000}%
\pgfsetdash{}{0pt}%
\pgfpathmoveto{\pgfqpoint{9.044900in}{5.310730in}}%
\pgfpathlineto{\pgfqpoint{5.849815in}{4.522824in}}%
\pgfusepath{stroke}%
\end{pgfscope}%
\begin{pgfscope}%
\pgfpathrectangle{\pgfqpoint{0.481978in}{0.331635in}}{\pgfqpoint{9.300000in}{7.700000in}}%
\pgfusepath{clip}%
\pgfsetrectcap%
\pgfsetroundjoin%
\pgfsetlinewidth{1.505625pt}%
\definecolor{currentstroke}{rgb}{0.552941,0.898039,0.631373}%
\pgfsetstrokecolor{currentstroke}%
\pgfsetstrokeopacity{0.200000}%
\pgfsetdash{}{0pt}%
\pgfpathmoveto{\pgfqpoint{8.024517in}{4.689990in}}%
\pgfpathlineto{\pgfqpoint{5.849815in}{4.522824in}}%
\pgfusepath{stroke}%
\end{pgfscope}%
\begin{pgfscope}%
\pgfpathrectangle{\pgfqpoint{0.481978in}{0.331635in}}{\pgfqpoint{9.300000in}{7.700000in}}%
\pgfusepath{clip}%
\pgfsetrectcap%
\pgfsetroundjoin%
\pgfsetlinewidth{1.505625pt}%
\definecolor{currentstroke}{rgb}{0.552941,0.898039,0.631373}%
\pgfsetstrokecolor{currentstroke}%
\pgfsetstrokeopacity{0.200000}%
\pgfsetdash{}{0pt}%
\pgfpathmoveto{\pgfqpoint{3.710429in}{4.755530in}}%
\pgfpathlineto{\pgfqpoint{5.849815in}{4.522824in}}%
\pgfusepath{stroke}%
\end{pgfscope}%
\begin{pgfscope}%
\pgfpathrectangle{\pgfqpoint{0.481978in}{0.331635in}}{\pgfqpoint{9.300000in}{7.700000in}}%
\pgfusepath{clip}%
\pgfsetrectcap%
\pgfsetroundjoin%
\pgfsetlinewidth{1.505625pt}%
\definecolor{currentstroke}{rgb}{0.552941,0.898039,0.631373}%
\pgfsetstrokecolor{currentstroke}%
\pgfsetstrokeopacity{0.200000}%
\pgfsetdash{}{0pt}%
\pgfpathmoveto{\pgfqpoint{5.690812in}{2.678714in}}%
\pgfpathlineto{\pgfqpoint{5.849815in}{4.522824in}}%
\pgfusepath{stroke}%
\end{pgfscope}%
\begin{pgfscope}%
\pgfpathrectangle{\pgfqpoint{0.481978in}{0.331635in}}{\pgfqpoint{9.300000in}{7.700000in}}%
\pgfusepath{clip}%
\pgfsetrectcap%
\pgfsetroundjoin%
\pgfsetlinewidth{1.505625pt}%
\definecolor{currentstroke}{rgb}{1.000000,0.623529,0.607843}%
\pgfsetstrokecolor{currentstroke}%
\pgfsetstrokeopacity{0.200000}%
\pgfsetdash{}{0pt}%
\pgfpathmoveto{\pgfqpoint{7.737423in}{5.882190in}}%
\pgfpathlineto{\pgfqpoint{5.342251in}{3.998981in}}%
\pgfusepath{stroke}%
\end{pgfscope}%
\begin{pgfscope}%
\pgfpathrectangle{\pgfqpoint{0.481978in}{0.331635in}}{\pgfqpoint{9.300000in}{7.700000in}}%
\pgfusepath{clip}%
\pgfsetrectcap%
\pgfsetroundjoin%
\pgfsetlinewidth{1.505625pt}%
\definecolor{currentstroke}{rgb}{1.000000,0.623529,0.607843}%
\pgfsetstrokecolor{currentstroke}%
\pgfsetstrokeopacity{0.200000}%
\pgfsetdash{}{0pt}%
\pgfpathmoveto{\pgfqpoint{8.478500in}{5.602262in}}%
\pgfpathlineto{\pgfqpoint{5.342251in}{3.998981in}}%
\pgfusepath{stroke}%
\end{pgfscope}%
\begin{pgfscope}%
\pgfpathrectangle{\pgfqpoint{0.481978in}{0.331635in}}{\pgfqpoint{9.300000in}{7.700000in}}%
\pgfusepath{clip}%
\pgfsetrectcap%
\pgfsetroundjoin%
\pgfsetlinewidth{1.505625pt}%
\definecolor{currentstroke}{rgb}{1.000000,0.623529,0.607843}%
\pgfsetstrokecolor{currentstroke}%
\pgfsetstrokeopacity{0.200000}%
\pgfsetdash{}{0pt}%
\pgfpathmoveto{\pgfqpoint{5.299357in}{2.456920in}}%
\pgfpathlineto{\pgfqpoint{5.342251in}{3.998981in}}%
\pgfusepath{stroke}%
\end{pgfscope}%
\begin{pgfscope}%
\pgfpathrectangle{\pgfqpoint{0.481978in}{0.331635in}}{\pgfqpoint{9.300000in}{7.700000in}}%
\pgfusepath{clip}%
\pgfsetrectcap%
\pgfsetroundjoin%
\pgfsetlinewidth{1.505625pt}%
\definecolor{currentstroke}{rgb}{1.000000,0.623529,0.607843}%
\pgfsetstrokecolor{currentstroke}%
\pgfsetstrokeopacity{0.200000}%
\pgfsetdash{}{0pt}%
\pgfpathmoveto{\pgfqpoint{5.050465in}{2.427924in}}%
\pgfpathlineto{\pgfqpoint{5.342251in}{3.998981in}}%
\pgfusepath{stroke}%
\end{pgfscope}%
\begin{pgfscope}%
\pgfpathrectangle{\pgfqpoint{0.481978in}{0.331635in}}{\pgfqpoint{9.300000in}{7.700000in}}%
\pgfusepath{clip}%
\pgfsetrectcap%
\pgfsetroundjoin%
\pgfsetlinewidth{1.505625pt}%
\definecolor{currentstroke}{rgb}{1.000000,0.623529,0.607843}%
\pgfsetstrokecolor{currentstroke}%
\pgfsetstrokeopacity{0.200000}%
\pgfsetdash{}{0pt}%
\pgfpathmoveto{\pgfqpoint{4.424855in}{3.491433in}}%
\pgfpathlineto{\pgfqpoint{5.342251in}{3.998981in}}%
\pgfusepath{stroke}%
\end{pgfscope}%
\begin{pgfscope}%
\pgfpathrectangle{\pgfqpoint{0.481978in}{0.331635in}}{\pgfqpoint{9.300000in}{7.700000in}}%
\pgfusepath{clip}%
\pgfsetrectcap%
\pgfsetroundjoin%
\pgfsetlinewidth{1.505625pt}%
\definecolor{currentstroke}{rgb}{1.000000,0.623529,0.607843}%
\pgfsetstrokecolor{currentstroke}%
\pgfsetstrokeopacity{0.200000}%
\pgfsetdash{}{0pt}%
\pgfpathmoveto{\pgfqpoint{4.703040in}{4.152158in}}%
\pgfpathlineto{\pgfqpoint{5.342251in}{3.998981in}}%
\pgfusepath{stroke}%
\end{pgfscope}%
\begin{pgfscope}%
\pgfpathrectangle{\pgfqpoint{0.481978in}{0.331635in}}{\pgfqpoint{9.300000in}{7.700000in}}%
\pgfusepath{clip}%
\pgfsetrectcap%
\pgfsetroundjoin%
\pgfsetlinewidth{1.505625pt}%
\definecolor{currentstroke}{rgb}{1.000000,0.623529,0.607843}%
\pgfsetstrokecolor{currentstroke}%
\pgfsetstrokeopacity{0.200000}%
\pgfsetdash{}{0pt}%
\pgfpathmoveto{\pgfqpoint{2.809853in}{3.881073in}}%
\pgfpathlineto{\pgfqpoint{5.342251in}{3.998981in}}%
\pgfusepath{stroke}%
\end{pgfscope}%
\begin{pgfscope}%
\pgfpathrectangle{\pgfqpoint{0.481978in}{0.331635in}}{\pgfqpoint{9.300000in}{7.700000in}}%
\pgfusepath{clip}%
\pgfsetrectcap%
\pgfsetroundjoin%
\pgfsetlinewidth{1.505625pt}%
\definecolor{currentstroke}{rgb}{1.000000,0.623529,0.607843}%
\pgfsetstrokecolor{currentstroke}%
\pgfsetstrokeopacity{0.200000}%
\pgfsetdash{}{0pt}%
\pgfpathmoveto{\pgfqpoint{3.971771in}{3.049785in}}%
\pgfpathlineto{\pgfqpoint{5.342251in}{3.998981in}}%
\pgfusepath{stroke}%
\end{pgfscope}%
\begin{pgfscope}%
\pgfpathrectangle{\pgfqpoint{0.481978in}{0.331635in}}{\pgfqpoint{9.300000in}{7.700000in}}%
\pgfusepath{clip}%
\pgfsetrectcap%
\pgfsetroundjoin%
\pgfsetlinewidth{1.505625pt}%
\definecolor{currentstroke}{rgb}{1.000000,0.623529,0.607843}%
\pgfsetstrokecolor{currentstroke}%
\pgfsetstrokeopacity{0.200000}%
\pgfsetdash{}{0pt}%
\pgfpathmoveto{\pgfqpoint{7.681317in}{5.102335in}}%
\pgfpathlineto{\pgfqpoint{5.342251in}{3.998981in}}%
\pgfusepath{stroke}%
\end{pgfscope}%
\begin{pgfscope}%
\pgfpathrectangle{\pgfqpoint{0.481978in}{0.331635in}}{\pgfqpoint{9.300000in}{7.700000in}}%
\pgfusepath{clip}%
\pgfsetrectcap%
\pgfsetroundjoin%
\pgfsetlinewidth{1.505625pt}%
\definecolor{currentstroke}{rgb}{1.000000,0.623529,0.607843}%
\pgfsetstrokecolor{currentstroke}%
\pgfsetstrokeopacity{0.200000}%
\pgfsetdash{}{0pt}%
\pgfpathmoveto{\pgfqpoint{4.922074in}{3.141067in}}%
\pgfpathlineto{\pgfqpoint{5.342251in}{3.998981in}}%
\pgfusepath{stroke}%
\end{pgfscope}%
\begin{pgfscope}%
\pgfpathrectangle{\pgfqpoint{0.481978in}{0.331635in}}{\pgfqpoint{9.300000in}{7.700000in}}%
\pgfusepath{clip}%
\pgfsetrectcap%
\pgfsetroundjoin%
\pgfsetlinewidth{1.505625pt}%
\definecolor{currentstroke}{rgb}{1.000000,0.623529,0.607843}%
\pgfsetstrokecolor{currentstroke}%
\pgfsetstrokeopacity{0.200000}%
\pgfsetdash{}{0pt}%
\pgfpathmoveto{\pgfqpoint{4.349051in}{3.146159in}}%
\pgfpathlineto{\pgfqpoint{5.342251in}{3.998981in}}%
\pgfusepath{stroke}%
\end{pgfscope}%
\begin{pgfscope}%
\pgfpathrectangle{\pgfqpoint{0.481978in}{0.331635in}}{\pgfqpoint{9.300000in}{7.700000in}}%
\pgfusepath{clip}%
\pgfsetrectcap%
\pgfsetroundjoin%
\pgfsetlinewidth{1.505625pt}%
\definecolor{currentstroke}{rgb}{1.000000,0.623529,0.607843}%
\pgfsetstrokecolor{currentstroke}%
\pgfsetstrokeopacity{0.200000}%
\pgfsetdash{}{0pt}%
\pgfpathmoveto{\pgfqpoint{4.285306in}{3.897257in}}%
\pgfpathlineto{\pgfqpoint{5.342251in}{3.998981in}}%
\pgfusepath{stroke}%
\end{pgfscope}%
\begin{pgfscope}%
\pgfpathrectangle{\pgfqpoint{0.481978in}{0.331635in}}{\pgfqpoint{9.300000in}{7.700000in}}%
\pgfusepath{clip}%
\pgfsetrectcap%
\pgfsetroundjoin%
\pgfsetlinewidth{1.505625pt}%
\definecolor{currentstroke}{rgb}{1.000000,0.623529,0.607843}%
\pgfsetstrokecolor{currentstroke}%
\pgfsetstrokeopacity{0.200000}%
\pgfsetdash{}{0pt}%
\pgfpathmoveto{\pgfqpoint{5.628759in}{1.359840in}}%
\pgfpathlineto{\pgfqpoint{5.342251in}{3.998981in}}%
\pgfusepath{stroke}%
\end{pgfscope}%
\begin{pgfscope}%
\pgfpathrectangle{\pgfqpoint{0.481978in}{0.331635in}}{\pgfqpoint{9.300000in}{7.700000in}}%
\pgfusepath{clip}%
\pgfsetrectcap%
\pgfsetroundjoin%
\pgfsetlinewidth{1.505625pt}%
\definecolor{currentstroke}{rgb}{1.000000,0.623529,0.607843}%
\pgfsetstrokecolor{currentstroke}%
\pgfsetstrokeopacity{0.200000}%
\pgfsetdash{}{0pt}%
\pgfpathmoveto{\pgfqpoint{2.208081in}{3.731947in}}%
\pgfpathlineto{\pgfqpoint{5.342251in}{3.998981in}}%
\pgfusepath{stroke}%
\end{pgfscope}%
\begin{pgfscope}%
\pgfpathrectangle{\pgfqpoint{0.481978in}{0.331635in}}{\pgfqpoint{9.300000in}{7.700000in}}%
\pgfusepath{clip}%
\pgfsetrectcap%
\pgfsetroundjoin%
\pgfsetlinewidth{1.505625pt}%
\definecolor{currentstroke}{rgb}{1.000000,0.623529,0.607843}%
\pgfsetstrokecolor{currentstroke}%
\pgfsetstrokeopacity{0.200000}%
\pgfsetdash{}{0pt}%
\pgfpathmoveto{\pgfqpoint{5.627987in}{1.357817in}}%
\pgfpathlineto{\pgfqpoint{5.342251in}{3.998981in}}%
\pgfusepath{stroke}%
\end{pgfscope}%
\begin{pgfscope}%
\pgfpathrectangle{\pgfqpoint{0.481978in}{0.331635in}}{\pgfqpoint{9.300000in}{7.700000in}}%
\pgfusepath{clip}%
\pgfsetrectcap%
\pgfsetroundjoin%
\pgfsetlinewidth{1.505625pt}%
\definecolor{currentstroke}{rgb}{1.000000,0.623529,0.607843}%
\pgfsetstrokecolor{currentstroke}%
\pgfsetstrokeopacity{0.200000}%
\pgfsetdash{}{0pt}%
\pgfpathmoveto{\pgfqpoint{7.678869in}{4.514349in}}%
\pgfpathlineto{\pgfqpoint{5.342251in}{3.998981in}}%
\pgfusepath{stroke}%
\end{pgfscope}%
\begin{pgfscope}%
\pgfpathrectangle{\pgfqpoint{0.481978in}{0.331635in}}{\pgfqpoint{9.300000in}{7.700000in}}%
\pgfusepath{clip}%
\pgfsetrectcap%
\pgfsetroundjoin%
\pgfsetlinewidth{1.505625pt}%
\definecolor{currentstroke}{rgb}{1.000000,0.623529,0.607843}%
\pgfsetstrokecolor{currentstroke}%
\pgfsetstrokeopacity{0.200000}%
\pgfsetdash{}{0pt}%
\pgfpathmoveto{\pgfqpoint{3.663088in}{2.022372in}}%
\pgfpathlineto{\pgfqpoint{5.342251in}{3.998981in}}%
\pgfusepath{stroke}%
\end{pgfscope}%
\begin{pgfscope}%
\pgfpathrectangle{\pgfqpoint{0.481978in}{0.331635in}}{\pgfqpoint{9.300000in}{7.700000in}}%
\pgfusepath{clip}%
\pgfsetrectcap%
\pgfsetroundjoin%
\pgfsetlinewidth{1.505625pt}%
\definecolor{currentstroke}{rgb}{1.000000,0.623529,0.607843}%
\pgfsetstrokecolor{currentstroke}%
\pgfsetstrokeopacity{0.200000}%
\pgfsetdash{}{0pt}%
\pgfpathmoveto{\pgfqpoint{2.974676in}{3.711091in}}%
\pgfpathlineto{\pgfqpoint{5.342251in}{3.998981in}}%
\pgfusepath{stroke}%
\end{pgfscope}%
\begin{pgfscope}%
\pgfpathrectangle{\pgfqpoint{0.481978in}{0.331635in}}{\pgfqpoint{9.300000in}{7.700000in}}%
\pgfusepath{clip}%
\pgfsetrectcap%
\pgfsetroundjoin%
\pgfsetlinewidth{1.505625pt}%
\definecolor{currentstroke}{rgb}{1.000000,0.623529,0.607843}%
\pgfsetstrokecolor{currentstroke}%
\pgfsetstrokeopacity{0.200000}%
\pgfsetdash{}{0pt}%
\pgfpathmoveto{\pgfqpoint{4.287456in}{3.591981in}}%
\pgfpathlineto{\pgfqpoint{5.342251in}{3.998981in}}%
\pgfusepath{stroke}%
\end{pgfscope}%
\begin{pgfscope}%
\pgfpathrectangle{\pgfqpoint{0.481978in}{0.331635in}}{\pgfqpoint{9.300000in}{7.700000in}}%
\pgfusepath{clip}%
\pgfsetrectcap%
\pgfsetroundjoin%
\pgfsetlinewidth{1.505625pt}%
\definecolor{currentstroke}{rgb}{1.000000,0.623529,0.607843}%
\pgfsetstrokecolor{currentstroke}%
\pgfsetstrokeopacity{0.200000}%
\pgfsetdash{}{0pt}%
\pgfpathmoveto{\pgfqpoint{1.967877in}{5.589834in}}%
\pgfpathlineto{\pgfqpoint{5.342251in}{3.998981in}}%
\pgfusepath{stroke}%
\end{pgfscope}%
\begin{pgfscope}%
\pgfpathrectangle{\pgfqpoint{0.481978in}{0.331635in}}{\pgfqpoint{9.300000in}{7.700000in}}%
\pgfusepath{clip}%
\pgfsetrectcap%
\pgfsetroundjoin%
\pgfsetlinewidth{1.505625pt}%
\definecolor{currentstroke}{rgb}{1.000000,0.623529,0.607843}%
\pgfsetstrokecolor{currentstroke}%
\pgfsetstrokeopacity{0.200000}%
\pgfsetdash{}{0pt}%
\pgfpathmoveto{\pgfqpoint{3.570734in}{2.934416in}}%
\pgfpathlineto{\pgfqpoint{5.342251in}{3.998981in}}%
\pgfusepath{stroke}%
\end{pgfscope}%
\begin{pgfscope}%
\pgfpathrectangle{\pgfqpoint{0.481978in}{0.331635in}}{\pgfqpoint{9.300000in}{7.700000in}}%
\pgfusepath{clip}%
\pgfsetrectcap%
\pgfsetroundjoin%
\pgfsetlinewidth{1.505625pt}%
\definecolor{currentstroke}{rgb}{1.000000,0.623529,0.607843}%
\pgfsetstrokecolor{currentstroke}%
\pgfsetstrokeopacity{0.200000}%
\pgfsetdash{}{0pt}%
\pgfpathmoveto{\pgfqpoint{5.661130in}{4.673478in}}%
\pgfpathlineto{\pgfqpoint{5.342251in}{3.998981in}}%
\pgfusepath{stroke}%
\end{pgfscope}%
\begin{pgfscope}%
\pgfpathrectangle{\pgfqpoint{0.481978in}{0.331635in}}{\pgfqpoint{9.300000in}{7.700000in}}%
\pgfusepath{clip}%
\pgfsetrectcap%
\pgfsetroundjoin%
\pgfsetlinewidth{1.505625pt}%
\definecolor{currentstroke}{rgb}{1.000000,0.623529,0.607843}%
\pgfsetstrokecolor{currentstroke}%
\pgfsetstrokeopacity{0.200000}%
\pgfsetdash{}{0pt}%
\pgfpathmoveto{\pgfqpoint{4.720327in}{4.363239in}}%
\pgfpathlineto{\pgfqpoint{5.342251in}{3.998981in}}%
\pgfusepath{stroke}%
\end{pgfscope}%
\begin{pgfscope}%
\pgfpathrectangle{\pgfqpoint{0.481978in}{0.331635in}}{\pgfqpoint{9.300000in}{7.700000in}}%
\pgfusepath{clip}%
\pgfsetrectcap%
\pgfsetroundjoin%
\pgfsetlinewidth{1.505625pt}%
\definecolor{currentstroke}{rgb}{1.000000,0.623529,0.607843}%
\pgfsetstrokecolor{currentstroke}%
\pgfsetstrokeopacity{0.200000}%
\pgfsetdash{}{0pt}%
\pgfpathmoveto{\pgfqpoint{2.339602in}{4.130396in}}%
\pgfpathlineto{\pgfqpoint{5.342251in}{3.998981in}}%
\pgfusepath{stroke}%
\end{pgfscope}%
\begin{pgfscope}%
\pgfpathrectangle{\pgfqpoint{0.481978in}{0.331635in}}{\pgfqpoint{9.300000in}{7.700000in}}%
\pgfusepath{clip}%
\pgfsetrectcap%
\pgfsetroundjoin%
\pgfsetlinewidth{1.505625pt}%
\definecolor{currentstroke}{rgb}{1.000000,0.623529,0.607843}%
\pgfsetstrokecolor{currentstroke}%
\pgfsetstrokeopacity{0.200000}%
\pgfsetdash{}{0pt}%
\pgfpathmoveto{\pgfqpoint{7.776459in}{5.380558in}}%
\pgfpathlineto{\pgfqpoint{5.342251in}{3.998981in}}%
\pgfusepath{stroke}%
\end{pgfscope}%
\begin{pgfscope}%
\pgfpathrectangle{\pgfqpoint{0.481978in}{0.331635in}}{\pgfqpoint{9.300000in}{7.700000in}}%
\pgfusepath{clip}%
\pgfsetrectcap%
\pgfsetroundjoin%
\pgfsetlinewidth{1.505625pt}%
\definecolor{currentstroke}{rgb}{1.000000,0.623529,0.607843}%
\pgfsetstrokecolor{currentstroke}%
\pgfsetstrokeopacity{0.200000}%
\pgfsetdash{}{0pt}%
\pgfpathmoveto{\pgfqpoint{5.832034in}{1.507144in}}%
\pgfpathlineto{\pgfqpoint{5.342251in}{3.998981in}}%
\pgfusepath{stroke}%
\end{pgfscope}%
\begin{pgfscope}%
\pgfpathrectangle{\pgfqpoint{0.481978in}{0.331635in}}{\pgfqpoint{9.300000in}{7.700000in}}%
\pgfusepath{clip}%
\pgfsetrectcap%
\pgfsetroundjoin%
\pgfsetlinewidth{1.505625pt}%
\definecolor{currentstroke}{rgb}{1.000000,0.623529,0.607843}%
\pgfsetstrokecolor{currentstroke}%
\pgfsetstrokeopacity{0.200000}%
\pgfsetdash{}{0pt}%
\pgfpathmoveto{\pgfqpoint{8.011004in}{6.276407in}}%
\pgfpathlineto{\pgfqpoint{5.342251in}{3.998981in}}%
\pgfusepath{stroke}%
\end{pgfscope}%
\begin{pgfscope}%
\pgfpathrectangle{\pgfqpoint{0.481978in}{0.331635in}}{\pgfqpoint{9.300000in}{7.700000in}}%
\pgfusepath{clip}%
\pgfsetrectcap%
\pgfsetroundjoin%
\pgfsetlinewidth{1.505625pt}%
\definecolor{currentstroke}{rgb}{1.000000,0.623529,0.607843}%
\pgfsetstrokecolor{currentstroke}%
\pgfsetstrokeopacity{0.200000}%
\pgfsetdash{}{0pt}%
\pgfpathmoveto{\pgfqpoint{2.908514in}{3.329410in}}%
\pgfpathlineto{\pgfqpoint{5.342251in}{3.998981in}}%
\pgfusepath{stroke}%
\end{pgfscope}%
\begin{pgfscope}%
\pgfpathrectangle{\pgfqpoint{0.481978in}{0.331635in}}{\pgfqpoint{9.300000in}{7.700000in}}%
\pgfusepath{clip}%
\pgfsetrectcap%
\pgfsetroundjoin%
\pgfsetlinewidth{1.505625pt}%
\definecolor{currentstroke}{rgb}{1.000000,0.623529,0.607843}%
\pgfsetstrokecolor{currentstroke}%
\pgfsetstrokeopacity{0.200000}%
\pgfsetdash{}{0pt}%
\pgfpathmoveto{\pgfqpoint{6.465338in}{2.366611in}}%
\pgfpathlineto{\pgfqpoint{5.342251in}{3.998981in}}%
\pgfusepath{stroke}%
\end{pgfscope}%
\begin{pgfscope}%
\pgfpathrectangle{\pgfqpoint{0.481978in}{0.331635in}}{\pgfqpoint{9.300000in}{7.700000in}}%
\pgfusepath{clip}%
\pgfsetrectcap%
\pgfsetroundjoin%
\pgfsetlinewidth{1.505625pt}%
\definecolor{currentstroke}{rgb}{1.000000,0.623529,0.607843}%
\pgfsetstrokecolor{currentstroke}%
\pgfsetstrokeopacity{0.200000}%
\pgfsetdash{}{0pt}%
\pgfpathmoveto{\pgfqpoint{8.106589in}{6.343346in}}%
\pgfpathlineto{\pgfqpoint{5.342251in}{3.998981in}}%
\pgfusepath{stroke}%
\end{pgfscope}%
\begin{pgfscope}%
\pgfpathrectangle{\pgfqpoint{0.481978in}{0.331635in}}{\pgfqpoint{9.300000in}{7.700000in}}%
\pgfusepath{clip}%
\pgfsetrectcap%
\pgfsetroundjoin%
\pgfsetlinewidth{1.505625pt}%
\definecolor{currentstroke}{rgb}{1.000000,0.623529,0.607843}%
\pgfsetstrokecolor{currentstroke}%
\pgfsetstrokeopacity{0.200000}%
\pgfsetdash{}{0pt}%
\pgfpathmoveto{\pgfqpoint{7.991875in}{5.264052in}}%
\pgfpathlineto{\pgfqpoint{5.342251in}{3.998981in}}%
\pgfusepath{stroke}%
\end{pgfscope}%
\begin{pgfscope}%
\pgfpathrectangle{\pgfqpoint{0.481978in}{0.331635in}}{\pgfqpoint{9.300000in}{7.700000in}}%
\pgfusepath{clip}%
\pgfsetrectcap%
\pgfsetroundjoin%
\pgfsetlinewidth{1.505625pt}%
\definecolor{currentstroke}{rgb}{1.000000,0.623529,0.607843}%
\pgfsetstrokecolor{currentstroke}%
\pgfsetstrokeopacity{0.200000}%
\pgfsetdash{}{0pt}%
\pgfpathmoveto{\pgfqpoint{7.423833in}{5.573837in}}%
\pgfpathlineto{\pgfqpoint{5.342251in}{3.998981in}}%
\pgfusepath{stroke}%
\end{pgfscope}%
\begin{pgfscope}%
\pgfpathrectangle{\pgfqpoint{0.481978in}{0.331635in}}{\pgfqpoint{9.300000in}{7.700000in}}%
\pgfusepath{clip}%
\pgfsetrectcap%
\pgfsetroundjoin%
\pgfsetlinewidth{1.505625pt}%
\definecolor{currentstroke}{rgb}{1.000000,0.623529,0.607843}%
\pgfsetstrokecolor{currentstroke}%
\pgfsetstrokeopacity{0.200000}%
\pgfsetdash{}{0pt}%
\pgfpathmoveto{\pgfqpoint{4.192970in}{5.950908in}}%
\pgfpathlineto{\pgfqpoint{5.342251in}{3.998981in}}%
\pgfusepath{stroke}%
\end{pgfscope}%
\begin{pgfscope}%
\pgfpathrectangle{\pgfqpoint{0.481978in}{0.331635in}}{\pgfqpoint{9.300000in}{7.700000in}}%
\pgfusepath{clip}%
\pgfsetrectcap%
\pgfsetroundjoin%
\pgfsetlinewidth{1.505625pt}%
\definecolor{currentstroke}{rgb}{1.000000,0.623529,0.607843}%
\pgfsetstrokecolor{currentstroke}%
\pgfsetstrokeopacity{0.200000}%
\pgfsetdash{}{0pt}%
\pgfpathmoveto{\pgfqpoint{6.770367in}{5.489298in}}%
\pgfpathlineto{\pgfqpoint{5.342251in}{3.998981in}}%
\pgfusepath{stroke}%
\end{pgfscope}%
\begin{pgfscope}%
\pgfpathrectangle{\pgfqpoint{0.481978in}{0.331635in}}{\pgfqpoint{9.300000in}{7.700000in}}%
\pgfusepath{clip}%
\pgfsetrectcap%
\pgfsetroundjoin%
\pgfsetlinewidth{1.505625pt}%
\definecolor{currentstroke}{rgb}{1.000000,0.623529,0.607843}%
\pgfsetstrokecolor{currentstroke}%
\pgfsetstrokeopacity{0.200000}%
\pgfsetdash{}{0pt}%
\pgfpathmoveto{\pgfqpoint{7.446866in}{5.698685in}}%
\pgfpathlineto{\pgfqpoint{5.342251in}{3.998981in}}%
\pgfusepath{stroke}%
\end{pgfscope}%
\begin{pgfscope}%
\pgfpathrectangle{\pgfqpoint{0.481978in}{0.331635in}}{\pgfqpoint{9.300000in}{7.700000in}}%
\pgfusepath{clip}%
\pgfsetrectcap%
\pgfsetroundjoin%
\pgfsetlinewidth{1.505625pt}%
\definecolor{currentstroke}{rgb}{1.000000,0.623529,0.607843}%
\pgfsetstrokecolor{currentstroke}%
\pgfsetstrokeopacity{0.200000}%
\pgfsetdash{}{0pt}%
\pgfpathmoveto{\pgfqpoint{4.422769in}{4.172874in}}%
\pgfpathlineto{\pgfqpoint{5.342251in}{3.998981in}}%
\pgfusepath{stroke}%
\end{pgfscope}%
\begin{pgfscope}%
\pgfpathrectangle{\pgfqpoint{0.481978in}{0.331635in}}{\pgfqpoint{9.300000in}{7.700000in}}%
\pgfusepath{clip}%
\pgfsetrectcap%
\pgfsetroundjoin%
\pgfsetlinewidth{1.505625pt}%
\definecolor{currentstroke}{rgb}{1.000000,0.623529,0.607843}%
\pgfsetstrokecolor{currentstroke}%
\pgfsetstrokeopacity{0.200000}%
\pgfsetdash{}{0pt}%
\pgfpathmoveto{\pgfqpoint{4.632997in}{1.495729in}}%
\pgfpathlineto{\pgfqpoint{5.342251in}{3.998981in}}%
\pgfusepath{stroke}%
\end{pgfscope}%
\begin{pgfscope}%
\pgfpathrectangle{\pgfqpoint{0.481978in}{0.331635in}}{\pgfqpoint{9.300000in}{7.700000in}}%
\pgfusepath{clip}%
\pgfsetrectcap%
\pgfsetroundjoin%
\pgfsetlinewidth{1.505625pt}%
\definecolor{currentstroke}{rgb}{1.000000,0.623529,0.607843}%
\pgfsetstrokecolor{currentstroke}%
\pgfsetstrokeopacity{0.200000}%
\pgfsetdash{}{0pt}%
\pgfpathmoveto{\pgfqpoint{3.319473in}{2.345162in}}%
\pgfpathlineto{\pgfqpoint{5.342251in}{3.998981in}}%
\pgfusepath{stroke}%
\end{pgfscope}%
\begin{pgfscope}%
\pgfpathrectangle{\pgfqpoint{0.481978in}{0.331635in}}{\pgfqpoint{9.300000in}{7.700000in}}%
\pgfusepath{clip}%
\pgfsetrectcap%
\pgfsetroundjoin%
\pgfsetlinewidth{1.505625pt}%
\definecolor{currentstroke}{rgb}{1.000000,0.623529,0.607843}%
\pgfsetstrokecolor{currentstroke}%
\pgfsetstrokeopacity{0.200000}%
\pgfsetdash{}{0pt}%
\pgfpathmoveto{\pgfqpoint{3.661029in}{2.526550in}}%
\pgfpathlineto{\pgfqpoint{5.342251in}{3.998981in}}%
\pgfusepath{stroke}%
\end{pgfscope}%
\begin{pgfscope}%
\pgfpathrectangle{\pgfqpoint{0.481978in}{0.331635in}}{\pgfqpoint{9.300000in}{7.700000in}}%
\pgfusepath{clip}%
\pgfsetrectcap%
\pgfsetroundjoin%
\pgfsetlinewidth{1.505625pt}%
\definecolor{currentstroke}{rgb}{1.000000,0.623529,0.607843}%
\pgfsetstrokecolor{currentstroke}%
\pgfsetstrokeopacity{0.200000}%
\pgfsetdash{}{0pt}%
\pgfpathmoveto{\pgfqpoint{3.845381in}{3.244228in}}%
\pgfpathlineto{\pgfqpoint{5.342251in}{3.998981in}}%
\pgfusepath{stroke}%
\end{pgfscope}%
\begin{pgfscope}%
\pgfpathrectangle{\pgfqpoint{0.481978in}{0.331635in}}{\pgfqpoint{9.300000in}{7.700000in}}%
\pgfusepath{clip}%
\pgfsetrectcap%
\pgfsetroundjoin%
\pgfsetlinewidth{1.505625pt}%
\definecolor{currentstroke}{rgb}{1.000000,0.623529,0.607843}%
\pgfsetstrokecolor{currentstroke}%
\pgfsetstrokeopacity{0.200000}%
\pgfsetdash{}{0pt}%
\pgfpathmoveto{\pgfqpoint{4.050059in}{3.536130in}}%
\pgfpathlineto{\pgfqpoint{5.342251in}{3.998981in}}%
\pgfusepath{stroke}%
\end{pgfscope}%
\begin{pgfscope}%
\pgfpathrectangle{\pgfqpoint{0.481978in}{0.331635in}}{\pgfqpoint{9.300000in}{7.700000in}}%
\pgfusepath{clip}%
\pgfsetrectcap%
\pgfsetroundjoin%
\pgfsetlinewidth{1.505625pt}%
\definecolor{currentstroke}{rgb}{1.000000,0.623529,0.607843}%
\pgfsetstrokecolor{currentstroke}%
\pgfsetstrokeopacity{0.200000}%
\pgfsetdash{}{0pt}%
\pgfpathmoveto{\pgfqpoint{4.925237in}{7.465473in}}%
\pgfpathlineto{\pgfqpoint{5.342251in}{3.998981in}}%
\pgfusepath{stroke}%
\end{pgfscope}%
\begin{pgfscope}%
\pgfpathrectangle{\pgfqpoint{0.481978in}{0.331635in}}{\pgfqpoint{9.300000in}{7.700000in}}%
\pgfusepath{clip}%
\pgfsetrectcap%
\pgfsetroundjoin%
\pgfsetlinewidth{1.505625pt}%
\definecolor{currentstroke}{rgb}{1.000000,0.623529,0.607843}%
\pgfsetstrokecolor{currentstroke}%
\pgfsetstrokeopacity{0.200000}%
\pgfsetdash{}{0pt}%
\pgfpathmoveto{\pgfqpoint{6.991308in}{3.587991in}}%
\pgfpathlineto{\pgfqpoint{5.342251in}{3.998981in}}%
\pgfusepath{stroke}%
\end{pgfscope}%
\begin{pgfscope}%
\pgfpathrectangle{\pgfqpoint{0.481978in}{0.331635in}}{\pgfqpoint{9.300000in}{7.700000in}}%
\pgfusepath{clip}%
\pgfsetrectcap%
\pgfsetroundjoin%
\pgfsetlinewidth{1.505625pt}%
\definecolor{currentstroke}{rgb}{1.000000,0.623529,0.607843}%
\pgfsetstrokecolor{currentstroke}%
\pgfsetstrokeopacity{0.200000}%
\pgfsetdash{}{0pt}%
\pgfpathmoveto{\pgfqpoint{8.618470in}{6.062569in}}%
\pgfpathlineto{\pgfqpoint{5.342251in}{3.998981in}}%
\pgfusepath{stroke}%
\end{pgfscope}%
\begin{pgfscope}%
\pgfpathrectangle{\pgfqpoint{0.481978in}{0.331635in}}{\pgfqpoint{9.300000in}{7.700000in}}%
\pgfusepath{clip}%
\pgfsetrectcap%
\pgfsetroundjoin%
\pgfsetlinewidth{1.505625pt}%
\definecolor{currentstroke}{rgb}{1.000000,0.623529,0.607843}%
\pgfsetstrokecolor{currentstroke}%
\pgfsetstrokeopacity{0.200000}%
\pgfsetdash{}{0pt}%
\pgfpathmoveto{\pgfqpoint{7.337079in}{5.258872in}}%
\pgfpathlineto{\pgfqpoint{5.342251in}{3.998981in}}%
\pgfusepath{stroke}%
\end{pgfscope}%
\begin{pgfscope}%
\pgfpathrectangle{\pgfqpoint{0.481978in}{0.331635in}}{\pgfqpoint{9.300000in}{7.700000in}}%
\pgfusepath{clip}%
\pgfsetrectcap%
\pgfsetroundjoin%
\pgfsetlinewidth{1.505625pt}%
\definecolor{currentstroke}{rgb}{1.000000,0.623529,0.607843}%
\pgfsetstrokecolor{currentstroke}%
\pgfsetstrokeopacity{0.200000}%
\pgfsetdash{}{0pt}%
\pgfpathmoveto{\pgfqpoint{4.553085in}{3.023307in}}%
\pgfpathlineto{\pgfqpoint{5.342251in}{3.998981in}}%
\pgfusepath{stroke}%
\end{pgfscope}%
\begin{pgfscope}%
\pgfpathrectangle{\pgfqpoint{0.481978in}{0.331635in}}{\pgfqpoint{9.300000in}{7.700000in}}%
\pgfusepath{clip}%
\pgfsetrectcap%
\pgfsetroundjoin%
\pgfsetlinewidth{1.505625pt}%
\definecolor{currentstroke}{rgb}{1.000000,0.623529,0.607843}%
\pgfsetstrokecolor{currentstroke}%
\pgfsetstrokeopacity{0.200000}%
\pgfsetdash{}{0pt}%
\pgfpathmoveto{\pgfqpoint{9.095716in}{5.890233in}}%
\pgfpathlineto{\pgfqpoint{5.342251in}{3.998981in}}%
\pgfusepath{stroke}%
\end{pgfscope}%
\begin{pgfscope}%
\pgfpathrectangle{\pgfqpoint{0.481978in}{0.331635in}}{\pgfqpoint{9.300000in}{7.700000in}}%
\pgfusepath{clip}%
\pgfsetrectcap%
\pgfsetroundjoin%
\pgfsetlinewidth{1.505625pt}%
\definecolor{currentstroke}{rgb}{1.000000,0.623529,0.607843}%
\pgfsetstrokecolor{currentstroke}%
\pgfsetstrokeopacity{0.200000}%
\pgfsetdash{}{0pt}%
\pgfpathmoveto{\pgfqpoint{2.403697in}{3.472029in}}%
\pgfpathlineto{\pgfqpoint{5.342251in}{3.998981in}}%
\pgfusepath{stroke}%
\end{pgfscope}%
\begin{pgfscope}%
\pgfpathrectangle{\pgfqpoint{0.481978in}{0.331635in}}{\pgfqpoint{9.300000in}{7.700000in}}%
\pgfusepath{clip}%
\pgfsetrectcap%
\pgfsetroundjoin%
\pgfsetlinewidth{1.505625pt}%
\definecolor{currentstroke}{rgb}{1.000000,0.623529,0.607843}%
\pgfsetstrokecolor{currentstroke}%
\pgfsetstrokeopacity{0.200000}%
\pgfsetdash{}{0pt}%
\pgfpathmoveto{\pgfqpoint{7.815154in}{4.452079in}}%
\pgfpathlineto{\pgfqpoint{5.342251in}{3.998981in}}%
\pgfusepath{stroke}%
\end{pgfscope}%
\begin{pgfscope}%
\pgfpathrectangle{\pgfqpoint{0.481978in}{0.331635in}}{\pgfqpoint{9.300000in}{7.700000in}}%
\pgfusepath{clip}%
\pgfsetrectcap%
\pgfsetroundjoin%
\pgfsetlinewidth{1.505625pt}%
\definecolor{currentstroke}{rgb}{1.000000,0.623529,0.607843}%
\pgfsetstrokecolor{currentstroke}%
\pgfsetstrokeopacity{0.200000}%
\pgfsetdash{}{0pt}%
\pgfpathmoveto{\pgfqpoint{4.473626in}{2.024232in}}%
\pgfpathlineto{\pgfqpoint{5.342251in}{3.998981in}}%
\pgfusepath{stroke}%
\end{pgfscope}%
\begin{pgfscope}%
\pgfpathrectangle{\pgfqpoint{0.481978in}{0.331635in}}{\pgfqpoint{9.300000in}{7.700000in}}%
\pgfusepath{clip}%
\pgfsetrectcap%
\pgfsetroundjoin%
\pgfsetlinewidth{1.505625pt}%
\definecolor{currentstroke}{rgb}{0.815686,0.733333,1.000000}%
\pgfsetstrokecolor{currentstroke}%
\pgfsetstrokeopacity{0.800000}%
\pgfsetdash{}{0pt}%
\pgfpathmoveto{\pgfqpoint{7.152848in}{5.392946in}}%
\pgfpathlineto{\pgfqpoint{5.293604in}{4.165376in}}%
\pgfusepath{stroke}%
\end{pgfscope}%
\begin{pgfscope}%
\pgfpathrectangle{\pgfqpoint{0.481978in}{0.331635in}}{\pgfqpoint{9.300000in}{7.700000in}}%
\pgfusepath{clip}%
\pgfsetrectcap%
\pgfsetroundjoin%
\pgfsetlinewidth{1.505625pt}%
\definecolor{currentstroke}{rgb}{0.815686,0.733333,1.000000}%
\pgfsetstrokecolor{currentstroke}%
\pgfsetstrokeopacity{0.800000}%
\pgfsetdash{}{0pt}%
\pgfpathmoveto{\pgfqpoint{7.702166in}{5.107203in}}%
\pgfpathlineto{\pgfqpoint{5.293604in}{4.165376in}}%
\pgfusepath{stroke}%
\end{pgfscope}%
\begin{pgfscope}%
\pgfpathrectangle{\pgfqpoint{0.481978in}{0.331635in}}{\pgfqpoint{9.300000in}{7.700000in}}%
\pgfusepath{clip}%
\pgfsetrectcap%
\pgfsetroundjoin%
\pgfsetlinewidth{1.505625pt}%
\definecolor{currentstroke}{rgb}{0.815686,0.733333,1.000000}%
\pgfsetstrokecolor{currentstroke}%
\pgfsetstrokeopacity{0.800000}%
\pgfsetdash{}{0pt}%
\pgfpathmoveto{\pgfqpoint{2.872669in}{3.217019in}}%
\pgfpathlineto{\pgfqpoint{5.293604in}{4.165376in}}%
\pgfusepath{stroke}%
\end{pgfscope}%
\begin{pgfscope}%
\pgfpathrectangle{\pgfqpoint{0.481978in}{0.331635in}}{\pgfqpoint{9.300000in}{7.700000in}}%
\pgfusepath{clip}%
\pgfsetrectcap%
\pgfsetroundjoin%
\pgfsetlinewidth{1.505625pt}%
\definecolor{currentstroke}{rgb}{0.815686,0.733333,1.000000}%
\pgfsetstrokecolor{currentstroke}%
\pgfsetstrokeopacity{0.800000}%
\pgfsetdash{}{0pt}%
\pgfpathmoveto{\pgfqpoint{3.619377in}{2.362886in}}%
\pgfpathlineto{\pgfqpoint{5.293604in}{4.165376in}}%
\pgfusepath{stroke}%
\end{pgfscope}%
\begin{pgfscope}%
\pgfpathrectangle{\pgfqpoint{0.481978in}{0.331635in}}{\pgfqpoint{9.300000in}{7.700000in}}%
\pgfusepath{clip}%
\pgfsetrectcap%
\pgfsetroundjoin%
\pgfsetlinewidth{1.505625pt}%
\definecolor{currentstroke}{rgb}{0.815686,0.733333,1.000000}%
\pgfsetstrokecolor{currentstroke}%
\pgfsetstrokeopacity{0.800000}%
\pgfsetdash{}{0pt}%
\pgfpathmoveto{\pgfqpoint{3.448106in}{2.512230in}}%
\pgfpathlineto{\pgfqpoint{5.293604in}{4.165376in}}%
\pgfusepath{stroke}%
\end{pgfscope}%
\begin{pgfscope}%
\pgfpathrectangle{\pgfqpoint{0.481978in}{0.331635in}}{\pgfqpoint{9.300000in}{7.700000in}}%
\pgfusepath{clip}%
\pgfsetrectcap%
\pgfsetroundjoin%
\pgfsetlinewidth{1.505625pt}%
\definecolor{currentstroke}{rgb}{0.815686,0.733333,1.000000}%
\pgfsetstrokecolor{currentstroke}%
\pgfsetstrokeopacity{0.800000}%
\pgfsetdash{}{0pt}%
\pgfpathmoveto{\pgfqpoint{6.968468in}{3.346064in}}%
\pgfpathlineto{\pgfqpoint{5.293604in}{4.165376in}}%
\pgfusepath{stroke}%
\end{pgfscope}%
\begin{pgfscope}%
\pgfpathrectangle{\pgfqpoint{0.481978in}{0.331635in}}{\pgfqpoint{9.300000in}{7.700000in}}%
\pgfusepath{clip}%
\pgfsetrectcap%
\pgfsetroundjoin%
\pgfsetlinewidth{1.505625pt}%
\definecolor{currentstroke}{rgb}{0.815686,0.733333,1.000000}%
\pgfsetstrokecolor{currentstroke}%
\pgfsetstrokeopacity{0.800000}%
\pgfsetdash{}{0pt}%
\pgfpathmoveto{\pgfqpoint{3.260742in}{7.031269in}}%
\pgfpathlineto{\pgfqpoint{5.293604in}{4.165376in}}%
\pgfusepath{stroke}%
\end{pgfscope}%
\begin{pgfscope}%
\pgfpathrectangle{\pgfqpoint{0.481978in}{0.331635in}}{\pgfqpoint{9.300000in}{7.700000in}}%
\pgfusepath{clip}%
\pgfsetrectcap%
\pgfsetroundjoin%
\pgfsetlinewidth{1.505625pt}%
\definecolor{currentstroke}{rgb}{0.815686,0.733333,1.000000}%
\pgfsetstrokecolor{currentstroke}%
\pgfsetstrokeopacity{0.800000}%
\pgfsetdash{}{0pt}%
\pgfpathmoveto{\pgfqpoint{3.172404in}{5.775014in}}%
\pgfpathlineto{\pgfqpoint{5.293604in}{4.165376in}}%
\pgfusepath{stroke}%
\end{pgfscope}%
\begin{pgfscope}%
\pgfpathrectangle{\pgfqpoint{0.481978in}{0.331635in}}{\pgfqpoint{9.300000in}{7.700000in}}%
\pgfusepath{clip}%
\pgfsetrectcap%
\pgfsetroundjoin%
\pgfsetlinewidth{1.505625pt}%
\definecolor{currentstroke}{rgb}{0.815686,0.733333,1.000000}%
\pgfsetstrokecolor{currentstroke}%
\pgfsetstrokeopacity{0.800000}%
\pgfsetdash{}{0pt}%
\pgfpathmoveto{\pgfqpoint{2.953247in}{2.700897in}}%
\pgfpathlineto{\pgfqpoint{5.293604in}{4.165376in}}%
\pgfusepath{stroke}%
\end{pgfscope}%
\begin{pgfscope}%
\pgfpathrectangle{\pgfqpoint{0.481978in}{0.331635in}}{\pgfqpoint{9.300000in}{7.700000in}}%
\pgfusepath{clip}%
\pgfsetrectcap%
\pgfsetroundjoin%
\pgfsetlinewidth{1.505625pt}%
\definecolor{currentstroke}{rgb}{0.815686,0.733333,1.000000}%
\pgfsetstrokecolor{currentstroke}%
\pgfsetstrokeopacity{0.800000}%
\pgfsetdash{}{0pt}%
\pgfpathmoveto{\pgfqpoint{8.116420in}{5.616284in}}%
\pgfpathlineto{\pgfqpoint{5.293604in}{4.165376in}}%
\pgfusepath{stroke}%
\end{pgfscope}%
\begin{pgfscope}%
\pgfpathrectangle{\pgfqpoint{0.481978in}{0.331635in}}{\pgfqpoint{9.300000in}{7.700000in}}%
\pgfusepath{clip}%
\pgfsetrectcap%
\pgfsetroundjoin%
\pgfsetlinewidth{1.505625pt}%
\definecolor{currentstroke}{rgb}{0.815686,0.733333,1.000000}%
\pgfsetstrokecolor{currentstroke}%
\pgfsetstrokeopacity{0.800000}%
\pgfsetdash{}{0pt}%
\pgfpathmoveto{\pgfqpoint{6.906230in}{4.730496in}}%
\pgfpathlineto{\pgfqpoint{5.293604in}{4.165376in}}%
\pgfusepath{stroke}%
\end{pgfscope}%
\begin{pgfscope}%
\pgfpathrectangle{\pgfqpoint{0.481978in}{0.331635in}}{\pgfqpoint{9.300000in}{7.700000in}}%
\pgfusepath{clip}%
\pgfsetrectcap%
\pgfsetroundjoin%
\pgfsetlinewidth{1.505625pt}%
\definecolor{currentstroke}{rgb}{0.815686,0.733333,1.000000}%
\pgfsetstrokecolor{currentstroke}%
\pgfsetstrokeopacity{0.800000}%
\pgfsetdash{}{0pt}%
\pgfpathmoveto{\pgfqpoint{3.815826in}{4.087628in}}%
\pgfpathlineto{\pgfqpoint{5.293604in}{4.165376in}}%
\pgfusepath{stroke}%
\end{pgfscope}%
\begin{pgfscope}%
\pgfpathrectangle{\pgfqpoint{0.481978in}{0.331635in}}{\pgfqpoint{9.300000in}{7.700000in}}%
\pgfusepath{clip}%
\pgfsetrectcap%
\pgfsetroundjoin%
\pgfsetlinewidth{1.505625pt}%
\definecolor{currentstroke}{rgb}{0.815686,0.733333,1.000000}%
\pgfsetstrokecolor{currentstroke}%
\pgfsetstrokeopacity{0.800000}%
\pgfsetdash{}{0pt}%
\pgfpathmoveto{\pgfqpoint{6.962413in}{5.802045in}}%
\pgfpathlineto{\pgfqpoint{5.293604in}{4.165376in}}%
\pgfusepath{stroke}%
\end{pgfscope}%
\begin{pgfscope}%
\pgfpathrectangle{\pgfqpoint{0.481978in}{0.331635in}}{\pgfqpoint{9.300000in}{7.700000in}}%
\pgfusepath{clip}%
\pgfsetrectcap%
\pgfsetroundjoin%
\pgfsetlinewidth{1.505625pt}%
\definecolor{currentstroke}{rgb}{0.815686,0.733333,1.000000}%
\pgfsetstrokecolor{currentstroke}%
\pgfsetstrokeopacity{0.800000}%
\pgfsetdash{}{0pt}%
\pgfpathmoveto{\pgfqpoint{7.236127in}{5.393044in}}%
\pgfpathlineto{\pgfqpoint{5.293604in}{4.165376in}}%
\pgfusepath{stroke}%
\end{pgfscope}%
\begin{pgfscope}%
\pgfpathrectangle{\pgfqpoint{0.481978in}{0.331635in}}{\pgfqpoint{9.300000in}{7.700000in}}%
\pgfusepath{clip}%
\pgfsetrectcap%
\pgfsetroundjoin%
\pgfsetlinewidth{1.505625pt}%
\definecolor{currentstroke}{rgb}{0.815686,0.733333,1.000000}%
\pgfsetstrokecolor{currentstroke}%
\pgfsetstrokeopacity{0.800000}%
\pgfsetdash{}{0pt}%
\pgfpathmoveto{\pgfqpoint{8.116846in}{6.524386in}}%
\pgfpathlineto{\pgfqpoint{5.293604in}{4.165376in}}%
\pgfusepath{stroke}%
\end{pgfscope}%
\begin{pgfscope}%
\pgfpathrectangle{\pgfqpoint{0.481978in}{0.331635in}}{\pgfqpoint{9.300000in}{7.700000in}}%
\pgfusepath{clip}%
\pgfsetrectcap%
\pgfsetroundjoin%
\pgfsetlinewidth{1.505625pt}%
\definecolor{currentstroke}{rgb}{0.815686,0.733333,1.000000}%
\pgfsetstrokecolor{currentstroke}%
\pgfsetstrokeopacity{0.800000}%
\pgfsetdash{}{0pt}%
\pgfpathmoveto{\pgfqpoint{3.449504in}{2.766701in}}%
\pgfpathlineto{\pgfqpoint{5.293604in}{4.165376in}}%
\pgfusepath{stroke}%
\end{pgfscope}%
\begin{pgfscope}%
\pgfpathrectangle{\pgfqpoint{0.481978in}{0.331635in}}{\pgfqpoint{9.300000in}{7.700000in}}%
\pgfusepath{clip}%
\pgfsetrectcap%
\pgfsetroundjoin%
\pgfsetlinewidth{1.505625pt}%
\definecolor{currentstroke}{rgb}{0.815686,0.733333,1.000000}%
\pgfsetstrokecolor{currentstroke}%
\pgfsetstrokeopacity{0.800000}%
\pgfsetdash{}{0pt}%
\pgfpathmoveto{\pgfqpoint{4.947086in}{5.421637in}}%
\pgfpathlineto{\pgfqpoint{5.293604in}{4.165376in}}%
\pgfusepath{stroke}%
\end{pgfscope}%
\begin{pgfscope}%
\pgfpathrectangle{\pgfqpoint{0.481978in}{0.331635in}}{\pgfqpoint{9.300000in}{7.700000in}}%
\pgfusepath{clip}%
\pgfsetrectcap%
\pgfsetroundjoin%
\pgfsetlinewidth{1.505625pt}%
\definecolor{currentstroke}{rgb}{0.815686,0.733333,1.000000}%
\pgfsetstrokecolor{currentstroke}%
\pgfsetstrokeopacity{0.800000}%
\pgfsetdash{}{0pt}%
\pgfpathmoveto{\pgfqpoint{6.925838in}{2.915473in}}%
\pgfpathlineto{\pgfqpoint{5.293604in}{4.165376in}}%
\pgfusepath{stroke}%
\end{pgfscope}%
\begin{pgfscope}%
\pgfpathrectangle{\pgfqpoint{0.481978in}{0.331635in}}{\pgfqpoint{9.300000in}{7.700000in}}%
\pgfusepath{clip}%
\pgfsetrectcap%
\pgfsetroundjoin%
\pgfsetlinewidth{1.505625pt}%
\definecolor{currentstroke}{rgb}{0.815686,0.733333,1.000000}%
\pgfsetstrokecolor{currentstroke}%
\pgfsetstrokeopacity{0.800000}%
\pgfsetdash{}{0pt}%
\pgfpathmoveto{\pgfqpoint{7.338785in}{3.335035in}}%
\pgfpathlineto{\pgfqpoint{5.293604in}{4.165376in}}%
\pgfusepath{stroke}%
\end{pgfscope}%
\begin{pgfscope}%
\pgfpathrectangle{\pgfqpoint{0.481978in}{0.331635in}}{\pgfqpoint{9.300000in}{7.700000in}}%
\pgfusepath{clip}%
\pgfsetrectcap%
\pgfsetroundjoin%
\pgfsetlinewidth{1.505625pt}%
\definecolor{currentstroke}{rgb}{0.815686,0.733333,1.000000}%
\pgfsetstrokecolor{currentstroke}%
\pgfsetstrokeopacity{0.800000}%
\pgfsetdash{}{0pt}%
\pgfpathmoveto{\pgfqpoint{3.955717in}{4.134945in}}%
\pgfpathlineto{\pgfqpoint{5.293604in}{4.165376in}}%
\pgfusepath{stroke}%
\end{pgfscope}%
\begin{pgfscope}%
\pgfpathrectangle{\pgfqpoint{0.481978in}{0.331635in}}{\pgfqpoint{9.300000in}{7.700000in}}%
\pgfusepath{clip}%
\pgfsetrectcap%
\pgfsetroundjoin%
\pgfsetlinewidth{1.505625pt}%
\definecolor{currentstroke}{rgb}{0.815686,0.733333,1.000000}%
\pgfsetstrokecolor{currentstroke}%
\pgfsetstrokeopacity{0.800000}%
\pgfsetdash{}{0pt}%
\pgfpathmoveto{\pgfqpoint{2.160067in}{1.924338in}}%
\pgfpathlineto{\pgfqpoint{5.293604in}{4.165376in}}%
\pgfusepath{stroke}%
\end{pgfscope}%
\begin{pgfscope}%
\pgfpathrectangle{\pgfqpoint{0.481978in}{0.331635in}}{\pgfqpoint{9.300000in}{7.700000in}}%
\pgfusepath{clip}%
\pgfsetrectcap%
\pgfsetroundjoin%
\pgfsetlinewidth{1.505625pt}%
\definecolor{currentstroke}{rgb}{0.815686,0.733333,1.000000}%
\pgfsetstrokecolor{currentstroke}%
\pgfsetstrokeopacity{0.800000}%
\pgfsetdash{}{0pt}%
\pgfpathmoveto{\pgfqpoint{4.198073in}{4.262822in}}%
\pgfpathlineto{\pgfqpoint{5.293604in}{4.165376in}}%
\pgfusepath{stroke}%
\end{pgfscope}%
\begin{pgfscope}%
\pgfpathrectangle{\pgfqpoint{0.481978in}{0.331635in}}{\pgfqpoint{9.300000in}{7.700000in}}%
\pgfusepath{clip}%
\pgfsetrectcap%
\pgfsetroundjoin%
\pgfsetlinewidth{1.505625pt}%
\definecolor{currentstroke}{rgb}{0.815686,0.733333,1.000000}%
\pgfsetstrokecolor{currentstroke}%
\pgfsetstrokeopacity{0.800000}%
\pgfsetdash{}{0pt}%
\pgfpathmoveto{\pgfqpoint{3.029876in}{4.931933in}}%
\pgfpathlineto{\pgfqpoint{5.293604in}{4.165376in}}%
\pgfusepath{stroke}%
\end{pgfscope}%
\begin{pgfscope}%
\pgfpathrectangle{\pgfqpoint{0.481978in}{0.331635in}}{\pgfqpoint{9.300000in}{7.700000in}}%
\pgfusepath{clip}%
\pgfsetrectcap%
\pgfsetroundjoin%
\pgfsetlinewidth{1.505625pt}%
\definecolor{currentstroke}{rgb}{0.815686,0.733333,1.000000}%
\pgfsetstrokecolor{currentstroke}%
\pgfsetstrokeopacity{0.800000}%
\pgfsetdash{}{0pt}%
\pgfpathmoveto{\pgfqpoint{1.310658in}{4.258773in}}%
\pgfpathlineto{\pgfqpoint{5.293604in}{4.165376in}}%
\pgfusepath{stroke}%
\end{pgfscope}%
\begin{pgfscope}%
\pgfpathrectangle{\pgfqpoint{0.481978in}{0.331635in}}{\pgfqpoint{9.300000in}{7.700000in}}%
\pgfusepath{clip}%
\pgfsetrectcap%
\pgfsetroundjoin%
\pgfsetlinewidth{1.505625pt}%
\definecolor{currentstroke}{rgb}{0.815686,0.733333,1.000000}%
\pgfsetstrokecolor{currentstroke}%
\pgfsetstrokeopacity{0.800000}%
\pgfsetdash{}{0pt}%
\pgfpathmoveto{\pgfqpoint{5.061821in}{4.710122in}}%
\pgfpathlineto{\pgfqpoint{5.293604in}{4.165376in}}%
\pgfusepath{stroke}%
\end{pgfscope}%
\begin{pgfscope}%
\pgfpathrectangle{\pgfqpoint{0.481978in}{0.331635in}}{\pgfqpoint{9.300000in}{7.700000in}}%
\pgfusepath{clip}%
\pgfsetrectcap%
\pgfsetroundjoin%
\pgfsetlinewidth{1.505625pt}%
\definecolor{currentstroke}{rgb}{0.815686,0.733333,1.000000}%
\pgfsetstrokecolor{currentstroke}%
\pgfsetstrokeopacity{0.800000}%
\pgfsetdash{}{0pt}%
\pgfpathmoveto{\pgfqpoint{5.046020in}{5.054040in}}%
\pgfpathlineto{\pgfqpoint{5.293604in}{4.165376in}}%
\pgfusepath{stroke}%
\end{pgfscope}%
\begin{pgfscope}%
\pgfpathrectangle{\pgfqpoint{0.481978in}{0.331635in}}{\pgfqpoint{9.300000in}{7.700000in}}%
\pgfusepath{clip}%
\pgfsetrectcap%
\pgfsetroundjoin%
\pgfsetlinewidth{1.505625pt}%
\definecolor{currentstroke}{rgb}{0.815686,0.733333,1.000000}%
\pgfsetstrokecolor{currentstroke}%
\pgfsetstrokeopacity{0.800000}%
\pgfsetdash{}{0pt}%
\pgfpathmoveto{\pgfqpoint{7.569943in}{3.679389in}}%
\pgfpathlineto{\pgfqpoint{5.293604in}{4.165376in}}%
\pgfusepath{stroke}%
\end{pgfscope}%
\begin{pgfscope}%
\pgfpathrectangle{\pgfqpoint{0.481978in}{0.331635in}}{\pgfqpoint{9.300000in}{7.700000in}}%
\pgfusepath{clip}%
\pgfsetrectcap%
\pgfsetroundjoin%
\pgfsetlinewidth{1.505625pt}%
\definecolor{currentstroke}{rgb}{0.815686,0.733333,1.000000}%
\pgfsetstrokecolor{currentstroke}%
\pgfsetstrokeopacity{0.800000}%
\pgfsetdash{}{0pt}%
\pgfpathmoveto{\pgfqpoint{7.674879in}{3.075539in}}%
\pgfpathlineto{\pgfqpoint{5.293604in}{4.165376in}}%
\pgfusepath{stroke}%
\end{pgfscope}%
\begin{pgfscope}%
\pgfpathrectangle{\pgfqpoint{0.481978in}{0.331635in}}{\pgfqpoint{9.300000in}{7.700000in}}%
\pgfusepath{clip}%
\pgfsetrectcap%
\pgfsetroundjoin%
\pgfsetlinewidth{1.505625pt}%
\definecolor{currentstroke}{rgb}{0.815686,0.733333,1.000000}%
\pgfsetstrokecolor{currentstroke}%
\pgfsetstrokeopacity{0.800000}%
\pgfsetdash{}{0pt}%
\pgfpathmoveto{\pgfqpoint{5.428071in}{3.114634in}}%
\pgfpathlineto{\pgfqpoint{5.293604in}{4.165376in}}%
\pgfusepath{stroke}%
\end{pgfscope}%
\begin{pgfscope}%
\pgfpathrectangle{\pgfqpoint{0.481978in}{0.331635in}}{\pgfqpoint{9.300000in}{7.700000in}}%
\pgfusepath{clip}%
\pgfsetrectcap%
\pgfsetroundjoin%
\pgfsetlinewidth{1.505625pt}%
\definecolor{currentstroke}{rgb}{0.815686,0.733333,1.000000}%
\pgfsetstrokecolor{currentstroke}%
\pgfsetstrokeopacity{0.800000}%
\pgfsetdash{}{0pt}%
\pgfpathmoveto{\pgfqpoint{2.403350in}{3.643277in}}%
\pgfpathlineto{\pgfqpoint{5.293604in}{4.165376in}}%
\pgfusepath{stroke}%
\end{pgfscope}%
\begin{pgfscope}%
\pgfpathrectangle{\pgfqpoint{0.481978in}{0.331635in}}{\pgfqpoint{9.300000in}{7.700000in}}%
\pgfusepath{clip}%
\pgfsetrectcap%
\pgfsetroundjoin%
\pgfsetlinewidth{1.505625pt}%
\definecolor{currentstroke}{rgb}{0.815686,0.733333,1.000000}%
\pgfsetstrokecolor{currentstroke}%
\pgfsetstrokeopacity{0.800000}%
\pgfsetdash{}{0pt}%
\pgfpathmoveto{\pgfqpoint{6.041612in}{5.634142in}}%
\pgfpathlineto{\pgfqpoint{5.293604in}{4.165376in}}%
\pgfusepath{stroke}%
\end{pgfscope}%
\begin{pgfscope}%
\pgfpathrectangle{\pgfqpoint{0.481978in}{0.331635in}}{\pgfqpoint{9.300000in}{7.700000in}}%
\pgfusepath{clip}%
\pgfsetrectcap%
\pgfsetroundjoin%
\pgfsetlinewidth{1.505625pt}%
\definecolor{currentstroke}{rgb}{0.815686,0.733333,1.000000}%
\pgfsetstrokecolor{currentstroke}%
\pgfsetstrokeopacity{0.800000}%
\pgfsetdash{}{0pt}%
\pgfpathmoveto{\pgfqpoint{8.383007in}{6.203756in}}%
\pgfpathlineto{\pgfqpoint{5.293604in}{4.165376in}}%
\pgfusepath{stroke}%
\end{pgfscope}%
\begin{pgfscope}%
\pgfpathrectangle{\pgfqpoint{0.481978in}{0.331635in}}{\pgfqpoint{9.300000in}{7.700000in}}%
\pgfusepath{clip}%
\pgfsetrectcap%
\pgfsetroundjoin%
\pgfsetlinewidth{1.505625pt}%
\definecolor{currentstroke}{rgb}{0.815686,0.733333,1.000000}%
\pgfsetstrokecolor{currentstroke}%
\pgfsetstrokeopacity{0.800000}%
\pgfsetdash{}{0pt}%
\pgfpathmoveto{\pgfqpoint{4.398769in}{4.567491in}}%
\pgfpathlineto{\pgfqpoint{5.293604in}{4.165376in}}%
\pgfusepath{stroke}%
\end{pgfscope}%
\begin{pgfscope}%
\pgfpathrectangle{\pgfqpoint{0.481978in}{0.331635in}}{\pgfqpoint{9.300000in}{7.700000in}}%
\pgfusepath{clip}%
\pgfsetrectcap%
\pgfsetroundjoin%
\pgfsetlinewidth{1.505625pt}%
\definecolor{currentstroke}{rgb}{0.815686,0.733333,1.000000}%
\pgfsetstrokecolor{currentstroke}%
\pgfsetstrokeopacity{0.800000}%
\pgfsetdash{}{0pt}%
\pgfpathmoveto{\pgfqpoint{8.682490in}{5.819369in}}%
\pgfpathlineto{\pgfqpoint{5.293604in}{4.165376in}}%
\pgfusepath{stroke}%
\end{pgfscope}%
\begin{pgfscope}%
\pgfpathrectangle{\pgfqpoint{0.481978in}{0.331635in}}{\pgfqpoint{9.300000in}{7.700000in}}%
\pgfusepath{clip}%
\pgfsetrectcap%
\pgfsetroundjoin%
\pgfsetlinewidth{1.505625pt}%
\definecolor{currentstroke}{rgb}{0.815686,0.733333,1.000000}%
\pgfsetstrokecolor{currentstroke}%
\pgfsetstrokeopacity{0.800000}%
\pgfsetdash{}{0pt}%
\pgfpathmoveto{\pgfqpoint{5.664598in}{1.155031in}}%
\pgfpathlineto{\pgfqpoint{5.293604in}{4.165376in}}%
\pgfusepath{stroke}%
\end{pgfscope}%
\begin{pgfscope}%
\pgfpathrectangle{\pgfqpoint{0.481978in}{0.331635in}}{\pgfqpoint{9.300000in}{7.700000in}}%
\pgfusepath{clip}%
\pgfsetrectcap%
\pgfsetroundjoin%
\pgfsetlinewidth{1.505625pt}%
\definecolor{currentstroke}{rgb}{0.815686,0.733333,1.000000}%
\pgfsetstrokecolor{currentstroke}%
\pgfsetstrokeopacity{0.800000}%
\pgfsetdash{}{0pt}%
\pgfpathmoveto{\pgfqpoint{1.939227in}{4.353523in}}%
\pgfpathlineto{\pgfqpoint{5.293604in}{4.165376in}}%
\pgfusepath{stroke}%
\end{pgfscope}%
\begin{pgfscope}%
\pgfpathrectangle{\pgfqpoint{0.481978in}{0.331635in}}{\pgfqpoint{9.300000in}{7.700000in}}%
\pgfusepath{clip}%
\pgfsetrectcap%
\pgfsetroundjoin%
\pgfsetlinewidth{1.505625pt}%
\definecolor{currentstroke}{rgb}{0.815686,0.733333,1.000000}%
\pgfsetstrokecolor{currentstroke}%
\pgfsetstrokeopacity{0.800000}%
\pgfsetdash{}{0pt}%
\pgfpathmoveto{\pgfqpoint{6.777031in}{4.449477in}}%
\pgfpathlineto{\pgfqpoint{5.293604in}{4.165376in}}%
\pgfusepath{stroke}%
\end{pgfscope}%
\begin{pgfscope}%
\pgfpathrectangle{\pgfqpoint{0.481978in}{0.331635in}}{\pgfqpoint{9.300000in}{7.700000in}}%
\pgfusepath{clip}%
\pgfsetrectcap%
\pgfsetroundjoin%
\pgfsetlinewidth{1.505625pt}%
\definecolor{currentstroke}{rgb}{0.815686,0.733333,1.000000}%
\pgfsetstrokecolor{currentstroke}%
\pgfsetstrokeopacity{0.800000}%
\pgfsetdash{}{0pt}%
\pgfpathmoveto{\pgfqpoint{3.798724in}{2.165621in}}%
\pgfpathlineto{\pgfqpoint{5.293604in}{4.165376in}}%
\pgfusepath{stroke}%
\end{pgfscope}%
\begin{pgfscope}%
\pgfpathrectangle{\pgfqpoint{0.481978in}{0.331635in}}{\pgfqpoint{9.300000in}{7.700000in}}%
\pgfusepath{clip}%
\pgfsetrectcap%
\pgfsetroundjoin%
\pgfsetlinewidth{1.505625pt}%
\definecolor{currentstroke}{rgb}{0.815686,0.733333,1.000000}%
\pgfsetstrokecolor{currentstroke}%
\pgfsetstrokeopacity{0.800000}%
\pgfsetdash{}{0pt}%
\pgfpathmoveto{\pgfqpoint{1.738451in}{3.903724in}}%
\pgfpathlineto{\pgfqpoint{5.293604in}{4.165376in}}%
\pgfusepath{stroke}%
\end{pgfscope}%
\begin{pgfscope}%
\pgfpathrectangle{\pgfqpoint{0.481978in}{0.331635in}}{\pgfqpoint{9.300000in}{7.700000in}}%
\pgfusepath{clip}%
\pgfsetrectcap%
\pgfsetroundjoin%
\pgfsetlinewidth{1.505625pt}%
\definecolor{currentstroke}{rgb}{0.815686,0.733333,1.000000}%
\pgfsetstrokecolor{currentstroke}%
\pgfsetstrokeopacity{0.800000}%
\pgfsetdash{}{0pt}%
\pgfpathmoveto{\pgfqpoint{8.508088in}{4.005881in}}%
\pgfpathlineto{\pgfqpoint{5.293604in}{4.165376in}}%
\pgfusepath{stroke}%
\end{pgfscope}%
\begin{pgfscope}%
\pgfpathrectangle{\pgfqpoint{0.481978in}{0.331635in}}{\pgfqpoint{9.300000in}{7.700000in}}%
\pgfusepath{clip}%
\pgfsetrectcap%
\pgfsetroundjoin%
\pgfsetlinewidth{1.505625pt}%
\definecolor{currentstroke}{rgb}{0.815686,0.733333,1.000000}%
\pgfsetstrokecolor{currentstroke}%
\pgfsetstrokeopacity{0.800000}%
\pgfsetdash{}{0pt}%
\pgfpathmoveto{\pgfqpoint{6.837191in}{4.890982in}}%
\pgfpathlineto{\pgfqpoint{5.293604in}{4.165376in}}%
\pgfusepath{stroke}%
\end{pgfscope}%
\begin{pgfscope}%
\pgfpathrectangle{\pgfqpoint{0.481978in}{0.331635in}}{\pgfqpoint{9.300000in}{7.700000in}}%
\pgfusepath{clip}%
\pgfsetrectcap%
\pgfsetroundjoin%
\pgfsetlinewidth{1.505625pt}%
\definecolor{currentstroke}{rgb}{0.815686,0.733333,1.000000}%
\pgfsetstrokecolor{currentstroke}%
\pgfsetstrokeopacity{0.800000}%
\pgfsetdash{}{0pt}%
\pgfpathmoveto{\pgfqpoint{4.101155in}{3.759624in}}%
\pgfpathlineto{\pgfqpoint{5.293604in}{4.165376in}}%
\pgfusepath{stroke}%
\end{pgfscope}%
\begin{pgfscope}%
\pgfpathrectangle{\pgfqpoint{0.481978in}{0.331635in}}{\pgfqpoint{9.300000in}{7.700000in}}%
\pgfusepath{clip}%
\pgfsetrectcap%
\pgfsetroundjoin%
\pgfsetlinewidth{1.505625pt}%
\definecolor{currentstroke}{rgb}{0.815686,0.733333,1.000000}%
\pgfsetstrokecolor{currentstroke}%
\pgfsetstrokeopacity{0.800000}%
\pgfsetdash{}{0pt}%
\pgfpathmoveto{\pgfqpoint{7.340850in}{3.294690in}}%
\pgfpathlineto{\pgfqpoint{5.293604in}{4.165376in}}%
\pgfusepath{stroke}%
\end{pgfscope}%
\begin{pgfscope}%
\pgfpathrectangle{\pgfqpoint{0.481978in}{0.331635in}}{\pgfqpoint{9.300000in}{7.700000in}}%
\pgfusepath{clip}%
\pgfsetrectcap%
\pgfsetroundjoin%
\pgfsetlinewidth{1.505625pt}%
\definecolor{currentstroke}{rgb}{0.815686,0.733333,1.000000}%
\pgfsetstrokecolor{currentstroke}%
\pgfsetstrokeopacity{0.800000}%
\pgfsetdash{}{0pt}%
\pgfpathmoveto{\pgfqpoint{4.878178in}{5.104000in}}%
\pgfpathlineto{\pgfqpoint{5.293604in}{4.165376in}}%
\pgfusepath{stroke}%
\end{pgfscope}%
\begin{pgfscope}%
\pgfpathrectangle{\pgfqpoint{0.481978in}{0.331635in}}{\pgfqpoint{9.300000in}{7.700000in}}%
\pgfusepath{clip}%
\pgfsetrectcap%
\pgfsetroundjoin%
\pgfsetlinewidth{1.505625pt}%
\definecolor{currentstroke}{rgb}{0.815686,0.733333,1.000000}%
\pgfsetstrokecolor{currentstroke}%
\pgfsetstrokeopacity{0.800000}%
\pgfsetdash{}{0pt}%
\pgfpathmoveto{\pgfqpoint{3.093874in}{4.888839in}}%
\pgfpathlineto{\pgfqpoint{5.293604in}{4.165376in}}%
\pgfusepath{stroke}%
\end{pgfscope}%
\begin{pgfscope}%
\pgfpathrectangle{\pgfqpoint{0.481978in}{0.331635in}}{\pgfqpoint{9.300000in}{7.700000in}}%
\pgfusepath{clip}%
\pgfsetrectcap%
\pgfsetroundjoin%
\pgfsetlinewidth{1.505625pt}%
\definecolor{currentstroke}{rgb}{0.815686,0.733333,1.000000}%
\pgfsetstrokecolor{currentstroke}%
\pgfsetstrokeopacity{0.800000}%
\pgfsetdash{}{0pt}%
\pgfpathmoveto{\pgfqpoint{5.273114in}{1.907651in}}%
\pgfpathlineto{\pgfqpoint{5.293604in}{4.165376in}}%
\pgfusepath{stroke}%
\end{pgfscope}%
\begin{pgfscope}%
\pgfpathrectangle{\pgfqpoint{0.481978in}{0.331635in}}{\pgfqpoint{9.300000in}{7.700000in}}%
\pgfusepath{clip}%
\pgfsetrectcap%
\pgfsetroundjoin%
\pgfsetlinewidth{1.505625pt}%
\definecolor{currentstroke}{rgb}{0.815686,0.733333,1.000000}%
\pgfsetstrokecolor{currentstroke}%
\pgfsetstrokeopacity{0.800000}%
\pgfsetdash{}{0pt}%
\pgfpathmoveto{\pgfqpoint{8.330701in}{5.648832in}}%
\pgfpathlineto{\pgfqpoint{5.293604in}{4.165376in}}%
\pgfusepath{stroke}%
\end{pgfscope}%
\begin{pgfscope}%
\pgfpathrectangle{\pgfqpoint{0.481978in}{0.331635in}}{\pgfqpoint{9.300000in}{7.700000in}}%
\pgfusepath{clip}%
\pgfsetrectcap%
\pgfsetroundjoin%
\pgfsetlinewidth{1.505625pt}%
\definecolor{currentstroke}{rgb}{0.815686,0.733333,1.000000}%
\pgfsetstrokecolor{currentstroke}%
\pgfsetstrokeopacity{0.800000}%
\pgfsetdash{}{0pt}%
\pgfpathmoveto{\pgfqpoint{7.364685in}{2.940268in}}%
\pgfpathlineto{\pgfqpoint{5.293604in}{4.165376in}}%
\pgfusepath{stroke}%
\end{pgfscope}%
\begin{pgfscope}%
\pgfpathrectangle{\pgfqpoint{0.481978in}{0.331635in}}{\pgfqpoint{9.300000in}{7.700000in}}%
\pgfusepath{clip}%
\pgfsetrectcap%
\pgfsetroundjoin%
\pgfsetlinewidth{1.505625pt}%
\definecolor{currentstroke}{rgb}{0.815686,0.733333,1.000000}%
\pgfsetstrokecolor{currentstroke}%
\pgfsetstrokeopacity{0.800000}%
\pgfsetdash{}{0pt}%
\pgfpathmoveto{\pgfqpoint{2.332471in}{2.516497in}}%
\pgfpathlineto{\pgfqpoint{5.293604in}{4.165376in}}%
\pgfusepath{stroke}%
\end{pgfscope}%
\begin{pgfscope}%
\pgfpathrectangle{\pgfqpoint{0.481978in}{0.331635in}}{\pgfqpoint{9.300000in}{7.700000in}}%
\pgfusepath{clip}%
\pgfsetrectcap%
\pgfsetroundjoin%
\pgfsetlinewidth{1.505625pt}%
\definecolor{currentstroke}{rgb}{0.815686,0.733333,1.000000}%
\pgfsetstrokecolor{currentstroke}%
\pgfsetstrokeopacity{0.800000}%
\pgfsetdash{}{0pt}%
\pgfpathmoveto{\pgfqpoint{6.392422in}{4.231341in}}%
\pgfpathlineto{\pgfqpoint{5.293604in}{4.165376in}}%
\pgfusepath{stroke}%
\end{pgfscope}%
\begin{pgfscope}%
\pgfpathrectangle{\pgfqpoint{0.481978in}{0.331635in}}{\pgfqpoint{9.300000in}{7.700000in}}%
\pgfusepath{clip}%
\pgfsetrectcap%
\pgfsetroundjoin%
\pgfsetlinewidth{1.505625pt}%
\definecolor{currentstroke}{rgb}{0.870588,0.733333,0.607843}%
\pgfsetstrokecolor{currentstroke}%
\pgfsetstrokeopacity{0.800000}%
\pgfsetdash{}{0pt}%
\pgfpathmoveto{\pgfqpoint{3.278906in}{4.376840in}}%
\pgfpathlineto{\pgfqpoint{4.473902in}{4.321654in}}%
\pgfusepath{stroke}%
\end{pgfscope}%
\begin{pgfscope}%
\pgfpathrectangle{\pgfqpoint{0.481978in}{0.331635in}}{\pgfqpoint{9.300000in}{7.700000in}}%
\pgfusepath{clip}%
\pgfsetrectcap%
\pgfsetroundjoin%
\pgfsetlinewidth{1.505625pt}%
\definecolor{currentstroke}{rgb}{0.870588,0.733333,0.607843}%
\pgfsetstrokecolor{currentstroke}%
\pgfsetstrokeopacity{0.800000}%
\pgfsetdash{}{0pt}%
\pgfpathmoveto{\pgfqpoint{1.806406in}{3.514068in}}%
\pgfpathlineto{\pgfqpoint{4.473902in}{4.321654in}}%
\pgfusepath{stroke}%
\end{pgfscope}%
\begin{pgfscope}%
\pgfpathrectangle{\pgfqpoint{0.481978in}{0.331635in}}{\pgfqpoint{9.300000in}{7.700000in}}%
\pgfusepath{clip}%
\pgfsetrectcap%
\pgfsetroundjoin%
\pgfsetlinewidth{1.505625pt}%
\definecolor{currentstroke}{rgb}{0.870588,0.733333,0.607843}%
\pgfsetstrokecolor{currentstroke}%
\pgfsetstrokeopacity{0.800000}%
\pgfsetdash{}{0pt}%
\pgfpathmoveto{\pgfqpoint{3.901522in}{2.752340in}}%
\pgfpathlineto{\pgfqpoint{4.473902in}{4.321654in}}%
\pgfusepath{stroke}%
\end{pgfscope}%
\begin{pgfscope}%
\pgfpathrectangle{\pgfqpoint{0.481978in}{0.331635in}}{\pgfqpoint{9.300000in}{7.700000in}}%
\pgfusepath{clip}%
\pgfsetrectcap%
\pgfsetroundjoin%
\pgfsetlinewidth{1.505625pt}%
\definecolor{currentstroke}{rgb}{0.870588,0.733333,0.607843}%
\pgfsetstrokecolor{currentstroke}%
\pgfsetstrokeopacity{0.800000}%
\pgfsetdash{}{0pt}%
\pgfpathmoveto{\pgfqpoint{2.577647in}{1.961490in}}%
\pgfpathlineto{\pgfqpoint{4.473902in}{4.321654in}}%
\pgfusepath{stroke}%
\end{pgfscope}%
\begin{pgfscope}%
\pgfpathrectangle{\pgfqpoint{0.481978in}{0.331635in}}{\pgfqpoint{9.300000in}{7.700000in}}%
\pgfusepath{clip}%
\pgfsetrectcap%
\pgfsetroundjoin%
\pgfsetlinewidth{1.505625pt}%
\definecolor{currentstroke}{rgb}{0.870588,0.733333,0.607843}%
\pgfsetstrokecolor{currentstroke}%
\pgfsetstrokeopacity{0.800000}%
\pgfsetdash{}{0pt}%
\pgfpathmoveto{\pgfqpoint{2.542882in}{4.429625in}}%
\pgfpathlineto{\pgfqpoint{4.473902in}{4.321654in}}%
\pgfusepath{stroke}%
\end{pgfscope}%
\begin{pgfscope}%
\pgfpathrectangle{\pgfqpoint{0.481978in}{0.331635in}}{\pgfqpoint{9.300000in}{7.700000in}}%
\pgfusepath{clip}%
\pgfsetrectcap%
\pgfsetroundjoin%
\pgfsetlinewidth{1.505625pt}%
\definecolor{currentstroke}{rgb}{0.870588,0.733333,0.607843}%
\pgfsetstrokecolor{currentstroke}%
\pgfsetstrokeopacity{0.800000}%
\pgfsetdash{}{0pt}%
\pgfpathmoveto{\pgfqpoint{6.100140in}{5.077934in}}%
\pgfpathlineto{\pgfqpoint{4.473902in}{4.321654in}}%
\pgfusepath{stroke}%
\end{pgfscope}%
\begin{pgfscope}%
\pgfpathrectangle{\pgfqpoint{0.481978in}{0.331635in}}{\pgfqpoint{9.300000in}{7.700000in}}%
\pgfusepath{clip}%
\pgfsetrectcap%
\pgfsetroundjoin%
\pgfsetlinewidth{1.505625pt}%
\definecolor{currentstroke}{rgb}{0.870588,0.733333,0.607843}%
\pgfsetstrokecolor{currentstroke}%
\pgfsetstrokeopacity{0.800000}%
\pgfsetdash{}{0pt}%
\pgfpathmoveto{\pgfqpoint{4.111074in}{4.571818in}}%
\pgfpathlineto{\pgfqpoint{4.473902in}{4.321654in}}%
\pgfusepath{stroke}%
\end{pgfscope}%
\begin{pgfscope}%
\pgfpathrectangle{\pgfqpoint{0.481978in}{0.331635in}}{\pgfqpoint{9.300000in}{7.700000in}}%
\pgfusepath{clip}%
\pgfsetrectcap%
\pgfsetroundjoin%
\pgfsetlinewidth{1.505625pt}%
\definecolor{currentstroke}{rgb}{0.870588,0.733333,0.607843}%
\pgfsetstrokecolor{currentstroke}%
\pgfsetstrokeopacity{0.800000}%
\pgfsetdash{}{0pt}%
\pgfpathmoveto{\pgfqpoint{2.567413in}{4.083544in}}%
\pgfpathlineto{\pgfqpoint{4.473902in}{4.321654in}}%
\pgfusepath{stroke}%
\end{pgfscope}%
\begin{pgfscope}%
\pgfpathrectangle{\pgfqpoint{0.481978in}{0.331635in}}{\pgfqpoint{9.300000in}{7.700000in}}%
\pgfusepath{clip}%
\pgfsetrectcap%
\pgfsetroundjoin%
\pgfsetlinewidth{1.505625pt}%
\definecolor{currentstroke}{rgb}{0.870588,0.733333,0.607843}%
\pgfsetstrokecolor{currentstroke}%
\pgfsetstrokeopacity{0.800000}%
\pgfsetdash{}{0pt}%
\pgfpathmoveto{\pgfqpoint{2.475746in}{4.004366in}}%
\pgfpathlineto{\pgfqpoint{4.473902in}{4.321654in}}%
\pgfusepath{stroke}%
\end{pgfscope}%
\begin{pgfscope}%
\pgfpathrectangle{\pgfqpoint{0.481978in}{0.331635in}}{\pgfqpoint{9.300000in}{7.700000in}}%
\pgfusepath{clip}%
\pgfsetrectcap%
\pgfsetroundjoin%
\pgfsetlinewidth{1.505625pt}%
\definecolor{currentstroke}{rgb}{0.870588,0.733333,0.607843}%
\pgfsetstrokecolor{currentstroke}%
\pgfsetstrokeopacity{0.800000}%
\pgfsetdash{}{0pt}%
\pgfpathmoveto{\pgfqpoint{7.321265in}{6.030797in}}%
\pgfpathlineto{\pgfqpoint{4.473902in}{4.321654in}}%
\pgfusepath{stroke}%
\end{pgfscope}%
\begin{pgfscope}%
\pgfpathrectangle{\pgfqpoint{0.481978in}{0.331635in}}{\pgfqpoint{9.300000in}{7.700000in}}%
\pgfusepath{clip}%
\pgfsetrectcap%
\pgfsetroundjoin%
\pgfsetlinewidth{1.505625pt}%
\definecolor{currentstroke}{rgb}{0.870588,0.733333,0.607843}%
\pgfsetstrokecolor{currentstroke}%
\pgfsetstrokeopacity{0.800000}%
\pgfsetdash{}{0pt}%
\pgfpathmoveto{\pgfqpoint{1.982524in}{5.206262in}}%
\pgfpathlineto{\pgfqpoint{4.473902in}{4.321654in}}%
\pgfusepath{stroke}%
\end{pgfscope}%
\begin{pgfscope}%
\pgfpathrectangle{\pgfqpoint{0.481978in}{0.331635in}}{\pgfqpoint{9.300000in}{7.700000in}}%
\pgfusepath{clip}%
\pgfsetrectcap%
\pgfsetroundjoin%
\pgfsetlinewidth{1.505625pt}%
\definecolor{currentstroke}{rgb}{0.870588,0.733333,0.607843}%
\pgfsetstrokecolor{currentstroke}%
\pgfsetstrokeopacity{0.800000}%
\pgfsetdash{}{0pt}%
\pgfpathmoveto{\pgfqpoint{2.801903in}{4.401537in}}%
\pgfpathlineto{\pgfqpoint{4.473902in}{4.321654in}}%
\pgfusepath{stroke}%
\end{pgfscope}%
\begin{pgfscope}%
\pgfpathrectangle{\pgfqpoint{0.481978in}{0.331635in}}{\pgfqpoint{9.300000in}{7.700000in}}%
\pgfusepath{clip}%
\pgfsetrectcap%
\pgfsetroundjoin%
\pgfsetlinewidth{1.505625pt}%
\definecolor{currentstroke}{rgb}{0.870588,0.733333,0.607843}%
\pgfsetstrokecolor{currentstroke}%
\pgfsetstrokeopacity{0.800000}%
\pgfsetdash{}{0pt}%
\pgfpathmoveto{\pgfqpoint{3.503709in}{4.313535in}}%
\pgfpathlineto{\pgfqpoint{4.473902in}{4.321654in}}%
\pgfusepath{stroke}%
\end{pgfscope}%
\begin{pgfscope}%
\pgfpathrectangle{\pgfqpoint{0.481978in}{0.331635in}}{\pgfqpoint{9.300000in}{7.700000in}}%
\pgfusepath{clip}%
\pgfsetrectcap%
\pgfsetroundjoin%
\pgfsetlinewidth{1.505625pt}%
\definecolor{currentstroke}{rgb}{0.870588,0.733333,0.607843}%
\pgfsetstrokecolor{currentstroke}%
\pgfsetstrokeopacity{0.800000}%
\pgfsetdash{}{0pt}%
\pgfpathmoveto{\pgfqpoint{7.049429in}{6.056502in}}%
\pgfpathlineto{\pgfqpoint{4.473902in}{4.321654in}}%
\pgfusepath{stroke}%
\end{pgfscope}%
\begin{pgfscope}%
\pgfpathrectangle{\pgfqpoint{0.481978in}{0.331635in}}{\pgfqpoint{9.300000in}{7.700000in}}%
\pgfusepath{clip}%
\pgfsetrectcap%
\pgfsetroundjoin%
\pgfsetlinewidth{1.505625pt}%
\definecolor{currentstroke}{rgb}{0.870588,0.733333,0.607843}%
\pgfsetstrokecolor{currentstroke}%
\pgfsetstrokeopacity{0.800000}%
\pgfsetdash{}{0pt}%
\pgfpathmoveto{\pgfqpoint{6.348280in}{4.846864in}}%
\pgfpathlineto{\pgfqpoint{4.473902in}{4.321654in}}%
\pgfusepath{stroke}%
\end{pgfscope}%
\begin{pgfscope}%
\pgfpathrectangle{\pgfqpoint{0.481978in}{0.331635in}}{\pgfqpoint{9.300000in}{7.700000in}}%
\pgfusepath{clip}%
\pgfsetrectcap%
\pgfsetroundjoin%
\pgfsetlinewidth{1.505625pt}%
\definecolor{currentstroke}{rgb}{0.870588,0.733333,0.607843}%
\pgfsetstrokecolor{currentstroke}%
\pgfsetstrokeopacity{0.800000}%
\pgfsetdash{}{0pt}%
\pgfpathmoveto{\pgfqpoint{5.923605in}{3.681854in}}%
\pgfpathlineto{\pgfqpoint{4.473902in}{4.321654in}}%
\pgfusepath{stroke}%
\end{pgfscope}%
\begin{pgfscope}%
\pgfpathrectangle{\pgfqpoint{0.481978in}{0.331635in}}{\pgfqpoint{9.300000in}{7.700000in}}%
\pgfusepath{clip}%
\pgfsetrectcap%
\pgfsetroundjoin%
\pgfsetlinewidth{1.505625pt}%
\definecolor{currentstroke}{rgb}{0.870588,0.733333,0.607843}%
\pgfsetstrokecolor{currentstroke}%
\pgfsetstrokeopacity{0.800000}%
\pgfsetdash{}{0pt}%
\pgfpathmoveto{\pgfqpoint{5.344914in}{4.036077in}}%
\pgfpathlineto{\pgfqpoint{4.473902in}{4.321654in}}%
\pgfusepath{stroke}%
\end{pgfscope}%
\begin{pgfscope}%
\pgfpathrectangle{\pgfqpoint{0.481978in}{0.331635in}}{\pgfqpoint{9.300000in}{7.700000in}}%
\pgfusepath{clip}%
\pgfsetrectcap%
\pgfsetroundjoin%
\pgfsetlinewidth{1.505625pt}%
\definecolor{currentstroke}{rgb}{0.870588,0.733333,0.607843}%
\pgfsetstrokecolor{currentstroke}%
\pgfsetstrokeopacity{0.800000}%
\pgfsetdash{}{0pt}%
\pgfpathmoveto{\pgfqpoint{4.527667in}{3.744276in}}%
\pgfpathlineto{\pgfqpoint{4.473902in}{4.321654in}}%
\pgfusepath{stroke}%
\end{pgfscope}%
\begin{pgfscope}%
\pgfpathrectangle{\pgfqpoint{0.481978in}{0.331635in}}{\pgfqpoint{9.300000in}{7.700000in}}%
\pgfusepath{clip}%
\pgfsetrectcap%
\pgfsetroundjoin%
\pgfsetlinewidth{1.505625pt}%
\definecolor{currentstroke}{rgb}{0.870588,0.733333,0.607843}%
\pgfsetstrokecolor{currentstroke}%
\pgfsetstrokeopacity{0.800000}%
\pgfsetdash{}{0pt}%
\pgfpathmoveto{\pgfqpoint{2.059850in}{4.036460in}}%
\pgfpathlineto{\pgfqpoint{4.473902in}{4.321654in}}%
\pgfusepath{stroke}%
\end{pgfscope}%
\begin{pgfscope}%
\pgfpathrectangle{\pgfqpoint{0.481978in}{0.331635in}}{\pgfqpoint{9.300000in}{7.700000in}}%
\pgfusepath{clip}%
\pgfsetrectcap%
\pgfsetroundjoin%
\pgfsetlinewidth{1.505625pt}%
\definecolor{currentstroke}{rgb}{0.870588,0.733333,0.607843}%
\pgfsetstrokecolor{currentstroke}%
\pgfsetstrokeopacity{0.800000}%
\pgfsetdash{}{0pt}%
\pgfpathmoveto{\pgfqpoint{5.974496in}{4.950112in}}%
\pgfpathlineto{\pgfqpoint{4.473902in}{4.321654in}}%
\pgfusepath{stroke}%
\end{pgfscope}%
\begin{pgfscope}%
\pgfpathrectangle{\pgfqpoint{0.481978in}{0.331635in}}{\pgfqpoint{9.300000in}{7.700000in}}%
\pgfusepath{clip}%
\pgfsetrectcap%
\pgfsetroundjoin%
\pgfsetlinewidth{1.505625pt}%
\definecolor{currentstroke}{rgb}{0.870588,0.733333,0.607843}%
\pgfsetstrokecolor{currentstroke}%
\pgfsetstrokeopacity{0.800000}%
\pgfsetdash{}{0pt}%
\pgfpathmoveto{\pgfqpoint{2.319348in}{4.794802in}}%
\pgfpathlineto{\pgfqpoint{4.473902in}{4.321654in}}%
\pgfusepath{stroke}%
\end{pgfscope}%
\begin{pgfscope}%
\pgfpathrectangle{\pgfqpoint{0.481978in}{0.331635in}}{\pgfqpoint{9.300000in}{7.700000in}}%
\pgfusepath{clip}%
\pgfsetrectcap%
\pgfsetroundjoin%
\pgfsetlinewidth{1.505625pt}%
\definecolor{currentstroke}{rgb}{0.870588,0.733333,0.607843}%
\pgfsetstrokecolor{currentstroke}%
\pgfsetstrokeopacity{0.800000}%
\pgfsetdash{}{0pt}%
\pgfpathmoveto{\pgfqpoint{6.238443in}{6.242431in}}%
\pgfpathlineto{\pgfqpoint{4.473902in}{4.321654in}}%
\pgfusepath{stroke}%
\end{pgfscope}%
\begin{pgfscope}%
\pgfpathrectangle{\pgfqpoint{0.481978in}{0.331635in}}{\pgfqpoint{9.300000in}{7.700000in}}%
\pgfusepath{clip}%
\pgfsetrectcap%
\pgfsetroundjoin%
\pgfsetlinewidth{1.505625pt}%
\definecolor{currentstroke}{rgb}{0.870588,0.733333,0.607843}%
\pgfsetstrokecolor{currentstroke}%
\pgfsetstrokeopacity{0.800000}%
\pgfsetdash{}{0pt}%
\pgfpathmoveto{\pgfqpoint{4.167128in}{3.897191in}}%
\pgfpathlineto{\pgfqpoint{4.473902in}{4.321654in}}%
\pgfusepath{stroke}%
\end{pgfscope}%
\begin{pgfscope}%
\pgfpathrectangle{\pgfqpoint{0.481978in}{0.331635in}}{\pgfqpoint{9.300000in}{7.700000in}}%
\pgfusepath{clip}%
\pgfsetrectcap%
\pgfsetroundjoin%
\pgfsetlinewidth{1.505625pt}%
\definecolor{currentstroke}{rgb}{0.870588,0.733333,0.607843}%
\pgfsetstrokecolor{currentstroke}%
\pgfsetstrokeopacity{0.800000}%
\pgfsetdash{}{0pt}%
\pgfpathmoveto{\pgfqpoint{4.383560in}{3.597829in}}%
\pgfpathlineto{\pgfqpoint{4.473902in}{4.321654in}}%
\pgfusepath{stroke}%
\end{pgfscope}%
\begin{pgfscope}%
\pgfpathrectangle{\pgfqpoint{0.481978in}{0.331635in}}{\pgfqpoint{9.300000in}{7.700000in}}%
\pgfusepath{clip}%
\pgfsetrectcap%
\pgfsetroundjoin%
\pgfsetlinewidth{1.505625pt}%
\definecolor{currentstroke}{rgb}{0.870588,0.733333,0.607843}%
\pgfsetstrokecolor{currentstroke}%
\pgfsetstrokeopacity{0.800000}%
\pgfsetdash{}{0pt}%
\pgfpathmoveto{\pgfqpoint{5.769577in}{4.001990in}}%
\pgfpathlineto{\pgfqpoint{4.473902in}{4.321654in}}%
\pgfusepath{stroke}%
\end{pgfscope}%
\begin{pgfscope}%
\pgfpathrectangle{\pgfqpoint{0.481978in}{0.331635in}}{\pgfqpoint{9.300000in}{7.700000in}}%
\pgfusepath{clip}%
\pgfsetrectcap%
\pgfsetroundjoin%
\pgfsetlinewidth{1.505625pt}%
\definecolor{currentstroke}{rgb}{0.870588,0.733333,0.607843}%
\pgfsetstrokecolor{currentstroke}%
\pgfsetstrokeopacity{0.800000}%
\pgfsetdash{}{0pt}%
\pgfpathmoveto{\pgfqpoint{9.359251in}{4.570076in}}%
\pgfpathlineto{\pgfqpoint{4.473902in}{4.321654in}}%
\pgfusepath{stroke}%
\end{pgfscope}%
\begin{pgfscope}%
\pgfpathrectangle{\pgfqpoint{0.481978in}{0.331635in}}{\pgfqpoint{9.300000in}{7.700000in}}%
\pgfusepath{clip}%
\pgfsetrectcap%
\pgfsetroundjoin%
\pgfsetlinewidth{1.505625pt}%
\definecolor{currentstroke}{rgb}{0.870588,0.733333,0.607843}%
\pgfsetstrokecolor{currentstroke}%
\pgfsetstrokeopacity{0.800000}%
\pgfsetdash{}{0pt}%
\pgfpathmoveto{\pgfqpoint{5.675309in}{2.672631in}}%
\pgfpathlineto{\pgfqpoint{4.473902in}{4.321654in}}%
\pgfusepath{stroke}%
\end{pgfscope}%
\begin{pgfscope}%
\pgfpathrectangle{\pgfqpoint{0.481978in}{0.331635in}}{\pgfqpoint{9.300000in}{7.700000in}}%
\pgfusepath{clip}%
\pgfsetrectcap%
\pgfsetroundjoin%
\pgfsetlinewidth{1.505625pt}%
\definecolor{currentstroke}{rgb}{0.870588,0.733333,0.607843}%
\pgfsetstrokecolor{currentstroke}%
\pgfsetstrokeopacity{0.800000}%
\pgfsetdash{}{0pt}%
\pgfpathmoveto{\pgfqpoint{3.026449in}{4.347964in}}%
\pgfpathlineto{\pgfqpoint{4.473902in}{4.321654in}}%
\pgfusepath{stroke}%
\end{pgfscope}%
\begin{pgfscope}%
\pgfpathrectangle{\pgfqpoint{0.481978in}{0.331635in}}{\pgfqpoint{9.300000in}{7.700000in}}%
\pgfusepath{clip}%
\pgfsetrectcap%
\pgfsetroundjoin%
\pgfsetlinewidth{1.505625pt}%
\definecolor{currentstroke}{rgb}{0.870588,0.733333,0.607843}%
\pgfsetstrokecolor{currentstroke}%
\pgfsetstrokeopacity{0.800000}%
\pgfsetdash{}{0pt}%
\pgfpathmoveto{\pgfqpoint{6.761280in}{5.866030in}}%
\pgfpathlineto{\pgfqpoint{4.473902in}{4.321654in}}%
\pgfusepath{stroke}%
\end{pgfscope}%
\begin{pgfscope}%
\pgfpathrectangle{\pgfqpoint{0.481978in}{0.331635in}}{\pgfqpoint{9.300000in}{7.700000in}}%
\pgfusepath{clip}%
\pgfsetrectcap%
\pgfsetroundjoin%
\pgfsetlinewidth{1.505625pt}%
\definecolor{currentstroke}{rgb}{0.870588,0.733333,0.607843}%
\pgfsetstrokecolor{currentstroke}%
\pgfsetstrokeopacity{0.800000}%
\pgfsetdash{}{0pt}%
\pgfpathmoveto{\pgfqpoint{2.445109in}{3.109579in}}%
\pgfpathlineto{\pgfqpoint{4.473902in}{4.321654in}}%
\pgfusepath{stroke}%
\end{pgfscope}%
\begin{pgfscope}%
\pgfpathrectangle{\pgfqpoint{0.481978in}{0.331635in}}{\pgfqpoint{9.300000in}{7.700000in}}%
\pgfusepath{clip}%
\pgfsetrectcap%
\pgfsetroundjoin%
\pgfsetlinewidth{1.505625pt}%
\definecolor{currentstroke}{rgb}{0.870588,0.733333,0.607843}%
\pgfsetstrokecolor{currentstroke}%
\pgfsetstrokeopacity{0.800000}%
\pgfsetdash{}{0pt}%
\pgfpathmoveto{\pgfqpoint{2.556572in}{4.487385in}}%
\pgfpathlineto{\pgfqpoint{4.473902in}{4.321654in}}%
\pgfusepath{stroke}%
\end{pgfscope}%
\begin{pgfscope}%
\pgfpathrectangle{\pgfqpoint{0.481978in}{0.331635in}}{\pgfqpoint{9.300000in}{7.700000in}}%
\pgfusepath{clip}%
\pgfsetrectcap%
\pgfsetroundjoin%
\pgfsetlinewidth{1.505625pt}%
\definecolor{currentstroke}{rgb}{0.870588,0.733333,0.607843}%
\pgfsetstrokecolor{currentstroke}%
\pgfsetstrokeopacity{0.800000}%
\pgfsetdash{}{0pt}%
\pgfpathmoveto{\pgfqpoint{3.956527in}{3.307551in}}%
\pgfpathlineto{\pgfqpoint{4.473902in}{4.321654in}}%
\pgfusepath{stroke}%
\end{pgfscope}%
\begin{pgfscope}%
\pgfpathrectangle{\pgfqpoint{0.481978in}{0.331635in}}{\pgfqpoint{9.300000in}{7.700000in}}%
\pgfusepath{clip}%
\pgfsetrectcap%
\pgfsetroundjoin%
\pgfsetlinewidth{1.505625pt}%
\definecolor{currentstroke}{rgb}{0.870588,0.733333,0.607843}%
\pgfsetstrokecolor{currentstroke}%
\pgfsetstrokeopacity{0.800000}%
\pgfsetdash{}{0pt}%
\pgfpathmoveto{\pgfqpoint{2.593480in}{4.742968in}}%
\pgfpathlineto{\pgfqpoint{4.473902in}{4.321654in}}%
\pgfusepath{stroke}%
\end{pgfscope}%
\begin{pgfscope}%
\pgfpathrectangle{\pgfqpoint{0.481978in}{0.331635in}}{\pgfqpoint{9.300000in}{7.700000in}}%
\pgfusepath{clip}%
\pgfsetrectcap%
\pgfsetroundjoin%
\pgfsetlinewidth{1.505625pt}%
\definecolor{currentstroke}{rgb}{0.870588,0.733333,0.607843}%
\pgfsetstrokecolor{currentstroke}%
\pgfsetstrokeopacity{0.800000}%
\pgfsetdash{}{0pt}%
\pgfpathmoveto{\pgfqpoint{1.766805in}{3.503655in}}%
\pgfpathlineto{\pgfqpoint{4.473902in}{4.321654in}}%
\pgfusepath{stroke}%
\end{pgfscope}%
\begin{pgfscope}%
\pgfpathrectangle{\pgfqpoint{0.481978in}{0.331635in}}{\pgfqpoint{9.300000in}{7.700000in}}%
\pgfusepath{clip}%
\pgfsetrectcap%
\pgfsetroundjoin%
\pgfsetlinewidth{1.505625pt}%
\definecolor{currentstroke}{rgb}{0.870588,0.733333,0.607843}%
\pgfsetstrokecolor{currentstroke}%
\pgfsetstrokeopacity{0.800000}%
\pgfsetdash{}{0pt}%
\pgfpathmoveto{\pgfqpoint{4.977562in}{4.282649in}}%
\pgfpathlineto{\pgfqpoint{4.473902in}{4.321654in}}%
\pgfusepath{stroke}%
\end{pgfscope}%
\begin{pgfscope}%
\pgfpathrectangle{\pgfqpoint{0.481978in}{0.331635in}}{\pgfqpoint{9.300000in}{7.700000in}}%
\pgfusepath{clip}%
\pgfsetrectcap%
\pgfsetroundjoin%
\pgfsetlinewidth{1.505625pt}%
\definecolor{currentstroke}{rgb}{0.870588,0.733333,0.607843}%
\pgfsetstrokecolor{currentstroke}%
\pgfsetstrokeopacity{0.800000}%
\pgfsetdash{}{0pt}%
\pgfpathmoveto{\pgfqpoint{5.616567in}{3.990039in}}%
\pgfpathlineto{\pgfqpoint{4.473902in}{4.321654in}}%
\pgfusepath{stroke}%
\end{pgfscope}%
\begin{pgfscope}%
\pgfpathrectangle{\pgfqpoint{0.481978in}{0.331635in}}{\pgfqpoint{9.300000in}{7.700000in}}%
\pgfusepath{clip}%
\pgfsetrectcap%
\pgfsetroundjoin%
\pgfsetlinewidth{1.505625pt}%
\definecolor{currentstroke}{rgb}{0.870588,0.733333,0.607843}%
\pgfsetstrokecolor{currentstroke}%
\pgfsetstrokeopacity{0.800000}%
\pgfsetdash{}{0pt}%
\pgfpathmoveto{\pgfqpoint{2.747420in}{2.580517in}}%
\pgfpathlineto{\pgfqpoint{4.473902in}{4.321654in}}%
\pgfusepath{stroke}%
\end{pgfscope}%
\begin{pgfscope}%
\pgfpathrectangle{\pgfqpoint{0.481978in}{0.331635in}}{\pgfqpoint{9.300000in}{7.700000in}}%
\pgfusepath{clip}%
\pgfsetrectcap%
\pgfsetroundjoin%
\pgfsetlinewidth{1.505625pt}%
\definecolor{currentstroke}{rgb}{0.870588,0.733333,0.607843}%
\pgfsetstrokecolor{currentstroke}%
\pgfsetstrokeopacity{0.800000}%
\pgfsetdash{}{0pt}%
\pgfpathmoveto{\pgfqpoint{3.710302in}{3.442909in}}%
\pgfpathlineto{\pgfqpoint{4.473902in}{4.321654in}}%
\pgfusepath{stroke}%
\end{pgfscope}%
\begin{pgfscope}%
\pgfpathrectangle{\pgfqpoint{0.481978in}{0.331635in}}{\pgfqpoint{9.300000in}{7.700000in}}%
\pgfusepath{clip}%
\pgfsetrectcap%
\pgfsetroundjoin%
\pgfsetlinewidth{1.505625pt}%
\definecolor{currentstroke}{rgb}{0.870588,0.733333,0.607843}%
\pgfsetstrokecolor{currentstroke}%
\pgfsetstrokeopacity{0.800000}%
\pgfsetdash{}{0pt}%
\pgfpathmoveto{\pgfqpoint{2.149626in}{4.883578in}}%
\pgfpathlineto{\pgfqpoint{4.473902in}{4.321654in}}%
\pgfusepath{stroke}%
\end{pgfscope}%
\begin{pgfscope}%
\pgfpathrectangle{\pgfqpoint{0.481978in}{0.331635in}}{\pgfqpoint{9.300000in}{7.700000in}}%
\pgfusepath{clip}%
\pgfsetrectcap%
\pgfsetroundjoin%
\pgfsetlinewidth{1.505625pt}%
\definecolor{currentstroke}{rgb}{0.870588,0.733333,0.607843}%
\pgfsetstrokecolor{currentstroke}%
\pgfsetstrokeopacity{0.800000}%
\pgfsetdash{}{0pt}%
\pgfpathmoveto{\pgfqpoint{2.077561in}{3.103422in}}%
\pgfpathlineto{\pgfqpoint{4.473902in}{4.321654in}}%
\pgfusepath{stroke}%
\end{pgfscope}%
\begin{pgfscope}%
\pgfpathrectangle{\pgfqpoint{0.481978in}{0.331635in}}{\pgfqpoint{9.300000in}{7.700000in}}%
\pgfusepath{clip}%
\pgfsetrectcap%
\pgfsetroundjoin%
\pgfsetlinewidth{1.505625pt}%
\definecolor{currentstroke}{rgb}{0.870588,0.733333,0.607843}%
\pgfsetstrokecolor{currentstroke}%
\pgfsetstrokeopacity{0.800000}%
\pgfsetdash{}{0pt}%
\pgfpathmoveto{\pgfqpoint{1.957320in}{5.229684in}}%
\pgfpathlineto{\pgfqpoint{4.473902in}{4.321654in}}%
\pgfusepath{stroke}%
\end{pgfscope}%
\begin{pgfscope}%
\pgfpathrectangle{\pgfqpoint{0.481978in}{0.331635in}}{\pgfqpoint{9.300000in}{7.700000in}}%
\pgfusepath{clip}%
\pgfsetrectcap%
\pgfsetroundjoin%
\pgfsetlinewidth{1.505625pt}%
\definecolor{currentstroke}{rgb}{0.870588,0.733333,0.607843}%
\pgfsetstrokecolor{currentstroke}%
\pgfsetstrokeopacity{0.800000}%
\pgfsetdash{}{0pt}%
\pgfpathmoveto{\pgfqpoint{2.660918in}{3.550587in}}%
\pgfpathlineto{\pgfqpoint{4.473902in}{4.321654in}}%
\pgfusepath{stroke}%
\end{pgfscope}%
\begin{pgfscope}%
\pgfpathrectangle{\pgfqpoint{0.481978in}{0.331635in}}{\pgfqpoint{9.300000in}{7.700000in}}%
\pgfusepath{clip}%
\pgfsetrectcap%
\pgfsetroundjoin%
\pgfsetlinewidth{1.505625pt}%
\definecolor{currentstroke}{rgb}{0.870588,0.733333,0.607843}%
\pgfsetstrokecolor{currentstroke}%
\pgfsetstrokeopacity{0.800000}%
\pgfsetdash{}{0pt}%
\pgfpathmoveto{\pgfqpoint{7.871019in}{5.550531in}}%
\pgfpathlineto{\pgfqpoint{4.473902in}{4.321654in}}%
\pgfusepath{stroke}%
\end{pgfscope}%
\begin{pgfscope}%
\pgfpathrectangle{\pgfqpoint{0.481978in}{0.331635in}}{\pgfqpoint{9.300000in}{7.700000in}}%
\pgfusepath{clip}%
\pgfsetrectcap%
\pgfsetroundjoin%
\pgfsetlinewidth{1.505625pt}%
\definecolor{currentstroke}{rgb}{0.870588,0.733333,0.607843}%
\pgfsetstrokecolor{currentstroke}%
\pgfsetstrokeopacity{0.800000}%
\pgfsetdash{}{0pt}%
\pgfpathmoveto{\pgfqpoint{5.367711in}{4.176744in}}%
\pgfpathlineto{\pgfqpoint{4.473902in}{4.321654in}}%
\pgfusepath{stroke}%
\end{pgfscope}%
\begin{pgfscope}%
\pgfpathrectangle{\pgfqpoint{0.481978in}{0.331635in}}{\pgfqpoint{9.300000in}{7.700000in}}%
\pgfusepath{clip}%
\pgfsetrectcap%
\pgfsetroundjoin%
\pgfsetlinewidth{1.505625pt}%
\definecolor{currentstroke}{rgb}{0.870588,0.733333,0.607843}%
\pgfsetstrokecolor{currentstroke}%
\pgfsetstrokeopacity{0.800000}%
\pgfsetdash{}{0pt}%
\pgfpathmoveto{\pgfqpoint{2.321412in}{4.244642in}}%
\pgfpathlineto{\pgfqpoint{4.473902in}{4.321654in}}%
\pgfusepath{stroke}%
\end{pgfscope}%
\begin{pgfscope}%
\pgfpathrectangle{\pgfqpoint{0.481978in}{0.331635in}}{\pgfqpoint{9.300000in}{7.700000in}}%
\pgfusepath{clip}%
\pgfsetrectcap%
\pgfsetroundjoin%
\pgfsetlinewidth{1.505625pt}%
\definecolor{currentstroke}{rgb}{0.870588,0.733333,0.607843}%
\pgfsetstrokecolor{currentstroke}%
\pgfsetstrokeopacity{0.800000}%
\pgfsetdash{}{0pt}%
\pgfpathmoveto{\pgfqpoint{7.119603in}{6.303649in}}%
\pgfpathlineto{\pgfqpoint{4.473902in}{4.321654in}}%
\pgfusepath{stroke}%
\end{pgfscope}%
\begin{pgfscope}%
\pgfpathrectangle{\pgfqpoint{0.481978in}{0.331635in}}{\pgfqpoint{9.300000in}{7.700000in}}%
\pgfusepath{clip}%
\pgfsetrectcap%
\pgfsetroundjoin%
\pgfsetlinewidth{1.505625pt}%
\definecolor{currentstroke}{rgb}{0.870588,0.733333,0.607843}%
\pgfsetstrokecolor{currentstroke}%
\pgfsetstrokeopacity{0.800000}%
\pgfsetdash{}{0pt}%
\pgfpathmoveto{\pgfqpoint{6.487284in}{5.504131in}}%
\pgfpathlineto{\pgfqpoint{4.473902in}{4.321654in}}%
\pgfusepath{stroke}%
\end{pgfscope}%
\begin{pgfscope}%
\pgfpathrectangle{\pgfqpoint{0.481978in}{0.331635in}}{\pgfqpoint{9.300000in}{7.700000in}}%
\pgfusepath{clip}%
\pgfsetrectcap%
\pgfsetroundjoin%
\pgfsetlinewidth{1.505625pt}%
\definecolor{currentstroke}{rgb}{0.870588,0.733333,0.607843}%
\pgfsetstrokecolor{currentstroke}%
\pgfsetstrokeopacity{0.800000}%
\pgfsetdash{}{0pt}%
\pgfpathmoveto{\pgfqpoint{8.920274in}{3.555200in}}%
\pgfpathlineto{\pgfqpoint{4.473902in}{4.321654in}}%
\pgfusepath{stroke}%
\end{pgfscope}%
\begin{pgfscope}%
\pgfpathrectangle{\pgfqpoint{0.481978in}{0.331635in}}{\pgfqpoint{9.300000in}{7.700000in}}%
\pgfusepath{clip}%
\pgfsetrectcap%
\pgfsetroundjoin%
\pgfsetlinewidth{1.505625pt}%
\definecolor{currentstroke}{rgb}{0.870588,0.733333,0.607843}%
\pgfsetstrokecolor{currentstroke}%
\pgfsetstrokeopacity{0.800000}%
\pgfsetdash{}{0pt}%
\pgfpathmoveto{\pgfqpoint{7.303090in}{6.145660in}}%
\pgfpathlineto{\pgfqpoint{4.473902in}{4.321654in}}%
\pgfusepath{stroke}%
\end{pgfscope}%
\begin{pgfscope}%
\pgfpathrectangle{\pgfqpoint{0.481978in}{0.331635in}}{\pgfqpoint{9.300000in}{7.700000in}}%
\pgfusepath{clip}%
\pgfsetrectcap%
\pgfsetroundjoin%
\pgfsetlinewidth{1.505625pt}%
\definecolor{currentstroke}{rgb}{0.870588,0.733333,0.607843}%
\pgfsetstrokecolor{currentstroke}%
\pgfsetstrokeopacity{0.800000}%
\pgfsetdash{}{0pt}%
\pgfpathmoveto{\pgfqpoint{9.189203in}{4.320348in}}%
\pgfpathlineto{\pgfqpoint{4.473902in}{4.321654in}}%
\pgfusepath{stroke}%
\end{pgfscope}%
\begin{pgfscope}%
\pgfsetrectcap%
\pgfsetmiterjoin%
\pgfsetlinewidth{0.803000pt}%
\definecolor{currentstroke}{rgb}{0.000000,0.000000,0.000000}%
\pgfsetstrokecolor{currentstroke}%
\pgfsetdash{}{0pt}%
\pgfpathmoveto{\pgfqpoint{0.481978in}{0.331635in}}%
\pgfpathlineto{\pgfqpoint{0.481978in}{8.031635in}}%
\pgfusepath{stroke}%
\end{pgfscope}%
\begin{pgfscope}%
\pgfsetrectcap%
\pgfsetmiterjoin%
\pgfsetlinewidth{0.803000pt}%
\definecolor{currentstroke}{rgb}{0.000000,0.000000,0.000000}%
\pgfsetstrokecolor{currentstroke}%
\pgfsetdash{}{0pt}%
\pgfpathmoveto{\pgfqpoint{9.781978in}{0.331635in}}%
\pgfpathlineto{\pgfqpoint{9.781978in}{8.031635in}}%
\pgfusepath{stroke}%
\end{pgfscope}%
\begin{pgfscope}%
\pgfsetrectcap%
\pgfsetmiterjoin%
\pgfsetlinewidth{0.803000pt}%
\definecolor{currentstroke}{rgb}{0.000000,0.000000,0.000000}%
\pgfsetstrokecolor{currentstroke}%
\pgfsetdash{}{0pt}%
\pgfpathmoveto{\pgfqpoint{0.481978in}{0.331635in}}%
\pgfpathlineto{\pgfqpoint{9.781978in}{0.331635in}}%
\pgfusepath{stroke}%
\end{pgfscope}%
\begin{pgfscope}%
\pgfsetrectcap%
\pgfsetmiterjoin%
\pgfsetlinewidth{0.803000pt}%
\definecolor{currentstroke}{rgb}{0.000000,0.000000,0.000000}%
\pgfsetstrokecolor{currentstroke}%
\pgfsetdash{}{0pt}%
\pgfpathmoveto{\pgfqpoint{0.481978in}{8.031635in}}%
\pgfpathlineto{\pgfqpoint{9.781978in}{8.031635in}}%
\pgfusepath{stroke}%
\end{pgfscope}%
\begin{pgfscope}%
\definecolor{textcolor}{rgb}{0.000000,0.000000,0.000000}%
\pgfsetstrokecolor{textcolor}%
\pgfsetfillcolor{textcolor}%
\pgftext[x=5.131978in,y=8.114968in,,base]{\color{textcolor}\sffamily\fontsize{12.000000}{14.400000}\selectfont T-SNE for chair images with domain randomisation}%
\end{pgfscope}%
\begin{pgfscope}%
\pgfsetbuttcap%
\pgfsetmiterjoin%
\definecolor{currentfill}{rgb}{1.000000,1.000000,1.000000}%
\pgfsetfillcolor{currentfill}%
\pgfsetfillopacity{0.800000}%
\pgfsetlinewidth{1.003750pt}%
\definecolor{currentstroke}{rgb}{0.800000,0.800000,0.800000}%
\pgfsetstrokecolor{currentstroke}%
\pgfsetstrokeopacity{0.800000}%
\pgfsetdash{}{0pt}%
\pgfpathmoveto{\pgfqpoint{9.879200in}{3.539566in}}%
\pgfpathlineto{\pgfqpoint{12.348384in}{3.539566in}}%
\pgfpathquadraticcurveto{\pgfqpoint{12.376162in}{3.539566in}}{\pgfqpoint{12.376162in}{3.567344in}}%
\pgfpathlineto{\pgfqpoint{12.376162in}{4.795926in}}%
\pgfpathquadraticcurveto{\pgfqpoint{12.376162in}{4.823704in}}{\pgfqpoint{12.348384in}{4.823704in}}%
\pgfpathlineto{\pgfqpoint{9.879200in}{4.823704in}}%
\pgfpathquadraticcurveto{\pgfqpoint{9.851422in}{4.823704in}}{\pgfqpoint{9.851422in}{4.795926in}}%
\pgfpathlineto{\pgfqpoint{9.851422in}{3.567344in}}%
\pgfpathquadraticcurveto{\pgfqpoint{9.851422in}{3.539566in}}{\pgfqpoint{9.879200in}{3.539566in}}%
\pgfpathclose%
\pgfusepath{stroke,fill}%
\end{pgfscope}%
\begin{pgfscope}%
\pgfsetbuttcap%
\pgfsetroundjoin%
\definecolor{currentfill}{rgb}{0.631373,0.788235,0.956863}%
\pgfsetfillcolor{currentfill}%
\pgfsetlinewidth{1.003750pt}%
\definecolor{currentstroke}{rgb}{0.631373,0.788235,0.956863}%
\pgfsetstrokecolor{currentstroke}%
\pgfsetdash{}{0pt}%
\pgfsys@defobject{currentmarker}{\pgfqpoint{-0.041667in}{-0.041667in}}{\pgfqpoint{0.041667in}{0.041667in}}{%
\pgfpathmoveto{\pgfqpoint{0.000000in}{-0.041667in}}%
\pgfpathcurveto{\pgfqpoint{0.011050in}{-0.041667in}}{\pgfqpoint{0.021649in}{-0.037276in}}{\pgfqpoint{0.029463in}{-0.029463in}}%
\pgfpathcurveto{\pgfqpoint{0.037276in}{-0.021649in}}{\pgfqpoint{0.041667in}{-0.011050in}}{\pgfqpoint{0.041667in}{0.000000in}}%
\pgfpathcurveto{\pgfqpoint{0.041667in}{0.011050in}}{\pgfqpoint{0.037276in}{0.021649in}}{\pgfqpoint{0.029463in}{0.029463in}}%
\pgfpathcurveto{\pgfqpoint{0.021649in}{0.037276in}}{\pgfqpoint{0.011050in}{0.041667in}}{\pgfqpoint{0.000000in}{0.041667in}}%
\pgfpathcurveto{\pgfqpoint{-0.011050in}{0.041667in}}{\pgfqpoint{-0.021649in}{0.037276in}}{\pgfqpoint{-0.029463in}{0.029463in}}%
\pgfpathcurveto{\pgfqpoint{-0.037276in}{0.021649in}}{\pgfqpoint{-0.041667in}{0.011050in}}{\pgfqpoint{-0.041667in}{0.000000in}}%
\pgfpathcurveto{\pgfqpoint{-0.041667in}{-0.011050in}}{\pgfqpoint{-0.037276in}{-0.021649in}}{\pgfqpoint{-0.029463in}{-0.029463in}}%
\pgfpathcurveto{\pgfqpoint{-0.021649in}{-0.037276in}}{\pgfqpoint{-0.011050in}{-0.041667in}}{\pgfqpoint{0.000000in}{-0.041667in}}%
\pgfpathclose%
\pgfusepath{stroke,fill}%
}%
\begin{pgfscope}%
\pgfsys@transformshift{10.045867in}{4.699084in}%
\pgfsys@useobject{currentmarker}{}%
\end{pgfscope}%
\end{pgfscope}%
\begin{pgfscope}%
\definecolor{textcolor}{rgb}{0.000000,0.000000,0.000000}%
\pgfsetstrokecolor{textcolor}%
\pgfsetfillcolor{textcolor}%
\pgftext[x=10.295867in,y=4.662625in,left,base]{\color{textcolor}\sffamily\fontsize{10.000000}{12.000000}\selectfont Pix3D}%
\end{pgfscope}%
\begin{pgfscope}%
\pgfsetbuttcap%
\pgfsetroundjoin%
\definecolor{currentfill}{rgb}{1.000000,0.705882,0.509804}%
\pgfsetfillcolor{currentfill}%
\pgfsetlinewidth{1.003750pt}%
\definecolor{currentstroke}{rgb}{1.000000,0.705882,0.509804}%
\pgfsetstrokecolor{currentstroke}%
\pgfsetdash{}{0pt}%
\pgfsys@defobject{currentmarker}{\pgfqpoint{-0.041667in}{-0.041667in}}{\pgfqpoint{0.041667in}{0.041667in}}{%
\pgfpathmoveto{\pgfqpoint{0.000000in}{-0.041667in}}%
\pgfpathcurveto{\pgfqpoint{0.011050in}{-0.041667in}}{\pgfqpoint{0.021649in}{-0.037276in}}{\pgfqpoint{0.029463in}{-0.029463in}}%
\pgfpathcurveto{\pgfqpoint{0.037276in}{-0.021649in}}{\pgfqpoint{0.041667in}{-0.011050in}}{\pgfqpoint{0.041667in}{0.000000in}}%
\pgfpathcurveto{\pgfqpoint{0.041667in}{0.011050in}}{\pgfqpoint{0.037276in}{0.021649in}}{\pgfqpoint{0.029463in}{0.029463in}}%
\pgfpathcurveto{\pgfqpoint{0.021649in}{0.037276in}}{\pgfqpoint{0.011050in}{0.041667in}}{\pgfqpoint{0.000000in}{0.041667in}}%
\pgfpathcurveto{\pgfqpoint{-0.011050in}{0.041667in}}{\pgfqpoint{-0.021649in}{0.037276in}}{\pgfqpoint{-0.029463in}{0.029463in}}%
\pgfpathcurveto{\pgfqpoint{-0.037276in}{0.021649in}}{\pgfqpoint{-0.041667in}{0.011050in}}{\pgfqpoint{-0.041667in}{0.000000in}}%
\pgfpathcurveto{\pgfqpoint{-0.041667in}{-0.011050in}}{\pgfqpoint{-0.037276in}{-0.021649in}}{\pgfqpoint{-0.029463in}{-0.029463in}}%
\pgfpathcurveto{\pgfqpoint{-0.021649in}{-0.037276in}}{\pgfqpoint{-0.011050in}{-0.041667in}}{\pgfqpoint{0.000000in}{-0.041667in}}%
\pgfpathclose%
\pgfusepath{stroke,fill}%
}%
\begin{pgfscope}%
\pgfsys@transformshift{10.045867in}{4.495226in}%
\pgfsys@useobject{currentmarker}{}%
\end{pgfscope}%
\end{pgfscope}%
\begin{pgfscope}%
\definecolor{textcolor}{rgb}{0.000000,0.000000,0.000000}%
\pgfsetstrokecolor{textcolor}%
\pgfsetfillcolor{textcolor}%
\pgftext[x=10.295867in,y=4.458768in,left,base]{\color{textcolor}\sffamily\fontsize{10.000000}{12.000000}\selectfont s2r3dfree\_textureless}%
\end{pgfscope}%
\begin{pgfscope}%
\pgfsetbuttcap%
\pgfsetroundjoin%
\definecolor{currentfill}{rgb}{0.552941,0.898039,0.631373}%
\pgfsetfillcolor{currentfill}%
\pgfsetlinewidth{1.003750pt}%
\definecolor{currentstroke}{rgb}{0.552941,0.898039,0.631373}%
\pgfsetstrokecolor{currentstroke}%
\pgfsetdash{}{0pt}%
\pgfsys@defobject{currentmarker}{\pgfqpoint{-0.041667in}{-0.041667in}}{\pgfqpoint{0.041667in}{0.041667in}}{%
\pgfpathmoveto{\pgfqpoint{0.000000in}{-0.041667in}}%
\pgfpathcurveto{\pgfqpoint{0.011050in}{-0.041667in}}{\pgfqpoint{0.021649in}{-0.037276in}}{\pgfqpoint{0.029463in}{-0.029463in}}%
\pgfpathcurveto{\pgfqpoint{0.037276in}{-0.021649in}}{\pgfqpoint{0.041667in}{-0.011050in}}{\pgfqpoint{0.041667in}{0.000000in}}%
\pgfpathcurveto{\pgfqpoint{0.041667in}{0.011050in}}{\pgfqpoint{0.037276in}{0.021649in}}{\pgfqpoint{0.029463in}{0.029463in}}%
\pgfpathcurveto{\pgfqpoint{0.021649in}{0.037276in}}{\pgfqpoint{0.011050in}{0.041667in}}{\pgfqpoint{0.000000in}{0.041667in}}%
\pgfpathcurveto{\pgfqpoint{-0.011050in}{0.041667in}}{\pgfqpoint{-0.021649in}{0.037276in}}{\pgfqpoint{-0.029463in}{0.029463in}}%
\pgfpathcurveto{\pgfqpoint{-0.037276in}{0.021649in}}{\pgfqpoint{-0.041667in}{0.011050in}}{\pgfqpoint{-0.041667in}{0.000000in}}%
\pgfpathcurveto{\pgfqpoint{-0.041667in}{-0.011050in}}{\pgfqpoint{-0.037276in}{-0.021649in}}{\pgfqpoint{-0.029463in}{-0.029463in}}%
\pgfpathcurveto{\pgfqpoint{-0.021649in}{-0.037276in}}{\pgfqpoint{-0.011050in}{-0.041667in}}{\pgfqpoint{0.000000in}{-0.041667in}}%
\pgfpathclose%
\pgfusepath{stroke,fill}%
}%
\begin{pgfscope}%
\pgfsys@transformshift{10.045867in}{4.287504in}%
\pgfsys@useobject{currentmarker}{}%
\end{pgfscope}%
\end{pgfscope}%
\begin{pgfscope}%
\definecolor{textcolor}{rgb}{0.000000,0.000000,0.000000}%
\pgfsetstrokecolor{textcolor}%
\pgfsetfillcolor{textcolor}%
\pgftext[x=10.295867in,y=4.251045in,left,base]{\color{textcolor}\sffamily\fontsize{10.000000}{12.000000}\selectfont s2r3dfree\_textureless\_light}%
\end{pgfscope}%
\begin{pgfscope}%
\pgfsetbuttcap%
\pgfsetroundjoin%
\definecolor{currentfill}{rgb}{1.000000,0.623529,0.607843}%
\pgfsetfillcolor{currentfill}%
\pgfsetlinewidth{1.003750pt}%
\definecolor{currentstroke}{rgb}{1.000000,0.623529,0.607843}%
\pgfsetstrokecolor{currentstroke}%
\pgfsetdash{}{0pt}%
\pgfsys@defobject{currentmarker}{\pgfqpoint{-0.041667in}{-0.041667in}}{\pgfqpoint{0.041667in}{0.041667in}}{%
\pgfpathmoveto{\pgfqpoint{0.000000in}{-0.041667in}}%
\pgfpathcurveto{\pgfqpoint{0.011050in}{-0.041667in}}{\pgfqpoint{0.021649in}{-0.037276in}}{\pgfqpoint{0.029463in}{-0.029463in}}%
\pgfpathcurveto{\pgfqpoint{0.037276in}{-0.021649in}}{\pgfqpoint{0.041667in}{-0.011050in}}{\pgfqpoint{0.041667in}{0.000000in}}%
\pgfpathcurveto{\pgfqpoint{0.041667in}{0.011050in}}{\pgfqpoint{0.037276in}{0.021649in}}{\pgfqpoint{0.029463in}{0.029463in}}%
\pgfpathcurveto{\pgfqpoint{0.021649in}{0.037276in}}{\pgfqpoint{0.011050in}{0.041667in}}{\pgfqpoint{0.000000in}{0.041667in}}%
\pgfpathcurveto{\pgfqpoint{-0.011050in}{0.041667in}}{\pgfqpoint{-0.021649in}{0.037276in}}{\pgfqpoint{-0.029463in}{0.029463in}}%
\pgfpathcurveto{\pgfqpoint{-0.037276in}{0.021649in}}{\pgfqpoint{-0.041667in}{0.011050in}}{\pgfqpoint{-0.041667in}{0.000000in}}%
\pgfpathcurveto{\pgfqpoint{-0.041667in}{-0.011050in}}{\pgfqpoint{-0.037276in}{-0.021649in}}{\pgfqpoint{-0.029463in}{-0.029463in}}%
\pgfpathcurveto{\pgfqpoint{-0.021649in}{-0.037276in}}{\pgfqpoint{-0.011050in}{-0.041667in}}{\pgfqpoint{0.000000in}{-0.041667in}}%
\pgfpathclose%
\pgfusepath{stroke,fill}%
}%
\begin{pgfscope}%
\pgfsys@transformshift{10.045867in}{4.079781in}%
\pgfsys@useobject{currentmarker}{}%
\end{pgfscope}%
\end{pgfscope}%
\begin{pgfscope}%
\definecolor{textcolor}{rgb}{0.000000,0.000000,0.000000}%
\pgfsetstrokecolor{textcolor}%
\pgfsetfillcolor{textcolor}%
\pgftext[x=10.295867in,y=4.043322in,left,base]{\color{textcolor}\sffamily\fontsize{10.000000}{12.000000}\selectfont s2r3dfree\_background}%
\end{pgfscope}%
\begin{pgfscope}%
\pgfsetbuttcap%
\pgfsetroundjoin%
\definecolor{currentfill}{rgb}{0.815686,0.733333,1.000000}%
\pgfsetfillcolor{currentfill}%
\pgfsetlinewidth{1.003750pt}%
\definecolor{currentstroke}{rgb}{0.815686,0.733333,1.000000}%
\pgfsetstrokecolor{currentstroke}%
\pgfsetdash{}{0pt}%
\pgfsys@defobject{currentmarker}{\pgfqpoint{-0.041667in}{-0.041667in}}{\pgfqpoint{0.041667in}{0.041667in}}{%
\pgfpathmoveto{\pgfqpoint{0.000000in}{-0.041667in}}%
\pgfpathcurveto{\pgfqpoint{0.011050in}{-0.041667in}}{\pgfqpoint{0.021649in}{-0.037276in}}{\pgfqpoint{0.029463in}{-0.029463in}}%
\pgfpathcurveto{\pgfqpoint{0.037276in}{-0.021649in}}{\pgfqpoint{0.041667in}{-0.011050in}}{\pgfqpoint{0.041667in}{0.000000in}}%
\pgfpathcurveto{\pgfqpoint{0.041667in}{0.011050in}}{\pgfqpoint{0.037276in}{0.021649in}}{\pgfqpoint{0.029463in}{0.029463in}}%
\pgfpathcurveto{\pgfqpoint{0.021649in}{0.037276in}}{\pgfqpoint{0.011050in}{0.041667in}}{\pgfqpoint{0.000000in}{0.041667in}}%
\pgfpathcurveto{\pgfqpoint{-0.011050in}{0.041667in}}{\pgfqpoint{-0.021649in}{0.037276in}}{\pgfqpoint{-0.029463in}{0.029463in}}%
\pgfpathcurveto{\pgfqpoint{-0.037276in}{0.021649in}}{\pgfqpoint{-0.041667in}{0.011050in}}{\pgfqpoint{-0.041667in}{0.000000in}}%
\pgfpathcurveto{\pgfqpoint{-0.041667in}{-0.011050in}}{\pgfqpoint{-0.037276in}{-0.021649in}}{\pgfqpoint{-0.029463in}{-0.029463in}}%
\pgfpathcurveto{\pgfqpoint{-0.021649in}{-0.037276in}}{\pgfqpoint{-0.011050in}{-0.041667in}}{\pgfqpoint{0.000000in}{-0.041667in}}%
\pgfpathclose%
\pgfusepath{stroke,fill}%
}%
\begin{pgfscope}%
\pgfsys@transformshift{10.045867in}{3.872058in}%
\pgfsys@useobject{currentmarker}{}%
\end{pgfscope}%
\end{pgfscope}%
\begin{pgfscope}%
\definecolor{textcolor}{rgb}{0.000000,0.000000,0.000000}%
\pgfsetstrokecolor{textcolor}%
\pgfsetfillcolor{textcolor}%
\pgftext[x=10.295867in,y=3.835600in,left,base]{\color{textcolor}\sffamily\fontsize{10.000000}{12.000000}\selectfont s2r3dfree\_background\_light2}%
\end{pgfscope}%
\begin{pgfscope}%
\pgfsetbuttcap%
\pgfsetroundjoin%
\definecolor{currentfill}{rgb}{0.870588,0.733333,0.607843}%
\pgfsetfillcolor{currentfill}%
\pgfsetlinewidth{1.003750pt}%
\definecolor{currentstroke}{rgb}{0.870588,0.733333,0.607843}%
\pgfsetstrokecolor{currentstroke}%
\pgfsetdash{}{0pt}%
\pgfsys@defobject{currentmarker}{\pgfqpoint{-0.041667in}{-0.041667in}}{\pgfqpoint{0.041667in}{0.041667in}}{%
\pgfpathmoveto{\pgfqpoint{0.000000in}{-0.041667in}}%
\pgfpathcurveto{\pgfqpoint{0.011050in}{-0.041667in}}{\pgfqpoint{0.021649in}{-0.037276in}}{\pgfqpoint{0.029463in}{-0.029463in}}%
\pgfpathcurveto{\pgfqpoint{0.037276in}{-0.021649in}}{\pgfqpoint{0.041667in}{-0.011050in}}{\pgfqpoint{0.041667in}{0.000000in}}%
\pgfpathcurveto{\pgfqpoint{0.041667in}{0.011050in}}{\pgfqpoint{0.037276in}{0.021649in}}{\pgfqpoint{0.029463in}{0.029463in}}%
\pgfpathcurveto{\pgfqpoint{0.021649in}{0.037276in}}{\pgfqpoint{0.011050in}{0.041667in}}{\pgfqpoint{0.000000in}{0.041667in}}%
\pgfpathcurveto{\pgfqpoint{-0.011050in}{0.041667in}}{\pgfqpoint{-0.021649in}{0.037276in}}{\pgfqpoint{-0.029463in}{0.029463in}}%
\pgfpathcurveto{\pgfqpoint{-0.037276in}{0.021649in}}{\pgfqpoint{-0.041667in}{0.011050in}}{\pgfqpoint{-0.041667in}{0.000000in}}%
\pgfpathcurveto{\pgfqpoint{-0.041667in}{-0.011050in}}{\pgfqpoint{-0.037276in}{-0.021649in}}{\pgfqpoint{-0.029463in}{-0.029463in}}%
\pgfpathcurveto{\pgfqpoint{-0.021649in}{-0.037276in}}{\pgfqpoint{-0.011050in}{-0.041667in}}{\pgfqpoint{0.000000in}{-0.041667in}}%
\pgfpathclose%
\pgfusepath{stroke,fill}%
}%
\begin{pgfscope}%
\pgfsys@transformshift{10.045867in}{3.664335in}%
\pgfsys@useobject{currentmarker}{}%
\end{pgfscope}%
\end{pgfscope}%
\begin{pgfscope}%
\definecolor{textcolor}{rgb}{0.000000,0.000000,0.000000}%
\pgfsetstrokecolor{textcolor}%
\pgfsetfillcolor{textcolor}%
\pgftext[x=10.295867in,y=3.627877in,left,base]{\color{textcolor}\sffamily\fontsize{10.000000}{12.000000}\selectfont s2r3dfree\_chair}%
\end{pgfscope}%
\end{pgfpicture}%
\makeatother%
\endgroup%
}
    \caption{T-SNE visualisation for chair images from Pix3d and \gls{free} dataset.}
    \label{fig:pix3dchair_s2r3dfreechair}
\end{figure}

\begin{figure}[!ht]
    \centering
    \resizebox{0.49\linewidth}{5cm}{%% Creator: Matplotlib, PGF backend
%%
%% To include the figure in your LaTeX document, write
%%   \input{<filename>.pgf}
%%
%% Make sure the required packages are loaded in your preamble
%%   \usepackage{pgf}
%%
%% Figures using additional raster images can only be included by \input if
%% they are in the same directory as the main LaTeX file. For loading figures
%% from other directories you can use the `import` package
%%   \usepackage{import}
%%
%% and then include the figures with
%%   \import{<path to file>}{<filename>.pgf}
%%
%% Matplotlib used the following preamble
%%   \usepackage{fontspec}
%%   \setmainfont{DejaVuSerif.ttf}[Path=\detokenize{/Users/apple/opt/anaconda3/envs/kaolin/lib/python3.7/site-packages/matplotlib/mpl-data/fonts/ttf/}]
%%   \setsansfont{DejaVuSans.ttf}[Path=\detokenize{/Users/apple/opt/anaconda3/envs/kaolin/lib/python3.7/site-packages/matplotlib/mpl-data/fonts/ttf/}]
%%   \setmonofont{DejaVuSansMono.ttf}[Path=\detokenize{/Users/apple/opt/anaconda3/envs/kaolin/lib/python3.7/site-packages/matplotlib/mpl-data/fonts/ttf/}]
%%
\begingroup%
\makeatletter%
\begin{pgfpicture}%
\pgfpathrectangle{\pgfpointorigin}{\pgfqpoint{11.959465in}{8.341596in}}%
\pgfusepath{use as bounding box, clip}%
\begin{pgfscope}%
\pgfsetbuttcap%
\pgfsetmiterjoin%
\definecolor{currentfill}{rgb}{1.000000,1.000000,1.000000}%
\pgfsetfillcolor{currentfill}%
\pgfsetlinewidth{0.000000pt}%
\definecolor{currentstroke}{rgb}{1.000000,1.000000,1.000000}%
\pgfsetstrokecolor{currentstroke}%
\pgfsetdash{}{0pt}%
\pgfpathmoveto{\pgfqpoint{0.000000in}{0.000000in}}%
\pgfpathlineto{\pgfqpoint{11.959465in}{0.000000in}}%
\pgfpathlineto{\pgfqpoint{11.959465in}{8.341596in}}%
\pgfpathlineto{\pgfqpoint{0.000000in}{8.341596in}}%
\pgfpathclose%
\pgfusepath{fill}%
\end{pgfscope}%
\begin{pgfscope}%
\pgfsetbuttcap%
\pgfsetmiterjoin%
\definecolor{currentfill}{rgb}{1.000000,1.000000,1.000000}%
\pgfsetfillcolor{currentfill}%
\pgfsetlinewidth{0.000000pt}%
\definecolor{currentstroke}{rgb}{0.000000,0.000000,0.000000}%
\pgfsetstrokecolor{currentstroke}%
\pgfsetstrokeopacity{0.000000}%
\pgfsetdash{}{0pt}%
\pgfpathmoveto{\pgfqpoint{0.481978in}{0.331635in}}%
\pgfpathlineto{\pgfqpoint{9.781978in}{0.331635in}}%
\pgfpathlineto{\pgfqpoint{9.781978in}{8.031635in}}%
\pgfpathlineto{\pgfqpoint{0.481978in}{8.031635in}}%
\pgfpathclose%
\pgfusepath{fill}%
\end{pgfscope}%
\begin{pgfscope}%
\pgfpathrectangle{\pgfqpoint{0.481978in}{0.331635in}}{\pgfqpoint{9.300000in}{7.700000in}}%
\pgfusepath{clip}%
\pgfsetbuttcap%
\pgfsetroundjoin%
\definecolor{currentfill}{rgb}{0.631373,0.788235,0.956863}%
\pgfsetfillcolor{currentfill}%
\pgfsetlinewidth{0.481800pt}%
\definecolor{currentstroke}{rgb}{1.000000,1.000000,1.000000}%
\pgfsetstrokecolor{currentstroke}%
\pgfsetdash{}{0pt}%
\pgfpathmoveto{\pgfqpoint{8.052863in}{1.510426in}}%
\pgfpathcurveto{\pgfqpoint{8.063914in}{1.510426in}}{\pgfqpoint{8.074513in}{1.514817in}}{\pgfqpoint{8.082326in}{1.522630in}}%
\pgfpathcurveto{\pgfqpoint{8.090140in}{1.530444in}}{\pgfqpoint{8.094530in}{1.541043in}}{\pgfqpoint{8.094530in}{1.552093in}}%
\pgfpathcurveto{\pgfqpoint{8.094530in}{1.563143in}}{\pgfqpoint{8.090140in}{1.573742in}}{\pgfqpoint{8.082326in}{1.581556in}}%
\pgfpathcurveto{\pgfqpoint{8.074513in}{1.589369in}}{\pgfqpoint{8.063914in}{1.593760in}}{\pgfqpoint{8.052863in}{1.593760in}}%
\pgfpathcurveto{\pgfqpoint{8.041813in}{1.593760in}}{\pgfqpoint{8.031214in}{1.589369in}}{\pgfqpoint{8.023401in}{1.581556in}}%
\pgfpathcurveto{\pgfqpoint{8.015587in}{1.573742in}}{\pgfqpoint{8.011197in}{1.563143in}}{\pgfqpoint{8.011197in}{1.552093in}}%
\pgfpathcurveto{\pgfqpoint{8.011197in}{1.541043in}}{\pgfqpoint{8.015587in}{1.530444in}}{\pgfqpoint{8.023401in}{1.522630in}}%
\pgfpathcurveto{\pgfqpoint{8.031214in}{1.514817in}}{\pgfqpoint{8.041813in}{1.510426in}}{\pgfqpoint{8.052863in}{1.510426in}}%
\pgfpathclose%
\pgfusepath{stroke,fill}%
\end{pgfscope}%
\begin{pgfscope}%
\pgfpathrectangle{\pgfqpoint{0.481978in}{0.331635in}}{\pgfqpoint{9.300000in}{7.700000in}}%
\pgfusepath{clip}%
\pgfsetbuttcap%
\pgfsetroundjoin%
\definecolor{currentfill}{rgb}{0.631373,0.788235,0.956863}%
\pgfsetfillcolor{currentfill}%
\pgfsetlinewidth{0.481800pt}%
\definecolor{currentstroke}{rgb}{1.000000,1.000000,1.000000}%
\pgfsetstrokecolor{currentstroke}%
\pgfsetdash{}{0pt}%
\pgfpathmoveto{\pgfqpoint{6.390490in}{4.831804in}}%
\pgfpathcurveto{\pgfqpoint{6.401540in}{4.831804in}}{\pgfqpoint{6.412139in}{4.836194in}}{\pgfqpoint{6.419952in}{4.844008in}}%
\pgfpathcurveto{\pgfqpoint{6.427766in}{4.851821in}}{\pgfqpoint{6.432156in}{4.862420in}}{\pgfqpoint{6.432156in}{4.873471in}}%
\pgfpathcurveto{\pgfqpoint{6.432156in}{4.884521in}}{\pgfqpoint{6.427766in}{4.895120in}}{\pgfqpoint{6.419952in}{4.902933in}}%
\pgfpathcurveto{\pgfqpoint{6.412139in}{4.910747in}}{\pgfqpoint{6.401540in}{4.915137in}}{\pgfqpoint{6.390490in}{4.915137in}}%
\pgfpathcurveto{\pgfqpoint{6.379439in}{4.915137in}}{\pgfqpoint{6.368840in}{4.910747in}}{\pgfqpoint{6.361027in}{4.902933in}}%
\pgfpathcurveto{\pgfqpoint{6.353213in}{4.895120in}}{\pgfqpoint{6.348823in}{4.884521in}}{\pgfqpoint{6.348823in}{4.873471in}}%
\pgfpathcurveto{\pgfqpoint{6.348823in}{4.862420in}}{\pgfqpoint{6.353213in}{4.851821in}}{\pgfqpoint{6.361027in}{4.844008in}}%
\pgfpathcurveto{\pgfqpoint{6.368840in}{4.836194in}}{\pgfqpoint{6.379439in}{4.831804in}}{\pgfqpoint{6.390490in}{4.831804in}}%
\pgfpathclose%
\pgfusepath{stroke,fill}%
\end{pgfscope}%
\begin{pgfscope}%
\pgfpathrectangle{\pgfqpoint{0.481978in}{0.331635in}}{\pgfqpoint{9.300000in}{7.700000in}}%
\pgfusepath{clip}%
\pgfsetbuttcap%
\pgfsetroundjoin%
\definecolor{currentfill}{rgb}{0.631373,0.788235,0.956863}%
\pgfsetfillcolor{currentfill}%
\pgfsetlinewidth{0.481800pt}%
\definecolor{currentstroke}{rgb}{1.000000,1.000000,1.000000}%
\pgfsetstrokecolor{currentstroke}%
\pgfsetdash{}{0pt}%
\pgfpathmoveto{\pgfqpoint{4.724066in}{3.560981in}}%
\pgfpathcurveto{\pgfqpoint{4.735116in}{3.560981in}}{\pgfqpoint{4.745715in}{3.565371in}}{\pgfqpoint{4.753529in}{3.573185in}}%
\pgfpathcurveto{\pgfqpoint{4.761342in}{3.580998in}}{\pgfqpoint{4.765733in}{3.591597in}}{\pgfqpoint{4.765733in}{3.602648in}}%
\pgfpathcurveto{\pgfqpoint{4.765733in}{3.613698in}}{\pgfqpoint{4.761342in}{3.624297in}}{\pgfqpoint{4.753529in}{3.632110in}}%
\pgfpathcurveto{\pgfqpoint{4.745715in}{3.639924in}}{\pgfqpoint{4.735116in}{3.644314in}}{\pgfqpoint{4.724066in}{3.644314in}}%
\pgfpathcurveto{\pgfqpoint{4.713016in}{3.644314in}}{\pgfqpoint{4.702417in}{3.639924in}}{\pgfqpoint{4.694603in}{3.632110in}}%
\pgfpathcurveto{\pgfqpoint{4.686790in}{3.624297in}}{\pgfqpoint{4.682399in}{3.613698in}}{\pgfqpoint{4.682399in}{3.602648in}}%
\pgfpathcurveto{\pgfqpoint{4.682399in}{3.591597in}}{\pgfqpoint{4.686790in}{3.580998in}}{\pgfqpoint{4.694603in}{3.573185in}}%
\pgfpathcurveto{\pgfqpoint{4.702417in}{3.565371in}}{\pgfqpoint{4.713016in}{3.560981in}}{\pgfqpoint{4.724066in}{3.560981in}}%
\pgfpathclose%
\pgfusepath{stroke,fill}%
\end{pgfscope}%
\begin{pgfscope}%
\pgfpathrectangle{\pgfqpoint{0.481978in}{0.331635in}}{\pgfqpoint{9.300000in}{7.700000in}}%
\pgfusepath{clip}%
\pgfsetbuttcap%
\pgfsetroundjoin%
\definecolor{currentfill}{rgb}{0.631373,0.788235,0.956863}%
\pgfsetfillcolor{currentfill}%
\pgfsetlinewidth{0.481800pt}%
\definecolor{currentstroke}{rgb}{1.000000,1.000000,1.000000}%
\pgfsetstrokecolor{currentstroke}%
\pgfsetdash{}{0pt}%
\pgfpathmoveto{\pgfqpoint{3.262933in}{3.133757in}}%
\pgfpathcurveto{\pgfqpoint{3.273983in}{3.133757in}}{\pgfqpoint{3.284582in}{3.138147in}}{\pgfqpoint{3.292396in}{3.145961in}}%
\pgfpathcurveto{\pgfqpoint{3.300209in}{3.153774in}}{\pgfqpoint{3.304600in}{3.164373in}}{\pgfqpoint{3.304600in}{3.175424in}}%
\pgfpathcurveto{\pgfqpoint{3.304600in}{3.186474in}}{\pgfqpoint{3.300209in}{3.197073in}}{\pgfqpoint{3.292396in}{3.204886in}}%
\pgfpathcurveto{\pgfqpoint{3.284582in}{3.212700in}}{\pgfqpoint{3.273983in}{3.217090in}}{\pgfqpoint{3.262933in}{3.217090in}}%
\pgfpathcurveto{\pgfqpoint{3.251883in}{3.217090in}}{\pgfqpoint{3.241284in}{3.212700in}}{\pgfqpoint{3.233470in}{3.204886in}}%
\pgfpathcurveto{\pgfqpoint{3.225657in}{3.197073in}}{\pgfqpoint{3.221266in}{3.186474in}}{\pgfqpoint{3.221266in}{3.175424in}}%
\pgfpathcurveto{\pgfqpoint{3.221266in}{3.164373in}}{\pgfqpoint{3.225657in}{3.153774in}}{\pgfqpoint{3.233470in}{3.145961in}}%
\pgfpathcurveto{\pgfqpoint{3.241284in}{3.138147in}}{\pgfqpoint{3.251883in}{3.133757in}}{\pgfqpoint{3.262933in}{3.133757in}}%
\pgfpathclose%
\pgfusepath{stroke,fill}%
\end{pgfscope}%
\begin{pgfscope}%
\pgfpathrectangle{\pgfqpoint{0.481978in}{0.331635in}}{\pgfqpoint{9.300000in}{7.700000in}}%
\pgfusepath{clip}%
\pgfsetbuttcap%
\pgfsetroundjoin%
\definecolor{currentfill}{rgb}{0.631373,0.788235,0.956863}%
\pgfsetfillcolor{currentfill}%
\pgfsetlinewidth{0.481800pt}%
\definecolor{currentstroke}{rgb}{1.000000,1.000000,1.000000}%
\pgfsetstrokecolor{currentstroke}%
\pgfsetdash{}{0pt}%
\pgfpathmoveto{\pgfqpoint{8.262238in}{1.069352in}}%
\pgfpathcurveto{\pgfqpoint{8.273288in}{1.069352in}}{\pgfqpoint{8.283887in}{1.073742in}}{\pgfqpoint{8.291700in}{1.081556in}}%
\pgfpathcurveto{\pgfqpoint{8.299514in}{1.089369in}}{\pgfqpoint{8.303904in}{1.099968in}}{\pgfqpoint{8.303904in}{1.111019in}}%
\pgfpathcurveto{\pgfqpoint{8.303904in}{1.122069in}}{\pgfqpoint{8.299514in}{1.132668in}}{\pgfqpoint{8.291700in}{1.140481in}}%
\pgfpathcurveto{\pgfqpoint{8.283887in}{1.148295in}}{\pgfqpoint{8.273288in}{1.152685in}}{\pgfqpoint{8.262238in}{1.152685in}}%
\pgfpathcurveto{\pgfqpoint{8.251188in}{1.152685in}}{\pgfqpoint{8.240589in}{1.148295in}}{\pgfqpoint{8.232775in}{1.140481in}}%
\pgfpathcurveto{\pgfqpoint{8.224961in}{1.132668in}}{\pgfqpoint{8.220571in}{1.122069in}}{\pgfqpoint{8.220571in}{1.111019in}}%
\pgfpathcurveto{\pgfqpoint{8.220571in}{1.099968in}}{\pgfqpoint{8.224961in}{1.089369in}}{\pgfqpoint{8.232775in}{1.081556in}}%
\pgfpathcurveto{\pgfqpoint{8.240589in}{1.073742in}}{\pgfqpoint{8.251188in}{1.069352in}}{\pgfqpoint{8.262238in}{1.069352in}}%
\pgfpathclose%
\pgfusepath{stroke,fill}%
\end{pgfscope}%
\begin{pgfscope}%
\pgfpathrectangle{\pgfqpoint{0.481978in}{0.331635in}}{\pgfqpoint{9.300000in}{7.700000in}}%
\pgfusepath{clip}%
\pgfsetbuttcap%
\pgfsetroundjoin%
\definecolor{currentfill}{rgb}{0.631373,0.788235,0.956863}%
\pgfsetfillcolor{currentfill}%
\pgfsetlinewidth{0.481800pt}%
\definecolor{currentstroke}{rgb}{1.000000,1.000000,1.000000}%
\pgfsetstrokecolor{currentstroke}%
\pgfsetdash{}{0pt}%
\pgfpathmoveto{\pgfqpoint{7.580160in}{5.017168in}}%
\pgfpathcurveto{\pgfqpoint{7.591210in}{5.017168in}}{\pgfqpoint{7.601809in}{5.021558in}}{\pgfqpoint{7.609622in}{5.029372in}}%
\pgfpathcurveto{\pgfqpoint{7.617436in}{5.037185in}}{\pgfqpoint{7.621826in}{5.047784in}}{\pgfqpoint{7.621826in}{5.058834in}}%
\pgfpathcurveto{\pgfqpoint{7.621826in}{5.069885in}}{\pgfqpoint{7.617436in}{5.080484in}}{\pgfqpoint{7.609622in}{5.088297in}}%
\pgfpathcurveto{\pgfqpoint{7.601809in}{5.096111in}}{\pgfqpoint{7.591210in}{5.100501in}}{\pgfqpoint{7.580160in}{5.100501in}}%
\pgfpathcurveto{\pgfqpoint{7.569110in}{5.100501in}}{\pgfqpoint{7.558511in}{5.096111in}}{\pgfqpoint{7.550697in}{5.088297in}}%
\pgfpathcurveto{\pgfqpoint{7.542883in}{5.080484in}}{\pgfqpoint{7.538493in}{5.069885in}}{\pgfqpoint{7.538493in}{5.058834in}}%
\pgfpathcurveto{\pgfqpoint{7.538493in}{5.047784in}}{\pgfqpoint{7.542883in}{5.037185in}}{\pgfqpoint{7.550697in}{5.029372in}}%
\pgfpathcurveto{\pgfqpoint{7.558511in}{5.021558in}}{\pgfqpoint{7.569110in}{5.017168in}}{\pgfqpoint{7.580160in}{5.017168in}}%
\pgfpathclose%
\pgfusepath{stroke,fill}%
\end{pgfscope}%
\begin{pgfscope}%
\pgfpathrectangle{\pgfqpoint{0.481978in}{0.331635in}}{\pgfqpoint{9.300000in}{7.700000in}}%
\pgfusepath{clip}%
\pgfsetbuttcap%
\pgfsetroundjoin%
\definecolor{currentfill}{rgb}{0.631373,0.788235,0.956863}%
\pgfsetfillcolor{currentfill}%
\pgfsetlinewidth{0.481800pt}%
\definecolor{currentstroke}{rgb}{1.000000,1.000000,1.000000}%
\pgfsetstrokecolor{currentstroke}%
\pgfsetdash{}{0pt}%
\pgfpathmoveto{\pgfqpoint{7.496625in}{2.019733in}}%
\pgfpathcurveto{\pgfqpoint{7.507675in}{2.019733in}}{\pgfqpoint{7.518274in}{2.024123in}}{\pgfqpoint{7.526088in}{2.031937in}}%
\pgfpathcurveto{\pgfqpoint{7.533901in}{2.039751in}}{\pgfqpoint{7.538292in}{2.050350in}}{\pgfqpoint{7.538292in}{2.061400in}}%
\pgfpathcurveto{\pgfqpoint{7.538292in}{2.072450in}}{\pgfqpoint{7.533901in}{2.083049in}}{\pgfqpoint{7.526088in}{2.090862in}}%
\pgfpathcurveto{\pgfqpoint{7.518274in}{2.098676in}}{\pgfqpoint{7.507675in}{2.103066in}}{\pgfqpoint{7.496625in}{2.103066in}}%
\pgfpathcurveto{\pgfqpoint{7.485575in}{2.103066in}}{\pgfqpoint{7.474976in}{2.098676in}}{\pgfqpoint{7.467162in}{2.090862in}}%
\pgfpathcurveto{\pgfqpoint{7.459349in}{2.083049in}}{\pgfqpoint{7.454958in}{2.072450in}}{\pgfqpoint{7.454958in}{2.061400in}}%
\pgfpathcurveto{\pgfqpoint{7.454958in}{2.050350in}}{\pgfqpoint{7.459349in}{2.039751in}}{\pgfqpoint{7.467162in}{2.031937in}}%
\pgfpathcurveto{\pgfqpoint{7.474976in}{2.024123in}}{\pgfqpoint{7.485575in}{2.019733in}}{\pgfqpoint{7.496625in}{2.019733in}}%
\pgfpathclose%
\pgfusepath{stroke,fill}%
\end{pgfscope}%
\begin{pgfscope}%
\pgfpathrectangle{\pgfqpoint{0.481978in}{0.331635in}}{\pgfqpoint{9.300000in}{7.700000in}}%
\pgfusepath{clip}%
\pgfsetbuttcap%
\pgfsetroundjoin%
\definecolor{currentfill}{rgb}{0.631373,0.788235,0.956863}%
\pgfsetfillcolor{currentfill}%
\pgfsetlinewidth{0.481800pt}%
\definecolor{currentstroke}{rgb}{1.000000,1.000000,1.000000}%
\pgfsetstrokecolor{currentstroke}%
\pgfsetdash{}{0pt}%
\pgfpathmoveto{\pgfqpoint{3.509030in}{2.716762in}}%
\pgfpathcurveto{\pgfqpoint{3.520080in}{2.716762in}}{\pgfqpoint{3.530679in}{2.721152in}}{\pgfqpoint{3.538493in}{2.728966in}}%
\pgfpathcurveto{\pgfqpoint{3.546307in}{2.736780in}}{\pgfqpoint{3.550697in}{2.747379in}}{\pgfqpoint{3.550697in}{2.758429in}}%
\pgfpathcurveto{\pgfqpoint{3.550697in}{2.769479in}}{\pgfqpoint{3.546307in}{2.780078in}}{\pgfqpoint{3.538493in}{2.787892in}}%
\pgfpathcurveto{\pgfqpoint{3.530679in}{2.795705in}}{\pgfqpoint{3.520080in}{2.800095in}}{\pgfqpoint{3.509030in}{2.800095in}}%
\pgfpathcurveto{\pgfqpoint{3.497980in}{2.800095in}}{\pgfqpoint{3.487381in}{2.795705in}}{\pgfqpoint{3.479567in}{2.787892in}}%
\pgfpathcurveto{\pgfqpoint{3.471754in}{2.780078in}}{\pgfqpoint{3.467364in}{2.769479in}}{\pgfqpoint{3.467364in}{2.758429in}}%
\pgfpathcurveto{\pgfqpoint{3.467364in}{2.747379in}}{\pgfqpoint{3.471754in}{2.736780in}}{\pgfqpoint{3.479567in}{2.728966in}}%
\pgfpathcurveto{\pgfqpoint{3.487381in}{2.721152in}}{\pgfqpoint{3.497980in}{2.716762in}}{\pgfqpoint{3.509030in}{2.716762in}}%
\pgfpathclose%
\pgfusepath{stroke,fill}%
\end{pgfscope}%
\begin{pgfscope}%
\pgfpathrectangle{\pgfqpoint{0.481978in}{0.331635in}}{\pgfqpoint{9.300000in}{7.700000in}}%
\pgfusepath{clip}%
\pgfsetbuttcap%
\pgfsetroundjoin%
\definecolor{currentfill}{rgb}{0.631373,0.788235,0.956863}%
\pgfsetfillcolor{currentfill}%
\pgfsetlinewidth{0.481800pt}%
\definecolor{currentstroke}{rgb}{1.000000,1.000000,1.000000}%
\pgfsetstrokecolor{currentstroke}%
\pgfsetdash{}{0pt}%
\pgfpathmoveto{\pgfqpoint{6.905988in}{4.369888in}}%
\pgfpathcurveto{\pgfqpoint{6.917038in}{4.369888in}}{\pgfqpoint{6.927637in}{4.374278in}}{\pgfqpoint{6.935451in}{4.382092in}}%
\pgfpathcurveto{\pgfqpoint{6.943265in}{4.389906in}}{\pgfqpoint{6.947655in}{4.400505in}}{\pgfqpoint{6.947655in}{4.411555in}}%
\pgfpathcurveto{\pgfqpoint{6.947655in}{4.422605in}}{\pgfqpoint{6.943265in}{4.433204in}}{\pgfqpoint{6.935451in}{4.441018in}}%
\pgfpathcurveto{\pgfqpoint{6.927637in}{4.448831in}}{\pgfqpoint{6.917038in}{4.453221in}}{\pgfqpoint{6.905988in}{4.453221in}}%
\pgfpathcurveto{\pgfqpoint{6.894938in}{4.453221in}}{\pgfqpoint{6.884339in}{4.448831in}}{\pgfqpoint{6.876525in}{4.441018in}}%
\pgfpathcurveto{\pgfqpoint{6.868712in}{4.433204in}}{\pgfqpoint{6.864321in}{4.422605in}}{\pgfqpoint{6.864321in}{4.411555in}}%
\pgfpathcurveto{\pgfqpoint{6.864321in}{4.400505in}}{\pgfqpoint{6.868712in}{4.389906in}}{\pgfqpoint{6.876525in}{4.382092in}}%
\pgfpathcurveto{\pgfqpoint{6.884339in}{4.374278in}}{\pgfqpoint{6.894938in}{4.369888in}}{\pgfqpoint{6.905988in}{4.369888in}}%
\pgfpathclose%
\pgfusepath{stroke,fill}%
\end{pgfscope}%
\begin{pgfscope}%
\pgfpathrectangle{\pgfqpoint{0.481978in}{0.331635in}}{\pgfqpoint{9.300000in}{7.700000in}}%
\pgfusepath{clip}%
\pgfsetbuttcap%
\pgfsetroundjoin%
\definecolor{currentfill}{rgb}{0.631373,0.788235,0.956863}%
\pgfsetfillcolor{currentfill}%
\pgfsetlinewidth{0.481800pt}%
\definecolor{currentstroke}{rgb}{1.000000,1.000000,1.000000}%
\pgfsetstrokecolor{currentstroke}%
\pgfsetdash{}{0pt}%
\pgfpathmoveto{\pgfqpoint{2.394843in}{4.902575in}}%
\pgfpathcurveto{\pgfqpoint{2.405893in}{4.902575in}}{\pgfqpoint{2.416492in}{4.906966in}}{\pgfqpoint{2.424306in}{4.914779in}}%
\pgfpathcurveto{\pgfqpoint{2.432120in}{4.922593in}}{\pgfqpoint{2.436510in}{4.933192in}}{\pgfqpoint{2.436510in}{4.944242in}}%
\pgfpathcurveto{\pgfqpoint{2.436510in}{4.955292in}}{\pgfqpoint{2.432120in}{4.965891in}}{\pgfqpoint{2.424306in}{4.973705in}}%
\pgfpathcurveto{\pgfqpoint{2.416492in}{4.981518in}}{\pgfqpoint{2.405893in}{4.985909in}}{\pgfqpoint{2.394843in}{4.985909in}}%
\pgfpathcurveto{\pgfqpoint{2.383793in}{4.985909in}}{\pgfqpoint{2.373194in}{4.981518in}}{\pgfqpoint{2.365380in}{4.973705in}}%
\pgfpathcurveto{\pgfqpoint{2.357567in}{4.965891in}}{\pgfqpoint{2.353176in}{4.955292in}}{\pgfqpoint{2.353176in}{4.944242in}}%
\pgfpathcurveto{\pgfqpoint{2.353176in}{4.933192in}}{\pgfqpoint{2.357567in}{4.922593in}}{\pgfqpoint{2.365380in}{4.914779in}}%
\pgfpathcurveto{\pgfqpoint{2.373194in}{4.906966in}}{\pgfqpoint{2.383793in}{4.902575in}}{\pgfqpoint{2.394843in}{4.902575in}}%
\pgfpathclose%
\pgfusepath{stroke,fill}%
\end{pgfscope}%
\begin{pgfscope}%
\pgfpathrectangle{\pgfqpoint{0.481978in}{0.331635in}}{\pgfqpoint{9.300000in}{7.700000in}}%
\pgfusepath{clip}%
\pgfsetbuttcap%
\pgfsetroundjoin%
\definecolor{currentfill}{rgb}{0.631373,0.788235,0.956863}%
\pgfsetfillcolor{currentfill}%
\pgfsetlinewidth{0.481800pt}%
\definecolor{currentstroke}{rgb}{1.000000,1.000000,1.000000}%
\pgfsetstrokecolor{currentstroke}%
\pgfsetdash{}{0pt}%
\pgfpathmoveto{\pgfqpoint{4.587397in}{5.212299in}}%
\pgfpathcurveto{\pgfqpoint{4.598447in}{5.212299in}}{\pgfqpoint{4.609046in}{5.216689in}}{\pgfqpoint{4.616859in}{5.224503in}}%
\pgfpathcurveto{\pgfqpoint{4.624673in}{5.232316in}}{\pgfqpoint{4.629063in}{5.242915in}}{\pgfqpoint{4.629063in}{5.253965in}}%
\pgfpathcurveto{\pgfqpoint{4.629063in}{5.265016in}}{\pgfqpoint{4.624673in}{5.275615in}}{\pgfqpoint{4.616859in}{5.283428in}}%
\pgfpathcurveto{\pgfqpoint{4.609046in}{5.291242in}}{\pgfqpoint{4.598447in}{5.295632in}}{\pgfqpoint{4.587397in}{5.295632in}}%
\pgfpathcurveto{\pgfqpoint{4.576346in}{5.295632in}}{\pgfqpoint{4.565747in}{5.291242in}}{\pgfqpoint{4.557934in}{5.283428in}}%
\pgfpathcurveto{\pgfqpoint{4.550120in}{5.275615in}}{\pgfqpoint{4.545730in}{5.265016in}}{\pgfqpoint{4.545730in}{5.253965in}}%
\pgfpathcurveto{\pgfqpoint{4.545730in}{5.242915in}}{\pgfqpoint{4.550120in}{5.232316in}}{\pgfqpoint{4.557934in}{5.224503in}}%
\pgfpathcurveto{\pgfqpoint{4.565747in}{5.216689in}}{\pgfqpoint{4.576346in}{5.212299in}}{\pgfqpoint{4.587397in}{5.212299in}}%
\pgfpathclose%
\pgfusepath{stroke,fill}%
\end{pgfscope}%
\begin{pgfscope}%
\pgfpathrectangle{\pgfqpoint{0.481978in}{0.331635in}}{\pgfqpoint{9.300000in}{7.700000in}}%
\pgfusepath{clip}%
\pgfsetbuttcap%
\pgfsetroundjoin%
\definecolor{currentfill}{rgb}{0.631373,0.788235,0.956863}%
\pgfsetfillcolor{currentfill}%
\pgfsetlinewidth{0.481800pt}%
\definecolor{currentstroke}{rgb}{1.000000,1.000000,1.000000}%
\pgfsetstrokecolor{currentstroke}%
\pgfsetdash{}{0pt}%
\pgfpathmoveto{\pgfqpoint{5.606626in}{4.828749in}}%
\pgfpathcurveto{\pgfqpoint{5.617676in}{4.828749in}}{\pgfqpoint{5.628275in}{4.833140in}}{\pgfqpoint{5.636089in}{4.840953in}}%
\pgfpathcurveto{\pgfqpoint{5.643902in}{4.848767in}}{\pgfqpoint{5.648293in}{4.859366in}}{\pgfqpoint{5.648293in}{4.870416in}}%
\pgfpathcurveto{\pgfqpoint{5.648293in}{4.881466in}}{\pgfqpoint{5.643902in}{4.892065in}}{\pgfqpoint{5.636089in}{4.899879in}}%
\pgfpathcurveto{\pgfqpoint{5.628275in}{4.907693in}}{\pgfqpoint{5.617676in}{4.912083in}}{\pgfqpoint{5.606626in}{4.912083in}}%
\pgfpathcurveto{\pgfqpoint{5.595576in}{4.912083in}}{\pgfqpoint{5.584977in}{4.907693in}}{\pgfqpoint{5.577163in}{4.899879in}}%
\pgfpathcurveto{\pgfqpoint{5.569349in}{4.892065in}}{\pgfqpoint{5.564959in}{4.881466in}}{\pgfqpoint{5.564959in}{4.870416in}}%
\pgfpathcurveto{\pgfqpoint{5.564959in}{4.859366in}}{\pgfqpoint{5.569349in}{4.848767in}}{\pgfqpoint{5.577163in}{4.840953in}}%
\pgfpathcurveto{\pgfqpoint{5.584977in}{4.833140in}}{\pgfqpoint{5.595576in}{4.828749in}}{\pgfqpoint{5.606626in}{4.828749in}}%
\pgfpathclose%
\pgfusepath{stroke,fill}%
\end{pgfscope}%
\begin{pgfscope}%
\pgfpathrectangle{\pgfqpoint{0.481978in}{0.331635in}}{\pgfqpoint{9.300000in}{7.700000in}}%
\pgfusepath{clip}%
\pgfsetbuttcap%
\pgfsetroundjoin%
\definecolor{currentfill}{rgb}{0.631373,0.788235,0.956863}%
\pgfsetfillcolor{currentfill}%
\pgfsetlinewidth{0.481800pt}%
\definecolor{currentstroke}{rgb}{1.000000,1.000000,1.000000}%
\pgfsetstrokecolor{currentstroke}%
\pgfsetdash{}{0pt}%
\pgfpathmoveto{\pgfqpoint{5.093085in}{5.604051in}}%
\pgfpathcurveto{\pgfqpoint{5.104136in}{5.604051in}}{\pgfqpoint{5.114735in}{5.608441in}}{\pgfqpoint{5.122548in}{5.616255in}}%
\pgfpathcurveto{\pgfqpoint{5.130362in}{5.624068in}}{\pgfqpoint{5.134752in}{5.634667in}}{\pgfqpoint{5.134752in}{5.645718in}}%
\pgfpathcurveto{\pgfqpoint{5.134752in}{5.656768in}}{\pgfqpoint{5.130362in}{5.667367in}}{\pgfqpoint{5.122548in}{5.675180in}}%
\pgfpathcurveto{\pgfqpoint{5.114735in}{5.682994in}}{\pgfqpoint{5.104136in}{5.687384in}}{\pgfqpoint{5.093085in}{5.687384in}}%
\pgfpathcurveto{\pgfqpoint{5.082035in}{5.687384in}}{\pgfqpoint{5.071436in}{5.682994in}}{\pgfqpoint{5.063623in}{5.675180in}}%
\pgfpathcurveto{\pgfqpoint{5.055809in}{5.667367in}}{\pgfqpoint{5.051419in}{5.656768in}}{\pgfqpoint{5.051419in}{5.645718in}}%
\pgfpathcurveto{\pgfqpoint{5.051419in}{5.634667in}}{\pgfqpoint{5.055809in}{5.624068in}}{\pgfqpoint{5.063623in}{5.616255in}}%
\pgfpathcurveto{\pgfqpoint{5.071436in}{5.608441in}}{\pgfqpoint{5.082035in}{5.604051in}}{\pgfqpoint{5.093085in}{5.604051in}}%
\pgfpathclose%
\pgfusepath{stroke,fill}%
\end{pgfscope}%
\begin{pgfscope}%
\pgfpathrectangle{\pgfqpoint{0.481978in}{0.331635in}}{\pgfqpoint{9.300000in}{7.700000in}}%
\pgfusepath{clip}%
\pgfsetbuttcap%
\pgfsetroundjoin%
\definecolor{currentfill}{rgb}{0.631373,0.788235,0.956863}%
\pgfsetfillcolor{currentfill}%
\pgfsetlinewidth{0.481800pt}%
\definecolor{currentstroke}{rgb}{1.000000,1.000000,1.000000}%
\pgfsetstrokecolor{currentstroke}%
\pgfsetdash{}{0pt}%
\pgfpathmoveto{\pgfqpoint{7.877868in}{4.769526in}}%
\pgfpathcurveto{\pgfqpoint{7.888918in}{4.769526in}}{\pgfqpoint{7.899517in}{4.773916in}}{\pgfqpoint{7.907331in}{4.781730in}}%
\pgfpathcurveto{\pgfqpoint{7.915144in}{4.789543in}}{\pgfqpoint{7.919535in}{4.800143in}}{\pgfqpoint{7.919535in}{4.811193in}}%
\pgfpathcurveto{\pgfqpoint{7.919535in}{4.822243in}}{\pgfqpoint{7.915144in}{4.832842in}}{\pgfqpoint{7.907331in}{4.840655in}}%
\pgfpathcurveto{\pgfqpoint{7.899517in}{4.848469in}}{\pgfqpoint{7.888918in}{4.852859in}}{\pgfqpoint{7.877868in}{4.852859in}}%
\pgfpathcurveto{\pgfqpoint{7.866818in}{4.852859in}}{\pgfqpoint{7.856219in}{4.848469in}}{\pgfqpoint{7.848405in}{4.840655in}}%
\pgfpathcurveto{\pgfqpoint{7.840591in}{4.832842in}}{\pgfqpoint{7.836201in}{4.822243in}}{\pgfqpoint{7.836201in}{4.811193in}}%
\pgfpathcurveto{\pgfqpoint{7.836201in}{4.800143in}}{\pgfqpoint{7.840591in}{4.789543in}}{\pgfqpoint{7.848405in}{4.781730in}}%
\pgfpathcurveto{\pgfqpoint{7.856219in}{4.773916in}}{\pgfqpoint{7.866818in}{4.769526in}}{\pgfqpoint{7.877868in}{4.769526in}}%
\pgfpathclose%
\pgfusepath{stroke,fill}%
\end{pgfscope}%
\begin{pgfscope}%
\pgfpathrectangle{\pgfqpoint{0.481978in}{0.331635in}}{\pgfqpoint{9.300000in}{7.700000in}}%
\pgfusepath{clip}%
\pgfsetbuttcap%
\pgfsetroundjoin%
\definecolor{currentfill}{rgb}{0.631373,0.788235,0.956863}%
\pgfsetfillcolor{currentfill}%
\pgfsetlinewidth{0.481800pt}%
\definecolor{currentstroke}{rgb}{1.000000,1.000000,1.000000}%
\pgfsetstrokecolor{currentstroke}%
\pgfsetdash{}{0pt}%
\pgfpathmoveto{\pgfqpoint{2.375730in}{2.359798in}}%
\pgfpathcurveto{\pgfqpoint{2.386780in}{2.359798in}}{\pgfqpoint{2.397379in}{2.364188in}}{\pgfqpoint{2.405192in}{2.372002in}}%
\pgfpathcurveto{\pgfqpoint{2.413006in}{2.379815in}}{\pgfqpoint{2.417396in}{2.390414in}}{\pgfqpoint{2.417396in}{2.401464in}}%
\pgfpathcurveto{\pgfqpoint{2.417396in}{2.412514in}}{\pgfqpoint{2.413006in}{2.423114in}}{\pgfqpoint{2.405192in}{2.430927in}}%
\pgfpathcurveto{\pgfqpoint{2.397379in}{2.438741in}}{\pgfqpoint{2.386780in}{2.443131in}}{\pgfqpoint{2.375730in}{2.443131in}}%
\pgfpathcurveto{\pgfqpoint{2.364680in}{2.443131in}}{\pgfqpoint{2.354081in}{2.438741in}}{\pgfqpoint{2.346267in}{2.430927in}}%
\pgfpathcurveto{\pgfqpoint{2.338453in}{2.423114in}}{\pgfqpoint{2.334063in}{2.412514in}}{\pgfqpoint{2.334063in}{2.401464in}}%
\pgfpathcurveto{\pgfqpoint{2.334063in}{2.390414in}}{\pgfqpoint{2.338453in}{2.379815in}}{\pgfqpoint{2.346267in}{2.372002in}}%
\pgfpathcurveto{\pgfqpoint{2.354081in}{2.364188in}}{\pgfqpoint{2.364680in}{2.359798in}}{\pgfqpoint{2.375730in}{2.359798in}}%
\pgfpathclose%
\pgfusepath{stroke,fill}%
\end{pgfscope}%
\begin{pgfscope}%
\pgfpathrectangle{\pgfqpoint{0.481978in}{0.331635in}}{\pgfqpoint{9.300000in}{7.700000in}}%
\pgfusepath{clip}%
\pgfsetbuttcap%
\pgfsetroundjoin%
\definecolor{currentfill}{rgb}{0.631373,0.788235,0.956863}%
\pgfsetfillcolor{currentfill}%
\pgfsetlinewidth{0.481800pt}%
\definecolor{currentstroke}{rgb}{1.000000,1.000000,1.000000}%
\pgfsetstrokecolor{currentstroke}%
\pgfsetdash{}{0pt}%
\pgfpathmoveto{\pgfqpoint{9.359251in}{4.500176in}}%
\pgfpathcurveto{\pgfqpoint{9.370301in}{4.500176in}}{\pgfqpoint{9.380900in}{4.504566in}}{\pgfqpoint{9.388713in}{4.512380in}}%
\pgfpathcurveto{\pgfqpoint{9.396527in}{4.520193in}}{\pgfqpoint{9.400917in}{4.530792in}}{\pgfqpoint{9.400917in}{4.541842in}}%
\pgfpathcurveto{\pgfqpoint{9.400917in}{4.552893in}}{\pgfqpoint{9.396527in}{4.563492in}}{\pgfqpoint{9.388713in}{4.571305in}}%
\pgfpathcurveto{\pgfqpoint{9.380900in}{4.579119in}}{\pgfqpoint{9.370301in}{4.583509in}}{\pgfqpoint{9.359251in}{4.583509in}}%
\pgfpathcurveto{\pgfqpoint{9.348201in}{4.583509in}}{\pgfqpoint{9.337601in}{4.579119in}}{\pgfqpoint{9.329788in}{4.571305in}}%
\pgfpathcurveto{\pgfqpoint{9.321974in}{4.563492in}}{\pgfqpoint{9.317584in}{4.552893in}}{\pgfqpoint{9.317584in}{4.541842in}}%
\pgfpathcurveto{\pgfqpoint{9.317584in}{4.530792in}}{\pgfqpoint{9.321974in}{4.520193in}}{\pgfqpoint{9.329788in}{4.512380in}}%
\pgfpathcurveto{\pgfqpoint{9.337601in}{4.504566in}}{\pgfqpoint{9.348201in}{4.500176in}}{\pgfqpoint{9.359251in}{4.500176in}}%
\pgfpathclose%
\pgfusepath{stroke,fill}%
\end{pgfscope}%
\begin{pgfscope}%
\pgfpathrectangle{\pgfqpoint{0.481978in}{0.331635in}}{\pgfqpoint{9.300000in}{7.700000in}}%
\pgfusepath{clip}%
\pgfsetbuttcap%
\pgfsetroundjoin%
\definecolor{currentfill}{rgb}{0.631373,0.788235,0.956863}%
\pgfsetfillcolor{currentfill}%
\pgfsetlinewidth{0.481800pt}%
\definecolor{currentstroke}{rgb}{1.000000,1.000000,1.000000}%
\pgfsetstrokecolor{currentstroke}%
\pgfsetdash{}{0pt}%
\pgfpathmoveto{\pgfqpoint{6.013955in}{5.344683in}}%
\pgfpathcurveto{\pgfqpoint{6.025005in}{5.344683in}}{\pgfqpoint{6.035604in}{5.349074in}}{\pgfqpoint{6.043417in}{5.356887in}}%
\pgfpathcurveto{\pgfqpoint{6.051231in}{5.364701in}}{\pgfqpoint{6.055621in}{5.375300in}}{\pgfqpoint{6.055621in}{5.386350in}}%
\pgfpathcurveto{\pgfqpoint{6.055621in}{5.397400in}}{\pgfqpoint{6.051231in}{5.407999in}}{\pgfqpoint{6.043417in}{5.415813in}}%
\pgfpathcurveto{\pgfqpoint{6.035604in}{5.423626in}}{\pgfqpoint{6.025005in}{5.428017in}}{\pgfqpoint{6.013955in}{5.428017in}}%
\pgfpathcurveto{\pgfqpoint{6.002904in}{5.428017in}}{\pgfqpoint{5.992305in}{5.423626in}}{\pgfqpoint{5.984492in}{5.415813in}}%
\pgfpathcurveto{\pgfqpoint{5.976678in}{5.407999in}}{\pgfqpoint{5.972288in}{5.397400in}}{\pgfqpoint{5.972288in}{5.386350in}}%
\pgfpathcurveto{\pgfqpoint{5.972288in}{5.375300in}}{\pgfqpoint{5.976678in}{5.364701in}}{\pgfqpoint{5.984492in}{5.356887in}}%
\pgfpathcurveto{\pgfqpoint{5.992305in}{5.349074in}}{\pgfqpoint{6.002904in}{5.344683in}}{\pgfqpoint{6.013955in}{5.344683in}}%
\pgfpathclose%
\pgfusepath{stroke,fill}%
\end{pgfscope}%
\begin{pgfscope}%
\pgfpathrectangle{\pgfqpoint{0.481978in}{0.331635in}}{\pgfqpoint{9.300000in}{7.700000in}}%
\pgfusepath{clip}%
\pgfsetbuttcap%
\pgfsetroundjoin%
\definecolor{currentfill}{rgb}{0.631373,0.788235,0.956863}%
\pgfsetfillcolor{currentfill}%
\pgfsetlinewidth{0.481800pt}%
\definecolor{currentstroke}{rgb}{1.000000,1.000000,1.000000}%
\pgfsetstrokecolor{currentstroke}%
\pgfsetdash{}{0pt}%
\pgfpathmoveto{\pgfqpoint{3.830222in}{6.778917in}}%
\pgfpathcurveto{\pgfqpoint{3.841272in}{6.778917in}}{\pgfqpoint{3.851871in}{6.783308in}}{\pgfqpoint{3.859685in}{6.791121in}}%
\pgfpathcurveto{\pgfqpoint{3.867499in}{6.798935in}}{\pgfqpoint{3.871889in}{6.809534in}}{\pgfqpoint{3.871889in}{6.820584in}}%
\pgfpathcurveto{\pgfqpoint{3.871889in}{6.831634in}}{\pgfqpoint{3.867499in}{6.842233in}}{\pgfqpoint{3.859685in}{6.850047in}}%
\pgfpathcurveto{\pgfqpoint{3.851871in}{6.857860in}}{\pgfqpoint{3.841272in}{6.862251in}}{\pgfqpoint{3.830222in}{6.862251in}}%
\pgfpathcurveto{\pgfqpoint{3.819172in}{6.862251in}}{\pgfqpoint{3.808573in}{6.857860in}}{\pgfqpoint{3.800759in}{6.850047in}}%
\pgfpathcurveto{\pgfqpoint{3.792946in}{6.842233in}}{\pgfqpoint{3.788556in}{6.831634in}}{\pgfqpoint{3.788556in}{6.820584in}}%
\pgfpathcurveto{\pgfqpoint{3.788556in}{6.809534in}}{\pgfqpoint{3.792946in}{6.798935in}}{\pgfqpoint{3.800759in}{6.791121in}}%
\pgfpathcurveto{\pgfqpoint{3.808573in}{6.783308in}}{\pgfqpoint{3.819172in}{6.778917in}}{\pgfqpoint{3.830222in}{6.778917in}}%
\pgfpathclose%
\pgfusepath{stroke,fill}%
\end{pgfscope}%
\begin{pgfscope}%
\pgfpathrectangle{\pgfqpoint{0.481978in}{0.331635in}}{\pgfqpoint{9.300000in}{7.700000in}}%
\pgfusepath{clip}%
\pgfsetbuttcap%
\pgfsetroundjoin%
\definecolor{currentfill}{rgb}{0.631373,0.788235,0.956863}%
\pgfsetfillcolor{currentfill}%
\pgfsetlinewidth{0.481800pt}%
\definecolor{currentstroke}{rgb}{1.000000,1.000000,1.000000}%
\pgfsetstrokecolor{currentstroke}%
\pgfsetdash{}{0pt}%
\pgfpathmoveto{\pgfqpoint{8.786314in}{1.252936in}}%
\pgfpathcurveto{\pgfqpoint{8.797364in}{1.252936in}}{\pgfqpoint{8.807963in}{1.257326in}}{\pgfqpoint{8.815777in}{1.265140in}}%
\pgfpathcurveto{\pgfqpoint{8.823591in}{1.272953in}}{\pgfqpoint{8.827981in}{1.283552in}}{\pgfqpoint{8.827981in}{1.294603in}}%
\pgfpathcurveto{\pgfqpoint{8.827981in}{1.305653in}}{\pgfqpoint{8.823591in}{1.316252in}}{\pgfqpoint{8.815777in}{1.324065in}}%
\pgfpathcurveto{\pgfqpoint{8.807963in}{1.331879in}}{\pgfqpoint{8.797364in}{1.336269in}}{\pgfqpoint{8.786314in}{1.336269in}}%
\pgfpathcurveto{\pgfqpoint{8.775264in}{1.336269in}}{\pgfqpoint{8.764665in}{1.331879in}}{\pgfqpoint{8.756851in}{1.324065in}}%
\pgfpathcurveto{\pgfqpoint{8.749038in}{1.316252in}}{\pgfqpoint{8.744648in}{1.305653in}}{\pgfqpoint{8.744648in}{1.294603in}}%
\pgfpathcurveto{\pgfqpoint{8.744648in}{1.283552in}}{\pgfqpoint{8.749038in}{1.272953in}}{\pgfqpoint{8.756851in}{1.265140in}}%
\pgfpathcurveto{\pgfqpoint{8.764665in}{1.257326in}}{\pgfqpoint{8.775264in}{1.252936in}}{\pgfqpoint{8.786314in}{1.252936in}}%
\pgfpathclose%
\pgfusepath{stroke,fill}%
\end{pgfscope}%
\begin{pgfscope}%
\pgfpathrectangle{\pgfqpoint{0.481978in}{0.331635in}}{\pgfqpoint{9.300000in}{7.700000in}}%
\pgfusepath{clip}%
\pgfsetbuttcap%
\pgfsetroundjoin%
\definecolor{currentfill}{rgb}{0.631373,0.788235,0.956863}%
\pgfsetfillcolor{currentfill}%
\pgfsetlinewidth{0.481800pt}%
\definecolor{currentstroke}{rgb}{1.000000,1.000000,1.000000}%
\pgfsetstrokecolor{currentstroke}%
\pgfsetdash{}{0pt}%
\pgfpathmoveto{\pgfqpoint{1.235540in}{2.404404in}}%
\pgfpathcurveto{\pgfqpoint{1.246590in}{2.404404in}}{\pgfqpoint{1.257190in}{2.408794in}}{\pgfqpoint{1.265003in}{2.416608in}}%
\pgfpathcurveto{\pgfqpoint{1.272817in}{2.424422in}}{\pgfqpoint{1.277207in}{2.435021in}}{\pgfqpoint{1.277207in}{2.446071in}}%
\pgfpathcurveto{\pgfqpoint{1.277207in}{2.457121in}}{\pgfqpoint{1.272817in}{2.467720in}}{\pgfqpoint{1.265003in}{2.475534in}}%
\pgfpathcurveto{\pgfqpoint{1.257190in}{2.483347in}}{\pgfqpoint{1.246590in}{2.487737in}}{\pgfqpoint{1.235540in}{2.487737in}}%
\pgfpathcurveto{\pgfqpoint{1.224490in}{2.487737in}}{\pgfqpoint{1.213891in}{2.483347in}}{\pgfqpoint{1.206078in}{2.475534in}}%
\pgfpathcurveto{\pgfqpoint{1.198264in}{2.467720in}}{\pgfqpoint{1.193874in}{2.457121in}}{\pgfqpoint{1.193874in}{2.446071in}}%
\pgfpathcurveto{\pgfqpoint{1.193874in}{2.435021in}}{\pgfqpoint{1.198264in}{2.424422in}}{\pgfqpoint{1.206078in}{2.416608in}}%
\pgfpathcurveto{\pgfqpoint{1.213891in}{2.408794in}}{\pgfqpoint{1.224490in}{2.404404in}}{\pgfqpoint{1.235540in}{2.404404in}}%
\pgfpathclose%
\pgfusepath{stroke,fill}%
\end{pgfscope}%
\begin{pgfscope}%
\pgfpathrectangle{\pgfqpoint{0.481978in}{0.331635in}}{\pgfqpoint{9.300000in}{7.700000in}}%
\pgfusepath{clip}%
\pgfsetbuttcap%
\pgfsetroundjoin%
\definecolor{currentfill}{rgb}{0.631373,0.788235,0.956863}%
\pgfsetfillcolor{currentfill}%
\pgfsetlinewidth{0.481800pt}%
\definecolor{currentstroke}{rgb}{1.000000,1.000000,1.000000}%
\pgfsetstrokecolor{currentstroke}%
\pgfsetdash{}{0pt}%
\pgfpathmoveto{\pgfqpoint{2.423803in}{3.142163in}}%
\pgfpathcurveto{\pgfqpoint{2.434853in}{3.142163in}}{\pgfqpoint{2.445452in}{3.146553in}}{\pgfqpoint{2.453265in}{3.154366in}}%
\pgfpathcurveto{\pgfqpoint{2.461079in}{3.162180in}}{\pgfqpoint{2.465469in}{3.172779in}}{\pgfqpoint{2.465469in}{3.183829in}}%
\pgfpathcurveto{\pgfqpoint{2.465469in}{3.194879in}}{\pgfqpoint{2.461079in}{3.205478in}}{\pgfqpoint{2.453265in}{3.213292in}}%
\pgfpathcurveto{\pgfqpoint{2.445452in}{3.221106in}}{\pgfqpoint{2.434853in}{3.225496in}}{\pgfqpoint{2.423803in}{3.225496in}}%
\pgfpathcurveto{\pgfqpoint{2.412752in}{3.225496in}}{\pgfqpoint{2.402153in}{3.221106in}}{\pgfqpoint{2.394340in}{3.213292in}}%
\pgfpathcurveto{\pgfqpoint{2.386526in}{3.205478in}}{\pgfqpoint{2.382136in}{3.194879in}}{\pgfqpoint{2.382136in}{3.183829in}}%
\pgfpathcurveto{\pgfqpoint{2.382136in}{3.172779in}}{\pgfqpoint{2.386526in}{3.162180in}}{\pgfqpoint{2.394340in}{3.154366in}}%
\pgfpathcurveto{\pgfqpoint{2.402153in}{3.146553in}}{\pgfqpoint{2.412752in}{3.142163in}}{\pgfqpoint{2.423803in}{3.142163in}}%
\pgfpathclose%
\pgfusepath{stroke,fill}%
\end{pgfscope}%
\begin{pgfscope}%
\pgfpathrectangle{\pgfqpoint{0.481978in}{0.331635in}}{\pgfqpoint{9.300000in}{7.700000in}}%
\pgfusepath{clip}%
\pgfsetbuttcap%
\pgfsetroundjoin%
\definecolor{currentfill}{rgb}{0.631373,0.788235,0.956863}%
\pgfsetfillcolor{currentfill}%
\pgfsetlinewidth{0.481800pt}%
\definecolor{currentstroke}{rgb}{1.000000,1.000000,1.000000}%
\pgfsetstrokecolor{currentstroke}%
\pgfsetdash{}{0pt}%
\pgfpathmoveto{\pgfqpoint{3.099413in}{4.257579in}}%
\pgfpathcurveto{\pgfqpoint{3.110463in}{4.257579in}}{\pgfqpoint{3.121062in}{4.261969in}}{\pgfqpoint{3.128876in}{4.269782in}}%
\pgfpathcurveto{\pgfqpoint{3.136690in}{4.277596in}}{\pgfqpoint{3.141080in}{4.288195in}}{\pgfqpoint{3.141080in}{4.299245in}}%
\pgfpathcurveto{\pgfqpoint{3.141080in}{4.310295in}}{\pgfqpoint{3.136690in}{4.320894in}}{\pgfqpoint{3.128876in}{4.328708in}}%
\pgfpathcurveto{\pgfqpoint{3.121062in}{4.336522in}}{\pgfqpoint{3.110463in}{4.340912in}}{\pgfqpoint{3.099413in}{4.340912in}}%
\pgfpathcurveto{\pgfqpoint{3.088363in}{4.340912in}}{\pgfqpoint{3.077764in}{4.336522in}}{\pgfqpoint{3.069950in}{4.328708in}}%
\pgfpathcurveto{\pgfqpoint{3.062137in}{4.320894in}}{\pgfqpoint{3.057746in}{4.310295in}}{\pgfqpoint{3.057746in}{4.299245in}}%
\pgfpathcurveto{\pgfqpoint{3.057746in}{4.288195in}}{\pgfqpoint{3.062137in}{4.277596in}}{\pgfqpoint{3.069950in}{4.269782in}}%
\pgfpathcurveto{\pgfqpoint{3.077764in}{4.261969in}}{\pgfqpoint{3.088363in}{4.257579in}}{\pgfqpoint{3.099413in}{4.257579in}}%
\pgfpathclose%
\pgfusepath{stroke,fill}%
\end{pgfscope}%
\begin{pgfscope}%
\pgfpathrectangle{\pgfqpoint{0.481978in}{0.331635in}}{\pgfqpoint{9.300000in}{7.700000in}}%
\pgfusepath{clip}%
\pgfsetbuttcap%
\pgfsetroundjoin%
\definecolor{currentfill}{rgb}{0.631373,0.788235,0.956863}%
\pgfsetfillcolor{currentfill}%
\pgfsetlinewidth{0.481800pt}%
\definecolor{currentstroke}{rgb}{1.000000,1.000000,1.000000}%
\pgfsetstrokecolor{currentstroke}%
\pgfsetdash{}{0pt}%
\pgfpathmoveto{\pgfqpoint{3.438982in}{4.455762in}}%
\pgfpathcurveto{\pgfqpoint{3.450032in}{4.455762in}}{\pgfqpoint{3.460631in}{4.460152in}}{\pgfqpoint{3.468445in}{4.467966in}}%
\pgfpathcurveto{\pgfqpoint{3.476258in}{4.475780in}}{\pgfqpoint{3.480648in}{4.486379in}}{\pgfqpoint{3.480648in}{4.497429in}}%
\pgfpathcurveto{\pgfqpoint{3.480648in}{4.508479in}}{\pgfqpoint{3.476258in}{4.519078in}}{\pgfqpoint{3.468445in}{4.526892in}}%
\pgfpathcurveto{\pgfqpoint{3.460631in}{4.534705in}}{\pgfqpoint{3.450032in}{4.539096in}}{\pgfqpoint{3.438982in}{4.539096in}}%
\pgfpathcurveto{\pgfqpoint{3.427932in}{4.539096in}}{\pgfqpoint{3.417333in}{4.534705in}}{\pgfqpoint{3.409519in}{4.526892in}}%
\pgfpathcurveto{\pgfqpoint{3.401705in}{4.519078in}}{\pgfqpoint{3.397315in}{4.508479in}}{\pgfqpoint{3.397315in}{4.497429in}}%
\pgfpathcurveto{\pgfqpoint{3.397315in}{4.486379in}}{\pgfqpoint{3.401705in}{4.475780in}}{\pgfqpoint{3.409519in}{4.467966in}}%
\pgfpathcurveto{\pgfqpoint{3.417333in}{4.460152in}}{\pgfqpoint{3.427932in}{4.455762in}}{\pgfqpoint{3.438982in}{4.455762in}}%
\pgfpathclose%
\pgfusepath{stroke,fill}%
\end{pgfscope}%
\begin{pgfscope}%
\pgfpathrectangle{\pgfqpoint{0.481978in}{0.331635in}}{\pgfqpoint{9.300000in}{7.700000in}}%
\pgfusepath{clip}%
\pgfsetbuttcap%
\pgfsetroundjoin%
\definecolor{currentfill}{rgb}{0.631373,0.788235,0.956863}%
\pgfsetfillcolor{currentfill}%
\pgfsetlinewidth{0.481800pt}%
\definecolor{currentstroke}{rgb}{1.000000,1.000000,1.000000}%
\pgfsetstrokecolor{currentstroke}%
\pgfsetdash{}{0pt}%
\pgfpathmoveto{\pgfqpoint{6.239241in}{3.949736in}}%
\pgfpathcurveto{\pgfqpoint{6.250292in}{3.949736in}}{\pgfqpoint{6.260891in}{3.954126in}}{\pgfqpoint{6.268704in}{3.961940in}}%
\pgfpathcurveto{\pgfqpoint{6.276518in}{3.969754in}}{\pgfqpoint{6.280908in}{3.980353in}}{\pgfqpoint{6.280908in}{3.991403in}}%
\pgfpathcurveto{\pgfqpoint{6.280908in}{4.002453in}}{\pgfqpoint{6.276518in}{4.013052in}}{\pgfqpoint{6.268704in}{4.020866in}}%
\pgfpathcurveto{\pgfqpoint{6.260891in}{4.028679in}}{\pgfqpoint{6.250292in}{4.033069in}}{\pgfqpoint{6.239241in}{4.033069in}}%
\pgfpathcurveto{\pgfqpoint{6.228191in}{4.033069in}}{\pgfqpoint{6.217592in}{4.028679in}}{\pgfqpoint{6.209779in}{4.020866in}}%
\pgfpathcurveto{\pgfqpoint{6.201965in}{4.013052in}}{\pgfqpoint{6.197575in}{4.002453in}}{\pgfqpoint{6.197575in}{3.991403in}}%
\pgfpathcurveto{\pgfqpoint{6.197575in}{3.980353in}}{\pgfqpoint{6.201965in}{3.969754in}}{\pgfqpoint{6.209779in}{3.961940in}}%
\pgfpathcurveto{\pgfqpoint{6.217592in}{3.954126in}}{\pgfqpoint{6.228191in}{3.949736in}}{\pgfqpoint{6.239241in}{3.949736in}}%
\pgfpathclose%
\pgfusepath{stroke,fill}%
\end{pgfscope}%
\begin{pgfscope}%
\pgfpathrectangle{\pgfqpoint{0.481978in}{0.331635in}}{\pgfqpoint{9.300000in}{7.700000in}}%
\pgfusepath{clip}%
\pgfsetbuttcap%
\pgfsetroundjoin%
\definecolor{currentfill}{rgb}{0.631373,0.788235,0.956863}%
\pgfsetfillcolor{currentfill}%
\pgfsetlinewidth{0.481800pt}%
\definecolor{currentstroke}{rgb}{1.000000,1.000000,1.000000}%
\pgfsetstrokecolor{currentstroke}%
\pgfsetdash{}{0pt}%
\pgfpathmoveto{\pgfqpoint{7.439054in}{1.158266in}}%
\pgfpathcurveto{\pgfqpoint{7.450104in}{1.158266in}}{\pgfqpoint{7.460703in}{1.162656in}}{\pgfqpoint{7.468516in}{1.170470in}}%
\pgfpathcurveto{\pgfqpoint{7.476330in}{1.178284in}}{\pgfqpoint{7.480720in}{1.188883in}}{\pgfqpoint{7.480720in}{1.199933in}}%
\pgfpathcurveto{\pgfqpoint{7.480720in}{1.210983in}}{\pgfqpoint{7.476330in}{1.221582in}}{\pgfqpoint{7.468516in}{1.229396in}}%
\pgfpathcurveto{\pgfqpoint{7.460703in}{1.237209in}}{\pgfqpoint{7.450104in}{1.241599in}}{\pgfqpoint{7.439054in}{1.241599in}}%
\pgfpathcurveto{\pgfqpoint{7.428003in}{1.241599in}}{\pgfqpoint{7.417404in}{1.237209in}}{\pgfqpoint{7.409591in}{1.229396in}}%
\pgfpathcurveto{\pgfqpoint{7.401777in}{1.221582in}}{\pgfqpoint{7.397387in}{1.210983in}}{\pgfqpoint{7.397387in}{1.199933in}}%
\pgfpathcurveto{\pgfqpoint{7.397387in}{1.188883in}}{\pgfqpoint{7.401777in}{1.178284in}}{\pgfqpoint{7.409591in}{1.170470in}}%
\pgfpathcurveto{\pgfqpoint{7.417404in}{1.162656in}}{\pgfqpoint{7.428003in}{1.158266in}}{\pgfqpoint{7.439054in}{1.158266in}}%
\pgfpathclose%
\pgfusepath{stroke,fill}%
\end{pgfscope}%
\begin{pgfscope}%
\pgfpathrectangle{\pgfqpoint{0.481978in}{0.331635in}}{\pgfqpoint{9.300000in}{7.700000in}}%
\pgfusepath{clip}%
\pgfsetbuttcap%
\pgfsetroundjoin%
\definecolor{currentfill}{rgb}{0.631373,0.788235,0.956863}%
\pgfsetfillcolor{currentfill}%
\pgfsetlinewidth{0.481800pt}%
\definecolor{currentstroke}{rgb}{1.000000,1.000000,1.000000}%
\pgfsetstrokecolor{currentstroke}%
\pgfsetdash{}{0pt}%
\pgfpathmoveto{\pgfqpoint{7.510101in}{1.642964in}}%
\pgfpathcurveto{\pgfqpoint{7.521151in}{1.642964in}}{\pgfqpoint{7.531750in}{1.647355in}}{\pgfqpoint{7.539564in}{1.655168in}}%
\pgfpathcurveto{\pgfqpoint{7.547377in}{1.662982in}}{\pgfqpoint{7.551768in}{1.673581in}}{\pgfqpoint{7.551768in}{1.684631in}}%
\pgfpathcurveto{\pgfqpoint{7.551768in}{1.695681in}}{\pgfqpoint{7.547377in}{1.706280in}}{\pgfqpoint{7.539564in}{1.714094in}}%
\pgfpathcurveto{\pgfqpoint{7.531750in}{1.721907in}}{\pgfqpoint{7.521151in}{1.726298in}}{\pgfqpoint{7.510101in}{1.726298in}}%
\pgfpathcurveto{\pgfqpoint{7.499051in}{1.726298in}}{\pgfqpoint{7.488452in}{1.721907in}}{\pgfqpoint{7.480638in}{1.714094in}}%
\pgfpathcurveto{\pgfqpoint{7.472824in}{1.706280in}}{\pgfqpoint{7.468434in}{1.695681in}}{\pgfqpoint{7.468434in}{1.684631in}}%
\pgfpathcurveto{\pgfqpoint{7.468434in}{1.673581in}}{\pgfqpoint{7.472824in}{1.662982in}}{\pgfqpoint{7.480638in}{1.655168in}}%
\pgfpathcurveto{\pgfqpoint{7.488452in}{1.647355in}}{\pgfqpoint{7.499051in}{1.642964in}}{\pgfqpoint{7.510101in}{1.642964in}}%
\pgfpathclose%
\pgfusepath{stroke,fill}%
\end{pgfscope}%
\begin{pgfscope}%
\pgfpathrectangle{\pgfqpoint{0.481978in}{0.331635in}}{\pgfqpoint{9.300000in}{7.700000in}}%
\pgfusepath{clip}%
\pgfsetbuttcap%
\pgfsetroundjoin%
\definecolor{currentfill}{rgb}{0.631373,0.788235,0.956863}%
\pgfsetfillcolor{currentfill}%
\pgfsetlinewidth{0.481800pt}%
\definecolor{currentstroke}{rgb}{1.000000,1.000000,1.000000}%
\pgfsetstrokecolor{currentstroke}%
\pgfsetdash{}{0pt}%
\pgfpathmoveto{\pgfqpoint{2.315914in}{4.192320in}}%
\pgfpathcurveto{\pgfqpoint{2.326964in}{4.192320in}}{\pgfqpoint{2.337563in}{4.196710in}}{\pgfqpoint{2.345376in}{4.204524in}}%
\pgfpathcurveto{\pgfqpoint{2.353190in}{4.212337in}}{\pgfqpoint{2.357580in}{4.222936in}}{\pgfqpoint{2.357580in}{4.233986in}}%
\pgfpathcurveto{\pgfqpoint{2.357580in}{4.245036in}}{\pgfqpoint{2.353190in}{4.255635in}}{\pgfqpoint{2.345376in}{4.263449in}}%
\pgfpathcurveto{\pgfqpoint{2.337563in}{4.271263in}}{\pgfqpoint{2.326964in}{4.275653in}}{\pgfqpoint{2.315914in}{4.275653in}}%
\pgfpathcurveto{\pgfqpoint{2.304863in}{4.275653in}}{\pgfqpoint{2.294264in}{4.271263in}}{\pgfqpoint{2.286451in}{4.263449in}}%
\pgfpathcurveto{\pgfqpoint{2.278637in}{4.255635in}}{\pgfqpoint{2.274247in}{4.245036in}}{\pgfqpoint{2.274247in}{4.233986in}}%
\pgfpathcurveto{\pgfqpoint{2.274247in}{4.222936in}}{\pgfqpoint{2.278637in}{4.212337in}}{\pgfqpoint{2.286451in}{4.204524in}}%
\pgfpathcurveto{\pgfqpoint{2.294264in}{4.196710in}}{\pgfqpoint{2.304863in}{4.192320in}}{\pgfqpoint{2.315914in}{4.192320in}}%
\pgfpathclose%
\pgfusepath{stroke,fill}%
\end{pgfscope}%
\begin{pgfscope}%
\pgfpathrectangle{\pgfqpoint{0.481978in}{0.331635in}}{\pgfqpoint{9.300000in}{7.700000in}}%
\pgfusepath{clip}%
\pgfsetbuttcap%
\pgfsetroundjoin%
\definecolor{currentfill}{rgb}{0.631373,0.788235,0.956863}%
\pgfsetfillcolor{currentfill}%
\pgfsetlinewidth{0.481800pt}%
\definecolor{currentstroke}{rgb}{1.000000,1.000000,1.000000}%
\pgfsetstrokecolor{currentstroke}%
\pgfsetdash{}{0pt}%
\pgfpathmoveto{\pgfqpoint{6.878899in}{1.071438in}}%
\pgfpathcurveto{\pgfqpoint{6.889949in}{1.071438in}}{\pgfqpoint{6.900548in}{1.075828in}}{\pgfqpoint{6.908362in}{1.083641in}}%
\pgfpathcurveto{\pgfqpoint{6.916175in}{1.091455in}}{\pgfqpoint{6.920566in}{1.102054in}}{\pgfqpoint{6.920566in}{1.113104in}}%
\pgfpathcurveto{\pgfqpoint{6.920566in}{1.124154in}}{\pgfqpoint{6.916175in}{1.134753in}}{\pgfqpoint{6.908362in}{1.142567in}}%
\pgfpathcurveto{\pgfqpoint{6.900548in}{1.150381in}}{\pgfqpoint{6.889949in}{1.154771in}}{\pgfqpoint{6.878899in}{1.154771in}}%
\pgfpathcurveto{\pgfqpoint{6.867849in}{1.154771in}}{\pgfqpoint{6.857250in}{1.150381in}}{\pgfqpoint{6.849436in}{1.142567in}}%
\pgfpathcurveto{\pgfqpoint{6.841623in}{1.134753in}}{\pgfqpoint{6.837232in}{1.124154in}}{\pgfqpoint{6.837232in}{1.113104in}}%
\pgfpathcurveto{\pgfqpoint{6.837232in}{1.102054in}}{\pgfqpoint{6.841623in}{1.091455in}}{\pgfqpoint{6.849436in}{1.083641in}}%
\pgfpathcurveto{\pgfqpoint{6.857250in}{1.075828in}}{\pgfqpoint{6.867849in}{1.071438in}}{\pgfqpoint{6.878899in}{1.071438in}}%
\pgfpathclose%
\pgfusepath{stroke,fill}%
\end{pgfscope}%
\begin{pgfscope}%
\pgfpathrectangle{\pgfqpoint{0.481978in}{0.331635in}}{\pgfqpoint{9.300000in}{7.700000in}}%
\pgfusepath{clip}%
\pgfsetbuttcap%
\pgfsetroundjoin%
\definecolor{currentfill}{rgb}{0.631373,0.788235,0.956863}%
\pgfsetfillcolor{currentfill}%
\pgfsetlinewidth{0.481800pt}%
\definecolor{currentstroke}{rgb}{1.000000,1.000000,1.000000}%
\pgfsetstrokecolor{currentstroke}%
\pgfsetdash{}{0pt}%
\pgfpathmoveto{\pgfqpoint{3.274514in}{1.534614in}}%
\pgfpathcurveto{\pgfqpoint{3.285565in}{1.534614in}}{\pgfqpoint{3.296164in}{1.539004in}}{\pgfqpoint{3.303977in}{1.546818in}}%
\pgfpathcurveto{\pgfqpoint{3.311791in}{1.554631in}}{\pgfqpoint{3.316181in}{1.565230in}}{\pgfqpoint{3.316181in}{1.576281in}}%
\pgfpathcurveto{\pgfqpoint{3.316181in}{1.587331in}}{\pgfqpoint{3.311791in}{1.597930in}}{\pgfqpoint{3.303977in}{1.605743in}}%
\pgfpathcurveto{\pgfqpoint{3.296164in}{1.613557in}}{\pgfqpoint{3.285565in}{1.617947in}}{\pgfqpoint{3.274514in}{1.617947in}}%
\pgfpathcurveto{\pgfqpoint{3.263464in}{1.617947in}}{\pgfqpoint{3.252865in}{1.613557in}}{\pgfqpoint{3.245052in}{1.605743in}}%
\pgfpathcurveto{\pgfqpoint{3.237238in}{1.597930in}}{\pgfqpoint{3.232848in}{1.587331in}}{\pgfqpoint{3.232848in}{1.576281in}}%
\pgfpathcurveto{\pgfqpoint{3.232848in}{1.565230in}}{\pgfqpoint{3.237238in}{1.554631in}}{\pgfqpoint{3.245052in}{1.546818in}}%
\pgfpathcurveto{\pgfqpoint{3.252865in}{1.539004in}}{\pgfqpoint{3.263464in}{1.534614in}}{\pgfqpoint{3.274514in}{1.534614in}}%
\pgfpathclose%
\pgfusepath{stroke,fill}%
\end{pgfscope}%
\begin{pgfscope}%
\pgfpathrectangle{\pgfqpoint{0.481978in}{0.331635in}}{\pgfqpoint{9.300000in}{7.700000in}}%
\pgfusepath{clip}%
\pgfsetbuttcap%
\pgfsetroundjoin%
\definecolor{currentfill}{rgb}{0.631373,0.788235,0.956863}%
\pgfsetfillcolor{currentfill}%
\pgfsetlinewidth{0.481800pt}%
\definecolor{currentstroke}{rgb}{1.000000,1.000000,1.000000}%
\pgfsetstrokecolor{currentstroke}%
\pgfsetdash{}{0pt}%
\pgfpathmoveto{\pgfqpoint{6.005151in}{4.440552in}}%
\pgfpathcurveto{\pgfqpoint{6.016201in}{4.440552in}}{\pgfqpoint{6.026800in}{4.444943in}}{\pgfqpoint{6.034613in}{4.452756in}}%
\pgfpathcurveto{\pgfqpoint{6.042427in}{4.460570in}}{\pgfqpoint{6.046817in}{4.471169in}}{\pgfqpoint{6.046817in}{4.482219in}}%
\pgfpathcurveto{\pgfqpoint{6.046817in}{4.493269in}}{\pgfqpoint{6.042427in}{4.503868in}}{\pgfqpoint{6.034613in}{4.511682in}}%
\pgfpathcurveto{\pgfqpoint{6.026800in}{4.519495in}}{\pgfqpoint{6.016201in}{4.523886in}}{\pgfqpoint{6.005151in}{4.523886in}}%
\pgfpathcurveto{\pgfqpoint{5.994100in}{4.523886in}}{\pgfqpoint{5.983501in}{4.519495in}}{\pgfqpoint{5.975688in}{4.511682in}}%
\pgfpathcurveto{\pgfqpoint{5.967874in}{4.503868in}}{\pgfqpoint{5.963484in}{4.493269in}}{\pgfqpoint{5.963484in}{4.482219in}}%
\pgfpathcurveto{\pgfqpoint{5.963484in}{4.471169in}}{\pgfqpoint{5.967874in}{4.460570in}}{\pgfqpoint{5.975688in}{4.452756in}}%
\pgfpathcurveto{\pgfqpoint{5.983501in}{4.444943in}}{\pgfqpoint{5.994100in}{4.440552in}}{\pgfqpoint{6.005151in}{4.440552in}}%
\pgfpathclose%
\pgfusepath{stroke,fill}%
\end{pgfscope}%
\begin{pgfscope}%
\pgfpathrectangle{\pgfqpoint{0.481978in}{0.331635in}}{\pgfqpoint{9.300000in}{7.700000in}}%
\pgfusepath{clip}%
\pgfsetbuttcap%
\pgfsetroundjoin%
\definecolor{currentfill}{rgb}{0.631373,0.788235,0.956863}%
\pgfsetfillcolor{currentfill}%
\pgfsetlinewidth{0.481800pt}%
\definecolor{currentstroke}{rgb}{1.000000,1.000000,1.000000}%
\pgfsetstrokecolor{currentstroke}%
\pgfsetdash{}{0pt}%
\pgfpathmoveto{\pgfqpoint{7.064826in}{4.790265in}}%
\pgfpathcurveto{\pgfqpoint{7.075876in}{4.790265in}}{\pgfqpoint{7.086475in}{4.794656in}}{\pgfqpoint{7.094289in}{4.802469in}}%
\pgfpathcurveto{\pgfqpoint{7.102102in}{4.810283in}}{\pgfqpoint{7.106493in}{4.820882in}}{\pgfqpoint{7.106493in}{4.831932in}}%
\pgfpathcurveto{\pgfqpoint{7.106493in}{4.842982in}}{\pgfqpoint{7.102102in}{4.853581in}}{\pgfqpoint{7.094289in}{4.861395in}}%
\pgfpathcurveto{\pgfqpoint{7.086475in}{4.869208in}}{\pgfqpoint{7.075876in}{4.873599in}}{\pgfqpoint{7.064826in}{4.873599in}}%
\pgfpathcurveto{\pgfqpoint{7.053776in}{4.873599in}}{\pgfqpoint{7.043177in}{4.869208in}}{\pgfqpoint{7.035363in}{4.861395in}}%
\pgfpathcurveto{\pgfqpoint{7.027550in}{4.853581in}}{\pgfqpoint{7.023159in}{4.842982in}}{\pgfqpoint{7.023159in}{4.831932in}}%
\pgfpathcurveto{\pgfqpoint{7.023159in}{4.820882in}}{\pgfqpoint{7.027550in}{4.810283in}}{\pgfqpoint{7.035363in}{4.802469in}}%
\pgfpathcurveto{\pgfqpoint{7.043177in}{4.794656in}}{\pgfqpoint{7.053776in}{4.790265in}}{\pgfqpoint{7.064826in}{4.790265in}}%
\pgfpathclose%
\pgfusepath{stroke,fill}%
\end{pgfscope}%
\begin{pgfscope}%
\pgfpathrectangle{\pgfqpoint{0.481978in}{0.331635in}}{\pgfqpoint{9.300000in}{7.700000in}}%
\pgfusepath{clip}%
\pgfsetbuttcap%
\pgfsetroundjoin%
\definecolor{currentfill}{rgb}{0.631373,0.788235,0.956863}%
\pgfsetfillcolor{currentfill}%
\pgfsetlinewidth{0.481800pt}%
\definecolor{currentstroke}{rgb}{1.000000,1.000000,1.000000}%
\pgfsetstrokecolor{currentstroke}%
\pgfsetdash{}{0pt}%
\pgfpathmoveto{\pgfqpoint{6.961902in}{1.575435in}}%
\pgfpathcurveto{\pgfqpoint{6.972953in}{1.575435in}}{\pgfqpoint{6.983552in}{1.579825in}}{\pgfqpoint{6.991365in}{1.587638in}}%
\pgfpathcurveto{\pgfqpoint{6.999179in}{1.595452in}}{\pgfqpoint{7.003569in}{1.606051in}}{\pgfqpoint{7.003569in}{1.617101in}}%
\pgfpathcurveto{\pgfqpoint{7.003569in}{1.628151in}}{\pgfqpoint{6.999179in}{1.638750in}}{\pgfqpoint{6.991365in}{1.646564in}}%
\pgfpathcurveto{\pgfqpoint{6.983552in}{1.654378in}}{\pgfqpoint{6.972953in}{1.658768in}}{\pgfqpoint{6.961902in}{1.658768in}}%
\pgfpathcurveto{\pgfqpoint{6.950852in}{1.658768in}}{\pgfqpoint{6.940253in}{1.654378in}}{\pgfqpoint{6.932440in}{1.646564in}}%
\pgfpathcurveto{\pgfqpoint{6.924626in}{1.638750in}}{\pgfqpoint{6.920236in}{1.628151in}}{\pgfqpoint{6.920236in}{1.617101in}}%
\pgfpathcurveto{\pgfqpoint{6.920236in}{1.606051in}}{\pgfqpoint{6.924626in}{1.595452in}}{\pgfqpoint{6.932440in}{1.587638in}}%
\pgfpathcurveto{\pgfqpoint{6.940253in}{1.579825in}}{\pgfqpoint{6.950852in}{1.575435in}}{\pgfqpoint{6.961902in}{1.575435in}}%
\pgfpathclose%
\pgfusepath{stroke,fill}%
\end{pgfscope}%
\begin{pgfscope}%
\pgfpathrectangle{\pgfqpoint{0.481978in}{0.331635in}}{\pgfqpoint{9.300000in}{7.700000in}}%
\pgfusepath{clip}%
\pgfsetbuttcap%
\pgfsetroundjoin%
\definecolor{currentfill}{rgb}{0.631373,0.788235,0.956863}%
\pgfsetfillcolor{currentfill}%
\pgfsetlinewidth{0.481800pt}%
\definecolor{currentstroke}{rgb}{1.000000,1.000000,1.000000}%
\pgfsetstrokecolor{currentstroke}%
\pgfsetdash{}{0pt}%
\pgfpathmoveto{\pgfqpoint{3.800404in}{4.844822in}}%
\pgfpathcurveto{\pgfqpoint{3.811455in}{4.844822in}}{\pgfqpoint{3.822054in}{4.849212in}}{\pgfqpoint{3.829867in}{4.857026in}}%
\pgfpathcurveto{\pgfqpoint{3.837681in}{4.864840in}}{\pgfqpoint{3.842071in}{4.875439in}}{\pgfqpoint{3.842071in}{4.886489in}}%
\pgfpathcurveto{\pgfqpoint{3.842071in}{4.897539in}}{\pgfqpoint{3.837681in}{4.908138in}}{\pgfqpoint{3.829867in}{4.915952in}}%
\pgfpathcurveto{\pgfqpoint{3.822054in}{4.923765in}}{\pgfqpoint{3.811455in}{4.928155in}}{\pgfqpoint{3.800404in}{4.928155in}}%
\pgfpathcurveto{\pgfqpoint{3.789354in}{4.928155in}}{\pgfqpoint{3.778755in}{4.923765in}}{\pgfqpoint{3.770942in}{4.915952in}}%
\pgfpathcurveto{\pgfqpoint{3.763128in}{4.908138in}}{\pgfqpoint{3.758738in}{4.897539in}}{\pgfqpoint{3.758738in}{4.886489in}}%
\pgfpathcurveto{\pgfqpoint{3.758738in}{4.875439in}}{\pgfqpoint{3.763128in}{4.864840in}}{\pgfqpoint{3.770942in}{4.857026in}}%
\pgfpathcurveto{\pgfqpoint{3.778755in}{4.849212in}}{\pgfqpoint{3.789354in}{4.844822in}}{\pgfqpoint{3.800404in}{4.844822in}}%
\pgfpathclose%
\pgfusepath{stroke,fill}%
\end{pgfscope}%
\begin{pgfscope}%
\pgfpathrectangle{\pgfqpoint{0.481978in}{0.331635in}}{\pgfqpoint{9.300000in}{7.700000in}}%
\pgfusepath{clip}%
\pgfsetbuttcap%
\pgfsetroundjoin%
\definecolor{currentfill}{rgb}{0.631373,0.788235,0.956863}%
\pgfsetfillcolor{currentfill}%
\pgfsetlinewidth{0.481800pt}%
\definecolor{currentstroke}{rgb}{1.000000,1.000000,1.000000}%
\pgfsetstrokecolor{currentstroke}%
\pgfsetdash{}{0pt}%
\pgfpathmoveto{\pgfqpoint{6.058849in}{1.448720in}}%
\pgfpathcurveto{\pgfqpoint{6.069899in}{1.448720in}}{\pgfqpoint{6.080498in}{1.453110in}}{\pgfqpoint{6.088312in}{1.460924in}}%
\pgfpathcurveto{\pgfqpoint{6.096125in}{1.468738in}}{\pgfqpoint{6.100516in}{1.479337in}}{\pgfqpoint{6.100516in}{1.490387in}}%
\pgfpathcurveto{\pgfqpoint{6.100516in}{1.501437in}}{\pgfqpoint{6.096125in}{1.512036in}}{\pgfqpoint{6.088312in}{1.519850in}}%
\pgfpathcurveto{\pgfqpoint{6.080498in}{1.527663in}}{\pgfqpoint{6.069899in}{1.532054in}}{\pgfqpoint{6.058849in}{1.532054in}}%
\pgfpathcurveto{\pgfqpoint{6.047799in}{1.532054in}}{\pgfqpoint{6.037200in}{1.527663in}}{\pgfqpoint{6.029386in}{1.519850in}}%
\pgfpathcurveto{\pgfqpoint{6.021573in}{1.512036in}}{\pgfqpoint{6.017182in}{1.501437in}}{\pgfqpoint{6.017182in}{1.490387in}}%
\pgfpathcurveto{\pgfqpoint{6.017182in}{1.479337in}}{\pgfqpoint{6.021573in}{1.468738in}}{\pgfqpoint{6.029386in}{1.460924in}}%
\pgfpathcurveto{\pgfqpoint{6.037200in}{1.453110in}}{\pgfqpoint{6.047799in}{1.448720in}}{\pgfqpoint{6.058849in}{1.448720in}}%
\pgfpathclose%
\pgfusepath{stroke,fill}%
\end{pgfscope}%
\begin{pgfscope}%
\pgfpathrectangle{\pgfqpoint{0.481978in}{0.331635in}}{\pgfqpoint{9.300000in}{7.700000in}}%
\pgfusepath{clip}%
\pgfsetbuttcap%
\pgfsetroundjoin%
\definecolor{currentfill}{rgb}{0.631373,0.788235,0.956863}%
\pgfsetfillcolor{currentfill}%
\pgfsetlinewidth{0.481800pt}%
\definecolor{currentstroke}{rgb}{1.000000,1.000000,1.000000}%
\pgfsetstrokecolor{currentstroke}%
\pgfsetdash{}{0pt}%
\pgfpathmoveto{\pgfqpoint{5.307606in}{5.921468in}}%
\pgfpathcurveto{\pgfqpoint{5.318656in}{5.921468in}}{\pgfqpoint{5.329255in}{5.925859in}}{\pgfqpoint{5.337068in}{5.933672in}}%
\pgfpathcurveto{\pgfqpoint{5.344882in}{5.941486in}}{\pgfqpoint{5.349272in}{5.952085in}}{\pgfqpoint{5.349272in}{5.963135in}}%
\pgfpathcurveto{\pgfqpoint{5.349272in}{5.974185in}}{\pgfqpoint{5.344882in}{5.984784in}}{\pgfqpoint{5.337068in}{5.992598in}}%
\pgfpathcurveto{\pgfqpoint{5.329255in}{6.000411in}}{\pgfqpoint{5.318656in}{6.004802in}}{\pgfqpoint{5.307606in}{6.004802in}}%
\pgfpathcurveto{\pgfqpoint{5.296555in}{6.004802in}}{\pgfqpoint{5.285956in}{6.000411in}}{\pgfqpoint{5.278143in}{5.992598in}}%
\pgfpathcurveto{\pgfqpoint{5.270329in}{5.984784in}}{\pgfqpoint{5.265939in}{5.974185in}}{\pgfqpoint{5.265939in}{5.963135in}}%
\pgfpathcurveto{\pgfqpoint{5.265939in}{5.952085in}}{\pgfqpoint{5.270329in}{5.941486in}}{\pgfqpoint{5.278143in}{5.933672in}}%
\pgfpathcurveto{\pgfqpoint{5.285956in}{5.925859in}}{\pgfqpoint{5.296555in}{5.921468in}}{\pgfqpoint{5.307606in}{5.921468in}}%
\pgfpathclose%
\pgfusepath{stroke,fill}%
\end{pgfscope}%
\begin{pgfscope}%
\pgfpathrectangle{\pgfqpoint{0.481978in}{0.331635in}}{\pgfqpoint{9.300000in}{7.700000in}}%
\pgfusepath{clip}%
\pgfsetbuttcap%
\pgfsetroundjoin%
\definecolor{currentfill}{rgb}{0.631373,0.788235,0.956863}%
\pgfsetfillcolor{currentfill}%
\pgfsetlinewidth{0.481800pt}%
\definecolor{currentstroke}{rgb}{1.000000,1.000000,1.000000}%
\pgfsetstrokecolor{currentstroke}%
\pgfsetdash{}{0pt}%
\pgfpathmoveto{\pgfqpoint{7.033678in}{3.796632in}}%
\pgfpathcurveto{\pgfqpoint{7.044728in}{3.796632in}}{\pgfqpoint{7.055327in}{3.801023in}}{\pgfqpoint{7.063140in}{3.808836in}}%
\pgfpathcurveto{\pgfqpoint{7.070954in}{3.816650in}}{\pgfqpoint{7.075344in}{3.827249in}}{\pgfqpoint{7.075344in}{3.838299in}}%
\pgfpathcurveto{\pgfqpoint{7.075344in}{3.849349in}}{\pgfqpoint{7.070954in}{3.859948in}}{\pgfqpoint{7.063140in}{3.867762in}}%
\pgfpathcurveto{\pgfqpoint{7.055327in}{3.875576in}}{\pgfqpoint{7.044728in}{3.879966in}}{\pgfqpoint{7.033678in}{3.879966in}}%
\pgfpathcurveto{\pgfqpoint{7.022627in}{3.879966in}}{\pgfqpoint{7.012028in}{3.875576in}}{\pgfqpoint{7.004215in}{3.867762in}}%
\pgfpathcurveto{\pgfqpoint{6.996401in}{3.859948in}}{\pgfqpoint{6.992011in}{3.849349in}}{\pgfqpoint{6.992011in}{3.838299in}}%
\pgfpathcurveto{\pgfqpoint{6.992011in}{3.827249in}}{\pgfqpoint{6.996401in}{3.816650in}}{\pgfqpoint{7.004215in}{3.808836in}}%
\pgfpathcurveto{\pgfqpoint{7.012028in}{3.801023in}}{\pgfqpoint{7.022627in}{3.796632in}}{\pgfqpoint{7.033678in}{3.796632in}}%
\pgfpathclose%
\pgfusepath{stroke,fill}%
\end{pgfscope}%
\begin{pgfscope}%
\pgfpathrectangle{\pgfqpoint{0.481978in}{0.331635in}}{\pgfqpoint{9.300000in}{7.700000in}}%
\pgfusepath{clip}%
\pgfsetbuttcap%
\pgfsetroundjoin%
\definecolor{currentfill}{rgb}{0.631373,0.788235,0.956863}%
\pgfsetfillcolor{currentfill}%
\pgfsetlinewidth{0.481800pt}%
\definecolor{currentstroke}{rgb}{1.000000,1.000000,1.000000}%
\pgfsetstrokecolor{currentstroke}%
\pgfsetdash{}{0pt}%
\pgfpathmoveto{\pgfqpoint{7.805428in}{0.806753in}}%
\pgfpathcurveto{\pgfqpoint{7.816478in}{0.806753in}}{\pgfqpoint{7.827077in}{0.811143in}}{\pgfqpoint{7.834891in}{0.818957in}}%
\pgfpathcurveto{\pgfqpoint{7.842704in}{0.826770in}}{\pgfqpoint{7.847095in}{0.837369in}}{\pgfqpoint{7.847095in}{0.848419in}}%
\pgfpathcurveto{\pgfqpoint{7.847095in}{0.859470in}}{\pgfqpoint{7.842704in}{0.870069in}}{\pgfqpoint{7.834891in}{0.877882in}}%
\pgfpathcurveto{\pgfqpoint{7.827077in}{0.885696in}}{\pgfqpoint{7.816478in}{0.890086in}}{\pgfqpoint{7.805428in}{0.890086in}}%
\pgfpathcurveto{\pgfqpoint{7.794378in}{0.890086in}}{\pgfqpoint{7.783779in}{0.885696in}}{\pgfqpoint{7.775965in}{0.877882in}}%
\pgfpathcurveto{\pgfqpoint{7.768151in}{0.870069in}}{\pgfqpoint{7.763761in}{0.859470in}}{\pgfqpoint{7.763761in}{0.848419in}}%
\pgfpathcurveto{\pgfqpoint{7.763761in}{0.837369in}}{\pgfqpoint{7.768151in}{0.826770in}}{\pgfqpoint{7.775965in}{0.818957in}}%
\pgfpathcurveto{\pgfqpoint{7.783779in}{0.811143in}}{\pgfqpoint{7.794378in}{0.806753in}}{\pgfqpoint{7.805428in}{0.806753in}}%
\pgfpathclose%
\pgfusepath{stroke,fill}%
\end{pgfscope}%
\begin{pgfscope}%
\pgfpathrectangle{\pgfqpoint{0.481978in}{0.331635in}}{\pgfqpoint{9.300000in}{7.700000in}}%
\pgfusepath{clip}%
\pgfsetbuttcap%
\pgfsetroundjoin%
\definecolor{currentfill}{rgb}{0.631373,0.788235,0.956863}%
\pgfsetfillcolor{currentfill}%
\pgfsetlinewidth{0.481800pt}%
\definecolor{currentstroke}{rgb}{1.000000,1.000000,1.000000}%
\pgfsetstrokecolor{currentstroke}%
\pgfsetdash{}{0pt}%
\pgfpathmoveto{\pgfqpoint{3.104471in}{4.739766in}}%
\pgfpathcurveto{\pgfqpoint{3.115521in}{4.739766in}}{\pgfqpoint{3.126120in}{4.744156in}}{\pgfqpoint{3.133933in}{4.751969in}}%
\pgfpathcurveto{\pgfqpoint{3.141747in}{4.759783in}}{\pgfqpoint{3.146137in}{4.770382in}}{\pgfqpoint{3.146137in}{4.781432in}}%
\pgfpathcurveto{\pgfqpoint{3.146137in}{4.792482in}}{\pgfqpoint{3.141747in}{4.803081in}}{\pgfqpoint{3.133933in}{4.810895in}}%
\pgfpathcurveto{\pgfqpoint{3.126120in}{4.818709in}}{\pgfqpoint{3.115521in}{4.823099in}}{\pgfqpoint{3.104471in}{4.823099in}}%
\pgfpathcurveto{\pgfqpoint{3.093421in}{4.823099in}}{\pgfqpoint{3.082822in}{4.818709in}}{\pgfqpoint{3.075008in}{4.810895in}}%
\pgfpathcurveto{\pgfqpoint{3.067194in}{4.803081in}}{\pgfqpoint{3.062804in}{4.792482in}}{\pgfqpoint{3.062804in}{4.781432in}}%
\pgfpathcurveto{\pgfqpoint{3.062804in}{4.770382in}}{\pgfqpoint{3.067194in}{4.759783in}}{\pgfqpoint{3.075008in}{4.751969in}}%
\pgfpathcurveto{\pgfqpoint{3.082822in}{4.744156in}}{\pgfqpoint{3.093421in}{4.739766in}}{\pgfqpoint{3.104471in}{4.739766in}}%
\pgfpathclose%
\pgfusepath{stroke,fill}%
\end{pgfscope}%
\begin{pgfscope}%
\pgfpathrectangle{\pgfqpoint{0.481978in}{0.331635in}}{\pgfqpoint{9.300000in}{7.700000in}}%
\pgfusepath{clip}%
\pgfsetbuttcap%
\pgfsetroundjoin%
\definecolor{currentfill}{rgb}{0.631373,0.788235,0.956863}%
\pgfsetfillcolor{currentfill}%
\pgfsetlinewidth{0.481800pt}%
\definecolor{currentstroke}{rgb}{1.000000,1.000000,1.000000}%
\pgfsetstrokecolor{currentstroke}%
\pgfsetdash{}{0pt}%
\pgfpathmoveto{\pgfqpoint{1.582673in}{3.689374in}}%
\pgfpathcurveto{\pgfqpoint{1.593723in}{3.689374in}}{\pgfqpoint{1.604322in}{3.693764in}}{\pgfqpoint{1.612135in}{3.701578in}}%
\pgfpathcurveto{\pgfqpoint{1.619949in}{3.709391in}}{\pgfqpoint{1.624339in}{3.719990in}}{\pgfqpoint{1.624339in}{3.731040in}}%
\pgfpathcurveto{\pgfqpoint{1.624339in}{3.742091in}}{\pgfqpoint{1.619949in}{3.752690in}}{\pgfqpoint{1.612135in}{3.760503in}}%
\pgfpathcurveto{\pgfqpoint{1.604322in}{3.768317in}}{\pgfqpoint{1.593723in}{3.772707in}}{\pgfqpoint{1.582673in}{3.772707in}}%
\pgfpathcurveto{\pgfqpoint{1.571622in}{3.772707in}}{\pgfqpoint{1.561023in}{3.768317in}}{\pgfqpoint{1.553210in}{3.760503in}}%
\pgfpathcurveto{\pgfqpoint{1.545396in}{3.752690in}}{\pgfqpoint{1.541006in}{3.742091in}}{\pgfqpoint{1.541006in}{3.731040in}}%
\pgfpathcurveto{\pgfqpoint{1.541006in}{3.719990in}}{\pgfqpoint{1.545396in}{3.709391in}}{\pgfqpoint{1.553210in}{3.701578in}}%
\pgfpathcurveto{\pgfqpoint{1.561023in}{3.693764in}}{\pgfqpoint{1.571622in}{3.689374in}}{\pgfqpoint{1.582673in}{3.689374in}}%
\pgfpathclose%
\pgfusepath{stroke,fill}%
\end{pgfscope}%
\begin{pgfscope}%
\pgfpathrectangle{\pgfqpoint{0.481978in}{0.331635in}}{\pgfqpoint{9.300000in}{7.700000in}}%
\pgfusepath{clip}%
\pgfsetbuttcap%
\pgfsetroundjoin%
\definecolor{currentfill}{rgb}{0.631373,0.788235,0.956863}%
\pgfsetfillcolor{currentfill}%
\pgfsetlinewidth{0.481800pt}%
\definecolor{currentstroke}{rgb}{1.000000,1.000000,1.000000}%
\pgfsetstrokecolor{currentstroke}%
\pgfsetdash{}{0pt}%
\pgfpathmoveto{\pgfqpoint{7.082906in}{5.476502in}}%
\pgfpathcurveto{\pgfqpoint{7.093957in}{5.476502in}}{\pgfqpoint{7.104556in}{5.480892in}}{\pgfqpoint{7.112369in}{5.488706in}}%
\pgfpathcurveto{\pgfqpoint{7.120183in}{5.496519in}}{\pgfqpoint{7.124573in}{5.507118in}}{\pgfqpoint{7.124573in}{5.518169in}}%
\pgfpathcurveto{\pgfqpoint{7.124573in}{5.529219in}}{\pgfqpoint{7.120183in}{5.539818in}}{\pgfqpoint{7.112369in}{5.547631in}}%
\pgfpathcurveto{\pgfqpoint{7.104556in}{5.555445in}}{\pgfqpoint{7.093957in}{5.559835in}}{\pgfqpoint{7.082906in}{5.559835in}}%
\pgfpathcurveto{\pgfqpoint{7.071856in}{5.559835in}}{\pgfqpoint{7.061257in}{5.555445in}}{\pgfqpoint{7.053444in}{5.547631in}}%
\pgfpathcurveto{\pgfqpoint{7.045630in}{5.539818in}}{\pgfqpoint{7.041240in}{5.529219in}}{\pgfqpoint{7.041240in}{5.518169in}}%
\pgfpathcurveto{\pgfqpoint{7.041240in}{5.507118in}}{\pgfqpoint{7.045630in}{5.496519in}}{\pgfqpoint{7.053444in}{5.488706in}}%
\pgfpathcurveto{\pgfqpoint{7.061257in}{5.480892in}}{\pgfqpoint{7.071856in}{5.476502in}}{\pgfqpoint{7.082906in}{5.476502in}}%
\pgfpathclose%
\pgfusepath{stroke,fill}%
\end{pgfscope}%
\begin{pgfscope}%
\pgfpathrectangle{\pgfqpoint{0.481978in}{0.331635in}}{\pgfqpoint{9.300000in}{7.700000in}}%
\pgfusepath{clip}%
\pgfsetbuttcap%
\pgfsetroundjoin%
\definecolor{currentfill}{rgb}{0.631373,0.788235,0.956863}%
\pgfsetfillcolor{currentfill}%
\pgfsetlinewidth{0.481800pt}%
\definecolor{currentstroke}{rgb}{1.000000,1.000000,1.000000}%
\pgfsetstrokecolor{currentstroke}%
\pgfsetdash{}{0pt}%
\pgfpathmoveto{\pgfqpoint{2.863295in}{2.519612in}}%
\pgfpathcurveto{\pgfqpoint{2.874345in}{2.519612in}}{\pgfqpoint{2.884944in}{2.524002in}}{\pgfqpoint{2.892758in}{2.531816in}}%
\pgfpathcurveto{\pgfqpoint{2.900572in}{2.539629in}}{\pgfqpoint{2.904962in}{2.550228in}}{\pgfqpoint{2.904962in}{2.561278in}}%
\pgfpathcurveto{\pgfqpoint{2.904962in}{2.572329in}}{\pgfqpoint{2.900572in}{2.582928in}}{\pgfqpoint{2.892758in}{2.590741in}}%
\pgfpathcurveto{\pgfqpoint{2.884944in}{2.598555in}}{\pgfqpoint{2.874345in}{2.602945in}}{\pgfqpoint{2.863295in}{2.602945in}}%
\pgfpathcurveto{\pgfqpoint{2.852245in}{2.602945in}}{\pgfqpoint{2.841646in}{2.598555in}}{\pgfqpoint{2.833833in}{2.590741in}}%
\pgfpathcurveto{\pgfqpoint{2.826019in}{2.582928in}}{\pgfqpoint{2.821629in}{2.572329in}}{\pgfqpoint{2.821629in}{2.561278in}}%
\pgfpathcurveto{\pgfqpoint{2.821629in}{2.550228in}}{\pgfqpoint{2.826019in}{2.539629in}}{\pgfqpoint{2.833833in}{2.531816in}}%
\pgfpathcurveto{\pgfqpoint{2.841646in}{2.524002in}}{\pgfqpoint{2.852245in}{2.519612in}}{\pgfqpoint{2.863295in}{2.519612in}}%
\pgfpathclose%
\pgfusepath{stroke,fill}%
\end{pgfscope}%
\begin{pgfscope}%
\pgfpathrectangle{\pgfqpoint{0.481978in}{0.331635in}}{\pgfqpoint{9.300000in}{7.700000in}}%
\pgfusepath{clip}%
\pgfsetbuttcap%
\pgfsetroundjoin%
\definecolor{currentfill}{rgb}{0.631373,0.788235,0.956863}%
\pgfsetfillcolor{currentfill}%
\pgfsetlinewidth{0.481800pt}%
\definecolor{currentstroke}{rgb}{1.000000,1.000000,1.000000}%
\pgfsetstrokecolor{currentstroke}%
\pgfsetdash{}{0pt}%
\pgfpathmoveto{\pgfqpoint{7.718781in}{2.730738in}}%
\pgfpathcurveto{\pgfqpoint{7.729831in}{2.730738in}}{\pgfqpoint{7.740431in}{2.735128in}}{\pgfqpoint{7.748244in}{2.742942in}}%
\pgfpathcurveto{\pgfqpoint{7.756058in}{2.750756in}}{\pgfqpoint{7.760448in}{2.761355in}}{\pgfqpoint{7.760448in}{2.772405in}}%
\pgfpathcurveto{\pgfqpoint{7.760448in}{2.783455in}}{\pgfqpoint{7.756058in}{2.794054in}}{\pgfqpoint{7.748244in}{2.801867in}}%
\pgfpathcurveto{\pgfqpoint{7.740431in}{2.809681in}}{\pgfqpoint{7.729831in}{2.814071in}}{\pgfqpoint{7.718781in}{2.814071in}}%
\pgfpathcurveto{\pgfqpoint{7.707731in}{2.814071in}}{\pgfqpoint{7.697132in}{2.809681in}}{\pgfqpoint{7.689319in}{2.801867in}}%
\pgfpathcurveto{\pgfqpoint{7.681505in}{2.794054in}}{\pgfqpoint{7.677115in}{2.783455in}}{\pgfqpoint{7.677115in}{2.772405in}}%
\pgfpathcurveto{\pgfqpoint{7.677115in}{2.761355in}}{\pgfqpoint{7.681505in}{2.750756in}}{\pgfqpoint{7.689319in}{2.742942in}}%
\pgfpathcurveto{\pgfqpoint{7.697132in}{2.735128in}}{\pgfqpoint{7.707731in}{2.730738in}}{\pgfqpoint{7.718781in}{2.730738in}}%
\pgfpathclose%
\pgfusepath{stroke,fill}%
\end{pgfscope}%
\begin{pgfscope}%
\pgfpathrectangle{\pgfqpoint{0.481978in}{0.331635in}}{\pgfqpoint{9.300000in}{7.700000in}}%
\pgfusepath{clip}%
\pgfsetbuttcap%
\pgfsetroundjoin%
\definecolor{currentfill}{rgb}{0.631373,0.788235,0.956863}%
\pgfsetfillcolor{currentfill}%
\pgfsetlinewidth{0.481800pt}%
\definecolor{currentstroke}{rgb}{1.000000,1.000000,1.000000}%
\pgfsetstrokecolor{currentstroke}%
\pgfsetdash{}{0pt}%
\pgfpathmoveto{\pgfqpoint{3.697003in}{3.898380in}}%
\pgfpathcurveto{\pgfqpoint{3.708053in}{3.898380in}}{\pgfqpoint{3.718652in}{3.902770in}}{\pgfqpoint{3.726465in}{3.910584in}}%
\pgfpathcurveto{\pgfqpoint{3.734279in}{3.918397in}}{\pgfqpoint{3.738669in}{3.928996in}}{\pgfqpoint{3.738669in}{3.940047in}}%
\pgfpathcurveto{\pgfqpoint{3.738669in}{3.951097in}}{\pgfqpoint{3.734279in}{3.961696in}}{\pgfqpoint{3.726465in}{3.969509in}}%
\pgfpathcurveto{\pgfqpoint{3.718652in}{3.977323in}}{\pgfqpoint{3.708053in}{3.981713in}}{\pgfqpoint{3.697003in}{3.981713in}}%
\pgfpathcurveto{\pgfqpoint{3.685953in}{3.981713in}}{\pgfqpoint{3.675354in}{3.977323in}}{\pgfqpoint{3.667540in}{3.969509in}}%
\pgfpathcurveto{\pgfqpoint{3.659726in}{3.961696in}}{\pgfqpoint{3.655336in}{3.951097in}}{\pgfqpoint{3.655336in}{3.940047in}}%
\pgfpathcurveto{\pgfqpoint{3.655336in}{3.928996in}}{\pgfqpoint{3.659726in}{3.918397in}}{\pgfqpoint{3.667540in}{3.910584in}}%
\pgfpathcurveto{\pgfqpoint{3.675354in}{3.902770in}}{\pgfqpoint{3.685953in}{3.898380in}}{\pgfqpoint{3.697003in}{3.898380in}}%
\pgfpathclose%
\pgfusepath{stroke,fill}%
\end{pgfscope}%
\begin{pgfscope}%
\pgfpathrectangle{\pgfqpoint{0.481978in}{0.331635in}}{\pgfqpoint{9.300000in}{7.700000in}}%
\pgfusepath{clip}%
\pgfsetbuttcap%
\pgfsetroundjoin%
\definecolor{currentfill}{rgb}{0.631373,0.788235,0.956863}%
\pgfsetfillcolor{currentfill}%
\pgfsetlinewidth{0.481800pt}%
\definecolor{currentstroke}{rgb}{1.000000,1.000000,1.000000}%
\pgfsetstrokecolor{currentstroke}%
\pgfsetdash{}{0pt}%
\pgfpathmoveto{\pgfqpoint{6.644743in}{0.639968in}}%
\pgfpathcurveto{\pgfqpoint{6.655793in}{0.639968in}}{\pgfqpoint{6.666392in}{0.644359in}}{\pgfqpoint{6.674206in}{0.652172in}}%
\pgfpathcurveto{\pgfqpoint{6.682019in}{0.659986in}}{\pgfqpoint{6.686409in}{0.670585in}}{\pgfqpoint{6.686409in}{0.681635in}}%
\pgfpathcurveto{\pgfqpoint{6.686409in}{0.692685in}}{\pgfqpoint{6.682019in}{0.703284in}}{\pgfqpoint{6.674206in}{0.711098in}}%
\pgfpathcurveto{\pgfqpoint{6.666392in}{0.718911in}}{\pgfqpoint{6.655793in}{0.723302in}}{\pgfqpoint{6.644743in}{0.723302in}}%
\pgfpathcurveto{\pgfqpoint{6.633693in}{0.723302in}}{\pgfqpoint{6.623094in}{0.718911in}}{\pgfqpoint{6.615280in}{0.711098in}}%
\pgfpathcurveto{\pgfqpoint{6.607466in}{0.703284in}}{\pgfqpoint{6.603076in}{0.692685in}}{\pgfqpoint{6.603076in}{0.681635in}}%
\pgfpathcurveto{\pgfqpoint{6.603076in}{0.670585in}}{\pgfqpoint{6.607466in}{0.659986in}}{\pgfqpoint{6.615280in}{0.652172in}}%
\pgfpathcurveto{\pgfqpoint{6.623094in}{0.644359in}}{\pgfqpoint{6.633693in}{0.639968in}}{\pgfqpoint{6.644743in}{0.639968in}}%
\pgfpathclose%
\pgfusepath{stroke,fill}%
\end{pgfscope}%
\begin{pgfscope}%
\pgfpathrectangle{\pgfqpoint{0.481978in}{0.331635in}}{\pgfqpoint{9.300000in}{7.700000in}}%
\pgfusepath{clip}%
\pgfsetbuttcap%
\pgfsetroundjoin%
\definecolor{currentfill}{rgb}{0.631373,0.788235,0.956863}%
\pgfsetfillcolor{currentfill}%
\pgfsetlinewidth{0.481800pt}%
\definecolor{currentstroke}{rgb}{1.000000,1.000000,1.000000}%
\pgfsetstrokecolor{currentstroke}%
\pgfsetdash{}{0pt}%
\pgfpathmoveto{\pgfqpoint{2.360953in}{1.944615in}}%
\pgfpathcurveto{\pgfqpoint{2.372003in}{1.944615in}}{\pgfqpoint{2.382602in}{1.949006in}}{\pgfqpoint{2.390416in}{1.956819in}}%
\pgfpathcurveto{\pgfqpoint{2.398230in}{1.964633in}}{\pgfqpoint{2.402620in}{1.975232in}}{\pgfqpoint{2.402620in}{1.986282in}}%
\pgfpathcurveto{\pgfqpoint{2.402620in}{1.997332in}}{\pgfqpoint{2.398230in}{2.007931in}}{\pgfqpoint{2.390416in}{2.015745in}}%
\pgfpathcurveto{\pgfqpoint{2.382602in}{2.023559in}}{\pgfqpoint{2.372003in}{2.027949in}}{\pgfqpoint{2.360953in}{2.027949in}}%
\pgfpathcurveto{\pgfqpoint{2.349903in}{2.027949in}}{\pgfqpoint{2.339304in}{2.023559in}}{\pgfqpoint{2.331490in}{2.015745in}}%
\pgfpathcurveto{\pgfqpoint{2.323677in}{2.007931in}}{\pgfqpoint{2.319286in}{1.997332in}}{\pgfqpoint{2.319286in}{1.986282in}}%
\pgfpathcurveto{\pgfqpoint{2.319286in}{1.975232in}}{\pgfqpoint{2.323677in}{1.964633in}}{\pgfqpoint{2.331490in}{1.956819in}}%
\pgfpathcurveto{\pgfqpoint{2.339304in}{1.949006in}}{\pgfqpoint{2.349903in}{1.944615in}}{\pgfqpoint{2.360953in}{1.944615in}}%
\pgfpathclose%
\pgfusepath{stroke,fill}%
\end{pgfscope}%
\begin{pgfscope}%
\pgfpathrectangle{\pgfqpoint{0.481978in}{0.331635in}}{\pgfqpoint{9.300000in}{7.700000in}}%
\pgfusepath{clip}%
\pgfsetbuttcap%
\pgfsetroundjoin%
\definecolor{currentfill}{rgb}{0.631373,0.788235,0.956863}%
\pgfsetfillcolor{currentfill}%
\pgfsetlinewidth{0.481800pt}%
\definecolor{currentstroke}{rgb}{1.000000,1.000000,1.000000}%
\pgfsetstrokecolor{currentstroke}%
\pgfsetdash{}{0pt}%
\pgfpathmoveto{\pgfqpoint{3.463144in}{2.136651in}}%
\pgfpathcurveto{\pgfqpoint{3.474194in}{2.136651in}}{\pgfqpoint{3.484793in}{2.141041in}}{\pgfqpoint{3.492606in}{2.148855in}}%
\pgfpathcurveto{\pgfqpoint{3.500420in}{2.156668in}}{\pgfqpoint{3.504810in}{2.167267in}}{\pgfqpoint{3.504810in}{2.178317in}}%
\pgfpathcurveto{\pgfqpoint{3.504810in}{2.189368in}}{\pgfqpoint{3.500420in}{2.199967in}}{\pgfqpoint{3.492606in}{2.207780in}}%
\pgfpathcurveto{\pgfqpoint{3.484793in}{2.215594in}}{\pgfqpoint{3.474194in}{2.219984in}}{\pgfqpoint{3.463144in}{2.219984in}}%
\pgfpathcurveto{\pgfqpoint{3.452094in}{2.219984in}}{\pgfqpoint{3.441495in}{2.215594in}}{\pgfqpoint{3.433681in}{2.207780in}}%
\pgfpathcurveto{\pgfqpoint{3.425867in}{2.199967in}}{\pgfqpoint{3.421477in}{2.189368in}}{\pgfqpoint{3.421477in}{2.178317in}}%
\pgfpathcurveto{\pgfqpoint{3.421477in}{2.167267in}}{\pgfqpoint{3.425867in}{2.156668in}}{\pgfqpoint{3.433681in}{2.148855in}}%
\pgfpathcurveto{\pgfqpoint{3.441495in}{2.141041in}}{\pgfqpoint{3.452094in}{2.136651in}}{\pgfqpoint{3.463144in}{2.136651in}}%
\pgfpathclose%
\pgfusepath{stroke,fill}%
\end{pgfscope}%
\begin{pgfscope}%
\pgfpathrectangle{\pgfqpoint{0.481978in}{0.331635in}}{\pgfqpoint{9.300000in}{7.700000in}}%
\pgfusepath{clip}%
\pgfsetbuttcap%
\pgfsetroundjoin%
\definecolor{currentfill}{rgb}{0.631373,0.788235,0.956863}%
\pgfsetfillcolor{currentfill}%
\pgfsetlinewidth{0.481800pt}%
\definecolor{currentstroke}{rgb}{1.000000,1.000000,1.000000}%
\pgfsetstrokecolor{currentstroke}%
\pgfsetdash{}{0pt}%
\pgfpathmoveto{\pgfqpoint{8.681814in}{2.091340in}}%
\pgfpathcurveto{\pgfqpoint{8.692865in}{2.091340in}}{\pgfqpoint{8.703464in}{2.095730in}}{\pgfqpoint{8.711277in}{2.103544in}}%
\pgfpathcurveto{\pgfqpoint{8.719091in}{2.111357in}}{\pgfqpoint{8.723481in}{2.121956in}}{\pgfqpoint{8.723481in}{2.133007in}}%
\pgfpathcurveto{\pgfqpoint{8.723481in}{2.144057in}}{\pgfqpoint{8.719091in}{2.154656in}}{\pgfqpoint{8.711277in}{2.162469in}}%
\pgfpathcurveto{\pgfqpoint{8.703464in}{2.170283in}}{\pgfqpoint{8.692865in}{2.174673in}}{\pgfqpoint{8.681814in}{2.174673in}}%
\pgfpathcurveto{\pgfqpoint{8.670764in}{2.174673in}}{\pgfqpoint{8.660165in}{2.170283in}}{\pgfqpoint{8.652352in}{2.162469in}}%
\pgfpathcurveto{\pgfqpoint{8.644538in}{2.154656in}}{\pgfqpoint{8.640148in}{2.144057in}}{\pgfqpoint{8.640148in}{2.133007in}}%
\pgfpathcurveto{\pgfqpoint{8.640148in}{2.121956in}}{\pgfqpoint{8.644538in}{2.111357in}}{\pgfqpoint{8.652352in}{2.103544in}}%
\pgfpathcurveto{\pgfqpoint{8.660165in}{2.095730in}}{\pgfqpoint{8.670764in}{2.091340in}}{\pgfqpoint{8.681814in}{2.091340in}}%
\pgfpathclose%
\pgfusepath{stroke,fill}%
\end{pgfscope}%
\begin{pgfscope}%
\pgfpathrectangle{\pgfqpoint{0.481978in}{0.331635in}}{\pgfqpoint{9.300000in}{7.700000in}}%
\pgfusepath{clip}%
\pgfsetbuttcap%
\pgfsetroundjoin%
\definecolor{currentfill}{rgb}{0.631373,0.788235,0.956863}%
\pgfsetfillcolor{currentfill}%
\pgfsetlinewidth{0.481800pt}%
\definecolor{currentstroke}{rgb}{1.000000,1.000000,1.000000}%
\pgfsetstrokecolor{currentstroke}%
\pgfsetdash{}{0pt}%
\pgfpathmoveto{\pgfqpoint{5.188038in}{5.085722in}}%
\pgfpathcurveto{\pgfqpoint{5.199089in}{5.085722in}}{\pgfqpoint{5.209688in}{5.090113in}}{\pgfqpoint{5.217501in}{5.097926in}}%
\pgfpathcurveto{\pgfqpoint{5.225315in}{5.105740in}}{\pgfqpoint{5.229705in}{5.116339in}}{\pgfqpoint{5.229705in}{5.127389in}}%
\pgfpathcurveto{\pgfqpoint{5.229705in}{5.138439in}}{\pgfqpoint{5.225315in}{5.149038in}}{\pgfqpoint{5.217501in}{5.156852in}}%
\pgfpathcurveto{\pgfqpoint{5.209688in}{5.164666in}}{\pgfqpoint{5.199089in}{5.169056in}}{\pgfqpoint{5.188038in}{5.169056in}}%
\pgfpathcurveto{\pgfqpoint{5.176988in}{5.169056in}}{\pgfqpoint{5.166389in}{5.164666in}}{\pgfqpoint{5.158576in}{5.156852in}}%
\pgfpathcurveto{\pgfqpoint{5.150762in}{5.149038in}}{\pgfqpoint{5.146372in}{5.138439in}}{\pgfqpoint{5.146372in}{5.127389in}}%
\pgfpathcurveto{\pgfqpoint{5.146372in}{5.116339in}}{\pgfqpoint{5.150762in}{5.105740in}}{\pgfqpoint{5.158576in}{5.097926in}}%
\pgfpathcurveto{\pgfqpoint{5.166389in}{5.090113in}}{\pgfqpoint{5.176988in}{5.085722in}}{\pgfqpoint{5.188038in}{5.085722in}}%
\pgfpathclose%
\pgfusepath{stroke,fill}%
\end{pgfscope}%
\begin{pgfscope}%
\pgfpathrectangle{\pgfqpoint{0.481978in}{0.331635in}}{\pgfqpoint{9.300000in}{7.700000in}}%
\pgfusepath{clip}%
\pgfsetbuttcap%
\pgfsetroundjoin%
\definecolor{currentfill}{rgb}{0.631373,0.788235,0.956863}%
\pgfsetfillcolor{currentfill}%
\pgfsetlinewidth{0.481800pt}%
\definecolor{currentstroke}{rgb}{1.000000,1.000000,1.000000}%
\pgfsetstrokecolor{currentstroke}%
\pgfsetdash{}{0pt}%
\pgfpathmoveto{\pgfqpoint{2.109676in}{2.935946in}}%
\pgfpathcurveto{\pgfqpoint{2.120726in}{2.935946in}}{\pgfqpoint{2.131325in}{2.940336in}}{\pgfqpoint{2.139139in}{2.948150in}}%
\pgfpathcurveto{\pgfqpoint{2.146952in}{2.955964in}}{\pgfqpoint{2.151342in}{2.966563in}}{\pgfqpoint{2.151342in}{2.977613in}}%
\pgfpathcurveto{\pgfqpoint{2.151342in}{2.988663in}}{\pgfqpoint{2.146952in}{2.999262in}}{\pgfqpoint{2.139139in}{3.007076in}}%
\pgfpathcurveto{\pgfqpoint{2.131325in}{3.014889in}}{\pgfqpoint{2.120726in}{3.019279in}}{\pgfqpoint{2.109676in}{3.019279in}}%
\pgfpathcurveto{\pgfqpoint{2.098626in}{3.019279in}}{\pgfqpoint{2.088027in}{3.014889in}}{\pgfqpoint{2.080213in}{3.007076in}}%
\pgfpathcurveto{\pgfqpoint{2.072399in}{2.999262in}}{\pgfqpoint{2.068009in}{2.988663in}}{\pgfqpoint{2.068009in}{2.977613in}}%
\pgfpathcurveto{\pgfqpoint{2.068009in}{2.966563in}}{\pgfqpoint{2.072399in}{2.955964in}}{\pgfqpoint{2.080213in}{2.948150in}}%
\pgfpathcurveto{\pgfqpoint{2.088027in}{2.940336in}}{\pgfqpoint{2.098626in}{2.935946in}}{\pgfqpoint{2.109676in}{2.935946in}}%
\pgfpathclose%
\pgfusepath{stroke,fill}%
\end{pgfscope}%
\begin{pgfscope}%
\pgfpathrectangle{\pgfqpoint{0.481978in}{0.331635in}}{\pgfqpoint{9.300000in}{7.700000in}}%
\pgfusepath{clip}%
\pgfsetbuttcap%
\pgfsetroundjoin%
\definecolor{currentfill}{rgb}{0.631373,0.788235,0.956863}%
\pgfsetfillcolor{currentfill}%
\pgfsetlinewidth{0.481800pt}%
\definecolor{currentstroke}{rgb}{1.000000,1.000000,1.000000}%
\pgfsetstrokecolor{currentstroke}%
\pgfsetdash{}{0pt}%
\pgfpathmoveto{\pgfqpoint{0.984432in}{4.164418in}}%
\pgfpathcurveto{\pgfqpoint{0.995482in}{4.164418in}}{\pgfqpoint{1.006081in}{4.168808in}}{\pgfqpoint{1.013895in}{4.176622in}}%
\pgfpathcurveto{\pgfqpoint{1.021709in}{4.184436in}}{\pgfqpoint{1.026099in}{4.195035in}}{\pgfqpoint{1.026099in}{4.206085in}}%
\pgfpathcurveto{\pgfqpoint{1.026099in}{4.217135in}}{\pgfqpoint{1.021709in}{4.227734in}}{\pgfqpoint{1.013895in}{4.235548in}}%
\pgfpathcurveto{\pgfqpoint{1.006081in}{4.243361in}}{\pgfqpoint{0.995482in}{4.247751in}}{\pgfqpoint{0.984432in}{4.247751in}}%
\pgfpathcurveto{\pgfqpoint{0.973382in}{4.247751in}}{\pgfqpoint{0.962783in}{4.243361in}}{\pgfqpoint{0.954970in}{4.235548in}}%
\pgfpathcurveto{\pgfqpoint{0.947156in}{4.227734in}}{\pgfqpoint{0.942766in}{4.217135in}}{\pgfqpoint{0.942766in}{4.206085in}}%
\pgfpathcurveto{\pgfqpoint{0.942766in}{4.195035in}}{\pgfqpoint{0.947156in}{4.184436in}}{\pgfqpoint{0.954970in}{4.176622in}}%
\pgfpathcurveto{\pgfqpoint{0.962783in}{4.168808in}}{\pgfqpoint{0.973382in}{4.164418in}}{\pgfqpoint{0.984432in}{4.164418in}}%
\pgfpathclose%
\pgfusepath{stroke,fill}%
\end{pgfscope}%
\begin{pgfscope}%
\pgfpathrectangle{\pgfqpoint{0.481978in}{0.331635in}}{\pgfqpoint{9.300000in}{7.700000in}}%
\pgfusepath{clip}%
\pgfsetbuttcap%
\pgfsetroundjoin%
\definecolor{currentfill}{rgb}{1.000000,0.705882,0.509804}%
\pgfsetfillcolor{currentfill}%
\pgfsetlinewidth{0.481800pt}%
\definecolor{currentstroke}{rgb}{1.000000,1.000000,1.000000}%
\pgfsetstrokecolor{currentstroke}%
\pgfsetdash{}{0pt}%
\pgfpathmoveto{\pgfqpoint{6.044554in}{2.136058in}}%
\pgfpathcurveto{\pgfqpoint{6.055604in}{2.136058in}}{\pgfqpoint{6.066203in}{2.140448in}}{\pgfqpoint{6.074017in}{2.148261in}}%
\pgfpathcurveto{\pgfqpoint{6.081831in}{2.156075in}}{\pgfqpoint{6.086221in}{2.166674in}}{\pgfqpoint{6.086221in}{2.177724in}}%
\pgfpathcurveto{\pgfqpoint{6.086221in}{2.188774in}}{\pgfqpoint{6.081831in}{2.199373in}}{\pgfqpoint{6.074017in}{2.207187in}}%
\pgfpathcurveto{\pgfqpoint{6.066203in}{2.215001in}}{\pgfqpoint{6.055604in}{2.219391in}}{\pgfqpoint{6.044554in}{2.219391in}}%
\pgfpathcurveto{\pgfqpoint{6.033504in}{2.219391in}}{\pgfqpoint{6.022905in}{2.215001in}}{\pgfqpoint{6.015091in}{2.207187in}}%
\pgfpathcurveto{\pgfqpoint{6.007278in}{2.199373in}}{\pgfqpoint{6.002887in}{2.188774in}}{\pgfqpoint{6.002887in}{2.177724in}}%
\pgfpathcurveto{\pgfqpoint{6.002887in}{2.166674in}}{\pgfqpoint{6.007278in}{2.156075in}}{\pgfqpoint{6.015091in}{2.148261in}}%
\pgfpathcurveto{\pgfqpoint{6.022905in}{2.140448in}}{\pgfqpoint{6.033504in}{2.136058in}}{\pgfqpoint{6.044554in}{2.136058in}}%
\pgfpathclose%
\pgfusepath{stroke,fill}%
\end{pgfscope}%
\begin{pgfscope}%
\pgfpathrectangle{\pgfqpoint{0.481978in}{0.331635in}}{\pgfqpoint{9.300000in}{7.700000in}}%
\pgfusepath{clip}%
\pgfsetbuttcap%
\pgfsetroundjoin%
\definecolor{currentfill}{rgb}{1.000000,0.705882,0.509804}%
\pgfsetfillcolor{currentfill}%
\pgfsetlinewidth{0.481800pt}%
\definecolor{currentstroke}{rgb}{1.000000,1.000000,1.000000}%
\pgfsetstrokecolor{currentstroke}%
\pgfsetdash{}{0pt}%
\pgfpathmoveto{\pgfqpoint{4.376435in}{1.830640in}}%
\pgfpathcurveto{\pgfqpoint{4.387485in}{1.830640in}}{\pgfqpoint{4.398084in}{1.835030in}}{\pgfqpoint{4.405897in}{1.842844in}}%
\pgfpathcurveto{\pgfqpoint{4.413711in}{1.850658in}}{\pgfqpoint{4.418101in}{1.861257in}}{\pgfqpoint{4.418101in}{1.872307in}}%
\pgfpathcurveto{\pgfqpoint{4.418101in}{1.883357in}}{\pgfqpoint{4.413711in}{1.893956in}}{\pgfqpoint{4.405897in}{1.901770in}}%
\pgfpathcurveto{\pgfqpoint{4.398084in}{1.909583in}}{\pgfqpoint{4.387485in}{1.913974in}}{\pgfqpoint{4.376435in}{1.913974in}}%
\pgfpathcurveto{\pgfqpoint{4.365385in}{1.913974in}}{\pgfqpoint{4.354785in}{1.909583in}}{\pgfqpoint{4.346972in}{1.901770in}}%
\pgfpathcurveto{\pgfqpoint{4.339158in}{1.893956in}}{\pgfqpoint{4.334768in}{1.883357in}}{\pgfqpoint{4.334768in}{1.872307in}}%
\pgfpathcurveto{\pgfqpoint{4.334768in}{1.861257in}}{\pgfqpoint{4.339158in}{1.850658in}}{\pgfqpoint{4.346972in}{1.842844in}}%
\pgfpathcurveto{\pgfqpoint{4.354785in}{1.835030in}}{\pgfqpoint{4.365385in}{1.830640in}}{\pgfqpoint{4.376435in}{1.830640in}}%
\pgfpathclose%
\pgfusepath{stroke,fill}%
\end{pgfscope}%
\begin{pgfscope}%
\pgfpathrectangle{\pgfqpoint{0.481978in}{0.331635in}}{\pgfqpoint{9.300000in}{7.700000in}}%
\pgfusepath{clip}%
\pgfsetbuttcap%
\pgfsetroundjoin%
\definecolor{currentfill}{rgb}{1.000000,0.705882,0.509804}%
\pgfsetfillcolor{currentfill}%
\pgfsetlinewidth{0.481800pt}%
\definecolor{currentstroke}{rgb}{1.000000,1.000000,1.000000}%
\pgfsetstrokecolor{currentstroke}%
\pgfsetdash{}{0pt}%
\pgfpathmoveto{\pgfqpoint{8.943163in}{5.720288in}}%
\pgfpathcurveto{\pgfqpoint{8.954213in}{5.720288in}}{\pgfqpoint{8.964812in}{5.724678in}}{\pgfqpoint{8.972626in}{5.732492in}}%
\pgfpathcurveto{\pgfqpoint{8.980439in}{5.740305in}}{\pgfqpoint{8.984830in}{5.750904in}}{\pgfqpoint{8.984830in}{5.761954in}}%
\pgfpathcurveto{\pgfqpoint{8.984830in}{5.773005in}}{\pgfqpoint{8.980439in}{5.783604in}}{\pgfqpoint{8.972626in}{5.791417in}}%
\pgfpathcurveto{\pgfqpoint{8.964812in}{5.799231in}}{\pgfqpoint{8.954213in}{5.803621in}}{\pgfqpoint{8.943163in}{5.803621in}}%
\pgfpathcurveto{\pgfqpoint{8.932113in}{5.803621in}}{\pgfqpoint{8.921514in}{5.799231in}}{\pgfqpoint{8.913700in}{5.791417in}}%
\pgfpathcurveto{\pgfqpoint{8.905887in}{5.783604in}}{\pgfqpoint{8.901496in}{5.773005in}}{\pgfqpoint{8.901496in}{5.761954in}}%
\pgfpathcurveto{\pgfqpoint{8.901496in}{5.750904in}}{\pgfqpoint{8.905887in}{5.740305in}}{\pgfqpoint{8.913700in}{5.732492in}}%
\pgfpathcurveto{\pgfqpoint{8.921514in}{5.724678in}}{\pgfqpoint{8.932113in}{5.720288in}}{\pgfqpoint{8.943163in}{5.720288in}}%
\pgfpathclose%
\pgfusepath{stroke,fill}%
\end{pgfscope}%
\begin{pgfscope}%
\pgfpathrectangle{\pgfqpoint{0.481978in}{0.331635in}}{\pgfqpoint{9.300000in}{7.700000in}}%
\pgfusepath{clip}%
\pgfsetbuttcap%
\pgfsetroundjoin%
\definecolor{currentfill}{rgb}{1.000000,0.705882,0.509804}%
\pgfsetfillcolor{currentfill}%
\pgfsetlinewidth{0.481800pt}%
\definecolor{currentstroke}{rgb}{1.000000,1.000000,1.000000}%
\pgfsetstrokecolor{currentstroke}%
\pgfsetdash{}{0pt}%
\pgfpathmoveto{\pgfqpoint{5.919865in}{2.990099in}}%
\pgfpathcurveto{\pgfqpoint{5.930915in}{2.990099in}}{\pgfqpoint{5.941514in}{2.994489in}}{\pgfqpoint{5.949328in}{3.002303in}}%
\pgfpathcurveto{\pgfqpoint{5.957141in}{3.010117in}}{\pgfqpoint{5.961532in}{3.020716in}}{\pgfqpoint{5.961532in}{3.031766in}}%
\pgfpathcurveto{\pgfqpoint{5.961532in}{3.042816in}}{\pgfqpoint{5.957141in}{3.053415in}}{\pgfqpoint{5.949328in}{3.061229in}}%
\pgfpathcurveto{\pgfqpoint{5.941514in}{3.069042in}}{\pgfqpoint{5.930915in}{3.073432in}}{\pgfqpoint{5.919865in}{3.073432in}}%
\pgfpathcurveto{\pgfqpoint{5.908815in}{3.073432in}}{\pgfqpoint{5.898216in}{3.069042in}}{\pgfqpoint{5.890402in}{3.061229in}}%
\pgfpathcurveto{\pgfqpoint{5.882589in}{3.053415in}}{\pgfqpoint{5.878198in}{3.042816in}}{\pgfqpoint{5.878198in}{3.031766in}}%
\pgfpathcurveto{\pgfqpoint{5.878198in}{3.020716in}}{\pgfqpoint{5.882589in}{3.010117in}}{\pgfqpoint{5.890402in}{3.002303in}}%
\pgfpathcurveto{\pgfqpoint{5.898216in}{2.994489in}}{\pgfqpoint{5.908815in}{2.990099in}}{\pgfqpoint{5.919865in}{2.990099in}}%
\pgfpathclose%
\pgfusepath{stroke,fill}%
\end{pgfscope}%
\begin{pgfscope}%
\pgfpathrectangle{\pgfqpoint{0.481978in}{0.331635in}}{\pgfqpoint{9.300000in}{7.700000in}}%
\pgfusepath{clip}%
\pgfsetbuttcap%
\pgfsetroundjoin%
\definecolor{currentfill}{rgb}{1.000000,0.705882,0.509804}%
\pgfsetfillcolor{currentfill}%
\pgfsetlinewidth{0.481800pt}%
\definecolor{currentstroke}{rgb}{1.000000,1.000000,1.000000}%
\pgfsetstrokecolor{currentstroke}%
\pgfsetdash{}{0pt}%
\pgfpathmoveto{\pgfqpoint{5.035065in}{3.870271in}}%
\pgfpathcurveto{\pgfqpoint{5.046115in}{3.870271in}}{\pgfqpoint{5.056714in}{3.874661in}}{\pgfqpoint{5.064528in}{3.882475in}}%
\pgfpathcurveto{\pgfqpoint{5.072341in}{3.890289in}}{\pgfqpoint{5.076731in}{3.900888in}}{\pgfqpoint{5.076731in}{3.911938in}}%
\pgfpathcurveto{\pgfqpoint{5.076731in}{3.922988in}}{\pgfqpoint{5.072341in}{3.933587in}}{\pgfqpoint{5.064528in}{3.941401in}}%
\pgfpathcurveto{\pgfqpoint{5.056714in}{3.949214in}}{\pgfqpoint{5.046115in}{3.953605in}}{\pgfqpoint{5.035065in}{3.953605in}}%
\pgfpathcurveto{\pgfqpoint{5.024015in}{3.953605in}}{\pgfqpoint{5.013416in}{3.949214in}}{\pgfqpoint{5.005602in}{3.941401in}}%
\pgfpathcurveto{\pgfqpoint{4.997788in}{3.933587in}}{\pgfqpoint{4.993398in}{3.922988in}}{\pgfqpoint{4.993398in}{3.911938in}}%
\pgfpathcurveto{\pgfqpoint{4.993398in}{3.900888in}}{\pgfqpoint{4.997788in}{3.890289in}}{\pgfqpoint{5.005602in}{3.882475in}}%
\pgfpathcurveto{\pgfqpoint{5.013416in}{3.874661in}}{\pgfqpoint{5.024015in}{3.870271in}}{\pgfqpoint{5.035065in}{3.870271in}}%
\pgfpathclose%
\pgfusepath{stroke,fill}%
\end{pgfscope}%
\begin{pgfscope}%
\pgfpathrectangle{\pgfqpoint{0.481978in}{0.331635in}}{\pgfqpoint{9.300000in}{7.700000in}}%
\pgfusepath{clip}%
\pgfsetbuttcap%
\pgfsetroundjoin%
\definecolor{currentfill}{rgb}{1.000000,0.705882,0.509804}%
\pgfsetfillcolor{currentfill}%
\pgfsetlinewidth{0.481800pt}%
\definecolor{currentstroke}{rgb}{1.000000,1.000000,1.000000}%
\pgfsetstrokecolor{currentstroke}%
\pgfsetdash{}{0pt}%
\pgfpathmoveto{\pgfqpoint{7.526077in}{7.039392in}}%
\pgfpathcurveto{\pgfqpoint{7.537127in}{7.039392in}}{\pgfqpoint{7.547726in}{7.043782in}}{\pgfqpoint{7.555540in}{7.051596in}}%
\pgfpathcurveto{\pgfqpoint{7.563353in}{7.059409in}}{\pgfqpoint{7.567744in}{7.070008in}}{\pgfqpoint{7.567744in}{7.081058in}}%
\pgfpathcurveto{\pgfqpoint{7.567744in}{7.092109in}}{\pgfqpoint{7.563353in}{7.102708in}}{\pgfqpoint{7.555540in}{7.110521in}}%
\pgfpathcurveto{\pgfqpoint{7.547726in}{7.118335in}}{\pgfqpoint{7.537127in}{7.122725in}}{\pgfqpoint{7.526077in}{7.122725in}}%
\pgfpathcurveto{\pgfqpoint{7.515027in}{7.122725in}}{\pgfqpoint{7.504428in}{7.118335in}}{\pgfqpoint{7.496614in}{7.110521in}}%
\pgfpathcurveto{\pgfqpoint{7.488800in}{7.102708in}}{\pgfqpoint{7.484410in}{7.092109in}}{\pgfqpoint{7.484410in}{7.081058in}}%
\pgfpathcurveto{\pgfqpoint{7.484410in}{7.070008in}}{\pgfqpoint{7.488800in}{7.059409in}}{\pgfqpoint{7.496614in}{7.051596in}}%
\pgfpathcurveto{\pgfqpoint{7.504428in}{7.043782in}}{\pgfqpoint{7.515027in}{7.039392in}}{\pgfqpoint{7.526077in}{7.039392in}}%
\pgfpathclose%
\pgfusepath{stroke,fill}%
\end{pgfscope}%
\begin{pgfscope}%
\pgfpathrectangle{\pgfqpoint{0.481978in}{0.331635in}}{\pgfqpoint{9.300000in}{7.700000in}}%
\pgfusepath{clip}%
\pgfsetbuttcap%
\pgfsetroundjoin%
\definecolor{currentfill}{rgb}{1.000000,0.705882,0.509804}%
\pgfsetfillcolor{currentfill}%
\pgfsetlinewidth{0.481800pt}%
\definecolor{currentstroke}{rgb}{1.000000,1.000000,1.000000}%
\pgfsetstrokecolor{currentstroke}%
\pgfsetdash{}{0pt}%
\pgfpathmoveto{\pgfqpoint{6.779267in}{7.330326in}}%
\pgfpathcurveto{\pgfqpoint{6.790317in}{7.330326in}}{\pgfqpoint{6.800916in}{7.334717in}}{\pgfqpoint{6.808730in}{7.342530in}}%
\pgfpathcurveto{\pgfqpoint{6.816544in}{7.350344in}}{\pgfqpoint{6.820934in}{7.360943in}}{\pgfqpoint{6.820934in}{7.371993in}}%
\pgfpathcurveto{\pgfqpoint{6.820934in}{7.383043in}}{\pgfqpoint{6.816544in}{7.393642in}}{\pgfqpoint{6.808730in}{7.401456in}}%
\pgfpathcurveto{\pgfqpoint{6.800916in}{7.409269in}}{\pgfqpoint{6.790317in}{7.413660in}}{\pgfqpoint{6.779267in}{7.413660in}}%
\pgfpathcurveto{\pgfqpoint{6.768217in}{7.413660in}}{\pgfqpoint{6.757618in}{7.409269in}}{\pgfqpoint{6.749804in}{7.401456in}}%
\pgfpathcurveto{\pgfqpoint{6.741991in}{7.393642in}}{\pgfqpoint{6.737600in}{7.383043in}}{\pgfqpoint{6.737600in}{7.371993in}}%
\pgfpathcurveto{\pgfqpoint{6.737600in}{7.360943in}}{\pgfqpoint{6.741991in}{7.350344in}}{\pgfqpoint{6.749804in}{7.342530in}}%
\pgfpathcurveto{\pgfqpoint{6.757618in}{7.334717in}}{\pgfqpoint{6.768217in}{7.330326in}}{\pgfqpoint{6.779267in}{7.330326in}}%
\pgfpathclose%
\pgfusepath{stroke,fill}%
\end{pgfscope}%
\begin{pgfscope}%
\pgfpathrectangle{\pgfqpoint{0.481978in}{0.331635in}}{\pgfqpoint{9.300000in}{7.700000in}}%
\pgfusepath{clip}%
\pgfsetbuttcap%
\pgfsetroundjoin%
\definecolor{currentfill}{rgb}{1.000000,0.705882,0.509804}%
\pgfsetfillcolor{currentfill}%
\pgfsetlinewidth{0.481800pt}%
\definecolor{currentstroke}{rgb}{1.000000,1.000000,1.000000}%
\pgfsetstrokecolor{currentstroke}%
\pgfsetdash{}{0pt}%
\pgfpathmoveto{\pgfqpoint{2.475590in}{6.035520in}}%
\pgfpathcurveto{\pgfqpoint{2.486640in}{6.035520in}}{\pgfqpoint{2.497239in}{6.039910in}}{\pgfqpoint{2.505053in}{6.047723in}}%
\pgfpathcurveto{\pgfqpoint{2.512866in}{6.055537in}}{\pgfqpoint{2.517257in}{6.066136in}}{\pgfqpoint{2.517257in}{6.077186in}}%
\pgfpathcurveto{\pgfqpoint{2.517257in}{6.088236in}}{\pgfqpoint{2.512866in}{6.098835in}}{\pgfqpoint{2.505053in}{6.106649in}}%
\pgfpathcurveto{\pgfqpoint{2.497239in}{6.114463in}}{\pgfqpoint{2.486640in}{6.118853in}}{\pgfqpoint{2.475590in}{6.118853in}}%
\pgfpathcurveto{\pgfqpoint{2.464540in}{6.118853in}}{\pgfqpoint{2.453941in}{6.114463in}}{\pgfqpoint{2.446127in}{6.106649in}}%
\pgfpathcurveto{\pgfqpoint{2.438314in}{6.098835in}}{\pgfqpoint{2.433923in}{6.088236in}}{\pgfqpoint{2.433923in}{6.077186in}}%
\pgfpathcurveto{\pgfqpoint{2.433923in}{6.066136in}}{\pgfqpoint{2.438314in}{6.055537in}}{\pgfqpoint{2.446127in}{6.047723in}}%
\pgfpathcurveto{\pgfqpoint{2.453941in}{6.039910in}}{\pgfqpoint{2.464540in}{6.035520in}}{\pgfqpoint{2.475590in}{6.035520in}}%
\pgfpathclose%
\pgfusepath{stroke,fill}%
\end{pgfscope}%
\begin{pgfscope}%
\pgfpathrectangle{\pgfqpoint{0.481978in}{0.331635in}}{\pgfqpoint{9.300000in}{7.700000in}}%
\pgfusepath{clip}%
\pgfsetbuttcap%
\pgfsetroundjoin%
\definecolor{currentfill}{rgb}{1.000000,0.705882,0.509804}%
\pgfsetfillcolor{currentfill}%
\pgfsetlinewidth{0.481800pt}%
\definecolor{currentstroke}{rgb}{1.000000,1.000000,1.000000}%
\pgfsetstrokecolor{currentstroke}%
\pgfsetdash{}{0pt}%
\pgfpathmoveto{\pgfqpoint{3.242777in}{3.542174in}}%
\pgfpathcurveto{\pgfqpoint{3.253827in}{3.542174in}}{\pgfqpoint{3.264426in}{3.546564in}}{\pgfqpoint{3.272240in}{3.554378in}}%
\pgfpathcurveto{\pgfqpoint{3.280054in}{3.562192in}}{\pgfqpoint{3.284444in}{3.572791in}}{\pgfqpoint{3.284444in}{3.583841in}}%
\pgfpathcurveto{\pgfqpoint{3.284444in}{3.594891in}}{\pgfqpoint{3.280054in}{3.605490in}}{\pgfqpoint{3.272240in}{3.613304in}}%
\pgfpathcurveto{\pgfqpoint{3.264426in}{3.621117in}}{\pgfqpoint{3.253827in}{3.625507in}}{\pgfqpoint{3.242777in}{3.625507in}}%
\pgfpathcurveto{\pgfqpoint{3.231727in}{3.625507in}}{\pgfqpoint{3.221128in}{3.621117in}}{\pgfqpoint{3.213314in}{3.613304in}}%
\pgfpathcurveto{\pgfqpoint{3.205501in}{3.605490in}}{\pgfqpoint{3.201111in}{3.594891in}}{\pgfqpoint{3.201111in}{3.583841in}}%
\pgfpathcurveto{\pgfqpoint{3.201111in}{3.572791in}}{\pgfqpoint{3.205501in}{3.562192in}}{\pgfqpoint{3.213314in}{3.554378in}}%
\pgfpathcurveto{\pgfqpoint{3.221128in}{3.546564in}}{\pgfqpoint{3.231727in}{3.542174in}}{\pgfqpoint{3.242777in}{3.542174in}}%
\pgfpathclose%
\pgfusepath{stroke,fill}%
\end{pgfscope}%
\begin{pgfscope}%
\pgfpathrectangle{\pgfqpoint{0.481978in}{0.331635in}}{\pgfqpoint{9.300000in}{7.700000in}}%
\pgfusepath{clip}%
\pgfsetbuttcap%
\pgfsetroundjoin%
\definecolor{currentfill}{rgb}{1.000000,0.705882,0.509804}%
\pgfsetfillcolor{currentfill}%
\pgfsetlinewidth{0.481800pt}%
\definecolor{currentstroke}{rgb}{1.000000,1.000000,1.000000}%
\pgfsetstrokecolor{currentstroke}%
\pgfsetdash{}{0pt}%
\pgfpathmoveto{\pgfqpoint{4.227999in}{4.356170in}}%
\pgfpathcurveto{\pgfqpoint{4.239049in}{4.356170in}}{\pgfqpoint{4.249648in}{4.360560in}}{\pgfqpoint{4.257461in}{4.368374in}}%
\pgfpathcurveto{\pgfqpoint{4.265275in}{4.376187in}}{\pgfqpoint{4.269665in}{4.386786in}}{\pgfqpoint{4.269665in}{4.397836in}}%
\pgfpathcurveto{\pgfqpoint{4.269665in}{4.408886in}}{\pgfqpoint{4.265275in}{4.419486in}}{\pgfqpoint{4.257461in}{4.427299in}}%
\pgfpathcurveto{\pgfqpoint{4.249648in}{4.435113in}}{\pgfqpoint{4.239049in}{4.439503in}}{\pgfqpoint{4.227999in}{4.439503in}}%
\pgfpathcurveto{\pgfqpoint{4.216948in}{4.439503in}}{\pgfqpoint{4.206349in}{4.435113in}}{\pgfqpoint{4.198536in}{4.427299in}}%
\pgfpathcurveto{\pgfqpoint{4.190722in}{4.419486in}}{\pgfqpoint{4.186332in}{4.408886in}}{\pgfqpoint{4.186332in}{4.397836in}}%
\pgfpathcurveto{\pgfqpoint{4.186332in}{4.386786in}}{\pgfqpoint{4.190722in}{4.376187in}}{\pgfqpoint{4.198536in}{4.368374in}}%
\pgfpathcurveto{\pgfqpoint{4.206349in}{4.360560in}}{\pgfqpoint{4.216948in}{4.356170in}}{\pgfqpoint{4.227999in}{4.356170in}}%
\pgfpathclose%
\pgfusepath{stroke,fill}%
\end{pgfscope}%
\begin{pgfscope}%
\pgfpathrectangle{\pgfqpoint{0.481978in}{0.331635in}}{\pgfqpoint{9.300000in}{7.700000in}}%
\pgfusepath{clip}%
\pgfsetbuttcap%
\pgfsetroundjoin%
\definecolor{currentfill}{rgb}{1.000000,0.705882,0.509804}%
\pgfsetfillcolor{currentfill}%
\pgfsetlinewidth{0.481800pt}%
\definecolor{currentstroke}{rgb}{1.000000,1.000000,1.000000}%
\pgfsetstrokecolor{currentstroke}%
\pgfsetdash{}{0pt}%
\pgfpathmoveto{\pgfqpoint{3.413419in}{5.316706in}}%
\pgfpathcurveto{\pgfqpoint{3.424469in}{5.316706in}}{\pgfqpoint{3.435068in}{5.321096in}}{\pgfqpoint{3.442882in}{5.328909in}}%
\pgfpathcurveto{\pgfqpoint{3.450695in}{5.336723in}}{\pgfqpoint{3.455086in}{5.347322in}}{\pgfqpoint{3.455086in}{5.358372in}}%
\pgfpathcurveto{\pgfqpoint{3.455086in}{5.369422in}}{\pgfqpoint{3.450695in}{5.380021in}}{\pgfqpoint{3.442882in}{5.387835in}}%
\pgfpathcurveto{\pgfqpoint{3.435068in}{5.395649in}}{\pgfqpoint{3.424469in}{5.400039in}}{\pgfqpoint{3.413419in}{5.400039in}}%
\pgfpathcurveto{\pgfqpoint{3.402369in}{5.400039in}}{\pgfqpoint{3.391770in}{5.395649in}}{\pgfqpoint{3.383956in}{5.387835in}}%
\pgfpathcurveto{\pgfqpoint{3.376142in}{5.380021in}}{\pgfqpoint{3.371752in}{5.369422in}}{\pgfqpoint{3.371752in}{5.358372in}}%
\pgfpathcurveto{\pgfqpoint{3.371752in}{5.347322in}}{\pgfqpoint{3.376142in}{5.336723in}}{\pgfqpoint{3.383956in}{5.328909in}}%
\pgfpathcurveto{\pgfqpoint{3.391770in}{5.321096in}}{\pgfqpoint{3.402369in}{5.316706in}}{\pgfqpoint{3.413419in}{5.316706in}}%
\pgfpathclose%
\pgfusepath{stroke,fill}%
\end{pgfscope}%
\begin{pgfscope}%
\pgfpathrectangle{\pgfqpoint{0.481978in}{0.331635in}}{\pgfqpoint{9.300000in}{7.700000in}}%
\pgfusepath{clip}%
\pgfsetbuttcap%
\pgfsetroundjoin%
\definecolor{currentfill}{rgb}{1.000000,0.705882,0.509804}%
\pgfsetfillcolor{currentfill}%
\pgfsetlinewidth{0.481800pt}%
\definecolor{currentstroke}{rgb}{1.000000,1.000000,1.000000}%
\pgfsetstrokecolor{currentstroke}%
\pgfsetdash{}{0pt}%
\pgfpathmoveto{\pgfqpoint{3.930822in}{5.610835in}}%
\pgfpathcurveto{\pgfqpoint{3.941872in}{5.610835in}}{\pgfqpoint{3.952471in}{5.615226in}}{\pgfqpoint{3.960285in}{5.623039in}}%
\pgfpathcurveto{\pgfqpoint{3.968099in}{5.630853in}}{\pgfqpoint{3.972489in}{5.641452in}}{\pgfqpoint{3.972489in}{5.652502in}}%
\pgfpathcurveto{\pgfqpoint{3.972489in}{5.663552in}}{\pgfqpoint{3.968099in}{5.674151in}}{\pgfqpoint{3.960285in}{5.681965in}}%
\pgfpathcurveto{\pgfqpoint{3.952471in}{5.689778in}}{\pgfqpoint{3.941872in}{5.694169in}}{\pgfqpoint{3.930822in}{5.694169in}}%
\pgfpathcurveto{\pgfqpoint{3.919772in}{5.694169in}}{\pgfqpoint{3.909173in}{5.689778in}}{\pgfqpoint{3.901359in}{5.681965in}}%
\pgfpathcurveto{\pgfqpoint{3.893546in}{5.674151in}}{\pgfqpoint{3.889156in}{5.663552in}}{\pgfqpoint{3.889156in}{5.652502in}}%
\pgfpathcurveto{\pgfqpoint{3.889156in}{5.641452in}}{\pgfqpoint{3.893546in}{5.630853in}}{\pgfqpoint{3.901359in}{5.623039in}}%
\pgfpathcurveto{\pgfqpoint{3.909173in}{5.615226in}}{\pgfqpoint{3.919772in}{5.610835in}}{\pgfqpoint{3.930822in}{5.610835in}}%
\pgfpathclose%
\pgfusepath{stroke,fill}%
\end{pgfscope}%
\begin{pgfscope}%
\pgfpathrectangle{\pgfqpoint{0.481978in}{0.331635in}}{\pgfqpoint{9.300000in}{7.700000in}}%
\pgfusepath{clip}%
\pgfsetbuttcap%
\pgfsetroundjoin%
\definecolor{currentfill}{rgb}{1.000000,0.705882,0.509804}%
\pgfsetfillcolor{currentfill}%
\pgfsetlinewidth{0.481800pt}%
\definecolor{currentstroke}{rgb}{1.000000,1.000000,1.000000}%
\pgfsetstrokecolor{currentstroke}%
\pgfsetdash{}{0pt}%
\pgfpathmoveto{\pgfqpoint{3.950735in}{3.042884in}}%
\pgfpathcurveto{\pgfqpoint{3.961785in}{3.042884in}}{\pgfqpoint{3.972384in}{3.047275in}}{\pgfqpoint{3.980198in}{3.055088in}}%
\pgfpathcurveto{\pgfqpoint{3.988011in}{3.062902in}}{\pgfqpoint{3.992402in}{3.073501in}}{\pgfqpoint{3.992402in}{3.084551in}}%
\pgfpathcurveto{\pgfqpoint{3.992402in}{3.095601in}}{\pgfqpoint{3.988011in}{3.106200in}}{\pgfqpoint{3.980198in}{3.114014in}}%
\pgfpathcurveto{\pgfqpoint{3.972384in}{3.121827in}}{\pgfqpoint{3.961785in}{3.126218in}}{\pgfqpoint{3.950735in}{3.126218in}}%
\pgfpathcurveto{\pgfqpoint{3.939685in}{3.126218in}}{\pgfqpoint{3.929086in}{3.121827in}}{\pgfqpoint{3.921272in}{3.114014in}}%
\pgfpathcurveto{\pgfqpoint{3.913459in}{3.106200in}}{\pgfqpoint{3.909068in}{3.095601in}}{\pgfqpoint{3.909068in}{3.084551in}}%
\pgfpathcurveto{\pgfqpoint{3.909068in}{3.073501in}}{\pgfqpoint{3.913459in}{3.062902in}}{\pgfqpoint{3.921272in}{3.055088in}}%
\pgfpathcurveto{\pgfqpoint{3.929086in}{3.047275in}}{\pgfqpoint{3.939685in}{3.042884in}}{\pgfqpoint{3.950735in}{3.042884in}}%
\pgfpathclose%
\pgfusepath{stroke,fill}%
\end{pgfscope}%
\begin{pgfscope}%
\pgfpathrectangle{\pgfqpoint{0.481978in}{0.331635in}}{\pgfqpoint{9.300000in}{7.700000in}}%
\pgfusepath{clip}%
\pgfsetbuttcap%
\pgfsetroundjoin%
\definecolor{currentfill}{rgb}{1.000000,0.705882,0.509804}%
\pgfsetfillcolor{currentfill}%
\pgfsetlinewidth{0.481800pt}%
\definecolor{currentstroke}{rgb}{1.000000,1.000000,1.000000}%
\pgfsetstrokecolor{currentstroke}%
\pgfsetdash{}{0pt}%
\pgfpathmoveto{\pgfqpoint{4.062075in}{3.437490in}}%
\pgfpathcurveto{\pgfqpoint{4.073125in}{3.437490in}}{\pgfqpoint{4.083724in}{3.441880in}}{\pgfqpoint{4.091537in}{3.449694in}}%
\pgfpathcurveto{\pgfqpoint{4.099351in}{3.457507in}}{\pgfqpoint{4.103741in}{3.468106in}}{\pgfqpoint{4.103741in}{3.479157in}}%
\pgfpathcurveto{\pgfqpoint{4.103741in}{3.490207in}}{\pgfqpoint{4.099351in}{3.500806in}}{\pgfqpoint{4.091537in}{3.508619in}}%
\pgfpathcurveto{\pgfqpoint{4.083724in}{3.516433in}}{\pgfqpoint{4.073125in}{3.520823in}}{\pgfqpoint{4.062075in}{3.520823in}}%
\pgfpathcurveto{\pgfqpoint{4.051024in}{3.520823in}}{\pgfqpoint{4.040425in}{3.516433in}}{\pgfqpoint{4.032612in}{3.508619in}}%
\pgfpathcurveto{\pgfqpoint{4.024798in}{3.500806in}}{\pgfqpoint{4.020408in}{3.490207in}}{\pgfqpoint{4.020408in}{3.479157in}}%
\pgfpathcurveto{\pgfqpoint{4.020408in}{3.468106in}}{\pgfqpoint{4.024798in}{3.457507in}}{\pgfqpoint{4.032612in}{3.449694in}}%
\pgfpathcurveto{\pgfqpoint{4.040425in}{3.441880in}}{\pgfqpoint{4.051024in}{3.437490in}}{\pgfqpoint{4.062075in}{3.437490in}}%
\pgfpathclose%
\pgfusepath{stroke,fill}%
\end{pgfscope}%
\begin{pgfscope}%
\pgfpathrectangle{\pgfqpoint{0.481978in}{0.331635in}}{\pgfqpoint{9.300000in}{7.700000in}}%
\pgfusepath{clip}%
\pgfsetbuttcap%
\pgfsetroundjoin%
\definecolor{currentfill}{rgb}{1.000000,0.705882,0.509804}%
\pgfsetfillcolor{currentfill}%
\pgfsetlinewidth{0.481800pt}%
\definecolor{currentstroke}{rgb}{1.000000,1.000000,1.000000}%
\pgfsetstrokecolor{currentstroke}%
\pgfsetdash{}{0pt}%
\pgfpathmoveto{\pgfqpoint{5.989908in}{3.404674in}}%
\pgfpathcurveto{\pgfqpoint{6.000958in}{3.404674in}}{\pgfqpoint{6.011557in}{3.409064in}}{\pgfqpoint{6.019370in}{3.416878in}}%
\pgfpathcurveto{\pgfqpoint{6.027184in}{3.424692in}}{\pgfqpoint{6.031574in}{3.435291in}}{\pgfqpoint{6.031574in}{3.446341in}}%
\pgfpathcurveto{\pgfqpoint{6.031574in}{3.457391in}}{\pgfqpoint{6.027184in}{3.467990in}}{\pgfqpoint{6.019370in}{3.475804in}}%
\pgfpathcurveto{\pgfqpoint{6.011557in}{3.483617in}}{\pgfqpoint{6.000958in}{3.488007in}}{\pgfqpoint{5.989908in}{3.488007in}}%
\pgfpathcurveto{\pgfqpoint{5.978857in}{3.488007in}}{\pgfqpoint{5.968258in}{3.483617in}}{\pgfqpoint{5.960445in}{3.475804in}}%
\pgfpathcurveto{\pgfqpoint{5.952631in}{3.467990in}}{\pgfqpoint{5.948241in}{3.457391in}}{\pgfqpoint{5.948241in}{3.446341in}}%
\pgfpathcurveto{\pgfqpoint{5.948241in}{3.435291in}}{\pgfqpoint{5.952631in}{3.424692in}}{\pgfqpoint{5.960445in}{3.416878in}}%
\pgfpathcurveto{\pgfqpoint{5.968258in}{3.409064in}}{\pgfqpoint{5.978857in}{3.404674in}}{\pgfqpoint{5.989908in}{3.404674in}}%
\pgfpathclose%
\pgfusepath{stroke,fill}%
\end{pgfscope}%
\begin{pgfscope}%
\pgfpathrectangle{\pgfqpoint{0.481978in}{0.331635in}}{\pgfqpoint{9.300000in}{7.700000in}}%
\pgfusepath{clip}%
\pgfsetbuttcap%
\pgfsetroundjoin%
\definecolor{currentfill}{rgb}{1.000000,0.705882,0.509804}%
\pgfsetfillcolor{currentfill}%
\pgfsetlinewidth{0.481800pt}%
\definecolor{currentstroke}{rgb}{1.000000,1.000000,1.000000}%
\pgfsetstrokecolor{currentstroke}%
\pgfsetdash{}{0pt}%
\pgfpathmoveto{\pgfqpoint{8.313309in}{4.090148in}}%
\pgfpathcurveto{\pgfqpoint{8.324359in}{4.090148in}}{\pgfqpoint{8.334958in}{4.094538in}}{\pgfqpoint{8.342772in}{4.102351in}}%
\pgfpathcurveto{\pgfqpoint{8.350586in}{4.110165in}}{\pgfqpoint{8.354976in}{4.120764in}}{\pgfqpoint{8.354976in}{4.131814in}}%
\pgfpathcurveto{\pgfqpoint{8.354976in}{4.142864in}}{\pgfqpoint{8.350586in}{4.153463in}}{\pgfqpoint{8.342772in}{4.161277in}}%
\pgfpathcurveto{\pgfqpoint{8.334958in}{4.169091in}}{\pgfqpoint{8.324359in}{4.173481in}}{\pgfqpoint{8.313309in}{4.173481in}}%
\pgfpathcurveto{\pgfqpoint{8.302259in}{4.173481in}}{\pgfqpoint{8.291660in}{4.169091in}}{\pgfqpoint{8.283846in}{4.161277in}}%
\pgfpathcurveto{\pgfqpoint{8.276033in}{4.153463in}}{\pgfqpoint{8.271642in}{4.142864in}}{\pgfqpoint{8.271642in}{4.131814in}}%
\pgfpathcurveto{\pgfqpoint{8.271642in}{4.120764in}}{\pgfqpoint{8.276033in}{4.110165in}}{\pgfqpoint{8.283846in}{4.102351in}}%
\pgfpathcurveto{\pgfqpoint{8.291660in}{4.094538in}}{\pgfqpoint{8.302259in}{4.090148in}}{\pgfqpoint{8.313309in}{4.090148in}}%
\pgfpathclose%
\pgfusepath{stroke,fill}%
\end{pgfscope}%
\begin{pgfscope}%
\pgfpathrectangle{\pgfqpoint{0.481978in}{0.331635in}}{\pgfqpoint{9.300000in}{7.700000in}}%
\pgfusepath{clip}%
\pgfsetbuttcap%
\pgfsetroundjoin%
\definecolor{currentfill}{rgb}{1.000000,0.705882,0.509804}%
\pgfsetfillcolor{currentfill}%
\pgfsetlinewidth{0.481800pt}%
\definecolor{currentstroke}{rgb}{1.000000,1.000000,1.000000}%
\pgfsetstrokecolor{currentstroke}%
\pgfsetdash{}{0pt}%
\pgfpathmoveto{\pgfqpoint{4.541721in}{3.238983in}}%
\pgfpathcurveto{\pgfqpoint{4.552771in}{3.238983in}}{\pgfqpoint{4.563371in}{3.243373in}}{\pgfqpoint{4.571184in}{3.251187in}}%
\pgfpathcurveto{\pgfqpoint{4.578998in}{3.259000in}}{\pgfqpoint{4.583388in}{3.269599in}}{\pgfqpoint{4.583388in}{3.280649in}}%
\pgfpathcurveto{\pgfqpoint{4.583388in}{3.291700in}}{\pgfqpoint{4.578998in}{3.302299in}}{\pgfqpoint{4.571184in}{3.310112in}}%
\pgfpathcurveto{\pgfqpoint{4.563371in}{3.317926in}}{\pgfqpoint{4.552771in}{3.322316in}}{\pgfqpoint{4.541721in}{3.322316in}}%
\pgfpathcurveto{\pgfqpoint{4.530671in}{3.322316in}}{\pgfqpoint{4.520072in}{3.317926in}}{\pgfqpoint{4.512259in}{3.310112in}}%
\pgfpathcurveto{\pgfqpoint{4.504445in}{3.302299in}}{\pgfqpoint{4.500055in}{3.291700in}}{\pgfqpoint{4.500055in}{3.280649in}}%
\pgfpathcurveto{\pgfqpoint{4.500055in}{3.269599in}}{\pgfqpoint{4.504445in}{3.259000in}}{\pgfqpoint{4.512259in}{3.251187in}}%
\pgfpathcurveto{\pgfqpoint{4.520072in}{3.243373in}}{\pgfqpoint{4.530671in}{3.238983in}}{\pgfqpoint{4.541721in}{3.238983in}}%
\pgfpathclose%
\pgfusepath{stroke,fill}%
\end{pgfscope}%
\begin{pgfscope}%
\pgfpathrectangle{\pgfqpoint{0.481978in}{0.331635in}}{\pgfqpoint{9.300000in}{7.700000in}}%
\pgfusepath{clip}%
\pgfsetbuttcap%
\pgfsetroundjoin%
\definecolor{currentfill}{rgb}{1.000000,0.705882,0.509804}%
\pgfsetfillcolor{currentfill}%
\pgfsetlinewidth{0.481800pt}%
\definecolor{currentstroke}{rgb}{1.000000,1.000000,1.000000}%
\pgfsetstrokecolor{currentstroke}%
\pgfsetdash{}{0pt}%
\pgfpathmoveto{\pgfqpoint{8.281939in}{6.665290in}}%
\pgfpathcurveto{\pgfqpoint{8.292989in}{6.665290in}}{\pgfqpoint{8.303588in}{6.669681in}}{\pgfqpoint{8.311402in}{6.677494in}}%
\pgfpathcurveto{\pgfqpoint{8.319216in}{6.685308in}}{\pgfqpoint{8.323606in}{6.695907in}}{\pgfqpoint{8.323606in}{6.706957in}}%
\pgfpathcurveto{\pgfqpoint{8.323606in}{6.718007in}}{\pgfqpoint{8.319216in}{6.728606in}}{\pgfqpoint{8.311402in}{6.736420in}}%
\pgfpathcurveto{\pgfqpoint{8.303588in}{6.744234in}}{\pgfqpoint{8.292989in}{6.748624in}}{\pgfqpoint{8.281939in}{6.748624in}}%
\pgfpathcurveto{\pgfqpoint{8.270889in}{6.748624in}}{\pgfqpoint{8.260290in}{6.744234in}}{\pgfqpoint{8.252476in}{6.736420in}}%
\pgfpathcurveto{\pgfqpoint{8.244663in}{6.728606in}}{\pgfqpoint{8.240273in}{6.718007in}}{\pgfqpoint{8.240273in}{6.706957in}}%
\pgfpathcurveto{\pgfqpoint{8.240273in}{6.695907in}}{\pgfqpoint{8.244663in}{6.685308in}}{\pgfqpoint{8.252476in}{6.677494in}}%
\pgfpathcurveto{\pgfqpoint{8.260290in}{6.669681in}}{\pgfqpoint{8.270889in}{6.665290in}}{\pgfqpoint{8.281939in}{6.665290in}}%
\pgfpathclose%
\pgfusepath{stroke,fill}%
\end{pgfscope}%
\begin{pgfscope}%
\pgfpathrectangle{\pgfqpoint{0.481978in}{0.331635in}}{\pgfqpoint{9.300000in}{7.700000in}}%
\pgfusepath{clip}%
\pgfsetbuttcap%
\pgfsetroundjoin%
\definecolor{currentfill}{rgb}{1.000000,0.705882,0.509804}%
\pgfsetfillcolor{currentfill}%
\pgfsetlinewidth{0.481800pt}%
\definecolor{currentstroke}{rgb}{1.000000,1.000000,1.000000}%
\pgfsetstrokecolor{currentstroke}%
\pgfsetdash{}{0pt}%
\pgfpathmoveto{\pgfqpoint{7.611554in}{7.512902in}}%
\pgfpathcurveto{\pgfqpoint{7.622605in}{7.512902in}}{\pgfqpoint{7.633204in}{7.517292in}}{\pgfqpoint{7.641017in}{7.525106in}}%
\pgfpathcurveto{\pgfqpoint{7.648831in}{7.532920in}}{\pgfqpoint{7.653221in}{7.543519in}}{\pgfqpoint{7.653221in}{7.554569in}}%
\pgfpathcurveto{\pgfqpoint{7.653221in}{7.565619in}}{\pgfqpoint{7.648831in}{7.576218in}}{\pgfqpoint{7.641017in}{7.584032in}}%
\pgfpathcurveto{\pgfqpoint{7.633204in}{7.591845in}}{\pgfqpoint{7.622605in}{7.596236in}}{\pgfqpoint{7.611554in}{7.596236in}}%
\pgfpathcurveto{\pgfqpoint{7.600504in}{7.596236in}}{\pgfqpoint{7.589905in}{7.591845in}}{\pgfqpoint{7.582092in}{7.584032in}}%
\pgfpathcurveto{\pgfqpoint{7.574278in}{7.576218in}}{\pgfqpoint{7.569888in}{7.565619in}}{\pgfqpoint{7.569888in}{7.554569in}}%
\pgfpathcurveto{\pgfqpoint{7.569888in}{7.543519in}}{\pgfqpoint{7.574278in}{7.532920in}}{\pgfqpoint{7.582092in}{7.525106in}}%
\pgfpathcurveto{\pgfqpoint{7.589905in}{7.517292in}}{\pgfqpoint{7.600504in}{7.512902in}}{\pgfqpoint{7.611554in}{7.512902in}}%
\pgfpathclose%
\pgfusepath{stroke,fill}%
\end{pgfscope}%
\begin{pgfscope}%
\pgfpathrectangle{\pgfqpoint{0.481978in}{0.331635in}}{\pgfqpoint{9.300000in}{7.700000in}}%
\pgfusepath{clip}%
\pgfsetbuttcap%
\pgfsetroundjoin%
\definecolor{currentfill}{rgb}{1.000000,0.705882,0.509804}%
\pgfsetfillcolor{currentfill}%
\pgfsetlinewidth{0.481800pt}%
\definecolor{currentstroke}{rgb}{1.000000,1.000000,1.000000}%
\pgfsetstrokecolor{currentstroke}%
\pgfsetdash{}{0pt}%
\pgfpathmoveto{\pgfqpoint{2.721741in}{3.669095in}}%
\pgfpathcurveto{\pgfqpoint{2.732792in}{3.669095in}}{\pgfqpoint{2.743391in}{3.673485in}}{\pgfqpoint{2.751204in}{3.681299in}}%
\pgfpathcurveto{\pgfqpoint{2.759018in}{3.689112in}}{\pgfqpoint{2.763408in}{3.699711in}}{\pgfqpoint{2.763408in}{3.710762in}}%
\pgfpathcurveto{\pgfqpoint{2.763408in}{3.721812in}}{\pgfqpoint{2.759018in}{3.732411in}}{\pgfqpoint{2.751204in}{3.740224in}}%
\pgfpathcurveto{\pgfqpoint{2.743391in}{3.748038in}}{\pgfqpoint{2.732792in}{3.752428in}}{\pgfqpoint{2.721741in}{3.752428in}}%
\pgfpathcurveto{\pgfqpoint{2.710691in}{3.752428in}}{\pgfqpoint{2.700092in}{3.748038in}}{\pgfqpoint{2.692279in}{3.740224in}}%
\pgfpathcurveto{\pgfqpoint{2.684465in}{3.732411in}}{\pgfqpoint{2.680075in}{3.721812in}}{\pgfqpoint{2.680075in}{3.710762in}}%
\pgfpathcurveto{\pgfqpoint{2.680075in}{3.699711in}}{\pgfqpoint{2.684465in}{3.689112in}}{\pgfqpoint{2.692279in}{3.681299in}}%
\pgfpathcurveto{\pgfqpoint{2.700092in}{3.673485in}}{\pgfqpoint{2.710691in}{3.669095in}}{\pgfqpoint{2.721741in}{3.669095in}}%
\pgfpathclose%
\pgfusepath{stroke,fill}%
\end{pgfscope}%
\begin{pgfscope}%
\pgfpathrectangle{\pgfqpoint{0.481978in}{0.331635in}}{\pgfqpoint{9.300000in}{7.700000in}}%
\pgfusepath{clip}%
\pgfsetbuttcap%
\pgfsetroundjoin%
\definecolor{currentfill}{rgb}{1.000000,0.705882,0.509804}%
\pgfsetfillcolor{currentfill}%
\pgfsetlinewidth{0.481800pt}%
\definecolor{currentstroke}{rgb}{1.000000,1.000000,1.000000}%
\pgfsetstrokecolor{currentstroke}%
\pgfsetdash{}{0pt}%
\pgfpathmoveto{\pgfqpoint{6.813584in}{2.399944in}}%
\pgfpathcurveto{\pgfqpoint{6.824634in}{2.399944in}}{\pgfqpoint{6.835233in}{2.404334in}}{\pgfqpoint{6.843047in}{2.412148in}}%
\pgfpathcurveto{\pgfqpoint{6.850860in}{2.419962in}}{\pgfqpoint{6.855251in}{2.430561in}}{\pgfqpoint{6.855251in}{2.441611in}}%
\pgfpathcurveto{\pgfqpoint{6.855251in}{2.452661in}}{\pgfqpoint{6.850860in}{2.463260in}}{\pgfqpoint{6.843047in}{2.471074in}}%
\pgfpathcurveto{\pgfqpoint{6.835233in}{2.478887in}}{\pgfqpoint{6.824634in}{2.483278in}}{\pgfqpoint{6.813584in}{2.483278in}}%
\pgfpathcurveto{\pgfqpoint{6.802534in}{2.483278in}}{\pgfqpoint{6.791935in}{2.478887in}}{\pgfqpoint{6.784121in}{2.471074in}}%
\pgfpathcurveto{\pgfqpoint{6.776308in}{2.463260in}}{\pgfqpoint{6.771917in}{2.452661in}}{\pgfqpoint{6.771917in}{2.441611in}}%
\pgfpathcurveto{\pgfqpoint{6.771917in}{2.430561in}}{\pgfqpoint{6.776308in}{2.419962in}}{\pgfqpoint{6.784121in}{2.412148in}}%
\pgfpathcurveto{\pgfqpoint{6.791935in}{2.404334in}}{\pgfqpoint{6.802534in}{2.399944in}}{\pgfqpoint{6.813584in}{2.399944in}}%
\pgfpathclose%
\pgfusepath{stroke,fill}%
\end{pgfscope}%
\begin{pgfscope}%
\pgfpathrectangle{\pgfqpoint{0.481978in}{0.331635in}}{\pgfqpoint{9.300000in}{7.700000in}}%
\pgfusepath{clip}%
\pgfsetbuttcap%
\pgfsetroundjoin%
\definecolor{currentfill}{rgb}{1.000000,0.705882,0.509804}%
\pgfsetfillcolor{currentfill}%
\pgfsetlinewidth{0.481800pt}%
\definecolor{currentstroke}{rgb}{1.000000,1.000000,1.000000}%
\pgfsetstrokecolor{currentstroke}%
\pgfsetdash{}{0pt}%
\pgfpathmoveto{\pgfqpoint{7.019489in}{7.639968in}}%
\pgfpathcurveto{\pgfqpoint{7.030540in}{7.639968in}}{\pgfqpoint{7.041139in}{7.644359in}}{\pgfqpoint{7.048952in}{7.652172in}}%
\pgfpathcurveto{\pgfqpoint{7.056766in}{7.659986in}}{\pgfqpoint{7.061156in}{7.670585in}}{\pgfqpoint{7.061156in}{7.681635in}}%
\pgfpathcurveto{\pgfqpoint{7.061156in}{7.692685in}}{\pgfqpoint{7.056766in}{7.703284in}}{\pgfqpoint{7.048952in}{7.711098in}}%
\pgfpathcurveto{\pgfqpoint{7.041139in}{7.718911in}}{\pgfqpoint{7.030540in}{7.723302in}}{\pgfqpoint{7.019489in}{7.723302in}}%
\pgfpathcurveto{\pgfqpoint{7.008439in}{7.723302in}}{\pgfqpoint{6.997840in}{7.718911in}}{\pgfqpoint{6.990027in}{7.711098in}}%
\pgfpathcurveto{\pgfqpoint{6.982213in}{7.703284in}}{\pgfqpoint{6.977823in}{7.692685in}}{\pgfqpoint{6.977823in}{7.681635in}}%
\pgfpathcurveto{\pgfqpoint{6.977823in}{7.670585in}}{\pgfqpoint{6.982213in}{7.659986in}}{\pgfqpoint{6.990027in}{7.652172in}}%
\pgfpathcurveto{\pgfqpoint{6.997840in}{7.644359in}}{\pgfqpoint{7.008439in}{7.639968in}}{\pgfqpoint{7.019489in}{7.639968in}}%
\pgfpathclose%
\pgfusepath{stroke,fill}%
\end{pgfscope}%
\begin{pgfscope}%
\pgfpathrectangle{\pgfqpoint{0.481978in}{0.331635in}}{\pgfqpoint{9.300000in}{7.700000in}}%
\pgfusepath{clip}%
\pgfsetbuttcap%
\pgfsetroundjoin%
\definecolor{currentfill}{rgb}{1.000000,0.705882,0.509804}%
\pgfsetfillcolor{currentfill}%
\pgfsetlinewidth{0.481800pt}%
\definecolor{currentstroke}{rgb}{1.000000,1.000000,1.000000}%
\pgfsetstrokecolor{currentstroke}%
\pgfsetdash{}{0pt}%
\pgfpathmoveto{\pgfqpoint{6.102927in}{7.240561in}}%
\pgfpathcurveto{\pgfqpoint{6.113977in}{7.240561in}}{\pgfqpoint{6.124576in}{7.244951in}}{\pgfqpoint{6.132389in}{7.252764in}}%
\pgfpathcurveto{\pgfqpoint{6.140203in}{7.260578in}}{\pgfqpoint{6.144593in}{7.271177in}}{\pgfqpoint{6.144593in}{7.282227in}}%
\pgfpathcurveto{\pgfqpoint{6.144593in}{7.293277in}}{\pgfqpoint{6.140203in}{7.303876in}}{\pgfqpoint{6.132389in}{7.311690in}}%
\pgfpathcurveto{\pgfqpoint{6.124576in}{7.319504in}}{\pgfqpoint{6.113977in}{7.323894in}}{\pgfqpoint{6.102927in}{7.323894in}}%
\pgfpathcurveto{\pgfqpoint{6.091876in}{7.323894in}}{\pgfqpoint{6.081277in}{7.319504in}}{\pgfqpoint{6.073464in}{7.311690in}}%
\pgfpathcurveto{\pgfqpoint{6.065650in}{7.303876in}}{\pgfqpoint{6.061260in}{7.293277in}}{\pgfqpoint{6.061260in}{7.282227in}}%
\pgfpathcurveto{\pgfqpoint{6.061260in}{7.271177in}}{\pgfqpoint{6.065650in}{7.260578in}}{\pgfqpoint{6.073464in}{7.252764in}}%
\pgfpathcurveto{\pgfqpoint{6.081277in}{7.244951in}}{\pgfqpoint{6.091876in}{7.240561in}}{\pgfqpoint{6.102927in}{7.240561in}}%
\pgfpathclose%
\pgfusepath{stroke,fill}%
\end{pgfscope}%
\begin{pgfscope}%
\pgfpathrectangle{\pgfqpoint{0.481978in}{0.331635in}}{\pgfqpoint{9.300000in}{7.700000in}}%
\pgfusepath{clip}%
\pgfsetbuttcap%
\pgfsetroundjoin%
\definecolor{currentfill}{rgb}{1.000000,0.705882,0.509804}%
\pgfsetfillcolor{currentfill}%
\pgfsetlinewidth{0.481800pt}%
\definecolor{currentstroke}{rgb}{1.000000,1.000000,1.000000}%
\pgfsetstrokecolor{currentstroke}%
\pgfsetdash{}{0pt}%
\pgfpathmoveto{\pgfqpoint{5.548025in}{3.709019in}}%
\pgfpathcurveto{\pgfqpoint{5.559075in}{3.709019in}}{\pgfqpoint{5.569674in}{3.713409in}}{\pgfqpoint{5.577488in}{3.721223in}}%
\pgfpathcurveto{\pgfqpoint{5.585301in}{3.729036in}}{\pgfqpoint{5.589692in}{3.739635in}}{\pgfqpoint{5.589692in}{3.750685in}}%
\pgfpathcurveto{\pgfqpoint{5.589692in}{3.761736in}}{\pgfqpoint{5.585301in}{3.772335in}}{\pgfqpoint{5.577488in}{3.780148in}}%
\pgfpathcurveto{\pgfqpoint{5.569674in}{3.787962in}}{\pgfqpoint{5.559075in}{3.792352in}}{\pgfqpoint{5.548025in}{3.792352in}}%
\pgfpathcurveto{\pgfqpoint{5.536975in}{3.792352in}}{\pgfqpoint{5.526376in}{3.787962in}}{\pgfqpoint{5.518562in}{3.780148in}}%
\pgfpathcurveto{\pgfqpoint{5.510748in}{3.772335in}}{\pgfqpoint{5.506358in}{3.761736in}}{\pgfqpoint{5.506358in}{3.750685in}}%
\pgfpathcurveto{\pgfqpoint{5.506358in}{3.739635in}}{\pgfqpoint{5.510748in}{3.729036in}}{\pgfqpoint{5.518562in}{3.721223in}}%
\pgfpathcurveto{\pgfqpoint{5.526376in}{3.713409in}}{\pgfqpoint{5.536975in}{3.709019in}}{\pgfqpoint{5.548025in}{3.709019in}}%
\pgfpathclose%
\pgfusepath{stroke,fill}%
\end{pgfscope}%
\begin{pgfscope}%
\pgfpathrectangle{\pgfqpoint{0.481978in}{0.331635in}}{\pgfqpoint{9.300000in}{7.700000in}}%
\pgfusepath{clip}%
\pgfsetbuttcap%
\pgfsetroundjoin%
\definecolor{currentfill}{rgb}{1.000000,0.705882,0.509804}%
\pgfsetfillcolor{currentfill}%
\pgfsetlinewidth{0.481800pt}%
\definecolor{currentstroke}{rgb}{1.000000,1.000000,1.000000}%
\pgfsetstrokecolor{currentstroke}%
\pgfsetdash{}{0pt}%
\pgfpathmoveto{\pgfqpoint{4.725147in}{4.229750in}}%
\pgfpathcurveto{\pgfqpoint{4.736197in}{4.229750in}}{\pgfqpoint{4.746796in}{4.234140in}}{\pgfqpoint{4.754610in}{4.241953in}}%
\pgfpathcurveto{\pgfqpoint{4.762423in}{4.249767in}}{\pgfqpoint{4.766814in}{4.260366in}}{\pgfqpoint{4.766814in}{4.271416in}}%
\pgfpathcurveto{\pgfqpoint{4.766814in}{4.282466in}}{\pgfqpoint{4.762423in}{4.293065in}}{\pgfqpoint{4.754610in}{4.300879in}}%
\pgfpathcurveto{\pgfqpoint{4.746796in}{4.308693in}}{\pgfqpoint{4.736197in}{4.313083in}}{\pgfqpoint{4.725147in}{4.313083in}}%
\pgfpathcurveto{\pgfqpoint{4.714097in}{4.313083in}}{\pgfqpoint{4.703498in}{4.308693in}}{\pgfqpoint{4.695684in}{4.300879in}}%
\pgfpathcurveto{\pgfqpoint{4.687870in}{4.293065in}}{\pgfqpoint{4.683480in}{4.282466in}}{\pgfqpoint{4.683480in}{4.271416in}}%
\pgfpathcurveto{\pgfqpoint{4.683480in}{4.260366in}}{\pgfqpoint{4.687870in}{4.249767in}}{\pgfqpoint{4.695684in}{4.241953in}}%
\pgfpathcurveto{\pgfqpoint{4.703498in}{4.234140in}}{\pgfqpoint{4.714097in}{4.229750in}}{\pgfqpoint{4.725147in}{4.229750in}}%
\pgfpathclose%
\pgfusepath{stroke,fill}%
\end{pgfscope}%
\begin{pgfscope}%
\pgfpathrectangle{\pgfqpoint{0.481978in}{0.331635in}}{\pgfqpoint{9.300000in}{7.700000in}}%
\pgfusepath{clip}%
\pgfsetbuttcap%
\pgfsetroundjoin%
\definecolor{currentfill}{rgb}{1.000000,0.705882,0.509804}%
\pgfsetfillcolor{currentfill}%
\pgfsetlinewidth{0.481800pt}%
\definecolor{currentstroke}{rgb}{1.000000,1.000000,1.000000}%
\pgfsetstrokecolor{currentstroke}%
\pgfsetdash{}{0pt}%
\pgfpathmoveto{\pgfqpoint{7.998254in}{3.604984in}}%
\pgfpathcurveto{\pgfqpoint{8.009304in}{3.604984in}}{\pgfqpoint{8.019903in}{3.609374in}}{\pgfqpoint{8.027716in}{3.617188in}}%
\pgfpathcurveto{\pgfqpoint{8.035530in}{3.625002in}}{\pgfqpoint{8.039920in}{3.635601in}}{\pgfqpoint{8.039920in}{3.646651in}}%
\pgfpathcurveto{\pgfqpoint{8.039920in}{3.657701in}}{\pgfqpoint{8.035530in}{3.668300in}}{\pgfqpoint{8.027716in}{3.676114in}}%
\pgfpathcurveto{\pgfqpoint{8.019903in}{3.683927in}}{\pgfqpoint{8.009304in}{3.688317in}}{\pgfqpoint{7.998254in}{3.688317in}}%
\pgfpathcurveto{\pgfqpoint{7.987203in}{3.688317in}}{\pgfqpoint{7.976604in}{3.683927in}}{\pgfqpoint{7.968791in}{3.676114in}}%
\pgfpathcurveto{\pgfqpoint{7.960977in}{3.668300in}}{\pgfqpoint{7.956587in}{3.657701in}}{\pgfqpoint{7.956587in}{3.646651in}}%
\pgfpathcurveto{\pgfqpoint{7.956587in}{3.635601in}}{\pgfqpoint{7.960977in}{3.625002in}}{\pgfqpoint{7.968791in}{3.617188in}}%
\pgfpathcurveto{\pgfqpoint{7.976604in}{3.609374in}}{\pgfqpoint{7.987203in}{3.604984in}}{\pgfqpoint{7.998254in}{3.604984in}}%
\pgfpathclose%
\pgfusepath{stroke,fill}%
\end{pgfscope}%
\begin{pgfscope}%
\pgfpathrectangle{\pgfqpoint{0.481978in}{0.331635in}}{\pgfqpoint{9.300000in}{7.700000in}}%
\pgfusepath{clip}%
\pgfsetbuttcap%
\pgfsetroundjoin%
\definecolor{currentfill}{rgb}{1.000000,0.705882,0.509804}%
\pgfsetfillcolor{currentfill}%
\pgfsetlinewidth{0.481800pt}%
\definecolor{currentstroke}{rgb}{1.000000,1.000000,1.000000}%
\pgfsetstrokecolor{currentstroke}%
\pgfsetdash{}{0pt}%
\pgfpathmoveto{\pgfqpoint{5.102532in}{4.485893in}}%
\pgfpathcurveto{\pgfqpoint{5.113582in}{4.485893in}}{\pgfqpoint{5.124181in}{4.490283in}}{\pgfqpoint{5.131995in}{4.498097in}}%
\pgfpathcurveto{\pgfqpoint{5.139808in}{4.505910in}}{\pgfqpoint{5.144199in}{4.516509in}}{\pgfqpoint{5.144199in}{4.527559in}}%
\pgfpathcurveto{\pgfqpoint{5.144199in}{4.538609in}}{\pgfqpoint{5.139808in}{4.549208in}}{\pgfqpoint{5.131995in}{4.557022in}}%
\pgfpathcurveto{\pgfqpoint{5.124181in}{4.564836in}}{\pgfqpoint{5.113582in}{4.569226in}}{\pgfqpoint{5.102532in}{4.569226in}}%
\pgfpathcurveto{\pgfqpoint{5.091482in}{4.569226in}}{\pgfqpoint{5.080883in}{4.564836in}}{\pgfqpoint{5.073069in}{4.557022in}}%
\pgfpathcurveto{\pgfqpoint{5.065256in}{4.549208in}}{\pgfqpoint{5.060865in}{4.538609in}}{\pgfqpoint{5.060865in}{4.527559in}}%
\pgfpathcurveto{\pgfqpoint{5.060865in}{4.516509in}}{\pgfqpoint{5.065256in}{4.505910in}}{\pgfqpoint{5.073069in}{4.498097in}}%
\pgfpathcurveto{\pgfqpoint{5.080883in}{4.490283in}}{\pgfqpoint{5.091482in}{4.485893in}}{\pgfqpoint{5.102532in}{4.485893in}}%
\pgfpathclose%
\pgfusepath{stroke,fill}%
\end{pgfscope}%
\begin{pgfscope}%
\pgfpathrectangle{\pgfqpoint{0.481978in}{0.331635in}}{\pgfqpoint{9.300000in}{7.700000in}}%
\pgfusepath{clip}%
\pgfsetbuttcap%
\pgfsetroundjoin%
\definecolor{currentfill}{rgb}{1.000000,0.705882,0.509804}%
\pgfsetfillcolor{currentfill}%
\pgfsetlinewidth{0.481800pt}%
\definecolor{currentstroke}{rgb}{1.000000,1.000000,1.000000}%
\pgfsetstrokecolor{currentstroke}%
\pgfsetdash{}{0pt}%
\pgfpathmoveto{\pgfqpoint{4.679104in}{4.702735in}}%
\pgfpathcurveto{\pgfqpoint{4.690154in}{4.702735in}}{\pgfqpoint{4.700753in}{4.707125in}}{\pgfqpoint{4.708567in}{4.714939in}}%
\pgfpathcurveto{\pgfqpoint{4.716381in}{4.722752in}}{\pgfqpoint{4.720771in}{4.733351in}}{\pgfqpoint{4.720771in}{4.744402in}}%
\pgfpathcurveto{\pgfqpoint{4.720771in}{4.755452in}}{\pgfqpoint{4.716381in}{4.766051in}}{\pgfqpoint{4.708567in}{4.773864in}}%
\pgfpathcurveto{\pgfqpoint{4.700753in}{4.781678in}}{\pgfqpoint{4.690154in}{4.786068in}}{\pgfqpoint{4.679104in}{4.786068in}}%
\pgfpathcurveto{\pgfqpoint{4.668054in}{4.786068in}}{\pgfqpoint{4.657455in}{4.781678in}}{\pgfqpoint{4.649641in}{4.773864in}}%
\pgfpathcurveto{\pgfqpoint{4.641828in}{4.766051in}}{\pgfqpoint{4.637437in}{4.755452in}}{\pgfqpoint{4.637437in}{4.744402in}}%
\pgfpathcurveto{\pgfqpoint{4.637437in}{4.733351in}}{\pgfqpoint{4.641828in}{4.722752in}}{\pgfqpoint{4.649641in}{4.714939in}}%
\pgfpathcurveto{\pgfqpoint{4.657455in}{4.707125in}}{\pgfqpoint{4.668054in}{4.702735in}}{\pgfqpoint{4.679104in}{4.702735in}}%
\pgfpathclose%
\pgfusepath{stroke,fill}%
\end{pgfscope}%
\begin{pgfscope}%
\pgfpathrectangle{\pgfqpoint{0.481978in}{0.331635in}}{\pgfqpoint{9.300000in}{7.700000in}}%
\pgfusepath{clip}%
\pgfsetbuttcap%
\pgfsetroundjoin%
\definecolor{currentfill}{rgb}{1.000000,0.705882,0.509804}%
\pgfsetfillcolor{currentfill}%
\pgfsetlinewidth{0.481800pt}%
\definecolor{currentstroke}{rgb}{1.000000,1.000000,1.000000}%
\pgfsetstrokecolor{currentstroke}%
\pgfsetdash{}{0pt}%
\pgfpathmoveto{\pgfqpoint{4.103990in}{2.254849in}}%
\pgfpathcurveto{\pgfqpoint{4.115040in}{2.254849in}}{\pgfqpoint{4.125639in}{2.259240in}}{\pgfqpoint{4.133453in}{2.267053in}}%
\pgfpathcurveto{\pgfqpoint{4.141266in}{2.274867in}}{\pgfqpoint{4.145657in}{2.285466in}}{\pgfqpoint{4.145657in}{2.296516in}}%
\pgfpathcurveto{\pgfqpoint{4.145657in}{2.307566in}}{\pgfqpoint{4.141266in}{2.318165in}}{\pgfqpoint{4.133453in}{2.325979in}}%
\pgfpathcurveto{\pgfqpoint{4.125639in}{2.333792in}}{\pgfqpoint{4.115040in}{2.338183in}}{\pgfqpoint{4.103990in}{2.338183in}}%
\pgfpathcurveto{\pgfqpoint{4.092940in}{2.338183in}}{\pgfqpoint{4.082341in}{2.333792in}}{\pgfqpoint{4.074527in}{2.325979in}}%
\pgfpathcurveto{\pgfqpoint{4.066714in}{2.318165in}}{\pgfqpoint{4.062323in}{2.307566in}}{\pgfqpoint{4.062323in}{2.296516in}}%
\pgfpathcurveto{\pgfqpoint{4.062323in}{2.285466in}}{\pgfqpoint{4.066714in}{2.274867in}}{\pgfqpoint{4.074527in}{2.267053in}}%
\pgfpathcurveto{\pgfqpoint{4.082341in}{2.259240in}}{\pgfqpoint{4.092940in}{2.254849in}}{\pgfqpoint{4.103990in}{2.254849in}}%
\pgfpathclose%
\pgfusepath{stroke,fill}%
\end{pgfscope}%
\begin{pgfscope}%
\pgfpathrectangle{\pgfqpoint{0.481978in}{0.331635in}}{\pgfqpoint{9.300000in}{7.700000in}}%
\pgfusepath{clip}%
\pgfsetbuttcap%
\pgfsetroundjoin%
\definecolor{currentfill}{rgb}{1.000000,0.705882,0.509804}%
\pgfsetfillcolor{currentfill}%
\pgfsetlinewidth{0.481800pt}%
\definecolor{currentstroke}{rgb}{1.000000,1.000000,1.000000}%
\pgfsetstrokecolor{currentstroke}%
\pgfsetdash{}{0pt}%
\pgfpathmoveto{\pgfqpoint{4.530179in}{2.861464in}}%
\pgfpathcurveto{\pgfqpoint{4.541229in}{2.861464in}}{\pgfqpoint{4.551828in}{2.865854in}}{\pgfqpoint{4.559642in}{2.873667in}}%
\pgfpathcurveto{\pgfqpoint{4.567455in}{2.881481in}}{\pgfqpoint{4.571846in}{2.892080in}}{\pgfqpoint{4.571846in}{2.903130in}}%
\pgfpathcurveto{\pgfqpoint{4.571846in}{2.914180in}}{\pgfqpoint{4.567455in}{2.924779in}}{\pgfqpoint{4.559642in}{2.932593in}}%
\pgfpathcurveto{\pgfqpoint{4.551828in}{2.940407in}}{\pgfqpoint{4.541229in}{2.944797in}}{\pgfqpoint{4.530179in}{2.944797in}}%
\pgfpathcurveto{\pgfqpoint{4.519129in}{2.944797in}}{\pgfqpoint{4.508530in}{2.940407in}}{\pgfqpoint{4.500716in}{2.932593in}}%
\pgfpathcurveto{\pgfqpoint{4.492902in}{2.924779in}}{\pgfqpoint{4.488512in}{2.914180in}}{\pgfqpoint{4.488512in}{2.903130in}}%
\pgfpathcurveto{\pgfqpoint{4.488512in}{2.892080in}}{\pgfqpoint{4.492902in}{2.881481in}}{\pgfqpoint{4.500716in}{2.873667in}}%
\pgfpathcurveto{\pgfqpoint{4.508530in}{2.865854in}}{\pgfqpoint{4.519129in}{2.861464in}}{\pgfqpoint{4.530179in}{2.861464in}}%
\pgfpathclose%
\pgfusepath{stroke,fill}%
\end{pgfscope}%
\begin{pgfscope}%
\pgfpathrectangle{\pgfqpoint{0.481978in}{0.331635in}}{\pgfqpoint{9.300000in}{7.700000in}}%
\pgfusepath{clip}%
\pgfsetbuttcap%
\pgfsetroundjoin%
\definecolor{currentfill}{rgb}{1.000000,0.705882,0.509804}%
\pgfsetfillcolor{currentfill}%
\pgfsetlinewidth{0.481800pt}%
\definecolor{currentstroke}{rgb}{1.000000,1.000000,1.000000}%
\pgfsetstrokecolor{currentstroke}%
\pgfsetdash{}{0pt}%
\pgfpathmoveto{\pgfqpoint{3.358304in}{5.994925in}}%
\pgfpathcurveto{\pgfqpoint{3.369354in}{5.994925in}}{\pgfqpoint{3.379953in}{5.999315in}}{\pgfqpoint{3.387767in}{6.007129in}}%
\pgfpathcurveto{\pgfqpoint{3.395581in}{6.014942in}}{\pgfqpoint{3.399971in}{6.025541in}}{\pgfqpoint{3.399971in}{6.036591in}}%
\pgfpathcurveto{\pgfqpoint{3.399971in}{6.047642in}}{\pgfqpoint{3.395581in}{6.058241in}}{\pgfqpoint{3.387767in}{6.066054in}}%
\pgfpathcurveto{\pgfqpoint{3.379953in}{6.073868in}}{\pgfqpoint{3.369354in}{6.078258in}}{\pgfqpoint{3.358304in}{6.078258in}}%
\pgfpathcurveto{\pgfqpoint{3.347254in}{6.078258in}}{\pgfqpoint{3.336655in}{6.073868in}}{\pgfqpoint{3.328841in}{6.066054in}}%
\pgfpathcurveto{\pgfqpoint{3.321028in}{6.058241in}}{\pgfqpoint{3.316638in}{6.047642in}}{\pgfqpoint{3.316638in}{6.036591in}}%
\pgfpathcurveto{\pgfqpoint{3.316638in}{6.025541in}}{\pgfqpoint{3.321028in}{6.014942in}}{\pgfqpoint{3.328841in}{6.007129in}}%
\pgfpathcurveto{\pgfqpoint{3.336655in}{5.999315in}}{\pgfqpoint{3.347254in}{5.994925in}}{\pgfqpoint{3.358304in}{5.994925in}}%
\pgfpathclose%
\pgfusepath{stroke,fill}%
\end{pgfscope}%
\begin{pgfscope}%
\pgfpathrectangle{\pgfqpoint{0.481978in}{0.331635in}}{\pgfqpoint{9.300000in}{7.700000in}}%
\pgfusepath{clip}%
\pgfsetbuttcap%
\pgfsetroundjoin%
\definecolor{currentfill}{rgb}{1.000000,0.705882,0.509804}%
\pgfsetfillcolor{currentfill}%
\pgfsetlinewidth{0.481800pt}%
\definecolor{currentstroke}{rgb}{1.000000,1.000000,1.000000}%
\pgfsetstrokecolor{currentstroke}%
\pgfsetdash{}{0pt}%
\pgfpathmoveto{\pgfqpoint{5.398084in}{2.590748in}}%
\pgfpathcurveto{\pgfqpoint{5.409134in}{2.590748in}}{\pgfqpoint{5.419733in}{2.595138in}}{\pgfqpoint{5.427547in}{2.602952in}}%
\pgfpathcurveto{\pgfqpoint{5.435360in}{2.610765in}}{\pgfqpoint{5.439750in}{2.621364in}}{\pgfqpoint{5.439750in}{2.632414in}}%
\pgfpathcurveto{\pgfqpoint{5.439750in}{2.643464in}}{\pgfqpoint{5.435360in}{2.654063in}}{\pgfqpoint{5.427547in}{2.661877in}}%
\pgfpathcurveto{\pgfqpoint{5.419733in}{2.669691in}}{\pgfqpoint{5.409134in}{2.674081in}}{\pgfqpoint{5.398084in}{2.674081in}}%
\pgfpathcurveto{\pgfqpoint{5.387034in}{2.674081in}}{\pgfqpoint{5.376435in}{2.669691in}}{\pgfqpoint{5.368621in}{2.661877in}}%
\pgfpathcurveto{\pgfqpoint{5.360807in}{2.654063in}}{\pgfqpoint{5.356417in}{2.643464in}}{\pgfqpoint{5.356417in}{2.632414in}}%
\pgfpathcurveto{\pgfqpoint{5.356417in}{2.621364in}}{\pgfqpoint{5.360807in}{2.610765in}}{\pgfqpoint{5.368621in}{2.602952in}}%
\pgfpathcurveto{\pgfqpoint{5.376435in}{2.595138in}}{\pgfqpoint{5.387034in}{2.590748in}}{\pgfqpoint{5.398084in}{2.590748in}}%
\pgfpathclose%
\pgfusepath{stroke,fill}%
\end{pgfscope}%
\begin{pgfscope}%
\pgfpathrectangle{\pgfqpoint{0.481978in}{0.331635in}}{\pgfqpoint{9.300000in}{7.700000in}}%
\pgfusepath{clip}%
\pgfsetbuttcap%
\pgfsetroundjoin%
\definecolor{currentfill}{rgb}{1.000000,0.705882,0.509804}%
\pgfsetfillcolor{currentfill}%
\pgfsetlinewidth{0.481800pt}%
\definecolor{currentstroke}{rgb}{1.000000,1.000000,1.000000}%
\pgfsetstrokecolor{currentstroke}%
\pgfsetdash{}{0pt}%
\pgfpathmoveto{\pgfqpoint{8.487669in}{6.396251in}}%
\pgfpathcurveto{\pgfqpoint{8.498719in}{6.396251in}}{\pgfqpoint{8.509318in}{6.400642in}}{\pgfqpoint{8.517131in}{6.408455in}}%
\pgfpathcurveto{\pgfqpoint{8.524945in}{6.416269in}}{\pgfqpoint{8.529335in}{6.426868in}}{\pgfqpoint{8.529335in}{6.437918in}}%
\pgfpathcurveto{\pgfqpoint{8.529335in}{6.448968in}}{\pgfqpoint{8.524945in}{6.459567in}}{\pgfqpoint{8.517131in}{6.467381in}}%
\pgfpathcurveto{\pgfqpoint{8.509318in}{6.475194in}}{\pgfqpoint{8.498719in}{6.479585in}}{\pgfqpoint{8.487669in}{6.479585in}}%
\pgfpathcurveto{\pgfqpoint{8.476619in}{6.479585in}}{\pgfqpoint{8.466020in}{6.475194in}}{\pgfqpoint{8.458206in}{6.467381in}}%
\pgfpathcurveto{\pgfqpoint{8.450392in}{6.459567in}}{\pgfqpoint{8.446002in}{6.448968in}}{\pgfqpoint{8.446002in}{6.437918in}}%
\pgfpathcurveto{\pgfqpoint{8.446002in}{6.426868in}}{\pgfqpoint{8.450392in}{6.416269in}}{\pgfqpoint{8.458206in}{6.408455in}}%
\pgfpathcurveto{\pgfqpoint{8.466020in}{6.400642in}}{\pgfqpoint{8.476619in}{6.396251in}}{\pgfqpoint{8.487669in}{6.396251in}}%
\pgfpathclose%
\pgfusepath{stroke,fill}%
\end{pgfscope}%
\begin{pgfscope}%
\pgfpathrectangle{\pgfqpoint{0.481978in}{0.331635in}}{\pgfqpoint{9.300000in}{7.700000in}}%
\pgfusepath{clip}%
\pgfsetbuttcap%
\pgfsetroundjoin%
\definecolor{currentfill}{rgb}{1.000000,0.705882,0.509804}%
\pgfsetfillcolor{currentfill}%
\pgfsetlinewidth{0.481800pt}%
\definecolor{currentstroke}{rgb}{1.000000,1.000000,1.000000}%
\pgfsetstrokecolor{currentstroke}%
\pgfsetdash{}{0pt}%
\pgfpathmoveto{\pgfqpoint{4.319461in}{3.830119in}}%
\pgfpathcurveto{\pgfqpoint{4.330511in}{3.830119in}}{\pgfqpoint{4.341110in}{3.834510in}}{\pgfqpoint{4.348923in}{3.842323in}}%
\pgfpathcurveto{\pgfqpoint{4.356737in}{3.850137in}}{\pgfqpoint{4.361127in}{3.860736in}}{\pgfqpoint{4.361127in}{3.871786in}}%
\pgfpathcurveto{\pgfqpoint{4.361127in}{3.882836in}}{\pgfqpoint{4.356737in}{3.893435in}}{\pgfqpoint{4.348923in}{3.901249in}}%
\pgfpathcurveto{\pgfqpoint{4.341110in}{3.909062in}}{\pgfqpoint{4.330511in}{3.913453in}}{\pgfqpoint{4.319461in}{3.913453in}}%
\pgfpathcurveto{\pgfqpoint{4.308410in}{3.913453in}}{\pgfqpoint{4.297811in}{3.909062in}}{\pgfqpoint{4.289998in}{3.901249in}}%
\pgfpathcurveto{\pgfqpoint{4.282184in}{3.893435in}}{\pgfqpoint{4.277794in}{3.882836in}}{\pgfqpoint{4.277794in}{3.871786in}}%
\pgfpathcurveto{\pgfqpoint{4.277794in}{3.860736in}}{\pgfqpoint{4.282184in}{3.850137in}}{\pgfqpoint{4.289998in}{3.842323in}}%
\pgfpathcurveto{\pgfqpoint{4.297811in}{3.834510in}}{\pgfqpoint{4.308410in}{3.830119in}}{\pgfqpoint{4.319461in}{3.830119in}}%
\pgfpathclose%
\pgfusepath{stroke,fill}%
\end{pgfscope}%
\begin{pgfscope}%
\pgfpathrectangle{\pgfqpoint{0.481978in}{0.331635in}}{\pgfqpoint{9.300000in}{7.700000in}}%
\pgfusepath{clip}%
\pgfsetbuttcap%
\pgfsetroundjoin%
\definecolor{currentfill}{rgb}{1.000000,0.705882,0.509804}%
\pgfsetfillcolor{currentfill}%
\pgfsetlinewidth{0.481800pt}%
\definecolor{currentstroke}{rgb}{1.000000,1.000000,1.000000}%
\pgfsetstrokecolor{currentstroke}%
\pgfsetdash{}{0pt}%
\pgfpathmoveto{\pgfqpoint{5.195031in}{3.280936in}}%
\pgfpathcurveto{\pgfqpoint{5.206081in}{3.280936in}}{\pgfqpoint{5.216680in}{3.285326in}}{\pgfqpoint{5.224494in}{3.293140in}}%
\pgfpathcurveto{\pgfqpoint{5.232307in}{3.300953in}}{\pgfqpoint{5.236698in}{3.311552in}}{\pgfqpoint{5.236698in}{3.322603in}}%
\pgfpathcurveto{\pgfqpoint{5.236698in}{3.333653in}}{\pgfqpoint{5.232307in}{3.344252in}}{\pgfqpoint{5.224494in}{3.352065in}}%
\pgfpathcurveto{\pgfqpoint{5.216680in}{3.359879in}}{\pgfqpoint{5.206081in}{3.364269in}}{\pgfqpoint{5.195031in}{3.364269in}}%
\pgfpathcurveto{\pgfqpoint{5.183981in}{3.364269in}}{\pgfqpoint{5.173382in}{3.359879in}}{\pgfqpoint{5.165568in}{3.352065in}}%
\pgfpathcurveto{\pgfqpoint{5.157755in}{3.344252in}}{\pgfqpoint{5.153364in}{3.333653in}}{\pgfqpoint{5.153364in}{3.322603in}}%
\pgfpathcurveto{\pgfqpoint{5.153364in}{3.311552in}}{\pgfqpoint{5.157755in}{3.300953in}}{\pgfqpoint{5.165568in}{3.293140in}}%
\pgfpathcurveto{\pgfqpoint{5.173382in}{3.285326in}}{\pgfqpoint{5.183981in}{3.280936in}}{\pgfqpoint{5.195031in}{3.280936in}}%
\pgfpathclose%
\pgfusepath{stroke,fill}%
\end{pgfscope}%
\begin{pgfscope}%
\pgfpathrectangle{\pgfqpoint{0.481978in}{0.331635in}}{\pgfqpoint{9.300000in}{7.700000in}}%
\pgfusepath{clip}%
\pgfsetbuttcap%
\pgfsetroundjoin%
\definecolor{currentfill}{rgb}{1.000000,0.705882,0.509804}%
\pgfsetfillcolor{currentfill}%
\pgfsetlinewidth{0.481800pt}%
\definecolor{currentstroke}{rgb}{1.000000,1.000000,1.000000}%
\pgfsetstrokecolor{currentstroke}%
\pgfsetdash{}{0pt}%
\pgfpathmoveto{\pgfqpoint{6.327559in}{7.566476in}}%
\pgfpathcurveto{\pgfqpoint{6.338609in}{7.566476in}}{\pgfqpoint{6.349208in}{7.570866in}}{\pgfqpoint{6.357022in}{7.578680in}}%
\pgfpathcurveto{\pgfqpoint{6.364836in}{7.586493in}}{\pgfqpoint{6.369226in}{7.597092in}}{\pgfqpoint{6.369226in}{7.608142in}}%
\pgfpathcurveto{\pgfqpoint{6.369226in}{7.619192in}}{\pgfqpoint{6.364836in}{7.629792in}}{\pgfqpoint{6.357022in}{7.637605in}}%
\pgfpathcurveto{\pgfqpoint{6.349208in}{7.645419in}}{\pgfqpoint{6.338609in}{7.649809in}}{\pgfqpoint{6.327559in}{7.649809in}}%
\pgfpathcurveto{\pgfqpoint{6.316509in}{7.649809in}}{\pgfqpoint{6.305910in}{7.645419in}}{\pgfqpoint{6.298096in}{7.637605in}}%
\pgfpathcurveto{\pgfqpoint{6.290283in}{7.629792in}}{\pgfqpoint{6.285892in}{7.619192in}}{\pgfqpoint{6.285892in}{7.608142in}}%
\pgfpathcurveto{\pgfqpoint{6.285892in}{7.597092in}}{\pgfqpoint{6.290283in}{7.586493in}}{\pgfqpoint{6.298096in}{7.578680in}}%
\pgfpathcurveto{\pgfqpoint{6.305910in}{7.570866in}}{\pgfqpoint{6.316509in}{7.566476in}}{\pgfqpoint{6.327559in}{7.566476in}}%
\pgfpathclose%
\pgfusepath{stroke,fill}%
\end{pgfscope}%
\begin{pgfscope}%
\pgfpathrectangle{\pgfqpoint{0.481978in}{0.331635in}}{\pgfqpoint{9.300000in}{7.700000in}}%
\pgfusepath{clip}%
\pgfsetbuttcap%
\pgfsetroundjoin%
\definecolor{currentfill}{rgb}{1.000000,0.705882,0.509804}%
\pgfsetfillcolor{currentfill}%
\pgfsetlinewidth{0.481800pt}%
\definecolor{currentstroke}{rgb}{1.000000,1.000000,1.000000}%
\pgfsetstrokecolor{currentstroke}%
\pgfsetdash{}{0pt}%
\pgfpathmoveto{\pgfqpoint{6.512708in}{3.228299in}}%
\pgfpathcurveto{\pgfqpoint{6.523758in}{3.228299in}}{\pgfqpoint{6.534357in}{3.232690in}}{\pgfqpoint{6.542171in}{3.240503in}}%
\pgfpathcurveto{\pgfqpoint{6.549984in}{3.248317in}}{\pgfqpoint{6.554375in}{3.258916in}}{\pgfqpoint{6.554375in}{3.269966in}}%
\pgfpathcurveto{\pgfqpoint{6.554375in}{3.281016in}}{\pgfqpoint{6.549984in}{3.291615in}}{\pgfqpoint{6.542171in}{3.299429in}}%
\pgfpathcurveto{\pgfqpoint{6.534357in}{3.307242in}}{\pgfqpoint{6.523758in}{3.311633in}}{\pgfqpoint{6.512708in}{3.311633in}}%
\pgfpathcurveto{\pgfqpoint{6.501658in}{3.311633in}}{\pgfqpoint{6.491059in}{3.307242in}}{\pgfqpoint{6.483245in}{3.299429in}}%
\pgfpathcurveto{\pgfqpoint{6.475432in}{3.291615in}}{\pgfqpoint{6.471041in}{3.281016in}}{\pgfqpoint{6.471041in}{3.269966in}}%
\pgfpathcurveto{\pgfqpoint{6.471041in}{3.258916in}}{\pgfqpoint{6.475432in}{3.248317in}}{\pgfqpoint{6.483245in}{3.240503in}}%
\pgfpathcurveto{\pgfqpoint{6.491059in}{3.232690in}}{\pgfqpoint{6.501658in}{3.228299in}}{\pgfqpoint{6.512708in}{3.228299in}}%
\pgfpathclose%
\pgfusepath{stroke,fill}%
\end{pgfscope}%
\begin{pgfscope}%
\pgfpathrectangle{\pgfqpoint{0.481978in}{0.331635in}}{\pgfqpoint{9.300000in}{7.700000in}}%
\pgfusepath{clip}%
\pgfsetbuttcap%
\pgfsetroundjoin%
\definecolor{currentfill}{rgb}{1.000000,0.705882,0.509804}%
\pgfsetfillcolor{currentfill}%
\pgfsetlinewidth{0.481800pt}%
\definecolor{currentstroke}{rgb}{1.000000,1.000000,1.000000}%
\pgfsetstrokecolor{currentstroke}%
\pgfsetdash{}{0pt}%
\pgfpathmoveto{\pgfqpoint{5.258302in}{2.755878in}}%
\pgfpathcurveto{\pgfqpoint{5.269352in}{2.755878in}}{\pgfqpoint{5.279951in}{2.760268in}}{\pgfqpoint{5.287765in}{2.768082in}}%
\pgfpathcurveto{\pgfqpoint{5.295578in}{2.775896in}}{\pgfqpoint{5.299968in}{2.786495in}}{\pgfqpoint{5.299968in}{2.797545in}}%
\pgfpathcurveto{\pgfqpoint{5.299968in}{2.808595in}}{\pgfqpoint{5.295578in}{2.819194in}}{\pgfqpoint{5.287765in}{2.827008in}}%
\pgfpathcurveto{\pgfqpoint{5.279951in}{2.834821in}}{\pgfqpoint{5.269352in}{2.839211in}}{\pgfqpoint{5.258302in}{2.839211in}}%
\pgfpathcurveto{\pgfqpoint{5.247252in}{2.839211in}}{\pgfqpoint{5.236653in}{2.834821in}}{\pgfqpoint{5.228839in}{2.827008in}}%
\pgfpathcurveto{\pgfqpoint{5.221025in}{2.819194in}}{\pgfqpoint{5.216635in}{2.808595in}}{\pgfqpoint{5.216635in}{2.797545in}}%
\pgfpathcurveto{\pgfqpoint{5.216635in}{2.786495in}}{\pgfqpoint{5.221025in}{2.775896in}}{\pgfqpoint{5.228839in}{2.768082in}}%
\pgfpathcurveto{\pgfqpoint{5.236653in}{2.760268in}}{\pgfqpoint{5.247252in}{2.755878in}}{\pgfqpoint{5.258302in}{2.755878in}}%
\pgfpathclose%
\pgfusepath{stroke,fill}%
\end{pgfscope}%
\begin{pgfscope}%
\pgfpathrectangle{\pgfqpoint{0.481978in}{0.331635in}}{\pgfqpoint{9.300000in}{7.700000in}}%
\pgfusepath{clip}%
\pgfsetbuttcap%
\pgfsetroundjoin%
\definecolor{currentfill}{rgb}{1.000000,0.705882,0.509804}%
\pgfsetfillcolor{currentfill}%
\pgfsetlinewidth{0.481800pt}%
\definecolor{currentstroke}{rgb}{1.000000,1.000000,1.000000}%
\pgfsetstrokecolor{currentstroke}%
\pgfsetdash{}{0pt}%
\pgfpathmoveto{\pgfqpoint{8.656102in}{6.052204in}}%
\pgfpathcurveto{\pgfqpoint{8.667152in}{6.052204in}}{\pgfqpoint{8.677751in}{6.056594in}}{\pgfqpoint{8.685565in}{6.064408in}}%
\pgfpathcurveto{\pgfqpoint{8.693379in}{6.072222in}}{\pgfqpoint{8.697769in}{6.082821in}}{\pgfqpoint{8.697769in}{6.093871in}}%
\pgfpathcurveto{\pgfqpoint{8.697769in}{6.104921in}}{\pgfqpoint{8.693379in}{6.115520in}}{\pgfqpoint{8.685565in}{6.123334in}}%
\pgfpathcurveto{\pgfqpoint{8.677751in}{6.131147in}}{\pgfqpoint{8.667152in}{6.135538in}}{\pgfqpoint{8.656102in}{6.135538in}}%
\pgfpathcurveto{\pgfqpoint{8.645052in}{6.135538in}}{\pgfqpoint{8.634453in}{6.131147in}}{\pgfqpoint{8.626639in}{6.123334in}}%
\pgfpathcurveto{\pgfqpoint{8.618826in}{6.115520in}}{\pgfqpoint{8.614435in}{6.104921in}}{\pgfqpoint{8.614435in}{6.093871in}}%
\pgfpathcurveto{\pgfqpoint{8.614435in}{6.082821in}}{\pgfqpoint{8.618826in}{6.072222in}}{\pgfqpoint{8.626639in}{6.064408in}}%
\pgfpathcurveto{\pgfqpoint{8.634453in}{6.056594in}}{\pgfqpoint{8.645052in}{6.052204in}}{\pgfqpoint{8.656102in}{6.052204in}}%
\pgfpathclose%
\pgfusepath{stroke,fill}%
\end{pgfscope}%
\begin{pgfscope}%
\pgfpathrectangle{\pgfqpoint{0.481978in}{0.331635in}}{\pgfqpoint{9.300000in}{7.700000in}}%
\pgfusepath{clip}%
\pgfsetbuttcap%
\pgfsetroundjoin%
\definecolor{currentfill}{rgb}{1.000000,0.705882,0.509804}%
\pgfsetfillcolor{currentfill}%
\pgfsetlinewidth{0.481800pt}%
\definecolor{currentstroke}{rgb}{1.000000,1.000000,1.000000}%
\pgfsetstrokecolor{currentstroke}%
\pgfsetdash{}{0pt}%
\pgfpathmoveto{\pgfqpoint{5.476507in}{4.174809in}}%
\pgfpathcurveto{\pgfqpoint{5.487557in}{4.174809in}}{\pgfqpoint{5.498156in}{4.179200in}}{\pgfqpoint{5.505970in}{4.187013in}}%
\pgfpathcurveto{\pgfqpoint{5.513783in}{4.194827in}}{\pgfqpoint{5.518173in}{4.205426in}}{\pgfqpoint{5.518173in}{4.216476in}}%
\pgfpathcurveto{\pgfqpoint{5.518173in}{4.227526in}}{\pgfqpoint{5.513783in}{4.238125in}}{\pgfqpoint{5.505970in}{4.245939in}}%
\pgfpathcurveto{\pgfqpoint{5.498156in}{4.253752in}}{\pgfqpoint{5.487557in}{4.258143in}}{\pgfqpoint{5.476507in}{4.258143in}}%
\pgfpathcurveto{\pgfqpoint{5.465457in}{4.258143in}}{\pgfqpoint{5.454858in}{4.253752in}}{\pgfqpoint{5.447044in}{4.245939in}}%
\pgfpathcurveto{\pgfqpoint{5.439230in}{4.238125in}}{\pgfqpoint{5.434840in}{4.227526in}}{\pgfqpoint{5.434840in}{4.216476in}}%
\pgfpathcurveto{\pgfqpoint{5.434840in}{4.205426in}}{\pgfqpoint{5.439230in}{4.194827in}}{\pgfqpoint{5.447044in}{4.187013in}}%
\pgfpathcurveto{\pgfqpoint{5.454858in}{4.179200in}}{\pgfqpoint{5.465457in}{4.174809in}}{\pgfqpoint{5.476507in}{4.174809in}}%
\pgfpathclose%
\pgfusepath{stroke,fill}%
\end{pgfscope}%
\begin{pgfscope}%
\pgfpathrectangle{\pgfqpoint{0.481978in}{0.331635in}}{\pgfqpoint{9.300000in}{7.700000in}}%
\pgfusepath{clip}%
\pgfsetbuttcap%
\pgfsetroundjoin%
\definecolor{currentfill}{rgb}{1.000000,0.705882,0.509804}%
\pgfsetfillcolor{currentfill}%
\pgfsetlinewidth{0.481800pt}%
\definecolor{currentstroke}{rgb}{1.000000,1.000000,1.000000}%
\pgfsetstrokecolor{currentstroke}%
\pgfsetdash{}{0pt}%
\pgfpathmoveto{\pgfqpoint{7.379869in}{3.247695in}}%
\pgfpathcurveto{\pgfqpoint{7.390920in}{3.247695in}}{\pgfqpoint{7.401519in}{3.252085in}}{\pgfqpoint{7.409332in}{3.259899in}}%
\pgfpathcurveto{\pgfqpoint{7.417146in}{3.267712in}}{\pgfqpoint{7.421536in}{3.278311in}}{\pgfqpoint{7.421536in}{3.289361in}}%
\pgfpathcurveto{\pgfqpoint{7.421536in}{3.300412in}}{\pgfqpoint{7.417146in}{3.311011in}}{\pgfqpoint{7.409332in}{3.318824in}}%
\pgfpathcurveto{\pgfqpoint{7.401519in}{3.326638in}}{\pgfqpoint{7.390920in}{3.331028in}}{\pgfqpoint{7.379869in}{3.331028in}}%
\pgfpathcurveto{\pgfqpoint{7.368819in}{3.331028in}}{\pgfqpoint{7.358220in}{3.326638in}}{\pgfqpoint{7.350407in}{3.318824in}}%
\pgfpathcurveto{\pgfqpoint{7.342593in}{3.311011in}}{\pgfqpoint{7.338203in}{3.300412in}}{\pgfqpoint{7.338203in}{3.289361in}}%
\pgfpathcurveto{\pgfqpoint{7.338203in}{3.278311in}}{\pgfqpoint{7.342593in}{3.267712in}}{\pgfqpoint{7.350407in}{3.259899in}}%
\pgfpathcurveto{\pgfqpoint{7.358220in}{3.252085in}}{\pgfqpoint{7.368819in}{3.247695in}}{\pgfqpoint{7.379869in}{3.247695in}}%
\pgfpathclose%
\pgfusepath{stroke,fill}%
\end{pgfscope}%
\begin{pgfscope}%
\pgfpathrectangle{\pgfqpoint{0.481978in}{0.331635in}}{\pgfqpoint{9.300000in}{7.700000in}}%
\pgfusepath{clip}%
\pgfsetbuttcap%
\pgfsetroundjoin%
\definecolor{currentfill}{rgb}{1.000000,0.705882,0.509804}%
\pgfsetfillcolor{currentfill}%
\pgfsetlinewidth{0.481800pt}%
\definecolor{currentstroke}{rgb}{1.000000,1.000000,1.000000}%
\pgfsetstrokecolor{currentstroke}%
\pgfsetdash{}{0pt}%
\pgfpathmoveto{\pgfqpoint{6.352024in}{2.733115in}}%
\pgfpathcurveto{\pgfqpoint{6.363074in}{2.733115in}}{\pgfqpoint{6.373673in}{2.737505in}}{\pgfqpoint{6.381486in}{2.745318in}}%
\pgfpathcurveto{\pgfqpoint{6.389300in}{2.753132in}}{\pgfqpoint{6.393690in}{2.763731in}}{\pgfqpoint{6.393690in}{2.774781in}}%
\pgfpathcurveto{\pgfqpoint{6.393690in}{2.785831in}}{\pgfqpoint{6.389300in}{2.796430in}}{\pgfqpoint{6.381486in}{2.804244in}}%
\pgfpathcurveto{\pgfqpoint{6.373673in}{2.812058in}}{\pgfqpoint{6.363074in}{2.816448in}}{\pgfqpoint{6.352024in}{2.816448in}}%
\pgfpathcurveto{\pgfqpoint{6.340974in}{2.816448in}}{\pgfqpoint{6.330375in}{2.812058in}}{\pgfqpoint{6.322561in}{2.804244in}}%
\pgfpathcurveto{\pgfqpoint{6.314747in}{2.796430in}}{\pgfqpoint{6.310357in}{2.785831in}}{\pgfqpoint{6.310357in}{2.774781in}}%
\pgfpathcurveto{\pgfqpoint{6.310357in}{2.763731in}}{\pgfqpoint{6.314747in}{2.753132in}}{\pgfqpoint{6.322561in}{2.745318in}}%
\pgfpathcurveto{\pgfqpoint{6.330375in}{2.737505in}}{\pgfqpoint{6.340974in}{2.733115in}}{\pgfqpoint{6.352024in}{2.733115in}}%
\pgfpathclose%
\pgfusepath{stroke,fill}%
\end{pgfscope}%
\begin{pgfscope}%
\pgfpathrectangle{\pgfqpoint{0.481978in}{0.331635in}}{\pgfqpoint{9.300000in}{7.700000in}}%
\pgfusepath{clip}%
\pgfsetbuttcap%
\pgfsetroundjoin%
\definecolor{currentfill}{rgb}{1.000000,0.705882,0.509804}%
\pgfsetfillcolor{currentfill}%
\pgfsetlinewidth{0.481800pt}%
\definecolor{currentstroke}{rgb}{1.000000,1.000000,1.000000}%
\pgfsetstrokecolor{currentstroke}%
\pgfsetdash{}{0pt}%
\pgfpathmoveto{\pgfqpoint{7.010012in}{6.613365in}}%
\pgfpathcurveto{\pgfqpoint{7.021062in}{6.613365in}}{\pgfqpoint{7.031662in}{6.617755in}}{\pgfqpoint{7.039475in}{6.625569in}}%
\pgfpathcurveto{\pgfqpoint{7.047289in}{6.633382in}}{\pgfqpoint{7.051679in}{6.643981in}}{\pgfqpoint{7.051679in}{6.655032in}}%
\pgfpathcurveto{\pgfqpoint{7.051679in}{6.666082in}}{\pgfqpoint{7.047289in}{6.676681in}}{\pgfqpoint{7.039475in}{6.684494in}}%
\pgfpathcurveto{\pgfqpoint{7.031662in}{6.692308in}}{\pgfqpoint{7.021062in}{6.696698in}}{\pgfqpoint{7.010012in}{6.696698in}}%
\pgfpathcurveto{\pgfqpoint{6.998962in}{6.696698in}}{\pgfqpoint{6.988363in}{6.692308in}}{\pgfqpoint{6.980550in}{6.684494in}}%
\pgfpathcurveto{\pgfqpoint{6.972736in}{6.676681in}}{\pgfqpoint{6.968346in}{6.666082in}}{\pgfqpoint{6.968346in}{6.655032in}}%
\pgfpathcurveto{\pgfqpoint{6.968346in}{6.643981in}}{\pgfqpoint{6.972736in}{6.633382in}}{\pgfqpoint{6.980550in}{6.625569in}}%
\pgfpathcurveto{\pgfqpoint{6.988363in}{6.617755in}}{\pgfqpoint{6.998962in}{6.613365in}}{\pgfqpoint{7.010012in}{6.613365in}}%
\pgfpathclose%
\pgfusepath{stroke,fill}%
\end{pgfscope}%
\begin{pgfscope}%
\pgfpathrectangle{\pgfqpoint{0.481978in}{0.331635in}}{\pgfqpoint{9.300000in}{7.700000in}}%
\pgfusepath{clip}%
\pgfsetbuttcap%
\pgfsetroundjoin%
\definecolor{currentfill}{rgb}{1.000000,0.705882,0.509804}%
\pgfsetfillcolor{currentfill}%
\pgfsetlinewidth{0.481800pt}%
\definecolor{currentstroke}{rgb}{1.000000,1.000000,1.000000}%
\pgfsetstrokecolor{currentstroke}%
\pgfsetdash{}{0pt}%
\pgfpathmoveto{\pgfqpoint{4.957439in}{2.224376in}}%
\pgfpathcurveto{\pgfqpoint{4.968489in}{2.224376in}}{\pgfqpoint{4.979088in}{2.228766in}}{\pgfqpoint{4.986902in}{2.236580in}}%
\pgfpathcurveto{\pgfqpoint{4.994715in}{2.244394in}}{\pgfqpoint{4.999106in}{2.254993in}}{\pgfqpoint{4.999106in}{2.266043in}}%
\pgfpathcurveto{\pgfqpoint{4.999106in}{2.277093in}}{\pgfqpoint{4.994715in}{2.287692in}}{\pgfqpoint{4.986902in}{2.295506in}}%
\pgfpathcurveto{\pgfqpoint{4.979088in}{2.303319in}}{\pgfqpoint{4.968489in}{2.307710in}}{\pgfqpoint{4.957439in}{2.307710in}}%
\pgfpathcurveto{\pgfqpoint{4.946389in}{2.307710in}}{\pgfqpoint{4.935790in}{2.303319in}}{\pgfqpoint{4.927976in}{2.295506in}}%
\pgfpathcurveto{\pgfqpoint{4.920163in}{2.287692in}}{\pgfqpoint{4.915772in}{2.277093in}}{\pgfqpoint{4.915772in}{2.266043in}}%
\pgfpathcurveto{\pgfqpoint{4.915772in}{2.254993in}}{\pgfqpoint{4.920163in}{2.244394in}}{\pgfqpoint{4.927976in}{2.236580in}}%
\pgfpathcurveto{\pgfqpoint{4.935790in}{2.228766in}}{\pgfqpoint{4.946389in}{2.224376in}}{\pgfqpoint{4.957439in}{2.224376in}}%
\pgfpathclose%
\pgfusepath{stroke,fill}%
\end{pgfscope}%
\begin{pgfscope}%
\pgfpathrectangle{\pgfqpoint{0.481978in}{0.331635in}}{\pgfqpoint{9.300000in}{7.700000in}}%
\pgfusepath{clip}%
\pgfsetbuttcap%
\pgfsetroundjoin%
\definecolor{currentfill}{rgb}{1.000000,0.705882,0.509804}%
\pgfsetfillcolor{currentfill}%
\pgfsetlinewidth{0.481800pt}%
\definecolor{currentstroke}{rgb}{1.000000,1.000000,1.000000}%
\pgfsetstrokecolor{currentstroke}%
\pgfsetdash{}{0pt}%
\pgfpathmoveto{\pgfqpoint{1.938901in}{5.922513in}}%
\pgfpathcurveto{\pgfqpoint{1.949952in}{5.922513in}}{\pgfqpoint{1.960551in}{5.926903in}}{\pgfqpoint{1.968364in}{5.934716in}}%
\pgfpathcurveto{\pgfqpoint{1.976178in}{5.942530in}}{\pgfqpoint{1.980568in}{5.953129in}}{\pgfqpoint{1.980568in}{5.964179in}}%
\pgfpathcurveto{\pgfqpoint{1.980568in}{5.975229in}}{\pgfqpoint{1.976178in}{5.985828in}}{\pgfqpoint{1.968364in}{5.993642in}}%
\pgfpathcurveto{\pgfqpoint{1.960551in}{6.001456in}}{\pgfqpoint{1.949952in}{6.005846in}}{\pgfqpoint{1.938901in}{6.005846in}}%
\pgfpathcurveto{\pgfqpoint{1.927851in}{6.005846in}}{\pgfqpoint{1.917252in}{6.001456in}}{\pgfqpoint{1.909439in}{5.993642in}}%
\pgfpathcurveto{\pgfqpoint{1.901625in}{5.985828in}}{\pgfqpoint{1.897235in}{5.975229in}}{\pgfqpoint{1.897235in}{5.964179in}}%
\pgfpathcurveto{\pgfqpoint{1.897235in}{5.953129in}}{\pgfqpoint{1.901625in}{5.942530in}}{\pgfqpoint{1.909439in}{5.934716in}}%
\pgfpathcurveto{\pgfqpoint{1.917252in}{5.926903in}}{\pgfqpoint{1.927851in}{5.922513in}}{\pgfqpoint{1.938901in}{5.922513in}}%
\pgfpathclose%
\pgfusepath{stroke,fill}%
\end{pgfscope}%
\begin{pgfscope}%
\pgfpathrectangle{\pgfqpoint{0.481978in}{0.331635in}}{\pgfqpoint{9.300000in}{7.700000in}}%
\pgfusepath{clip}%
\pgfsetbuttcap%
\pgfsetroundjoin%
\definecolor{currentfill}{rgb}{1.000000,0.705882,0.509804}%
\pgfsetfillcolor{currentfill}%
\pgfsetlinewidth{0.481800pt}%
\definecolor{currentstroke}{rgb}{1.000000,1.000000,1.000000}%
\pgfsetstrokecolor{currentstroke}%
\pgfsetdash{}{0pt}%
\pgfpathmoveto{\pgfqpoint{6.653857in}{6.971094in}}%
\pgfpathcurveto{\pgfqpoint{6.664907in}{6.971094in}}{\pgfqpoint{6.675506in}{6.975485in}}{\pgfqpoint{6.683320in}{6.983298in}}%
\pgfpathcurveto{\pgfqpoint{6.691133in}{6.991112in}}{\pgfqpoint{6.695524in}{7.001711in}}{\pgfqpoint{6.695524in}{7.012761in}}%
\pgfpathcurveto{\pgfqpoint{6.695524in}{7.023811in}}{\pgfqpoint{6.691133in}{7.034410in}}{\pgfqpoint{6.683320in}{7.042224in}}%
\pgfpathcurveto{\pgfqpoint{6.675506in}{7.050037in}}{\pgfqpoint{6.664907in}{7.054428in}}{\pgfqpoint{6.653857in}{7.054428in}}%
\pgfpathcurveto{\pgfqpoint{6.642807in}{7.054428in}}{\pgfqpoint{6.632208in}{7.050037in}}{\pgfqpoint{6.624394in}{7.042224in}}%
\pgfpathcurveto{\pgfqpoint{6.616581in}{7.034410in}}{\pgfqpoint{6.612190in}{7.023811in}}{\pgfqpoint{6.612190in}{7.012761in}}%
\pgfpathcurveto{\pgfqpoint{6.612190in}{7.001711in}}{\pgfqpoint{6.616581in}{6.991112in}}{\pgfqpoint{6.624394in}{6.983298in}}%
\pgfpathcurveto{\pgfqpoint{6.632208in}{6.975485in}}{\pgfqpoint{6.642807in}{6.971094in}}{\pgfqpoint{6.653857in}{6.971094in}}%
\pgfpathclose%
\pgfusepath{stroke,fill}%
\end{pgfscope}%
\begin{pgfscope}%
\pgfpathrectangle{\pgfqpoint{0.481978in}{0.331635in}}{\pgfqpoint{9.300000in}{7.700000in}}%
\pgfusepath{clip}%
\pgfsetbuttcap%
\pgfsetroundjoin%
\definecolor{currentfill}{rgb}{1.000000,0.705882,0.509804}%
\pgfsetfillcolor{currentfill}%
\pgfsetlinewidth{0.481800pt}%
\definecolor{currentstroke}{rgb}{1.000000,1.000000,1.000000}%
\pgfsetstrokecolor{currentstroke}%
\pgfsetdash{}{0pt}%
\pgfpathmoveto{\pgfqpoint{7.825213in}{7.117074in}}%
\pgfpathcurveto{\pgfqpoint{7.836263in}{7.117074in}}{\pgfqpoint{7.846862in}{7.121465in}}{\pgfqpoint{7.854676in}{7.129278in}}%
\pgfpathcurveto{\pgfqpoint{7.862489in}{7.137092in}}{\pgfqpoint{7.866880in}{7.147691in}}{\pgfqpoint{7.866880in}{7.158741in}}%
\pgfpathcurveto{\pgfqpoint{7.866880in}{7.169791in}}{\pgfqpoint{7.862489in}{7.180390in}}{\pgfqpoint{7.854676in}{7.188204in}}%
\pgfpathcurveto{\pgfqpoint{7.846862in}{7.196017in}}{\pgfqpoint{7.836263in}{7.200408in}}{\pgfqpoint{7.825213in}{7.200408in}}%
\pgfpathcurveto{\pgfqpoint{7.814163in}{7.200408in}}{\pgfqpoint{7.803564in}{7.196017in}}{\pgfqpoint{7.795750in}{7.188204in}}%
\pgfpathcurveto{\pgfqpoint{7.787937in}{7.180390in}}{\pgfqpoint{7.783546in}{7.169791in}}{\pgfqpoint{7.783546in}{7.158741in}}%
\pgfpathcurveto{\pgfqpoint{7.783546in}{7.147691in}}{\pgfqpoint{7.787937in}{7.137092in}}{\pgfqpoint{7.795750in}{7.129278in}}%
\pgfpathcurveto{\pgfqpoint{7.803564in}{7.121465in}}{\pgfqpoint{7.814163in}{7.117074in}}{\pgfqpoint{7.825213in}{7.117074in}}%
\pgfpathclose%
\pgfusepath{stroke,fill}%
\end{pgfscope}%
\begin{pgfscope}%
\pgfpathrectangle{\pgfqpoint{0.481978in}{0.331635in}}{\pgfqpoint{9.300000in}{7.700000in}}%
\pgfusepath{clip}%
\pgfsetbuttcap%
\pgfsetroundjoin%
\definecolor{currentfill}{rgb}{1.000000,0.705882,0.509804}%
\pgfsetfillcolor{currentfill}%
\pgfsetlinewidth{0.481800pt}%
\definecolor{currentstroke}{rgb}{1.000000,1.000000,1.000000}%
\pgfsetstrokecolor{currentstroke}%
\pgfsetdash{}{0pt}%
\pgfpathmoveto{\pgfqpoint{4.408915in}{2.505647in}}%
\pgfpathcurveto{\pgfqpoint{4.419965in}{2.505647in}}{\pgfqpoint{4.430564in}{2.510037in}}{\pgfqpoint{4.438378in}{2.517850in}}%
\pgfpathcurveto{\pgfqpoint{4.446191in}{2.525664in}}{\pgfqpoint{4.450582in}{2.536263in}}{\pgfqpoint{4.450582in}{2.547313in}}%
\pgfpathcurveto{\pgfqpoint{4.450582in}{2.558363in}}{\pgfqpoint{4.446191in}{2.568962in}}{\pgfqpoint{4.438378in}{2.576776in}}%
\pgfpathcurveto{\pgfqpoint{4.430564in}{2.584590in}}{\pgfqpoint{4.419965in}{2.588980in}}{\pgfqpoint{4.408915in}{2.588980in}}%
\pgfpathcurveto{\pgfqpoint{4.397865in}{2.588980in}}{\pgfqpoint{4.387266in}{2.584590in}}{\pgfqpoint{4.379452in}{2.576776in}}%
\pgfpathcurveto{\pgfqpoint{4.371638in}{2.568962in}}{\pgfqpoint{4.367248in}{2.558363in}}{\pgfqpoint{4.367248in}{2.547313in}}%
\pgfpathcurveto{\pgfqpoint{4.367248in}{2.536263in}}{\pgfqpoint{4.371638in}{2.525664in}}{\pgfqpoint{4.379452in}{2.517850in}}%
\pgfpathcurveto{\pgfqpoint{4.387266in}{2.510037in}}{\pgfqpoint{4.397865in}{2.505647in}}{\pgfqpoint{4.408915in}{2.505647in}}%
\pgfpathclose%
\pgfusepath{stroke,fill}%
\end{pgfscope}%
\begin{pgfscope}%
\pgfpathrectangle{\pgfqpoint{0.481978in}{0.331635in}}{\pgfqpoint{9.300000in}{7.700000in}}%
\pgfusepath{clip}%
\pgfsetbuttcap%
\pgfsetroundjoin%
\definecolor{currentfill}{rgb}{1.000000,0.705882,0.509804}%
\pgfsetfillcolor{currentfill}%
\pgfsetlinewidth{0.481800pt}%
\definecolor{currentstroke}{rgb}{1.000000,1.000000,1.000000}%
\pgfsetstrokecolor{currentstroke}%
\pgfsetdash{}{0pt}%
\pgfpathmoveto{\pgfqpoint{0.904705in}{3.119494in}}%
\pgfpathcurveto{\pgfqpoint{0.915755in}{3.119494in}}{\pgfqpoint{0.926354in}{3.123885in}}{\pgfqpoint{0.934168in}{3.131698in}}%
\pgfpathcurveto{\pgfqpoint{0.941982in}{3.139512in}}{\pgfqpoint{0.946372in}{3.150111in}}{\pgfqpoint{0.946372in}{3.161161in}}%
\pgfpathcurveto{\pgfqpoint{0.946372in}{3.172211in}}{\pgfqpoint{0.941982in}{3.182810in}}{\pgfqpoint{0.934168in}{3.190624in}}%
\pgfpathcurveto{\pgfqpoint{0.926354in}{3.198437in}}{\pgfqpoint{0.915755in}{3.202828in}}{\pgfqpoint{0.904705in}{3.202828in}}%
\pgfpathcurveto{\pgfqpoint{0.893655in}{3.202828in}}{\pgfqpoint{0.883056in}{3.198437in}}{\pgfqpoint{0.875242in}{3.190624in}}%
\pgfpathcurveto{\pgfqpoint{0.867429in}{3.182810in}}{\pgfqpoint{0.863039in}{3.172211in}}{\pgfqpoint{0.863039in}{3.161161in}}%
\pgfpathcurveto{\pgfqpoint{0.863039in}{3.150111in}}{\pgfqpoint{0.867429in}{3.139512in}}{\pgfqpoint{0.875242in}{3.131698in}}%
\pgfpathcurveto{\pgfqpoint{0.883056in}{3.123885in}}{\pgfqpoint{0.893655in}{3.119494in}}{\pgfqpoint{0.904705in}{3.119494in}}%
\pgfpathclose%
\pgfusepath{stroke,fill}%
\end{pgfscope}%
\begin{pgfscope}%
\pgfpathrectangle{\pgfqpoint{0.481978in}{0.331635in}}{\pgfqpoint{9.300000in}{7.700000in}}%
\pgfusepath{clip}%
\pgfsetbuttcap%
\pgfsetroundjoin%
\definecolor{currentfill}{rgb}{1.000000,0.705882,0.509804}%
\pgfsetfillcolor{currentfill}%
\pgfsetlinewidth{0.481800pt}%
\definecolor{currentstroke}{rgb}{1.000000,1.000000,1.000000}%
\pgfsetstrokecolor{currentstroke}%
\pgfsetdash{}{0pt}%
\pgfpathmoveto{\pgfqpoint{8.224572in}{3.304265in}}%
\pgfpathcurveto{\pgfqpoint{8.235623in}{3.304265in}}{\pgfqpoint{8.246222in}{3.308655in}}{\pgfqpoint{8.254035in}{3.316469in}}%
\pgfpathcurveto{\pgfqpoint{8.261849in}{3.324282in}}{\pgfqpoint{8.266239in}{3.334881in}}{\pgfqpoint{8.266239in}{3.345931in}}%
\pgfpathcurveto{\pgfqpoint{8.266239in}{3.356981in}}{\pgfqpoint{8.261849in}{3.367580in}}{\pgfqpoint{8.254035in}{3.375394in}}%
\pgfpathcurveto{\pgfqpoint{8.246222in}{3.383208in}}{\pgfqpoint{8.235623in}{3.387598in}}{\pgfqpoint{8.224572in}{3.387598in}}%
\pgfpathcurveto{\pgfqpoint{8.213522in}{3.387598in}}{\pgfqpoint{8.202923in}{3.383208in}}{\pgfqpoint{8.195110in}{3.375394in}}%
\pgfpathcurveto{\pgfqpoint{8.187296in}{3.367580in}}{\pgfqpoint{8.182906in}{3.356981in}}{\pgfqpoint{8.182906in}{3.345931in}}%
\pgfpathcurveto{\pgfqpoint{8.182906in}{3.334881in}}{\pgfqpoint{8.187296in}{3.324282in}}{\pgfqpoint{8.195110in}{3.316469in}}%
\pgfpathcurveto{\pgfqpoint{8.202923in}{3.308655in}}{\pgfqpoint{8.213522in}{3.304265in}}{\pgfqpoint{8.224572in}{3.304265in}}%
\pgfpathclose%
\pgfusepath{stroke,fill}%
\end{pgfscope}%
\begin{pgfscope}%
\pgfpathrectangle{\pgfqpoint{0.481978in}{0.331635in}}{\pgfqpoint{9.300000in}{7.700000in}}%
\pgfusepath{clip}%
\pgfsetbuttcap%
\pgfsetroundjoin%
\definecolor{currentfill}{rgb}{0.631373,0.788235,0.956863}%
\pgfsetfillcolor{currentfill}%
\pgfsetlinewidth{1.003750pt}%
\definecolor{currentstroke}{rgb}{0.631373,0.788235,0.956863}%
\pgfsetstrokecolor{currentstroke}%
\pgfsetdash{}{0pt}%
\pgfsys@defobject{currentmarker}{\pgfqpoint{-0.041667in}{-0.041667in}}{\pgfqpoint{0.041667in}{0.041667in}}{%
\pgfpathmoveto{\pgfqpoint{0.000000in}{-0.041667in}}%
\pgfpathcurveto{\pgfqpoint{0.011050in}{-0.041667in}}{\pgfqpoint{0.021649in}{-0.037276in}}{\pgfqpoint{0.029463in}{-0.029463in}}%
\pgfpathcurveto{\pgfqpoint{0.037276in}{-0.021649in}}{\pgfqpoint{0.041667in}{-0.011050in}}{\pgfqpoint{0.041667in}{0.000000in}}%
\pgfpathcurveto{\pgfqpoint{0.041667in}{0.011050in}}{\pgfqpoint{0.037276in}{0.021649in}}{\pgfqpoint{0.029463in}{0.029463in}}%
\pgfpathcurveto{\pgfqpoint{0.021649in}{0.037276in}}{\pgfqpoint{0.011050in}{0.041667in}}{\pgfqpoint{0.000000in}{0.041667in}}%
\pgfpathcurveto{\pgfqpoint{-0.011050in}{0.041667in}}{\pgfqpoint{-0.021649in}{0.037276in}}{\pgfqpoint{-0.029463in}{0.029463in}}%
\pgfpathcurveto{\pgfqpoint{-0.037276in}{0.021649in}}{\pgfqpoint{-0.041667in}{0.011050in}}{\pgfqpoint{-0.041667in}{0.000000in}}%
\pgfpathcurveto{\pgfqpoint{-0.041667in}{-0.011050in}}{\pgfqpoint{-0.037276in}{-0.021649in}}{\pgfqpoint{-0.029463in}{-0.029463in}}%
\pgfpathcurveto{\pgfqpoint{-0.021649in}{-0.037276in}}{\pgfqpoint{-0.011050in}{-0.041667in}}{\pgfqpoint{0.000000in}{-0.041667in}}%
\pgfpathclose%
\pgfusepath{stroke,fill}%
}%
\end{pgfscope}%
\begin{pgfscope}%
\pgfpathrectangle{\pgfqpoint{0.481978in}{0.331635in}}{\pgfqpoint{9.300000in}{7.700000in}}%
\pgfusepath{clip}%
\pgfsetbuttcap%
\pgfsetroundjoin%
\definecolor{currentfill}{rgb}{1.000000,0.705882,0.509804}%
\pgfsetfillcolor{currentfill}%
\pgfsetlinewidth{1.003750pt}%
\definecolor{currentstroke}{rgb}{1.000000,0.705882,0.509804}%
\pgfsetstrokecolor{currentstroke}%
\pgfsetdash{}{0pt}%
\pgfsys@defobject{currentmarker}{\pgfqpoint{-0.041667in}{-0.041667in}}{\pgfqpoint{0.041667in}{0.041667in}}{%
\pgfpathmoveto{\pgfqpoint{0.000000in}{-0.041667in}}%
\pgfpathcurveto{\pgfqpoint{0.011050in}{-0.041667in}}{\pgfqpoint{0.021649in}{-0.037276in}}{\pgfqpoint{0.029463in}{-0.029463in}}%
\pgfpathcurveto{\pgfqpoint{0.037276in}{-0.021649in}}{\pgfqpoint{0.041667in}{-0.011050in}}{\pgfqpoint{0.041667in}{0.000000in}}%
\pgfpathcurveto{\pgfqpoint{0.041667in}{0.011050in}}{\pgfqpoint{0.037276in}{0.021649in}}{\pgfqpoint{0.029463in}{0.029463in}}%
\pgfpathcurveto{\pgfqpoint{0.021649in}{0.037276in}}{\pgfqpoint{0.011050in}{0.041667in}}{\pgfqpoint{0.000000in}{0.041667in}}%
\pgfpathcurveto{\pgfqpoint{-0.011050in}{0.041667in}}{\pgfqpoint{-0.021649in}{0.037276in}}{\pgfqpoint{-0.029463in}{0.029463in}}%
\pgfpathcurveto{\pgfqpoint{-0.037276in}{0.021649in}}{\pgfqpoint{-0.041667in}{0.011050in}}{\pgfqpoint{-0.041667in}{0.000000in}}%
\pgfpathcurveto{\pgfqpoint{-0.041667in}{-0.011050in}}{\pgfqpoint{-0.037276in}{-0.021649in}}{\pgfqpoint{-0.029463in}{-0.029463in}}%
\pgfpathcurveto{\pgfqpoint{-0.021649in}{-0.037276in}}{\pgfqpoint{-0.011050in}{-0.041667in}}{\pgfqpoint{0.000000in}{-0.041667in}}%
\pgfpathclose%
\pgfusepath{stroke,fill}%
}%
\end{pgfscope}%
\begin{pgfscope}%
\pgfsetbuttcap%
\pgfsetroundjoin%
\definecolor{currentfill}{rgb}{0.000000,0.000000,0.000000}%
\pgfsetfillcolor{currentfill}%
\pgfsetlinewidth{0.803000pt}%
\definecolor{currentstroke}{rgb}{0.000000,0.000000,0.000000}%
\pgfsetstrokecolor{currentstroke}%
\pgfsetdash{}{0pt}%
\pgfsys@defobject{currentmarker}{\pgfqpoint{0.000000in}{-0.048611in}}{\pgfqpoint{0.000000in}{0.000000in}}{%
\pgfpathmoveto{\pgfqpoint{0.000000in}{0.000000in}}%
\pgfpathlineto{\pgfqpoint{0.000000in}{-0.048611in}}%
\pgfusepath{stroke,fill}%
}%
\begin{pgfscope}%
\pgfsys@transformshift{0.741718in}{0.331635in}%
\pgfsys@useobject{currentmarker}{}%
\end{pgfscope}%
\end{pgfscope}%
\begin{pgfscope}%
\definecolor{textcolor}{rgb}{0.000000,0.000000,0.000000}%
\pgfsetstrokecolor{textcolor}%
\pgfsetfillcolor{textcolor}%
\pgftext[x=0.741718in,y=0.234413in,,top]{\color{textcolor}\sffamily\fontsize{10.000000}{12.000000}\selectfont \ensuremath{-}60}%
\end{pgfscope}%
\begin{pgfscope}%
\pgfsetbuttcap%
\pgfsetroundjoin%
\definecolor{currentfill}{rgb}{0.000000,0.000000,0.000000}%
\pgfsetfillcolor{currentfill}%
\pgfsetlinewidth{0.803000pt}%
\definecolor{currentstroke}{rgb}{0.000000,0.000000,0.000000}%
\pgfsetstrokecolor{currentstroke}%
\pgfsetdash{}{0pt}%
\pgfsys@defobject{currentmarker}{\pgfqpoint{0.000000in}{-0.048611in}}{\pgfqpoint{0.000000in}{0.000000in}}{%
\pgfpathmoveto{\pgfqpoint{0.000000in}{0.000000in}}%
\pgfpathlineto{\pgfqpoint{0.000000in}{-0.048611in}}%
\pgfusepath{stroke,fill}%
}%
\begin{pgfscope}%
\pgfsys@transformshift{2.361715in}{0.331635in}%
\pgfsys@useobject{currentmarker}{}%
\end{pgfscope}%
\end{pgfscope}%
\begin{pgfscope}%
\definecolor{textcolor}{rgb}{0.000000,0.000000,0.000000}%
\pgfsetstrokecolor{textcolor}%
\pgfsetfillcolor{textcolor}%
\pgftext[x=2.361715in,y=0.234413in,,top]{\color{textcolor}\sffamily\fontsize{10.000000}{12.000000}\selectfont \ensuremath{-}40}%
\end{pgfscope}%
\begin{pgfscope}%
\pgfsetbuttcap%
\pgfsetroundjoin%
\definecolor{currentfill}{rgb}{0.000000,0.000000,0.000000}%
\pgfsetfillcolor{currentfill}%
\pgfsetlinewidth{0.803000pt}%
\definecolor{currentstroke}{rgb}{0.000000,0.000000,0.000000}%
\pgfsetstrokecolor{currentstroke}%
\pgfsetdash{}{0pt}%
\pgfsys@defobject{currentmarker}{\pgfqpoint{0.000000in}{-0.048611in}}{\pgfqpoint{0.000000in}{0.000000in}}{%
\pgfpathmoveto{\pgfqpoint{0.000000in}{0.000000in}}%
\pgfpathlineto{\pgfqpoint{0.000000in}{-0.048611in}}%
\pgfusepath{stroke,fill}%
}%
\begin{pgfscope}%
\pgfsys@transformshift{3.981712in}{0.331635in}%
\pgfsys@useobject{currentmarker}{}%
\end{pgfscope}%
\end{pgfscope}%
\begin{pgfscope}%
\definecolor{textcolor}{rgb}{0.000000,0.000000,0.000000}%
\pgfsetstrokecolor{textcolor}%
\pgfsetfillcolor{textcolor}%
\pgftext[x=3.981712in,y=0.234413in,,top]{\color{textcolor}\sffamily\fontsize{10.000000}{12.000000}\selectfont \ensuremath{-}20}%
\end{pgfscope}%
\begin{pgfscope}%
\pgfsetbuttcap%
\pgfsetroundjoin%
\definecolor{currentfill}{rgb}{0.000000,0.000000,0.000000}%
\pgfsetfillcolor{currentfill}%
\pgfsetlinewidth{0.803000pt}%
\definecolor{currentstroke}{rgb}{0.000000,0.000000,0.000000}%
\pgfsetstrokecolor{currentstroke}%
\pgfsetdash{}{0pt}%
\pgfsys@defobject{currentmarker}{\pgfqpoint{0.000000in}{-0.048611in}}{\pgfqpoint{0.000000in}{0.000000in}}{%
\pgfpathmoveto{\pgfqpoint{0.000000in}{0.000000in}}%
\pgfpathlineto{\pgfqpoint{0.000000in}{-0.048611in}}%
\pgfusepath{stroke,fill}%
}%
\begin{pgfscope}%
\pgfsys@transformshift{5.601708in}{0.331635in}%
\pgfsys@useobject{currentmarker}{}%
\end{pgfscope}%
\end{pgfscope}%
\begin{pgfscope}%
\definecolor{textcolor}{rgb}{0.000000,0.000000,0.000000}%
\pgfsetstrokecolor{textcolor}%
\pgfsetfillcolor{textcolor}%
\pgftext[x=5.601708in,y=0.234413in,,top]{\color{textcolor}\sffamily\fontsize{10.000000}{12.000000}\selectfont 0}%
\end{pgfscope}%
\begin{pgfscope}%
\pgfsetbuttcap%
\pgfsetroundjoin%
\definecolor{currentfill}{rgb}{0.000000,0.000000,0.000000}%
\pgfsetfillcolor{currentfill}%
\pgfsetlinewidth{0.803000pt}%
\definecolor{currentstroke}{rgb}{0.000000,0.000000,0.000000}%
\pgfsetstrokecolor{currentstroke}%
\pgfsetdash{}{0pt}%
\pgfsys@defobject{currentmarker}{\pgfqpoint{0.000000in}{-0.048611in}}{\pgfqpoint{0.000000in}{0.000000in}}{%
\pgfpathmoveto{\pgfqpoint{0.000000in}{0.000000in}}%
\pgfpathlineto{\pgfqpoint{0.000000in}{-0.048611in}}%
\pgfusepath{stroke,fill}%
}%
\begin{pgfscope}%
\pgfsys@transformshift{7.221705in}{0.331635in}%
\pgfsys@useobject{currentmarker}{}%
\end{pgfscope}%
\end{pgfscope}%
\begin{pgfscope}%
\definecolor{textcolor}{rgb}{0.000000,0.000000,0.000000}%
\pgfsetstrokecolor{textcolor}%
\pgfsetfillcolor{textcolor}%
\pgftext[x=7.221705in,y=0.234413in,,top]{\color{textcolor}\sffamily\fontsize{10.000000}{12.000000}\selectfont 20}%
\end{pgfscope}%
\begin{pgfscope}%
\pgfsetbuttcap%
\pgfsetroundjoin%
\definecolor{currentfill}{rgb}{0.000000,0.000000,0.000000}%
\pgfsetfillcolor{currentfill}%
\pgfsetlinewidth{0.803000pt}%
\definecolor{currentstroke}{rgb}{0.000000,0.000000,0.000000}%
\pgfsetstrokecolor{currentstroke}%
\pgfsetdash{}{0pt}%
\pgfsys@defobject{currentmarker}{\pgfqpoint{0.000000in}{-0.048611in}}{\pgfqpoint{0.000000in}{0.000000in}}{%
\pgfpathmoveto{\pgfqpoint{0.000000in}{0.000000in}}%
\pgfpathlineto{\pgfqpoint{0.000000in}{-0.048611in}}%
\pgfusepath{stroke,fill}%
}%
\begin{pgfscope}%
\pgfsys@transformshift{8.841702in}{0.331635in}%
\pgfsys@useobject{currentmarker}{}%
\end{pgfscope}%
\end{pgfscope}%
\begin{pgfscope}%
\definecolor{textcolor}{rgb}{0.000000,0.000000,0.000000}%
\pgfsetstrokecolor{textcolor}%
\pgfsetfillcolor{textcolor}%
\pgftext[x=8.841702in,y=0.234413in,,top]{\color{textcolor}\sffamily\fontsize{10.000000}{12.000000}\selectfont 40}%
\end{pgfscope}%
\begin{pgfscope}%
\pgfsetbuttcap%
\pgfsetroundjoin%
\definecolor{currentfill}{rgb}{0.000000,0.000000,0.000000}%
\pgfsetfillcolor{currentfill}%
\pgfsetlinewidth{0.803000pt}%
\definecolor{currentstroke}{rgb}{0.000000,0.000000,0.000000}%
\pgfsetstrokecolor{currentstroke}%
\pgfsetdash{}{0pt}%
\pgfsys@defobject{currentmarker}{\pgfqpoint{-0.048611in}{0.000000in}}{\pgfqpoint{-0.000000in}{0.000000in}}{%
\pgfpathmoveto{\pgfqpoint{-0.000000in}{0.000000in}}%
\pgfpathlineto{\pgfqpoint{-0.048611in}{0.000000in}}%
\pgfusepath{stroke,fill}%
}%
\begin{pgfscope}%
\pgfsys@transformshift{0.481978in}{0.879543in}%
\pgfsys@useobject{currentmarker}{}%
\end{pgfscope}%
\end{pgfscope}%
\begin{pgfscope}%
\definecolor{textcolor}{rgb}{0.000000,0.000000,0.000000}%
\pgfsetstrokecolor{textcolor}%
\pgfsetfillcolor{textcolor}%
\pgftext[x=0.100000in, y=0.826781in, left, base]{\color{textcolor}\sffamily\fontsize{10.000000}{12.000000}\selectfont \ensuremath{-}75}%
\end{pgfscope}%
\begin{pgfscope}%
\pgfsetbuttcap%
\pgfsetroundjoin%
\definecolor{currentfill}{rgb}{0.000000,0.000000,0.000000}%
\pgfsetfillcolor{currentfill}%
\pgfsetlinewidth{0.803000pt}%
\definecolor{currentstroke}{rgb}{0.000000,0.000000,0.000000}%
\pgfsetstrokecolor{currentstroke}%
\pgfsetdash{}{0pt}%
\pgfsys@defobject{currentmarker}{\pgfqpoint{-0.048611in}{0.000000in}}{\pgfqpoint{-0.000000in}{0.000000in}}{%
\pgfpathmoveto{\pgfqpoint{-0.000000in}{0.000000in}}%
\pgfpathlineto{\pgfqpoint{-0.048611in}{0.000000in}}%
\pgfusepath{stroke,fill}%
}%
\begin{pgfscope}%
\pgfsys@transformshift{0.481978in}{1.925520in}%
\pgfsys@useobject{currentmarker}{}%
\end{pgfscope}%
\end{pgfscope}%
\begin{pgfscope}%
\definecolor{textcolor}{rgb}{0.000000,0.000000,0.000000}%
\pgfsetstrokecolor{textcolor}%
\pgfsetfillcolor{textcolor}%
\pgftext[x=0.100000in, y=1.872759in, left, base]{\color{textcolor}\sffamily\fontsize{10.000000}{12.000000}\selectfont \ensuremath{-}50}%
\end{pgfscope}%
\begin{pgfscope}%
\pgfsetbuttcap%
\pgfsetroundjoin%
\definecolor{currentfill}{rgb}{0.000000,0.000000,0.000000}%
\pgfsetfillcolor{currentfill}%
\pgfsetlinewidth{0.803000pt}%
\definecolor{currentstroke}{rgb}{0.000000,0.000000,0.000000}%
\pgfsetstrokecolor{currentstroke}%
\pgfsetdash{}{0pt}%
\pgfsys@defobject{currentmarker}{\pgfqpoint{-0.048611in}{0.000000in}}{\pgfqpoint{-0.000000in}{0.000000in}}{%
\pgfpathmoveto{\pgfqpoint{-0.000000in}{0.000000in}}%
\pgfpathlineto{\pgfqpoint{-0.048611in}{0.000000in}}%
\pgfusepath{stroke,fill}%
}%
\begin{pgfscope}%
\pgfsys@transformshift{0.481978in}{2.971498in}%
\pgfsys@useobject{currentmarker}{}%
\end{pgfscope}%
\end{pgfscope}%
\begin{pgfscope}%
\definecolor{textcolor}{rgb}{0.000000,0.000000,0.000000}%
\pgfsetstrokecolor{textcolor}%
\pgfsetfillcolor{textcolor}%
\pgftext[x=0.100000in, y=2.918736in, left, base]{\color{textcolor}\sffamily\fontsize{10.000000}{12.000000}\selectfont \ensuremath{-}25}%
\end{pgfscope}%
\begin{pgfscope}%
\pgfsetbuttcap%
\pgfsetroundjoin%
\definecolor{currentfill}{rgb}{0.000000,0.000000,0.000000}%
\pgfsetfillcolor{currentfill}%
\pgfsetlinewidth{0.803000pt}%
\definecolor{currentstroke}{rgb}{0.000000,0.000000,0.000000}%
\pgfsetstrokecolor{currentstroke}%
\pgfsetdash{}{0pt}%
\pgfsys@defobject{currentmarker}{\pgfqpoint{-0.048611in}{0.000000in}}{\pgfqpoint{-0.000000in}{0.000000in}}{%
\pgfpathmoveto{\pgfqpoint{-0.000000in}{0.000000in}}%
\pgfpathlineto{\pgfqpoint{-0.048611in}{0.000000in}}%
\pgfusepath{stroke,fill}%
}%
\begin{pgfscope}%
\pgfsys@transformshift{0.481978in}{4.017475in}%
\pgfsys@useobject{currentmarker}{}%
\end{pgfscope}%
\end{pgfscope}%
\begin{pgfscope}%
\definecolor{textcolor}{rgb}{0.000000,0.000000,0.000000}%
\pgfsetstrokecolor{textcolor}%
\pgfsetfillcolor{textcolor}%
\pgftext[x=0.296390in, y=3.964714in, left, base]{\color{textcolor}\sffamily\fontsize{10.000000}{12.000000}\selectfont 0}%
\end{pgfscope}%
\begin{pgfscope}%
\pgfsetbuttcap%
\pgfsetroundjoin%
\definecolor{currentfill}{rgb}{0.000000,0.000000,0.000000}%
\pgfsetfillcolor{currentfill}%
\pgfsetlinewidth{0.803000pt}%
\definecolor{currentstroke}{rgb}{0.000000,0.000000,0.000000}%
\pgfsetstrokecolor{currentstroke}%
\pgfsetdash{}{0pt}%
\pgfsys@defobject{currentmarker}{\pgfqpoint{-0.048611in}{0.000000in}}{\pgfqpoint{-0.000000in}{0.000000in}}{%
\pgfpathmoveto{\pgfqpoint{-0.000000in}{0.000000in}}%
\pgfpathlineto{\pgfqpoint{-0.048611in}{0.000000in}}%
\pgfusepath{stroke,fill}%
}%
\begin{pgfscope}%
\pgfsys@transformshift{0.481978in}{5.063453in}%
\pgfsys@useobject{currentmarker}{}%
\end{pgfscope}%
\end{pgfscope}%
\begin{pgfscope}%
\definecolor{textcolor}{rgb}{0.000000,0.000000,0.000000}%
\pgfsetstrokecolor{textcolor}%
\pgfsetfillcolor{textcolor}%
\pgftext[x=0.208025in, y=5.010691in, left, base]{\color{textcolor}\sffamily\fontsize{10.000000}{12.000000}\selectfont 25}%
\end{pgfscope}%
\begin{pgfscope}%
\pgfsetbuttcap%
\pgfsetroundjoin%
\definecolor{currentfill}{rgb}{0.000000,0.000000,0.000000}%
\pgfsetfillcolor{currentfill}%
\pgfsetlinewidth{0.803000pt}%
\definecolor{currentstroke}{rgb}{0.000000,0.000000,0.000000}%
\pgfsetstrokecolor{currentstroke}%
\pgfsetdash{}{0pt}%
\pgfsys@defobject{currentmarker}{\pgfqpoint{-0.048611in}{0.000000in}}{\pgfqpoint{-0.000000in}{0.000000in}}{%
\pgfpathmoveto{\pgfqpoint{-0.000000in}{0.000000in}}%
\pgfpathlineto{\pgfqpoint{-0.048611in}{0.000000in}}%
\pgfusepath{stroke,fill}%
}%
\begin{pgfscope}%
\pgfsys@transformshift{0.481978in}{6.109431in}%
\pgfsys@useobject{currentmarker}{}%
\end{pgfscope}%
\end{pgfscope}%
\begin{pgfscope}%
\definecolor{textcolor}{rgb}{0.000000,0.000000,0.000000}%
\pgfsetstrokecolor{textcolor}%
\pgfsetfillcolor{textcolor}%
\pgftext[x=0.208025in, y=6.056669in, left, base]{\color{textcolor}\sffamily\fontsize{10.000000}{12.000000}\selectfont 50}%
\end{pgfscope}%
\begin{pgfscope}%
\pgfsetbuttcap%
\pgfsetroundjoin%
\definecolor{currentfill}{rgb}{0.000000,0.000000,0.000000}%
\pgfsetfillcolor{currentfill}%
\pgfsetlinewidth{0.803000pt}%
\definecolor{currentstroke}{rgb}{0.000000,0.000000,0.000000}%
\pgfsetstrokecolor{currentstroke}%
\pgfsetdash{}{0pt}%
\pgfsys@defobject{currentmarker}{\pgfqpoint{-0.048611in}{0.000000in}}{\pgfqpoint{-0.000000in}{0.000000in}}{%
\pgfpathmoveto{\pgfqpoint{-0.000000in}{0.000000in}}%
\pgfpathlineto{\pgfqpoint{-0.048611in}{0.000000in}}%
\pgfusepath{stroke,fill}%
}%
\begin{pgfscope}%
\pgfsys@transformshift{0.481978in}{7.155408in}%
\pgfsys@useobject{currentmarker}{}%
\end{pgfscope}%
\end{pgfscope}%
\begin{pgfscope}%
\definecolor{textcolor}{rgb}{0.000000,0.000000,0.000000}%
\pgfsetstrokecolor{textcolor}%
\pgfsetfillcolor{textcolor}%
\pgftext[x=0.208025in, y=7.102647in, left, base]{\color{textcolor}\sffamily\fontsize{10.000000}{12.000000}\selectfont 75}%
\end{pgfscope}%
\begin{pgfscope}%
\pgfpathrectangle{\pgfqpoint{0.481978in}{0.331635in}}{\pgfqpoint{9.300000in}{7.700000in}}%
\pgfusepath{clip}%
\pgfsetrectcap%
\pgfsetroundjoin%
\pgfsetlinewidth{1.505625pt}%
\definecolor{currentstroke}{rgb}{0.631373,0.788235,0.956863}%
\pgfsetstrokecolor{currentstroke}%
\pgfsetstrokeopacity{0.800000}%
\pgfsetdash{}{0pt}%
\pgfpathmoveto{\pgfqpoint{8.052863in}{1.552093in}}%
\pgfpathlineto{\pgfqpoint{5.229698in}{3.467077in}}%
\pgfusepath{stroke}%
\end{pgfscope}%
\begin{pgfscope}%
\pgfpathrectangle{\pgfqpoint{0.481978in}{0.331635in}}{\pgfqpoint{9.300000in}{7.700000in}}%
\pgfusepath{clip}%
\pgfsetrectcap%
\pgfsetroundjoin%
\pgfsetlinewidth{1.505625pt}%
\definecolor{currentstroke}{rgb}{0.631373,0.788235,0.956863}%
\pgfsetstrokecolor{currentstroke}%
\pgfsetstrokeopacity{0.800000}%
\pgfsetdash{}{0pt}%
\pgfpathmoveto{\pgfqpoint{6.390490in}{4.873471in}}%
\pgfpathlineto{\pgfqpoint{5.229698in}{3.467077in}}%
\pgfusepath{stroke}%
\end{pgfscope}%
\begin{pgfscope}%
\pgfpathrectangle{\pgfqpoint{0.481978in}{0.331635in}}{\pgfqpoint{9.300000in}{7.700000in}}%
\pgfusepath{clip}%
\pgfsetrectcap%
\pgfsetroundjoin%
\pgfsetlinewidth{1.505625pt}%
\definecolor{currentstroke}{rgb}{0.631373,0.788235,0.956863}%
\pgfsetstrokecolor{currentstroke}%
\pgfsetstrokeopacity{0.800000}%
\pgfsetdash{}{0pt}%
\pgfpathmoveto{\pgfqpoint{4.724066in}{3.602648in}}%
\pgfpathlineto{\pgfqpoint{5.229698in}{3.467077in}}%
\pgfusepath{stroke}%
\end{pgfscope}%
\begin{pgfscope}%
\pgfpathrectangle{\pgfqpoint{0.481978in}{0.331635in}}{\pgfqpoint{9.300000in}{7.700000in}}%
\pgfusepath{clip}%
\pgfsetrectcap%
\pgfsetroundjoin%
\pgfsetlinewidth{1.505625pt}%
\definecolor{currentstroke}{rgb}{0.631373,0.788235,0.956863}%
\pgfsetstrokecolor{currentstroke}%
\pgfsetstrokeopacity{0.800000}%
\pgfsetdash{}{0pt}%
\pgfpathmoveto{\pgfqpoint{3.262933in}{3.175424in}}%
\pgfpathlineto{\pgfqpoint{5.229698in}{3.467077in}}%
\pgfusepath{stroke}%
\end{pgfscope}%
\begin{pgfscope}%
\pgfpathrectangle{\pgfqpoint{0.481978in}{0.331635in}}{\pgfqpoint{9.300000in}{7.700000in}}%
\pgfusepath{clip}%
\pgfsetrectcap%
\pgfsetroundjoin%
\pgfsetlinewidth{1.505625pt}%
\definecolor{currentstroke}{rgb}{0.631373,0.788235,0.956863}%
\pgfsetstrokecolor{currentstroke}%
\pgfsetstrokeopacity{0.800000}%
\pgfsetdash{}{0pt}%
\pgfpathmoveto{\pgfqpoint{8.262238in}{1.111019in}}%
\pgfpathlineto{\pgfqpoint{5.229698in}{3.467077in}}%
\pgfusepath{stroke}%
\end{pgfscope}%
\begin{pgfscope}%
\pgfpathrectangle{\pgfqpoint{0.481978in}{0.331635in}}{\pgfqpoint{9.300000in}{7.700000in}}%
\pgfusepath{clip}%
\pgfsetrectcap%
\pgfsetroundjoin%
\pgfsetlinewidth{1.505625pt}%
\definecolor{currentstroke}{rgb}{0.631373,0.788235,0.956863}%
\pgfsetstrokecolor{currentstroke}%
\pgfsetstrokeopacity{0.800000}%
\pgfsetdash{}{0pt}%
\pgfpathmoveto{\pgfqpoint{7.580160in}{5.058834in}}%
\pgfpathlineto{\pgfqpoint{5.229698in}{3.467077in}}%
\pgfusepath{stroke}%
\end{pgfscope}%
\begin{pgfscope}%
\pgfpathrectangle{\pgfqpoint{0.481978in}{0.331635in}}{\pgfqpoint{9.300000in}{7.700000in}}%
\pgfusepath{clip}%
\pgfsetrectcap%
\pgfsetroundjoin%
\pgfsetlinewidth{1.505625pt}%
\definecolor{currentstroke}{rgb}{0.631373,0.788235,0.956863}%
\pgfsetstrokecolor{currentstroke}%
\pgfsetstrokeopacity{0.800000}%
\pgfsetdash{}{0pt}%
\pgfpathmoveto{\pgfqpoint{7.496625in}{2.061400in}}%
\pgfpathlineto{\pgfqpoint{5.229698in}{3.467077in}}%
\pgfusepath{stroke}%
\end{pgfscope}%
\begin{pgfscope}%
\pgfpathrectangle{\pgfqpoint{0.481978in}{0.331635in}}{\pgfqpoint{9.300000in}{7.700000in}}%
\pgfusepath{clip}%
\pgfsetrectcap%
\pgfsetroundjoin%
\pgfsetlinewidth{1.505625pt}%
\definecolor{currentstroke}{rgb}{0.631373,0.788235,0.956863}%
\pgfsetstrokecolor{currentstroke}%
\pgfsetstrokeopacity{0.800000}%
\pgfsetdash{}{0pt}%
\pgfpathmoveto{\pgfqpoint{3.509030in}{2.758429in}}%
\pgfpathlineto{\pgfqpoint{5.229698in}{3.467077in}}%
\pgfusepath{stroke}%
\end{pgfscope}%
\begin{pgfscope}%
\pgfpathrectangle{\pgfqpoint{0.481978in}{0.331635in}}{\pgfqpoint{9.300000in}{7.700000in}}%
\pgfusepath{clip}%
\pgfsetrectcap%
\pgfsetroundjoin%
\pgfsetlinewidth{1.505625pt}%
\definecolor{currentstroke}{rgb}{0.631373,0.788235,0.956863}%
\pgfsetstrokecolor{currentstroke}%
\pgfsetstrokeopacity{0.800000}%
\pgfsetdash{}{0pt}%
\pgfpathmoveto{\pgfqpoint{6.905988in}{4.411555in}}%
\pgfpathlineto{\pgfqpoint{5.229698in}{3.467077in}}%
\pgfusepath{stroke}%
\end{pgfscope}%
\begin{pgfscope}%
\pgfpathrectangle{\pgfqpoint{0.481978in}{0.331635in}}{\pgfqpoint{9.300000in}{7.700000in}}%
\pgfusepath{clip}%
\pgfsetrectcap%
\pgfsetroundjoin%
\pgfsetlinewidth{1.505625pt}%
\definecolor{currentstroke}{rgb}{0.631373,0.788235,0.956863}%
\pgfsetstrokecolor{currentstroke}%
\pgfsetstrokeopacity{0.800000}%
\pgfsetdash{}{0pt}%
\pgfpathmoveto{\pgfqpoint{2.394843in}{4.944242in}}%
\pgfpathlineto{\pgfqpoint{5.229698in}{3.467077in}}%
\pgfusepath{stroke}%
\end{pgfscope}%
\begin{pgfscope}%
\pgfpathrectangle{\pgfqpoint{0.481978in}{0.331635in}}{\pgfqpoint{9.300000in}{7.700000in}}%
\pgfusepath{clip}%
\pgfsetrectcap%
\pgfsetroundjoin%
\pgfsetlinewidth{1.505625pt}%
\definecolor{currentstroke}{rgb}{0.631373,0.788235,0.956863}%
\pgfsetstrokecolor{currentstroke}%
\pgfsetstrokeopacity{0.800000}%
\pgfsetdash{}{0pt}%
\pgfpathmoveto{\pgfqpoint{4.587397in}{5.253965in}}%
\pgfpathlineto{\pgfqpoint{5.229698in}{3.467077in}}%
\pgfusepath{stroke}%
\end{pgfscope}%
\begin{pgfscope}%
\pgfpathrectangle{\pgfqpoint{0.481978in}{0.331635in}}{\pgfqpoint{9.300000in}{7.700000in}}%
\pgfusepath{clip}%
\pgfsetrectcap%
\pgfsetroundjoin%
\pgfsetlinewidth{1.505625pt}%
\definecolor{currentstroke}{rgb}{0.631373,0.788235,0.956863}%
\pgfsetstrokecolor{currentstroke}%
\pgfsetstrokeopacity{0.800000}%
\pgfsetdash{}{0pt}%
\pgfpathmoveto{\pgfqpoint{5.606626in}{4.870416in}}%
\pgfpathlineto{\pgfqpoint{5.229698in}{3.467077in}}%
\pgfusepath{stroke}%
\end{pgfscope}%
\begin{pgfscope}%
\pgfpathrectangle{\pgfqpoint{0.481978in}{0.331635in}}{\pgfqpoint{9.300000in}{7.700000in}}%
\pgfusepath{clip}%
\pgfsetrectcap%
\pgfsetroundjoin%
\pgfsetlinewidth{1.505625pt}%
\definecolor{currentstroke}{rgb}{0.631373,0.788235,0.956863}%
\pgfsetstrokecolor{currentstroke}%
\pgfsetstrokeopacity{0.800000}%
\pgfsetdash{}{0pt}%
\pgfpathmoveto{\pgfqpoint{5.093085in}{5.645718in}}%
\pgfpathlineto{\pgfqpoint{5.229698in}{3.467077in}}%
\pgfusepath{stroke}%
\end{pgfscope}%
\begin{pgfscope}%
\pgfpathrectangle{\pgfqpoint{0.481978in}{0.331635in}}{\pgfqpoint{9.300000in}{7.700000in}}%
\pgfusepath{clip}%
\pgfsetrectcap%
\pgfsetroundjoin%
\pgfsetlinewidth{1.505625pt}%
\definecolor{currentstroke}{rgb}{0.631373,0.788235,0.956863}%
\pgfsetstrokecolor{currentstroke}%
\pgfsetstrokeopacity{0.800000}%
\pgfsetdash{}{0pt}%
\pgfpathmoveto{\pgfqpoint{7.877868in}{4.811193in}}%
\pgfpathlineto{\pgfqpoint{5.229698in}{3.467077in}}%
\pgfusepath{stroke}%
\end{pgfscope}%
\begin{pgfscope}%
\pgfpathrectangle{\pgfqpoint{0.481978in}{0.331635in}}{\pgfqpoint{9.300000in}{7.700000in}}%
\pgfusepath{clip}%
\pgfsetrectcap%
\pgfsetroundjoin%
\pgfsetlinewidth{1.505625pt}%
\definecolor{currentstroke}{rgb}{0.631373,0.788235,0.956863}%
\pgfsetstrokecolor{currentstroke}%
\pgfsetstrokeopacity{0.800000}%
\pgfsetdash{}{0pt}%
\pgfpathmoveto{\pgfqpoint{2.375730in}{2.401464in}}%
\pgfpathlineto{\pgfqpoint{5.229698in}{3.467077in}}%
\pgfusepath{stroke}%
\end{pgfscope}%
\begin{pgfscope}%
\pgfpathrectangle{\pgfqpoint{0.481978in}{0.331635in}}{\pgfqpoint{9.300000in}{7.700000in}}%
\pgfusepath{clip}%
\pgfsetrectcap%
\pgfsetroundjoin%
\pgfsetlinewidth{1.505625pt}%
\definecolor{currentstroke}{rgb}{0.631373,0.788235,0.956863}%
\pgfsetstrokecolor{currentstroke}%
\pgfsetstrokeopacity{0.800000}%
\pgfsetdash{}{0pt}%
\pgfpathmoveto{\pgfqpoint{9.359251in}{4.541842in}}%
\pgfpathlineto{\pgfqpoint{5.229698in}{3.467077in}}%
\pgfusepath{stroke}%
\end{pgfscope}%
\begin{pgfscope}%
\pgfpathrectangle{\pgfqpoint{0.481978in}{0.331635in}}{\pgfqpoint{9.300000in}{7.700000in}}%
\pgfusepath{clip}%
\pgfsetrectcap%
\pgfsetroundjoin%
\pgfsetlinewidth{1.505625pt}%
\definecolor{currentstroke}{rgb}{0.631373,0.788235,0.956863}%
\pgfsetstrokecolor{currentstroke}%
\pgfsetstrokeopacity{0.800000}%
\pgfsetdash{}{0pt}%
\pgfpathmoveto{\pgfqpoint{6.013955in}{5.386350in}}%
\pgfpathlineto{\pgfqpoint{5.229698in}{3.467077in}}%
\pgfusepath{stroke}%
\end{pgfscope}%
\begin{pgfscope}%
\pgfpathrectangle{\pgfqpoint{0.481978in}{0.331635in}}{\pgfqpoint{9.300000in}{7.700000in}}%
\pgfusepath{clip}%
\pgfsetrectcap%
\pgfsetroundjoin%
\pgfsetlinewidth{1.505625pt}%
\definecolor{currentstroke}{rgb}{0.631373,0.788235,0.956863}%
\pgfsetstrokecolor{currentstroke}%
\pgfsetstrokeopacity{0.800000}%
\pgfsetdash{}{0pt}%
\pgfpathmoveto{\pgfqpoint{3.830222in}{6.820584in}}%
\pgfpathlineto{\pgfqpoint{5.229698in}{3.467077in}}%
\pgfusepath{stroke}%
\end{pgfscope}%
\begin{pgfscope}%
\pgfpathrectangle{\pgfqpoint{0.481978in}{0.331635in}}{\pgfqpoint{9.300000in}{7.700000in}}%
\pgfusepath{clip}%
\pgfsetrectcap%
\pgfsetroundjoin%
\pgfsetlinewidth{1.505625pt}%
\definecolor{currentstroke}{rgb}{0.631373,0.788235,0.956863}%
\pgfsetstrokecolor{currentstroke}%
\pgfsetstrokeopacity{0.800000}%
\pgfsetdash{}{0pt}%
\pgfpathmoveto{\pgfqpoint{8.786314in}{1.294603in}}%
\pgfpathlineto{\pgfqpoint{5.229698in}{3.467077in}}%
\pgfusepath{stroke}%
\end{pgfscope}%
\begin{pgfscope}%
\pgfpathrectangle{\pgfqpoint{0.481978in}{0.331635in}}{\pgfqpoint{9.300000in}{7.700000in}}%
\pgfusepath{clip}%
\pgfsetrectcap%
\pgfsetroundjoin%
\pgfsetlinewidth{1.505625pt}%
\definecolor{currentstroke}{rgb}{0.631373,0.788235,0.956863}%
\pgfsetstrokecolor{currentstroke}%
\pgfsetstrokeopacity{0.800000}%
\pgfsetdash{}{0pt}%
\pgfpathmoveto{\pgfqpoint{1.235540in}{2.446071in}}%
\pgfpathlineto{\pgfqpoint{5.229698in}{3.467077in}}%
\pgfusepath{stroke}%
\end{pgfscope}%
\begin{pgfscope}%
\pgfpathrectangle{\pgfqpoint{0.481978in}{0.331635in}}{\pgfqpoint{9.300000in}{7.700000in}}%
\pgfusepath{clip}%
\pgfsetrectcap%
\pgfsetroundjoin%
\pgfsetlinewidth{1.505625pt}%
\definecolor{currentstroke}{rgb}{0.631373,0.788235,0.956863}%
\pgfsetstrokecolor{currentstroke}%
\pgfsetstrokeopacity{0.800000}%
\pgfsetdash{}{0pt}%
\pgfpathmoveto{\pgfqpoint{2.423803in}{3.183829in}}%
\pgfpathlineto{\pgfqpoint{5.229698in}{3.467077in}}%
\pgfusepath{stroke}%
\end{pgfscope}%
\begin{pgfscope}%
\pgfpathrectangle{\pgfqpoint{0.481978in}{0.331635in}}{\pgfqpoint{9.300000in}{7.700000in}}%
\pgfusepath{clip}%
\pgfsetrectcap%
\pgfsetroundjoin%
\pgfsetlinewidth{1.505625pt}%
\definecolor{currentstroke}{rgb}{0.631373,0.788235,0.956863}%
\pgfsetstrokecolor{currentstroke}%
\pgfsetstrokeopacity{0.800000}%
\pgfsetdash{}{0pt}%
\pgfpathmoveto{\pgfqpoint{3.099413in}{4.299245in}}%
\pgfpathlineto{\pgfqpoint{5.229698in}{3.467077in}}%
\pgfusepath{stroke}%
\end{pgfscope}%
\begin{pgfscope}%
\pgfpathrectangle{\pgfqpoint{0.481978in}{0.331635in}}{\pgfqpoint{9.300000in}{7.700000in}}%
\pgfusepath{clip}%
\pgfsetrectcap%
\pgfsetroundjoin%
\pgfsetlinewidth{1.505625pt}%
\definecolor{currentstroke}{rgb}{0.631373,0.788235,0.956863}%
\pgfsetstrokecolor{currentstroke}%
\pgfsetstrokeopacity{0.800000}%
\pgfsetdash{}{0pt}%
\pgfpathmoveto{\pgfqpoint{3.438982in}{4.497429in}}%
\pgfpathlineto{\pgfqpoint{5.229698in}{3.467077in}}%
\pgfusepath{stroke}%
\end{pgfscope}%
\begin{pgfscope}%
\pgfpathrectangle{\pgfqpoint{0.481978in}{0.331635in}}{\pgfqpoint{9.300000in}{7.700000in}}%
\pgfusepath{clip}%
\pgfsetrectcap%
\pgfsetroundjoin%
\pgfsetlinewidth{1.505625pt}%
\definecolor{currentstroke}{rgb}{0.631373,0.788235,0.956863}%
\pgfsetstrokecolor{currentstroke}%
\pgfsetstrokeopacity{0.800000}%
\pgfsetdash{}{0pt}%
\pgfpathmoveto{\pgfqpoint{6.239241in}{3.991403in}}%
\pgfpathlineto{\pgfqpoint{5.229698in}{3.467077in}}%
\pgfusepath{stroke}%
\end{pgfscope}%
\begin{pgfscope}%
\pgfpathrectangle{\pgfqpoint{0.481978in}{0.331635in}}{\pgfqpoint{9.300000in}{7.700000in}}%
\pgfusepath{clip}%
\pgfsetrectcap%
\pgfsetroundjoin%
\pgfsetlinewidth{1.505625pt}%
\definecolor{currentstroke}{rgb}{0.631373,0.788235,0.956863}%
\pgfsetstrokecolor{currentstroke}%
\pgfsetstrokeopacity{0.800000}%
\pgfsetdash{}{0pt}%
\pgfpathmoveto{\pgfqpoint{7.439054in}{1.199933in}}%
\pgfpathlineto{\pgfqpoint{5.229698in}{3.467077in}}%
\pgfusepath{stroke}%
\end{pgfscope}%
\begin{pgfscope}%
\pgfpathrectangle{\pgfqpoint{0.481978in}{0.331635in}}{\pgfqpoint{9.300000in}{7.700000in}}%
\pgfusepath{clip}%
\pgfsetrectcap%
\pgfsetroundjoin%
\pgfsetlinewidth{1.505625pt}%
\definecolor{currentstroke}{rgb}{0.631373,0.788235,0.956863}%
\pgfsetstrokecolor{currentstroke}%
\pgfsetstrokeopacity{0.800000}%
\pgfsetdash{}{0pt}%
\pgfpathmoveto{\pgfqpoint{7.510101in}{1.684631in}}%
\pgfpathlineto{\pgfqpoint{5.229698in}{3.467077in}}%
\pgfusepath{stroke}%
\end{pgfscope}%
\begin{pgfscope}%
\pgfpathrectangle{\pgfqpoint{0.481978in}{0.331635in}}{\pgfqpoint{9.300000in}{7.700000in}}%
\pgfusepath{clip}%
\pgfsetrectcap%
\pgfsetroundjoin%
\pgfsetlinewidth{1.505625pt}%
\definecolor{currentstroke}{rgb}{0.631373,0.788235,0.956863}%
\pgfsetstrokecolor{currentstroke}%
\pgfsetstrokeopacity{0.800000}%
\pgfsetdash{}{0pt}%
\pgfpathmoveto{\pgfqpoint{2.315914in}{4.233986in}}%
\pgfpathlineto{\pgfqpoint{5.229698in}{3.467077in}}%
\pgfusepath{stroke}%
\end{pgfscope}%
\begin{pgfscope}%
\pgfpathrectangle{\pgfqpoint{0.481978in}{0.331635in}}{\pgfqpoint{9.300000in}{7.700000in}}%
\pgfusepath{clip}%
\pgfsetrectcap%
\pgfsetroundjoin%
\pgfsetlinewidth{1.505625pt}%
\definecolor{currentstroke}{rgb}{0.631373,0.788235,0.956863}%
\pgfsetstrokecolor{currentstroke}%
\pgfsetstrokeopacity{0.800000}%
\pgfsetdash{}{0pt}%
\pgfpathmoveto{\pgfqpoint{6.878899in}{1.113104in}}%
\pgfpathlineto{\pgfqpoint{5.229698in}{3.467077in}}%
\pgfusepath{stroke}%
\end{pgfscope}%
\begin{pgfscope}%
\pgfpathrectangle{\pgfqpoint{0.481978in}{0.331635in}}{\pgfqpoint{9.300000in}{7.700000in}}%
\pgfusepath{clip}%
\pgfsetrectcap%
\pgfsetroundjoin%
\pgfsetlinewidth{1.505625pt}%
\definecolor{currentstroke}{rgb}{0.631373,0.788235,0.956863}%
\pgfsetstrokecolor{currentstroke}%
\pgfsetstrokeopacity{0.800000}%
\pgfsetdash{}{0pt}%
\pgfpathmoveto{\pgfqpoint{3.274514in}{1.576281in}}%
\pgfpathlineto{\pgfqpoint{5.229698in}{3.467077in}}%
\pgfusepath{stroke}%
\end{pgfscope}%
\begin{pgfscope}%
\pgfpathrectangle{\pgfqpoint{0.481978in}{0.331635in}}{\pgfqpoint{9.300000in}{7.700000in}}%
\pgfusepath{clip}%
\pgfsetrectcap%
\pgfsetroundjoin%
\pgfsetlinewidth{1.505625pt}%
\definecolor{currentstroke}{rgb}{0.631373,0.788235,0.956863}%
\pgfsetstrokecolor{currentstroke}%
\pgfsetstrokeopacity{0.800000}%
\pgfsetdash{}{0pt}%
\pgfpathmoveto{\pgfqpoint{6.005151in}{4.482219in}}%
\pgfpathlineto{\pgfqpoint{5.229698in}{3.467077in}}%
\pgfusepath{stroke}%
\end{pgfscope}%
\begin{pgfscope}%
\pgfpathrectangle{\pgfqpoint{0.481978in}{0.331635in}}{\pgfqpoint{9.300000in}{7.700000in}}%
\pgfusepath{clip}%
\pgfsetrectcap%
\pgfsetroundjoin%
\pgfsetlinewidth{1.505625pt}%
\definecolor{currentstroke}{rgb}{0.631373,0.788235,0.956863}%
\pgfsetstrokecolor{currentstroke}%
\pgfsetstrokeopacity{0.800000}%
\pgfsetdash{}{0pt}%
\pgfpathmoveto{\pgfqpoint{7.064826in}{4.831932in}}%
\pgfpathlineto{\pgfqpoint{5.229698in}{3.467077in}}%
\pgfusepath{stroke}%
\end{pgfscope}%
\begin{pgfscope}%
\pgfpathrectangle{\pgfqpoint{0.481978in}{0.331635in}}{\pgfqpoint{9.300000in}{7.700000in}}%
\pgfusepath{clip}%
\pgfsetrectcap%
\pgfsetroundjoin%
\pgfsetlinewidth{1.505625pt}%
\definecolor{currentstroke}{rgb}{0.631373,0.788235,0.956863}%
\pgfsetstrokecolor{currentstroke}%
\pgfsetstrokeopacity{0.800000}%
\pgfsetdash{}{0pt}%
\pgfpathmoveto{\pgfqpoint{6.961902in}{1.617101in}}%
\pgfpathlineto{\pgfqpoint{5.229698in}{3.467077in}}%
\pgfusepath{stroke}%
\end{pgfscope}%
\begin{pgfscope}%
\pgfpathrectangle{\pgfqpoint{0.481978in}{0.331635in}}{\pgfqpoint{9.300000in}{7.700000in}}%
\pgfusepath{clip}%
\pgfsetrectcap%
\pgfsetroundjoin%
\pgfsetlinewidth{1.505625pt}%
\definecolor{currentstroke}{rgb}{0.631373,0.788235,0.956863}%
\pgfsetstrokecolor{currentstroke}%
\pgfsetstrokeopacity{0.800000}%
\pgfsetdash{}{0pt}%
\pgfpathmoveto{\pgfqpoint{3.800404in}{4.886489in}}%
\pgfpathlineto{\pgfqpoint{5.229698in}{3.467077in}}%
\pgfusepath{stroke}%
\end{pgfscope}%
\begin{pgfscope}%
\pgfpathrectangle{\pgfqpoint{0.481978in}{0.331635in}}{\pgfqpoint{9.300000in}{7.700000in}}%
\pgfusepath{clip}%
\pgfsetrectcap%
\pgfsetroundjoin%
\pgfsetlinewidth{1.505625pt}%
\definecolor{currentstroke}{rgb}{0.631373,0.788235,0.956863}%
\pgfsetstrokecolor{currentstroke}%
\pgfsetstrokeopacity{0.800000}%
\pgfsetdash{}{0pt}%
\pgfpathmoveto{\pgfqpoint{6.058849in}{1.490387in}}%
\pgfpathlineto{\pgfqpoint{5.229698in}{3.467077in}}%
\pgfusepath{stroke}%
\end{pgfscope}%
\begin{pgfscope}%
\pgfpathrectangle{\pgfqpoint{0.481978in}{0.331635in}}{\pgfqpoint{9.300000in}{7.700000in}}%
\pgfusepath{clip}%
\pgfsetrectcap%
\pgfsetroundjoin%
\pgfsetlinewidth{1.505625pt}%
\definecolor{currentstroke}{rgb}{0.631373,0.788235,0.956863}%
\pgfsetstrokecolor{currentstroke}%
\pgfsetstrokeopacity{0.800000}%
\pgfsetdash{}{0pt}%
\pgfpathmoveto{\pgfqpoint{5.307606in}{5.963135in}}%
\pgfpathlineto{\pgfqpoint{5.229698in}{3.467077in}}%
\pgfusepath{stroke}%
\end{pgfscope}%
\begin{pgfscope}%
\pgfpathrectangle{\pgfqpoint{0.481978in}{0.331635in}}{\pgfqpoint{9.300000in}{7.700000in}}%
\pgfusepath{clip}%
\pgfsetrectcap%
\pgfsetroundjoin%
\pgfsetlinewidth{1.505625pt}%
\definecolor{currentstroke}{rgb}{0.631373,0.788235,0.956863}%
\pgfsetstrokecolor{currentstroke}%
\pgfsetstrokeopacity{0.800000}%
\pgfsetdash{}{0pt}%
\pgfpathmoveto{\pgfqpoint{7.033678in}{3.838299in}}%
\pgfpathlineto{\pgfqpoint{5.229698in}{3.467077in}}%
\pgfusepath{stroke}%
\end{pgfscope}%
\begin{pgfscope}%
\pgfpathrectangle{\pgfqpoint{0.481978in}{0.331635in}}{\pgfqpoint{9.300000in}{7.700000in}}%
\pgfusepath{clip}%
\pgfsetrectcap%
\pgfsetroundjoin%
\pgfsetlinewidth{1.505625pt}%
\definecolor{currentstroke}{rgb}{0.631373,0.788235,0.956863}%
\pgfsetstrokecolor{currentstroke}%
\pgfsetstrokeopacity{0.800000}%
\pgfsetdash{}{0pt}%
\pgfpathmoveto{\pgfqpoint{7.805428in}{0.848419in}}%
\pgfpathlineto{\pgfqpoint{5.229698in}{3.467077in}}%
\pgfusepath{stroke}%
\end{pgfscope}%
\begin{pgfscope}%
\pgfpathrectangle{\pgfqpoint{0.481978in}{0.331635in}}{\pgfqpoint{9.300000in}{7.700000in}}%
\pgfusepath{clip}%
\pgfsetrectcap%
\pgfsetroundjoin%
\pgfsetlinewidth{1.505625pt}%
\definecolor{currentstroke}{rgb}{0.631373,0.788235,0.956863}%
\pgfsetstrokecolor{currentstroke}%
\pgfsetstrokeopacity{0.800000}%
\pgfsetdash{}{0pt}%
\pgfpathmoveto{\pgfqpoint{3.104471in}{4.781432in}}%
\pgfpathlineto{\pgfqpoint{5.229698in}{3.467077in}}%
\pgfusepath{stroke}%
\end{pgfscope}%
\begin{pgfscope}%
\pgfpathrectangle{\pgfqpoint{0.481978in}{0.331635in}}{\pgfqpoint{9.300000in}{7.700000in}}%
\pgfusepath{clip}%
\pgfsetrectcap%
\pgfsetroundjoin%
\pgfsetlinewidth{1.505625pt}%
\definecolor{currentstroke}{rgb}{0.631373,0.788235,0.956863}%
\pgfsetstrokecolor{currentstroke}%
\pgfsetstrokeopacity{0.800000}%
\pgfsetdash{}{0pt}%
\pgfpathmoveto{\pgfqpoint{1.582673in}{3.731040in}}%
\pgfpathlineto{\pgfqpoint{5.229698in}{3.467077in}}%
\pgfusepath{stroke}%
\end{pgfscope}%
\begin{pgfscope}%
\pgfpathrectangle{\pgfqpoint{0.481978in}{0.331635in}}{\pgfqpoint{9.300000in}{7.700000in}}%
\pgfusepath{clip}%
\pgfsetrectcap%
\pgfsetroundjoin%
\pgfsetlinewidth{1.505625pt}%
\definecolor{currentstroke}{rgb}{0.631373,0.788235,0.956863}%
\pgfsetstrokecolor{currentstroke}%
\pgfsetstrokeopacity{0.800000}%
\pgfsetdash{}{0pt}%
\pgfpathmoveto{\pgfqpoint{7.082906in}{5.518169in}}%
\pgfpathlineto{\pgfqpoint{5.229698in}{3.467077in}}%
\pgfusepath{stroke}%
\end{pgfscope}%
\begin{pgfscope}%
\pgfpathrectangle{\pgfqpoint{0.481978in}{0.331635in}}{\pgfqpoint{9.300000in}{7.700000in}}%
\pgfusepath{clip}%
\pgfsetrectcap%
\pgfsetroundjoin%
\pgfsetlinewidth{1.505625pt}%
\definecolor{currentstroke}{rgb}{0.631373,0.788235,0.956863}%
\pgfsetstrokecolor{currentstroke}%
\pgfsetstrokeopacity{0.800000}%
\pgfsetdash{}{0pt}%
\pgfpathmoveto{\pgfqpoint{2.863295in}{2.561278in}}%
\pgfpathlineto{\pgfqpoint{5.229698in}{3.467077in}}%
\pgfusepath{stroke}%
\end{pgfscope}%
\begin{pgfscope}%
\pgfpathrectangle{\pgfqpoint{0.481978in}{0.331635in}}{\pgfqpoint{9.300000in}{7.700000in}}%
\pgfusepath{clip}%
\pgfsetrectcap%
\pgfsetroundjoin%
\pgfsetlinewidth{1.505625pt}%
\definecolor{currentstroke}{rgb}{0.631373,0.788235,0.956863}%
\pgfsetstrokecolor{currentstroke}%
\pgfsetstrokeopacity{0.800000}%
\pgfsetdash{}{0pt}%
\pgfpathmoveto{\pgfqpoint{7.718781in}{2.772405in}}%
\pgfpathlineto{\pgfqpoint{5.229698in}{3.467077in}}%
\pgfusepath{stroke}%
\end{pgfscope}%
\begin{pgfscope}%
\pgfpathrectangle{\pgfqpoint{0.481978in}{0.331635in}}{\pgfqpoint{9.300000in}{7.700000in}}%
\pgfusepath{clip}%
\pgfsetrectcap%
\pgfsetroundjoin%
\pgfsetlinewidth{1.505625pt}%
\definecolor{currentstroke}{rgb}{0.631373,0.788235,0.956863}%
\pgfsetstrokecolor{currentstroke}%
\pgfsetstrokeopacity{0.800000}%
\pgfsetdash{}{0pt}%
\pgfpathmoveto{\pgfqpoint{3.697003in}{3.940047in}}%
\pgfpathlineto{\pgfqpoint{5.229698in}{3.467077in}}%
\pgfusepath{stroke}%
\end{pgfscope}%
\begin{pgfscope}%
\pgfpathrectangle{\pgfqpoint{0.481978in}{0.331635in}}{\pgfqpoint{9.300000in}{7.700000in}}%
\pgfusepath{clip}%
\pgfsetrectcap%
\pgfsetroundjoin%
\pgfsetlinewidth{1.505625pt}%
\definecolor{currentstroke}{rgb}{0.631373,0.788235,0.956863}%
\pgfsetstrokecolor{currentstroke}%
\pgfsetstrokeopacity{0.800000}%
\pgfsetdash{}{0pt}%
\pgfpathmoveto{\pgfqpoint{6.644743in}{0.681635in}}%
\pgfpathlineto{\pgfqpoint{5.229698in}{3.467077in}}%
\pgfusepath{stroke}%
\end{pgfscope}%
\begin{pgfscope}%
\pgfpathrectangle{\pgfqpoint{0.481978in}{0.331635in}}{\pgfqpoint{9.300000in}{7.700000in}}%
\pgfusepath{clip}%
\pgfsetrectcap%
\pgfsetroundjoin%
\pgfsetlinewidth{1.505625pt}%
\definecolor{currentstroke}{rgb}{0.631373,0.788235,0.956863}%
\pgfsetstrokecolor{currentstroke}%
\pgfsetstrokeopacity{0.800000}%
\pgfsetdash{}{0pt}%
\pgfpathmoveto{\pgfqpoint{2.360953in}{1.986282in}}%
\pgfpathlineto{\pgfqpoint{5.229698in}{3.467077in}}%
\pgfusepath{stroke}%
\end{pgfscope}%
\begin{pgfscope}%
\pgfpathrectangle{\pgfqpoint{0.481978in}{0.331635in}}{\pgfqpoint{9.300000in}{7.700000in}}%
\pgfusepath{clip}%
\pgfsetrectcap%
\pgfsetroundjoin%
\pgfsetlinewidth{1.505625pt}%
\definecolor{currentstroke}{rgb}{0.631373,0.788235,0.956863}%
\pgfsetstrokecolor{currentstroke}%
\pgfsetstrokeopacity{0.800000}%
\pgfsetdash{}{0pt}%
\pgfpathmoveto{\pgfqpoint{3.463144in}{2.178317in}}%
\pgfpathlineto{\pgfqpoint{5.229698in}{3.467077in}}%
\pgfusepath{stroke}%
\end{pgfscope}%
\begin{pgfscope}%
\pgfpathrectangle{\pgfqpoint{0.481978in}{0.331635in}}{\pgfqpoint{9.300000in}{7.700000in}}%
\pgfusepath{clip}%
\pgfsetrectcap%
\pgfsetroundjoin%
\pgfsetlinewidth{1.505625pt}%
\definecolor{currentstroke}{rgb}{0.631373,0.788235,0.956863}%
\pgfsetstrokecolor{currentstroke}%
\pgfsetstrokeopacity{0.800000}%
\pgfsetdash{}{0pt}%
\pgfpathmoveto{\pgfqpoint{8.681814in}{2.133007in}}%
\pgfpathlineto{\pgfqpoint{5.229698in}{3.467077in}}%
\pgfusepath{stroke}%
\end{pgfscope}%
\begin{pgfscope}%
\pgfpathrectangle{\pgfqpoint{0.481978in}{0.331635in}}{\pgfqpoint{9.300000in}{7.700000in}}%
\pgfusepath{clip}%
\pgfsetrectcap%
\pgfsetroundjoin%
\pgfsetlinewidth{1.505625pt}%
\definecolor{currentstroke}{rgb}{0.631373,0.788235,0.956863}%
\pgfsetstrokecolor{currentstroke}%
\pgfsetstrokeopacity{0.800000}%
\pgfsetdash{}{0pt}%
\pgfpathmoveto{\pgfqpoint{5.188038in}{5.127389in}}%
\pgfpathlineto{\pgfqpoint{5.229698in}{3.467077in}}%
\pgfusepath{stroke}%
\end{pgfscope}%
\begin{pgfscope}%
\pgfpathrectangle{\pgfqpoint{0.481978in}{0.331635in}}{\pgfqpoint{9.300000in}{7.700000in}}%
\pgfusepath{clip}%
\pgfsetrectcap%
\pgfsetroundjoin%
\pgfsetlinewidth{1.505625pt}%
\definecolor{currentstroke}{rgb}{0.631373,0.788235,0.956863}%
\pgfsetstrokecolor{currentstroke}%
\pgfsetstrokeopacity{0.800000}%
\pgfsetdash{}{0pt}%
\pgfpathmoveto{\pgfqpoint{2.109676in}{2.977613in}}%
\pgfpathlineto{\pgfqpoint{5.229698in}{3.467077in}}%
\pgfusepath{stroke}%
\end{pgfscope}%
\begin{pgfscope}%
\pgfpathrectangle{\pgfqpoint{0.481978in}{0.331635in}}{\pgfqpoint{9.300000in}{7.700000in}}%
\pgfusepath{clip}%
\pgfsetrectcap%
\pgfsetroundjoin%
\pgfsetlinewidth{1.505625pt}%
\definecolor{currentstroke}{rgb}{0.631373,0.788235,0.956863}%
\pgfsetstrokecolor{currentstroke}%
\pgfsetstrokeopacity{0.800000}%
\pgfsetdash{}{0pt}%
\pgfpathmoveto{\pgfqpoint{0.984432in}{4.206085in}}%
\pgfpathlineto{\pgfqpoint{5.229698in}{3.467077in}}%
\pgfusepath{stroke}%
\end{pgfscope}%
\begin{pgfscope}%
\pgfpathrectangle{\pgfqpoint{0.481978in}{0.331635in}}{\pgfqpoint{9.300000in}{7.700000in}}%
\pgfusepath{clip}%
\pgfsetrectcap%
\pgfsetroundjoin%
\pgfsetlinewidth{1.505625pt}%
\definecolor{currentstroke}{rgb}{1.000000,0.705882,0.509804}%
\pgfsetstrokecolor{currentstroke}%
\pgfsetstrokeopacity{0.800000}%
\pgfsetdash{}{0pt}%
\pgfpathmoveto{\pgfqpoint{6.044554in}{2.177724in}}%
\pgfpathlineto{\pgfqpoint{5.573130in}{4.513635in}}%
\pgfusepath{stroke}%
\end{pgfscope}%
\begin{pgfscope}%
\pgfpathrectangle{\pgfqpoint{0.481978in}{0.331635in}}{\pgfqpoint{9.300000in}{7.700000in}}%
\pgfusepath{clip}%
\pgfsetrectcap%
\pgfsetroundjoin%
\pgfsetlinewidth{1.505625pt}%
\definecolor{currentstroke}{rgb}{1.000000,0.705882,0.509804}%
\pgfsetstrokecolor{currentstroke}%
\pgfsetstrokeopacity{0.800000}%
\pgfsetdash{}{0pt}%
\pgfpathmoveto{\pgfqpoint{4.376435in}{1.872307in}}%
\pgfpathlineto{\pgfqpoint{5.573130in}{4.513635in}}%
\pgfusepath{stroke}%
\end{pgfscope}%
\begin{pgfscope}%
\pgfpathrectangle{\pgfqpoint{0.481978in}{0.331635in}}{\pgfqpoint{9.300000in}{7.700000in}}%
\pgfusepath{clip}%
\pgfsetrectcap%
\pgfsetroundjoin%
\pgfsetlinewidth{1.505625pt}%
\definecolor{currentstroke}{rgb}{1.000000,0.705882,0.509804}%
\pgfsetstrokecolor{currentstroke}%
\pgfsetstrokeopacity{0.800000}%
\pgfsetdash{}{0pt}%
\pgfpathmoveto{\pgfqpoint{8.943163in}{5.761954in}}%
\pgfpathlineto{\pgfqpoint{5.573130in}{4.513635in}}%
\pgfusepath{stroke}%
\end{pgfscope}%
\begin{pgfscope}%
\pgfpathrectangle{\pgfqpoint{0.481978in}{0.331635in}}{\pgfqpoint{9.300000in}{7.700000in}}%
\pgfusepath{clip}%
\pgfsetrectcap%
\pgfsetroundjoin%
\pgfsetlinewidth{1.505625pt}%
\definecolor{currentstroke}{rgb}{1.000000,0.705882,0.509804}%
\pgfsetstrokecolor{currentstroke}%
\pgfsetstrokeopacity{0.800000}%
\pgfsetdash{}{0pt}%
\pgfpathmoveto{\pgfqpoint{5.919865in}{3.031766in}}%
\pgfpathlineto{\pgfqpoint{5.573130in}{4.513635in}}%
\pgfusepath{stroke}%
\end{pgfscope}%
\begin{pgfscope}%
\pgfpathrectangle{\pgfqpoint{0.481978in}{0.331635in}}{\pgfqpoint{9.300000in}{7.700000in}}%
\pgfusepath{clip}%
\pgfsetrectcap%
\pgfsetroundjoin%
\pgfsetlinewidth{1.505625pt}%
\definecolor{currentstroke}{rgb}{1.000000,0.705882,0.509804}%
\pgfsetstrokecolor{currentstroke}%
\pgfsetstrokeopacity{0.800000}%
\pgfsetdash{}{0pt}%
\pgfpathmoveto{\pgfqpoint{5.035065in}{3.911938in}}%
\pgfpathlineto{\pgfqpoint{5.573130in}{4.513635in}}%
\pgfusepath{stroke}%
\end{pgfscope}%
\begin{pgfscope}%
\pgfpathrectangle{\pgfqpoint{0.481978in}{0.331635in}}{\pgfqpoint{9.300000in}{7.700000in}}%
\pgfusepath{clip}%
\pgfsetrectcap%
\pgfsetroundjoin%
\pgfsetlinewidth{1.505625pt}%
\definecolor{currentstroke}{rgb}{1.000000,0.705882,0.509804}%
\pgfsetstrokecolor{currentstroke}%
\pgfsetstrokeopacity{0.800000}%
\pgfsetdash{}{0pt}%
\pgfpathmoveto{\pgfqpoint{7.526077in}{7.081058in}}%
\pgfpathlineto{\pgfqpoint{5.573130in}{4.513635in}}%
\pgfusepath{stroke}%
\end{pgfscope}%
\begin{pgfscope}%
\pgfpathrectangle{\pgfqpoint{0.481978in}{0.331635in}}{\pgfqpoint{9.300000in}{7.700000in}}%
\pgfusepath{clip}%
\pgfsetrectcap%
\pgfsetroundjoin%
\pgfsetlinewidth{1.505625pt}%
\definecolor{currentstroke}{rgb}{1.000000,0.705882,0.509804}%
\pgfsetstrokecolor{currentstroke}%
\pgfsetstrokeopacity{0.800000}%
\pgfsetdash{}{0pt}%
\pgfpathmoveto{\pgfqpoint{6.779267in}{7.371993in}}%
\pgfpathlineto{\pgfqpoint{5.573130in}{4.513635in}}%
\pgfusepath{stroke}%
\end{pgfscope}%
\begin{pgfscope}%
\pgfpathrectangle{\pgfqpoint{0.481978in}{0.331635in}}{\pgfqpoint{9.300000in}{7.700000in}}%
\pgfusepath{clip}%
\pgfsetrectcap%
\pgfsetroundjoin%
\pgfsetlinewidth{1.505625pt}%
\definecolor{currentstroke}{rgb}{1.000000,0.705882,0.509804}%
\pgfsetstrokecolor{currentstroke}%
\pgfsetstrokeopacity{0.800000}%
\pgfsetdash{}{0pt}%
\pgfpathmoveto{\pgfqpoint{2.475590in}{6.077186in}}%
\pgfpathlineto{\pgfqpoint{5.573130in}{4.513635in}}%
\pgfusepath{stroke}%
\end{pgfscope}%
\begin{pgfscope}%
\pgfpathrectangle{\pgfqpoint{0.481978in}{0.331635in}}{\pgfqpoint{9.300000in}{7.700000in}}%
\pgfusepath{clip}%
\pgfsetrectcap%
\pgfsetroundjoin%
\pgfsetlinewidth{1.505625pt}%
\definecolor{currentstroke}{rgb}{1.000000,0.705882,0.509804}%
\pgfsetstrokecolor{currentstroke}%
\pgfsetstrokeopacity{0.800000}%
\pgfsetdash{}{0pt}%
\pgfpathmoveto{\pgfqpoint{3.242777in}{3.583841in}}%
\pgfpathlineto{\pgfqpoint{5.573130in}{4.513635in}}%
\pgfusepath{stroke}%
\end{pgfscope}%
\begin{pgfscope}%
\pgfpathrectangle{\pgfqpoint{0.481978in}{0.331635in}}{\pgfqpoint{9.300000in}{7.700000in}}%
\pgfusepath{clip}%
\pgfsetrectcap%
\pgfsetroundjoin%
\pgfsetlinewidth{1.505625pt}%
\definecolor{currentstroke}{rgb}{1.000000,0.705882,0.509804}%
\pgfsetstrokecolor{currentstroke}%
\pgfsetstrokeopacity{0.800000}%
\pgfsetdash{}{0pt}%
\pgfpathmoveto{\pgfqpoint{4.227999in}{4.397836in}}%
\pgfpathlineto{\pgfqpoint{5.573130in}{4.513635in}}%
\pgfusepath{stroke}%
\end{pgfscope}%
\begin{pgfscope}%
\pgfpathrectangle{\pgfqpoint{0.481978in}{0.331635in}}{\pgfqpoint{9.300000in}{7.700000in}}%
\pgfusepath{clip}%
\pgfsetrectcap%
\pgfsetroundjoin%
\pgfsetlinewidth{1.505625pt}%
\definecolor{currentstroke}{rgb}{1.000000,0.705882,0.509804}%
\pgfsetstrokecolor{currentstroke}%
\pgfsetstrokeopacity{0.800000}%
\pgfsetdash{}{0pt}%
\pgfpathmoveto{\pgfqpoint{3.413419in}{5.358372in}}%
\pgfpathlineto{\pgfqpoint{5.573130in}{4.513635in}}%
\pgfusepath{stroke}%
\end{pgfscope}%
\begin{pgfscope}%
\pgfpathrectangle{\pgfqpoint{0.481978in}{0.331635in}}{\pgfqpoint{9.300000in}{7.700000in}}%
\pgfusepath{clip}%
\pgfsetrectcap%
\pgfsetroundjoin%
\pgfsetlinewidth{1.505625pt}%
\definecolor{currentstroke}{rgb}{1.000000,0.705882,0.509804}%
\pgfsetstrokecolor{currentstroke}%
\pgfsetstrokeopacity{0.800000}%
\pgfsetdash{}{0pt}%
\pgfpathmoveto{\pgfqpoint{3.930822in}{5.652502in}}%
\pgfpathlineto{\pgfqpoint{5.573130in}{4.513635in}}%
\pgfusepath{stroke}%
\end{pgfscope}%
\begin{pgfscope}%
\pgfpathrectangle{\pgfqpoint{0.481978in}{0.331635in}}{\pgfqpoint{9.300000in}{7.700000in}}%
\pgfusepath{clip}%
\pgfsetrectcap%
\pgfsetroundjoin%
\pgfsetlinewidth{1.505625pt}%
\definecolor{currentstroke}{rgb}{1.000000,0.705882,0.509804}%
\pgfsetstrokecolor{currentstroke}%
\pgfsetstrokeopacity{0.800000}%
\pgfsetdash{}{0pt}%
\pgfpathmoveto{\pgfqpoint{3.950735in}{3.084551in}}%
\pgfpathlineto{\pgfqpoint{5.573130in}{4.513635in}}%
\pgfusepath{stroke}%
\end{pgfscope}%
\begin{pgfscope}%
\pgfpathrectangle{\pgfqpoint{0.481978in}{0.331635in}}{\pgfqpoint{9.300000in}{7.700000in}}%
\pgfusepath{clip}%
\pgfsetrectcap%
\pgfsetroundjoin%
\pgfsetlinewidth{1.505625pt}%
\definecolor{currentstroke}{rgb}{1.000000,0.705882,0.509804}%
\pgfsetstrokecolor{currentstroke}%
\pgfsetstrokeopacity{0.800000}%
\pgfsetdash{}{0pt}%
\pgfpathmoveto{\pgfqpoint{4.062075in}{3.479157in}}%
\pgfpathlineto{\pgfqpoint{5.573130in}{4.513635in}}%
\pgfusepath{stroke}%
\end{pgfscope}%
\begin{pgfscope}%
\pgfpathrectangle{\pgfqpoint{0.481978in}{0.331635in}}{\pgfqpoint{9.300000in}{7.700000in}}%
\pgfusepath{clip}%
\pgfsetrectcap%
\pgfsetroundjoin%
\pgfsetlinewidth{1.505625pt}%
\definecolor{currentstroke}{rgb}{1.000000,0.705882,0.509804}%
\pgfsetstrokecolor{currentstroke}%
\pgfsetstrokeopacity{0.800000}%
\pgfsetdash{}{0pt}%
\pgfpathmoveto{\pgfqpoint{5.989908in}{3.446341in}}%
\pgfpathlineto{\pgfqpoint{5.573130in}{4.513635in}}%
\pgfusepath{stroke}%
\end{pgfscope}%
\begin{pgfscope}%
\pgfpathrectangle{\pgfqpoint{0.481978in}{0.331635in}}{\pgfqpoint{9.300000in}{7.700000in}}%
\pgfusepath{clip}%
\pgfsetrectcap%
\pgfsetroundjoin%
\pgfsetlinewidth{1.505625pt}%
\definecolor{currentstroke}{rgb}{1.000000,0.705882,0.509804}%
\pgfsetstrokecolor{currentstroke}%
\pgfsetstrokeopacity{0.800000}%
\pgfsetdash{}{0pt}%
\pgfpathmoveto{\pgfqpoint{8.313309in}{4.131814in}}%
\pgfpathlineto{\pgfqpoint{5.573130in}{4.513635in}}%
\pgfusepath{stroke}%
\end{pgfscope}%
\begin{pgfscope}%
\pgfpathrectangle{\pgfqpoint{0.481978in}{0.331635in}}{\pgfqpoint{9.300000in}{7.700000in}}%
\pgfusepath{clip}%
\pgfsetrectcap%
\pgfsetroundjoin%
\pgfsetlinewidth{1.505625pt}%
\definecolor{currentstroke}{rgb}{1.000000,0.705882,0.509804}%
\pgfsetstrokecolor{currentstroke}%
\pgfsetstrokeopacity{0.800000}%
\pgfsetdash{}{0pt}%
\pgfpathmoveto{\pgfqpoint{4.541721in}{3.280649in}}%
\pgfpathlineto{\pgfqpoint{5.573130in}{4.513635in}}%
\pgfusepath{stroke}%
\end{pgfscope}%
\begin{pgfscope}%
\pgfpathrectangle{\pgfqpoint{0.481978in}{0.331635in}}{\pgfqpoint{9.300000in}{7.700000in}}%
\pgfusepath{clip}%
\pgfsetrectcap%
\pgfsetroundjoin%
\pgfsetlinewidth{1.505625pt}%
\definecolor{currentstroke}{rgb}{1.000000,0.705882,0.509804}%
\pgfsetstrokecolor{currentstroke}%
\pgfsetstrokeopacity{0.800000}%
\pgfsetdash{}{0pt}%
\pgfpathmoveto{\pgfqpoint{8.281939in}{6.706957in}}%
\pgfpathlineto{\pgfqpoint{5.573130in}{4.513635in}}%
\pgfusepath{stroke}%
\end{pgfscope}%
\begin{pgfscope}%
\pgfpathrectangle{\pgfqpoint{0.481978in}{0.331635in}}{\pgfqpoint{9.300000in}{7.700000in}}%
\pgfusepath{clip}%
\pgfsetrectcap%
\pgfsetroundjoin%
\pgfsetlinewidth{1.505625pt}%
\definecolor{currentstroke}{rgb}{1.000000,0.705882,0.509804}%
\pgfsetstrokecolor{currentstroke}%
\pgfsetstrokeopacity{0.800000}%
\pgfsetdash{}{0pt}%
\pgfpathmoveto{\pgfqpoint{7.611554in}{7.554569in}}%
\pgfpathlineto{\pgfqpoint{5.573130in}{4.513635in}}%
\pgfusepath{stroke}%
\end{pgfscope}%
\begin{pgfscope}%
\pgfpathrectangle{\pgfqpoint{0.481978in}{0.331635in}}{\pgfqpoint{9.300000in}{7.700000in}}%
\pgfusepath{clip}%
\pgfsetrectcap%
\pgfsetroundjoin%
\pgfsetlinewidth{1.505625pt}%
\definecolor{currentstroke}{rgb}{1.000000,0.705882,0.509804}%
\pgfsetstrokecolor{currentstroke}%
\pgfsetstrokeopacity{0.800000}%
\pgfsetdash{}{0pt}%
\pgfpathmoveto{\pgfqpoint{2.721741in}{3.710762in}}%
\pgfpathlineto{\pgfqpoint{5.573130in}{4.513635in}}%
\pgfusepath{stroke}%
\end{pgfscope}%
\begin{pgfscope}%
\pgfpathrectangle{\pgfqpoint{0.481978in}{0.331635in}}{\pgfqpoint{9.300000in}{7.700000in}}%
\pgfusepath{clip}%
\pgfsetrectcap%
\pgfsetroundjoin%
\pgfsetlinewidth{1.505625pt}%
\definecolor{currentstroke}{rgb}{1.000000,0.705882,0.509804}%
\pgfsetstrokecolor{currentstroke}%
\pgfsetstrokeopacity{0.800000}%
\pgfsetdash{}{0pt}%
\pgfpathmoveto{\pgfqpoint{6.813584in}{2.441611in}}%
\pgfpathlineto{\pgfqpoint{5.573130in}{4.513635in}}%
\pgfusepath{stroke}%
\end{pgfscope}%
\begin{pgfscope}%
\pgfpathrectangle{\pgfqpoint{0.481978in}{0.331635in}}{\pgfqpoint{9.300000in}{7.700000in}}%
\pgfusepath{clip}%
\pgfsetrectcap%
\pgfsetroundjoin%
\pgfsetlinewidth{1.505625pt}%
\definecolor{currentstroke}{rgb}{1.000000,0.705882,0.509804}%
\pgfsetstrokecolor{currentstroke}%
\pgfsetstrokeopacity{0.800000}%
\pgfsetdash{}{0pt}%
\pgfpathmoveto{\pgfqpoint{7.019489in}{7.681635in}}%
\pgfpathlineto{\pgfqpoint{5.573130in}{4.513635in}}%
\pgfusepath{stroke}%
\end{pgfscope}%
\begin{pgfscope}%
\pgfpathrectangle{\pgfqpoint{0.481978in}{0.331635in}}{\pgfqpoint{9.300000in}{7.700000in}}%
\pgfusepath{clip}%
\pgfsetrectcap%
\pgfsetroundjoin%
\pgfsetlinewidth{1.505625pt}%
\definecolor{currentstroke}{rgb}{1.000000,0.705882,0.509804}%
\pgfsetstrokecolor{currentstroke}%
\pgfsetstrokeopacity{0.800000}%
\pgfsetdash{}{0pt}%
\pgfpathmoveto{\pgfqpoint{6.102927in}{7.282227in}}%
\pgfpathlineto{\pgfqpoint{5.573130in}{4.513635in}}%
\pgfusepath{stroke}%
\end{pgfscope}%
\begin{pgfscope}%
\pgfpathrectangle{\pgfqpoint{0.481978in}{0.331635in}}{\pgfqpoint{9.300000in}{7.700000in}}%
\pgfusepath{clip}%
\pgfsetrectcap%
\pgfsetroundjoin%
\pgfsetlinewidth{1.505625pt}%
\definecolor{currentstroke}{rgb}{1.000000,0.705882,0.509804}%
\pgfsetstrokecolor{currentstroke}%
\pgfsetstrokeopacity{0.800000}%
\pgfsetdash{}{0pt}%
\pgfpathmoveto{\pgfqpoint{5.548025in}{3.750685in}}%
\pgfpathlineto{\pgfqpoint{5.573130in}{4.513635in}}%
\pgfusepath{stroke}%
\end{pgfscope}%
\begin{pgfscope}%
\pgfpathrectangle{\pgfqpoint{0.481978in}{0.331635in}}{\pgfqpoint{9.300000in}{7.700000in}}%
\pgfusepath{clip}%
\pgfsetrectcap%
\pgfsetroundjoin%
\pgfsetlinewidth{1.505625pt}%
\definecolor{currentstroke}{rgb}{1.000000,0.705882,0.509804}%
\pgfsetstrokecolor{currentstroke}%
\pgfsetstrokeopacity{0.800000}%
\pgfsetdash{}{0pt}%
\pgfpathmoveto{\pgfqpoint{4.725147in}{4.271416in}}%
\pgfpathlineto{\pgfqpoint{5.573130in}{4.513635in}}%
\pgfusepath{stroke}%
\end{pgfscope}%
\begin{pgfscope}%
\pgfpathrectangle{\pgfqpoint{0.481978in}{0.331635in}}{\pgfqpoint{9.300000in}{7.700000in}}%
\pgfusepath{clip}%
\pgfsetrectcap%
\pgfsetroundjoin%
\pgfsetlinewidth{1.505625pt}%
\definecolor{currentstroke}{rgb}{1.000000,0.705882,0.509804}%
\pgfsetstrokecolor{currentstroke}%
\pgfsetstrokeopacity{0.800000}%
\pgfsetdash{}{0pt}%
\pgfpathmoveto{\pgfqpoint{7.998254in}{3.646651in}}%
\pgfpathlineto{\pgfqpoint{5.573130in}{4.513635in}}%
\pgfusepath{stroke}%
\end{pgfscope}%
\begin{pgfscope}%
\pgfpathrectangle{\pgfqpoint{0.481978in}{0.331635in}}{\pgfqpoint{9.300000in}{7.700000in}}%
\pgfusepath{clip}%
\pgfsetrectcap%
\pgfsetroundjoin%
\pgfsetlinewidth{1.505625pt}%
\definecolor{currentstroke}{rgb}{1.000000,0.705882,0.509804}%
\pgfsetstrokecolor{currentstroke}%
\pgfsetstrokeopacity{0.800000}%
\pgfsetdash{}{0pt}%
\pgfpathmoveto{\pgfqpoint{5.102532in}{4.527559in}}%
\pgfpathlineto{\pgfqpoint{5.573130in}{4.513635in}}%
\pgfusepath{stroke}%
\end{pgfscope}%
\begin{pgfscope}%
\pgfpathrectangle{\pgfqpoint{0.481978in}{0.331635in}}{\pgfqpoint{9.300000in}{7.700000in}}%
\pgfusepath{clip}%
\pgfsetrectcap%
\pgfsetroundjoin%
\pgfsetlinewidth{1.505625pt}%
\definecolor{currentstroke}{rgb}{1.000000,0.705882,0.509804}%
\pgfsetstrokecolor{currentstroke}%
\pgfsetstrokeopacity{0.800000}%
\pgfsetdash{}{0pt}%
\pgfpathmoveto{\pgfqpoint{4.679104in}{4.744402in}}%
\pgfpathlineto{\pgfqpoint{5.573130in}{4.513635in}}%
\pgfusepath{stroke}%
\end{pgfscope}%
\begin{pgfscope}%
\pgfpathrectangle{\pgfqpoint{0.481978in}{0.331635in}}{\pgfqpoint{9.300000in}{7.700000in}}%
\pgfusepath{clip}%
\pgfsetrectcap%
\pgfsetroundjoin%
\pgfsetlinewidth{1.505625pt}%
\definecolor{currentstroke}{rgb}{1.000000,0.705882,0.509804}%
\pgfsetstrokecolor{currentstroke}%
\pgfsetstrokeopacity{0.800000}%
\pgfsetdash{}{0pt}%
\pgfpathmoveto{\pgfqpoint{4.103990in}{2.296516in}}%
\pgfpathlineto{\pgfqpoint{5.573130in}{4.513635in}}%
\pgfusepath{stroke}%
\end{pgfscope}%
\begin{pgfscope}%
\pgfpathrectangle{\pgfqpoint{0.481978in}{0.331635in}}{\pgfqpoint{9.300000in}{7.700000in}}%
\pgfusepath{clip}%
\pgfsetrectcap%
\pgfsetroundjoin%
\pgfsetlinewidth{1.505625pt}%
\definecolor{currentstroke}{rgb}{1.000000,0.705882,0.509804}%
\pgfsetstrokecolor{currentstroke}%
\pgfsetstrokeopacity{0.800000}%
\pgfsetdash{}{0pt}%
\pgfpathmoveto{\pgfqpoint{4.530179in}{2.903130in}}%
\pgfpathlineto{\pgfqpoint{5.573130in}{4.513635in}}%
\pgfusepath{stroke}%
\end{pgfscope}%
\begin{pgfscope}%
\pgfpathrectangle{\pgfqpoint{0.481978in}{0.331635in}}{\pgfqpoint{9.300000in}{7.700000in}}%
\pgfusepath{clip}%
\pgfsetrectcap%
\pgfsetroundjoin%
\pgfsetlinewidth{1.505625pt}%
\definecolor{currentstroke}{rgb}{1.000000,0.705882,0.509804}%
\pgfsetstrokecolor{currentstroke}%
\pgfsetstrokeopacity{0.800000}%
\pgfsetdash{}{0pt}%
\pgfpathmoveto{\pgfqpoint{3.358304in}{6.036591in}}%
\pgfpathlineto{\pgfqpoint{5.573130in}{4.513635in}}%
\pgfusepath{stroke}%
\end{pgfscope}%
\begin{pgfscope}%
\pgfpathrectangle{\pgfqpoint{0.481978in}{0.331635in}}{\pgfqpoint{9.300000in}{7.700000in}}%
\pgfusepath{clip}%
\pgfsetrectcap%
\pgfsetroundjoin%
\pgfsetlinewidth{1.505625pt}%
\definecolor{currentstroke}{rgb}{1.000000,0.705882,0.509804}%
\pgfsetstrokecolor{currentstroke}%
\pgfsetstrokeopacity{0.800000}%
\pgfsetdash{}{0pt}%
\pgfpathmoveto{\pgfqpoint{5.398084in}{2.632414in}}%
\pgfpathlineto{\pgfqpoint{5.573130in}{4.513635in}}%
\pgfusepath{stroke}%
\end{pgfscope}%
\begin{pgfscope}%
\pgfpathrectangle{\pgfqpoint{0.481978in}{0.331635in}}{\pgfqpoint{9.300000in}{7.700000in}}%
\pgfusepath{clip}%
\pgfsetrectcap%
\pgfsetroundjoin%
\pgfsetlinewidth{1.505625pt}%
\definecolor{currentstroke}{rgb}{1.000000,0.705882,0.509804}%
\pgfsetstrokecolor{currentstroke}%
\pgfsetstrokeopacity{0.800000}%
\pgfsetdash{}{0pt}%
\pgfpathmoveto{\pgfqpoint{8.487669in}{6.437918in}}%
\pgfpathlineto{\pgfqpoint{5.573130in}{4.513635in}}%
\pgfusepath{stroke}%
\end{pgfscope}%
\begin{pgfscope}%
\pgfpathrectangle{\pgfqpoint{0.481978in}{0.331635in}}{\pgfqpoint{9.300000in}{7.700000in}}%
\pgfusepath{clip}%
\pgfsetrectcap%
\pgfsetroundjoin%
\pgfsetlinewidth{1.505625pt}%
\definecolor{currentstroke}{rgb}{1.000000,0.705882,0.509804}%
\pgfsetstrokecolor{currentstroke}%
\pgfsetstrokeopacity{0.800000}%
\pgfsetdash{}{0pt}%
\pgfpathmoveto{\pgfqpoint{4.319461in}{3.871786in}}%
\pgfpathlineto{\pgfqpoint{5.573130in}{4.513635in}}%
\pgfusepath{stroke}%
\end{pgfscope}%
\begin{pgfscope}%
\pgfpathrectangle{\pgfqpoint{0.481978in}{0.331635in}}{\pgfqpoint{9.300000in}{7.700000in}}%
\pgfusepath{clip}%
\pgfsetrectcap%
\pgfsetroundjoin%
\pgfsetlinewidth{1.505625pt}%
\definecolor{currentstroke}{rgb}{1.000000,0.705882,0.509804}%
\pgfsetstrokecolor{currentstroke}%
\pgfsetstrokeopacity{0.800000}%
\pgfsetdash{}{0pt}%
\pgfpathmoveto{\pgfqpoint{5.195031in}{3.322603in}}%
\pgfpathlineto{\pgfqpoint{5.573130in}{4.513635in}}%
\pgfusepath{stroke}%
\end{pgfscope}%
\begin{pgfscope}%
\pgfpathrectangle{\pgfqpoint{0.481978in}{0.331635in}}{\pgfqpoint{9.300000in}{7.700000in}}%
\pgfusepath{clip}%
\pgfsetrectcap%
\pgfsetroundjoin%
\pgfsetlinewidth{1.505625pt}%
\definecolor{currentstroke}{rgb}{1.000000,0.705882,0.509804}%
\pgfsetstrokecolor{currentstroke}%
\pgfsetstrokeopacity{0.800000}%
\pgfsetdash{}{0pt}%
\pgfpathmoveto{\pgfqpoint{6.327559in}{7.608142in}}%
\pgfpathlineto{\pgfqpoint{5.573130in}{4.513635in}}%
\pgfusepath{stroke}%
\end{pgfscope}%
\begin{pgfscope}%
\pgfpathrectangle{\pgfqpoint{0.481978in}{0.331635in}}{\pgfqpoint{9.300000in}{7.700000in}}%
\pgfusepath{clip}%
\pgfsetrectcap%
\pgfsetroundjoin%
\pgfsetlinewidth{1.505625pt}%
\definecolor{currentstroke}{rgb}{1.000000,0.705882,0.509804}%
\pgfsetstrokecolor{currentstroke}%
\pgfsetstrokeopacity{0.800000}%
\pgfsetdash{}{0pt}%
\pgfpathmoveto{\pgfqpoint{6.512708in}{3.269966in}}%
\pgfpathlineto{\pgfqpoint{5.573130in}{4.513635in}}%
\pgfusepath{stroke}%
\end{pgfscope}%
\begin{pgfscope}%
\pgfpathrectangle{\pgfqpoint{0.481978in}{0.331635in}}{\pgfqpoint{9.300000in}{7.700000in}}%
\pgfusepath{clip}%
\pgfsetrectcap%
\pgfsetroundjoin%
\pgfsetlinewidth{1.505625pt}%
\definecolor{currentstroke}{rgb}{1.000000,0.705882,0.509804}%
\pgfsetstrokecolor{currentstroke}%
\pgfsetstrokeopacity{0.800000}%
\pgfsetdash{}{0pt}%
\pgfpathmoveto{\pgfqpoint{5.258302in}{2.797545in}}%
\pgfpathlineto{\pgfqpoint{5.573130in}{4.513635in}}%
\pgfusepath{stroke}%
\end{pgfscope}%
\begin{pgfscope}%
\pgfpathrectangle{\pgfqpoint{0.481978in}{0.331635in}}{\pgfqpoint{9.300000in}{7.700000in}}%
\pgfusepath{clip}%
\pgfsetrectcap%
\pgfsetroundjoin%
\pgfsetlinewidth{1.505625pt}%
\definecolor{currentstroke}{rgb}{1.000000,0.705882,0.509804}%
\pgfsetstrokecolor{currentstroke}%
\pgfsetstrokeopacity{0.800000}%
\pgfsetdash{}{0pt}%
\pgfpathmoveto{\pgfqpoint{8.656102in}{6.093871in}}%
\pgfpathlineto{\pgfqpoint{5.573130in}{4.513635in}}%
\pgfusepath{stroke}%
\end{pgfscope}%
\begin{pgfscope}%
\pgfpathrectangle{\pgfqpoint{0.481978in}{0.331635in}}{\pgfqpoint{9.300000in}{7.700000in}}%
\pgfusepath{clip}%
\pgfsetrectcap%
\pgfsetroundjoin%
\pgfsetlinewidth{1.505625pt}%
\definecolor{currentstroke}{rgb}{1.000000,0.705882,0.509804}%
\pgfsetstrokecolor{currentstroke}%
\pgfsetstrokeopacity{0.800000}%
\pgfsetdash{}{0pt}%
\pgfpathmoveto{\pgfqpoint{5.476507in}{4.216476in}}%
\pgfpathlineto{\pgfqpoint{5.573130in}{4.513635in}}%
\pgfusepath{stroke}%
\end{pgfscope}%
\begin{pgfscope}%
\pgfpathrectangle{\pgfqpoint{0.481978in}{0.331635in}}{\pgfqpoint{9.300000in}{7.700000in}}%
\pgfusepath{clip}%
\pgfsetrectcap%
\pgfsetroundjoin%
\pgfsetlinewidth{1.505625pt}%
\definecolor{currentstroke}{rgb}{1.000000,0.705882,0.509804}%
\pgfsetstrokecolor{currentstroke}%
\pgfsetstrokeopacity{0.800000}%
\pgfsetdash{}{0pt}%
\pgfpathmoveto{\pgfqpoint{7.379869in}{3.289361in}}%
\pgfpathlineto{\pgfqpoint{5.573130in}{4.513635in}}%
\pgfusepath{stroke}%
\end{pgfscope}%
\begin{pgfscope}%
\pgfpathrectangle{\pgfqpoint{0.481978in}{0.331635in}}{\pgfqpoint{9.300000in}{7.700000in}}%
\pgfusepath{clip}%
\pgfsetrectcap%
\pgfsetroundjoin%
\pgfsetlinewidth{1.505625pt}%
\definecolor{currentstroke}{rgb}{1.000000,0.705882,0.509804}%
\pgfsetstrokecolor{currentstroke}%
\pgfsetstrokeopacity{0.800000}%
\pgfsetdash{}{0pt}%
\pgfpathmoveto{\pgfqpoint{6.352024in}{2.774781in}}%
\pgfpathlineto{\pgfqpoint{5.573130in}{4.513635in}}%
\pgfusepath{stroke}%
\end{pgfscope}%
\begin{pgfscope}%
\pgfpathrectangle{\pgfqpoint{0.481978in}{0.331635in}}{\pgfqpoint{9.300000in}{7.700000in}}%
\pgfusepath{clip}%
\pgfsetrectcap%
\pgfsetroundjoin%
\pgfsetlinewidth{1.505625pt}%
\definecolor{currentstroke}{rgb}{1.000000,0.705882,0.509804}%
\pgfsetstrokecolor{currentstroke}%
\pgfsetstrokeopacity{0.800000}%
\pgfsetdash{}{0pt}%
\pgfpathmoveto{\pgfqpoint{7.010012in}{6.655032in}}%
\pgfpathlineto{\pgfqpoint{5.573130in}{4.513635in}}%
\pgfusepath{stroke}%
\end{pgfscope}%
\begin{pgfscope}%
\pgfpathrectangle{\pgfqpoint{0.481978in}{0.331635in}}{\pgfqpoint{9.300000in}{7.700000in}}%
\pgfusepath{clip}%
\pgfsetrectcap%
\pgfsetroundjoin%
\pgfsetlinewidth{1.505625pt}%
\definecolor{currentstroke}{rgb}{1.000000,0.705882,0.509804}%
\pgfsetstrokecolor{currentstroke}%
\pgfsetstrokeopacity{0.800000}%
\pgfsetdash{}{0pt}%
\pgfpathmoveto{\pgfqpoint{4.957439in}{2.266043in}}%
\pgfpathlineto{\pgfqpoint{5.573130in}{4.513635in}}%
\pgfusepath{stroke}%
\end{pgfscope}%
\begin{pgfscope}%
\pgfpathrectangle{\pgfqpoint{0.481978in}{0.331635in}}{\pgfqpoint{9.300000in}{7.700000in}}%
\pgfusepath{clip}%
\pgfsetrectcap%
\pgfsetroundjoin%
\pgfsetlinewidth{1.505625pt}%
\definecolor{currentstroke}{rgb}{1.000000,0.705882,0.509804}%
\pgfsetstrokecolor{currentstroke}%
\pgfsetstrokeopacity{0.800000}%
\pgfsetdash{}{0pt}%
\pgfpathmoveto{\pgfqpoint{1.938901in}{5.964179in}}%
\pgfpathlineto{\pgfqpoint{5.573130in}{4.513635in}}%
\pgfusepath{stroke}%
\end{pgfscope}%
\begin{pgfscope}%
\pgfpathrectangle{\pgfqpoint{0.481978in}{0.331635in}}{\pgfqpoint{9.300000in}{7.700000in}}%
\pgfusepath{clip}%
\pgfsetrectcap%
\pgfsetroundjoin%
\pgfsetlinewidth{1.505625pt}%
\definecolor{currentstroke}{rgb}{1.000000,0.705882,0.509804}%
\pgfsetstrokecolor{currentstroke}%
\pgfsetstrokeopacity{0.800000}%
\pgfsetdash{}{0pt}%
\pgfpathmoveto{\pgfqpoint{6.653857in}{7.012761in}}%
\pgfpathlineto{\pgfqpoint{5.573130in}{4.513635in}}%
\pgfusepath{stroke}%
\end{pgfscope}%
\begin{pgfscope}%
\pgfpathrectangle{\pgfqpoint{0.481978in}{0.331635in}}{\pgfqpoint{9.300000in}{7.700000in}}%
\pgfusepath{clip}%
\pgfsetrectcap%
\pgfsetroundjoin%
\pgfsetlinewidth{1.505625pt}%
\definecolor{currentstroke}{rgb}{1.000000,0.705882,0.509804}%
\pgfsetstrokecolor{currentstroke}%
\pgfsetstrokeopacity{0.800000}%
\pgfsetdash{}{0pt}%
\pgfpathmoveto{\pgfqpoint{7.825213in}{7.158741in}}%
\pgfpathlineto{\pgfqpoint{5.573130in}{4.513635in}}%
\pgfusepath{stroke}%
\end{pgfscope}%
\begin{pgfscope}%
\pgfpathrectangle{\pgfqpoint{0.481978in}{0.331635in}}{\pgfqpoint{9.300000in}{7.700000in}}%
\pgfusepath{clip}%
\pgfsetrectcap%
\pgfsetroundjoin%
\pgfsetlinewidth{1.505625pt}%
\definecolor{currentstroke}{rgb}{1.000000,0.705882,0.509804}%
\pgfsetstrokecolor{currentstroke}%
\pgfsetstrokeopacity{0.800000}%
\pgfsetdash{}{0pt}%
\pgfpathmoveto{\pgfqpoint{4.408915in}{2.547313in}}%
\pgfpathlineto{\pgfqpoint{5.573130in}{4.513635in}}%
\pgfusepath{stroke}%
\end{pgfscope}%
\begin{pgfscope}%
\pgfpathrectangle{\pgfqpoint{0.481978in}{0.331635in}}{\pgfqpoint{9.300000in}{7.700000in}}%
\pgfusepath{clip}%
\pgfsetrectcap%
\pgfsetroundjoin%
\pgfsetlinewidth{1.505625pt}%
\definecolor{currentstroke}{rgb}{1.000000,0.705882,0.509804}%
\pgfsetstrokecolor{currentstroke}%
\pgfsetstrokeopacity{0.800000}%
\pgfsetdash{}{0pt}%
\pgfpathmoveto{\pgfqpoint{0.904705in}{3.161161in}}%
\pgfpathlineto{\pgfqpoint{5.573130in}{4.513635in}}%
\pgfusepath{stroke}%
\end{pgfscope}%
\begin{pgfscope}%
\pgfpathrectangle{\pgfqpoint{0.481978in}{0.331635in}}{\pgfqpoint{9.300000in}{7.700000in}}%
\pgfusepath{clip}%
\pgfsetrectcap%
\pgfsetroundjoin%
\pgfsetlinewidth{1.505625pt}%
\definecolor{currentstroke}{rgb}{1.000000,0.705882,0.509804}%
\pgfsetstrokecolor{currentstroke}%
\pgfsetstrokeopacity{0.800000}%
\pgfsetdash{}{0pt}%
\pgfpathmoveto{\pgfqpoint{8.224572in}{3.345931in}}%
\pgfpathlineto{\pgfqpoint{5.573130in}{4.513635in}}%
\pgfusepath{stroke}%
\end{pgfscope}%
\begin{pgfscope}%
\pgfsetrectcap%
\pgfsetmiterjoin%
\pgfsetlinewidth{0.803000pt}%
\definecolor{currentstroke}{rgb}{0.000000,0.000000,0.000000}%
\pgfsetstrokecolor{currentstroke}%
\pgfsetdash{}{0pt}%
\pgfpathmoveto{\pgfqpoint{0.481978in}{0.331635in}}%
\pgfpathlineto{\pgfqpoint{0.481978in}{8.031635in}}%
\pgfusepath{stroke}%
\end{pgfscope}%
\begin{pgfscope}%
\pgfsetrectcap%
\pgfsetmiterjoin%
\pgfsetlinewidth{0.803000pt}%
\definecolor{currentstroke}{rgb}{0.000000,0.000000,0.000000}%
\pgfsetstrokecolor{currentstroke}%
\pgfsetdash{}{0pt}%
\pgfpathmoveto{\pgfqpoint{9.781978in}{0.331635in}}%
\pgfpathlineto{\pgfqpoint{9.781978in}{8.031635in}}%
\pgfusepath{stroke}%
\end{pgfscope}%
\begin{pgfscope}%
\pgfsetrectcap%
\pgfsetmiterjoin%
\pgfsetlinewidth{0.803000pt}%
\definecolor{currentstroke}{rgb}{0.000000,0.000000,0.000000}%
\pgfsetstrokecolor{currentstroke}%
\pgfsetdash{}{0pt}%
\pgfpathmoveto{\pgfqpoint{0.481978in}{0.331635in}}%
\pgfpathlineto{\pgfqpoint{9.781978in}{0.331635in}}%
\pgfusepath{stroke}%
\end{pgfscope}%
\begin{pgfscope}%
\pgfsetrectcap%
\pgfsetmiterjoin%
\pgfsetlinewidth{0.803000pt}%
\definecolor{currentstroke}{rgb}{0.000000,0.000000,0.000000}%
\pgfsetstrokecolor{currentstroke}%
\pgfsetdash{}{0pt}%
\pgfpathmoveto{\pgfqpoint{0.481978in}{8.031635in}}%
\pgfpathlineto{\pgfqpoint{9.781978in}{8.031635in}}%
\pgfusepath{stroke}%
\end{pgfscope}%
\begin{pgfscope}%
\definecolor{textcolor}{rgb}{0.000000,0.000000,0.000000}%
\pgfsetstrokecolor{textcolor}%
\pgfsetfillcolor{textcolor}%
\pgftext[x=5.131978in,y=8.114968in,,base]{\color{textcolor}\sffamily\fontsize{12.000000}{14.400000}\selectfont T-SNE for chair images (s2r3dfree\_textureless)}%
\end{pgfscope}%
\begin{pgfscope}%
\pgfsetbuttcap%
\pgfsetmiterjoin%
\definecolor{currentfill}{rgb}{1.000000,1.000000,1.000000}%
\pgfsetfillcolor{currentfill}%
\pgfsetfillopacity{0.800000}%
\pgfsetlinewidth{1.003750pt}%
\definecolor{currentstroke}{rgb}{0.800000,0.800000,0.800000}%
\pgfsetstrokecolor{currentstroke}%
\pgfsetstrokeopacity{0.800000}%
\pgfsetdash{}{0pt}%
\pgfpathmoveto{\pgfqpoint{9.879200in}{3.955012in}}%
\pgfpathlineto{\pgfqpoint{11.831688in}{3.955012in}}%
\pgfpathquadraticcurveto{\pgfqpoint{11.859465in}{3.955012in}}{\pgfqpoint{11.859465in}{3.982789in}}%
\pgfpathlineto{\pgfqpoint{11.859465in}{4.380481in}}%
\pgfpathquadraticcurveto{\pgfqpoint{11.859465in}{4.408258in}}{\pgfqpoint{11.831688in}{4.408258in}}%
\pgfpathlineto{\pgfqpoint{9.879200in}{4.408258in}}%
\pgfpathquadraticcurveto{\pgfqpoint{9.851422in}{4.408258in}}{\pgfqpoint{9.851422in}{4.380481in}}%
\pgfpathlineto{\pgfqpoint{9.851422in}{3.982789in}}%
\pgfpathquadraticcurveto{\pgfqpoint{9.851422in}{3.955012in}}{\pgfqpoint{9.879200in}{3.955012in}}%
\pgfpathclose%
\pgfusepath{stroke,fill}%
\end{pgfscope}%
\begin{pgfscope}%
\pgfsetbuttcap%
\pgfsetroundjoin%
\definecolor{currentfill}{rgb}{0.631373,0.788235,0.956863}%
\pgfsetfillcolor{currentfill}%
\pgfsetlinewidth{1.003750pt}%
\definecolor{currentstroke}{rgb}{0.631373,0.788235,0.956863}%
\pgfsetstrokecolor{currentstroke}%
\pgfsetdash{}{0pt}%
\pgfsys@defobject{currentmarker}{\pgfqpoint{-0.041667in}{-0.041667in}}{\pgfqpoint{0.041667in}{0.041667in}}{%
\pgfpathmoveto{\pgfqpoint{0.000000in}{-0.041667in}}%
\pgfpathcurveto{\pgfqpoint{0.011050in}{-0.041667in}}{\pgfqpoint{0.021649in}{-0.037276in}}{\pgfqpoint{0.029463in}{-0.029463in}}%
\pgfpathcurveto{\pgfqpoint{0.037276in}{-0.021649in}}{\pgfqpoint{0.041667in}{-0.011050in}}{\pgfqpoint{0.041667in}{0.000000in}}%
\pgfpathcurveto{\pgfqpoint{0.041667in}{0.011050in}}{\pgfqpoint{0.037276in}{0.021649in}}{\pgfqpoint{0.029463in}{0.029463in}}%
\pgfpathcurveto{\pgfqpoint{0.021649in}{0.037276in}}{\pgfqpoint{0.011050in}{0.041667in}}{\pgfqpoint{0.000000in}{0.041667in}}%
\pgfpathcurveto{\pgfqpoint{-0.011050in}{0.041667in}}{\pgfqpoint{-0.021649in}{0.037276in}}{\pgfqpoint{-0.029463in}{0.029463in}}%
\pgfpathcurveto{\pgfqpoint{-0.037276in}{0.021649in}}{\pgfqpoint{-0.041667in}{0.011050in}}{\pgfqpoint{-0.041667in}{0.000000in}}%
\pgfpathcurveto{\pgfqpoint{-0.041667in}{-0.011050in}}{\pgfqpoint{-0.037276in}{-0.021649in}}{\pgfqpoint{-0.029463in}{-0.029463in}}%
\pgfpathcurveto{\pgfqpoint{-0.021649in}{-0.037276in}}{\pgfqpoint{-0.011050in}{-0.041667in}}{\pgfqpoint{0.000000in}{-0.041667in}}%
\pgfpathclose%
\pgfusepath{stroke,fill}%
}%
\begin{pgfscope}%
\pgfsys@transformshift{10.045867in}{4.283638in}%
\pgfsys@useobject{currentmarker}{}%
\end{pgfscope}%
\end{pgfscope}%
\begin{pgfscope}%
\definecolor{textcolor}{rgb}{0.000000,0.000000,0.000000}%
\pgfsetstrokecolor{textcolor}%
\pgfsetfillcolor{textcolor}%
\pgftext[x=10.295867in,y=4.247180in,left,base]{\color{textcolor}\sffamily\fontsize{10.000000}{12.000000}\selectfont Pix3D}%
\end{pgfscope}%
\begin{pgfscope}%
\pgfsetbuttcap%
\pgfsetroundjoin%
\definecolor{currentfill}{rgb}{1.000000,0.705882,0.509804}%
\pgfsetfillcolor{currentfill}%
\pgfsetlinewidth{1.003750pt}%
\definecolor{currentstroke}{rgb}{1.000000,0.705882,0.509804}%
\pgfsetstrokecolor{currentstroke}%
\pgfsetdash{}{0pt}%
\pgfsys@defobject{currentmarker}{\pgfqpoint{-0.041667in}{-0.041667in}}{\pgfqpoint{0.041667in}{0.041667in}}{%
\pgfpathmoveto{\pgfqpoint{0.000000in}{-0.041667in}}%
\pgfpathcurveto{\pgfqpoint{0.011050in}{-0.041667in}}{\pgfqpoint{0.021649in}{-0.037276in}}{\pgfqpoint{0.029463in}{-0.029463in}}%
\pgfpathcurveto{\pgfqpoint{0.037276in}{-0.021649in}}{\pgfqpoint{0.041667in}{-0.011050in}}{\pgfqpoint{0.041667in}{0.000000in}}%
\pgfpathcurveto{\pgfqpoint{0.041667in}{0.011050in}}{\pgfqpoint{0.037276in}{0.021649in}}{\pgfqpoint{0.029463in}{0.029463in}}%
\pgfpathcurveto{\pgfqpoint{0.021649in}{0.037276in}}{\pgfqpoint{0.011050in}{0.041667in}}{\pgfqpoint{0.000000in}{0.041667in}}%
\pgfpathcurveto{\pgfqpoint{-0.011050in}{0.041667in}}{\pgfqpoint{-0.021649in}{0.037276in}}{\pgfqpoint{-0.029463in}{0.029463in}}%
\pgfpathcurveto{\pgfqpoint{-0.037276in}{0.021649in}}{\pgfqpoint{-0.041667in}{0.011050in}}{\pgfqpoint{-0.041667in}{0.000000in}}%
\pgfpathcurveto{\pgfqpoint{-0.041667in}{-0.011050in}}{\pgfqpoint{-0.037276in}{-0.021649in}}{\pgfqpoint{-0.029463in}{-0.029463in}}%
\pgfpathcurveto{\pgfqpoint{-0.021649in}{-0.037276in}}{\pgfqpoint{-0.011050in}{-0.041667in}}{\pgfqpoint{0.000000in}{-0.041667in}}%
\pgfpathclose%
\pgfusepath{stroke,fill}%
}%
\begin{pgfscope}%
\pgfsys@transformshift{10.045867in}{4.079781in}%
\pgfsys@useobject{currentmarker}{}%
\end{pgfscope}%
\end{pgfscope}%
\begin{pgfscope}%
\definecolor{textcolor}{rgb}{0.000000,0.000000,0.000000}%
\pgfsetstrokecolor{textcolor}%
\pgfsetfillcolor{textcolor}%
\pgftext[x=10.295867in,y=4.043322in,left,base]{\color{textcolor}\sffamily\fontsize{10.000000}{12.000000}\selectfont s2r3dfree\_textureless}%
\end{pgfscope}%
\end{pgfpicture}%
\makeatother%
\endgroup%
}
    \resizebox{0.49\linewidth}{5cm}{%% Creator: Matplotlib, PGF backend
%%
%% To include the figure in your LaTeX document, write
%%   \input{<filename>.pgf}
%%
%% Make sure the required packages are loaded in your preamble
%%   \usepackage{pgf}
%%
%% Figures using additional raster images can only be included by \input if
%% they are in the same directory as the main LaTeX file. For loading figures
%% from other directories you can use the `import` package
%%   \usepackage{import}
%%
%% and then include the figures with
%%   \import{<path to file>}{<filename>.pgf}
%%
%% Matplotlib used the following preamble
%%   \usepackage{fontspec}
%%   \setmainfont{DejaVuSerif.ttf}[Path=\detokenize{/Users/apple/opt/anaconda3/envs/kaolin/lib/python3.7/site-packages/matplotlib/mpl-data/fonts/ttf/}]
%%   \setsansfont{DejaVuSans.ttf}[Path=\detokenize{/Users/apple/opt/anaconda3/envs/kaolin/lib/python3.7/site-packages/matplotlib/mpl-data/fonts/ttf/}]
%%   \setmonofont{DejaVuSansMono.ttf}[Path=\detokenize{/Users/apple/opt/anaconda3/envs/kaolin/lib/python3.7/site-packages/matplotlib/mpl-data/fonts/ttf/}]
%%
\begingroup%
\makeatletter%
\begin{pgfpicture}%
\pgfpathrectangle{\pgfpointorigin}{\pgfqpoint{12.248365in}{8.341596in}}%
\pgfusepath{use as bounding box, clip}%
\begin{pgfscope}%
\pgfsetbuttcap%
\pgfsetmiterjoin%
\definecolor{currentfill}{rgb}{1.000000,1.000000,1.000000}%
\pgfsetfillcolor{currentfill}%
\pgfsetlinewidth{0.000000pt}%
\definecolor{currentstroke}{rgb}{1.000000,1.000000,1.000000}%
\pgfsetstrokecolor{currentstroke}%
\pgfsetdash{}{0pt}%
\pgfpathmoveto{\pgfqpoint{-0.000000in}{0.000000in}}%
\pgfpathlineto{\pgfqpoint{12.248365in}{0.000000in}}%
\pgfpathlineto{\pgfqpoint{12.248365in}{8.341596in}}%
\pgfpathlineto{\pgfqpoint{-0.000000in}{8.341596in}}%
\pgfpathclose%
\pgfusepath{fill}%
\end{pgfscope}%
\begin{pgfscope}%
\pgfsetbuttcap%
\pgfsetmiterjoin%
\definecolor{currentfill}{rgb}{1.000000,1.000000,1.000000}%
\pgfsetfillcolor{currentfill}%
\pgfsetlinewidth{0.000000pt}%
\definecolor{currentstroke}{rgb}{0.000000,0.000000,0.000000}%
\pgfsetstrokecolor{currentstroke}%
\pgfsetstrokeopacity{0.000000}%
\pgfsetdash{}{0pt}%
\pgfpathmoveto{\pgfqpoint{0.393613in}{0.331635in}}%
\pgfpathlineto{\pgfqpoint{9.693613in}{0.331635in}}%
\pgfpathlineto{\pgfqpoint{9.693613in}{8.031635in}}%
\pgfpathlineto{\pgfqpoint{0.393613in}{8.031635in}}%
\pgfpathclose%
\pgfusepath{fill}%
\end{pgfscope}%
\begin{pgfscope}%
\pgfpathrectangle{\pgfqpoint{0.393613in}{0.331635in}}{\pgfqpoint{9.300000in}{7.700000in}}%
\pgfusepath{clip}%
\pgfsetbuttcap%
\pgfsetroundjoin%
\definecolor{currentfill}{rgb}{0.631373,0.788235,0.956863}%
\pgfsetfillcolor{currentfill}%
\pgfsetlinewidth{0.481800pt}%
\definecolor{currentstroke}{rgb}{1.000000,1.000000,1.000000}%
\pgfsetstrokecolor{currentstroke}%
\pgfsetdash{}{0pt}%
\pgfpathmoveto{\pgfqpoint{2.878164in}{1.516475in}}%
\pgfpathcurveto{\pgfqpoint{2.889214in}{1.516475in}}{\pgfqpoint{2.899813in}{1.520865in}}{\pgfqpoint{2.907626in}{1.528679in}}%
\pgfpathcurveto{\pgfqpoint{2.915440in}{1.536492in}}{\pgfqpoint{2.919830in}{1.547091in}}{\pgfqpoint{2.919830in}{1.558141in}}%
\pgfpathcurveto{\pgfqpoint{2.919830in}{1.569191in}}{\pgfqpoint{2.915440in}{1.579790in}}{\pgfqpoint{2.907626in}{1.587604in}}%
\pgfpathcurveto{\pgfqpoint{2.899813in}{1.595418in}}{\pgfqpoint{2.889214in}{1.599808in}}{\pgfqpoint{2.878164in}{1.599808in}}%
\pgfpathcurveto{\pgfqpoint{2.867113in}{1.599808in}}{\pgfqpoint{2.856514in}{1.595418in}}{\pgfqpoint{2.848701in}{1.587604in}}%
\pgfpathcurveto{\pgfqpoint{2.840887in}{1.579790in}}{\pgfqpoint{2.836497in}{1.569191in}}{\pgfqpoint{2.836497in}{1.558141in}}%
\pgfpathcurveto{\pgfqpoint{2.836497in}{1.547091in}}{\pgfqpoint{2.840887in}{1.536492in}}{\pgfqpoint{2.848701in}{1.528679in}}%
\pgfpathcurveto{\pgfqpoint{2.856514in}{1.520865in}}{\pgfqpoint{2.867113in}{1.516475in}}{\pgfqpoint{2.878164in}{1.516475in}}%
\pgfpathclose%
\pgfusepath{stroke,fill}%
\end{pgfscope}%
\begin{pgfscope}%
\pgfpathrectangle{\pgfqpoint{0.393613in}{0.331635in}}{\pgfqpoint{9.300000in}{7.700000in}}%
\pgfusepath{clip}%
\pgfsetbuttcap%
\pgfsetroundjoin%
\definecolor{currentfill}{rgb}{0.631373,0.788235,0.956863}%
\pgfsetfillcolor{currentfill}%
\pgfsetlinewidth{0.481800pt}%
\definecolor{currentstroke}{rgb}{1.000000,1.000000,1.000000}%
\pgfsetstrokecolor{currentstroke}%
\pgfsetdash{}{0pt}%
\pgfpathmoveto{\pgfqpoint{1.610944in}{5.101202in}}%
\pgfpathcurveto{\pgfqpoint{1.621994in}{5.101202in}}{\pgfqpoint{1.632593in}{5.105593in}}{\pgfqpoint{1.640407in}{5.113406in}}%
\pgfpathcurveto{\pgfqpoint{1.648220in}{5.121220in}}{\pgfqpoint{1.652611in}{5.131819in}}{\pgfqpoint{1.652611in}{5.142869in}}%
\pgfpathcurveto{\pgfqpoint{1.652611in}{5.153919in}}{\pgfqpoint{1.648220in}{5.164518in}}{\pgfqpoint{1.640407in}{5.172332in}}%
\pgfpathcurveto{\pgfqpoint{1.632593in}{5.180145in}}{\pgfqpoint{1.621994in}{5.184536in}}{\pgfqpoint{1.610944in}{5.184536in}}%
\pgfpathcurveto{\pgfqpoint{1.599894in}{5.184536in}}{\pgfqpoint{1.589295in}{5.180145in}}{\pgfqpoint{1.581481in}{5.172332in}}%
\pgfpathcurveto{\pgfqpoint{1.573667in}{5.164518in}}{\pgfqpoint{1.569277in}{5.153919in}}{\pgfqpoint{1.569277in}{5.142869in}}%
\pgfpathcurveto{\pgfqpoint{1.569277in}{5.131819in}}{\pgfqpoint{1.573667in}{5.121220in}}{\pgfqpoint{1.581481in}{5.113406in}}%
\pgfpathcurveto{\pgfqpoint{1.589295in}{5.105593in}}{\pgfqpoint{1.599894in}{5.101202in}}{\pgfqpoint{1.610944in}{5.101202in}}%
\pgfpathclose%
\pgfusepath{stroke,fill}%
\end{pgfscope}%
\begin{pgfscope}%
\pgfpathrectangle{\pgfqpoint{0.393613in}{0.331635in}}{\pgfqpoint{9.300000in}{7.700000in}}%
\pgfusepath{clip}%
\pgfsetbuttcap%
\pgfsetroundjoin%
\definecolor{currentfill}{rgb}{0.631373,0.788235,0.956863}%
\pgfsetfillcolor{currentfill}%
\pgfsetlinewidth{0.481800pt}%
\definecolor{currentstroke}{rgb}{1.000000,1.000000,1.000000}%
\pgfsetstrokecolor{currentstroke}%
\pgfsetdash{}{0pt}%
\pgfpathmoveto{\pgfqpoint{4.136764in}{5.848270in}}%
\pgfpathcurveto{\pgfqpoint{4.147814in}{5.848270in}}{\pgfqpoint{4.158413in}{5.852660in}}{\pgfqpoint{4.166227in}{5.860474in}}%
\pgfpathcurveto{\pgfqpoint{4.174040in}{5.868288in}}{\pgfqpoint{4.178431in}{5.878887in}}{\pgfqpoint{4.178431in}{5.889937in}}%
\pgfpathcurveto{\pgfqpoint{4.178431in}{5.900987in}}{\pgfqpoint{4.174040in}{5.911586in}}{\pgfqpoint{4.166227in}{5.919399in}}%
\pgfpathcurveto{\pgfqpoint{4.158413in}{5.927213in}}{\pgfqpoint{4.147814in}{5.931603in}}{\pgfqpoint{4.136764in}{5.931603in}}%
\pgfpathcurveto{\pgfqpoint{4.125714in}{5.931603in}}{\pgfqpoint{4.115115in}{5.927213in}}{\pgfqpoint{4.107301in}{5.919399in}}%
\pgfpathcurveto{\pgfqpoint{4.099488in}{5.911586in}}{\pgfqpoint{4.095097in}{5.900987in}}{\pgfqpoint{4.095097in}{5.889937in}}%
\pgfpathcurveto{\pgfqpoint{4.095097in}{5.878887in}}{\pgfqpoint{4.099488in}{5.868288in}}{\pgfqpoint{4.107301in}{5.860474in}}%
\pgfpathcurveto{\pgfqpoint{4.115115in}{5.852660in}}{\pgfqpoint{4.125714in}{5.848270in}}{\pgfqpoint{4.136764in}{5.848270in}}%
\pgfpathclose%
\pgfusepath{stroke,fill}%
\end{pgfscope}%
\begin{pgfscope}%
\pgfpathrectangle{\pgfqpoint{0.393613in}{0.331635in}}{\pgfqpoint{9.300000in}{7.700000in}}%
\pgfusepath{clip}%
\pgfsetbuttcap%
\pgfsetroundjoin%
\definecolor{currentfill}{rgb}{0.631373,0.788235,0.956863}%
\pgfsetfillcolor{currentfill}%
\pgfsetlinewidth{0.481800pt}%
\definecolor{currentstroke}{rgb}{1.000000,1.000000,1.000000}%
\pgfsetstrokecolor{currentstroke}%
\pgfsetdash{}{0pt}%
\pgfpathmoveto{\pgfqpoint{3.118402in}{4.346395in}}%
\pgfpathcurveto{\pgfqpoint{3.129452in}{4.346395in}}{\pgfqpoint{3.140051in}{4.350785in}}{\pgfqpoint{3.147865in}{4.358599in}}%
\pgfpathcurveto{\pgfqpoint{3.155678in}{4.366412in}}{\pgfqpoint{3.160068in}{4.377011in}}{\pgfqpoint{3.160068in}{4.388061in}}%
\pgfpathcurveto{\pgfqpoint{3.160068in}{4.399112in}}{\pgfqpoint{3.155678in}{4.409711in}}{\pgfqpoint{3.147865in}{4.417524in}}%
\pgfpathcurveto{\pgfqpoint{3.140051in}{4.425338in}}{\pgfqpoint{3.129452in}{4.429728in}}{\pgfqpoint{3.118402in}{4.429728in}}%
\pgfpathcurveto{\pgfqpoint{3.107352in}{4.429728in}}{\pgfqpoint{3.096753in}{4.425338in}}{\pgfqpoint{3.088939in}{4.417524in}}%
\pgfpathcurveto{\pgfqpoint{3.081125in}{4.409711in}}{\pgfqpoint{3.076735in}{4.399112in}}{\pgfqpoint{3.076735in}{4.388061in}}%
\pgfpathcurveto{\pgfqpoint{3.076735in}{4.377011in}}{\pgfqpoint{3.081125in}{4.366412in}}{\pgfqpoint{3.088939in}{4.358599in}}%
\pgfpathcurveto{\pgfqpoint{3.096753in}{4.350785in}}{\pgfqpoint{3.107352in}{4.346395in}}{\pgfqpoint{3.118402in}{4.346395in}}%
\pgfpathclose%
\pgfusepath{stroke,fill}%
\end{pgfscope}%
\begin{pgfscope}%
\pgfpathrectangle{\pgfqpoint{0.393613in}{0.331635in}}{\pgfqpoint{9.300000in}{7.700000in}}%
\pgfusepath{clip}%
\pgfsetbuttcap%
\pgfsetroundjoin%
\definecolor{currentfill}{rgb}{0.631373,0.788235,0.956863}%
\pgfsetfillcolor{currentfill}%
\pgfsetlinewidth{0.481800pt}%
\definecolor{currentstroke}{rgb}{1.000000,1.000000,1.000000}%
\pgfsetstrokecolor{currentstroke}%
\pgfsetdash{}{0pt}%
\pgfpathmoveto{\pgfqpoint{3.792263in}{1.975665in}}%
\pgfpathcurveto{\pgfqpoint{3.803313in}{1.975665in}}{\pgfqpoint{3.813912in}{1.980055in}}{\pgfqpoint{3.821725in}{1.987869in}}%
\pgfpathcurveto{\pgfqpoint{3.829539in}{1.995682in}}{\pgfqpoint{3.833929in}{2.006282in}}{\pgfqpoint{3.833929in}{2.017332in}}%
\pgfpathcurveto{\pgfqpoint{3.833929in}{2.028382in}}{\pgfqpoint{3.829539in}{2.038981in}}{\pgfqpoint{3.821725in}{2.046794in}}%
\pgfpathcurveto{\pgfqpoint{3.813912in}{2.054608in}}{\pgfqpoint{3.803313in}{2.058998in}}{\pgfqpoint{3.792263in}{2.058998in}}%
\pgfpathcurveto{\pgfqpoint{3.781212in}{2.058998in}}{\pgfqpoint{3.770613in}{2.054608in}}{\pgfqpoint{3.762800in}{2.046794in}}%
\pgfpathcurveto{\pgfqpoint{3.754986in}{2.038981in}}{\pgfqpoint{3.750596in}{2.028382in}}{\pgfqpoint{3.750596in}{2.017332in}}%
\pgfpathcurveto{\pgfqpoint{3.750596in}{2.006282in}}{\pgfqpoint{3.754986in}{1.995682in}}{\pgfqpoint{3.762800in}{1.987869in}}%
\pgfpathcurveto{\pgfqpoint{3.770613in}{1.980055in}}{\pgfqpoint{3.781212in}{1.975665in}}{\pgfqpoint{3.792263in}{1.975665in}}%
\pgfpathclose%
\pgfusepath{stroke,fill}%
\end{pgfscope}%
\begin{pgfscope}%
\pgfpathrectangle{\pgfqpoint{0.393613in}{0.331635in}}{\pgfqpoint{9.300000in}{7.700000in}}%
\pgfusepath{clip}%
\pgfsetbuttcap%
\pgfsetroundjoin%
\definecolor{currentfill}{rgb}{0.631373,0.788235,0.956863}%
\pgfsetfillcolor{currentfill}%
\pgfsetlinewidth{0.481800pt}%
\definecolor{currentstroke}{rgb}{1.000000,1.000000,1.000000}%
\pgfsetstrokecolor{currentstroke}%
\pgfsetdash{}{0pt}%
\pgfpathmoveto{\pgfqpoint{1.145785in}{5.805643in}}%
\pgfpathcurveto{\pgfqpoint{1.156835in}{5.805643in}}{\pgfqpoint{1.167434in}{5.810034in}}{\pgfqpoint{1.175247in}{5.817847in}}%
\pgfpathcurveto{\pgfqpoint{1.183061in}{5.825661in}}{\pgfqpoint{1.187451in}{5.836260in}}{\pgfqpoint{1.187451in}{5.847310in}}%
\pgfpathcurveto{\pgfqpoint{1.187451in}{5.858360in}}{\pgfqpoint{1.183061in}{5.868959in}}{\pgfqpoint{1.175247in}{5.876773in}}%
\pgfpathcurveto{\pgfqpoint{1.167434in}{5.884586in}}{\pgfqpoint{1.156835in}{5.888977in}}{\pgfqpoint{1.145785in}{5.888977in}}%
\pgfpathcurveto{\pgfqpoint{1.134734in}{5.888977in}}{\pgfqpoint{1.124135in}{5.884586in}}{\pgfqpoint{1.116322in}{5.876773in}}%
\pgfpathcurveto{\pgfqpoint{1.108508in}{5.868959in}}{\pgfqpoint{1.104118in}{5.858360in}}{\pgfqpoint{1.104118in}{5.847310in}}%
\pgfpathcurveto{\pgfqpoint{1.104118in}{5.836260in}}{\pgfqpoint{1.108508in}{5.825661in}}{\pgfqpoint{1.116322in}{5.817847in}}%
\pgfpathcurveto{\pgfqpoint{1.124135in}{5.810034in}}{\pgfqpoint{1.134734in}{5.805643in}}{\pgfqpoint{1.145785in}{5.805643in}}%
\pgfpathclose%
\pgfusepath{stroke,fill}%
\end{pgfscope}%
\begin{pgfscope}%
\pgfpathrectangle{\pgfqpoint{0.393613in}{0.331635in}}{\pgfqpoint{9.300000in}{7.700000in}}%
\pgfusepath{clip}%
\pgfsetbuttcap%
\pgfsetroundjoin%
\definecolor{currentfill}{rgb}{0.631373,0.788235,0.956863}%
\pgfsetfillcolor{currentfill}%
\pgfsetlinewidth{0.481800pt}%
\definecolor{currentstroke}{rgb}{1.000000,1.000000,1.000000}%
\pgfsetstrokecolor{currentstroke}%
\pgfsetdash{}{0pt}%
\pgfpathmoveto{\pgfqpoint{2.952892in}{1.892306in}}%
\pgfpathcurveto{\pgfqpoint{2.963942in}{1.892306in}}{\pgfqpoint{2.974541in}{1.896697in}}{\pgfqpoint{2.982355in}{1.904510in}}%
\pgfpathcurveto{\pgfqpoint{2.990168in}{1.912324in}}{\pgfqpoint{2.994559in}{1.922923in}}{\pgfqpoint{2.994559in}{1.933973in}}%
\pgfpathcurveto{\pgfqpoint{2.994559in}{1.945023in}}{\pgfqpoint{2.990168in}{1.955622in}}{\pgfqpoint{2.982355in}{1.963436in}}%
\pgfpathcurveto{\pgfqpoint{2.974541in}{1.971249in}}{\pgfqpoint{2.963942in}{1.975640in}}{\pgfqpoint{2.952892in}{1.975640in}}%
\pgfpathcurveto{\pgfqpoint{2.941842in}{1.975640in}}{\pgfqpoint{2.931243in}{1.971249in}}{\pgfqpoint{2.923429in}{1.963436in}}%
\pgfpathcurveto{\pgfqpoint{2.915616in}{1.955622in}}{\pgfqpoint{2.911225in}{1.945023in}}{\pgfqpoint{2.911225in}{1.933973in}}%
\pgfpathcurveto{\pgfqpoint{2.911225in}{1.922923in}}{\pgfqpoint{2.915616in}{1.912324in}}{\pgfqpoint{2.923429in}{1.904510in}}%
\pgfpathcurveto{\pgfqpoint{2.931243in}{1.896697in}}{\pgfqpoint{2.941842in}{1.892306in}}{\pgfqpoint{2.952892in}{1.892306in}}%
\pgfpathclose%
\pgfusepath{stroke,fill}%
\end{pgfscope}%
\begin{pgfscope}%
\pgfpathrectangle{\pgfqpoint{0.393613in}{0.331635in}}{\pgfqpoint{9.300000in}{7.700000in}}%
\pgfusepath{clip}%
\pgfsetbuttcap%
\pgfsetroundjoin%
\definecolor{currentfill}{rgb}{0.631373,0.788235,0.956863}%
\pgfsetfillcolor{currentfill}%
\pgfsetlinewidth{0.481800pt}%
\definecolor{currentstroke}{rgb}{1.000000,1.000000,1.000000}%
\pgfsetstrokecolor{currentstroke}%
\pgfsetdash{}{0pt}%
\pgfpathmoveto{\pgfqpoint{4.435471in}{5.526080in}}%
\pgfpathcurveto{\pgfqpoint{4.446521in}{5.526080in}}{\pgfqpoint{4.457120in}{5.530471in}}{\pgfqpoint{4.464933in}{5.538284in}}%
\pgfpathcurveto{\pgfqpoint{4.472747in}{5.546098in}}{\pgfqpoint{4.477137in}{5.556697in}}{\pgfqpoint{4.477137in}{5.567747in}}%
\pgfpathcurveto{\pgfqpoint{4.477137in}{5.578797in}}{\pgfqpoint{4.472747in}{5.589396in}}{\pgfqpoint{4.464933in}{5.597210in}}%
\pgfpathcurveto{\pgfqpoint{4.457120in}{5.605023in}}{\pgfqpoint{4.446521in}{5.609414in}}{\pgfqpoint{4.435471in}{5.609414in}}%
\pgfpathcurveto{\pgfqpoint{4.424421in}{5.609414in}}{\pgfqpoint{4.413822in}{5.605023in}}{\pgfqpoint{4.406008in}{5.597210in}}%
\pgfpathcurveto{\pgfqpoint{4.398194in}{5.589396in}}{\pgfqpoint{4.393804in}{5.578797in}}{\pgfqpoint{4.393804in}{5.567747in}}%
\pgfpathcurveto{\pgfqpoint{4.393804in}{5.556697in}}{\pgfqpoint{4.398194in}{5.546098in}}{\pgfqpoint{4.406008in}{5.538284in}}%
\pgfpathcurveto{\pgfqpoint{4.413822in}{5.530471in}}{\pgfqpoint{4.424421in}{5.526080in}}{\pgfqpoint{4.435471in}{5.526080in}}%
\pgfpathclose%
\pgfusepath{stroke,fill}%
\end{pgfscope}%
\begin{pgfscope}%
\pgfpathrectangle{\pgfqpoint{0.393613in}{0.331635in}}{\pgfqpoint{9.300000in}{7.700000in}}%
\pgfusepath{clip}%
\pgfsetbuttcap%
\pgfsetroundjoin%
\definecolor{currentfill}{rgb}{0.631373,0.788235,0.956863}%
\pgfsetfillcolor{currentfill}%
\pgfsetlinewidth{0.481800pt}%
\definecolor{currentstroke}{rgb}{1.000000,1.000000,1.000000}%
\pgfsetstrokecolor{currentstroke}%
\pgfsetdash{}{0pt}%
\pgfpathmoveto{\pgfqpoint{1.974238in}{5.767666in}}%
\pgfpathcurveto{\pgfqpoint{1.985288in}{5.767666in}}{\pgfqpoint{1.995887in}{5.772056in}}{\pgfqpoint{2.003700in}{5.779869in}}%
\pgfpathcurveto{\pgfqpoint{2.011514in}{5.787683in}}{\pgfqpoint{2.015904in}{5.798282in}}{\pgfqpoint{2.015904in}{5.809332in}}%
\pgfpathcurveto{\pgfqpoint{2.015904in}{5.820382in}}{\pgfqpoint{2.011514in}{5.830981in}}{\pgfqpoint{2.003700in}{5.838795in}}%
\pgfpathcurveto{\pgfqpoint{1.995887in}{5.846609in}}{\pgfqpoint{1.985288in}{5.850999in}}{\pgfqpoint{1.974238in}{5.850999in}}%
\pgfpathcurveto{\pgfqpoint{1.963187in}{5.850999in}}{\pgfqpoint{1.952588in}{5.846609in}}{\pgfqpoint{1.944775in}{5.838795in}}%
\pgfpathcurveto{\pgfqpoint{1.936961in}{5.830981in}}{\pgfqpoint{1.932571in}{5.820382in}}{\pgfqpoint{1.932571in}{5.809332in}}%
\pgfpathcurveto{\pgfqpoint{1.932571in}{5.798282in}}{\pgfqpoint{1.936961in}{5.787683in}}{\pgfqpoint{1.944775in}{5.779869in}}%
\pgfpathcurveto{\pgfqpoint{1.952588in}{5.772056in}}{\pgfqpoint{1.963187in}{5.767666in}}{\pgfqpoint{1.974238in}{5.767666in}}%
\pgfpathclose%
\pgfusepath{stroke,fill}%
\end{pgfscope}%
\begin{pgfscope}%
\pgfpathrectangle{\pgfqpoint{0.393613in}{0.331635in}}{\pgfqpoint{9.300000in}{7.700000in}}%
\pgfusepath{clip}%
\pgfsetbuttcap%
\pgfsetroundjoin%
\definecolor{currentfill}{rgb}{0.631373,0.788235,0.956863}%
\pgfsetfillcolor{currentfill}%
\pgfsetlinewidth{0.481800pt}%
\definecolor{currentstroke}{rgb}{1.000000,1.000000,1.000000}%
\pgfsetstrokecolor{currentstroke}%
\pgfsetdash{}{0pt}%
\pgfpathmoveto{\pgfqpoint{2.379728in}{6.669063in}}%
\pgfpathcurveto{\pgfqpoint{2.390778in}{6.669063in}}{\pgfqpoint{2.401377in}{6.673453in}}{\pgfqpoint{2.409191in}{6.681267in}}%
\pgfpathcurveto{\pgfqpoint{2.417004in}{6.689081in}}{\pgfqpoint{2.421395in}{6.699680in}}{\pgfqpoint{2.421395in}{6.710730in}}%
\pgfpathcurveto{\pgfqpoint{2.421395in}{6.721780in}}{\pgfqpoint{2.417004in}{6.732379in}}{\pgfqpoint{2.409191in}{6.740192in}}%
\pgfpathcurveto{\pgfqpoint{2.401377in}{6.748006in}}{\pgfqpoint{2.390778in}{6.752396in}}{\pgfqpoint{2.379728in}{6.752396in}}%
\pgfpathcurveto{\pgfqpoint{2.368678in}{6.752396in}}{\pgfqpoint{2.358079in}{6.748006in}}{\pgfqpoint{2.350265in}{6.740192in}}%
\pgfpathcurveto{\pgfqpoint{2.342451in}{6.732379in}}{\pgfqpoint{2.338061in}{6.721780in}}{\pgfqpoint{2.338061in}{6.710730in}}%
\pgfpathcurveto{\pgfqpoint{2.338061in}{6.699680in}}{\pgfqpoint{2.342451in}{6.689081in}}{\pgfqpoint{2.350265in}{6.681267in}}%
\pgfpathcurveto{\pgfqpoint{2.358079in}{6.673453in}}{\pgfqpoint{2.368678in}{6.669063in}}{\pgfqpoint{2.379728in}{6.669063in}}%
\pgfpathclose%
\pgfusepath{stroke,fill}%
\end{pgfscope}%
\begin{pgfscope}%
\pgfpathrectangle{\pgfqpoint{0.393613in}{0.331635in}}{\pgfqpoint{9.300000in}{7.700000in}}%
\pgfusepath{clip}%
\pgfsetbuttcap%
\pgfsetroundjoin%
\definecolor{currentfill}{rgb}{0.631373,0.788235,0.956863}%
\pgfsetfillcolor{currentfill}%
\pgfsetlinewidth{0.481800pt}%
\definecolor{currentstroke}{rgb}{1.000000,1.000000,1.000000}%
\pgfsetstrokecolor{currentstroke}%
\pgfsetdash{}{0pt}%
\pgfpathmoveto{\pgfqpoint{2.765242in}{4.850114in}}%
\pgfpathcurveto{\pgfqpoint{2.776292in}{4.850114in}}{\pgfqpoint{2.786891in}{4.854505in}}{\pgfqpoint{2.794705in}{4.862318in}}%
\pgfpathcurveto{\pgfqpoint{2.802518in}{4.870132in}}{\pgfqpoint{2.806908in}{4.880731in}}{\pgfqpoint{2.806908in}{4.891781in}}%
\pgfpathcurveto{\pgfqpoint{2.806908in}{4.902831in}}{\pgfqpoint{2.802518in}{4.913430in}}{\pgfqpoint{2.794705in}{4.921244in}}%
\pgfpathcurveto{\pgfqpoint{2.786891in}{4.929057in}}{\pgfqpoint{2.776292in}{4.933448in}}{\pgfqpoint{2.765242in}{4.933448in}}%
\pgfpathcurveto{\pgfqpoint{2.754192in}{4.933448in}}{\pgfqpoint{2.743593in}{4.929057in}}{\pgfqpoint{2.735779in}{4.921244in}}%
\pgfpathcurveto{\pgfqpoint{2.727965in}{4.913430in}}{\pgfqpoint{2.723575in}{4.902831in}}{\pgfqpoint{2.723575in}{4.891781in}}%
\pgfpathcurveto{\pgfqpoint{2.723575in}{4.880731in}}{\pgfqpoint{2.727965in}{4.870132in}}{\pgfqpoint{2.735779in}{4.862318in}}%
\pgfpathcurveto{\pgfqpoint{2.743593in}{4.854505in}}{\pgfqpoint{2.754192in}{4.850114in}}{\pgfqpoint{2.765242in}{4.850114in}}%
\pgfpathclose%
\pgfusepath{stroke,fill}%
\end{pgfscope}%
\begin{pgfscope}%
\pgfpathrectangle{\pgfqpoint{0.393613in}{0.331635in}}{\pgfqpoint{9.300000in}{7.700000in}}%
\pgfusepath{clip}%
\pgfsetbuttcap%
\pgfsetroundjoin%
\definecolor{currentfill}{rgb}{0.631373,0.788235,0.956863}%
\pgfsetfillcolor{currentfill}%
\pgfsetlinewidth{0.481800pt}%
\definecolor{currentstroke}{rgb}{1.000000,1.000000,1.000000}%
\pgfsetstrokecolor{currentstroke}%
\pgfsetdash{}{0pt}%
\pgfpathmoveto{\pgfqpoint{2.626630in}{6.501501in}}%
\pgfpathcurveto{\pgfqpoint{2.637680in}{6.501501in}}{\pgfqpoint{2.648279in}{6.505892in}}{\pgfqpoint{2.656093in}{6.513705in}}%
\pgfpathcurveto{\pgfqpoint{2.663906in}{6.521519in}}{\pgfqpoint{2.668296in}{6.532118in}}{\pgfqpoint{2.668296in}{6.543168in}}%
\pgfpathcurveto{\pgfqpoint{2.668296in}{6.554218in}}{\pgfqpoint{2.663906in}{6.564817in}}{\pgfqpoint{2.656093in}{6.572631in}}%
\pgfpathcurveto{\pgfqpoint{2.648279in}{6.580445in}}{\pgfqpoint{2.637680in}{6.584835in}}{\pgfqpoint{2.626630in}{6.584835in}}%
\pgfpathcurveto{\pgfqpoint{2.615580in}{6.584835in}}{\pgfqpoint{2.604981in}{6.580445in}}{\pgfqpoint{2.597167in}{6.572631in}}%
\pgfpathcurveto{\pgfqpoint{2.589353in}{6.564817in}}{\pgfqpoint{2.584963in}{6.554218in}}{\pgfqpoint{2.584963in}{6.543168in}}%
\pgfpathcurveto{\pgfqpoint{2.584963in}{6.532118in}}{\pgfqpoint{2.589353in}{6.521519in}}{\pgfqpoint{2.597167in}{6.513705in}}%
\pgfpathcurveto{\pgfqpoint{2.604981in}{6.505892in}}{\pgfqpoint{2.615580in}{6.501501in}}{\pgfqpoint{2.626630in}{6.501501in}}%
\pgfpathclose%
\pgfusepath{stroke,fill}%
\end{pgfscope}%
\begin{pgfscope}%
\pgfpathrectangle{\pgfqpoint{0.393613in}{0.331635in}}{\pgfqpoint{9.300000in}{7.700000in}}%
\pgfusepath{clip}%
\pgfsetbuttcap%
\pgfsetroundjoin%
\definecolor{currentfill}{rgb}{0.631373,0.788235,0.956863}%
\pgfsetfillcolor{currentfill}%
\pgfsetlinewidth{0.481800pt}%
\definecolor{currentstroke}{rgb}{1.000000,1.000000,1.000000}%
\pgfsetstrokecolor{currentstroke}%
\pgfsetdash{}{0pt}%
\pgfpathmoveto{\pgfqpoint{3.301695in}{5.285888in}}%
\pgfpathcurveto{\pgfqpoint{3.312745in}{5.285888in}}{\pgfqpoint{3.323344in}{5.290279in}}{\pgfqpoint{3.331157in}{5.298092in}}%
\pgfpathcurveto{\pgfqpoint{3.338971in}{5.305906in}}{\pgfqpoint{3.343361in}{5.316505in}}{\pgfqpoint{3.343361in}{5.327555in}}%
\pgfpathcurveto{\pgfqpoint{3.343361in}{5.338605in}}{\pgfqpoint{3.338971in}{5.349204in}}{\pgfqpoint{3.331157in}{5.357018in}}%
\pgfpathcurveto{\pgfqpoint{3.323344in}{5.364831in}}{\pgfqpoint{3.312745in}{5.369222in}}{\pgfqpoint{3.301695in}{5.369222in}}%
\pgfpathcurveto{\pgfqpoint{3.290644in}{5.369222in}}{\pgfqpoint{3.280045in}{5.364831in}}{\pgfqpoint{3.272232in}{5.357018in}}%
\pgfpathcurveto{\pgfqpoint{3.264418in}{5.349204in}}{\pgfqpoint{3.260028in}{5.338605in}}{\pgfqpoint{3.260028in}{5.327555in}}%
\pgfpathcurveto{\pgfqpoint{3.260028in}{5.316505in}}{\pgfqpoint{3.264418in}{5.305906in}}{\pgfqpoint{3.272232in}{5.298092in}}%
\pgfpathcurveto{\pgfqpoint{3.280045in}{5.290279in}}{\pgfqpoint{3.290644in}{5.285888in}}{\pgfqpoint{3.301695in}{5.285888in}}%
\pgfpathclose%
\pgfusepath{stroke,fill}%
\end{pgfscope}%
\begin{pgfscope}%
\pgfpathrectangle{\pgfqpoint{0.393613in}{0.331635in}}{\pgfqpoint{9.300000in}{7.700000in}}%
\pgfusepath{clip}%
\pgfsetbuttcap%
\pgfsetroundjoin%
\definecolor{currentfill}{rgb}{0.631373,0.788235,0.956863}%
\pgfsetfillcolor{currentfill}%
\pgfsetlinewidth{0.481800pt}%
\definecolor{currentstroke}{rgb}{1.000000,1.000000,1.000000}%
\pgfsetstrokecolor{currentstroke}%
\pgfsetdash{}{0pt}%
\pgfpathmoveto{\pgfqpoint{0.968107in}{5.855717in}}%
\pgfpathcurveto{\pgfqpoint{0.979157in}{5.855717in}}{\pgfqpoint{0.989756in}{5.860108in}}{\pgfqpoint{0.997570in}{5.867921in}}%
\pgfpathcurveto{\pgfqpoint{1.005384in}{5.875735in}}{\pgfqpoint{1.009774in}{5.886334in}}{\pgfqpoint{1.009774in}{5.897384in}}%
\pgfpathcurveto{\pgfqpoint{1.009774in}{5.908434in}}{\pgfqpoint{1.005384in}{5.919033in}}{\pgfqpoint{0.997570in}{5.926847in}}%
\pgfpathcurveto{\pgfqpoint{0.989756in}{5.934660in}}{\pgfqpoint{0.979157in}{5.939051in}}{\pgfqpoint{0.968107in}{5.939051in}}%
\pgfpathcurveto{\pgfqpoint{0.957057in}{5.939051in}}{\pgfqpoint{0.946458in}{5.934660in}}{\pgfqpoint{0.938644in}{5.926847in}}%
\pgfpathcurveto{\pgfqpoint{0.930831in}{5.919033in}}{\pgfqpoint{0.926441in}{5.908434in}}{\pgfqpoint{0.926441in}{5.897384in}}%
\pgfpathcurveto{\pgfqpoint{0.926441in}{5.886334in}}{\pgfqpoint{0.930831in}{5.875735in}}{\pgfqpoint{0.938644in}{5.867921in}}%
\pgfpathcurveto{\pgfqpoint{0.946458in}{5.860108in}}{\pgfqpoint{0.957057in}{5.855717in}}{\pgfqpoint{0.968107in}{5.855717in}}%
\pgfpathclose%
\pgfusepath{stroke,fill}%
\end{pgfscope}%
\begin{pgfscope}%
\pgfpathrectangle{\pgfqpoint{0.393613in}{0.331635in}}{\pgfqpoint{9.300000in}{7.700000in}}%
\pgfusepath{clip}%
\pgfsetbuttcap%
\pgfsetroundjoin%
\definecolor{currentfill}{rgb}{0.631373,0.788235,0.956863}%
\pgfsetfillcolor{currentfill}%
\pgfsetlinewidth{0.481800pt}%
\definecolor{currentstroke}{rgb}{1.000000,1.000000,1.000000}%
\pgfsetstrokecolor{currentstroke}%
\pgfsetdash{}{0pt}%
\pgfpathmoveto{\pgfqpoint{4.228775in}{6.130862in}}%
\pgfpathcurveto{\pgfqpoint{4.239826in}{6.130862in}}{\pgfqpoint{4.250425in}{6.135252in}}{\pgfqpoint{4.258238in}{6.143066in}}%
\pgfpathcurveto{\pgfqpoint{4.266052in}{6.150879in}}{\pgfqpoint{4.270442in}{6.161478in}}{\pgfqpoint{4.270442in}{6.172529in}}%
\pgfpathcurveto{\pgfqpoint{4.270442in}{6.183579in}}{\pgfqpoint{4.266052in}{6.194178in}}{\pgfqpoint{4.258238in}{6.201991in}}%
\pgfpathcurveto{\pgfqpoint{4.250425in}{6.209805in}}{\pgfqpoint{4.239826in}{6.214195in}}{\pgfqpoint{4.228775in}{6.214195in}}%
\pgfpathcurveto{\pgfqpoint{4.217725in}{6.214195in}}{\pgfqpoint{4.207126in}{6.209805in}}{\pgfqpoint{4.199313in}{6.201991in}}%
\pgfpathcurveto{\pgfqpoint{4.191499in}{6.194178in}}{\pgfqpoint{4.187109in}{6.183579in}}{\pgfqpoint{4.187109in}{6.172529in}}%
\pgfpathcurveto{\pgfqpoint{4.187109in}{6.161478in}}{\pgfqpoint{4.191499in}{6.150879in}}{\pgfqpoint{4.199313in}{6.143066in}}%
\pgfpathcurveto{\pgfqpoint{4.207126in}{6.135252in}}{\pgfqpoint{4.217725in}{6.130862in}}{\pgfqpoint{4.228775in}{6.130862in}}%
\pgfpathclose%
\pgfusepath{stroke,fill}%
\end{pgfscope}%
\begin{pgfscope}%
\pgfpathrectangle{\pgfqpoint{0.393613in}{0.331635in}}{\pgfqpoint{9.300000in}{7.700000in}}%
\pgfusepath{clip}%
\pgfsetbuttcap%
\pgfsetroundjoin%
\definecolor{currentfill}{rgb}{0.631373,0.788235,0.956863}%
\pgfsetfillcolor{currentfill}%
\pgfsetlinewidth{0.481800pt}%
\definecolor{currentstroke}{rgb}{1.000000,1.000000,1.000000}%
\pgfsetstrokecolor{currentstroke}%
\pgfsetdash{}{0pt}%
\pgfpathmoveto{\pgfqpoint{2.425460in}{6.990714in}}%
\pgfpathcurveto{\pgfqpoint{2.436510in}{6.990714in}}{\pgfqpoint{2.447109in}{6.995104in}}{\pgfqpoint{2.454922in}{7.002918in}}%
\pgfpathcurveto{\pgfqpoint{2.462736in}{7.010731in}}{\pgfqpoint{2.467126in}{7.021330in}}{\pgfqpoint{2.467126in}{7.032380in}}%
\pgfpathcurveto{\pgfqpoint{2.467126in}{7.043431in}}{\pgfqpoint{2.462736in}{7.054030in}}{\pgfqpoint{2.454922in}{7.061843in}}%
\pgfpathcurveto{\pgfqpoint{2.447109in}{7.069657in}}{\pgfqpoint{2.436510in}{7.074047in}}{\pgfqpoint{2.425460in}{7.074047in}}%
\pgfpathcurveto{\pgfqpoint{2.414409in}{7.074047in}}{\pgfqpoint{2.403810in}{7.069657in}}{\pgfqpoint{2.395997in}{7.061843in}}%
\pgfpathcurveto{\pgfqpoint{2.388183in}{7.054030in}}{\pgfqpoint{2.383793in}{7.043431in}}{\pgfqpoint{2.383793in}{7.032380in}}%
\pgfpathcurveto{\pgfqpoint{2.383793in}{7.021330in}}{\pgfqpoint{2.388183in}{7.010731in}}{\pgfqpoint{2.395997in}{7.002918in}}%
\pgfpathcurveto{\pgfqpoint{2.403810in}{6.995104in}}{\pgfqpoint{2.414409in}{6.990714in}}{\pgfqpoint{2.425460in}{6.990714in}}%
\pgfpathclose%
\pgfusepath{stroke,fill}%
\end{pgfscope}%
\begin{pgfscope}%
\pgfpathrectangle{\pgfqpoint{0.393613in}{0.331635in}}{\pgfqpoint{9.300000in}{7.700000in}}%
\pgfusepath{clip}%
\pgfsetbuttcap%
\pgfsetroundjoin%
\definecolor{currentfill}{rgb}{0.631373,0.788235,0.956863}%
\pgfsetfillcolor{currentfill}%
\pgfsetlinewidth{0.481800pt}%
\definecolor{currentstroke}{rgb}{1.000000,1.000000,1.000000}%
\pgfsetstrokecolor{currentstroke}%
\pgfsetdash{}{0pt}%
\pgfpathmoveto{\pgfqpoint{2.491424in}{5.665706in}}%
\pgfpathcurveto{\pgfqpoint{2.502474in}{5.665706in}}{\pgfqpoint{2.513073in}{5.670096in}}{\pgfqpoint{2.520887in}{5.677910in}}%
\pgfpathcurveto{\pgfqpoint{2.528700in}{5.685723in}}{\pgfqpoint{2.533091in}{5.696322in}}{\pgfqpoint{2.533091in}{5.707373in}}%
\pgfpathcurveto{\pgfqpoint{2.533091in}{5.718423in}}{\pgfqpoint{2.528700in}{5.729022in}}{\pgfqpoint{2.520887in}{5.736835in}}%
\pgfpathcurveto{\pgfqpoint{2.513073in}{5.744649in}}{\pgfqpoint{2.502474in}{5.749039in}}{\pgfqpoint{2.491424in}{5.749039in}}%
\pgfpathcurveto{\pgfqpoint{2.480374in}{5.749039in}}{\pgfqpoint{2.469775in}{5.744649in}}{\pgfqpoint{2.461961in}{5.736835in}}%
\pgfpathcurveto{\pgfqpoint{2.454148in}{5.729022in}}{\pgfqpoint{2.449757in}{5.718423in}}{\pgfqpoint{2.449757in}{5.707373in}}%
\pgfpathcurveto{\pgfqpoint{2.449757in}{5.696322in}}{\pgfqpoint{2.454148in}{5.685723in}}{\pgfqpoint{2.461961in}{5.677910in}}%
\pgfpathcurveto{\pgfqpoint{2.469775in}{5.670096in}}{\pgfqpoint{2.480374in}{5.665706in}}{\pgfqpoint{2.491424in}{5.665706in}}%
\pgfpathclose%
\pgfusepath{stroke,fill}%
\end{pgfscope}%
\begin{pgfscope}%
\pgfpathrectangle{\pgfqpoint{0.393613in}{0.331635in}}{\pgfqpoint{9.300000in}{7.700000in}}%
\pgfusepath{clip}%
\pgfsetbuttcap%
\pgfsetroundjoin%
\definecolor{currentfill}{rgb}{0.631373,0.788235,0.956863}%
\pgfsetfillcolor{currentfill}%
\pgfsetlinewidth{0.481800pt}%
\definecolor{currentstroke}{rgb}{1.000000,1.000000,1.000000}%
\pgfsetstrokecolor{currentstroke}%
\pgfsetdash{}{0pt}%
\pgfpathmoveto{\pgfqpoint{5.405752in}{7.030029in}}%
\pgfpathcurveto{\pgfqpoint{5.416802in}{7.030029in}}{\pgfqpoint{5.427401in}{7.034419in}}{\pgfqpoint{5.435215in}{7.042233in}}%
\pgfpathcurveto{\pgfqpoint{5.443029in}{7.050046in}}{\pgfqpoint{5.447419in}{7.060645in}}{\pgfqpoint{5.447419in}{7.071695in}}%
\pgfpathcurveto{\pgfqpoint{5.447419in}{7.082746in}}{\pgfqpoint{5.443029in}{7.093345in}}{\pgfqpoint{5.435215in}{7.101158in}}%
\pgfpathcurveto{\pgfqpoint{5.427401in}{7.108972in}}{\pgfqpoint{5.416802in}{7.113362in}}{\pgfqpoint{5.405752in}{7.113362in}}%
\pgfpathcurveto{\pgfqpoint{5.394702in}{7.113362in}}{\pgfqpoint{5.384103in}{7.108972in}}{\pgfqpoint{5.376289in}{7.101158in}}%
\pgfpathcurveto{\pgfqpoint{5.368476in}{7.093345in}}{\pgfqpoint{5.364086in}{7.082746in}}{\pgfqpoint{5.364086in}{7.071695in}}%
\pgfpathcurveto{\pgfqpoint{5.364086in}{7.060645in}}{\pgfqpoint{5.368476in}{7.050046in}}{\pgfqpoint{5.376289in}{7.042233in}}%
\pgfpathcurveto{\pgfqpoint{5.384103in}{7.034419in}}{\pgfqpoint{5.394702in}{7.030029in}}{\pgfqpoint{5.405752in}{7.030029in}}%
\pgfpathclose%
\pgfusepath{stroke,fill}%
\end{pgfscope}%
\begin{pgfscope}%
\pgfpathrectangle{\pgfqpoint{0.393613in}{0.331635in}}{\pgfqpoint{9.300000in}{7.700000in}}%
\pgfusepath{clip}%
\pgfsetbuttcap%
\pgfsetroundjoin%
\definecolor{currentfill}{rgb}{0.631373,0.788235,0.956863}%
\pgfsetfillcolor{currentfill}%
\pgfsetlinewidth{0.481800pt}%
\definecolor{currentstroke}{rgb}{1.000000,1.000000,1.000000}%
\pgfsetstrokecolor{currentstroke}%
\pgfsetdash{}{0pt}%
\pgfpathmoveto{\pgfqpoint{4.029243in}{1.937457in}}%
\pgfpathcurveto{\pgfqpoint{4.040293in}{1.937457in}}{\pgfqpoint{4.050892in}{1.941847in}}{\pgfqpoint{4.058706in}{1.949661in}}%
\pgfpathcurveto{\pgfqpoint{4.066520in}{1.957475in}}{\pgfqpoint{4.070910in}{1.968074in}}{\pgfqpoint{4.070910in}{1.979124in}}%
\pgfpathcurveto{\pgfqpoint{4.070910in}{1.990174in}}{\pgfqpoint{4.066520in}{2.000773in}}{\pgfqpoint{4.058706in}{2.008587in}}%
\pgfpathcurveto{\pgfqpoint{4.050892in}{2.016400in}}{\pgfqpoint{4.040293in}{2.020790in}}{\pgfqpoint{4.029243in}{2.020790in}}%
\pgfpathcurveto{\pgfqpoint{4.018193in}{2.020790in}}{\pgfqpoint{4.007594in}{2.016400in}}{\pgfqpoint{3.999780in}{2.008587in}}%
\pgfpathcurveto{\pgfqpoint{3.991967in}{2.000773in}}{\pgfqpoint{3.987576in}{1.990174in}}{\pgfqpoint{3.987576in}{1.979124in}}%
\pgfpathcurveto{\pgfqpoint{3.987576in}{1.968074in}}{\pgfqpoint{3.991967in}{1.957475in}}{\pgfqpoint{3.999780in}{1.949661in}}%
\pgfpathcurveto{\pgfqpoint{4.007594in}{1.941847in}}{\pgfqpoint{4.018193in}{1.937457in}}{\pgfqpoint{4.029243in}{1.937457in}}%
\pgfpathclose%
\pgfusepath{stroke,fill}%
\end{pgfscope}%
\begin{pgfscope}%
\pgfpathrectangle{\pgfqpoint{0.393613in}{0.331635in}}{\pgfqpoint{9.300000in}{7.700000in}}%
\pgfusepath{clip}%
\pgfsetbuttcap%
\pgfsetroundjoin%
\definecolor{currentfill}{rgb}{0.631373,0.788235,0.956863}%
\pgfsetfillcolor{currentfill}%
\pgfsetlinewidth{0.481800pt}%
\definecolor{currentstroke}{rgb}{1.000000,1.000000,1.000000}%
\pgfsetstrokecolor{currentstroke}%
\pgfsetdash{}{0pt}%
\pgfpathmoveto{\pgfqpoint{3.912187in}{3.956741in}}%
\pgfpathcurveto{\pgfqpoint{3.923237in}{3.956741in}}{\pgfqpoint{3.933836in}{3.961131in}}{\pgfqpoint{3.941650in}{3.968945in}}%
\pgfpathcurveto{\pgfqpoint{3.949463in}{3.976759in}}{\pgfqpoint{3.953853in}{3.987358in}}{\pgfqpoint{3.953853in}{3.998408in}}%
\pgfpathcurveto{\pgfqpoint{3.953853in}{4.009458in}}{\pgfqpoint{3.949463in}{4.020057in}}{\pgfqpoint{3.941650in}{4.027871in}}%
\pgfpathcurveto{\pgfqpoint{3.933836in}{4.035684in}}{\pgfqpoint{3.923237in}{4.040074in}}{\pgfqpoint{3.912187in}{4.040074in}}%
\pgfpathcurveto{\pgfqpoint{3.901137in}{4.040074in}}{\pgfqpoint{3.890538in}{4.035684in}}{\pgfqpoint{3.882724in}{4.027871in}}%
\pgfpathcurveto{\pgfqpoint{3.874910in}{4.020057in}}{\pgfqpoint{3.870520in}{4.009458in}}{\pgfqpoint{3.870520in}{3.998408in}}%
\pgfpathcurveto{\pgfqpoint{3.870520in}{3.987358in}}{\pgfqpoint{3.874910in}{3.976759in}}{\pgfqpoint{3.882724in}{3.968945in}}%
\pgfpathcurveto{\pgfqpoint{3.890538in}{3.961131in}}{\pgfqpoint{3.901137in}{3.956741in}}{\pgfqpoint{3.912187in}{3.956741in}}%
\pgfpathclose%
\pgfusepath{stroke,fill}%
\end{pgfscope}%
\begin{pgfscope}%
\pgfpathrectangle{\pgfqpoint{0.393613in}{0.331635in}}{\pgfqpoint{9.300000in}{7.700000in}}%
\pgfusepath{clip}%
\pgfsetbuttcap%
\pgfsetroundjoin%
\definecolor{currentfill}{rgb}{0.631373,0.788235,0.956863}%
\pgfsetfillcolor{currentfill}%
\pgfsetlinewidth{0.481800pt}%
\definecolor{currentstroke}{rgb}{1.000000,1.000000,1.000000}%
\pgfsetstrokecolor{currentstroke}%
\pgfsetdash{}{0pt}%
\pgfpathmoveto{\pgfqpoint{3.189809in}{3.244936in}}%
\pgfpathcurveto{\pgfqpoint{3.200859in}{3.244936in}}{\pgfqpoint{3.211458in}{3.249326in}}{\pgfqpoint{3.219271in}{3.257140in}}%
\pgfpathcurveto{\pgfqpoint{3.227085in}{3.264954in}}{\pgfqpoint{3.231475in}{3.275553in}}{\pgfqpoint{3.231475in}{3.286603in}}%
\pgfpathcurveto{\pgfqpoint{3.231475in}{3.297653in}}{\pgfqpoint{3.227085in}{3.308252in}}{\pgfqpoint{3.219271in}{3.316065in}}%
\pgfpathcurveto{\pgfqpoint{3.211458in}{3.323879in}}{\pgfqpoint{3.200859in}{3.328269in}}{\pgfqpoint{3.189809in}{3.328269in}}%
\pgfpathcurveto{\pgfqpoint{3.178758in}{3.328269in}}{\pgfqpoint{3.168159in}{3.323879in}}{\pgfqpoint{3.160346in}{3.316065in}}%
\pgfpathcurveto{\pgfqpoint{3.152532in}{3.308252in}}{\pgfqpoint{3.148142in}{3.297653in}}{\pgfqpoint{3.148142in}{3.286603in}}%
\pgfpathcurveto{\pgfqpoint{3.148142in}{3.275553in}}{\pgfqpoint{3.152532in}{3.264954in}}{\pgfqpoint{3.160346in}{3.257140in}}%
\pgfpathcurveto{\pgfqpoint{3.168159in}{3.249326in}}{\pgfqpoint{3.178758in}{3.244936in}}{\pgfqpoint{3.189809in}{3.244936in}}%
\pgfpathclose%
\pgfusepath{stroke,fill}%
\end{pgfscope}%
\begin{pgfscope}%
\pgfpathrectangle{\pgfqpoint{0.393613in}{0.331635in}}{\pgfqpoint{9.300000in}{7.700000in}}%
\pgfusepath{clip}%
\pgfsetbuttcap%
\pgfsetroundjoin%
\definecolor{currentfill}{rgb}{0.631373,0.788235,0.956863}%
\pgfsetfillcolor{currentfill}%
\pgfsetlinewidth{0.481800pt}%
\definecolor{currentstroke}{rgb}{1.000000,1.000000,1.000000}%
\pgfsetstrokecolor{currentstroke}%
\pgfsetdash{}{0pt}%
\pgfpathmoveto{\pgfqpoint{2.615024in}{4.448849in}}%
\pgfpathcurveto{\pgfqpoint{2.626074in}{4.448849in}}{\pgfqpoint{2.636673in}{4.453239in}}{\pgfqpoint{2.644487in}{4.461053in}}%
\pgfpathcurveto{\pgfqpoint{2.652300in}{4.468867in}}{\pgfqpoint{2.656691in}{4.479466in}}{\pgfqpoint{2.656691in}{4.490516in}}%
\pgfpathcurveto{\pgfqpoint{2.656691in}{4.501566in}}{\pgfqpoint{2.652300in}{4.512165in}}{\pgfqpoint{2.644487in}{4.519979in}}%
\pgfpathcurveto{\pgfqpoint{2.636673in}{4.527792in}}{\pgfqpoint{2.626074in}{4.532182in}}{\pgfqpoint{2.615024in}{4.532182in}}%
\pgfpathcurveto{\pgfqpoint{2.603974in}{4.532182in}}{\pgfqpoint{2.593375in}{4.527792in}}{\pgfqpoint{2.585561in}{4.519979in}}%
\pgfpathcurveto{\pgfqpoint{2.577747in}{4.512165in}}{\pgfqpoint{2.573357in}{4.501566in}}{\pgfqpoint{2.573357in}{4.490516in}}%
\pgfpathcurveto{\pgfqpoint{2.573357in}{4.479466in}}{\pgfqpoint{2.577747in}{4.468867in}}{\pgfqpoint{2.585561in}{4.461053in}}%
\pgfpathcurveto{\pgfqpoint{2.593375in}{4.453239in}}{\pgfqpoint{2.603974in}{4.448849in}}{\pgfqpoint{2.615024in}{4.448849in}}%
\pgfpathclose%
\pgfusepath{stroke,fill}%
\end{pgfscope}%
\begin{pgfscope}%
\pgfpathrectangle{\pgfqpoint{0.393613in}{0.331635in}}{\pgfqpoint{9.300000in}{7.700000in}}%
\pgfusepath{clip}%
\pgfsetbuttcap%
\pgfsetroundjoin%
\definecolor{currentfill}{rgb}{0.631373,0.788235,0.956863}%
\pgfsetfillcolor{currentfill}%
\pgfsetlinewidth{0.481800pt}%
\definecolor{currentstroke}{rgb}{1.000000,1.000000,1.000000}%
\pgfsetstrokecolor{currentstroke}%
\pgfsetdash{}{0pt}%
\pgfpathmoveto{\pgfqpoint{2.589699in}{5.150005in}}%
\pgfpathcurveto{\pgfqpoint{2.600749in}{5.150005in}}{\pgfqpoint{2.611348in}{5.154395in}}{\pgfqpoint{2.619161in}{5.162209in}}%
\pgfpathcurveto{\pgfqpoint{2.626975in}{5.170022in}}{\pgfqpoint{2.631365in}{5.180621in}}{\pgfqpoint{2.631365in}{5.191672in}}%
\pgfpathcurveto{\pgfqpoint{2.631365in}{5.202722in}}{\pgfqpoint{2.626975in}{5.213321in}}{\pgfqpoint{2.619161in}{5.221134in}}%
\pgfpathcurveto{\pgfqpoint{2.611348in}{5.228948in}}{\pgfqpoint{2.600749in}{5.233338in}}{\pgfqpoint{2.589699in}{5.233338in}}%
\pgfpathcurveto{\pgfqpoint{2.578649in}{5.233338in}}{\pgfqpoint{2.568050in}{5.228948in}}{\pgfqpoint{2.560236in}{5.221134in}}%
\pgfpathcurveto{\pgfqpoint{2.552422in}{5.213321in}}{\pgfqpoint{2.548032in}{5.202722in}}{\pgfqpoint{2.548032in}{5.191672in}}%
\pgfpathcurveto{\pgfqpoint{2.548032in}{5.180621in}}{\pgfqpoint{2.552422in}{5.170022in}}{\pgfqpoint{2.560236in}{5.162209in}}%
\pgfpathcurveto{\pgfqpoint{2.568050in}{5.154395in}}{\pgfqpoint{2.578649in}{5.150005in}}{\pgfqpoint{2.589699in}{5.150005in}}%
\pgfpathclose%
\pgfusepath{stroke,fill}%
\end{pgfscope}%
\begin{pgfscope}%
\pgfpathrectangle{\pgfqpoint{0.393613in}{0.331635in}}{\pgfqpoint{9.300000in}{7.700000in}}%
\pgfusepath{clip}%
\pgfsetbuttcap%
\pgfsetroundjoin%
\definecolor{currentfill}{rgb}{0.631373,0.788235,0.956863}%
\pgfsetfillcolor{currentfill}%
\pgfsetlinewidth{0.481800pt}%
\definecolor{currentstroke}{rgb}{1.000000,1.000000,1.000000}%
\pgfsetstrokecolor{currentstroke}%
\pgfsetdash{}{0pt}%
\pgfpathmoveto{\pgfqpoint{1.233235in}{3.927163in}}%
\pgfpathcurveto{\pgfqpoint{1.244285in}{3.927163in}}{\pgfqpoint{1.254884in}{3.931553in}}{\pgfqpoint{1.262698in}{3.939367in}}%
\pgfpathcurveto{\pgfqpoint{1.270512in}{3.947181in}}{\pgfqpoint{1.274902in}{3.957780in}}{\pgfqpoint{1.274902in}{3.968830in}}%
\pgfpathcurveto{\pgfqpoint{1.274902in}{3.979880in}}{\pgfqpoint{1.270512in}{3.990479in}}{\pgfqpoint{1.262698in}{3.998292in}}%
\pgfpathcurveto{\pgfqpoint{1.254884in}{4.006106in}}{\pgfqpoint{1.244285in}{4.010496in}}{\pgfqpoint{1.233235in}{4.010496in}}%
\pgfpathcurveto{\pgfqpoint{1.222185in}{4.010496in}}{\pgfqpoint{1.211586in}{4.006106in}}{\pgfqpoint{1.203772in}{3.998292in}}%
\pgfpathcurveto{\pgfqpoint{1.195959in}{3.990479in}}{\pgfqpoint{1.191569in}{3.979880in}}{\pgfqpoint{1.191569in}{3.968830in}}%
\pgfpathcurveto{\pgfqpoint{1.191569in}{3.957780in}}{\pgfqpoint{1.195959in}{3.947181in}}{\pgfqpoint{1.203772in}{3.939367in}}%
\pgfpathcurveto{\pgfqpoint{1.211586in}{3.931553in}}{\pgfqpoint{1.222185in}{3.927163in}}{\pgfqpoint{1.233235in}{3.927163in}}%
\pgfpathclose%
\pgfusepath{stroke,fill}%
\end{pgfscope}%
\begin{pgfscope}%
\pgfpathrectangle{\pgfqpoint{0.393613in}{0.331635in}}{\pgfqpoint{9.300000in}{7.700000in}}%
\pgfusepath{clip}%
\pgfsetbuttcap%
\pgfsetroundjoin%
\definecolor{currentfill}{rgb}{0.631373,0.788235,0.956863}%
\pgfsetfillcolor{currentfill}%
\pgfsetlinewidth{0.481800pt}%
\definecolor{currentstroke}{rgb}{1.000000,1.000000,1.000000}%
\pgfsetstrokecolor{currentstroke}%
\pgfsetdash{}{0pt}%
\pgfpathmoveto{\pgfqpoint{3.810785in}{1.904817in}}%
\pgfpathcurveto{\pgfqpoint{3.821835in}{1.904817in}}{\pgfqpoint{3.832434in}{1.909207in}}{\pgfqpoint{3.840248in}{1.917021in}}%
\pgfpathcurveto{\pgfqpoint{3.848061in}{1.924835in}}{\pgfqpoint{3.852452in}{1.935434in}}{\pgfqpoint{3.852452in}{1.946484in}}%
\pgfpathcurveto{\pgfqpoint{3.852452in}{1.957534in}}{\pgfqpoint{3.848061in}{1.968133in}}{\pgfqpoint{3.840248in}{1.975946in}}%
\pgfpathcurveto{\pgfqpoint{3.832434in}{1.983760in}}{\pgfqpoint{3.821835in}{1.988150in}}{\pgfqpoint{3.810785in}{1.988150in}}%
\pgfpathcurveto{\pgfqpoint{3.799735in}{1.988150in}}{\pgfqpoint{3.789136in}{1.983760in}}{\pgfqpoint{3.781322in}{1.975946in}}%
\pgfpathcurveto{\pgfqpoint{3.773509in}{1.968133in}}{\pgfqpoint{3.769118in}{1.957534in}}{\pgfqpoint{3.769118in}{1.946484in}}%
\pgfpathcurveto{\pgfqpoint{3.769118in}{1.935434in}}{\pgfqpoint{3.773509in}{1.924835in}}{\pgfqpoint{3.781322in}{1.917021in}}%
\pgfpathcurveto{\pgfqpoint{3.789136in}{1.909207in}}{\pgfqpoint{3.799735in}{1.904817in}}{\pgfqpoint{3.810785in}{1.904817in}}%
\pgfpathclose%
\pgfusepath{stroke,fill}%
\end{pgfscope}%
\begin{pgfscope}%
\pgfpathrectangle{\pgfqpoint{0.393613in}{0.331635in}}{\pgfqpoint{9.300000in}{7.700000in}}%
\pgfusepath{clip}%
\pgfsetbuttcap%
\pgfsetroundjoin%
\definecolor{currentfill}{rgb}{0.631373,0.788235,0.956863}%
\pgfsetfillcolor{currentfill}%
\pgfsetlinewidth{0.481800pt}%
\definecolor{currentstroke}{rgb}{1.000000,1.000000,1.000000}%
\pgfsetstrokecolor{currentstroke}%
\pgfsetdash{}{0pt}%
\pgfpathmoveto{\pgfqpoint{2.870620in}{1.733520in}}%
\pgfpathcurveto{\pgfqpoint{2.881670in}{1.733520in}}{\pgfqpoint{2.892269in}{1.737911in}}{\pgfqpoint{2.900083in}{1.745724in}}%
\pgfpathcurveto{\pgfqpoint{2.907897in}{1.753538in}}{\pgfqpoint{2.912287in}{1.764137in}}{\pgfqpoint{2.912287in}{1.775187in}}%
\pgfpathcurveto{\pgfqpoint{2.912287in}{1.786237in}}{\pgfqpoint{2.907897in}{1.796836in}}{\pgfqpoint{2.900083in}{1.804650in}}%
\pgfpathcurveto{\pgfqpoint{2.892269in}{1.812463in}}{\pgfqpoint{2.881670in}{1.816854in}}{\pgfqpoint{2.870620in}{1.816854in}}%
\pgfpathcurveto{\pgfqpoint{2.859570in}{1.816854in}}{\pgfqpoint{2.848971in}{1.812463in}}{\pgfqpoint{2.841158in}{1.804650in}}%
\pgfpathcurveto{\pgfqpoint{2.833344in}{1.796836in}}{\pgfqpoint{2.828954in}{1.786237in}}{\pgfqpoint{2.828954in}{1.775187in}}%
\pgfpathcurveto{\pgfqpoint{2.828954in}{1.764137in}}{\pgfqpoint{2.833344in}{1.753538in}}{\pgfqpoint{2.841158in}{1.745724in}}%
\pgfpathcurveto{\pgfqpoint{2.848971in}{1.737911in}}{\pgfqpoint{2.859570in}{1.733520in}}{\pgfqpoint{2.870620in}{1.733520in}}%
\pgfpathclose%
\pgfusepath{stroke,fill}%
\end{pgfscope}%
\begin{pgfscope}%
\pgfpathrectangle{\pgfqpoint{0.393613in}{0.331635in}}{\pgfqpoint{9.300000in}{7.700000in}}%
\pgfusepath{clip}%
\pgfsetbuttcap%
\pgfsetroundjoin%
\definecolor{currentfill}{rgb}{0.631373,0.788235,0.956863}%
\pgfsetfillcolor{currentfill}%
\pgfsetlinewidth{0.481800pt}%
\definecolor{currentstroke}{rgb}{1.000000,1.000000,1.000000}%
\pgfsetstrokecolor{currentstroke}%
\pgfsetdash{}{0pt}%
\pgfpathmoveto{\pgfqpoint{2.247275in}{4.426248in}}%
\pgfpathcurveto{\pgfqpoint{2.258325in}{4.426248in}}{\pgfqpoint{2.268924in}{4.430639in}}{\pgfqpoint{2.276738in}{4.438452in}}%
\pgfpathcurveto{\pgfqpoint{2.284552in}{4.446266in}}{\pgfqpoint{2.288942in}{4.456865in}}{\pgfqpoint{2.288942in}{4.467915in}}%
\pgfpathcurveto{\pgfqpoint{2.288942in}{4.478965in}}{\pgfqpoint{2.284552in}{4.489564in}}{\pgfqpoint{2.276738in}{4.497378in}}%
\pgfpathcurveto{\pgfqpoint{2.268924in}{4.505191in}}{\pgfqpoint{2.258325in}{4.509582in}}{\pgfqpoint{2.247275in}{4.509582in}}%
\pgfpathcurveto{\pgfqpoint{2.236225in}{4.509582in}}{\pgfqpoint{2.225626in}{4.505191in}}{\pgfqpoint{2.217812in}{4.497378in}}%
\pgfpathcurveto{\pgfqpoint{2.209999in}{4.489564in}}{\pgfqpoint{2.205609in}{4.478965in}}{\pgfqpoint{2.205609in}{4.467915in}}%
\pgfpathcurveto{\pgfqpoint{2.205609in}{4.456865in}}{\pgfqpoint{2.209999in}{4.446266in}}{\pgfqpoint{2.217812in}{4.438452in}}%
\pgfpathcurveto{\pgfqpoint{2.225626in}{4.430639in}}{\pgfqpoint{2.236225in}{4.426248in}}{\pgfqpoint{2.247275in}{4.426248in}}%
\pgfpathclose%
\pgfusepath{stroke,fill}%
\end{pgfscope}%
\begin{pgfscope}%
\pgfpathrectangle{\pgfqpoint{0.393613in}{0.331635in}}{\pgfqpoint{9.300000in}{7.700000in}}%
\pgfusepath{clip}%
\pgfsetbuttcap%
\pgfsetroundjoin%
\definecolor{currentfill}{rgb}{0.631373,0.788235,0.956863}%
\pgfsetfillcolor{currentfill}%
\pgfsetlinewidth{0.481800pt}%
\definecolor{currentstroke}{rgb}{1.000000,1.000000,1.000000}%
\pgfsetstrokecolor{currentstroke}%
\pgfsetdash{}{0pt}%
\pgfpathmoveto{\pgfqpoint{3.278707in}{2.466573in}}%
\pgfpathcurveto{\pgfqpoint{3.289757in}{2.466573in}}{\pgfqpoint{3.300356in}{2.470964in}}{\pgfqpoint{3.308169in}{2.478777in}}%
\pgfpathcurveto{\pgfqpoint{3.315983in}{2.486591in}}{\pgfqpoint{3.320373in}{2.497190in}}{\pgfqpoint{3.320373in}{2.508240in}}%
\pgfpathcurveto{\pgfqpoint{3.320373in}{2.519290in}}{\pgfqpoint{3.315983in}{2.529889in}}{\pgfqpoint{3.308169in}{2.537703in}}%
\pgfpathcurveto{\pgfqpoint{3.300356in}{2.545516in}}{\pgfqpoint{3.289757in}{2.549907in}}{\pgfqpoint{3.278707in}{2.549907in}}%
\pgfpathcurveto{\pgfqpoint{3.267656in}{2.549907in}}{\pgfqpoint{3.257057in}{2.545516in}}{\pgfqpoint{3.249244in}{2.537703in}}%
\pgfpathcurveto{\pgfqpoint{3.241430in}{2.529889in}}{\pgfqpoint{3.237040in}{2.519290in}}{\pgfqpoint{3.237040in}{2.508240in}}%
\pgfpathcurveto{\pgfqpoint{3.237040in}{2.497190in}}{\pgfqpoint{3.241430in}{2.486591in}}{\pgfqpoint{3.249244in}{2.478777in}}%
\pgfpathcurveto{\pgfqpoint{3.257057in}{2.470964in}}{\pgfqpoint{3.267656in}{2.466573in}}{\pgfqpoint{3.278707in}{2.466573in}}%
\pgfpathclose%
\pgfusepath{stroke,fill}%
\end{pgfscope}%
\begin{pgfscope}%
\pgfpathrectangle{\pgfqpoint{0.393613in}{0.331635in}}{\pgfqpoint{9.300000in}{7.700000in}}%
\pgfusepath{clip}%
\pgfsetbuttcap%
\pgfsetroundjoin%
\definecolor{currentfill}{rgb}{0.631373,0.788235,0.956863}%
\pgfsetfillcolor{currentfill}%
\pgfsetlinewidth{0.481800pt}%
\definecolor{currentstroke}{rgb}{1.000000,1.000000,1.000000}%
\pgfsetstrokecolor{currentstroke}%
\pgfsetdash{}{0pt}%
\pgfpathmoveto{\pgfqpoint{4.507490in}{0.639968in}}%
\pgfpathcurveto{\pgfqpoint{4.518540in}{0.639968in}}{\pgfqpoint{4.529139in}{0.644359in}}{\pgfqpoint{4.536953in}{0.652172in}}%
\pgfpathcurveto{\pgfqpoint{4.544766in}{0.659986in}}{\pgfqpoint{4.549157in}{0.670585in}}{\pgfqpoint{4.549157in}{0.681635in}}%
\pgfpathcurveto{\pgfqpoint{4.549157in}{0.692685in}}{\pgfqpoint{4.544766in}{0.703284in}}{\pgfqpoint{4.536953in}{0.711098in}}%
\pgfpathcurveto{\pgfqpoint{4.529139in}{0.718911in}}{\pgfqpoint{4.518540in}{0.723302in}}{\pgfqpoint{4.507490in}{0.723302in}}%
\pgfpathcurveto{\pgfqpoint{4.496440in}{0.723302in}}{\pgfqpoint{4.485841in}{0.718911in}}{\pgfqpoint{4.478027in}{0.711098in}}%
\pgfpathcurveto{\pgfqpoint{4.470214in}{0.703284in}}{\pgfqpoint{4.465823in}{0.692685in}}{\pgfqpoint{4.465823in}{0.681635in}}%
\pgfpathcurveto{\pgfqpoint{4.465823in}{0.670585in}}{\pgfqpoint{4.470214in}{0.659986in}}{\pgfqpoint{4.478027in}{0.652172in}}%
\pgfpathcurveto{\pgfqpoint{4.485841in}{0.644359in}}{\pgfqpoint{4.496440in}{0.639968in}}{\pgfqpoint{4.507490in}{0.639968in}}%
\pgfpathclose%
\pgfusepath{stroke,fill}%
\end{pgfscope}%
\begin{pgfscope}%
\pgfpathrectangle{\pgfqpoint{0.393613in}{0.331635in}}{\pgfqpoint{9.300000in}{7.700000in}}%
\pgfusepath{clip}%
\pgfsetbuttcap%
\pgfsetroundjoin%
\definecolor{currentfill}{rgb}{0.631373,0.788235,0.956863}%
\pgfsetfillcolor{currentfill}%
\pgfsetlinewidth{0.481800pt}%
\definecolor{currentstroke}{rgb}{1.000000,1.000000,1.000000}%
\pgfsetstrokecolor{currentstroke}%
\pgfsetdash{}{0pt}%
\pgfpathmoveto{\pgfqpoint{2.850278in}{7.132769in}}%
\pgfpathcurveto{\pgfqpoint{2.861328in}{7.132769in}}{\pgfqpoint{2.871927in}{7.137159in}}{\pgfqpoint{2.879740in}{7.144973in}}%
\pgfpathcurveto{\pgfqpoint{2.887554in}{7.152787in}}{\pgfqpoint{2.891944in}{7.163386in}}{\pgfqpoint{2.891944in}{7.174436in}}%
\pgfpathcurveto{\pgfqpoint{2.891944in}{7.185486in}}{\pgfqpoint{2.887554in}{7.196085in}}{\pgfqpoint{2.879740in}{7.203899in}}%
\pgfpathcurveto{\pgfqpoint{2.871927in}{7.211712in}}{\pgfqpoint{2.861328in}{7.216103in}}{\pgfqpoint{2.850278in}{7.216103in}}%
\pgfpathcurveto{\pgfqpoint{2.839227in}{7.216103in}}{\pgfqpoint{2.828628in}{7.211712in}}{\pgfqpoint{2.820815in}{7.203899in}}%
\pgfpathcurveto{\pgfqpoint{2.813001in}{7.196085in}}{\pgfqpoint{2.808611in}{7.185486in}}{\pgfqpoint{2.808611in}{7.174436in}}%
\pgfpathcurveto{\pgfqpoint{2.808611in}{7.163386in}}{\pgfqpoint{2.813001in}{7.152787in}}{\pgfqpoint{2.820815in}{7.144973in}}%
\pgfpathcurveto{\pgfqpoint{2.828628in}{7.137159in}}{\pgfqpoint{2.839227in}{7.132769in}}{\pgfqpoint{2.850278in}{7.132769in}}%
\pgfpathclose%
\pgfusepath{stroke,fill}%
\end{pgfscope}%
\begin{pgfscope}%
\pgfpathrectangle{\pgfqpoint{0.393613in}{0.331635in}}{\pgfqpoint{9.300000in}{7.700000in}}%
\pgfusepath{clip}%
\pgfsetbuttcap%
\pgfsetroundjoin%
\definecolor{currentfill}{rgb}{0.631373,0.788235,0.956863}%
\pgfsetfillcolor{currentfill}%
\pgfsetlinewidth{0.481800pt}%
\definecolor{currentstroke}{rgb}{1.000000,1.000000,1.000000}%
\pgfsetstrokecolor{currentstroke}%
\pgfsetdash{}{0pt}%
\pgfpathmoveto{\pgfqpoint{1.938431in}{4.996291in}}%
\pgfpathcurveto{\pgfqpoint{1.949481in}{4.996291in}}{\pgfqpoint{1.960080in}{5.000682in}}{\pgfqpoint{1.967894in}{5.008495in}}%
\pgfpathcurveto{\pgfqpoint{1.975708in}{5.016309in}}{\pgfqpoint{1.980098in}{5.026908in}}{\pgfqpoint{1.980098in}{5.037958in}}%
\pgfpathcurveto{\pgfqpoint{1.980098in}{5.049008in}}{\pgfqpoint{1.975708in}{5.059607in}}{\pgfqpoint{1.967894in}{5.067421in}}%
\pgfpathcurveto{\pgfqpoint{1.960080in}{5.075234in}}{\pgfqpoint{1.949481in}{5.079625in}}{\pgfqpoint{1.938431in}{5.079625in}}%
\pgfpathcurveto{\pgfqpoint{1.927381in}{5.079625in}}{\pgfqpoint{1.916782in}{5.075234in}}{\pgfqpoint{1.908968in}{5.067421in}}%
\pgfpathcurveto{\pgfqpoint{1.901155in}{5.059607in}}{\pgfqpoint{1.896764in}{5.049008in}}{\pgfqpoint{1.896764in}{5.037958in}}%
\pgfpathcurveto{\pgfqpoint{1.896764in}{5.026908in}}{\pgfqpoint{1.901155in}{5.016309in}}{\pgfqpoint{1.908968in}{5.008495in}}%
\pgfpathcurveto{\pgfqpoint{1.916782in}{5.000682in}}{\pgfqpoint{1.927381in}{4.996291in}}{\pgfqpoint{1.938431in}{4.996291in}}%
\pgfpathclose%
\pgfusepath{stroke,fill}%
\end{pgfscope}%
\begin{pgfscope}%
\pgfpathrectangle{\pgfqpoint{0.393613in}{0.331635in}}{\pgfqpoint{9.300000in}{7.700000in}}%
\pgfusepath{clip}%
\pgfsetbuttcap%
\pgfsetroundjoin%
\definecolor{currentfill}{rgb}{0.631373,0.788235,0.956863}%
\pgfsetfillcolor{currentfill}%
\pgfsetlinewidth{0.481800pt}%
\definecolor{currentstroke}{rgb}{1.000000,1.000000,1.000000}%
\pgfsetstrokecolor{currentstroke}%
\pgfsetdash{}{0pt}%
\pgfpathmoveto{\pgfqpoint{2.919129in}{0.889402in}}%
\pgfpathcurveto{\pgfqpoint{2.930179in}{0.889402in}}{\pgfqpoint{2.940778in}{0.893792in}}{\pgfqpoint{2.948592in}{0.901606in}}%
\pgfpathcurveto{\pgfqpoint{2.956406in}{0.909420in}}{\pgfqpoint{2.960796in}{0.920019in}}{\pgfqpoint{2.960796in}{0.931069in}}%
\pgfpathcurveto{\pgfqpoint{2.960796in}{0.942119in}}{\pgfqpoint{2.956406in}{0.952718in}}{\pgfqpoint{2.948592in}{0.960532in}}%
\pgfpathcurveto{\pgfqpoint{2.940778in}{0.968345in}}{\pgfqpoint{2.930179in}{0.972735in}}{\pgfqpoint{2.919129in}{0.972735in}}%
\pgfpathcurveto{\pgfqpoint{2.908079in}{0.972735in}}{\pgfqpoint{2.897480in}{0.968345in}}{\pgfqpoint{2.889666in}{0.960532in}}%
\pgfpathcurveto{\pgfqpoint{2.881853in}{0.952718in}}{\pgfqpoint{2.877462in}{0.942119in}}{\pgfqpoint{2.877462in}{0.931069in}}%
\pgfpathcurveto{\pgfqpoint{2.877462in}{0.920019in}}{\pgfqpoint{2.881853in}{0.909420in}}{\pgfqpoint{2.889666in}{0.901606in}}%
\pgfpathcurveto{\pgfqpoint{2.897480in}{0.893792in}}{\pgfqpoint{2.908079in}{0.889402in}}{\pgfqpoint{2.919129in}{0.889402in}}%
\pgfpathclose%
\pgfusepath{stroke,fill}%
\end{pgfscope}%
\begin{pgfscope}%
\pgfpathrectangle{\pgfqpoint{0.393613in}{0.331635in}}{\pgfqpoint{9.300000in}{7.700000in}}%
\pgfusepath{clip}%
\pgfsetbuttcap%
\pgfsetroundjoin%
\definecolor{currentfill}{rgb}{0.631373,0.788235,0.956863}%
\pgfsetfillcolor{currentfill}%
\pgfsetlinewidth{0.481800pt}%
\definecolor{currentstroke}{rgb}{1.000000,1.000000,1.000000}%
\pgfsetstrokecolor{currentstroke}%
\pgfsetdash{}{0pt}%
\pgfpathmoveto{\pgfqpoint{2.553778in}{6.288010in}}%
\pgfpathcurveto{\pgfqpoint{2.564829in}{6.288010in}}{\pgfqpoint{2.575428in}{6.292401in}}{\pgfqpoint{2.583241in}{6.300214in}}%
\pgfpathcurveto{\pgfqpoint{2.591055in}{6.308028in}}{\pgfqpoint{2.595445in}{6.318627in}}{\pgfqpoint{2.595445in}{6.329677in}}%
\pgfpathcurveto{\pgfqpoint{2.595445in}{6.340727in}}{\pgfqpoint{2.591055in}{6.351326in}}{\pgfqpoint{2.583241in}{6.359140in}}%
\pgfpathcurveto{\pgfqpoint{2.575428in}{6.366953in}}{\pgfqpoint{2.564829in}{6.371344in}}{\pgfqpoint{2.553778in}{6.371344in}}%
\pgfpathcurveto{\pgfqpoint{2.542728in}{6.371344in}}{\pgfqpoint{2.532129in}{6.366953in}}{\pgfqpoint{2.524316in}{6.359140in}}%
\pgfpathcurveto{\pgfqpoint{2.516502in}{6.351326in}}{\pgfqpoint{2.512112in}{6.340727in}}{\pgfqpoint{2.512112in}{6.329677in}}%
\pgfpathcurveto{\pgfqpoint{2.512112in}{6.318627in}}{\pgfqpoint{2.516502in}{6.308028in}}{\pgfqpoint{2.524316in}{6.300214in}}%
\pgfpathcurveto{\pgfqpoint{2.532129in}{6.292401in}}{\pgfqpoint{2.542728in}{6.288010in}}{\pgfqpoint{2.553778in}{6.288010in}}%
\pgfpathclose%
\pgfusepath{stroke,fill}%
\end{pgfscope}%
\begin{pgfscope}%
\pgfpathrectangle{\pgfqpoint{0.393613in}{0.331635in}}{\pgfqpoint{9.300000in}{7.700000in}}%
\pgfusepath{clip}%
\pgfsetbuttcap%
\pgfsetroundjoin%
\definecolor{currentfill}{rgb}{0.631373,0.788235,0.956863}%
\pgfsetfillcolor{currentfill}%
\pgfsetlinewidth{0.481800pt}%
\definecolor{currentstroke}{rgb}{1.000000,1.000000,1.000000}%
\pgfsetstrokecolor{currentstroke}%
\pgfsetdash{}{0pt}%
\pgfpathmoveto{\pgfqpoint{2.490513in}{3.733398in}}%
\pgfpathcurveto{\pgfqpoint{2.501563in}{3.733398in}}{\pgfqpoint{2.512162in}{3.737788in}}{\pgfqpoint{2.519975in}{3.745602in}}%
\pgfpathcurveto{\pgfqpoint{2.527789in}{3.753416in}}{\pgfqpoint{2.532179in}{3.764015in}}{\pgfqpoint{2.532179in}{3.775065in}}%
\pgfpathcurveto{\pgfqpoint{2.532179in}{3.786115in}}{\pgfqpoint{2.527789in}{3.796714in}}{\pgfqpoint{2.519975in}{3.804528in}}%
\pgfpathcurveto{\pgfqpoint{2.512162in}{3.812341in}}{\pgfqpoint{2.501563in}{3.816731in}}{\pgfqpoint{2.490513in}{3.816731in}}%
\pgfpathcurveto{\pgfqpoint{2.479463in}{3.816731in}}{\pgfqpoint{2.468863in}{3.812341in}}{\pgfqpoint{2.461050in}{3.804528in}}%
\pgfpathcurveto{\pgfqpoint{2.453236in}{3.796714in}}{\pgfqpoint{2.448846in}{3.786115in}}{\pgfqpoint{2.448846in}{3.775065in}}%
\pgfpathcurveto{\pgfqpoint{2.448846in}{3.764015in}}{\pgfqpoint{2.453236in}{3.753416in}}{\pgfqpoint{2.461050in}{3.745602in}}%
\pgfpathcurveto{\pgfqpoint{2.468863in}{3.737788in}}{\pgfqpoint{2.479463in}{3.733398in}}{\pgfqpoint{2.490513in}{3.733398in}}%
\pgfpathclose%
\pgfusepath{stroke,fill}%
\end{pgfscope}%
\begin{pgfscope}%
\pgfpathrectangle{\pgfqpoint{0.393613in}{0.331635in}}{\pgfqpoint{9.300000in}{7.700000in}}%
\pgfusepath{clip}%
\pgfsetbuttcap%
\pgfsetroundjoin%
\definecolor{currentfill}{rgb}{0.631373,0.788235,0.956863}%
\pgfsetfillcolor{currentfill}%
\pgfsetlinewidth{0.481800pt}%
\definecolor{currentstroke}{rgb}{1.000000,1.000000,1.000000}%
\pgfsetstrokecolor{currentstroke}%
\pgfsetdash{}{0pt}%
\pgfpathmoveto{\pgfqpoint{0.816340in}{4.715929in}}%
\pgfpathcurveto{\pgfqpoint{0.827390in}{4.715929in}}{\pgfqpoint{0.837989in}{4.720319in}}{\pgfqpoint{0.845803in}{4.728133in}}%
\pgfpathcurveto{\pgfqpoint{0.853616in}{4.735946in}}{\pgfqpoint{0.858006in}{4.746545in}}{\pgfqpoint{0.858006in}{4.757595in}}%
\pgfpathcurveto{\pgfqpoint{0.858006in}{4.768646in}}{\pgfqpoint{0.853616in}{4.779245in}}{\pgfqpoint{0.845803in}{4.787058in}}%
\pgfpathcurveto{\pgfqpoint{0.837989in}{4.794872in}}{\pgfqpoint{0.827390in}{4.799262in}}{\pgfqpoint{0.816340in}{4.799262in}}%
\pgfpathcurveto{\pgfqpoint{0.805290in}{4.799262in}}{\pgfqpoint{0.794691in}{4.794872in}}{\pgfqpoint{0.786877in}{4.787058in}}%
\pgfpathcurveto{\pgfqpoint{0.779063in}{4.779245in}}{\pgfqpoint{0.774673in}{4.768646in}}{\pgfqpoint{0.774673in}{4.757595in}}%
\pgfpathcurveto{\pgfqpoint{0.774673in}{4.746545in}}{\pgfqpoint{0.779063in}{4.735946in}}{\pgfqpoint{0.786877in}{4.728133in}}%
\pgfpathcurveto{\pgfqpoint{0.794691in}{4.720319in}}{\pgfqpoint{0.805290in}{4.715929in}}{\pgfqpoint{0.816340in}{4.715929in}}%
\pgfpathclose%
\pgfusepath{stroke,fill}%
\end{pgfscope}%
\begin{pgfscope}%
\pgfpathrectangle{\pgfqpoint{0.393613in}{0.331635in}}{\pgfqpoint{9.300000in}{7.700000in}}%
\pgfusepath{clip}%
\pgfsetbuttcap%
\pgfsetroundjoin%
\definecolor{currentfill}{rgb}{0.631373,0.788235,0.956863}%
\pgfsetfillcolor{currentfill}%
\pgfsetlinewidth{0.481800pt}%
\definecolor{currentstroke}{rgb}{1.000000,1.000000,1.000000}%
\pgfsetstrokecolor{currentstroke}%
\pgfsetdash{}{0pt}%
\pgfpathmoveto{\pgfqpoint{1.972243in}{3.973088in}}%
\pgfpathcurveto{\pgfqpoint{1.983293in}{3.973088in}}{\pgfqpoint{1.993892in}{3.977478in}}{\pgfqpoint{2.001706in}{3.985292in}}%
\pgfpathcurveto{\pgfqpoint{2.009519in}{3.993105in}}{\pgfqpoint{2.013910in}{4.003704in}}{\pgfqpoint{2.013910in}{4.014754in}}%
\pgfpathcurveto{\pgfqpoint{2.013910in}{4.025805in}}{\pgfqpoint{2.009519in}{4.036404in}}{\pgfqpoint{2.001706in}{4.044217in}}%
\pgfpathcurveto{\pgfqpoint{1.993892in}{4.052031in}}{\pgfqpoint{1.983293in}{4.056421in}}{\pgfqpoint{1.972243in}{4.056421in}}%
\pgfpathcurveto{\pgfqpoint{1.961193in}{4.056421in}}{\pgfqpoint{1.950594in}{4.052031in}}{\pgfqpoint{1.942780in}{4.044217in}}%
\pgfpathcurveto{\pgfqpoint{1.934967in}{4.036404in}}{\pgfqpoint{1.930576in}{4.025805in}}{\pgfqpoint{1.930576in}{4.014754in}}%
\pgfpathcurveto{\pgfqpoint{1.930576in}{4.003704in}}{\pgfqpoint{1.934967in}{3.993105in}}{\pgfqpoint{1.942780in}{3.985292in}}%
\pgfpathcurveto{\pgfqpoint{1.950594in}{3.977478in}}{\pgfqpoint{1.961193in}{3.973088in}}{\pgfqpoint{1.972243in}{3.973088in}}%
\pgfpathclose%
\pgfusepath{stroke,fill}%
\end{pgfscope}%
\begin{pgfscope}%
\pgfpathrectangle{\pgfqpoint{0.393613in}{0.331635in}}{\pgfqpoint{9.300000in}{7.700000in}}%
\pgfusepath{clip}%
\pgfsetbuttcap%
\pgfsetroundjoin%
\definecolor{currentfill}{rgb}{0.631373,0.788235,0.956863}%
\pgfsetfillcolor{currentfill}%
\pgfsetlinewidth{0.481800pt}%
\definecolor{currentstroke}{rgb}{1.000000,1.000000,1.000000}%
\pgfsetstrokecolor{currentstroke}%
\pgfsetdash{}{0pt}%
\pgfpathmoveto{\pgfqpoint{3.271288in}{1.418894in}}%
\pgfpathcurveto{\pgfqpoint{3.282338in}{1.418894in}}{\pgfqpoint{3.292937in}{1.423284in}}{\pgfqpoint{3.300751in}{1.431098in}}%
\pgfpathcurveto{\pgfqpoint{3.308564in}{1.438911in}}{\pgfqpoint{3.312955in}{1.449510in}}{\pgfqpoint{3.312955in}{1.460560in}}%
\pgfpathcurveto{\pgfqpoint{3.312955in}{1.471611in}}{\pgfqpoint{3.308564in}{1.482210in}}{\pgfqpoint{3.300751in}{1.490023in}}%
\pgfpathcurveto{\pgfqpoint{3.292937in}{1.497837in}}{\pgfqpoint{3.282338in}{1.502227in}}{\pgfqpoint{3.271288in}{1.502227in}}%
\pgfpathcurveto{\pgfqpoint{3.260238in}{1.502227in}}{\pgfqpoint{3.249639in}{1.497837in}}{\pgfqpoint{3.241825in}{1.490023in}}%
\pgfpathcurveto{\pgfqpoint{3.234011in}{1.482210in}}{\pgfqpoint{3.229621in}{1.471611in}}{\pgfqpoint{3.229621in}{1.460560in}}%
\pgfpathcurveto{\pgfqpoint{3.229621in}{1.449510in}}{\pgfqpoint{3.234011in}{1.438911in}}{\pgfqpoint{3.241825in}{1.431098in}}%
\pgfpathcurveto{\pgfqpoint{3.249639in}{1.423284in}}{\pgfqpoint{3.260238in}{1.418894in}}{\pgfqpoint{3.271288in}{1.418894in}}%
\pgfpathclose%
\pgfusepath{stroke,fill}%
\end{pgfscope}%
\begin{pgfscope}%
\pgfpathrectangle{\pgfqpoint{0.393613in}{0.331635in}}{\pgfqpoint{9.300000in}{7.700000in}}%
\pgfusepath{clip}%
\pgfsetbuttcap%
\pgfsetroundjoin%
\definecolor{currentfill}{rgb}{0.631373,0.788235,0.956863}%
\pgfsetfillcolor{currentfill}%
\pgfsetlinewidth{0.481800pt}%
\definecolor{currentstroke}{rgb}{1.000000,1.000000,1.000000}%
\pgfsetstrokecolor{currentstroke}%
\pgfsetdash{}{0pt}%
\pgfpathmoveto{\pgfqpoint{2.443533in}{5.210649in}}%
\pgfpathcurveto{\pgfqpoint{2.454583in}{5.210649in}}{\pgfqpoint{2.465182in}{5.215039in}}{\pgfqpoint{2.472996in}{5.222853in}}%
\pgfpathcurveto{\pgfqpoint{2.480809in}{5.230666in}}{\pgfqpoint{2.485200in}{5.241265in}}{\pgfqpoint{2.485200in}{5.252316in}}%
\pgfpathcurveto{\pgfqpoint{2.485200in}{5.263366in}}{\pgfqpoint{2.480809in}{5.273965in}}{\pgfqpoint{2.472996in}{5.281778in}}%
\pgfpathcurveto{\pgfqpoint{2.465182in}{5.289592in}}{\pgfqpoint{2.454583in}{5.293982in}}{\pgfqpoint{2.443533in}{5.293982in}}%
\pgfpathcurveto{\pgfqpoint{2.432483in}{5.293982in}}{\pgfqpoint{2.421884in}{5.289592in}}{\pgfqpoint{2.414070in}{5.281778in}}%
\pgfpathcurveto{\pgfqpoint{2.406257in}{5.273965in}}{\pgfqpoint{2.401866in}{5.263366in}}{\pgfqpoint{2.401866in}{5.252316in}}%
\pgfpathcurveto{\pgfqpoint{2.401866in}{5.241265in}}{\pgfqpoint{2.406257in}{5.230666in}}{\pgfqpoint{2.414070in}{5.222853in}}%
\pgfpathcurveto{\pgfqpoint{2.421884in}{5.215039in}}{\pgfqpoint{2.432483in}{5.210649in}}{\pgfqpoint{2.443533in}{5.210649in}}%
\pgfpathclose%
\pgfusepath{stroke,fill}%
\end{pgfscope}%
\begin{pgfscope}%
\pgfpathrectangle{\pgfqpoint{0.393613in}{0.331635in}}{\pgfqpoint{9.300000in}{7.700000in}}%
\pgfusepath{clip}%
\pgfsetbuttcap%
\pgfsetroundjoin%
\definecolor{currentfill}{rgb}{0.631373,0.788235,0.956863}%
\pgfsetfillcolor{currentfill}%
\pgfsetlinewidth{0.481800pt}%
\definecolor{currentstroke}{rgb}{1.000000,1.000000,1.000000}%
\pgfsetstrokecolor{currentstroke}%
\pgfsetdash{}{0pt}%
\pgfpathmoveto{\pgfqpoint{3.808024in}{7.639968in}}%
\pgfpathcurveto{\pgfqpoint{3.819074in}{7.639968in}}{\pgfqpoint{3.829673in}{7.644359in}}{\pgfqpoint{3.837487in}{7.652172in}}%
\pgfpathcurveto{\pgfqpoint{3.845300in}{7.659986in}}{\pgfqpoint{3.849690in}{7.670585in}}{\pgfqpoint{3.849690in}{7.681635in}}%
\pgfpathcurveto{\pgfqpoint{3.849690in}{7.692685in}}{\pgfqpoint{3.845300in}{7.703284in}}{\pgfqpoint{3.837487in}{7.711098in}}%
\pgfpathcurveto{\pgfqpoint{3.829673in}{7.718911in}}{\pgfqpoint{3.819074in}{7.723302in}}{\pgfqpoint{3.808024in}{7.723302in}}%
\pgfpathcurveto{\pgfqpoint{3.796974in}{7.723302in}}{\pgfqpoint{3.786375in}{7.718911in}}{\pgfqpoint{3.778561in}{7.711098in}}%
\pgfpathcurveto{\pgfqpoint{3.770747in}{7.703284in}}{\pgfqpoint{3.766357in}{7.692685in}}{\pgfqpoint{3.766357in}{7.681635in}}%
\pgfpathcurveto{\pgfqpoint{3.766357in}{7.670585in}}{\pgfqpoint{3.770747in}{7.659986in}}{\pgfqpoint{3.778561in}{7.652172in}}%
\pgfpathcurveto{\pgfqpoint{3.786375in}{7.644359in}}{\pgfqpoint{3.796974in}{7.639968in}}{\pgfqpoint{3.808024in}{7.639968in}}%
\pgfpathclose%
\pgfusepath{stroke,fill}%
\end{pgfscope}%
\begin{pgfscope}%
\pgfpathrectangle{\pgfqpoint{0.393613in}{0.331635in}}{\pgfqpoint{9.300000in}{7.700000in}}%
\pgfusepath{clip}%
\pgfsetbuttcap%
\pgfsetroundjoin%
\definecolor{currentfill}{rgb}{0.631373,0.788235,0.956863}%
\pgfsetfillcolor{currentfill}%
\pgfsetlinewidth{0.481800pt}%
\definecolor{currentstroke}{rgb}{1.000000,1.000000,1.000000}%
\pgfsetstrokecolor{currentstroke}%
\pgfsetdash{}{0pt}%
\pgfpathmoveto{\pgfqpoint{1.545120in}{6.799076in}}%
\pgfpathcurveto{\pgfqpoint{1.556170in}{6.799076in}}{\pgfqpoint{1.566769in}{6.803467in}}{\pgfqpoint{1.574582in}{6.811280in}}%
\pgfpathcurveto{\pgfqpoint{1.582396in}{6.819094in}}{\pgfqpoint{1.586786in}{6.829693in}}{\pgfqpoint{1.586786in}{6.840743in}}%
\pgfpathcurveto{\pgfqpoint{1.586786in}{6.851793in}}{\pgfqpoint{1.582396in}{6.862392in}}{\pgfqpoint{1.574582in}{6.870206in}}%
\pgfpathcurveto{\pgfqpoint{1.566769in}{6.878020in}}{\pgfqpoint{1.556170in}{6.882410in}}{\pgfqpoint{1.545120in}{6.882410in}}%
\pgfpathcurveto{\pgfqpoint{1.534069in}{6.882410in}}{\pgfqpoint{1.523470in}{6.878020in}}{\pgfqpoint{1.515657in}{6.870206in}}%
\pgfpathcurveto{\pgfqpoint{1.507843in}{6.862392in}}{\pgfqpoint{1.503453in}{6.851793in}}{\pgfqpoint{1.503453in}{6.840743in}}%
\pgfpathcurveto{\pgfqpoint{1.503453in}{6.829693in}}{\pgfqpoint{1.507843in}{6.819094in}}{\pgfqpoint{1.515657in}{6.811280in}}%
\pgfpathcurveto{\pgfqpoint{1.523470in}{6.803467in}}{\pgfqpoint{1.534069in}{6.799076in}}{\pgfqpoint{1.545120in}{6.799076in}}%
\pgfpathclose%
\pgfusepath{stroke,fill}%
\end{pgfscope}%
\begin{pgfscope}%
\pgfpathrectangle{\pgfqpoint{0.393613in}{0.331635in}}{\pgfqpoint{9.300000in}{7.700000in}}%
\pgfusepath{clip}%
\pgfsetbuttcap%
\pgfsetroundjoin%
\definecolor{currentfill}{rgb}{0.631373,0.788235,0.956863}%
\pgfsetfillcolor{currentfill}%
\pgfsetlinewidth{0.481800pt}%
\definecolor{currentstroke}{rgb}{1.000000,1.000000,1.000000}%
\pgfsetstrokecolor{currentstroke}%
\pgfsetdash{}{0pt}%
\pgfpathmoveto{\pgfqpoint{4.495356in}{6.292143in}}%
\pgfpathcurveto{\pgfqpoint{4.506406in}{6.292143in}}{\pgfqpoint{4.517005in}{6.296534in}}{\pgfqpoint{4.524819in}{6.304347in}}%
\pgfpathcurveto{\pgfqpoint{4.532632in}{6.312161in}}{\pgfqpoint{4.537023in}{6.322760in}}{\pgfqpoint{4.537023in}{6.333810in}}%
\pgfpathcurveto{\pgfqpoint{4.537023in}{6.344860in}}{\pgfqpoint{4.532632in}{6.355459in}}{\pgfqpoint{4.524819in}{6.363273in}}%
\pgfpathcurveto{\pgfqpoint{4.517005in}{6.371087in}}{\pgfqpoint{4.506406in}{6.375477in}}{\pgfqpoint{4.495356in}{6.375477in}}%
\pgfpathcurveto{\pgfqpoint{4.484306in}{6.375477in}}{\pgfqpoint{4.473707in}{6.371087in}}{\pgfqpoint{4.465893in}{6.363273in}}%
\pgfpathcurveto{\pgfqpoint{4.458080in}{6.355459in}}{\pgfqpoint{4.453689in}{6.344860in}}{\pgfqpoint{4.453689in}{6.333810in}}%
\pgfpathcurveto{\pgfqpoint{4.453689in}{6.322760in}}{\pgfqpoint{4.458080in}{6.312161in}}{\pgfqpoint{4.465893in}{6.304347in}}%
\pgfpathcurveto{\pgfqpoint{4.473707in}{6.296534in}}{\pgfqpoint{4.484306in}{6.292143in}}{\pgfqpoint{4.495356in}{6.292143in}}%
\pgfpathclose%
\pgfusepath{stroke,fill}%
\end{pgfscope}%
\begin{pgfscope}%
\pgfpathrectangle{\pgfqpoint{0.393613in}{0.331635in}}{\pgfqpoint{9.300000in}{7.700000in}}%
\pgfusepath{clip}%
\pgfsetbuttcap%
\pgfsetroundjoin%
\definecolor{currentfill}{rgb}{0.631373,0.788235,0.956863}%
\pgfsetfillcolor{currentfill}%
\pgfsetlinewidth{0.481800pt}%
\definecolor{currentstroke}{rgb}{1.000000,1.000000,1.000000}%
\pgfsetstrokecolor{currentstroke}%
\pgfsetdash{}{0pt}%
\pgfpathmoveto{\pgfqpoint{2.111783in}{3.157576in}}%
\pgfpathcurveto{\pgfqpoint{2.122833in}{3.157576in}}{\pgfqpoint{2.133432in}{3.161966in}}{\pgfqpoint{2.141245in}{3.169780in}}%
\pgfpathcurveto{\pgfqpoint{2.149059in}{3.177593in}}{\pgfqpoint{2.153449in}{3.188192in}}{\pgfqpoint{2.153449in}{3.199242in}}%
\pgfpathcurveto{\pgfqpoint{2.153449in}{3.210292in}}{\pgfqpoint{2.149059in}{3.220891in}}{\pgfqpoint{2.141245in}{3.228705in}}%
\pgfpathcurveto{\pgfqpoint{2.133432in}{3.236519in}}{\pgfqpoint{2.122833in}{3.240909in}}{\pgfqpoint{2.111783in}{3.240909in}}%
\pgfpathcurveto{\pgfqpoint{2.100732in}{3.240909in}}{\pgfqpoint{2.090133in}{3.236519in}}{\pgfqpoint{2.082320in}{3.228705in}}%
\pgfpathcurveto{\pgfqpoint{2.074506in}{3.220891in}}{\pgfqpoint{2.070116in}{3.210292in}}{\pgfqpoint{2.070116in}{3.199242in}}%
\pgfpathcurveto{\pgfqpoint{2.070116in}{3.188192in}}{\pgfqpoint{2.074506in}{3.177593in}}{\pgfqpoint{2.082320in}{3.169780in}}%
\pgfpathcurveto{\pgfqpoint{2.090133in}{3.161966in}}{\pgfqpoint{2.100732in}{3.157576in}}{\pgfqpoint{2.111783in}{3.157576in}}%
\pgfpathclose%
\pgfusepath{stroke,fill}%
\end{pgfscope}%
\begin{pgfscope}%
\pgfpathrectangle{\pgfqpoint{0.393613in}{0.331635in}}{\pgfqpoint{9.300000in}{7.700000in}}%
\pgfusepath{clip}%
\pgfsetbuttcap%
\pgfsetroundjoin%
\definecolor{currentfill}{rgb}{0.631373,0.788235,0.956863}%
\pgfsetfillcolor{currentfill}%
\pgfsetlinewidth{0.481800pt}%
\definecolor{currentstroke}{rgb}{1.000000,1.000000,1.000000}%
\pgfsetstrokecolor{currentstroke}%
\pgfsetdash{}{0pt}%
\pgfpathmoveto{\pgfqpoint{2.968097in}{5.645878in}}%
\pgfpathcurveto{\pgfqpoint{2.979147in}{5.645878in}}{\pgfqpoint{2.989746in}{5.650268in}}{\pgfqpoint{2.997560in}{5.658082in}}%
\pgfpathcurveto{\pgfqpoint{3.005373in}{5.665896in}}{\pgfqpoint{3.009764in}{5.676495in}}{\pgfqpoint{3.009764in}{5.687545in}}%
\pgfpathcurveto{\pgfqpoint{3.009764in}{5.698595in}}{\pgfqpoint{3.005373in}{5.709194in}}{\pgfqpoint{2.997560in}{5.717008in}}%
\pgfpathcurveto{\pgfqpoint{2.989746in}{5.724821in}}{\pgfqpoint{2.979147in}{5.729212in}}{\pgfqpoint{2.968097in}{5.729212in}}%
\pgfpathcurveto{\pgfqpoint{2.957047in}{5.729212in}}{\pgfqpoint{2.946448in}{5.724821in}}{\pgfqpoint{2.938634in}{5.717008in}}%
\pgfpathcurveto{\pgfqpoint{2.930821in}{5.709194in}}{\pgfqpoint{2.926430in}{5.698595in}}{\pgfqpoint{2.926430in}{5.687545in}}%
\pgfpathcurveto{\pgfqpoint{2.926430in}{5.676495in}}{\pgfqpoint{2.930821in}{5.665896in}}{\pgfqpoint{2.938634in}{5.658082in}}%
\pgfpathcurveto{\pgfqpoint{2.946448in}{5.650268in}}{\pgfqpoint{2.957047in}{5.645878in}}{\pgfqpoint{2.968097in}{5.645878in}}%
\pgfpathclose%
\pgfusepath{stroke,fill}%
\end{pgfscope}%
\begin{pgfscope}%
\pgfpathrectangle{\pgfqpoint{0.393613in}{0.331635in}}{\pgfqpoint{9.300000in}{7.700000in}}%
\pgfusepath{clip}%
\pgfsetbuttcap%
\pgfsetroundjoin%
\definecolor{currentfill}{rgb}{0.631373,0.788235,0.956863}%
\pgfsetfillcolor{currentfill}%
\pgfsetlinewidth{0.481800pt}%
\definecolor{currentstroke}{rgb}{1.000000,1.000000,1.000000}%
\pgfsetstrokecolor{currentstroke}%
\pgfsetdash{}{0pt}%
\pgfpathmoveto{\pgfqpoint{2.017579in}{2.370733in}}%
\pgfpathcurveto{\pgfqpoint{2.028629in}{2.370733in}}{\pgfqpoint{2.039228in}{2.375123in}}{\pgfqpoint{2.047042in}{2.382937in}}%
\pgfpathcurveto{\pgfqpoint{2.054856in}{2.390750in}}{\pgfqpoint{2.059246in}{2.401350in}}{\pgfqpoint{2.059246in}{2.412400in}}%
\pgfpathcurveto{\pgfqpoint{2.059246in}{2.423450in}}{\pgfqpoint{2.054856in}{2.434049in}}{\pgfqpoint{2.047042in}{2.441862in}}%
\pgfpathcurveto{\pgfqpoint{2.039228in}{2.449676in}}{\pgfqpoint{2.028629in}{2.454066in}}{\pgfqpoint{2.017579in}{2.454066in}}%
\pgfpathcurveto{\pgfqpoint{2.006529in}{2.454066in}}{\pgfqpoint{1.995930in}{2.449676in}}{\pgfqpoint{1.988116in}{2.441862in}}%
\pgfpathcurveto{\pgfqpoint{1.980303in}{2.434049in}}{\pgfqpoint{1.975912in}{2.423450in}}{\pgfqpoint{1.975912in}{2.412400in}}%
\pgfpathcurveto{\pgfqpoint{1.975912in}{2.401350in}}{\pgfqpoint{1.980303in}{2.390750in}}{\pgfqpoint{1.988116in}{2.382937in}}%
\pgfpathcurveto{\pgfqpoint{1.995930in}{2.375123in}}{\pgfqpoint{2.006529in}{2.370733in}}{\pgfqpoint{2.017579in}{2.370733in}}%
\pgfpathclose%
\pgfusepath{stroke,fill}%
\end{pgfscope}%
\begin{pgfscope}%
\pgfpathrectangle{\pgfqpoint{0.393613in}{0.331635in}}{\pgfqpoint{9.300000in}{7.700000in}}%
\pgfusepath{clip}%
\pgfsetbuttcap%
\pgfsetroundjoin%
\definecolor{currentfill}{rgb}{0.631373,0.788235,0.956863}%
\pgfsetfillcolor{currentfill}%
\pgfsetlinewidth{0.481800pt}%
\definecolor{currentstroke}{rgb}{1.000000,1.000000,1.000000}%
\pgfsetstrokecolor{currentstroke}%
\pgfsetdash{}{0pt}%
\pgfpathmoveto{\pgfqpoint{4.875280in}{6.831971in}}%
\pgfpathcurveto{\pgfqpoint{4.886330in}{6.831971in}}{\pgfqpoint{4.896929in}{6.836361in}}{\pgfqpoint{4.904743in}{6.844175in}}%
\pgfpathcurveto{\pgfqpoint{4.912556in}{6.851989in}}{\pgfqpoint{4.916947in}{6.862588in}}{\pgfqpoint{4.916947in}{6.873638in}}%
\pgfpathcurveto{\pgfqpoint{4.916947in}{6.884688in}}{\pgfqpoint{4.912556in}{6.895287in}}{\pgfqpoint{4.904743in}{6.903100in}}%
\pgfpathcurveto{\pgfqpoint{4.896929in}{6.910914in}}{\pgfqpoint{4.886330in}{6.915304in}}{\pgfqpoint{4.875280in}{6.915304in}}%
\pgfpathcurveto{\pgfqpoint{4.864230in}{6.915304in}}{\pgfqpoint{4.853631in}{6.910914in}}{\pgfqpoint{4.845817in}{6.903100in}}%
\pgfpathcurveto{\pgfqpoint{4.838004in}{6.895287in}}{\pgfqpoint{4.833613in}{6.884688in}}{\pgfqpoint{4.833613in}{6.873638in}}%
\pgfpathcurveto{\pgfqpoint{4.833613in}{6.862588in}}{\pgfqpoint{4.838004in}{6.851989in}}{\pgfqpoint{4.845817in}{6.844175in}}%
\pgfpathcurveto{\pgfqpoint{4.853631in}{6.836361in}}{\pgfqpoint{4.864230in}{6.831971in}}{\pgfqpoint{4.875280in}{6.831971in}}%
\pgfpathclose%
\pgfusepath{stroke,fill}%
\end{pgfscope}%
\begin{pgfscope}%
\pgfpathrectangle{\pgfqpoint{0.393613in}{0.331635in}}{\pgfqpoint{9.300000in}{7.700000in}}%
\pgfusepath{clip}%
\pgfsetbuttcap%
\pgfsetroundjoin%
\definecolor{currentfill}{rgb}{0.631373,0.788235,0.956863}%
\pgfsetfillcolor{currentfill}%
\pgfsetlinewidth{0.481800pt}%
\definecolor{currentstroke}{rgb}{1.000000,1.000000,1.000000}%
\pgfsetstrokecolor{currentstroke}%
\pgfsetdash{}{0pt}%
\pgfpathmoveto{\pgfqpoint{4.821671in}{6.252549in}}%
\pgfpathcurveto{\pgfqpoint{4.832721in}{6.252549in}}{\pgfqpoint{4.843320in}{6.256939in}}{\pgfqpoint{4.851134in}{6.264753in}}%
\pgfpathcurveto{\pgfqpoint{4.858947in}{6.272566in}}{\pgfqpoint{4.863338in}{6.283165in}}{\pgfqpoint{4.863338in}{6.294216in}}%
\pgfpathcurveto{\pgfqpoint{4.863338in}{6.305266in}}{\pgfqpoint{4.858947in}{6.315865in}}{\pgfqpoint{4.851134in}{6.323678in}}%
\pgfpathcurveto{\pgfqpoint{4.843320in}{6.331492in}}{\pgfqpoint{4.832721in}{6.335882in}}{\pgfqpoint{4.821671in}{6.335882in}}%
\pgfpathcurveto{\pgfqpoint{4.810621in}{6.335882in}}{\pgfqpoint{4.800022in}{6.331492in}}{\pgfqpoint{4.792208in}{6.323678in}}%
\pgfpathcurveto{\pgfqpoint{4.784395in}{6.315865in}}{\pgfqpoint{4.780004in}{6.305266in}}{\pgfqpoint{4.780004in}{6.294216in}}%
\pgfpathcurveto{\pgfqpoint{4.780004in}{6.283165in}}{\pgfqpoint{4.784395in}{6.272566in}}{\pgfqpoint{4.792208in}{6.264753in}}%
\pgfpathcurveto{\pgfqpoint{4.800022in}{6.256939in}}{\pgfqpoint{4.810621in}{6.252549in}}{\pgfqpoint{4.821671in}{6.252549in}}%
\pgfpathclose%
\pgfusepath{stroke,fill}%
\end{pgfscope}%
\begin{pgfscope}%
\pgfpathrectangle{\pgfqpoint{0.393613in}{0.331635in}}{\pgfqpoint{9.300000in}{7.700000in}}%
\pgfusepath{clip}%
\pgfsetbuttcap%
\pgfsetroundjoin%
\definecolor{currentfill}{rgb}{0.631373,0.788235,0.956863}%
\pgfsetfillcolor{currentfill}%
\pgfsetlinewidth{0.481800pt}%
\definecolor{currentstroke}{rgb}{1.000000,1.000000,1.000000}%
\pgfsetstrokecolor{currentstroke}%
\pgfsetdash{}{0pt}%
\pgfpathmoveto{\pgfqpoint{2.038680in}{2.394137in}}%
\pgfpathcurveto{\pgfqpoint{2.049730in}{2.394137in}}{\pgfqpoint{2.060329in}{2.398527in}}{\pgfqpoint{2.068142in}{2.406340in}}%
\pgfpathcurveto{\pgfqpoint{2.075956in}{2.414154in}}{\pgfqpoint{2.080346in}{2.424753in}}{\pgfqpoint{2.080346in}{2.435803in}}%
\pgfpathcurveto{\pgfqpoint{2.080346in}{2.446853in}}{\pgfqpoint{2.075956in}{2.457452in}}{\pgfqpoint{2.068142in}{2.465266in}}%
\pgfpathcurveto{\pgfqpoint{2.060329in}{2.473080in}}{\pgfqpoint{2.049730in}{2.477470in}}{\pgfqpoint{2.038680in}{2.477470in}}%
\pgfpathcurveto{\pgfqpoint{2.027629in}{2.477470in}}{\pgfqpoint{2.017030in}{2.473080in}}{\pgfqpoint{2.009217in}{2.465266in}}%
\pgfpathcurveto{\pgfqpoint{2.001403in}{2.457452in}}{\pgfqpoint{1.997013in}{2.446853in}}{\pgfqpoint{1.997013in}{2.435803in}}%
\pgfpathcurveto{\pgfqpoint{1.997013in}{2.424753in}}{\pgfqpoint{2.001403in}{2.414154in}}{\pgfqpoint{2.009217in}{2.406340in}}%
\pgfpathcurveto{\pgfqpoint{2.017030in}{2.398527in}}{\pgfqpoint{2.027629in}{2.394137in}}{\pgfqpoint{2.038680in}{2.394137in}}%
\pgfpathclose%
\pgfusepath{stroke,fill}%
\end{pgfscope}%
\begin{pgfscope}%
\pgfpathrectangle{\pgfqpoint{0.393613in}{0.331635in}}{\pgfqpoint{9.300000in}{7.700000in}}%
\pgfusepath{clip}%
\pgfsetbuttcap%
\pgfsetroundjoin%
\definecolor{currentfill}{rgb}{0.631373,0.788235,0.956863}%
\pgfsetfillcolor{currentfill}%
\pgfsetlinewidth{0.481800pt}%
\definecolor{currentstroke}{rgb}{1.000000,1.000000,1.000000}%
\pgfsetstrokecolor{currentstroke}%
\pgfsetdash{}{0pt}%
\pgfpathmoveto{\pgfqpoint{3.158987in}{6.265025in}}%
\pgfpathcurveto{\pgfqpoint{3.170037in}{6.265025in}}{\pgfqpoint{3.180636in}{6.269415in}}{\pgfqpoint{3.188449in}{6.277229in}}%
\pgfpathcurveto{\pgfqpoint{3.196263in}{6.285043in}}{\pgfqpoint{3.200653in}{6.295642in}}{\pgfqpoint{3.200653in}{6.306692in}}%
\pgfpathcurveto{\pgfqpoint{3.200653in}{6.317742in}}{\pgfqpoint{3.196263in}{6.328341in}}{\pgfqpoint{3.188449in}{6.336155in}}%
\pgfpathcurveto{\pgfqpoint{3.180636in}{6.343968in}}{\pgfqpoint{3.170037in}{6.348358in}}{\pgfqpoint{3.158987in}{6.348358in}}%
\pgfpathcurveto{\pgfqpoint{3.147936in}{6.348358in}}{\pgfqpoint{3.137337in}{6.343968in}}{\pgfqpoint{3.129524in}{6.336155in}}%
\pgfpathcurveto{\pgfqpoint{3.121710in}{6.328341in}}{\pgfqpoint{3.117320in}{6.317742in}}{\pgfqpoint{3.117320in}{6.306692in}}%
\pgfpathcurveto{\pgfqpoint{3.117320in}{6.295642in}}{\pgfqpoint{3.121710in}{6.285043in}}{\pgfqpoint{3.129524in}{6.277229in}}%
\pgfpathcurveto{\pgfqpoint{3.137337in}{6.269415in}}{\pgfqpoint{3.147936in}{6.265025in}}{\pgfqpoint{3.158987in}{6.265025in}}%
\pgfpathclose%
\pgfusepath{stroke,fill}%
\end{pgfscope}%
\begin{pgfscope}%
\pgfpathrectangle{\pgfqpoint{0.393613in}{0.331635in}}{\pgfqpoint{9.300000in}{7.700000in}}%
\pgfusepath{clip}%
\pgfsetbuttcap%
\pgfsetroundjoin%
\definecolor{currentfill}{rgb}{0.631373,0.788235,0.956863}%
\pgfsetfillcolor{currentfill}%
\pgfsetlinewidth{0.481800pt}%
\definecolor{currentstroke}{rgb}{1.000000,1.000000,1.000000}%
\pgfsetstrokecolor{currentstroke}%
\pgfsetdash{}{0pt}%
\pgfpathmoveto{\pgfqpoint{3.163926in}{3.153448in}}%
\pgfpathcurveto{\pgfqpoint{3.174976in}{3.153448in}}{\pgfqpoint{3.185575in}{3.157838in}}{\pgfqpoint{3.193389in}{3.165652in}}%
\pgfpathcurveto{\pgfqpoint{3.201203in}{3.173465in}}{\pgfqpoint{3.205593in}{3.184064in}}{\pgfqpoint{3.205593in}{3.195114in}}%
\pgfpathcurveto{\pgfqpoint{3.205593in}{3.206165in}}{\pgfqpoint{3.201203in}{3.216764in}}{\pgfqpoint{3.193389in}{3.224577in}}%
\pgfpathcurveto{\pgfqpoint{3.185575in}{3.232391in}}{\pgfqpoint{3.174976in}{3.236781in}}{\pgfqpoint{3.163926in}{3.236781in}}%
\pgfpathcurveto{\pgfqpoint{3.152876in}{3.236781in}}{\pgfqpoint{3.142277in}{3.232391in}}{\pgfqpoint{3.134463in}{3.224577in}}%
\pgfpathcurveto{\pgfqpoint{3.126650in}{3.216764in}}{\pgfqpoint{3.122259in}{3.206165in}}{\pgfqpoint{3.122259in}{3.195114in}}%
\pgfpathcurveto{\pgfqpoint{3.122259in}{3.184064in}}{\pgfqpoint{3.126650in}{3.173465in}}{\pgfqpoint{3.134463in}{3.165652in}}%
\pgfpathcurveto{\pgfqpoint{3.142277in}{3.157838in}}{\pgfqpoint{3.152876in}{3.153448in}}{\pgfqpoint{3.163926in}{3.153448in}}%
\pgfpathclose%
\pgfusepath{stroke,fill}%
\end{pgfscope}%
\begin{pgfscope}%
\pgfpathrectangle{\pgfqpoint{0.393613in}{0.331635in}}{\pgfqpoint{9.300000in}{7.700000in}}%
\pgfusepath{clip}%
\pgfsetbuttcap%
\pgfsetroundjoin%
\definecolor{currentfill}{rgb}{0.631373,0.788235,0.956863}%
\pgfsetfillcolor{currentfill}%
\pgfsetlinewidth{0.481800pt}%
\definecolor{currentstroke}{rgb}{1.000000,1.000000,1.000000}%
\pgfsetstrokecolor{currentstroke}%
\pgfsetdash{}{0pt}%
\pgfpathmoveto{\pgfqpoint{3.868734in}{6.834554in}}%
\pgfpathcurveto{\pgfqpoint{3.879784in}{6.834554in}}{\pgfqpoint{3.890383in}{6.838945in}}{\pgfqpoint{3.898197in}{6.846758in}}%
\pgfpathcurveto{\pgfqpoint{3.906010in}{6.854572in}}{\pgfqpoint{3.910401in}{6.865171in}}{\pgfqpoint{3.910401in}{6.876221in}}%
\pgfpathcurveto{\pgfqpoint{3.910401in}{6.887271in}}{\pgfqpoint{3.906010in}{6.897870in}}{\pgfqpoint{3.898197in}{6.905684in}}%
\pgfpathcurveto{\pgfqpoint{3.890383in}{6.913497in}}{\pgfqpoint{3.879784in}{6.917888in}}{\pgfqpoint{3.868734in}{6.917888in}}%
\pgfpathcurveto{\pgfqpoint{3.857684in}{6.917888in}}{\pgfqpoint{3.847085in}{6.913497in}}{\pgfqpoint{3.839271in}{6.905684in}}%
\pgfpathcurveto{\pgfqpoint{3.831458in}{6.897870in}}{\pgfqpoint{3.827067in}{6.887271in}}{\pgfqpoint{3.827067in}{6.876221in}}%
\pgfpathcurveto{\pgfqpoint{3.827067in}{6.865171in}}{\pgfqpoint{3.831458in}{6.854572in}}{\pgfqpoint{3.839271in}{6.846758in}}%
\pgfpathcurveto{\pgfqpoint{3.847085in}{6.838945in}}{\pgfqpoint{3.857684in}{6.834554in}}{\pgfqpoint{3.868734in}{6.834554in}}%
\pgfpathclose%
\pgfusepath{stroke,fill}%
\end{pgfscope}%
\begin{pgfscope}%
\pgfpathrectangle{\pgfqpoint{0.393613in}{0.331635in}}{\pgfqpoint{9.300000in}{7.700000in}}%
\pgfusepath{clip}%
\pgfsetbuttcap%
\pgfsetroundjoin%
\definecolor{currentfill}{rgb}{1.000000,0.705882,0.509804}%
\pgfsetfillcolor{currentfill}%
\pgfsetlinewidth{0.481800pt}%
\definecolor{currentstroke}{rgb}{1.000000,1.000000,1.000000}%
\pgfsetstrokecolor{currentstroke}%
\pgfsetdash{}{0pt}%
\pgfpathmoveto{\pgfqpoint{8.729535in}{4.279064in}}%
\pgfpathcurveto{\pgfqpoint{8.740586in}{4.279064in}}{\pgfqpoint{8.751185in}{4.283454in}}{\pgfqpoint{8.758998in}{4.291268in}}%
\pgfpathcurveto{\pgfqpoint{8.766812in}{4.299081in}}{\pgfqpoint{8.771202in}{4.309680in}}{\pgfqpoint{8.771202in}{4.320731in}}%
\pgfpathcurveto{\pgfqpoint{8.771202in}{4.331781in}}{\pgfqpoint{8.766812in}{4.342380in}}{\pgfqpoint{8.758998in}{4.350193in}}%
\pgfpathcurveto{\pgfqpoint{8.751185in}{4.358007in}}{\pgfqpoint{8.740586in}{4.362397in}}{\pgfqpoint{8.729535in}{4.362397in}}%
\pgfpathcurveto{\pgfqpoint{8.718485in}{4.362397in}}{\pgfqpoint{8.707886in}{4.358007in}}{\pgfqpoint{8.700073in}{4.350193in}}%
\pgfpathcurveto{\pgfqpoint{8.692259in}{4.342380in}}{\pgfqpoint{8.687869in}{4.331781in}}{\pgfqpoint{8.687869in}{4.320731in}}%
\pgfpathcurveto{\pgfqpoint{8.687869in}{4.309680in}}{\pgfqpoint{8.692259in}{4.299081in}}{\pgfqpoint{8.700073in}{4.291268in}}%
\pgfpathcurveto{\pgfqpoint{8.707886in}{4.283454in}}{\pgfqpoint{8.718485in}{4.279064in}}{\pgfqpoint{8.729535in}{4.279064in}}%
\pgfpathclose%
\pgfusepath{stroke,fill}%
\end{pgfscope}%
\begin{pgfscope}%
\pgfpathrectangle{\pgfqpoint{0.393613in}{0.331635in}}{\pgfqpoint{9.300000in}{7.700000in}}%
\pgfusepath{clip}%
\pgfsetbuttcap%
\pgfsetroundjoin%
\definecolor{currentfill}{rgb}{1.000000,0.705882,0.509804}%
\pgfsetfillcolor{currentfill}%
\pgfsetlinewidth{0.481800pt}%
\definecolor{currentstroke}{rgb}{1.000000,1.000000,1.000000}%
\pgfsetstrokecolor{currentstroke}%
\pgfsetdash{}{0pt}%
\pgfpathmoveto{\pgfqpoint{8.008246in}{2.574721in}}%
\pgfpathcurveto{\pgfqpoint{8.019296in}{2.574721in}}{\pgfqpoint{8.029895in}{2.579111in}}{\pgfqpoint{8.037709in}{2.586924in}}%
\pgfpathcurveto{\pgfqpoint{8.045522in}{2.594738in}}{\pgfqpoint{8.049913in}{2.605337in}}{\pgfqpoint{8.049913in}{2.616387in}}%
\pgfpathcurveto{\pgfqpoint{8.049913in}{2.627437in}}{\pgfqpoint{8.045522in}{2.638036in}}{\pgfqpoint{8.037709in}{2.645850in}}%
\pgfpathcurveto{\pgfqpoint{8.029895in}{2.653664in}}{\pgfqpoint{8.019296in}{2.658054in}}{\pgfqpoint{8.008246in}{2.658054in}}%
\pgfpathcurveto{\pgfqpoint{7.997196in}{2.658054in}}{\pgfqpoint{7.986597in}{2.653664in}}{\pgfqpoint{7.978783in}{2.645850in}}%
\pgfpathcurveto{\pgfqpoint{7.970970in}{2.638036in}}{\pgfqpoint{7.966579in}{2.627437in}}{\pgfqpoint{7.966579in}{2.616387in}}%
\pgfpathcurveto{\pgfqpoint{7.966579in}{2.605337in}}{\pgfqpoint{7.970970in}{2.594738in}}{\pgfqpoint{7.978783in}{2.586924in}}%
\pgfpathcurveto{\pgfqpoint{7.986597in}{2.579111in}}{\pgfqpoint{7.997196in}{2.574721in}}{\pgfqpoint{8.008246in}{2.574721in}}%
\pgfpathclose%
\pgfusepath{stroke,fill}%
\end{pgfscope}%
\begin{pgfscope}%
\pgfpathrectangle{\pgfqpoint{0.393613in}{0.331635in}}{\pgfqpoint{9.300000in}{7.700000in}}%
\pgfusepath{clip}%
\pgfsetbuttcap%
\pgfsetroundjoin%
\definecolor{currentfill}{rgb}{1.000000,0.705882,0.509804}%
\pgfsetfillcolor{currentfill}%
\pgfsetlinewidth{0.481800pt}%
\definecolor{currentstroke}{rgb}{1.000000,1.000000,1.000000}%
\pgfsetstrokecolor{currentstroke}%
\pgfsetdash{}{0pt}%
\pgfpathmoveto{\pgfqpoint{6.843430in}{2.204420in}}%
\pgfpathcurveto{\pgfqpoint{6.854480in}{2.204420in}}{\pgfqpoint{6.865079in}{2.208810in}}{\pgfqpoint{6.872893in}{2.216624in}}%
\pgfpathcurveto{\pgfqpoint{6.880706in}{2.224437in}}{\pgfqpoint{6.885097in}{2.235036in}}{\pgfqpoint{6.885097in}{2.246087in}}%
\pgfpathcurveto{\pgfqpoint{6.885097in}{2.257137in}}{\pgfqpoint{6.880706in}{2.267736in}}{\pgfqpoint{6.872893in}{2.275549in}}%
\pgfpathcurveto{\pgfqpoint{6.865079in}{2.283363in}}{\pgfqpoint{6.854480in}{2.287753in}}{\pgfqpoint{6.843430in}{2.287753in}}%
\pgfpathcurveto{\pgfqpoint{6.832380in}{2.287753in}}{\pgfqpoint{6.821781in}{2.283363in}}{\pgfqpoint{6.813967in}{2.275549in}}%
\pgfpathcurveto{\pgfqpoint{6.806153in}{2.267736in}}{\pgfqpoint{6.801763in}{2.257137in}}{\pgfqpoint{6.801763in}{2.246087in}}%
\pgfpathcurveto{\pgfqpoint{6.801763in}{2.235036in}}{\pgfqpoint{6.806153in}{2.224437in}}{\pgfqpoint{6.813967in}{2.216624in}}%
\pgfpathcurveto{\pgfqpoint{6.821781in}{2.208810in}}{\pgfqpoint{6.832380in}{2.204420in}}{\pgfqpoint{6.843430in}{2.204420in}}%
\pgfpathclose%
\pgfusepath{stroke,fill}%
\end{pgfscope}%
\begin{pgfscope}%
\pgfpathrectangle{\pgfqpoint{0.393613in}{0.331635in}}{\pgfqpoint{9.300000in}{7.700000in}}%
\pgfusepath{clip}%
\pgfsetbuttcap%
\pgfsetroundjoin%
\definecolor{currentfill}{rgb}{1.000000,0.705882,0.509804}%
\pgfsetfillcolor{currentfill}%
\pgfsetlinewidth{0.481800pt}%
\definecolor{currentstroke}{rgb}{1.000000,1.000000,1.000000}%
\pgfsetstrokecolor{currentstroke}%
\pgfsetdash{}{0pt}%
\pgfpathmoveto{\pgfqpoint{6.884025in}{6.061158in}}%
\pgfpathcurveto{\pgfqpoint{6.895076in}{6.061158in}}{\pgfqpoint{6.905675in}{6.065548in}}{\pgfqpoint{6.913488in}{6.073362in}}%
\pgfpathcurveto{\pgfqpoint{6.921302in}{6.081176in}}{\pgfqpoint{6.925692in}{6.091775in}}{\pgfqpoint{6.925692in}{6.102825in}}%
\pgfpathcurveto{\pgfqpoint{6.925692in}{6.113875in}}{\pgfqpoint{6.921302in}{6.124474in}}{\pgfqpoint{6.913488in}{6.132288in}}%
\pgfpathcurveto{\pgfqpoint{6.905675in}{6.140101in}}{\pgfqpoint{6.895076in}{6.144492in}}{\pgfqpoint{6.884025in}{6.144492in}}%
\pgfpathcurveto{\pgfqpoint{6.872975in}{6.144492in}}{\pgfqpoint{6.862376in}{6.140101in}}{\pgfqpoint{6.854563in}{6.132288in}}%
\pgfpathcurveto{\pgfqpoint{6.846749in}{6.124474in}}{\pgfqpoint{6.842359in}{6.113875in}}{\pgfqpoint{6.842359in}{6.102825in}}%
\pgfpathcurveto{\pgfqpoint{6.842359in}{6.091775in}}{\pgfqpoint{6.846749in}{6.081176in}}{\pgfqpoint{6.854563in}{6.073362in}}%
\pgfpathcurveto{\pgfqpoint{6.862376in}{6.065548in}}{\pgfqpoint{6.872975in}{6.061158in}}{\pgfqpoint{6.884025in}{6.061158in}}%
\pgfpathclose%
\pgfusepath{stroke,fill}%
\end{pgfscope}%
\begin{pgfscope}%
\pgfpathrectangle{\pgfqpoint{0.393613in}{0.331635in}}{\pgfqpoint{9.300000in}{7.700000in}}%
\pgfusepath{clip}%
\pgfsetbuttcap%
\pgfsetroundjoin%
\definecolor{currentfill}{rgb}{1.000000,0.705882,0.509804}%
\pgfsetfillcolor{currentfill}%
\pgfsetlinewidth{0.481800pt}%
\definecolor{currentstroke}{rgb}{1.000000,1.000000,1.000000}%
\pgfsetstrokecolor{currentstroke}%
\pgfsetdash{}{0pt}%
\pgfpathmoveto{\pgfqpoint{8.217971in}{5.450756in}}%
\pgfpathcurveto{\pgfqpoint{8.229022in}{5.450756in}}{\pgfqpoint{8.239621in}{5.455146in}}{\pgfqpoint{8.247434in}{5.462960in}}%
\pgfpathcurveto{\pgfqpoint{8.255248in}{5.470774in}}{\pgfqpoint{8.259638in}{5.481373in}}{\pgfqpoint{8.259638in}{5.492423in}}%
\pgfpathcurveto{\pgfqpoint{8.259638in}{5.503473in}}{\pgfqpoint{8.255248in}{5.514072in}}{\pgfqpoint{8.247434in}{5.521886in}}%
\pgfpathcurveto{\pgfqpoint{8.239621in}{5.529699in}}{\pgfqpoint{8.229022in}{5.534089in}}{\pgfqpoint{8.217971in}{5.534089in}}%
\pgfpathcurveto{\pgfqpoint{8.206921in}{5.534089in}}{\pgfqpoint{8.196322in}{5.529699in}}{\pgfqpoint{8.188509in}{5.521886in}}%
\pgfpathcurveto{\pgfqpoint{8.180695in}{5.514072in}}{\pgfqpoint{8.176305in}{5.503473in}}{\pgfqpoint{8.176305in}{5.492423in}}%
\pgfpathcurveto{\pgfqpoint{8.176305in}{5.481373in}}{\pgfqpoint{8.180695in}{5.470774in}}{\pgfqpoint{8.188509in}{5.462960in}}%
\pgfpathcurveto{\pgfqpoint{8.196322in}{5.455146in}}{\pgfqpoint{8.206921in}{5.450756in}}{\pgfqpoint{8.217971in}{5.450756in}}%
\pgfpathclose%
\pgfusepath{stroke,fill}%
\end{pgfscope}%
\begin{pgfscope}%
\pgfpathrectangle{\pgfqpoint{0.393613in}{0.331635in}}{\pgfqpoint{9.300000in}{7.700000in}}%
\pgfusepath{clip}%
\pgfsetbuttcap%
\pgfsetroundjoin%
\definecolor{currentfill}{rgb}{1.000000,0.705882,0.509804}%
\pgfsetfillcolor{currentfill}%
\pgfsetlinewidth{0.481800pt}%
\definecolor{currentstroke}{rgb}{1.000000,1.000000,1.000000}%
\pgfsetstrokecolor{currentstroke}%
\pgfsetdash{}{0pt}%
\pgfpathmoveto{\pgfqpoint{7.449270in}{4.145806in}}%
\pgfpathcurveto{\pgfqpoint{7.460320in}{4.145806in}}{\pgfqpoint{7.470919in}{4.150197in}}{\pgfqpoint{7.478733in}{4.158010in}}%
\pgfpathcurveto{\pgfqpoint{7.486546in}{4.165824in}}{\pgfqpoint{7.490937in}{4.176423in}}{\pgfqpoint{7.490937in}{4.187473in}}%
\pgfpathcurveto{\pgfqpoint{7.490937in}{4.198523in}}{\pgfqpoint{7.486546in}{4.209122in}}{\pgfqpoint{7.478733in}{4.216936in}}%
\pgfpathcurveto{\pgfqpoint{7.470919in}{4.224750in}}{\pgfqpoint{7.460320in}{4.229140in}}{\pgfqpoint{7.449270in}{4.229140in}}%
\pgfpathcurveto{\pgfqpoint{7.438220in}{4.229140in}}{\pgfqpoint{7.427621in}{4.224750in}}{\pgfqpoint{7.419807in}{4.216936in}}%
\pgfpathcurveto{\pgfqpoint{7.411994in}{4.209122in}}{\pgfqpoint{7.407603in}{4.198523in}}{\pgfqpoint{7.407603in}{4.187473in}}%
\pgfpathcurveto{\pgfqpoint{7.407603in}{4.176423in}}{\pgfqpoint{7.411994in}{4.165824in}}{\pgfqpoint{7.419807in}{4.158010in}}%
\pgfpathcurveto{\pgfqpoint{7.427621in}{4.150197in}}{\pgfqpoint{7.438220in}{4.145806in}}{\pgfqpoint{7.449270in}{4.145806in}}%
\pgfpathclose%
\pgfusepath{stroke,fill}%
\end{pgfscope}%
\begin{pgfscope}%
\pgfpathrectangle{\pgfqpoint{0.393613in}{0.331635in}}{\pgfqpoint{9.300000in}{7.700000in}}%
\pgfusepath{clip}%
\pgfsetbuttcap%
\pgfsetroundjoin%
\definecolor{currentfill}{rgb}{1.000000,0.705882,0.509804}%
\pgfsetfillcolor{currentfill}%
\pgfsetlinewidth{0.481800pt}%
\definecolor{currentstroke}{rgb}{1.000000,1.000000,1.000000}%
\pgfsetstrokecolor{currentstroke}%
\pgfsetdash{}{0pt}%
\pgfpathmoveto{\pgfqpoint{7.634735in}{5.895591in}}%
\pgfpathcurveto{\pgfqpoint{7.645786in}{5.895591in}}{\pgfqpoint{7.656385in}{5.899981in}}{\pgfqpoint{7.664198in}{5.907795in}}%
\pgfpathcurveto{\pgfqpoint{7.672012in}{5.915609in}}{\pgfqpoint{7.676402in}{5.926208in}}{\pgfqpoint{7.676402in}{5.937258in}}%
\pgfpathcurveto{\pgfqpoint{7.676402in}{5.948308in}}{\pgfqpoint{7.672012in}{5.958907in}}{\pgfqpoint{7.664198in}{5.966721in}}%
\pgfpathcurveto{\pgfqpoint{7.656385in}{5.974534in}}{\pgfqpoint{7.645786in}{5.978925in}}{\pgfqpoint{7.634735in}{5.978925in}}%
\pgfpathcurveto{\pgfqpoint{7.623685in}{5.978925in}}{\pgfqpoint{7.613086in}{5.974534in}}{\pgfqpoint{7.605273in}{5.966721in}}%
\pgfpathcurveto{\pgfqpoint{7.597459in}{5.958907in}}{\pgfqpoint{7.593069in}{5.948308in}}{\pgfqpoint{7.593069in}{5.937258in}}%
\pgfpathcurveto{\pgfqpoint{7.593069in}{5.926208in}}{\pgfqpoint{7.597459in}{5.915609in}}{\pgfqpoint{7.605273in}{5.907795in}}%
\pgfpathcurveto{\pgfqpoint{7.613086in}{5.899981in}}{\pgfqpoint{7.623685in}{5.895591in}}{\pgfqpoint{7.634735in}{5.895591in}}%
\pgfpathclose%
\pgfusepath{stroke,fill}%
\end{pgfscope}%
\begin{pgfscope}%
\pgfpathrectangle{\pgfqpoint{0.393613in}{0.331635in}}{\pgfqpoint{9.300000in}{7.700000in}}%
\pgfusepath{clip}%
\pgfsetbuttcap%
\pgfsetroundjoin%
\definecolor{currentfill}{rgb}{1.000000,0.705882,0.509804}%
\pgfsetfillcolor{currentfill}%
\pgfsetlinewidth{0.481800pt}%
\definecolor{currentstroke}{rgb}{1.000000,1.000000,1.000000}%
\pgfsetstrokecolor{currentstroke}%
\pgfsetdash{}{0pt}%
\pgfpathmoveto{\pgfqpoint{7.737786in}{3.811293in}}%
\pgfpathcurveto{\pgfqpoint{7.748836in}{3.811293in}}{\pgfqpoint{7.759435in}{3.815683in}}{\pgfqpoint{7.767249in}{3.823497in}}%
\pgfpathcurveto{\pgfqpoint{7.775062in}{3.831310in}}{\pgfqpoint{7.779453in}{3.841909in}}{\pgfqpoint{7.779453in}{3.852959in}}%
\pgfpathcurveto{\pgfqpoint{7.779453in}{3.864009in}}{\pgfqpoint{7.775062in}{3.874608in}}{\pgfqpoint{7.767249in}{3.882422in}}%
\pgfpathcurveto{\pgfqpoint{7.759435in}{3.890236in}}{\pgfqpoint{7.748836in}{3.894626in}}{\pgfqpoint{7.737786in}{3.894626in}}%
\pgfpathcurveto{\pgfqpoint{7.726736in}{3.894626in}}{\pgfqpoint{7.716137in}{3.890236in}}{\pgfqpoint{7.708323in}{3.882422in}}%
\pgfpathcurveto{\pgfqpoint{7.700510in}{3.874608in}}{\pgfqpoint{7.696119in}{3.864009in}}{\pgfqpoint{7.696119in}{3.852959in}}%
\pgfpathcurveto{\pgfqpoint{7.696119in}{3.841909in}}{\pgfqpoint{7.700510in}{3.831310in}}{\pgfqpoint{7.708323in}{3.823497in}}%
\pgfpathcurveto{\pgfqpoint{7.716137in}{3.815683in}}{\pgfqpoint{7.726736in}{3.811293in}}{\pgfqpoint{7.737786in}{3.811293in}}%
\pgfpathclose%
\pgfusepath{stroke,fill}%
\end{pgfscope}%
\begin{pgfscope}%
\pgfpathrectangle{\pgfqpoint{0.393613in}{0.331635in}}{\pgfqpoint{9.300000in}{7.700000in}}%
\pgfusepath{clip}%
\pgfsetbuttcap%
\pgfsetroundjoin%
\definecolor{currentfill}{rgb}{1.000000,0.705882,0.509804}%
\pgfsetfillcolor{currentfill}%
\pgfsetlinewidth{0.481800pt}%
\definecolor{currentstroke}{rgb}{1.000000,1.000000,1.000000}%
\pgfsetstrokecolor{currentstroke}%
\pgfsetdash{}{0pt}%
\pgfpathmoveto{\pgfqpoint{6.985227in}{5.946910in}}%
\pgfpathcurveto{\pgfqpoint{6.996277in}{5.946910in}}{\pgfqpoint{7.006876in}{5.951301in}}{\pgfqpoint{7.014689in}{5.959114in}}%
\pgfpathcurveto{\pgfqpoint{7.022503in}{5.966928in}}{\pgfqpoint{7.026893in}{5.977527in}}{\pgfqpoint{7.026893in}{5.988577in}}%
\pgfpathcurveto{\pgfqpoint{7.026893in}{5.999627in}}{\pgfqpoint{7.022503in}{6.010226in}}{\pgfqpoint{7.014689in}{6.018040in}}%
\pgfpathcurveto{\pgfqpoint{7.006876in}{6.025853in}}{\pgfqpoint{6.996277in}{6.030244in}}{\pgfqpoint{6.985227in}{6.030244in}}%
\pgfpathcurveto{\pgfqpoint{6.974177in}{6.030244in}}{\pgfqpoint{6.963578in}{6.025853in}}{\pgfqpoint{6.955764in}{6.018040in}}%
\pgfpathcurveto{\pgfqpoint{6.947950in}{6.010226in}}{\pgfqpoint{6.943560in}{5.999627in}}{\pgfqpoint{6.943560in}{5.988577in}}%
\pgfpathcurveto{\pgfqpoint{6.943560in}{5.977527in}}{\pgfqpoint{6.947950in}{5.966928in}}{\pgfqpoint{6.955764in}{5.959114in}}%
\pgfpathcurveto{\pgfqpoint{6.963578in}{5.951301in}}{\pgfqpoint{6.974177in}{5.946910in}}{\pgfqpoint{6.985227in}{5.946910in}}%
\pgfpathclose%
\pgfusepath{stroke,fill}%
\end{pgfscope}%
\begin{pgfscope}%
\pgfpathrectangle{\pgfqpoint{0.393613in}{0.331635in}}{\pgfqpoint{9.300000in}{7.700000in}}%
\pgfusepath{clip}%
\pgfsetbuttcap%
\pgfsetroundjoin%
\definecolor{currentfill}{rgb}{1.000000,0.705882,0.509804}%
\pgfsetfillcolor{currentfill}%
\pgfsetlinewidth{0.481800pt}%
\definecolor{currentstroke}{rgb}{1.000000,1.000000,1.000000}%
\pgfsetstrokecolor{currentstroke}%
\pgfsetdash{}{0pt}%
\pgfpathmoveto{\pgfqpoint{8.526151in}{3.449595in}}%
\pgfpathcurveto{\pgfqpoint{8.537202in}{3.449595in}}{\pgfqpoint{8.547801in}{3.453986in}}{\pgfqpoint{8.555614in}{3.461799in}}%
\pgfpathcurveto{\pgfqpoint{8.563428in}{3.469613in}}{\pgfqpoint{8.567818in}{3.480212in}}{\pgfqpoint{8.567818in}{3.491262in}}%
\pgfpathcurveto{\pgfqpoint{8.567818in}{3.502312in}}{\pgfqpoint{8.563428in}{3.512911in}}{\pgfqpoint{8.555614in}{3.520725in}}%
\pgfpathcurveto{\pgfqpoint{8.547801in}{3.528538in}}{\pgfqpoint{8.537202in}{3.532929in}}{\pgfqpoint{8.526151in}{3.532929in}}%
\pgfpathcurveto{\pgfqpoint{8.515101in}{3.532929in}}{\pgfqpoint{8.504502in}{3.528538in}}{\pgfqpoint{8.496689in}{3.520725in}}%
\pgfpathcurveto{\pgfqpoint{8.488875in}{3.512911in}}{\pgfqpoint{8.484485in}{3.502312in}}{\pgfqpoint{8.484485in}{3.491262in}}%
\pgfpathcurveto{\pgfqpoint{8.484485in}{3.480212in}}{\pgfqpoint{8.488875in}{3.469613in}}{\pgfqpoint{8.496689in}{3.461799in}}%
\pgfpathcurveto{\pgfqpoint{8.504502in}{3.453986in}}{\pgfqpoint{8.515101in}{3.449595in}}{\pgfqpoint{8.526151in}{3.449595in}}%
\pgfpathclose%
\pgfusepath{stroke,fill}%
\end{pgfscope}%
\begin{pgfscope}%
\pgfpathrectangle{\pgfqpoint{0.393613in}{0.331635in}}{\pgfqpoint{9.300000in}{7.700000in}}%
\pgfusepath{clip}%
\pgfsetbuttcap%
\pgfsetroundjoin%
\definecolor{currentfill}{rgb}{1.000000,0.705882,0.509804}%
\pgfsetfillcolor{currentfill}%
\pgfsetlinewidth{0.481800pt}%
\definecolor{currentstroke}{rgb}{1.000000,1.000000,1.000000}%
\pgfsetstrokecolor{currentstroke}%
\pgfsetdash{}{0pt}%
\pgfpathmoveto{\pgfqpoint{4.970295in}{4.640782in}}%
\pgfpathcurveto{\pgfqpoint{4.981345in}{4.640782in}}{\pgfqpoint{4.991944in}{4.645172in}}{\pgfqpoint{4.999758in}{4.652985in}}%
\pgfpathcurveto{\pgfqpoint{5.007571in}{4.660799in}}{\pgfqpoint{5.011962in}{4.671398in}}{\pgfqpoint{5.011962in}{4.682448in}}%
\pgfpathcurveto{\pgfqpoint{5.011962in}{4.693498in}}{\pgfqpoint{5.007571in}{4.704097in}}{\pgfqpoint{4.999758in}{4.711911in}}%
\pgfpathcurveto{\pgfqpoint{4.991944in}{4.719725in}}{\pgfqpoint{4.981345in}{4.724115in}}{\pgfqpoint{4.970295in}{4.724115in}}%
\pgfpathcurveto{\pgfqpoint{4.959245in}{4.724115in}}{\pgfqpoint{4.948646in}{4.719725in}}{\pgfqpoint{4.940832in}{4.711911in}}%
\pgfpathcurveto{\pgfqpoint{4.933018in}{4.704097in}}{\pgfqpoint{4.928628in}{4.693498in}}{\pgfqpoint{4.928628in}{4.682448in}}%
\pgfpathcurveto{\pgfqpoint{4.928628in}{4.671398in}}{\pgfqpoint{4.933018in}{4.660799in}}{\pgfqpoint{4.940832in}{4.652985in}}%
\pgfpathcurveto{\pgfqpoint{4.948646in}{4.645172in}}{\pgfqpoint{4.959245in}{4.640782in}}{\pgfqpoint{4.970295in}{4.640782in}}%
\pgfpathclose%
\pgfusepath{stroke,fill}%
\end{pgfscope}%
\begin{pgfscope}%
\pgfpathrectangle{\pgfqpoint{0.393613in}{0.331635in}}{\pgfqpoint{9.300000in}{7.700000in}}%
\pgfusepath{clip}%
\pgfsetbuttcap%
\pgfsetroundjoin%
\definecolor{currentfill}{rgb}{1.000000,0.705882,0.509804}%
\pgfsetfillcolor{currentfill}%
\pgfsetlinewidth{0.481800pt}%
\definecolor{currentstroke}{rgb}{1.000000,1.000000,1.000000}%
\pgfsetstrokecolor{currentstroke}%
\pgfsetdash{}{0pt}%
\pgfpathmoveto{\pgfqpoint{4.066721in}{5.258266in}}%
\pgfpathcurveto{\pgfqpoint{4.077771in}{5.258266in}}{\pgfqpoint{4.088370in}{5.262656in}}{\pgfqpoint{4.096184in}{5.270470in}}%
\pgfpathcurveto{\pgfqpoint{4.103997in}{5.278284in}}{\pgfqpoint{4.108388in}{5.288883in}}{\pgfqpoint{4.108388in}{5.299933in}}%
\pgfpathcurveto{\pgfqpoint{4.108388in}{5.310983in}}{\pgfqpoint{4.103997in}{5.321582in}}{\pgfqpoint{4.096184in}{5.329396in}}%
\pgfpathcurveto{\pgfqpoint{4.088370in}{5.337209in}}{\pgfqpoint{4.077771in}{5.341600in}}{\pgfqpoint{4.066721in}{5.341600in}}%
\pgfpathcurveto{\pgfqpoint{4.055671in}{5.341600in}}{\pgfqpoint{4.045072in}{5.337209in}}{\pgfqpoint{4.037258in}{5.329396in}}%
\pgfpathcurveto{\pgfqpoint{4.029444in}{5.321582in}}{\pgfqpoint{4.025054in}{5.310983in}}{\pgfqpoint{4.025054in}{5.299933in}}%
\pgfpathcurveto{\pgfqpoint{4.025054in}{5.288883in}}{\pgfqpoint{4.029444in}{5.278284in}}{\pgfqpoint{4.037258in}{5.270470in}}%
\pgfpathcurveto{\pgfqpoint{4.045072in}{5.262656in}}{\pgfqpoint{4.055671in}{5.258266in}}{\pgfqpoint{4.066721in}{5.258266in}}%
\pgfpathclose%
\pgfusepath{stroke,fill}%
\end{pgfscope}%
\begin{pgfscope}%
\pgfpathrectangle{\pgfqpoint{0.393613in}{0.331635in}}{\pgfqpoint{9.300000in}{7.700000in}}%
\pgfusepath{clip}%
\pgfsetbuttcap%
\pgfsetroundjoin%
\definecolor{currentfill}{rgb}{1.000000,0.705882,0.509804}%
\pgfsetfillcolor{currentfill}%
\pgfsetlinewidth{0.481800pt}%
\definecolor{currentstroke}{rgb}{1.000000,1.000000,1.000000}%
\pgfsetstrokecolor{currentstroke}%
\pgfsetdash{}{0pt}%
\pgfpathmoveto{\pgfqpoint{5.047977in}{5.087951in}}%
\pgfpathcurveto{\pgfqpoint{5.059027in}{5.087951in}}{\pgfqpoint{5.069626in}{5.092341in}}{\pgfqpoint{5.077439in}{5.100155in}}%
\pgfpathcurveto{\pgfqpoint{5.085253in}{5.107968in}}{\pgfqpoint{5.089643in}{5.118567in}}{\pgfqpoint{5.089643in}{5.129617in}}%
\pgfpathcurveto{\pgfqpoint{5.089643in}{5.140668in}}{\pgfqpoint{5.085253in}{5.151267in}}{\pgfqpoint{5.077439in}{5.159080in}}%
\pgfpathcurveto{\pgfqpoint{5.069626in}{5.166894in}}{\pgfqpoint{5.059027in}{5.171284in}}{\pgfqpoint{5.047977in}{5.171284in}}%
\pgfpathcurveto{\pgfqpoint{5.036927in}{5.171284in}}{\pgfqpoint{5.026328in}{5.166894in}}{\pgfqpoint{5.018514in}{5.159080in}}%
\pgfpathcurveto{\pgfqpoint{5.010700in}{5.151267in}}{\pgfqpoint{5.006310in}{5.140668in}}{\pgfqpoint{5.006310in}{5.129617in}}%
\pgfpathcurveto{\pgfqpoint{5.006310in}{5.118567in}}{\pgfqpoint{5.010700in}{5.107968in}}{\pgfqpoint{5.018514in}{5.100155in}}%
\pgfpathcurveto{\pgfqpoint{5.026328in}{5.092341in}}{\pgfqpoint{5.036927in}{5.087951in}}{\pgfqpoint{5.047977in}{5.087951in}}%
\pgfpathclose%
\pgfusepath{stroke,fill}%
\end{pgfscope}%
\begin{pgfscope}%
\pgfpathrectangle{\pgfqpoint{0.393613in}{0.331635in}}{\pgfqpoint{9.300000in}{7.700000in}}%
\pgfusepath{clip}%
\pgfsetbuttcap%
\pgfsetroundjoin%
\definecolor{currentfill}{rgb}{1.000000,0.705882,0.509804}%
\pgfsetfillcolor{currentfill}%
\pgfsetlinewidth{0.481800pt}%
\definecolor{currentstroke}{rgb}{1.000000,1.000000,1.000000}%
\pgfsetstrokecolor{currentstroke}%
\pgfsetdash{}{0pt}%
\pgfpathmoveto{\pgfqpoint{8.708537in}{3.107892in}}%
\pgfpathcurveto{\pgfqpoint{8.719587in}{3.107892in}}{\pgfqpoint{8.730186in}{3.112282in}}{\pgfqpoint{8.738000in}{3.120096in}}%
\pgfpathcurveto{\pgfqpoint{8.745813in}{3.127909in}}{\pgfqpoint{8.750204in}{3.138508in}}{\pgfqpoint{8.750204in}{3.149559in}}%
\pgfpathcurveto{\pgfqpoint{8.750204in}{3.160609in}}{\pgfqpoint{8.745813in}{3.171208in}}{\pgfqpoint{8.738000in}{3.179021in}}%
\pgfpathcurveto{\pgfqpoint{8.730186in}{3.186835in}}{\pgfqpoint{8.719587in}{3.191225in}}{\pgfqpoint{8.708537in}{3.191225in}}%
\pgfpathcurveto{\pgfqpoint{8.697487in}{3.191225in}}{\pgfqpoint{8.686888in}{3.186835in}}{\pgfqpoint{8.679074in}{3.179021in}}%
\pgfpathcurveto{\pgfqpoint{8.671261in}{3.171208in}}{\pgfqpoint{8.666870in}{3.160609in}}{\pgfqpoint{8.666870in}{3.149559in}}%
\pgfpathcurveto{\pgfqpoint{8.666870in}{3.138508in}}{\pgfqpoint{8.671261in}{3.127909in}}{\pgfqpoint{8.679074in}{3.120096in}}%
\pgfpathcurveto{\pgfqpoint{8.686888in}{3.112282in}}{\pgfqpoint{8.697487in}{3.107892in}}{\pgfqpoint{8.708537in}{3.107892in}}%
\pgfpathclose%
\pgfusepath{stroke,fill}%
\end{pgfscope}%
\begin{pgfscope}%
\pgfpathrectangle{\pgfqpoint{0.393613in}{0.331635in}}{\pgfqpoint{9.300000in}{7.700000in}}%
\pgfusepath{clip}%
\pgfsetbuttcap%
\pgfsetroundjoin%
\definecolor{currentfill}{rgb}{1.000000,0.705882,0.509804}%
\pgfsetfillcolor{currentfill}%
\pgfsetlinewidth{0.481800pt}%
\definecolor{currentstroke}{rgb}{1.000000,1.000000,1.000000}%
\pgfsetstrokecolor{currentstroke}%
\pgfsetdash{}{0pt}%
\pgfpathmoveto{\pgfqpoint{6.611184in}{2.557264in}}%
\pgfpathcurveto{\pgfqpoint{6.622235in}{2.557264in}}{\pgfqpoint{6.632834in}{2.561655in}}{\pgfqpoint{6.640647in}{2.569468in}}%
\pgfpathcurveto{\pgfqpoint{6.648461in}{2.577282in}}{\pgfqpoint{6.652851in}{2.587881in}}{\pgfqpoint{6.652851in}{2.598931in}}%
\pgfpathcurveto{\pgfqpoint{6.652851in}{2.609981in}}{\pgfqpoint{6.648461in}{2.620580in}}{\pgfqpoint{6.640647in}{2.628394in}}%
\pgfpathcurveto{\pgfqpoint{6.632834in}{2.636208in}}{\pgfqpoint{6.622235in}{2.640598in}}{\pgfqpoint{6.611184in}{2.640598in}}%
\pgfpathcurveto{\pgfqpoint{6.600134in}{2.640598in}}{\pgfqpoint{6.589535in}{2.636208in}}{\pgfqpoint{6.581722in}{2.628394in}}%
\pgfpathcurveto{\pgfqpoint{6.573908in}{2.620580in}}{\pgfqpoint{6.569518in}{2.609981in}}{\pgfqpoint{6.569518in}{2.598931in}}%
\pgfpathcurveto{\pgfqpoint{6.569518in}{2.587881in}}{\pgfqpoint{6.573908in}{2.577282in}}{\pgfqpoint{6.581722in}{2.569468in}}%
\pgfpathcurveto{\pgfqpoint{6.589535in}{2.561655in}}{\pgfqpoint{6.600134in}{2.557264in}}{\pgfqpoint{6.611184in}{2.557264in}}%
\pgfpathclose%
\pgfusepath{stroke,fill}%
\end{pgfscope}%
\begin{pgfscope}%
\pgfpathrectangle{\pgfqpoint{0.393613in}{0.331635in}}{\pgfqpoint{9.300000in}{7.700000in}}%
\pgfusepath{clip}%
\pgfsetbuttcap%
\pgfsetroundjoin%
\definecolor{currentfill}{rgb}{1.000000,0.705882,0.509804}%
\pgfsetfillcolor{currentfill}%
\pgfsetlinewidth{0.481800pt}%
\definecolor{currentstroke}{rgb}{1.000000,1.000000,1.000000}%
\pgfsetstrokecolor{currentstroke}%
\pgfsetdash{}{0pt}%
\pgfpathmoveto{\pgfqpoint{5.696576in}{3.888908in}}%
\pgfpathcurveto{\pgfqpoint{5.707627in}{3.888908in}}{\pgfqpoint{5.718226in}{3.893298in}}{\pgfqpoint{5.726039in}{3.901112in}}%
\pgfpathcurveto{\pgfqpoint{5.733853in}{3.908925in}}{\pgfqpoint{5.738243in}{3.919525in}}{\pgfqpoint{5.738243in}{3.930575in}}%
\pgfpathcurveto{\pgfqpoint{5.738243in}{3.941625in}}{\pgfqpoint{5.733853in}{3.952224in}}{\pgfqpoint{5.726039in}{3.960037in}}%
\pgfpathcurveto{\pgfqpoint{5.718226in}{3.967851in}}{\pgfqpoint{5.707627in}{3.972241in}}{\pgfqpoint{5.696576in}{3.972241in}}%
\pgfpathcurveto{\pgfqpoint{5.685526in}{3.972241in}}{\pgfqpoint{5.674927in}{3.967851in}}{\pgfqpoint{5.667114in}{3.960037in}}%
\pgfpathcurveto{\pgfqpoint{5.659300in}{3.952224in}}{\pgfqpoint{5.654910in}{3.941625in}}{\pgfqpoint{5.654910in}{3.930575in}}%
\pgfpathcurveto{\pgfqpoint{5.654910in}{3.919525in}}{\pgfqpoint{5.659300in}{3.908925in}}{\pgfqpoint{5.667114in}{3.901112in}}%
\pgfpathcurveto{\pgfqpoint{5.674927in}{3.893298in}}{\pgfqpoint{5.685526in}{3.888908in}}{\pgfqpoint{5.696576in}{3.888908in}}%
\pgfpathclose%
\pgfusepath{stroke,fill}%
\end{pgfscope}%
\begin{pgfscope}%
\pgfpathrectangle{\pgfqpoint{0.393613in}{0.331635in}}{\pgfqpoint{9.300000in}{7.700000in}}%
\pgfusepath{clip}%
\pgfsetbuttcap%
\pgfsetroundjoin%
\definecolor{currentfill}{rgb}{1.000000,0.705882,0.509804}%
\pgfsetfillcolor{currentfill}%
\pgfsetlinewidth{0.481800pt}%
\definecolor{currentstroke}{rgb}{1.000000,1.000000,1.000000}%
\pgfsetstrokecolor{currentstroke}%
\pgfsetdash{}{0pt}%
\pgfpathmoveto{\pgfqpoint{4.929999in}{4.103506in}}%
\pgfpathcurveto{\pgfqpoint{4.941049in}{4.103506in}}{\pgfqpoint{4.951648in}{4.107896in}}{\pgfqpoint{4.959462in}{4.115710in}}%
\pgfpathcurveto{\pgfqpoint{4.967275in}{4.123524in}}{\pgfqpoint{4.971666in}{4.134123in}}{\pgfqpoint{4.971666in}{4.145173in}}%
\pgfpathcurveto{\pgfqpoint{4.971666in}{4.156223in}}{\pgfqpoint{4.967275in}{4.166822in}}{\pgfqpoint{4.959462in}{4.174636in}}%
\pgfpathcurveto{\pgfqpoint{4.951648in}{4.182449in}}{\pgfqpoint{4.941049in}{4.186840in}}{\pgfqpoint{4.929999in}{4.186840in}}%
\pgfpathcurveto{\pgfqpoint{4.918949in}{4.186840in}}{\pgfqpoint{4.908350in}{4.182449in}}{\pgfqpoint{4.900536in}{4.174636in}}%
\pgfpathcurveto{\pgfqpoint{4.892722in}{4.166822in}}{\pgfqpoint{4.888332in}{4.156223in}}{\pgfqpoint{4.888332in}{4.145173in}}%
\pgfpathcurveto{\pgfqpoint{4.888332in}{4.134123in}}{\pgfqpoint{4.892722in}{4.123524in}}{\pgfqpoint{4.900536in}{4.115710in}}%
\pgfpathcurveto{\pgfqpoint{4.908350in}{4.107896in}}{\pgfqpoint{4.918949in}{4.103506in}}{\pgfqpoint{4.929999in}{4.103506in}}%
\pgfpathclose%
\pgfusepath{stroke,fill}%
\end{pgfscope}%
\begin{pgfscope}%
\pgfpathrectangle{\pgfqpoint{0.393613in}{0.331635in}}{\pgfqpoint{9.300000in}{7.700000in}}%
\pgfusepath{clip}%
\pgfsetbuttcap%
\pgfsetroundjoin%
\definecolor{currentfill}{rgb}{1.000000,0.705882,0.509804}%
\pgfsetfillcolor{currentfill}%
\pgfsetlinewidth{0.481800pt}%
\definecolor{currentstroke}{rgb}{1.000000,1.000000,1.000000}%
\pgfsetstrokecolor{currentstroke}%
\pgfsetdash{}{0pt}%
\pgfpathmoveto{\pgfqpoint{4.410117in}{4.593834in}}%
\pgfpathcurveto{\pgfqpoint{4.421167in}{4.593834in}}{\pgfqpoint{4.431766in}{4.598224in}}{\pgfqpoint{4.439580in}{4.606038in}}%
\pgfpathcurveto{\pgfqpoint{4.447393in}{4.613851in}}{\pgfqpoint{4.451784in}{4.624450in}}{\pgfqpoint{4.451784in}{4.635500in}}%
\pgfpathcurveto{\pgfqpoint{4.451784in}{4.646551in}}{\pgfqpoint{4.447393in}{4.657150in}}{\pgfqpoint{4.439580in}{4.664963in}}%
\pgfpathcurveto{\pgfqpoint{4.431766in}{4.672777in}}{\pgfqpoint{4.421167in}{4.677167in}}{\pgfqpoint{4.410117in}{4.677167in}}%
\pgfpathcurveto{\pgfqpoint{4.399067in}{4.677167in}}{\pgfqpoint{4.388468in}{4.672777in}}{\pgfqpoint{4.380654in}{4.664963in}}%
\pgfpathcurveto{\pgfqpoint{4.372841in}{4.657150in}}{\pgfqpoint{4.368450in}{4.646551in}}{\pgfqpoint{4.368450in}{4.635500in}}%
\pgfpathcurveto{\pgfqpoint{4.368450in}{4.624450in}}{\pgfqpoint{4.372841in}{4.613851in}}{\pgfqpoint{4.380654in}{4.606038in}}%
\pgfpathcurveto{\pgfqpoint{4.388468in}{4.598224in}}{\pgfqpoint{4.399067in}{4.593834in}}{\pgfqpoint{4.410117in}{4.593834in}}%
\pgfpathclose%
\pgfusepath{stroke,fill}%
\end{pgfscope}%
\begin{pgfscope}%
\pgfpathrectangle{\pgfqpoint{0.393613in}{0.331635in}}{\pgfqpoint{9.300000in}{7.700000in}}%
\pgfusepath{clip}%
\pgfsetbuttcap%
\pgfsetroundjoin%
\definecolor{currentfill}{rgb}{1.000000,0.705882,0.509804}%
\pgfsetfillcolor{currentfill}%
\pgfsetlinewidth{0.481800pt}%
\definecolor{currentstroke}{rgb}{1.000000,1.000000,1.000000}%
\pgfsetstrokecolor{currentstroke}%
\pgfsetdash{}{0pt}%
\pgfpathmoveto{\pgfqpoint{8.226679in}{3.197436in}}%
\pgfpathcurveto{\pgfqpoint{8.237730in}{3.197436in}}{\pgfqpoint{8.248329in}{3.201826in}}{\pgfqpoint{8.256142in}{3.209640in}}%
\pgfpathcurveto{\pgfqpoint{8.263956in}{3.217453in}}{\pgfqpoint{8.268346in}{3.228052in}}{\pgfqpoint{8.268346in}{3.239102in}}%
\pgfpathcurveto{\pgfqpoint{8.268346in}{3.250153in}}{\pgfqpoint{8.263956in}{3.260752in}}{\pgfqpoint{8.256142in}{3.268565in}}%
\pgfpathcurveto{\pgfqpoint{8.248329in}{3.276379in}}{\pgfqpoint{8.237730in}{3.280769in}}{\pgfqpoint{8.226679in}{3.280769in}}%
\pgfpathcurveto{\pgfqpoint{8.215629in}{3.280769in}}{\pgfqpoint{8.205030in}{3.276379in}}{\pgfqpoint{8.197217in}{3.268565in}}%
\pgfpathcurveto{\pgfqpoint{8.189403in}{3.260752in}}{\pgfqpoint{8.185013in}{3.250153in}}{\pgfqpoint{8.185013in}{3.239102in}}%
\pgfpathcurveto{\pgfqpoint{8.185013in}{3.228052in}}{\pgfqpoint{8.189403in}{3.217453in}}{\pgfqpoint{8.197217in}{3.209640in}}%
\pgfpathcurveto{\pgfqpoint{8.205030in}{3.201826in}}{\pgfqpoint{8.215629in}{3.197436in}}{\pgfqpoint{8.226679in}{3.197436in}}%
\pgfpathclose%
\pgfusepath{stroke,fill}%
\end{pgfscope}%
\begin{pgfscope}%
\pgfpathrectangle{\pgfqpoint{0.393613in}{0.331635in}}{\pgfqpoint{9.300000in}{7.700000in}}%
\pgfusepath{clip}%
\pgfsetbuttcap%
\pgfsetroundjoin%
\definecolor{currentfill}{rgb}{1.000000,0.705882,0.509804}%
\pgfsetfillcolor{currentfill}%
\pgfsetlinewidth{0.481800pt}%
\definecolor{currentstroke}{rgb}{1.000000,1.000000,1.000000}%
\pgfsetstrokecolor{currentstroke}%
\pgfsetdash{}{0pt}%
\pgfpathmoveto{\pgfqpoint{8.462150in}{4.732949in}}%
\pgfpathcurveto{\pgfqpoint{8.473200in}{4.732949in}}{\pgfqpoint{8.483799in}{4.737339in}}{\pgfqpoint{8.491612in}{4.745152in}}%
\pgfpathcurveto{\pgfqpoint{8.499426in}{4.752966in}}{\pgfqpoint{8.503816in}{4.763565in}}{\pgfqpoint{8.503816in}{4.774615in}}%
\pgfpathcurveto{\pgfqpoint{8.503816in}{4.785665in}}{\pgfqpoint{8.499426in}{4.796264in}}{\pgfqpoint{8.491612in}{4.804078in}}%
\pgfpathcurveto{\pgfqpoint{8.483799in}{4.811892in}}{\pgfqpoint{8.473200in}{4.816282in}}{\pgfqpoint{8.462150in}{4.816282in}}%
\pgfpathcurveto{\pgfqpoint{8.451099in}{4.816282in}}{\pgfqpoint{8.440500in}{4.811892in}}{\pgfqpoint{8.432687in}{4.804078in}}%
\pgfpathcurveto{\pgfqpoint{8.424873in}{4.796264in}}{\pgfqpoint{8.420483in}{4.785665in}}{\pgfqpoint{8.420483in}{4.774615in}}%
\pgfpathcurveto{\pgfqpoint{8.420483in}{4.763565in}}{\pgfqpoint{8.424873in}{4.752966in}}{\pgfqpoint{8.432687in}{4.745152in}}%
\pgfpathcurveto{\pgfqpoint{8.440500in}{4.737339in}}{\pgfqpoint{8.451099in}{4.732949in}}{\pgfqpoint{8.462150in}{4.732949in}}%
\pgfpathclose%
\pgfusepath{stroke,fill}%
\end{pgfscope}%
\begin{pgfscope}%
\pgfpathrectangle{\pgfqpoint{0.393613in}{0.331635in}}{\pgfqpoint{9.300000in}{7.700000in}}%
\pgfusepath{clip}%
\pgfsetbuttcap%
\pgfsetroundjoin%
\definecolor{currentfill}{rgb}{1.000000,0.705882,0.509804}%
\pgfsetfillcolor{currentfill}%
\pgfsetlinewidth{0.481800pt}%
\definecolor{currentstroke}{rgb}{1.000000,1.000000,1.000000}%
\pgfsetstrokecolor{currentstroke}%
\pgfsetdash{}{0pt}%
\pgfpathmoveto{\pgfqpoint{5.762899in}{4.672439in}}%
\pgfpathcurveto{\pgfqpoint{5.773949in}{4.672439in}}{\pgfqpoint{5.784548in}{4.676829in}}{\pgfqpoint{5.792362in}{4.684643in}}%
\pgfpathcurveto{\pgfqpoint{5.800176in}{4.692456in}}{\pgfqpoint{5.804566in}{4.703055in}}{\pgfqpoint{5.804566in}{4.714105in}}%
\pgfpathcurveto{\pgfqpoint{5.804566in}{4.725156in}}{\pgfqpoint{5.800176in}{4.735755in}}{\pgfqpoint{5.792362in}{4.743568in}}%
\pgfpathcurveto{\pgfqpoint{5.784548in}{4.751382in}}{\pgfqpoint{5.773949in}{4.755772in}}{\pgfqpoint{5.762899in}{4.755772in}}%
\pgfpathcurveto{\pgfqpoint{5.751849in}{4.755772in}}{\pgfqpoint{5.741250in}{4.751382in}}{\pgfqpoint{5.733436in}{4.743568in}}%
\pgfpathcurveto{\pgfqpoint{5.725623in}{4.735755in}}{\pgfqpoint{5.721233in}{4.725156in}}{\pgfqpoint{5.721233in}{4.714105in}}%
\pgfpathcurveto{\pgfqpoint{5.721233in}{4.703055in}}{\pgfqpoint{5.725623in}{4.692456in}}{\pgfqpoint{5.733436in}{4.684643in}}%
\pgfpathcurveto{\pgfqpoint{5.741250in}{4.676829in}}{\pgfqpoint{5.751849in}{4.672439in}}{\pgfqpoint{5.762899in}{4.672439in}}%
\pgfpathclose%
\pgfusepath{stroke,fill}%
\end{pgfscope}%
\begin{pgfscope}%
\pgfpathrectangle{\pgfqpoint{0.393613in}{0.331635in}}{\pgfqpoint{9.300000in}{7.700000in}}%
\pgfusepath{clip}%
\pgfsetbuttcap%
\pgfsetroundjoin%
\definecolor{currentfill}{rgb}{1.000000,0.705882,0.509804}%
\pgfsetfillcolor{currentfill}%
\pgfsetlinewidth{0.481800pt}%
\definecolor{currentstroke}{rgb}{1.000000,1.000000,1.000000}%
\pgfsetstrokecolor{currentstroke}%
\pgfsetdash{}{0pt}%
\pgfpathmoveto{\pgfqpoint{8.593098in}{3.906325in}}%
\pgfpathcurveto{\pgfqpoint{8.604148in}{3.906325in}}{\pgfqpoint{8.614747in}{3.910715in}}{\pgfqpoint{8.622561in}{3.918529in}}%
\pgfpathcurveto{\pgfqpoint{8.630374in}{3.926342in}}{\pgfqpoint{8.634765in}{3.936941in}}{\pgfqpoint{8.634765in}{3.947991in}}%
\pgfpathcurveto{\pgfqpoint{8.634765in}{3.959042in}}{\pgfqpoint{8.630374in}{3.969641in}}{\pgfqpoint{8.622561in}{3.977454in}}%
\pgfpathcurveto{\pgfqpoint{8.614747in}{3.985268in}}{\pgfqpoint{8.604148in}{3.989658in}}{\pgfqpoint{8.593098in}{3.989658in}}%
\pgfpathcurveto{\pgfqpoint{8.582048in}{3.989658in}}{\pgfqpoint{8.571449in}{3.985268in}}{\pgfqpoint{8.563635in}{3.977454in}}%
\pgfpathcurveto{\pgfqpoint{8.555822in}{3.969641in}}{\pgfqpoint{8.551431in}{3.959042in}}{\pgfqpoint{8.551431in}{3.947991in}}%
\pgfpathcurveto{\pgfqpoint{8.551431in}{3.936941in}}{\pgfqpoint{8.555822in}{3.926342in}}{\pgfqpoint{8.563635in}{3.918529in}}%
\pgfpathcurveto{\pgfqpoint{8.571449in}{3.910715in}}{\pgfqpoint{8.582048in}{3.906325in}}{\pgfqpoint{8.593098in}{3.906325in}}%
\pgfpathclose%
\pgfusepath{stroke,fill}%
\end{pgfscope}%
\begin{pgfscope}%
\pgfpathrectangle{\pgfqpoint{0.393613in}{0.331635in}}{\pgfqpoint{9.300000in}{7.700000in}}%
\pgfusepath{clip}%
\pgfsetbuttcap%
\pgfsetroundjoin%
\definecolor{currentfill}{rgb}{1.000000,0.705882,0.509804}%
\pgfsetfillcolor{currentfill}%
\pgfsetlinewidth{0.481800pt}%
\definecolor{currentstroke}{rgb}{1.000000,1.000000,1.000000}%
\pgfsetstrokecolor{currentstroke}%
\pgfsetdash{}{0pt}%
\pgfpathmoveto{\pgfqpoint{5.266097in}{2.274799in}}%
\pgfpathcurveto{\pgfqpoint{5.277148in}{2.274799in}}{\pgfqpoint{5.287747in}{2.279189in}}{\pgfqpoint{5.295560in}{2.287003in}}%
\pgfpathcurveto{\pgfqpoint{5.303374in}{2.294816in}}{\pgfqpoint{5.307764in}{2.305415in}}{\pgfqpoint{5.307764in}{2.316466in}}%
\pgfpathcurveto{\pgfqpoint{5.307764in}{2.327516in}}{\pgfqpoint{5.303374in}{2.338115in}}{\pgfqpoint{5.295560in}{2.345928in}}%
\pgfpathcurveto{\pgfqpoint{5.287747in}{2.353742in}}{\pgfqpoint{5.277148in}{2.358132in}}{\pgfqpoint{5.266097in}{2.358132in}}%
\pgfpathcurveto{\pgfqpoint{5.255047in}{2.358132in}}{\pgfqpoint{5.244448in}{2.353742in}}{\pgfqpoint{5.236635in}{2.345928in}}%
\pgfpathcurveto{\pgfqpoint{5.228821in}{2.338115in}}{\pgfqpoint{5.224431in}{2.327516in}}{\pgfqpoint{5.224431in}{2.316466in}}%
\pgfpathcurveto{\pgfqpoint{5.224431in}{2.305415in}}{\pgfqpoint{5.228821in}{2.294816in}}{\pgfqpoint{5.236635in}{2.287003in}}%
\pgfpathcurveto{\pgfqpoint{5.244448in}{2.279189in}}{\pgfqpoint{5.255047in}{2.274799in}}{\pgfqpoint{5.266097in}{2.274799in}}%
\pgfpathclose%
\pgfusepath{stroke,fill}%
\end{pgfscope}%
\begin{pgfscope}%
\pgfpathrectangle{\pgfqpoint{0.393613in}{0.331635in}}{\pgfqpoint{9.300000in}{7.700000in}}%
\pgfusepath{clip}%
\pgfsetbuttcap%
\pgfsetroundjoin%
\definecolor{currentfill}{rgb}{1.000000,0.705882,0.509804}%
\pgfsetfillcolor{currentfill}%
\pgfsetlinewidth{0.481800pt}%
\definecolor{currentstroke}{rgb}{1.000000,1.000000,1.000000}%
\pgfsetstrokecolor{currentstroke}%
\pgfsetdash{}{0pt}%
\pgfpathmoveto{\pgfqpoint{7.489133in}{2.337951in}}%
\pgfpathcurveto{\pgfqpoint{7.500183in}{2.337951in}}{\pgfqpoint{7.510782in}{2.342341in}}{\pgfqpoint{7.518596in}{2.350155in}}%
\pgfpathcurveto{\pgfqpoint{7.526410in}{2.357968in}}{\pgfqpoint{7.530800in}{2.368567in}}{\pgfqpoint{7.530800in}{2.379617in}}%
\pgfpathcurveto{\pgfqpoint{7.530800in}{2.390668in}}{\pgfqpoint{7.526410in}{2.401267in}}{\pgfqpoint{7.518596in}{2.409080in}}%
\pgfpathcurveto{\pgfqpoint{7.510782in}{2.416894in}}{\pgfqpoint{7.500183in}{2.421284in}}{\pgfqpoint{7.489133in}{2.421284in}}%
\pgfpathcurveto{\pgfqpoint{7.478083in}{2.421284in}}{\pgfqpoint{7.467484in}{2.416894in}}{\pgfqpoint{7.459670in}{2.409080in}}%
\pgfpathcurveto{\pgfqpoint{7.451857in}{2.401267in}}{\pgfqpoint{7.447467in}{2.390668in}}{\pgfqpoint{7.447467in}{2.379617in}}%
\pgfpathcurveto{\pgfqpoint{7.447467in}{2.368567in}}{\pgfqpoint{7.451857in}{2.357968in}}{\pgfqpoint{7.459670in}{2.350155in}}%
\pgfpathcurveto{\pgfqpoint{7.467484in}{2.342341in}}{\pgfqpoint{7.478083in}{2.337951in}}{\pgfqpoint{7.489133in}{2.337951in}}%
\pgfpathclose%
\pgfusepath{stroke,fill}%
\end{pgfscope}%
\begin{pgfscope}%
\pgfpathrectangle{\pgfqpoint{0.393613in}{0.331635in}}{\pgfqpoint{9.300000in}{7.700000in}}%
\pgfusepath{clip}%
\pgfsetbuttcap%
\pgfsetroundjoin%
\definecolor{currentfill}{rgb}{1.000000,0.705882,0.509804}%
\pgfsetfillcolor{currentfill}%
\pgfsetlinewidth{0.481800pt}%
\definecolor{currentstroke}{rgb}{1.000000,1.000000,1.000000}%
\pgfsetstrokecolor{currentstroke}%
\pgfsetdash{}{0pt}%
\pgfpathmoveto{\pgfqpoint{7.375591in}{4.687021in}}%
\pgfpathcurveto{\pgfqpoint{7.386641in}{4.687021in}}{\pgfqpoint{7.397240in}{4.691411in}}{\pgfqpoint{7.405054in}{4.699225in}}%
\pgfpathcurveto{\pgfqpoint{7.412868in}{4.707038in}}{\pgfqpoint{7.417258in}{4.717637in}}{\pgfqpoint{7.417258in}{4.728687in}}%
\pgfpathcurveto{\pgfqpoint{7.417258in}{4.739737in}}{\pgfqpoint{7.412868in}{4.750337in}}{\pgfqpoint{7.405054in}{4.758150in}}%
\pgfpathcurveto{\pgfqpoint{7.397240in}{4.765964in}}{\pgfqpoint{7.386641in}{4.770354in}}{\pgfqpoint{7.375591in}{4.770354in}}%
\pgfpathcurveto{\pgfqpoint{7.364541in}{4.770354in}}{\pgfqpoint{7.353942in}{4.765964in}}{\pgfqpoint{7.346128in}{4.758150in}}%
\pgfpathcurveto{\pgfqpoint{7.338315in}{4.750337in}}{\pgfqpoint{7.333924in}{4.739737in}}{\pgfqpoint{7.333924in}{4.728687in}}%
\pgfpathcurveto{\pgfqpoint{7.333924in}{4.717637in}}{\pgfqpoint{7.338315in}{4.707038in}}{\pgfqpoint{7.346128in}{4.699225in}}%
\pgfpathcurveto{\pgfqpoint{7.353942in}{4.691411in}}{\pgfqpoint{7.364541in}{4.687021in}}{\pgfqpoint{7.375591in}{4.687021in}}%
\pgfpathclose%
\pgfusepath{stroke,fill}%
\end{pgfscope}%
\begin{pgfscope}%
\pgfpathrectangle{\pgfqpoint{0.393613in}{0.331635in}}{\pgfqpoint{9.300000in}{7.700000in}}%
\pgfusepath{clip}%
\pgfsetbuttcap%
\pgfsetroundjoin%
\definecolor{currentfill}{rgb}{1.000000,0.705882,0.509804}%
\pgfsetfillcolor{currentfill}%
\pgfsetlinewidth{0.481800pt}%
\definecolor{currentstroke}{rgb}{1.000000,1.000000,1.000000}%
\pgfsetstrokecolor{currentstroke}%
\pgfsetdash{}{0pt}%
\pgfpathmoveto{\pgfqpoint{5.281132in}{3.971102in}}%
\pgfpathcurveto{\pgfqpoint{5.292182in}{3.971102in}}{\pgfqpoint{5.302781in}{3.975492in}}{\pgfqpoint{5.310594in}{3.983305in}}%
\pgfpathcurveto{\pgfqpoint{5.318408in}{3.991119in}}{\pgfqpoint{5.322798in}{4.001718in}}{\pgfqpoint{5.322798in}{4.012768in}}%
\pgfpathcurveto{\pgfqpoint{5.322798in}{4.023818in}}{\pgfqpoint{5.318408in}{4.034417in}}{\pgfqpoint{5.310594in}{4.042231in}}%
\pgfpathcurveto{\pgfqpoint{5.302781in}{4.050045in}}{\pgfqpoint{5.292182in}{4.054435in}}{\pgfqpoint{5.281132in}{4.054435in}}%
\pgfpathcurveto{\pgfqpoint{5.270082in}{4.054435in}}{\pgfqpoint{5.259483in}{4.050045in}}{\pgfqpoint{5.251669in}{4.042231in}}%
\pgfpathcurveto{\pgfqpoint{5.243855in}{4.034417in}}{\pgfqpoint{5.239465in}{4.023818in}}{\pgfqpoint{5.239465in}{4.012768in}}%
\pgfpathcurveto{\pgfqpoint{5.239465in}{4.001718in}}{\pgfqpoint{5.243855in}{3.991119in}}{\pgfqpoint{5.251669in}{3.983305in}}%
\pgfpathcurveto{\pgfqpoint{5.259483in}{3.975492in}}{\pgfqpoint{5.270082in}{3.971102in}}{\pgfqpoint{5.281132in}{3.971102in}}%
\pgfpathclose%
\pgfusepath{stroke,fill}%
\end{pgfscope}%
\begin{pgfscope}%
\pgfpathrectangle{\pgfqpoint{0.393613in}{0.331635in}}{\pgfqpoint{9.300000in}{7.700000in}}%
\pgfusepath{clip}%
\pgfsetbuttcap%
\pgfsetroundjoin%
\definecolor{currentfill}{rgb}{1.000000,0.705882,0.509804}%
\pgfsetfillcolor{currentfill}%
\pgfsetlinewidth{0.481800pt}%
\definecolor{currentstroke}{rgb}{1.000000,1.000000,1.000000}%
\pgfsetstrokecolor{currentstroke}%
\pgfsetdash{}{0pt}%
\pgfpathmoveto{\pgfqpoint{8.775963in}{3.764502in}}%
\pgfpathcurveto{\pgfqpoint{8.787013in}{3.764502in}}{\pgfqpoint{8.797612in}{3.768892in}}{\pgfqpoint{8.805425in}{3.776706in}}%
\pgfpathcurveto{\pgfqpoint{8.813239in}{3.784519in}}{\pgfqpoint{8.817629in}{3.795118in}}{\pgfqpoint{8.817629in}{3.806168in}}%
\pgfpathcurveto{\pgfqpoint{8.817629in}{3.817219in}}{\pgfqpoint{8.813239in}{3.827818in}}{\pgfqpoint{8.805425in}{3.835631in}}%
\pgfpathcurveto{\pgfqpoint{8.797612in}{3.843445in}}{\pgfqpoint{8.787013in}{3.847835in}}{\pgfqpoint{8.775963in}{3.847835in}}%
\pgfpathcurveto{\pgfqpoint{8.764912in}{3.847835in}}{\pgfqpoint{8.754313in}{3.843445in}}{\pgfqpoint{8.746500in}{3.835631in}}%
\pgfpathcurveto{\pgfqpoint{8.738686in}{3.827818in}}{\pgfqpoint{8.734296in}{3.817219in}}{\pgfqpoint{8.734296in}{3.806168in}}%
\pgfpathcurveto{\pgfqpoint{8.734296in}{3.795118in}}{\pgfqpoint{8.738686in}{3.784519in}}{\pgfqpoint{8.746500in}{3.776706in}}%
\pgfpathcurveto{\pgfqpoint{8.754313in}{3.768892in}}{\pgfqpoint{8.764912in}{3.764502in}}{\pgfqpoint{8.775963in}{3.764502in}}%
\pgfpathclose%
\pgfusepath{stroke,fill}%
\end{pgfscope}%
\begin{pgfscope}%
\pgfpathrectangle{\pgfqpoint{0.393613in}{0.331635in}}{\pgfqpoint{9.300000in}{7.700000in}}%
\pgfusepath{clip}%
\pgfsetbuttcap%
\pgfsetroundjoin%
\definecolor{currentfill}{rgb}{1.000000,0.705882,0.509804}%
\pgfsetfillcolor{currentfill}%
\pgfsetlinewidth{0.481800pt}%
\definecolor{currentstroke}{rgb}{1.000000,1.000000,1.000000}%
\pgfsetstrokecolor{currentstroke}%
\pgfsetdash{}{0pt}%
\pgfpathmoveto{\pgfqpoint{9.270885in}{4.190667in}}%
\pgfpathcurveto{\pgfqpoint{9.281935in}{4.190667in}}{\pgfqpoint{9.292534in}{4.195057in}}{\pgfqpoint{9.300348in}{4.202871in}}%
\pgfpathcurveto{\pgfqpoint{9.308162in}{4.210684in}}{\pgfqpoint{9.312552in}{4.221283in}}{\pgfqpoint{9.312552in}{4.232334in}}%
\pgfpathcurveto{\pgfqpoint{9.312552in}{4.243384in}}{\pgfqpoint{9.308162in}{4.253983in}}{\pgfqpoint{9.300348in}{4.261796in}}%
\pgfpathcurveto{\pgfqpoint{9.292534in}{4.269610in}}{\pgfqpoint{9.281935in}{4.274000in}}{\pgfqpoint{9.270885in}{4.274000in}}%
\pgfpathcurveto{\pgfqpoint{9.259835in}{4.274000in}}{\pgfqpoint{9.249236in}{4.269610in}}{\pgfqpoint{9.241422in}{4.261796in}}%
\pgfpathcurveto{\pgfqpoint{9.233609in}{4.253983in}}{\pgfqpoint{9.229219in}{4.243384in}}{\pgfqpoint{9.229219in}{4.232334in}}%
\pgfpathcurveto{\pgfqpoint{9.229219in}{4.221283in}}{\pgfqpoint{9.233609in}{4.210684in}}{\pgfqpoint{9.241422in}{4.202871in}}%
\pgfpathcurveto{\pgfqpoint{9.249236in}{4.195057in}}{\pgfqpoint{9.259835in}{4.190667in}}{\pgfqpoint{9.270885in}{4.190667in}}%
\pgfpathclose%
\pgfusepath{stroke,fill}%
\end{pgfscope}%
\begin{pgfscope}%
\pgfpathrectangle{\pgfqpoint{0.393613in}{0.331635in}}{\pgfqpoint{9.300000in}{7.700000in}}%
\pgfusepath{clip}%
\pgfsetbuttcap%
\pgfsetroundjoin%
\definecolor{currentfill}{rgb}{1.000000,0.705882,0.509804}%
\pgfsetfillcolor{currentfill}%
\pgfsetlinewidth{0.481800pt}%
\definecolor{currentstroke}{rgb}{1.000000,1.000000,1.000000}%
\pgfsetstrokecolor{currentstroke}%
\pgfsetdash{}{0pt}%
\pgfpathmoveto{\pgfqpoint{5.221194in}{3.827244in}}%
\pgfpathcurveto{\pgfqpoint{5.232244in}{3.827244in}}{\pgfqpoint{5.242843in}{3.831635in}}{\pgfqpoint{5.250657in}{3.839448in}}%
\pgfpathcurveto{\pgfqpoint{5.258470in}{3.847262in}}{\pgfqpoint{5.262861in}{3.857861in}}{\pgfqpoint{5.262861in}{3.868911in}}%
\pgfpathcurveto{\pgfqpoint{5.262861in}{3.879961in}}{\pgfqpoint{5.258470in}{3.890560in}}{\pgfqpoint{5.250657in}{3.898374in}}%
\pgfpathcurveto{\pgfqpoint{5.242843in}{3.906187in}}{\pgfqpoint{5.232244in}{3.910578in}}{\pgfqpoint{5.221194in}{3.910578in}}%
\pgfpathcurveto{\pgfqpoint{5.210144in}{3.910578in}}{\pgfqpoint{5.199545in}{3.906187in}}{\pgfqpoint{5.191731in}{3.898374in}}%
\pgfpathcurveto{\pgfqpoint{5.183917in}{3.890560in}}{\pgfqpoint{5.179527in}{3.879961in}}{\pgfqpoint{5.179527in}{3.868911in}}%
\pgfpathcurveto{\pgfqpoint{5.179527in}{3.857861in}}{\pgfqpoint{5.183917in}{3.847262in}}{\pgfqpoint{5.191731in}{3.839448in}}%
\pgfpathcurveto{\pgfqpoint{5.199545in}{3.831635in}}{\pgfqpoint{5.210144in}{3.827244in}}{\pgfqpoint{5.221194in}{3.827244in}}%
\pgfpathclose%
\pgfusepath{stroke,fill}%
\end{pgfscope}%
\begin{pgfscope}%
\pgfpathrectangle{\pgfqpoint{0.393613in}{0.331635in}}{\pgfqpoint{9.300000in}{7.700000in}}%
\pgfusepath{clip}%
\pgfsetbuttcap%
\pgfsetroundjoin%
\definecolor{currentfill}{rgb}{1.000000,0.705882,0.509804}%
\pgfsetfillcolor{currentfill}%
\pgfsetlinewidth{0.481800pt}%
\definecolor{currentstroke}{rgb}{1.000000,1.000000,1.000000}%
\pgfsetstrokecolor{currentstroke}%
\pgfsetdash{}{0pt}%
\pgfpathmoveto{\pgfqpoint{4.726563in}{5.063919in}}%
\pgfpathcurveto{\pgfqpoint{4.737613in}{5.063919in}}{\pgfqpoint{4.748212in}{5.068309in}}{\pgfqpoint{4.756026in}{5.076123in}}%
\pgfpathcurveto{\pgfqpoint{4.763839in}{5.083936in}}{\pgfqpoint{4.768230in}{5.094535in}}{\pgfqpoint{4.768230in}{5.105585in}}%
\pgfpathcurveto{\pgfqpoint{4.768230in}{5.116636in}}{\pgfqpoint{4.763839in}{5.127235in}}{\pgfqpoint{4.756026in}{5.135048in}}%
\pgfpathcurveto{\pgfqpoint{4.748212in}{5.142862in}}{\pgfqpoint{4.737613in}{5.147252in}}{\pgfqpoint{4.726563in}{5.147252in}}%
\pgfpathcurveto{\pgfqpoint{4.715513in}{5.147252in}}{\pgfqpoint{4.704914in}{5.142862in}}{\pgfqpoint{4.697100in}{5.135048in}}%
\pgfpathcurveto{\pgfqpoint{4.689286in}{5.127235in}}{\pgfqpoint{4.684896in}{5.116636in}}{\pgfqpoint{4.684896in}{5.105585in}}%
\pgfpathcurveto{\pgfqpoint{4.684896in}{5.094535in}}{\pgfqpoint{4.689286in}{5.083936in}}{\pgfqpoint{4.697100in}{5.076123in}}%
\pgfpathcurveto{\pgfqpoint{4.704914in}{5.068309in}}{\pgfqpoint{4.715513in}{5.063919in}}{\pgfqpoint{4.726563in}{5.063919in}}%
\pgfpathclose%
\pgfusepath{stroke,fill}%
\end{pgfscope}%
\begin{pgfscope}%
\pgfpathrectangle{\pgfqpoint{0.393613in}{0.331635in}}{\pgfqpoint{9.300000in}{7.700000in}}%
\pgfusepath{clip}%
\pgfsetbuttcap%
\pgfsetroundjoin%
\definecolor{currentfill}{rgb}{1.000000,0.705882,0.509804}%
\pgfsetfillcolor{currentfill}%
\pgfsetlinewidth{0.481800pt}%
\definecolor{currentstroke}{rgb}{1.000000,1.000000,1.000000}%
\pgfsetstrokecolor{currentstroke}%
\pgfsetdash{}{0pt}%
\pgfpathmoveto{\pgfqpoint{5.210620in}{2.259081in}}%
\pgfpathcurveto{\pgfqpoint{5.221670in}{2.259081in}}{\pgfqpoint{5.232269in}{2.263471in}}{\pgfqpoint{5.240083in}{2.271285in}}%
\pgfpathcurveto{\pgfqpoint{5.247897in}{2.279098in}}{\pgfqpoint{5.252287in}{2.289698in}}{\pgfqpoint{5.252287in}{2.300748in}}%
\pgfpathcurveto{\pgfqpoint{5.252287in}{2.311798in}}{\pgfqpoint{5.247897in}{2.322397in}}{\pgfqpoint{5.240083in}{2.330210in}}%
\pgfpathcurveto{\pgfqpoint{5.232269in}{2.338024in}}{\pgfqpoint{5.221670in}{2.342414in}}{\pgfqpoint{5.210620in}{2.342414in}}%
\pgfpathcurveto{\pgfqpoint{5.199570in}{2.342414in}}{\pgfqpoint{5.188971in}{2.338024in}}{\pgfqpoint{5.181158in}{2.330210in}}%
\pgfpathcurveto{\pgfqpoint{5.173344in}{2.322397in}}{\pgfqpoint{5.168954in}{2.311798in}}{\pgfqpoint{5.168954in}{2.300748in}}%
\pgfpathcurveto{\pgfqpoint{5.168954in}{2.289698in}}{\pgfqpoint{5.173344in}{2.279098in}}{\pgfqpoint{5.181158in}{2.271285in}}%
\pgfpathcurveto{\pgfqpoint{5.188971in}{2.263471in}}{\pgfqpoint{5.199570in}{2.259081in}}{\pgfqpoint{5.210620in}{2.259081in}}%
\pgfpathclose%
\pgfusepath{stroke,fill}%
\end{pgfscope}%
\begin{pgfscope}%
\pgfpathrectangle{\pgfqpoint{0.393613in}{0.331635in}}{\pgfqpoint{9.300000in}{7.700000in}}%
\pgfusepath{clip}%
\pgfsetbuttcap%
\pgfsetroundjoin%
\definecolor{currentfill}{rgb}{1.000000,0.705882,0.509804}%
\pgfsetfillcolor{currentfill}%
\pgfsetlinewidth{0.481800pt}%
\definecolor{currentstroke}{rgb}{1.000000,1.000000,1.000000}%
\pgfsetstrokecolor{currentstroke}%
\pgfsetdash{}{0pt}%
\pgfpathmoveto{\pgfqpoint{6.611736in}{2.781108in}}%
\pgfpathcurveto{\pgfqpoint{6.622786in}{2.781108in}}{\pgfqpoint{6.633385in}{2.785498in}}{\pgfqpoint{6.641199in}{2.793312in}}%
\pgfpathcurveto{\pgfqpoint{6.649013in}{2.801125in}}{\pgfqpoint{6.653403in}{2.811724in}}{\pgfqpoint{6.653403in}{2.822774in}}%
\pgfpathcurveto{\pgfqpoint{6.653403in}{2.833824in}}{\pgfqpoint{6.649013in}{2.844423in}}{\pgfqpoint{6.641199in}{2.852237in}}%
\pgfpathcurveto{\pgfqpoint{6.633385in}{2.860051in}}{\pgfqpoint{6.622786in}{2.864441in}}{\pgfqpoint{6.611736in}{2.864441in}}%
\pgfpathcurveto{\pgfqpoint{6.600686in}{2.864441in}}{\pgfqpoint{6.590087in}{2.860051in}}{\pgfqpoint{6.582273in}{2.852237in}}%
\pgfpathcurveto{\pgfqpoint{6.574460in}{2.844423in}}{\pgfqpoint{6.570069in}{2.833824in}}{\pgfqpoint{6.570069in}{2.822774in}}%
\pgfpathcurveto{\pgfqpoint{6.570069in}{2.811724in}}{\pgfqpoint{6.574460in}{2.801125in}}{\pgfqpoint{6.582273in}{2.793312in}}%
\pgfpathcurveto{\pgfqpoint{6.590087in}{2.785498in}}{\pgfqpoint{6.600686in}{2.781108in}}{\pgfqpoint{6.611736in}{2.781108in}}%
\pgfpathclose%
\pgfusepath{stroke,fill}%
\end{pgfscope}%
\begin{pgfscope}%
\pgfpathrectangle{\pgfqpoint{0.393613in}{0.331635in}}{\pgfqpoint{9.300000in}{7.700000in}}%
\pgfusepath{clip}%
\pgfsetbuttcap%
\pgfsetroundjoin%
\definecolor{currentfill}{rgb}{1.000000,0.705882,0.509804}%
\pgfsetfillcolor{currentfill}%
\pgfsetlinewidth{0.481800pt}%
\definecolor{currentstroke}{rgb}{1.000000,1.000000,1.000000}%
\pgfsetstrokecolor{currentstroke}%
\pgfsetdash{}{0pt}%
\pgfpathmoveto{\pgfqpoint{4.838508in}{5.640517in}}%
\pgfpathcurveto{\pgfqpoint{4.849558in}{5.640517in}}{\pgfqpoint{4.860157in}{5.644907in}}{\pgfqpoint{4.867971in}{5.652721in}}%
\pgfpathcurveto{\pgfqpoint{4.875784in}{5.660535in}}{\pgfqpoint{4.880175in}{5.671134in}}{\pgfqpoint{4.880175in}{5.682184in}}%
\pgfpathcurveto{\pgfqpoint{4.880175in}{5.693234in}}{\pgfqpoint{4.875784in}{5.703833in}}{\pgfqpoint{4.867971in}{5.711647in}}%
\pgfpathcurveto{\pgfqpoint{4.860157in}{5.719460in}}{\pgfqpoint{4.849558in}{5.723850in}}{\pgfqpoint{4.838508in}{5.723850in}}%
\pgfpathcurveto{\pgfqpoint{4.827458in}{5.723850in}}{\pgfqpoint{4.816859in}{5.719460in}}{\pgfqpoint{4.809045in}{5.711647in}}%
\pgfpathcurveto{\pgfqpoint{4.801232in}{5.703833in}}{\pgfqpoint{4.796841in}{5.693234in}}{\pgfqpoint{4.796841in}{5.682184in}}%
\pgfpathcurveto{\pgfqpoint{4.796841in}{5.671134in}}{\pgfqpoint{4.801232in}{5.660535in}}{\pgfqpoint{4.809045in}{5.652721in}}%
\pgfpathcurveto{\pgfqpoint{4.816859in}{5.644907in}}{\pgfqpoint{4.827458in}{5.640517in}}{\pgfqpoint{4.838508in}{5.640517in}}%
\pgfpathclose%
\pgfusepath{stroke,fill}%
\end{pgfscope}%
\begin{pgfscope}%
\pgfpathrectangle{\pgfqpoint{0.393613in}{0.331635in}}{\pgfqpoint{9.300000in}{7.700000in}}%
\pgfusepath{clip}%
\pgfsetbuttcap%
\pgfsetroundjoin%
\definecolor{currentfill}{rgb}{1.000000,0.705882,0.509804}%
\pgfsetfillcolor{currentfill}%
\pgfsetlinewidth{0.481800pt}%
\definecolor{currentstroke}{rgb}{1.000000,1.000000,1.000000}%
\pgfsetstrokecolor{currentstroke}%
\pgfsetdash{}{0pt}%
\pgfpathmoveto{\pgfqpoint{7.784989in}{5.637904in}}%
\pgfpathcurveto{\pgfqpoint{7.796039in}{5.637904in}}{\pgfqpoint{7.806638in}{5.642294in}}{\pgfqpoint{7.814452in}{5.650108in}}%
\pgfpathcurveto{\pgfqpoint{7.822265in}{5.657921in}}{\pgfqpoint{7.826656in}{5.668520in}}{\pgfqpoint{7.826656in}{5.679570in}}%
\pgfpathcurveto{\pgfqpoint{7.826656in}{5.690621in}}{\pgfqpoint{7.822265in}{5.701220in}}{\pgfqpoint{7.814452in}{5.709033in}}%
\pgfpathcurveto{\pgfqpoint{7.806638in}{5.716847in}}{\pgfqpoint{7.796039in}{5.721237in}}{\pgfqpoint{7.784989in}{5.721237in}}%
\pgfpathcurveto{\pgfqpoint{7.773939in}{5.721237in}}{\pgfqpoint{7.763340in}{5.716847in}}{\pgfqpoint{7.755526in}{5.709033in}}%
\pgfpathcurveto{\pgfqpoint{7.747713in}{5.701220in}}{\pgfqpoint{7.743322in}{5.690621in}}{\pgfqpoint{7.743322in}{5.679570in}}%
\pgfpathcurveto{\pgfqpoint{7.743322in}{5.668520in}}{\pgfqpoint{7.747713in}{5.657921in}}{\pgfqpoint{7.755526in}{5.650108in}}%
\pgfpathcurveto{\pgfqpoint{7.763340in}{5.642294in}}{\pgfqpoint{7.773939in}{5.637904in}}{\pgfqpoint{7.784989in}{5.637904in}}%
\pgfpathclose%
\pgfusepath{stroke,fill}%
\end{pgfscope}%
\begin{pgfscope}%
\pgfpathrectangle{\pgfqpoint{0.393613in}{0.331635in}}{\pgfqpoint{9.300000in}{7.700000in}}%
\pgfusepath{clip}%
\pgfsetbuttcap%
\pgfsetroundjoin%
\definecolor{currentfill}{rgb}{1.000000,0.705882,0.509804}%
\pgfsetfillcolor{currentfill}%
\pgfsetlinewidth{0.481800pt}%
\definecolor{currentstroke}{rgb}{1.000000,1.000000,1.000000}%
\pgfsetstrokecolor{currentstroke}%
\pgfsetdash{}{0pt}%
\pgfpathmoveto{\pgfqpoint{9.033656in}{3.832510in}}%
\pgfpathcurveto{\pgfqpoint{9.044706in}{3.832510in}}{\pgfqpoint{9.055305in}{3.836901in}}{\pgfqpoint{9.063119in}{3.844714in}}%
\pgfpathcurveto{\pgfqpoint{9.070933in}{3.852528in}}{\pgfqpoint{9.075323in}{3.863127in}}{\pgfqpoint{9.075323in}{3.874177in}}%
\pgfpathcurveto{\pgfqpoint{9.075323in}{3.885227in}}{\pgfqpoint{9.070933in}{3.895826in}}{\pgfqpoint{9.063119in}{3.903640in}}%
\pgfpathcurveto{\pgfqpoint{9.055305in}{3.911454in}}{\pgfqpoint{9.044706in}{3.915844in}}{\pgfqpoint{9.033656in}{3.915844in}}%
\pgfpathcurveto{\pgfqpoint{9.022606in}{3.915844in}}{\pgfqpoint{9.012007in}{3.911454in}}{\pgfqpoint{9.004193in}{3.903640in}}%
\pgfpathcurveto{\pgfqpoint{8.996380in}{3.895826in}}{\pgfqpoint{8.991990in}{3.885227in}}{\pgfqpoint{8.991990in}{3.874177in}}%
\pgfpathcurveto{\pgfqpoint{8.991990in}{3.863127in}}{\pgfqpoint{8.996380in}{3.852528in}}{\pgfqpoint{9.004193in}{3.844714in}}%
\pgfpathcurveto{\pgfqpoint{9.012007in}{3.836901in}}{\pgfqpoint{9.022606in}{3.832510in}}{\pgfqpoint{9.033656in}{3.832510in}}%
\pgfpathclose%
\pgfusepath{stroke,fill}%
\end{pgfscope}%
\begin{pgfscope}%
\pgfpathrectangle{\pgfqpoint{0.393613in}{0.331635in}}{\pgfqpoint{9.300000in}{7.700000in}}%
\pgfusepath{clip}%
\pgfsetbuttcap%
\pgfsetroundjoin%
\definecolor{currentfill}{rgb}{1.000000,0.705882,0.509804}%
\pgfsetfillcolor{currentfill}%
\pgfsetlinewidth{0.481800pt}%
\definecolor{currentstroke}{rgb}{1.000000,1.000000,1.000000}%
\pgfsetstrokecolor{currentstroke}%
\pgfsetdash{}{0pt}%
\pgfpathmoveto{\pgfqpoint{4.512348in}{4.999893in}}%
\pgfpathcurveto{\pgfqpoint{4.523398in}{4.999893in}}{\pgfqpoint{4.533997in}{5.004283in}}{\pgfqpoint{4.541810in}{5.012097in}}%
\pgfpathcurveto{\pgfqpoint{4.549624in}{5.019910in}}{\pgfqpoint{4.554014in}{5.030509in}}{\pgfqpoint{4.554014in}{5.041560in}}%
\pgfpathcurveto{\pgfqpoint{4.554014in}{5.052610in}}{\pgfqpoint{4.549624in}{5.063209in}}{\pgfqpoint{4.541810in}{5.071022in}}%
\pgfpathcurveto{\pgfqpoint{4.533997in}{5.078836in}}{\pgfqpoint{4.523398in}{5.083226in}}{\pgfqpoint{4.512348in}{5.083226in}}%
\pgfpathcurveto{\pgfqpoint{4.501297in}{5.083226in}}{\pgfqpoint{4.490698in}{5.078836in}}{\pgfqpoint{4.482885in}{5.071022in}}%
\pgfpathcurveto{\pgfqpoint{4.475071in}{5.063209in}}{\pgfqpoint{4.470681in}{5.052610in}}{\pgfqpoint{4.470681in}{5.041560in}}%
\pgfpathcurveto{\pgfqpoint{4.470681in}{5.030509in}}{\pgfqpoint{4.475071in}{5.019910in}}{\pgfqpoint{4.482885in}{5.012097in}}%
\pgfpathcurveto{\pgfqpoint{4.490698in}{5.004283in}}{\pgfqpoint{4.501297in}{4.999893in}}{\pgfqpoint{4.512348in}{4.999893in}}%
\pgfpathclose%
\pgfusepath{stroke,fill}%
\end{pgfscope}%
\begin{pgfscope}%
\pgfpathrectangle{\pgfqpoint{0.393613in}{0.331635in}}{\pgfqpoint{9.300000in}{7.700000in}}%
\pgfusepath{clip}%
\pgfsetbuttcap%
\pgfsetroundjoin%
\definecolor{currentfill}{rgb}{1.000000,0.705882,0.509804}%
\pgfsetfillcolor{currentfill}%
\pgfsetlinewidth{0.481800pt}%
\definecolor{currentstroke}{rgb}{1.000000,1.000000,1.000000}%
\pgfsetstrokecolor{currentstroke}%
\pgfsetdash{}{0pt}%
\pgfpathmoveto{\pgfqpoint{5.861358in}{5.827605in}}%
\pgfpathcurveto{\pgfqpoint{5.872408in}{5.827605in}}{\pgfqpoint{5.883007in}{5.831995in}}{\pgfqpoint{5.890820in}{5.839809in}}%
\pgfpathcurveto{\pgfqpoint{5.898634in}{5.847623in}}{\pgfqpoint{5.903024in}{5.858222in}}{\pgfqpoint{5.903024in}{5.869272in}}%
\pgfpathcurveto{\pgfqpoint{5.903024in}{5.880322in}}{\pgfqpoint{5.898634in}{5.890921in}}{\pgfqpoint{5.890820in}{5.898734in}}%
\pgfpathcurveto{\pgfqpoint{5.883007in}{5.906548in}}{\pgfqpoint{5.872408in}{5.910938in}}{\pgfqpoint{5.861358in}{5.910938in}}%
\pgfpathcurveto{\pgfqpoint{5.850307in}{5.910938in}}{\pgfqpoint{5.839708in}{5.906548in}}{\pgfqpoint{5.831895in}{5.898734in}}%
\pgfpathcurveto{\pgfqpoint{5.824081in}{5.890921in}}{\pgfqpoint{5.819691in}{5.880322in}}{\pgfqpoint{5.819691in}{5.869272in}}%
\pgfpathcurveto{\pgfqpoint{5.819691in}{5.858222in}}{\pgfqpoint{5.824081in}{5.847623in}}{\pgfqpoint{5.831895in}{5.839809in}}%
\pgfpathcurveto{\pgfqpoint{5.839708in}{5.831995in}}{\pgfqpoint{5.850307in}{5.827605in}}{\pgfqpoint{5.861358in}{5.827605in}}%
\pgfpathclose%
\pgfusepath{stroke,fill}%
\end{pgfscope}%
\begin{pgfscope}%
\pgfpathrectangle{\pgfqpoint{0.393613in}{0.331635in}}{\pgfqpoint{9.300000in}{7.700000in}}%
\pgfusepath{clip}%
\pgfsetbuttcap%
\pgfsetroundjoin%
\definecolor{currentfill}{rgb}{1.000000,0.705882,0.509804}%
\pgfsetfillcolor{currentfill}%
\pgfsetlinewidth{0.481800pt}%
\definecolor{currentstroke}{rgb}{1.000000,1.000000,1.000000}%
\pgfsetstrokecolor{currentstroke}%
\pgfsetdash{}{0pt}%
\pgfpathmoveto{\pgfqpoint{5.626156in}{2.786395in}}%
\pgfpathcurveto{\pgfqpoint{5.637207in}{2.786395in}}{\pgfqpoint{5.647806in}{2.790785in}}{\pgfqpoint{5.655619in}{2.798599in}}%
\pgfpathcurveto{\pgfqpoint{5.663433in}{2.806413in}}{\pgfqpoint{5.667823in}{2.817012in}}{\pgfqpoint{5.667823in}{2.828062in}}%
\pgfpathcurveto{\pgfqpoint{5.667823in}{2.839112in}}{\pgfqpoint{5.663433in}{2.849711in}}{\pgfqpoint{5.655619in}{2.857525in}}%
\pgfpathcurveto{\pgfqpoint{5.647806in}{2.865338in}}{\pgfqpoint{5.637207in}{2.869729in}}{\pgfqpoint{5.626156in}{2.869729in}}%
\pgfpathcurveto{\pgfqpoint{5.615106in}{2.869729in}}{\pgfqpoint{5.604507in}{2.865338in}}{\pgfqpoint{5.596694in}{2.857525in}}%
\pgfpathcurveto{\pgfqpoint{5.588880in}{2.849711in}}{\pgfqpoint{5.584490in}{2.839112in}}{\pgfqpoint{5.584490in}{2.828062in}}%
\pgfpathcurveto{\pgfqpoint{5.584490in}{2.817012in}}{\pgfqpoint{5.588880in}{2.806413in}}{\pgfqpoint{5.596694in}{2.798599in}}%
\pgfpathcurveto{\pgfqpoint{5.604507in}{2.790785in}}{\pgfqpoint{5.615106in}{2.786395in}}{\pgfqpoint{5.626156in}{2.786395in}}%
\pgfpathclose%
\pgfusepath{stroke,fill}%
\end{pgfscope}%
\begin{pgfscope}%
\pgfpathrectangle{\pgfqpoint{0.393613in}{0.331635in}}{\pgfqpoint{9.300000in}{7.700000in}}%
\pgfusepath{clip}%
\pgfsetbuttcap%
\pgfsetroundjoin%
\definecolor{currentfill}{rgb}{1.000000,0.705882,0.509804}%
\pgfsetfillcolor{currentfill}%
\pgfsetlinewidth{0.481800pt}%
\definecolor{currentstroke}{rgb}{1.000000,1.000000,1.000000}%
\pgfsetstrokecolor{currentstroke}%
\pgfsetdash{}{0pt}%
\pgfpathmoveto{\pgfqpoint{4.868691in}{3.435194in}}%
\pgfpathcurveto{\pgfqpoint{4.879741in}{3.435194in}}{\pgfqpoint{4.890340in}{3.439584in}}{\pgfqpoint{4.898154in}{3.447397in}}%
\pgfpathcurveto{\pgfqpoint{4.905967in}{3.455211in}}{\pgfqpoint{4.910358in}{3.465810in}}{\pgfqpoint{4.910358in}{3.476860in}}%
\pgfpathcurveto{\pgfqpoint{4.910358in}{3.487910in}}{\pgfqpoint{4.905967in}{3.498509in}}{\pgfqpoint{4.898154in}{3.506323in}}%
\pgfpathcurveto{\pgfqpoint{4.890340in}{3.514137in}}{\pgfqpoint{4.879741in}{3.518527in}}{\pgfqpoint{4.868691in}{3.518527in}}%
\pgfpathcurveto{\pgfqpoint{4.857641in}{3.518527in}}{\pgfqpoint{4.847042in}{3.514137in}}{\pgfqpoint{4.839228in}{3.506323in}}%
\pgfpathcurveto{\pgfqpoint{4.831414in}{3.498509in}}{\pgfqpoint{4.827024in}{3.487910in}}{\pgfqpoint{4.827024in}{3.476860in}}%
\pgfpathcurveto{\pgfqpoint{4.827024in}{3.465810in}}{\pgfqpoint{4.831414in}{3.455211in}}{\pgfqpoint{4.839228in}{3.447397in}}%
\pgfpathcurveto{\pgfqpoint{4.847042in}{3.439584in}}{\pgfqpoint{4.857641in}{3.435194in}}{\pgfqpoint{4.868691in}{3.435194in}}%
\pgfpathclose%
\pgfusepath{stroke,fill}%
\end{pgfscope}%
\begin{pgfscope}%
\pgfpathrectangle{\pgfqpoint{0.393613in}{0.331635in}}{\pgfqpoint{9.300000in}{7.700000in}}%
\pgfusepath{clip}%
\pgfsetbuttcap%
\pgfsetroundjoin%
\definecolor{currentfill}{rgb}{1.000000,0.705882,0.509804}%
\pgfsetfillcolor{currentfill}%
\pgfsetlinewidth{0.481800pt}%
\definecolor{currentstroke}{rgb}{1.000000,1.000000,1.000000}%
\pgfsetstrokecolor{currentstroke}%
\pgfsetdash{}{0pt}%
\pgfpathmoveto{\pgfqpoint{7.939732in}{4.099002in}}%
\pgfpathcurveto{\pgfqpoint{7.950782in}{4.099002in}}{\pgfqpoint{7.961381in}{4.103392in}}{\pgfqpoint{7.969195in}{4.111205in}}%
\pgfpathcurveto{\pgfqpoint{7.977009in}{4.119019in}}{\pgfqpoint{7.981399in}{4.129618in}}{\pgfqpoint{7.981399in}{4.140668in}}%
\pgfpathcurveto{\pgfqpoint{7.981399in}{4.151718in}}{\pgfqpoint{7.977009in}{4.162317in}}{\pgfqpoint{7.969195in}{4.170131in}}%
\pgfpathcurveto{\pgfqpoint{7.961381in}{4.177945in}}{\pgfqpoint{7.950782in}{4.182335in}}{\pgfqpoint{7.939732in}{4.182335in}}%
\pgfpathcurveto{\pgfqpoint{7.928682in}{4.182335in}}{\pgfqpoint{7.918083in}{4.177945in}}{\pgfqpoint{7.910269in}{4.170131in}}%
\pgfpathcurveto{\pgfqpoint{7.902456in}{4.162317in}}{\pgfqpoint{7.898065in}{4.151718in}}{\pgfqpoint{7.898065in}{4.140668in}}%
\pgfpathcurveto{\pgfqpoint{7.898065in}{4.129618in}}{\pgfqpoint{7.902456in}{4.119019in}}{\pgfqpoint{7.910269in}{4.111205in}}%
\pgfpathcurveto{\pgfqpoint{7.918083in}{4.103392in}}{\pgfqpoint{7.928682in}{4.099002in}}{\pgfqpoint{7.939732in}{4.099002in}}%
\pgfpathclose%
\pgfusepath{stroke,fill}%
\end{pgfscope}%
\begin{pgfscope}%
\pgfpathrectangle{\pgfqpoint{0.393613in}{0.331635in}}{\pgfqpoint{9.300000in}{7.700000in}}%
\pgfusepath{clip}%
\pgfsetbuttcap%
\pgfsetroundjoin%
\definecolor{currentfill}{rgb}{1.000000,0.705882,0.509804}%
\pgfsetfillcolor{currentfill}%
\pgfsetlinewidth{0.481800pt}%
\definecolor{currentstroke}{rgb}{1.000000,1.000000,1.000000}%
\pgfsetstrokecolor{currentstroke}%
\pgfsetdash{}{0pt}%
\pgfpathmoveto{\pgfqpoint{5.975649in}{5.859945in}}%
\pgfpathcurveto{\pgfqpoint{5.986700in}{5.859945in}}{\pgfqpoint{5.997299in}{5.864336in}}{\pgfqpoint{6.005112in}{5.872149in}}%
\pgfpathcurveto{\pgfqpoint{6.012926in}{5.879963in}}{\pgfqpoint{6.017316in}{5.890562in}}{\pgfqpoint{6.017316in}{5.901612in}}%
\pgfpathcurveto{\pgfqpoint{6.017316in}{5.912662in}}{\pgfqpoint{6.012926in}{5.923261in}}{\pgfqpoint{6.005112in}{5.931075in}}%
\pgfpathcurveto{\pgfqpoint{5.997299in}{5.938888in}}{\pgfqpoint{5.986700in}{5.943279in}}{\pgfqpoint{5.975649in}{5.943279in}}%
\pgfpathcurveto{\pgfqpoint{5.964599in}{5.943279in}}{\pgfqpoint{5.954000in}{5.938888in}}{\pgfqpoint{5.946187in}{5.931075in}}%
\pgfpathcurveto{\pgfqpoint{5.938373in}{5.923261in}}{\pgfqpoint{5.933983in}{5.912662in}}{\pgfqpoint{5.933983in}{5.901612in}}%
\pgfpathcurveto{\pgfqpoint{5.933983in}{5.890562in}}{\pgfqpoint{5.938373in}{5.879963in}}{\pgfqpoint{5.946187in}{5.872149in}}%
\pgfpathcurveto{\pgfqpoint{5.954000in}{5.864336in}}{\pgfqpoint{5.964599in}{5.859945in}}{\pgfqpoint{5.975649in}{5.859945in}}%
\pgfpathclose%
\pgfusepath{stroke,fill}%
\end{pgfscope}%
\begin{pgfscope}%
\pgfpathrectangle{\pgfqpoint{0.393613in}{0.331635in}}{\pgfqpoint{9.300000in}{7.700000in}}%
\pgfusepath{clip}%
\pgfsetbuttcap%
\pgfsetroundjoin%
\definecolor{currentfill}{rgb}{1.000000,0.705882,0.509804}%
\pgfsetfillcolor{currentfill}%
\pgfsetlinewidth{0.481800pt}%
\definecolor{currentstroke}{rgb}{1.000000,1.000000,1.000000}%
\pgfsetstrokecolor{currentstroke}%
\pgfsetdash{}{0pt}%
\pgfpathmoveto{\pgfqpoint{5.533563in}{3.205376in}}%
\pgfpathcurveto{\pgfqpoint{5.544613in}{3.205376in}}{\pgfqpoint{5.555212in}{3.209767in}}{\pgfqpoint{5.563026in}{3.217580in}}%
\pgfpathcurveto{\pgfqpoint{5.570840in}{3.225394in}}{\pgfqpoint{5.575230in}{3.235993in}}{\pgfqpoint{5.575230in}{3.247043in}}%
\pgfpathcurveto{\pgfqpoint{5.575230in}{3.258093in}}{\pgfqpoint{5.570840in}{3.268692in}}{\pgfqpoint{5.563026in}{3.276506in}}%
\pgfpathcurveto{\pgfqpoint{5.555212in}{3.284320in}}{\pgfqpoint{5.544613in}{3.288710in}}{\pgfqpoint{5.533563in}{3.288710in}}%
\pgfpathcurveto{\pgfqpoint{5.522513in}{3.288710in}}{\pgfqpoint{5.511914in}{3.284320in}}{\pgfqpoint{5.504100in}{3.276506in}}%
\pgfpathcurveto{\pgfqpoint{5.496287in}{3.268692in}}{\pgfqpoint{5.491897in}{3.258093in}}{\pgfqpoint{5.491897in}{3.247043in}}%
\pgfpathcurveto{\pgfqpoint{5.491897in}{3.235993in}}{\pgfqpoint{5.496287in}{3.225394in}}{\pgfqpoint{5.504100in}{3.217580in}}%
\pgfpathcurveto{\pgfqpoint{5.511914in}{3.209767in}}{\pgfqpoint{5.522513in}{3.205376in}}{\pgfqpoint{5.533563in}{3.205376in}}%
\pgfpathclose%
\pgfusepath{stroke,fill}%
\end{pgfscope}%
\begin{pgfscope}%
\pgfpathrectangle{\pgfqpoint{0.393613in}{0.331635in}}{\pgfqpoint{9.300000in}{7.700000in}}%
\pgfusepath{clip}%
\pgfsetbuttcap%
\pgfsetroundjoin%
\definecolor{currentfill}{rgb}{1.000000,0.705882,0.509804}%
\pgfsetfillcolor{currentfill}%
\pgfsetlinewidth{0.481800pt}%
\definecolor{currentstroke}{rgb}{1.000000,1.000000,1.000000}%
\pgfsetstrokecolor{currentstroke}%
\pgfsetdash{}{0pt}%
\pgfpathmoveto{\pgfqpoint{8.879060in}{4.485403in}}%
\pgfpathcurveto{\pgfqpoint{8.890110in}{4.485403in}}{\pgfqpoint{8.900709in}{4.489794in}}{\pgfqpoint{8.908523in}{4.497607in}}%
\pgfpathcurveto{\pgfqpoint{8.916336in}{4.505421in}}{\pgfqpoint{8.920726in}{4.516020in}}{\pgfqpoint{8.920726in}{4.527070in}}%
\pgfpathcurveto{\pgfqpoint{8.920726in}{4.538120in}}{\pgfqpoint{8.916336in}{4.548719in}}{\pgfqpoint{8.908523in}{4.556533in}}%
\pgfpathcurveto{\pgfqpoint{8.900709in}{4.564347in}}{\pgfqpoint{8.890110in}{4.568737in}}{\pgfqpoint{8.879060in}{4.568737in}}%
\pgfpathcurveto{\pgfqpoint{8.868010in}{4.568737in}}{\pgfqpoint{8.857411in}{4.564347in}}{\pgfqpoint{8.849597in}{4.556533in}}%
\pgfpathcurveto{\pgfqpoint{8.841783in}{4.548719in}}{\pgfqpoint{8.837393in}{4.538120in}}{\pgfqpoint{8.837393in}{4.527070in}}%
\pgfpathcurveto{\pgfqpoint{8.837393in}{4.516020in}}{\pgfqpoint{8.841783in}{4.505421in}}{\pgfqpoint{8.849597in}{4.497607in}}%
\pgfpathcurveto{\pgfqpoint{8.857411in}{4.489794in}}{\pgfqpoint{8.868010in}{4.485403in}}{\pgfqpoint{8.879060in}{4.485403in}}%
\pgfpathclose%
\pgfusepath{stroke,fill}%
\end{pgfscope}%
\begin{pgfscope}%
\pgfpathrectangle{\pgfqpoint{0.393613in}{0.331635in}}{\pgfqpoint{9.300000in}{7.700000in}}%
\pgfusepath{clip}%
\pgfsetbuttcap%
\pgfsetroundjoin%
\definecolor{currentfill}{rgb}{1.000000,0.705882,0.509804}%
\pgfsetfillcolor{currentfill}%
\pgfsetlinewidth{0.481800pt}%
\definecolor{currentstroke}{rgb}{1.000000,1.000000,1.000000}%
\pgfsetstrokecolor{currentstroke}%
\pgfsetdash{}{0pt}%
\pgfpathmoveto{\pgfqpoint{4.426686in}{4.638474in}}%
\pgfpathcurveto{\pgfqpoint{4.437736in}{4.638474in}}{\pgfqpoint{4.448335in}{4.642864in}}{\pgfqpoint{4.456149in}{4.650678in}}%
\pgfpathcurveto{\pgfqpoint{4.463963in}{4.658492in}}{\pgfqpoint{4.468353in}{4.669091in}}{\pgfqpoint{4.468353in}{4.680141in}}%
\pgfpathcurveto{\pgfqpoint{4.468353in}{4.691191in}}{\pgfqpoint{4.463963in}{4.701790in}}{\pgfqpoint{4.456149in}{4.709603in}}%
\pgfpathcurveto{\pgfqpoint{4.448335in}{4.717417in}}{\pgfqpoint{4.437736in}{4.721807in}}{\pgfqpoint{4.426686in}{4.721807in}}%
\pgfpathcurveto{\pgfqpoint{4.415636in}{4.721807in}}{\pgfqpoint{4.405037in}{4.717417in}}{\pgfqpoint{4.397223in}{4.709603in}}%
\pgfpathcurveto{\pgfqpoint{4.389410in}{4.701790in}}{\pgfqpoint{4.385020in}{4.691191in}}{\pgfqpoint{4.385020in}{4.680141in}}%
\pgfpathcurveto{\pgfqpoint{4.385020in}{4.669091in}}{\pgfqpoint{4.389410in}{4.658492in}}{\pgfqpoint{4.397223in}{4.650678in}}%
\pgfpathcurveto{\pgfqpoint{4.405037in}{4.642864in}}{\pgfqpoint{4.415636in}{4.638474in}}{\pgfqpoint{4.426686in}{4.638474in}}%
\pgfpathclose%
\pgfusepath{stroke,fill}%
\end{pgfscope}%
\begin{pgfscope}%
\pgfpathrectangle{\pgfqpoint{0.393613in}{0.331635in}}{\pgfqpoint{9.300000in}{7.700000in}}%
\pgfusepath{clip}%
\pgfsetbuttcap%
\pgfsetroundjoin%
\definecolor{currentfill}{rgb}{1.000000,0.705882,0.509804}%
\pgfsetfillcolor{currentfill}%
\pgfsetlinewidth{0.481800pt}%
\definecolor{currentstroke}{rgb}{1.000000,1.000000,1.000000}%
\pgfsetstrokecolor{currentstroke}%
\pgfsetdash{}{0pt}%
\pgfpathmoveto{\pgfqpoint{6.115172in}{3.487692in}}%
\pgfpathcurveto{\pgfqpoint{6.126222in}{3.487692in}}{\pgfqpoint{6.136821in}{3.492083in}}{\pgfqpoint{6.144635in}{3.499896in}}%
\pgfpathcurveto{\pgfqpoint{6.152449in}{3.507710in}}{\pgfqpoint{6.156839in}{3.518309in}}{\pgfqpoint{6.156839in}{3.529359in}}%
\pgfpathcurveto{\pgfqpoint{6.156839in}{3.540409in}}{\pgfqpoint{6.152449in}{3.551008in}}{\pgfqpoint{6.144635in}{3.558822in}}%
\pgfpathcurveto{\pgfqpoint{6.136821in}{3.566636in}}{\pgfqpoint{6.126222in}{3.571026in}}{\pgfqpoint{6.115172in}{3.571026in}}%
\pgfpathcurveto{\pgfqpoint{6.104122in}{3.571026in}}{\pgfqpoint{6.093523in}{3.566636in}}{\pgfqpoint{6.085709in}{3.558822in}}%
\pgfpathcurveto{\pgfqpoint{6.077896in}{3.551008in}}{\pgfqpoint{6.073506in}{3.540409in}}{\pgfqpoint{6.073506in}{3.529359in}}%
\pgfpathcurveto{\pgfqpoint{6.073506in}{3.518309in}}{\pgfqpoint{6.077896in}{3.507710in}}{\pgfqpoint{6.085709in}{3.499896in}}%
\pgfpathcurveto{\pgfqpoint{6.093523in}{3.492083in}}{\pgfqpoint{6.104122in}{3.487692in}}{\pgfqpoint{6.115172in}{3.487692in}}%
\pgfpathclose%
\pgfusepath{stroke,fill}%
\end{pgfscope}%
\begin{pgfscope}%
\pgfpathrectangle{\pgfqpoint{0.393613in}{0.331635in}}{\pgfqpoint{9.300000in}{7.700000in}}%
\pgfusepath{clip}%
\pgfsetbuttcap%
\pgfsetroundjoin%
\definecolor{currentfill}{rgb}{1.000000,0.705882,0.509804}%
\pgfsetfillcolor{currentfill}%
\pgfsetlinewidth{0.481800pt}%
\definecolor{currentstroke}{rgb}{1.000000,1.000000,1.000000}%
\pgfsetstrokecolor{currentstroke}%
\pgfsetdash{}{0pt}%
\pgfpathmoveto{\pgfqpoint{5.118229in}{2.365376in}}%
\pgfpathcurveto{\pgfqpoint{5.129279in}{2.365376in}}{\pgfqpoint{5.139878in}{2.369767in}}{\pgfqpoint{5.147691in}{2.377580in}}%
\pgfpathcurveto{\pgfqpoint{5.155505in}{2.385394in}}{\pgfqpoint{5.159895in}{2.395993in}}{\pgfqpoint{5.159895in}{2.407043in}}%
\pgfpathcurveto{\pgfqpoint{5.159895in}{2.418093in}}{\pgfqpoint{5.155505in}{2.428692in}}{\pgfqpoint{5.147691in}{2.436506in}}%
\pgfpathcurveto{\pgfqpoint{5.139878in}{2.444319in}}{\pgfqpoint{5.129279in}{2.448710in}}{\pgfqpoint{5.118229in}{2.448710in}}%
\pgfpathcurveto{\pgfqpoint{5.107178in}{2.448710in}}{\pgfqpoint{5.096579in}{2.444319in}}{\pgfqpoint{5.088766in}{2.436506in}}%
\pgfpathcurveto{\pgfqpoint{5.080952in}{2.428692in}}{\pgfqpoint{5.076562in}{2.418093in}}{\pgfqpoint{5.076562in}{2.407043in}}%
\pgfpathcurveto{\pgfqpoint{5.076562in}{2.395993in}}{\pgfqpoint{5.080952in}{2.385394in}}{\pgfqpoint{5.088766in}{2.377580in}}%
\pgfpathcurveto{\pgfqpoint{5.096579in}{2.369767in}}{\pgfqpoint{5.107178in}{2.365376in}}{\pgfqpoint{5.118229in}{2.365376in}}%
\pgfpathclose%
\pgfusepath{stroke,fill}%
\end{pgfscope}%
\begin{pgfscope}%
\pgfpathrectangle{\pgfqpoint{0.393613in}{0.331635in}}{\pgfqpoint{9.300000in}{7.700000in}}%
\pgfusepath{clip}%
\pgfsetbuttcap%
\pgfsetroundjoin%
\definecolor{currentfill}{rgb}{1.000000,0.705882,0.509804}%
\pgfsetfillcolor{currentfill}%
\pgfsetlinewidth{0.481800pt}%
\definecolor{currentstroke}{rgb}{1.000000,1.000000,1.000000}%
\pgfsetstrokecolor{currentstroke}%
\pgfsetdash{}{0pt}%
\pgfpathmoveto{\pgfqpoint{6.785951in}{6.307098in}}%
\pgfpathcurveto{\pgfqpoint{6.797001in}{6.307098in}}{\pgfqpoint{6.807600in}{6.311489in}}{\pgfqpoint{6.815414in}{6.319302in}}%
\pgfpathcurveto{\pgfqpoint{6.823227in}{6.327116in}}{\pgfqpoint{6.827617in}{6.337715in}}{\pgfqpoint{6.827617in}{6.348765in}}%
\pgfpathcurveto{\pgfqpoint{6.827617in}{6.359815in}}{\pgfqpoint{6.823227in}{6.370414in}}{\pgfqpoint{6.815414in}{6.378228in}}%
\pgfpathcurveto{\pgfqpoint{6.807600in}{6.386042in}}{\pgfqpoint{6.797001in}{6.390432in}}{\pgfqpoint{6.785951in}{6.390432in}}%
\pgfpathcurveto{\pgfqpoint{6.774901in}{6.390432in}}{\pgfqpoint{6.764302in}{6.386042in}}{\pgfqpoint{6.756488in}{6.378228in}}%
\pgfpathcurveto{\pgfqpoint{6.748674in}{6.370414in}}{\pgfqpoint{6.744284in}{6.359815in}}{\pgfqpoint{6.744284in}{6.348765in}}%
\pgfpathcurveto{\pgfqpoint{6.744284in}{6.337715in}}{\pgfqpoint{6.748674in}{6.327116in}}{\pgfqpoint{6.756488in}{6.319302in}}%
\pgfpathcurveto{\pgfqpoint{6.764302in}{6.311489in}}{\pgfqpoint{6.774901in}{6.307098in}}{\pgfqpoint{6.785951in}{6.307098in}}%
\pgfpathclose%
\pgfusepath{stroke,fill}%
\end{pgfscope}%
\begin{pgfscope}%
\pgfpathrectangle{\pgfqpoint{0.393613in}{0.331635in}}{\pgfqpoint{9.300000in}{7.700000in}}%
\pgfusepath{clip}%
\pgfsetbuttcap%
\pgfsetroundjoin%
\definecolor{currentfill}{rgb}{1.000000,0.705882,0.509804}%
\pgfsetfillcolor{currentfill}%
\pgfsetlinewidth{0.481800pt}%
\definecolor{currentstroke}{rgb}{1.000000,1.000000,1.000000}%
\pgfsetstrokecolor{currentstroke}%
\pgfsetdash{}{0pt}%
\pgfpathmoveto{\pgfqpoint{4.549543in}{2.828118in}}%
\pgfpathcurveto{\pgfqpoint{4.560593in}{2.828118in}}{\pgfqpoint{4.571192in}{2.832508in}}{\pgfqpoint{4.579006in}{2.840322in}}%
\pgfpathcurveto{\pgfqpoint{4.586820in}{2.848135in}}{\pgfqpoint{4.591210in}{2.858734in}}{\pgfqpoint{4.591210in}{2.869784in}}%
\pgfpathcurveto{\pgfqpoint{4.591210in}{2.880835in}}{\pgfqpoint{4.586820in}{2.891434in}}{\pgfqpoint{4.579006in}{2.899247in}}%
\pgfpathcurveto{\pgfqpoint{4.571192in}{2.907061in}}{\pgfqpoint{4.560593in}{2.911451in}}{\pgfqpoint{4.549543in}{2.911451in}}%
\pgfpathcurveto{\pgfqpoint{4.538493in}{2.911451in}}{\pgfqpoint{4.527894in}{2.907061in}}{\pgfqpoint{4.520080in}{2.899247in}}%
\pgfpathcurveto{\pgfqpoint{4.512267in}{2.891434in}}{\pgfqpoint{4.507876in}{2.880835in}}{\pgfqpoint{4.507876in}{2.869784in}}%
\pgfpathcurveto{\pgfqpoint{4.507876in}{2.858734in}}{\pgfqpoint{4.512267in}{2.848135in}}{\pgfqpoint{4.520080in}{2.840322in}}%
\pgfpathcurveto{\pgfqpoint{4.527894in}{2.832508in}}{\pgfqpoint{4.538493in}{2.828118in}}{\pgfqpoint{4.549543in}{2.828118in}}%
\pgfpathclose%
\pgfusepath{stroke,fill}%
\end{pgfscope}%
\begin{pgfscope}%
\pgfpathrectangle{\pgfqpoint{0.393613in}{0.331635in}}{\pgfqpoint{9.300000in}{7.700000in}}%
\pgfusepath{clip}%
\pgfsetbuttcap%
\pgfsetroundjoin%
\definecolor{currentfill}{rgb}{1.000000,0.705882,0.509804}%
\pgfsetfillcolor{currentfill}%
\pgfsetlinewidth{0.481800pt}%
\definecolor{currentstroke}{rgb}{1.000000,1.000000,1.000000}%
\pgfsetstrokecolor{currentstroke}%
\pgfsetdash{}{0pt}%
\pgfpathmoveto{\pgfqpoint{7.362784in}{4.414752in}}%
\pgfpathcurveto{\pgfqpoint{7.373834in}{4.414752in}}{\pgfqpoint{7.384433in}{4.419142in}}{\pgfqpoint{7.392247in}{4.426956in}}%
\pgfpathcurveto{\pgfqpoint{7.400060in}{4.434770in}}{\pgfqpoint{7.404451in}{4.445369in}}{\pgfqpoint{7.404451in}{4.456419in}}%
\pgfpathcurveto{\pgfqpoint{7.404451in}{4.467469in}}{\pgfqpoint{7.400060in}{4.478068in}}{\pgfqpoint{7.392247in}{4.485881in}}%
\pgfpathcurveto{\pgfqpoint{7.384433in}{4.493695in}}{\pgfqpoint{7.373834in}{4.498085in}}{\pgfqpoint{7.362784in}{4.498085in}}%
\pgfpathcurveto{\pgfqpoint{7.351734in}{4.498085in}}{\pgfqpoint{7.341135in}{4.493695in}}{\pgfqpoint{7.333321in}{4.485881in}}%
\pgfpathcurveto{\pgfqpoint{7.325508in}{4.478068in}}{\pgfqpoint{7.321117in}{4.467469in}}{\pgfqpoint{7.321117in}{4.456419in}}%
\pgfpathcurveto{\pgfqpoint{7.321117in}{4.445369in}}{\pgfqpoint{7.325508in}{4.434770in}}{\pgfqpoint{7.333321in}{4.426956in}}%
\pgfpathcurveto{\pgfqpoint{7.341135in}{4.419142in}}{\pgfqpoint{7.351734in}{4.414752in}}{\pgfqpoint{7.362784in}{4.414752in}}%
\pgfpathclose%
\pgfusepath{stroke,fill}%
\end{pgfscope}%
\begin{pgfscope}%
\pgfpathrectangle{\pgfqpoint{0.393613in}{0.331635in}}{\pgfqpoint{9.300000in}{7.700000in}}%
\pgfusepath{clip}%
\pgfsetbuttcap%
\pgfsetroundjoin%
\definecolor{currentfill}{rgb}{1.000000,0.705882,0.509804}%
\pgfsetfillcolor{currentfill}%
\pgfsetlinewidth{0.481800pt}%
\definecolor{currentstroke}{rgb}{1.000000,1.000000,1.000000}%
\pgfsetstrokecolor{currentstroke}%
\pgfsetdash{}{0pt}%
\pgfpathmoveto{\pgfqpoint{3.789706in}{4.934223in}}%
\pgfpathcurveto{\pgfqpoint{3.800756in}{4.934223in}}{\pgfqpoint{3.811355in}{4.938613in}}{\pgfqpoint{3.819168in}{4.946427in}}%
\pgfpathcurveto{\pgfqpoint{3.826982in}{4.954240in}}{\pgfqpoint{3.831372in}{4.964839in}}{\pgfqpoint{3.831372in}{4.975890in}}%
\pgfpathcurveto{\pgfqpoint{3.831372in}{4.986940in}}{\pgfqpoint{3.826982in}{4.997539in}}{\pgfqpoint{3.819168in}{5.005352in}}%
\pgfpathcurveto{\pgfqpoint{3.811355in}{5.013166in}}{\pgfqpoint{3.800756in}{5.017556in}}{\pgfqpoint{3.789706in}{5.017556in}}%
\pgfpathcurveto{\pgfqpoint{3.778655in}{5.017556in}}{\pgfqpoint{3.768056in}{5.013166in}}{\pgfqpoint{3.760243in}{5.005352in}}%
\pgfpathcurveto{\pgfqpoint{3.752429in}{4.997539in}}{\pgfqpoint{3.748039in}{4.986940in}}{\pgfqpoint{3.748039in}{4.975890in}}%
\pgfpathcurveto{\pgfqpoint{3.748039in}{4.964839in}}{\pgfqpoint{3.752429in}{4.954240in}}{\pgfqpoint{3.760243in}{4.946427in}}%
\pgfpathcurveto{\pgfqpoint{3.768056in}{4.938613in}}{\pgfqpoint{3.778655in}{4.934223in}}{\pgfqpoint{3.789706in}{4.934223in}}%
\pgfpathclose%
\pgfusepath{stroke,fill}%
\end{pgfscope}%
\begin{pgfscope}%
\pgfpathrectangle{\pgfqpoint{0.393613in}{0.331635in}}{\pgfqpoint{9.300000in}{7.700000in}}%
\pgfusepath{clip}%
\pgfsetbuttcap%
\pgfsetroundjoin%
\definecolor{currentfill}{rgb}{0.631373,0.788235,0.956863}%
\pgfsetfillcolor{currentfill}%
\pgfsetlinewidth{1.003750pt}%
\definecolor{currentstroke}{rgb}{0.631373,0.788235,0.956863}%
\pgfsetstrokecolor{currentstroke}%
\pgfsetdash{}{0pt}%
\pgfsys@defobject{currentmarker}{\pgfqpoint{-0.041667in}{-0.041667in}}{\pgfqpoint{0.041667in}{0.041667in}}{%
\pgfpathmoveto{\pgfqpoint{0.000000in}{-0.041667in}}%
\pgfpathcurveto{\pgfqpoint{0.011050in}{-0.041667in}}{\pgfqpoint{0.021649in}{-0.037276in}}{\pgfqpoint{0.029463in}{-0.029463in}}%
\pgfpathcurveto{\pgfqpoint{0.037276in}{-0.021649in}}{\pgfqpoint{0.041667in}{-0.011050in}}{\pgfqpoint{0.041667in}{0.000000in}}%
\pgfpathcurveto{\pgfqpoint{0.041667in}{0.011050in}}{\pgfqpoint{0.037276in}{0.021649in}}{\pgfqpoint{0.029463in}{0.029463in}}%
\pgfpathcurveto{\pgfqpoint{0.021649in}{0.037276in}}{\pgfqpoint{0.011050in}{0.041667in}}{\pgfqpoint{0.000000in}{0.041667in}}%
\pgfpathcurveto{\pgfqpoint{-0.011050in}{0.041667in}}{\pgfqpoint{-0.021649in}{0.037276in}}{\pgfqpoint{-0.029463in}{0.029463in}}%
\pgfpathcurveto{\pgfqpoint{-0.037276in}{0.021649in}}{\pgfqpoint{-0.041667in}{0.011050in}}{\pgfqpoint{-0.041667in}{0.000000in}}%
\pgfpathcurveto{\pgfqpoint{-0.041667in}{-0.011050in}}{\pgfqpoint{-0.037276in}{-0.021649in}}{\pgfqpoint{-0.029463in}{-0.029463in}}%
\pgfpathcurveto{\pgfqpoint{-0.021649in}{-0.037276in}}{\pgfqpoint{-0.011050in}{-0.041667in}}{\pgfqpoint{0.000000in}{-0.041667in}}%
\pgfpathclose%
\pgfusepath{stroke,fill}%
}%
\end{pgfscope}%
\begin{pgfscope}%
\pgfpathrectangle{\pgfqpoint{0.393613in}{0.331635in}}{\pgfqpoint{9.300000in}{7.700000in}}%
\pgfusepath{clip}%
\pgfsetbuttcap%
\pgfsetroundjoin%
\definecolor{currentfill}{rgb}{1.000000,0.705882,0.509804}%
\pgfsetfillcolor{currentfill}%
\pgfsetlinewidth{1.003750pt}%
\definecolor{currentstroke}{rgb}{1.000000,0.705882,0.509804}%
\pgfsetstrokecolor{currentstroke}%
\pgfsetdash{}{0pt}%
\pgfsys@defobject{currentmarker}{\pgfqpoint{-0.041667in}{-0.041667in}}{\pgfqpoint{0.041667in}{0.041667in}}{%
\pgfpathmoveto{\pgfqpoint{0.000000in}{-0.041667in}}%
\pgfpathcurveto{\pgfqpoint{0.011050in}{-0.041667in}}{\pgfqpoint{0.021649in}{-0.037276in}}{\pgfqpoint{0.029463in}{-0.029463in}}%
\pgfpathcurveto{\pgfqpoint{0.037276in}{-0.021649in}}{\pgfqpoint{0.041667in}{-0.011050in}}{\pgfqpoint{0.041667in}{0.000000in}}%
\pgfpathcurveto{\pgfqpoint{0.041667in}{0.011050in}}{\pgfqpoint{0.037276in}{0.021649in}}{\pgfqpoint{0.029463in}{0.029463in}}%
\pgfpathcurveto{\pgfqpoint{0.021649in}{0.037276in}}{\pgfqpoint{0.011050in}{0.041667in}}{\pgfqpoint{0.000000in}{0.041667in}}%
\pgfpathcurveto{\pgfqpoint{-0.011050in}{0.041667in}}{\pgfqpoint{-0.021649in}{0.037276in}}{\pgfqpoint{-0.029463in}{0.029463in}}%
\pgfpathcurveto{\pgfqpoint{-0.037276in}{0.021649in}}{\pgfqpoint{-0.041667in}{0.011050in}}{\pgfqpoint{-0.041667in}{0.000000in}}%
\pgfpathcurveto{\pgfqpoint{-0.041667in}{-0.011050in}}{\pgfqpoint{-0.037276in}{-0.021649in}}{\pgfqpoint{-0.029463in}{-0.029463in}}%
\pgfpathcurveto{\pgfqpoint{-0.021649in}{-0.037276in}}{\pgfqpoint{-0.011050in}{-0.041667in}}{\pgfqpoint{0.000000in}{-0.041667in}}%
\pgfpathclose%
\pgfusepath{stroke,fill}%
}%
\end{pgfscope}%
\begin{pgfscope}%
\pgfsetbuttcap%
\pgfsetroundjoin%
\definecolor{currentfill}{rgb}{0.000000,0.000000,0.000000}%
\pgfsetfillcolor{currentfill}%
\pgfsetlinewidth{0.803000pt}%
\definecolor{currentstroke}{rgb}{0.000000,0.000000,0.000000}%
\pgfsetstrokecolor{currentstroke}%
\pgfsetdash{}{0pt}%
\pgfsys@defobject{currentmarker}{\pgfqpoint{0.000000in}{-0.048611in}}{\pgfqpoint{0.000000in}{0.000000in}}{%
\pgfpathmoveto{\pgfqpoint{0.000000in}{0.000000in}}%
\pgfpathlineto{\pgfqpoint{0.000000in}{-0.048611in}}%
\pgfusepath{stroke,fill}%
}%
\begin{pgfscope}%
\pgfsys@transformshift{1.503282in}{0.331635in}%
\pgfsys@useobject{currentmarker}{}%
\end{pgfscope}%
\end{pgfscope}%
\begin{pgfscope}%
\definecolor{textcolor}{rgb}{0.000000,0.000000,0.000000}%
\pgfsetstrokecolor{textcolor}%
\pgfsetfillcolor{textcolor}%
\pgftext[x=1.503282in,y=0.234413in,,top]{\color{textcolor}\sffamily\fontsize{10.000000}{12.000000}\selectfont \ensuremath{-}10}%
\end{pgfscope}%
\begin{pgfscope}%
\pgfsetbuttcap%
\pgfsetroundjoin%
\definecolor{currentfill}{rgb}{0.000000,0.000000,0.000000}%
\pgfsetfillcolor{currentfill}%
\pgfsetlinewidth{0.803000pt}%
\definecolor{currentstroke}{rgb}{0.000000,0.000000,0.000000}%
\pgfsetstrokecolor{currentstroke}%
\pgfsetdash{}{0pt}%
\pgfsys@defobject{currentmarker}{\pgfqpoint{0.000000in}{-0.048611in}}{\pgfqpoint{0.000000in}{0.000000in}}{%
\pgfpathmoveto{\pgfqpoint{0.000000in}{0.000000in}}%
\pgfpathlineto{\pgfqpoint{0.000000in}{-0.048611in}}%
\pgfusepath{stroke,fill}%
}%
\begin{pgfscope}%
\pgfsys@transformshift{2.759833in}{0.331635in}%
\pgfsys@useobject{currentmarker}{}%
\end{pgfscope}%
\end{pgfscope}%
\begin{pgfscope}%
\definecolor{textcolor}{rgb}{0.000000,0.000000,0.000000}%
\pgfsetstrokecolor{textcolor}%
\pgfsetfillcolor{textcolor}%
\pgftext[x=2.759833in,y=0.234413in,,top]{\color{textcolor}\sffamily\fontsize{10.000000}{12.000000}\selectfont \ensuremath{-}8}%
\end{pgfscope}%
\begin{pgfscope}%
\pgfsetbuttcap%
\pgfsetroundjoin%
\definecolor{currentfill}{rgb}{0.000000,0.000000,0.000000}%
\pgfsetfillcolor{currentfill}%
\pgfsetlinewidth{0.803000pt}%
\definecolor{currentstroke}{rgb}{0.000000,0.000000,0.000000}%
\pgfsetstrokecolor{currentstroke}%
\pgfsetdash{}{0pt}%
\pgfsys@defobject{currentmarker}{\pgfqpoint{0.000000in}{-0.048611in}}{\pgfqpoint{0.000000in}{0.000000in}}{%
\pgfpathmoveto{\pgfqpoint{0.000000in}{0.000000in}}%
\pgfpathlineto{\pgfqpoint{0.000000in}{-0.048611in}}%
\pgfusepath{stroke,fill}%
}%
\begin{pgfscope}%
\pgfsys@transformshift{4.016384in}{0.331635in}%
\pgfsys@useobject{currentmarker}{}%
\end{pgfscope}%
\end{pgfscope}%
\begin{pgfscope}%
\definecolor{textcolor}{rgb}{0.000000,0.000000,0.000000}%
\pgfsetstrokecolor{textcolor}%
\pgfsetfillcolor{textcolor}%
\pgftext[x=4.016384in,y=0.234413in,,top]{\color{textcolor}\sffamily\fontsize{10.000000}{12.000000}\selectfont \ensuremath{-}6}%
\end{pgfscope}%
\begin{pgfscope}%
\pgfsetbuttcap%
\pgfsetroundjoin%
\definecolor{currentfill}{rgb}{0.000000,0.000000,0.000000}%
\pgfsetfillcolor{currentfill}%
\pgfsetlinewidth{0.803000pt}%
\definecolor{currentstroke}{rgb}{0.000000,0.000000,0.000000}%
\pgfsetstrokecolor{currentstroke}%
\pgfsetdash{}{0pt}%
\pgfsys@defobject{currentmarker}{\pgfqpoint{0.000000in}{-0.048611in}}{\pgfqpoint{0.000000in}{0.000000in}}{%
\pgfpathmoveto{\pgfqpoint{0.000000in}{0.000000in}}%
\pgfpathlineto{\pgfqpoint{0.000000in}{-0.048611in}}%
\pgfusepath{stroke,fill}%
}%
\begin{pgfscope}%
\pgfsys@transformshift{5.272935in}{0.331635in}%
\pgfsys@useobject{currentmarker}{}%
\end{pgfscope}%
\end{pgfscope}%
\begin{pgfscope}%
\definecolor{textcolor}{rgb}{0.000000,0.000000,0.000000}%
\pgfsetstrokecolor{textcolor}%
\pgfsetfillcolor{textcolor}%
\pgftext[x=5.272935in,y=0.234413in,,top]{\color{textcolor}\sffamily\fontsize{10.000000}{12.000000}\selectfont \ensuremath{-}4}%
\end{pgfscope}%
\begin{pgfscope}%
\pgfsetbuttcap%
\pgfsetroundjoin%
\definecolor{currentfill}{rgb}{0.000000,0.000000,0.000000}%
\pgfsetfillcolor{currentfill}%
\pgfsetlinewidth{0.803000pt}%
\definecolor{currentstroke}{rgb}{0.000000,0.000000,0.000000}%
\pgfsetstrokecolor{currentstroke}%
\pgfsetdash{}{0pt}%
\pgfsys@defobject{currentmarker}{\pgfqpoint{0.000000in}{-0.048611in}}{\pgfqpoint{0.000000in}{0.000000in}}{%
\pgfpathmoveto{\pgfqpoint{0.000000in}{0.000000in}}%
\pgfpathlineto{\pgfqpoint{0.000000in}{-0.048611in}}%
\pgfusepath{stroke,fill}%
}%
\begin{pgfscope}%
\pgfsys@transformshift{6.529487in}{0.331635in}%
\pgfsys@useobject{currentmarker}{}%
\end{pgfscope}%
\end{pgfscope}%
\begin{pgfscope}%
\definecolor{textcolor}{rgb}{0.000000,0.000000,0.000000}%
\pgfsetstrokecolor{textcolor}%
\pgfsetfillcolor{textcolor}%
\pgftext[x=6.529487in,y=0.234413in,,top]{\color{textcolor}\sffamily\fontsize{10.000000}{12.000000}\selectfont \ensuremath{-}2}%
\end{pgfscope}%
\begin{pgfscope}%
\pgfsetbuttcap%
\pgfsetroundjoin%
\definecolor{currentfill}{rgb}{0.000000,0.000000,0.000000}%
\pgfsetfillcolor{currentfill}%
\pgfsetlinewidth{0.803000pt}%
\definecolor{currentstroke}{rgb}{0.000000,0.000000,0.000000}%
\pgfsetstrokecolor{currentstroke}%
\pgfsetdash{}{0pt}%
\pgfsys@defobject{currentmarker}{\pgfqpoint{0.000000in}{-0.048611in}}{\pgfqpoint{0.000000in}{0.000000in}}{%
\pgfpathmoveto{\pgfqpoint{0.000000in}{0.000000in}}%
\pgfpathlineto{\pgfqpoint{0.000000in}{-0.048611in}}%
\pgfusepath{stroke,fill}%
}%
\begin{pgfscope}%
\pgfsys@transformshift{7.786038in}{0.331635in}%
\pgfsys@useobject{currentmarker}{}%
\end{pgfscope}%
\end{pgfscope}%
\begin{pgfscope}%
\definecolor{textcolor}{rgb}{0.000000,0.000000,0.000000}%
\pgfsetstrokecolor{textcolor}%
\pgfsetfillcolor{textcolor}%
\pgftext[x=7.786038in,y=0.234413in,,top]{\color{textcolor}\sffamily\fontsize{10.000000}{12.000000}\selectfont 0}%
\end{pgfscope}%
\begin{pgfscope}%
\pgfsetbuttcap%
\pgfsetroundjoin%
\definecolor{currentfill}{rgb}{0.000000,0.000000,0.000000}%
\pgfsetfillcolor{currentfill}%
\pgfsetlinewidth{0.803000pt}%
\definecolor{currentstroke}{rgb}{0.000000,0.000000,0.000000}%
\pgfsetstrokecolor{currentstroke}%
\pgfsetdash{}{0pt}%
\pgfsys@defobject{currentmarker}{\pgfqpoint{0.000000in}{-0.048611in}}{\pgfqpoint{0.000000in}{0.000000in}}{%
\pgfpathmoveto{\pgfqpoint{0.000000in}{0.000000in}}%
\pgfpathlineto{\pgfqpoint{0.000000in}{-0.048611in}}%
\pgfusepath{stroke,fill}%
}%
\begin{pgfscope}%
\pgfsys@transformshift{9.042589in}{0.331635in}%
\pgfsys@useobject{currentmarker}{}%
\end{pgfscope}%
\end{pgfscope}%
\begin{pgfscope}%
\definecolor{textcolor}{rgb}{0.000000,0.000000,0.000000}%
\pgfsetstrokecolor{textcolor}%
\pgfsetfillcolor{textcolor}%
\pgftext[x=9.042589in,y=0.234413in,,top]{\color{textcolor}\sffamily\fontsize{10.000000}{12.000000}\selectfont 2}%
\end{pgfscope}%
\begin{pgfscope}%
\pgfsetbuttcap%
\pgfsetroundjoin%
\definecolor{currentfill}{rgb}{0.000000,0.000000,0.000000}%
\pgfsetfillcolor{currentfill}%
\pgfsetlinewidth{0.803000pt}%
\definecolor{currentstroke}{rgb}{0.000000,0.000000,0.000000}%
\pgfsetstrokecolor{currentstroke}%
\pgfsetdash{}{0pt}%
\pgfsys@defobject{currentmarker}{\pgfqpoint{-0.048611in}{0.000000in}}{\pgfqpoint{-0.000000in}{0.000000in}}{%
\pgfpathmoveto{\pgfqpoint{-0.000000in}{0.000000in}}%
\pgfpathlineto{\pgfqpoint{-0.048611in}{0.000000in}}%
\pgfusepath{stroke,fill}%
}%
\begin{pgfscope}%
\pgfsys@transformshift{0.393613in}{0.678543in}%
\pgfsys@useobject{currentmarker}{}%
\end{pgfscope}%
\end{pgfscope}%
\begin{pgfscope}%
\definecolor{textcolor}{rgb}{0.000000,0.000000,0.000000}%
\pgfsetstrokecolor{textcolor}%
\pgfsetfillcolor{textcolor}%
\pgftext[x=0.100000in, y=0.625781in, left, base]{\color{textcolor}\sffamily\fontsize{10.000000}{12.000000}\selectfont \ensuremath{-}2}%
\end{pgfscope}%
\begin{pgfscope}%
\pgfsetbuttcap%
\pgfsetroundjoin%
\definecolor{currentfill}{rgb}{0.000000,0.000000,0.000000}%
\pgfsetfillcolor{currentfill}%
\pgfsetlinewidth{0.803000pt}%
\definecolor{currentstroke}{rgb}{0.000000,0.000000,0.000000}%
\pgfsetstrokecolor{currentstroke}%
\pgfsetdash{}{0pt}%
\pgfsys@defobject{currentmarker}{\pgfqpoint{-0.048611in}{0.000000in}}{\pgfqpoint{-0.000000in}{0.000000in}}{%
\pgfpathmoveto{\pgfqpoint{-0.000000in}{0.000000in}}%
\pgfpathlineto{\pgfqpoint{-0.048611in}{0.000000in}}%
\pgfusepath{stroke,fill}%
}%
\begin{pgfscope}%
\pgfsys@transformshift{0.393613in}{2.018524in}%
\pgfsys@useobject{currentmarker}{}%
\end{pgfscope}%
\end{pgfscope}%
\begin{pgfscope}%
\definecolor{textcolor}{rgb}{0.000000,0.000000,0.000000}%
\pgfsetstrokecolor{textcolor}%
\pgfsetfillcolor{textcolor}%
\pgftext[x=0.208025in, y=1.965762in, left, base]{\color{textcolor}\sffamily\fontsize{10.000000}{12.000000}\selectfont 0}%
\end{pgfscope}%
\begin{pgfscope}%
\pgfsetbuttcap%
\pgfsetroundjoin%
\definecolor{currentfill}{rgb}{0.000000,0.000000,0.000000}%
\pgfsetfillcolor{currentfill}%
\pgfsetlinewidth{0.803000pt}%
\definecolor{currentstroke}{rgb}{0.000000,0.000000,0.000000}%
\pgfsetstrokecolor{currentstroke}%
\pgfsetdash{}{0pt}%
\pgfsys@defobject{currentmarker}{\pgfqpoint{-0.048611in}{0.000000in}}{\pgfqpoint{-0.000000in}{0.000000in}}{%
\pgfpathmoveto{\pgfqpoint{-0.000000in}{0.000000in}}%
\pgfpathlineto{\pgfqpoint{-0.048611in}{0.000000in}}%
\pgfusepath{stroke,fill}%
}%
\begin{pgfscope}%
\pgfsys@transformshift{0.393613in}{3.358505in}%
\pgfsys@useobject{currentmarker}{}%
\end{pgfscope}%
\end{pgfscope}%
\begin{pgfscope}%
\definecolor{textcolor}{rgb}{0.000000,0.000000,0.000000}%
\pgfsetstrokecolor{textcolor}%
\pgfsetfillcolor{textcolor}%
\pgftext[x=0.208025in, y=3.305743in, left, base]{\color{textcolor}\sffamily\fontsize{10.000000}{12.000000}\selectfont 2}%
\end{pgfscope}%
\begin{pgfscope}%
\pgfsetbuttcap%
\pgfsetroundjoin%
\definecolor{currentfill}{rgb}{0.000000,0.000000,0.000000}%
\pgfsetfillcolor{currentfill}%
\pgfsetlinewidth{0.803000pt}%
\definecolor{currentstroke}{rgb}{0.000000,0.000000,0.000000}%
\pgfsetstrokecolor{currentstroke}%
\pgfsetdash{}{0pt}%
\pgfsys@defobject{currentmarker}{\pgfqpoint{-0.048611in}{0.000000in}}{\pgfqpoint{-0.000000in}{0.000000in}}{%
\pgfpathmoveto{\pgfqpoint{-0.000000in}{0.000000in}}%
\pgfpathlineto{\pgfqpoint{-0.048611in}{0.000000in}}%
\pgfusepath{stroke,fill}%
}%
\begin{pgfscope}%
\pgfsys@transformshift{0.393613in}{4.698486in}%
\pgfsys@useobject{currentmarker}{}%
\end{pgfscope}%
\end{pgfscope}%
\begin{pgfscope}%
\definecolor{textcolor}{rgb}{0.000000,0.000000,0.000000}%
\pgfsetstrokecolor{textcolor}%
\pgfsetfillcolor{textcolor}%
\pgftext[x=0.208025in, y=4.645724in, left, base]{\color{textcolor}\sffamily\fontsize{10.000000}{12.000000}\selectfont 4}%
\end{pgfscope}%
\begin{pgfscope}%
\pgfsetbuttcap%
\pgfsetroundjoin%
\definecolor{currentfill}{rgb}{0.000000,0.000000,0.000000}%
\pgfsetfillcolor{currentfill}%
\pgfsetlinewidth{0.803000pt}%
\definecolor{currentstroke}{rgb}{0.000000,0.000000,0.000000}%
\pgfsetstrokecolor{currentstroke}%
\pgfsetdash{}{0pt}%
\pgfsys@defobject{currentmarker}{\pgfqpoint{-0.048611in}{0.000000in}}{\pgfqpoint{-0.000000in}{0.000000in}}{%
\pgfpathmoveto{\pgfqpoint{-0.000000in}{0.000000in}}%
\pgfpathlineto{\pgfqpoint{-0.048611in}{0.000000in}}%
\pgfusepath{stroke,fill}%
}%
\begin{pgfscope}%
\pgfsys@transformshift{0.393613in}{6.038467in}%
\pgfsys@useobject{currentmarker}{}%
\end{pgfscope}%
\end{pgfscope}%
\begin{pgfscope}%
\definecolor{textcolor}{rgb}{0.000000,0.000000,0.000000}%
\pgfsetstrokecolor{textcolor}%
\pgfsetfillcolor{textcolor}%
\pgftext[x=0.208025in, y=5.985705in, left, base]{\color{textcolor}\sffamily\fontsize{10.000000}{12.000000}\selectfont 6}%
\end{pgfscope}%
\begin{pgfscope}%
\pgfsetbuttcap%
\pgfsetroundjoin%
\definecolor{currentfill}{rgb}{0.000000,0.000000,0.000000}%
\pgfsetfillcolor{currentfill}%
\pgfsetlinewidth{0.803000pt}%
\definecolor{currentstroke}{rgb}{0.000000,0.000000,0.000000}%
\pgfsetstrokecolor{currentstroke}%
\pgfsetdash{}{0pt}%
\pgfsys@defobject{currentmarker}{\pgfqpoint{-0.048611in}{0.000000in}}{\pgfqpoint{-0.000000in}{0.000000in}}{%
\pgfpathmoveto{\pgfqpoint{-0.000000in}{0.000000in}}%
\pgfpathlineto{\pgfqpoint{-0.048611in}{0.000000in}}%
\pgfusepath{stroke,fill}%
}%
\begin{pgfscope}%
\pgfsys@transformshift{0.393613in}{7.378448in}%
\pgfsys@useobject{currentmarker}{}%
\end{pgfscope}%
\end{pgfscope}%
\begin{pgfscope}%
\definecolor{textcolor}{rgb}{0.000000,0.000000,0.000000}%
\pgfsetstrokecolor{textcolor}%
\pgfsetfillcolor{textcolor}%
\pgftext[x=0.208025in, y=7.325687in, left, base]{\color{textcolor}\sffamily\fontsize{10.000000}{12.000000}\selectfont 8}%
\end{pgfscope}%
\begin{pgfscope}%
\pgfpathrectangle{\pgfqpoint{0.393613in}{0.331635in}}{\pgfqpoint{9.300000in}{7.700000in}}%
\pgfusepath{clip}%
\pgfsetrectcap%
\pgfsetroundjoin%
\pgfsetlinewidth{1.505625pt}%
\definecolor{currentstroke}{rgb}{0.631373,0.788235,0.956863}%
\pgfsetstrokecolor{currentstroke}%
\pgfsetstrokeopacity{0.800000}%
\pgfsetdash{}{0pt}%
\pgfpathmoveto{\pgfqpoint{2.878164in}{1.558141in}}%
\pgfpathlineto{\pgfqpoint{2.941011in}{4.613488in}}%
\pgfusepath{stroke}%
\end{pgfscope}%
\begin{pgfscope}%
\pgfpathrectangle{\pgfqpoint{0.393613in}{0.331635in}}{\pgfqpoint{9.300000in}{7.700000in}}%
\pgfusepath{clip}%
\pgfsetrectcap%
\pgfsetroundjoin%
\pgfsetlinewidth{1.505625pt}%
\definecolor{currentstroke}{rgb}{0.631373,0.788235,0.956863}%
\pgfsetstrokecolor{currentstroke}%
\pgfsetstrokeopacity{0.800000}%
\pgfsetdash{}{0pt}%
\pgfpathmoveto{\pgfqpoint{1.610944in}{5.142869in}}%
\pgfpathlineto{\pgfqpoint{2.941011in}{4.613488in}}%
\pgfusepath{stroke}%
\end{pgfscope}%
\begin{pgfscope}%
\pgfpathrectangle{\pgfqpoint{0.393613in}{0.331635in}}{\pgfqpoint{9.300000in}{7.700000in}}%
\pgfusepath{clip}%
\pgfsetrectcap%
\pgfsetroundjoin%
\pgfsetlinewidth{1.505625pt}%
\definecolor{currentstroke}{rgb}{0.631373,0.788235,0.956863}%
\pgfsetstrokecolor{currentstroke}%
\pgfsetstrokeopacity{0.800000}%
\pgfsetdash{}{0pt}%
\pgfpathmoveto{\pgfqpoint{4.136764in}{5.889937in}}%
\pgfpathlineto{\pgfqpoint{2.941011in}{4.613488in}}%
\pgfusepath{stroke}%
\end{pgfscope}%
\begin{pgfscope}%
\pgfpathrectangle{\pgfqpoint{0.393613in}{0.331635in}}{\pgfqpoint{9.300000in}{7.700000in}}%
\pgfusepath{clip}%
\pgfsetrectcap%
\pgfsetroundjoin%
\pgfsetlinewidth{1.505625pt}%
\definecolor{currentstroke}{rgb}{0.631373,0.788235,0.956863}%
\pgfsetstrokecolor{currentstroke}%
\pgfsetstrokeopacity{0.800000}%
\pgfsetdash{}{0pt}%
\pgfpathmoveto{\pgfqpoint{3.118402in}{4.388061in}}%
\pgfpathlineto{\pgfqpoint{2.941011in}{4.613488in}}%
\pgfusepath{stroke}%
\end{pgfscope}%
\begin{pgfscope}%
\pgfpathrectangle{\pgfqpoint{0.393613in}{0.331635in}}{\pgfqpoint{9.300000in}{7.700000in}}%
\pgfusepath{clip}%
\pgfsetrectcap%
\pgfsetroundjoin%
\pgfsetlinewidth{1.505625pt}%
\definecolor{currentstroke}{rgb}{0.631373,0.788235,0.956863}%
\pgfsetstrokecolor{currentstroke}%
\pgfsetstrokeopacity{0.800000}%
\pgfsetdash{}{0pt}%
\pgfpathmoveto{\pgfqpoint{3.792263in}{2.017332in}}%
\pgfpathlineto{\pgfqpoint{2.941011in}{4.613488in}}%
\pgfusepath{stroke}%
\end{pgfscope}%
\begin{pgfscope}%
\pgfpathrectangle{\pgfqpoint{0.393613in}{0.331635in}}{\pgfqpoint{9.300000in}{7.700000in}}%
\pgfusepath{clip}%
\pgfsetrectcap%
\pgfsetroundjoin%
\pgfsetlinewidth{1.505625pt}%
\definecolor{currentstroke}{rgb}{0.631373,0.788235,0.956863}%
\pgfsetstrokecolor{currentstroke}%
\pgfsetstrokeopacity{0.800000}%
\pgfsetdash{}{0pt}%
\pgfpathmoveto{\pgfqpoint{1.145785in}{5.847310in}}%
\pgfpathlineto{\pgfqpoint{2.941011in}{4.613488in}}%
\pgfusepath{stroke}%
\end{pgfscope}%
\begin{pgfscope}%
\pgfpathrectangle{\pgfqpoint{0.393613in}{0.331635in}}{\pgfqpoint{9.300000in}{7.700000in}}%
\pgfusepath{clip}%
\pgfsetrectcap%
\pgfsetroundjoin%
\pgfsetlinewidth{1.505625pt}%
\definecolor{currentstroke}{rgb}{0.631373,0.788235,0.956863}%
\pgfsetstrokecolor{currentstroke}%
\pgfsetstrokeopacity{0.800000}%
\pgfsetdash{}{0pt}%
\pgfpathmoveto{\pgfqpoint{2.952892in}{1.933973in}}%
\pgfpathlineto{\pgfqpoint{2.941011in}{4.613488in}}%
\pgfusepath{stroke}%
\end{pgfscope}%
\begin{pgfscope}%
\pgfpathrectangle{\pgfqpoint{0.393613in}{0.331635in}}{\pgfqpoint{9.300000in}{7.700000in}}%
\pgfusepath{clip}%
\pgfsetrectcap%
\pgfsetroundjoin%
\pgfsetlinewidth{1.505625pt}%
\definecolor{currentstroke}{rgb}{0.631373,0.788235,0.956863}%
\pgfsetstrokecolor{currentstroke}%
\pgfsetstrokeopacity{0.800000}%
\pgfsetdash{}{0pt}%
\pgfpathmoveto{\pgfqpoint{4.435471in}{5.567747in}}%
\pgfpathlineto{\pgfqpoint{2.941011in}{4.613488in}}%
\pgfusepath{stroke}%
\end{pgfscope}%
\begin{pgfscope}%
\pgfpathrectangle{\pgfqpoint{0.393613in}{0.331635in}}{\pgfqpoint{9.300000in}{7.700000in}}%
\pgfusepath{clip}%
\pgfsetrectcap%
\pgfsetroundjoin%
\pgfsetlinewidth{1.505625pt}%
\definecolor{currentstroke}{rgb}{0.631373,0.788235,0.956863}%
\pgfsetstrokecolor{currentstroke}%
\pgfsetstrokeopacity{0.800000}%
\pgfsetdash{}{0pt}%
\pgfpathmoveto{\pgfqpoint{1.974238in}{5.809332in}}%
\pgfpathlineto{\pgfqpoint{2.941011in}{4.613488in}}%
\pgfusepath{stroke}%
\end{pgfscope}%
\begin{pgfscope}%
\pgfpathrectangle{\pgfqpoint{0.393613in}{0.331635in}}{\pgfqpoint{9.300000in}{7.700000in}}%
\pgfusepath{clip}%
\pgfsetrectcap%
\pgfsetroundjoin%
\pgfsetlinewidth{1.505625pt}%
\definecolor{currentstroke}{rgb}{0.631373,0.788235,0.956863}%
\pgfsetstrokecolor{currentstroke}%
\pgfsetstrokeopacity{0.800000}%
\pgfsetdash{}{0pt}%
\pgfpathmoveto{\pgfqpoint{2.379728in}{6.710730in}}%
\pgfpathlineto{\pgfqpoint{2.941011in}{4.613488in}}%
\pgfusepath{stroke}%
\end{pgfscope}%
\begin{pgfscope}%
\pgfpathrectangle{\pgfqpoint{0.393613in}{0.331635in}}{\pgfqpoint{9.300000in}{7.700000in}}%
\pgfusepath{clip}%
\pgfsetrectcap%
\pgfsetroundjoin%
\pgfsetlinewidth{1.505625pt}%
\definecolor{currentstroke}{rgb}{0.631373,0.788235,0.956863}%
\pgfsetstrokecolor{currentstroke}%
\pgfsetstrokeopacity{0.800000}%
\pgfsetdash{}{0pt}%
\pgfpathmoveto{\pgfqpoint{2.765242in}{4.891781in}}%
\pgfpathlineto{\pgfqpoint{2.941011in}{4.613488in}}%
\pgfusepath{stroke}%
\end{pgfscope}%
\begin{pgfscope}%
\pgfpathrectangle{\pgfqpoint{0.393613in}{0.331635in}}{\pgfqpoint{9.300000in}{7.700000in}}%
\pgfusepath{clip}%
\pgfsetrectcap%
\pgfsetroundjoin%
\pgfsetlinewidth{1.505625pt}%
\definecolor{currentstroke}{rgb}{0.631373,0.788235,0.956863}%
\pgfsetstrokecolor{currentstroke}%
\pgfsetstrokeopacity{0.800000}%
\pgfsetdash{}{0pt}%
\pgfpathmoveto{\pgfqpoint{2.626630in}{6.543168in}}%
\pgfpathlineto{\pgfqpoint{2.941011in}{4.613488in}}%
\pgfusepath{stroke}%
\end{pgfscope}%
\begin{pgfscope}%
\pgfpathrectangle{\pgfqpoint{0.393613in}{0.331635in}}{\pgfqpoint{9.300000in}{7.700000in}}%
\pgfusepath{clip}%
\pgfsetrectcap%
\pgfsetroundjoin%
\pgfsetlinewidth{1.505625pt}%
\definecolor{currentstroke}{rgb}{0.631373,0.788235,0.956863}%
\pgfsetstrokecolor{currentstroke}%
\pgfsetstrokeopacity{0.800000}%
\pgfsetdash{}{0pt}%
\pgfpathmoveto{\pgfqpoint{3.301695in}{5.327555in}}%
\pgfpathlineto{\pgfqpoint{2.941011in}{4.613488in}}%
\pgfusepath{stroke}%
\end{pgfscope}%
\begin{pgfscope}%
\pgfpathrectangle{\pgfqpoint{0.393613in}{0.331635in}}{\pgfqpoint{9.300000in}{7.700000in}}%
\pgfusepath{clip}%
\pgfsetrectcap%
\pgfsetroundjoin%
\pgfsetlinewidth{1.505625pt}%
\definecolor{currentstroke}{rgb}{0.631373,0.788235,0.956863}%
\pgfsetstrokecolor{currentstroke}%
\pgfsetstrokeopacity{0.800000}%
\pgfsetdash{}{0pt}%
\pgfpathmoveto{\pgfqpoint{0.968107in}{5.897384in}}%
\pgfpathlineto{\pgfqpoint{2.941011in}{4.613488in}}%
\pgfusepath{stroke}%
\end{pgfscope}%
\begin{pgfscope}%
\pgfpathrectangle{\pgfqpoint{0.393613in}{0.331635in}}{\pgfqpoint{9.300000in}{7.700000in}}%
\pgfusepath{clip}%
\pgfsetrectcap%
\pgfsetroundjoin%
\pgfsetlinewidth{1.505625pt}%
\definecolor{currentstroke}{rgb}{0.631373,0.788235,0.956863}%
\pgfsetstrokecolor{currentstroke}%
\pgfsetstrokeopacity{0.800000}%
\pgfsetdash{}{0pt}%
\pgfpathmoveto{\pgfqpoint{4.228775in}{6.172529in}}%
\pgfpathlineto{\pgfqpoint{2.941011in}{4.613488in}}%
\pgfusepath{stroke}%
\end{pgfscope}%
\begin{pgfscope}%
\pgfpathrectangle{\pgfqpoint{0.393613in}{0.331635in}}{\pgfqpoint{9.300000in}{7.700000in}}%
\pgfusepath{clip}%
\pgfsetrectcap%
\pgfsetroundjoin%
\pgfsetlinewidth{1.505625pt}%
\definecolor{currentstroke}{rgb}{0.631373,0.788235,0.956863}%
\pgfsetstrokecolor{currentstroke}%
\pgfsetstrokeopacity{0.800000}%
\pgfsetdash{}{0pt}%
\pgfpathmoveto{\pgfqpoint{2.425460in}{7.032380in}}%
\pgfpathlineto{\pgfqpoint{2.941011in}{4.613488in}}%
\pgfusepath{stroke}%
\end{pgfscope}%
\begin{pgfscope}%
\pgfpathrectangle{\pgfqpoint{0.393613in}{0.331635in}}{\pgfqpoint{9.300000in}{7.700000in}}%
\pgfusepath{clip}%
\pgfsetrectcap%
\pgfsetroundjoin%
\pgfsetlinewidth{1.505625pt}%
\definecolor{currentstroke}{rgb}{0.631373,0.788235,0.956863}%
\pgfsetstrokecolor{currentstroke}%
\pgfsetstrokeopacity{0.800000}%
\pgfsetdash{}{0pt}%
\pgfpathmoveto{\pgfqpoint{2.491424in}{5.707373in}}%
\pgfpathlineto{\pgfqpoint{2.941011in}{4.613488in}}%
\pgfusepath{stroke}%
\end{pgfscope}%
\begin{pgfscope}%
\pgfpathrectangle{\pgfqpoint{0.393613in}{0.331635in}}{\pgfqpoint{9.300000in}{7.700000in}}%
\pgfusepath{clip}%
\pgfsetrectcap%
\pgfsetroundjoin%
\pgfsetlinewidth{1.505625pt}%
\definecolor{currentstroke}{rgb}{0.631373,0.788235,0.956863}%
\pgfsetstrokecolor{currentstroke}%
\pgfsetstrokeopacity{0.800000}%
\pgfsetdash{}{0pt}%
\pgfpathmoveto{\pgfqpoint{5.405752in}{7.071695in}}%
\pgfpathlineto{\pgfqpoint{2.941011in}{4.613488in}}%
\pgfusepath{stroke}%
\end{pgfscope}%
\begin{pgfscope}%
\pgfpathrectangle{\pgfqpoint{0.393613in}{0.331635in}}{\pgfqpoint{9.300000in}{7.700000in}}%
\pgfusepath{clip}%
\pgfsetrectcap%
\pgfsetroundjoin%
\pgfsetlinewidth{1.505625pt}%
\definecolor{currentstroke}{rgb}{0.631373,0.788235,0.956863}%
\pgfsetstrokecolor{currentstroke}%
\pgfsetstrokeopacity{0.800000}%
\pgfsetdash{}{0pt}%
\pgfpathmoveto{\pgfqpoint{4.029243in}{1.979124in}}%
\pgfpathlineto{\pgfqpoint{2.941011in}{4.613488in}}%
\pgfusepath{stroke}%
\end{pgfscope}%
\begin{pgfscope}%
\pgfpathrectangle{\pgfqpoint{0.393613in}{0.331635in}}{\pgfqpoint{9.300000in}{7.700000in}}%
\pgfusepath{clip}%
\pgfsetrectcap%
\pgfsetroundjoin%
\pgfsetlinewidth{1.505625pt}%
\definecolor{currentstroke}{rgb}{0.631373,0.788235,0.956863}%
\pgfsetstrokecolor{currentstroke}%
\pgfsetstrokeopacity{0.800000}%
\pgfsetdash{}{0pt}%
\pgfpathmoveto{\pgfqpoint{3.912187in}{3.998408in}}%
\pgfpathlineto{\pgfqpoint{2.941011in}{4.613488in}}%
\pgfusepath{stroke}%
\end{pgfscope}%
\begin{pgfscope}%
\pgfpathrectangle{\pgfqpoint{0.393613in}{0.331635in}}{\pgfqpoint{9.300000in}{7.700000in}}%
\pgfusepath{clip}%
\pgfsetrectcap%
\pgfsetroundjoin%
\pgfsetlinewidth{1.505625pt}%
\definecolor{currentstroke}{rgb}{0.631373,0.788235,0.956863}%
\pgfsetstrokecolor{currentstroke}%
\pgfsetstrokeopacity{0.800000}%
\pgfsetdash{}{0pt}%
\pgfpathmoveto{\pgfqpoint{3.189809in}{3.286603in}}%
\pgfpathlineto{\pgfqpoint{2.941011in}{4.613488in}}%
\pgfusepath{stroke}%
\end{pgfscope}%
\begin{pgfscope}%
\pgfpathrectangle{\pgfqpoint{0.393613in}{0.331635in}}{\pgfqpoint{9.300000in}{7.700000in}}%
\pgfusepath{clip}%
\pgfsetrectcap%
\pgfsetroundjoin%
\pgfsetlinewidth{1.505625pt}%
\definecolor{currentstroke}{rgb}{0.631373,0.788235,0.956863}%
\pgfsetstrokecolor{currentstroke}%
\pgfsetstrokeopacity{0.800000}%
\pgfsetdash{}{0pt}%
\pgfpathmoveto{\pgfqpoint{2.615024in}{4.490516in}}%
\pgfpathlineto{\pgfqpoint{2.941011in}{4.613488in}}%
\pgfusepath{stroke}%
\end{pgfscope}%
\begin{pgfscope}%
\pgfpathrectangle{\pgfqpoint{0.393613in}{0.331635in}}{\pgfqpoint{9.300000in}{7.700000in}}%
\pgfusepath{clip}%
\pgfsetrectcap%
\pgfsetroundjoin%
\pgfsetlinewidth{1.505625pt}%
\definecolor{currentstroke}{rgb}{0.631373,0.788235,0.956863}%
\pgfsetstrokecolor{currentstroke}%
\pgfsetstrokeopacity{0.800000}%
\pgfsetdash{}{0pt}%
\pgfpathmoveto{\pgfqpoint{2.589699in}{5.191672in}}%
\pgfpathlineto{\pgfqpoint{2.941011in}{4.613488in}}%
\pgfusepath{stroke}%
\end{pgfscope}%
\begin{pgfscope}%
\pgfpathrectangle{\pgfqpoint{0.393613in}{0.331635in}}{\pgfqpoint{9.300000in}{7.700000in}}%
\pgfusepath{clip}%
\pgfsetrectcap%
\pgfsetroundjoin%
\pgfsetlinewidth{1.505625pt}%
\definecolor{currentstroke}{rgb}{0.631373,0.788235,0.956863}%
\pgfsetstrokecolor{currentstroke}%
\pgfsetstrokeopacity{0.800000}%
\pgfsetdash{}{0pt}%
\pgfpathmoveto{\pgfqpoint{1.233235in}{3.968830in}}%
\pgfpathlineto{\pgfqpoint{2.941011in}{4.613488in}}%
\pgfusepath{stroke}%
\end{pgfscope}%
\begin{pgfscope}%
\pgfpathrectangle{\pgfqpoint{0.393613in}{0.331635in}}{\pgfqpoint{9.300000in}{7.700000in}}%
\pgfusepath{clip}%
\pgfsetrectcap%
\pgfsetroundjoin%
\pgfsetlinewidth{1.505625pt}%
\definecolor{currentstroke}{rgb}{0.631373,0.788235,0.956863}%
\pgfsetstrokecolor{currentstroke}%
\pgfsetstrokeopacity{0.800000}%
\pgfsetdash{}{0pt}%
\pgfpathmoveto{\pgfqpoint{3.810785in}{1.946484in}}%
\pgfpathlineto{\pgfqpoint{2.941011in}{4.613488in}}%
\pgfusepath{stroke}%
\end{pgfscope}%
\begin{pgfscope}%
\pgfpathrectangle{\pgfqpoint{0.393613in}{0.331635in}}{\pgfqpoint{9.300000in}{7.700000in}}%
\pgfusepath{clip}%
\pgfsetrectcap%
\pgfsetroundjoin%
\pgfsetlinewidth{1.505625pt}%
\definecolor{currentstroke}{rgb}{0.631373,0.788235,0.956863}%
\pgfsetstrokecolor{currentstroke}%
\pgfsetstrokeopacity{0.800000}%
\pgfsetdash{}{0pt}%
\pgfpathmoveto{\pgfqpoint{2.870620in}{1.775187in}}%
\pgfpathlineto{\pgfqpoint{2.941011in}{4.613488in}}%
\pgfusepath{stroke}%
\end{pgfscope}%
\begin{pgfscope}%
\pgfpathrectangle{\pgfqpoint{0.393613in}{0.331635in}}{\pgfqpoint{9.300000in}{7.700000in}}%
\pgfusepath{clip}%
\pgfsetrectcap%
\pgfsetroundjoin%
\pgfsetlinewidth{1.505625pt}%
\definecolor{currentstroke}{rgb}{0.631373,0.788235,0.956863}%
\pgfsetstrokecolor{currentstroke}%
\pgfsetstrokeopacity{0.800000}%
\pgfsetdash{}{0pt}%
\pgfpathmoveto{\pgfqpoint{2.247275in}{4.467915in}}%
\pgfpathlineto{\pgfqpoint{2.941011in}{4.613488in}}%
\pgfusepath{stroke}%
\end{pgfscope}%
\begin{pgfscope}%
\pgfpathrectangle{\pgfqpoint{0.393613in}{0.331635in}}{\pgfqpoint{9.300000in}{7.700000in}}%
\pgfusepath{clip}%
\pgfsetrectcap%
\pgfsetroundjoin%
\pgfsetlinewidth{1.505625pt}%
\definecolor{currentstroke}{rgb}{0.631373,0.788235,0.956863}%
\pgfsetstrokecolor{currentstroke}%
\pgfsetstrokeopacity{0.800000}%
\pgfsetdash{}{0pt}%
\pgfpathmoveto{\pgfqpoint{3.278707in}{2.508240in}}%
\pgfpathlineto{\pgfqpoint{2.941011in}{4.613488in}}%
\pgfusepath{stroke}%
\end{pgfscope}%
\begin{pgfscope}%
\pgfpathrectangle{\pgfqpoint{0.393613in}{0.331635in}}{\pgfqpoint{9.300000in}{7.700000in}}%
\pgfusepath{clip}%
\pgfsetrectcap%
\pgfsetroundjoin%
\pgfsetlinewidth{1.505625pt}%
\definecolor{currentstroke}{rgb}{0.631373,0.788235,0.956863}%
\pgfsetstrokecolor{currentstroke}%
\pgfsetstrokeopacity{0.800000}%
\pgfsetdash{}{0pt}%
\pgfpathmoveto{\pgfqpoint{4.507490in}{0.681635in}}%
\pgfpathlineto{\pgfqpoint{2.941011in}{4.613488in}}%
\pgfusepath{stroke}%
\end{pgfscope}%
\begin{pgfscope}%
\pgfpathrectangle{\pgfqpoint{0.393613in}{0.331635in}}{\pgfqpoint{9.300000in}{7.700000in}}%
\pgfusepath{clip}%
\pgfsetrectcap%
\pgfsetroundjoin%
\pgfsetlinewidth{1.505625pt}%
\definecolor{currentstroke}{rgb}{0.631373,0.788235,0.956863}%
\pgfsetstrokecolor{currentstroke}%
\pgfsetstrokeopacity{0.800000}%
\pgfsetdash{}{0pt}%
\pgfpathmoveto{\pgfqpoint{2.850278in}{7.174436in}}%
\pgfpathlineto{\pgfqpoint{2.941011in}{4.613488in}}%
\pgfusepath{stroke}%
\end{pgfscope}%
\begin{pgfscope}%
\pgfpathrectangle{\pgfqpoint{0.393613in}{0.331635in}}{\pgfqpoint{9.300000in}{7.700000in}}%
\pgfusepath{clip}%
\pgfsetrectcap%
\pgfsetroundjoin%
\pgfsetlinewidth{1.505625pt}%
\definecolor{currentstroke}{rgb}{0.631373,0.788235,0.956863}%
\pgfsetstrokecolor{currentstroke}%
\pgfsetstrokeopacity{0.800000}%
\pgfsetdash{}{0pt}%
\pgfpathmoveto{\pgfqpoint{1.938431in}{5.037958in}}%
\pgfpathlineto{\pgfqpoint{2.941011in}{4.613488in}}%
\pgfusepath{stroke}%
\end{pgfscope}%
\begin{pgfscope}%
\pgfpathrectangle{\pgfqpoint{0.393613in}{0.331635in}}{\pgfqpoint{9.300000in}{7.700000in}}%
\pgfusepath{clip}%
\pgfsetrectcap%
\pgfsetroundjoin%
\pgfsetlinewidth{1.505625pt}%
\definecolor{currentstroke}{rgb}{0.631373,0.788235,0.956863}%
\pgfsetstrokecolor{currentstroke}%
\pgfsetstrokeopacity{0.800000}%
\pgfsetdash{}{0pt}%
\pgfpathmoveto{\pgfqpoint{2.919129in}{0.931069in}}%
\pgfpathlineto{\pgfqpoint{2.941011in}{4.613488in}}%
\pgfusepath{stroke}%
\end{pgfscope}%
\begin{pgfscope}%
\pgfpathrectangle{\pgfqpoint{0.393613in}{0.331635in}}{\pgfqpoint{9.300000in}{7.700000in}}%
\pgfusepath{clip}%
\pgfsetrectcap%
\pgfsetroundjoin%
\pgfsetlinewidth{1.505625pt}%
\definecolor{currentstroke}{rgb}{0.631373,0.788235,0.956863}%
\pgfsetstrokecolor{currentstroke}%
\pgfsetstrokeopacity{0.800000}%
\pgfsetdash{}{0pt}%
\pgfpathmoveto{\pgfqpoint{2.553778in}{6.329677in}}%
\pgfpathlineto{\pgfqpoint{2.941011in}{4.613488in}}%
\pgfusepath{stroke}%
\end{pgfscope}%
\begin{pgfscope}%
\pgfpathrectangle{\pgfqpoint{0.393613in}{0.331635in}}{\pgfqpoint{9.300000in}{7.700000in}}%
\pgfusepath{clip}%
\pgfsetrectcap%
\pgfsetroundjoin%
\pgfsetlinewidth{1.505625pt}%
\definecolor{currentstroke}{rgb}{0.631373,0.788235,0.956863}%
\pgfsetstrokecolor{currentstroke}%
\pgfsetstrokeopacity{0.800000}%
\pgfsetdash{}{0pt}%
\pgfpathmoveto{\pgfqpoint{2.490513in}{3.775065in}}%
\pgfpathlineto{\pgfqpoint{2.941011in}{4.613488in}}%
\pgfusepath{stroke}%
\end{pgfscope}%
\begin{pgfscope}%
\pgfpathrectangle{\pgfqpoint{0.393613in}{0.331635in}}{\pgfqpoint{9.300000in}{7.700000in}}%
\pgfusepath{clip}%
\pgfsetrectcap%
\pgfsetroundjoin%
\pgfsetlinewidth{1.505625pt}%
\definecolor{currentstroke}{rgb}{0.631373,0.788235,0.956863}%
\pgfsetstrokecolor{currentstroke}%
\pgfsetstrokeopacity{0.800000}%
\pgfsetdash{}{0pt}%
\pgfpathmoveto{\pgfqpoint{0.816340in}{4.757595in}}%
\pgfpathlineto{\pgfqpoint{2.941011in}{4.613488in}}%
\pgfusepath{stroke}%
\end{pgfscope}%
\begin{pgfscope}%
\pgfpathrectangle{\pgfqpoint{0.393613in}{0.331635in}}{\pgfqpoint{9.300000in}{7.700000in}}%
\pgfusepath{clip}%
\pgfsetrectcap%
\pgfsetroundjoin%
\pgfsetlinewidth{1.505625pt}%
\definecolor{currentstroke}{rgb}{0.631373,0.788235,0.956863}%
\pgfsetstrokecolor{currentstroke}%
\pgfsetstrokeopacity{0.800000}%
\pgfsetdash{}{0pt}%
\pgfpathmoveto{\pgfqpoint{1.972243in}{4.014754in}}%
\pgfpathlineto{\pgfqpoint{2.941011in}{4.613488in}}%
\pgfusepath{stroke}%
\end{pgfscope}%
\begin{pgfscope}%
\pgfpathrectangle{\pgfqpoint{0.393613in}{0.331635in}}{\pgfqpoint{9.300000in}{7.700000in}}%
\pgfusepath{clip}%
\pgfsetrectcap%
\pgfsetroundjoin%
\pgfsetlinewidth{1.505625pt}%
\definecolor{currentstroke}{rgb}{0.631373,0.788235,0.956863}%
\pgfsetstrokecolor{currentstroke}%
\pgfsetstrokeopacity{0.800000}%
\pgfsetdash{}{0pt}%
\pgfpathmoveto{\pgfqpoint{3.271288in}{1.460560in}}%
\pgfpathlineto{\pgfqpoint{2.941011in}{4.613488in}}%
\pgfusepath{stroke}%
\end{pgfscope}%
\begin{pgfscope}%
\pgfpathrectangle{\pgfqpoint{0.393613in}{0.331635in}}{\pgfqpoint{9.300000in}{7.700000in}}%
\pgfusepath{clip}%
\pgfsetrectcap%
\pgfsetroundjoin%
\pgfsetlinewidth{1.505625pt}%
\definecolor{currentstroke}{rgb}{0.631373,0.788235,0.956863}%
\pgfsetstrokecolor{currentstroke}%
\pgfsetstrokeopacity{0.800000}%
\pgfsetdash{}{0pt}%
\pgfpathmoveto{\pgfqpoint{2.443533in}{5.252316in}}%
\pgfpathlineto{\pgfqpoint{2.941011in}{4.613488in}}%
\pgfusepath{stroke}%
\end{pgfscope}%
\begin{pgfscope}%
\pgfpathrectangle{\pgfqpoint{0.393613in}{0.331635in}}{\pgfqpoint{9.300000in}{7.700000in}}%
\pgfusepath{clip}%
\pgfsetrectcap%
\pgfsetroundjoin%
\pgfsetlinewidth{1.505625pt}%
\definecolor{currentstroke}{rgb}{0.631373,0.788235,0.956863}%
\pgfsetstrokecolor{currentstroke}%
\pgfsetstrokeopacity{0.800000}%
\pgfsetdash{}{0pt}%
\pgfpathmoveto{\pgfqpoint{3.808024in}{7.681635in}}%
\pgfpathlineto{\pgfqpoint{2.941011in}{4.613488in}}%
\pgfusepath{stroke}%
\end{pgfscope}%
\begin{pgfscope}%
\pgfpathrectangle{\pgfqpoint{0.393613in}{0.331635in}}{\pgfqpoint{9.300000in}{7.700000in}}%
\pgfusepath{clip}%
\pgfsetrectcap%
\pgfsetroundjoin%
\pgfsetlinewidth{1.505625pt}%
\definecolor{currentstroke}{rgb}{0.631373,0.788235,0.956863}%
\pgfsetstrokecolor{currentstroke}%
\pgfsetstrokeopacity{0.800000}%
\pgfsetdash{}{0pt}%
\pgfpathmoveto{\pgfqpoint{1.545120in}{6.840743in}}%
\pgfpathlineto{\pgfqpoint{2.941011in}{4.613488in}}%
\pgfusepath{stroke}%
\end{pgfscope}%
\begin{pgfscope}%
\pgfpathrectangle{\pgfqpoint{0.393613in}{0.331635in}}{\pgfqpoint{9.300000in}{7.700000in}}%
\pgfusepath{clip}%
\pgfsetrectcap%
\pgfsetroundjoin%
\pgfsetlinewidth{1.505625pt}%
\definecolor{currentstroke}{rgb}{0.631373,0.788235,0.956863}%
\pgfsetstrokecolor{currentstroke}%
\pgfsetstrokeopacity{0.800000}%
\pgfsetdash{}{0pt}%
\pgfpathmoveto{\pgfqpoint{4.495356in}{6.333810in}}%
\pgfpathlineto{\pgfqpoint{2.941011in}{4.613488in}}%
\pgfusepath{stroke}%
\end{pgfscope}%
\begin{pgfscope}%
\pgfpathrectangle{\pgfqpoint{0.393613in}{0.331635in}}{\pgfqpoint{9.300000in}{7.700000in}}%
\pgfusepath{clip}%
\pgfsetrectcap%
\pgfsetroundjoin%
\pgfsetlinewidth{1.505625pt}%
\definecolor{currentstroke}{rgb}{0.631373,0.788235,0.956863}%
\pgfsetstrokecolor{currentstroke}%
\pgfsetstrokeopacity{0.800000}%
\pgfsetdash{}{0pt}%
\pgfpathmoveto{\pgfqpoint{2.111783in}{3.199242in}}%
\pgfpathlineto{\pgfqpoint{2.941011in}{4.613488in}}%
\pgfusepath{stroke}%
\end{pgfscope}%
\begin{pgfscope}%
\pgfpathrectangle{\pgfqpoint{0.393613in}{0.331635in}}{\pgfqpoint{9.300000in}{7.700000in}}%
\pgfusepath{clip}%
\pgfsetrectcap%
\pgfsetroundjoin%
\pgfsetlinewidth{1.505625pt}%
\definecolor{currentstroke}{rgb}{0.631373,0.788235,0.956863}%
\pgfsetstrokecolor{currentstroke}%
\pgfsetstrokeopacity{0.800000}%
\pgfsetdash{}{0pt}%
\pgfpathmoveto{\pgfqpoint{2.968097in}{5.687545in}}%
\pgfpathlineto{\pgfqpoint{2.941011in}{4.613488in}}%
\pgfusepath{stroke}%
\end{pgfscope}%
\begin{pgfscope}%
\pgfpathrectangle{\pgfqpoint{0.393613in}{0.331635in}}{\pgfqpoint{9.300000in}{7.700000in}}%
\pgfusepath{clip}%
\pgfsetrectcap%
\pgfsetroundjoin%
\pgfsetlinewidth{1.505625pt}%
\definecolor{currentstroke}{rgb}{0.631373,0.788235,0.956863}%
\pgfsetstrokecolor{currentstroke}%
\pgfsetstrokeopacity{0.800000}%
\pgfsetdash{}{0pt}%
\pgfpathmoveto{\pgfqpoint{2.017579in}{2.412400in}}%
\pgfpathlineto{\pgfqpoint{2.941011in}{4.613488in}}%
\pgfusepath{stroke}%
\end{pgfscope}%
\begin{pgfscope}%
\pgfpathrectangle{\pgfqpoint{0.393613in}{0.331635in}}{\pgfqpoint{9.300000in}{7.700000in}}%
\pgfusepath{clip}%
\pgfsetrectcap%
\pgfsetroundjoin%
\pgfsetlinewidth{1.505625pt}%
\definecolor{currentstroke}{rgb}{0.631373,0.788235,0.956863}%
\pgfsetstrokecolor{currentstroke}%
\pgfsetstrokeopacity{0.800000}%
\pgfsetdash{}{0pt}%
\pgfpathmoveto{\pgfqpoint{4.875280in}{6.873638in}}%
\pgfpathlineto{\pgfqpoint{2.941011in}{4.613488in}}%
\pgfusepath{stroke}%
\end{pgfscope}%
\begin{pgfscope}%
\pgfpathrectangle{\pgfqpoint{0.393613in}{0.331635in}}{\pgfqpoint{9.300000in}{7.700000in}}%
\pgfusepath{clip}%
\pgfsetrectcap%
\pgfsetroundjoin%
\pgfsetlinewidth{1.505625pt}%
\definecolor{currentstroke}{rgb}{0.631373,0.788235,0.956863}%
\pgfsetstrokecolor{currentstroke}%
\pgfsetstrokeopacity{0.800000}%
\pgfsetdash{}{0pt}%
\pgfpathmoveto{\pgfqpoint{4.821671in}{6.294216in}}%
\pgfpathlineto{\pgfqpoint{2.941011in}{4.613488in}}%
\pgfusepath{stroke}%
\end{pgfscope}%
\begin{pgfscope}%
\pgfpathrectangle{\pgfqpoint{0.393613in}{0.331635in}}{\pgfqpoint{9.300000in}{7.700000in}}%
\pgfusepath{clip}%
\pgfsetrectcap%
\pgfsetroundjoin%
\pgfsetlinewidth{1.505625pt}%
\definecolor{currentstroke}{rgb}{0.631373,0.788235,0.956863}%
\pgfsetstrokecolor{currentstroke}%
\pgfsetstrokeopacity{0.800000}%
\pgfsetdash{}{0pt}%
\pgfpathmoveto{\pgfqpoint{2.038680in}{2.435803in}}%
\pgfpathlineto{\pgfqpoint{2.941011in}{4.613488in}}%
\pgfusepath{stroke}%
\end{pgfscope}%
\begin{pgfscope}%
\pgfpathrectangle{\pgfqpoint{0.393613in}{0.331635in}}{\pgfqpoint{9.300000in}{7.700000in}}%
\pgfusepath{clip}%
\pgfsetrectcap%
\pgfsetroundjoin%
\pgfsetlinewidth{1.505625pt}%
\definecolor{currentstroke}{rgb}{0.631373,0.788235,0.956863}%
\pgfsetstrokecolor{currentstroke}%
\pgfsetstrokeopacity{0.800000}%
\pgfsetdash{}{0pt}%
\pgfpathmoveto{\pgfqpoint{3.158987in}{6.306692in}}%
\pgfpathlineto{\pgfqpoint{2.941011in}{4.613488in}}%
\pgfusepath{stroke}%
\end{pgfscope}%
\begin{pgfscope}%
\pgfpathrectangle{\pgfqpoint{0.393613in}{0.331635in}}{\pgfqpoint{9.300000in}{7.700000in}}%
\pgfusepath{clip}%
\pgfsetrectcap%
\pgfsetroundjoin%
\pgfsetlinewidth{1.505625pt}%
\definecolor{currentstroke}{rgb}{0.631373,0.788235,0.956863}%
\pgfsetstrokecolor{currentstroke}%
\pgfsetstrokeopacity{0.800000}%
\pgfsetdash{}{0pt}%
\pgfpathmoveto{\pgfqpoint{3.163926in}{3.195114in}}%
\pgfpathlineto{\pgfqpoint{2.941011in}{4.613488in}}%
\pgfusepath{stroke}%
\end{pgfscope}%
\begin{pgfscope}%
\pgfpathrectangle{\pgfqpoint{0.393613in}{0.331635in}}{\pgfqpoint{9.300000in}{7.700000in}}%
\pgfusepath{clip}%
\pgfsetrectcap%
\pgfsetroundjoin%
\pgfsetlinewidth{1.505625pt}%
\definecolor{currentstroke}{rgb}{0.631373,0.788235,0.956863}%
\pgfsetstrokecolor{currentstroke}%
\pgfsetstrokeopacity{0.800000}%
\pgfsetdash{}{0pt}%
\pgfpathmoveto{\pgfqpoint{3.868734in}{6.876221in}}%
\pgfpathlineto{\pgfqpoint{2.941011in}{4.613488in}}%
\pgfusepath{stroke}%
\end{pgfscope}%
\begin{pgfscope}%
\pgfpathrectangle{\pgfqpoint{0.393613in}{0.331635in}}{\pgfqpoint{9.300000in}{7.700000in}}%
\pgfusepath{clip}%
\pgfsetrectcap%
\pgfsetroundjoin%
\pgfsetlinewidth{1.505625pt}%
\definecolor{currentstroke}{rgb}{1.000000,0.705882,0.509804}%
\pgfsetstrokecolor{currentstroke}%
\pgfsetstrokeopacity{0.800000}%
\pgfsetdash{}{0pt}%
\pgfpathmoveto{\pgfqpoint{8.729535in}{4.320731in}}%
\pgfpathlineto{\pgfqpoint{6.534666in}{4.191861in}}%
\pgfusepath{stroke}%
\end{pgfscope}%
\begin{pgfscope}%
\pgfpathrectangle{\pgfqpoint{0.393613in}{0.331635in}}{\pgfqpoint{9.300000in}{7.700000in}}%
\pgfusepath{clip}%
\pgfsetrectcap%
\pgfsetroundjoin%
\pgfsetlinewidth{1.505625pt}%
\definecolor{currentstroke}{rgb}{1.000000,0.705882,0.509804}%
\pgfsetstrokecolor{currentstroke}%
\pgfsetstrokeopacity{0.800000}%
\pgfsetdash{}{0pt}%
\pgfpathmoveto{\pgfqpoint{8.008246in}{2.616387in}}%
\pgfpathlineto{\pgfqpoint{6.534666in}{4.191861in}}%
\pgfusepath{stroke}%
\end{pgfscope}%
\begin{pgfscope}%
\pgfpathrectangle{\pgfqpoint{0.393613in}{0.331635in}}{\pgfqpoint{9.300000in}{7.700000in}}%
\pgfusepath{clip}%
\pgfsetrectcap%
\pgfsetroundjoin%
\pgfsetlinewidth{1.505625pt}%
\definecolor{currentstroke}{rgb}{1.000000,0.705882,0.509804}%
\pgfsetstrokecolor{currentstroke}%
\pgfsetstrokeopacity{0.800000}%
\pgfsetdash{}{0pt}%
\pgfpathmoveto{\pgfqpoint{6.843430in}{2.246087in}}%
\pgfpathlineto{\pgfqpoint{6.534666in}{4.191861in}}%
\pgfusepath{stroke}%
\end{pgfscope}%
\begin{pgfscope}%
\pgfpathrectangle{\pgfqpoint{0.393613in}{0.331635in}}{\pgfqpoint{9.300000in}{7.700000in}}%
\pgfusepath{clip}%
\pgfsetrectcap%
\pgfsetroundjoin%
\pgfsetlinewidth{1.505625pt}%
\definecolor{currentstroke}{rgb}{1.000000,0.705882,0.509804}%
\pgfsetstrokecolor{currentstroke}%
\pgfsetstrokeopacity{0.800000}%
\pgfsetdash{}{0pt}%
\pgfpathmoveto{\pgfqpoint{6.884025in}{6.102825in}}%
\pgfpathlineto{\pgfqpoint{6.534666in}{4.191861in}}%
\pgfusepath{stroke}%
\end{pgfscope}%
\begin{pgfscope}%
\pgfpathrectangle{\pgfqpoint{0.393613in}{0.331635in}}{\pgfqpoint{9.300000in}{7.700000in}}%
\pgfusepath{clip}%
\pgfsetrectcap%
\pgfsetroundjoin%
\pgfsetlinewidth{1.505625pt}%
\definecolor{currentstroke}{rgb}{1.000000,0.705882,0.509804}%
\pgfsetstrokecolor{currentstroke}%
\pgfsetstrokeopacity{0.800000}%
\pgfsetdash{}{0pt}%
\pgfpathmoveto{\pgfqpoint{8.217971in}{5.492423in}}%
\pgfpathlineto{\pgfqpoint{6.534666in}{4.191861in}}%
\pgfusepath{stroke}%
\end{pgfscope}%
\begin{pgfscope}%
\pgfpathrectangle{\pgfqpoint{0.393613in}{0.331635in}}{\pgfqpoint{9.300000in}{7.700000in}}%
\pgfusepath{clip}%
\pgfsetrectcap%
\pgfsetroundjoin%
\pgfsetlinewidth{1.505625pt}%
\definecolor{currentstroke}{rgb}{1.000000,0.705882,0.509804}%
\pgfsetstrokecolor{currentstroke}%
\pgfsetstrokeopacity{0.800000}%
\pgfsetdash{}{0pt}%
\pgfpathmoveto{\pgfqpoint{7.449270in}{4.187473in}}%
\pgfpathlineto{\pgfqpoint{6.534666in}{4.191861in}}%
\pgfusepath{stroke}%
\end{pgfscope}%
\begin{pgfscope}%
\pgfpathrectangle{\pgfqpoint{0.393613in}{0.331635in}}{\pgfqpoint{9.300000in}{7.700000in}}%
\pgfusepath{clip}%
\pgfsetrectcap%
\pgfsetroundjoin%
\pgfsetlinewidth{1.505625pt}%
\definecolor{currentstroke}{rgb}{1.000000,0.705882,0.509804}%
\pgfsetstrokecolor{currentstroke}%
\pgfsetstrokeopacity{0.800000}%
\pgfsetdash{}{0pt}%
\pgfpathmoveto{\pgfqpoint{7.634735in}{5.937258in}}%
\pgfpathlineto{\pgfqpoint{6.534666in}{4.191861in}}%
\pgfusepath{stroke}%
\end{pgfscope}%
\begin{pgfscope}%
\pgfpathrectangle{\pgfqpoint{0.393613in}{0.331635in}}{\pgfqpoint{9.300000in}{7.700000in}}%
\pgfusepath{clip}%
\pgfsetrectcap%
\pgfsetroundjoin%
\pgfsetlinewidth{1.505625pt}%
\definecolor{currentstroke}{rgb}{1.000000,0.705882,0.509804}%
\pgfsetstrokecolor{currentstroke}%
\pgfsetstrokeopacity{0.800000}%
\pgfsetdash{}{0pt}%
\pgfpathmoveto{\pgfqpoint{7.737786in}{3.852959in}}%
\pgfpathlineto{\pgfqpoint{6.534666in}{4.191861in}}%
\pgfusepath{stroke}%
\end{pgfscope}%
\begin{pgfscope}%
\pgfpathrectangle{\pgfqpoint{0.393613in}{0.331635in}}{\pgfqpoint{9.300000in}{7.700000in}}%
\pgfusepath{clip}%
\pgfsetrectcap%
\pgfsetroundjoin%
\pgfsetlinewidth{1.505625pt}%
\definecolor{currentstroke}{rgb}{1.000000,0.705882,0.509804}%
\pgfsetstrokecolor{currentstroke}%
\pgfsetstrokeopacity{0.800000}%
\pgfsetdash{}{0pt}%
\pgfpathmoveto{\pgfqpoint{6.985227in}{5.988577in}}%
\pgfpathlineto{\pgfqpoint{6.534666in}{4.191861in}}%
\pgfusepath{stroke}%
\end{pgfscope}%
\begin{pgfscope}%
\pgfpathrectangle{\pgfqpoint{0.393613in}{0.331635in}}{\pgfqpoint{9.300000in}{7.700000in}}%
\pgfusepath{clip}%
\pgfsetrectcap%
\pgfsetroundjoin%
\pgfsetlinewidth{1.505625pt}%
\definecolor{currentstroke}{rgb}{1.000000,0.705882,0.509804}%
\pgfsetstrokecolor{currentstroke}%
\pgfsetstrokeopacity{0.800000}%
\pgfsetdash{}{0pt}%
\pgfpathmoveto{\pgfqpoint{8.526151in}{3.491262in}}%
\pgfpathlineto{\pgfqpoint{6.534666in}{4.191861in}}%
\pgfusepath{stroke}%
\end{pgfscope}%
\begin{pgfscope}%
\pgfpathrectangle{\pgfqpoint{0.393613in}{0.331635in}}{\pgfqpoint{9.300000in}{7.700000in}}%
\pgfusepath{clip}%
\pgfsetrectcap%
\pgfsetroundjoin%
\pgfsetlinewidth{1.505625pt}%
\definecolor{currentstroke}{rgb}{1.000000,0.705882,0.509804}%
\pgfsetstrokecolor{currentstroke}%
\pgfsetstrokeopacity{0.800000}%
\pgfsetdash{}{0pt}%
\pgfpathmoveto{\pgfqpoint{4.970295in}{4.682448in}}%
\pgfpathlineto{\pgfqpoint{6.534666in}{4.191861in}}%
\pgfusepath{stroke}%
\end{pgfscope}%
\begin{pgfscope}%
\pgfpathrectangle{\pgfqpoint{0.393613in}{0.331635in}}{\pgfqpoint{9.300000in}{7.700000in}}%
\pgfusepath{clip}%
\pgfsetrectcap%
\pgfsetroundjoin%
\pgfsetlinewidth{1.505625pt}%
\definecolor{currentstroke}{rgb}{1.000000,0.705882,0.509804}%
\pgfsetstrokecolor{currentstroke}%
\pgfsetstrokeopacity{0.800000}%
\pgfsetdash{}{0pt}%
\pgfpathmoveto{\pgfqpoint{4.066721in}{5.299933in}}%
\pgfpathlineto{\pgfqpoint{6.534666in}{4.191861in}}%
\pgfusepath{stroke}%
\end{pgfscope}%
\begin{pgfscope}%
\pgfpathrectangle{\pgfqpoint{0.393613in}{0.331635in}}{\pgfqpoint{9.300000in}{7.700000in}}%
\pgfusepath{clip}%
\pgfsetrectcap%
\pgfsetroundjoin%
\pgfsetlinewidth{1.505625pt}%
\definecolor{currentstroke}{rgb}{1.000000,0.705882,0.509804}%
\pgfsetstrokecolor{currentstroke}%
\pgfsetstrokeopacity{0.800000}%
\pgfsetdash{}{0pt}%
\pgfpathmoveto{\pgfqpoint{5.047977in}{5.129617in}}%
\pgfpathlineto{\pgfqpoint{6.534666in}{4.191861in}}%
\pgfusepath{stroke}%
\end{pgfscope}%
\begin{pgfscope}%
\pgfpathrectangle{\pgfqpoint{0.393613in}{0.331635in}}{\pgfqpoint{9.300000in}{7.700000in}}%
\pgfusepath{clip}%
\pgfsetrectcap%
\pgfsetroundjoin%
\pgfsetlinewidth{1.505625pt}%
\definecolor{currentstroke}{rgb}{1.000000,0.705882,0.509804}%
\pgfsetstrokecolor{currentstroke}%
\pgfsetstrokeopacity{0.800000}%
\pgfsetdash{}{0pt}%
\pgfpathmoveto{\pgfqpoint{8.708537in}{3.149559in}}%
\pgfpathlineto{\pgfqpoint{6.534666in}{4.191861in}}%
\pgfusepath{stroke}%
\end{pgfscope}%
\begin{pgfscope}%
\pgfpathrectangle{\pgfqpoint{0.393613in}{0.331635in}}{\pgfqpoint{9.300000in}{7.700000in}}%
\pgfusepath{clip}%
\pgfsetrectcap%
\pgfsetroundjoin%
\pgfsetlinewidth{1.505625pt}%
\definecolor{currentstroke}{rgb}{1.000000,0.705882,0.509804}%
\pgfsetstrokecolor{currentstroke}%
\pgfsetstrokeopacity{0.800000}%
\pgfsetdash{}{0pt}%
\pgfpathmoveto{\pgfqpoint{6.611184in}{2.598931in}}%
\pgfpathlineto{\pgfqpoint{6.534666in}{4.191861in}}%
\pgfusepath{stroke}%
\end{pgfscope}%
\begin{pgfscope}%
\pgfpathrectangle{\pgfqpoint{0.393613in}{0.331635in}}{\pgfqpoint{9.300000in}{7.700000in}}%
\pgfusepath{clip}%
\pgfsetrectcap%
\pgfsetroundjoin%
\pgfsetlinewidth{1.505625pt}%
\definecolor{currentstroke}{rgb}{1.000000,0.705882,0.509804}%
\pgfsetstrokecolor{currentstroke}%
\pgfsetstrokeopacity{0.800000}%
\pgfsetdash{}{0pt}%
\pgfpathmoveto{\pgfqpoint{5.696576in}{3.930575in}}%
\pgfpathlineto{\pgfqpoint{6.534666in}{4.191861in}}%
\pgfusepath{stroke}%
\end{pgfscope}%
\begin{pgfscope}%
\pgfpathrectangle{\pgfqpoint{0.393613in}{0.331635in}}{\pgfqpoint{9.300000in}{7.700000in}}%
\pgfusepath{clip}%
\pgfsetrectcap%
\pgfsetroundjoin%
\pgfsetlinewidth{1.505625pt}%
\definecolor{currentstroke}{rgb}{1.000000,0.705882,0.509804}%
\pgfsetstrokecolor{currentstroke}%
\pgfsetstrokeopacity{0.800000}%
\pgfsetdash{}{0pt}%
\pgfpathmoveto{\pgfqpoint{4.929999in}{4.145173in}}%
\pgfpathlineto{\pgfqpoint{6.534666in}{4.191861in}}%
\pgfusepath{stroke}%
\end{pgfscope}%
\begin{pgfscope}%
\pgfpathrectangle{\pgfqpoint{0.393613in}{0.331635in}}{\pgfqpoint{9.300000in}{7.700000in}}%
\pgfusepath{clip}%
\pgfsetrectcap%
\pgfsetroundjoin%
\pgfsetlinewidth{1.505625pt}%
\definecolor{currentstroke}{rgb}{1.000000,0.705882,0.509804}%
\pgfsetstrokecolor{currentstroke}%
\pgfsetstrokeopacity{0.800000}%
\pgfsetdash{}{0pt}%
\pgfpathmoveto{\pgfqpoint{4.410117in}{4.635500in}}%
\pgfpathlineto{\pgfqpoint{6.534666in}{4.191861in}}%
\pgfusepath{stroke}%
\end{pgfscope}%
\begin{pgfscope}%
\pgfpathrectangle{\pgfqpoint{0.393613in}{0.331635in}}{\pgfqpoint{9.300000in}{7.700000in}}%
\pgfusepath{clip}%
\pgfsetrectcap%
\pgfsetroundjoin%
\pgfsetlinewidth{1.505625pt}%
\definecolor{currentstroke}{rgb}{1.000000,0.705882,0.509804}%
\pgfsetstrokecolor{currentstroke}%
\pgfsetstrokeopacity{0.800000}%
\pgfsetdash{}{0pt}%
\pgfpathmoveto{\pgfqpoint{8.226679in}{3.239102in}}%
\pgfpathlineto{\pgfqpoint{6.534666in}{4.191861in}}%
\pgfusepath{stroke}%
\end{pgfscope}%
\begin{pgfscope}%
\pgfpathrectangle{\pgfqpoint{0.393613in}{0.331635in}}{\pgfqpoint{9.300000in}{7.700000in}}%
\pgfusepath{clip}%
\pgfsetrectcap%
\pgfsetroundjoin%
\pgfsetlinewidth{1.505625pt}%
\definecolor{currentstroke}{rgb}{1.000000,0.705882,0.509804}%
\pgfsetstrokecolor{currentstroke}%
\pgfsetstrokeopacity{0.800000}%
\pgfsetdash{}{0pt}%
\pgfpathmoveto{\pgfqpoint{8.462150in}{4.774615in}}%
\pgfpathlineto{\pgfqpoint{6.534666in}{4.191861in}}%
\pgfusepath{stroke}%
\end{pgfscope}%
\begin{pgfscope}%
\pgfpathrectangle{\pgfqpoint{0.393613in}{0.331635in}}{\pgfqpoint{9.300000in}{7.700000in}}%
\pgfusepath{clip}%
\pgfsetrectcap%
\pgfsetroundjoin%
\pgfsetlinewidth{1.505625pt}%
\definecolor{currentstroke}{rgb}{1.000000,0.705882,0.509804}%
\pgfsetstrokecolor{currentstroke}%
\pgfsetstrokeopacity{0.800000}%
\pgfsetdash{}{0pt}%
\pgfpathmoveto{\pgfqpoint{5.762899in}{4.714105in}}%
\pgfpathlineto{\pgfqpoint{6.534666in}{4.191861in}}%
\pgfusepath{stroke}%
\end{pgfscope}%
\begin{pgfscope}%
\pgfpathrectangle{\pgfqpoint{0.393613in}{0.331635in}}{\pgfqpoint{9.300000in}{7.700000in}}%
\pgfusepath{clip}%
\pgfsetrectcap%
\pgfsetroundjoin%
\pgfsetlinewidth{1.505625pt}%
\definecolor{currentstroke}{rgb}{1.000000,0.705882,0.509804}%
\pgfsetstrokecolor{currentstroke}%
\pgfsetstrokeopacity{0.800000}%
\pgfsetdash{}{0pt}%
\pgfpathmoveto{\pgfqpoint{8.593098in}{3.947991in}}%
\pgfpathlineto{\pgfqpoint{6.534666in}{4.191861in}}%
\pgfusepath{stroke}%
\end{pgfscope}%
\begin{pgfscope}%
\pgfpathrectangle{\pgfqpoint{0.393613in}{0.331635in}}{\pgfqpoint{9.300000in}{7.700000in}}%
\pgfusepath{clip}%
\pgfsetrectcap%
\pgfsetroundjoin%
\pgfsetlinewidth{1.505625pt}%
\definecolor{currentstroke}{rgb}{1.000000,0.705882,0.509804}%
\pgfsetstrokecolor{currentstroke}%
\pgfsetstrokeopacity{0.800000}%
\pgfsetdash{}{0pt}%
\pgfpathmoveto{\pgfqpoint{5.266097in}{2.316466in}}%
\pgfpathlineto{\pgfqpoint{6.534666in}{4.191861in}}%
\pgfusepath{stroke}%
\end{pgfscope}%
\begin{pgfscope}%
\pgfpathrectangle{\pgfqpoint{0.393613in}{0.331635in}}{\pgfqpoint{9.300000in}{7.700000in}}%
\pgfusepath{clip}%
\pgfsetrectcap%
\pgfsetroundjoin%
\pgfsetlinewidth{1.505625pt}%
\definecolor{currentstroke}{rgb}{1.000000,0.705882,0.509804}%
\pgfsetstrokecolor{currentstroke}%
\pgfsetstrokeopacity{0.800000}%
\pgfsetdash{}{0pt}%
\pgfpathmoveto{\pgfqpoint{7.489133in}{2.379617in}}%
\pgfpathlineto{\pgfqpoint{6.534666in}{4.191861in}}%
\pgfusepath{stroke}%
\end{pgfscope}%
\begin{pgfscope}%
\pgfpathrectangle{\pgfqpoint{0.393613in}{0.331635in}}{\pgfqpoint{9.300000in}{7.700000in}}%
\pgfusepath{clip}%
\pgfsetrectcap%
\pgfsetroundjoin%
\pgfsetlinewidth{1.505625pt}%
\definecolor{currentstroke}{rgb}{1.000000,0.705882,0.509804}%
\pgfsetstrokecolor{currentstroke}%
\pgfsetstrokeopacity{0.800000}%
\pgfsetdash{}{0pt}%
\pgfpathmoveto{\pgfqpoint{7.375591in}{4.728687in}}%
\pgfpathlineto{\pgfqpoint{6.534666in}{4.191861in}}%
\pgfusepath{stroke}%
\end{pgfscope}%
\begin{pgfscope}%
\pgfpathrectangle{\pgfqpoint{0.393613in}{0.331635in}}{\pgfqpoint{9.300000in}{7.700000in}}%
\pgfusepath{clip}%
\pgfsetrectcap%
\pgfsetroundjoin%
\pgfsetlinewidth{1.505625pt}%
\definecolor{currentstroke}{rgb}{1.000000,0.705882,0.509804}%
\pgfsetstrokecolor{currentstroke}%
\pgfsetstrokeopacity{0.800000}%
\pgfsetdash{}{0pt}%
\pgfpathmoveto{\pgfqpoint{5.281132in}{4.012768in}}%
\pgfpathlineto{\pgfqpoint{6.534666in}{4.191861in}}%
\pgfusepath{stroke}%
\end{pgfscope}%
\begin{pgfscope}%
\pgfpathrectangle{\pgfqpoint{0.393613in}{0.331635in}}{\pgfqpoint{9.300000in}{7.700000in}}%
\pgfusepath{clip}%
\pgfsetrectcap%
\pgfsetroundjoin%
\pgfsetlinewidth{1.505625pt}%
\definecolor{currentstroke}{rgb}{1.000000,0.705882,0.509804}%
\pgfsetstrokecolor{currentstroke}%
\pgfsetstrokeopacity{0.800000}%
\pgfsetdash{}{0pt}%
\pgfpathmoveto{\pgfqpoint{8.775963in}{3.806168in}}%
\pgfpathlineto{\pgfqpoint{6.534666in}{4.191861in}}%
\pgfusepath{stroke}%
\end{pgfscope}%
\begin{pgfscope}%
\pgfpathrectangle{\pgfqpoint{0.393613in}{0.331635in}}{\pgfqpoint{9.300000in}{7.700000in}}%
\pgfusepath{clip}%
\pgfsetrectcap%
\pgfsetroundjoin%
\pgfsetlinewidth{1.505625pt}%
\definecolor{currentstroke}{rgb}{1.000000,0.705882,0.509804}%
\pgfsetstrokecolor{currentstroke}%
\pgfsetstrokeopacity{0.800000}%
\pgfsetdash{}{0pt}%
\pgfpathmoveto{\pgfqpoint{9.270885in}{4.232334in}}%
\pgfpathlineto{\pgfqpoint{6.534666in}{4.191861in}}%
\pgfusepath{stroke}%
\end{pgfscope}%
\begin{pgfscope}%
\pgfpathrectangle{\pgfqpoint{0.393613in}{0.331635in}}{\pgfqpoint{9.300000in}{7.700000in}}%
\pgfusepath{clip}%
\pgfsetrectcap%
\pgfsetroundjoin%
\pgfsetlinewidth{1.505625pt}%
\definecolor{currentstroke}{rgb}{1.000000,0.705882,0.509804}%
\pgfsetstrokecolor{currentstroke}%
\pgfsetstrokeopacity{0.800000}%
\pgfsetdash{}{0pt}%
\pgfpathmoveto{\pgfqpoint{5.221194in}{3.868911in}}%
\pgfpathlineto{\pgfqpoint{6.534666in}{4.191861in}}%
\pgfusepath{stroke}%
\end{pgfscope}%
\begin{pgfscope}%
\pgfpathrectangle{\pgfqpoint{0.393613in}{0.331635in}}{\pgfqpoint{9.300000in}{7.700000in}}%
\pgfusepath{clip}%
\pgfsetrectcap%
\pgfsetroundjoin%
\pgfsetlinewidth{1.505625pt}%
\definecolor{currentstroke}{rgb}{1.000000,0.705882,0.509804}%
\pgfsetstrokecolor{currentstroke}%
\pgfsetstrokeopacity{0.800000}%
\pgfsetdash{}{0pt}%
\pgfpathmoveto{\pgfqpoint{4.726563in}{5.105585in}}%
\pgfpathlineto{\pgfqpoint{6.534666in}{4.191861in}}%
\pgfusepath{stroke}%
\end{pgfscope}%
\begin{pgfscope}%
\pgfpathrectangle{\pgfqpoint{0.393613in}{0.331635in}}{\pgfqpoint{9.300000in}{7.700000in}}%
\pgfusepath{clip}%
\pgfsetrectcap%
\pgfsetroundjoin%
\pgfsetlinewidth{1.505625pt}%
\definecolor{currentstroke}{rgb}{1.000000,0.705882,0.509804}%
\pgfsetstrokecolor{currentstroke}%
\pgfsetstrokeopacity{0.800000}%
\pgfsetdash{}{0pt}%
\pgfpathmoveto{\pgfqpoint{5.210620in}{2.300748in}}%
\pgfpathlineto{\pgfqpoint{6.534666in}{4.191861in}}%
\pgfusepath{stroke}%
\end{pgfscope}%
\begin{pgfscope}%
\pgfpathrectangle{\pgfqpoint{0.393613in}{0.331635in}}{\pgfqpoint{9.300000in}{7.700000in}}%
\pgfusepath{clip}%
\pgfsetrectcap%
\pgfsetroundjoin%
\pgfsetlinewidth{1.505625pt}%
\definecolor{currentstroke}{rgb}{1.000000,0.705882,0.509804}%
\pgfsetstrokecolor{currentstroke}%
\pgfsetstrokeopacity{0.800000}%
\pgfsetdash{}{0pt}%
\pgfpathmoveto{\pgfqpoint{6.611736in}{2.822774in}}%
\pgfpathlineto{\pgfqpoint{6.534666in}{4.191861in}}%
\pgfusepath{stroke}%
\end{pgfscope}%
\begin{pgfscope}%
\pgfpathrectangle{\pgfqpoint{0.393613in}{0.331635in}}{\pgfqpoint{9.300000in}{7.700000in}}%
\pgfusepath{clip}%
\pgfsetrectcap%
\pgfsetroundjoin%
\pgfsetlinewidth{1.505625pt}%
\definecolor{currentstroke}{rgb}{1.000000,0.705882,0.509804}%
\pgfsetstrokecolor{currentstroke}%
\pgfsetstrokeopacity{0.800000}%
\pgfsetdash{}{0pt}%
\pgfpathmoveto{\pgfqpoint{4.838508in}{5.682184in}}%
\pgfpathlineto{\pgfqpoint{6.534666in}{4.191861in}}%
\pgfusepath{stroke}%
\end{pgfscope}%
\begin{pgfscope}%
\pgfpathrectangle{\pgfqpoint{0.393613in}{0.331635in}}{\pgfqpoint{9.300000in}{7.700000in}}%
\pgfusepath{clip}%
\pgfsetrectcap%
\pgfsetroundjoin%
\pgfsetlinewidth{1.505625pt}%
\definecolor{currentstroke}{rgb}{1.000000,0.705882,0.509804}%
\pgfsetstrokecolor{currentstroke}%
\pgfsetstrokeopacity{0.800000}%
\pgfsetdash{}{0pt}%
\pgfpathmoveto{\pgfqpoint{7.784989in}{5.679570in}}%
\pgfpathlineto{\pgfqpoint{6.534666in}{4.191861in}}%
\pgfusepath{stroke}%
\end{pgfscope}%
\begin{pgfscope}%
\pgfpathrectangle{\pgfqpoint{0.393613in}{0.331635in}}{\pgfqpoint{9.300000in}{7.700000in}}%
\pgfusepath{clip}%
\pgfsetrectcap%
\pgfsetroundjoin%
\pgfsetlinewidth{1.505625pt}%
\definecolor{currentstroke}{rgb}{1.000000,0.705882,0.509804}%
\pgfsetstrokecolor{currentstroke}%
\pgfsetstrokeopacity{0.800000}%
\pgfsetdash{}{0pt}%
\pgfpathmoveto{\pgfqpoint{9.033656in}{3.874177in}}%
\pgfpathlineto{\pgfqpoint{6.534666in}{4.191861in}}%
\pgfusepath{stroke}%
\end{pgfscope}%
\begin{pgfscope}%
\pgfpathrectangle{\pgfqpoint{0.393613in}{0.331635in}}{\pgfqpoint{9.300000in}{7.700000in}}%
\pgfusepath{clip}%
\pgfsetrectcap%
\pgfsetroundjoin%
\pgfsetlinewidth{1.505625pt}%
\definecolor{currentstroke}{rgb}{1.000000,0.705882,0.509804}%
\pgfsetstrokecolor{currentstroke}%
\pgfsetstrokeopacity{0.800000}%
\pgfsetdash{}{0pt}%
\pgfpathmoveto{\pgfqpoint{4.512348in}{5.041560in}}%
\pgfpathlineto{\pgfqpoint{6.534666in}{4.191861in}}%
\pgfusepath{stroke}%
\end{pgfscope}%
\begin{pgfscope}%
\pgfpathrectangle{\pgfqpoint{0.393613in}{0.331635in}}{\pgfqpoint{9.300000in}{7.700000in}}%
\pgfusepath{clip}%
\pgfsetrectcap%
\pgfsetroundjoin%
\pgfsetlinewidth{1.505625pt}%
\definecolor{currentstroke}{rgb}{1.000000,0.705882,0.509804}%
\pgfsetstrokecolor{currentstroke}%
\pgfsetstrokeopacity{0.800000}%
\pgfsetdash{}{0pt}%
\pgfpathmoveto{\pgfqpoint{5.861358in}{5.869272in}}%
\pgfpathlineto{\pgfqpoint{6.534666in}{4.191861in}}%
\pgfusepath{stroke}%
\end{pgfscope}%
\begin{pgfscope}%
\pgfpathrectangle{\pgfqpoint{0.393613in}{0.331635in}}{\pgfqpoint{9.300000in}{7.700000in}}%
\pgfusepath{clip}%
\pgfsetrectcap%
\pgfsetroundjoin%
\pgfsetlinewidth{1.505625pt}%
\definecolor{currentstroke}{rgb}{1.000000,0.705882,0.509804}%
\pgfsetstrokecolor{currentstroke}%
\pgfsetstrokeopacity{0.800000}%
\pgfsetdash{}{0pt}%
\pgfpathmoveto{\pgfqpoint{5.626156in}{2.828062in}}%
\pgfpathlineto{\pgfqpoint{6.534666in}{4.191861in}}%
\pgfusepath{stroke}%
\end{pgfscope}%
\begin{pgfscope}%
\pgfpathrectangle{\pgfqpoint{0.393613in}{0.331635in}}{\pgfqpoint{9.300000in}{7.700000in}}%
\pgfusepath{clip}%
\pgfsetrectcap%
\pgfsetroundjoin%
\pgfsetlinewidth{1.505625pt}%
\definecolor{currentstroke}{rgb}{1.000000,0.705882,0.509804}%
\pgfsetstrokecolor{currentstroke}%
\pgfsetstrokeopacity{0.800000}%
\pgfsetdash{}{0pt}%
\pgfpathmoveto{\pgfqpoint{4.868691in}{3.476860in}}%
\pgfpathlineto{\pgfqpoint{6.534666in}{4.191861in}}%
\pgfusepath{stroke}%
\end{pgfscope}%
\begin{pgfscope}%
\pgfpathrectangle{\pgfqpoint{0.393613in}{0.331635in}}{\pgfqpoint{9.300000in}{7.700000in}}%
\pgfusepath{clip}%
\pgfsetrectcap%
\pgfsetroundjoin%
\pgfsetlinewidth{1.505625pt}%
\definecolor{currentstroke}{rgb}{1.000000,0.705882,0.509804}%
\pgfsetstrokecolor{currentstroke}%
\pgfsetstrokeopacity{0.800000}%
\pgfsetdash{}{0pt}%
\pgfpathmoveto{\pgfqpoint{7.939732in}{4.140668in}}%
\pgfpathlineto{\pgfqpoint{6.534666in}{4.191861in}}%
\pgfusepath{stroke}%
\end{pgfscope}%
\begin{pgfscope}%
\pgfpathrectangle{\pgfqpoint{0.393613in}{0.331635in}}{\pgfqpoint{9.300000in}{7.700000in}}%
\pgfusepath{clip}%
\pgfsetrectcap%
\pgfsetroundjoin%
\pgfsetlinewidth{1.505625pt}%
\definecolor{currentstroke}{rgb}{1.000000,0.705882,0.509804}%
\pgfsetstrokecolor{currentstroke}%
\pgfsetstrokeopacity{0.800000}%
\pgfsetdash{}{0pt}%
\pgfpathmoveto{\pgfqpoint{5.975649in}{5.901612in}}%
\pgfpathlineto{\pgfqpoint{6.534666in}{4.191861in}}%
\pgfusepath{stroke}%
\end{pgfscope}%
\begin{pgfscope}%
\pgfpathrectangle{\pgfqpoint{0.393613in}{0.331635in}}{\pgfqpoint{9.300000in}{7.700000in}}%
\pgfusepath{clip}%
\pgfsetrectcap%
\pgfsetroundjoin%
\pgfsetlinewidth{1.505625pt}%
\definecolor{currentstroke}{rgb}{1.000000,0.705882,0.509804}%
\pgfsetstrokecolor{currentstroke}%
\pgfsetstrokeopacity{0.800000}%
\pgfsetdash{}{0pt}%
\pgfpathmoveto{\pgfqpoint{5.533563in}{3.247043in}}%
\pgfpathlineto{\pgfqpoint{6.534666in}{4.191861in}}%
\pgfusepath{stroke}%
\end{pgfscope}%
\begin{pgfscope}%
\pgfpathrectangle{\pgfqpoint{0.393613in}{0.331635in}}{\pgfqpoint{9.300000in}{7.700000in}}%
\pgfusepath{clip}%
\pgfsetrectcap%
\pgfsetroundjoin%
\pgfsetlinewidth{1.505625pt}%
\definecolor{currentstroke}{rgb}{1.000000,0.705882,0.509804}%
\pgfsetstrokecolor{currentstroke}%
\pgfsetstrokeopacity{0.800000}%
\pgfsetdash{}{0pt}%
\pgfpathmoveto{\pgfqpoint{8.879060in}{4.527070in}}%
\pgfpathlineto{\pgfqpoint{6.534666in}{4.191861in}}%
\pgfusepath{stroke}%
\end{pgfscope}%
\begin{pgfscope}%
\pgfpathrectangle{\pgfqpoint{0.393613in}{0.331635in}}{\pgfqpoint{9.300000in}{7.700000in}}%
\pgfusepath{clip}%
\pgfsetrectcap%
\pgfsetroundjoin%
\pgfsetlinewidth{1.505625pt}%
\definecolor{currentstroke}{rgb}{1.000000,0.705882,0.509804}%
\pgfsetstrokecolor{currentstroke}%
\pgfsetstrokeopacity{0.800000}%
\pgfsetdash{}{0pt}%
\pgfpathmoveto{\pgfqpoint{4.426686in}{4.680141in}}%
\pgfpathlineto{\pgfqpoint{6.534666in}{4.191861in}}%
\pgfusepath{stroke}%
\end{pgfscope}%
\begin{pgfscope}%
\pgfpathrectangle{\pgfqpoint{0.393613in}{0.331635in}}{\pgfqpoint{9.300000in}{7.700000in}}%
\pgfusepath{clip}%
\pgfsetrectcap%
\pgfsetroundjoin%
\pgfsetlinewidth{1.505625pt}%
\definecolor{currentstroke}{rgb}{1.000000,0.705882,0.509804}%
\pgfsetstrokecolor{currentstroke}%
\pgfsetstrokeopacity{0.800000}%
\pgfsetdash{}{0pt}%
\pgfpathmoveto{\pgfqpoint{6.115172in}{3.529359in}}%
\pgfpathlineto{\pgfqpoint{6.534666in}{4.191861in}}%
\pgfusepath{stroke}%
\end{pgfscope}%
\begin{pgfscope}%
\pgfpathrectangle{\pgfqpoint{0.393613in}{0.331635in}}{\pgfqpoint{9.300000in}{7.700000in}}%
\pgfusepath{clip}%
\pgfsetrectcap%
\pgfsetroundjoin%
\pgfsetlinewidth{1.505625pt}%
\definecolor{currentstroke}{rgb}{1.000000,0.705882,0.509804}%
\pgfsetstrokecolor{currentstroke}%
\pgfsetstrokeopacity{0.800000}%
\pgfsetdash{}{0pt}%
\pgfpathmoveto{\pgfqpoint{5.118229in}{2.407043in}}%
\pgfpathlineto{\pgfqpoint{6.534666in}{4.191861in}}%
\pgfusepath{stroke}%
\end{pgfscope}%
\begin{pgfscope}%
\pgfpathrectangle{\pgfqpoint{0.393613in}{0.331635in}}{\pgfqpoint{9.300000in}{7.700000in}}%
\pgfusepath{clip}%
\pgfsetrectcap%
\pgfsetroundjoin%
\pgfsetlinewidth{1.505625pt}%
\definecolor{currentstroke}{rgb}{1.000000,0.705882,0.509804}%
\pgfsetstrokecolor{currentstroke}%
\pgfsetstrokeopacity{0.800000}%
\pgfsetdash{}{0pt}%
\pgfpathmoveto{\pgfqpoint{6.785951in}{6.348765in}}%
\pgfpathlineto{\pgfqpoint{6.534666in}{4.191861in}}%
\pgfusepath{stroke}%
\end{pgfscope}%
\begin{pgfscope}%
\pgfpathrectangle{\pgfqpoint{0.393613in}{0.331635in}}{\pgfqpoint{9.300000in}{7.700000in}}%
\pgfusepath{clip}%
\pgfsetrectcap%
\pgfsetroundjoin%
\pgfsetlinewidth{1.505625pt}%
\definecolor{currentstroke}{rgb}{1.000000,0.705882,0.509804}%
\pgfsetstrokecolor{currentstroke}%
\pgfsetstrokeopacity{0.800000}%
\pgfsetdash{}{0pt}%
\pgfpathmoveto{\pgfqpoint{4.549543in}{2.869784in}}%
\pgfpathlineto{\pgfqpoint{6.534666in}{4.191861in}}%
\pgfusepath{stroke}%
\end{pgfscope}%
\begin{pgfscope}%
\pgfpathrectangle{\pgfqpoint{0.393613in}{0.331635in}}{\pgfqpoint{9.300000in}{7.700000in}}%
\pgfusepath{clip}%
\pgfsetrectcap%
\pgfsetroundjoin%
\pgfsetlinewidth{1.505625pt}%
\definecolor{currentstroke}{rgb}{1.000000,0.705882,0.509804}%
\pgfsetstrokecolor{currentstroke}%
\pgfsetstrokeopacity{0.800000}%
\pgfsetdash{}{0pt}%
\pgfpathmoveto{\pgfqpoint{7.362784in}{4.456419in}}%
\pgfpathlineto{\pgfqpoint{6.534666in}{4.191861in}}%
\pgfusepath{stroke}%
\end{pgfscope}%
\begin{pgfscope}%
\pgfpathrectangle{\pgfqpoint{0.393613in}{0.331635in}}{\pgfqpoint{9.300000in}{7.700000in}}%
\pgfusepath{clip}%
\pgfsetrectcap%
\pgfsetroundjoin%
\pgfsetlinewidth{1.505625pt}%
\definecolor{currentstroke}{rgb}{1.000000,0.705882,0.509804}%
\pgfsetstrokecolor{currentstroke}%
\pgfsetstrokeopacity{0.800000}%
\pgfsetdash{}{0pt}%
\pgfpathmoveto{\pgfqpoint{3.789706in}{4.975890in}}%
\pgfpathlineto{\pgfqpoint{6.534666in}{4.191861in}}%
\pgfusepath{stroke}%
\end{pgfscope}%
\begin{pgfscope}%
\pgfsetrectcap%
\pgfsetmiterjoin%
\pgfsetlinewidth{0.803000pt}%
\definecolor{currentstroke}{rgb}{0.000000,0.000000,0.000000}%
\pgfsetstrokecolor{currentstroke}%
\pgfsetdash{}{0pt}%
\pgfpathmoveto{\pgfqpoint{0.393613in}{0.331635in}}%
\pgfpathlineto{\pgfqpoint{0.393613in}{8.031635in}}%
\pgfusepath{stroke}%
\end{pgfscope}%
\begin{pgfscope}%
\pgfsetrectcap%
\pgfsetmiterjoin%
\pgfsetlinewidth{0.803000pt}%
\definecolor{currentstroke}{rgb}{0.000000,0.000000,0.000000}%
\pgfsetstrokecolor{currentstroke}%
\pgfsetdash{}{0pt}%
\pgfpathmoveto{\pgfqpoint{9.693613in}{0.331635in}}%
\pgfpathlineto{\pgfqpoint{9.693613in}{8.031635in}}%
\pgfusepath{stroke}%
\end{pgfscope}%
\begin{pgfscope}%
\pgfsetrectcap%
\pgfsetmiterjoin%
\pgfsetlinewidth{0.803000pt}%
\definecolor{currentstroke}{rgb}{0.000000,0.000000,0.000000}%
\pgfsetstrokecolor{currentstroke}%
\pgfsetdash{}{0pt}%
\pgfpathmoveto{\pgfqpoint{0.393613in}{0.331635in}}%
\pgfpathlineto{\pgfqpoint{9.693612in}{0.331635in}}%
\pgfusepath{stroke}%
\end{pgfscope}%
\begin{pgfscope}%
\pgfsetrectcap%
\pgfsetmiterjoin%
\pgfsetlinewidth{0.803000pt}%
\definecolor{currentstroke}{rgb}{0.000000,0.000000,0.000000}%
\pgfsetstrokecolor{currentstroke}%
\pgfsetdash{}{0pt}%
\pgfpathmoveto{\pgfqpoint{0.393613in}{8.031635in}}%
\pgfpathlineto{\pgfqpoint{9.693612in}{8.031635in}}%
\pgfusepath{stroke}%
\end{pgfscope}%
\begin{pgfscope}%
\definecolor{textcolor}{rgb}{0.000000,0.000000,0.000000}%
\pgfsetstrokecolor{textcolor}%
\pgfsetfillcolor{textcolor}%
\pgftext[x=5.043613in,y=8.114968in,,base]{\color{textcolor}\sffamily\fontsize{12.000000}{14.400000}\selectfont T-SNE for chair images (s2r3dfree\_textureless\_light)}%
\end{pgfscope}%
\begin{pgfscope}%
\pgfsetbuttcap%
\pgfsetmiterjoin%
\definecolor{currentfill}{rgb}{1.000000,1.000000,1.000000}%
\pgfsetfillcolor{currentfill}%
\pgfsetfillopacity{0.800000}%
\pgfsetlinewidth{1.003750pt}%
\definecolor{currentstroke}{rgb}{0.800000,0.800000,0.800000}%
\pgfsetstrokecolor{currentstroke}%
\pgfsetstrokeopacity{0.800000}%
\pgfsetdash{}{0pt}%
\pgfpathmoveto{\pgfqpoint{9.790835in}{3.955012in}}%
\pgfpathlineto{\pgfqpoint{12.120587in}{3.955012in}}%
\pgfpathquadraticcurveto{\pgfqpoint{12.148365in}{3.955012in}}{\pgfqpoint{12.148365in}{3.982789in}}%
\pgfpathlineto{\pgfqpoint{12.148365in}{4.380481in}}%
\pgfpathquadraticcurveto{\pgfqpoint{12.148365in}{4.408258in}}{\pgfqpoint{12.120587in}{4.408258in}}%
\pgfpathlineto{\pgfqpoint{9.790835in}{4.408258in}}%
\pgfpathquadraticcurveto{\pgfqpoint{9.763057in}{4.408258in}}{\pgfqpoint{9.763057in}{4.380481in}}%
\pgfpathlineto{\pgfqpoint{9.763057in}{3.982789in}}%
\pgfpathquadraticcurveto{\pgfqpoint{9.763057in}{3.955012in}}{\pgfqpoint{9.790835in}{3.955012in}}%
\pgfpathclose%
\pgfusepath{stroke,fill}%
\end{pgfscope}%
\begin{pgfscope}%
\pgfsetbuttcap%
\pgfsetroundjoin%
\definecolor{currentfill}{rgb}{0.631373,0.788235,0.956863}%
\pgfsetfillcolor{currentfill}%
\pgfsetlinewidth{1.003750pt}%
\definecolor{currentstroke}{rgb}{0.631373,0.788235,0.956863}%
\pgfsetstrokecolor{currentstroke}%
\pgfsetdash{}{0pt}%
\pgfsys@defobject{currentmarker}{\pgfqpoint{-0.041667in}{-0.041667in}}{\pgfqpoint{0.041667in}{0.041667in}}{%
\pgfpathmoveto{\pgfqpoint{0.000000in}{-0.041667in}}%
\pgfpathcurveto{\pgfqpoint{0.011050in}{-0.041667in}}{\pgfqpoint{0.021649in}{-0.037276in}}{\pgfqpoint{0.029463in}{-0.029463in}}%
\pgfpathcurveto{\pgfqpoint{0.037276in}{-0.021649in}}{\pgfqpoint{0.041667in}{-0.011050in}}{\pgfqpoint{0.041667in}{0.000000in}}%
\pgfpathcurveto{\pgfqpoint{0.041667in}{0.011050in}}{\pgfqpoint{0.037276in}{0.021649in}}{\pgfqpoint{0.029463in}{0.029463in}}%
\pgfpathcurveto{\pgfqpoint{0.021649in}{0.037276in}}{\pgfqpoint{0.011050in}{0.041667in}}{\pgfqpoint{0.000000in}{0.041667in}}%
\pgfpathcurveto{\pgfqpoint{-0.011050in}{0.041667in}}{\pgfqpoint{-0.021649in}{0.037276in}}{\pgfqpoint{-0.029463in}{0.029463in}}%
\pgfpathcurveto{\pgfqpoint{-0.037276in}{0.021649in}}{\pgfqpoint{-0.041667in}{0.011050in}}{\pgfqpoint{-0.041667in}{0.000000in}}%
\pgfpathcurveto{\pgfqpoint{-0.041667in}{-0.011050in}}{\pgfqpoint{-0.037276in}{-0.021649in}}{\pgfqpoint{-0.029463in}{-0.029463in}}%
\pgfpathcurveto{\pgfqpoint{-0.021649in}{-0.037276in}}{\pgfqpoint{-0.011050in}{-0.041667in}}{\pgfqpoint{0.000000in}{-0.041667in}}%
\pgfpathclose%
\pgfusepath{stroke,fill}%
}%
\begin{pgfscope}%
\pgfsys@transformshift{9.957501in}{4.283638in}%
\pgfsys@useobject{currentmarker}{}%
\end{pgfscope}%
\end{pgfscope}%
\begin{pgfscope}%
\definecolor{textcolor}{rgb}{0.000000,0.000000,0.000000}%
\pgfsetstrokecolor{textcolor}%
\pgfsetfillcolor{textcolor}%
\pgftext[x=10.207501in,y=4.247180in,left,base]{\color{textcolor}\sffamily\fontsize{10.000000}{12.000000}\selectfont Pix3D}%
\end{pgfscope}%
\begin{pgfscope}%
\pgfsetbuttcap%
\pgfsetroundjoin%
\definecolor{currentfill}{rgb}{1.000000,0.705882,0.509804}%
\pgfsetfillcolor{currentfill}%
\pgfsetlinewidth{1.003750pt}%
\definecolor{currentstroke}{rgb}{1.000000,0.705882,0.509804}%
\pgfsetstrokecolor{currentstroke}%
\pgfsetdash{}{0pt}%
\pgfsys@defobject{currentmarker}{\pgfqpoint{-0.041667in}{-0.041667in}}{\pgfqpoint{0.041667in}{0.041667in}}{%
\pgfpathmoveto{\pgfqpoint{0.000000in}{-0.041667in}}%
\pgfpathcurveto{\pgfqpoint{0.011050in}{-0.041667in}}{\pgfqpoint{0.021649in}{-0.037276in}}{\pgfqpoint{0.029463in}{-0.029463in}}%
\pgfpathcurveto{\pgfqpoint{0.037276in}{-0.021649in}}{\pgfqpoint{0.041667in}{-0.011050in}}{\pgfqpoint{0.041667in}{0.000000in}}%
\pgfpathcurveto{\pgfqpoint{0.041667in}{0.011050in}}{\pgfqpoint{0.037276in}{0.021649in}}{\pgfqpoint{0.029463in}{0.029463in}}%
\pgfpathcurveto{\pgfqpoint{0.021649in}{0.037276in}}{\pgfqpoint{0.011050in}{0.041667in}}{\pgfqpoint{0.000000in}{0.041667in}}%
\pgfpathcurveto{\pgfqpoint{-0.011050in}{0.041667in}}{\pgfqpoint{-0.021649in}{0.037276in}}{\pgfqpoint{-0.029463in}{0.029463in}}%
\pgfpathcurveto{\pgfqpoint{-0.037276in}{0.021649in}}{\pgfqpoint{-0.041667in}{0.011050in}}{\pgfqpoint{-0.041667in}{0.000000in}}%
\pgfpathcurveto{\pgfqpoint{-0.041667in}{-0.011050in}}{\pgfqpoint{-0.037276in}{-0.021649in}}{\pgfqpoint{-0.029463in}{-0.029463in}}%
\pgfpathcurveto{\pgfqpoint{-0.021649in}{-0.037276in}}{\pgfqpoint{-0.011050in}{-0.041667in}}{\pgfqpoint{0.000000in}{-0.041667in}}%
\pgfpathclose%
\pgfusepath{stroke,fill}%
}%
\begin{pgfscope}%
\pgfsys@transformshift{9.957501in}{4.079781in}%
\pgfsys@useobject{currentmarker}{}%
\end{pgfscope}%
\end{pgfscope}%
\begin{pgfscope}%
\definecolor{textcolor}{rgb}{0.000000,0.000000,0.000000}%
\pgfsetstrokecolor{textcolor}%
\pgfsetfillcolor{textcolor}%
\pgftext[x=10.207501in,y=4.043322in,left,base]{\color{textcolor}\sffamily\fontsize{10.000000}{12.000000}\selectfont s2r3dfree\_textureless\_light}%
\end{pgfscope}%
\end{pgfpicture}%
\makeatother%
\endgroup%
}\\
    \resizebox{0.49\linewidth}{5cm}{%% Creator: Matplotlib, PGF backend
%%
%% To include the figure in your LaTeX document, write
%%   \input{<filename>.pgf}
%%
%% Make sure the required packages are loaded in your preamble
%%   \usepackage{pgf}
%%
%% Figures using additional raster images can only be included by \input if
%% they are in the same directory as the main LaTeX file. For loading figures
%% from other directories you can use the `import` package
%%   \usepackage{import}
%%
%% and then include the figures with
%%   \import{<path to file>}{<filename>.pgf}
%%
%% Matplotlib used the following preamble
%%   \usepackage{fontspec}
%%   \setmainfont{DejaVuSerif.ttf}[Path=\detokenize{/Users/apple/opt/anaconda3/envs/kaolin/lib/python3.7/site-packages/matplotlib/mpl-data/fonts/ttf/}]
%%   \setsansfont{DejaVuSans.ttf}[Path=\detokenize{/Users/apple/opt/anaconda3/envs/kaolin/lib/python3.7/site-packages/matplotlib/mpl-data/fonts/ttf/}]
%%   \setmonofont{DejaVuSansMono.ttf}[Path=\detokenize{/Users/apple/opt/anaconda3/envs/kaolin/lib/python3.7/site-packages/matplotlib/mpl-data/fonts/ttf/}]
%%
\begingroup%
\makeatletter%
\begin{pgfpicture}%
\pgfpathrectangle{\pgfpointorigin}{\pgfqpoint{12.010532in}{8.341596in}}%
\pgfusepath{use as bounding box, clip}%
\begin{pgfscope}%
\pgfsetbuttcap%
\pgfsetmiterjoin%
\definecolor{currentfill}{rgb}{1.000000,1.000000,1.000000}%
\pgfsetfillcolor{currentfill}%
\pgfsetlinewidth{0.000000pt}%
\definecolor{currentstroke}{rgb}{1.000000,1.000000,1.000000}%
\pgfsetstrokecolor{currentstroke}%
\pgfsetdash{}{0pt}%
\pgfpathmoveto{\pgfqpoint{0.000000in}{0.000000in}}%
\pgfpathlineto{\pgfqpoint{12.010532in}{0.000000in}}%
\pgfpathlineto{\pgfqpoint{12.010532in}{8.341596in}}%
\pgfpathlineto{\pgfqpoint{0.000000in}{8.341596in}}%
\pgfpathclose%
\pgfusepath{fill}%
\end{pgfscope}%
\begin{pgfscope}%
\pgfsetbuttcap%
\pgfsetmiterjoin%
\definecolor{currentfill}{rgb}{1.000000,1.000000,1.000000}%
\pgfsetfillcolor{currentfill}%
\pgfsetlinewidth{0.000000pt}%
\definecolor{currentstroke}{rgb}{0.000000,0.000000,0.000000}%
\pgfsetstrokecolor{currentstroke}%
\pgfsetstrokeopacity{0.000000}%
\pgfsetdash{}{0pt}%
\pgfpathmoveto{\pgfqpoint{0.481978in}{0.331635in}}%
\pgfpathlineto{\pgfqpoint{9.781978in}{0.331635in}}%
\pgfpathlineto{\pgfqpoint{9.781978in}{8.031635in}}%
\pgfpathlineto{\pgfqpoint{0.481978in}{8.031635in}}%
\pgfpathclose%
\pgfusepath{fill}%
\end{pgfscope}%
\begin{pgfscope}%
\pgfpathrectangle{\pgfqpoint{0.481978in}{0.331635in}}{\pgfqpoint{9.300000in}{7.700000in}}%
\pgfusepath{clip}%
\pgfsetbuttcap%
\pgfsetroundjoin%
\definecolor{currentfill}{rgb}{0.631373,0.788235,0.956863}%
\pgfsetfillcolor{currentfill}%
\pgfsetlinewidth{0.481800pt}%
\definecolor{currentstroke}{rgb}{1.000000,1.000000,1.000000}%
\pgfsetstrokecolor{currentstroke}%
\pgfsetdash{}{0pt}%
\pgfpathmoveto{\pgfqpoint{6.274584in}{7.535334in}}%
\pgfpathcurveto{\pgfqpoint{6.285634in}{7.535334in}}{\pgfqpoint{6.296233in}{7.539724in}}{\pgfqpoint{6.304047in}{7.547538in}}%
\pgfpathcurveto{\pgfqpoint{6.311860in}{7.555352in}}{\pgfqpoint{6.316251in}{7.565951in}}{\pgfqpoint{6.316251in}{7.577001in}}%
\pgfpathcurveto{\pgfqpoint{6.316251in}{7.588051in}}{\pgfqpoint{6.311860in}{7.598650in}}{\pgfqpoint{6.304047in}{7.606463in}}%
\pgfpathcurveto{\pgfqpoint{6.296233in}{7.614277in}}{\pgfqpoint{6.285634in}{7.618667in}}{\pgfqpoint{6.274584in}{7.618667in}}%
\pgfpathcurveto{\pgfqpoint{6.263534in}{7.618667in}}{\pgfqpoint{6.252935in}{7.614277in}}{\pgfqpoint{6.245121in}{7.606463in}}%
\pgfpathcurveto{\pgfqpoint{6.237308in}{7.598650in}}{\pgfqpoint{6.232917in}{7.588051in}}{\pgfqpoint{6.232917in}{7.577001in}}%
\pgfpathcurveto{\pgfqpoint{6.232917in}{7.565951in}}{\pgfqpoint{6.237308in}{7.555352in}}{\pgfqpoint{6.245121in}{7.547538in}}%
\pgfpathcurveto{\pgfqpoint{6.252935in}{7.539724in}}{\pgfqpoint{6.263534in}{7.535334in}}{\pgfqpoint{6.274584in}{7.535334in}}%
\pgfpathclose%
\pgfusepath{stroke,fill}%
\end{pgfscope}%
\begin{pgfscope}%
\pgfpathrectangle{\pgfqpoint{0.481978in}{0.331635in}}{\pgfqpoint{9.300000in}{7.700000in}}%
\pgfusepath{clip}%
\pgfsetbuttcap%
\pgfsetroundjoin%
\definecolor{currentfill}{rgb}{0.631373,0.788235,0.956863}%
\pgfsetfillcolor{currentfill}%
\pgfsetlinewidth{0.481800pt}%
\definecolor{currentstroke}{rgb}{1.000000,1.000000,1.000000}%
\pgfsetstrokecolor{currentstroke}%
\pgfsetdash{}{0pt}%
\pgfpathmoveto{\pgfqpoint{2.282958in}{5.043748in}}%
\pgfpathcurveto{\pgfqpoint{2.294008in}{5.043748in}}{\pgfqpoint{2.304607in}{5.048138in}}{\pgfqpoint{2.312421in}{5.055952in}}%
\pgfpathcurveto{\pgfqpoint{2.320234in}{5.063765in}}{\pgfqpoint{2.324625in}{5.074364in}}{\pgfqpoint{2.324625in}{5.085415in}}%
\pgfpathcurveto{\pgfqpoint{2.324625in}{5.096465in}}{\pgfqpoint{2.320234in}{5.107064in}}{\pgfqpoint{2.312421in}{5.114877in}}%
\pgfpathcurveto{\pgfqpoint{2.304607in}{5.122691in}}{\pgfqpoint{2.294008in}{5.127081in}}{\pgfqpoint{2.282958in}{5.127081in}}%
\pgfpathcurveto{\pgfqpoint{2.271908in}{5.127081in}}{\pgfqpoint{2.261309in}{5.122691in}}{\pgfqpoint{2.253495in}{5.114877in}}%
\pgfpathcurveto{\pgfqpoint{2.245682in}{5.107064in}}{\pgfqpoint{2.241291in}{5.096465in}}{\pgfqpoint{2.241291in}{5.085415in}}%
\pgfpathcurveto{\pgfqpoint{2.241291in}{5.074364in}}{\pgfqpoint{2.245682in}{5.063765in}}{\pgfqpoint{2.253495in}{5.055952in}}%
\pgfpathcurveto{\pgfqpoint{2.261309in}{5.048138in}}{\pgfqpoint{2.271908in}{5.043748in}}{\pgfqpoint{2.282958in}{5.043748in}}%
\pgfpathclose%
\pgfusepath{stroke,fill}%
\end{pgfscope}%
\begin{pgfscope}%
\pgfpathrectangle{\pgfqpoint{0.481978in}{0.331635in}}{\pgfqpoint{9.300000in}{7.700000in}}%
\pgfusepath{clip}%
\pgfsetbuttcap%
\pgfsetroundjoin%
\definecolor{currentfill}{rgb}{0.631373,0.788235,0.956863}%
\pgfsetfillcolor{currentfill}%
\pgfsetlinewidth{0.481800pt}%
\definecolor{currentstroke}{rgb}{1.000000,1.000000,1.000000}%
\pgfsetstrokecolor{currentstroke}%
\pgfsetdash{}{0pt}%
\pgfpathmoveto{\pgfqpoint{4.404087in}{5.116998in}}%
\pgfpathcurveto{\pgfqpoint{4.415137in}{5.116998in}}{\pgfqpoint{4.425736in}{5.121388in}}{\pgfqpoint{4.433550in}{5.129202in}}%
\pgfpathcurveto{\pgfqpoint{4.441363in}{5.137015in}}{\pgfqpoint{4.445754in}{5.147614in}}{\pgfqpoint{4.445754in}{5.158664in}}%
\pgfpathcurveto{\pgfqpoint{4.445754in}{5.169715in}}{\pgfqpoint{4.441363in}{5.180314in}}{\pgfqpoint{4.433550in}{5.188127in}}%
\pgfpathcurveto{\pgfqpoint{4.425736in}{5.195941in}}{\pgfqpoint{4.415137in}{5.200331in}}{\pgfqpoint{4.404087in}{5.200331in}}%
\pgfpathcurveto{\pgfqpoint{4.393037in}{5.200331in}}{\pgfqpoint{4.382438in}{5.195941in}}{\pgfqpoint{4.374624in}{5.188127in}}%
\pgfpathcurveto{\pgfqpoint{4.366811in}{5.180314in}}{\pgfqpoint{4.362420in}{5.169715in}}{\pgfqpoint{4.362420in}{5.158664in}}%
\pgfpathcurveto{\pgfqpoint{4.362420in}{5.147614in}}{\pgfqpoint{4.366811in}{5.137015in}}{\pgfqpoint{4.374624in}{5.129202in}}%
\pgfpathcurveto{\pgfqpoint{4.382438in}{5.121388in}}{\pgfqpoint{4.393037in}{5.116998in}}{\pgfqpoint{4.404087in}{5.116998in}}%
\pgfpathclose%
\pgfusepath{stroke,fill}%
\end{pgfscope}%
\begin{pgfscope}%
\pgfpathrectangle{\pgfqpoint{0.481978in}{0.331635in}}{\pgfqpoint{9.300000in}{7.700000in}}%
\pgfusepath{clip}%
\pgfsetbuttcap%
\pgfsetroundjoin%
\definecolor{currentfill}{rgb}{0.631373,0.788235,0.956863}%
\pgfsetfillcolor{currentfill}%
\pgfsetlinewidth{0.481800pt}%
\definecolor{currentstroke}{rgb}{1.000000,1.000000,1.000000}%
\pgfsetstrokecolor{currentstroke}%
\pgfsetdash{}{0pt}%
\pgfpathmoveto{\pgfqpoint{4.342376in}{5.615016in}}%
\pgfpathcurveto{\pgfqpoint{4.353426in}{5.615016in}}{\pgfqpoint{4.364025in}{5.619406in}}{\pgfqpoint{4.371839in}{5.627220in}}%
\pgfpathcurveto{\pgfqpoint{4.379653in}{5.635033in}}{\pgfqpoint{4.384043in}{5.645632in}}{\pgfqpoint{4.384043in}{5.656682in}}%
\pgfpathcurveto{\pgfqpoint{4.384043in}{5.667733in}}{\pgfqpoint{4.379653in}{5.678332in}}{\pgfqpoint{4.371839in}{5.686145in}}%
\pgfpathcurveto{\pgfqpoint{4.364025in}{5.693959in}}{\pgfqpoint{4.353426in}{5.698349in}}{\pgfqpoint{4.342376in}{5.698349in}}%
\pgfpathcurveto{\pgfqpoint{4.331326in}{5.698349in}}{\pgfqpoint{4.320727in}{5.693959in}}{\pgfqpoint{4.312913in}{5.686145in}}%
\pgfpathcurveto{\pgfqpoint{4.305100in}{5.678332in}}{\pgfqpoint{4.300710in}{5.667733in}}{\pgfqpoint{4.300710in}{5.656682in}}%
\pgfpathcurveto{\pgfqpoint{4.300710in}{5.645632in}}{\pgfqpoint{4.305100in}{5.635033in}}{\pgfqpoint{4.312913in}{5.627220in}}%
\pgfpathcurveto{\pgfqpoint{4.320727in}{5.619406in}}{\pgfqpoint{4.331326in}{5.615016in}}{\pgfqpoint{4.342376in}{5.615016in}}%
\pgfpathclose%
\pgfusepath{stroke,fill}%
\end{pgfscope}%
\begin{pgfscope}%
\pgfpathrectangle{\pgfqpoint{0.481978in}{0.331635in}}{\pgfqpoint{9.300000in}{7.700000in}}%
\pgfusepath{clip}%
\pgfsetbuttcap%
\pgfsetroundjoin%
\definecolor{currentfill}{rgb}{0.631373,0.788235,0.956863}%
\pgfsetfillcolor{currentfill}%
\pgfsetlinewidth{0.481800pt}%
\definecolor{currentstroke}{rgb}{1.000000,1.000000,1.000000}%
\pgfsetstrokecolor{currentstroke}%
\pgfsetdash{}{0pt}%
\pgfpathmoveto{\pgfqpoint{6.817897in}{7.226647in}}%
\pgfpathcurveto{\pgfqpoint{6.828948in}{7.226647in}}{\pgfqpoint{6.839547in}{7.231037in}}{\pgfqpoint{6.847360in}{7.238851in}}%
\pgfpathcurveto{\pgfqpoint{6.855174in}{7.246665in}}{\pgfqpoint{6.859564in}{7.257264in}}{\pgfqpoint{6.859564in}{7.268314in}}%
\pgfpathcurveto{\pgfqpoint{6.859564in}{7.279364in}}{\pgfqpoint{6.855174in}{7.289963in}}{\pgfqpoint{6.847360in}{7.297777in}}%
\pgfpathcurveto{\pgfqpoint{6.839547in}{7.305590in}}{\pgfqpoint{6.828948in}{7.309980in}}{\pgfqpoint{6.817897in}{7.309980in}}%
\pgfpathcurveto{\pgfqpoint{6.806847in}{7.309980in}}{\pgfqpoint{6.796248in}{7.305590in}}{\pgfqpoint{6.788435in}{7.297777in}}%
\pgfpathcurveto{\pgfqpoint{6.780621in}{7.289963in}}{\pgfqpoint{6.776231in}{7.279364in}}{\pgfqpoint{6.776231in}{7.268314in}}%
\pgfpathcurveto{\pgfqpoint{6.776231in}{7.257264in}}{\pgfqpoint{6.780621in}{7.246665in}}{\pgfqpoint{6.788435in}{7.238851in}}%
\pgfpathcurveto{\pgfqpoint{6.796248in}{7.231037in}}{\pgfqpoint{6.806847in}{7.226647in}}{\pgfqpoint{6.817897in}{7.226647in}}%
\pgfpathclose%
\pgfusepath{stroke,fill}%
\end{pgfscope}%
\begin{pgfscope}%
\pgfpathrectangle{\pgfqpoint{0.481978in}{0.331635in}}{\pgfqpoint{9.300000in}{7.700000in}}%
\pgfusepath{clip}%
\pgfsetbuttcap%
\pgfsetroundjoin%
\definecolor{currentfill}{rgb}{0.631373,0.788235,0.956863}%
\pgfsetfillcolor{currentfill}%
\pgfsetlinewidth{0.481800pt}%
\definecolor{currentstroke}{rgb}{1.000000,1.000000,1.000000}%
\pgfsetstrokecolor{currentstroke}%
\pgfsetdash{}{0pt}%
\pgfpathmoveto{\pgfqpoint{1.831950in}{3.925507in}}%
\pgfpathcurveto{\pgfqpoint{1.843001in}{3.925507in}}{\pgfqpoint{1.853600in}{3.929898in}}{\pgfqpoint{1.861413in}{3.937711in}}%
\pgfpathcurveto{\pgfqpoint{1.869227in}{3.945525in}}{\pgfqpoint{1.873617in}{3.956124in}}{\pgfqpoint{1.873617in}{3.967174in}}%
\pgfpathcurveto{\pgfqpoint{1.873617in}{3.978224in}}{\pgfqpoint{1.869227in}{3.988823in}}{\pgfqpoint{1.861413in}{3.996637in}}%
\pgfpathcurveto{\pgfqpoint{1.853600in}{4.004450in}}{\pgfqpoint{1.843001in}{4.008841in}}{\pgfqpoint{1.831950in}{4.008841in}}%
\pgfpathcurveto{\pgfqpoint{1.820900in}{4.008841in}}{\pgfqpoint{1.810301in}{4.004450in}}{\pgfqpoint{1.802488in}{3.996637in}}%
\pgfpathcurveto{\pgfqpoint{1.794674in}{3.988823in}}{\pgfqpoint{1.790284in}{3.978224in}}{\pgfqpoint{1.790284in}{3.967174in}}%
\pgfpathcurveto{\pgfqpoint{1.790284in}{3.956124in}}{\pgfqpoint{1.794674in}{3.945525in}}{\pgfqpoint{1.802488in}{3.937711in}}%
\pgfpathcurveto{\pgfqpoint{1.810301in}{3.929898in}}{\pgfqpoint{1.820900in}{3.925507in}}{\pgfqpoint{1.831950in}{3.925507in}}%
\pgfpathclose%
\pgfusepath{stroke,fill}%
\end{pgfscope}%
\begin{pgfscope}%
\pgfpathrectangle{\pgfqpoint{0.481978in}{0.331635in}}{\pgfqpoint{9.300000in}{7.700000in}}%
\pgfusepath{clip}%
\pgfsetbuttcap%
\pgfsetroundjoin%
\definecolor{currentfill}{rgb}{0.631373,0.788235,0.956863}%
\pgfsetfillcolor{currentfill}%
\pgfsetlinewidth{0.481800pt}%
\definecolor{currentstroke}{rgb}{1.000000,1.000000,1.000000}%
\pgfsetstrokecolor{currentstroke}%
\pgfsetdash{}{0pt}%
\pgfpathmoveto{\pgfqpoint{6.806036in}{5.897988in}}%
\pgfpathcurveto{\pgfqpoint{6.817086in}{5.897988in}}{\pgfqpoint{6.827685in}{5.902378in}}{\pgfqpoint{6.835499in}{5.910192in}}%
\pgfpathcurveto{\pgfqpoint{6.843313in}{5.918005in}}{\pgfqpoint{6.847703in}{5.928604in}}{\pgfqpoint{6.847703in}{5.939655in}}%
\pgfpathcurveto{\pgfqpoint{6.847703in}{5.950705in}}{\pgfqpoint{6.843313in}{5.961304in}}{\pgfqpoint{6.835499in}{5.969117in}}%
\pgfpathcurveto{\pgfqpoint{6.827685in}{5.976931in}}{\pgfqpoint{6.817086in}{5.981321in}}{\pgfqpoint{6.806036in}{5.981321in}}%
\pgfpathcurveto{\pgfqpoint{6.794986in}{5.981321in}}{\pgfqpoint{6.784387in}{5.976931in}}{\pgfqpoint{6.776574in}{5.969117in}}%
\pgfpathcurveto{\pgfqpoint{6.768760in}{5.961304in}}{\pgfqpoint{6.764370in}{5.950705in}}{\pgfqpoint{6.764370in}{5.939655in}}%
\pgfpathcurveto{\pgfqpoint{6.764370in}{5.928604in}}{\pgfqpoint{6.768760in}{5.918005in}}{\pgfqpoint{6.776574in}{5.910192in}}%
\pgfpathcurveto{\pgfqpoint{6.784387in}{5.902378in}}{\pgfqpoint{6.794986in}{5.897988in}}{\pgfqpoint{6.806036in}{5.897988in}}%
\pgfpathclose%
\pgfusepath{stroke,fill}%
\end{pgfscope}%
\begin{pgfscope}%
\pgfpathrectangle{\pgfqpoint{0.481978in}{0.331635in}}{\pgfqpoint{9.300000in}{7.700000in}}%
\pgfusepath{clip}%
\pgfsetbuttcap%
\pgfsetroundjoin%
\definecolor{currentfill}{rgb}{0.631373,0.788235,0.956863}%
\pgfsetfillcolor{currentfill}%
\pgfsetlinewidth{0.481800pt}%
\definecolor{currentstroke}{rgb}{1.000000,1.000000,1.000000}%
\pgfsetstrokecolor{currentstroke}%
\pgfsetdash{}{0pt}%
\pgfpathmoveto{\pgfqpoint{3.985143in}{5.351284in}}%
\pgfpathcurveto{\pgfqpoint{3.996194in}{5.351284in}}{\pgfqpoint{4.006793in}{5.355675in}}{\pgfqpoint{4.014606in}{5.363488in}}%
\pgfpathcurveto{\pgfqpoint{4.022420in}{5.371302in}}{\pgfqpoint{4.026810in}{5.381901in}}{\pgfqpoint{4.026810in}{5.392951in}}%
\pgfpathcurveto{\pgfqpoint{4.026810in}{5.404001in}}{\pgfqpoint{4.022420in}{5.414600in}}{\pgfqpoint{4.014606in}{5.422414in}}%
\pgfpathcurveto{\pgfqpoint{4.006793in}{5.430227in}}{\pgfqpoint{3.996194in}{5.434618in}}{\pgfqpoint{3.985143in}{5.434618in}}%
\pgfpathcurveto{\pgfqpoint{3.974093in}{5.434618in}}{\pgfqpoint{3.963494in}{5.430227in}}{\pgfqpoint{3.955681in}{5.422414in}}%
\pgfpathcurveto{\pgfqpoint{3.947867in}{5.414600in}}{\pgfqpoint{3.943477in}{5.404001in}}{\pgfqpoint{3.943477in}{5.392951in}}%
\pgfpathcurveto{\pgfqpoint{3.943477in}{5.381901in}}{\pgfqpoint{3.947867in}{5.371302in}}{\pgfqpoint{3.955681in}{5.363488in}}%
\pgfpathcurveto{\pgfqpoint{3.963494in}{5.355675in}}{\pgfqpoint{3.974093in}{5.351284in}}{\pgfqpoint{3.985143in}{5.351284in}}%
\pgfpathclose%
\pgfusepath{stroke,fill}%
\end{pgfscope}%
\begin{pgfscope}%
\pgfpathrectangle{\pgfqpoint{0.481978in}{0.331635in}}{\pgfqpoint{9.300000in}{7.700000in}}%
\pgfusepath{clip}%
\pgfsetbuttcap%
\pgfsetroundjoin%
\definecolor{currentfill}{rgb}{0.631373,0.788235,0.956863}%
\pgfsetfillcolor{currentfill}%
\pgfsetlinewidth{0.481800pt}%
\definecolor{currentstroke}{rgb}{1.000000,1.000000,1.000000}%
\pgfsetstrokecolor{currentstroke}%
\pgfsetdash{}{0pt}%
\pgfpathmoveto{\pgfqpoint{4.751471in}{3.564492in}}%
\pgfpathcurveto{\pgfqpoint{4.762521in}{3.564492in}}{\pgfqpoint{4.773120in}{3.568883in}}{\pgfqpoint{4.780934in}{3.576696in}}%
\pgfpathcurveto{\pgfqpoint{4.788747in}{3.584510in}}{\pgfqpoint{4.793137in}{3.595109in}}{\pgfqpoint{4.793137in}{3.606159in}}%
\pgfpathcurveto{\pgfqpoint{4.793137in}{3.617209in}}{\pgfqpoint{4.788747in}{3.627808in}}{\pgfqpoint{4.780934in}{3.635622in}}%
\pgfpathcurveto{\pgfqpoint{4.773120in}{3.643435in}}{\pgfqpoint{4.762521in}{3.647826in}}{\pgfqpoint{4.751471in}{3.647826in}}%
\pgfpathcurveto{\pgfqpoint{4.740421in}{3.647826in}}{\pgfqpoint{4.729822in}{3.643435in}}{\pgfqpoint{4.722008in}{3.635622in}}%
\pgfpathcurveto{\pgfqpoint{4.714194in}{3.627808in}}{\pgfqpoint{4.709804in}{3.617209in}}{\pgfqpoint{4.709804in}{3.606159in}}%
\pgfpathcurveto{\pgfqpoint{4.709804in}{3.595109in}}{\pgfqpoint{4.714194in}{3.584510in}}{\pgfqpoint{4.722008in}{3.576696in}}%
\pgfpathcurveto{\pgfqpoint{4.729822in}{3.568883in}}{\pgfqpoint{4.740421in}{3.564492in}}{\pgfqpoint{4.751471in}{3.564492in}}%
\pgfpathclose%
\pgfusepath{stroke,fill}%
\end{pgfscope}%
\begin{pgfscope}%
\pgfpathrectangle{\pgfqpoint{0.481978in}{0.331635in}}{\pgfqpoint{9.300000in}{7.700000in}}%
\pgfusepath{clip}%
\pgfsetbuttcap%
\pgfsetroundjoin%
\definecolor{currentfill}{rgb}{0.631373,0.788235,0.956863}%
\pgfsetfillcolor{currentfill}%
\pgfsetlinewidth{0.481800pt}%
\definecolor{currentstroke}{rgb}{1.000000,1.000000,1.000000}%
\pgfsetstrokecolor{currentstroke}%
\pgfsetdash{}{0pt}%
\pgfpathmoveto{\pgfqpoint{0.904705in}{3.307906in}}%
\pgfpathcurveto{\pgfqpoint{0.915755in}{3.307906in}}{\pgfqpoint{0.926354in}{3.312296in}}{\pgfqpoint{0.934168in}{3.320110in}}%
\pgfpathcurveto{\pgfqpoint{0.941982in}{3.327923in}}{\pgfqpoint{0.946372in}{3.338522in}}{\pgfqpoint{0.946372in}{3.349572in}}%
\pgfpathcurveto{\pgfqpoint{0.946372in}{3.360622in}}{\pgfqpoint{0.941982in}{3.371221in}}{\pgfqpoint{0.934168in}{3.379035in}}%
\pgfpathcurveto{\pgfqpoint{0.926354in}{3.386849in}}{\pgfqpoint{0.915755in}{3.391239in}}{\pgfqpoint{0.904705in}{3.391239in}}%
\pgfpathcurveto{\pgfqpoint{0.893655in}{3.391239in}}{\pgfqpoint{0.883056in}{3.386849in}}{\pgfqpoint{0.875242in}{3.379035in}}%
\pgfpathcurveto{\pgfqpoint{0.867429in}{3.371221in}}{\pgfqpoint{0.863039in}{3.360622in}}{\pgfqpoint{0.863039in}{3.349572in}}%
\pgfpathcurveto{\pgfqpoint{0.863039in}{3.338522in}}{\pgfqpoint{0.867429in}{3.327923in}}{\pgfqpoint{0.875242in}{3.320110in}}%
\pgfpathcurveto{\pgfqpoint{0.883056in}{3.312296in}}{\pgfqpoint{0.893655in}{3.307906in}}{\pgfqpoint{0.904705in}{3.307906in}}%
\pgfpathclose%
\pgfusepath{stroke,fill}%
\end{pgfscope}%
\begin{pgfscope}%
\pgfpathrectangle{\pgfqpoint{0.481978in}{0.331635in}}{\pgfqpoint{9.300000in}{7.700000in}}%
\pgfusepath{clip}%
\pgfsetbuttcap%
\pgfsetroundjoin%
\definecolor{currentfill}{rgb}{0.631373,0.788235,0.956863}%
\pgfsetfillcolor{currentfill}%
\pgfsetlinewidth{0.481800pt}%
\definecolor{currentstroke}{rgb}{1.000000,1.000000,1.000000}%
\pgfsetstrokecolor{currentstroke}%
\pgfsetdash{}{0pt}%
\pgfpathmoveto{\pgfqpoint{5.294893in}{4.882562in}}%
\pgfpathcurveto{\pgfqpoint{5.305943in}{4.882562in}}{\pgfqpoint{5.316542in}{4.886952in}}{\pgfqpoint{5.324355in}{4.894766in}}%
\pgfpathcurveto{\pgfqpoint{5.332169in}{4.902579in}}{\pgfqpoint{5.336559in}{4.913178in}}{\pgfqpoint{5.336559in}{4.924228in}}%
\pgfpathcurveto{\pgfqpoint{5.336559in}{4.935278in}}{\pgfqpoint{5.332169in}{4.945877in}}{\pgfqpoint{5.324355in}{4.953691in}}%
\pgfpathcurveto{\pgfqpoint{5.316542in}{4.961505in}}{\pgfqpoint{5.305943in}{4.965895in}}{\pgfqpoint{5.294893in}{4.965895in}}%
\pgfpathcurveto{\pgfqpoint{5.283842in}{4.965895in}}{\pgfqpoint{5.273243in}{4.961505in}}{\pgfqpoint{5.265430in}{4.953691in}}%
\pgfpathcurveto{\pgfqpoint{5.257616in}{4.945877in}}{\pgfqpoint{5.253226in}{4.935278in}}{\pgfqpoint{5.253226in}{4.924228in}}%
\pgfpathcurveto{\pgfqpoint{5.253226in}{4.913178in}}{\pgfqpoint{5.257616in}{4.902579in}}{\pgfqpoint{5.265430in}{4.894766in}}%
\pgfpathcurveto{\pgfqpoint{5.273243in}{4.886952in}}{\pgfqpoint{5.283842in}{4.882562in}}{\pgfqpoint{5.294893in}{4.882562in}}%
\pgfpathclose%
\pgfusepath{stroke,fill}%
\end{pgfscope}%
\begin{pgfscope}%
\pgfpathrectangle{\pgfqpoint{0.481978in}{0.331635in}}{\pgfqpoint{9.300000in}{7.700000in}}%
\pgfusepath{clip}%
\pgfsetbuttcap%
\pgfsetroundjoin%
\definecolor{currentfill}{rgb}{0.631373,0.788235,0.956863}%
\pgfsetfillcolor{currentfill}%
\pgfsetlinewidth{0.481800pt}%
\definecolor{currentstroke}{rgb}{1.000000,1.000000,1.000000}%
\pgfsetstrokecolor{currentstroke}%
\pgfsetdash{}{0pt}%
\pgfpathmoveto{\pgfqpoint{1.539916in}{3.164925in}}%
\pgfpathcurveto{\pgfqpoint{1.550966in}{3.164925in}}{\pgfqpoint{1.561565in}{3.169316in}}{\pgfqpoint{1.569378in}{3.177129in}}%
\pgfpathcurveto{\pgfqpoint{1.577192in}{3.184943in}}{\pgfqpoint{1.581582in}{3.195542in}}{\pgfqpoint{1.581582in}{3.206592in}}%
\pgfpathcurveto{\pgfqpoint{1.581582in}{3.217642in}}{\pgfqpoint{1.577192in}{3.228241in}}{\pgfqpoint{1.569378in}{3.236055in}}%
\pgfpathcurveto{\pgfqpoint{1.561565in}{3.243868in}}{\pgfqpoint{1.550966in}{3.248259in}}{\pgfqpoint{1.539916in}{3.248259in}}%
\pgfpathcurveto{\pgfqpoint{1.528866in}{3.248259in}}{\pgfqpoint{1.518267in}{3.243868in}}{\pgfqpoint{1.510453in}{3.236055in}}%
\pgfpathcurveto{\pgfqpoint{1.502639in}{3.228241in}}{\pgfqpoint{1.498249in}{3.217642in}}{\pgfqpoint{1.498249in}{3.206592in}}%
\pgfpathcurveto{\pgfqpoint{1.498249in}{3.195542in}}{\pgfqpoint{1.502639in}{3.184943in}}{\pgfqpoint{1.510453in}{3.177129in}}%
\pgfpathcurveto{\pgfqpoint{1.518267in}{3.169316in}}{\pgfqpoint{1.528866in}{3.164925in}}{\pgfqpoint{1.539916in}{3.164925in}}%
\pgfpathclose%
\pgfusepath{stroke,fill}%
\end{pgfscope}%
\begin{pgfscope}%
\pgfpathrectangle{\pgfqpoint{0.481978in}{0.331635in}}{\pgfqpoint{9.300000in}{7.700000in}}%
\pgfusepath{clip}%
\pgfsetbuttcap%
\pgfsetroundjoin%
\definecolor{currentfill}{rgb}{0.631373,0.788235,0.956863}%
\pgfsetfillcolor{currentfill}%
\pgfsetlinewidth{0.481800pt}%
\definecolor{currentstroke}{rgb}{1.000000,1.000000,1.000000}%
\pgfsetstrokecolor{currentstroke}%
\pgfsetdash{}{0pt}%
\pgfpathmoveto{\pgfqpoint{6.009667in}{2.081225in}}%
\pgfpathcurveto{\pgfqpoint{6.020717in}{2.081225in}}{\pgfqpoint{6.031316in}{2.085616in}}{\pgfqpoint{6.039130in}{2.093429in}}%
\pgfpathcurveto{\pgfqpoint{6.046944in}{2.101243in}}{\pgfqpoint{6.051334in}{2.111842in}}{\pgfqpoint{6.051334in}{2.122892in}}%
\pgfpathcurveto{\pgfqpoint{6.051334in}{2.133942in}}{\pgfqpoint{6.046944in}{2.144541in}}{\pgfqpoint{6.039130in}{2.152355in}}%
\pgfpathcurveto{\pgfqpoint{6.031316in}{2.160168in}}{\pgfqpoint{6.020717in}{2.164559in}}{\pgfqpoint{6.009667in}{2.164559in}}%
\pgfpathcurveto{\pgfqpoint{5.998617in}{2.164559in}}{\pgfqpoint{5.988018in}{2.160168in}}{\pgfqpoint{5.980205in}{2.152355in}}%
\pgfpathcurveto{\pgfqpoint{5.972391in}{2.144541in}}{\pgfqpoint{5.968001in}{2.133942in}}{\pgfqpoint{5.968001in}{2.122892in}}%
\pgfpathcurveto{\pgfqpoint{5.968001in}{2.111842in}}{\pgfqpoint{5.972391in}{2.101243in}}{\pgfqpoint{5.980205in}{2.093429in}}%
\pgfpathcurveto{\pgfqpoint{5.988018in}{2.085616in}}{\pgfqpoint{5.998617in}{2.081225in}}{\pgfqpoint{6.009667in}{2.081225in}}%
\pgfpathclose%
\pgfusepath{stroke,fill}%
\end{pgfscope}%
\begin{pgfscope}%
\pgfpathrectangle{\pgfqpoint{0.481978in}{0.331635in}}{\pgfqpoint{9.300000in}{7.700000in}}%
\pgfusepath{clip}%
\pgfsetbuttcap%
\pgfsetroundjoin%
\definecolor{currentfill}{rgb}{0.631373,0.788235,0.956863}%
\pgfsetfillcolor{currentfill}%
\pgfsetlinewidth{0.481800pt}%
\definecolor{currentstroke}{rgb}{1.000000,1.000000,1.000000}%
\pgfsetstrokecolor{currentstroke}%
\pgfsetdash{}{0pt}%
\pgfpathmoveto{\pgfqpoint{1.634066in}{4.386210in}}%
\pgfpathcurveto{\pgfqpoint{1.645116in}{4.386210in}}{\pgfqpoint{1.655715in}{4.390600in}}{\pgfqpoint{1.663529in}{4.398413in}}%
\pgfpathcurveto{\pgfqpoint{1.671342in}{4.406227in}}{\pgfqpoint{1.675733in}{4.416826in}}{\pgfqpoint{1.675733in}{4.427876in}}%
\pgfpathcurveto{\pgfqpoint{1.675733in}{4.438926in}}{\pgfqpoint{1.671342in}{4.449525in}}{\pgfqpoint{1.663529in}{4.457339in}}%
\pgfpathcurveto{\pgfqpoint{1.655715in}{4.465153in}}{\pgfqpoint{1.645116in}{4.469543in}}{\pgfqpoint{1.634066in}{4.469543in}}%
\pgfpathcurveto{\pgfqpoint{1.623016in}{4.469543in}}{\pgfqpoint{1.612417in}{4.465153in}}{\pgfqpoint{1.604603in}{4.457339in}}%
\pgfpathcurveto{\pgfqpoint{1.596789in}{4.449525in}}{\pgfqpoint{1.592399in}{4.438926in}}{\pgfqpoint{1.592399in}{4.427876in}}%
\pgfpathcurveto{\pgfqpoint{1.592399in}{4.416826in}}{\pgfqpoint{1.596789in}{4.406227in}}{\pgfqpoint{1.604603in}{4.398413in}}%
\pgfpathcurveto{\pgfqpoint{1.612417in}{4.390600in}}{\pgfqpoint{1.623016in}{4.386210in}}{\pgfqpoint{1.634066in}{4.386210in}}%
\pgfpathclose%
\pgfusepath{stroke,fill}%
\end{pgfscope}%
\begin{pgfscope}%
\pgfpathrectangle{\pgfqpoint{0.481978in}{0.331635in}}{\pgfqpoint{9.300000in}{7.700000in}}%
\pgfusepath{clip}%
\pgfsetbuttcap%
\pgfsetroundjoin%
\definecolor{currentfill}{rgb}{0.631373,0.788235,0.956863}%
\pgfsetfillcolor{currentfill}%
\pgfsetlinewidth{0.481800pt}%
\definecolor{currentstroke}{rgb}{1.000000,1.000000,1.000000}%
\pgfsetstrokecolor{currentstroke}%
\pgfsetdash{}{0pt}%
\pgfpathmoveto{\pgfqpoint{4.786472in}{1.456378in}}%
\pgfpathcurveto{\pgfqpoint{4.797522in}{1.456378in}}{\pgfqpoint{4.808121in}{1.460768in}}{\pgfqpoint{4.815935in}{1.468582in}}%
\pgfpathcurveto{\pgfqpoint{4.823748in}{1.476395in}}{\pgfqpoint{4.828139in}{1.486994in}}{\pgfqpoint{4.828139in}{1.498044in}}%
\pgfpathcurveto{\pgfqpoint{4.828139in}{1.509094in}}{\pgfqpoint{4.823748in}{1.519694in}}{\pgfqpoint{4.815935in}{1.527507in}}%
\pgfpathcurveto{\pgfqpoint{4.808121in}{1.535321in}}{\pgfqpoint{4.797522in}{1.539711in}}{\pgfqpoint{4.786472in}{1.539711in}}%
\pgfpathcurveto{\pgfqpoint{4.775422in}{1.539711in}}{\pgfqpoint{4.764823in}{1.535321in}}{\pgfqpoint{4.757009in}{1.527507in}}%
\pgfpathcurveto{\pgfqpoint{4.749196in}{1.519694in}}{\pgfqpoint{4.744805in}{1.509094in}}{\pgfqpoint{4.744805in}{1.498044in}}%
\pgfpathcurveto{\pgfqpoint{4.744805in}{1.486994in}}{\pgfqpoint{4.749196in}{1.476395in}}{\pgfqpoint{4.757009in}{1.468582in}}%
\pgfpathcurveto{\pgfqpoint{4.764823in}{1.460768in}}{\pgfqpoint{4.775422in}{1.456378in}}{\pgfqpoint{4.786472in}{1.456378in}}%
\pgfpathclose%
\pgfusepath{stroke,fill}%
\end{pgfscope}%
\begin{pgfscope}%
\pgfpathrectangle{\pgfqpoint{0.481978in}{0.331635in}}{\pgfqpoint{9.300000in}{7.700000in}}%
\pgfusepath{clip}%
\pgfsetbuttcap%
\pgfsetroundjoin%
\definecolor{currentfill}{rgb}{0.631373,0.788235,0.956863}%
\pgfsetfillcolor{currentfill}%
\pgfsetlinewidth{0.481800pt}%
\definecolor{currentstroke}{rgb}{1.000000,1.000000,1.000000}%
\pgfsetstrokecolor{currentstroke}%
\pgfsetdash{}{0pt}%
\pgfpathmoveto{\pgfqpoint{1.153250in}{2.877892in}}%
\pgfpathcurveto{\pgfqpoint{1.164300in}{2.877892in}}{\pgfqpoint{1.174899in}{2.882282in}}{\pgfqpoint{1.182713in}{2.890096in}}%
\pgfpathcurveto{\pgfqpoint{1.190526in}{2.897909in}}{\pgfqpoint{1.194917in}{2.908508in}}{\pgfqpoint{1.194917in}{2.919559in}}%
\pgfpathcurveto{\pgfqpoint{1.194917in}{2.930609in}}{\pgfqpoint{1.190526in}{2.941208in}}{\pgfqpoint{1.182713in}{2.949021in}}%
\pgfpathcurveto{\pgfqpoint{1.174899in}{2.956835in}}{\pgfqpoint{1.164300in}{2.961225in}}{\pgfqpoint{1.153250in}{2.961225in}}%
\pgfpathcurveto{\pgfqpoint{1.142200in}{2.961225in}}{\pgfqpoint{1.131601in}{2.956835in}}{\pgfqpoint{1.123787in}{2.949021in}}%
\pgfpathcurveto{\pgfqpoint{1.115974in}{2.941208in}}{\pgfqpoint{1.111583in}{2.930609in}}{\pgfqpoint{1.111583in}{2.919559in}}%
\pgfpathcurveto{\pgfqpoint{1.111583in}{2.908508in}}{\pgfqpoint{1.115974in}{2.897909in}}{\pgfqpoint{1.123787in}{2.890096in}}%
\pgfpathcurveto{\pgfqpoint{1.131601in}{2.882282in}}{\pgfqpoint{1.142200in}{2.877892in}}{\pgfqpoint{1.153250in}{2.877892in}}%
\pgfpathclose%
\pgfusepath{stroke,fill}%
\end{pgfscope}%
\begin{pgfscope}%
\pgfpathrectangle{\pgfqpoint{0.481978in}{0.331635in}}{\pgfqpoint{9.300000in}{7.700000in}}%
\pgfusepath{clip}%
\pgfsetbuttcap%
\pgfsetroundjoin%
\definecolor{currentfill}{rgb}{0.631373,0.788235,0.956863}%
\pgfsetfillcolor{currentfill}%
\pgfsetlinewidth{0.481800pt}%
\definecolor{currentstroke}{rgb}{1.000000,1.000000,1.000000}%
\pgfsetstrokecolor{currentstroke}%
\pgfsetdash{}{0pt}%
\pgfpathmoveto{\pgfqpoint{2.365283in}{3.823443in}}%
\pgfpathcurveto{\pgfqpoint{2.376334in}{3.823443in}}{\pgfqpoint{2.386933in}{3.827833in}}{\pgfqpoint{2.394746in}{3.835647in}}%
\pgfpathcurveto{\pgfqpoint{2.402560in}{3.843461in}}{\pgfqpoint{2.406950in}{3.854060in}}{\pgfqpoint{2.406950in}{3.865110in}}%
\pgfpathcurveto{\pgfqpoint{2.406950in}{3.876160in}}{\pgfqpoint{2.402560in}{3.886759in}}{\pgfqpoint{2.394746in}{3.894573in}}%
\pgfpathcurveto{\pgfqpoint{2.386933in}{3.902386in}}{\pgfqpoint{2.376334in}{3.906776in}}{\pgfqpoint{2.365283in}{3.906776in}}%
\pgfpathcurveto{\pgfqpoint{2.354233in}{3.906776in}}{\pgfqpoint{2.343634in}{3.902386in}}{\pgfqpoint{2.335821in}{3.894573in}}%
\pgfpathcurveto{\pgfqpoint{2.328007in}{3.886759in}}{\pgfqpoint{2.323617in}{3.876160in}}{\pgfqpoint{2.323617in}{3.865110in}}%
\pgfpathcurveto{\pgfqpoint{2.323617in}{3.854060in}}{\pgfqpoint{2.328007in}{3.843461in}}{\pgfqpoint{2.335821in}{3.835647in}}%
\pgfpathcurveto{\pgfqpoint{2.343634in}{3.827833in}}{\pgfqpoint{2.354233in}{3.823443in}}{\pgfqpoint{2.365283in}{3.823443in}}%
\pgfpathclose%
\pgfusepath{stroke,fill}%
\end{pgfscope}%
\begin{pgfscope}%
\pgfpathrectangle{\pgfqpoint{0.481978in}{0.331635in}}{\pgfqpoint{9.300000in}{7.700000in}}%
\pgfusepath{clip}%
\pgfsetbuttcap%
\pgfsetroundjoin%
\definecolor{currentfill}{rgb}{0.631373,0.788235,0.956863}%
\pgfsetfillcolor{currentfill}%
\pgfsetlinewidth{0.481800pt}%
\definecolor{currentstroke}{rgb}{1.000000,1.000000,1.000000}%
\pgfsetstrokecolor{currentstroke}%
\pgfsetdash{}{0pt}%
\pgfpathmoveto{\pgfqpoint{4.851746in}{2.476807in}}%
\pgfpathcurveto{\pgfqpoint{4.862796in}{2.476807in}}{\pgfqpoint{4.873395in}{2.481197in}}{\pgfqpoint{4.881209in}{2.489011in}}%
\pgfpathcurveto{\pgfqpoint{4.889022in}{2.496825in}}{\pgfqpoint{4.893413in}{2.507424in}}{\pgfqpoint{4.893413in}{2.518474in}}%
\pgfpathcurveto{\pgfqpoint{4.893413in}{2.529524in}}{\pgfqpoint{4.889022in}{2.540123in}}{\pgfqpoint{4.881209in}{2.547936in}}%
\pgfpathcurveto{\pgfqpoint{4.873395in}{2.555750in}}{\pgfqpoint{4.862796in}{2.560140in}}{\pgfqpoint{4.851746in}{2.560140in}}%
\pgfpathcurveto{\pgfqpoint{4.840696in}{2.560140in}}{\pgfqpoint{4.830097in}{2.555750in}}{\pgfqpoint{4.822283in}{2.547936in}}%
\pgfpathcurveto{\pgfqpoint{4.814470in}{2.540123in}}{\pgfqpoint{4.810079in}{2.529524in}}{\pgfqpoint{4.810079in}{2.518474in}}%
\pgfpathcurveto{\pgfqpoint{4.810079in}{2.507424in}}{\pgfqpoint{4.814470in}{2.496825in}}{\pgfqpoint{4.822283in}{2.489011in}}%
\pgfpathcurveto{\pgfqpoint{4.830097in}{2.481197in}}{\pgfqpoint{4.840696in}{2.476807in}}{\pgfqpoint{4.851746in}{2.476807in}}%
\pgfpathclose%
\pgfusepath{stroke,fill}%
\end{pgfscope}%
\begin{pgfscope}%
\pgfpathrectangle{\pgfqpoint{0.481978in}{0.331635in}}{\pgfqpoint{9.300000in}{7.700000in}}%
\pgfusepath{clip}%
\pgfsetbuttcap%
\pgfsetroundjoin%
\definecolor{currentfill}{rgb}{0.631373,0.788235,0.956863}%
\pgfsetfillcolor{currentfill}%
\pgfsetlinewidth{0.481800pt}%
\definecolor{currentstroke}{rgb}{1.000000,1.000000,1.000000}%
\pgfsetstrokecolor{currentstroke}%
\pgfsetdash{}{0pt}%
\pgfpathmoveto{\pgfqpoint{7.001514in}{7.639968in}}%
\pgfpathcurveto{\pgfqpoint{7.012564in}{7.639968in}}{\pgfqpoint{7.023163in}{7.644359in}}{\pgfqpoint{7.030977in}{7.652172in}}%
\pgfpathcurveto{\pgfqpoint{7.038791in}{7.659986in}}{\pgfqpoint{7.043181in}{7.670585in}}{\pgfqpoint{7.043181in}{7.681635in}}%
\pgfpathcurveto{\pgfqpoint{7.043181in}{7.692685in}}{\pgfqpoint{7.038791in}{7.703284in}}{\pgfqpoint{7.030977in}{7.711098in}}%
\pgfpathcurveto{\pgfqpoint{7.023163in}{7.718911in}}{\pgfqpoint{7.012564in}{7.723302in}}{\pgfqpoint{7.001514in}{7.723302in}}%
\pgfpathcurveto{\pgfqpoint{6.990464in}{7.723302in}}{\pgfqpoint{6.979865in}{7.718911in}}{\pgfqpoint{6.972051in}{7.711098in}}%
\pgfpathcurveto{\pgfqpoint{6.964238in}{7.703284in}}{\pgfqpoint{6.959848in}{7.692685in}}{\pgfqpoint{6.959848in}{7.681635in}}%
\pgfpathcurveto{\pgfqpoint{6.959848in}{7.670585in}}{\pgfqpoint{6.964238in}{7.659986in}}{\pgfqpoint{6.972051in}{7.652172in}}%
\pgfpathcurveto{\pgfqpoint{6.979865in}{7.644359in}}{\pgfqpoint{6.990464in}{7.639968in}}{\pgfqpoint{7.001514in}{7.639968in}}%
\pgfpathclose%
\pgfusepath{stroke,fill}%
\end{pgfscope}%
\begin{pgfscope}%
\pgfpathrectangle{\pgfqpoint{0.481978in}{0.331635in}}{\pgfqpoint{9.300000in}{7.700000in}}%
\pgfusepath{clip}%
\pgfsetbuttcap%
\pgfsetroundjoin%
\definecolor{currentfill}{rgb}{0.631373,0.788235,0.956863}%
\pgfsetfillcolor{currentfill}%
\pgfsetlinewidth{0.481800pt}%
\definecolor{currentstroke}{rgb}{1.000000,1.000000,1.000000}%
\pgfsetstrokecolor{currentstroke}%
\pgfsetdash{}{0pt}%
\pgfpathmoveto{\pgfqpoint{3.390380in}{6.615658in}}%
\pgfpathcurveto{\pgfqpoint{3.401430in}{6.615658in}}{\pgfqpoint{3.412029in}{6.620048in}}{\pgfqpoint{3.419843in}{6.627862in}}%
\pgfpathcurveto{\pgfqpoint{3.427657in}{6.635675in}}{\pgfqpoint{3.432047in}{6.646274in}}{\pgfqpoint{3.432047in}{6.657325in}}%
\pgfpathcurveto{\pgfqpoint{3.432047in}{6.668375in}}{\pgfqpoint{3.427657in}{6.678974in}}{\pgfqpoint{3.419843in}{6.686787in}}%
\pgfpathcurveto{\pgfqpoint{3.412029in}{6.694601in}}{\pgfqpoint{3.401430in}{6.698991in}}{\pgfqpoint{3.390380in}{6.698991in}}%
\pgfpathcurveto{\pgfqpoint{3.379330in}{6.698991in}}{\pgfqpoint{3.368731in}{6.694601in}}{\pgfqpoint{3.360917in}{6.686787in}}%
\pgfpathcurveto{\pgfqpoint{3.353104in}{6.678974in}}{\pgfqpoint{3.348714in}{6.668375in}}{\pgfqpoint{3.348714in}{6.657325in}}%
\pgfpathcurveto{\pgfqpoint{3.348714in}{6.646274in}}{\pgfqpoint{3.353104in}{6.635675in}}{\pgfqpoint{3.360917in}{6.627862in}}%
\pgfpathcurveto{\pgfqpoint{3.368731in}{6.620048in}}{\pgfqpoint{3.379330in}{6.615658in}}{\pgfqpoint{3.390380in}{6.615658in}}%
\pgfpathclose%
\pgfusepath{stroke,fill}%
\end{pgfscope}%
\begin{pgfscope}%
\pgfpathrectangle{\pgfqpoint{0.481978in}{0.331635in}}{\pgfqpoint{9.300000in}{7.700000in}}%
\pgfusepath{clip}%
\pgfsetbuttcap%
\pgfsetroundjoin%
\definecolor{currentfill}{rgb}{0.631373,0.788235,0.956863}%
\pgfsetfillcolor{currentfill}%
\pgfsetlinewidth{0.481800pt}%
\definecolor{currentstroke}{rgb}{1.000000,1.000000,1.000000}%
\pgfsetstrokecolor{currentstroke}%
\pgfsetdash{}{0pt}%
\pgfpathmoveto{\pgfqpoint{5.562444in}{6.314545in}}%
\pgfpathcurveto{\pgfqpoint{5.573494in}{6.314545in}}{\pgfqpoint{5.584093in}{6.318935in}}{\pgfqpoint{5.591907in}{6.326749in}}%
\pgfpathcurveto{\pgfqpoint{5.599721in}{6.334563in}}{\pgfqpoint{5.604111in}{6.345162in}}{\pgfqpoint{5.604111in}{6.356212in}}%
\pgfpathcurveto{\pgfqpoint{5.604111in}{6.367262in}}{\pgfqpoint{5.599721in}{6.377861in}}{\pgfqpoint{5.591907in}{6.385675in}}%
\pgfpathcurveto{\pgfqpoint{5.584093in}{6.393488in}}{\pgfqpoint{5.573494in}{6.397879in}}{\pgfqpoint{5.562444in}{6.397879in}}%
\pgfpathcurveto{\pgfqpoint{5.551394in}{6.397879in}}{\pgfqpoint{5.540795in}{6.393488in}}{\pgfqpoint{5.532981in}{6.385675in}}%
\pgfpathcurveto{\pgfqpoint{5.525168in}{6.377861in}}{\pgfqpoint{5.520778in}{6.367262in}}{\pgfqpoint{5.520778in}{6.356212in}}%
\pgfpathcurveto{\pgfqpoint{5.520778in}{6.345162in}}{\pgfqpoint{5.525168in}{6.334563in}}{\pgfqpoint{5.532981in}{6.326749in}}%
\pgfpathcurveto{\pgfqpoint{5.540795in}{6.318935in}}{\pgfqpoint{5.551394in}{6.314545in}}{\pgfqpoint{5.562444in}{6.314545in}}%
\pgfpathclose%
\pgfusepath{stroke,fill}%
\end{pgfscope}%
\begin{pgfscope}%
\pgfpathrectangle{\pgfqpoint{0.481978in}{0.331635in}}{\pgfqpoint{9.300000in}{7.700000in}}%
\pgfusepath{clip}%
\pgfsetbuttcap%
\pgfsetroundjoin%
\definecolor{currentfill}{rgb}{0.631373,0.788235,0.956863}%
\pgfsetfillcolor{currentfill}%
\pgfsetlinewidth{0.481800pt}%
\definecolor{currentstroke}{rgb}{1.000000,1.000000,1.000000}%
\pgfsetstrokecolor{currentstroke}%
\pgfsetdash{}{0pt}%
\pgfpathmoveto{\pgfqpoint{4.975949in}{5.326503in}}%
\pgfpathcurveto{\pgfqpoint{4.986999in}{5.326503in}}{\pgfqpoint{4.997598in}{5.330893in}}{\pgfqpoint{5.005411in}{5.338707in}}%
\pgfpathcurveto{\pgfqpoint{5.013225in}{5.346521in}}{\pgfqpoint{5.017615in}{5.357120in}}{\pgfqpoint{5.017615in}{5.368170in}}%
\pgfpathcurveto{\pgfqpoint{5.017615in}{5.379220in}}{\pgfqpoint{5.013225in}{5.389819in}}{\pgfqpoint{5.005411in}{5.397633in}}%
\pgfpathcurveto{\pgfqpoint{4.997598in}{5.405446in}}{\pgfqpoint{4.986999in}{5.409837in}}{\pgfqpoint{4.975949in}{5.409837in}}%
\pgfpathcurveto{\pgfqpoint{4.964899in}{5.409837in}}{\pgfqpoint{4.954300in}{5.405446in}}{\pgfqpoint{4.946486in}{5.397633in}}%
\pgfpathcurveto{\pgfqpoint{4.938672in}{5.389819in}}{\pgfqpoint{4.934282in}{5.379220in}}{\pgfqpoint{4.934282in}{5.368170in}}%
\pgfpathcurveto{\pgfqpoint{4.934282in}{5.357120in}}{\pgfqpoint{4.938672in}{5.346521in}}{\pgfqpoint{4.946486in}{5.338707in}}%
\pgfpathcurveto{\pgfqpoint{4.954300in}{5.330893in}}{\pgfqpoint{4.964899in}{5.326503in}}{\pgfqpoint{4.975949in}{5.326503in}}%
\pgfpathclose%
\pgfusepath{stroke,fill}%
\end{pgfscope}%
\begin{pgfscope}%
\pgfpathrectangle{\pgfqpoint{0.481978in}{0.331635in}}{\pgfqpoint{9.300000in}{7.700000in}}%
\pgfusepath{clip}%
\pgfsetbuttcap%
\pgfsetroundjoin%
\definecolor{currentfill}{rgb}{0.631373,0.788235,0.956863}%
\pgfsetfillcolor{currentfill}%
\pgfsetlinewidth{0.481800pt}%
\definecolor{currentstroke}{rgb}{1.000000,1.000000,1.000000}%
\pgfsetstrokecolor{currentstroke}%
\pgfsetdash{}{0pt}%
\pgfpathmoveto{\pgfqpoint{4.416722in}{4.321108in}}%
\pgfpathcurveto{\pgfqpoint{4.427772in}{4.321108in}}{\pgfqpoint{4.438371in}{4.325499in}}{\pgfqpoint{4.446185in}{4.333312in}}%
\pgfpathcurveto{\pgfqpoint{4.453998in}{4.341126in}}{\pgfqpoint{4.458389in}{4.351725in}}{\pgfqpoint{4.458389in}{4.362775in}}%
\pgfpathcurveto{\pgfqpoint{4.458389in}{4.373825in}}{\pgfqpoint{4.453998in}{4.384424in}}{\pgfqpoint{4.446185in}{4.392238in}}%
\pgfpathcurveto{\pgfqpoint{4.438371in}{4.400051in}}{\pgfqpoint{4.427772in}{4.404442in}}{\pgfqpoint{4.416722in}{4.404442in}}%
\pgfpathcurveto{\pgfqpoint{4.405672in}{4.404442in}}{\pgfqpoint{4.395073in}{4.400051in}}{\pgfqpoint{4.387259in}{4.392238in}}%
\pgfpathcurveto{\pgfqpoint{4.379446in}{4.384424in}}{\pgfqpoint{4.375055in}{4.373825in}}{\pgfqpoint{4.375055in}{4.362775in}}%
\pgfpathcurveto{\pgfqpoint{4.375055in}{4.351725in}}{\pgfqpoint{4.379446in}{4.341126in}}{\pgfqpoint{4.387259in}{4.333312in}}%
\pgfpathcurveto{\pgfqpoint{4.395073in}{4.325499in}}{\pgfqpoint{4.405672in}{4.321108in}}{\pgfqpoint{4.416722in}{4.321108in}}%
\pgfpathclose%
\pgfusepath{stroke,fill}%
\end{pgfscope}%
\begin{pgfscope}%
\pgfpathrectangle{\pgfqpoint{0.481978in}{0.331635in}}{\pgfqpoint{9.300000in}{7.700000in}}%
\pgfusepath{clip}%
\pgfsetbuttcap%
\pgfsetroundjoin%
\definecolor{currentfill}{rgb}{0.631373,0.788235,0.956863}%
\pgfsetfillcolor{currentfill}%
\pgfsetlinewidth{0.481800pt}%
\definecolor{currentstroke}{rgb}{1.000000,1.000000,1.000000}%
\pgfsetstrokecolor{currentstroke}%
\pgfsetdash{}{0pt}%
\pgfpathmoveto{\pgfqpoint{5.411196in}{3.789287in}}%
\pgfpathcurveto{\pgfqpoint{5.422247in}{3.789287in}}{\pgfqpoint{5.432846in}{3.793677in}}{\pgfqpoint{5.440659in}{3.801491in}}%
\pgfpathcurveto{\pgfqpoint{5.448473in}{3.809304in}}{\pgfqpoint{5.452863in}{3.819903in}}{\pgfqpoint{5.452863in}{3.830953in}}%
\pgfpathcurveto{\pgfqpoint{5.452863in}{3.842004in}}{\pgfqpoint{5.448473in}{3.852603in}}{\pgfqpoint{5.440659in}{3.860416in}}%
\pgfpathcurveto{\pgfqpoint{5.432846in}{3.868230in}}{\pgfqpoint{5.422247in}{3.872620in}}{\pgfqpoint{5.411196in}{3.872620in}}%
\pgfpathcurveto{\pgfqpoint{5.400146in}{3.872620in}}{\pgfqpoint{5.389547in}{3.868230in}}{\pgfqpoint{5.381734in}{3.860416in}}%
\pgfpathcurveto{\pgfqpoint{5.373920in}{3.852603in}}{\pgfqpoint{5.369530in}{3.842004in}}{\pgfqpoint{5.369530in}{3.830953in}}%
\pgfpathcurveto{\pgfqpoint{5.369530in}{3.819903in}}{\pgfqpoint{5.373920in}{3.809304in}}{\pgfqpoint{5.381734in}{3.801491in}}%
\pgfpathcurveto{\pgfqpoint{5.389547in}{3.793677in}}{\pgfqpoint{5.400146in}{3.789287in}}{\pgfqpoint{5.411196in}{3.789287in}}%
\pgfpathclose%
\pgfusepath{stroke,fill}%
\end{pgfscope}%
\begin{pgfscope}%
\pgfpathrectangle{\pgfqpoint{0.481978in}{0.331635in}}{\pgfqpoint{9.300000in}{7.700000in}}%
\pgfusepath{clip}%
\pgfsetbuttcap%
\pgfsetroundjoin%
\definecolor{currentfill}{rgb}{0.631373,0.788235,0.956863}%
\pgfsetfillcolor{currentfill}%
\pgfsetlinewidth{0.481800pt}%
\definecolor{currentstroke}{rgb}{1.000000,1.000000,1.000000}%
\pgfsetstrokecolor{currentstroke}%
\pgfsetdash{}{0pt}%
\pgfpathmoveto{\pgfqpoint{6.107702in}{6.012769in}}%
\pgfpathcurveto{\pgfqpoint{6.118752in}{6.012769in}}{\pgfqpoint{6.129351in}{6.017160in}}{\pgfqpoint{6.137165in}{6.024973in}}%
\pgfpathcurveto{\pgfqpoint{6.144979in}{6.032787in}}{\pgfqpoint{6.149369in}{6.043386in}}{\pgfqpoint{6.149369in}{6.054436in}}%
\pgfpathcurveto{\pgfqpoint{6.149369in}{6.065486in}}{\pgfqpoint{6.144979in}{6.076085in}}{\pgfqpoint{6.137165in}{6.083899in}}%
\pgfpathcurveto{\pgfqpoint{6.129351in}{6.091712in}}{\pgfqpoint{6.118752in}{6.096103in}}{\pgfqpoint{6.107702in}{6.096103in}}%
\pgfpathcurveto{\pgfqpoint{6.096652in}{6.096103in}}{\pgfqpoint{6.086053in}{6.091712in}}{\pgfqpoint{6.078240in}{6.083899in}}%
\pgfpathcurveto{\pgfqpoint{6.070426in}{6.076085in}}{\pgfqpoint{6.066036in}{6.065486in}}{\pgfqpoint{6.066036in}{6.054436in}}%
\pgfpathcurveto{\pgfqpoint{6.066036in}{6.043386in}}{\pgfqpoint{6.070426in}{6.032787in}}{\pgfqpoint{6.078240in}{6.024973in}}%
\pgfpathcurveto{\pgfqpoint{6.086053in}{6.017160in}}{\pgfqpoint{6.096652in}{6.012769in}}{\pgfqpoint{6.107702in}{6.012769in}}%
\pgfpathclose%
\pgfusepath{stroke,fill}%
\end{pgfscope}%
\begin{pgfscope}%
\pgfpathrectangle{\pgfqpoint{0.481978in}{0.331635in}}{\pgfqpoint{9.300000in}{7.700000in}}%
\pgfusepath{clip}%
\pgfsetbuttcap%
\pgfsetroundjoin%
\definecolor{currentfill}{rgb}{0.631373,0.788235,0.956863}%
\pgfsetfillcolor{currentfill}%
\pgfsetlinewidth{0.481800pt}%
\definecolor{currentstroke}{rgb}{1.000000,1.000000,1.000000}%
\pgfsetstrokecolor{currentstroke}%
\pgfsetdash{}{0pt}%
\pgfpathmoveto{\pgfqpoint{6.308017in}{7.174317in}}%
\pgfpathcurveto{\pgfqpoint{6.319067in}{7.174317in}}{\pgfqpoint{6.329666in}{7.178707in}}{\pgfqpoint{6.337479in}{7.186521in}}%
\pgfpathcurveto{\pgfqpoint{6.345293in}{7.194334in}}{\pgfqpoint{6.349683in}{7.204933in}}{\pgfqpoint{6.349683in}{7.215984in}}%
\pgfpathcurveto{\pgfqpoint{6.349683in}{7.227034in}}{\pgfqpoint{6.345293in}{7.237633in}}{\pgfqpoint{6.337479in}{7.245446in}}%
\pgfpathcurveto{\pgfqpoint{6.329666in}{7.253260in}}{\pgfqpoint{6.319067in}{7.257650in}}{\pgfqpoint{6.308017in}{7.257650in}}%
\pgfpathcurveto{\pgfqpoint{6.296966in}{7.257650in}}{\pgfqpoint{6.286367in}{7.253260in}}{\pgfqpoint{6.278554in}{7.245446in}}%
\pgfpathcurveto{\pgfqpoint{6.270740in}{7.237633in}}{\pgfqpoint{6.266350in}{7.227034in}}{\pgfqpoint{6.266350in}{7.215984in}}%
\pgfpathcurveto{\pgfqpoint{6.266350in}{7.204933in}}{\pgfqpoint{6.270740in}{7.194334in}}{\pgfqpoint{6.278554in}{7.186521in}}%
\pgfpathcurveto{\pgfqpoint{6.286367in}{7.178707in}}{\pgfqpoint{6.296966in}{7.174317in}}{\pgfqpoint{6.308017in}{7.174317in}}%
\pgfpathclose%
\pgfusepath{stroke,fill}%
\end{pgfscope}%
\begin{pgfscope}%
\pgfpathrectangle{\pgfqpoint{0.481978in}{0.331635in}}{\pgfqpoint{9.300000in}{7.700000in}}%
\pgfusepath{clip}%
\pgfsetbuttcap%
\pgfsetroundjoin%
\definecolor{currentfill}{rgb}{0.631373,0.788235,0.956863}%
\pgfsetfillcolor{currentfill}%
\pgfsetlinewidth{0.481800pt}%
\definecolor{currentstroke}{rgb}{1.000000,1.000000,1.000000}%
\pgfsetstrokecolor{currentstroke}%
\pgfsetdash{}{0pt}%
\pgfpathmoveto{\pgfqpoint{1.614740in}{1.165369in}}%
\pgfpathcurveto{\pgfqpoint{1.625791in}{1.165369in}}{\pgfqpoint{1.636390in}{1.169759in}}{\pgfqpoint{1.644203in}{1.177573in}}%
\pgfpathcurveto{\pgfqpoint{1.652017in}{1.185387in}}{\pgfqpoint{1.656407in}{1.195986in}}{\pgfqpoint{1.656407in}{1.207036in}}%
\pgfpathcurveto{\pgfqpoint{1.656407in}{1.218086in}}{\pgfqpoint{1.652017in}{1.228685in}}{\pgfqpoint{1.644203in}{1.236499in}}%
\pgfpathcurveto{\pgfqpoint{1.636390in}{1.244312in}}{\pgfqpoint{1.625791in}{1.248702in}}{\pgfqpoint{1.614740in}{1.248702in}}%
\pgfpathcurveto{\pgfqpoint{1.603690in}{1.248702in}}{\pgfqpoint{1.593091in}{1.244312in}}{\pgfqpoint{1.585278in}{1.236499in}}%
\pgfpathcurveto{\pgfqpoint{1.577464in}{1.228685in}}{\pgfqpoint{1.573074in}{1.218086in}}{\pgfqpoint{1.573074in}{1.207036in}}%
\pgfpathcurveto{\pgfqpoint{1.573074in}{1.195986in}}{\pgfqpoint{1.577464in}{1.185387in}}{\pgfqpoint{1.585278in}{1.177573in}}%
\pgfpathcurveto{\pgfqpoint{1.593091in}{1.169759in}}{\pgfqpoint{1.603690in}{1.165369in}}{\pgfqpoint{1.614740in}{1.165369in}}%
\pgfpathclose%
\pgfusepath{stroke,fill}%
\end{pgfscope}%
\begin{pgfscope}%
\pgfpathrectangle{\pgfqpoint{0.481978in}{0.331635in}}{\pgfqpoint{9.300000in}{7.700000in}}%
\pgfusepath{clip}%
\pgfsetbuttcap%
\pgfsetroundjoin%
\definecolor{currentfill}{rgb}{0.631373,0.788235,0.956863}%
\pgfsetfillcolor{currentfill}%
\pgfsetlinewidth{0.481800pt}%
\definecolor{currentstroke}{rgb}{1.000000,1.000000,1.000000}%
\pgfsetstrokecolor{currentstroke}%
\pgfsetdash{}{0pt}%
\pgfpathmoveto{\pgfqpoint{5.680299in}{6.980771in}}%
\pgfpathcurveto{\pgfqpoint{5.691349in}{6.980771in}}{\pgfqpoint{5.701948in}{6.985161in}}{\pgfqpoint{5.709762in}{6.992975in}}%
\pgfpathcurveto{\pgfqpoint{5.717576in}{7.000788in}}{\pgfqpoint{5.721966in}{7.011387in}}{\pgfqpoint{5.721966in}{7.022438in}}%
\pgfpathcurveto{\pgfqpoint{5.721966in}{7.033488in}}{\pgfqpoint{5.717576in}{7.044087in}}{\pgfqpoint{5.709762in}{7.051900in}}%
\pgfpathcurveto{\pgfqpoint{5.701948in}{7.059714in}}{\pgfqpoint{5.691349in}{7.064104in}}{\pgfqpoint{5.680299in}{7.064104in}}%
\pgfpathcurveto{\pgfqpoint{5.669249in}{7.064104in}}{\pgfqpoint{5.658650in}{7.059714in}}{\pgfqpoint{5.650836in}{7.051900in}}%
\pgfpathcurveto{\pgfqpoint{5.643023in}{7.044087in}}{\pgfqpoint{5.638633in}{7.033488in}}{\pgfqpoint{5.638633in}{7.022438in}}%
\pgfpathcurveto{\pgfqpoint{5.638633in}{7.011387in}}{\pgfqpoint{5.643023in}{7.000788in}}{\pgfqpoint{5.650836in}{6.992975in}}%
\pgfpathcurveto{\pgfqpoint{5.658650in}{6.985161in}}{\pgfqpoint{5.669249in}{6.980771in}}{\pgfqpoint{5.680299in}{6.980771in}}%
\pgfpathclose%
\pgfusepath{stroke,fill}%
\end{pgfscope}%
\begin{pgfscope}%
\pgfpathrectangle{\pgfqpoint{0.481978in}{0.331635in}}{\pgfqpoint{9.300000in}{7.700000in}}%
\pgfusepath{clip}%
\pgfsetbuttcap%
\pgfsetroundjoin%
\definecolor{currentfill}{rgb}{0.631373,0.788235,0.956863}%
\pgfsetfillcolor{currentfill}%
\pgfsetlinewidth{0.481800pt}%
\definecolor{currentstroke}{rgb}{1.000000,1.000000,1.000000}%
\pgfsetstrokecolor{currentstroke}%
\pgfsetdash{}{0pt}%
\pgfpathmoveto{\pgfqpoint{4.473102in}{6.270419in}}%
\pgfpathcurveto{\pgfqpoint{4.484152in}{6.270419in}}{\pgfqpoint{4.494751in}{6.274809in}}{\pgfqpoint{4.502564in}{6.282623in}}%
\pgfpathcurveto{\pgfqpoint{4.510378in}{6.290437in}}{\pgfqpoint{4.514768in}{6.301036in}}{\pgfqpoint{4.514768in}{6.312086in}}%
\pgfpathcurveto{\pgfqpoint{4.514768in}{6.323136in}}{\pgfqpoint{4.510378in}{6.333735in}}{\pgfqpoint{4.502564in}{6.341549in}}%
\pgfpathcurveto{\pgfqpoint{4.494751in}{6.349362in}}{\pgfqpoint{4.484152in}{6.353752in}}{\pgfqpoint{4.473102in}{6.353752in}}%
\pgfpathcurveto{\pgfqpoint{4.462051in}{6.353752in}}{\pgfqpoint{4.451452in}{6.349362in}}{\pgfqpoint{4.443639in}{6.341549in}}%
\pgfpathcurveto{\pgfqpoint{4.435825in}{6.333735in}}{\pgfqpoint{4.431435in}{6.323136in}}{\pgfqpoint{4.431435in}{6.312086in}}%
\pgfpathcurveto{\pgfqpoint{4.431435in}{6.301036in}}{\pgfqpoint{4.435825in}{6.290437in}}{\pgfqpoint{4.443639in}{6.282623in}}%
\pgfpathcurveto{\pgfqpoint{4.451452in}{6.274809in}}{\pgfqpoint{4.462051in}{6.270419in}}{\pgfqpoint{4.473102in}{6.270419in}}%
\pgfpathclose%
\pgfusepath{stroke,fill}%
\end{pgfscope}%
\begin{pgfscope}%
\pgfpathrectangle{\pgfqpoint{0.481978in}{0.331635in}}{\pgfqpoint{9.300000in}{7.700000in}}%
\pgfusepath{clip}%
\pgfsetbuttcap%
\pgfsetroundjoin%
\definecolor{currentfill}{rgb}{0.631373,0.788235,0.956863}%
\pgfsetfillcolor{currentfill}%
\pgfsetlinewidth{0.481800pt}%
\definecolor{currentstroke}{rgb}{1.000000,1.000000,1.000000}%
\pgfsetstrokecolor{currentstroke}%
\pgfsetdash{}{0pt}%
\pgfpathmoveto{\pgfqpoint{2.894322in}{3.546196in}}%
\pgfpathcurveto{\pgfqpoint{2.905372in}{3.546196in}}{\pgfqpoint{2.915971in}{3.550587in}}{\pgfqpoint{2.923785in}{3.558400in}}%
\pgfpathcurveto{\pgfqpoint{2.931599in}{3.566214in}}{\pgfqpoint{2.935989in}{3.576813in}}{\pgfqpoint{2.935989in}{3.587863in}}%
\pgfpathcurveto{\pgfqpoint{2.935989in}{3.598913in}}{\pgfqpoint{2.931599in}{3.609512in}}{\pgfqpoint{2.923785in}{3.617326in}}%
\pgfpathcurveto{\pgfqpoint{2.915971in}{3.625139in}}{\pgfqpoint{2.905372in}{3.629530in}}{\pgfqpoint{2.894322in}{3.629530in}}%
\pgfpathcurveto{\pgfqpoint{2.883272in}{3.629530in}}{\pgfqpoint{2.872673in}{3.625139in}}{\pgfqpoint{2.864859in}{3.617326in}}%
\pgfpathcurveto{\pgfqpoint{2.857046in}{3.609512in}}{\pgfqpoint{2.852656in}{3.598913in}}{\pgfqpoint{2.852656in}{3.587863in}}%
\pgfpathcurveto{\pgfqpoint{2.852656in}{3.576813in}}{\pgfqpoint{2.857046in}{3.566214in}}{\pgfqpoint{2.864859in}{3.558400in}}%
\pgfpathcurveto{\pgfqpoint{2.872673in}{3.550587in}}{\pgfqpoint{2.883272in}{3.546196in}}{\pgfqpoint{2.894322in}{3.546196in}}%
\pgfpathclose%
\pgfusepath{stroke,fill}%
\end{pgfscope}%
\begin{pgfscope}%
\pgfpathrectangle{\pgfqpoint{0.481978in}{0.331635in}}{\pgfqpoint{9.300000in}{7.700000in}}%
\pgfusepath{clip}%
\pgfsetbuttcap%
\pgfsetroundjoin%
\definecolor{currentfill}{rgb}{0.631373,0.788235,0.956863}%
\pgfsetfillcolor{currentfill}%
\pgfsetlinewidth{0.481800pt}%
\definecolor{currentstroke}{rgb}{1.000000,1.000000,1.000000}%
\pgfsetstrokecolor{currentstroke}%
\pgfsetdash{}{0pt}%
\pgfpathmoveto{\pgfqpoint{2.835916in}{4.175572in}}%
\pgfpathcurveto{\pgfqpoint{2.846966in}{4.175572in}}{\pgfqpoint{2.857565in}{4.179962in}}{\pgfqpoint{2.865378in}{4.187776in}}%
\pgfpathcurveto{\pgfqpoint{2.873192in}{4.195589in}}{\pgfqpoint{2.877582in}{4.206189in}}{\pgfqpoint{2.877582in}{4.217239in}}%
\pgfpathcurveto{\pgfqpoint{2.877582in}{4.228289in}}{\pgfqpoint{2.873192in}{4.238888in}}{\pgfqpoint{2.865378in}{4.246701in}}%
\pgfpathcurveto{\pgfqpoint{2.857565in}{4.254515in}}{\pgfqpoint{2.846966in}{4.258905in}}{\pgfqpoint{2.835916in}{4.258905in}}%
\pgfpathcurveto{\pgfqpoint{2.824865in}{4.258905in}}{\pgfqpoint{2.814266in}{4.254515in}}{\pgfqpoint{2.806453in}{4.246701in}}%
\pgfpathcurveto{\pgfqpoint{2.798639in}{4.238888in}}{\pgfqpoint{2.794249in}{4.228289in}}{\pgfqpoint{2.794249in}{4.217239in}}%
\pgfpathcurveto{\pgfqpoint{2.794249in}{4.206189in}}{\pgfqpoint{2.798639in}{4.195589in}}{\pgfqpoint{2.806453in}{4.187776in}}%
\pgfpathcurveto{\pgfqpoint{2.814266in}{4.179962in}}{\pgfqpoint{2.824865in}{4.175572in}}{\pgfqpoint{2.835916in}{4.175572in}}%
\pgfpathclose%
\pgfusepath{stroke,fill}%
\end{pgfscope}%
\begin{pgfscope}%
\pgfpathrectangle{\pgfqpoint{0.481978in}{0.331635in}}{\pgfqpoint{9.300000in}{7.700000in}}%
\pgfusepath{clip}%
\pgfsetbuttcap%
\pgfsetroundjoin%
\definecolor{currentfill}{rgb}{0.631373,0.788235,0.956863}%
\pgfsetfillcolor{currentfill}%
\pgfsetlinewidth{0.481800pt}%
\definecolor{currentstroke}{rgb}{1.000000,1.000000,1.000000}%
\pgfsetstrokecolor{currentstroke}%
\pgfsetdash{}{0pt}%
\pgfpathmoveto{\pgfqpoint{6.230328in}{6.601066in}}%
\pgfpathcurveto{\pgfqpoint{6.241378in}{6.601066in}}{\pgfqpoint{6.251977in}{6.605456in}}{\pgfqpoint{6.259791in}{6.613270in}}%
\pgfpathcurveto{\pgfqpoint{6.267605in}{6.621084in}}{\pgfqpoint{6.271995in}{6.631683in}}{\pgfqpoint{6.271995in}{6.642733in}}%
\pgfpathcurveto{\pgfqpoint{6.271995in}{6.653783in}}{\pgfqpoint{6.267605in}{6.664382in}}{\pgfqpoint{6.259791in}{6.672195in}}%
\pgfpathcurveto{\pgfqpoint{6.251977in}{6.680009in}}{\pgfqpoint{6.241378in}{6.684399in}}{\pgfqpoint{6.230328in}{6.684399in}}%
\pgfpathcurveto{\pgfqpoint{6.219278in}{6.684399in}}{\pgfqpoint{6.208679in}{6.680009in}}{\pgfqpoint{6.200866in}{6.672195in}}%
\pgfpathcurveto{\pgfqpoint{6.193052in}{6.664382in}}{\pgfqpoint{6.188662in}{6.653783in}}{\pgfqpoint{6.188662in}{6.642733in}}%
\pgfpathcurveto{\pgfqpoint{6.188662in}{6.631683in}}{\pgfqpoint{6.193052in}{6.621084in}}{\pgfqpoint{6.200866in}{6.613270in}}%
\pgfpathcurveto{\pgfqpoint{6.208679in}{6.605456in}}{\pgfqpoint{6.219278in}{6.601066in}}{\pgfqpoint{6.230328in}{6.601066in}}%
\pgfpathclose%
\pgfusepath{stroke,fill}%
\end{pgfscope}%
\begin{pgfscope}%
\pgfpathrectangle{\pgfqpoint{0.481978in}{0.331635in}}{\pgfqpoint{9.300000in}{7.700000in}}%
\pgfusepath{clip}%
\pgfsetbuttcap%
\pgfsetroundjoin%
\definecolor{currentfill}{rgb}{0.631373,0.788235,0.956863}%
\pgfsetfillcolor{currentfill}%
\pgfsetlinewidth{0.481800pt}%
\definecolor{currentstroke}{rgb}{1.000000,1.000000,1.000000}%
\pgfsetstrokecolor{currentstroke}%
\pgfsetdash{}{0pt}%
\pgfpathmoveto{\pgfqpoint{4.009340in}{4.285681in}}%
\pgfpathcurveto{\pgfqpoint{4.020390in}{4.285681in}}{\pgfqpoint{4.030989in}{4.290071in}}{\pgfqpoint{4.038803in}{4.297885in}}%
\pgfpathcurveto{\pgfqpoint{4.046617in}{4.305699in}}{\pgfqpoint{4.051007in}{4.316298in}}{\pgfqpoint{4.051007in}{4.327348in}}%
\pgfpathcurveto{\pgfqpoint{4.051007in}{4.338398in}}{\pgfqpoint{4.046617in}{4.348997in}}{\pgfqpoint{4.038803in}{4.356811in}}%
\pgfpathcurveto{\pgfqpoint{4.030989in}{4.364624in}}{\pgfqpoint{4.020390in}{4.369014in}}{\pgfqpoint{4.009340in}{4.369014in}}%
\pgfpathcurveto{\pgfqpoint{3.998290in}{4.369014in}}{\pgfqpoint{3.987691in}{4.364624in}}{\pgfqpoint{3.979877in}{4.356811in}}%
\pgfpathcurveto{\pgfqpoint{3.972064in}{4.348997in}}{\pgfqpoint{3.967674in}{4.338398in}}{\pgfqpoint{3.967674in}{4.327348in}}%
\pgfpathcurveto{\pgfqpoint{3.967674in}{4.316298in}}{\pgfqpoint{3.972064in}{4.305699in}}{\pgfqpoint{3.979877in}{4.297885in}}%
\pgfpathcurveto{\pgfqpoint{3.987691in}{4.290071in}}{\pgfqpoint{3.998290in}{4.285681in}}{\pgfqpoint{4.009340in}{4.285681in}}%
\pgfpathclose%
\pgfusepath{stroke,fill}%
\end{pgfscope}%
\begin{pgfscope}%
\pgfpathrectangle{\pgfqpoint{0.481978in}{0.331635in}}{\pgfqpoint{9.300000in}{7.700000in}}%
\pgfusepath{clip}%
\pgfsetbuttcap%
\pgfsetroundjoin%
\definecolor{currentfill}{rgb}{0.631373,0.788235,0.956863}%
\pgfsetfillcolor{currentfill}%
\pgfsetlinewidth{0.481800pt}%
\definecolor{currentstroke}{rgb}{1.000000,1.000000,1.000000}%
\pgfsetstrokecolor{currentstroke}%
\pgfsetdash{}{0pt}%
\pgfpathmoveto{\pgfqpoint{5.539390in}{5.683934in}}%
\pgfpathcurveto{\pgfqpoint{5.550440in}{5.683934in}}{\pgfqpoint{5.561039in}{5.688324in}}{\pgfqpoint{5.568853in}{5.696138in}}%
\pgfpathcurveto{\pgfqpoint{5.576667in}{5.703951in}}{\pgfqpoint{5.581057in}{5.714550in}}{\pgfqpoint{5.581057in}{5.725601in}}%
\pgfpathcurveto{\pgfqpoint{5.581057in}{5.736651in}}{\pgfqpoint{5.576667in}{5.747250in}}{\pgfqpoint{5.568853in}{5.755063in}}%
\pgfpathcurveto{\pgfqpoint{5.561039in}{5.762877in}}{\pgfqpoint{5.550440in}{5.767267in}}{\pgfqpoint{5.539390in}{5.767267in}}%
\pgfpathcurveto{\pgfqpoint{5.528340in}{5.767267in}}{\pgfqpoint{5.517741in}{5.762877in}}{\pgfqpoint{5.509928in}{5.755063in}}%
\pgfpathcurveto{\pgfqpoint{5.502114in}{5.747250in}}{\pgfqpoint{5.497724in}{5.736651in}}{\pgfqpoint{5.497724in}{5.725601in}}%
\pgfpathcurveto{\pgfqpoint{5.497724in}{5.714550in}}{\pgfqpoint{5.502114in}{5.703951in}}{\pgfqpoint{5.509928in}{5.696138in}}%
\pgfpathcurveto{\pgfqpoint{5.517741in}{5.688324in}}{\pgfqpoint{5.528340in}{5.683934in}}{\pgfqpoint{5.539390in}{5.683934in}}%
\pgfpathclose%
\pgfusepath{stroke,fill}%
\end{pgfscope}%
\begin{pgfscope}%
\pgfpathrectangle{\pgfqpoint{0.481978in}{0.331635in}}{\pgfqpoint{9.300000in}{7.700000in}}%
\pgfusepath{clip}%
\pgfsetbuttcap%
\pgfsetroundjoin%
\definecolor{currentfill}{rgb}{0.631373,0.788235,0.956863}%
\pgfsetfillcolor{currentfill}%
\pgfsetlinewidth{0.481800pt}%
\definecolor{currentstroke}{rgb}{1.000000,1.000000,1.000000}%
\pgfsetstrokecolor{currentstroke}%
\pgfsetdash{}{0pt}%
\pgfpathmoveto{\pgfqpoint{2.431291in}{1.829495in}}%
\pgfpathcurveto{\pgfqpoint{2.442341in}{1.829495in}}{\pgfqpoint{2.452940in}{1.833885in}}{\pgfqpoint{2.460754in}{1.841699in}}%
\pgfpathcurveto{\pgfqpoint{2.468568in}{1.849512in}}{\pgfqpoint{2.472958in}{1.860111in}}{\pgfqpoint{2.472958in}{1.871162in}}%
\pgfpathcurveto{\pgfqpoint{2.472958in}{1.882212in}}{\pgfqpoint{2.468568in}{1.892811in}}{\pgfqpoint{2.460754in}{1.900624in}}%
\pgfpathcurveto{\pgfqpoint{2.452940in}{1.908438in}}{\pgfqpoint{2.442341in}{1.912828in}}{\pgfqpoint{2.431291in}{1.912828in}}%
\pgfpathcurveto{\pgfqpoint{2.420241in}{1.912828in}}{\pgfqpoint{2.409642in}{1.908438in}}{\pgfqpoint{2.401828in}{1.900624in}}%
\pgfpathcurveto{\pgfqpoint{2.394015in}{1.892811in}}{\pgfqpoint{2.389624in}{1.882212in}}{\pgfqpoint{2.389624in}{1.871162in}}%
\pgfpathcurveto{\pgfqpoint{2.389624in}{1.860111in}}{\pgfqpoint{2.394015in}{1.849512in}}{\pgfqpoint{2.401828in}{1.841699in}}%
\pgfpathcurveto{\pgfqpoint{2.409642in}{1.833885in}}{\pgfqpoint{2.420241in}{1.829495in}}{\pgfqpoint{2.431291in}{1.829495in}}%
\pgfpathclose%
\pgfusepath{stroke,fill}%
\end{pgfscope}%
\begin{pgfscope}%
\pgfpathrectangle{\pgfqpoint{0.481978in}{0.331635in}}{\pgfqpoint{9.300000in}{7.700000in}}%
\pgfusepath{clip}%
\pgfsetbuttcap%
\pgfsetroundjoin%
\definecolor{currentfill}{rgb}{0.631373,0.788235,0.956863}%
\pgfsetfillcolor{currentfill}%
\pgfsetlinewidth{0.481800pt}%
\definecolor{currentstroke}{rgb}{1.000000,1.000000,1.000000}%
\pgfsetstrokecolor{currentstroke}%
\pgfsetdash{}{0pt}%
\pgfpathmoveto{\pgfqpoint{6.872681in}{1.417795in}}%
\pgfpathcurveto{\pgfqpoint{6.883731in}{1.417795in}}{\pgfqpoint{6.894330in}{1.422185in}}{\pgfqpoint{6.902143in}{1.429999in}}%
\pgfpathcurveto{\pgfqpoint{6.909957in}{1.437812in}}{\pgfqpoint{6.914347in}{1.448411in}}{\pgfqpoint{6.914347in}{1.459461in}}%
\pgfpathcurveto{\pgfqpoint{6.914347in}{1.470511in}}{\pgfqpoint{6.909957in}{1.481110in}}{\pgfqpoint{6.902143in}{1.488924in}}%
\pgfpathcurveto{\pgfqpoint{6.894330in}{1.496738in}}{\pgfqpoint{6.883731in}{1.501128in}}{\pgfqpoint{6.872681in}{1.501128in}}%
\pgfpathcurveto{\pgfqpoint{6.861630in}{1.501128in}}{\pgfqpoint{6.851031in}{1.496738in}}{\pgfqpoint{6.843218in}{1.488924in}}%
\pgfpathcurveto{\pgfqpoint{6.835404in}{1.481110in}}{\pgfqpoint{6.831014in}{1.470511in}}{\pgfqpoint{6.831014in}{1.459461in}}%
\pgfpathcurveto{\pgfqpoint{6.831014in}{1.448411in}}{\pgfqpoint{6.835404in}{1.437812in}}{\pgfqpoint{6.843218in}{1.429999in}}%
\pgfpathcurveto{\pgfqpoint{6.851031in}{1.422185in}}{\pgfqpoint{6.861630in}{1.417795in}}{\pgfqpoint{6.872681in}{1.417795in}}%
\pgfpathclose%
\pgfusepath{stroke,fill}%
\end{pgfscope}%
\begin{pgfscope}%
\pgfpathrectangle{\pgfqpoint{0.481978in}{0.331635in}}{\pgfqpoint{9.300000in}{7.700000in}}%
\pgfusepath{clip}%
\pgfsetbuttcap%
\pgfsetroundjoin%
\definecolor{currentfill}{rgb}{0.631373,0.788235,0.956863}%
\pgfsetfillcolor{currentfill}%
\pgfsetlinewidth{0.481800pt}%
\definecolor{currentstroke}{rgb}{1.000000,1.000000,1.000000}%
\pgfsetstrokecolor{currentstroke}%
\pgfsetdash{}{0pt}%
\pgfpathmoveto{\pgfqpoint{6.738984in}{6.562070in}}%
\pgfpathcurveto{\pgfqpoint{6.750034in}{6.562070in}}{\pgfqpoint{6.760633in}{6.566460in}}{\pgfqpoint{6.768447in}{6.574274in}}%
\pgfpathcurveto{\pgfqpoint{6.776260in}{6.582087in}}{\pgfqpoint{6.780650in}{6.592687in}}{\pgfqpoint{6.780650in}{6.603737in}}%
\pgfpathcurveto{\pgfqpoint{6.780650in}{6.614787in}}{\pgfqpoint{6.776260in}{6.625386in}}{\pgfqpoint{6.768447in}{6.633199in}}%
\pgfpathcurveto{\pgfqpoint{6.760633in}{6.641013in}}{\pgfqpoint{6.750034in}{6.645403in}}{\pgfqpoint{6.738984in}{6.645403in}}%
\pgfpathcurveto{\pgfqpoint{6.727934in}{6.645403in}}{\pgfqpoint{6.717335in}{6.641013in}}{\pgfqpoint{6.709521in}{6.633199in}}%
\pgfpathcurveto{\pgfqpoint{6.701707in}{6.625386in}}{\pgfqpoint{6.697317in}{6.614787in}}{\pgfqpoint{6.697317in}{6.603737in}}%
\pgfpathcurveto{\pgfqpoint{6.697317in}{6.592687in}}{\pgfqpoint{6.701707in}{6.582087in}}{\pgfqpoint{6.709521in}{6.574274in}}%
\pgfpathcurveto{\pgfqpoint{6.717335in}{6.566460in}}{\pgfqpoint{6.727934in}{6.562070in}}{\pgfqpoint{6.738984in}{6.562070in}}%
\pgfpathclose%
\pgfusepath{stroke,fill}%
\end{pgfscope}%
\begin{pgfscope}%
\pgfpathrectangle{\pgfqpoint{0.481978in}{0.331635in}}{\pgfqpoint{9.300000in}{7.700000in}}%
\pgfusepath{clip}%
\pgfsetbuttcap%
\pgfsetroundjoin%
\definecolor{currentfill}{rgb}{0.631373,0.788235,0.956863}%
\pgfsetfillcolor{currentfill}%
\pgfsetlinewidth{0.481800pt}%
\definecolor{currentstroke}{rgb}{1.000000,1.000000,1.000000}%
\pgfsetstrokecolor{currentstroke}%
\pgfsetdash{}{0pt}%
\pgfpathmoveto{\pgfqpoint{2.165107in}{3.332644in}}%
\pgfpathcurveto{\pgfqpoint{2.176158in}{3.332644in}}{\pgfqpoint{2.186757in}{3.337035in}}{\pgfqpoint{2.194570in}{3.344848in}}%
\pgfpathcurveto{\pgfqpoint{2.202384in}{3.352662in}}{\pgfqpoint{2.206774in}{3.363261in}}{\pgfqpoint{2.206774in}{3.374311in}}%
\pgfpathcurveto{\pgfqpoint{2.206774in}{3.385361in}}{\pgfqpoint{2.202384in}{3.395960in}}{\pgfqpoint{2.194570in}{3.403774in}}%
\pgfpathcurveto{\pgfqpoint{2.186757in}{3.411587in}}{\pgfqpoint{2.176158in}{3.415978in}}{\pgfqpoint{2.165107in}{3.415978in}}%
\pgfpathcurveto{\pgfqpoint{2.154057in}{3.415978in}}{\pgfqpoint{2.143458in}{3.411587in}}{\pgfqpoint{2.135645in}{3.403774in}}%
\pgfpathcurveto{\pgfqpoint{2.127831in}{3.395960in}}{\pgfqpoint{2.123441in}{3.385361in}}{\pgfqpoint{2.123441in}{3.374311in}}%
\pgfpathcurveto{\pgfqpoint{2.123441in}{3.363261in}}{\pgfqpoint{2.127831in}{3.352662in}}{\pgfqpoint{2.135645in}{3.344848in}}%
\pgfpathcurveto{\pgfqpoint{2.143458in}{3.337035in}}{\pgfqpoint{2.154057in}{3.332644in}}{\pgfqpoint{2.165107in}{3.332644in}}%
\pgfpathclose%
\pgfusepath{stroke,fill}%
\end{pgfscope}%
\begin{pgfscope}%
\pgfpathrectangle{\pgfqpoint{0.481978in}{0.331635in}}{\pgfqpoint{9.300000in}{7.700000in}}%
\pgfusepath{clip}%
\pgfsetbuttcap%
\pgfsetroundjoin%
\definecolor{currentfill}{rgb}{0.631373,0.788235,0.956863}%
\pgfsetfillcolor{currentfill}%
\pgfsetlinewidth{0.481800pt}%
\definecolor{currentstroke}{rgb}{1.000000,1.000000,1.000000}%
\pgfsetstrokecolor{currentstroke}%
\pgfsetdash{}{0pt}%
\pgfpathmoveto{\pgfqpoint{5.720593in}{1.115651in}}%
\pgfpathcurveto{\pgfqpoint{5.731643in}{1.115651in}}{\pgfqpoint{5.742242in}{1.120041in}}{\pgfqpoint{5.750056in}{1.127855in}}%
\pgfpathcurveto{\pgfqpoint{5.757869in}{1.135668in}}{\pgfqpoint{5.762260in}{1.146267in}}{\pgfqpoint{5.762260in}{1.157318in}}%
\pgfpathcurveto{\pgfqpoint{5.762260in}{1.168368in}}{\pgfqpoint{5.757869in}{1.178967in}}{\pgfqpoint{5.750056in}{1.186780in}}%
\pgfpathcurveto{\pgfqpoint{5.742242in}{1.194594in}}{\pgfqpoint{5.731643in}{1.198984in}}{\pgfqpoint{5.720593in}{1.198984in}}%
\pgfpathcurveto{\pgfqpoint{5.709543in}{1.198984in}}{\pgfqpoint{5.698944in}{1.194594in}}{\pgfqpoint{5.691130in}{1.186780in}}%
\pgfpathcurveto{\pgfqpoint{5.683317in}{1.178967in}}{\pgfqpoint{5.678926in}{1.168368in}}{\pgfqpoint{5.678926in}{1.157318in}}%
\pgfpathcurveto{\pgfqpoint{5.678926in}{1.146267in}}{\pgfqpoint{5.683317in}{1.135668in}}{\pgfqpoint{5.691130in}{1.127855in}}%
\pgfpathcurveto{\pgfqpoint{5.698944in}{1.120041in}}{\pgfqpoint{5.709543in}{1.115651in}}{\pgfqpoint{5.720593in}{1.115651in}}%
\pgfpathclose%
\pgfusepath{stroke,fill}%
\end{pgfscope}%
\begin{pgfscope}%
\pgfpathrectangle{\pgfqpoint{0.481978in}{0.331635in}}{\pgfqpoint{9.300000in}{7.700000in}}%
\pgfusepath{clip}%
\pgfsetbuttcap%
\pgfsetroundjoin%
\definecolor{currentfill}{rgb}{0.631373,0.788235,0.956863}%
\pgfsetfillcolor{currentfill}%
\pgfsetlinewidth{0.481800pt}%
\definecolor{currentstroke}{rgb}{1.000000,1.000000,1.000000}%
\pgfsetstrokecolor{currentstroke}%
\pgfsetdash{}{0pt}%
\pgfpathmoveto{\pgfqpoint{1.454724in}{5.319174in}}%
\pgfpathcurveto{\pgfqpoint{1.465774in}{5.319174in}}{\pgfqpoint{1.476373in}{5.323564in}}{\pgfqpoint{1.484187in}{5.331378in}}%
\pgfpathcurveto{\pgfqpoint{1.492000in}{5.339192in}}{\pgfqpoint{1.496391in}{5.349791in}}{\pgfqpoint{1.496391in}{5.360841in}}%
\pgfpathcurveto{\pgfqpoint{1.496391in}{5.371891in}}{\pgfqpoint{1.492000in}{5.382490in}}{\pgfqpoint{1.484187in}{5.390303in}}%
\pgfpathcurveto{\pgfqpoint{1.476373in}{5.398117in}}{\pgfqpoint{1.465774in}{5.402507in}}{\pgfqpoint{1.454724in}{5.402507in}}%
\pgfpathcurveto{\pgfqpoint{1.443674in}{5.402507in}}{\pgfqpoint{1.433075in}{5.398117in}}{\pgfqpoint{1.425261in}{5.390303in}}%
\pgfpathcurveto{\pgfqpoint{1.417448in}{5.382490in}}{\pgfqpoint{1.413057in}{5.371891in}}{\pgfqpoint{1.413057in}{5.360841in}}%
\pgfpathcurveto{\pgfqpoint{1.413057in}{5.349791in}}{\pgfqpoint{1.417448in}{5.339192in}}{\pgfqpoint{1.425261in}{5.331378in}}%
\pgfpathcurveto{\pgfqpoint{1.433075in}{5.323564in}}{\pgfqpoint{1.443674in}{5.319174in}}{\pgfqpoint{1.454724in}{5.319174in}}%
\pgfpathclose%
\pgfusepath{stroke,fill}%
\end{pgfscope}%
\begin{pgfscope}%
\pgfpathrectangle{\pgfqpoint{0.481978in}{0.331635in}}{\pgfqpoint{9.300000in}{7.700000in}}%
\pgfusepath{clip}%
\pgfsetbuttcap%
\pgfsetroundjoin%
\definecolor{currentfill}{rgb}{0.631373,0.788235,0.956863}%
\pgfsetfillcolor{currentfill}%
\pgfsetlinewidth{0.481800pt}%
\definecolor{currentstroke}{rgb}{1.000000,1.000000,1.000000}%
\pgfsetstrokecolor{currentstroke}%
\pgfsetdash{}{0pt}%
\pgfpathmoveto{\pgfqpoint{4.584792in}{1.847011in}}%
\pgfpathcurveto{\pgfqpoint{4.595842in}{1.847011in}}{\pgfqpoint{4.606441in}{1.851402in}}{\pgfqpoint{4.614255in}{1.859215in}}%
\pgfpathcurveto{\pgfqpoint{4.622069in}{1.867029in}}{\pgfqpoint{4.626459in}{1.877628in}}{\pgfqpoint{4.626459in}{1.888678in}}%
\pgfpathcurveto{\pgfqpoint{4.626459in}{1.899728in}}{\pgfqpoint{4.622069in}{1.910327in}}{\pgfqpoint{4.614255in}{1.918141in}}%
\pgfpathcurveto{\pgfqpoint{4.606441in}{1.925954in}}{\pgfqpoint{4.595842in}{1.930345in}}{\pgfqpoint{4.584792in}{1.930345in}}%
\pgfpathcurveto{\pgfqpoint{4.573742in}{1.930345in}}{\pgfqpoint{4.563143in}{1.925954in}}{\pgfqpoint{4.555330in}{1.918141in}}%
\pgfpathcurveto{\pgfqpoint{4.547516in}{1.910327in}}{\pgfqpoint{4.543126in}{1.899728in}}{\pgfqpoint{4.543126in}{1.888678in}}%
\pgfpathcurveto{\pgfqpoint{4.543126in}{1.877628in}}{\pgfqpoint{4.547516in}{1.867029in}}{\pgfqpoint{4.555330in}{1.859215in}}%
\pgfpathcurveto{\pgfqpoint{4.563143in}{1.851402in}}{\pgfqpoint{4.573742in}{1.847011in}}{\pgfqpoint{4.584792in}{1.847011in}}%
\pgfpathclose%
\pgfusepath{stroke,fill}%
\end{pgfscope}%
\begin{pgfscope}%
\pgfpathrectangle{\pgfqpoint{0.481978in}{0.331635in}}{\pgfqpoint{9.300000in}{7.700000in}}%
\pgfusepath{clip}%
\pgfsetbuttcap%
\pgfsetroundjoin%
\definecolor{currentfill}{rgb}{0.631373,0.788235,0.956863}%
\pgfsetfillcolor{currentfill}%
\pgfsetlinewidth{0.481800pt}%
\definecolor{currentstroke}{rgb}{1.000000,1.000000,1.000000}%
\pgfsetstrokecolor{currentstroke}%
\pgfsetdash{}{0pt}%
\pgfpathmoveto{\pgfqpoint{4.904799in}{5.869876in}}%
\pgfpathcurveto{\pgfqpoint{4.915849in}{5.869876in}}{\pgfqpoint{4.926448in}{5.874266in}}{\pgfqpoint{4.934262in}{5.882080in}}%
\pgfpathcurveto{\pgfqpoint{4.942075in}{5.889893in}}{\pgfqpoint{4.946466in}{5.900492in}}{\pgfqpoint{4.946466in}{5.911543in}}%
\pgfpathcurveto{\pgfqpoint{4.946466in}{5.922593in}}{\pgfqpoint{4.942075in}{5.933192in}}{\pgfqpoint{4.934262in}{5.941005in}}%
\pgfpathcurveto{\pgfqpoint{4.926448in}{5.948819in}}{\pgfqpoint{4.915849in}{5.953209in}}{\pgfqpoint{4.904799in}{5.953209in}}%
\pgfpathcurveto{\pgfqpoint{4.893749in}{5.953209in}}{\pgfqpoint{4.883150in}{5.948819in}}{\pgfqpoint{4.875336in}{5.941005in}}%
\pgfpathcurveto{\pgfqpoint{4.867523in}{5.933192in}}{\pgfqpoint{4.863132in}{5.922593in}}{\pgfqpoint{4.863132in}{5.911543in}}%
\pgfpathcurveto{\pgfqpoint{4.863132in}{5.900492in}}{\pgfqpoint{4.867523in}{5.889893in}}{\pgfqpoint{4.875336in}{5.882080in}}%
\pgfpathcurveto{\pgfqpoint{4.883150in}{5.874266in}}{\pgfqpoint{4.893749in}{5.869876in}}{\pgfqpoint{4.904799in}{5.869876in}}%
\pgfpathclose%
\pgfusepath{stroke,fill}%
\end{pgfscope}%
\begin{pgfscope}%
\pgfpathrectangle{\pgfqpoint{0.481978in}{0.331635in}}{\pgfqpoint{9.300000in}{7.700000in}}%
\pgfusepath{clip}%
\pgfsetbuttcap%
\pgfsetroundjoin%
\definecolor{currentfill}{rgb}{0.631373,0.788235,0.956863}%
\pgfsetfillcolor{currentfill}%
\pgfsetlinewidth{0.481800pt}%
\definecolor{currentstroke}{rgb}{1.000000,1.000000,1.000000}%
\pgfsetstrokecolor{currentstroke}%
\pgfsetdash{}{0pt}%
\pgfpathmoveto{\pgfqpoint{4.893406in}{4.156647in}}%
\pgfpathcurveto{\pgfqpoint{4.904456in}{4.156647in}}{\pgfqpoint{4.915055in}{4.161037in}}{\pgfqpoint{4.922869in}{4.168851in}}%
\pgfpathcurveto{\pgfqpoint{4.930682in}{4.176665in}}{\pgfqpoint{4.935072in}{4.187264in}}{\pgfqpoint{4.935072in}{4.198314in}}%
\pgfpathcurveto{\pgfqpoint{4.935072in}{4.209364in}}{\pgfqpoint{4.930682in}{4.219963in}}{\pgfqpoint{4.922869in}{4.227777in}}%
\pgfpathcurveto{\pgfqpoint{4.915055in}{4.235590in}}{\pgfqpoint{4.904456in}{4.239981in}}{\pgfqpoint{4.893406in}{4.239981in}}%
\pgfpathcurveto{\pgfqpoint{4.882356in}{4.239981in}}{\pgfqpoint{4.871757in}{4.235590in}}{\pgfqpoint{4.863943in}{4.227777in}}%
\pgfpathcurveto{\pgfqpoint{4.856129in}{4.219963in}}{\pgfqpoint{4.851739in}{4.209364in}}{\pgfqpoint{4.851739in}{4.198314in}}%
\pgfpathcurveto{\pgfqpoint{4.851739in}{4.187264in}}{\pgfqpoint{4.856129in}{4.176665in}}{\pgfqpoint{4.863943in}{4.168851in}}%
\pgfpathcurveto{\pgfqpoint{4.871757in}{4.161037in}}{\pgfqpoint{4.882356in}{4.156647in}}{\pgfqpoint{4.893406in}{4.156647in}}%
\pgfpathclose%
\pgfusepath{stroke,fill}%
\end{pgfscope}%
\begin{pgfscope}%
\pgfpathrectangle{\pgfqpoint{0.481978in}{0.331635in}}{\pgfqpoint{9.300000in}{7.700000in}}%
\pgfusepath{clip}%
\pgfsetbuttcap%
\pgfsetroundjoin%
\definecolor{currentfill}{rgb}{0.631373,0.788235,0.956863}%
\pgfsetfillcolor{currentfill}%
\pgfsetlinewidth{0.481800pt}%
\definecolor{currentstroke}{rgb}{1.000000,1.000000,1.000000}%
\pgfsetstrokecolor{currentstroke}%
\pgfsetdash{}{0pt}%
\pgfpathmoveto{\pgfqpoint{5.321160in}{7.572579in}}%
\pgfpathcurveto{\pgfqpoint{5.332210in}{7.572579in}}{\pgfqpoint{5.342809in}{7.576969in}}{\pgfqpoint{5.350623in}{7.584783in}}%
\pgfpathcurveto{\pgfqpoint{5.358437in}{7.592597in}}{\pgfqpoint{5.362827in}{7.603196in}}{\pgfqpoint{5.362827in}{7.614246in}}%
\pgfpathcurveto{\pgfqpoint{5.362827in}{7.625296in}}{\pgfqpoint{5.358437in}{7.635895in}}{\pgfqpoint{5.350623in}{7.643708in}}%
\pgfpathcurveto{\pgfqpoint{5.342809in}{7.651522in}}{\pgfqpoint{5.332210in}{7.655912in}}{\pgfqpoint{5.321160in}{7.655912in}}%
\pgfpathcurveto{\pgfqpoint{5.310110in}{7.655912in}}{\pgfqpoint{5.299511in}{7.651522in}}{\pgfqpoint{5.291697in}{7.643708in}}%
\pgfpathcurveto{\pgfqpoint{5.283884in}{7.635895in}}{\pgfqpoint{5.279494in}{7.625296in}}{\pgfqpoint{5.279494in}{7.614246in}}%
\pgfpathcurveto{\pgfqpoint{5.279494in}{7.603196in}}{\pgfqpoint{5.283884in}{7.592597in}}{\pgfqpoint{5.291697in}{7.584783in}}%
\pgfpathcurveto{\pgfqpoint{5.299511in}{7.576969in}}{\pgfqpoint{5.310110in}{7.572579in}}{\pgfqpoint{5.321160in}{7.572579in}}%
\pgfpathclose%
\pgfusepath{stroke,fill}%
\end{pgfscope}%
\begin{pgfscope}%
\pgfpathrectangle{\pgfqpoint{0.481978in}{0.331635in}}{\pgfqpoint{9.300000in}{7.700000in}}%
\pgfusepath{clip}%
\pgfsetbuttcap%
\pgfsetroundjoin%
\definecolor{currentfill}{rgb}{0.631373,0.788235,0.956863}%
\pgfsetfillcolor{currentfill}%
\pgfsetlinewidth{0.481800pt}%
\definecolor{currentstroke}{rgb}{1.000000,1.000000,1.000000}%
\pgfsetstrokecolor{currentstroke}%
\pgfsetdash{}{0pt}%
\pgfpathmoveto{\pgfqpoint{5.212048in}{1.903878in}}%
\pgfpathcurveto{\pgfqpoint{5.223098in}{1.903878in}}{\pgfqpoint{5.233697in}{1.908269in}}{\pgfqpoint{5.241511in}{1.916082in}}%
\pgfpathcurveto{\pgfqpoint{5.249324in}{1.923896in}}{\pgfqpoint{5.253715in}{1.934495in}}{\pgfqpoint{5.253715in}{1.945545in}}%
\pgfpathcurveto{\pgfqpoint{5.253715in}{1.956595in}}{\pgfqpoint{5.249324in}{1.967194in}}{\pgfqpoint{5.241511in}{1.975008in}}%
\pgfpathcurveto{\pgfqpoint{5.233697in}{1.982822in}}{\pgfqpoint{5.223098in}{1.987212in}}{\pgfqpoint{5.212048in}{1.987212in}}%
\pgfpathcurveto{\pgfqpoint{5.200998in}{1.987212in}}{\pgfqpoint{5.190399in}{1.982822in}}{\pgfqpoint{5.182585in}{1.975008in}}%
\pgfpathcurveto{\pgfqpoint{5.174771in}{1.967194in}}{\pgfqpoint{5.170381in}{1.956595in}}{\pgfqpoint{5.170381in}{1.945545in}}%
\pgfpathcurveto{\pgfqpoint{5.170381in}{1.934495in}}{\pgfqpoint{5.174771in}{1.923896in}}{\pgfqpoint{5.182585in}{1.916082in}}%
\pgfpathcurveto{\pgfqpoint{5.190399in}{1.908269in}}{\pgfqpoint{5.200998in}{1.903878in}}{\pgfqpoint{5.212048in}{1.903878in}}%
\pgfpathclose%
\pgfusepath{stroke,fill}%
\end{pgfscope}%
\begin{pgfscope}%
\pgfpathrectangle{\pgfqpoint{0.481978in}{0.331635in}}{\pgfqpoint{9.300000in}{7.700000in}}%
\pgfusepath{clip}%
\pgfsetbuttcap%
\pgfsetroundjoin%
\definecolor{currentfill}{rgb}{0.631373,0.788235,0.956863}%
\pgfsetfillcolor{currentfill}%
\pgfsetlinewidth{0.481800pt}%
\definecolor{currentstroke}{rgb}{1.000000,1.000000,1.000000}%
\pgfsetstrokecolor{currentstroke}%
\pgfsetdash{}{0pt}%
\pgfpathmoveto{\pgfqpoint{3.803419in}{5.836066in}}%
\pgfpathcurveto{\pgfqpoint{3.814469in}{5.836066in}}{\pgfqpoint{3.825068in}{5.840456in}}{\pgfqpoint{3.832881in}{5.848270in}}%
\pgfpathcurveto{\pgfqpoint{3.840695in}{5.856084in}}{\pgfqpoint{3.845085in}{5.866683in}}{\pgfqpoint{3.845085in}{5.877733in}}%
\pgfpathcurveto{\pgfqpoint{3.845085in}{5.888783in}}{\pgfqpoint{3.840695in}{5.899382in}}{\pgfqpoint{3.832881in}{5.907196in}}%
\pgfpathcurveto{\pgfqpoint{3.825068in}{5.915009in}}{\pgfqpoint{3.814469in}{5.919400in}}{\pgfqpoint{3.803419in}{5.919400in}}%
\pgfpathcurveto{\pgfqpoint{3.792369in}{5.919400in}}{\pgfqpoint{3.781769in}{5.915009in}}{\pgfqpoint{3.773956in}{5.907196in}}%
\pgfpathcurveto{\pgfqpoint{3.766142in}{5.899382in}}{\pgfqpoint{3.761752in}{5.888783in}}{\pgfqpoint{3.761752in}{5.877733in}}%
\pgfpathcurveto{\pgfqpoint{3.761752in}{5.866683in}}{\pgfqpoint{3.766142in}{5.856084in}}{\pgfqpoint{3.773956in}{5.848270in}}%
\pgfpathcurveto{\pgfqpoint{3.781769in}{5.840456in}}{\pgfqpoint{3.792369in}{5.836066in}}{\pgfqpoint{3.803419in}{5.836066in}}%
\pgfpathclose%
\pgfusepath{stroke,fill}%
\end{pgfscope}%
\begin{pgfscope}%
\pgfpathrectangle{\pgfqpoint{0.481978in}{0.331635in}}{\pgfqpoint{9.300000in}{7.700000in}}%
\pgfusepath{clip}%
\pgfsetbuttcap%
\pgfsetroundjoin%
\definecolor{currentfill}{rgb}{0.631373,0.788235,0.956863}%
\pgfsetfillcolor{currentfill}%
\pgfsetlinewidth{0.481800pt}%
\definecolor{currentstroke}{rgb}{1.000000,1.000000,1.000000}%
\pgfsetstrokecolor{currentstroke}%
\pgfsetdash{}{0pt}%
\pgfpathmoveto{\pgfqpoint{4.981886in}{7.209319in}}%
\pgfpathcurveto{\pgfqpoint{4.992937in}{7.209319in}}{\pgfqpoint{5.003536in}{7.213709in}}{\pgfqpoint{5.011349in}{7.221522in}}%
\pgfpathcurveto{\pgfqpoint{5.019163in}{7.229336in}}{\pgfqpoint{5.023553in}{7.239935in}}{\pgfqpoint{5.023553in}{7.250985in}}%
\pgfpathcurveto{\pgfqpoint{5.023553in}{7.262035in}}{\pgfqpoint{5.019163in}{7.272634in}}{\pgfqpoint{5.011349in}{7.280448in}}%
\pgfpathcurveto{\pgfqpoint{5.003536in}{7.288262in}}{\pgfqpoint{4.992937in}{7.292652in}}{\pgfqpoint{4.981886in}{7.292652in}}%
\pgfpathcurveto{\pgfqpoint{4.970836in}{7.292652in}}{\pgfqpoint{4.960237in}{7.288262in}}{\pgfqpoint{4.952424in}{7.280448in}}%
\pgfpathcurveto{\pgfqpoint{4.944610in}{7.272634in}}{\pgfqpoint{4.940220in}{7.262035in}}{\pgfqpoint{4.940220in}{7.250985in}}%
\pgfpathcurveto{\pgfqpoint{4.940220in}{7.239935in}}{\pgfqpoint{4.944610in}{7.229336in}}{\pgfqpoint{4.952424in}{7.221522in}}%
\pgfpathcurveto{\pgfqpoint{4.960237in}{7.213709in}}{\pgfqpoint{4.970836in}{7.209319in}}{\pgfqpoint{4.981886in}{7.209319in}}%
\pgfpathclose%
\pgfusepath{stroke,fill}%
\end{pgfscope}%
\begin{pgfscope}%
\pgfpathrectangle{\pgfqpoint{0.481978in}{0.331635in}}{\pgfqpoint{9.300000in}{7.700000in}}%
\pgfusepath{clip}%
\pgfsetbuttcap%
\pgfsetroundjoin%
\definecolor{currentfill}{rgb}{0.631373,0.788235,0.956863}%
\pgfsetfillcolor{currentfill}%
\pgfsetlinewidth{0.481800pt}%
\definecolor{currentstroke}{rgb}{1.000000,1.000000,1.000000}%
\pgfsetstrokecolor{currentstroke}%
\pgfsetdash{}{0pt}%
\pgfpathmoveto{\pgfqpoint{3.839349in}{3.870023in}}%
\pgfpathcurveto{\pgfqpoint{3.850399in}{3.870023in}}{\pgfqpoint{3.860998in}{3.874414in}}{\pgfqpoint{3.868812in}{3.882227in}}%
\pgfpathcurveto{\pgfqpoint{3.876626in}{3.890041in}}{\pgfqpoint{3.881016in}{3.900640in}}{\pgfqpoint{3.881016in}{3.911690in}}%
\pgfpathcurveto{\pgfqpoint{3.881016in}{3.922740in}}{\pgfqpoint{3.876626in}{3.933339in}}{\pgfqpoint{3.868812in}{3.941153in}}%
\pgfpathcurveto{\pgfqpoint{3.860998in}{3.948967in}}{\pgfqpoint{3.850399in}{3.953357in}}{\pgfqpoint{3.839349in}{3.953357in}}%
\pgfpathcurveto{\pgfqpoint{3.828299in}{3.953357in}}{\pgfqpoint{3.817700in}{3.948967in}}{\pgfqpoint{3.809886in}{3.941153in}}%
\pgfpathcurveto{\pgfqpoint{3.802073in}{3.933339in}}{\pgfqpoint{3.797682in}{3.922740in}}{\pgfqpoint{3.797682in}{3.911690in}}%
\pgfpathcurveto{\pgfqpoint{3.797682in}{3.900640in}}{\pgfqpoint{3.802073in}{3.890041in}}{\pgfqpoint{3.809886in}{3.882227in}}%
\pgfpathcurveto{\pgfqpoint{3.817700in}{3.874414in}}{\pgfqpoint{3.828299in}{3.870023in}}{\pgfqpoint{3.839349in}{3.870023in}}%
\pgfpathclose%
\pgfusepath{stroke,fill}%
\end{pgfscope}%
\begin{pgfscope}%
\pgfpathrectangle{\pgfqpoint{0.481978in}{0.331635in}}{\pgfqpoint{9.300000in}{7.700000in}}%
\pgfusepath{clip}%
\pgfsetbuttcap%
\pgfsetroundjoin%
\definecolor{currentfill}{rgb}{0.631373,0.788235,0.956863}%
\pgfsetfillcolor{currentfill}%
\pgfsetlinewidth{0.481800pt}%
\definecolor{currentstroke}{rgb}{1.000000,1.000000,1.000000}%
\pgfsetstrokecolor{currentstroke}%
\pgfsetdash{}{0pt}%
\pgfpathmoveto{\pgfqpoint{6.849912in}{4.108125in}}%
\pgfpathcurveto{\pgfqpoint{6.860962in}{4.108125in}}{\pgfqpoint{6.871561in}{4.112515in}}{\pgfqpoint{6.879375in}{4.120329in}}%
\pgfpathcurveto{\pgfqpoint{6.887188in}{4.128142in}}{\pgfqpoint{6.891579in}{4.138741in}}{\pgfqpoint{6.891579in}{4.149791in}}%
\pgfpathcurveto{\pgfqpoint{6.891579in}{4.160841in}}{\pgfqpoint{6.887188in}{4.171440in}}{\pgfqpoint{6.879375in}{4.179254in}}%
\pgfpathcurveto{\pgfqpoint{6.871561in}{4.187068in}}{\pgfqpoint{6.860962in}{4.191458in}}{\pgfqpoint{6.849912in}{4.191458in}}%
\pgfpathcurveto{\pgfqpoint{6.838862in}{4.191458in}}{\pgfqpoint{6.828263in}{4.187068in}}{\pgfqpoint{6.820449in}{4.179254in}}%
\pgfpathcurveto{\pgfqpoint{6.812636in}{4.171440in}}{\pgfqpoint{6.808245in}{4.160841in}}{\pgfqpoint{6.808245in}{4.149791in}}%
\pgfpathcurveto{\pgfqpoint{6.808245in}{4.138741in}}{\pgfqpoint{6.812636in}{4.128142in}}{\pgfqpoint{6.820449in}{4.120329in}}%
\pgfpathcurveto{\pgfqpoint{6.828263in}{4.112515in}}{\pgfqpoint{6.838862in}{4.108125in}}{\pgfqpoint{6.849912in}{4.108125in}}%
\pgfpathclose%
\pgfusepath{stroke,fill}%
\end{pgfscope}%
\begin{pgfscope}%
\pgfpathrectangle{\pgfqpoint{0.481978in}{0.331635in}}{\pgfqpoint{9.300000in}{7.700000in}}%
\pgfusepath{clip}%
\pgfsetbuttcap%
\pgfsetroundjoin%
\definecolor{currentfill}{rgb}{0.631373,0.788235,0.956863}%
\pgfsetfillcolor{currentfill}%
\pgfsetlinewidth{0.481800pt}%
\definecolor{currentstroke}{rgb}{1.000000,1.000000,1.000000}%
\pgfsetstrokecolor{currentstroke}%
\pgfsetdash{}{0pt}%
\pgfpathmoveto{\pgfqpoint{4.052180in}{2.328627in}}%
\pgfpathcurveto{\pgfqpoint{4.063230in}{2.328627in}}{\pgfqpoint{4.073829in}{2.333017in}}{\pgfqpoint{4.081642in}{2.340831in}}%
\pgfpathcurveto{\pgfqpoint{4.089456in}{2.348644in}}{\pgfqpoint{4.093846in}{2.359243in}}{\pgfqpoint{4.093846in}{2.370294in}}%
\pgfpathcurveto{\pgfqpoint{4.093846in}{2.381344in}}{\pgfqpoint{4.089456in}{2.391943in}}{\pgfqpoint{4.081642in}{2.399756in}}%
\pgfpathcurveto{\pgfqpoint{4.073829in}{2.407570in}}{\pgfqpoint{4.063230in}{2.411960in}}{\pgfqpoint{4.052180in}{2.411960in}}%
\pgfpathcurveto{\pgfqpoint{4.041129in}{2.411960in}}{\pgfqpoint{4.030530in}{2.407570in}}{\pgfqpoint{4.022717in}{2.399756in}}%
\pgfpathcurveto{\pgfqpoint{4.014903in}{2.391943in}}{\pgfqpoint{4.010513in}{2.381344in}}{\pgfqpoint{4.010513in}{2.370294in}}%
\pgfpathcurveto{\pgfqpoint{4.010513in}{2.359243in}}{\pgfqpoint{4.014903in}{2.348644in}}{\pgfqpoint{4.022717in}{2.340831in}}%
\pgfpathcurveto{\pgfqpoint{4.030530in}{2.333017in}}{\pgfqpoint{4.041129in}{2.328627in}}{\pgfqpoint{4.052180in}{2.328627in}}%
\pgfpathclose%
\pgfusepath{stroke,fill}%
\end{pgfscope}%
\begin{pgfscope}%
\pgfpathrectangle{\pgfqpoint{0.481978in}{0.331635in}}{\pgfqpoint{9.300000in}{7.700000in}}%
\pgfusepath{clip}%
\pgfsetbuttcap%
\pgfsetroundjoin%
\definecolor{currentfill}{rgb}{1.000000,0.705882,0.509804}%
\pgfsetfillcolor{currentfill}%
\pgfsetlinewidth{0.481800pt}%
\definecolor{currentstroke}{rgb}{1.000000,1.000000,1.000000}%
\pgfsetstrokecolor{currentstroke}%
\pgfsetdash{}{0pt}%
\pgfpathmoveto{\pgfqpoint{6.020800in}{4.975388in}}%
\pgfpathcurveto{\pgfqpoint{6.031850in}{4.975388in}}{\pgfqpoint{6.042449in}{4.979778in}}{\pgfqpoint{6.050263in}{4.987592in}}%
\pgfpathcurveto{\pgfqpoint{6.058077in}{4.995405in}}{\pgfqpoint{6.062467in}{5.006004in}}{\pgfqpoint{6.062467in}{5.017054in}}%
\pgfpathcurveto{\pgfqpoint{6.062467in}{5.028105in}}{\pgfqpoint{6.058077in}{5.038704in}}{\pgfqpoint{6.050263in}{5.046517in}}%
\pgfpathcurveto{\pgfqpoint{6.042449in}{5.054331in}}{\pgfqpoint{6.031850in}{5.058721in}}{\pgfqpoint{6.020800in}{5.058721in}}%
\pgfpathcurveto{\pgfqpoint{6.009750in}{5.058721in}}{\pgfqpoint{5.999151in}{5.054331in}}{\pgfqpoint{5.991338in}{5.046517in}}%
\pgfpathcurveto{\pgfqpoint{5.983524in}{5.038704in}}{\pgfqpoint{5.979134in}{5.028105in}}{\pgfqpoint{5.979134in}{5.017054in}}%
\pgfpathcurveto{\pgfqpoint{5.979134in}{5.006004in}}{\pgfqpoint{5.983524in}{4.995405in}}{\pgfqpoint{5.991338in}{4.987592in}}%
\pgfpathcurveto{\pgfqpoint{5.999151in}{4.979778in}}{\pgfqpoint{6.009750in}{4.975388in}}{\pgfqpoint{6.020800in}{4.975388in}}%
\pgfpathclose%
\pgfusepath{stroke,fill}%
\end{pgfscope}%
\begin{pgfscope}%
\pgfpathrectangle{\pgfqpoint{0.481978in}{0.331635in}}{\pgfqpoint{9.300000in}{7.700000in}}%
\pgfusepath{clip}%
\pgfsetbuttcap%
\pgfsetroundjoin%
\definecolor{currentfill}{rgb}{1.000000,0.705882,0.509804}%
\pgfsetfillcolor{currentfill}%
\pgfsetlinewidth{0.481800pt}%
\definecolor{currentstroke}{rgb}{1.000000,1.000000,1.000000}%
\pgfsetstrokecolor{currentstroke}%
\pgfsetdash{}{0pt}%
\pgfpathmoveto{\pgfqpoint{7.511979in}{7.121641in}}%
\pgfpathcurveto{\pgfqpoint{7.523029in}{7.121641in}}{\pgfqpoint{7.533628in}{7.126031in}}{\pgfqpoint{7.541441in}{7.133844in}}%
\pgfpathcurveto{\pgfqpoint{7.549255in}{7.141658in}}{\pgfqpoint{7.553645in}{7.152257in}}{\pgfqpoint{7.553645in}{7.163307in}}%
\pgfpathcurveto{\pgfqpoint{7.553645in}{7.174357in}}{\pgfqpoint{7.549255in}{7.184956in}}{\pgfqpoint{7.541441in}{7.192770in}}%
\pgfpathcurveto{\pgfqpoint{7.533628in}{7.200584in}}{\pgfqpoint{7.523029in}{7.204974in}}{\pgfqpoint{7.511979in}{7.204974in}}%
\pgfpathcurveto{\pgfqpoint{7.500929in}{7.204974in}}{\pgfqpoint{7.490330in}{7.200584in}}{\pgfqpoint{7.482516in}{7.192770in}}%
\pgfpathcurveto{\pgfqpoint{7.474702in}{7.184956in}}{\pgfqpoint{7.470312in}{7.174357in}}{\pgfqpoint{7.470312in}{7.163307in}}%
\pgfpathcurveto{\pgfqpoint{7.470312in}{7.152257in}}{\pgfqpoint{7.474702in}{7.141658in}}{\pgfqpoint{7.482516in}{7.133844in}}%
\pgfpathcurveto{\pgfqpoint{7.490330in}{7.126031in}}{\pgfqpoint{7.500929in}{7.121641in}}{\pgfqpoint{7.511979in}{7.121641in}}%
\pgfpathclose%
\pgfusepath{stroke,fill}%
\end{pgfscope}%
\begin{pgfscope}%
\pgfpathrectangle{\pgfqpoint{0.481978in}{0.331635in}}{\pgfqpoint{9.300000in}{7.700000in}}%
\pgfusepath{clip}%
\pgfsetbuttcap%
\pgfsetroundjoin%
\definecolor{currentfill}{rgb}{1.000000,0.705882,0.509804}%
\pgfsetfillcolor{currentfill}%
\pgfsetlinewidth{0.481800pt}%
\definecolor{currentstroke}{rgb}{1.000000,1.000000,1.000000}%
\pgfsetstrokecolor{currentstroke}%
\pgfsetdash{}{0pt}%
\pgfpathmoveto{\pgfqpoint{6.803906in}{4.819840in}}%
\pgfpathcurveto{\pgfqpoint{6.814956in}{4.819840in}}{\pgfqpoint{6.825555in}{4.824230in}}{\pgfqpoint{6.833369in}{4.832044in}}%
\pgfpathcurveto{\pgfqpoint{6.841182in}{4.839857in}}{\pgfqpoint{6.845573in}{4.850456in}}{\pgfqpoint{6.845573in}{4.861506in}}%
\pgfpathcurveto{\pgfqpoint{6.845573in}{4.872557in}}{\pgfqpoint{6.841182in}{4.883156in}}{\pgfqpoint{6.833369in}{4.890969in}}%
\pgfpathcurveto{\pgfqpoint{6.825555in}{4.898783in}}{\pgfqpoint{6.814956in}{4.903173in}}{\pgfqpoint{6.803906in}{4.903173in}}%
\pgfpathcurveto{\pgfqpoint{6.792856in}{4.903173in}}{\pgfqpoint{6.782257in}{4.898783in}}{\pgfqpoint{6.774443in}{4.890969in}}%
\pgfpathcurveto{\pgfqpoint{6.766629in}{4.883156in}}{\pgfqpoint{6.762239in}{4.872557in}}{\pgfqpoint{6.762239in}{4.861506in}}%
\pgfpathcurveto{\pgfqpoint{6.762239in}{4.850456in}}{\pgfqpoint{6.766629in}{4.839857in}}{\pgfqpoint{6.774443in}{4.832044in}}%
\pgfpathcurveto{\pgfqpoint{6.782257in}{4.824230in}}{\pgfqpoint{6.792856in}{4.819840in}}{\pgfqpoint{6.803906in}{4.819840in}}%
\pgfpathclose%
\pgfusepath{stroke,fill}%
\end{pgfscope}%
\begin{pgfscope}%
\pgfpathrectangle{\pgfqpoint{0.481978in}{0.331635in}}{\pgfqpoint{9.300000in}{7.700000in}}%
\pgfusepath{clip}%
\pgfsetbuttcap%
\pgfsetroundjoin%
\definecolor{currentfill}{rgb}{1.000000,0.705882,0.509804}%
\pgfsetfillcolor{currentfill}%
\pgfsetlinewidth{0.481800pt}%
\definecolor{currentstroke}{rgb}{1.000000,1.000000,1.000000}%
\pgfsetstrokecolor{currentstroke}%
\pgfsetdash{}{0pt}%
\pgfpathmoveto{\pgfqpoint{8.639111in}{5.424385in}}%
\pgfpathcurveto{\pgfqpoint{8.650161in}{5.424385in}}{\pgfqpoint{8.660760in}{5.428775in}}{\pgfqpoint{8.668574in}{5.436589in}}%
\pgfpathcurveto{\pgfqpoint{8.676388in}{5.444403in}}{\pgfqpoint{8.680778in}{5.455002in}}{\pgfqpoint{8.680778in}{5.466052in}}%
\pgfpathcurveto{\pgfqpoint{8.680778in}{5.477102in}}{\pgfqpoint{8.676388in}{5.487701in}}{\pgfqpoint{8.668574in}{5.495515in}}%
\pgfpathcurveto{\pgfqpoint{8.660760in}{5.503328in}}{\pgfqpoint{8.650161in}{5.507718in}}{\pgfqpoint{8.639111in}{5.507718in}}%
\pgfpathcurveto{\pgfqpoint{8.628061in}{5.507718in}}{\pgfqpoint{8.617462in}{5.503328in}}{\pgfqpoint{8.609648in}{5.495515in}}%
\pgfpathcurveto{\pgfqpoint{8.601835in}{5.487701in}}{\pgfqpoint{8.597444in}{5.477102in}}{\pgfqpoint{8.597444in}{5.466052in}}%
\pgfpathcurveto{\pgfqpoint{8.597444in}{5.455002in}}{\pgfqpoint{8.601835in}{5.444403in}}{\pgfqpoint{8.609648in}{5.436589in}}%
\pgfpathcurveto{\pgfqpoint{8.617462in}{5.428775in}}{\pgfqpoint{8.628061in}{5.424385in}}{\pgfqpoint{8.639111in}{5.424385in}}%
\pgfpathclose%
\pgfusepath{stroke,fill}%
\end{pgfscope}%
\begin{pgfscope}%
\pgfpathrectangle{\pgfqpoint{0.481978in}{0.331635in}}{\pgfqpoint{9.300000in}{7.700000in}}%
\pgfusepath{clip}%
\pgfsetbuttcap%
\pgfsetroundjoin%
\definecolor{currentfill}{rgb}{1.000000,0.705882,0.509804}%
\pgfsetfillcolor{currentfill}%
\pgfsetlinewidth{0.481800pt}%
\definecolor{currentstroke}{rgb}{1.000000,1.000000,1.000000}%
\pgfsetstrokecolor{currentstroke}%
\pgfsetdash{}{0pt}%
\pgfpathmoveto{\pgfqpoint{7.878998in}{5.782490in}}%
\pgfpathcurveto{\pgfqpoint{7.890048in}{5.782490in}}{\pgfqpoint{7.900647in}{5.786880in}}{\pgfqpoint{7.908461in}{5.794694in}}%
\pgfpathcurveto{\pgfqpoint{7.916274in}{5.802508in}}{\pgfqpoint{7.920664in}{5.813107in}}{\pgfqpoint{7.920664in}{5.824157in}}%
\pgfpathcurveto{\pgfqpoint{7.920664in}{5.835207in}}{\pgfqpoint{7.916274in}{5.845806in}}{\pgfqpoint{7.908461in}{5.853620in}}%
\pgfpathcurveto{\pgfqpoint{7.900647in}{5.861433in}}{\pgfqpoint{7.890048in}{5.865824in}}{\pgfqpoint{7.878998in}{5.865824in}}%
\pgfpathcurveto{\pgfqpoint{7.867948in}{5.865824in}}{\pgfqpoint{7.857349in}{5.861433in}}{\pgfqpoint{7.849535in}{5.853620in}}%
\pgfpathcurveto{\pgfqpoint{7.841721in}{5.845806in}}{\pgfqpoint{7.837331in}{5.835207in}}{\pgfqpoint{7.837331in}{5.824157in}}%
\pgfpathcurveto{\pgfqpoint{7.837331in}{5.813107in}}{\pgfqpoint{7.841721in}{5.802508in}}{\pgfqpoint{7.849535in}{5.794694in}}%
\pgfpathcurveto{\pgfqpoint{7.857349in}{5.786880in}}{\pgfqpoint{7.867948in}{5.782490in}}{\pgfqpoint{7.878998in}{5.782490in}}%
\pgfpathclose%
\pgfusepath{stroke,fill}%
\end{pgfscope}%
\begin{pgfscope}%
\pgfpathrectangle{\pgfqpoint{0.481978in}{0.331635in}}{\pgfqpoint{9.300000in}{7.700000in}}%
\pgfusepath{clip}%
\pgfsetbuttcap%
\pgfsetroundjoin%
\definecolor{currentfill}{rgb}{1.000000,0.705882,0.509804}%
\pgfsetfillcolor{currentfill}%
\pgfsetlinewidth{0.481800pt}%
\definecolor{currentstroke}{rgb}{1.000000,1.000000,1.000000}%
\pgfsetstrokecolor{currentstroke}%
\pgfsetdash{}{0pt}%
\pgfpathmoveto{\pgfqpoint{4.344121in}{0.639968in}}%
\pgfpathcurveto{\pgfqpoint{4.355171in}{0.639968in}}{\pgfqpoint{4.365770in}{0.644359in}}{\pgfqpoint{4.373583in}{0.652172in}}%
\pgfpathcurveto{\pgfqpoint{4.381397in}{0.659986in}}{\pgfqpoint{4.385787in}{0.670585in}}{\pgfqpoint{4.385787in}{0.681635in}}%
\pgfpathcurveto{\pgfqpoint{4.385787in}{0.692685in}}{\pgfqpoint{4.381397in}{0.703284in}}{\pgfqpoint{4.373583in}{0.711098in}}%
\pgfpathcurveto{\pgfqpoint{4.365770in}{0.718911in}}{\pgfqpoint{4.355171in}{0.723302in}}{\pgfqpoint{4.344121in}{0.723302in}}%
\pgfpathcurveto{\pgfqpoint{4.333070in}{0.723302in}}{\pgfqpoint{4.322471in}{0.718911in}}{\pgfqpoint{4.314658in}{0.711098in}}%
\pgfpathcurveto{\pgfqpoint{4.306844in}{0.703284in}}{\pgfqpoint{4.302454in}{0.692685in}}{\pgfqpoint{4.302454in}{0.681635in}}%
\pgfpathcurveto{\pgfqpoint{4.302454in}{0.670585in}}{\pgfqpoint{4.306844in}{0.659986in}}{\pgfqpoint{4.314658in}{0.652172in}}%
\pgfpathcurveto{\pgfqpoint{4.322471in}{0.644359in}}{\pgfqpoint{4.333070in}{0.639968in}}{\pgfqpoint{4.344121in}{0.639968in}}%
\pgfpathclose%
\pgfusepath{stroke,fill}%
\end{pgfscope}%
\begin{pgfscope}%
\pgfpathrectangle{\pgfqpoint{0.481978in}{0.331635in}}{\pgfqpoint{9.300000in}{7.700000in}}%
\pgfusepath{clip}%
\pgfsetbuttcap%
\pgfsetroundjoin%
\definecolor{currentfill}{rgb}{1.000000,0.705882,0.509804}%
\pgfsetfillcolor{currentfill}%
\pgfsetlinewidth{0.481800pt}%
\definecolor{currentstroke}{rgb}{1.000000,1.000000,1.000000}%
\pgfsetstrokecolor{currentstroke}%
\pgfsetdash{}{0pt}%
\pgfpathmoveto{\pgfqpoint{6.745741in}{2.549999in}}%
\pgfpathcurveto{\pgfqpoint{6.756792in}{2.549999in}}{\pgfqpoint{6.767391in}{2.554389in}}{\pgfqpoint{6.775204in}{2.562203in}}%
\pgfpathcurveto{\pgfqpoint{6.783018in}{2.570017in}}{\pgfqpoint{6.787408in}{2.580616in}}{\pgfqpoint{6.787408in}{2.591666in}}%
\pgfpathcurveto{\pgfqpoint{6.787408in}{2.602716in}}{\pgfqpoint{6.783018in}{2.613315in}}{\pgfqpoint{6.775204in}{2.621129in}}%
\pgfpathcurveto{\pgfqpoint{6.767391in}{2.628942in}}{\pgfqpoint{6.756792in}{2.633333in}}{\pgfqpoint{6.745741in}{2.633333in}}%
\pgfpathcurveto{\pgfqpoint{6.734691in}{2.633333in}}{\pgfqpoint{6.724092in}{2.628942in}}{\pgfqpoint{6.716279in}{2.621129in}}%
\pgfpathcurveto{\pgfqpoint{6.708465in}{2.613315in}}{\pgfqpoint{6.704075in}{2.602716in}}{\pgfqpoint{6.704075in}{2.591666in}}%
\pgfpathcurveto{\pgfqpoint{6.704075in}{2.580616in}}{\pgfqpoint{6.708465in}{2.570017in}}{\pgfqpoint{6.716279in}{2.562203in}}%
\pgfpathcurveto{\pgfqpoint{6.724092in}{2.554389in}}{\pgfqpoint{6.734691in}{2.549999in}}{\pgfqpoint{6.745741in}{2.549999in}}%
\pgfpathclose%
\pgfusepath{stroke,fill}%
\end{pgfscope}%
\begin{pgfscope}%
\pgfpathrectangle{\pgfqpoint{0.481978in}{0.331635in}}{\pgfqpoint{9.300000in}{7.700000in}}%
\pgfusepath{clip}%
\pgfsetbuttcap%
\pgfsetroundjoin%
\definecolor{currentfill}{rgb}{1.000000,0.705882,0.509804}%
\pgfsetfillcolor{currentfill}%
\pgfsetlinewidth{0.481800pt}%
\definecolor{currentstroke}{rgb}{1.000000,1.000000,1.000000}%
\pgfsetstrokecolor{currentstroke}%
\pgfsetdash{}{0pt}%
\pgfpathmoveto{\pgfqpoint{8.003363in}{4.771723in}}%
\pgfpathcurveto{\pgfqpoint{8.014413in}{4.771723in}}{\pgfqpoint{8.025012in}{4.776113in}}{\pgfqpoint{8.032825in}{4.783926in}}%
\pgfpathcurveto{\pgfqpoint{8.040639in}{4.791740in}}{\pgfqpoint{8.045029in}{4.802339in}}{\pgfqpoint{8.045029in}{4.813389in}}%
\pgfpathcurveto{\pgfqpoint{8.045029in}{4.824439in}}{\pgfqpoint{8.040639in}{4.835038in}}{\pgfqpoint{8.032825in}{4.842852in}}%
\pgfpathcurveto{\pgfqpoint{8.025012in}{4.850666in}}{\pgfqpoint{8.014413in}{4.855056in}}{\pgfqpoint{8.003363in}{4.855056in}}%
\pgfpathcurveto{\pgfqpoint{7.992312in}{4.855056in}}{\pgfqpoint{7.981713in}{4.850666in}}{\pgfqpoint{7.973900in}{4.842852in}}%
\pgfpathcurveto{\pgfqpoint{7.966086in}{4.835038in}}{\pgfqpoint{7.961696in}{4.824439in}}{\pgfqpoint{7.961696in}{4.813389in}}%
\pgfpathcurveto{\pgfqpoint{7.961696in}{4.802339in}}{\pgfqpoint{7.966086in}{4.791740in}}{\pgfqpoint{7.973900in}{4.783926in}}%
\pgfpathcurveto{\pgfqpoint{7.981713in}{4.776113in}}{\pgfqpoint{7.992312in}{4.771723in}}{\pgfqpoint{8.003363in}{4.771723in}}%
\pgfpathclose%
\pgfusepath{stroke,fill}%
\end{pgfscope}%
\begin{pgfscope}%
\pgfpathrectangle{\pgfqpoint{0.481978in}{0.331635in}}{\pgfqpoint{9.300000in}{7.700000in}}%
\pgfusepath{clip}%
\pgfsetbuttcap%
\pgfsetroundjoin%
\definecolor{currentfill}{rgb}{1.000000,0.705882,0.509804}%
\pgfsetfillcolor{currentfill}%
\pgfsetlinewidth{0.481800pt}%
\definecolor{currentstroke}{rgb}{1.000000,1.000000,1.000000}%
\pgfsetstrokecolor{currentstroke}%
\pgfsetdash{}{0pt}%
\pgfpathmoveto{\pgfqpoint{8.570858in}{4.574385in}}%
\pgfpathcurveto{\pgfqpoint{8.581908in}{4.574385in}}{\pgfqpoint{8.592507in}{4.578775in}}{\pgfqpoint{8.600321in}{4.586588in}}%
\pgfpathcurveto{\pgfqpoint{8.608134in}{4.594402in}}{\pgfqpoint{8.612524in}{4.605001in}}{\pgfqpoint{8.612524in}{4.616051in}}%
\pgfpathcurveto{\pgfqpoint{8.612524in}{4.627101in}}{\pgfqpoint{8.608134in}{4.637700in}}{\pgfqpoint{8.600321in}{4.645514in}}%
\pgfpathcurveto{\pgfqpoint{8.592507in}{4.653328in}}{\pgfqpoint{8.581908in}{4.657718in}}{\pgfqpoint{8.570858in}{4.657718in}}%
\pgfpathcurveto{\pgfqpoint{8.559808in}{4.657718in}}{\pgfqpoint{8.549209in}{4.653328in}}{\pgfqpoint{8.541395in}{4.645514in}}%
\pgfpathcurveto{\pgfqpoint{8.533581in}{4.637700in}}{\pgfqpoint{8.529191in}{4.627101in}}{\pgfqpoint{8.529191in}{4.616051in}}%
\pgfpathcurveto{\pgfqpoint{8.529191in}{4.605001in}}{\pgfqpoint{8.533581in}{4.594402in}}{\pgfqpoint{8.541395in}{4.586588in}}%
\pgfpathcurveto{\pgfqpoint{8.549209in}{4.578775in}}{\pgfqpoint{8.559808in}{4.574385in}}{\pgfqpoint{8.570858in}{4.574385in}}%
\pgfpathclose%
\pgfusepath{stroke,fill}%
\end{pgfscope}%
\begin{pgfscope}%
\pgfpathrectangle{\pgfqpoint{0.481978in}{0.331635in}}{\pgfqpoint{9.300000in}{7.700000in}}%
\pgfusepath{clip}%
\pgfsetbuttcap%
\pgfsetroundjoin%
\definecolor{currentfill}{rgb}{1.000000,0.705882,0.509804}%
\pgfsetfillcolor{currentfill}%
\pgfsetlinewidth{0.481800pt}%
\definecolor{currentstroke}{rgb}{1.000000,1.000000,1.000000}%
\pgfsetstrokecolor{currentstroke}%
\pgfsetdash{}{0pt}%
\pgfpathmoveto{\pgfqpoint{7.484442in}{4.976959in}}%
\pgfpathcurveto{\pgfqpoint{7.495492in}{4.976959in}}{\pgfqpoint{7.506091in}{4.981349in}}{\pgfqpoint{7.513904in}{4.989163in}}%
\pgfpathcurveto{\pgfqpoint{7.521718in}{4.996976in}}{\pgfqpoint{7.526108in}{5.007575in}}{\pgfqpoint{7.526108in}{5.018626in}}%
\pgfpathcurveto{\pgfqpoint{7.526108in}{5.029676in}}{\pgfqpoint{7.521718in}{5.040275in}}{\pgfqpoint{7.513904in}{5.048088in}}%
\pgfpathcurveto{\pgfqpoint{7.506091in}{5.055902in}}{\pgfqpoint{7.495492in}{5.060292in}}{\pgfqpoint{7.484442in}{5.060292in}}%
\pgfpathcurveto{\pgfqpoint{7.473391in}{5.060292in}}{\pgfqpoint{7.462792in}{5.055902in}}{\pgfqpoint{7.454979in}{5.048088in}}%
\pgfpathcurveto{\pgfqpoint{7.447165in}{5.040275in}}{\pgfqpoint{7.442775in}{5.029676in}}{\pgfqpoint{7.442775in}{5.018626in}}%
\pgfpathcurveto{\pgfqpoint{7.442775in}{5.007575in}}{\pgfqpoint{7.447165in}{4.996976in}}{\pgfqpoint{7.454979in}{4.989163in}}%
\pgfpathcurveto{\pgfqpoint{7.462792in}{4.981349in}}{\pgfqpoint{7.473391in}{4.976959in}}{\pgfqpoint{7.484442in}{4.976959in}}%
\pgfpathclose%
\pgfusepath{stroke,fill}%
\end{pgfscope}%
\begin{pgfscope}%
\pgfpathrectangle{\pgfqpoint{0.481978in}{0.331635in}}{\pgfqpoint{9.300000in}{7.700000in}}%
\pgfusepath{clip}%
\pgfsetbuttcap%
\pgfsetroundjoin%
\definecolor{currentfill}{rgb}{1.000000,0.705882,0.509804}%
\pgfsetfillcolor{currentfill}%
\pgfsetlinewidth{0.481800pt}%
\definecolor{currentstroke}{rgb}{1.000000,1.000000,1.000000}%
\pgfsetstrokecolor{currentstroke}%
\pgfsetdash{}{0pt}%
\pgfpathmoveto{\pgfqpoint{3.493710in}{0.989024in}}%
\pgfpathcurveto{\pgfqpoint{3.504760in}{0.989024in}}{\pgfqpoint{3.515360in}{0.993414in}}{\pgfqpoint{3.523173in}{1.001228in}}%
\pgfpathcurveto{\pgfqpoint{3.530987in}{1.009042in}}{\pgfqpoint{3.535377in}{1.019641in}}{\pgfqpoint{3.535377in}{1.030691in}}%
\pgfpathcurveto{\pgfqpoint{3.535377in}{1.041741in}}{\pgfqpoint{3.530987in}{1.052340in}}{\pgfqpoint{3.523173in}{1.060154in}}%
\pgfpathcurveto{\pgfqpoint{3.515360in}{1.067967in}}{\pgfqpoint{3.504760in}{1.072357in}}{\pgfqpoint{3.493710in}{1.072357in}}%
\pgfpathcurveto{\pgfqpoint{3.482660in}{1.072357in}}{\pgfqpoint{3.472061in}{1.067967in}}{\pgfqpoint{3.464248in}{1.060154in}}%
\pgfpathcurveto{\pgfqpoint{3.456434in}{1.052340in}}{\pgfqpoint{3.452044in}{1.041741in}}{\pgfqpoint{3.452044in}{1.030691in}}%
\pgfpathcurveto{\pgfqpoint{3.452044in}{1.019641in}}{\pgfqpoint{3.456434in}{1.009042in}}{\pgfqpoint{3.464248in}{1.001228in}}%
\pgfpathcurveto{\pgfqpoint{3.472061in}{0.993414in}}{\pgfqpoint{3.482660in}{0.989024in}}{\pgfqpoint{3.493710in}{0.989024in}}%
\pgfpathclose%
\pgfusepath{stroke,fill}%
\end{pgfscope}%
\begin{pgfscope}%
\pgfpathrectangle{\pgfqpoint{0.481978in}{0.331635in}}{\pgfqpoint{9.300000in}{7.700000in}}%
\pgfusepath{clip}%
\pgfsetbuttcap%
\pgfsetroundjoin%
\definecolor{currentfill}{rgb}{1.000000,0.705882,0.509804}%
\pgfsetfillcolor{currentfill}%
\pgfsetlinewidth{0.481800pt}%
\definecolor{currentstroke}{rgb}{1.000000,1.000000,1.000000}%
\pgfsetstrokecolor{currentstroke}%
\pgfsetdash{}{0pt}%
\pgfpathmoveto{\pgfqpoint{5.177601in}{3.179990in}}%
\pgfpathcurveto{\pgfqpoint{5.188651in}{3.179990in}}{\pgfqpoint{5.199250in}{3.184380in}}{\pgfqpoint{5.207064in}{3.192194in}}%
\pgfpathcurveto{\pgfqpoint{5.214877in}{3.200007in}}{\pgfqpoint{5.219267in}{3.210607in}}{\pgfqpoint{5.219267in}{3.221657in}}%
\pgfpathcurveto{\pgfqpoint{5.219267in}{3.232707in}}{\pgfqpoint{5.214877in}{3.243306in}}{\pgfqpoint{5.207064in}{3.251119in}}%
\pgfpathcurveto{\pgfqpoint{5.199250in}{3.258933in}}{\pgfqpoint{5.188651in}{3.263323in}}{\pgfqpoint{5.177601in}{3.263323in}}%
\pgfpathcurveto{\pgfqpoint{5.166551in}{3.263323in}}{\pgfqpoint{5.155952in}{3.258933in}}{\pgfqpoint{5.148138in}{3.251119in}}%
\pgfpathcurveto{\pgfqpoint{5.140324in}{3.243306in}}{\pgfqpoint{5.135934in}{3.232707in}}{\pgfqpoint{5.135934in}{3.221657in}}%
\pgfpathcurveto{\pgfqpoint{5.135934in}{3.210607in}}{\pgfqpoint{5.140324in}{3.200007in}}{\pgfqpoint{5.148138in}{3.192194in}}%
\pgfpathcurveto{\pgfqpoint{5.155952in}{3.184380in}}{\pgfqpoint{5.166551in}{3.179990in}}{\pgfqpoint{5.177601in}{3.179990in}}%
\pgfpathclose%
\pgfusepath{stroke,fill}%
\end{pgfscope}%
\begin{pgfscope}%
\pgfpathrectangle{\pgfqpoint{0.481978in}{0.331635in}}{\pgfqpoint{9.300000in}{7.700000in}}%
\pgfusepath{clip}%
\pgfsetbuttcap%
\pgfsetroundjoin%
\definecolor{currentfill}{rgb}{1.000000,0.705882,0.509804}%
\pgfsetfillcolor{currentfill}%
\pgfsetlinewidth{0.481800pt}%
\definecolor{currentstroke}{rgb}{1.000000,1.000000,1.000000}%
\pgfsetstrokecolor{currentstroke}%
\pgfsetdash{}{0pt}%
\pgfpathmoveto{\pgfqpoint{4.181982in}{2.915533in}}%
\pgfpathcurveto{\pgfqpoint{4.193033in}{2.915533in}}{\pgfqpoint{4.203632in}{2.919923in}}{\pgfqpoint{4.211445in}{2.927737in}}%
\pgfpathcurveto{\pgfqpoint{4.219259in}{2.935550in}}{\pgfqpoint{4.223649in}{2.946149in}}{\pgfqpoint{4.223649in}{2.957199in}}%
\pgfpathcurveto{\pgfqpoint{4.223649in}{2.968250in}}{\pgfqpoint{4.219259in}{2.978849in}}{\pgfqpoint{4.211445in}{2.986662in}}%
\pgfpathcurveto{\pgfqpoint{4.203632in}{2.994476in}}{\pgfqpoint{4.193033in}{2.998866in}}{\pgfqpoint{4.181982in}{2.998866in}}%
\pgfpathcurveto{\pgfqpoint{4.170932in}{2.998866in}}{\pgfqpoint{4.160333in}{2.994476in}}{\pgfqpoint{4.152520in}{2.986662in}}%
\pgfpathcurveto{\pgfqpoint{4.144706in}{2.978849in}}{\pgfqpoint{4.140316in}{2.968250in}}{\pgfqpoint{4.140316in}{2.957199in}}%
\pgfpathcurveto{\pgfqpoint{4.140316in}{2.946149in}}{\pgfqpoint{4.144706in}{2.935550in}}{\pgfqpoint{4.152520in}{2.927737in}}%
\pgfpathcurveto{\pgfqpoint{4.160333in}{2.919923in}}{\pgfqpoint{4.170932in}{2.915533in}}{\pgfqpoint{4.181982in}{2.915533in}}%
\pgfpathclose%
\pgfusepath{stroke,fill}%
\end{pgfscope}%
\begin{pgfscope}%
\pgfpathrectangle{\pgfqpoint{0.481978in}{0.331635in}}{\pgfqpoint{9.300000in}{7.700000in}}%
\pgfusepath{clip}%
\pgfsetbuttcap%
\pgfsetroundjoin%
\definecolor{currentfill}{rgb}{1.000000,0.705882,0.509804}%
\pgfsetfillcolor{currentfill}%
\pgfsetlinewidth{0.481800pt}%
\definecolor{currentstroke}{rgb}{1.000000,1.000000,1.000000}%
\pgfsetstrokecolor{currentstroke}%
\pgfsetdash{}{0pt}%
\pgfpathmoveto{\pgfqpoint{9.127562in}{5.199542in}}%
\pgfpathcurveto{\pgfqpoint{9.138612in}{5.199542in}}{\pgfqpoint{9.149211in}{5.203932in}}{\pgfqpoint{9.157025in}{5.211746in}}%
\pgfpathcurveto{\pgfqpoint{9.164838in}{5.219560in}}{\pgfqpoint{9.169228in}{5.230159in}}{\pgfqpoint{9.169228in}{5.241209in}}%
\pgfpathcurveto{\pgfqpoint{9.169228in}{5.252259in}}{\pgfqpoint{9.164838in}{5.262858in}}{\pgfqpoint{9.157025in}{5.270671in}}%
\pgfpathcurveto{\pgfqpoint{9.149211in}{5.278485in}}{\pgfqpoint{9.138612in}{5.282875in}}{\pgfqpoint{9.127562in}{5.282875in}}%
\pgfpathcurveto{\pgfqpoint{9.116512in}{5.282875in}}{\pgfqpoint{9.105913in}{5.278485in}}{\pgfqpoint{9.098099in}{5.270671in}}%
\pgfpathcurveto{\pgfqpoint{9.090285in}{5.262858in}}{\pgfqpoint{9.085895in}{5.252259in}}{\pgfqpoint{9.085895in}{5.241209in}}%
\pgfpathcurveto{\pgfqpoint{9.085895in}{5.230159in}}{\pgfqpoint{9.090285in}{5.219560in}}{\pgfqpoint{9.098099in}{5.211746in}}%
\pgfpathcurveto{\pgfqpoint{9.105913in}{5.203932in}}{\pgfqpoint{9.116512in}{5.199542in}}{\pgfqpoint{9.127562in}{5.199542in}}%
\pgfpathclose%
\pgfusepath{stroke,fill}%
\end{pgfscope}%
\begin{pgfscope}%
\pgfpathrectangle{\pgfqpoint{0.481978in}{0.331635in}}{\pgfqpoint{9.300000in}{7.700000in}}%
\pgfusepath{clip}%
\pgfsetbuttcap%
\pgfsetroundjoin%
\definecolor{currentfill}{rgb}{1.000000,0.705882,0.509804}%
\pgfsetfillcolor{currentfill}%
\pgfsetlinewidth{0.481800pt}%
\definecolor{currentstroke}{rgb}{1.000000,1.000000,1.000000}%
\pgfsetstrokecolor{currentstroke}%
\pgfsetdash{}{0pt}%
\pgfpathmoveto{\pgfqpoint{3.346801in}{3.757238in}}%
\pgfpathcurveto{\pgfqpoint{3.357851in}{3.757238in}}{\pgfqpoint{3.368450in}{3.761628in}}{\pgfqpoint{3.376264in}{3.769442in}}%
\pgfpathcurveto{\pgfqpoint{3.384077in}{3.777256in}}{\pgfqpoint{3.388467in}{3.787855in}}{\pgfqpoint{3.388467in}{3.798905in}}%
\pgfpathcurveto{\pgfqpoint{3.388467in}{3.809955in}}{\pgfqpoint{3.384077in}{3.820554in}}{\pgfqpoint{3.376264in}{3.828368in}}%
\pgfpathcurveto{\pgfqpoint{3.368450in}{3.836181in}}{\pgfqpoint{3.357851in}{3.840572in}}{\pgfqpoint{3.346801in}{3.840572in}}%
\pgfpathcurveto{\pgfqpoint{3.335751in}{3.840572in}}{\pgfqpoint{3.325152in}{3.836181in}}{\pgfqpoint{3.317338in}{3.828368in}}%
\pgfpathcurveto{\pgfqpoint{3.309524in}{3.820554in}}{\pgfqpoint{3.305134in}{3.809955in}}{\pgfqpoint{3.305134in}{3.798905in}}%
\pgfpathcurveto{\pgfqpoint{3.305134in}{3.787855in}}{\pgfqpoint{3.309524in}{3.777256in}}{\pgfqpoint{3.317338in}{3.769442in}}%
\pgfpathcurveto{\pgfqpoint{3.325152in}{3.761628in}}{\pgfqpoint{3.335751in}{3.757238in}}{\pgfqpoint{3.346801in}{3.757238in}}%
\pgfpathclose%
\pgfusepath{stroke,fill}%
\end{pgfscope}%
\begin{pgfscope}%
\pgfpathrectangle{\pgfqpoint{0.481978in}{0.331635in}}{\pgfqpoint{9.300000in}{7.700000in}}%
\pgfusepath{clip}%
\pgfsetbuttcap%
\pgfsetroundjoin%
\definecolor{currentfill}{rgb}{1.000000,0.705882,0.509804}%
\pgfsetfillcolor{currentfill}%
\pgfsetlinewidth{0.481800pt}%
\definecolor{currentstroke}{rgb}{1.000000,1.000000,1.000000}%
\pgfsetstrokecolor{currentstroke}%
\pgfsetdash{}{0pt}%
\pgfpathmoveto{\pgfqpoint{2.661207in}{2.830051in}}%
\pgfpathcurveto{\pgfqpoint{2.672258in}{2.830051in}}{\pgfqpoint{2.682857in}{2.834441in}}{\pgfqpoint{2.690670in}{2.842254in}}%
\pgfpathcurveto{\pgfqpoint{2.698484in}{2.850068in}}{\pgfqpoint{2.702874in}{2.860667in}}{\pgfqpoint{2.702874in}{2.871717in}}%
\pgfpathcurveto{\pgfqpoint{2.702874in}{2.882767in}}{\pgfqpoint{2.698484in}{2.893366in}}{\pgfqpoint{2.690670in}{2.901180in}}%
\pgfpathcurveto{\pgfqpoint{2.682857in}{2.908994in}}{\pgfqpoint{2.672258in}{2.913384in}}{\pgfqpoint{2.661207in}{2.913384in}}%
\pgfpathcurveto{\pgfqpoint{2.650157in}{2.913384in}}{\pgfqpoint{2.639558in}{2.908994in}}{\pgfqpoint{2.631745in}{2.901180in}}%
\pgfpathcurveto{\pgfqpoint{2.623931in}{2.893366in}}{\pgfqpoint{2.619541in}{2.882767in}}{\pgfqpoint{2.619541in}{2.871717in}}%
\pgfpathcurveto{\pgfqpoint{2.619541in}{2.860667in}}{\pgfqpoint{2.623931in}{2.850068in}}{\pgfqpoint{2.631745in}{2.842254in}}%
\pgfpathcurveto{\pgfqpoint{2.639558in}{2.834441in}}{\pgfqpoint{2.650157in}{2.830051in}}{\pgfqpoint{2.661207in}{2.830051in}}%
\pgfpathclose%
\pgfusepath{stroke,fill}%
\end{pgfscope}%
\begin{pgfscope}%
\pgfpathrectangle{\pgfqpoint{0.481978in}{0.331635in}}{\pgfqpoint{9.300000in}{7.700000in}}%
\pgfusepath{clip}%
\pgfsetbuttcap%
\pgfsetroundjoin%
\definecolor{currentfill}{rgb}{1.000000,0.705882,0.509804}%
\pgfsetfillcolor{currentfill}%
\pgfsetlinewidth{0.481800pt}%
\definecolor{currentstroke}{rgb}{1.000000,1.000000,1.000000}%
\pgfsetstrokecolor{currentstroke}%
\pgfsetdash{}{0pt}%
\pgfpathmoveto{\pgfqpoint{6.147513in}{3.373345in}}%
\pgfpathcurveto{\pgfqpoint{6.158563in}{3.373345in}}{\pgfqpoint{6.169162in}{3.377735in}}{\pgfqpoint{6.176976in}{3.385549in}}%
\pgfpathcurveto{\pgfqpoint{6.184789in}{3.393362in}}{\pgfqpoint{6.189179in}{3.403961in}}{\pgfqpoint{6.189179in}{3.415011in}}%
\pgfpathcurveto{\pgfqpoint{6.189179in}{3.426062in}}{\pgfqpoint{6.184789in}{3.436661in}}{\pgfqpoint{6.176976in}{3.444474in}}%
\pgfpathcurveto{\pgfqpoint{6.169162in}{3.452288in}}{\pgfqpoint{6.158563in}{3.456678in}}{\pgfqpoint{6.147513in}{3.456678in}}%
\pgfpathcurveto{\pgfqpoint{6.136463in}{3.456678in}}{\pgfqpoint{6.125864in}{3.452288in}}{\pgfqpoint{6.118050in}{3.444474in}}%
\pgfpathcurveto{\pgfqpoint{6.110236in}{3.436661in}}{\pgfqpoint{6.105846in}{3.426062in}}{\pgfqpoint{6.105846in}{3.415011in}}%
\pgfpathcurveto{\pgfqpoint{6.105846in}{3.403961in}}{\pgfqpoint{6.110236in}{3.393362in}}{\pgfqpoint{6.118050in}{3.385549in}}%
\pgfpathcurveto{\pgfqpoint{6.125864in}{3.377735in}}{\pgfqpoint{6.136463in}{3.373345in}}{\pgfqpoint{6.147513in}{3.373345in}}%
\pgfpathclose%
\pgfusepath{stroke,fill}%
\end{pgfscope}%
\begin{pgfscope}%
\pgfpathrectangle{\pgfqpoint{0.481978in}{0.331635in}}{\pgfqpoint{9.300000in}{7.700000in}}%
\pgfusepath{clip}%
\pgfsetbuttcap%
\pgfsetroundjoin%
\definecolor{currentfill}{rgb}{1.000000,0.705882,0.509804}%
\pgfsetfillcolor{currentfill}%
\pgfsetlinewidth{0.481800pt}%
\definecolor{currentstroke}{rgb}{1.000000,1.000000,1.000000}%
\pgfsetstrokecolor{currentstroke}%
\pgfsetdash{}{0pt}%
\pgfpathmoveto{\pgfqpoint{8.123188in}{4.340553in}}%
\pgfpathcurveto{\pgfqpoint{8.134238in}{4.340553in}}{\pgfqpoint{8.144837in}{4.344943in}}{\pgfqpoint{8.152651in}{4.352757in}}%
\pgfpathcurveto{\pgfqpoint{8.160465in}{4.360571in}}{\pgfqpoint{8.164855in}{4.371170in}}{\pgfqpoint{8.164855in}{4.382220in}}%
\pgfpathcurveto{\pgfqpoint{8.164855in}{4.393270in}}{\pgfqpoint{8.160465in}{4.403869in}}{\pgfqpoint{8.152651in}{4.411683in}}%
\pgfpathcurveto{\pgfqpoint{8.144837in}{4.419496in}}{\pgfqpoint{8.134238in}{4.423886in}}{\pgfqpoint{8.123188in}{4.423886in}}%
\pgfpathcurveto{\pgfqpoint{8.112138in}{4.423886in}}{\pgfqpoint{8.101539in}{4.419496in}}{\pgfqpoint{8.093725in}{4.411683in}}%
\pgfpathcurveto{\pgfqpoint{8.085912in}{4.403869in}}{\pgfqpoint{8.081522in}{4.393270in}}{\pgfqpoint{8.081522in}{4.382220in}}%
\pgfpathcurveto{\pgfqpoint{8.081522in}{4.371170in}}{\pgfqpoint{8.085912in}{4.360571in}}{\pgfqpoint{8.093725in}{4.352757in}}%
\pgfpathcurveto{\pgfqpoint{8.101539in}{4.344943in}}{\pgfqpoint{8.112138in}{4.340553in}}{\pgfqpoint{8.123188in}{4.340553in}}%
\pgfpathclose%
\pgfusepath{stroke,fill}%
\end{pgfscope}%
\begin{pgfscope}%
\pgfpathrectangle{\pgfqpoint{0.481978in}{0.331635in}}{\pgfqpoint{9.300000in}{7.700000in}}%
\pgfusepath{clip}%
\pgfsetbuttcap%
\pgfsetroundjoin%
\definecolor{currentfill}{rgb}{1.000000,0.705882,0.509804}%
\pgfsetfillcolor{currentfill}%
\pgfsetlinewidth{0.481800pt}%
\definecolor{currentstroke}{rgb}{1.000000,1.000000,1.000000}%
\pgfsetstrokecolor{currentstroke}%
\pgfsetdash{}{0pt}%
\pgfpathmoveto{\pgfqpoint{4.044825in}{4.771544in}}%
\pgfpathcurveto{\pgfqpoint{4.055875in}{4.771544in}}{\pgfqpoint{4.066474in}{4.775934in}}{\pgfqpoint{4.074287in}{4.783748in}}%
\pgfpathcurveto{\pgfqpoint{4.082101in}{4.791561in}}{\pgfqpoint{4.086491in}{4.802160in}}{\pgfqpoint{4.086491in}{4.813210in}}%
\pgfpathcurveto{\pgfqpoint{4.086491in}{4.824261in}}{\pgfqpoint{4.082101in}{4.834860in}}{\pgfqpoint{4.074287in}{4.842673in}}%
\pgfpathcurveto{\pgfqpoint{4.066474in}{4.850487in}}{\pgfqpoint{4.055875in}{4.854877in}}{\pgfqpoint{4.044825in}{4.854877in}}%
\pgfpathcurveto{\pgfqpoint{4.033774in}{4.854877in}}{\pgfqpoint{4.023175in}{4.850487in}}{\pgfqpoint{4.015362in}{4.842673in}}%
\pgfpathcurveto{\pgfqpoint{4.007548in}{4.834860in}}{\pgfqpoint{4.003158in}{4.824261in}}{\pgfqpoint{4.003158in}{4.813210in}}%
\pgfpathcurveto{\pgfqpoint{4.003158in}{4.802160in}}{\pgfqpoint{4.007548in}{4.791561in}}{\pgfqpoint{4.015362in}{4.783748in}}%
\pgfpathcurveto{\pgfqpoint{4.023175in}{4.775934in}}{\pgfqpoint{4.033774in}{4.771544in}}{\pgfqpoint{4.044825in}{4.771544in}}%
\pgfpathclose%
\pgfusepath{stroke,fill}%
\end{pgfscope}%
\begin{pgfscope}%
\pgfpathrectangle{\pgfqpoint{0.481978in}{0.331635in}}{\pgfqpoint{9.300000in}{7.700000in}}%
\pgfusepath{clip}%
\pgfsetbuttcap%
\pgfsetroundjoin%
\definecolor{currentfill}{rgb}{1.000000,0.705882,0.509804}%
\pgfsetfillcolor{currentfill}%
\pgfsetlinewidth{0.481800pt}%
\definecolor{currentstroke}{rgb}{1.000000,1.000000,1.000000}%
\pgfsetstrokecolor{currentstroke}%
\pgfsetdash{}{0pt}%
\pgfpathmoveto{\pgfqpoint{3.633094in}{2.755684in}}%
\pgfpathcurveto{\pgfqpoint{3.644144in}{2.755684in}}{\pgfqpoint{3.654743in}{2.760074in}}{\pgfqpoint{3.662557in}{2.767888in}}%
\pgfpathcurveto{\pgfqpoint{3.670371in}{2.775701in}}{\pgfqpoint{3.674761in}{2.786300in}}{\pgfqpoint{3.674761in}{2.797350in}}%
\pgfpathcurveto{\pgfqpoint{3.674761in}{2.808401in}}{\pgfqpoint{3.670371in}{2.819000in}}{\pgfqpoint{3.662557in}{2.826813in}}%
\pgfpathcurveto{\pgfqpoint{3.654743in}{2.834627in}}{\pgfqpoint{3.644144in}{2.839017in}}{\pgfqpoint{3.633094in}{2.839017in}}%
\pgfpathcurveto{\pgfqpoint{3.622044in}{2.839017in}}{\pgfqpoint{3.611445in}{2.834627in}}{\pgfqpoint{3.603631in}{2.826813in}}%
\pgfpathcurveto{\pgfqpoint{3.595818in}{2.819000in}}{\pgfqpoint{3.591427in}{2.808401in}}{\pgfqpoint{3.591427in}{2.797350in}}%
\pgfpathcurveto{\pgfqpoint{3.591427in}{2.786300in}}{\pgfqpoint{3.595818in}{2.775701in}}{\pgfqpoint{3.603631in}{2.767888in}}%
\pgfpathcurveto{\pgfqpoint{3.611445in}{2.760074in}}{\pgfqpoint{3.622044in}{2.755684in}}{\pgfqpoint{3.633094in}{2.755684in}}%
\pgfpathclose%
\pgfusepath{stroke,fill}%
\end{pgfscope}%
\begin{pgfscope}%
\pgfpathrectangle{\pgfqpoint{0.481978in}{0.331635in}}{\pgfqpoint{9.300000in}{7.700000in}}%
\pgfusepath{clip}%
\pgfsetbuttcap%
\pgfsetroundjoin%
\definecolor{currentfill}{rgb}{1.000000,0.705882,0.509804}%
\pgfsetfillcolor{currentfill}%
\pgfsetlinewidth{0.481800pt}%
\definecolor{currentstroke}{rgb}{1.000000,1.000000,1.000000}%
\pgfsetstrokecolor{currentstroke}%
\pgfsetdash{}{0pt}%
\pgfpathmoveto{\pgfqpoint{4.595278in}{3.081442in}}%
\pgfpathcurveto{\pgfqpoint{4.606328in}{3.081442in}}{\pgfqpoint{4.616927in}{3.085832in}}{\pgfqpoint{4.624741in}{3.093646in}}%
\pgfpathcurveto{\pgfqpoint{4.632554in}{3.101459in}}{\pgfqpoint{4.636945in}{3.112058in}}{\pgfqpoint{4.636945in}{3.123108in}}%
\pgfpathcurveto{\pgfqpoint{4.636945in}{3.134158in}}{\pgfqpoint{4.632554in}{3.144758in}}{\pgfqpoint{4.624741in}{3.152571in}}%
\pgfpathcurveto{\pgfqpoint{4.616927in}{3.160385in}}{\pgfqpoint{4.606328in}{3.164775in}}{\pgfqpoint{4.595278in}{3.164775in}}%
\pgfpathcurveto{\pgfqpoint{4.584228in}{3.164775in}}{\pgfqpoint{4.573629in}{3.160385in}}{\pgfqpoint{4.565815in}{3.152571in}}%
\pgfpathcurveto{\pgfqpoint{4.558001in}{3.144758in}}{\pgfqpoint{4.553611in}{3.134158in}}{\pgfqpoint{4.553611in}{3.123108in}}%
\pgfpathcurveto{\pgfqpoint{4.553611in}{3.112058in}}{\pgfqpoint{4.558001in}{3.101459in}}{\pgfqpoint{4.565815in}{3.093646in}}%
\pgfpathcurveto{\pgfqpoint{4.573629in}{3.085832in}}{\pgfqpoint{4.584228in}{3.081442in}}{\pgfqpoint{4.595278in}{3.081442in}}%
\pgfpathclose%
\pgfusepath{stroke,fill}%
\end{pgfscope}%
\begin{pgfscope}%
\pgfpathrectangle{\pgfqpoint{0.481978in}{0.331635in}}{\pgfqpoint{9.300000in}{7.700000in}}%
\pgfusepath{clip}%
\pgfsetbuttcap%
\pgfsetroundjoin%
\definecolor{currentfill}{rgb}{1.000000,0.705882,0.509804}%
\pgfsetfillcolor{currentfill}%
\pgfsetlinewidth{0.481800pt}%
\definecolor{currentstroke}{rgb}{1.000000,1.000000,1.000000}%
\pgfsetstrokecolor{currentstroke}%
\pgfsetdash{}{0pt}%
\pgfpathmoveto{\pgfqpoint{3.313622in}{1.749779in}}%
\pgfpathcurveto{\pgfqpoint{3.324672in}{1.749779in}}{\pgfqpoint{3.335271in}{1.754169in}}{\pgfqpoint{3.343085in}{1.761982in}}%
\pgfpathcurveto{\pgfqpoint{3.350899in}{1.769796in}}{\pgfqpoint{3.355289in}{1.780395in}}{\pgfqpoint{3.355289in}{1.791445in}}%
\pgfpathcurveto{\pgfqpoint{3.355289in}{1.802495in}}{\pgfqpoint{3.350899in}{1.813094in}}{\pgfqpoint{3.343085in}{1.820908in}}%
\pgfpathcurveto{\pgfqpoint{3.335271in}{1.828722in}}{\pgfqpoint{3.324672in}{1.833112in}}{\pgfqpoint{3.313622in}{1.833112in}}%
\pgfpathcurveto{\pgfqpoint{3.302572in}{1.833112in}}{\pgfqpoint{3.291973in}{1.828722in}}{\pgfqpoint{3.284159in}{1.820908in}}%
\pgfpathcurveto{\pgfqpoint{3.276346in}{1.813094in}}{\pgfqpoint{3.271956in}{1.802495in}}{\pgfqpoint{3.271956in}{1.791445in}}%
\pgfpathcurveto{\pgfqpoint{3.271956in}{1.780395in}}{\pgfqpoint{3.276346in}{1.769796in}}{\pgfqpoint{3.284159in}{1.761982in}}%
\pgfpathcurveto{\pgfqpoint{3.291973in}{1.754169in}}{\pgfqpoint{3.302572in}{1.749779in}}{\pgfqpoint{3.313622in}{1.749779in}}%
\pgfpathclose%
\pgfusepath{stroke,fill}%
\end{pgfscope}%
\begin{pgfscope}%
\pgfpathrectangle{\pgfqpoint{0.481978in}{0.331635in}}{\pgfqpoint{9.300000in}{7.700000in}}%
\pgfusepath{clip}%
\pgfsetbuttcap%
\pgfsetroundjoin%
\definecolor{currentfill}{rgb}{1.000000,0.705882,0.509804}%
\pgfsetfillcolor{currentfill}%
\pgfsetlinewidth{0.481800pt}%
\definecolor{currentstroke}{rgb}{1.000000,1.000000,1.000000}%
\pgfsetstrokecolor{currentstroke}%
\pgfsetdash{}{0pt}%
\pgfpathmoveto{\pgfqpoint{3.782039in}{3.254788in}}%
\pgfpathcurveto{\pgfqpoint{3.793089in}{3.254788in}}{\pgfqpoint{3.803689in}{3.259178in}}{\pgfqpoint{3.811502in}{3.266992in}}%
\pgfpathcurveto{\pgfqpoint{3.819316in}{3.274805in}}{\pgfqpoint{3.823706in}{3.285404in}}{\pgfqpoint{3.823706in}{3.296454in}}%
\pgfpathcurveto{\pgfqpoint{3.823706in}{3.307505in}}{\pgfqpoint{3.819316in}{3.318104in}}{\pgfqpoint{3.811502in}{3.325917in}}%
\pgfpathcurveto{\pgfqpoint{3.803689in}{3.333731in}}{\pgfqpoint{3.793089in}{3.338121in}}{\pgfqpoint{3.782039in}{3.338121in}}%
\pgfpathcurveto{\pgfqpoint{3.770989in}{3.338121in}}{\pgfqpoint{3.760390in}{3.333731in}}{\pgfqpoint{3.752577in}{3.325917in}}%
\pgfpathcurveto{\pgfqpoint{3.744763in}{3.318104in}}{\pgfqpoint{3.740373in}{3.307505in}}{\pgfqpoint{3.740373in}{3.296454in}}%
\pgfpathcurveto{\pgfqpoint{3.740373in}{3.285404in}}{\pgfqpoint{3.744763in}{3.274805in}}{\pgfqpoint{3.752577in}{3.266992in}}%
\pgfpathcurveto{\pgfqpoint{3.760390in}{3.259178in}}{\pgfqpoint{3.770989in}{3.254788in}}{\pgfqpoint{3.782039in}{3.254788in}}%
\pgfpathclose%
\pgfusepath{stroke,fill}%
\end{pgfscope}%
\begin{pgfscope}%
\pgfpathrectangle{\pgfqpoint{0.481978in}{0.331635in}}{\pgfqpoint{9.300000in}{7.700000in}}%
\pgfusepath{clip}%
\pgfsetbuttcap%
\pgfsetroundjoin%
\definecolor{currentfill}{rgb}{1.000000,0.705882,0.509804}%
\pgfsetfillcolor{currentfill}%
\pgfsetlinewidth{0.481800pt}%
\definecolor{currentstroke}{rgb}{1.000000,1.000000,1.000000}%
\pgfsetstrokecolor{currentstroke}%
\pgfsetdash{}{0pt}%
\pgfpathmoveto{\pgfqpoint{3.355525in}{4.278688in}}%
\pgfpathcurveto{\pgfqpoint{3.366575in}{4.278688in}}{\pgfqpoint{3.377174in}{4.283079in}}{\pgfqpoint{3.384987in}{4.290892in}}%
\pgfpathcurveto{\pgfqpoint{3.392801in}{4.298706in}}{\pgfqpoint{3.397191in}{4.309305in}}{\pgfqpoint{3.397191in}{4.320355in}}%
\pgfpathcurveto{\pgfqpoint{3.397191in}{4.331405in}}{\pgfqpoint{3.392801in}{4.342004in}}{\pgfqpoint{3.384987in}{4.349818in}}%
\pgfpathcurveto{\pgfqpoint{3.377174in}{4.357631in}}{\pgfqpoint{3.366575in}{4.362022in}}{\pgfqpoint{3.355525in}{4.362022in}}%
\pgfpathcurveto{\pgfqpoint{3.344474in}{4.362022in}}{\pgfqpoint{3.333875in}{4.357631in}}{\pgfqpoint{3.326062in}{4.349818in}}%
\pgfpathcurveto{\pgfqpoint{3.318248in}{4.342004in}}{\pgfqpoint{3.313858in}{4.331405in}}{\pgfqpoint{3.313858in}{4.320355in}}%
\pgfpathcurveto{\pgfqpoint{3.313858in}{4.309305in}}{\pgfqpoint{3.318248in}{4.298706in}}{\pgfqpoint{3.326062in}{4.290892in}}%
\pgfpathcurveto{\pgfqpoint{3.333875in}{4.283079in}}{\pgfqpoint{3.344474in}{4.278688in}}{\pgfqpoint{3.355525in}{4.278688in}}%
\pgfpathclose%
\pgfusepath{stroke,fill}%
\end{pgfscope}%
\begin{pgfscope}%
\pgfpathrectangle{\pgfqpoint{0.481978in}{0.331635in}}{\pgfqpoint{9.300000in}{7.700000in}}%
\pgfusepath{clip}%
\pgfsetbuttcap%
\pgfsetroundjoin%
\definecolor{currentfill}{rgb}{1.000000,0.705882,0.509804}%
\pgfsetfillcolor{currentfill}%
\pgfsetlinewidth{0.481800pt}%
\definecolor{currentstroke}{rgb}{1.000000,1.000000,1.000000}%
\pgfsetstrokecolor{currentstroke}%
\pgfsetdash{}{0pt}%
\pgfpathmoveto{\pgfqpoint{3.639977in}{1.350357in}}%
\pgfpathcurveto{\pgfqpoint{3.651028in}{1.350357in}}{\pgfqpoint{3.661627in}{1.354748in}}{\pgfqpoint{3.669440in}{1.362561in}}%
\pgfpathcurveto{\pgfqpoint{3.677254in}{1.370375in}}{\pgfqpoint{3.681644in}{1.380974in}}{\pgfqpoint{3.681644in}{1.392024in}}%
\pgfpathcurveto{\pgfqpoint{3.681644in}{1.403074in}}{\pgfqpoint{3.677254in}{1.413673in}}{\pgfqpoint{3.669440in}{1.421487in}}%
\pgfpathcurveto{\pgfqpoint{3.661627in}{1.429300in}}{\pgfqpoint{3.651028in}{1.433691in}}{\pgfqpoint{3.639977in}{1.433691in}}%
\pgfpathcurveto{\pgfqpoint{3.628927in}{1.433691in}}{\pgfqpoint{3.618328in}{1.429300in}}{\pgfqpoint{3.610515in}{1.421487in}}%
\pgfpathcurveto{\pgfqpoint{3.602701in}{1.413673in}}{\pgfqpoint{3.598311in}{1.403074in}}{\pgfqpoint{3.598311in}{1.392024in}}%
\pgfpathcurveto{\pgfqpoint{3.598311in}{1.380974in}}{\pgfqpoint{3.602701in}{1.370375in}}{\pgfqpoint{3.610515in}{1.362561in}}%
\pgfpathcurveto{\pgfqpoint{3.618328in}{1.354748in}}{\pgfqpoint{3.628927in}{1.350357in}}{\pgfqpoint{3.639977in}{1.350357in}}%
\pgfpathclose%
\pgfusepath{stroke,fill}%
\end{pgfscope}%
\begin{pgfscope}%
\pgfpathrectangle{\pgfqpoint{0.481978in}{0.331635in}}{\pgfqpoint{9.300000in}{7.700000in}}%
\pgfusepath{clip}%
\pgfsetbuttcap%
\pgfsetroundjoin%
\definecolor{currentfill}{rgb}{1.000000,0.705882,0.509804}%
\pgfsetfillcolor{currentfill}%
\pgfsetlinewidth{0.481800pt}%
\definecolor{currentstroke}{rgb}{1.000000,1.000000,1.000000}%
\pgfsetstrokecolor{currentstroke}%
\pgfsetdash{}{0pt}%
\pgfpathmoveto{\pgfqpoint{3.357786in}{4.857081in}}%
\pgfpathcurveto{\pgfqpoint{3.368836in}{4.857081in}}{\pgfqpoint{3.379435in}{4.861472in}}{\pgfqpoint{3.387249in}{4.869285in}}%
\pgfpathcurveto{\pgfqpoint{3.395062in}{4.877099in}}{\pgfqpoint{3.399453in}{4.887698in}}{\pgfqpoint{3.399453in}{4.898748in}}%
\pgfpathcurveto{\pgfqpoint{3.399453in}{4.909798in}}{\pgfqpoint{3.395062in}{4.920397in}}{\pgfqpoint{3.387249in}{4.928211in}}%
\pgfpathcurveto{\pgfqpoint{3.379435in}{4.936024in}}{\pgfqpoint{3.368836in}{4.940415in}}{\pgfqpoint{3.357786in}{4.940415in}}%
\pgfpathcurveto{\pgfqpoint{3.346736in}{4.940415in}}{\pgfqpoint{3.336137in}{4.936024in}}{\pgfqpoint{3.328323in}{4.928211in}}%
\pgfpathcurveto{\pgfqpoint{3.320510in}{4.920397in}}{\pgfqpoint{3.316119in}{4.909798in}}{\pgfqpoint{3.316119in}{4.898748in}}%
\pgfpathcurveto{\pgfqpoint{3.316119in}{4.887698in}}{\pgfqpoint{3.320510in}{4.877099in}}{\pgfqpoint{3.328323in}{4.869285in}}%
\pgfpathcurveto{\pgfqpoint{3.336137in}{4.861472in}}{\pgfqpoint{3.346736in}{4.857081in}}{\pgfqpoint{3.357786in}{4.857081in}}%
\pgfpathclose%
\pgfusepath{stroke,fill}%
\end{pgfscope}%
\begin{pgfscope}%
\pgfpathrectangle{\pgfqpoint{0.481978in}{0.331635in}}{\pgfqpoint{9.300000in}{7.700000in}}%
\pgfusepath{clip}%
\pgfsetbuttcap%
\pgfsetroundjoin%
\definecolor{currentfill}{rgb}{1.000000,0.705882,0.509804}%
\pgfsetfillcolor{currentfill}%
\pgfsetlinewidth{0.481800pt}%
\definecolor{currentstroke}{rgb}{1.000000,1.000000,1.000000}%
\pgfsetstrokecolor{currentstroke}%
\pgfsetdash{}{0pt}%
\pgfpathmoveto{\pgfqpoint{7.382152in}{2.768990in}}%
\pgfpathcurveto{\pgfqpoint{7.393202in}{2.768990in}}{\pgfqpoint{7.403802in}{2.773380in}}{\pgfqpoint{7.411615in}{2.781194in}}%
\pgfpathcurveto{\pgfqpoint{7.419429in}{2.789008in}}{\pgfqpoint{7.423819in}{2.799607in}}{\pgfqpoint{7.423819in}{2.810657in}}%
\pgfpathcurveto{\pgfqpoint{7.423819in}{2.821707in}}{\pgfqpoint{7.419429in}{2.832306in}}{\pgfqpoint{7.411615in}{2.840119in}}%
\pgfpathcurveto{\pgfqpoint{7.403802in}{2.847933in}}{\pgfqpoint{7.393202in}{2.852323in}}{\pgfqpoint{7.382152in}{2.852323in}}%
\pgfpathcurveto{\pgfqpoint{7.371102in}{2.852323in}}{\pgfqpoint{7.360503in}{2.847933in}}{\pgfqpoint{7.352690in}{2.840119in}}%
\pgfpathcurveto{\pgfqpoint{7.344876in}{2.832306in}}{\pgfqpoint{7.340486in}{2.821707in}}{\pgfqpoint{7.340486in}{2.810657in}}%
\pgfpathcurveto{\pgfqpoint{7.340486in}{2.799607in}}{\pgfqpoint{7.344876in}{2.789008in}}{\pgfqpoint{7.352690in}{2.781194in}}%
\pgfpathcurveto{\pgfqpoint{7.360503in}{2.773380in}}{\pgfqpoint{7.371102in}{2.768990in}}{\pgfqpoint{7.382152in}{2.768990in}}%
\pgfpathclose%
\pgfusepath{stroke,fill}%
\end{pgfscope}%
\begin{pgfscope}%
\pgfpathrectangle{\pgfqpoint{0.481978in}{0.331635in}}{\pgfqpoint{9.300000in}{7.700000in}}%
\pgfusepath{clip}%
\pgfsetbuttcap%
\pgfsetroundjoin%
\definecolor{currentfill}{rgb}{1.000000,0.705882,0.509804}%
\pgfsetfillcolor{currentfill}%
\pgfsetlinewidth{0.481800pt}%
\definecolor{currentstroke}{rgb}{1.000000,1.000000,1.000000}%
\pgfsetstrokecolor{currentstroke}%
\pgfsetdash{}{0pt}%
\pgfpathmoveto{\pgfqpoint{8.231517in}{5.934610in}}%
\pgfpathcurveto{\pgfqpoint{8.242567in}{5.934610in}}{\pgfqpoint{8.253166in}{5.939000in}}{\pgfqpoint{8.260979in}{5.946814in}}%
\pgfpathcurveto{\pgfqpoint{8.268793in}{5.954628in}}{\pgfqpoint{8.273183in}{5.965227in}}{\pgfqpoint{8.273183in}{5.976277in}}%
\pgfpathcurveto{\pgfqpoint{8.273183in}{5.987327in}}{\pgfqpoint{8.268793in}{5.997926in}}{\pgfqpoint{8.260979in}{6.005739in}}%
\pgfpathcurveto{\pgfqpoint{8.253166in}{6.013553in}}{\pgfqpoint{8.242567in}{6.017943in}}{\pgfqpoint{8.231517in}{6.017943in}}%
\pgfpathcurveto{\pgfqpoint{8.220467in}{6.017943in}}{\pgfqpoint{8.209868in}{6.013553in}}{\pgfqpoint{8.202054in}{6.005739in}}%
\pgfpathcurveto{\pgfqpoint{8.194240in}{5.997926in}}{\pgfqpoint{8.189850in}{5.987327in}}{\pgfqpoint{8.189850in}{5.976277in}}%
\pgfpathcurveto{\pgfqpoint{8.189850in}{5.965227in}}{\pgfqpoint{8.194240in}{5.954628in}}{\pgfqpoint{8.202054in}{5.946814in}}%
\pgfpathcurveto{\pgfqpoint{8.209868in}{5.939000in}}{\pgfqpoint{8.220467in}{5.934610in}}{\pgfqpoint{8.231517in}{5.934610in}}%
\pgfpathclose%
\pgfusepath{stroke,fill}%
\end{pgfscope}%
\begin{pgfscope}%
\pgfpathrectangle{\pgfqpoint{0.481978in}{0.331635in}}{\pgfqpoint{9.300000in}{7.700000in}}%
\pgfusepath{clip}%
\pgfsetbuttcap%
\pgfsetroundjoin%
\definecolor{currentfill}{rgb}{1.000000,0.705882,0.509804}%
\pgfsetfillcolor{currentfill}%
\pgfsetlinewidth{0.481800pt}%
\definecolor{currentstroke}{rgb}{1.000000,1.000000,1.000000}%
\pgfsetstrokecolor{currentstroke}%
\pgfsetdash{}{0pt}%
\pgfpathmoveto{\pgfqpoint{6.777913in}{5.188332in}}%
\pgfpathcurveto{\pgfqpoint{6.788964in}{5.188332in}}{\pgfqpoint{6.799563in}{5.192722in}}{\pgfqpoint{6.807376in}{5.200535in}}%
\pgfpathcurveto{\pgfqpoint{6.815190in}{5.208349in}}{\pgfqpoint{6.819580in}{5.218948in}}{\pgfqpoint{6.819580in}{5.229998in}}%
\pgfpathcurveto{\pgfqpoint{6.819580in}{5.241048in}}{\pgfqpoint{6.815190in}{5.251647in}}{\pgfqpoint{6.807376in}{5.259461in}}%
\pgfpathcurveto{\pgfqpoint{6.799563in}{5.267275in}}{\pgfqpoint{6.788964in}{5.271665in}}{\pgfqpoint{6.777913in}{5.271665in}}%
\pgfpathcurveto{\pgfqpoint{6.766863in}{5.271665in}}{\pgfqpoint{6.756264in}{5.267275in}}{\pgfqpoint{6.748451in}{5.259461in}}%
\pgfpathcurveto{\pgfqpoint{6.740637in}{5.251647in}}{\pgfqpoint{6.736247in}{5.241048in}}{\pgfqpoint{6.736247in}{5.229998in}}%
\pgfpathcurveto{\pgfqpoint{6.736247in}{5.218948in}}{\pgfqpoint{6.740637in}{5.208349in}}{\pgfqpoint{6.748451in}{5.200535in}}%
\pgfpathcurveto{\pgfqpoint{6.756264in}{5.192722in}}{\pgfqpoint{6.766863in}{5.188332in}}{\pgfqpoint{6.777913in}{5.188332in}}%
\pgfpathclose%
\pgfusepath{stroke,fill}%
\end{pgfscope}%
\begin{pgfscope}%
\pgfpathrectangle{\pgfqpoint{0.481978in}{0.331635in}}{\pgfqpoint{9.300000in}{7.700000in}}%
\pgfusepath{clip}%
\pgfsetbuttcap%
\pgfsetroundjoin%
\definecolor{currentfill}{rgb}{1.000000,0.705882,0.509804}%
\pgfsetfillcolor{currentfill}%
\pgfsetlinewidth{0.481800pt}%
\definecolor{currentstroke}{rgb}{1.000000,1.000000,1.000000}%
\pgfsetstrokecolor{currentstroke}%
\pgfsetdash{}{0pt}%
\pgfpathmoveto{\pgfqpoint{4.360013in}{3.888389in}}%
\pgfpathcurveto{\pgfqpoint{4.371063in}{3.888389in}}{\pgfqpoint{4.381662in}{3.892780in}}{\pgfqpoint{4.389475in}{3.900593in}}%
\pgfpathcurveto{\pgfqpoint{4.397289in}{3.908407in}}{\pgfqpoint{4.401679in}{3.919006in}}{\pgfqpoint{4.401679in}{3.930056in}}%
\pgfpathcurveto{\pgfqpoint{4.401679in}{3.941106in}}{\pgfqpoint{4.397289in}{3.951705in}}{\pgfqpoint{4.389475in}{3.959519in}}%
\pgfpathcurveto{\pgfqpoint{4.381662in}{3.967333in}}{\pgfqpoint{4.371063in}{3.971723in}}{\pgfqpoint{4.360013in}{3.971723in}}%
\pgfpathcurveto{\pgfqpoint{4.348962in}{3.971723in}}{\pgfqpoint{4.338363in}{3.967333in}}{\pgfqpoint{4.330550in}{3.959519in}}%
\pgfpathcurveto{\pgfqpoint{4.322736in}{3.951705in}}{\pgfqpoint{4.318346in}{3.941106in}}{\pgfqpoint{4.318346in}{3.930056in}}%
\pgfpathcurveto{\pgfqpoint{4.318346in}{3.919006in}}{\pgfqpoint{4.322736in}{3.908407in}}{\pgfqpoint{4.330550in}{3.900593in}}%
\pgfpathcurveto{\pgfqpoint{4.338363in}{3.892780in}}{\pgfqpoint{4.348962in}{3.888389in}}{\pgfqpoint{4.360013in}{3.888389in}}%
\pgfpathclose%
\pgfusepath{stroke,fill}%
\end{pgfscope}%
\begin{pgfscope}%
\pgfpathrectangle{\pgfqpoint{0.481978in}{0.331635in}}{\pgfqpoint{9.300000in}{7.700000in}}%
\pgfusepath{clip}%
\pgfsetbuttcap%
\pgfsetroundjoin%
\definecolor{currentfill}{rgb}{1.000000,0.705882,0.509804}%
\pgfsetfillcolor{currentfill}%
\pgfsetlinewidth{0.481800pt}%
\definecolor{currentstroke}{rgb}{1.000000,1.000000,1.000000}%
\pgfsetstrokecolor{currentstroke}%
\pgfsetdash{}{0pt}%
\pgfpathmoveto{\pgfqpoint{5.351509in}{4.324819in}}%
\pgfpathcurveto{\pgfqpoint{5.362560in}{4.324819in}}{\pgfqpoint{5.373159in}{4.329209in}}{\pgfqpoint{5.380972in}{4.337023in}}%
\pgfpathcurveto{\pgfqpoint{5.388786in}{4.344836in}}{\pgfqpoint{5.393176in}{4.355435in}}{\pgfqpoint{5.393176in}{4.366486in}}%
\pgfpathcurveto{\pgfqpoint{5.393176in}{4.377536in}}{\pgfqpoint{5.388786in}{4.388135in}}{\pgfqpoint{5.380972in}{4.395948in}}%
\pgfpathcurveto{\pgfqpoint{5.373159in}{4.403762in}}{\pgfqpoint{5.362560in}{4.408152in}}{\pgfqpoint{5.351509in}{4.408152in}}%
\pgfpathcurveto{\pgfqpoint{5.340459in}{4.408152in}}{\pgfqpoint{5.329860in}{4.403762in}}{\pgfqpoint{5.322047in}{4.395948in}}%
\pgfpathcurveto{\pgfqpoint{5.314233in}{4.388135in}}{\pgfqpoint{5.309843in}{4.377536in}}{\pgfqpoint{5.309843in}{4.366486in}}%
\pgfpathcurveto{\pgfqpoint{5.309843in}{4.355435in}}{\pgfqpoint{5.314233in}{4.344836in}}{\pgfqpoint{5.322047in}{4.337023in}}%
\pgfpathcurveto{\pgfqpoint{5.329860in}{4.329209in}}{\pgfqpoint{5.340459in}{4.324819in}}{\pgfqpoint{5.351509in}{4.324819in}}%
\pgfpathclose%
\pgfusepath{stroke,fill}%
\end{pgfscope}%
\begin{pgfscope}%
\pgfpathrectangle{\pgfqpoint{0.481978in}{0.331635in}}{\pgfqpoint{9.300000in}{7.700000in}}%
\pgfusepath{clip}%
\pgfsetbuttcap%
\pgfsetroundjoin%
\definecolor{currentfill}{rgb}{1.000000,0.705882,0.509804}%
\pgfsetfillcolor{currentfill}%
\pgfsetlinewidth{0.481800pt}%
\definecolor{currentstroke}{rgb}{1.000000,1.000000,1.000000}%
\pgfsetstrokecolor{currentstroke}%
\pgfsetdash{}{0pt}%
\pgfpathmoveto{\pgfqpoint{2.852372in}{4.752667in}}%
\pgfpathcurveto{\pgfqpoint{2.863422in}{4.752667in}}{\pgfqpoint{2.874021in}{4.757057in}}{\pgfqpoint{2.881835in}{4.764871in}}%
\pgfpathcurveto{\pgfqpoint{2.889648in}{4.772685in}}{\pgfqpoint{2.894038in}{4.783284in}}{\pgfqpoint{2.894038in}{4.794334in}}%
\pgfpathcurveto{\pgfqpoint{2.894038in}{4.805384in}}{\pgfqpoint{2.889648in}{4.815983in}}{\pgfqpoint{2.881835in}{4.823797in}}%
\pgfpathcurveto{\pgfqpoint{2.874021in}{4.831610in}}{\pgfqpoint{2.863422in}{4.836000in}}{\pgfqpoint{2.852372in}{4.836000in}}%
\pgfpathcurveto{\pgfqpoint{2.841322in}{4.836000in}}{\pgfqpoint{2.830723in}{4.831610in}}{\pgfqpoint{2.822909in}{4.823797in}}%
\pgfpathcurveto{\pgfqpoint{2.815095in}{4.815983in}}{\pgfqpoint{2.810705in}{4.805384in}}{\pgfqpoint{2.810705in}{4.794334in}}%
\pgfpathcurveto{\pgfqpoint{2.810705in}{4.783284in}}{\pgfqpoint{2.815095in}{4.772685in}}{\pgfqpoint{2.822909in}{4.764871in}}%
\pgfpathcurveto{\pgfqpoint{2.830723in}{4.757057in}}{\pgfqpoint{2.841322in}{4.752667in}}{\pgfqpoint{2.852372in}{4.752667in}}%
\pgfpathclose%
\pgfusepath{stroke,fill}%
\end{pgfscope}%
\begin{pgfscope}%
\pgfpathrectangle{\pgfqpoint{0.481978in}{0.331635in}}{\pgfqpoint{9.300000in}{7.700000in}}%
\pgfusepath{clip}%
\pgfsetbuttcap%
\pgfsetroundjoin%
\definecolor{currentfill}{rgb}{1.000000,0.705882,0.509804}%
\pgfsetfillcolor{currentfill}%
\pgfsetlinewidth{0.481800pt}%
\definecolor{currentstroke}{rgb}{1.000000,1.000000,1.000000}%
\pgfsetstrokecolor{currentstroke}%
\pgfsetdash{}{0pt}%
\pgfpathmoveto{\pgfqpoint{2.417331in}{4.439055in}}%
\pgfpathcurveto{\pgfqpoint{2.428381in}{4.439055in}}{\pgfqpoint{2.438980in}{4.443446in}}{\pgfqpoint{2.446793in}{4.451259in}}%
\pgfpathcurveto{\pgfqpoint{2.454607in}{4.459073in}}{\pgfqpoint{2.458997in}{4.469672in}}{\pgfqpoint{2.458997in}{4.480722in}}%
\pgfpathcurveto{\pgfqpoint{2.458997in}{4.491772in}}{\pgfqpoint{2.454607in}{4.502371in}}{\pgfqpoint{2.446793in}{4.510185in}}%
\pgfpathcurveto{\pgfqpoint{2.438980in}{4.517998in}}{\pgfqpoint{2.428381in}{4.522389in}}{\pgfqpoint{2.417331in}{4.522389in}}%
\pgfpathcurveto{\pgfqpoint{2.406280in}{4.522389in}}{\pgfqpoint{2.395681in}{4.517998in}}{\pgfqpoint{2.387868in}{4.510185in}}%
\pgfpathcurveto{\pgfqpoint{2.380054in}{4.502371in}}{\pgfqpoint{2.375664in}{4.491772in}}{\pgfqpoint{2.375664in}{4.480722in}}%
\pgfpathcurveto{\pgfqpoint{2.375664in}{4.469672in}}{\pgfqpoint{2.380054in}{4.459073in}}{\pgfqpoint{2.387868in}{4.451259in}}%
\pgfpathcurveto{\pgfqpoint{2.395681in}{4.443446in}}{\pgfqpoint{2.406280in}{4.439055in}}{\pgfqpoint{2.417331in}{4.439055in}}%
\pgfpathclose%
\pgfusepath{stroke,fill}%
\end{pgfscope}%
\begin{pgfscope}%
\pgfpathrectangle{\pgfqpoint{0.481978in}{0.331635in}}{\pgfqpoint{9.300000in}{7.700000in}}%
\pgfusepath{clip}%
\pgfsetbuttcap%
\pgfsetroundjoin%
\definecolor{currentfill}{rgb}{1.000000,0.705882,0.509804}%
\pgfsetfillcolor{currentfill}%
\pgfsetlinewidth{0.481800pt}%
\definecolor{currentstroke}{rgb}{1.000000,1.000000,1.000000}%
\pgfsetstrokecolor{currentstroke}%
\pgfsetdash{}{0pt}%
\pgfpathmoveto{\pgfqpoint{2.136437in}{2.578612in}}%
\pgfpathcurveto{\pgfqpoint{2.147487in}{2.578612in}}{\pgfqpoint{2.158086in}{2.583002in}}{\pgfqpoint{2.165900in}{2.590815in}}%
\pgfpathcurveto{\pgfqpoint{2.173713in}{2.598629in}}{\pgfqpoint{2.178103in}{2.609228in}}{\pgfqpoint{2.178103in}{2.620278in}}%
\pgfpathcurveto{\pgfqpoint{2.178103in}{2.631328in}}{\pgfqpoint{2.173713in}{2.641927in}}{\pgfqpoint{2.165900in}{2.649741in}}%
\pgfpathcurveto{\pgfqpoint{2.158086in}{2.657555in}}{\pgfqpoint{2.147487in}{2.661945in}}{\pgfqpoint{2.136437in}{2.661945in}}%
\pgfpathcurveto{\pgfqpoint{2.125387in}{2.661945in}}{\pgfqpoint{2.114788in}{2.657555in}}{\pgfqpoint{2.106974in}{2.649741in}}%
\pgfpathcurveto{\pgfqpoint{2.099160in}{2.641927in}}{\pgfqpoint{2.094770in}{2.631328in}}{\pgfqpoint{2.094770in}{2.620278in}}%
\pgfpathcurveto{\pgfqpoint{2.094770in}{2.609228in}}{\pgfqpoint{2.099160in}{2.598629in}}{\pgfqpoint{2.106974in}{2.590815in}}%
\pgfpathcurveto{\pgfqpoint{2.114788in}{2.583002in}}{\pgfqpoint{2.125387in}{2.578612in}}{\pgfqpoint{2.136437in}{2.578612in}}%
\pgfpathclose%
\pgfusepath{stroke,fill}%
\end{pgfscope}%
\begin{pgfscope}%
\pgfpathrectangle{\pgfqpoint{0.481978in}{0.331635in}}{\pgfqpoint{9.300000in}{7.700000in}}%
\pgfusepath{clip}%
\pgfsetbuttcap%
\pgfsetroundjoin%
\definecolor{currentfill}{rgb}{1.000000,0.705882,0.509804}%
\pgfsetfillcolor{currentfill}%
\pgfsetlinewidth{0.481800pt}%
\definecolor{currentstroke}{rgb}{1.000000,1.000000,1.000000}%
\pgfsetstrokecolor{currentstroke}%
\pgfsetdash{}{0pt}%
\pgfpathmoveto{\pgfqpoint{3.078254in}{2.408578in}}%
\pgfpathcurveto{\pgfqpoint{3.089304in}{2.408578in}}{\pgfqpoint{3.099903in}{2.412968in}}{\pgfqpoint{3.107717in}{2.420782in}}%
\pgfpathcurveto{\pgfqpoint{3.115531in}{2.428595in}}{\pgfqpoint{3.119921in}{2.439194in}}{\pgfqpoint{3.119921in}{2.450245in}}%
\pgfpathcurveto{\pgfqpoint{3.119921in}{2.461295in}}{\pgfqpoint{3.115531in}{2.471894in}}{\pgfqpoint{3.107717in}{2.479707in}}%
\pgfpathcurveto{\pgfqpoint{3.099903in}{2.487521in}}{\pgfqpoint{3.089304in}{2.491911in}}{\pgfqpoint{3.078254in}{2.491911in}}%
\pgfpathcurveto{\pgfqpoint{3.067204in}{2.491911in}}{\pgfqpoint{3.056605in}{2.487521in}}{\pgfqpoint{3.048791in}{2.479707in}}%
\pgfpathcurveto{\pgfqpoint{3.040978in}{2.471894in}}{\pgfqpoint{3.036587in}{2.461295in}}{\pgfqpoint{3.036587in}{2.450245in}}%
\pgfpathcurveto{\pgfqpoint{3.036587in}{2.439194in}}{\pgfqpoint{3.040978in}{2.428595in}}{\pgfqpoint{3.048791in}{2.420782in}}%
\pgfpathcurveto{\pgfqpoint{3.056605in}{2.412968in}}{\pgfqpoint{3.067204in}{2.408578in}}{\pgfqpoint{3.078254in}{2.408578in}}%
\pgfpathclose%
\pgfusepath{stroke,fill}%
\end{pgfscope}%
\begin{pgfscope}%
\pgfpathrectangle{\pgfqpoint{0.481978in}{0.331635in}}{\pgfqpoint{9.300000in}{7.700000in}}%
\pgfusepath{clip}%
\pgfsetbuttcap%
\pgfsetroundjoin%
\definecolor{currentfill}{rgb}{1.000000,0.705882,0.509804}%
\pgfsetfillcolor{currentfill}%
\pgfsetlinewidth{0.481800pt}%
\definecolor{currentstroke}{rgb}{1.000000,1.000000,1.000000}%
\pgfsetstrokecolor{currentstroke}%
\pgfsetdash{}{0pt}%
\pgfpathmoveto{\pgfqpoint{9.359251in}{4.608884in}}%
\pgfpathcurveto{\pgfqpoint{9.370301in}{4.608884in}}{\pgfqpoint{9.380900in}{4.613274in}}{\pgfqpoint{9.388713in}{4.621087in}}%
\pgfpathcurveto{\pgfqpoint{9.396527in}{4.628901in}}{\pgfqpoint{9.400917in}{4.639500in}}{\pgfqpoint{9.400917in}{4.650550in}}%
\pgfpathcurveto{\pgfqpoint{9.400917in}{4.661600in}}{\pgfqpoint{9.396527in}{4.672199in}}{\pgfqpoint{9.388713in}{4.680013in}}%
\pgfpathcurveto{\pgfqpoint{9.380900in}{4.687827in}}{\pgfqpoint{9.370301in}{4.692217in}}{\pgfqpoint{9.359251in}{4.692217in}}%
\pgfpathcurveto{\pgfqpoint{9.348201in}{4.692217in}}{\pgfqpoint{9.337601in}{4.687827in}}{\pgfqpoint{9.329788in}{4.680013in}}%
\pgfpathcurveto{\pgfqpoint{9.321974in}{4.672199in}}{\pgfqpoint{9.317584in}{4.661600in}}{\pgfqpoint{9.317584in}{4.650550in}}%
\pgfpathcurveto{\pgfqpoint{9.317584in}{4.639500in}}{\pgfqpoint{9.321974in}{4.628901in}}{\pgfqpoint{9.329788in}{4.621087in}}%
\pgfpathcurveto{\pgfqpoint{9.337601in}{4.613274in}}{\pgfqpoint{9.348201in}{4.608884in}}{\pgfqpoint{9.359251in}{4.608884in}}%
\pgfpathclose%
\pgfusepath{stroke,fill}%
\end{pgfscope}%
\begin{pgfscope}%
\pgfpathrectangle{\pgfqpoint{0.481978in}{0.331635in}}{\pgfqpoint{9.300000in}{7.700000in}}%
\pgfusepath{clip}%
\pgfsetbuttcap%
\pgfsetroundjoin%
\definecolor{currentfill}{rgb}{1.000000,0.705882,0.509804}%
\pgfsetfillcolor{currentfill}%
\pgfsetlinewidth{0.481800pt}%
\definecolor{currentstroke}{rgb}{1.000000,1.000000,1.000000}%
\pgfsetstrokecolor{currentstroke}%
\pgfsetdash{}{0pt}%
\pgfpathmoveto{\pgfqpoint{5.664273in}{3.152762in}}%
\pgfpathcurveto{\pgfqpoint{5.675323in}{3.152762in}}{\pgfqpoint{5.685922in}{3.157152in}}{\pgfqpoint{5.693736in}{3.164966in}}%
\pgfpathcurveto{\pgfqpoint{5.701549in}{3.172780in}}{\pgfqpoint{5.705940in}{3.183379in}}{\pgfqpoint{5.705940in}{3.194429in}}%
\pgfpathcurveto{\pgfqpoint{5.705940in}{3.205479in}}{\pgfqpoint{5.701549in}{3.216078in}}{\pgfqpoint{5.693736in}{3.223892in}}%
\pgfpathcurveto{\pgfqpoint{5.685922in}{3.231705in}}{\pgfqpoint{5.675323in}{3.236095in}}{\pgfqpoint{5.664273in}{3.236095in}}%
\pgfpathcurveto{\pgfqpoint{5.653223in}{3.236095in}}{\pgfqpoint{5.642624in}{3.231705in}}{\pgfqpoint{5.634810in}{3.223892in}}%
\pgfpathcurveto{\pgfqpoint{5.626997in}{3.216078in}}{\pgfqpoint{5.622606in}{3.205479in}}{\pgfqpoint{5.622606in}{3.194429in}}%
\pgfpathcurveto{\pgfqpoint{5.622606in}{3.183379in}}{\pgfqpoint{5.626997in}{3.172780in}}{\pgfqpoint{5.634810in}{3.164966in}}%
\pgfpathcurveto{\pgfqpoint{5.642624in}{3.157152in}}{\pgfqpoint{5.653223in}{3.152762in}}{\pgfqpoint{5.664273in}{3.152762in}}%
\pgfpathclose%
\pgfusepath{stroke,fill}%
\end{pgfscope}%
\begin{pgfscope}%
\pgfpathrectangle{\pgfqpoint{0.481978in}{0.331635in}}{\pgfqpoint{9.300000in}{7.700000in}}%
\pgfusepath{clip}%
\pgfsetbuttcap%
\pgfsetroundjoin%
\definecolor{currentfill}{rgb}{1.000000,0.705882,0.509804}%
\pgfsetfillcolor{currentfill}%
\pgfsetlinewidth{0.481800pt}%
\definecolor{currentstroke}{rgb}{1.000000,1.000000,1.000000}%
\pgfsetstrokecolor{currentstroke}%
\pgfsetdash{}{0pt}%
\pgfpathmoveto{\pgfqpoint{8.498557in}{4.998420in}}%
\pgfpathcurveto{\pgfqpoint{8.509607in}{4.998420in}}{\pgfqpoint{8.520206in}{5.002810in}}{\pgfqpoint{8.528020in}{5.010624in}}%
\pgfpathcurveto{\pgfqpoint{8.535834in}{5.018438in}}{\pgfqpoint{8.540224in}{5.029037in}}{\pgfqpoint{8.540224in}{5.040087in}}%
\pgfpathcurveto{\pgfqpoint{8.540224in}{5.051137in}}{\pgfqpoint{8.535834in}{5.061736in}}{\pgfqpoint{8.528020in}{5.069549in}}%
\pgfpathcurveto{\pgfqpoint{8.520206in}{5.077363in}}{\pgfqpoint{8.509607in}{5.081753in}}{\pgfqpoint{8.498557in}{5.081753in}}%
\pgfpathcurveto{\pgfqpoint{8.487507in}{5.081753in}}{\pgfqpoint{8.476908in}{5.077363in}}{\pgfqpoint{8.469094in}{5.069549in}}%
\pgfpathcurveto{\pgfqpoint{8.461281in}{5.061736in}}{\pgfqpoint{8.456891in}{5.051137in}}{\pgfqpoint{8.456891in}{5.040087in}}%
\pgfpathcurveto{\pgfqpoint{8.456891in}{5.029037in}}{\pgfqpoint{8.461281in}{5.018438in}}{\pgfqpoint{8.469094in}{5.010624in}}%
\pgfpathcurveto{\pgfqpoint{8.476908in}{5.002810in}}{\pgfqpoint{8.487507in}{4.998420in}}{\pgfqpoint{8.498557in}{4.998420in}}%
\pgfpathclose%
\pgfusepath{stroke,fill}%
\end{pgfscope}%
\begin{pgfscope}%
\pgfpathrectangle{\pgfqpoint{0.481978in}{0.331635in}}{\pgfqpoint{9.300000in}{7.700000in}}%
\pgfusepath{clip}%
\pgfsetbuttcap%
\pgfsetroundjoin%
\definecolor{currentfill}{rgb}{1.000000,0.705882,0.509804}%
\pgfsetfillcolor{currentfill}%
\pgfsetlinewidth{0.481800pt}%
\definecolor{currentstroke}{rgb}{1.000000,1.000000,1.000000}%
\pgfsetstrokecolor{currentstroke}%
\pgfsetdash{}{0pt}%
\pgfpathmoveto{\pgfqpoint{6.107881in}{4.191119in}}%
\pgfpathcurveto{\pgfqpoint{6.118931in}{4.191119in}}{\pgfqpoint{6.129530in}{4.195510in}}{\pgfqpoint{6.137344in}{4.203323in}}%
\pgfpathcurveto{\pgfqpoint{6.145157in}{4.211137in}}{\pgfqpoint{6.149548in}{4.221736in}}{\pgfqpoint{6.149548in}{4.232786in}}%
\pgfpathcurveto{\pgfqpoint{6.149548in}{4.243836in}}{\pgfqpoint{6.145157in}{4.254435in}}{\pgfqpoint{6.137344in}{4.262249in}}%
\pgfpathcurveto{\pgfqpoint{6.129530in}{4.270062in}}{\pgfqpoint{6.118931in}{4.274453in}}{\pgfqpoint{6.107881in}{4.274453in}}%
\pgfpathcurveto{\pgfqpoint{6.096831in}{4.274453in}}{\pgfqpoint{6.086232in}{4.270062in}}{\pgfqpoint{6.078418in}{4.262249in}}%
\pgfpathcurveto{\pgfqpoint{6.070605in}{4.254435in}}{\pgfqpoint{6.066214in}{4.243836in}}{\pgfqpoint{6.066214in}{4.232786in}}%
\pgfpathcurveto{\pgfqpoint{6.066214in}{4.221736in}}{\pgfqpoint{6.070605in}{4.211137in}}{\pgfqpoint{6.078418in}{4.203323in}}%
\pgfpathcurveto{\pgfqpoint{6.086232in}{4.195510in}}{\pgfqpoint{6.096831in}{4.191119in}}{\pgfqpoint{6.107881in}{4.191119in}}%
\pgfpathclose%
\pgfusepath{stroke,fill}%
\end{pgfscope}%
\begin{pgfscope}%
\pgfpathrectangle{\pgfqpoint{0.481978in}{0.331635in}}{\pgfqpoint{9.300000in}{7.700000in}}%
\pgfusepath{clip}%
\pgfsetbuttcap%
\pgfsetroundjoin%
\definecolor{currentfill}{rgb}{1.000000,0.705882,0.509804}%
\pgfsetfillcolor{currentfill}%
\pgfsetlinewidth{0.481800pt}%
\definecolor{currentstroke}{rgb}{1.000000,1.000000,1.000000}%
\pgfsetstrokecolor{currentstroke}%
\pgfsetdash{}{0pt}%
\pgfpathmoveto{\pgfqpoint{7.335472in}{5.502267in}}%
\pgfpathcurveto{\pgfqpoint{7.346522in}{5.502267in}}{\pgfqpoint{7.357121in}{5.506657in}}{\pgfqpoint{7.364934in}{5.514471in}}%
\pgfpathcurveto{\pgfqpoint{7.372748in}{5.522284in}}{\pgfqpoint{7.377138in}{5.532883in}}{\pgfqpoint{7.377138in}{5.543933in}}%
\pgfpathcurveto{\pgfqpoint{7.377138in}{5.554984in}}{\pgfqpoint{7.372748in}{5.565583in}}{\pgfqpoint{7.364934in}{5.573396in}}%
\pgfpathcurveto{\pgfqpoint{7.357121in}{5.581210in}}{\pgfqpoint{7.346522in}{5.585600in}}{\pgfqpoint{7.335472in}{5.585600in}}%
\pgfpathcurveto{\pgfqpoint{7.324422in}{5.585600in}}{\pgfqpoint{7.313822in}{5.581210in}}{\pgfqpoint{7.306009in}{5.573396in}}%
\pgfpathcurveto{\pgfqpoint{7.298195in}{5.565583in}}{\pgfqpoint{7.293805in}{5.554984in}}{\pgfqpoint{7.293805in}{5.543933in}}%
\pgfpathcurveto{\pgfqpoint{7.293805in}{5.532883in}}{\pgfqpoint{7.298195in}{5.522284in}}{\pgfqpoint{7.306009in}{5.514471in}}%
\pgfpathcurveto{\pgfqpoint{7.313822in}{5.506657in}}{\pgfqpoint{7.324422in}{5.502267in}}{\pgfqpoint{7.335472in}{5.502267in}}%
\pgfpathclose%
\pgfusepath{stroke,fill}%
\end{pgfscope}%
\begin{pgfscope}%
\pgfpathrectangle{\pgfqpoint{0.481978in}{0.331635in}}{\pgfqpoint{9.300000in}{7.700000in}}%
\pgfusepath{clip}%
\pgfsetbuttcap%
\pgfsetroundjoin%
\definecolor{currentfill}{rgb}{1.000000,0.705882,0.509804}%
\pgfsetfillcolor{currentfill}%
\pgfsetlinewidth{0.481800pt}%
\definecolor{currentstroke}{rgb}{1.000000,1.000000,1.000000}%
\pgfsetstrokecolor{currentstroke}%
\pgfsetdash{}{0pt}%
\pgfpathmoveto{\pgfqpoint{8.971590in}{5.799204in}}%
\pgfpathcurveto{\pgfqpoint{8.982641in}{5.799204in}}{\pgfqpoint{8.993240in}{5.803594in}}{\pgfqpoint{9.001053in}{5.811408in}}%
\pgfpathcurveto{\pgfqpoint{9.008867in}{5.819222in}}{\pgfqpoint{9.013257in}{5.829821in}}{\pgfqpoint{9.013257in}{5.840871in}}%
\pgfpathcurveto{\pgfqpoint{9.013257in}{5.851921in}}{\pgfqpoint{9.008867in}{5.862520in}}{\pgfqpoint{9.001053in}{5.870334in}}%
\pgfpathcurveto{\pgfqpoint{8.993240in}{5.878147in}}{\pgfqpoint{8.982641in}{5.882537in}}{\pgfqpoint{8.971590in}{5.882537in}}%
\pgfpathcurveto{\pgfqpoint{8.960540in}{5.882537in}}{\pgfqpoint{8.949941in}{5.878147in}}{\pgfqpoint{8.942128in}{5.870334in}}%
\pgfpathcurveto{\pgfqpoint{8.934314in}{5.862520in}}{\pgfqpoint{8.929924in}{5.851921in}}{\pgfqpoint{8.929924in}{5.840871in}}%
\pgfpathcurveto{\pgfqpoint{8.929924in}{5.829821in}}{\pgfqpoint{8.934314in}{5.819222in}}{\pgfqpoint{8.942128in}{5.811408in}}%
\pgfpathcurveto{\pgfqpoint{8.949941in}{5.803594in}}{\pgfqpoint{8.960540in}{5.799204in}}{\pgfqpoint{8.971590in}{5.799204in}}%
\pgfpathclose%
\pgfusepath{stroke,fill}%
\end{pgfscope}%
\begin{pgfscope}%
\pgfpathrectangle{\pgfqpoint{0.481978in}{0.331635in}}{\pgfqpoint{9.300000in}{7.700000in}}%
\pgfusepath{clip}%
\pgfsetbuttcap%
\pgfsetroundjoin%
\definecolor{currentfill}{rgb}{1.000000,0.705882,0.509804}%
\pgfsetfillcolor{currentfill}%
\pgfsetlinewidth{0.481800pt}%
\definecolor{currentstroke}{rgb}{1.000000,1.000000,1.000000}%
\pgfsetstrokecolor{currentstroke}%
\pgfsetdash{}{0pt}%
\pgfpathmoveto{\pgfqpoint{4.777742in}{4.712947in}}%
\pgfpathcurveto{\pgfqpoint{4.788792in}{4.712947in}}{\pgfqpoint{4.799391in}{4.717337in}}{\pgfqpoint{4.807205in}{4.725150in}}%
\pgfpathcurveto{\pgfqpoint{4.815018in}{4.732964in}}{\pgfqpoint{4.819409in}{4.743563in}}{\pgfqpoint{4.819409in}{4.754613in}}%
\pgfpathcurveto{\pgfqpoint{4.819409in}{4.765663in}}{\pgfqpoint{4.815018in}{4.776262in}}{\pgfqpoint{4.807205in}{4.784076in}}%
\pgfpathcurveto{\pgfqpoint{4.799391in}{4.791890in}}{\pgfqpoint{4.788792in}{4.796280in}}{\pgfqpoint{4.777742in}{4.796280in}}%
\pgfpathcurveto{\pgfqpoint{4.766692in}{4.796280in}}{\pgfqpoint{4.756093in}{4.791890in}}{\pgfqpoint{4.748279in}{4.784076in}}%
\pgfpathcurveto{\pgfqpoint{4.740465in}{4.776262in}}{\pgfqpoint{4.736075in}{4.765663in}}{\pgfqpoint{4.736075in}{4.754613in}}%
\pgfpathcurveto{\pgfqpoint{4.736075in}{4.743563in}}{\pgfqpoint{4.740465in}{4.732964in}}{\pgfqpoint{4.748279in}{4.725150in}}%
\pgfpathcurveto{\pgfqpoint{4.756093in}{4.717337in}}{\pgfqpoint{4.766692in}{4.712947in}}{\pgfqpoint{4.777742in}{4.712947in}}%
\pgfpathclose%
\pgfusepath{stroke,fill}%
\end{pgfscope}%
\begin{pgfscope}%
\pgfpathrectangle{\pgfqpoint{0.481978in}{0.331635in}}{\pgfqpoint{9.300000in}{7.700000in}}%
\pgfusepath{clip}%
\pgfsetbuttcap%
\pgfsetroundjoin%
\definecolor{currentfill}{rgb}{1.000000,0.705882,0.509804}%
\pgfsetfillcolor{currentfill}%
\pgfsetlinewidth{0.481800pt}%
\definecolor{currentstroke}{rgb}{1.000000,1.000000,1.000000}%
\pgfsetstrokecolor{currentstroke}%
\pgfsetdash{}{0pt}%
\pgfpathmoveto{\pgfqpoint{2.999362in}{0.989798in}}%
\pgfpathcurveto{\pgfqpoint{3.010412in}{0.989798in}}{\pgfqpoint{3.021011in}{0.994188in}}{\pgfqpoint{3.028825in}{1.002002in}}%
\pgfpathcurveto{\pgfqpoint{3.036638in}{1.009815in}}{\pgfqpoint{3.041028in}{1.020414in}}{\pgfqpoint{3.041028in}{1.031465in}}%
\pgfpathcurveto{\pgfqpoint{3.041028in}{1.042515in}}{\pgfqpoint{3.036638in}{1.053114in}}{\pgfqpoint{3.028825in}{1.060927in}}%
\pgfpathcurveto{\pgfqpoint{3.021011in}{1.068741in}}{\pgfqpoint{3.010412in}{1.073131in}}{\pgfqpoint{2.999362in}{1.073131in}}%
\pgfpathcurveto{\pgfqpoint{2.988312in}{1.073131in}}{\pgfqpoint{2.977713in}{1.068741in}}{\pgfqpoint{2.969899in}{1.060927in}}%
\pgfpathcurveto{\pgfqpoint{2.962085in}{1.053114in}}{\pgfqpoint{2.957695in}{1.042515in}}{\pgfqpoint{2.957695in}{1.031465in}}%
\pgfpathcurveto{\pgfqpoint{2.957695in}{1.020414in}}{\pgfqpoint{2.962085in}{1.009815in}}{\pgfqpoint{2.969899in}{1.002002in}}%
\pgfpathcurveto{\pgfqpoint{2.977713in}{0.994188in}}{\pgfqpoint{2.988312in}{0.989798in}}{\pgfqpoint{2.999362in}{0.989798in}}%
\pgfpathclose%
\pgfusepath{stroke,fill}%
\end{pgfscope}%
\begin{pgfscope}%
\pgfpathrectangle{\pgfqpoint{0.481978in}{0.331635in}}{\pgfqpoint{9.300000in}{7.700000in}}%
\pgfusepath{clip}%
\pgfsetbuttcap%
\pgfsetroundjoin%
\definecolor{currentfill}{rgb}{1.000000,0.705882,0.509804}%
\pgfsetfillcolor{currentfill}%
\pgfsetlinewidth{0.481800pt}%
\definecolor{currentstroke}{rgb}{1.000000,1.000000,1.000000}%
\pgfsetstrokecolor{currentstroke}%
\pgfsetdash{}{0pt}%
\pgfpathmoveto{\pgfqpoint{3.001304in}{5.589493in}}%
\pgfpathcurveto{\pgfqpoint{3.012354in}{5.589493in}}{\pgfqpoint{3.022953in}{5.593883in}}{\pgfqpoint{3.030767in}{5.601697in}}%
\pgfpathcurveto{\pgfqpoint{3.038581in}{5.609510in}}{\pgfqpoint{3.042971in}{5.620109in}}{\pgfqpoint{3.042971in}{5.631160in}}%
\pgfpathcurveto{\pgfqpoint{3.042971in}{5.642210in}}{\pgfqpoint{3.038581in}{5.652809in}}{\pgfqpoint{3.030767in}{5.660622in}}%
\pgfpathcurveto{\pgfqpoint{3.022953in}{5.668436in}}{\pgfqpoint{3.012354in}{5.672826in}}{\pgfqpoint{3.001304in}{5.672826in}}%
\pgfpathcurveto{\pgfqpoint{2.990254in}{5.672826in}}{\pgfqpoint{2.979655in}{5.668436in}}{\pgfqpoint{2.971841in}{5.660622in}}%
\pgfpathcurveto{\pgfqpoint{2.964028in}{5.652809in}}{\pgfqpoint{2.959637in}{5.642210in}}{\pgfqpoint{2.959637in}{5.631160in}}%
\pgfpathcurveto{\pgfqpoint{2.959637in}{5.620109in}}{\pgfqpoint{2.964028in}{5.609510in}}{\pgfqpoint{2.971841in}{5.601697in}}%
\pgfpathcurveto{\pgfqpoint{2.979655in}{5.593883in}}{\pgfqpoint{2.990254in}{5.589493in}}{\pgfqpoint{3.001304in}{5.589493in}}%
\pgfpathclose%
\pgfusepath{stroke,fill}%
\end{pgfscope}%
\begin{pgfscope}%
\pgfpathrectangle{\pgfqpoint{0.481978in}{0.331635in}}{\pgfqpoint{9.300000in}{7.700000in}}%
\pgfusepath{clip}%
\pgfsetbuttcap%
\pgfsetroundjoin%
\definecolor{currentfill}{rgb}{1.000000,0.705882,0.509804}%
\pgfsetfillcolor{currentfill}%
\pgfsetlinewidth{0.481800pt}%
\definecolor{currentstroke}{rgb}{1.000000,1.000000,1.000000}%
\pgfsetstrokecolor{currentstroke}%
\pgfsetdash{}{0pt}%
\pgfpathmoveto{\pgfqpoint{3.332958in}{3.126459in}}%
\pgfpathcurveto{\pgfqpoint{3.344008in}{3.126459in}}{\pgfqpoint{3.354608in}{3.130849in}}{\pgfqpoint{3.362421in}{3.138663in}}%
\pgfpathcurveto{\pgfqpoint{3.370235in}{3.146476in}}{\pgfqpoint{3.374625in}{3.157075in}}{\pgfqpoint{3.374625in}{3.168126in}}%
\pgfpathcurveto{\pgfqpoint{3.374625in}{3.179176in}}{\pgfqpoint{3.370235in}{3.189775in}}{\pgfqpoint{3.362421in}{3.197588in}}%
\pgfpathcurveto{\pgfqpoint{3.354608in}{3.205402in}}{\pgfqpoint{3.344008in}{3.209792in}}{\pgfqpoint{3.332958in}{3.209792in}}%
\pgfpathcurveto{\pgfqpoint{3.321908in}{3.209792in}}{\pgfqpoint{3.311309in}{3.205402in}}{\pgfqpoint{3.303496in}{3.197588in}}%
\pgfpathcurveto{\pgfqpoint{3.295682in}{3.189775in}}{\pgfqpoint{3.291292in}{3.179176in}}{\pgfqpoint{3.291292in}{3.168126in}}%
\pgfpathcurveto{\pgfqpoint{3.291292in}{3.157075in}}{\pgfqpoint{3.295682in}{3.146476in}}{\pgfqpoint{3.303496in}{3.138663in}}%
\pgfpathcurveto{\pgfqpoint{3.311309in}{3.130849in}}{\pgfqpoint{3.321908in}{3.126459in}}{\pgfqpoint{3.332958in}{3.126459in}}%
\pgfpathclose%
\pgfusepath{stroke,fill}%
\end{pgfscope}%
\begin{pgfscope}%
\pgfpathrectangle{\pgfqpoint{0.481978in}{0.331635in}}{\pgfqpoint{9.300000in}{7.700000in}}%
\pgfusepath{clip}%
\pgfsetbuttcap%
\pgfsetroundjoin%
\definecolor{currentfill}{rgb}{1.000000,0.705882,0.509804}%
\pgfsetfillcolor{currentfill}%
\pgfsetlinewidth{0.481800pt}%
\definecolor{currentstroke}{rgb}{1.000000,1.000000,1.000000}%
\pgfsetstrokecolor{currentstroke}%
\pgfsetdash{}{0pt}%
\pgfpathmoveto{\pgfqpoint{8.075703in}{5.244500in}}%
\pgfpathcurveto{\pgfqpoint{8.086753in}{5.244500in}}{\pgfqpoint{8.097352in}{5.248890in}}{\pgfqpoint{8.105166in}{5.256704in}}%
\pgfpathcurveto{\pgfqpoint{8.112979in}{5.264518in}}{\pgfqpoint{8.117369in}{5.275117in}}{\pgfqpoint{8.117369in}{5.286167in}}%
\pgfpathcurveto{\pgfqpoint{8.117369in}{5.297217in}}{\pgfqpoint{8.112979in}{5.307816in}}{\pgfqpoint{8.105166in}{5.315630in}}%
\pgfpathcurveto{\pgfqpoint{8.097352in}{5.323443in}}{\pgfqpoint{8.086753in}{5.327833in}}{\pgfqpoint{8.075703in}{5.327833in}}%
\pgfpathcurveto{\pgfqpoint{8.064653in}{5.327833in}}{\pgfqpoint{8.054054in}{5.323443in}}{\pgfqpoint{8.046240in}{5.315630in}}%
\pgfpathcurveto{\pgfqpoint{8.038426in}{5.307816in}}{\pgfqpoint{8.034036in}{5.297217in}}{\pgfqpoint{8.034036in}{5.286167in}}%
\pgfpathcurveto{\pgfqpoint{8.034036in}{5.275117in}}{\pgfqpoint{8.038426in}{5.264518in}}{\pgfqpoint{8.046240in}{5.256704in}}%
\pgfpathcurveto{\pgfqpoint{8.054054in}{5.248890in}}{\pgfqpoint{8.064653in}{5.244500in}}{\pgfqpoint{8.075703in}{5.244500in}}%
\pgfpathclose%
\pgfusepath{stroke,fill}%
\end{pgfscope}%
\begin{pgfscope}%
\pgfpathrectangle{\pgfqpoint{0.481978in}{0.331635in}}{\pgfqpoint{9.300000in}{7.700000in}}%
\pgfusepath{clip}%
\pgfsetbuttcap%
\pgfsetroundjoin%
\definecolor{currentfill}{rgb}{1.000000,0.705882,0.509804}%
\pgfsetfillcolor{currentfill}%
\pgfsetlinewidth{0.481800pt}%
\definecolor{currentstroke}{rgb}{1.000000,1.000000,1.000000}%
\pgfsetstrokecolor{currentstroke}%
\pgfsetdash{}{0pt}%
\pgfpathmoveto{\pgfqpoint{4.123221in}{3.486220in}}%
\pgfpathcurveto{\pgfqpoint{4.134271in}{3.486220in}}{\pgfqpoint{4.144870in}{3.490610in}}{\pgfqpoint{4.152683in}{3.498424in}}%
\pgfpathcurveto{\pgfqpoint{4.160497in}{3.506238in}}{\pgfqpoint{4.164887in}{3.516837in}}{\pgfqpoint{4.164887in}{3.527887in}}%
\pgfpathcurveto{\pgfqpoint{4.164887in}{3.538937in}}{\pgfqpoint{4.160497in}{3.549536in}}{\pgfqpoint{4.152683in}{3.557350in}}%
\pgfpathcurveto{\pgfqpoint{4.144870in}{3.565163in}}{\pgfqpoint{4.134271in}{3.569553in}}{\pgfqpoint{4.123221in}{3.569553in}}%
\pgfpathcurveto{\pgfqpoint{4.112170in}{3.569553in}}{\pgfqpoint{4.101571in}{3.565163in}}{\pgfqpoint{4.093758in}{3.557350in}}%
\pgfpathcurveto{\pgfqpoint{4.085944in}{3.549536in}}{\pgfqpoint{4.081554in}{3.538937in}}{\pgfqpoint{4.081554in}{3.527887in}}%
\pgfpathcurveto{\pgfqpoint{4.081554in}{3.516837in}}{\pgfqpoint{4.085944in}{3.506238in}}{\pgfqpoint{4.093758in}{3.498424in}}%
\pgfpathcurveto{\pgfqpoint{4.101571in}{3.490610in}}{\pgfqpoint{4.112170in}{3.486220in}}{\pgfqpoint{4.123221in}{3.486220in}}%
\pgfpathclose%
\pgfusepath{stroke,fill}%
\end{pgfscope}%
\begin{pgfscope}%
\pgfpathrectangle{\pgfqpoint{0.481978in}{0.331635in}}{\pgfqpoint{9.300000in}{7.700000in}}%
\pgfusepath{clip}%
\pgfsetbuttcap%
\pgfsetroundjoin%
\definecolor{currentfill}{rgb}{1.000000,0.705882,0.509804}%
\pgfsetfillcolor{currentfill}%
\pgfsetlinewidth{0.481800pt}%
\definecolor{currentstroke}{rgb}{1.000000,1.000000,1.000000}%
\pgfsetstrokecolor{currentstroke}%
\pgfsetdash{}{0pt}%
\pgfpathmoveto{\pgfqpoint{1.536500in}{2.189662in}}%
\pgfpathcurveto{\pgfqpoint{1.547550in}{2.189662in}}{\pgfqpoint{1.558149in}{2.194053in}}{\pgfqpoint{1.565963in}{2.201866in}}%
\pgfpathcurveto{\pgfqpoint{1.573777in}{2.209680in}}{\pgfqpoint{1.578167in}{2.220279in}}{\pgfqpoint{1.578167in}{2.231329in}}%
\pgfpathcurveto{\pgfqpoint{1.578167in}{2.242379in}}{\pgfqpoint{1.573777in}{2.252978in}}{\pgfqpoint{1.565963in}{2.260792in}}%
\pgfpathcurveto{\pgfqpoint{1.558149in}{2.268605in}}{\pgfqpoint{1.547550in}{2.272996in}}{\pgfqpoint{1.536500in}{2.272996in}}%
\pgfpathcurveto{\pgfqpoint{1.525450in}{2.272996in}}{\pgfqpoint{1.514851in}{2.268605in}}{\pgfqpoint{1.507038in}{2.260792in}}%
\pgfpathcurveto{\pgfqpoint{1.499224in}{2.252978in}}{\pgfqpoint{1.494834in}{2.242379in}}{\pgfqpoint{1.494834in}{2.231329in}}%
\pgfpathcurveto{\pgfqpoint{1.494834in}{2.220279in}}{\pgfqpoint{1.499224in}{2.209680in}}{\pgfqpoint{1.507038in}{2.201866in}}%
\pgfpathcurveto{\pgfqpoint{1.514851in}{2.194053in}}{\pgfqpoint{1.525450in}{2.189662in}}{\pgfqpoint{1.536500in}{2.189662in}}%
\pgfpathclose%
\pgfusepath{stroke,fill}%
\end{pgfscope}%
\begin{pgfscope}%
\pgfpathrectangle{\pgfqpoint{0.481978in}{0.331635in}}{\pgfqpoint{9.300000in}{7.700000in}}%
\pgfusepath{clip}%
\pgfsetbuttcap%
\pgfsetroundjoin%
\definecolor{currentfill}{rgb}{1.000000,0.705882,0.509804}%
\pgfsetfillcolor{currentfill}%
\pgfsetlinewidth{0.481800pt}%
\definecolor{currentstroke}{rgb}{1.000000,1.000000,1.000000}%
\pgfsetstrokecolor{currentstroke}%
\pgfsetdash{}{0pt}%
\pgfpathmoveto{\pgfqpoint{5.550194in}{2.720047in}}%
\pgfpathcurveto{\pgfqpoint{5.561244in}{2.720047in}}{\pgfqpoint{5.571843in}{2.724438in}}{\pgfqpoint{5.579657in}{2.732251in}}%
\pgfpathcurveto{\pgfqpoint{5.587471in}{2.740065in}}{\pgfqpoint{5.591861in}{2.750664in}}{\pgfqpoint{5.591861in}{2.761714in}}%
\pgfpathcurveto{\pgfqpoint{5.591861in}{2.772764in}}{\pgfqpoint{5.587471in}{2.783363in}}{\pgfqpoint{5.579657in}{2.791177in}}%
\pgfpathcurveto{\pgfqpoint{5.571843in}{2.798990in}}{\pgfqpoint{5.561244in}{2.803381in}}{\pgfqpoint{5.550194in}{2.803381in}}%
\pgfpathcurveto{\pgfqpoint{5.539144in}{2.803381in}}{\pgfqpoint{5.528545in}{2.798990in}}{\pgfqpoint{5.520731in}{2.791177in}}%
\pgfpathcurveto{\pgfqpoint{5.512918in}{2.783363in}}{\pgfqpoint{5.508527in}{2.772764in}}{\pgfqpoint{5.508527in}{2.761714in}}%
\pgfpathcurveto{\pgfqpoint{5.508527in}{2.750664in}}{\pgfqpoint{5.512918in}{2.740065in}}{\pgfqpoint{5.520731in}{2.732251in}}%
\pgfpathcurveto{\pgfqpoint{5.528545in}{2.724438in}}{\pgfqpoint{5.539144in}{2.720047in}}{\pgfqpoint{5.550194in}{2.720047in}}%
\pgfpathclose%
\pgfusepath{stroke,fill}%
\end{pgfscope}%
\begin{pgfscope}%
\pgfpathrectangle{\pgfqpoint{0.481978in}{0.331635in}}{\pgfqpoint{9.300000in}{7.700000in}}%
\pgfusepath{clip}%
\pgfsetbuttcap%
\pgfsetroundjoin%
\definecolor{currentfill}{rgb}{1.000000,0.705882,0.509804}%
\pgfsetfillcolor{currentfill}%
\pgfsetlinewidth{0.481800pt}%
\definecolor{currentstroke}{rgb}{1.000000,1.000000,1.000000}%
\pgfsetstrokecolor{currentstroke}%
\pgfsetdash{}{0pt}%
\pgfpathmoveto{\pgfqpoint{2.791505in}{3.043200in}}%
\pgfpathcurveto{\pgfqpoint{2.802555in}{3.043200in}}{\pgfqpoint{2.813154in}{3.047591in}}{\pgfqpoint{2.820967in}{3.055404in}}%
\pgfpathcurveto{\pgfqpoint{2.828781in}{3.063218in}}{\pgfqpoint{2.833171in}{3.073817in}}{\pgfqpoint{2.833171in}{3.084867in}}%
\pgfpathcurveto{\pgfqpoint{2.833171in}{3.095917in}}{\pgfqpoint{2.828781in}{3.106516in}}{\pgfqpoint{2.820967in}{3.114330in}}%
\pgfpathcurveto{\pgfqpoint{2.813154in}{3.122144in}}{\pgfqpoint{2.802555in}{3.126534in}}{\pgfqpoint{2.791505in}{3.126534in}}%
\pgfpathcurveto{\pgfqpoint{2.780455in}{3.126534in}}{\pgfqpoint{2.769856in}{3.122144in}}{\pgfqpoint{2.762042in}{3.114330in}}%
\pgfpathcurveto{\pgfqpoint{2.754228in}{3.106516in}}{\pgfqpoint{2.749838in}{3.095917in}}{\pgfqpoint{2.749838in}{3.084867in}}%
\pgfpathcurveto{\pgfqpoint{2.749838in}{3.073817in}}{\pgfqpoint{2.754228in}{3.063218in}}{\pgfqpoint{2.762042in}{3.055404in}}%
\pgfpathcurveto{\pgfqpoint{2.769856in}{3.047591in}}{\pgfqpoint{2.780455in}{3.043200in}}{\pgfqpoint{2.791505in}{3.043200in}}%
\pgfpathclose%
\pgfusepath{stroke,fill}%
\end{pgfscope}%
\begin{pgfscope}%
\pgfpathrectangle{\pgfqpoint{0.481978in}{0.331635in}}{\pgfqpoint{9.300000in}{7.700000in}}%
\pgfusepath{clip}%
\pgfsetbuttcap%
\pgfsetroundjoin%
\definecolor{currentfill}{rgb}{0.631373,0.788235,0.956863}%
\pgfsetfillcolor{currentfill}%
\pgfsetlinewidth{1.003750pt}%
\definecolor{currentstroke}{rgb}{0.631373,0.788235,0.956863}%
\pgfsetstrokecolor{currentstroke}%
\pgfsetdash{}{0pt}%
\pgfsys@defobject{currentmarker}{\pgfqpoint{-0.041667in}{-0.041667in}}{\pgfqpoint{0.041667in}{0.041667in}}{%
\pgfpathmoveto{\pgfqpoint{0.000000in}{-0.041667in}}%
\pgfpathcurveto{\pgfqpoint{0.011050in}{-0.041667in}}{\pgfqpoint{0.021649in}{-0.037276in}}{\pgfqpoint{0.029463in}{-0.029463in}}%
\pgfpathcurveto{\pgfqpoint{0.037276in}{-0.021649in}}{\pgfqpoint{0.041667in}{-0.011050in}}{\pgfqpoint{0.041667in}{0.000000in}}%
\pgfpathcurveto{\pgfqpoint{0.041667in}{0.011050in}}{\pgfqpoint{0.037276in}{0.021649in}}{\pgfqpoint{0.029463in}{0.029463in}}%
\pgfpathcurveto{\pgfqpoint{0.021649in}{0.037276in}}{\pgfqpoint{0.011050in}{0.041667in}}{\pgfqpoint{0.000000in}{0.041667in}}%
\pgfpathcurveto{\pgfqpoint{-0.011050in}{0.041667in}}{\pgfqpoint{-0.021649in}{0.037276in}}{\pgfqpoint{-0.029463in}{0.029463in}}%
\pgfpathcurveto{\pgfqpoint{-0.037276in}{0.021649in}}{\pgfqpoint{-0.041667in}{0.011050in}}{\pgfqpoint{-0.041667in}{0.000000in}}%
\pgfpathcurveto{\pgfqpoint{-0.041667in}{-0.011050in}}{\pgfqpoint{-0.037276in}{-0.021649in}}{\pgfqpoint{-0.029463in}{-0.029463in}}%
\pgfpathcurveto{\pgfqpoint{-0.021649in}{-0.037276in}}{\pgfqpoint{-0.011050in}{-0.041667in}}{\pgfqpoint{0.000000in}{-0.041667in}}%
\pgfpathclose%
\pgfusepath{stroke,fill}%
}%
\end{pgfscope}%
\begin{pgfscope}%
\pgfpathrectangle{\pgfqpoint{0.481978in}{0.331635in}}{\pgfqpoint{9.300000in}{7.700000in}}%
\pgfusepath{clip}%
\pgfsetbuttcap%
\pgfsetroundjoin%
\definecolor{currentfill}{rgb}{1.000000,0.705882,0.509804}%
\pgfsetfillcolor{currentfill}%
\pgfsetlinewidth{1.003750pt}%
\definecolor{currentstroke}{rgb}{1.000000,0.705882,0.509804}%
\pgfsetstrokecolor{currentstroke}%
\pgfsetdash{}{0pt}%
\pgfsys@defobject{currentmarker}{\pgfqpoint{-0.041667in}{-0.041667in}}{\pgfqpoint{0.041667in}{0.041667in}}{%
\pgfpathmoveto{\pgfqpoint{0.000000in}{-0.041667in}}%
\pgfpathcurveto{\pgfqpoint{0.011050in}{-0.041667in}}{\pgfqpoint{0.021649in}{-0.037276in}}{\pgfqpoint{0.029463in}{-0.029463in}}%
\pgfpathcurveto{\pgfqpoint{0.037276in}{-0.021649in}}{\pgfqpoint{0.041667in}{-0.011050in}}{\pgfqpoint{0.041667in}{0.000000in}}%
\pgfpathcurveto{\pgfqpoint{0.041667in}{0.011050in}}{\pgfqpoint{0.037276in}{0.021649in}}{\pgfqpoint{0.029463in}{0.029463in}}%
\pgfpathcurveto{\pgfqpoint{0.021649in}{0.037276in}}{\pgfqpoint{0.011050in}{0.041667in}}{\pgfqpoint{0.000000in}{0.041667in}}%
\pgfpathcurveto{\pgfqpoint{-0.011050in}{0.041667in}}{\pgfqpoint{-0.021649in}{0.037276in}}{\pgfqpoint{-0.029463in}{0.029463in}}%
\pgfpathcurveto{\pgfqpoint{-0.037276in}{0.021649in}}{\pgfqpoint{-0.041667in}{0.011050in}}{\pgfqpoint{-0.041667in}{0.000000in}}%
\pgfpathcurveto{\pgfqpoint{-0.041667in}{-0.011050in}}{\pgfqpoint{-0.037276in}{-0.021649in}}{\pgfqpoint{-0.029463in}{-0.029463in}}%
\pgfpathcurveto{\pgfqpoint{-0.021649in}{-0.037276in}}{\pgfqpoint{-0.011050in}{-0.041667in}}{\pgfqpoint{0.000000in}{-0.041667in}}%
\pgfpathclose%
\pgfusepath{stroke,fill}%
}%
\end{pgfscope}%
\begin{pgfscope}%
\pgfsetbuttcap%
\pgfsetroundjoin%
\definecolor{currentfill}{rgb}{0.000000,0.000000,0.000000}%
\pgfsetfillcolor{currentfill}%
\pgfsetlinewidth{0.803000pt}%
\definecolor{currentstroke}{rgb}{0.000000,0.000000,0.000000}%
\pgfsetstrokecolor{currentstroke}%
\pgfsetdash{}{0pt}%
\pgfsys@defobject{currentmarker}{\pgfqpoint{0.000000in}{-0.048611in}}{\pgfqpoint{0.000000in}{0.000000in}}{%
\pgfpathmoveto{\pgfqpoint{0.000000in}{0.000000in}}%
\pgfpathlineto{\pgfqpoint{0.000000in}{-0.048611in}}%
\pgfusepath{stroke,fill}%
}%
\begin{pgfscope}%
\pgfsys@transformshift{1.241160in}{0.331635in}%
\pgfsys@useobject{currentmarker}{}%
\end{pgfscope}%
\end{pgfscope}%
\begin{pgfscope}%
\definecolor{textcolor}{rgb}{0.000000,0.000000,0.000000}%
\pgfsetstrokecolor{textcolor}%
\pgfsetfillcolor{textcolor}%
\pgftext[x=1.241160in,y=0.234413in,,top]{\color{textcolor}\sffamily\fontsize{10.000000}{12.000000}\selectfont \ensuremath{-}60}%
\end{pgfscope}%
\begin{pgfscope}%
\pgfsetbuttcap%
\pgfsetroundjoin%
\definecolor{currentfill}{rgb}{0.000000,0.000000,0.000000}%
\pgfsetfillcolor{currentfill}%
\pgfsetlinewidth{0.803000pt}%
\definecolor{currentstroke}{rgb}{0.000000,0.000000,0.000000}%
\pgfsetstrokecolor{currentstroke}%
\pgfsetdash{}{0pt}%
\pgfsys@defobject{currentmarker}{\pgfqpoint{0.000000in}{-0.048611in}}{\pgfqpoint{0.000000in}{0.000000in}}{%
\pgfpathmoveto{\pgfqpoint{0.000000in}{0.000000in}}%
\pgfpathlineto{\pgfqpoint{0.000000in}{-0.048611in}}%
\pgfusepath{stroke,fill}%
}%
\begin{pgfscope}%
\pgfsys@transformshift{2.528283in}{0.331635in}%
\pgfsys@useobject{currentmarker}{}%
\end{pgfscope}%
\end{pgfscope}%
\begin{pgfscope}%
\definecolor{textcolor}{rgb}{0.000000,0.000000,0.000000}%
\pgfsetstrokecolor{textcolor}%
\pgfsetfillcolor{textcolor}%
\pgftext[x=2.528283in,y=0.234413in,,top]{\color{textcolor}\sffamily\fontsize{10.000000}{12.000000}\selectfont \ensuremath{-}40}%
\end{pgfscope}%
\begin{pgfscope}%
\pgfsetbuttcap%
\pgfsetroundjoin%
\definecolor{currentfill}{rgb}{0.000000,0.000000,0.000000}%
\pgfsetfillcolor{currentfill}%
\pgfsetlinewidth{0.803000pt}%
\definecolor{currentstroke}{rgb}{0.000000,0.000000,0.000000}%
\pgfsetstrokecolor{currentstroke}%
\pgfsetdash{}{0pt}%
\pgfsys@defobject{currentmarker}{\pgfqpoint{0.000000in}{-0.048611in}}{\pgfqpoint{0.000000in}{0.000000in}}{%
\pgfpathmoveto{\pgfqpoint{0.000000in}{0.000000in}}%
\pgfpathlineto{\pgfqpoint{0.000000in}{-0.048611in}}%
\pgfusepath{stroke,fill}%
}%
\begin{pgfscope}%
\pgfsys@transformshift{3.815406in}{0.331635in}%
\pgfsys@useobject{currentmarker}{}%
\end{pgfscope}%
\end{pgfscope}%
\begin{pgfscope}%
\definecolor{textcolor}{rgb}{0.000000,0.000000,0.000000}%
\pgfsetstrokecolor{textcolor}%
\pgfsetfillcolor{textcolor}%
\pgftext[x=3.815406in,y=0.234413in,,top]{\color{textcolor}\sffamily\fontsize{10.000000}{12.000000}\selectfont \ensuremath{-}20}%
\end{pgfscope}%
\begin{pgfscope}%
\pgfsetbuttcap%
\pgfsetroundjoin%
\definecolor{currentfill}{rgb}{0.000000,0.000000,0.000000}%
\pgfsetfillcolor{currentfill}%
\pgfsetlinewidth{0.803000pt}%
\definecolor{currentstroke}{rgb}{0.000000,0.000000,0.000000}%
\pgfsetstrokecolor{currentstroke}%
\pgfsetdash{}{0pt}%
\pgfsys@defobject{currentmarker}{\pgfqpoint{0.000000in}{-0.048611in}}{\pgfqpoint{0.000000in}{0.000000in}}{%
\pgfpathmoveto{\pgfqpoint{0.000000in}{0.000000in}}%
\pgfpathlineto{\pgfqpoint{0.000000in}{-0.048611in}}%
\pgfusepath{stroke,fill}%
}%
\begin{pgfscope}%
\pgfsys@transformshift{5.102528in}{0.331635in}%
\pgfsys@useobject{currentmarker}{}%
\end{pgfscope}%
\end{pgfscope}%
\begin{pgfscope}%
\definecolor{textcolor}{rgb}{0.000000,0.000000,0.000000}%
\pgfsetstrokecolor{textcolor}%
\pgfsetfillcolor{textcolor}%
\pgftext[x=5.102528in,y=0.234413in,,top]{\color{textcolor}\sffamily\fontsize{10.000000}{12.000000}\selectfont 0}%
\end{pgfscope}%
\begin{pgfscope}%
\pgfsetbuttcap%
\pgfsetroundjoin%
\definecolor{currentfill}{rgb}{0.000000,0.000000,0.000000}%
\pgfsetfillcolor{currentfill}%
\pgfsetlinewidth{0.803000pt}%
\definecolor{currentstroke}{rgb}{0.000000,0.000000,0.000000}%
\pgfsetstrokecolor{currentstroke}%
\pgfsetdash{}{0pt}%
\pgfsys@defobject{currentmarker}{\pgfqpoint{0.000000in}{-0.048611in}}{\pgfqpoint{0.000000in}{0.000000in}}{%
\pgfpathmoveto{\pgfqpoint{0.000000in}{0.000000in}}%
\pgfpathlineto{\pgfqpoint{0.000000in}{-0.048611in}}%
\pgfusepath{stroke,fill}%
}%
\begin{pgfscope}%
\pgfsys@transformshift{6.389651in}{0.331635in}%
\pgfsys@useobject{currentmarker}{}%
\end{pgfscope}%
\end{pgfscope}%
\begin{pgfscope}%
\definecolor{textcolor}{rgb}{0.000000,0.000000,0.000000}%
\pgfsetstrokecolor{textcolor}%
\pgfsetfillcolor{textcolor}%
\pgftext[x=6.389651in,y=0.234413in,,top]{\color{textcolor}\sffamily\fontsize{10.000000}{12.000000}\selectfont 20}%
\end{pgfscope}%
\begin{pgfscope}%
\pgfsetbuttcap%
\pgfsetroundjoin%
\definecolor{currentfill}{rgb}{0.000000,0.000000,0.000000}%
\pgfsetfillcolor{currentfill}%
\pgfsetlinewidth{0.803000pt}%
\definecolor{currentstroke}{rgb}{0.000000,0.000000,0.000000}%
\pgfsetstrokecolor{currentstroke}%
\pgfsetdash{}{0pt}%
\pgfsys@defobject{currentmarker}{\pgfqpoint{0.000000in}{-0.048611in}}{\pgfqpoint{0.000000in}{0.000000in}}{%
\pgfpathmoveto{\pgfqpoint{0.000000in}{0.000000in}}%
\pgfpathlineto{\pgfqpoint{0.000000in}{-0.048611in}}%
\pgfusepath{stroke,fill}%
}%
\begin{pgfscope}%
\pgfsys@transformshift{7.676774in}{0.331635in}%
\pgfsys@useobject{currentmarker}{}%
\end{pgfscope}%
\end{pgfscope}%
\begin{pgfscope}%
\definecolor{textcolor}{rgb}{0.000000,0.000000,0.000000}%
\pgfsetstrokecolor{textcolor}%
\pgfsetfillcolor{textcolor}%
\pgftext[x=7.676774in,y=0.234413in,,top]{\color{textcolor}\sffamily\fontsize{10.000000}{12.000000}\selectfont 40}%
\end{pgfscope}%
\begin{pgfscope}%
\pgfsetbuttcap%
\pgfsetroundjoin%
\definecolor{currentfill}{rgb}{0.000000,0.000000,0.000000}%
\pgfsetfillcolor{currentfill}%
\pgfsetlinewidth{0.803000pt}%
\definecolor{currentstroke}{rgb}{0.000000,0.000000,0.000000}%
\pgfsetstrokecolor{currentstroke}%
\pgfsetdash{}{0pt}%
\pgfsys@defobject{currentmarker}{\pgfqpoint{0.000000in}{-0.048611in}}{\pgfqpoint{0.000000in}{0.000000in}}{%
\pgfpathmoveto{\pgfqpoint{0.000000in}{0.000000in}}%
\pgfpathlineto{\pgfqpoint{0.000000in}{-0.048611in}}%
\pgfusepath{stroke,fill}%
}%
\begin{pgfscope}%
\pgfsys@transformshift{8.963897in}{0.331635in}%
\pgfsys@useobject{currentmarker}{}%
\end{pgfscope}%
\end{pgfscope}%
\begin{pgfscope}%
\definecolor{textcolor}{rgb}{0.000000,0.000000,0.000000}%
\pgfsetstrokecolor{textcolor}%
\pgfsetfillcolor{textcolor}%
\pgftext[x=8.963897in,y=0.234413in,,top]{\color{textcolor}\sffamily\fontsize{10.000000}{12.000000}\selectfont 60}%
\end{pgfscope}%
\begin{pgfscope}%
\pgfsetbuttcap%
\pgfsetroundjoin%
\definecolor{currentfill}{rgb}{0.000000,0.000000,0.000000}%
\pgfsetfillcolor{currentfill}%
\pgfsetlinewidth{0.803000pt}%
\definecolor{currentstroke}{rgb}{0.000000,0.000000,0.000000}%
\pgfsetstrokecolor{currentstroke}%
\pgfsetdash{}{0pt}%
\pgfsys@defobject{currentmarker}{\pgfqpoint{-0.048611in}{0.000000in}}{\pgfqpoint{-0.000000in}{0.000000in}}{%
\pgfpathmoveto{\pgfqpoint{-0.000000in}{0.000000in}}%
\pgfpathlineto{\pgfqpoint{-0.048611in}{0.000000in}}%
\pgfusepath{stroke,fill}%
}%
\begin{pgfscope}%
\pgfsys@transformshift{0.481978in}{0.586442in}%
\pgfsys@useobject{currentmarker}{}%
\end{pgfscope}%
\end{pgfscope}%
\begin{pgfscope}%
\definecolor{textcolor}{rgb}{0.000000,0.000000,0.000000}%
\pgfsetstrokecolor{textcolor}%
\pgfsetfillcolor{textcolor}%
\pgftext[x=0.100000in, y=0.533681in, left, base]{\color{textcolor}\sffamily\fontsize{10.000000}{12.000000}\selectfont \ensuremath{-}60}%
\end{pgfscope}%
\begin{pgfscope}%
\pgfsetbuttcap%
\pgfsetroundjoin%
\definecolor{currentfill}{rgb}{0.000000,0.000000,0.000000}%
\pgfsetfillcolor{currentfill}%
\pgfsetlinewidth{0.803000pt}%
\definecolor{currentstroke}{rgb}{0.000000,0.000000,0.000000}%
\pgfsetstrokecolor{currentstroke}%
\pgfsetdash{}{0pt}%
\pgfsys@defobject{currentmarker}{\pgfqpoint{-0.048611in}{0.000000in}}{\pgfqpoint{-0.000000in}{0.000000in}}{%
\pgfpathmoveto{\pgfqpoint{-0.000000in}{0.000000in}}%
\pgfpathlineto{\pgfqpoint{-0.048611in}{0.000000in}}%
\pgfusepath{stroke,fill}%
}%
\begin{pgfscope}%
\pgfsys@transformshift{0.481978in}{1.784019in}%
\pgfsys@useobject{currentmarker}{}%
\end{pgfscope}%
\end{pgfscope}%
\begin{pgfscope}%
\definecolor{textcolor}{rgb}{0.000000,0.000000,0.000000}%
\pgfsetstrokecolor{textcolor}%
\pgfsetfillcolor{textcolor}%
\pgftext[x=0.100000in, y=1.731258in, left, base]{\color{textcolor}\sffamily\fontsize{10.000000}{12.000000}\selectfont \ensuremath{-}40}%
\end{pgfscope}%
\begin{pgfscope}%
\pgfsetbuttcap%
\pgfsetroundjoin%
\definecolor{currentfill}{rgb}{0.000000,0.000000,0.000000}%
\pgfsetfillcolor{currentfill}%
\pgfsetlinewidth{0.803000pt}%
\definecolor{currentstroke}{rgb}{0.000000,0.000000,0.000000}%
\pgfsetstrokecolor{currentstroke}%
\pgfsetdash{}{0pt}%
\pgfsys@defobject{currentmarker}{\pgfqpoint{-0.048611in}{0.000000in}}{\pgfqpoint{-0.000000in}{0.000000in}}{%
\pgfpathmoveto{\pgfqpoint{-0.000000in}{0.000000in}}%
\pgfpathlineto{\pgfqpoint{-0.048611in}{0.000000in}}%
\pgfusepath{stroke,fill}%
}%
\begin{pgfscope}%
\pgfsys@transformshift{0.481978in}{2.981597in}%
\pgfsys@useobject{currentmarker}{}%
\end{pgfscope}%
\end{pgfscope}%
\begin{pgfscope}%
\definecolor{textcolor}{rgb}{0.000000,0.000000,0.000000}%
\pgfsetstrokecolor{textcolor}%
\pgfsetfillcolor{textcolor}%
\pgftext[x=0.100000in, y=2.928835in, left, base]{\color{textcolor}\sffamily\fontsize{10.000000}{12.000000}\selectfont \ensuremath{-}20}%
\end{pgfscope}%
\begin{pgfscope}%
\pgfsetbuttcap%
\pgfsetroundjoin%
\definecolor{currentfill}{rgb}{0.000000,0.000000,0.000000}%
\pgfsetfillcolor{currentfill}%
\pgfsetlinewidth{0.803000pt}%
\definecolor{currentstroke}{rgb}{0.000000,0.000000,0.000000}%
\pgfsetstrokecolor{currentstroke}%
\pgfsetdash{}{0pt}%
\pgfsys@defobject{currentmarker}{\pgfqpoint{-0.048611in}{0.000000in}}{\pgfqpoint{-0.000000in}{0.000000in}}{%
\pgfpathmoveto{\pgfqpoint{-0.000000in}{0.000000in}}%
\pgfpathlineto{\pgfqpoint{-0.048611in}{0.000000in}}%
\pgfusepath{stroke,fill}%
}%
\begin{pgfscope}%
\pgfsys@transformshift{0.481978in}{4.179174in}%
\pgfsys@useobject{currentmarker}{}%
\end{pgfscope}%
\end{pgfscope}%
\begin{pgfscope}%
\definecolor{textcolor}{rgb}{0.000000,0.000000,0.000000}%
\pgfsetstrokecolor{textcolor}%
\pgfsetfillcolor{textcolor}%
\pgftext[x=0.296390in, y=4.126412in, left, base]{\color{textcolor}\sffamily\fontsize{10.000000}{12.000000}\selectfont 0}%
\end{pgfscope}%
\begin{pgfscope}%
\pgfsetbuttcap%
\pgfsetroundjoin%
\definecolor{currentfill}{rgb}{0.000000,0.000000,0.000000}%
\pgfsetfillcolor{currentfill}%
\pgfsetlinewidth{0.803000pt}%
\definecolor{currentstroke}{rgb}{0.000000,0.000000,0.000000}%
\pgfsetstrokecolor{currentstroke}%
\pgfsetdash{}{0pt}%
\pgfsys@defobject{currentmarker}{\pgfqpoint{-0.048611in}{0.000000in}}{\pgfqpoint{-0.000000in}{0.000000in}}{%
\pgfpathmoveto{\pgfqpoint{-0.000000in}{0.000000in}}%
\pgfpathlineto{\pgfqpoint{-0.048611in}{0.000000in}}%
\pgfusepath{stroke,fill}%
}%
\begin{pgfscope}%
\pgfsys@transformshift{0.481978in}{5.376751in}%
\pgfsys@useobject{currentmarker}{}%
\end{pgfscope}%
\end{pgfscope}%
\begin{pgfscope}%
\definecolor{textcolor}{rgb}{0.000000,0.000000,0.000000}%
\pgfsetstrokecolor{textcolor}%
\pgfsetfillcolor{textcolor}%
\pgftext[x=0.208025in, y=5.323990in, left, base]{\color{textcolor}\sffamily\fontsize{10.000000}{12.000000}\selectfont 20}%
\end{pgfscope}%
\begin{pgfscope}%
\pgfsetbuttcap%
\pgfsetroundjoin%
\definecolor{currentfill}{rgb}{0.000000,0.000000,0.000000}%
\pgfsetfillcolor{currentfill}%
\pgfsetlinewidth{0.803000pt}%
\definecolor{currentstroke}{rgb}{0.000000,0.000000,0.000000}%
\pgfsetstrokecolor{currentstroke}%
\pgfsetdash{}{0pt}%
\pgfsys@defobject{currentmarker}{\pgfqpoint{-0.048611in}{0.000000in}}{\pgfqpoint{-0.000000in}{0.000000in}}{%
\pgfpathmoveto{\pgfqpoint{-0.000000in}{0.000000in}}%
\pgfpathlineto{\pgfqpoint{-0.048611in}{0.000000in}}%
\pgfusepath{stroke,fill}%
}%
\begin{pgfscope}%
\pgfsys@transformshift{0.481978in}{6.574329in}%
\pgfsys@useobject{currentmarker}{}%
\end{pgfscope}%
\end{pgfscope}%
\begin{pgfscope}%
\definecolor{textcolor}{rgb}{0.000000,0.000000,0.000000}%
\pgfsetstrokecolor{textcolor}%
\pgfsetfillcolor{textcolor}%
\pgftext[x=0.208025in, y=6.521567in, left, base]{\color{textcolor}\sffamily\fontsize{10.000000}{12.000000}\selectfont 40}%
\end{pgfscope}%
\begin{pgfscope}%
\pgfsetbuttcap%
\pgfsetroundjoin%
\definecolor{currentfill}{rgb}{0.000000,0.000000,0.000000}%
\pgfsetfillcolor{currentfill}%
\pgfsetlinewidth{0.803000pt}%
\definecolor{currentstroke}{rgb}{0.000000,0.000000,0.000000}%
\pgfsetstrokecolor{currentstroke}%
\pgfsetdash{}{0pt}%
\pgfsys@defobject{currentmarker}{\pgfqpoint{-0.048611in}{0.000000in}}{\pgfqpoint{-0.000000in}{0.000000in}}{%
\pgfpathmoveto{\pgfqpoint{-0.000000in}{0.000000in}}%
\pgfpathlineto{\pgfqpoint{-0.048611in}{0.000000in}}%
\pgfusepath{stroke,fill}%
}%
\begin{pgfscope}%
\pgfsys@transformshift{0.481978in}{7.771906in}%
\pgfsys@useobject{currentmarker}{}%
\end{pgfscope}%
\end{pgfscope}%
\begin{pgfscope}%
\definecolor{textcolor}{rgb}{0.000000,0.000000,0.000000}%
\pgfsetstrokecolor{textcolor}%
\pgfsetfillcolor{textcolor}%
\pgftext[x=0.208025in, y=7.719144in, left, base]{\color{textcolor}\sffamily\fontsize{10.000000}{12.000000}\selectfont 60}%
\end{pgfscope}%
\begin{pgfscope}%
\pgfpathrectangle{\pgfqpoint{0.481978in}{0.331635in}}{\pgfqpoint{9.300000in}{7.700000in}}%
\pgfusepath{clip}%
\pgfsetrectcap%
\pgfsetroundjoin%
\pgfsetlinewidth{1.505625pt}%
\definecolor{currentstroke}{rgb}{0.631373,0.788235,0.956863}%
\pgfsetstrokecolor{currentstroke}%
\pgfsetstrokeopacity{0.800000}%
\pgfsetdash{}{0pt}%
\pgfpathmoveto{\pgfqpoint{6.274584in}{7.577001in}}%
\pgfpathlineto{\pgfqpoint{4.406284in}{4.599997in}}%
\pgfusepath{stroke}%
\end{pgfscope}%
\begin{pgfscope}%
\pgfpathrectangle{\pgfqpoint{0.481978in}{0.331635in}}{\pgfqpoint{9.300000in}{7.700000in}}%
\pgfusepath{clip}%
\pgfsetrectcap%
\pgfsetroundjoin%
\pgfsetlinewidth{1.505625pt}%
\definecolor{currentstroke}{rgb}{0.631373,0.788235,0.956863}%
\pgfsetstrokecolor{currentstroke}%
\pgfsetstrokeopacity{0.800000}%
\pgfsetdash{}{0pt}%
\pgfpathmoveto{\pgfqpoint{2.282958in}{5.085415in}}%
\pgfpathlineto{\pgfqpoint{4.406284in}{4.599997in}}%
\pgfusepath{stroke}%
\end{pgfscope}%
\begin{pgfscope}%
\pgfpathrectangle{\pgfqpoint{0.481978in}{0.331635in}}{\pgfqpoint{9.300000in}{7.700000in}}%
\pgfusepath{clip}%
\pgfsetrectcap%
\pgfsetroundjoin%
\pgfsetlinewidth{1.505625pt}%
\definecolor{currentstroke}{rgb}{0.631373,0.788235,0.956863}%
\pgfsetstrokecolor{currentstroke}%
\pgfsetstrokeopacity{0.800000}%
\pgfsetdash{}{0pt}%
\pgfpathmoveto{\pgfqpoint{4.404087in}{5.158664in}}%
\pgfpathlineto{\pgfqpoint{4.406284in}{4.599997in}}%
\pgfusepath{stroke}%
\end{pgfscope}%
\begin{pgfscope}%
\pgfpathrectangle{\pgfqpoint{0.481978in}{0.331635in}}{\pgfqpoint{9.300000in}{7.700000in}}%
\pgfusepath{clip}%
\pgfsetrectcap%
\pgfsetroundjoin%
\pgfsetlinewidth{1.505625pt}%
\definecolor{currentstroke}{rgb}{0.631373,0.788235,0.956863}%
\pgfsetstrokecolor{currentstroke}%
\pgfsetstrokeopacity{0.800000}%
\pgfsetdash{}{0pt}%
\pgfpathmoveto{\pgfqpoint{4.342376in}{5.656682in}}%
\pgfpathlineto{\pgfqpoint{4.406284in}{4.599997in}}%
\pgfusepath{stroke}%
\end{pgfscope}%
\begin{pgfscope}%
\pgfpathrectangle{\pgfqpoint{0.481978in}{0.331635in}}{\pgfqpoint{9.300000in}{7.700000in}}%
\pgfusepath{clip}%
\pgfsetrectcap%
\pgfsetroundjoin%
\pgfsetlinewidth{1.505625pt}%
\definecolor{currentstroke}{rgb}{0.631373,0.788235,0.956863}%
\pgfsetstrokecolor{currentstroke}%
\pgfsetstrokeopacity{0.800000}%
\pgfsetdash{}{0pt}%
\pgfpathmoveto{\pgfqpoint{6.817897in}{7.268314in}}%
\pgfpathlineto{\pgfqpoint{4.406284in}{4.599997in}}%
\pgfusepath{stroke}%
\end{pgfscope}%
\begin{pgfscope}%
\pgfpathrectangle{\pgfqpoint{0.481978in}{0.331635in}}{\pgfqpoint{9.300000in}{7.700000in}}%
\pgfusepath{clip}%
\pgfsetrectcap%
\pgfsetroundjoin%
\pgfsetlinewidth{1.505625pt}%
\definecolor{currentstroke}{rgb}{0.631373,0.788235,0.956863}%
\pgfsetstrokecolor{currentstroke}%
\pgfsetstrokeopacity{0.800000}%
\pgfsetdash{}{0pt}%
\pgfpathmoveto{\pgfqpoint{1.831950in}{3.967174in}}%
\pgfpathlineto{\pgfqpoint{4.406284in}{4.599997in}}%
\pgfusepath{stroke}%
\end{pgfscope}%
\begin{pgfscope}%
\pgfpathrectangle{\pgfqpoint{0.481978in}{0.331635in}}{\pgfqpoint{9.300000in}{7.700000in}}%
\pgfusepath{clip}%
\pgfsetrectcap%
\pgfsetroundjoin%
\pgfsetlinewidth{1.505625pt}%
\definecolor{currentstroke}{rgb}{0.631373,0.788235,0.956863}%
\pgfsetstrokecolor{currentstroke}%
\pgfsetstrokeopacity{0.800000}%
\pgfsetdash{}{0pt}%
\pgfpathmoveto{\pgfqpoint{6.806036in}{5.939655in}}%
\pgfpathlineto{\pgfqpoint{4.406284in}{4.599997in}}%
\pgfusepath{stroke}%
\end{pgfscope}%
\begin{pgfscope}%
\pgfpathrectangle{\pgfqpoint{0.481978in}{0.331635in}}{\pgfqpoint{9.300000in}{7.700000in}}%
\pgfusepath{clip}%
\pgfsetrectcap%
\pgfsetroundjoin%
\pgfsetlinewidth{1.505625pt}%
\definecolor{currentstroke}{rgb}{0.631373,0.788235,0.956863}%
\pgfsetstrokecolor{currentstroke}%
\pgfsetstrokeopacity{0.800000}%
\pgfsetdash{}{0pt}%
\pgfpathmoveto{\pgfqpoint{3.985143in}{5.392951in}}%
\pgfpathlineto{\pgfqpoint{4.406284in}{4.599997in}}%
\pgfusepath{stroke}%
\end{pgfscope}%
\begin{pgfscope}%
\pgfpathrectangle{\pgfqpoint{0.481978in}{0.331635in}}{\pgfqpoint{9.300000in}{7.700000in}}%
\pgfusepath{clip}%
\pgfsetrectcap%
\pgfsetroundjoin%
\pgfsetlinewidth{1.505625pt}%
\definecolor{currentstroke}{rgb}{0.631373,0.788235,0.956863}%
\pgfsetstrokecolor{currentstroke}%
\pgfsetstrokeopacity{0.800000}%
\pgfsetdash{}{0pt}%
\pgfpathmoveto{\pgfqpoint{4.751471in}{3.606159in}}%
\pgfpathlineto{\pgfqpoint{4.406284in}{4.599997in}}%
\pgfusepath{stroke}%
\end{pgfscope}%
\begin{pgfscope}%
\pgfpathrectangle{\pgfqpoint{0.481978in}{0.331635in}}{\pgfqpoint{9.300000in}{7.700000in}}%
\pgfusepath{clip}%
\pgfsetrectcap%
\pgfsetroundjoin%
\pgfsetlinewidth{1.505625pt}%
\definecolor{currentstroke}{rgb}{0.631373,0.788235,0.956863}%
\pgfsetstrokecolor{currentstroke}%
\pgfsetstrokeopacity{0.800000}%
\pgfsetdash{}{0pt}%
\pgfpathmoveto{\pgfqpoint{0.904705in}{3.349572in}}%
\pgfpathlineto{\pgfqpoint{4.406284in}{4.599997in}}%
\pgfusepath{stroke}%
\end{pgfscope}%
\begin{pgfscope}%
\pgfpathrectangle{\pgfqpoint{0.481978in}{0.331635in}}{\pgfqpoint{9.300000in}{7.700000in}}%
\pgfusepath{clip}%
\pgfsetrectcap%
\pgfsetroundjoin%
\pgfsetlinewidth{1.505625pt}%
\definecolor{currentstroke}{rgb}{0.631373,0.788235,0.956863}%
\pgfsetstrokecolor{currentstroke}%
\pgfsetstrokeopacity{0.800000}%
\pgfsetdash{}{0pt}%
\pgfpathmoveto{\pgfqpoint{5.294893in}{4.924228in}}%
\pgfpathlineto{\pgfqpoint{4.406284in}{4.599997in}}%
\pgfusepath{stroke}%
\end{pgfscope}%
\begin{pgfscope}%
\pgfpathrectangle{\pgfqpoint{0.481978in}{0.331635in}}{\pgfqpoint{9.300000in}{7.700000in}}%
\pgfusepath{clip}%
\pgfsetrectcap%
\pgfsetroundjoin%
\pgfsetlinewidth{1.505625pt}%
\definecolor{currentstroke}{rgb}{0.631373,0.788235,0.956863}%
\pgfsetstrokecolor{currentstroke}%
\pgfsetstrokeopacity{0.800000}%
\pgfsetdash{}{0pt}%
\pgfpathmoveto{\pgfqpoint{1.539916in}{3.206592in}}%
\pgfpathlineto{\pgfqpoint{4.406284in}{4.599997in}}%
\pgfusepath{stroke}%
\end{pgfscope}%
\begin{pgfscope}%
\pgfpathrectangle{\pgfqpoint{0.481978in}{0.331635in}}{\pgfqpoint{9.300000in}{7.700000in}}%
\pgfusepath{clip}%
\pgfsetrectcap%
\pgfsetroundjoin%
\pgfsetlinewidth{1.505625pt}%
\definecolor{currentstroke}{rgb}{0.631373,0.788235,0.956863}%
\pgfsetstrokecolor{currentstroke}%
\pgfsetstrokeopacity{0.800000}%
\pgfsetdash{}{0pt}%
\pgfpathmoveto{\pgfqpoint{6.009667in}{2.122892in}}%
\pgfpathlineto{\pgfqpoint{4.406284in}{4.599997in}}%
\pgfusepath{stroke}%
\end{pgfscope}%
\begin{pgfscope}%
\pgfpathrectangle{\pgfqpoint{0.481978in}{0.331635in}}{\pgfqpoint{9.300000in}{7.700000in}}%
\pgfusepath{clip}%
\pgfsetrectcap%
\pgfsetroundjoin%
\pgfsetlinewidth{1.505625pt}%
\definecolor{currentstroke}{rgb}{0.631373,0.788235,0.956863}%
\pgfsetstrokecolor{currentstroke}%
\pgfsetstrokeopacity{0.800000}%
\pgfsetdash{}{0pt}%
\pgfpathmoveto{\pgfqpoint{1.634066in}{4.427876in}}%
\pgfpathlineto{\pgfqpoint{4.406284in}{4.599997in}}%
\pgfusepath{stroke}%
\end{pgfscope}%
\begin{pgfscope}%
\pgfpathrectangle{\pgfqpoint{0.481978in}{0.331635in}}{\pgfqpoint{9.300000in}{7.700000in}}%
\pgfusepath{clip}%
\pgfsetrectcap%
\pgfsetroundjoin%
\pgfsetlinewidth{1.505625pt}%
\definecolor{currentstroke}{rgb}{0.631373,0.788235,0.956863}%
\pgfsetstrokecolor{currentstroke}%
\pgfsetstrokeopacity{0.800000}%
\pgfsetdash{}{0pt}%
\pgfpathmoveto{\pgfqpoint{4.786472in}{1.498044in}}%
\pgfpathlineto{\pgfqpoint{4.406284in}{4.599997in}}%
\pgfusepath{stroke}%
\end{pgfscope}%
\begin{pgfscope}%
\pgfpathrectangle{\pgfqpoint{0.481978in}{0.331635in}}{\pgfqpoint{9.300000in}{7.700000in}}%
\pgfusepath{clip}%
\pgfsetrectcap%
\pgfsetroundjoin%
\pgfsetlinewidth{1.505625pt}%
\definecolor{currentstroke}{rgb}{0.631373,0.788235,0.956863}%
\pgfsetstrokecolor{currentstroke}%
\pgfsetstrokeopacity{0.800000}%
\pgfsetdash{}{0pt}%
\pgfpathmoveto{\pgfqpoint{1.153250in}{2.919559in}}%
\pgfpathlineto{\pgfqpoint{4.406284in}{4.599997in}}%
\pgfusepath{stroke}%
\end{pgfscope}%
\begin{pgfscope}%
\pgfpathrectangle{\pgfqpoint{0.481978in}{0.331635in}}{\pgfqpoint{9.300000in}{7.700000in}}%
\pgfusepath{clip}%
\pgfsetrectcap%
\pgfsetroundjoin%
\pgfsetlinewidth{1.505625pt}%
\definecolor{currentstroke}{rgb}{0.631373,0.788235,0.956863}%
\pgfsetstrokecolor{currentstroke}%
\pgfsetstrokeopacity{0.800000}%
\pgfsetdash{}{0pt}%
\pgfpathmoveto{\pgfqpoint{2.365283in}{3.865110in}}%
\pgfpathlineto{\pgfqpoint{4.406284in}{4.599997in}}%
\pgfusepath{stroke}%
\end{pgfscope}%
\begin{pgfscope}%
\pgfpathrectangle{\pgfqpoint{0.481978in}{0.331635in}}{\pgfqpoint{9.300000in}{7.700000in}}%
\pgfusepath{clip}%
\pgfsetrectcap%
\pgfsetroundjoin%
\pgfsetlinewidth{1.505625pt}%
\definecolor{currentstroke}{rgb}{0.631373,0.788235,0.956863}%
\pgfsetstrokecolor{currentstroke}%
\pgfsetstrokeopacity{0.800000}%
\pgfsetdash{}{0pt}%
\pgfpathmoveto{\pgfqpoint{4.851746in}{2.518474in}}%
\pgfpathlineto{\pgfqpoint{4.406284in}{4.599997in}}%
\pgfusepath{stroke}%
\end{pgfscope}%
\begin{pgfscope}%
\pgfpathrectangle{\pgfqpoint{0.481978in}{0.331635in}}{\pgfqpoint{9.300000in}{7.700000in}}%
\pgfusepath{clip}%
\pgfsetrectcap%
\pgfsetroundjoin%
\pgfsetlinewidth{1.505625pt}%
\definecolor{currentstroke}{rgb}{0.631373,0.788235,0.956863}%
\pgfsetstrokecolor{currentstroke}%
\pgfsetstrokeopacity{0.800000}%
\pgfsetdash{}{0pt}%
\pgfpathmoveto{\pgfqpoint{7.001514in}{7.681635in}}%
\pgfpathlineto{\pgfqpoint{4.406284in}{4.599997in}}%
\pgfusepath{stroke}%
\end{pgfscope}%
\begin{pgfscope}%
\pgfpathrectangle{\pgfqpoint{0.481978in}{0.331635in}}{\pgfqpoint{9.300000in}{7.700000in}}%
\pgfusepath{clip}%
\pgfsetrectcap%
\pgfsetroundjoin%
\pgfsetlinewidth{1.505625pt}%
\definecolor{currentstroke}{rgb}{0.631373,0.788235,0.956863}%
\pgfsetstrokecolor{currentstroke}%
\pgfsetstrokeopacity{0.800000}%
\pgfsetdash{}{0pt}%
\pgfpathmoveto{\pgfqpoint{3.390380in}{6.657325in}}%
\pgfpathlineto{\pgfqpoint{4.406284in}{4.599997in}}%
\pgfusepath{stroke}%
\end{pgfscope}%
\begin{pgfscope}%
\pgfpathrectangle{\pgfqpoint{0.481978in}{0.331635in}}{\pgfqpoint{9.300000in}{7.700000in}}%
\pgfusepath{clip}%
\pgfsetrectcap%
\pgfsetroundjoin%
\pgfsetlinewidth{1.505625pt}%
\definecolor{currentstroke}{rgb}{0.631373,0.788235,0.956863}%
\pgfsetstrokecolor{currentstroke}%
\pgfsetstrokeopacity{0.800000}%
\pgfsetdash{}{0pt}%
\pgfpathmoveto{\pgfqpoint{5.562444in}{6.356212in}}%
\pgfpathlineto{\pgfqpoint{4.406284in}{4.599997in}}%
\pgfusepath{stroke}%
\end{pgfscope}%
\begin{pgfscope}%
\pgfpathrectangle{\pgfqpoint{0.481978in}{0.331635in}}{\pgfqpoint{9.300000in}{7.700000in}}%
\pgfusepath{clip}%
\pgfsetrectcap%
\pgfsetroundjoin%
\pgfsetlinewidth{1.505625pt}%
\definecolor{currentstroke}{rgb}{0.631373,0.788235,0.956863}%
\pgfsetstrokecolor{currentstroke}%
\pgfsetstrokeopacity{0.800000}%
\pgfsetdash{}{0pt}%
\pgfpathmoveto{\pgfqpoint{4.975949in}{5.368170in}}%
\pgfpathlineto{\pgfqpoint{4.406284in}{4.599997in}}%
\pgfusepath{stroke}%
\end{pgfscope}%
\begin{pgfscope}%
\pgfpathrectangle{\pgfqpoint{0.481978in}{0.331635in}}{\pgfqpoint{9.300000in}{7.700000in}}%
\pgfusepath{clip}%
\pgfsetrectcap%
\pgfsetroundjoin%
\pgfsetlinewidth{1.505625pt}%
\definecolor{currentstroke}{rgb}{0.631373,0.788235,0.956863}%
\pgfsetstrokecolor{currentstroke}%
\pgfsetstrokeopacity{0.800000}%
\pgfsetdash{}{0pt}%
\pgfpathmoveto{\pgfqpoint{4.416722in}{4.362775in}}%
\pgfpathlineto{\pgfqpoint{4.406284in}{4.599997in}}%
\pgfusepath{stroke}%
\end{pgfscope}%
\begin{pgfscope}%
\pgfpathrectangle{\pgfqpoint{0.481978in}{0.331635in}}{\pgfqpoint{9.300000in}{7.700000in}}%
\pgfusepath{clip}%
\pgfsetrectcap%
\pgfsetroundjoin%
\pgfsetlinewidth{1.505625pt}%
\definecolor{currentstroke}{rgb}{0.631373,0.788235,0.956863}%
\pgfsetstrokecolor{currentstroke}%
\pgfsetstrokeopacity{0.800000}%
\pgfsetdash{}{0pt}%
\pgfpathmoveto{\pgfqpoint{5.411196in}{3.830953in}}%
\pgfpathlineto{\pgfqpoint{4.406284in}{4.599997in}}%
\pgfusepath{stroke}%
\end{pgfscope}%
\begin{pgfscope}%
\pgfpathrectangle{\pgfqpoint{0.481978in}{0.331635in}}{\pgfqpoint{9.300000in}{7.700000in}}%
\pgfusepath{clip}%
\pgfsetrectcap%
\pgfsetroundjoin%
\pgfsetlinewidth{1.505625pt}%
\definecolor{currentstroke}{rgb}{0.631373,0.788235,0.956863}%
\pgfsetstrokecolor{currentstroke}%
\pgfsetstrokeopacity{0.800000}%
\pgfsetdash{}{0pt}%
\pgfpathmoveto{\pgfqpoint{6.107702in}{6.054436in}}%
\pgfpathlineto{\pgfqpoint{4.406284in}{4.599997in}}%
\pgfusepath{stroke}%
\end{pgfscope}%
\begin{pgfscope}%
\pgfpathrectangle{\pgfqpoint{0.481978in}{0.331635in}}{\pgfqpoint{9.300000in}{7.700000in}}%
\pgfusepath{clip}%
\pgfsetrectcap%
\pgfsetroundjoin%
\pgfsetlinewidth{1.505625pt}%
\definecolor{currentstroke}{rgb}{0.631373,0.788235,0.956863}%
\pgfsetstrokecolor{currentstroke}%
\pgfsetstrokeopacity{0.800000}%
\pgfsetdash{}{0pt}%
\pgfpathmoveto{\pgfqpoint{6.308017in}{7.215984in}}%
\pgfpathlineto{\pgfqpoint{4.406284in}{4.599997in}}%
\pgfusepath{stroke}%
\end{pgfscope}%
\begin{pgfscope}%
\pgfpathrectangle{\pgfqpoint{0.481978in}{0.331635in}}{\pgfqpoint{9.300000in}{7.700000in}}%
\pgfusepath{clip}%
\pgfsetrectcap%
\pgfsetroundjoin%
\pgfsetlinewidth{1.505625pt}%
\definecolor{currentstroke}{rgb}{0.631373,0.788235,0.956863}%
\pgfsetstrokecolor{currentstroke}%
\pgfsetstrokeopacity{0.800000}%
\pgfsetdash{}{0pt}%
\pgfpathmoveto{\pgfqpoint{1.614740in}{1.207036in}}%
\pgfpathlineto{\pgfqpoint{4.406284in}{4.599997in}}%
\pgfusepath{stroke}%
\end{pgfscope}%
\begin{pgfscope}%
\pgfpathrectangle{\pgfqpoint{0.481978in}{0.331635in}}{\pgfqpoint{9.300000in}{7.700000in}}%
\pgfusepath{clip}%
\pgfsetrectcap%
\pgfsetroundjoin%
\pgfsetlinewidth{1.505625pt}%
\definecolor{currentstroke}{rgb}{0.631373,0.788235,0.956863}%
\pgfsetstrokecolor{currentstroke}%
\pgfsetstrokeopacity{0.800000}%
\pgfsetdash{}{0pt}%
\pgfpathmoveto{\pgfqpoint{5.680299in}{7.022438in}}%
\pgfpathlineto{\pgfqpoint{4.406284in}{4.599997in}}%
\pgfusepath{stroke}%
\end{pgfscope}%
\begin{pgfscope}%
\pgfpathrectangle{\pgfqpoint{0.481978in}{0.331635in}}{\pgfqpoint{9.300000in}{7.700000in}}%
\pgfusepath{clip}%
\pgfsetrectcap%
\pgfsetroundjoin%
\pgfsetlinewidth{1.505625pt}%
\definecolor{currentstroke}{rgb}{0.631373,0.788235,0.956863}%
\pgfsetstrokecolor{currentstroke}%
\pgfsetstrokeopacity{0.800000}%
\pgfsetdash{}{0pt}%
\pgfpathmoveto{\pgfqpoint{4.473102in}{6.312086in}}%
\pgfpathlineto{\pgfqpoint{4.406284in}{4.599997in}}%
\pgfusepath{stroke}%
\end{pgfscope}%
\begin{pgfscope}%
\pgfpathrectangle{\pgfqpoint{0.481978in}{0.331635in}}{\pgfqpoint{9.300000in}{7.700000in}}%
\pgfusepath{clip}%
\pgfsetrectcap%
\pgfsetroundjoin%
\pgfsetlinewidth{1.505625pt}%
\definecolor{currentstroke}{rgb}{0.631373,0.788235,0.956863}%
\pgfsetstrokecolor{currentstroke}%
\pgfsetstrokeopacity{0.800000}%
\pgfsetdash{}{0pt}%
\pgfpathmoveto{\pgfqpoint{2.894322in}{3.587863in}}%
\pgfpathlineto{\pgfqpoint{4.406284in}{4.599997in}}%
\pgfusepath{stroke}%
\end{pgfscope}%
\begin{pgfscope}%
\pgfpathrectangle{\pgfqpoint{0.481978in}{0.331635in}}{\pgfqpoint{9.300000in}{7.700000in}}%
\pgfusepath{clip}%
\pgfsetrectcap%
\pgfsetroundjoin%
\pgfsetlinewidth{1.505625pt}%
\definecolor{currentstroke}{rgb}{0.631373,0.788235,0.956863}%
\pgfsetstrokecolor{currentstroke}%
\pgfsetstrokeopacity{0.800000}%
\pgfsetdash{}{0pt}%
\pgfpathmoveto{\pgfqpoint{2.835916in}{4.217239in}}%
\pgfpathlineto{\pgfqpoint{4.406284in}{4.599997in}}%
\pgfusepath{stroke}%
\end{pgfscope}%
\begin{pgfscope}%
\pgfpathrectangle{\pgfqpoint{0.481978in}{0.331635in}}{\pgfqpoint{9.300000in}{7.700000in}}%
\pgfusepath{clip}%
\pgfsetrectcap%
\pgfsetroundjoin%
\pgfsetlinewidth{1.505625pt}%
\definecolor{currentstroke}{rgb}{0.631373,0.788235,0.956863}%
\pgfsetstrokecolor{currentstroke}%
\pgfsetstrokeopacity{0.800000}%
\pgfsetdash{}{0pt}%
\pgfpathmoveto{\pgfqpoint{6.230328in}{6.642733in}}%
\pgfpathlineto{\pgfqpoint{4.406284in}{4.599997in}}%
\pgfusepath{stroke}%
\end{pgfscope}%
\begin{pgfscope}%
\pgfpathrectangle{\pgfqpoint{0.481978in}{0.331635in}}{\pgfqpoint{9.300000in}{7.700000in}}%
\pgfusepath{clip}%
\pgfsetrectcap%
\pgfsetroundjoin%
\pgfsetlinewidth{1.505625pt}%
\definecolor{currentstroke}{rgb}{0.631373,0.788235,0.956863}%
\pgfsetstrokecolor{currentstroke}%
\pgfsetstrokeopacity{0.800000}%
\pgfsetdash{}{0pt}%
\pgfpathmoveto{\pgfqpoint{4.009340in}{4.327348in}}%
\pgfpathlineto{\pgfqpoint{4.406284in}{4.599997in}}%
\pgfusepath{stroke}%
\end{pgfscope}%
\begin{pgfscope}%
\pgfpathrectangle{\pgfqpoint{0.481978in}{0.331635in}}{\pgfqpoint{9.300000in}{7.700000in}}%
\pgfusepath{clip}%
\pgfsetrectcap%
\pgfsetroundjoin%
\pgfsetlinewidth{1.505625pt}%
\definecolor{currentstroke}{rgb}{0.631373,0.788235,0.956863}%
\pgfsetstrokecolor{currentstroke}%
\pgfsetstrokeopacity{0.800000}%
\pgfsetdash{}{0pt}%
\pgfpathmoveto{\pgfqpoint{5.539390in}{5.725601in}}%
\pgfpathlineto{\pgfqpoint{4.406284in}{4.599997in}}%
\pgfusepath{stroke}%
\end{pgfscope}%
\begin{pgfscope}%
\pgfpathrectangle{\pgfqpoint{0.481978in}{0.331635in}}{\pgfqpoint{9.300000in}{7.700000in}}%
\pgfusepath{clip}%
\pgfsetrectcap%
\pgfsetroundjoin%
\pgfsetlinewidth{1.505625pt}%
\definecolor{currentstroke}{rgb}{0.631373,0.788235,0.956863}%
\pgfsetstrokecolor{currentstroke}%
\pgfsetstrokeopacity{0.800000}%
\pgfsetdash{}{0pt}%
\pgfpathmoveto{\pgfqpoint{2.431291in}{1.871162in}}%
\pgfpathlineto{\pgfqpoint{4.406284in}{4.599997in}}%
\pgfusepath{stroke}%
\end{pgfscope}%
\begin{pgfscope}%
\pgfpathrectangle{\pgfqpoint{0.481978in}{0.331635in}}{\pgfqpoint{9.300000in}{7.700000in}}%
\pgfusepath{clip}%
\pgfsetrectcap%
\pgfsetroundjoin%
\pgfsetlinewidth{1.505625pt}%
\definecolor{currentstroke}{rgb}{0.631373,0.788235,0.956863}%
\pgfsetstrokecolor{currentstroke}%
\pgfsetstrokeopacity{0.800000}%
\pgfsetdash{}{0pt}%
\pgfpathmoveto{\pgfqpoint{6.872681in}{1.459461in}}%
\pgfpathlineto{\pgfqpoint{4.406284in}{4.599997in}}%
\pgfusepath{stroke}%
\end{pgfscope}%
\begin{pgfscope}%
\pgfpathrectangle{\pgfqpoint{0.481978in}{0.331635in}}{\pgfqpoint{9.300000in}{7.700000in}}%
\pgfusepath{clip}%
\pgfsetrectcap%
\pgfsetroundjoin%
\pgfsetlinewidth{1.505625pt}%
\definecolor{currentstroke}{rgb}{0.631373,0.788235,0.956863}%
\pgfsetstrokecolor{currentstroke}%
\pgfsetstrokeopacity{0.800000}%
\pgfsetdash{}{0pt}%
\pgfpathmoveto{\pgfqpoint{6.738984in}{6.603737in}}%
\pgfpathlineto{\pgfqpoint{4.406284in}{4.599997in}}%
\pgfusepath{stroke}%
\end{pgfscope}%
\begin{pgfscope}%
\pgfpathrectangle{\pgfqpoint{0.481978in}{0.331635in}}{\pgfqpoint{9.300000in}{7.700000in}}%
\pgfusepath{clip}%
\pgfsetrectcap%
\pgfsetroundjoin%
\pgfsetlinewidth{1.505625pt}%
\definecolor{currentstroke}{rgb}{0.631373,0.788235,0.956863}%
\pgfsetstrokecolor{currentstroke}%
\pgfsetstrokeopacity{0.800000}%
\pgfsetdash{}{0pt}%
\pgfpathmoveto{\pgfqpoint{2.165107in}{3.374311in}}%
\pgfpathlineto{\pgfqpoint{4.406284in}{4.599997in}}%
\pgfusepath{stroke}%
\end{pgfscope}%
\begin{pgfscope}%
\pgfpathrectangle{\pgfqpoint{0.481978in}{0.331635in}}{\pgfqpoint{9.300000in}{7.700000in}}%
\pgfusepath{clip}%
\pgfsetrectcap%
\pgfsetroundjoin%
\pgfsetlinewidth{1.505625pt}%
\definecolor{currentstroke}{rgb}{0.631373,0.788235,0.956863}%
\pgfsetstrokecolor{currentstroke}%
\pgfsetstrokeopacity{0.800000}%
\pgfsetdash{}{0pt}%
\pgfpathmoveto{\pgfqpoint{5.720593in}{1.157318in}}%
\pgfpathlineto{\pgfqpoint{4.406284in}{4.599997in}}%
\pgfusepath{stroke}%
\end{pgfscope}%
\begin{pgfscope}%
\pgfpathrectangle{\pgfqpoint{0.481978in}{0.331635in}}{\pgfqpoint{9.300000in}{7.700000in}}%
\pgfusepath{clip}%
\pgfsetrectcap%
\pgfsetroundjoin%
\pgfsetlinewidth{1.505625pt}%
\definecolor{currentstroke}{rgb}{0.631373,0.788235,0.956863}%
\pgfsetstrokecolor{currentstroke}%
\pgfsetstrokeopacity{0.800000}%
\pgfsetdash{}{0pt}%
\pgfpathmoveto{\pgfqpoint{1.454724in}{5.360841in}}%
\pgfpathlineto{\pgfqpoint{4.406284in}{4.599997in}}%
\pgfusepath{stroke}%
\end{pgfscope}%
\begin{pgfscope}%
\pgfpathrectangle{\pgfqpoint{0.481978in}{0.331635in}}{\pgfqpoint{9.300000in}{7.700000in}}%
\pgfusepath{clip}%
\pgfsetrectcap%
\pgfsetroundjoin%
\pgfsetlinewidth{1.505625pt}%
\definecolor{currentstroke}{rgb}{0.631373,0.788235,0.956863}%
\pgfsetstrokecolor{currentstroke}%
\pgfsetstrokeopacity{0.800000}%
\pgfsetdash{}{0pt}%
\pgfpathmoveto{\pgfqpoint{4.584792in}{1.888678in}}%
\pgfpathlineto{\pgfqpoint{4.406284in}{4.599997in}}%
\pgfusepath{stroke}%
\end{pgfscope}%
\begin{pgfscope}%
\pgfpathrectangle{\pgfqpoint{0.481978in}{0.331635in}}{\pgfqpoint{9.300000in}{7.700000in}}%
\pgfusepath{clip}%
\pgfsetrectcap%
\pgfsetroundjoin%
\pgfsetlinewidth{1.505625pt}%
\definecolor{currentstroke}{rgb}{0.631373,0.788235,0.956863}%
\pgfsetstrokecolor{currentstroke}%
\pgfsetstrokeopacity{0.800000}%
\pgfsetdash{}{0pt}%
\pgfpathmoveto{\pgfqpoint{4.904799in}{5.911543in}}%
\pgfpathlineto{\pgfqpoint{4.406284in}{4.599997in}}%
\pgfusepath{stroke}%
\end{pgfscope}%
\begin{pgfscope}%
\pgfpathrectangle{\pgfqpoint{0.481978in}{0.331635in}}{\pgfqpoint{9.300000in}{7.700000in}}%
\pgfusepath{clip}%
\pgfsetrectcap%
\pgfsetroundjoin%
\pgfsetlinewidth{1.505625pt}%
\definecolor{currentstroke}{rgb}{0.631373,0.788235,0.956863}%
\pgfsetstrokecolor{currentstroke}%
\pgfsetstrokeopacity{0.800000}%
\pgfsetdash{}{0pt}%
\pgfpathmoveto{\pgfqpoint{4.893406in}{4.198314in}}%
\pgfpathlineto{\pgfqpoint{4.406284in}{4.599997in}}%
\pgfusepath{stroke}%
\end{pgfscope}%
\begin{pgfscope}%
\pgfpathrectangle{\pgfqpoint{0.481978in}{0.331635in}}{\pgfqpoint{9.300000in}{7.700000in}}%
\pgfusepath{clip}%
\pgfsetrectcap%
\pgfsetroundjoin%
\pgfsetlinewidth{1.505625pt}%
\definecolor{currentstroke}{rgb}{0.631373,0.788235,0.956863}%
\pgfsetstrokecolor{currentstroke}%
\pgfsetstrokeopacity{0.800000}%
\pgfsetdash{}{0pt}%
\pgfpathmoveto{\pgfqpoint{5.321160in}{7.614246in}}%
\pgfpathlineto{\pgfqpoint{4.406284in}{4.599997in}}%
\pgfusepath{stroke}%
\end{pgfscope}%
\begin{pgfscope}%
\pgfpathrectangle{\pgfqpoint{0.481978in}{0.331635in}}{\pgfqpoint{9.300000in}{7.700000in}}%
\pgfusepath{clip}%
\pgfsetrectcap%
\pgfsetroundjoin%
\pgfsetlinewidth{1.505625pt}%
\definecolor{currentstroke}{rgb}{0.631373,0.788235,0.956863}%
\pgfsetstrokecolor{currentstroke}%
\pgfsetstrokeopacity{0.800000}%
\pgfsetdash{}{0pt}%
\pgfpathmoveto{\pgfqpoint{5.212048in}{1.945545in}}%
\pgfpathlineto{\pgfqpoint{4.406284in}{4.599997in}}%
\pgfusepath{stroke}%
\end{pgfscope}%
\begin{pgfscope}%
\pgfpathrectangle{\pgfqpoint{0.481978in}{0.331635in}}{\pgfqpoint{9.300000in}{7.700000in}}%
\pgfusepath{clip}%
\pgfsetrectcap%
\pgfsetroundjoin%
\pgfsetlinewidth{1.505625pt}%
\definecolor{currentstroke}{rgb}{0.631373,0.788235,0.956863}%
\pgfsetstrokecolor{currentstroke}%
\pgfsetstrokeopacity{0.800000}%
\pgfsetdash{}{0pt}%
\pgfpathmoveto{\pgfqpoint{3.803419in}{5.877733in}}%
\pgfpathlineto{\pgfqpoint{4.406284in}{4.599997in}}%
\pgfusepath{stroke}%
\end{pgfscope}%
\begin{pgfscope}%
\pgfpathrectangle{\pgfqpoint{0.481978in}{0.331635in}}{\pgfqpoint{9.300000in}{7.700000in}}%
\pgfusepath{clip}%
\pgfsetrectcap%
\pgfsetroundjoin%
\pgfsetlinewidth{1.505625pt}%
\definecolor{currentstroke}{rgb}{0.631373,0.788235,0.956863}%
\pgfsetstrokecolor{currentstroke}%
\pgfsetstrokeopacity{0.800000}%
\pgfsetdash{}{0pt}%
\pgfpathmoveto{\pgfqpoint{4.981886in}{7.250985in}}%
\pgfpathlineto{\pgfqpoint{4.406284in}{4.599997in}}%
\pgfusepath{stroke}%
\end{pgfscope}%
\begin{pgfscope}%
\pgfpathrectangle{\pgfqpoint{0.481978in}{0.331635in}}{\pgfqpoint{9.300000in}{7.700000in}}%
\pgfusepath{clip}%
\pgfsetrectcap%
\pgfsetroundjoin%
\pgfsetlinewidth{1.505625pt}%
\definecolor{currentstroke}{rgb}{0.631373,0.788235,0.956863}%
\pgfsetstrokecolor{currentstroke}%
\pgfsetstrokeopacity{0.800000}%
\pgfsetdash{}{0pt}%
\pgfpathmoveto{\pgfqpoint{3.839349in}{3.911690in}}%
\pgfpathlineto{\pgfqpoint{4.406284in}{4.599997in}}%
\pgfusepath{stroke}%
\end{pgfscope}%
\begin{pgfscope}%
\pgfpathrectangle{\pgfqpoint{0.481978in}{0.331635in}}{\pgfqpoint{9.300000in}{7.700000in}}%
\pgfusepath{clip}%
\pgfsetrectcap%
\pgfsetroundjoin%
\pgfsetlinewidth{1.505625pt}%
\definecolor{currentstroke}{rgb}{0.631373,0.788235,0.956863}%
\pgfsetstrokecolor{currentstroke}%
\pgfsetstrokeopacity{0.800000}%
\pgfsetdash{}{0pt}%
\pgfpathmoveto{\pgfqpoint{6.849912in}{4.149791in}}%
\pgfpathlineto{\pgfqpoint{4.406284in}{4.599997in}}%
\pgfusepath{stroke}%
\end{pgfscope}%
\begin{pgfscope}%
\pgfpathrectangle{\pgfqpoint{0.481978in}{0.331635in}}{\pgfqpoint{9.300000in}{7.700000in}}%
\pgfusepath{clip}%
\pgfsetrectcap%
\pgfsetroundjoin%
\pgfsetlinewidth{1.505625pt}%
\definecolor{currentstroke}{rgb}{0.631373,0.788235,0.956863}%
\pgfsetstrokecolor{currentstroke}%
\pgfsetstrokeopacity{0.800000}%
\pgfsetdash{}{0pt}%
\pgfpathmoveto{\pgfqpoint{4.052180in}{2.370294in}}%
\pgfpathlineto{\pgfqpoint{4.406284in}{4.599997in}}%
\pgfusepath{stroke}%
\end{pgfscope}%
\begin{pgfscope}%
\pgfpathrectangle{\pgfqpoint{0.481978in}{0.331635in}}{\pgfqpoint{9.300000in}{7.700000in}}%
\pgfusepath{clip}%
\pgfsetrectcap%
\pgfsetroundjoin%
\pgfsetlinewidth{1.505625pt}%
\definecolor{currentstroke}{rgb}{1.000000,0.705882,0.509804}%
\pgfsetstrokecolor{currentstroke}%
\pgfsetstrokeopacity{0.800000}%
\pgfsetdash{}{0pt}%
\pgfpathmoveto{\pgfqpoint{6.020800in}{5.017054in}}%
\pgfpathlineto{\pgfqpoint{5.373961in}{3.920876in}}%
\pgfusepath{stroke}%
\end{pgfscope}%
\begin{pgfscope}%
\pgfpathrectangle{\pgfqpoint{0.481978in}{0.331635in}}{\pgfqpoint{9.300000in}{7.700000in}}%
\pgfusepath{clip}%
\pgfsetrectcap%
\pgfsetroundjoin%
\pgfsetlinewidth{1.505625pt}%
\definecolor{currentstroke}{rgb}{1.000000,0.705882,0.509804}%
\pgfsetstrokecolor{currentstroke}%
\pgfsetstrokeopacity{0.800000}%
\pgfsetdash{}{0pt}%
\pgfpathmoveto{\pgfqpoint{7.511979in}{7.163307in}}%
\pgfpathlineto{\pgfqpoint{5.373961in}{3.920876in}}%
\pgfusepath{stroke}%
\end{pgfscope}%
\begin{pgfscope}%
\pgfpathrectangle{\pgfqpoint{0.481978in}{0.331635in}}{\pgfqpoint{9.300000in}{7.700000in}}%
\pgfusepath{clip}%
\pgfsetrectcap%
\pgfsetroundjoin%
\pgfsetlinewidth{1.505625pt}%
\definecolor{currentstroke}{rgb}{1.000000,0.705882,0.509804}%
\pgfsetstrokecolor{currentstroke}%
\pgfsetstrokeopacity{0.800000}%
\pgfsetdash{}{0pt}%
\pgfpathmoveto{\pgfqpoint{6.803906in}{4.861506in}}%
\pgfpathlineto{\pgfqpoint{5.373961in}{3.920876in}}%
\pgfusepath{stroke}%
\end{pgfscope}%
\begin{pgfscope}%
\pgfpathrectangle{\pgfqpoint{0.481978in}{0.331635in}}{\pgfqpoint{9.300000in}{7.700000in}}%
\pgfusepath{clip}%
\pgfsetrectcap%
\pgfsetroundjoin%
\pgfsetlinewidth{1.505625pt}%
\definecolor{currentstroke}{rgb}{1.000000,0.705882,0.509804}%
\pgfsetstrokecolor{currentstroke}%
\pgfsetstrokeopacity{0.800000}%
\pgfsetdash{}{0pt}%
\pgfpathmoveto{\pgfqpoint{8.639111in}{5.466052in}}%
\pgfpathlineto{\pgfqpoint{5.373961in}{3.920876in}}%
\pgfusepath{stroke}%
\end{pgfscope}%
\begin{pgfscope}%
\pgfpathrectangle{\pgfqpoint{0.481978in}{0.331635in}}{\pgfqpoint{9.300000in}{7.700000in}}%
\pgfusepath{clip}%
\pgfsetrectcap%
\pgfsetroundjoin%
\pgfsetlinewidth{1.505625pt}%
\definecolor{currentstroke}{rgb}{1.000000,0.705882,0.509804}%
\pgfsetstrokecolor{currentstroke}%
\pgfsetstrokeopacity{0.800000}%
\pgfsetdash{}{0pt}%
\pgfpathmoveto{\pgfqpoint{7.878998in}{5.824157in}}%
\pgfpathlineto{\pgfqpoint{5.373961in}{3.920876in}}%
\pgfusepath{stroke}%
\end{pgfscope}%
\begin{pgfscope}%
\pgfpathrectangle{\pgfqpoint{0.481978in}{0.331635in}}{\pgfqpoint{9.300000in}{7.700000in}}%
\pgfusepath{clip}%
\pgfsetrectcap%
\pgfsetroundjoin%
\pgfsetlinewidth{1.505625pt}%
\definecolor{currentstroke}{rgb}{1.000000,0.705882,0.509804}%
\pgfsetstrokecolor{currentstroke}%
\pgfsetstrokeopacity{0.800000}%
\pgfsetdash{}{0pt}%
\pgfpathmoveto{\pgfqpoint{4.344121in}{0.681635in}}%
\pgfpathlineto{\pgfqpoint{5.373961in}{3.920876in}}%
\pgfusepath{stroke}%
\end{pgfscope}%
\begin{pgfscope}%
\pgfpathrectangle{\pgfqpoint{0.481978in}{0.331635in}}{\pgfqpoint{9.300000in}{7.700000in}}%
\pgfusepath{clip}%
\pgfsetrectcap%
\pgfsetroundjoin%
\pgfsetlinewidth{1.505625pt}%
\definecolor{currentstroke}{rgb}{1.000000,0.705882,0.509804}%
\pgfsetstrokecolor{currentstroke}%
\pgfsetstrokeopacity{0.800000}%
\pgfsetdash{}{0pt}%
\pgfpathmoveto{\pgfqpoint{6.745741in}{2.591666in}}%
\pgfpathlineto{\pgfqpoint{5.373961in}{3.920876in}}%
\pgfusepath{stroke}%
\end{pgfscope}%
\begin{pgfscope}%
\pgfpathrectangle{\pgfqpoint{0.481978in}{0.331635in}}{\pgfqpoint{9.300000in}{7.700000in}}%
\pgfusepath{clip}%
\pgfsetrectcap%
\pgfsetroundjoin%
\pgfsetlinewidth{1.505625pt}%
\definecolor{currentstroke}{rgb}{1.000000,0.705882,0.509804}%
\pgfsetstrokecolor{currentstroke}%
\pgfsetstrokeopacity{0.800000}%
\pgfsetdash{}{0pt}%
\pgfpathmoveto{\pgfqpoint{8.003363in}{4.813389in}}%
\pgfpathlineto{\pgfqpoint{5.373961in}{3.920876in}}%
\pgfusepath{stroke}%
\end{pgfscope}%
\begin{pgfscope}%
\pgfpathrectangle{\pgfqpoint{0.481978in}{0.331635in}}{\pgfqpoint{9.300000in}{7.700000in}}%
\pgfusepath{clip}%
\pgfsetrectcap%
\pgfsetroundjoin%
\pgfsetlinewidth{1.505625pt}%
\definecolor{currentstroke}{rgb}{1.000000,0.705882,0.509804}%
\pgfsetstrokecolor{currentstroke}%
\pgfsetstrokeopacity{0.800000}%
\pgfsetdash{}{0pt}%
\pgfpathmoveto{\pgfqpoint{8.570858in}{4.616051in}}%
\pgfpathlineto{\pgfqpoint{5.373961in}{3.920876in}}%
\pgfusepath{stroke}%
\end{pgfscope}%
\begin{pgfscope}%
\pgfpathrectangle{\pgfqpoint{0.481978in}{0.331635in}}{\pgfqpoint{9.300000in}{7.700000in}}%
\pgfusepath{clip}%
\pgfsetrectcap%
\pgfsetroundjoin%
\pgfsetlinewidth{1.505625pt}%
\definecolor{currentstroke}{rgb}{1.000000,0.705882,0.509804}%
\pgfsetstrokecolor{currentstroke}%
\pgfsetstrokeopacity{0.800000}%
\pgfsetdash{}{0pt}%
\pgfpathmoveto{\pgfqpoint{7.484442in}{5.018626in}}%
\pgfpathlineto{\pgfqpoint{5.373961in}{3.920876in}}%
\pgfusepath{stroke}%
\end{pgfscope}%
\begin{pgfscope}%
\pgfpathrectangle{\pgfqpoint{0.481978in}{0.331635in}}{\pgfqpoint{9.300000in}{7.700000in}}%
\pgfusepath{clip}%
\pgfsetrectcap%
\pgfsetroundjoin%
\pgfsetlinewidth{1.505625pt}%
\definecolor{currentstroke}{rgb}{1.000000,0.705882,0.509804}%
\pgfsetstrokecolor{currentstroke}%
\pgfsetstrokeopacity{0.800000}%
\pgfsetdash{}{0pt}%
\pgfpathmoveto{\pgfqpoint{3.493710in}{1.030691in}}%
\pgfpathlineto{\pgfqpoint{5.373961in}{3.920876in}}%
\pgfusepath{stroke}%
\end{pgfscope}%
\begin{pgfscope}%
\pgfpathrectangle{\pgfqpoint{0.481978in}{0.331635in}}{\pgfqpoint{9.300000in}{7.700000in}}%
\pgfusepath{clip}%
\pgfsetrectcap%
\pgfsetroundjoin%
\pgfsetlinewidth{1.505625pt}%
\definecolor{currentstroke}{rgb}{1.000000,0.705882,0.509804}%
\pgfsetstrokecolor{currentstroke}%
\pgfsetstrokeopacity{0.800000}%
\pgfsetdash{}{0pt}%
\pgfpathmoveto{\pgfqpoint{5.177601in}{3.221657in}}%
\pgfpathlineto{\pgfqpoint{5.373961in}{3.920876in}}%
\pgfusepath{stroke}%
\end{pgfscope}%
\begin{pgfscope}%
\pgfpathrectangle{\pgfqpoint{0.481978in}{0.331635in}}{\pgfqpoint{9.300000in}{7.700000in}}%
\pgfusepath{clip}%
\pgfsetrectcap%
\pgfsetroundjoin%
\pgfsetlinewidth{1.505625pt}%
\definecolor{currentstroke}{rgb}{1.000000,0.705882,0.509804}%
\pgfsetstrokecolor{currentstroke}%
\pgfsetstrokeopacity{0.800000}%
\pgfsetdash{}{0pt}%
\pgfpathmoveto{\pgfqpoint{4.181982in}{2.957199in}}%
\pgfpathlineto{\pgfqpoint{5.373961in}{3.920876in}}%
\pgfusepath{stroke}%
\end{pgfscope}%
\begin{pgfscope}%
\pgfpathrectangle{\pgfqpoint{0.481978in}{0.331635in}}{\pgfqpoint{9.300000in}{7.700000in}}%
\pgfusepath{clip}%
\pgfsetrectcap%
\pgfsetroundjoin%
\pgfsetlinewidth{1.505625pt}%
\definecolor{currentstroke}{rgb}{1.000000,0.705882,0.509804}%
\pgfsetstrokecolor{currentstroke}%
\pgfsetstrokeopacity{0.800000}%
\pgfsetdash{}{0pt}%
\pgfpathmoveto{\pgfqpoint{9.127562in}{5.241209in}}%
\pgfpathlineto{\pgfqpoint{5.373961in}{3.920876in}}%
\pgfusepath{stroke}%
\end{pgfscope}%
\begin{pgfscope}%
\pgfpathrectangle{\pgfqpoint{0.481978in}{0.331635in}}{\pgfqpoint{9.300000in}{7.700000in}}%
\pgfusepath{clip}%
\pgfsetrectcap%
\pgfsetroundjoin%
\pgfsetlinewidth{1.505625pt}%
\definecolor{currentstroke}{rgb}{1.000000,0.705882,0.509804}%
\pgfsetstrokecolor{currentstroke}%
\pgfsetstrokeopacity{0.800000}%
\pgfsetdash{}{0pt}%
\pgfpathmoveto{\pgfqpoint{3.346801in}{3.798905in}}%
\pgfpathlineto{\pgfqpoint{5.373961in}{3.920876in}}%
\pgfusepath{stroke}%
\end{pgfscope}%
\begin{pgfscope}%
\pgfpathrectangle{\pgfqpoint{0.481978in}{0.331635in}}{\pgfqpoint{9.300000in}{7.700000in}}%
\pgfusepath{clip}%
\pgfsetrectcap%
\pgfsetroundjoin%
\pgfsetlinewidth{1.505625pt}%
\definecolor{currentstroke}{rgb}{1.000000,0.705882,0.509804}%
\pgfsetstrokecolor{currentstroke}%
\pgfsetstrokeopacity{0.800000}%
\pgfsetdash{}{0pt}%
\pgfpathmoveto{\pgfqpoint{2.661207in}{2.871717in}}%
\pgfpathlineto{\pgfqpoint{5.373961in}{3.920876in}}%
\pgfusepath{stroke}%
\end{pgfscope}%
\begin{pgfscope}%
\pgfpathrectangle{\pgfqpoint{0.481978in}{0.331635in}}{\pgfqpoint{9.300000in}{7.700000in}}%
\pgfusepath{clip}%
\pgfsetrectcap%
\pgfsetroundjoin%
\pgfsetlinewidth{1.505625pt}%
\definecolor{currentstroke}{rgb}{1.000000,0.705882,0.509804}%
\pgfsetstrokecolor{currentstroke}%
\pgfsetstrokeopacity{0.800000}%
\pgfsetdash{}{0pt}%
\pgfpathmoveto{\pgfqpoint{6.147513in}{3.415011in}}%
\pgfpathlineto{\pgfqpoint{5.373961in}{3.920876in}}%
\pgfusepath{stroke}%
\end{pgfscope}%
\begin{pgfscope}%
\pgfpathrectangle{\pgfqpoint{0.481978in}{0.331635in}}{\pgfqpoint{9.300000in}{7.700000in}}%
\pgfusepath{clip}%
\pgfsetrectcap%
\pgfsetroundjoin%
\pgfsetlinewidth{1.505625pt}%
\definecolor{currentstroke}{rgb}{1.000000,0.705882,0.509804}%
\pgfsetstrokecolor{currentstroke}%
\pgfsetstrokeopacity{0.800000}%
\pgfsetdash{}{0pt}%
\pgfpathmoveto{\pgfqpoint{8.123188in}{4.382220in}}%
\pgfpathlineto{\pgfqpoint{5.373961in}{3.920876in}}%
\pgfusepath{stroke}%
\end{pgfscope}%
\begin{pgfscope}%
\pgfpathrectangle{\pgfqpoint{0.481978in}{0.331635in}}{\pgfqpoint{9.300000in}{7.700000in}}%
\pgfusepath{clip}%
\pgfsetrectcap%
\pgfsetroundjoin%
\pgfsetlinewidth{1.505625pt}%
\definecolor{currentstroke}{rgb}{1.000000,0.705882,0.509804}%
\pgfsetstrokecolor{currentstroke}%
\pgfsetstrokeopacity{0.800000}%
\pgfsetdash{}{0pt}%
\pgfpathmoveto{\pgfqpoint{4.044825in}{4.813210in}}%
\pgfpathlineto{\pgfqpoint{5.373961in}{3.920876in}}%
\pgfusepath{stroke}%
\end{pgfscope}%
\begin{pgfscope}%
\pgfpathrectangle{\pgfqpoint{0.481978in}{0.331635in}}{\pgfqpoint{9.300000in}{7.700000in}}%
\pgfusepath{clip}%
\pgfsetrectcap%
\pgfsetroundjoin%
\pgfsetlinewidth{1.505625pt}%
\definecolor{currentstroke}{rgb}{1.000000,0.705882,0.509804}%
\pgfsetstrokecolor{currentstroke}%
\pgfsetstrokeopacity{0.800000}%
\pgfsetdash{}{0pt}%
\pgfpathmoveto{\pgfqpoint{3.633094in}{2.797350in}}%
\pgfpathlineto{\pgfqpoint{5.373961in}{3.920876in}}%
\pgfusepath{stroke}%
\end{pgfscope}%
\begin{pgfscope}%
\pgfpathrectangle{\pgfqpoint{0.481978in}{0.331635in}}{\pgfqpoint{9.300000in}{7.700000in}}%
\pgfusepath{clip}%
\pgfsetrectcap%
\pgfsetroundjoin%
\pgfsetlinewidth{1.505625pt}%
\definecolor{currentstroke}{rgb}{1.000000,0.705882,0.509804}%
\pgfsetstrokecolor{currentstroke}%
\pgfsetstrokeopacity{0.800000}%
\pgfsetdash{}{0pt}%
\pgfpathmoveto{\pgfqpoint{4.595278in}{3.123108in}}%
\pgfpathlineto{\pgfqpoint{5.373961in}{3.920876in}}%
\pgfusepath{stroke}%
\end{pgfscope}%
\begin{pgfscope}%
\pgfpathrectangle{\pgfqpoint{0.481978in}{0.331635in}}{\pgfqpoint{9.300000in}{7.700000in}}%
\pgfusepath{clip}%
\pgfsetrectcap%
\pgfsetroundjoin%
\pgfsetlinewidth{1.505625pt}%
\definecolor{currentstroke}{rgb}{1.000000,0.705882,0.509804}%
\pgfsetstrokecolor{currentstroke}%
\pgfsetstrokeopacity{0.800000}%
\pgfsetdash{}{0pt}%
\pgfpathmoveto{\pgfqpoint{3.313622in}{1.791445in}}%
\pgfpathlineto{\pgfqpoint{5.373961in}{3.920876in}}%
\pgfusepath{stroke}%
\end{pgfscope}%
\begin{pgfscope}%
\pgfpathrectangle{\pgfqpoint{0.481978in}{0.331635in}}{\pgfqpoint{9.300000in}{7.700000in}}%
\pgfusepath{clip}%
\pgfsetrectcap%
\pgfsetroundjoin%
\pgfsetlinewidth{1.505625pt}%
\definecolor{currentstroke}{rgb}{1.000000,0.705882,0.509804}%
\pgfsetstrokecolor{currentstroke}%
\pgfsetstrokeopacity{0.800000}%
\pgfsetdash{}{0pt}%
\pgfpathmoveto{\pgfqpoint{3.782039in}{3.296454in}}%
\pgfpathlineto{\pgfqpoint{5.373961in}{3.920876in}}%
\pgfusepath{stroke}%
\end{pgfscope}%
\begin{pgfscope}%
\pgfpathrectangle{\pgfqpoint{0.481978in}{0.331635in}}{\pgfqpoint{9.300000in}{7.700000in}}%
\pgfusepath{clip}%
\pgfsetrectcap%
\pgfsetroundjoin%
\pgfsetlinewidth{1.505625pt}%
\definecolor{currentstroke}{rgb}{1.000000,0.705882,0.509804}%
\pgfsetstrokecolor{currentstroke}%
\pgfsetstrokeopacity{0.800000}%
\pgfsetdash{}{0pt}%
\pgfpathmoveto{\pgfqpoint{3.355525in}{4.320355in}}%
\pgfpathlineto{\pgfqpoint{5.373961in}{3.920876in}}%
\pgfusepath{stroke}%
\end{pgfscope}%
\begin{pgfscope}%
\pgfpathrectangle{\pgfqpoint{0.481978in}{0.331635in}}{\pgfqpoint{9.300000in}{7.700000in}}%
\pgfusepath{clip}%
\pgfsetrectcap%
\pgfsetroundjoin%
\pgfsetlinewidth{1.505625pt}%
\definecolor{currentstroke}{rgb}{1.000000,0.705882,0.509804}%
\pgfsetstrokecolor{currentstroke}%
\pgfsetstrokeopacity{0.800000}%
\pgfsetdash{}{0pt}%
\pgfpathmoveto{\pgfqpoint{3.639977in}{1.392024in}}%
\pgfpathlineto{\pgfqpoint{5.373961in}{3.920876in}}%
\pgfusepath{stroke}%
\end{pgfscope}%
\begin{pgfscope}%
\pgfpathrectangle{\pgfqpoint{0.481978in}{0.331635in}}{\pgfqpoint{9.300000in}{7.700000in}}%
\pgfusepath{clip}%
\pgfsetrectcap%
\pgfsetroundjoin%
\pgfsetlinewidth{1.505625pt}%
\definecolor{currentstroke}{rgb}{1.000000,0.705882,0.509804}%
\pgfsetstrokecolor{currentstroke}%
\pgfsetstrokeopacity{0.800000}%
\pgfsetdash{}{0pt}%
\pgfpathmoveto{\pgfqpoint{3.357786in}{4.898748in}}%
\pgfpathlineto{\pgfqpoint{5.373961in}{3.920876in}}%
\pgfusepath{stroke}%
\end{pgfscope}%
\begin{pgfscope}%
\pgfpathrectangle{\pgfqpoint{0.481978in}{0.331635in}}{\pgfqpoint{9.300000in}{7.700000in}}%
\pgfusepath{clip}%
\pgfsetrectcap%
\pgfsetroundjoin%
\pgfsetlinewidth{1.505625pt}%
\definecolor{currentstroke}{rgb}{1.000000,0.705882,0.509804}%
\pgfsetstrokecolor{currentstroke}%
\pgfsetstrokeopacity{0.800000}%
\pgfsetdash{}{0pt}%
\pgfpathmoveto{\pgfqpoint{7.382152in}{2.810657in}}%
\pgfpathlineto{\pgfqpoint{5.373961in}{3.920876in}}%
\pgfusepath{stroke}%
\end{pgfscope}%
\begin{pgfscope}%
\pgfpathrectangle{\pgfqpoint{0.481978in}{0.331635in}}{\pgfqpoint{9.300000in}{7.700000in}}%
\pgfusepath{clip}%
\pgfsetrectcap%
\pgfsetroundjoin%
\pgfsetlinewidth{1.505625pt}%
\definecolor{currentstroke}{rgb}{1.000000,0.705882,0.509804}%
\pgfsetstrokecolor{currentstroke}%
\pgfsetstrokeopacity{0.800000}%
\pgfsetdash{}{0pt}%
\pgfpathmoveto{\pgfqpoint{8.231517in}{5.976277in}}%
\pgfpathlineto{\pgfqpoint{5.373961in}{3.920876in}}%
\pgfusepath{stroke}%
\end{pgfscope}%
\begin{pgfscope}%
\pgfpathrectangle{\pgfqpoint{0.481978in}{0.331635in}}{\pgfqpoint{9.300000in}{7.700000in}}%
\pgfusepath{clip}%
\pgfsetrectcap%
\pgfsetroundjoin%
\pgfsetlinewidth{1.505625pt}%
\definecolor{currentstroke}{rgb}{1.000000,0.705882,0.509804}%
\pgfsetstrokecolor{currentstroke}%
\pgfsetstrokeopacity{0.800000}%
\pgfsetdash{}{0pt}%
\pgfpathmoveto{\pgfqpoint{6.777913in}{5.229998in}}%
\pgfpathlineto{\pgfqpoint{5.373961in}{3.920876in}}%
\pgfusepath{stroke}%
\end{pgfscope}%
\begin{pgfscope}%
\pgfpathrectangle{\pgfqpoint{0.481978in}{0.331635in}}{\pgfqpoint{9.300000in}{7.700000in}}%
\pgfusepath{clip}%
\pgfsetrectcap%
\pgfsetroundjoin%
\pgfsetlinewidth{1.505625pt}%
\definecolor{currentstroke}{rgb}{1.000000,0.705882,0.509804}%
\pgfsetstrokecolor{currentstroke}%
\pgfsetstrokeopacity{0.800000}%
\pgfsetdash{}{0pt}%
\pgfpathmoveto{\pgfqpoint{4.360013in}{3.930056in}}%
\pgfpathlineto{\pgfqpoint{5.373961in}{3.920876in}}%
\pgfusepath{stroke}%
\end{pgfscope}%
\begin{pgfscope}%
\pgfpathrectangle{\pgfqpoint{0.481978in}{0.331635in}}{\pgfqpoint{9.300000in}{7.700000in}}%
\pgfusepath{clip}%
\pgfsetrectcap%
\pgfsetroundjoin%
\pgfsetlinewidth{1.505625pt}%
\definecolor{currentstroke}{rgb}{1.000000,0.705882,0.509804}%
\pgfsetstrokecolor{currentstroke}%
\pgfsetstrokeopacity{0.800000}%
\pgfsetdash{}{0pt}%
\pgfpathmoveto{\pgfqpoint{5.351509in}{4.366486in}}%
\pgfpathlineto{\pgfqpoint{5.373961in}{3.920876in}}%
\pgfusepath{stroke}%
\end{pgfscope}%
\begin{pgfscope}%
\pgfpathrectangle{\pgfqpoint{0.481978in}{0.331635in}}{\pgfqpoint{9.300000in}{7.700000in}}%
\pgfusepath{clip}%
\pgfsetrectcap%
\pgfsetroundjoin%
\pgfsetlinewidth{1.505625pt}%
\definecolor{currentstroke}{rgb}{1.000000,0.705882,0.509804}%
\pgfsetstrokecolor{currentstroke}%
\pgfsetstrokeopacity{0.800000}%
\pgfsetdash{}{0pt}%
\pgfpathmoveto{\pgfqpoint{2.852372in}{4.794334in}}%
\pgfpathlineto{\pgfqpoint{5.373961in}{3.920876in}}%
\pgfusepath{stroke}%
\end{pgfscope}%
\begin{pgfscope}%
\pgfpathrectangle{\pgfqpoint{0.481978in}{0.331635in}}{\pgfqpoint{9.300000in}{7.700000in}}%
\pgfusepath{clip}%
\pgfsetrectcap%
\pgfsetroundjoin%
\pgfsetlinewidth{1.505625pt}%
\definecolor{currentstroke}{rgb}{1.000000,0.705882,0.509804}%
\pgfsetstrokecolor{currentstroke}%
\pgfsetstrokeopacity{0.800000}%
\pgfsetdash{}{0pt}%
\pgfpathmoveto{\pgfqpoint{2.417331in}{4.480722in}}%
\pgfpathlineto{\pgfqpoint{5.373961in}{3.920876in}}%
\pgfusepath{stroke}%
\end{pgfscope}%
\begin{pgfscope}%
\pgfpathrectangle{\pgfqpoint{0.481978in}{0.331635in}}{\pgfqpoint{9.300000in}{7.700000in}}%
\pgfusepath{clip}%
\pgfsetrectcap%
\pgfsetroundjoin%
\pgfsetlinewidth{1.505625pt}%
\definecolor{currentstroke}{rgb}{1.000000,0.705882,0.509804}%
\pgfsetstrokecolor{currentstroke}%
\pgfsetstrokeopacity{0.800000}%
\pgfsetdash{}{0pt}%
\pgfpathmoveto{\pgfqpoint{2.136437in}{2.620278in}}%
\pgfpathlineto{\pgfqpoint{5.373961in}{3.920876in}}%
\pgfusepath{stroke}%
\end{pgfscope}%
\begin{pgfscope}%
\pgfpathrectangle{\pgfqpoint{0.481978in}{0.331635in}}{\pgfqpoint{9.300000in}{7.700000in}}%
\pgfusepath{clip}%
\pgfsetrectcap%
\pgfsetroundjoin%
\pgfsetlinewidth{1.505625pt}%
\definecolor{currentstroke}{rgb}{1.000000,0.705882,0.509804}%
\pgfsetstrokecolor{currentstroke}%
\pgfsetstrokeopacity{0.800000}%
\pgfsetdash{}{0pt}%
\pgfpathmoveto{\pgfqpoint{3.078254in}{2.450245in}}%
\pgfpathlineto{\pgfqpoint{5.373961in}{3.920876in}}%
\pgfusepath{stroke}%
\end{pgfscope}%
\begin{pgfscope}%
\pgfpathrectangle{\pgfqpoint{0.481978in}{0.331635in}}{\pgfqpoint{9.300000in}{7.700000in}}%
\pgfusepath{clip}%
\pgfsetrectcap%
\pgfsetroundjoin%
\pgfsetlinewidth{1.505625pt}%
\definecolor{currentstroke}{rgb}{1.000000,0.705882,0.509804}%
\pgfsetstrokecolor{currentstroke}%
\pgfsetstrokeopacity{0.800000}%
\pgfsetdash{}{0pt}%
\pgfpathmoveto{\pgfqpoint{9.359251in}{4.650550in}}%
\pgfpathlineto{\pgfqpoint{5.373961in}{3.920876in}}%
\pgfusepath{stroke}%
\end{pgfscope}%
\begin{pgfscope}%
\pgfpathrectangle{\pgfqpoint{0.481978in}{0.331635in}}{\pgfqpoint{9.300000in}{7.700000in}}%
\pgfusepath{clip}%
\pgfsetrectcap%
\pgfsetroundjoin%
\pgfsetlinewidth{1.505625pt}%
\definecolor{currentstroke}{rgb}{1.000000,0.705882,0.509804}%
\pgfsetstrokecolor{currentstroke}%
\pgfsetstrokeopacity{0.800000}%
\pgfsetdash{}{0pt}%
\pgfpathmoveto{\pgfqpoint{5.664273in}{3.194429in}}%
\pgfpathlineto{\pgfqpoint{5.373961in}{3.920876in}}%
\pgfusepath{stroke}%
\end{pgfscope}%
\begin{pgfscope}%
\pgfpathrectangle{\pgfqpoint{0.481978in}{0.331635in}}{\pgfqpoint{9.300000in}{7.700000in}}%
\pgfusepath{clip}%
\pgfsetrectcap%
\pgfsetroundjoin%
\pgfsetlinewidth{1.505625pt}%
\definecolor{currentstroke}{rgb}{1.000000,0.705882,0.509804}%
\pgfsetstrokecolor{currentstroke}%
\pgfsetstrokeopacity{0.800000}%
\pgfsetdash{}{0pt}%
\pgfpathmoveto{\pgfqpoint{8.498557in}{5.040087in}}%
\pgfpathlineto{\pgfqpoint{5.373961in}{3.920876in}}%
\pgfusepath{stroke}%
\end{pgfscope}%
\begin{pgfscope}%
\pgfpathrectangle{\pgfqpoint{0.481978in}{0.331635in}}{\pgfqpoint{9.300000in}{7.700000in}}%
\pgfusepath{clip}%
\pgfsetrectcap%
\pgfsetroundjoin%
\pgfsetlinewidth{1.505625pt}%
\definecolor{currentstroke}{rgb}{1.000000,0.705882,0.509804}%
\pgfsetstrokecolor{currentstroke}%
\pgfsetstrokeopacity{0.800000}%
\pgfsetdash{}{0pt}%
\pgfpathmoveto{\pgfqpoint{6.107881in}{4.232786in}}%
\pgfpathlineto{\pgfqpoint{5.373961in}{3.920876in}}%
\pgfusepath{stroke}%
\end{pgfscope}%
\begin{pgfscope}%
\pgfpathrectangle{\pgfqpoint{0.481978in}{0.331635in}}{\pgfqpoint{9.300000in}{7.700000in}}%
\pgfusepath{clip}%
\pgfsetrectcap%
\pgfsetroundjoin%
\pgfsetlinewidth{1.505625pt}%
\definecolor{currentstroke}{rgb}{1.000000,0.705882,0.509804}%
\pgfsetstrokecolor{currentstroke}%
\pgfsetstrokeopacity{0.800000}%
\pgfsetdash{}{0pt}%
\pgfpathmoveto{\pgfqpoint{7.335472in}{5.543933in}}%
\pgfpathlineto{\pgfqpoint{5.373961in}{3.920876in}}%
\pgfusepath{stroke}%
\end{pgfscope}%
\begin{pgfscope}%
\pgfpathrectangle{\pgfqpoint{0.481978in}{0.331635in}}{\pgfqpoint{9.300000in}{7.700000in}}%
\pgfusepath{clip}%
\pgfsetrectcap%
\pgfsetroundjoin%
\pgfsetlinewidth{1.505625pt}%
\definecolor{currentstroke}{rgb}{1.000000,0.705882,0.509804}%
\pgfsetstrokecolor{currentstroke}%
\pgfsetstrokeopacity{0.800000}%
\pgfsetdash{}{0pt}%
\pgfpathmoveto{\pgfqpoint{8.971590in}{5.840871in}}%
\pgfpathlineto{\pgfqpoint{5.373961in}{3.920876in}}%
\pgfusepath{stroke}%
\end{pgfscope}%
\begin{pgfscope}%
\pgfpathrectangle{\pgfqpoint{0.481978in}{0.331635in}}{\pgfqpoint{9.300000in}{7.700000in}}%
\pgfusepath{clip}%
\pgfsetrectcap%
\pgfsetroundjoin%
\pgfsetlinewidth{1.505625pt}%
\definecolor{currentstroke}{rgb}{1.000000,0.705882,0.509804}%
\pgfsetstrokecolor{currentstroke}%
\pgfsetstrokeopacity{0.800000}%
\pgfsetdash{}{0pt}%
\pgfpathmoveto{\pgfqpoint{4.777742in}{4.754613in}}%
\pgfpathlineto{\pgfqpoint{5.373961in}{3.920876in}}%
\pgfusepath{stroke}%
\end{pgfscope}%
\begin{pgfscope}%
\pgfpathrectangle{\pgfqpoint{0.481978in}{0.331635in}}{\pgfqpoint{9.300000in}{7.700000in}}%
\pgfusepath{clip}%
\pgfsetrectcap%
\pgfsetroundjoin%
\pgfsetlinewidth{1.505625pt}%
\definecolor{currentstroke}{rgb}{1.000000,0.705882,0.509804}%
\pgfsetstrokecolor{currentstroke}%
\pgfsetstrokeopacity{0.800000}%
\pgfsetdash{}{0pt}%
\pgfpathmoveto{\pgfqpoint{2.999362in}{1.031465in}}%
\pgfpathlineto{\pgfqpoint{5.373961in}{3.920876in}}%
\pgfusepath{stroke}%
\end{pgfscope}%
\begin{pgfscope}%
\pgfpathrectangle{\pgfqpoint{0.481978in}{0.331635in}}{\pgfqpoint{9.300000in}{7.700000in}}%
\pgfusepath{clip}%
\pgfsetrectcap%
\pgfsetroundjoin%
\pgfsetlinewidth{1.505625pt}%
\definecolor{currentstroke}{rgb}{1.000000,0.705882,0.509804}%
\pgfsetstrokecolor{currentstroke}%
\pgfsetstrokeopacity{0.800000}%
\pgfsetdash{}{0pt}%
\pgfpathmoveto{\pgfqpoint{3.001304in}{5.631160in}}%
\pgfpathlineto{\pgfqpoint{5.373961in}{3.920876in}}%
\pgfusepath{stroke}%
\end{pgfscope}%
\begin{pgfscope}%
\pgfpathrectangle{\pgfqpoint{0.481978in}{0.331635in}}{\pgfqpoint{9.300000in}{7.700000in}}%
\pgfusepath{clip}%
\pgfsetrectcap%
\pgfsetroundjoin%
\pgfsetlinewidth{1.505625pt}%
\definecolor{currentstroke}{rgb}{1.000000,0.705882,0.509804}%
\pgfsetstrokecolor{currentstroke}%
\pgfsetstrokeopacity{0.800000}%
\pgfsetdash{}{0pt}%
\pgfpathmoveto{\pgfqpoint{3.332958in}{3.168126in}}%
\pgfpathlineto{\pgfqpoint{5.373961in}{3.920876in}}%
\pgfusepath{stroke}%
\end{pgfscope}%
\begin{pgfscope}%
\pgfpathrectangle{\pgfqpoint{0.481978in}{0.331635in}}{\pgfqpoint{9.300000in}{7.700000in}}%
\pgfusepath{clip}%
\pgfsetrectcap%
\pgfsetroundjoin%
\pgfsetlinewidth{1.505625pt}%
\definecolor{currentstroke}{rgb}{1.000000,0.705882,0.509804}%
\pgfsetstrokecolor{currentstroke}%
\pgfsetstrokeopacity{0.800000}%
\pgfsetdash{}{0pt}%
\pgfpathmoveto{\pgfqpoint{8.075703in}{5.286167in}}%
\pgfpathlineto{\pgfqpoint{5.373961in}{3.920876in}}%
\pgfusepath{stroke}%
\end{pgfscope}%
\begin{pgfscope}%
\pgfpathrectangle{\pgfqpoint{0.481978in}{0.331635in}}{\pgfqpoint{9.300000in}{7.700000in}}%
\pgfusepath{clip}%
\pgfsetrectcap%
\pgfsetroundjoin%
\pgfsetlinewidth{1.505625pt}%
\definecolor{currentstroke}{rgb}{1.000000,0.705882,0.509804}%
\pgfsetstrokecolor{currentstroke}%
\pgfsetstrokeopacity{0.800000}%
\pgfsetdash{}{0pt}%
\pgfpathmoveto{\pgfqpoint{4.123221in}{3.527887in}}%
\pgfpathlineto{\pgfqpoint{5.373961in}{3.920876in}}%
\pgfusepath{stroke}%
\end{pgfscope}%
\begin{pgfscope}%
\pgfpathrectangle{\pgfqpoint{0.481978in}{0.331635in}}{\pgfqpoint{9.300000in}{7.700000in}}%
\pgfusepath{clip}%
\pgfsetrectcap%
\pgfsetroundjoin%
\pgfsetlinewidth{1.505625pt}%
\definecolor{currentstroke}{rgb}{1.000000,0.705882,0.509804}%
\pgfsetstrokecolor{currentstroke}%
\pgfsetstrokeopacity{0.800000}%
\pgfsetdash{}{0pt}%
\pgfpathmoveto{\pgfqpoint{1.536500in}{2.231329in}}%
\pgfpathlineto{\pgfqpoint{5.373961in}{3.920876in}}%
\pgfusepath{stroke}%
\end{pgfscope}%
\begin{pgfscope}%
\pgfpathrectangle{\pgfqpoint{0.481978in}{0.331635in}}{\pgfqpoint{9.300000in}{7.700000in}}%
\pgfusepath{clip}%
\pgfsetrectcap%
\pgfsetroundjoin%
\pgfsetlinewidth{1.505625pt}%
\definecolor{currentstroke}{rgb}{1.000000,0.705882,0.509804}%
\pgfsetstrokecolor{currentstroke}%
\pgfsetstrokeopacity{0.800000}%
\pgfsetdash{}{0pt}%
\pgfpathmoveto{\pgfqpoint{5.550194in}{2.761714in}}%
\pgfpathlineto{\pgfqpoint{5.373961in}{3.920876in}}%
\pgfusepath{stroke}%
\end{pgfscope}%
\begin{pgfscope}%
\pgfpathrectangle{\pgfqpoint{0.481978in}{0.331635in}}{\pgfqpoint{9.300000in}{7.700000in}}%
\pgfusepath{clip}%
\pgfsetrectcap%
\pgfsetroundjoin%
\pgfsetlinewidth{1.505625pt}%
\definecolor{currentstroke}{rgb}{1.000000,0.705882,0.509804}%
\pgfsetstrokecolor{currentstroke}%
\pgfsetstrokeopacity{0.800000}%
\pgfsetdash{}{0pt}%
\pgfpathmoveto{\pgfqpoint{2.791505in}{3.084867in}}%
\pgfpathlineto{\pgfqpoint{5.373961in}{3.920876in}}%
\pgfusepath{stroke}%
\end{pgfscope}%
\begin{pgfscope}%
\pgfsetrectcap%
\pgfsetmiterjoin%
\pgfsetlinewidth{0.803000pt}%
\definecolor{currentstroke}{rgb}{0.000000,0.000000,0.000000}%
\pgfsetstrokecolor{currentstroke}%
\pgfsetdash{}{0pt}%
\pgfpathmoveto{\pgfqpoint{0.481978in}{0.331635in}}%
\pgfpathlineto{\pgfqpoint{0.481978in}{8.031635in}}%
\pgfusepath{stroke}%
\end{pgfscope}%
\begin{pgfscope}%
\pgfsetrectcap%
\pgfsetmiterjoin%
\pgfsetlinewidth{0.803000pt}%
\definecolor{currentstroke}{rgb}{0.000000,0.000000,0.000000}%
\pgfsetstrokecolor{currentstroke}%
\pgfsetdash{}{0pt}%
\pgfpathmoveto{\pgfqpoint{9.781978in}{0.331635in}}%
\pgfpathlineto{\pgfqpoint{9.781978in}{8.031635in}}%
\pgfusepath{stroke}%
\end{pgfscope}%
\begin{pgfscope}%
\pgfsetrectcap%
\pgfsetmiterjoin%
\pgfsetlinewidth{0.803000pt}%
\definecolor{currentstroke}{rgb}{0.000000,0.000000,0.000000}%
\pgfsetstrokecolor{currentstroke}%
\pgfsetdash{}{0pt}%
\pgfpathmoveto{\pgfqpoint{0.481978in}{0.331635in}}%
\pgfpathlineto{\pgfqpoint{9.781978in}{0.331635in}}%
\pgfusepath{stroke}%
\end{pgfscope}%
\begin{pgfscope}%
\pgfsetrectcap%
\pgfsetmiterjoin%
\pgfsetlinewidth{0.803000pt}%
\definecolor{currentstroke}{rgb}{0.000000,0.000000,0.000000}%
\pgfsetstrokecolor{currentstroke}%
\pgfsetdash{}{0pt}%
\pgfpathmoveto{\pgfqpoint{0.481978in}{8.031635in}}%
\pgfpathlineto{\pgfqpoint{9.781978in}{8.031635in}}%
\pgfusepath{stroke}%
\end{pgfscope}%
\begin{pgfscope}%
\definecolor{textcolor}{rgb}{0.000000,0.000000,0.000000}%
\pgfsetstrokecolor{textcolor}%
\pgfsetfillcolor{textcolor}%
\pgftext[x=5.131978in,y=8.114968in,,base]{\color{textcolor}\sffamily\fontsize{12.000000}{14.400000}\selectfont T-SNE for chair images (s2r3dfree\_background)}%
\end{pgfscope}%
\begin{pgfscope}%
\pgfsetbuttcap%
\pgfsetmiterjoin%
\definecolor{currentfill}{rgb}{1.000000,1.000000,1.000000}%
\pgfsetfillcolor{currentfill}%
\pgfsetfillopacity{0.800000}%
\pgfsetlinewidth{1.003750pt}%
\definecolor{currentstroke}{rgb}{0.800000,0.800000,0.800000}%
\pgfsetstrokecolor{currentstroke}%
\pgfsetstrokeopacity{0.800000}%
\pgfsetdash{}{0pt}%
\pgfpathmoveto{\pgfqpoint{9.879200in}{3.955012in}}%
\pgfpathlineto{\pgfqpoint{11.882754in}{3.955012in}}%
\pgfpathquadraticcurveto{\pgfqpoint{11.910532in}{3.955012in}}{\pgfqpoint{11.910532in}{3.982789in}}%
\pgfpathlineto{\pgfqpoint{11.910532in}{4.380481in}}%
\pgfpathquadraticcurveto{\pgfqpoint{11.910532in}{4.408258in}}{\pgfqpoint{11.882754in}{4.408258in}}%
\pgfpathlineto{\pgfqpoint{9.879200in}{4.408258in}}%
\pgfpathquadraticcurveto{\pgfqpoint{9.851422in}{4.408258in}}{\pgfqpoint{9.851422in}{4.380481in}}%
\pgfpathlineto{\pgfqpoint{9.851422in}{3.982789in}}%
\pgfpathquadraticcurveto{\pgfqpoint{9.851422in}{3.955012in}}{\pgfqpoint{9.879200in}{3.955012in}}%
\pgfpathclose%
\pgfusepath{stroke,fill}%
\end{pgfscope}%
\begin{pgfscope}%
\pgfsetbuttcap%
\pgfsetroundjoin%
\definecolor{currentfill}{rgb}{0.631373,0.788235,0.956863}%
\pgfsetfillcolor{currentfill}%
\pgfsetlinewidth{1.003750pt}%
\definecolor{currentstroke}{rgb}{0.631373,0.788235,0.956863}%
\pgfsetstrokecolor{currentstroke}%
\pgfsetdash{}{0pt}%
\pgfsys@defobject{currentmarker}{\pgfqpoint{-0.041667in}{-0.041667in}}{\pgfqpoint{0.041667in}{0.041667in}}{%
\pgfpathmoveto{\pgfqpoint{0.000000in}{-0.041667in}}%
\pgfpathcurveto{\pgfqpoint{0.011050in}{-0.041667in}}{\pgfqpoint{0.021649in}{-0.037276in}}{\pgfqpoint{0.029463in}{-0.029463in}}%
\pgfpathcurveto{\pgfqpoint{0.037276in}{-0.021649in}}{\pgfqpoint{0.041667in}{-0.011050in}}{\pgfqpoint{0.041667in}{0.000000in}}%
\pgfpathcurveto{\pgfqpoint{0.041667in}{0.011050in}}{\pgfqpoint{0.037276in}{0.021649in}}{\pgfqpoint{0.029463in}{0.029463in}}%
\pgfpathcurveto{\pgfqpoint{0.021649in}{0.037276in}}{\pgfqpoint{0.011050in}{0.041667in}}{\pgfqpoint{0.000000in}{0.041667in}}%
\pgfpathcurveto{\pgfqpoint{-0.011050in}{0.041667in}}{\pgfqpoint{-0.021649in}{0.037276in}}{\pgfqpoint{-0.029463in}{0.029463in}}%
\pgfpathcurveto{\pgfqpoint{-0.037276in}{0.021649in}}{\pgfqpoint{-0.041667in}{0.011050in}}{\pgfqpoint{-0.041667in}{0.000000in}}%
\pgfpathcurveto{\pgfqpoint{-0.041667in}{-0.011050in}}{\pgfqpoint{-0.037276in}{-0.021649in}}{\pgfqpoint{-0.029463in}{-0.029463in}}%
\pgfpathcurveto{\pgfqpoint{-0.021649in}{-0.037276in}}{\pgfqpoint{-0.011050in}{-0.041667in}}{\pgfqpoint{0.000000in}{-0.041667in}}%
\pgfpathclose%
\pgfusepath{stroke,fill}%
}%
\begin{pgfscope}%
\pgfsys@transformshift{10.045867in}{4.283638in}%
\pgfsys@useobject{currentmarker}{}%
\end{pgfscope}%
\end{pgfscope}%
\begin{pgfscope}%
\definecolor{textcolor}{rgb}{0.000000,0.000000,0.000000}%
\pgfsetstrokecolor{textcolor}%
\pgfsetfillcolor{textcolor}%
\pgftext[x=10.295867in,y=4.247180in,left,base]{\color{textcolor}\sffamily\fontsize{10.000000}{12.000000}\selectfont Pix3D}%
\end{pgfscope}%
\begin{pgfscope}%
\pgfsetbuttcap%
\pgfsetroundjoin%
\definecolor{currentfill}{rgb}{1.000000,0.705882,0.509804}%
\pgfsetfillcolor{currentfill}%
\pgfsetlinewidth{1.003750pt}%
\definecolor{currentstroke}{rgb}{1.000000,0.705882,0.509804}%
\pgfsetstrokecolor{currentstroke}%
\pgfsetdash{}{0pt}%
\pgfsys@defobject{currentmarker}{\pgfqpoint{-0.041667in}{-0.041667in}}{\pgfqpoint{0.041667in}{0.041667in}}{%
\pgfpathmoveto{\pgfqpoint{0.000000in}{-0.041667in}}%
\pgfpathcurveto{\pgfqpoint{0.011050in}{-0.041667in}}{\pgfqpoint{0.021649in}{-0.037276in}}{\pgfqpoint{0.029463in}{-0.029463in}}%
\pgfpathcurveto{\pgfqpoint{0.037276in}{-0.021649in}}{\pgfqpoint{0.041667in}{-0.011050in}}{\pgfqpoint{0.041667in}{0.000000in}}%
\pgfpathcurveto{\pgfqpoint{0.041667in}{0.011050in}}{\pgfqpoint{0.037276in}{0.021649in}}{\pgfqpoint{0.029463in}{0.029463in}}%
\pgfpathcurveto{\pgfqpoint{0.021649in}{0.037276in}}{\pgfqpoint{0.011050in}{0.041667in}}{\pgfqpoint{0.000000in}{0.041667in}}%
\pgfpathcurveto{\pgfqpoint{-0.011050in}{0.041667in}}{\pgfqpoint{-0.021649in}{0.037276in}}{\pgfqpoint{-0.029463in}{0.029463in}}%
\pgfpathcurveto{\pgfqpoint{-0.037276in}{0.021649in}}{\pgfqpoint{-0.041667in}{0.011050in}}{\pgfqpoint{-0.041667in}{0.000000in}}%
\pgfpathcurveto{\pgfqpoint{-0.041667in}{-0.011050in}}{\pgfqpoint{-0.037276in}{-0.021649in}}{\pgfqpoint{-0.029463in}{-0.029463in}}%
\pgfpathcurveto{\pgfqpoint{-0.021649in}{-0.037276in}}{\pgfqpoint{-0.011050in}{-0.041667in}}{\pgfqpoint{0.000000in}{-0.041667in}}%
\pgfpathclose%
\pgfusepath{stroke,fill}%
}%
\begin{pgfscope}%
\pgfsys@transformshift{10.045867in}{4.079781in}%
\pgfsys@useobject{currentmarker}{}%
\end{pgfscope}%
\end{pgfscope}%
\begin{pgfscope}%
\definecolor{textcolor}{rgb}{0.000000,0.000000,0.000000}%
\pgfsetstrokecolor{textcolor}%
\pgfsetfillcolor{textcolor}%
\pgftext[x=10.295867in,y=4.043322in,left,base]{\color{textcolor}\sffamily\fontsize{10.000000}{12.000000}\selectfont s2r3dfree\_background}%
\end{pgfscope}%
\end{pgfpicture}%
\makeatother%
\endgroup%
}
    \resizebox{0.49\linewidth}{5cm}{%% Creator: Matplotlib, PGF backend
%%
%% To include the figure in your LaTeX document, write
%%   \input{<filename>.pgf}
%%
%% Make sure the required packages are loaded in your preamble
%%   \usepackage{pgf}
%%
%% Figures using additional raster images can only be included by \input if
%% they are in the same directory as the main LaTeX file. For loading figures
%% from other directories you can use the `import` package
%%   \usepackage{import}
%%
%% and then include the figures with
%%   \import{<path to file>}{<filename>.pgf}
%%
%% Matplotlib used the following preamble
%%   \usepackage{fontspec}
%%   \setmainfont{DejaVuSerif.ttf}[Path=\detokenize{/Users/apple/opt/anaconda3/envs/kaolin/lib/python3.7/site-packages/matplotlib/mpl-data/fonts/ttf/}]
%%   \setsansfont{DejaVuSans.ttf}[Path=\detokenize{/Users/apple/opt/anaconda3/envs/kaolin/lib/python3.7/site-packages/matplotlib/mpl-data/fonts/ttf/}]
%%   \setmonofont{DejaVuSansMono.ttf}[Path=\detokenize{/Users/apple/opt/anaconda3/envs/kaolin/lib/python3.7/site-packages/matplotlib/mpl-data/fonts/ttf/}]
%%
\begingroup%
\makeatletter%
\begin{pgfpicture}%
\pgfpathrectangle{\pgfpointorigin}{\pgfqpoint{12.564527in}{8.341596in}}%
\pgfusepath{use as bounding box, clip}%
\begin{pgfscope}%
\pgfsetbuttcap%
\pgfsetmiterjoin%
\definecolor{currentfill}{rgb}{1.000000,1.000000,1.000000}%
\pgfsetfillcolor{currentfill}%
\pgfsetlinewidth{0.000000pt}%
\definecolor{currentstroke}{rgb}{1.000000,1.000000,1.000000}%
\pgfsetstrokecolor{currentstroke}%
\pgfsetdash{}{0pt}%
\pgfpathmoveto{\pgfqpoint{0.000000in}{0.000000in}}%
\pgfpathlineto{\pgfqpoint{12.564527in}{0.000000in}}%
\pgfpathlineto{\pgfqpoint{12.564527in}{8.341596in}}%
\pgfpathlineto{\pgfqpoint{0.000000in}{8.341596in}}%
\pgfpathclose%
\pgfusepath{fill}%
\end{pgfscope}%
\begin{pgfscope}%
\pgfsetbuttcap%
\pgfsetmiterjoin%
\definecolor{currentfill}{rgb}{1.000000,1.000000,1.000000}%
\pgfsetfillcolor{currentfill}%
\pgfsetlinewidth{0.000000pt}%
\definecolor{currentstroke}{rgb}{0.000000,0.000000,0.000000}%
\pgfsetstrokecolor{currentstroke}%
\pgfsetstrokeopacity{0.000000}%
\pgfsetdash{}{0pt}%
\pgfpathmoveto{\pgfqpoint{0.570343in}{0.331635in}}%
\pgfpathlineto{\pgfqpoint{9.870343in}{0.331635in}}%
\pgfpathlineto{\pgfqpoint{9.870343in}{8.031635in}}%
\pgfpathlineto{\pgfqpoint{0.570343in}{8.031635in}}%
\pgfpathclose%
\pgfusepath{fill}%
\end{pgfscope}%
\begin{pgfscope}%
\pgfpathrectangle{\pgfqpoint{0.570343in}{0.331635in}}{\pgfqpoint{9.300000in}{7.700000in}}%
\pgfusepath{clip}%
\pgfsetbuttcap%
\pgfsetroundjoin%
\definecolor{currentfill}{rgb}{0.631373,0.788235,0.956863}%
\pgfsetfillcolor{currentfill}%
\pgfsetlinewidth{0.481800pt}%
\definecolor{currentstroke}{rgb}{1.000000,1.000000,1.000000}%
\pgfsetstrokecolor{currentstroke}%
\pgfsetdash{}{0pt}%
\pgfpathmoveto{\pgfqpoint{5.786839in}{4.744793in}}%
\pgfpathcurveto{\pgfqpoint{5.797889in}{4.744793in}}{\pgfqpoint{5.808488in}{4.749183in}}{\pgfqpoint{5.816302in}{4.756997in}}%
\pgfpathcurveto{\pgfqpoint{5.824116in}{4.764810in}}{\pgfqpoint{5.828506in}{4.775410in}}{\pgfqpoint{5.828506in}{4.786460in}}%
\pgfpathcurveto{\pgfqpoint{5.828506in}{4.797510in}}{\pgfqpoint{5.824116in}{4.808109in}}{\pgfqpoint{5.816302in}{4.815922in}}%
\pgfpathcurveto{\pgfqpoint{5.808488in}{4.823736in}}{\pgfqpoint{5.797889in}{4.828126in}}{\pgfqpoint{5.786839in}{4.828126in}}%
\pgfpathcurveto{\pgfqpoint{5.775789in}{4.828126in}}{\pgfqpoint{5.765190in}{4.823736in}}{\pgfqpoint{5.757376in}{4.815922in}}%
\pgfpathcurveto{\pgfqpoint{5.749563in}{4.808109in}}{\pgfqpoint{5.745173in}{4.797510in}}{\pgfqpoint{5.745173in}{4.786460in}}%
\pgfpathcurveto{\pgfqpoint{5.745173in}{4.775410in}}{\pgfqpoint{5.749563in}{4.764810in}}{\pgfqpoint{5.757376in}{4.756997in}}%
\pgfpathcurveto{\pgfqpoint{5.765190in}{4.749183in}}{\pgfqpoint{5.775789in}{4.744793in}}{\pgfqpoint{5.786839in}{4.744793in}}%
\pgfpathclose%
\pgfusepath{stroke,fill}%
\end{pgfscope}%
\begin{pgfscope}%
\pgfpathrectangle{\pgfqpoint{0.570343in}{0.331635in}}{\pgfqpoint{9.300000in}{7.700000in}}%
\pgfusepath{clip}%
\pgfsetbuttcap%
\pgfsetroundjoin%
\definecolor{currentfill}{rgb}{0.631373,0.788235,0.956863}%
\pgfsetfillcolor{currentfill}%
\pgfsetlinewidth{0.481800pt}%
\definecolor{currentstroke}{rgb}{1.000000,1.000000,1.000000}%
\pgfsetstrokecolor{currentstroke}%
\pgfsetdash{}{0pt}%
\pgfpathmoveto{\pgfqpoint{6.090094in}{6.702331in}}%
\pgfpathcurveto{\pgfqpoint{6.101144in}{6.702331in}}{\pgfqpoint{6.111743in}{6.706721in}}{\pgfqpoint{6.119557in}{6.714535in}}%
\pgfpathcurveto{\pgfqpoint{6.127370in}{6.722348in}}{\pgfqpoint{6.131761in}{6.732947in}}{\pgfqpoint{6.131761in}{6.743997in}}%
\pgfpathcurveto{\pgfqpoint{6.131761in}{6.755048in}}{\pgfqpoint{6.127370in}{6.765647in}}{\pgfqpoint{6.119557in}{6.773460in}}%
\pgfpathcurveto{\pgfqpoint{6.111743in}{6.781274in}}{\pgfqpoint{6.101144in}{6.785664in}}{\pgfqpoint{6.090094in}{6.785664in}}%
\pgfpathcurveto{\pgfqpoint{6.079044in}{6.785664in}}{\pgfqpoint{6.068445in}{6.781274in}}{\pgfqpoint{6.060631in}{6.773460in}}%
\pgfpathcurveto{\pgfqpoint{6.052817in}{6.765647in}}{\pgfqpoint{6.048427in}{6.755048in}}{\pgfqpoint{6.048427in}{6.743997in}}%
\pgfpathcurveto{\pgfqpoint{6.048427in}{6.732947in}}{\pgfqpoint{6.052817in}{6.722348in}}{\pgfqpoint{6.060631in}{6.714535in}}%
\pgfpathcurveto{\pgfqpoint{6.068445in}{6.706721in}}{\pgfqpoint{6.079044in}{6.702331in}}{\pgfqpoint{6.090094in}{6.702331in}}%
\pgfpathclose%
\pgfusepath{stroke,fill}%
\end{pgfscope}%
\begin{pgfscope}%
\pgfpathrectangle{\pgfqpoint{0.570343in}{0.331635in}}{\pgfqpoint{9.300000in}{7.700000in}}%
\pgfusepath{clip}%
\pgfsetbuttcap%
\pgfsetroundjoin%
\definecolor{currentfill}{rgb}{0.631373,0.788235,0.956863}%
\pgfsetfillcolor{currentfill}%
\pgfsetlinewidth{0.481800pt}%
\definecolor{currentstroke}{rgb}{1.000000,1.000000,1.000000}%
\pgfsetstrokecolor{currentstroke}%
\pgfsetdash{}{0pt}%
\pgfpathmoveto{\pgfqpoint{6.096102in}{3.996168in}}%
\pgfpathcurveto{\pgfqpoint{6.107152in}{3.996168in}}{\pgfqpoint{6.117751in}{4.000559in}}{\pgfqpoint{6.125565in}{4.008372in}}%
\pgfpathcurveto{\pgfqpoint{6.133378in}{4.016186in}}{\pgfqpoint{6.137769in}{4.026785in}}{\pgfqpoint{6.137769in}{4.037835in}}%
\pgfpathcurveto{\pgfqpoint{6.137769in}{4.048885in}}{\pgfqpoint{6.133378in}{4.059484in}}{\pgfqpoint{6.125565in}{4.067298in}}%
\pgfpathcurveto{\pgfqpoint{6.117751in}{4.075111in}}{\pgfqpoint{6.107152in}{4.079502in}}{\pgfqpoint{6.096102in}{4.079502in}}%
\pgfpathcurveto{\pgfqpoint{6.085052in}{4.079502in}}{\pgfqpoint{6.074453in}{4.075111in}}{\pgfqpoint{6.066639in}{4.067298in}}%
\pgfpathcurveto{\pgfqpoint{6.058825in}{4.059484in}}{\pgfqpoint{6.054435in}{4.048885in}}{\pgfqpoint{6.054435in}{4.037835in}}%
\pgfpathcurveto{\pgfqpoint{6.054435in}{4.026785in}}{\pgfqpoint{6.058825in}{4.016186in}}{\pgfqpoint{6.066639in}{4.008372in}}%
\pgfpathcurveto{\pgfqpoint{6.074453in}{4.000559in}}{\pgfqpoint{6.085052in}{3.996168in}}{\pgfqpoint{6.096102in}{3.996168in}}%
\pgfpathclose%
\pgfusepath{stroke,fill}%
\end{pgfscope}%
\begin{pgfscope}%
\pgfpathrectangle{\pgfqpoint{0.570343in}{0.331635in}}{\pgfqpoint{9.300000in}{7.700000in}}%
\pgfusepath{clip}%
\pgfsetbuttcap%
\pgfsetroundjoin%
\definecolor{currentfill}{rgb}{0.631373,0.788235,0.956863}%
\pgfsetfillcolor{currentfill}%
\pgfsetlinewidth{0.481800pt}%
\definecolor{currentstroke}{rgb}{1.000000,1.000000,1.000000}%
\pgfsetstrokecolor{currentstroke}%
\pgfsetdash{}{0pt}%
\pgfpathmoveto{\pgfqpoint{6.085130in}{3.550185in}}%
\pgfpathcurveto{\pgfqpoint{6.096180in}{3.550185in}}{\pgfqpoint{6.106779in}{3.554575in}}{\pgfqpoint{6.114593in}{3.562389in}}%
\pgfpathcurveto{\pgfqpoint{6.122407in}{3.570202in}}{\pgfqpoint{6.126797in}{3.580801in}}{\pgfqpoint{6.126797in}{3.591851in}}%
\pgfpathcurveto{\pgfqpoint{6.126797in}{3.602902in}}{\pgfqpoint{6.122407in}{3.613501in}}{\pgfqpoint{6.114593in}{3.621314in}}%
\pgfpathcurveto{\pgfqpoint{6.106779in}{3.629128in}}{\pgfqpoint{6.096180in}{3.633518in}}{\pgfqpoint{6.085130in}{3.633518in}}%
\pgfpathcurveto{\pgfqpoint{6.074080in}{3.633518in}}{\pgfqpoint{6.063481in}{3.629128in}}{\pgfqpoint{6.055667in}{3.621314in}}%
\pgfpathcurveto{\pgfqpoint{6.047854in}{3.613501in}}{\pgfqpoint{6.043464in}{3.602902in}}{\pgfqpoint{6.043464in}{3.591851in}}%
\pgfpathcurveto{\pgfqpoint{6.043464in}{3.580801in}}{\pgfqpoint{6.047854in}{3.570202in}}{\pgfqpoint{6.055667in}{3.562389in}}%
\pgfpathcurveto{\pgfqpoint{6.063481in}{3.554575in}}{\pgfqpoint{6.074080in}{3.550185in}}{\pgfqpoint{6.085130in}{3.550185in}}%
\pgfpathclose%
\pgfusepath{stroke,fill}%
\end{pgfscope}%
\begin{pgfscope}%
\pgfpathrectangle{\pgfqpoint{0.570343in}{0.331635in}}{\pgfqpoint{9.300000in}{7.700000in}}%
\pgfusepath{clip}%
\pgfsetbuttcap%
\pgfsetroundjoin%
\definecolor{currentfill}{rgb}{0.631373,0.788235,0.956863}%
\pgfsetfillcolor{currentfill}%
\pgfsetlinewidth{0.481800pt}%
\definecolor{currentstroke}{rgb}{1.000000,1.000000,1.000000}%
\pgfsetstrokecolor{currentstroke}%
\pgfsetdash{}{0pt}%
\pgfpathmoveto{\pgfqpoint{5.699074in}{3.875738in}}%
\pgfpathcurveto{\pgfqpoint{5.710124in}{3.875738in}}{\pgfqpoint{5.720723in}{3.880128in}}{\pgfqpoint{5.728537in}{3.887941in}}%
\pgfpathcurveto{\pgfqpoint{5.736350in}{3.895755in}}{\pgfqpoint{5.740741in}{3.906354in}}{\pgfqpoint{5.740741in}{3.917404in}}%
\pgfpathcurveto{\pgfqpoint{5.740741in}{3.928454in}}{\pgfqpoint{5.736350in}{3.939053in}}{\pgfqpoint{5.728537in}{3.946867in}}%
\pgfpathcurveto{\pgfqpoint{5.720723in}{3.954681in}}{\pgfqpoint{5.710124in}{3.959071in}}{\pgfqpoint{5.699074in}{3.959071in}}%
\pgfpathcurveto{\pgfqpoint{5.688024in}{3.959071in}}{\pgfqpoint{5.677425in}{3.954681in}}{\pgfqpoint{5.669611in}{3.946867in}}%
\pgfpathcurveto{\pgfqpoint{5.661797in}{3.939053in}}{\pgfqpoint{5.657407in}{3.928454in}}{\pgfqpoint{5.657407in}{3.917404in}}%
\pgfpathcurveto{\pgfqpoint{5.657407in}{3.906354in}}{\pgfqpoint{5.661797in}{3.895755in}}{\pgfqpoint{5.669611in}{3.887941in}}%
\pgfpathcurveto{\pgfqpoint{5.677425in}{3.880128in}}{\pgfqpoint{5.688024in}{3.875738in}}{\pgfqpoint{5.699074in}{3.875738in}}%
\pgfpathclose%
\pgfusepath{stroke,fill}%
\end{pgfscope}%
\begin{pgfscope}%
\pgfpathrectangle{\pgfqpoint{0.570343in}{0.331635in}}{\pgfqpoint{9.300000in}{7.700000in}}%
\pgfusepath{clip}%
\pgfsetbuttcap%
\pgfsetroundjoin%
\definecolor{currentfill}{rgb}{0.631373,0.788235,0.956863}%
\pgfsetfillcolor{currentfill}%
\pgfsetlinewidth{0.481800pt}%
\definecolor{currentstroke}{rgb}{1.000000,1.000000,1.000000}%
\pgfsetstrokecolor{currentstroke}%
\pgfsetdash{}{0pt}%
\pgfpathmoveto{\pgfqpoint{6.170060in}{7.070477in}}%
\pgfpathcurveto{\pgfqpoint{6.181110in}{7.070477in}}{\pgfqpoint{6.191709in}{7.074867in}}{\pgfqpoint{6.199523in}{7.082681in}}%
\pgfpathcurveto{\pgfqpoint{6.207337in}{7.090494in}}{\pgfqpoint{6.211727in}{7.101093in}}{\pgfqpoint{6.211727in}{7.112143in}}%
\pgfpathcurveto{\pgfqpoint{6.211727in}{7.123193in}}{\pgfqpoint{6.207337in}{7.133793in}}{\pgfqpoint{6.199523in}{7.141606in}}%
\pgfpathcurveto{\pgfqpoint{6.191709in}{7.149420in}}{\pgfqpoint{6.181110in}{7.153810in}}{\pgfqpoint{6.170060in}{7.153810in}}%
\pgfpathcurveto{\pgfqpoint{6.159010in}{7.153810in}}{\pgfqpoint{6.148411in}{7.149420in}}{\pgfqpoint{6.140597in}{7.141606in}}%
\pgfpathcurveto{\pgfqpoint{6.132784in}{7.133793in}}{\pgfqpoint{6.128394in}{7.123193in}}{\pgfqpoint{6.128394in}{7.112143in}}%
\pgfpathcurveto{\pgfqpoint{6.128394in}{7.101093in}}{\pgfqpoint{6.132784in}{7.090494in}}{\pgfqpoint{6.140597in}{7.082681in}}%
\pgfpathcurveto{\pgfqpoint{6.148411in}{7.074867in}}{\pgfqpoint{6.159010in}{7.070477in}}{\pgfqpoint{6.170060in}{7.070477in}}%
\pgfpathclose%
\pgfusepath{stroke,fill}%
\end{pgfscope}%
\begin{pgfscope}%
\pgfpathrectangle{\pgfqpoint{0.570343in}{0.331635in}}{\pgfqpoint{9.300000in}{7.700000in}}%
\pgfusepath{clip}%
\pgfsetbuttcap%
\pgfsetroundjoin%
\definecolor{currentfill}{rgb}{0.631373,0.788235,0.956863}%
\pgfsetfillcolor{currentfill}%
\pgfsetlinewidth{0.481800pt}%
\definecolor{currentstroke}{rgb}{1.000000,1.000000,1.000000}%
\pgfsetstrokecolor{currentstroke}%
\pgfsetdash{}{0pt}%
\pgfpathmoveto{\pgfqpoint{5.816201in}{5.312139in}}%
\pgfpathcurveto{\pgfqpoint{5.827251in}{5.312139in}}{\pgfqpoint{5.837850in}{5.316529in}}{\pgfqpoint{5.845663in}{5.324343in}}%
\pgfpathcurveto{\pgfqpoint{5.853477in}{5.332157in}}{\pgfqpoint{5.857867in}{5.342756in}}{\pgfqpoint{5.857867in}{5.353806in}}%
\pgfpathcurveto{\pgfqpoint{5.857867in}{5.364856in}}{\pgfqpoint{5.853477in}{5.375455in}}{\pgfqpoint{5.845663in}{5.383269in}}%
\pgfpathcurveto{\pgfqpoint{5.837850in}{5.391082in}}{\pgfqpoint{5.827251in}{5.395472in}}{\pgfqpoint{5.816201in}{5.395472in}}%
\pgfpathcurveto{\pgfqpoint{5.805151in}{5.395472in}}{\pgfqpoint{5.794552in}{5.391082in}}{\pgfqpoint{5.786738in}{5.383269in}}%
\pgfpathcurveto{\pgfqpoint{5.778924in}{5.375455in}}{\pgfqpoint{5.774534in}{5.364856in}}{\pgfqpoint{5.774534in}{5.353806in}}%
\pgfpathcurveto{\pgfqpoint{5.774534in}{5.342756in}}{\pgfqpoint{5.778924in}{5.332157in}}{\pgfqpoint{5.786738in}{5.324343in}}%
\pgfpathcurveto{\pgfqpoint{5.794552in}{5.316529in}}{\pgfqpoint{5.805151in}{5.312139in}}{\pgfqpoint{5.816201in}{5.312139in}}%
\pgfpathclose%
\pgfusepath{stroke,fill}%
\end{pgfscope}%
\begin{pgfscope}%
\pgfpathrectangle{\pgfqpoint{0.570343in}{0.331635in}}{\pgfqpoint{9.300000in}{7.700000in}}%
\pgfusepath{clip}%
\pgfsetbuttcap%
\pgfsetroundjoin%
\definecolor{currentfill}{rgb}{0.631373,0.788235,0.956863}%
\pgfsetfillcolor{currentfill}%
\pgfsetlinewidth{0.481800pt}%
\definecolor{currentstroke}{rgb}{1.000000,1.000000,1.000000}%
\pgfsetstrokecolor{currentstroke}%
\pgfsetdash{}{0pt}%
\pgfpathmoveto{\pgfqpoint{6.056307in}{3.080409in}}%
\pgfpathcurveto{\pgfqpoint{6.067357in}{3.080409in}}{\pgfqpoint{6.077956in}{3.084799in}}{\pgfqpoint{6.085770in}{3.092612in}}%
\pgfpathcurveto{\pgfqpoint{6.093584in}{3.100426in}}{\pgfqpoint{6.097974in}{3.111025in}}{\pgfqpoint{6.097974in}{3.122075in}}%
\pgfpathcurveto{\pgfqpoint{6.097974in}{3.133125in}}{\pgfqpoint{6.093584in}{3.143724in}}{\pgfqpoint{6.085770in}{3.151538in}}%
\pgfpathcurveto{\pgfqpoint{6.077956in}{3.159352in}}{\pgfqpoint{6.067357in}{3.163742in}}{\pgfqpoint{6.056307in}{3.163742in}}%
\pgfpathcurveto{\pgfqpoint{6.045257in}{3.163742in}}{\pgfqpoint{6.034658in}{3.159352in}}{\pgfqpoint{6.026844in}{3.151538in}}%
\pgfpathcurveto{\pgfqpoint{6.019031in}{3.143724in}}{\pgfqpoint{6.014640in}{3.133125in}}{\pgfqpoint{6.014640in}{3.122075in}}%
\pgfpathcurveto{\pgfqpoint{6.014640in}{3.111025in}}{\pgfqpoint{6.019031in}{3.100426in}}{\pgfqpoint{6.026844in}{3.092612in}}%
\pgfpathcurveto{\pgfqpoint{6.034658in}{3.084799in}}{\pgfqpoint{6.045257in}{3.080409in}}{\pgfqpoint{6.056307in}{3.080409in}}%
\pgfpathclose%
\pgfusepath{stroke,fill}%
\end{pgfscope}%
\begin{pgfscope}%
\pgfpathrectangle{\pgfqpoint{0.570343in}{0.331635in}}{\pgfqpoint{9.300000in}{7.700000in}}%
\pgfusepath{clip}%
\pgfsetbuttcap%
\pgfsetroundjoin%
\definecolor{currentfill}{rgb}{0.631373,0.788235,0.956863}%
\pgfsetfillcolor{currentfill}%
\pgfsetlinewidth{0.481800pt}%
\definecolor{currentstroke}{rgb}{1.000000,1.000000,1.000000}%
\pgfsetstrokecolor{currentstroke}%
\pgfsetdash{}{0pt}%
\pgfpathmoveto{\pgfqpoint{6.035930in}{4.622081in}}%
\pgfpathcurveto{\pgfqpoint{6.046981in}{4.622081in}}{\pgfqpoint{6.057580in}{4.626471in}}{\pgfqpoint{6.065393in}{4.634284in}}%
\pgfpathcurveto{\pgfqpoint{6.073207in}{4.642098in}}{\pgfqpoint{6.077597in}{4.652697in}}{\pgfqpoint{6.077597in}{4.663747in}}%
\pgfpathcurveto{\pgfqpoint{6.077597in}{4.674797in}}{\pgfqpoint{6.073207in}{4.685396in}}{\pgfqpoint{6.065393in}{4.693210in}}%
\pgfpathcurveto{\pgfqpoint{6.057580in}{4.701024in}}{\pgfqpoint{6.046981in}{4.705414in}}{\pgfqpoint{6.035930in}{4.705414in}}%
\pgfpathcurveto{\pgfqpoint{6.024880in}{4.705414in}}{\pgfqpoint{6.014281in}{4.701024in}}{\pgfqpoint{6.006468in}{4.693210in}}%
\pgfpathcurveto{\pgfqpoint{5.998654in}{4.685396in}}{\pgfqpoint{5.994264in}{4.674797in}}{\pgfqpoint{5.994264in}{4.663747in}}%
\pgfpathcurveto{\pgfqpoint{5.994264in}{4.652697in}}{\pgfqpoint{5.998654in}{4.642098in}}{\pgfqpoint{6.006468in}{4.634284in}}%
\pgfpathcurveto{\pgfqpoint{6.014281in}{4.626471in}}{\pgfqpoint{6.024880in}{4.622081in}}{\pgfqpoint{6.035930in}{4.622081in}}%
\pgfpathclose%
\pgfusepath{stroke,fill}%
\end{pgfscope}%
\begin{pgfscope}%
\pgfpathrectangle{\pgfqpoint{0.570343in}{0.331635in}}{\pgfqpoint{9.300000in}{7.700000in}}%
\pgfusepath{clip}%
\pgfsetbuttcap%
\pgfsetroundjoin%
\definecolor{currentfill}{rgb}{0.631373,0.788235,0.956863}%
\pgfsetfillcolor{currentfill}%
\pgfsetlinewidth{0.481800pt}%
\definecolor{currentstroke}{rgb}{1.000000,1.000000,1.000000}%
\pgfsetstrokecolor{currentstroke}%
\pgfsetdash{}{0pt}%
\pgfpathmoveto{\pgfqpoint{6.279500in}{0.791270in}}%
\pgfpathcurveto{\pgfqpoint{6.290550in}{0.791270in}}{\pgfqpoint{6.301149in}{0.795660in}}{\pgfqpoint{6.308963in}{0.803474in}}%
\pgfpathcurveto{\pgfqpoint{6.316777in}{0.811288in}}{\pgfqpoint{6.321167in}{0.821887in}}{\pgfqpoint{6.321167in}{0.832937in}}%
\pgfpathcurveto{\pgfqpoint{6.321167in}{0.843987in}}{\pgfqpoint{6.316777in}{0.854586in}}{\pgfqpoint{6.308963in}{0.862400in}}%
\pgfpathcurveto{\pgfqpoint{6.301149in}{0.870213in}}{\pgfqpoint{6.290550in}{0.874603in}}{\pgfqpoint{6.279500in}{0.874603in}}%
\pgfpathcurveto{\pgfqpoint{6.268450in}{0.874603in}}{\pgfqpoint{6.257851in}{0.870213in}}{\pgfqpoint{6.250038in}{0.862400in}}%
\pgfpathcurveto{\pgfqpoint{6.242224in}{0.854586in}}{\pgfqpoint{6.237834in}{0.843987in}}{\pgfqpoint{6.237834in}{0.832937in}}%
\pgfpathcurveto{\pgfqpoint{6.237834in}{0.821887in}}{\pgfqpoint{6.242224in}{0.811288in}}{\pgfqpoint{6.250038in}{0.803474in}}%
\pgfpathcurveto{\pgfqpoint{6.257851in}{0.795660in}}{\pgfqpoint{6.268450in}{0.791270in}}{\pgfqpoint{6.279500in}{0.791270in}}%
\pgfpathclose%
\pgfusepath{stroke,fill}%
\end{pgfscope}%
\begin{pgfscope}%
\pgfpathrectangle{\pgfqpoint{0.570343in}{0.331635in}}{\pgfqpoint{9.300000in}{7.700000in}}%
\pgfusepath{clip}%
\pgfsetbuttcap%
\pgfsetroundjoin%
\definecolor{currentfill}{rgb}{0.631373,0.788235,0.956863}%
\pgfsetfillcolor{currentfill}%
\pgfsetlinewidth{0.481800pt}%
\definecolor{currentstroke}{rgb}{1.000000,1.000000,1.000000}%
\pgfsetstrokecolor{currentstroke}%
\pgfsetdash{}{0pt}%
\pgfpathmoveto{\pgfqpoint{6.307476in}{6.103375in}}%
\pgfpathcurveto{\pgfqpoint{6.318527in}{6.103375in}}{\pgfqpoint{6.329126in}{6.107765in}}{\pgfqpoint{6.336939in}{6.115579in}}%
\pgfpathcurveto{\pgfqpoint{6.344753in}{6.123393in}}{\pgfqpoint{6.349143in}{6.133992in}}{\pgfqpoint{6.349143in}{6.145042in}}%
\pgfpathcurveto{\pgfqpoint{6.349143in}{6.156092in}}{\pgfqpoint{6.344753in}{6.166691in}}{\pgfqpoint{6.336939in}{6.174504in}}%
\pgfpathcurveto{\pgfqpoint{6.329126in}{6.182318in}}{\pgfqpoint{6.318527in}{6.186708in}}{\pgfqpoint{6.307476in}{6.186708in}}%
\pgfpathcurveto{\pgfqpoint{6.296426in}{6.186708in}}{\pgfqpoint{6.285827in}{6.182318in}}{\pgfqpoint{6.278014in}{6.174504in}}%
\pgfpathcurveto{\pgfqpoint{6.270200in}{6.166691in}}{\pgfqpoint{6.265810in}{6.156092in}}{\pgfqpoint{6.265810in}{6.145042in}}%
\pgfpathcurveto{\pgfqpoint{6.265810in}{6.133992in}}{\pgfqpoint{6.270200in}{6.123393in}}{\pgfqpoint{6.278014in}{6.115579in}}%
\pgfpathcurveto{\pgfqpoint{6.285827in}{6.107765in}}{\pgfqpoint{6.296426in}{6.103375in}}{\pgfqpoint{6.307476in}{6.103375in}}%
\pgfpathclose%
\pgfusepath{stroke,fill}%
\end{pgfscope}%
\begin{pgfscope}%
\pgfpathrectangle{\pgfqpoint{0.570343in}{0.331635in}}{\pgfqpoint{9.300000in}{7.700000in}}%
\pgfusepath{clip}%
\pgfsetbuttcap%
\pgfsetroundjoin%
\definecolor{currentfill}{rgb}{0.631373,0.788235,0.956863}%
\pgfsetfillcolor{currentfill}%
\pgfsetlinewidth{0.481800pt}%
\definecolor{currentstroke}{rgb}{1.000000,1.000000,1.000000}%
\pgfsetstrokecolor{currentstroke}%
\pgfsetdash{}{0pt}%
\pgfpathmoveto{\pgfqpoint{6.108461in}{7.491204in}}%
\pgfpathcurveto{\pgfqpoint{6.119511in}{7.491204in}}{\pgfqpoint{6.130110in}{7.495594in}}{\pgfqpoint{6.137924in}{7.503408in}}%
\pgfpathcurveto{\pgfqpoint{6.145737in}{7.511221in}}{\pgfqpoint{6.150128in}{7.521821in}}{\pgfqpoint{6.150128in}{7.532871in}}%
\pgfpathcurveto{\pgfqpoint{6.150128in}{7.543921in}}{\pgfqpoint{6.145737in}{7.554520in}}{\pgfqpoint{6.137924in}{7.562333in}}%
\pgfpathcurveto{\pgfqpoint{6.130110in}{7.570147in}}{\pgfqpoint{6.119511in}{7.574537in}}{\pgfqpoint{6.108461in}{7.574537in}}%
\pgfpathcurveto{\pgfqpoint{6.097411in}{7.574537in}}{\pgfqpoint{6.086812in}{7.570147in}}{\pgfqpoint{6.078998in}{7.562333in}}%
\pgfpathcurveto{\pgfqpoint{6.071185in}{7.554520in}}{\pgfqpoint{6.066794in}{7.543921in}}{\pgfqpoint{6.066794in}{7.532871in}}%
\pgfpathcurveto{\pgfqpoint{6.066794in}{7.521821in}}{\pgfqpoint{6.071185in}{7.511221in}}{\pgfqpoint{6.078998in}{7.503408in}}%
\pgfpathcurveto{\pgfqpoint{6.086812in}{7.495594in}}{\pgfqpoint{6.097411in}{7.491204in}}{\pgfqpoint{6.108461in}{7.491204in}}%
\pgfpathclose%
\pgfusepath{stroke,fill}%
\end{pgfscope}%
\begin{pgfscope}%
\pgfpathrectangle{\pgfqpoint{0.570343in}{0.331635in}}{\pgfqpoint{9.300000in}{7.700000in}}%
\pgfusepath{clip}%
\pgfsetbuttcap%
\pgfsetroundjoin%
\definecolor{currentfill}{rgb}{0.631373,0.788235,0.956863}%
\pgfsetfillcolor{currentfill}%
\pgfsetlinewidth{0.481800pt}%
\definecolor{currentstroke}{rgb}{1.000000,1.000000,1.000000}%
\pgfsetstrokecolor{currentstroke}%
\pgfsetdash{}{0pt}%
\pgfpathmoveto{\pgfqpoint{6.404411in}{1.033468in}}%
\pgfpathcurveto{\pgfqpoint{6.415462in}{1.033468in}}{\pgfqpoint{6.426061in}{1.037858in}}{\pgfqpoint{6.433874in}{1.045672in}}%
\pgfpathcurveto{\pgfqpoint{6.441688in}{1.053486in}}{\pgfqpoint{6.446078in}{1.064085in}}{\pgfqpoint{6.446078in}{1.075135in}}%
\pgfpathcurveto{\pgfqpoint{6.446078in}{1.086185in}}{\pgfqpoint{6.441688in}{1.096784in}}{\pgfqpoint{6.433874in}{1.104598in}}%
\pgfpathcurveto{\pgfqpoint{6.426061in}{1.112411in}}{\pgfqpoint{6.415462in}{1.116801in}}{\pgfqpoint{6.404411in}{1.116801in}}%
\pgfpathcurveto{\pgfqpoint{6.393361in}{1.116801in}}{\pgfqpoint{6.382762in}{1.112411in}}{\pgfqpoint{6.374949in}{1.104598in}}%
\pgfpathcurveto{\pgfqpoint{6.367135in}{1.096784in}}{\pgfqpoint{6.362745in}{1.086185in}}{\pgfqpoint{6.362745in}{1.075135in}}%
\pgfpathcurveto{\pgfqpoint{6.362745in}{1.064085in}}{\pgfqpoint{6.367135in}{1.053486in}}{\pgfqpoint{6.374949in}{1.045672in}}%
\pgfpathcurveto{\pgfqpoint{6.382762in}{1.037858in}}{\pgfqpoint{6.393361in}{1.033468in}}{\pgfqpoint{6.404411in}{1.033468in}}%
\pgfpathclose%
\pgfusepath{stroke,fill}%
\end{pgfscope}%
\begin{pgfscope}%
\pgfpathrectangle{\pgfqpoint{0.570343in}{0.331635in}}{\pgfqpoint{9.300000in}{7.700000in}}%
\pgfusepath{clip}%
\pgfsetbuttcap%
\pgfsetroundjoin%
\definecolor{currentfill}{rgb}{0.631373,0.788235,0.956863}%
\pgfsetfillcolor{currentfill}%
\pgfsetlinewidth{0.481800pt}%
\definecolor{currentstroke}{rgb}{1.000000,1.000000,1.000000}%
\pgfsetstrokecolor{currentstroke}%
\pgfsetdash{}{0pt}%
\pgfpathmoveto{\pgfqpoint{6.178369in}{7.639968in}}%
\pgfpathcurveto{\pgfqpoint{6.189419in}{7.639968in}}{\pgfqpoint{6.200018in}{7.644359in}}{\pgfqpoint{6.207832in}{7.652172in}}%
\pgfpathcurveto{\pgfqpoint{6.215646in}{7.659986in}}{\pgfqpoint{6.220036in}{7.670585in}}{\pgfqpoint{6.220036in}{7.681635in}}%
\pgfpathcurveto{\pgfqpoint{6.220036in}{7.692685in}}{\pgfqpoint{6.215646in}{7.703284in}}{\pgfqpoint{6.207832in}{7.711098in}}%
\pgfpathcurveto{\pgfqpoint{6.200018in}{7.718911in}}{\pgfqpoint{6.189419in}{7.723302in}}{\pgfqpoint{6.178369in}{7.723302in}}%
\pgfpathcurveto{\pgfqpoint{6.167319in}{7.723302in}}{\pgfqpoint{6.156720in}{7.718911in}}{\pgfqpoint{6.148906in}{7.711098in}}%
\pgfpathcurveto{\pgfqpoint{6.141093in}{7.703284in}}{\pgfqpoint{6.136703in}{7.692685in}}{\pgfqpoint{6.136703in}{7.681635in}}%
\pgfpathcurveto{\pgfqpoint{6.136703in}{7.670585in}}{\pgfqpoint{6.141093in}{7.659986in}}{\pgfqpoint{6.148906in}{7.652172in}}%
\pgfpathcurveto{\pgfqpoint{6.156720in}{7.644359in}}{\pgfqpoint{6.167319in}{7.639968in}}{\pgfqpoint{6.178369in}{7.639968in}}%
\pgfpathclose%
\pgfusepath{stroke,fill}%
\end{pgfscope}%
\begin{pgfscope}%
\pgfpathrectangle{\pgfqpoint{0.570343in}{0.331635in}}{\pgfqpoint{9.300000in}{7.700000in}}%
\pgfusepath{clip}%
\pgfsetbuttcap%
\pgfsetroundjoin%
\definecolor{currentfill}{rgb}{0.631373,0.788235,0.956863}%
\pgfsetfillcolor{currentfill}%
\pgfsetlinewidth{0.481800pt}%
\definecolor{currentstroke}{rgb}{1.000000,1.000000,1.000000}%
\pgfsetstrokecolor{currentstroke}%
\pgfsetdash{}{0pt}%
\pgfpathmoveto{\pgfqpoint{6.121234in}{3.059476in}}%
\pgfpathcurveto{\pgfqpoint{6.132284in}{3.059476in}}{\pgfqpoint{6.142883in}{3.063866in}}{\pgfqpoint{6.150696in}{3.071680in}}%
\pgfpathcurveto{\pgfqpoint{6.158510in}{3.079494in}}{\pgfqpoint{6.162900in}{3.090093in}}{\pgfqpoint{6.162900in}{3.101143in}}%
\pgfpathcurveto{\pgfqpoint{6.162900in}{3.112193in}}{\pgfqpoint{6.158510in}{3.122792in}}{\pgfqpoint{6.150696in}{3.130606in}}%
\pgfpathcurveto{\pgfqpoint{6.142883in}{3.138419in}}{\pgfqpoint{6.132284in}{3.142810in}}{\pgfqpoint{6.121234in}{3.142810in}}%
\pgfpathcurveto{\pgfqpoint{6.110184in}{3.142810in}}{\pgfqpoint{6.099584in}{3.138419in}}{\pgfqpoint{6.091771in}{3.130606in}}%
\pgfpathcurveto{\pgfqpoint{6.083957in}{3.122792in}}{\pgfqpoint{6.079567in}{3.112193in}}{\pgfqpoint{6.079567in}{3.101143in}}%
\pgfpathcurveto{\pgfqpoint{6.079567in}{3.090093in}}{\pgfqpoint{6.083957in}{3.079494in}}{\pgfqpoint{6.091771in}{3.071680in}}%
\pgfpathcurveto{\pgfqpoint{6.099584in}{3.063866in}}{\pgfqpoint{6.110184in}{3.059476in}}{\pgfqpoint{6.121234in}{3.059476in}}%
\pgfpathclose%
\pgfusepath{stroke,fill}%
\end{pgfscope}%
\begin{pgfscope}%
\pgfpathrectangle{\pgfqpoint{0.570343in}{0.331635in}}{\pgfqpoint{9.300000in}{7.700000in}}%
\pgfusepath{clip}%
\pgfsetbuttcap%
\pgfsetroundjoin%
\definecolor{currentfill}{rgb}{0.631373,0.788235,0.956863}%
\pgfsetfillcolor{currentfill}%
\pgfsetlinewidth{0.481800pt}%
\definecolor{currentstroke}{rgb}{1.000000,1.000000,1.000000}%
\pgfsetstrokecolor{currentstroke}%
\pgfsetdash{}{0pt}%
\pgfpathmoveto{\pgfqpoint{6.298085in}{7.628616in}}%
\pgfpathcurveto{\pgfqpoint{6.309136in}{7.628616in}}{\pgfqpoint{6.319735in}{7.633006in}}{\pgfqpoint{6.327548in}{7.640820in}}%
\pgfpathcurveto{\pgfqpoint{6.335362in}{7.648634in}}{\pgfqpoint{6.339752in}{7.659233in}}{\pgfqpoint{6.339752in}{7.670283in}}%
\pgfpathcurveto{\pgfqpoint{6.339752in}{7.681333in}}{\pgfqpoint{6.335362in}{7.691932in}}{\pgfqpoint{6.327548in}{7.699746in}}%
\pgfpathcurveto{\pgfqpoint{6.319735in}{7.707559in}}{\pgfqpoint{6.309136in}{7.711949in}}{\pgfqpoint{6.298085in}{7.711949in}}%
\pgfpathcurveto{\pgfqpoint{6.287035in}{7.711949in}}{\pgfqpoint{6.276436in}{7.707559in}}{\pgfqpoint{6.268623in}{7.699746in}}%
\pgfpathcurveto{\pgfqpoint{6.260809in}{7.691932in}}{\pgfqpoint{6.256419in}{7.681333in}}{\pgfqpoint{6.256419in}{7.670283in}}%
\pgfpathcurveto{\pgfqpoint{6.256419in}{7.659233in}}{\pgfqpoint{6.260809in}{7.648634in}}{\pgfqpoint{6.268623in}{7.640820in}}%
\pgfpathcurveto{\pgfqpoint{6.276436in}{7.633006in}}{\pgfqpoint{6.287035in}{7.628616in}}{\pgfqpoint{6.298085in}{7.628616in}}%
\pgfpathclose%
\pgfusepath{stroke,fill}%
\end{pgfscope}%
\begin{pgfscope}%
\pgfpathrectangle{\pgfqpoint{0.570343in}{0.331635in}}{\pgfqpoint{9.300000in}{7.700000in}}%
\pgfusepath{clip}%
\pgfsetbuttcap%
\pgfsetroundjoin%
\definecolor{currentfill}{rgb}{0.631373,0.788235,0.956863}%
\pgfsetfillcolor{currentfill}%
\pgfsetlinewidth{0.481800pt}%
\definecolor{currentstroke}{rgb}{1.000000,1.000000,1.000000}%
\pgfsetstrokecolor{currentstroke}%
\pgfsetdash{}{0pt}%
\pgfpathmoveto{\pgfqpoint{6.239187in}{7.282471in}}%
\pgfpathcurveto{\pgfqpoint{6.250237in}{7.282471in}}{\pgfqpoint{6.260836in}{7.286861in}}{\pgfqpoint{6.268649in}{7.294675in}}%
\pgfpathcurveto{\pgfqpoint{6.276463in}{7.302489in}}{\pgfqpoint{6.280853in}{7.313088in}}{\pgfqpoint{6.280853in}{7.324138in}}%
\pgfpathcurveto{\pgfqpoint{6.280853in}{7.335188in}}{\pgfqpoint{6.276463in}{7.345787in}}{\pgfqpoint{6.268649in}{7.353601in}}%
\pgfpathcurveto{\pgfqpoint{6.260836in}{7.361414in}}{\pgfqpoint{6.250237in}{7.365805in}}{\pgfqpoint{6.239187in}{7.365805in}}%
\pgfpathcurveto{\pgfqpoint{6.228137in}{7.365805in}}{\pgfqpoint{6.217538in}{7.361414in}}{\pgfqpoint{6.209724in}{7.353601in}}%
\pgfpathcurveto{\pgfqpoint{6.201910in}{7.345787in}}{\pgfqpoint{6.197520in}{7.335188in}}{\pgfqpoint{6.197520in}{7.324138in}}%
\pgfpathcurveto{\pgfqpoint{6.197520in}{7.313088in}}{\pgfqpoint{6.201910in}{7.302489in}}{\pgfqpoint{6.209724in}{7.294675in}}%
\pgfpathcurveto{\pgfqpoint{6.217538in}{7.286861in}}{\pgfqpoint{6.228137in}{7.282471in}}{\pgfqpoint{6.239187in}{7.282471in}}%
\pgfpathclose%
\pgfusepath{stroke,fill}%
\end{pgfscope}%
\begin{pgfscope}%
\pgfpathrectangle{\pgfqpoint{0.570343in}{0.331635in}}{\pgfqpoint{9.300000in}{7.700000in}}%
\pgfusepath{clip}%
\pgfsetbuttcap%
\pgfsetroundjoin%
\definecolor{currentfill}{rgb}{0.631373,0.788235,0.956863}%
\pgfsetfillcolor{currentfill}%
\pgfsetlinewidth{0.481800pt}%
\definecolor{currentstroke}{rgb}{1.000000,1.000000,1.000000}%
\pgfsetstrokecolor{currentstroke}%
\pgfsetdash{}{0pt}%
\pgfpathmoveto{\pgfqpoint{5.924096in}{5.544005in}}%
\pgfpathcurveto{\pgfqpoint{5.935146in}{5.544005in}}{\pgfqpoint{5.945745in}{5.548395in}}{\pgfqpoint{5.953559in}{5.556209in}}%
\pgfpathcurveto{\pgfqpoint{5.961373in}{5.564022in}}{\pgfqpoint{5.965763in}{5.574621in}}{\pgfqpoint{5.965763in}{5.585671in}}%
\pgfpathcurveto{\pgfqpoint{5.965763in}{5.596722in}}{\pgfqpoint{5.961373in}{5.607321in}}{\pgfqpoint{5.953559in}{5.615134in}}%
\pgfpathcurveto{\pgfqpoint{5.945745in}{5.622948in}}{\pgfqpoint{5.935146in}{5.627338in}}{\pgfqpoint{5.924096in}{5.627338in}}%
\pgfpathcurveto{\pgfqpoint{5.913046in}{5.627338in}}{\pgfqpoint{5.902447in}{5.622948in}}{\pgfqpoint{5.894633in}{5.615134in}}%
\pgfpathcurveto{\pgfqpoint{5.886820in}{5.607321in}}{\pgfqpoint{5.882430in}{5.596722in}}{\pgfqpoint{5.882430in}{5.585671in}}%
\pgfpathcurveto{\pgfqpoint{5.882430in}{5.574621in}}{\pgfqpoint{5.886820in}{5.564022in}}{\pgfqpoint{5.894633in}{5.556209in}}%
\pgfpathcurveto{\pgfqpoint{5.902447in}{5.548395in}}{\pgfqpoint{5.913046in}{5.544005in}}{\pgfqpoint{5.924096in}{5.544005in}}%
\pgfpathclose%
\pgfusepath{stroke,fill}%
\end{pgfscope}%
\begin{pgfscope}%
\pgfpathrectangle{\pgfqpoint{0.570343in}{0.331635in}}{\pgfqpoint{9.300000in}{7.700000in}}%
\pgfusepath{clip}%
\pgfsetbuttcap%
\pgfsetroundjoin%
\definecolor{currentfill}{rgb}{0.631373,0.788235,0.956863}%
\pgfsetfillcolor{currentfill}%
\pgfsetlinewidth{0.481800pt}%
\definecolor{currentstroke}{rgb}{1.000000,1.000000,1.000000}%
\pgfsetstrokecolor{currentstroke}%
\pgfsetdash{}{0pt}%
\pgfpathmoveto{\pgfqpoint{5.694819in}{4.412253in}}%
\pgfpathcurveto{\pgfqpoint{5.705869in}{4.412253in}}{\pgfqpoint{5.716469in}{4.416643in}}{\pgfqpoint{5.724282in}{4.424457in}}%
\pgfpathcurveto{\pgfqpoint{5.732096in}{4.432270in}}{\pgfqpoint{5.736486in}{4.442869in}}{\pgfqpoint{5.736486in}{4.453919in}}%
\pgfpathcurveto{\pgfqpoint{5.736486in}{4.464970in}}{\pgfqpoint{5.732096in}{4.475569in}}{\pgfqpoint{5.724282in}{4.483382in}}%
\pgfpathcurveto{\pgfqpoint{5.716469in}{4.491196in}}{\pgfqpoint{5.705869in}{4.495586in}}{\pgfqpoint{5.694819in}{4.495586in}}%
\pgfpathcurveto{\pgfqpoint{5.683769in}{4.495586in}}{\pgfqpoint{5.673170in}{4.491196in}}{\pgfqpoint{5.665357in}{4.483382in}}%
\pgfpathcurveto{\pgfqpoint{5.657543in}{4.475569in}}{\pgfqpoint{5.653153in}{4.464970in}}{\pgfqpoint{5.653153in}{4.453919in}}%
\pgfpathcurveto{\pgfqpoint{5.653153in}{4.442869in}}{\pgfqpoint{5.657543in}{4.432270in}}{\pgfqpoint{5.665357in}{4.424457in}}%
\pgfpathcurveto{\pgfqpoint{5.673170in}{4.416643in}}{\pgfqpoint{5.683769in}{4.412253in}}{\pgfqpoint{5.694819in}{4.412253in}}%
\pgfpathclose%
\pgfusepath{stroke,fill}%
\end{pgfscope}%
\begin{pgfscope}%
\pgfpathrectangle{\pgfqpoint{0.570343in}{0.331635in}}{\pgfqpoint{9.300000in}{7.700000in}}%
\pgfusepath{clip}%
\pgfsetbuttcap%
\pgfsetroundjoin%
\definecolor{currentfill}{rgb}{0.631373,0.788235,0.956863}%
\pgfsetfillcolor{currentfill}%
\pgfsetlinewidth{0.481800pt}%
\definecolor{currentstroke}{rgb}{1.000000,1.000000,1.000000}%
\pgfsetstrokecolor{currentstroke}%
\pgfsetdash{}{0pt}%
\pgfpathmoveto{\pgfqpoint{6.024489in}{2.416007in}}%
\pgfpathcurveto{\pgfqpoint{6.035539in}{2.416007in}}{\pgfqpoint{6.046138in}{2.420397in}}{\pgfqpoint{6.053952in}{2.428211in}}%
\pgfpathcurveto{\pgfqpoint{6.061765in}{2.436025in}}{\pgfqpoint{6.066155in}{2.446624in}}{\pgfqpoint{6.066155in}{2.457674in}}%
\pgfpathcurveto{\pgfqpoint{6.066155in}{2.468724in}}{\pgfqpoint{6.061765in}{2.479323in}}{\pgfqpoint{6.053952in}{2.487137in}}%
\pgfpathcurveto{\pgfqpoint{6.046138in}{2.494950in}}{\pgfqpoint{6.035539in}{2.499340in}}{\pgfqpoint{6.024489in}{2.499340in}}%
\pgfpathcurveto{\pgfqpoint{6.013439in}{2.499340in}}{\pgfqpoint{6.002840in}{2.494950in}}{\pgfqpoint{5.995026in}{2.487137in}}%
\pgfpathcurveto{\pgfqpoint{5.987212in}{2.479323in}}{\pgfqpoint{5.982822in}{2.468724in}}{\pgfqpoint{5.982822in}{2.457674in}}%
\pgfpathcurveto{\pgfqpoint{5.982822in}{2.446624in}}{\pgfqpoint{5.987212in}{2.436025in}}{\pgfqpoint{5.995026in}{2.428211in}}%
\pgfpathcurveto{\pgfqpoint{6.002840in}{2.420397in}}{\pgfqpoint{6.013439in}{2.416007in}}{\pgfqpoint{6.024489in}{2.416007in}}%
\pgfpathclose%
\pgfusepath{stroke,fill}%
\end{pgfscope}%
\begin{pgfscope}%
\pgfpathrectangle{\pgfqpoint{0.570343in}{0.331635in}}{\pgfqpoint{9.300000in}{7.700000in}}%
\pgfusepath{clip}%
\pgfsetbuttcap%
\pgfsetroundjoin%
\definecolor{currentfill}{rgb}{0.631373,0.788235,0.956863}%
\pgfsetfillcolor{currentfill}%
\pgfsetlinewidth{0.481800pt}%
\definecolor{currentstroke}{rgb}{1.000000,1.000000,1.000000}%
\pgfsetstrokecolor{currentstroke}%
\pgfsetdash{}{0pt}%
\pgfpathmoveto{\pgfqpoint{6.436145in}{4.355918in}}%
\pgfpathcurveto{\pgfqpoint{6.447195in}{4.355918in}}{\pgfqpoint{6.457794in}{4.360308in}}{\pgfqpoint{6.465607in}{4.368121in}}%
\pgfpathcurveto{\pgfqpoint{6.473421in}{4.375935in}}{\pgfqpoint{6.477811in}{4.386534in}}{\pgfqpoint{6.477811in}{4.397584in}}%
\pgfpathcurveto{\pgfqpoint{6.477811in}{4.408634in}}{\pgfqpoint{6.473421in}{4.419233in}}{\pgfqpoint{6.465607in}{4.427047in}}%
\pgfpathcurveto{\pgfqpoint{6.457794in}{4.434861in}}{\pgfqpoint{6.447195in}{4.439251in}}{\pgfqpoint{6.436145in}{4.439251in}}%
\pgfpathcurveto{\pgfqpoint{6.425094in}{4.439251in}}{\pgfqpoint{6.414495in}{4.434861in}}{\pgfqpoint{6.406682in}{4.427047in}}%
\pgfpathcurveto{\pgfqpoint{6.398868in}{4.419233in}}{\pgfqpoint{6.394478in}{4.408634in}}{\pgfqpoint{6.394478in}{4.397584in}}%
\pgfpathcurveto{\pgfqpoint{6.394478in}{4.386534in}}{\pgfqpoint{6.398868in}{4.375935in}}{\pgfqpoint{6.406682in}{4.368121in}}%
\pgfpathcurveto{\pgfqpoint{6.414495in}{4.360308in}}{\pgfqpoint{6.425094in}{4.355918in}}{\pgfqpoint{6.436145in}{4.355918in}}%
\pgfpathclose%
\pgfusepath{stroke,fill}%
\end{pgfscope}%
\begin{pgfscope}%
\pgfpathrectangle{\pgfqpoint{0.570343in}{0.331635in}}{\pgfqpoint{9.300000in}{7.700000in}}%
\pgfusepath{clip}%
\pgfsetbuttcap%
\pgfsetroundjoin%
\definecolor{currentfill}{rgb}{0.631373,0.788235,0.956863}%
\pgfsetfillcolor{currentfill}%
\pgfsetlinewidth{0.481800pt}%
\definecolor{currentstroke}{rgb}{1.000000,1.000000,1.000000}%
\pgfsetstrokecolor{currentstroke}%
\pgfsetdash{}{0pt}%
\pgfpathmoveto{\pgfqpoint{6.011284in}{5.161591in}}%
\pgfpathcurveto{\pgfqpoint{6.022334in}{5.161591in}}{\pgfqpoint{6.032934in}{5.165981in}}{\pgfqpoint{6.040747in}{5.173795in}}%
\pgfpathcurveto{\pgfqpoint{6.048561in}{5.181608in}}{\pgfqpoint{6.052951in}{5.192207in}}{\pgfqpoint{6.052951in}{5.203258in}}%
\pgfpathcurveto{\pgfqpoint{6.052951in}{5.214308in}}{\pgfqpoint{6.048561in}{5.224907in}}{\pgfqpoint{6.040747in}{5.232720in}}%
\pgfpathcurveto{\pgfqpoint{6.032934in}{5.240534in}}{\pgfqpoint{6.022334in}{5.244924in}}{\pgfqpoint{6.011284in}{5.244924in}}%
\pgfpathcurveto{\pgfqpoint{6.000234in}{5.244924in}}{\pgfqpoint{5.989635in}{5.240534in}}{\pgfqpoint{5.981822in}{5.232720in}}%
\pgfpathcurveto{\pgfqpoint{5.974008in}{5.224907in}}{\pgfqpoint{5.969618in}{5.214308in}}{\pgfqpoint{5.969618in}{5.203258in}}%
\pgfpathcurveto{\pgfqpoint{5.969618in}{5.192207in}}{\pgfqpoint{5.974008in}{5.181608in}}{\pgfqpoint{5.981822in}{5.173795in}}%
\pgfpathcurveto{\pgfqpoint{5.989635in}{5.165981in}}{\pgfqpoint{6.000234in}{5.161591in}}{\pgfqpoint{6.011284in}{5.161591in}}%
\pgfpathclose%
\pgfusepath{stroke,fill}%
\end{pgfscope}%
\begin{pgfscope}%
\pgfpathrectangle{\pgfqpoint{0.570343in}{0.331635in}}{\pgfqpoint{9.300000in}{7.700000in}}%
\pgfusepath{clip}%
\pgfsetbuttcap%
\pgfsetroundjoin%
\definecolor{currentfill}{rgb}{0.631373,0.788235,0.956863}%
\pgfsetfillcolor{currentfill}%
\pgfsetlinewidth{0.481800pt}%
\definecolor{currentstroke}{rgb}{1.000000,1.000000,1.000000}%
\pgfsetstrokecolor{currentstroke}%
\pgfsetdash{}{0pt}%
\pgfpathmoveto{\pgfqpoint{6.108299in}{5.922048in}}%
\pgfpathcurveto{\pgfqpoint{6.119349in}{5.922048in}}{\pgfqpoint{6.129948in}{5.926438in}}{\pgfqpoint{6.137762in}{5.934252in}}%
\pgfpathcurveto{\pgfqpoint{6.145575in}{5.942066in}}{\pgfqpoint{6.149966in}{5.952665in}}{\pgfqpoint{6.149966in}{5.963715in}}%
\pgfpathcurveto{\pgfqpoint{6.149966in}{5.974765in}}{\pgfqpoint{6.145575in}{5.985364in}}{\pgfqpoint{6.137762in}{5.993178in}}%
\pgfpathcurveto{\pgfqpoint{6.129948in}{6.000991in}}{\pgfqpoint{6.119349in}{6.005381in}}{\pgfqpoint{6.108299in}{6.005381in}}%
\pgfpathcurveto{\pgfqpoint{6.097249in}{6.005381in}}{\pgfqpoint{6.086650in}{6.000991in}}{\pgfqpoint{6.078836in}{5.993178in}}%
\pgfpathcurveto{\pgfqpoint{6.071023in}{5.985364in}}{\pgfqpoint{6.066632in}{5.974765in}}{\pgfqpoint{6.066632in}{5.963715in}}%
\pgfpathcurveto{\pgfqpoint{6.066632in}{5.952665in}}{\pgfqpoint{6.071023in}{5.942066in}}{\pgfqpoint{6.078836in}{5.934252in}}%
\pgfpathcurveto{\pgfqpoint{6.086650in}{5.926438in}}{\pgfqpoint{6.097249in}{5.922048in}}{\pgfqpoint{6.108299in}{5.922048in}}%
\pgfpathclose%
\pgfusepath{stroke,fill}%
\end{pgfscope}%
\begin{pgfscope}%
\pgfpathrectangle{\pgfqpoint{0.570343in}{0.331635in}}{\pgfqpoint{9.300000in}{7.700000in}}%
\pgfusepath{clip}%
\pgfsetbuttcap%
\pgfsetroundjoin%
\definecolor{currentfill}{rgb}{0.631373,0.788235,0.956863}%
\pgfsetfillcolor{currentfill}%
\pgfsetlinewidth{0.481800pt}%
\definecolor{currentstroke}{rgb}{1.000000,1.000000,1.000000}%
\pgfsetstrokecolor{currentstroke}%
\pgfsetdash{}{0pt}%
\pgfpathmoveto{\pgfqpoint{6.344974in}{3.764795in}}%
\pgfpathcurveto{\pgfqpoint{6.356024in}{3.764795in}}{\pgfqpoint{6.366623in}{3.769186in}}{\pgfqpoint{6.374437in}{3.776999in}}%
\pgfpathcurveto{\pgfqpoint{6.382251in}{3.784813in}}{\pgfqpoint{6.386641in}{3.795412in}}{\pgfqpoint{6.386641in}{3.806462in}}%
\pgfpathcurveto{\pgfqpoint{6.386641in}{3.817512in}}{\pgfqpoint{6.382251in}{3.828111in}}{\pgfqpoint{6.374437in}{3.835925in}}%
\pgfpathcurveto{\pgfqpoint{6.366623in}{3.843738in}}{\pgfqpoint{6.356024in}{3.848129in}}{\pgfqpoint{6.344974in}{3.848129in}}%
\pgfpathcurveto{\pgfqpoint{6.333924in}{3.848129in}}{\pgfqpoint{6.323325in}{3.843738in}}{\pgfqpoint{6.315511in}{3.835925in}}%
\pgfpathcurveto{\pgfqpoint{6.307698in}{3.828111in}}{\pgfqpoint{6.303308in}{3.817512in}}{\pgfqpoint{6.303308in}{3.806462in}}%
\pgfpathcurveto{\pgfqpoint{6.303308in}{3.795412in}}{\pgfqpoint{6.307698in}{3.784813in}}{\pgfqpoint{6.315511in}{3.776999in}}%
\pgfpathcurveto{\pgfqpoint{6.323325in}{3.769186in}}{\pgfqpoint{6.333924in}{3.764795in}}{\pgfqpoint{6.344974in}{3.764795in}}%
\pgfpathclose%
\pgfusepath{stroke,fill}%
\end{pgfscope}%
\begin{pgfscope}%
\pgfpathrectangle{\pgfqpoint{0.570343in}{0.331635in}}{\pgfqpoint{9.300000in}{7.700000in}}%
\pgfusepath{clip}%
\pgfsetbuttcap%
\pgfsetroundjoin%
\definecolor{currentfill}{rgb}{0.631373,0.788235,0.956863}%
\pgfsetfillcolor{currentfill}%
\pgfsetlinewidth{0.481800pt}%
\definecolor{currentstroke}{rgb}{1.000000,1.000000,1.000000}%
\pgfsetstrokecolor{currentstroke}%
\pgfsetdash{}{0pt}%
\pgfpathmoveto{\pgfqpoint{6.386006in}{3.059107in}}%
\pgfpathcurveto{\pgfqpoint{6.397056in}{3.059107in}}{\pgfqpoint{6.407655in}{3.063497in}}{\pgfqpoint{6.415468in}{3.071311in}}%
\pgfpathcurveto{\pgfqpoint{6.423282in}{3.079124in}}{\pgfqpoint{6.427672in}{3.089723in}}{\pgfqpoint{6.427672in}{3.100773in}}%
\pgfpathcurveto{\pgfqpoint{6.427672in}{3.111824in}}{\pgfqpoint{6.423282in}{3.122423in}}{\pgfqpoint{6.415468in}{3.130236in}}%
\pgfpathcurveto{\pgfqpoint{6.407655in}{3.138050in}}{\pgfqpoint{6.397056in}{3.142440in}}{\pgfqpoint{6.386006in}{3.142440in}}%
\pgfpathcurveto{\pgfqpoint{6.374955in}{3.142440in}}{\pgfqpoint{6.364356in}{3.138050in}}{\pgfqpoint{6.356543in}{3.130236in}}%
\pgfpathcurveto{\pgfqpoint{6.348729in}{3.122423in}}{\pgfqpoint{6.344339in}{3.111824in}}{\pgfqpoint{6.344339in}{3.100773in}}%
\pgfpathcurveto{\pgfqpoint{6.344339in}{3.089723in}}{\pgfqpoint{6.348729in}{3.079124in}}{\pgfqpoint{6.356543in}{3.071311in}}%
\pgfpathcurveto{\pgfqpoint{6.364356in}{3.063497in}}{\pgfqpoint{6.374955in}{3.059107in}}{\pgfqpoint{6.386006in}{3.059107in}}%
\pgfpathclose%
\pgfusepath{stroke,fill}%
\end{pgfscope}%
\begin{pgfscope}%
\pgfpathrectangle{\pgfqpoint{0.570343in}{0.331635in}}{\pgfqpoint{9.300000in}{7.700000in}}%
\pgfusepath{clip}%
\pgfsetbuttcap%
\pgfsetroundjoin%
\definecolor{currentfill}{rgb}{0.631373,0.788235,0.956863}%
\pgfsetfillcolor{currentfill}%
\pgfsetlinewidth{0.481800pt}%
\definecolor{currentstroke}{rgb}{1.000000,1.000000,1.000000}%
\pgfsetstrokecolor{currentstroke}%
\pgfsetdash{}{0pt}%
\pgfpathmoveto{\pgfqpoint{5.830354in}{4.657339in}}%
\pgfpathcurveto{\pgfqpoint{5.841405in}{4.657339in}}{\pgfqpoint{5.852004in}{4.661730in}}{\pgfqpoint{5.859817in}{4.669543in}}%
\pgfpathcurveto{\pgfqpoint{5.867631in}{4.677357in}}{\pgfqpoint{5.872021in}{4.687956in}}{\pgfqpoint{5.872021in}{4.699006in}}%
\pgfpathcurveto{\pgfqpoint{5.872021in}{4.710056in}}{\pgfqpoint{5.867631in}{4.720655in}}{\pgfqpoint{5.859817in}{4.728469in}}%
\pgfpathcurveto{\pgfqpoint{5.852004in}{4.736282in}}{\pgfqpoint{5.841405in}{4.740673in}}{\pgfqpoint{5.830354in}{4.740673in}}%
\pgfpathcurveto{\pgfqpoint{5.819304in}{4.740673in}}{\pgfqpoint{5.808705in}{4.736282in}}{\pgfqpoint{5.800892in}{4.728469in}}%
\pgfpathcurveto{\pgfqpoint{5.793078in}{4.720655in}}{\pgfqpoint{5.788688in}{4.710056in}}{\pgfqpoint{5.788688in}{4.699006in}}%
\pgfpathcurveto{\pgfqpoint{5.788688in}{4.687956in}}{\pgfqpoint{5.793078in}{4.677357in}}{\pgfqpoint{5.800892in}{4.669543in}}%
\pgfpathcurveto{\pgfqpoint{5.808705in}{4.661730in}}{\pgfqpoint{5.819304in}{4.657339in}}{\pgfqpoint{5.830354in}{4.657339in}}%
\pgfpathclose%
\pgfusepath{stroke,fill}%
\end{pgfscope}%
\begin{pgfscope}%
\pgfpathrectangle{\pgfqpoint{0.570343in}{0.331635in}}{\pgfqpoint{9.300000in}{7.700000in}}%
\pgfusepath{clip}%
\pgfsetbuttcap%
\pgfsetroundjoin%
\definecolor{currentfill}{rgb}{0.631373,0.788235,0.956863}%
\pgfsetfillcolor{currentfill}%
\pgfsetlinewidth{0.481800pt}%
\definecolor{currentstroke}{rgb}{1.000000,1.000000,1.000000}%
\pgfsetstrokecolor{currentstroke}%
\pgfsetdash{}{0pt}%
\pgfpathmoveto{\pgfqpoint{6.363645in}{4.738210in}}%
\pgfpathcurveto{\pgfqpoint{6.374696in}{4.738210in}}{\pgfqpoint{6.385295in}{4.742600in}}{\pgfqpoint{6.393108in}{4.750414in}}%
\pgfpathcurveto{\pgfqpoint{6.400922in}{4.758227in}}{\pgfqpoint{6.405312in}{4.768826in}}{\pgfqpoint{6.405312in}{4.779876in}}%
\pgfpathcurveto{\pgfqpoint{6.405312in}{4.790926in}}{\pgfqpoint{6.400922in}{4.801525in}}{\pgfqpoint{6.393108in}{4.809339in}}%
\pgfpathcurveto{\pgfqpoint{6.385295in}{4.817153in}}{\pgfqpoint{6.374696in}{4.821543in}}{\pgfqpoint{6.363645in}{4.821543in}}%
\pgfpathcurveto{\pgfqpoint{6.352595in}{4.821543in}}{\pgfqpoint{6.341996in}{4.817153in}}{\pgfqpoint{6.334183in}{4.809339in}}%
\pgfpathcurveto{\pgfqpoint{6.326369in}{4.801525in}}{\pgfqpoint{6.321979in}{4.790926in}}{\pgfqpoint{6.321979in}{4.779876in}}%
\pgfpathcurveto{\pgfqpoint{6.321979in}{4.768826in}}{\pgfqpoint{6.326369in}{4.758227in}}{\pgfqpoint{6.334183in}{4.750414in}}%
\pgfpathcurveto{\pgfqpoint{6.341996in}{4.742600in}}{\pgfqpoint{6.352595in}{4.738210in}}{\pgfqpoint{6.363645in}{4.738210in}}%
\pgfpathclose%
\pgfusepath{stroke,fill}%
\end{pgfscope}%
\begin{pgfscope}%
\pgfpathrectangle{\pgfqpoint{0.570343in}{0.331635in}}{\pgfqpoint{9.300000in}{7.700000in}}%
\pgfusepath{clip}%
\pgfsetbuttcap%
\pgfsetroundjoin%
\definecolor{currentfill}{rgb}{0.631373,0.788235,0.956863}%
\pgfsetfillcolor{currentfill}%
\pgfsetlinewidth{0.481800pt}%
\definecolor{currentstroke}{rgb}{1.000000,1.000000,1.000000}%
\pgfsetstrokecolor{currentstroke}%
\pgfsetdash{}{0pt}%
\pgfpathmoveto{\pgfqpoint{6.416113in}{3.360636in}}%
\pgfpathcurveto{\pgfqpoint{6.427163in}{3.360636in}}{\pgfqpoint{6.437762in}{3.365026in}}{\pgfqpoint{6.445576in}{3.372839in}}%
\pgfpathcurveto{\pgfqpoint{6.453389in}{3.380653in}}{\pgfqpoint{6.457779in}{3.391252in}}{\pgfqpoint{6.457779in}{3.402302in}}%
\pgfpathcurveto{\pgfqpoint{6.457779in}{3.413352in}}{\pgfqpoint{6.453389in}{3.423951in}}{\pgfqpoint{6.445576in}{3.431765in}}%
\pgfpathcurveto{\pgfqpoint{6.437762in}{3.439579in}}{\pgfqpoint{6.427163in}{3.443969in}}{\pgfqpoint{6.416113in}{3.443969in}}%
\pgfpathcurveto{\pgfqpoint{6.405063in}{3.443969in}}{\pgfqpoint{6.394464in}{3.439579in}}{\pgfqpoint{6.386650in}{3.431765in}}%
\pgfpathcurveto{\pgfqpoint{6.378836in}{3.423951in}}{\pgfqpoint{6.374446in}{3.413352in}}{\pgfqpoint{6.374446in}{3.402302in}}%
\pgfpathcurveto{\pgfqpoint{6.374446in}{3.391252in}}{\pgfqpoint{6.378836in}{3.380653in}}{\pgfqpoint{6.386650in}{3.372839in}}%
\pgfpathcurveto{\pgfqpoint{6.394464in}{3.365026in}}{\pgfqpoint{6.405063in}{3.360636in}}{\pgfqpoint{6.416113in}{3.360636in}}%
\pgfpathclose%
\pgfusepath{stroke,fill}%
\end{pgfscope}%
\begin{pgfscope}%
\pgfpathrectangle{\pgfqpoint{0.570343in}{0.331635in}}{\pgfqpoint{9.300000in}{7.700000in}}%
\pgfusepath{clip}%
\pgfsetbuttcap%
\pgfsetroundjoin%
\definecolor{currentfill}{rgb}{0.631373,0.788235,0.956863}%
\pgfsetfillcolor{currentfill}%
\pgfsetlinewidth{0.481800pt}%
\definecolor{currentstroke}{rgb}{1.000000,1.000000,1.000000}%
\pgfsetstrokecolor{currentstroke}%
\pgfsetdash{}{0pt}%
\pgfpathmoveto{\pgfqpoint{5.950563in}{1.866560in}}%
\pgfpathcurveto{\pgfqpoint{5.961613in}{1.866560in}}{\pgfqpoint{5.972212in}{1.870950in}}{\pgfqpoint{5.980026in}{1.878763in}}%
\pgfpathcurveto{\pgfqpoint{5.987840in}{1.886577in}}{\pgfqpoint{5.992230in}{1.897176in}}{\pgfqpoint{5.992230in}{1.908226in}}%
\pgfpathcurveto{\pgfqpoint{5.992230in}{1.919276in}}{\pgfqpoint{5.987840in}{1.929875in}}{\pgfqpoint{5.980026in}{1.937689in}}%
\pgfpathcurveto{\pgfqpoint{5.972212in}{1.945503in}}{\pgfqpoint{5.961613in}{1.949893in}}{\pgfqpoint{5.950563in}{1.949893in}}%
\pgfpathcurveto{\pgfqpoint{5.939513in}{1.949893in}}{\pgfqpoint{5.928914in}{1.945503in}}{\pgfqpoint{5.921100in}{1.937689in}}%
\pgfpathcurveto{\pgfqpoint{5.913287in}{1.929875in}}{\pgfqpoint{5.908897in}{1.919276in}}{\pgfqpoint{5.908897in}{1.908226in}}%
\pgfpathcurveto{\pgfqpoint{5.908897in}{1.897176in}}{\pgfqpoint{5.913287in}{1.886577in}}{\pgfqpoint{5.921100in}{1.878763in}}%
\pgfpathcurveto{\pgfqpoint{5.928914in}{1.870950in}}{\pgfqpoint{5.939513in}{1.866560in}}{\pgfqpoint{5.950563in}{1.866560in}}%
\pgfpathclose%
\pgfusepath{stroke,fill}%
\end{pgfscope}%
\begin{pgfscope}%
\pgfpathrectangle{\pgfqpoint{0.570343in}{0.331635in}}{\pgfqpoint{9.300000in}{7.700000in}}%
\pgfusepath{clip}%
\pgfsetbuttcap%
\pgfsetroundjoin%
\definecolor{currentfill}{rgb}{0.631373,0.788235,0.956863}%
\pgfsetfillcolor{currentfill}%
\pgfsetlinewidth{0.481800pt}%
\definecolor{currentstroke}{rgb}{1.000000,1.000000,1.000000}%
\pgfsetstrokecolor{currentstroke}%
\pgfsetdash{}{0pt}%
\pgfpathmoveto{\pgfqpoint{6.254873in}{6.674586in}}%
\pgfpathcurveto{\pgfqpoint{6.265923in}{6.674586in}}{\pgfqpoint{6.276522in}{6.678977in}}{\pgfqpoint{6.284336in}{6.686790in}}%
\pgfpathcurveto{\pgfqpoint{6.292150in}{6.694604in}}{\pgfqpoint{6.296540in}{6.705203in}}{\pgfqpoint{6.296540in}{6.716253in}}%
\pgfpathcurveto{\pgfqpoint{6.296540in}{6.727303in}}{\pgfqpoint{6.292150in}{6.737902in}}{\pgfqpoint{6.284336in}{6.745716in}}%
\pgfpathcurveto{\pgfqpoint{6.276522in}{6.753530in}}{\pgfqpoint{6.265923in}{6.757920in}}{\pgfqpoint{6.254873in}{6.757920in}}%
\pgfpathcurveto{\pgfqpoint{6.243823in}{6.757920in}}{\pgfqpoint{6.233224in}{6.753530in}}{\pgfqpoint{6.225410in}{6.745716in}}%
\pgfpathcurveto{\pgfqpoint{6.217597in}{6.737902in}}{\pgfqpoint{6.213206in}{6.727303in}}{\pgfqpoint{6.213206in}{6.716253in}}%
\pgfpathcurveto{\pgfqpoint{6.213206in}{6.705203in}}{\pgfqpoint{6.217597in}{6.694604in}}{\pgfqpoint{6.225410in}{6.686790in}}%
\pgfpathcurveto{\pgfqpoint{6.233224in}{6.678977in}}{\pgfqpoint{6.243823in}{6.674586in}}{\pgfqpoint{6.254873in}{6.674586in}}%
\pgfpathclose%
\pgfusepath{stroke,fill}%
\end{pgfscope}%
\begin{pgfscope}%
\pgfpathrectangle{\pgfqpoint{0.570343in}{0.331635in}}{\pgfqpoint{9.300000in}{7.700000in}}%
\pgfusepath{clip}%
\pgfsetbuttcap%
\pgfsetroundjoin%
\definecolor{currentfill}{rgb}{0.631373,0.788235,0.956863}%
\pgfsetfillcolor{currentfill}%
\pgfsetlinewidth{0.481800pt}%
\definecolor{currentstroke}{rgb}{1.000000,1.000000,1.000000}%
\pgfsetstrokecolor{currentstroke}%
\pgfsetdash{}{0pt}%
\pgfpathmoveto{\pgfqpoint{6.143314in}{6.510792in}}%
\pgfpathcurveto{\pgfqpoint{6.154364in}{6.510792in}}{\pgfqpoint{6.164963in}{6.515182in}}{\pgfqpoint{6.172777in}{6.522995in}}%
\pgfpathcurveto{\pgfqpoint{6.180590in}{6.530809in}}{\pgfqpoint{6.184980in}{6.541408in}}{\pgfqpoint{6.184980in}{6.552458in}}%
\pgfpathcurveto{\pgfqpoint{6.184980in}{6.563508in}}{\pgfqpoint{6.180590in}{6.574107in}}{\pgfqpoint{6.172777in}{6.581921in}}%
\pgfpathcurveto{\pgfqpoint{6.164963in}{6.589735in}}{\pgfqpoint{6.154364in}{6.594125in}}{\pgfqpoint{6.143314in}{6.594125in}}%
\pgfpathcurveto{\pgfqpoint{6.132264in}{6.594125in}}{\pgfqpoint{6.121665in}{6.589735in}}{\pgfqpoint{6.113851in}{6.581921in}}%
\pgfpathcurveto{\pgfqpoint{6.106037in}{6.574107in}}{\pgfqpoint{6.101647in}{6.563508in}}{\pgfqpoint{6.101647in}{6.552458in}}%
\pgfpathcurveto{\pgfqpoint{6.101647in}{6.541408in}}{\pgfqpoint{6.106037in}{6.530809in}}{\pgfqpoint{6.113851in}{6.522995in}}%
\pgfpathcurveto{\pgfqpoint{6.121665in}{6.515182in}}{\pgfqpoint{6.132264in}{6.510792in}}{\pgfqpoint{6.143314in}{6.510792in}}%
\pgfpathclose%
\pgfusepath{stroke,fill}%
\end{pgfscope}%
\begin{pgfscope}%
\pgfpathrectangle{\pgfqpoint{0.570343in}{0.331635in}}{\pgfqpoint{9.300000in}{7.700000in}}%
\pgfusepath{clip}%
\pgfsetbuttcap%
\pgfsetroundjoin%
\definecolor{currentfill}{rgb}{0.631373,0.788235,0.956863}%
\pgfsetfillcolor{currentfill}%
\pgfsetlinewidth{0.481800pt}%
\definecolor{currentstroke}{rgb}{1.000000,1.000000,1.000000}%
\pgfsetstrokecolor{currentstroke}%
\pgfsetdash{}{0pt}%
\pgfpathmoveto{\pgfqpoint{5.740772in}{5.142425in}}%
\pgfpathcurveto{\pgfqpoint{5.751822in}{5.142425in}}{\pgfqpoint{5.762421in}{5.146815in}}{\pgfqpoint{5.770235in}{5.154628in}}%
\pgfpathcurveto{\pgfqpoint{5.778048in}{5.162442in}}{\pgfqpoint{5.782438in}{5.173041in}}{\pgfqpoint{5.782438in}{5.184091in}}%
\pgfpathcurveto{\pgfqpoint{5.782438in}{5.195141in}}{\pgfqpoint{5.778048in}{5.205740in}}{\pgfqpoint{5.770235in}{5.213554in}}%
\pgfpathcurveto{\pgfqpoint{5.762421in}{5.221368in}}{\pgfqpoint{5.751822in}{5.225758in}}{\pgfqpoint{5.740772in}{5.225758in}}%
\pgfpathcurveto{\pgfqpoint{5.729722in}{5.225758in}}{\pgfqpoint{5.719123in}{5.221368in}}{\pgfqpoint{5.711309in}{5.213554in}}%
\pgfpathcurveto{\pgfqpoint{5.703495in}{5.205740in}}{\pgfqpoint{5.699105in}{5.195141in}}{\pgfqpoint{5.699105in}{5.184091in}}%
\pgfpathcurveto{\pgfqpoint{5.699105in}{5.173041in}}{\pgfqpoint{5.703495in}{5.162442in}}{\pgfqpoint{5.711309in}{5.154628in}}%
\pgfpathcurveto{\pgfqpoint{5.719123in}{5.146815in}}{\pgfqpoint{5.729722in}{5.142425in}}{\pgfqpoint{5.740772in}{5.142425in}}%
\pgfpathclose%
\pgfusepath{stroke,fill}%
\end{pgfscope}%
\begin{pgfscope}%
\pgfpathrectangle{\pgfqpoint{0.570343in}{0.331635in}}{\pgfqpoint{9.300000in}{7.700000in}}%
\pgfusepath{clip}%
\pgfsetbuttcap%
\pgfsetroundjoin%
\definecolor{currentfill}{rgb}{0.631373,0.788235,0.956863}%
\pgfsetfillcolor{currentfill}%
\pgfsetlinewidth{0.481800pt}%
\definecolor{currentstroke}{rgb}{1.000000,1.000000,1.000000}%
\pgfsetstrokecolor{currentstroke}%
\pgfsetdash{}{0pt}%
\pgfpathmoveto{\pgfqpoint{6.169499in}{5.937613in}}%
\pgfpathcurveto{\pgfqpoint{6.180549in}{5.937613in}}{\pgfqpoint{6.191148in}{5.942003in}}{\pgfqpoint{6.198962in}{5.949817in}}%
\pgfpathcurveto{\pgfqpoint{6.206776in}{5.957630in}}{\pgfqpoint{6.211166in}{5.968229in}}{\pgfqpoint{6.211166in}{5.979279in}}%
\pgfpathcurveto{\pgfqpoint{6.211166in}{5.990330in}}{\pgfqpoint{6.206776in}{6.000929in}}{\pgfqpoint{6.198962in}{6.008742in}}%
\pgfpathcurveto{\pgfqpoint{6.191148in}{6.016556in}}{\pgfqpoint{6.180549in}{6.020946in}}{\pgfqpoint{6.169499in}{6.020946in}}%
\pgfpathcurveto{\pgfqpoint{6.158449in}{6.020946in}}{\pgfqpoint{6.147850in}{6.016556in}}{\pgfqpoint{6.140036in}{6.008742in}}%
\pgfpathcurveto{\pgfqpoint{6.132223in}{6.000929in}}{\pgfqpoint{6.127832in}{5.990330in}}{\pgfqpoint{6.127832in}{5.979279in}}%
\pgfpathcurveto{\pgfqpoint{6.127832in}{5.968229in}}{\pgfqpoint{6.132223in}{5.957630in}}{\pgfqpoint{6.140036in}{5.949817in}}%
\pgfpathcurveto{\pgfqpoint{6.147850in}{5.942003in}}{\pgfqpoint{6.158449in}{5.937613in}}{\pgfqpoint{6.169499in}{5.937613in}}%
\pgfpathclose%
\pgfusepath{stroke,fill}%
\end{pgfscope}%
\begin{pgfscope}%
\pgfpathrectangle{\pgfqpoint{0.570343in}{0.331635in}}{\pgfqpoint{9.300000in}{7.700000in}}%
\pgfusepath{clip}%
\pgfsetbuttcap%
\pgfsetroundjoin%
\definecolor{currentfill}{rgb}{0.631373,0.788235,0.956863}%
\pgfsetfillcolor{currentfill}%
\pgfsetlinewidth{0.481800pt}%
\definecolor{currentstroke}{rgb}{1.000000,1.000000,1.000000}%
\pgfsetstrokecolor{currentstroke}%
\pgfsetdash{}{0pt}%
\pgfpathmoveto{\pgfqpoint{6.359818in}{6.155696in}}%
\pgfpathcurveto{\pgfqpoint{6.370868in}{6.155696in}}{\pgfqpoint{6.381467in}{6.160087in}}{\pgfqpoint{6.389280in}{6.167900in}}%
\pgfpathcurveto{\pgfqpoint{6.397094in}{6.175714in}}{\pgfqpoint{6.401484in}{6.186313in}}{\pgfqpoint{6.401484in}{6.197363in}}%
\pgfpathcurveto{\pgfqpoint{6.401484in}{6.208413in}}{\pgfqpoint{6.397094in}{6.219012in}}{\pgfqpoint{6.389280in}{6.226826in}}%
\pgfpathcurveto{\pgfqpoint{6.381467in}{6.234639in}}{\pgfqpoint{6.370868in}{6.239030in}}{\pgfqpoint{6.359818in}{6.239030in}}%
\pgfpathcurveto{\pgfqpoint{6.348767in}{6.239030in}}{\pgfqpoint{6.338168in}{6.234639in}}{\pgfqpoint{6.330355in}{6.226826in}}%
\pgfpathcurveto{\pgfqpoint{6.322541in}{6.219012in}}{\pgfqpoint{6.318151in}{6.208413in}}{\pgfqpoint{6.318151in}{6.197363in}}%
\pgfpathcurveto{\pgfqpoint{6.318151in}{6.186313in}}{\pgfqpoint{6.322541in}{6.175714in}}{\pgfqpoint{6.330355in}{6.167900in}}%
\pgfpathcurveto{\pgfqpoint{6.338168in}{6.160087in}}{\pgfqpoint{6.348767in}{6.155696in}}{\pgfqpoint{6.359818in}{6.155696in}}%
\pgfpathclose%
\pgfusepath{stroke,fill}%
\end{pgfscope}%
\begin{pgfscope}%
\pgfpathrectangle{\pgfqpoint{0.570343in}{0.331635in}}{\pgfqpoint{9.300000in}{7.700000in}}%
\pgfusepath{clip}%
\pgfsetbuttcap%
\pgfsetroundjoin%
\definecolor{currentfill}{rgb}{0.631373,0.788235,0.956863}%
\pgfsetfillcolor{currentfill}%
\pgfsetlinewidth{0.481800pt}%
\definecolor{currentstroke}{rgb}{1.000000,1.000000,1.000000}%
\pgfsetstrokecolor{currentstroke}%
\pgfsetdash{}{0pt}%
\pgfpathmoveto{\pgfqpoint{5.850943in}{6.774460in}}%
\pgfpathcurveto{\pgfqpoint{5.861994in}{6.774460in}}{\pgfqpoint{5.872593in}{6.778850in}}{\pgfqpoint{5.880406in}{6.786663in}}%
\pgfpathcurveto{\pgfqpoint{5.888220in}{6.794477in}}{\pgfqpoint{5.892610in}{6.805076in}}{\pgfqpoint{5.892610in}{6.816126in}}%
\pgfpathcurveto{\pgfqpoint{5.892610in}{6.827176in}}{\pgfqpoint{5.888220in}{6.837775in}}{\pgfqpoint{5.880406in}{6.845589in}}%
\pgfpathcurveto{\pgfqpoint{5.872593in}{6.853403in}}{\pgfqpoint{5.861994in}{6.857793in}}{\pgfqpoint{5.850943in}{6.857793in}}%
\pgfpathcurveto{\pgfqpoint{5.839893in}{6.857793in}}{\pgfqpoint{5.829294in}{6.853403in}}{\pgfqpoint{5.821481in}{6.845589in}}%
\pgfpathcurveto{\pgfqpoint{5.813667in}{6.837775in}}{\pgfqpoint{5.809277in}{6.827176in}}{\pgfqpoint{5.809277in}{6.816126in}}%
\pgfpathcurveto{\pgfqpoint{5.809277in}{6.805076in}}{\pgfqpoint{5.813667in}{6.794477in}}{\pgfqpoint{5.821481in}{6.786663in}}%
\pgfpathcurveto{\pgfqpoint{5.829294in}{6.778850in}}{\pgfqpoint{5.839893in}{6.774460in}}{\pgfqpoint{5.850943in}{6.774460in}}%
\pgfpathclose%
\pgfusepath{stroke,fill}%
\end{pgfscope}%
\begin{pgfscope}%
\pgfpathrectangle{\pgfqpoint{0.570343in}{0.331635in}}{\pgfqpoint{9.300000in}{7.700000in}}%
\pgfusepath{clip}%
\pgfsetbuttcap%
\pgfsetroundjoin%
\definecolor{currentfill}{rgb}{0.631373,0.788235,0.956863}%
\pgfsetfillcolor{currentfill}%
\pgfsetlinewidth{0.481800pt}%
\definecolor{currentstroke}{rgb}{1.000000,1.000000,1.000000}%
\pgfsetstrokecolor{currentstroke}%
\pgfsetdash{}{0pt}%
\pgfpathmoveto{\pgfqpoint{6.323724in}{4.356024in}}%
\pgfpathcurveto{\pgfqpoint{6.334774in}{4.356024in}}{\pgfqpoint{6.345373in}{4.360414in}}{\pgfqpoint{6.353186in}{4.368228in}}%
\pgfpathcurveto{\pgfqpoint{6.361000in}{4.376041in}}{\pgfqpoint{6.365390in}{4.386640in}}{\pgfqpoint{6.365390in}{4.397691in}}%
\pgfpathcurveto{\pgfqpoint{6.365390in}{4.408741in}}{\pgfqpoint{6.361000in}{4.419340in}}{\pgfqpoint{6.353186in}{4.427153in}}%
\pgfpathcurveto{\pgfqpoint{6.345373in}{4.434967in}}{\pgfqpoint{6.334774in}{4.439357in}}{\pgfqpoint{6.323724in}{4.439357in}}%
\pgfpathcurveto{\pgfqpoint{6.312674in}{4.439357in}}{\pgfqpoint{6.302075in}{4.434967in}}{\pgfqpoint{6.294261in}{4.427153in}}%
\pgfpathcurveto{\pgfqpoint{6.286447in}{4.419340in}}{\pgfqpoint{6.282057in}{4.408741in}}{\pgfqpoint{6.282057in}{4.397691in}}%
\pgfpathcurveto{\pgfqpoint{6.282057in}{4.386640in}}{\pgfqpoint{6.286447in}{4.376041in}}{\pgfqpoint{6.294261in}{4.368228in}}%
\pgfpathcurveto{\pgfqpoint{6.302075in}{4.360414in}}{\pgfqpoint{6.312674in}{4.356024in}}{\pgfqpoint{6.323724in}{4.356024in}}%
\pgfpathclose%
\pgfusepath{stroke,fill}%
\end{pgfscope}%
\begin{pgfscope}%
\pgfpathrectangle{\pgfqpoint{0.570343in}{0.331635in}}{\pgfqpoint{9.300000in}{7.700000in}}%
\pgfusepath{clip}%
\pgfsetbuttcap%
\pgfsetroundjoin%
\definecolor{currentfill}{rgb}{0.631373,0.788235,0.956863}%
\pgfsetfillcolor{currentfill}%
\pgfsetlinewidth{0.481800pt}%
\definecolor{currentstroke}{rgb}{1.000000,1.000000,1.000000}%
\pgfsetstrokecolor{currentstroke}%
\pgfsetdash{}{0pt}%
\pgfpathmoveto{\pgfqpoint{5.756078in}{4.076255in}}%
\pgfpathcurveto{\pgfqpoint{5.767128in}{4.076255in}}{\pgfqpoint{5.777727in}{4.080645in}}{\pgfqpoint{5.785541in}{4.088459in}}%
\pgfpathcurveto{\pgfqpoint{5.793355in}{4.096272in}}{\pgfqpoint{5.797745in}{4.106872in}}{\pgfqpoint{5.797745in}{4.117922in}}%
\pgfpathcurveto{\pgfqpoint{5.797745in}{4.128972in}}{\pgfqpoint{5.793355in}{4.139571in}}{\pgfqpoint{5.785541in}{4.147384in}}%
\pgfpathcurveto{\pgfqpoint{5.777727in}{4.155198in}}{\pgfqpoint{5.767128in}{4.159588in}}{\pgfqpoint{5.756078in}{4.159588in}}%
\pgfpathcurveto{\pgfqpoint{5.745028in}{4.159588in}}{\pgfqpoint{5.734429in}{4.155198in}}{\pgfqpoint{5.726615in}{4.147384in}}%
\pgfpathcurveto{\pgfqpoint{5.718802in}{4.139571in}}{\pgfqpoint{5.714412in}{4.128972in}}{\pgfqpoint{5.714412in}{4.117922in}}%
\pgfpathcurveto{\pgfqpoint{5.714412in}{4.106872in}}{\pgfqpoint{5.718802in}{4.096272in}}{\pgfqpoint{5.726615in}{4.088459in}}%
\pgfpathcurveto{\pgfqpoint{5.734429in}{4.080645in}}{\pgfqpoint{5.745028in}{4.076255in}}{\pgfqpoint{5.756078in}{4.076255in}}%
\pgfpathclose%
\pgfusepath{stroke,fill}%
\end{pgfscope}%
\begin{pgfscope}%
\pgfpathrectangle{\pgfqpoint{0.570343in}{0.331635in}}{\pgfqpoint{9.300000in}{7.700000in}}%
\pgfusepath{clip}%
\pgfsetbuttcap%
\pgfsetroundjoin%
\definecolor{currentfill}{rgb}{0.631373,0.788235,0.956863}%
\pgfsetfillcolor{currentfill}%
\pgfsetlinewidth{0.481800pt}%
\definecolor{currentstroke}{rgb}{1.000000,1.000000,1.000000}%
\pgfsetstrokecolor{currentstroke}%
\pgfsetdash{}{0pt}%
\pgfpathmoveto{\pgfqpoint{6.208523in}{6.231132in}}%
\pgfpathcurveto{\pgfqpoint{6.219573in}{6.231132in}}{\pgfqpoint{6.230172in}{6.235522in}}{\pgfqpoint{6.237986in}{6.243336in}}%
\pgfpathcurveto{\pgfqpoint{6.245799in}{6.251149in}}{\pgfqpoint{6.250190in}{6.261748in}}{\pgfqpoint{6.250190in}{6.272798in}}%
\pgfpathcurveto{\pgfqpoint{6.250190in}{6.283848in}}{\pgfqpoint{6.245799in}{6.294447in}}{\pgfqpoint{6.237986in}{6.302261in}}%
\pgfpathcurveto{\pgfqpoint{6.230172in}{6.310075in}}{\pgfqpoint{6.219573in}{6.314465in}}{\pgfqpoint{6.208523in}{6.314465in}}%
\pgfpathcurveto{\pgfqpoint{6.197473in}{6.314465in}}{\pgfqpoint{6.186874in}{6.310075in}}{\pgfqpoint{6.179060in}{6.302261in}}%
\pgfpathcurveto{\pgfqpoint{6.171246in}{6.294447in}}{\pgfqpoint{6.166856in}{6.283848in}}{\pgfqpoint{6.166856in}{6.272798in}}%
\pgfpathcurveto{\pgfqpoint{6.166856in}{6.261748in}}{\pgfqpoint{6.171246in}{6.251149in}}{\pgfqpoint{6.179060in}{6.243336in}}%
\pgfpathcurveto{\pgfqpoint{6.186874in}{6.235522in}}{\pgfqpoint{6.197473in}{6.231132in}}{\pgfqpoint{6.208523in}{6.231132in}}%
\pgfpathclose%
\pgfusepath{stroke,fill}%
\end{pgfscope}%
\begin{pgfscope}%
\pgfpathrectangle{\pgfqpoint{0.570343in}{0.331635in}}{\pgfqpoint{9.300000in}{7.700000in}}%
\pgfusepath{clip}%
\pgfsetbuttcap%
\pgfsetroundjoin%
\definecolor{currentfill}{rgb}{0.631373,0.788235,0.956863}%
\pgfsetfillcolor{currentfill}%
\pgfsetlinewidth{0.481800pt}%
\definecolor{currentstroke}{rgb}{1.000000,1.000000,1.000000}%
\pgfsetstrokecolor{currentstroke}%
\pgfsetdash{}{0pt}%
\pgfpathmoveto{\pgfqpoint{5.947265in}{2.970538in}}%
\pgfpathcurveto{\pgfqpoint{5.958315in}{2.970538in}}{\pgfqpoint{5.968914in}{2.974928in}}{\pgfqpoint{5.976728in}{2.982742in}}%
\pgfpathcurveto{\pgfqpoint{5.984541in}{2.990556in}}{\pgfqpoint{5.988932in}{3.001155in}}{\pgfqpoint{5.988932in}{3.012205in}}%
\pgfpathcurveto{\pgfqpoint{5.988932in}{3.023255in}}{\pgfqpoint{5.984541in}{3.033854in}}{\pgfqpoint{5.976728in}{3.041668in}}%
\pgfpathcurveto{\pgfqpoint{5.968914in}{3.049481in}}{\pgfqpoint{5.958315in}{3.053871in}}{\pgfqpoint{5.947265in}{3.053871in}}%
\pgfpathcurveto{\pgfqpoint{5.936215in}{3.053871in}}{\pgfqpoint{5.925616in}{3.049481in}}{\pgfqpoint{5.917802in}{3.041668in}}%
\pgfpathcurveto{\pgfqpoint{5.909989in}{3.033854in}}{\pgfqpoint{5.905598in}{3.023255in}}{\pgfqpoint{5.905598in}{3.012205in}}%
\pgfpathcurveto{\pgfqpoint{5.905598in}{3.001155in}}{\pgfqpoint{5.909989in}{2.990556in}}{\pgfqpoint{5.917802in}{2.982742in}}%
\pgfpathcurveto{\pgfqpoint{5.925616in}{2.974928in}}{\pgfqpoint{5.936215in}{2.970538in}}{\pgfqpoint{5.947265in}{2.970538in}}%
\pgfpathclose%
\pgfusepath{stroke,fill}%
\end{pgfscope}%
\begin{pgfscope}%
\pgfpathrectangle{\pgfqpoint{0.570343in}{0.331635in}}{\pgfqpoint{9.300000in}{7.700000in}}%
\pgfusepath{clip}%
\pgfsetbuttcap%
\pgfsetroundjoin%
\definecolor{currentfill}{rgb}{0.631373,0.788235,0.956863}%
\pgfsetfillcolor{currentfill}%
\pgfsetlinewidth{0.481800pt}%
\definecolor{currentstroke}{rgb}{1.000000,1.000000,1.000000}%
\pgfsetstrokecolor{currentstroke}%
\pgfsetdash{}{0pt}%
\pgfpathmoveto{\pgfqpoint{5.997400in}{7.439855in}}%
\pgfpathcurveto{\pgfqpoint{6.008450in}{7.439855in}}{\pgfqpoint{6.019049in}{7.444245in}}{\pgfqpoint{6.026863in}{7.452059in}}%
\pgfpathcurveto{\pgfqpoint{6.034676in}{7.459872in}}{\pgfqpoint{6.039067in}{7.470471in}}{\pgfqpoint{6.039067in}{7.481522in}}%
\pgfpathcurveto{\pgfqpoint{6.039067in}{7.492572in}}{\pgfqpoint{6.034676in}{7.503171in}}{\pgfqpoint{6.026863in}{7.510984in}}%
\pgfpathcurveto{\pgfqpoint{6.019049in}{7.518798in}}{\pgfqpoint{6.008450in}{7.523188in}}{\pgfqpoint{5.997400in}{7.523188in}}%
\pgfpathcurveto{\pgfqpoint{5.986350in}{7.523188in}}{\pgfqpoint{5.975751in}{7.518798in}}{\pgfqpoint{5.967937in}{7.510984in}}%
\pgfpathcurveto{\pgfqpoint{5.960124in}{7.503171in}}{\pgfqpoint{5.955733in}{7.492572in}}{\pgfqpoint{5.955733in}{7.481522in}}%
\pgfpathcurveto{\pgfqpoint{5.955733in}{7.470471in}}{\pgfqpoint{5.960124in}{7.459872in}}{\pgfqpoint{5.967937in}{7.452059in}}%
\pgfpathcurveto{\pgfqpoint{5.975751in}{7.444245in}}{\pgfqpoint{5.986350in}{7.439855in}}{\pgfqpoint{5.997400in}{7.439855in}}%
\pgfpathclose%
\pgfusepath{stroke,fill}%
\end{pgfscope}%
\begin{pgfscope}%
\pgfpathrectangle{\pgfqpoint{0.570343in}{0.331635in}}{\pgfqpoint{9.300000in}{7.700000in}}%
\pgfusepath{clip}%
\pgfsetbuttcap%
\pgfsetroundjoin%
\definecolor{currentfill}{rgb}{0.631373,0.788235,0.956863}%
\pgfsetfillcolor{currentfill}%
\pgfsetlinewidth{0.481800pt}%
\definecolor{currentstroke}{rgb}{1.000000,1.000000,1.000000}%
\pgfsetstrokecolor{currentstroke}%
\pgfsetdash{}{0pt}%
\pgfpathmoveto{\pgfqpoint{6.104525in}{2.646878in}}%
\pgfpathcurveto{\pgfqpoint{6.115575in}{2.646878in}}{\pgfqpoint{6.126174in}{2.651268in}}{\pgfqpoint{6.133988in}{2.659082in}}%
\pgfpathcurveto{\pgfqpoint{6.141802in}{2.666896in}}{\pgfqpoint{6.146192in}{2.677495in}}{\pgfqpoint{6.146192in}{2.688545in}}%
\pgfpathcurveto{\pgfqpoint{6.146192in}{2.699595in}}{\pgfqpoint{6.141802in}{2.710194in}}{\pgfqpoint{6.133988in}{2.718008in}}%
\pgfpathcurveto{\pgfqpoint{6.126174in}{2.725821in}}{\pgfqpoint{6.115575in}{2.730212in}}{\pgfqpoint{6.104525in}{2.730212in}}%
\pgfpathcurveto{\pgfqpoint{6.093475in}{2.730212in}}{\pgfqpoint{6.082876in}{2.725821in}}{\pgfqpoint{6.075062in}{2.718008in}}%
\pgfpathcurveto{\pgfqpoint{6.067249in}{2.710194in}}{\pgfqpoint{6.062859in}{2.699595in}}{\pgfqpoint{6.062859in}{2.688545in}}%
\pgfpathcurveto{\pgfqpoint{6.062859in}{2.677495in}}{\pgfqpoint{6.067249in}{2.666896in}}{\pgfqpoint{6.075062in}{2.659082in}}%
\pgfpathcurveto{\pgfqpoint{6.082876in}{2.651268in}}{\pgfqpoint{6.093475in}{2.646878in}}{\pgfqpoint{6.104525in}{2.646878in}}%
\pgfpathclose%
\pgfusepath{stroke,fill}%
\end{pgfscope}%
\begin{pgfscope}%
\pgfpathrectangle{\pgfqpoint{0.570343in}{0.331635in}}{\pgfqpoint{9.300000in}{7.700000in}}%
\pgfusepath{clip}%
\pgfsetbuttcap%
\pgfsetroundjoin%
\definecolor{currentfill}{rgb}{0.631373,0.788235,0.956863}%
\pgfsetfillcolor{currentfill}%
\pgfsetlinewidth{0.481800pt}%
\definecolor{currentstroke}{rgb}{1.000000,1.000000,1.000000}%
\pgfsetstrokecolor{currentstroke}%
\pgfsetdash{}{0pt}%
\pgfpathmoveto{\pgfqpoint{5.949816in}{4.757751in}}%
\pgfpathcurveto{\pgfqpoint{5.960866in}{4.757751in}}{\pgfqpoint{5.971465in}{4.762141in}}{\pgfqpoint{5.979279in}{4.769955in}}%
\pgfpathcurveto{\pgfqpoint{5.987093in}{4.777769in}}{\pgfqpoint{5.991483in}{4.788368in}}{\pgfqpoint{5.991483in}{4.799418in}}%
\pgfpathcurveto{\pgfqpoint{5.991483in}{4.810468in}}{\pgfqpoint{5.987093in}{4.821067in}}{\pgfqpoint{5.979279in}{4.828881in}}%
\pgfpathcurveto{\pgfqpoint{5.971465in}{4.836694in}}{\pgfqpoint{5.960866in}{4.841084in}}{\pgfqpoint{5.949816in}{4.841084in}}%
\pgfpathcurveto{\pgfqpoint{5.938766in}{4.841084in}}{\pgfqpoint{5.928167in}{4.836694in}}{\pgfqpoint{5.920353in}{4.828881in}}%
\pgfpathcurveto{\pgfqpoint{5.912540in}{4.821067in}}{\pgfqpoint{5.908150in}{4.810468in}}{\pgfqpoint{5.908150in}{4.799418in}}%
\pgfpathcurveto{\pgfqpoint{5.908150in}{4.788368in}}{\pgfqpoint{5.912540in}{4.777769in}}{\pgfqpoint{5.920353in}{4.769955in}}%
\pgfpathcurveto{\pgfqpoint{5.928167in}{4.762141in}}{\pgfqpoint{5.938766in}{4.757751in}}{\pgfqpoint{5.949816in}{4.757751in}}%
\pgfpathclose%
\pgfusepath{stroke,fill}%
\end{pgfscope}%
\begin{pgfscope}%
\pgfpathrectangle{\pgfqpoint{0.570343in}{0.331635in}}{\pgfqpoint{9.300000in}{7.700000in}}%
\pgfusepath{clip}%
\pgfsetbuttcap%
\pgfsetroundjoin%
\definecolor{currentfill}{rgb}{0.631373,0.788235,0.956863}%
\pgfsetfillcolor{currentfill}%
\pgfsetlinewidth{0.481800pt}%
\definecolor{currentstroke}{rgb}{1.000000,1.000000,1.000000}%
\pgfsetstrokecolor{currentstroke}%
\pgfsetdash{}{0pt}%
\pgfpathmoveto{\pgfqpoint{6.020291in}{4.040187in}}%
\pgfpathcurveto{\pgfqpoint{6.031342in}{4.040187in}}{\pgfqpoint{6.041941in}{4.044577in}}{\pgfqpoint{6.049754in}{4.052391in}}%
\pgfpathcurveto{\pgfqpoint{6.057568in}{4.060204in}}{\pgfqpoint{6.061958in}{4.070803in}}{\pgfqpoint{6.061958in}{4.081854in}}%
\pgfpathcurveto{\pgfqpoint{6.061958in}{4.092904in}}{\pgfqpoint{6.057568in}{4.103503in}}{\pgfqpoint{6.049754in}{4.111316in}}%
\pgfpathcurveto{\pgfqpoint{6.041941in}{4.119130in}}{\pgfqpoint{6.031342in}{4.123520in}}{\pgfqpoint{6.020291in}{4.123520in}}%
\pgfpathcurveto{\pgfqpoint{6.009241in}{4.123520in}}{\pgfqpoint{5.998642in}{4.119130in}}{\pgfqpoint{5.990829in}{4.111316in}}%
\pgfpathcurveto{\pgfqpoint{5.983015in}{4.103503in}}{\pgfqpoint{5.978625in}{4.092904in}}{\pgfqpoint{5.978625in}{4.081854in}}%
\pgfpathcurveto{\pgfqpoint{5.978625in}{4.070803in}}{\pgfqpoint{5.983015in}{4.060204in}}{\pgfqpoint{5.990829in}{4.052391in}}%
\pgfpathcurveto{\pgfqpoint{5.998642in}{4.044577in}}{\pgfqpoint{6.009241in}{4.040187in}}{\pgfqpoint{6.020291in}{4.040187in}}%
\pgfpathclose%
\pgfusepath{stroke,fill}%
\end{pgfscope}%
\begin{pgfscope}%
\pgfpathrectangle{\pgfqpoint{0.570343in}{0.331635in}}{\pgfqpoint{9.300000in}{7.700000in}}%
\pgfusepath{clip}%
\pgfsetbuttcap%
\pgfsetroundjoin%
\definecolor{currentfill}{rgb}{0.631373,0.788235,0.956863}%
\pgfsetfillcolor{currentfill}%
\pgfsetlinewidth{0.481800pt}%
\definecolor{currentstroke}{rgb}{1.000000,1.000000,1.000000}%
\pgfsetstrokecolor{currentstroke}%
\pgfsetdash{}{0pt}%
\pgfpathmoveto{\pgfqpoint{5.824951in}{3.427244in}}%
\pgfpathcurveto{\pgfqpoint{5.836001in}{3.427244in}}{\pgfqpoint{5.846600in}{3.431634in}}{\pgfqpoint{5.854413in}{3.439448in}}%
\pgfpathcurveto{\pgfqpoint{5.862227in}{3.447262in}}{\pgfqpoint{5.866617in}{3.457861in}}{\pgfqpoint{5.866617in}{3.468911in}}%
\pgfpathcurveto{\pgfqpoint{5.866617in}{3.479961in}}{\pgfqpoint{5.862227in}{3.490560in}}{\pgfqpoint{5.854413in}{3.498374in}}%
\pgfpathcurveto{\pgfqpoint{5.846600in}{3.506187in}}{\pgfqpoint{5.836001in}{3.510577in}}{\pgfqpoint{5.824951in}{3.510577in}}%
\pgfpathcurveto{\pgfqpoint{5.813901in}{3.510577in}}{\pgfqpoint{5.803302in}{3.506187in}}{\pgfqpoint{5.795488in}{3.498374in}}%
\pgfpathcurveto{\pgfqpoint{5.787674in}{3.490560in}}{\pgfqpoint{5.783284in}{3.479961in}}{\pgfqpoint{5.783284in}{3.468911in}}%
\pgfpathcurveto{\pgfqpoint{5.783284in}{3.457861in}}{\pgfqpoint{5.787674in}{3.447262in}}{\pgfqpoint{5.795488in}{3.439448in}}%
\pgfpathcurveto{\pgfqpoint{5.803302in}{3.431634in}}{\pgfqpoint{5.813901in}{3.427244in}}{\pgfqpoint{5.824951in}{3.427244in}}%
\pgfpathclose%
\pgfusepath{stroke,fill}%
\end{pgfscope}%
\begin{pgfscope}%
\pgfpathrectangle{\pgfqpoint{0.570343in}{0.331635in}}{\pgfqpoint{9.300000in}{7.700000in}}%
\pgfusepath{clip}%
\pgfsetbuttcap%
\pgfsetroundjoin%
\definecolor{currentfill}{rgb}{0.631373,0.788235,0.956863}%
\pgfsetfillcolor{currentfill}%
\pgfsetlinewidth{0.481800pt}%
\definecolor{currentstroke}{rgb}{1.000000,1.000000,1.000000}%
\pgfsetstrokecolor{currentstroke}%
\pgfsetdash{}{0pt}%
\pgfpathmoveto{\pgfqpoint{6.330522in}{1.667275in}}%
\pgfpathcurveto{\pgfqpoint{6.341572in}{1.667275in}}{\pgfqpoint{6.352171in}{1.671666in}}{\pgfqpoint{6.359984in}{1.679479in}}%
\pgfpathcurveto{\pgfqpoint{6.367798in}{1.687293in}}{\pgfqpoint{6.372188in}{1.697892in}}{\pgfqpoint{6.372188in}{1.708942in}}%
\pgfpathcurveto{\pgfqpoint{6.372188in}{1.719992in}}{\pgfqpoint{6.367798in}{1.730591in}}{\pgfqpoint{6.359984in}{1.738405in}}%
\pgfpathcurveto{\pgfqpoint{6.352171in}{1.746219in}}{\pgfqpoint{6.341572in}{1.750609in}}{\pgfqpoint{6.330522in}{1.750609in}}%
\pgfpathcurveto{\pgfqpoint{6.319471in}{1.750609in}}{\pgfqpoint{6.308872in}{1.746219in}}{\pgfqpoint{6.301059in}{1.738405in}}%
\pgfpathcurveto{\pgfqpoint{6.293245in}{1.730591in}}{\pgfqpoint{6.288855in}{1.719992in}}{\pgfqpoint{6.288855in}{1.708942in}}%
\pgfpathcurveto{\pgfqpoint{6.288855in}{1.697892in}}{\pgfqpoint{6.293245in}{1.687293in}}{\pgfqpoint{6.301059in}{1.679479in}}%
\pgfpathcurveto{\pgfqpoint{6.308872in}{1.671666in}}{\pgfqpoint{6.319471in}{1.667275in}}{\pgfqpoint{6.330522in}{1.667275in}}%
\pgfpathclose%
\pgfusepath{stroke,fill}%
\end{pgfscope}%
\begin{pgfscope}%
\pgfpathrectangle{\pgfqpoint{0.570343in}{0.331635in}}{\pgfqpoint{9.300000in}{7.700000in}}%
\pgfusepath{clip}%
\pgfsetbuttcap%
\pgfsetroundjoin%
\definecolor{currentfill}{rgb}{0.631373,0.788235,0.956863}%
\pgfsetfillcolor{currentfill}%
\pgfsetlinewidth{0.481800pt}%
\definecolor{currentstroke}{rgb}{1.000000,1.000000,1.000000}%
\pgfsetstrokecolor{currentstroke}%
\pgfsetdash{}{0pt}%
\pgfpathmoveto{\pgfqpoint{6.013977in}{3.390140in}}%
\pgfpathcurveto{\pgfqpoint{6.025027in}{3.390140in}}{\pgfqpoint{6.035626in}{3.394531in}}{\pgfqpoint{6.043439in}{3.402344in}}%
\pgfpathcurveto{\pgfqpoint{6.051253in}{3.410158in}}{\pgfqpoint{6.055643in}{3.420757in}}{\pgfqpoint{6.055643in}{3.431807in}}%
\pgfpathcurveto{\pgfqpoint{6.055643in}{3.442857in}}{\pgfqpoint{6.051253in}{3.453456in}}{\pgfqpoint{6.043439in}{3.461270in}}%
\pgfpathcurveto{\pgfqpoint{6.035626in}{3.469083in}}{\pgfqpoint{6.025027in}{3.473474in}}{\pgfqpoint{6.013977in}{3.473474in}}%
\pgfpathcurveto{\pgfqpoint{6.002927in}{3.473474in}}{\pgfqpoint{5.992328in}{3.469083in}}{\pgfqpoint{5.984514in}{3.461270in}}%
\pgfpathcurveto{\pgfqpoint{5.976700in}{3.453456in}}{\pgfqpoint{5.972310in}{3.442857in}}{\pgfqpoint{5.972310in}{3.431807in}}%
\pgfpathcurveto{\pgfqpoint{5.972310in}{3.420757in}}{\pgfqpoint{5.976700in}{3.410158in}}{\pgfqpoint{5.984514in}{3.402344in}}%
\pgfpathcurveto{\pgfqpoint{5.992328in}{3.394531in}}{\pgfqpoint{6.002927in}{3.390140in}}{\pgfqpoint{6.013977in}{3.390140in}}%
\pgfpathclose%
\pgfusepath{stroke,fill}%
\end{pgfscope}%
\begin{pgfscope}%
\pgfpathrectangle{\pgfqpoint{0.570343in}{0.331635in}}{\pgfqpoint{9.300000in}{7.700000in}}%
\pgfusepath{clip}%
\pgfsetbuttcap%
\pgfsetroundjoin%
\definecolor{currentfill}{rgb}{0.631373,0.788235,0.956863}%
\pgfsetfillcolor{currentfill}%
\pgfsetlinewidth{0.481800pt}%
\definecolor{currentstroke}{rgb}{1.000000,1.000000,1.000000}%
\pgfsetstrokecolor{currentstroke}%
\pgfsetdash{}{0pt}%
\pgfpathmoveto{\pgfqpoint{5.861429in}{4.100307in}}%
\pgfpathcurveto{\pgfqpoint{5.872479in}{4.100307in}}{\pgfqpoint{5.883078in}{4.104698in}}{\pgfqpoint{5.890891in}{4.112511in}}%
\pgfpathcurveto{\pgfqpoint{5.898705in}{4.120325in}}{\pgfqpoint{5.903095in}{4.130924in}}{\pgfqpoint{5.903095in}{4.141974in}}%
\pgfpathcurveto{\pgfqpoint{5.903095in}{4.153024in}}{\pgfqpoint{5.898705in}{4.163623in}}{\pgfqpoint{5.890891in}{4.171437in}}%
\pgfpathcurveto{\pgfqpoint{5.883078in}{4.179251in}}{\pgfqpoint{5.872479in}{4.183641in}}{\pgfqpoint{5.861429in}{4.183641in}}%
\pgfpathcurveto{\pgfqpoint{5.850378in}{4.183641in}}{\pgfqpoint{5.839779in}{4.179251in}}{\pgfqpoint{5.831966in}{4.171437in}}%
\pgfpathcurveto{\pgfqpoint{5.824152in}{4.163623in}}{\pgfqpoint{5.819762in}{4.153024in}}{\pgfqpoint{5.819762in}{4.141974in}}%
\pgfpathcurveto{\pgfqpoint{5.819762in}{4.130924in}}{\pgfqpoint{5.824152in}{4.120325in}}{\pgfqpoint{5.831966in}{4.112511in}}%
\pgfpathcurveto{\pgfqpoint{5.839779in}{4.104698in}}{\pgfqpoint{5.850378in}{4.100307in}}{\pgfqpoint{5.861429in}{4.100307in}}%
\pgfpathclose%
\pgfusepath{stroke,fill}%
\end{pgfscope}%
\begin{pgfscope}%
\pgfpathrectangle{\pgfqpoint{0.570343in}{0.331635in}}{\pgfqpoint{9.300000in}{7.700000in}}%
\pgfusepath{clip}%
\pgfsetbuttcap%
\pgfsetroundjoin%
\definecolor{currentfill}{rgb}{0.631373,0.788235,0.956863}%
\pgfsetfillcolor{currentfill}%
\pgfsetlinewidth{0.481800pt}%
\definecolor{currentstroke}{rgb}{1.000000,1.000000,1.000000}%
\pgfsetstrokecolor{currentstroke}%
\pgfsetdash{}{0pt}%
\pgfpathmoveto{\pgfqpoint{6.215095in}{2.935014in}}%
\pgfpathcurveto{\pgfqpoint{6.226145in}{2.935014in}}{\pgfqpoint{6.236744in}{2.939404in}}{\pgfqpoint{6.244558in}{2.947218in}}%
\pgfpathcurveto{\pgfqpoint{6.252371in}{2.955032in}}{\pgfqpoint{6.256761in}{2.965631in}}{\pgfqpoint{6.256761in}{2.976681in}}%
\pgfpathcurveto{\pgfqpoint{6.256761in}{2.987731in}}{\pgfqpoint{6.252371in}{2.998330in}}{\pgfqpoint{6.244558in}{3.006144in}}%
\pgfpathcurveto{\pgfqpoint{6.236744in}{3.013957in}}{\pgfqpoint{6.226145in}{3.018348in}}{\pgfqpoint{6.215095in}{3.018348in}}%
\pgfpathcurveto{\pgfqpoint{6.204045in}{3.018348in}}{\pgfqpoint{6.193446in}{3.013957in}}{\pgfqpoint{6.185632in}{3.006144in}}%
\pgfpathcurveto{\pgfqpoint{6.177818in}{2.998330in}}{\pgfqpoint{6.173428in}{2.987731in}}{\pgfqpoint{6.173428in}{2.976681in}}%
\pgfpathcurveto{\pgfqpoint{6.173428in}{2.965631in}}{\pgfqpoint{6.177818in}{2.955032in}}{\pgfqpoint{6.185632in}{2.947218in}}%
\pgfpathcurveto{\pgfqpoint{6.193446in}{2.939404in}}{\pgfqpoint{6.204045in}{2.935014in}}{\pgfqpoint{6.215095in}{2.935014in}}%
\pgfpathclose%
\pgfusepath{stroke,fill}%
\end{pgfscope}%
\begin{pgfscope}%
\pgfpathrectangle{\pgfqpoint{0.570343in}{0.331635in}}{\pgfqpoint{9.300000in}{7.700000in}}%
\pgfusepath{clip}%
\pgfsetbuttcap%
\pgfsetroundjoin%
\definecolor{currentfill}{rgb}{0.631373,0.788235,0.956863}%
\pgfsetfillcolor{currentfill}%
\pgfsetlinewidth{0.481800pt}%
\definecolor{currentstroke}{rgb}{1.000000,1.000000,1.000000}%
\pgfsetstrokecolor{currentstroke}%
\pgfsetdash{}{0pt}%
\pgfpathmoveto{\pgfqpoint{6.424663in}{3.993303in}}%
\pgfpathcurveto{\pgfqpoint{6.435713in}{3.993303in}}{\pgfqpoint{6.446312in}{3.997693in}}{\pgfqpoint{6.454126in}{4.005507in}}%
\pgfpathcurveto{\pgfqpoint{6.461939in}{4.013320in}}{\pgfqpoint{6.466329in}{4.023919in}}{\pgfqpoint{6.466329in}{4.034970in}}%
\pgfpathcurveto{\pgfqpoint{6.466329in}{4.046020in}}{\pgfqpoint{6.461939in}{4.056619in}}{\pgfqpoint{6.454126in}{4.064432in}}%
\pgfpathcurveto{\pgfqpoint{6.446312in}{4.072246in}}{\pgfqpoint{6.435713in}{4.076636in}}{\pgfqpoint{6.424663in}{4.076636in}}%
\pgfpathcurveto{\pgfqpoint{6.413613in}{4.076636in}}{\pgfqpoint{6.403014in}{4.072246in}}{\pgfqpoint{6.395200in}{4.064432in}}%
\pgfpathcurveto{\pgfqpoint{6.387386in}{4.056619in}}{\pgfqpoint{6.382996in}{4.046020in}}{\pgfqpoint{6.382996in}{4.034970in}}%
\pgfpathcurveto{\pgfqpoint{6.382996in}{4.023919in}}{\pgfqpoint{6.387386in}{4.013320in}}{\pgfqpoint{6.395200in}{4.005507in}}%
\pgfpathcurveto{\pgfqpoint{6.403014in}{3.997693in}}{\pgfqpoint{6.413613in}{3.993303in}}{\pgfqpoint{6.424663in}{3.993303in}}%
\pgfpathclose%
\pgfusepath{stroke,fill}%
\end{pgfscope}%
\begin{pgfscope}%
\pgfpathrectangle{\pgfqpoint{0.570343in}{0.331635in}}{\pgfqpoint{9.300000in}{7.700000in}}%
\pgfusepath{clip}%
\pgfsetbuttcap%
\pgfsetroundjoin%
\definecolor{currentfill}{rgb}{0.631373,0.788235,0.956863}%
\pgfsetfillcolor{currentfill}%
\pgfsetlinewidth{0.481800pt}%
\definecolor{currentstroke}{rgb}{1.000000,1.000000,1.000000}%
\pgfsetstrokecolor{currentstroke}%
\pgfsetdash{}{0pt}%
\pgfpathmoveto{\pgfqpoint{5.983136in}{6.892882in}}%
\pgfpathcurveto{\pgfqpoint{5.994186in}{6.892882in}}{\pgfqpoint{6.004785in}{6.897272in}}{\pgfqpoint{6.012599in}{6.905086in}}%
\pgfpathcurveto{\pgfqpoint{6.020412in}{6.912899in}}{\pgfqpoint{6.024802in}{6.923498in}}{\pgfqpoint{6.024802in}{6.934548in}}%
\pgfpathcurveto{\pgfqpoint{6.024802in}{6.945598in}}{\pgfqpoint{6.020412in}{6.956197in}}{\pgfqpoint{6.012599in}{6.964011in}}%
\pgfpathcurveto{\pgfqpoint{6.004785in}{6.971825in}}{\pgfqpoint{5.994186in}{6.976215in}}{\pgfqpoint{5.983136in}{6.976215in}}%
\pgfpathcurveto{\pgfqpoint{5.972086in}{6.976215in}}{\pgfqpoint{5.961487in}{6.971825in}}{\pgfqpoint{5.953673in}{6.964011in}}%
\pgfpathcurveto{\pgfqpoint{5.945859in}{6.956197in}}{\pgfqpoint{5.941469in}{6.945598in}}{\pgfqpoint{5.941469in}{6.934548in}}%
\pgfpathcurveto{\pgfqpoint{5.941469in}{6.923498in}}{\pgfqpoint{5.945859in}{6.912899in}}{\pgfqpoint{5.953673in}{6.905086in}}%
\pgfpathcurveto{\pgfqpoint{5.961487in}{6.897272in}}{\pgfqpoint{5.972086in}{6.892882in}}{\pgfqpoint{5.983136in}{6.892882in}}%
\pgfpathclose%
\pgfusepath{stroke,fill}%
\end{pgfscope}%
\begin{pgfscope}%
\pgfpathrectangle{\pgfqpoint{0.570343in}{0.331635in}}{\pgfqpoint{9.300000in}{7.700000in}}%
\pgfusepath{clip}%
\pgfsetbuttcap%
\pgfsetroundjoin%
\definecolor{currentfill}{rgb}{1.000000,0.705882,0.509804}%
\pgfsetfillcolor{currentfill}%
\pgfsetlinewidth{0.481800pt}%
\definecolor{currentstroke}{rgb}{1.000000,1.000000,1.000000}%
\pgfsetstrokecolor{currentstroke}%
\pgfsetdash{}{0pt}%
\pgfpathmoveto{\pgfqpoint{6.442953in}{5.796011in}}%
\pgfpathcurveto{\pgfqpoint{6.454003in}{5.796011in}}{\pgfqpoint{6.464602in}{5.800401in}}{\pgfqpoint{6.472416in}{5.808215in}}%
\pgfpathcurveto{\pgfqpoint{6.480229in}{5.816029in}}{\pgfqpoint{6.484620in}{5.826628in}}{\pgfqpoint{6.484620in}{5.837678in}}%
\pgfpathcurveto{\pgfqpoint{6.484620in}{5.848728in}}{\pgfqpoint{6.480229in}{5.859327in}}{\pgfqpoint{6.472416in}{5.867141in}}%
\pgfpathcurveto{\pgfqpoint{6.464602in}{5.874954in}}{\pgfqpoint{6.454003in}{5.879345in}}{\pgfqpoint{6.442953in}{5.879345in}}%
\pgfpathcurveto{\pgfqpoint{6.431903in}{5.879345in}}{\pgfqpoint{6.421304in}{5.874954in}}{\pgfqpoint{6.413490in}{5.867141in}}%
\pgfpathcurveto{\pgfqpoint{6.405677in}{5.859327in}}{\pgfqpoint{6.401286in}{5.848728in}}{\pgfqpoint{6.401286in}{5.837678in}}%
\pgfpathcurveto{\pgfqpoint{6.401286in}{5.826628in}}{\pgfqpoint{6.405677in}{5.816029in}}{\pgfqpoint{6.413490in}{5.808215in}}%
\pgfpathcurveto{\pgfqpoint{6.421304in}{5.800401in}}{\pgfqpoint{6.431903in}{5.796011in}}{\pgfqpoint{6.442953in}{5.796011in}}%
\pgfpathclose%
\pgfusepath{stroke,fill}%
\end{pgfscope}%
\begin{pgfscope}%
\pgfpathrectangle{\pgfqpoint{0.570343in}{0.331635in}}{\pgfqpoint{9.300000in}{7.700000in}}%
\pgfusepath{clip}%
\pgfsetbuttcap%
\pgfsetroundjoin%
\definecolor{currentfill}{rgb}{1.000000,0.705882,0.509804}%
\pgfsetfillcolor{currentfill}%
\pgfsetlinewidth{0.481800pt}%
\definecolor{currentstroke}{rgb}{1.000000,1.000000,1.000000}%
\pgfsetstrokecolor{currentstroke}%
\pgfsetdash{}{0pt}%
\pgfpathmoveto{\pgfqpoint{6.303552in}{3.014193in}}%
\pgfpathcurveto{\pgfqpoint{6.314602in}{3.014193in}}{\pgfqpoint{6.325201in}{3.018583in}}{\pgfqpoint{6.333014in}{3.026397in}}%
\pgfpathcurveto{\pgfqpoint{6.340828in}{3.034211in}}{\pgfqpoint{6.345218in}{3.044810in}}{\pgfqpoint{6.345218in}{3.055860in}}%
\pgfpathcurveto{\pgfqpoint{6.345218in}{3.066910in}}{\pgfqpoint{6.340828in}{3.077509in}}{\pgfqpoint{6.333014in}{3.085323in}}%
\pgfpathcurveto{\pgfqpoint{6.325201in}{3.093136in}}{\pgfqpoint{6.314602in}{3.097526in}}{\pgfqpoint{6.303552in}{3.097526in}}%
\pgfpathcurveto{\pgfqpoint{6.292501in}{3.097526in}}{\pgfqpoint{6.281902in}{3.093136in}}{\pgfqpoint{6.274089in}{3.085323in}}%
\pgfpathcurveto{\pgfqpoint{6.266275in}{3.077509in}}{\pgfqpoint{6.261885in}{3.066910in}}{\pgfqpoint{6.261885in}{3.055860in}}%
\pgfpathcurveto{\pgfqpoint{6.261885in}{3.044810in}}{\pgfqpoint{6.266275in}{3.034211in}}{\pgfqpoint{6.274089in}{3.026397in}}%
\pgfpathcurveto{\pgfqpoint{6.281902in}{3.018583in}}{\pgfqpoint{6.292501in}{3.014193in}}{\pgfqpoint{6.303552in}{3.014193in}}%
\pgfpathclose%
\pgfusepath{stroke,fill}%
\end{pgfscope}%
\begin{pgfscope}%
\pgfpathrectangle{\pgfqpoint{0.570343in}{0.331635in}}{\pgfqpoint{9.300000in}{7.700000in}}%
\pgfusepath{clip}%
\pgfsetbuttcap%
\pgfsetroundjoin%
\definecolor{currentfill}{rgb}{1.000000,0.705882,0.509804}%
\pgfsetfillcolor{currentfill}%
\pgfsetlinewidth{0.481800pt}%
\definecolor{currentstroke}{rgb}{1.000000,1.000000,1.000000}%
\pgfsetstrokecolor{currentstroke}%
\pgfsetdash{}{0pt}%
\pgfpathmoveto{\pgfqpoint{6.657819in}{4.268625in}}%
\pgfpathcurveto{\pgfqpoint{6.668869in}{4.268625in}}{\pgfqpoint{6.679468in}{4.273015in}}{\pgfqpoint{6.687282in}{4.280829in}}%
\pgfpathcurveto{\pgfqpoint{6.695095in}{4.288642in}}{\pgfqpoint{6.699486in}{4.299241in}}{\pgfqpoint{6.699486in}{4.310291in}}%
\pgfpathcurveto{\pgfqpoint{6.699486in}{4.321342in}}{\pgfqpoint{6.695095in}{4.331941in}}{\pgfqpoint{6.687282in}{4.339754in}}%
\pgfpathcurveto{\pgfqpoint{6.679468in}{4.347568in}}{\pgfqpoint{6.668869in}{4.351958in}}{\pgfqpoint{6.657819in}{4.351958in}}%
\pgfpathcurveto{\pgfqpoint{6.646769in}{4.351958in}}{\pgfqpoint{6.636170in}{4.347568in}}{\pgfqpoint{6.628356in}{4.339754in}}%
\pgfpathcurveto{\pgfqpoint{6.620543in}{4.331941in}}{\pgfqpoint{6.616152in}{4.321342in}}{\pgfqpoint{6.616152in}{4.310291in}}%
\pgfpathcurveto{\pgfqpoint{6.616152in}{4.299241in}}{\pgfqpoint{6.620543in}{4.288642in}}{\pgfqpoint{6.628356in}{4.280829in}}%
\pgfpathcurveto{\pgfqpoint{6.636170in}{4.273015in}}{\pgfqpoint{6.646769in}{4.268625in}}{\pgfqpoint{6.657819in}{4.268625in}}%
\pgfpathclose%
\pgfusepath{stroke,fill}%
\end{pgfscope}%
\begin{pgfscope}%
\pgfpathrectangle{\pgfqpoint{0.570343in}{0.331635in}}{\pgfqpoint{9.300000in}{7.700000in}}%
\pgfusepath{clip}%
\pgfsetbuttcap%
\pgfsetroundjoin%
\definecolor{currentfill}{rgb}{1.000000,0.705882,0.509804}%
\pgfsetfillcolor{currentfill}%
\pgfsetlinewidth{0.481800pt}%
\definecolor{currentstroke}{rgb}{1.000000,1.000000,1.000000}%
\pgfsetstrokecolor{currentstroke}%
\pgfsetdash{}{0pt}%
\pgfpathmoveto{\pgfqpoint{6.208605in}{1.978462in}}%
\pgfpathcurveto{\pgfqpoint{6.219655in}{1.978462in}}{\pgfqpoint{6.230254in}{1.982853in}}{\pgfqpoint{6.238068in}{1.990666in}}%
\pgfpathcurveto{\pgfqpoint{6.245881in}{1.998480in}}{\pgfqpoint{6.250272in}{2.009079in}}{\pgfqpoint{6.250272in}{2.020129in}}%
\pgfpathcurveto{\pgfqpoint{6.250272in}{2.031179in}}{\pgfqpoint{6.245881in}{2.041778in}}{\pgfqpoint{6.238068in}{2.049592in}}%
\pgfpathcurveto{\pgfqpoint{6.230254in}{2.057405in}}{\pgfqpoint{6.219655in}{2.061796in}}{\pgfqpoint{6.208605in}{2.061796in}}%
\pgfpathcurveto{\pgfqpoint{6.197555in}{2.061796in}}{\pgfqpoint{6.186956in}{2.057405in}}{\pgfqpoint{6.179142in}{2.049592in}}%
\pgfpathcurveto{\pgfqpoint{6.171329in}{2.041778in}}{\pgfqpoint{6.166938in}{2.031179in}}{\pgfqpoint{6.166938in}{2.020129in}}%
\pgfpathcurveto{\pgfqpoint{6.166938in}{2.009079in}}{\pgfqpoint{6.171329in}{1.998480in}}{\pgfqpoint{6.179142in}{1.990666in}}%
\pgfpathcurveto{\pgfqpoint{6.186956in}{1.982853in}}{\pgfqpoint{6.197555in}{1.978462in}}{\pgfqpoint{6.208605in}{1.978462in}}%
\pgfpathclose%
\pgfusepath{stroke,fill}%
\end{pgfscope}%
\begin{pgfscope}%
\pgfpathrectangle{\pgfqpoint{0.570343in}{0.331635in}}{\pgfqpoint{9.300000in}{7.700000in}}%
\pgfusepath{clip}%
\pgfsetbuttcap%
\pgfsetroundjoin%
\definecolor{currentfill}{rgb}{1.000000,0.705882,0.509804}%
\pgfsetfillcolor{currentfill}%
\pgfsetlinewidth{0.481800pt}%
\definecolor{currentstroke}{rgb}{1.000000,1.000000,1.000000}%
\pgfsetstrokecolor{currentstroke}%
\pgfsetdash{}{0pt}%
\pgfpathmoveto{\pgfqpoint{6.335839in}{5.352357in}}%
\pgfpathcurveto{\pgfqpoint{6.346890in}{5.352357in}}{\pgfqpoint{6.357489in}{5.356747in}}{\pgfqpoint{6.365302in}{5.364561in}}%
\pgfpathcurveto{\pgfqpoint{6.373116in}{5.372375in}}{\pgfqpoint{6.377506in}{5.382974in}}{\pgfqpoint{6.377506in}{5.394024in}}%
\pgfpathcurveto{\pgfqpoint{6.377506in}{5.405074in}}{\pgfqpoint{6.373116in}{5.415673in}}{\pgfqpoint{6.365302in}{5.423487in}}%
\pgfpathcurveto{\pgfqpoint{6.357489in}{5.431300in}}{\pgfqpoint{6.346890in}{5.435690in}}{\pgfqpoint{6.335839in}{5.435690in}}%
\pgfpathcurveto{\pgfqpoint{6.324789in}{5.435690in}}{\pgfqpoint{6.314190in}{5.431300in}}{\pgfqpoint{6.306377in}{5.423487in}}%
\pgfpathcurveto{\pgfqpoint{6.298563in}{5.415673in}}{\pgfqpoint{6.294173in}{5.405074in}}{\pgfqpoint{6.294173in}{5.394024in}}%
\pgfpathcurveto{\pgfqpoint{6.294173in}{5.382974in}}{\pgfqpoint{6.298563in}{5.372375in}}{\pgfqpoint{6.306377in}{5.364561in}}%
\pgfpathcurveto{\pgfqpoint{6.314190in}{5.356747in}}{\pgfqpoint{6.324789in}{5.352357in}}{\pgfqpoint{6.335839in}{5.352357in}}%
\pgfpathclose%
\pgfusepath{stroke,fill}%
\end{pgfscope}%
\begin{pgfscope}%
\pgfpathrectangle{\pgfqpoint{0.570343in}{0.331635in}}{\pgfqpoint{9.300000in}{7.700000in}}%
\pgfusepath{clip}%
\pgfsetbuttcap%
\pgfsetroundjoin%
\definecolor{currentfill}{rgb}{1.000000,0.705882,0.509804}%
\pgfsetfillcolor{currentfill}%
\pgfsetlinewidth{0.481800pt}%
\definecolor{currentstroke}{rgb}{1.000000,1.000000,1.000000}%
\pgfsetstrokecolor{currentstroke}%
\pgfsetdash{}{0pt}%
\pgfpathmoveto{\pgfqpoint{7.346402in}{3.701025in}}%
\pgfpathcurveto{\pgfqpoint{7.357452in}{3.701025in}}{\pgfqpoint{7.368051in}{3.705415in}}{\pgfqpoint{7.375865in}{3.713229in}}%
\pgfpathcurveto{\pgfqpoint{7.383678in}{3.721042in}}{\pgfqpoint{7.388069in}{3.731641in}}{\pgfqpoint{7.388069in}{3.742692in}}%
\pgfpathcurveto{\pgfqpoint{7.388069in}{3.753742in}}{\pgfqpoint{7.383678in}{3.764341in}}{\pgfqpoint{7.375865in}{3.772154in}}%
\pgfpathcurveto{\pgfqpoint{7.368051in}{3.779968in}}{\pgfqpoint{7.357452in}{3.784358in}}{\pgfqpoint{7.346402in}{3.784358in}}%
\pgfpathcurveto{\pgfqpoint{7.335352in}{3.784358in}}{\pgfqpoint{7.324753in}{3.779968in}}{\pgfqpoint{7.316939in}{3.772154in}}%
\pgfpathcurveto{\pgfqpoint{7.309126in}{3.764341in}}{\pgfqpoint{7.304735in}{3.753742in}}{\pgfqpoint{7.304735in}{3.742692in}}%
\pgfpathcurveto{\pgfqpoint{7.304735in}{3.731641in}}{\pgfqpoint{7.309126in}{3.721042in}}{\pgfqpoint{7.316939in}{3.713229in}}%
\pgfpathcurveto{\pgfqpoint{7.324753in}{3.705415in}}{\pgfqpoint{7.335352in}{3.701025in}}{\pgfqpoint{7.346402in}{3.701025in}}%
\pgfpathclose%
\pgfusepath{stroke,fill}%
\end{pgfscope}%
\begin{pgfscope}%
\pgfpathrectangle{\pgfqpoint{0.570343in}{0.331635in}}{\pgfqpoint{9.300000in}{7.700000in}}%
\pgfusepath{clip}%
\pgfsetbuttcap%
\pgfsetroundjoin%
\definecolor{currentfill}{rgb}{1.000000,0.705882,0.509804}%
\pgfsetfillcolor{currentfill}%
\pgfsetlinewidth{0.481800pt}%
\definecolor{currentstroke}{rgb}{1.000000,1.000000,1.000000}%
\pgfsetstrokecolor{currentstroke}%
\pgfsetdash{}{0pt}%
\pgfpathmoveto{\pgfqpoint{6.089195in}{5.053494in}}%
\pgfpathcurveto{\pgfqpoint{6.100245in}{5.053494in}}{\pgfqpoint{6.110844in}{5.057884in}}{\pgfqpoint{6.118658in}{5.065698in}}%
\pgfpathcurveto{\pgfqpoint{6.126471in}{5.073511in}}{\pgfqpoint{6.130862in}{5.084110in}}{\pgfqpoint{6.130862in}{5.095160in}}%
\pgfpathcurveto{\pgfqpoint{6.130862in}{5.106211in}}{\pgfqpoint{6.126471in}{5.116810in}}{\pgfqpoint{6.118658in}{5.124623in}}%
\pgfpathcurveto{\pgfqpoint{6.110844in}{5.132437in}}{\pgfqpoint{6.100245in}{5.136827in}}{\pgfqpoint{6.089195in}{5.136827in}}%
\pgfpathcurveto{\pgfqpoint{6.078145in}{5.136827in}}{\pgfqpoint{6.067546in}{5.132437in}}{\pgfqpoint{6.059732in}{5.124623in}}%
\pgfpathcurveto{\pgfqpoint{6.051919in}{5.116810in}}{\pgfqpoint{6.047528in}{5.106211in}}{\pgfqpoint{6.047528in}{5.095160in}}%
\pgfpathcurveto{\pgfqpoint{6.047528in}{5.084110in}}{\pgfqpoint{6.051919in}{5.073511in}}{\pgfqpoint{6.059732in}{5.065698in}}%
\pgfpathcurveto{\pgfqpoint{6.067546in}{5.057884in}}{\pgfqpoint{6.078145in}{5.053494in}}{\pgfqpoint{6.089195in}{5.053494in}}%
\pgfpathclose%
\pgfusepath{stroke,fill}%
\end{pgfscope}%
\begin{pgfscope}%
\pgfpathrectangle{\pgfqpoint{0.570343in}{0.331635in}}{\pgfqpoint{9.300000in}{7.700000in}}%
\pgfusepath{clip}%
\pgfsetbuttcap%
\pgfsetroundjoin%
\definecolor{currentfill}{rgb}{1.000000,0.705882,0.509804}%
\pgfsetfillcolor{currentfill}%
\pgfsetlinewidth{0.481800pt}%
\definecolor{currentstroke}{rgb}{1.000000,1.000000,1.000000}%
\pgfsetstrokecolor{currentstroke}%
\pgfsetdash{}{0pt}%
\pgfpathmoveto{\pgfqpoint{6.210432in}{5.149938in}}%
\pgfpathcurveto{\pgfqpoint{6.221482in}{5.149938in}}{\pgfqpoint{6.232081in}{5.154328in}}{\pgfqpoint{6.239895in}{5.162142in}}%
\pgfpathcurveto{\pgfqpoint{6.247708in}{5.169956in}}{\pgfqpoint{6.252099in}{5.180555in}}{\pgfqpoint{6.252099in}{5.191605in}}%
\pgfpathcurveto{\pgfqpoint{6.252099in}{5.202655in}}{\pgfqpoint{6.247708in}{5.213254in}}{\pgfqpoint{6.239895in}{5.221068in}}%
\pgfpathcurveto{\pgfqpoint{6.232081in}{5.228881in}}{\pgfqpoint{6.221482in}{5.233272in}}{\pgfqpoint{6.210432in}{5.233272in}}%
\pgfpathcurveto{\pgfqpoint{6.199382in}{5.233272in}}{\pgfqpoint{6.188783in}{5.228881in}}{\pgfqpoint{6.180969in}{5.221068in}}%
\pgfpathcurveto{\pgfqpoint{6.173155in}{5.213254in}}{\pgfqpoint{6.168765in}{5.202655in}}{\pgfqpoint{6.168765in}{5.191605in}}%
\pgfpathcurveto{\pgfqpoint{6.168765in}{5.180555in}}{\pgfqpoint{6.173155in}{5.169956in}}{\pgfqpoint{6.180969in}{5.162142in}}%
\pgfpathcurveto{\pgfqpoint{6.188783in}{5.154328in}}{\pgfqpoint{6.199382in}{5.149938in}}{\pgfqpoint{6.210432in}{5.149938in}}%
\pgfpathclose%
\pgfusepath{stroke,fill}%
\end{pgfscope}%
\begin{pgfscope}%
\pgfpathrectangle{\pgfqpoint{0.570343in}{0.331635in}}{\pgfqpoint{9.300000in}{7.700000in}}%
\pgfusepath{clip}%
\pgfsetbuttcap%
\pgfsetroundjoin%
\definecolor{currentfill}{rgb}{1.000000,0.705882,0.509804}%
\pgfsetfillcolor{currentfill}%
\pgfsetlinewidth{0.481800pt}%
\definecolor{currentstroke}{rgb}{1.000000,1.000000,1.000000}%
\pgfsetstrokecolor{currentstroke}%
\pgfsetdash{}{0pt}%
\pgfpathmoveto{\pgfqpoint{0.993071in}{4.907190in}}%
\pgfpathcurveto{\pgfqpoint{1.004121in}{4.907190in}}{\pgfqpoint{1.014720in}{4.911581in}}{\pgfqpoint{1.022533in}{4.919394in}}%
\pgfpathcurveto{\pgfqpoint{1.030347in}{4.927208in}}{\pgfqpoint{1.034737in}{4.937807in}}{\pgfqpoint{1.034737in}{4.948857in}}%
\pgfpathcurveto{\pgfqpoint{1.034737in}{4.959907in}}{\pgfqpoint{1.030347in}{4.970506in}}{\pgfqpoint{1.022533in}{4.978320in}}%
\pgfpathcurveto{\pgfqpoint{1.014720in}{4.986133in}}{\pgfqpoint{1.004121in}{4.990524in}}{\pgfqpoint{0.993071in}{4.990524in}}%
\pgfpathcurveto{\pgfqpoint{0.982020in}{4.990524in}}{\pgfqpoint{0.971421in}{4.986133in}}{\pgfqpoint{0.963608in}{4.978320in}}%
\pgfpathcurveto{\pgfqpoint{0.955794in}{4.970506in}}{\pgfqpoint{0.951404in}{4.959907in}}{\pgfqpoint{0.951404in}{4.948857in}}%
\pgfpathcurveto{\pgfqpoint{0.951404in}{4.937807in}}{\pgfqpoint{0.955794in}{4.927208in}}{\pgfqpoint{0.963608in}{4.919394in}}%
\pgfpathcurveto{\pgfqpoint{0.971421in}{4.911581in}}{\pgfqpoint{0.982020in}{4.907190in}}{\pgfqpoint{0.993071in}{4.907190in}}%
\pgfpathclose%
\pgfusepath{stroke,fill}%
\end{pgfscope}%
\begin{pgfscope}%
\pgfpathrectangle{\pgfqpoint{0.570343in}{0.331635in}}{\pgfqpoint{9.300000in}{7.700000in}}%
\pgfusepath{clip}%
\pgfsetbuttcap%
\pgfsetroundjoin%
\definecolor{currentfill}{rgb}{1.000000,0.705882,0.509804}%
\pgfsetfillcolor{currentfill}%
\pgfsetlinewidth{0.481800pt}%
\definecolor{currentstroke}{rgb}{1.000000,1.000000,1.000000}%
\pgfsetstrokecolor{currentstroke}%
\pgfsetdash{}{0pt}%
\pgfpathmoveto{\pgfqpoint{6.146652in}{5.039314in}}%
\pgfpathcurveto{\pgfqpoint{6.157703in}{5.039314in}}{\pgfqpoint{6.168302in}{5.043704in}}{\pgfqpoint{6.176115in}{5.051517in}}%
\pgfpathcurveto{\pgfqpoint{6.183929in}{5.059331in}}{\pgfqpoint{6.188319in}{5.069930in}}{\pgfqpoint{6.188319in}{5.080980in}}%
\pgfpathcurveto{\pgfqpoint{6.188319in}{5.092030in}}{\pgfqpoint{6.183929in}{5.102629in}}{\pgfqpoint{6.176115in}{5.110443in}}%
\pgfpathcurveto{\pgfqpoint{6.168302in}{5.118257in}}{\pgfqpoint{6.157703in}{5.122647in}}{\pgfqpoint{6.146652in}{5.122647in}}%
\pgfpathcurveto{\pgfqpoint{6.135602in}{5.122647in}}{\pgfqpoint{6.125003in}{5.118257in}}{\pgfqpoint{6.117190in}{5.110443in}}%
\pgfpathcurveto{\pgfqpoint{6.109376in}{5.102629in}}{\pgfqpoint{6.104986in}{5.092030in}}{\pgfqpoint{6.104986in}{5.080980in}}%
\pgfpathcurveto{\pgfqpoint{6.104986in}{5.069930in}}{\pgfqpoint{6.109376in}{5.059331in}}{\pgfqpoint{6.117190in}{5.051517in}}%
\pgfpathcurveto{\pgfqpoint{6.125003in}{5.043704in}}{\pgfqpoint{6.135602in}{5.039314in}}{\pgfqpoint{6.146652in}{5.039314in}}%
\pgfpathclose%
\pgfusepath{stroke,fill}%
\end{pgfscope}%
\begin{pgfscope}%
\pgfpathrectangle{\pgfqpoint{0.570343in}{0.331635in}}{\pgfqpoint{9.300000in}{7.700000in}}%
\pgfusepath{clip}%
\pgfsetbuttcap%
\pgfsetroundjoin%
\definecolor{currentfill}{rgb}{1.000000,0.705882,0.509804}%
\pgfsetfillcolor{currentfill}%
\pgfsetlinewidth{0.481800pt}%
\definecolor{currentstroke}{rgb}{1.000000,1.000000,1.000000}%
\pgfsetstrokecolor{currentstroke}%
\pgfsetdash{}{0pt}%
\pgfpathmoveto{\pgfqpoint{6.252198in}{5.784076in}}%
\pgfpathcurveto{\pgfqpoint{6.263249in}{5.784076in}}{\pgfqpoint{6.273848in}{5.788466in}}{\pgfqpoint{6.281661in}{5.796280in}}%
\pgfpathcurveto{\pgfqpoint{6.289475in}{5.804094in}}{\pgfqpoint{6.293865in}{5.814693in}}{\pgfqpoint{6.293865in}{5.825743in}}%
\pgfpathcurveto{\pgfqpoint{6.293865in}{5.836793in}}{\pgfqpoint{6.289475in}{5.847392in}}{\pgfqpoint{6.281661in}{5.855206in}}%
\pgfpathcurveto{\pgfqpoint{6.273848in}{5.863019in}}{\pgfqpoint{6.263249in}{5.867410in}}{\pgfqpoint{6.252198in}{5.867410in}}%
\pgfpathcurveto{\pgfqpoint{6.241148in}{5.867410in}}{\pgfqpoint{6.230549in}{5.863019in}}{\pgfqpoint{6.222736in}{5.855206in}}%
\pgfpathcurveto{\pgfqpoint{6.214922in}{5.847392in}}{\pgfqpoint{6.210532in}{5.836793in}}{\pgfqpoint{6.210532in}{5.825743in}}%
\pgfpathcurveto{\pgfqpoint{6.210532in}{5.814693in}}{\pgfqpoint{6.214922in}{5.804094in}}{\pgfqpoint{6.222736in}{5.796280in}}%
\pgfpathcurveto{\pgfqpoint{6.230549in}{5.788466in}}{\pgfqpoint{6.241148in}{5.784076in}}{\pgfqpoint{6.252198in}{5.784076in}}%
\pgfpathclose%
\pgfusepath{stroke,fill}%
\end{pgfscope}%
\begin{pgfscope}%
\pgfpathrectangle{\pgfqpoint{0.570343in}{0.331635in}}{\pgfqpoint{9.300000in}{7.700000in}}%
\pgfusepath{clip}%
\pgfsetbuttcap%
\pgfsetroundjoin%
\definecolor{currentfill}{rgb}{1.000000,0.705882,0.509804}%
\pgfsetfillcolor{currentfill}%
\pgfsetlinewidth{0.481800pt}%
\definecolor{currentstroke}{rgb}{1.000000,1.000000,1.000000}%
\pgfsetstrokecolor{currentstroke}%
\pgfsetdash{}{0pt}%
\pgfpathmoveto{\pgfqpoint{6.265171in}{3.571872in}}%
\pgfpathcurveto{\pgfqpoint{6.276221in}{3.571872in}}{\pgfqpoint{6.286820in}{3.576262in}}{\pgfqpoint{6.294634in}{3.584076in}}%
\pgfpathcurveto{\pgfqpoint{6.302447in}{3.591889in}}{\pgfqpoint{6.306838in}{3.602488in}}{\pgfqpoint{6.306838in}{3.613538in}}%
\pgfpathcurveto{\pgfqpoint{6.306838in}{3.624588in}}{\pgfqpoint{6.302447in}{3.635187in}}{\pgfqpoint{6.294634in}{3.643001in}}%
\pgfpathcurveto{\pgfqpoint{6.286820in}{3.650815in}}{\pgfqpoint{6.276221in}{3.655205in}}{\pgfqpoint{6.265171in}{3.655205in}}%
\pgfpathcurveto{\pgfqpoint{6.254121in}{3.655205in}}{\pgfqpoint{6.243522in}{3.650815in}}{\pgfqpoint{6.235708in}{3.643001in}}%
\pgfpathcurveto{\pgfqpoint{6.227895in}{3.635187in}}{\pgfqpoint{6.223504in}{3.624588in}}{\pgfqpoint{6.223504in}{3.613538in}}%
\pgfpathcurveto{\pgfqpoint{6.223504in}{3.602488in}}{\pgfqpoint{6.227895in}{3.591889in}}{\pgfqpoint{6.235708in}{3.584076in}}%
\pgfpathcurveto{\pgfqpoint{6.243522in}{3.576262in}}{\pgfqpoint{6.254121in}{3.571872in}}{\pgfqpoint{6.265171in}{3.571872in}}%
\pgfpathclose%
\pgfusepath{stroke,fill}%
\end{pgfscope}%
\begin{pgfscope}%
\pgfpathrectangle{\pgfqpoint{0.570343in}{0.331635in}}{\pgfqpoint{9.300000in}{7.700000in}}%
\pgfusepath{clip}%
\pgfsetbuttcap%
\pgfsetroundjoin%
\definecolor{currentfill}{rgb}{1.000000,0.705882,0.509804}%
\pgfsetfillcolor{currentfill}%
\pgfsetlinewidth{0.481800pt}%
\definecolor{currentstroke}{rgb}{1.000000,1.000000,1.000000}%
\pgfsetstrokecolor{currentstroke}%
\pgfsetdash{}{0pt}%
\pgfpathmoveto{\pgfqpoint{6.235593in}{4.049478in}}%
\pgfpathcurveto{\pgfqpoint{6.246643in}{4.049478in}}{\pgfqpoint{6.257242in}{4.053868in}}{\pgfqpoint{6.265056in}{4.061682in}}%
\pgfpathcurveto{\pgfqpoint{6.272870in}{4.069495in}}{\pgfqpoint{6.277260in}{4.080094in}}{\pgfqpoint{6.277260in}{4.091144in}}%
\pgfpathcurveto{\pgfqpoint{6.277260in}{4.102195in}}{\pgfqpoint{6.272870in}{4.112794in}}{\pgfqpoint{6.265056in}{4.120607in}}%
\pgfpathcurveto{\pgfqpoint{6.257242in}{4.128421in}}{\pgfqpoint{6.246643in}{4.132811in}}{\pgfqpoint{6.235593in}{4.132811in}}%
\pgfpathcurveto{\pgfqpoint{6.224543in}{4.132811in}}{\pgfqpoint{6.213944in}{4.128421in}}{\pgfqpoint{6.206131in}{4.120607in}}%
\pgfpathcurveto{\pgfqpoint{6.198317in}{4.112794in}}{\pgfqpoint{6.193927in}{4.102195in}}{\pgfqpoint{6.193927in}{4.091144in}}%
\pgfpathcurveto{\pgfqpoint{6.193927in}{4.080094in}}{\pgfqpoint{6.198317in}{4.069495in}}{\pgfqpoint{6.206131in}{4.061682in}}%
\pgfpathcurveto{\pgfqpoint{6.213944in}{4.053868in}}{\pgfqpoint{6.224543in}{4.049478in}}{\pgfqpoint{6.235593in}{4.049478in}}%
\pgfpathclose%
\pgfusepath{stroke,fill}%
\end{pgfscope}%
\begin{pgfscope}%
\pgfpathrectangle{\pgfqpoint{0.570343in}{0.331635in}}{\pgfqpoint{9.300000in}{7.700000in}}%
\pgfusepath{clip}%
\pgfsetbuttcap%
\pgfsetroundjoin%
\definecolor{currentfill}{rgb}{1.000000,0.705882,0.509804}%
\pgfsetfillcolor{currentfill}%
\pgfsetlinewidth{0.481800pt}%
\definecolor{currentstroke}{rgb}{1.000000,1.000000,1.000000}%
\pgfsetstrokecolor{currentstroke}%
\pgfsetdash{}{0pt}%
\pgfpathmoveto{\pgfqpoint{5.626465in}{1.779731in}}%
\pgfpathcurveto{\pgfqpoint{5.637515in}{1.779731in}}{\pgfqpoint{5.648114in}{1.784122in}}{\pgfqpoint{5.655928in}{1.791935in}}%
\pgfpathcurveto{\pgfqpoint{5.663741in}{1.799749in}}{\pgfqpoint{5.668131in}{1.810348in}}{\pgfqpoint{5.668131in}{1.821398in}}%
\pgfpathcurveto{\pgfqpoint{5.668131in}{1.832448in}}{\pgfqpoint{5.663741in}{1.843047in}}{\pgfqpoint{5.655928in}{1.850861in}}%
\pgfpathcurveto{\pgfqpoint{5.648114in}{1.858674in}}{\pgfqpoint{5.637515in}{1.863065in}}{\pgfqpoint{5.626465in}{1.863065in}}%
\pgfpathcurveto{\pgfqpoint{5.615415in}{1.863065in}}{\pgfqpoint{5.604816in}{1.858674in}}{\pgfqpoint{5.597002in}{1.850861in}}%
\pgfpathcurveto{\pgfqpoint{5.589188in}{1.843047in}}{\pgfqpoint{5.584798in}{1.832448in}}{\pgfqpoint{5.584798in}{1.821398in}}%
\pgfpathcurveto{\pgfqpoint{5.584798in}{1.810348in}}{\pgfqpoint{5.589188in}{1.799749in}}{\pgfqpoint{5.597002in}{1.791935in}}%
\pgfpathcurveto{\pgfqpoint{5.604816in}{1.784122in}}{\pgfqpoint{5.615415in}{1.779731in}}{\pgfqpoint{5.626465in}{1.779731in}}%
\pgfpathclose%
\pgfusepath{stroke,fill}%
\end{pgfscope}%
\begin{pgfscope}%
\pgfpathrectangle{\pgfqpoint{0.570343in}{0.331635in}}{\pgfqpoint{9.300000in}{7.700000in}}%
\pgfusepath{clip}%
\pgfsetbuttcap%
\pgfsetroundjoin%
\definecolor{currentfill}{rgb}{1.000000,0.705882,0.509804}%
\pgfsetfillcolor{currentfill}%
\pgfsetlinewidth{0.481800pt}%
\definecolor{currentstroke}{rgb}{1.000000,1.000000,1.000000}%
\pgfsetstrokecolor{currentstroke}%
\pgfsetdash{}{0pt}%
\pgfpathmoveto{\pgfqpoint{6.206878in}{3.645884in}}%
\pgfpathcurveto{\pgfqpoint{6.217928in}{3.645884in}}{\pgfqpoint{6.228527in}{3.650274in}}{\pgfqpoint{6.236341in}{3.658088in}}%
\pgfpathcurveto{\pgfqpoint{6.244155in}{3.665901in}}{\pgfqpoint{6.248545in}{3.676500in}}{\pgfqpoint{6.248545in}{3.687550in}}%
\pgfpathcurveto{\pgfqpoint{6.248545in}{3.698600in}}{\pgfqpoint{6.244155in}{3.709200in}}{\pgfqpoint{6.236341in}{3.717013in}}%
\pgfpathcurveto{\pgfqpoint{6.228527in}{3.724827in}}{\pgfqpoint{6.217928in}{3.729217in}}{\pgfqpoint{6.206878in}{3.729217in}}%
\pgfpathcurveto{\pgfqpoint{6.195828in}{3.729217in}}{\pgfqpoint{6.185229in}{3.724827in}}{\pgfqpoint{6.177415in}{3.717013in}}%
\pgfpathcurveto{\pgfqpoint{6.169602in}{3.709200in}}{\pgfqpoint{6.165212in}{3.698600in}}{\pgfqpoint{6.165212in}{3.687550in}}%
\pgfpathcurveto{\pgfqpoint{6.165212in}{3.676500in}}{\pgfqpoint{6.169602in}{3.665901in}}{\pgfqpoint{6.177415in}{3.658088in}}%
\pgfpathcurveto{\pgfqpoint{6.185229in}{3.650274in}}{\pgfqpoint{6.195828in}{3.645884in}}{\pgfqpoint{6.206878in}{3.645884in}}%
\pgfpathclose%
\pgfusepath{stroke,fill}%
\end{pgfscope}%
\begin{pgfscope}%
\pgfpathrectangle{\pgfqpoint{0.570343in}{0.331635in}}{\pgfqpoint{9.300000in}{7.700000in}}%
\pgfusepath{clip}%
\pgfsetbuttcap%
\pgfsetroundjoin%
\definecolor{currentfill}{rgb}{1.000000,0.705882,0.509804}%
\pgfsetfillcolor{currentfill}%
\pgfsetlinewidth{0.481800pt}%
\definecolor{currentstroke}{rgb}{1.000000,1.000000,1.000000}%
\pgfsetstrokecolor{currentstroke}%
\pgfsetdash{}{0pt}%
\pgfpathmoveto{\pgfqpoint{6.595639in}{2.719570in}}%
\pgfpathcurveto{\pgfqpoint{6.606689in}{2.719570in}}{\pgfqpoint{6.617288in}{2.723960in}}{\pgfqpoint{6.625102in}{2.731774in}}%
\pgfpathcurveto{\pgfqpoint{6.632915in}{2.739588in}}{\pgfqpoint{6.637306in}{2.750187in}}{\pgfqpoint{6.637306in}{2.761237in}}%
\pgfpathcurveto{\pgfqpoint{6.637306in}{2.772287in}}{\pgfqpoint{6.632915in}{2.782886in}}{\pgfqpoint{6.625102in}{2.790700in}}%
\pgfpathcurveto{\pgfqpoint{6.617288in}{2.798513in}}{\pgfqpoint{6.606689in}{2.802903in}}{\pgfqpoint{6.595639in}{2.802903in}}%
\pgfpathcurveto{\pgfqpoint{6.584589in}{2.802903in}}{\pgfqpoint{6.573990in}{2.798513in}}{\pgfqpoint{6.566176in}{2.790700in}}%
\pgfpathcurveto{\pgfqpoint{6.558363in}{2.782886in}}{\pgfqpoint{6.553972in}{2.772287in}}{\pgfqpoint{6.553972in}{2.761237in}}%
\pgfpathcurveto{\pgfqpoint{6.553972in}{2.750187in}}{\pgfqpoint{6.558363in}{2.739588in}}{\pgfqpoint{6.566176in}{2.731774in}}%
\pgfpathcurveto{\pgfqpoint{6.573990in}{2.723960in}}{\pgfqpoint{6.584589in}{2.719570in}}{\pgfqpoint{6.595639in}{2.719570in}}%
\pgfpathclose%
\pgfusepath{stroke,fill}%
\end{pgfscope}%
\begin{pgfscope}%
\pgfpathrectangle{\pgfqpoint{0.570343in}{0.331635in}}{\pgfqpoint{9.300000in}{7.700000in}}%
\pgfusepath{clip}%
\pgfsetbuttcap%
\pgfsetroundjoin%
\definecolor{currentfill}{rgb}{1.000000,0.705882,0.509804}%
\pgfsetfillcolor{currentfill}%
\pgfsetlinewidth{0.481800pt}%
\definecolor{currentstroke}{rgb}{1.000000,1.000000,1.000000}%
\pgfsetstrokecolor{currentstroke}%
\pgfsetdash{}{0pt}%
\pgfpathmoveto{\pgfqpoint{6.543720in}{5.564756in}}%
\pgfpathcurveto{\pgfqpoint{6.554770in}{5.564756in}}{\pgfqpoint{6.565369in}{5.569147in}}{\pgfqpoint{6.573183in}{5.576960in}}%
\pgfpathcurveto{\pgfqpoint{6.580996in}{5.584774in}}{\pgfqpoint{6.585387in}{5.595373in}}{\pgfqpoint{6.585387in}{5.606423in}}%
\pgfpathcurveto{\pgfqpoint{6.585387in}{5.617473in}}{\pgfqpoint{6.580996in}{5.628072in}}{\pgfqpoint{6.573183in}{5.635886in}}%
\pgfpathcurveto{\pgfqpoint{6.565369in}{5.643700in}}{\pgfqpoint{6.554770in}{5.648090in}}{\pgfqpoint{6.543720in}{5.648090in}}%
\pgfpathcurveto{\pgfqpoint{6.532670in}{5.648090in}}{\pgfqpoint{6.522071in}{5.643700in}}{\pgfqpoint{6.514257in}{5.635886in}}%
\pgfpathcurveto{\pgfqpoint{6.506444in}{5.628072in}}{\pgfqpoint{6.502053in}{5.617473in}}{\pgfqpoint{6.502053in}{5.606423in}}%
\pgfpathcurveto{\pgfqpoint{6.502053in}{5.595373in}}{\pgfqpoint{6.506444in}{5.584774in}}{\pgfqpoint{6.514257in}{5.576960in}}%
\pgfpathcurveto{\pgfqpoint{6.522071in}{5.569147in}}{\pgfqpoint{6.532670in}{5.564756in}}{\pgfqpoint{6.543720in}{5.564756in}}%
\pgfpathclose%
\pgfusepath{stroke,fill}%
\end{pgfscope}%
\begin{pgfscope}%
\pgfpathrectangle{\pgfqpoint{0.570343in}{0.331635in}}{\pgfqpoint{9.300000in}{7.700000in}}%
\pgfusepath{clip}%
\pgfsetbuttcap%
\pgfsetroundjoin%
\definecolor{currentfill}{rgb}{1.000000,0.705882,0.509804}%
\pgfsetfillcolor{currentfill}%
\pgfsetlinewidth{0.481800pt}%
\definecolor{currentstroke}{rgb}{1.000000,1.000000,1.000000}%
\pgfsetstrokecolor{currentstroke}%
\pgfsetdash{}{0pt}%
\pgfpathmoveto{\pgfqpoint{6.147267in}{5.492362in}}%
\pgfpathcurveto{\pgfqpoint{6.158317in}{5.492362in}}{\pgfqpoint{6.168916in}{5.496752in}}{\pgfqpoint{6.176730in}{5.504566in}}%
\pgfpathcurveto{\pgfqpoint{6.184544in}{5.512379in}}{\pgfqpoint{6.188934in}{5.522978in}}{\pgfqpoint{6.188934in}{5.534029in}}%
\pgfpathcurveto{\pgfqpoint{6.188934in}{5.545079in}}{\pgfqpoint{6.184544in}{5.555678in}}{\pgfqpoint{6.176730in}{5.563491in}}%
\pgfpathcurveto{\pgfqpoint{6.168916in}{5.571305in}}{\pgfqpoint{6.158317in}{5.575695in}}{\pgfqpoint{6.147267in}{5.575695in}}%
\pgfpathcurveto{\pgfqpoint{6.136217in}{5.575695in}}{\pgfqpoint{6.125618in}{5.571305in}}{\pgfqpoint{6.117804in}{5.563491in}}%
\pgfpathcurveto{\pgfqpoint{6.109991in}{5.555678in}}{\pgfqpoint{6.105601in}{5.545079in}}{\pgfqpoint{6.105601in}{5.534029in}}%
\pgfpathcurveto{\pgfqpoint{6.105601in}{5.522978in}}{\pgfqpoint{6.109991in}{5.512379in}}{\pgfqpoint{6.117804in}{5.504566in}}%
\pgfpathcurveto{\pgfqpoint{6.125618in}{5.496752in}}{\pgfqpoint{6.136217in}{5.492362in}}{\pgfqpoint{6.147267in}{5.492362in}}%
\pgfpathclose%
\pgfusepath{stroke,fill}%
\end{pgfscope}%
\begin{pgfscope}%
\pgfpathrectangle{\pgfqpoint{0.570343in}{0.331635in}}{\pgfqpoint{9.300000in}{7.700000in}}%
\pgfusepath{clip}%
\pgfsetbuttcap%
\pgfsetroundjoin%
\definecolor{currentfill}{rgb}{1.000000,0.705882,0.509804}%
\pgfsetfillcolor{currentfill}%
\pgfsetlinewidth{0.481800pt}%
\definecolor{currentstroke}{rgb}{1.000000,1.000000,1.000000}%
\pgfsetstrokecolor{currentstroke}%
\pgfsetdash{}{0pt}%
\pgfpathmoveto{\pgfqpoint{6.478915in}{2.421630in}}%
\pgfpathcurveto{\pgfqpoint{6.489965in}{2.421630in}}{\pgfqpoint{6.500564in}{2.426020in}}{\pgfqpoint{6.508378in}{2.433833in}}%
\pgfpathcurveto{\pgfqpoint{6.516191in}{2.441647in}}{\pgfqpoint{6.520582in}{2.452246in}}{\pgfqpoint{6.520582in}{2.463296in}}%
\pgfpathcurveto{\pgfqpoint{6.520582in}{2.474346in}}{\pgfqpoint{6.516191in}{2.484945in}}{\pgfqpoint{6.508378in}{2.492759in}}%
\pgfpathcurveto{\pgfqpoint{6.500564in}{2.500573in}}{\pgfqpoint{6.489965in}{2.504963in}}{\pgfqpoint{6.478915in}{2.504963in}}%
\pgfpathcurveto{\pgfqpoint{6.467865in}{2.504963in}}{\pgfqpoint{6.457266in}{2.500573in}}{\pgfqpoint{6.449452in}{2.492759in}}%
\pgfpathcurveto{\pgfqpoint{6.441639in}{2.484945in}}{\pgfqpoint{6.437248in}{2.474346in}}{\pgfqpoint{6.437248in}{2.463296in}}%
\pgfpathcurveto{\pgfqpoint{6.437248in}{2.452246in}}{\pgfqpoint{6.441639in}{2.441647in}}{\pgfqpoint{6.449452in}{2.433833in}}%
\pgfpathcurveto{\pgfqpoint{6.457266in}{2.426020in}}{\pgfqpoint{6.467865in}{2.421630in}}{\pgfqpoint{6.478915in}{2.421630in}}%
\pgfpathclose%
\pgfusepath{stroke,fill}%
\end{pgfscope}%
\begin{pgfscope}%
\pgfpathrectangle{\pgfqpoint{0.570343in}{0.331635in}}{\pgfqpoint{9.300000in}{7.700000in}}%
\pgfusepath{clip}%
\pgfsetbuttcap%
\pgfsetroundjoin%
\definecolor{currentfill}{rgb}{1.000000,0.705882,0.509804}%
\pgfsetfillcolor{currentfill}%
\pgfsetlinewidth{0.481800pt}%
\definecolor{currentstroke}{rgb}{1.000000,1.000000,1.000000}%
\pgfsetstrokecolor{currentstroke}%
\pgfsetdash{}{0pt}%
\pgfpathmoveto{\pgfqpoint{6.474508in}{4.910497in}}%
\pgfpathcurveto{\pgfqpoint{6.485558in}{4.910497in}}{\pgfqpoint{6.496157in}{4.914887in}}{\pgfqpoint{6.503971in}{4.922700in}}%
\pgfpathcurveto{\pgfqpoint{6.511784in}{4.930514in}}{\pgfqpoint{6.516175in}{4.941113in}}{\pgfqpoint{6.516175in}{4.952163in}}%
\pgfpathcurveto{\pgfqpoint{6.516175in}{4.963213in}}{\pgfqpoint{6.511784in}{4.973812in}}{\pgfqpoint{6.503971in}{4.981626in}}%
\pgfpathcurveto{\pgfqpoint{6.496157in}{4.989440in}}{\pgfqpoint{6.485558in}{4.993830in}}{\pgfqpoint{6.474508in}{4.993830in}}%
\pgfpathcurveto{\pgfqpoint{6.463458in}{4.993830in}}{\pgfqpoint{6.452859in}{4.989440in}}{\pgfqpoint{6.445045in}{4.981626in}}%
\pgfpathcurveto{\pgfqpoint{6.437232in}{4.973812in}}{\pgfqpoint{6.432841in}{4.963213in}}{\pgfqpoint{6.432841in}{4.952163in}}%
\pgfpathcurveto{\pgfqpoint{6.432841in}{4.941113in}}{\pgfqpoint{6.437232in}{4.930514in}}{\pgfqpoint{6.445045in}{4.922700in}}%
\pgfpathcurveto{\pgfqpoint{6.452859in}{4.914887in}}{\pgfqpoint{6.463458in}{4.910497in}}{\pgfqpoint{6.474508in}{4.910497in}}%
\pgfpathclose%
\pgfusepath{stroke,fill}%
\end{pgfscope}%
\begin{pgfscope}%
\pgfpathrectangle{\pgfqpoint{0.570343in}{0.331635in}}{\pgfqpoint{9.300000in}{7.700000in}}%
\pgfusepath{clip}%
\pgfsetbuttcap%
\pgfsetroundjoin%
\definecolor{currentfill}{rgb}{1.000000,0.705882,0.509804}%
\pgfsetfillcolor{currentfill}%
\pgfsetlinewidth{0.481800pt}%
\definecolor{currentstroke}{rgb}{1.000000,1.000000,1.000000}%
\pgfsetstrokecolor{currentstroke}%
\pgfsetdash{}{0pt}%
\pgfpathmoveto{\pgfqpoint{6.197218in}{4.709497in}}%
\pgfpathcurveto{\pgfqpoint{6.208268in}{4.709497in}}{\pgfqpoint{6.218867in}{4.713887in}}{\pgfqpoint{6.226681in}{4.721701in}}%
\pgfpathcurveto{\pgfqpoint{6.234494in}{4.729514in}}{\pgfqpoint{6.238885in}{4.740113in}}{\pgfqpoint{6.238885in}{4.751164in}}%
\pgfpathcurveto{\pgfqpoint{6.238885in}{4.762214in}}{\pgfqpoint{6.234494in}{4.772813in}}{\pgfqpoint{6.226681in}{4.780626in}}%
\pgfpathcurveto{\pgfqpoint{6.218867in}{4.788440in}}{\pgfqpoint{6.208268in}{4.792830in}}{\pgfqpoint{6.197218in}{4.792830in}}%
\pgfpathcurveto{\pgfqpoint{6.186168in}{4.792830in}}{\pgfqpoint{6.175569in}{4.788440in}}{\pgfqpoint{6.167755in}{4.780626in}}%
\pgfpathcurveto{\pgfqpoint{6.159941in}{4.772813in}}{\pgfqpoint{6.155551in}{4.762214in}}{\pgfqpoint{6.155551in}{4.751164in}}%
\pgfpathcurveto{\pgfqpoint{6.155551in}{4.740113in}}{\pgfqpoint{6.159941in}{4.729514in}}{\pgfqpoint{6.167755in}{4.721701in}}%
\pgfpathcurveto{\pgfqpoint{6.175569in}{4.713887in}}{\pgfqpoint{6.186168in}{4.709497in}}{\pgfqpoint{6.197218in}{4.709497in}}%
\pgfpathclose%
\pgfusepath{stroke,fill}%
\end{pgfscope}%
\begin{pgfscope}%
\pgfpathrectangle{\pgfqpoint{0.570343in}{0.331635in}}{\pgfqpoint{9.300000in}{7.700000in}}%
\pgfusepath{clip}%
\pgfsetbuttcap%
\pgfsetroundjoin%
\definecolor{currentfill}{rgb}{1.000000,0.705882,0.509804}%
\pgfsetfillcolor{currentfill}%
\pgfsetlinewidth{0.481800pt}%
\definecolor{currentstroke}{rgb}{1.000000,1.000000,1.000000}%
\pgfsetstrokecolor{currentstroke}%
\pgfsetdash{}{0pt}%
\pgfpathmoveto{\pgfqpoint{5.938120in}{3.836822in}}%
\pgfpathcurveto{\pgfqpoint{5.949170in}{3.836822in}}{\pgfqpoint{5.959769in}{3.841212in}}{\pgfqpoint{5.967583in}{3.849026in}}%
\pgfpathcurveto{\pgfqpoint{5.975397in}{3.856839in}}{\pgfqpoint{5.979787in}{3.867438in}}{\pgfqpoint{5.979787in}{3.878488in}}%
\pgfpathcurveto{\pgfqpoint{5.979787in}{3.889538in}}{\pgfqpoint{5.975397in}{3.900138in}}{\pgfqpoint{5.967583in}{3.907951in}}%
\pgfpathcurveto{\pgfqpoint{5.959769in}{3.915765in}}{\pgfqpoint{5.949170in}{3.920155in}}{\pgfqpoint{5.938120in}{3.920155in}}%
\pgfpathcurveto{\pgfqpoint{5.927070in}{3.920155in}}{\pgfqpoint{5.916471in}{3.915765in}}{\pgfqpoint{5.908658in}{3.907951in}}%
\pgfpathcurveto{\pgfqpoint{5.900844in}{3.900138in}}{\pgfqpoint{5.896454in}{3.889538in}}{\pgfqpoint{5.896454in}{3.878488in}}%
\pgfpathcurveto{\pgfqpoint{5.896454in}{3.867438in}}{\pgfqpoint{5.900844in}{3.856839in}}{\pgfqpoint{5.908658in}{3.849026in}}%
\pgfpathcurveto{\pgfqpoint{5.916471in}{3.841212in}}{\pgfqpoint{5.927070in}{3.836822in}}{\pgfqpoint{5.938120in}{3.836822in}}%
\pgfpathclose%
\pgfusepath{stroke,fill}%
\end{pgfscope}%
\begin{pgfscope}%
\pgfpathrectangle{\pgfqpoint{0.570343in}{0.331635in}}{\pgfqpoint{9.300000in}{7.700000in}}%
\pgfusepath{clip}%
\pgfsetbuttcap%
\pgfsetroundjoin%
\definecolor{currentfill}{rgb}{1.000000,0.705882,0.509804}%
\pgfsetfillcolor{currentfill}%
\pgfsetlinewidth{0.481800pt}%
\definecolor{currentstroke}{rgb}{1.000000,1.000000,1.000000}%
\pgfsetstrokecolor{currentstroke}%
\pgfsetdash{}{0pt}%
\pgfpathmoveto{\pgfqpoint{6.148378in}{1.282690in}}%
\pgfpathcurveto{\pgfqpoint{6.159428in}{1.282690in}}{\pgfqpoint{6.170027in}{1.287080in}}{\pgfqpoint{6.177841in}{1.294894in}}%
\pgfpathcurveto{\pgfqpoint{6.185654in}{1.302707in}}{\pgfqpoint{6.190045in}{1.313307in}}{\pgfqpoint{6.190045in}{1.324357in}}%
\pgfpathcurveto{\pgfqpoint{6.190045in}{1.335407in}}{\pgfqpoint{6.185654in}{1.346006in}}{\pgfqpoint{6.177841in}{1.353819in}}%
\pgfpathcurveto{\pgfqpoint{6.170027in}{1.361633in}}{\pgfqpoint{6.159428in}{1.366023in}}{\pgfqpoint{6.148378in}{1.366023in}}%
\pgfpathcurveto{\pgfqpoint{6.137328in}{1.366023in}}{\pgfqpoint{6.126729in}{1.361633in}}{\pgfqpoint{6.118915in}{1.353819in}}%
\pgfpathcurveto{\pgfqpoint{6.111102in}{1.346006in}}{\pgfqpoint{6.106711in}{1.335407in}}{\pgfqpoint{6.106711in}{1.324357in}}%
\pgfpathcurveto{\pgfqpoint{6.106711in}{1.313307in}}{\pgfqpoint{6.111102in}{1.302707in}}{\pgfqpoint{6.118915in}{1.294894in}}%
\pgfpathcurveto{\pgfqpoint{6.126729in}{1.287080in}}{\pgfqpoint{6.137328in}{1.282690in}}{\pgfqpoint{6.148378in}{1.282690in}}%
\pgfpathclose%
\pgfusepath{stroke,fill}%
\end{pgfscope}%
\begin{pgfscope}%
\pgfpathrectangle{\pgfqpoint{0.570343in}{0.331635in}}{\pgfqpoint{9.300000in}{7.700000in}}%
\pgfusepath{clip}%
\pgfsetbuttcap%
\pgfsetroundjoin%
\definecolor{currentfill}{rgb}{1.000000,0.705882,0.509804}%
\pgfsetfillcolor{currentfill}%
\pgfsetlinewidth{0.481800pt}%
\definecolor{currentstroke}{rgb}{1.000000,1.000000,1.000000}%
\pgfsetstrokecolor{currentstroke}%
\pgfsetdash{}{0pt}%
\pgfpathmoveto{\pgfqpoint{6.264144in}{5.257015in}}%
\pgfpathcurveto{\pgfqpoint{6.275194in}{5.257015in}}{\pgfqpoint{6.285793in}{5.261405in}}{\pgfqpoint{6.293607in}{5.269219in}}%
\pgfpathcurveto{\pgfqpoint{6.301421in}{5.277033in}}{\pgfqpoint{6.305811in}{5.287632in}}{\pgfqpoint{6.305811in}{5.298682in}}%
\pgfpathcurveto{\pgfqpoint{6.305811in}{5.309732in}}{\pgfqpoint{6.301421in}{5.320331in}}{\pgfqpoint{6.293607in}{5.328144in}}%
\pgfpathcurveto{\pgfqpoint{6.285793in}{5.335958in}}{\pgfqpoint{6.275194in}{5.340348in}}{\pgfqpoint{6.264144in}{5.340348in}}%
\pgfpathcurveto{\pgfqpoint{6.253094in}{5.340348in}}{\pgfqpoint{6.242495in}{5.335958in}}{\pgfqpoint{6.234681in}{5.328144in}}%
\pgfpathcurveto{\pgfqpoint{6.226868in}{5.320331in}}{\pgfqpoint{6.222478in}{5.309732in}}{\pgfqpoint{6.222478in}{5.298682in}}%
\pgfpathcurveto{\pgfqpoint{6.222478in}{5.287632in}}{\pgfqpoint{6.226868in}{5.277033in}}{\pgfqpoint{6.234681in}{5.269219in}}%
\pgfpathcurveto{\pgfqpoint{6.242495in}{5.261405in}}{\pgfqpoint{6.253094in}{5.257015in}}{\pgfqpoint{6.264144in}{5.257015in}}%
\pgfpathclose%
\pgfusepath{stroke,fill}%
\end{pgfscope}%
\begin{pgfscope}%
\pgfpathrectangle{\pgfqpoint{0.570343in}{0.331635in}}{\pgfqpoint{9.300000in}{7.700000in}}%
\pgfusepath{clip}%
\pgfsetbuttcap%
\pgfsetroundjoin%
\definecolor{currentfill}{rgb}{1.000000,0.705882,0.509804}%
\pgfsetfillcolor{currentfill}%
\pgfsetlinewidth{0.481800pt}%
\definecolor{currentstroke}{rgb}{1.000000,1.000000,1.000000}%
\pgfsetstrokecolor{currentstroke}%
\pgfsetdash{}{0pt}%
\pgfpathmoveto{\pgfqpoint{6.250398in}{2.253374in}}%
\pgfpathcurveto{\pgfqpoint{6.261448in}{2.253374in}}{\pgfqpoint{6.272047in}{2.257764in}}{\pgfqpoint{6.279861in}{2.265578in}}%
\pgfpathcurveto{\pgfqpoint{6.287675in}{2.273392in}}{\pgfqpoint{6.292065in}{2.283991in}}{\pgfqpoint{6.292065in}{2.295041in}}%
\pgfpathcurveto{\pgfqpoint{6.292065in}{2.306091in}}{\pgfqpoint{6.287675in}{2.316690in}}{\pgfqpoint{6.279861in}{2.324503in}}%
\pgfpathcurveto{\pgfqpoint{6.272047in}{2.332317in}}{\pgfqpoint{6.261448in}{2.336707in}}{\pgfqpoint{6.250398in}{2.336707in}}%
\pgfpathcurveto{\pgfqpoint{6.239348in}{2.336707in}}{\pgfqpoint{6.228749in}{2.332317in}}{\pgfqpoint{6.220936in}{2.324503in}}%
\pgfpathcurveto{\pgfqpoint{6.213122in}{2.316690in}}{\pgfqpoint{6.208732in}{2.306091in}}{\pgfqpoint{6.208732in}{2.295041in}}%
\pgfpathcurveto{\pgfqpoint{6.208732in}{2.283991in}}{\pgfqpoint{6.213122in}{2.273392in}}{\pgfqpoint{6.220936in}{2.265578in}}%
\pgfpathcurveto{\pgfqpoint{6.228749in}{2.257764in}}{\pgfqpoint{6.239348in}{2.253374in}}{\pgfqpoint{6.250398in}{2.253374in}}%
\pgfpathclose%
\pgfusepath{stroke,fill}%
\end{pgfscope}%
\begin{pgfscope}%
\pgfpathrectangle{\pgfqpoint{0.570343in}{0.331635in}}{\pgfqpoint{9.300000in}{7.700000in}}%
\pgfusepath{clip}%
\pgfsetbuttcap%
\pgfsetroundjoin%
\definecolor{currentfill}{rgb}{1.000000,0.705882,0.509804}%
\pgfsetfillcolor{currentfill}%
\pgfsetlinewidth{0.481800pt}%
\definecolor{currentstroke}{rgb}{1.000000,1.000000,1.000000}%
\pgfsetstrokecolor{currentstroke}%
\pgfsetdash{}{0pt}%
\pgfpathmoveto{\pgfqpoint{6.335293in}{6.972145in}}%
\pgfpathcurveto{\pgfqpoint{6.346343in}{6.972145in}}{\pgfqpoint{6.356942in}{6.976535in}}{\pgfqpoint{6.364756in}{6.984349in}}%
\pgfpathcurveto{\pgfqpoint{6.372569in}{6.992163in}}{\pgfqpoint{6.376960in}{7.002762in}}{\pgfqpoint{6.376960in}{7.013812in}}%
\pgfpathcurveto{\pgfqpoint{6.376960in}{7.024862in}}{\pgfqpoint{6.372569in}{7.035461in}}{\pgfqpoint{6.364756in}{7.043275in}}%
\pgfpathcurveto{\pgfqpoint{6.356942in}{7.051088in}}{\pgfqpoint{6.346343in}{7.055479in}}{\pgfqpoint{6.335293in}{7.055479in}}%
\pgfpathcurveto{\pgfqpoint{6.324243in}{7.055479in}}{\pgfqpoint{6.313644in}{7.051088in}}{\pgfqpoint{6.305830in}{7.043275in}}%
\pgfpathcurveto{\pgfqpoint{6.298016in}{7.035461in}}{\pgfqpoint{6.293626in}{7.024862in}}{\pgfqpoint{6.293626in}{7.013812in}}%
\pgfpathcurveto{\pgfqpoint{6.293626in}{7.002762in}}{\pgfqpoint{6.298016in}{6.992163in}}{\pgfqpoint{6.305830in}{6.984349in}}%
\pgfpathcurveto{\pgfqpoint{6.313644in}{6.976535in}}{\pgfqpoint{6.324243in}{6.972145in}}{\pgfqpoint{6.335293in}{6.972145in}}%
\pgfpathclose%
\pgfusepath{stroke,fill}%
\end{pgfscope}%
\begin{pgfscope}%
\pgfpathrectangle{\pgfqpoint{0.570343in}{0.331635in}}{\pgfqpoint{9.300000in}{7.700000in}}%
\pgfusepath{clip}%
\pgfsetbuttcap%
\pgfsetroundjoin%
\definecolor{currentfill}{rgb}{1.000000,0.705882,0.509804}%
\pgfsetfillcolor{currentfill}%
\pgfsetlinewidth{0.481800pt}%
\definecolor{currentstroke}{rgb}{1.000000,1.000000,1.000000}%
\pgfsetstrokecolor{currentstroke}%
\pgfsetdash{}{0pt}%
\pgfpathmoveto{\pgfqpoint{9.447616in}{2.029073in}}%
\pgfpathcurveto{\pgfqpoint{9.458666in}{2.029073in}}{\pgfqpoint{9.469265in}{2.033463in}}{\pgfqpoint{9.477079in}{2.041277in}}%
\pgfpathcurveto{\pgfqpoint{9.484892in}{2.049090in}}{\pgfqpoint{9.489283in}{2.059689in}}{\pgfqpoint{9.489283in}{2.070739in}}%
\pgfpathcurveto{\pgfqpoint{9.489283in}{2.081789in}}{\pgfqpoint{9.484892in}{2.092388in}}{\pgfqpoint{9.477079in}{2.100202in}}%
\pgfpathcurveto{\pgfqpoint{9.469265in}{2.108016in}}{\pgfqpoint{9.458666in}{2.112406in}}{\pgfqpoint{9.447616in}{2.112406in}}%
\pgfpathcurveto{\pgfqpoint{9.436566in}{2.112406in}}{\pgfqpoint{9.425967in}{2.108016in}}{\pgfqpoint{9.418153in}{2.100202in}}%
\pgfpathcurveto{\pgfqpoint{9.410340in}{2.092388in}}{\pgfqpoint{9.405949in}{2.081789in}}{\pgfqpoint{9.405949in}{2.070739in}}%
\pgfpathcurveto{\pgfqpoint{9.405949in}{2.059689in}}{\pgfqpoint{9.410340in}{2.049090in}}{\pgfqpoint{9.418153in}{2.041277in}}%
\pgfpathcurveto{\pgfqpoint{9.425967in}{2.033463in}}{\pgfqpoint{9.436566in}{2.029073in}}{\pgfqpoint{9.447616in}{2.029073in}}%
\pgfpathclose%
\pgfusepath{stroke,fill}%
\end{pgfscope}%
\begin{pgfscope}%
\pgfpathrectangle{\pgfqpoint{0.570343in}{0.331635in}}{\pgfqpoint{9.300000in}{7.700000in}}%
\pgfusepath{clip}%
\pgfsetbuttcap%
\pgfsetroundjoin%
\definecolor{currentfill}{rgb}{1.000000,0.705882,0.509804}%
\pgfsetfillcolor{currentfill}%
\pgfsetlinewidth{0.481800pt}%
\definecolor{currentstroke}{rgb}{1.000000,1.000000,1.000000}%
\pgfsetstrokecolor{currentstroke}%
\pgfsetdash{}{0pt}%
\pgfpathmoveto{\pgfqpoint{6.163528in}{3.854886in}}%
\pgfpathcurveto{\pgfqpoint{6.174578in}{3.854886in}}{\pgfqpoint{6.185177in}{3.859276in}}{\pgfqpoint{6.192990in}{3.867090in}}%
\pgfpathcurveto{\pgfqpoint{6.200804in}{3.874903in}}{\pgfqpoint{6.205194in}{3.885502in}}{\pgfqpoint{6.205194in}{3.896552in}}%
\pgfpathcurveto{\pgfqpoint{6.205194in}{3.907602in}}{\pgfqpoint{6.200804in}{3.918202in}}{\pgfqpoint{6.192990in}{3.926015in}}%
\pgfpathcurveto{\pgfqpoint{6.185177in}{3.933829in}}{\pgfqpoint{6.174578in}{3.938219in}}{\pgfqpoint{6.163528in}{3.938219in}}%
\pgfpathcurveto{\pgfqpoint{6.152478in}{3.938219in}}{\pgfqpoint{6.141879in}{3.933829in}}{\pgfqpoint{6.134065in}{3.926015in}}%
\pgfpathcurveto{\pgfqpoint{6.126251in}{3.918202in}}{\pgfqpoint{6.121861in}{3.907602in}}{\pgfqpoint{6.121861in}{3.896552in}}%
\pgfpathcurveto{\pgfqpoint{6.121861in}{3.885502in}}{\pgfqpoint{6.126251in}{3.874903in}}{\pgfqpoint{6.134065in}{3.867090in}}%
\pgfpathcurveto{\pgfqpoint{6.141879in}{3.859276in}}{\pgfqpoint{6.152478in}{3.854886in}}{\pgfqpoint{6.163528in}{3.854886in}}%
\pgfpathclose%
\pgfusepath{stroke,fill}%
\end{pgfscope}%
\begin{pgfscope}%
\pgfpathrectangle{\pgfqpoint{0.570343in}{0.331635in}}{\pgfqpoint{9.300000in}{7.700000in}}%
\pgfusepath{clip}%
\pgfsetbuttcap%
\pgfsetroundjoin%
\definecolor{currentfill}{rgb}{1.000000,0.705882,0.509804}%
\pgfsetfillcolor{currentfill}%
\pgfsetlinewidth{0.481800pt}%
\definecolor{currentstroke}{rgb}{1.000000,1.000000,1.000000}%
\pgfsetstrokecolor{currentstroke}%
\pgfsetdash{}{0pt}%
\pgfpathmoveto{\pgfqpoint{6.074829in}{5.518293in}}%
\pgfpathcurveto{\pgfqpoint{6.085879in}{5.518293in}}{\pgfqpoint{6.096478in}{5.522684in}}{\pgfqpoint{6.104292in}{5.530497in}}%
\pgfpathcurveto{\pgfqpoint{6.112106in}{5.538311in}}{\pgfqpoint{6.116496in}{5.548910in}}{\pgfqpoint{6.116496in}{5.559960in}}%
\pgfpathcurveto{\pgfqpoint{6.116496in}{5.571010in}}{\pgfqpoint{6.112106in}{5.581609in}}{\pgfqpoint{6.104292in}{5.589423in}}%
\pgfpathcurveto{\pgfqpoint{6.096478in}{5.597236in}}{\pgfqpoint{6.085879in}{5.601627in}}{\pgfqpoint{6.074829in}{5.601627in}}%
\pgfpathcurveto{\pgfqpoint{6.063779in}{5.601627in}}{\pgfqpoint{6.053180in}{5.597236in}}{\pgfqpoint{6.045366in}{5.589423in}}%
\pgfpathcurveto{\pgfqpoint{6.037553in}{5.581609in}}{\pgfqpoint{6.033162in}{5.571010in}}{\pgfqpoint{6.033162in}{5.559960in}}%
\pgfpathcurveto{\pgfqpoint{6.033162in}{5.548910in}}{\pgfqpoint{6.037553in}{5.538311in}}{\pgfqpoint{6.045366in}{5.530497in}}%
\pgfpathcurveto{\pgfqpoint{6.053180in}{5.522684in}}{\pgfqpoint{6.063779in}{5.518293in}}{\pgfqpoint{6.074829in}{5.518293in}}%
\pgfpathclose%
\pgfusepath{stroke,fill}%
\end{pgfscope}%
\begin{pgfscope}%
\pgfpathrectangle{\pgfqpoint{0.570343in}{0.331635in}}{\pgfqpoint{9.300000in}{7.700000in}}%
\pgfusepath{clip}%
\pgfsetbuttcap%
\pgfsetroundjoin%
\definecolor{currentfill}{rgb}{1.000000,0.705882,0.509804}%
\pgfsetfillcolor{currentfill}%
\pgfsetlinewidth{0.481800pt}%
\definecolor{currentstroke}{rgb}{1.000000,1.000000,1.000000}%
\pgfsetstrokecolor{currentstroke}%
\pgfsetdash{}{0pt}%
\pgfpathmoveto{\pgfqpoint{6.249575in}{4.553794in}}%
\pgfpathcurveto{\pgfqpoint{6.260625in}{4.553794in}}{\pgfqpoint{6.271224in}{4.558184in}}{\pgfqpoint{6.279038in}{4.565998in}}%
\pgfpathcurveto{\pgfqpoint{6.286852in}{4.573812in}}{\pgfqpoint{6.291242in}{4.584411in}}{\pgfqpoint{6.291242in}{4.595461in}}%
\pgfpathcurveto{\pgfqpoint{6.291242in}{4.606511in}}{\pgfqpoint{6.286852in}{4.617110in}}{\pgfqpoint{6.279038in}{4.624924in}}%
\pgfpathcurveto{\pgfqpoint{6.271224in}{4.632737in}}{\pgfqpoint{6.260625in}{4.637127in}}{\pgfqpoint{6.249575in}{4.637127in}}%
\pgfpathcurveto{\pgfqpoint{6.238525in}{4.637127in}}{\pgfqpoint{6.227926in}{4.632737in}}{\pgfqpoint{6.220113in}{4.624924in}}%
\pgfpathcurveto{\pgfqpoint{6.212299in}{4.617110in}}{\pgfqpoint{6.207909in}{4.606511in}}{\pgfqpoint{6.207909in}{4.595461in}}%
\pgfpathcurveto{\pgfqpoint{6.207909in}{4.584411in}}{\pgfqpoint{6.212299in}{4.573812in}}{\pgfqpoint{6.220113in}{4.565998in}}%
\pgfpathcurveto{\pgfqpoint{6.227926in}{4.558184in}}{\pgfqpoint{6.238525in}{4.553794in}}{\pgfqpoint{6.249575in}{4.553794in}}%
\pgfpathclose%
\pgfusepath{stroke,fill}%
\end{pgfscope}%
\begin{pgfscope}%
\pgfpathrectangle{\pgfqpoint{0.570343in}{0.331635in}}{\pgfqpoint{9.300000in}{7.700000in}}%
\pgfusepath{clip}%
\pgfsetbuttcap%
\pgfsetroundjoin%
\definecolor{currentfill}{rgb}{1.000000,0.705882,0.509804}%
\pgfsetfillcolor{currentfill}%
\pgfsetlinewidth{0.481800pt}%
\definecolor{currentstroke}{rgb}{1.000000,1.000000,1.000000}%
\pgfsetstrokecolor{currentstroke}%
\pgfsetdash{}{0pt}%
\pgfpathmoveto{\pgfqpoint{6.295225in}{2.267425in}}%
\pgfpathcurveto{\pgfqpoint{6.306276in}{2.267425in}}{\pgfqpoint{6.316875in}{2.271815in}}{\pgfqpoint{6.324688in}{2.279628in}}%
\pgfpathcurveto{\pgfqpoint{6.332502in}{2.287442in}}{\pgfqpoint{6.336892in}{2.298041in}}{\pgfqpoint{6.336892in}{2.309091in}}%
\pgfpathcurveto{\pgfqpoint{6.336892in}{2.320141in}}{\pgfqpoint{6.332502in}{2.330740in}}{\pgfqpoint{6.324688in}{2.338554in}}%
\pgfpathcurveto{\pgfqpoint{6.316875in}{2.346368in}}{\pgfqpoint{6.306276in}{2.350758in}}{\pgfqpoint{6.295225in}{2.350758in}}%
\pgfpathcurveto{\pgfqpoint{6.284175in}{2.350758in}}{\pgfqpoint{6.273576in}{2.346368in}}{\pgfqpoint{6.265763in}{2.338554in}}%
\pgfpathcurveto{\pgfqpoint{6.257949in}{2.330740in}}{\pgfqpoint{6.253559in}{2.320141in}}{\pgfqpoint{6.253559in}{2.309091in}}%
\pgfpathcurveto{\pgfqpoint{6.253559in}{2.298041in}}{\pgfqpoint{6.257949in}{2.287442in}}{\pgfqpoint{6.265763in}{2.279628in}}%
\pgfpathcurveto{\pgfqpoint{6.273576in}{2.271815in}}{\pgfqpoint{6.284175in}{2.267425in}}{\pgfqpoint{6.295225in}{2.267425in}}%
\pgfpathclose%
\pgfusepath{stroke,fill}%
\end{pgfscope}%
\begin{pgfscope}%
\pgfpathrectangle{\pgfqpoint{0.570343in}{0.331635in}}{\pgfqpoint{9.300000in}{7.700000in}}%
\pgfusepath{clip}%
\pgfsetbuttcap%
\pgfsetroundjoin%
\definecolor{currentfill}{rgb}{1.000000,0.705882,0.509804}%
\pgfsetfillcolor{currentfill}%
\pgfsetlinewidth{0.481800pt}%
\definecolor{currentstroke}{rgb}{1.000000,1.000000,1.000000}%
\pgfsetstrokecolor{currentstroke}%
\pgfsetdash{}{0pt}%
\pgfpathmoveto{\pgfqpoint{6.124439in}{4.496361in}}%
\pgfpathcurveto{\pgfqpoint{6.135489in}{4.496361in}}{\pgfqpoint{6.146089in}{4.500751in}}{\pgfqpoint{6.153902in}{4.508565in}}%
\pgfpathcurveto{\pgfqpoint{6.161716in}{4.516378in}}{\pgfqpoint{6.166106in}{4.526977in}}{\pgfqpoint{6.166106in}{4.538027in}}%
\pgfpathcurveto{\pgfqpoint{6.166106in}{4.549078in}}{\pgfqpoint{6.161716in}{4.559677in}}{\pgfqpoint{6.153902in}{4.567490in}}%
\pgfpathcurveto{\pgfqpoint{6.146089in}{4.575304in}}{\pgfqpoint{6.135489in}{4.579694in}}{\pgfqpoint{6.124439in}{4.579694in}}%
\pgfpathcurveto{\pgfqpoint{6.113389in}{4.579694in}}{\pgfqpoint{6.102790in}{4.575304in}}{\pgfqpoint{6.094977in}{4.567490in}}%
\pgfpathcurveto{\pgfqpoint{6.087163in}{4.559677in}}{\pgfqpoint{6.082773in}{4.549078in}}{\pgfqpoint{6.082773in}{4.538027in}}%
\pgfpathcurveto{\pgfqpoint{6.082773in}{4.526977in}}{\pgfqpoint{6.087163in}{4.516378in}}{\pgfqpoint{6.094977in}{4.508565in}}%
\pgfpathcurveto{\pgfqpoint{6.102790in}{4.500751in}}{\pgfqpoint{6.113389in}{4.496361in}}{\pgfqpoint{6.124439in}{4.496361in}}%
\pgfpathclose%
\pgfusepath{stroke,fill}%
\end{pgfscope}%
\begin{pgfscope}%
\pgfpathrectangle{\pgfqpoint{0.570343in}{0.331635in}}{\pgfqpoint{9.300000in}{7.700000in}}%
\pgfusepath{clip}%
\pgfsetbuttcap%
\pgfsetroundjoin%
\definecolor{currentfill}{rgb}{1.000000,0.705882,0.509804}%
\pgfsetfillcolor{currentfill}%
\pgfsetlinewidth{0.481800pt}%
\definecolor{currentstroke}{rgb}{1.000000,1.000000,1.000000}%
\pgfsetstrokecolor{currentstroke}%
\pgfsetdash{}{0pt}%
\pgfpathmoveto{\pgfqpoint{6.053358in}{6.320598in}}%
\pgfpathcurveto{\pgfqpoint{6.064408in}{6.320598in}}{\pgfqpoint{6.075007in}{6.324988in}}{\pgfqpoint{6.082821in}{6.332802in}}%
\pgfpathcurveto{\pgfqpoint{6.090634in}{6.340615in}}{\pgfqpoint{6.095024in}{6.351214in}}{\pgfqpoint{6.095024in}{6.362264in}}%
\pgfpathcurveto{\pgfqpoint{6.095024in}{6.373315in}}{\pgfqpoint{6.090634in}{6.383914in}}{\pgfqpoint{6.082821in}{6.391727in}}%
\pgfpathcurveto{\pgfqpoint{6.075007in}{6.399541in}}{\pgfqpoint{6.064408in}{6.403931in}}{\pgfqpoint{6.053358in}{6.403931in}}%
\pgfpathcurveto{\pgfqpoint{6.042308in}{6.403931in}}{\pgfqpoint{6.031709in}{6.399541in}}{\pgfqpoint{6.023895in}{6.391727in}}%
\pgfpathcurveto{\pgfqpoint{6.016081in}{6.383914in}}{\pgfqpoint{6.011691in}{6.373315in}}{\pgfqpoint{6.011691in}{6.362264in}}%
\pgfpathcurveto{\pgfqpoint{6.011691in}{6.351214in}}{\pgfqpoint{6.016081in}{6.340615in}}{\pgfqpoint{6.023895in}{6.332802in}}%
\pgfpathcurveto{\pgfqpoint{6.031709in}{6.324988in}}{\pgfqpoint{6.042308in}{6.320598in}}{\pgfqpoint{6.053358in}{6.320598in}}%
\pgfpathclose%
\pgfusepath{stroke,fill}%
\end{pgfscope}%
\begin{pgfscope}%
\pgfpathrectangle{\pgfqpoint{0.570343in}{0.331635in}}{\pgfqpoint{9.300000in}{7.700000in}}%
\pgfusepath{clip}%
\pgfsetbuttcap%
\pgfsetroundjoin%
\definecolor{currentfill}{rgb}{1.000000,0.705882,0.509804}%
\pgfsetfillcolor{currentfill}%
\pgfsetlinewidth{0.481800pt}%
\definecolor{currentstroke}{rgb}{1.000000,1.000000,1.000000}%
\pgfsetstrokecolor{currentstroke}%
\pgfsetdash{}{0pt}%
\pgfpathmoveto{\pgfqpoint{6.175797in}{4.240191in}}%
\pgfpathcurveto{\pgfqpoint{6.186847in}{4.240191in}}{\pgfqpoint{6.197446in}{4.244582in}}{\pgfqpoint{6.205260in}{4.252395in}}%
\pgfpathcurveto{\pgfqpoint{6.213073in}{4.260209in}}{\pgfqpoint{6.217463in}{4.270808in}}{\pgfqpoint{6.217463in}{4.281858in}}%
\pgfpathcurveto{\pgfqpoint{6.217463in}{4.292908in}}{\pgfqpoint{6.213073in}{4.303507in}}{\pgfqpoint{6.205260in}{4.311321in}}%
\pgfpathcurveto{\pgfqpoint{6.197446in}{4.319134in}}{\pgfqpoint{6.186847in}{4.323525in}}{\pgfqpoint{6.175797in}{4.323525in}}%
\pgfpathcurveto{\pgfqpoint{6.164747in}{4.323525in}}{\pgfqpoint{6.154148in}{4.319134in}}{\pgfqpoint{6.146334in}{4.311321in}}%
\pgfpathcurveto{\pgfqpoint{6.138520in}{4.303507in}}{\pgfqpoint{6.134130in}{4.292908in}}{\pgfqpoint{6.134130in}{4.281858in}}%
\pgfpathcurveto{\pgfqpoint{6.134130in}{4.270808in}}{\pgfqpoint{6.138520in}{4.260209in}}{\pgfqpoint{6.146334in}{4.252395in}}%
\pgfpathcurveto{\pgfqpoint{6.154148in}{4.244582in}}{\pgfqpoint{6.164747in}{4.240191in}}{\pgfqpoint{6.175797in}{4.240191in}}%
\pgfpathclose%
\pgfusepath{stroke,fill}%
\end{pgfscope}%
\begin{pgfscope}%
\pgfpathrectangle{\pgfqpoint{0.570343in}{0.331635in}}{\pgfqpoint{9.300000in}{7.700000in}}%
\pgfusepath{clip}%
\pgfsetbuttcap%
\pgfsetroundjoin%
\definecolor{currentfill}{rgb}{1.000000,0.705882,0.509804}%
\pgfsetfillcolor{currentfill}%
\pgfsetlinewidth{0.481800pt}%
\definecolor{currentstroke}{rgb}{1.000000,1.000000,1.000000}%
\pgfsetstrokecolor{currentstroke}%
\pgfsetdash{}{0pt}%
\pgfpathmoveto{\pgfqpoint{6.585001in}{1.377648in}}%
\pgfpathcurveto{\pgfqpoint{6.596051in}{1.377648in}}{\pgfqpoint{6.606650in}{1.382038in}}{\pgfqpoint{6.614464in}{1.389852in}}%
\pgfpathcurveto{\pgfqpoint{6.622277in}{1.397665in}}{\pgfqpoint{6.626668in}{1.408265in}}{\pgfqpoint{6.626668in}{1.419315in}}%
\pgfpathcurveto{\pgfqpoint{6.626668in}{1.430365in}}{\pgfqpoint{6.622277in}{1.440964in}}{\pgfqpoint{6.614464in}{1.448777in}}%
\pgfpathcurveto{\pgfqpoint{6.606650in}{1.456591in}}{\pgfqpoint{6.596051in}{1.460981in}}{\pgfqpoint{6.585001in}{1.460981in}}%
\pgfpathcurveto{\pgfqpoint{6.573951in}{1.460981in}}{\pgfqpoint{6.563352in}{1.456591in}}{\pgfqpoint{6.555538in}{1.448777in}}%
\pgfpathcurveto{\pgfqpoint{6.547725in}{1.440964in}}{\pgfqpoint{6.543334in}{1.430365in}}{\pgfqpoint{6.543334in}{1.419315in}}%
\pgfpathcurveto{\pgfqpoint{6.543334in}{1.408265in}}{\pgfqpoint{6.547725in}{1.397665in}}{\pgfqpoint{6.555538in}{1.389852in}}%
\pgfpathcurveto{\pgfqpoint{6.563352in}{1.382038in}}{\pgfqpoint{6.573951in}{1.377648in}}{\pgfqpoint{6.585001in}{1.377648in}}%
\pgfpathclose%
\pgfusepath{stroke,fill}%
\end{pgfscope}%
\begin{pgfscope}%
\pgfpathrectangle{\pgfqpoint{0.570343in}{0.331635in}}{\pgfqpoint{9.300000in}{7.700000in}}%
\pgfusepath{clip}%
\pgfsetbuttcap%
\pgfsetroundjoin%
\definecolor{currentfill}{rgb}{1.000000,0.705882,0.509804}%
\pgfsetfillcolor{currentfill}%
\pgfsetlinewidth{0.481800pt}%
\definecolor{currentstroke}{rgb}{1.000000,1.000000,1.000000}%
\pgfsetstrokecolor{currentstroke}%
\pgfsetdash{}{0pt}%
\pgfpathmoveto{\pgfqpoint{6.280311in}{4.900351in}}%
\pgfpathcurveto{\pgfqpoint{6.291361in}{4.900351in}}{\pgfqpoint{6.301960in}{4.904741in}}{\pgfqpoint{6.309774in}{4.912555in}}%
\pgfpathcurveto{\pgfqpoint{6.317588in}{4.920369in}}{\pgfqpoint{6.321978in}{4.930968in}}{\pgfqpoint{6.321978in}{4.942018in}}%
\pgfpathcurveto{\pgfqpoint{6.321978in}{4.953068in}}{\pgfqpoint{6.317588in}{4.963667in}}{\pgfqpoint{6.309774in}{4.971481in}}%
\pgfpathcurveto{\pgfqpoint{6.301960in}{4.979294in}}{\pgfqpoint{6.291361in}{4.983685in}}{\pgfqpoint{6.280311in}{4.983685in}}%
\pgfpathcurveto{\pgfqpoint{6.269261in}{4.983685in}}{\pgfqpoint{6.258662in}{4.979294in}}{\pgfqpoint{6.250848in}{4.971481in}}%
\pgfpathcurveto{\pgfqpoint{6.243035in}{4.963667in}}{\pgfqpoint{6.238645in}{4.953068in}}{\pgfqpoint{6.238645in}{4.942018in}}%
\pgfpathcurveto{\pgfqpoint{6.238645in}{4.930968in}}{\pgfqpoint{6.243035in}{4.920369in}}{\pgfqpoint{6.250848in}{4.912555in}}%
\pgfpathcurveto{\pgfqpoint{6.258662in}{4.904741in}}{\pgfqpoint{6.269261in}{4.900351in}}{\pgfqpoint{6.280311in}{4.900351in}}%
\pgfpathclose%
\pgfusepath{stroke,fill}%
\end{pgfscope}%
\begin{pgfscope}%
\pgfpathrectangle{\pgfqpoint{0.570343in}{0.331635in}}{\pgfqpoint{9.300000in}{7.700000in}}%
\pgfusepath{clip}%
\pgfsetbuttcap%
\pgfsetroundjoin%
\definecolor{currentfill}{rgb}{1.000000,0.705882,0.509804}%
\pgfsetfillcolor{currentfill}%
\pgfsetlinewidth{0.481800pt}%
\definecolor{currentstroke}{rgb}{1.000000,1.000000,1.000000}%
\pgfsetstrokecolor{currentstroke}%
\pgfsetdash{}{0pt}%
\pgfpathmoveto{\pgfqpoint{6.490755in}{2.810294in}}%
\pgfpathcurveto{\pgfqpoint{6.501805in}{2.810294in}}{\pgfqpoint{6.512404in}{2.814684in}}{\pgfqpoint{6.520218in}{2.822498in}}%
\pgfpathcurveto{\pgfqpoint{6.528031in}{2.830312in}}{\pgfqpoint{6.532422in}{2.840911in}}{\pgfqpoint{6.532422in}{2.851961in}}%
\pgfpathcurveto{\pgfqpoint{6.532422in}{2.863011in}}{\pgfqpoint{6.528031in}{2.873610in}}{\pgfqpoint{6.520218in}{2.881424in}}%
\pgfpathcurveto{\pgfqpoint{6.512404in}{2.889237in}}{\pgfqpoint{6.501805in}{2.893628in}}{\pgfqpoint{6.490755in}{2.893628in}}%
\pgfpathcurveto{\pgfqpoint{6.479705in}{2.893628in}}{\pgfqpoint{6.469106in}{2.889237in}}{\pgfqpoint{6.461292in}{2.881424in}}%
\pgfpathcurveto{\pgfqpoint{6.453479in}{2.873610in}}{\pgfqpoint{6.449088in}{2.863011in}}{\pgfqpoint{6.449088in}{2.851961in}}%
\pgfpathcurveto{\pgfqpoint{6.449088in}{2.840911in}}{\pgfqpoint{6.453479in}{2.830312in}}{\pgfqpoint{6.461292in}{2.822498in}}%
\pgfpathcurveto{\pgfqpoint{6.469106in}{2.814684in}}{\pgfqpoint{6.479705in}{2.810294in}}{\pgfqpoint{6.490755in}{2.810294in}}%
\pgfpathclose%
\pgfusepath{stroke,fill}%
\end{pgfscope}%
\begin{pgfscope}%
\pgfpathrectangle{\pgfqpoint{0.570343in}{0.331635in}}{\pgfqpoint{9.300000in}{7.700000in}}%
\pgfusepath{clip}%
\pgfsetbuttcap%
\pgfsetroundjoin%
\definecolor{currentfill}{rgb}{1.000000,0.705882,0.509804}%
\pgfsetfillcolor{currentfill}%
\pgfsetlinewidth{0.481800pt}%
\definecolor{currentstroke}{rgb}{1.000000,1.000000,1.000000}%
\pgfsetstrokecolor{currentstroke}%
\pgfsetdash{}{0pt}%
\pgfpathmoveto{\pgfqpoint{6.415715in}{5.367353in}}%
\pgfpathcurveto{\pgfqpoint{6.426765in}{5.367353in}}{\pgfqpoint{6.437364in}{5.371744in}}{\pgfqpoint{6.445178in}{5.379557in}}%
\pgfpathcurveto{\pgfqpoint{6.452991in}{5.387371in}}{\pgfqpoint{6.457382in}{5.397970in}}{\pgfqpoint{6.457382in}{5.409020in}}%
\pgfpathcurveto{\pgfqpoint{6.457382in}{5.420070in}}{\pgfqpoint{6.452991in}{5.430669in}}{\pgfqpoint{6.445178in}{5.438483in}}%
\pgfpathcurveto{\pgfqpoint{6.437364in}{5.446296in}}{\pgfqpoint{6.426765in}{5.450687in}}{\pgfqpoint{6.415715in}{5.450687in}}%
\pgfpathcurveto{\pgfqpoint{6.404665in}{5.450687in}}{\pgfqpoint{6.394066in}{5.446296in}}{\pgfqpoint{6.386252in}{5.438483in}}%
\pgfpathcurveto{\pgfqpoint{6.378439in}{5.430669in}}{\pgfqpoint{6.374048in}{5.420070in}}{\pgfqpoint{6.374048in}{5.409020in}}%
\pgfpathcurveto{\pgfqpoint{6.374048in}{5.397970in}}{\pgfqpoint{6.378439in}{5.387371in}}{\pgfqpoint{6.386252in}{5.379557in}}%
\pgfpathcurveto{\pgfqpoint{6.394066in}{5.371744in}}{\pgfqpoint{6.404665in}{5.367353in}}{\pgfqpoint{6.415715in}{5.367353in}}%
\pgfpathclose%
\pgfusepath{stroke,fill}%
\end{pgfscope}%
\begin{pgfscope}%
\pgfpathrectangle{\pgfqpoint{0.570343in}{0.331635in}}{\pgfqpoint{9.300000in}{7.700000in}}%
\pgfusepath{clip}%
\pgfsetbuttcap%
\pgfsetroundjoin%
\definecolor{currentfill}{rgb}{1.000000,0.705882,0.509804}%
\pgfsetfillcolor{currentfill}%
\pgfsetlinewidth{0.481800pt}%
\definecolor{currentstroke}{rgb}{1.000000,1.000000,1.000000}%
\pgfsetstrokecolor{currentstroke}%
\pgfsetdash{}{0pt}%
\pgfpathmoveto{\pgfqpoint{6.578206in}{3.578201in}}%
\pgfpathcurveto{\pgfqpoint{6.589256in}{3.578201in}}{\pgfqpoint{6.599855in}{3.582591in}}{\pgfqpoint{6.607669in}{3.590404in}}%
\pgfpathcurveto{\pgfqpoint{6.615482in}{3.598218in}}{\pgfqpoint{6.619873in}{3.608817in}}{\pgfqpoint{6.619873in}{3.619867in}}%
\pgfpathcurveto{\pgfqpoint{6.619873in}{3.630917in}}{\pgfqpoint{6.615482in}{3.641516in}}{\pgfqpoint{6.607669in}{3.649330in}}%
\pgfpathcurveto{\pgfqpoint{6.599855in}{3.657144in}}{\pgfqpoint{6.589256in}{3.661534in}}{\pgfqpoint{6.578206in}{3.661534in}}%
\pgfpathcurveto{\pgfqpoint{6.567156in}{3.661534in}}{\pgfqpoint{6.556557in}{3.657144in}}{\pgfqpoint{6.548743in}{3.649330in}}%
\pgfpathcurveto{\pgfqpoint{6.540930in}{3.641516in}}{\pgfqpoint{6.536539in}{3.630917in}}{\pgfqpoint{6.536539in}{3.619867in}}%
\pgfpathcurveto{\pgfqpoint{6.536539in}{3.608817in}}{\pgfqpoint{6.540930in}{3.598218in}}{\pgfqpoint{6.548743in}{3.590404in}}%
\pgfpathcurveto{\pgfqpoint{6.556557in}{3.582591in}}{\pgfqpoint{6.567156in}{3.578201in}}{\pgfqpoint{6.578206in}{3.578201in}}%
\pgfpathclose%
\pgfusepath{stroke,fill}%
\end{pgfscope}%
\begin{pgfscope}%
\pgfpathrectangle{\pgfqpoint{0.570343in}{0.331635in}}{\pgfqpoint{9.300000in}{7.700000in}}%
\pgfusepath{clip}%
\pgfsetbuttcap%
\pgfsetroundjoin%
\definecolor{currentfill}{rgb}{1.000000,0.705882,0.509804}%
\pgfsetfillcolor{currentfill}%
\pgfsetlinewidth{0.481800pt}%
\definecolor{currentstroke}{rgb}{1.000000,1.000000,1.000000}%
\pgfsetstrokecolor{currentstroke}%
\pgfsetdash{}{0pt}%
\pgfpathmoveto{\pgfqpoint{6.570870in}{4.649512in}}%
\pgfpathcurveto{\pgfqpoint{6.581920in}{4.649512in}}{\pgfqpoint{6.592519in}{4.653902in}}{\pgfqpoint{6.600333in}{4.661716in}}%
\pgfpathcurveto{\pgfqpoint{6.608146in}{4.669529in}}{\pgfqpoint{6.612536in}{4.680128in}}{\pgfqpoint{6.612536in}{4.691179in}}%
\pgfpathcurveto{\pgfqpoint{6.612536in}{4.702229in}}{\pgfqpoint{6.608146in}{4.712828in}}{\pgfqpoint{6.600333in}{4.720641in}}%
\pgfpathcurveto{\pgfqpoint{6.592519in}{4.728455in}}{\pgfqpoint{6.581920in}{4.732845in}}{\pgfqpoint{6.570870in}{4.732845in}}%
\pgfpathcurveto{\pgfqpoint{6.559820in}{4.732845in}}{\pgfqpoint{6.549221in}{4.728455in}}{\pgfqpoint{6.541407in}{4.720641in}}%
\pgfpathcurveto{\pgfqpoint{6.533593in}{4.712828in}}{\pgfqpoint{6.529203in}{4.702229in}}{\pgfqpoint{6.529203in}{4.691179in}}%
\pgfpathcurveto{\pgfqpoint{6.529203in}{4.680128in}}{\pgfqpoint{6.533593in}{4.669529in}}{\pgfqpoint{6.541407in}{4.661716in}}%
\pgfpathcurveto{\pgfqpoint{6.549221in}{4.653902in}}{\pgfqpoint{6.559820in}{4.649512in}}{\pgfqpoint{6.570870in}{4.649512in}}%
\pgfpathclose%
\pgfusepath{stroke,fill}%
\end{pgfscope}%
\begin{pgfscope}%
\pgfpathrectangle{\pgfqpoint{0.570343in}{0.331635in}}{\pgfqpoint{9.300000in}{7.700000in}}%
\pgfusepath{clip}%
\pgfsetbuttcap%
\pgfsetroundjoin%
\definecolor{currentfill}{rgb}{1.000000,0.705882,0.509804}%
\pgfsetfillcolor{currentfill}%
\pgfsetlinewidth{0.481800pt}%
\definecolor{currentstroke}{rgb}{1.000000,1.000000,1.000000}%
\pgfsetstrokecolor{currentstroke}%
\pgfsetdash{}{0pt}%
\pgfpathmoveto{\pgfqpoint{6.605909in}{5.160112in}}%
\pgfpathcurveto{\pgfqpoint{6.616959in}{5.160112in}}{\pgfqpoint{6.627558in}{5.164502in}}{\pgfqpoint{6.635372in}{5.172316in}}%
\pgfpathcurveto{\pgfqpoint{6.643186in}{5.180129in}}{\pgfqpoint{6.647576in}{5.190728in}}{\pgfqpoint{6.647576in}{5.201778in}}%
\pgfpathcurveto{\pgfqpoint{6.647576in}{5.212828in}}{\pgfqpoint{6.643186in}{5.223427in}}{\pgfqpoint{6.635372in}{5.231241in}}%
\pgfpathcurveto{\pgfqpoint{6.627558in}{5.239055in}}{\pgfqpoint{6.616959in}{5.243445in}}{\pgfqpoint{6.605909in}{5.243445in}}%
\pgfpathcurveto{\pgfqpoint{6.594859in}{5.243445in}}{\pgfqpoint{6.584260in}{5.239055in}}{\pgfqpoint{6.576446in}{5.231241in}}%
\pgfpathcurveto{\pgfqpoint{6.568633in}{5.223427in}}{\pgfqpoint{6.564242in}{5.212828in}}{\pgfqpoint{6.564242in}{5.201778in}}%
\pgfpathcurveto{\pgfqpoint{6.564242in}{5.190728in}}{\pgfqpoint{6.568633in}{5.180129in}}{\pgfqpoint{6.576446in}{5.172316in}}%
\pgfpathcurveto{\pgfqpoint{6.584260in}{5.164502in}}{\pgfqpoint{6.594859in}{5.160112in}}{\pgfqpoint{6.605909in}{5.160112in}}%
\pgfpathclose%
\pgfusepath{stroke,fill}%
\end{pgfscope}%
\begin{pgfscope}%
\pgfpathrectangle{\pgfqpoint{0.570343in}{0.331635in}}{\pgfqpoint{9.300000in}{7.700000in}}%
\pgfusepath{clip}%
\pgfsetbuttcap%
\pgfsetroundjoin%
\definecolor{currentfill}{rgb}{1.000000,0.705882,0.509804}%
\pgfsetfillcolor{currentfill}%
\pgfsetlinewidth{0.481800pt}%
\definecolor{currentstroke}{rgb}{1.000000,1.000000,1.000000}%
\pgfsetstrokecolor{currentstroke}%
\pgfsetdash{}{0pt}%
\pgfpathmoveto{\pgfqpoint{6.543745in}{2.381706in}}%
\pgfpathcurveto{\pgfqpoint{6.554795in}{2.381706in}}{\pgfqpoint{6.565394in}{2.386096in}}{\pgfqpoint{6.573208in}{2.393910in}}%
\pgfpathcurveto{\pgfqpoint{6.581022in}{2.401723in}}{\pgfqpoint{6.585412in}{2.412322in}}{\pgfqpoint{6.585412in}{2.423372in}}%
\pgfpathcurveto{\pgfqpoint{6.585412in}{2.434422in}}{\pgfqpoint{6.581022in}{2.445021in}}{\pgfqpoint{6.573208in}{2.452835in}}%
\pgfpathcurveto{\pgfqpoint{6.565394in}{2.460649in}}{\pgfqpoint{6.554795in}{2.465039in}}{\pgfqpoint{6.543745in}{2.465039in}}%
\pgfpathcurveto{\pgfqpoint{6.532695in}{2.465039in}}{\pgfqpoint{6.522096in}{2.460649in}}{\pgfqpoint{6.514282in}{2.452835in}}%
\pgfpathcurveto{\pgfqpoint{6.506469in}{2.445021in}}{\pgfqpoint{6.502078in}{2.434422in}}{\pgfqpoint{6.502078in}{2.423372in}}%
\pgfpathcurveto{\pgfqpoint{6.502078in}{2.412322in}}{\pgfqpoint{6.506469in}{2.401723in}}{\pgfqpoint{6.514282in}{2.393910in}}%
\pgfpathcurveto{\pgfqpoint{6.522096in}{2.386096in}}{\pgfqpoint{6.532695in}{2.381706in}}{\pgfqpoint{6.543745in}{2.381706in}}%
\pgfpathclose%
\pgfusepath{stroke,fill}%
\end{pgfscope}%
\begin{pgfscope}%
\pgfpathrectangle{\pgfqpoint{0.570343in}{0.331635in}}{\pgfqpoint{9.300000in}{7.700000in}}%
\pgfusepath{clip}%
\pgfsetbuttcap%
\pgfsetroundjoin%
\definecolor{currentfill}{rgb}{1.000000,0.705882,0.509804}%
\pgfsetfillcolor{currentfill}%
\pgfsetlinewidth{0.481800pt}%
\definecolor{currentstroke}{rgb}{1.000000,1.000000,1.000000}%
\pgfsetstrokecolor{currentstroke}%
\pgfsetdash{}{0pt}%
\pgfpathmoveto{\pgfqpoint{6.129683in}{0.639968in}}%
\pgfpathcurveto{\pgfqpoint{6.140734in}{0.639968in}}{\pgfqpoint{6.151333in}{0.644359in}}{\pgfqpoint{6.159146in}{0.652172in}}%
\pgfpathcurveto{\pgfqpoint{6.166960in}{0.659986in}}{\pgfqpoint{6.171350in}{0.670585in}}{\pgfqpoint{6.171350in}{0.681635in}}%
\pgfpathcurveto{\pgfqpoint{6.171350in}{0.692685in}}{\pgfqpoint{6.166960in}{0.703284in}}{\pgfqpoint{6.159146in}{0.711098in}}%
\pgfpathcurveto{\pgfqpoint{6.151333in}{0.718911in}}{\pgfqpoint{6.140734in}{0.723302in}}{\pgfqpoint{6.129683in}{0.723302in}}%
\pgfpathcurveto{\pgfqpoint{6.118633in}{0.723302in}}{\pgfqpoint{6.108034in}{0.718911in}}{\pgfqpoint{6.100221in}{0.711098in}}%
\pgfpathcurveto{\pgfqpoint{6.092407in}{0.703284in}}{\pgfqpoint{6.088017in}{0.692685in}}{\pgfqpoint{6.088017in}{0.681635in}}%
\pgfpathcurveto{\pgfqpoint{6.088017in}{0.670585in}}{\pgfqpoint{6.092407in}{0.659986in}}{\pgfqpoint{6.100221in}{0.652172in}}%
\pgfpathcurveto{\pgfqpoint{6.108034in}{0.644359in}}{\pgfqpoint{6.118633in}{0.639968in}}{\pgfqpoint{6.129683in}{0.639968in}}%
\pgfpathclose%
\pgfusepath{stroke,fill}%
\end{pgfscope}%
\begin{pgfscope}%
\pgfpathrectangle{\pgfqpoint{0.570343in}{0.331635in}}{\pgfqpoint{9.300000in}{7.700000in}}%
\pgfusepath{clip}%
\pgfsetbuttcap%
\pgfsetroundjoin%
\definecolor{currentfill}{rgb}{1.000000,0.705882,0.509804}%
\pgfsetfillcolor{currentfill}%
\pgfsetlinewidth{0.481800pt}%
\definecolor{currentstroke}{rgb}{1.000000,1.000000,1.000000}%
\pgfsetstrokecolor{currentstroke}%
\pgfsetdash{}{0pt}%
\pgfpathmoveto{\pgfqpoint{6.658623in}{5.192603in}}%
\pgfpathcurveto{\pgfqpoint{6.669673in}{5.192603in}}{\pgfqpoint{6.680272in}{5.196993in}}{\pgfqpoint{6.688086in}{5.204807in}}%
\pgfpathcurveto{\pgfqpoint{6.695900in}{5.212621in}}{\pgfqpoint{6.700290in}{5.223220in}}{\pgfqpoint{6.700290in}{5.234270in}}%
\pgfpathcurveto{\pgfqpoint{6.700290in}{5.245320in}}{\pgfqpoint{6.695900in}{5.255919in}}{\pgfqpoint{6.688086in}{5.263732in}}%
\pgfpathcurveto{\pgfqpoint{6.680272in}{5.271546in}}{\pgfqpoint{6.669673in}{5.275936in}}{\pgfqpoint{6.658623in}{5.275936in}}%
\pgfpathcurveto{\pgfqpoint{6.647573in}{5.275936in}}{\pgfqpoint{6.636974in}{5.271546in}}{\pgfqpoint{6.629161in}{5.263732in}}%
\pgfpathcurveto{\pgfqpoint{6.621347in}{5.255919in}}{\pgfqpoint{6.616957in}{5.245320in}}{\pgfqpoint{6.616957in}{5.234270in}}%
\pgfpathcurveto{\pgfqpoint{6.616957in}{5.223220in}}{\pgfqpoint{6.621347in}{5.212621in}}{\pgfqpoint{6.629161in}{5.204807in}}%
\pgfpathcurveto{\pgfqpoint{6.636974in}{5.196993in}}{\pgfqpoint{6.647573in}{5.192603in}}{\pgfqpoint{6.658623in}{5.192603in}}%
\pgfpathclose%
\pgfusepath{stroke,fill}%
\end{pgfscope}%
\begin{pgfscope}%
\pgfpathrectangle{\pgfqpoint{0.570343in}{0.331635in}}{\pgfqpoint{9.300000in}{7.700000in}}%
\pgfusepath{clip}%
\pgfsetbuttcap%
\pgfsetroundjoin%
\definecolor{currentfill}{rgb}{1.000000,0.705882,0.509804}%
\pgfsetfillcolor{currentfill}%
\pgfsetlinewidth{0.481800pt}%
\definecolor{currentstroke}{rgb}{1.000000,1.000000,1.000000}%
\pgfsetstrokecolor{currentstroke}%
\pgfsetdash{}{0pt}%
\pgfpathmoveto{\pgfqpoint{6.207206in}{5.509257in}}%
\pgfpathcurveto{\pgfqpoint{6.218257in}{5.509257in}}{\pgfqpoint{6.228856in}{5.513648in}}{\pgfqpoint{6.236669in}{5.521461in}}%
\pgfpathcurveto{\pgfqpoint{6.244483in}{5.529275in}}{\pgfqpoint{6.248873in}{5.539874in}}{\pgfqpoint{6.248873in}{5.550924in}}%
\pgfpathcurveto{\pgfqpoint{6.248873in}{5.561974in}}{\pgfqpoint{6.244483in}{5.572573in}}{\pgfqpoint{6.236669in}{5.580387in}}%
\pgfpathcurveto{\pgfqpoint{6.228856in}{5.588200in}}{\pgfqpoint{6.218257in}{5.592591in}}{\pgfqpoint{6.207206in}{5.592591in}}%
\pgfpathcurveto{\pgfqpoint{6.196156in}{5.592591in}}{\pgfqpoint{6.185557in}{5.588200in}}{\pgfqpoint{6.177744in}{5.580387in}}%
\pgfpathcurveto{\pgfqpoint{6.169930in}{5.572573in}}{\pgfqpoint{6.165540in}{5.561974in}}{\pgfqpoint{6.165540in}{5.550924in}}%
\pgfpathcurveto{\pgfqpoint{6.165540in}{5.539874in}}{\pgfqpoint{6.169930in}{5.529275in}}{\pgfqpoint{6.177744in}{5.521461in}}%
\pgfpathcurveto{\pgfqpoint{6.185557in}{5.513648in}}{\pgfqpoint{6.196156in}{5.509257in}}{\pgfqpoint{6.207206in}{5.509257in}}%
\pgfpathclose%
\pgfusepath{stroke,fill}%
\end{pgfscope}%
\begin{pgfscope}%
\pgfpathrectangle{\pgfqpoint{0.570343in}{0.331635in}}{\pgfqpoint{9.300000in}{7.700000in}}%
\pgfusepath{clip}%
\pgfsetbuttcap%
\pgfsetroundjoin%
\definecolor{currentfill}{rgb}{1.000000,0.705882,0.509804}%
\pgfsetfillcolor{currentfill}%
\pgfsetlinewidth{0.481800pt}%
\definecolor{currentstroke}{rgb}{1.000000,1.000000,1.000000}%
\pgfsetstrokecolor{currentstroke}%
\pgfsetdash{}{0pt}%
\pgfpathmoveto{\pgfqpoint{6.593399in}{1.957875in}}%
\pgfpathcurveto{\pgfqpoint{6.604449in}{1.957875in}}{\pgfqpoint{6.615048in}{1.962266in}}{\pgfqpoint{6.622862in}{1.970079in}}%
\pgfpathcurveto{\pgfqpoint{6.630676in}{1.977893in}}{\pgfqpoint{6.635066in}{1.988492in}}{\pgfqpoint{6.635066in}{1.999542in}}%
\pgfpathcurveto{\pgfqpoint{6.635066in}{2.010592in}}{\pgfqpoint{6.630676in}{2.021191in}}{\pgfqpoint{6.622862in}{2.029005in}}%
\pgfpathcurveto{\pgfqpoint{6.615048in}{2.036818in}}{\pgfqpoint{6.604449in}{2.041209in}}{\pgfqpoint{6.593399in}{2.041209in}}%
\pgfpathcurveto{\pgfqpoint{6.582349in}{2.041209in}}{\pgfqpoint{6.571750in}{2.036818in}}{\pgfqpoint{6.563937in}{2.029005in}}%
\pgfpathcurveto{\pgfqpoint{6.556123in}{2.021191in}}{\pgfqpoint{6.551733in}{2.010592in}}{\pgfqpoint{6.551733in}{1.999542in}}%
\pgfpathcurveto{\pgfqpoint{6.551733in}{1.988492in}}{\pgfqpoint{6.556123in}{1.977893in}}{\pgfqpoint{6.563937in}{1.970079in}}%
\pgfpathcurveto{\pgfqpoint{6.571750in}{1.962266in}}{\pgfqpoint{6.582349in}{1.957875in}}{\pgfqpoint{6.593399in}{1.957875in}}%
\pgfpathclose%
\pgfusepath{stroke,fill}%
\end{pgfscope}%
\begin{pgfscope}%
\pgfpathrectangle{\pgfqpoint{0.570343in}{0.331635in}}{\pgfqpoint{9.300000in}{7.700000in}}%
\pgfusepath{clip}%
\pgfsetbuttcap%
\pgfsetroundjoin%
\definecolor{currentfill}{rgb}{1.000000,0.705882,0.509804}%
\pgfsetfillcolor{currentfill}%
\pgfsetlinewidth{0.481800pt}%
\definecolor{currentstroke}{rgb}{1.000000,1.000000,1.000000}%
\pgfsetstrokecolor{currentstroke}%
\pgfsetdash{}{0pt}%
\pgfpathmoveto{\pgfqpoint{6.036485in}{5.729768in}}%
\pgfpathcurveto{\pgfqpoint{6.047535in}{5.729768in}}{\pgfqpoint{6.058134in}{5.734159in}}{\pgfqpoint{6.065947in}{5.741972in}}%
\pgfpathcurveto{\pgfqpoint{6.073761in}{5.749786in}}{\pgfqpoint{6.078151in}{5.760385in}}{\pgfqpoint{6.078151in}{5.771435in}}%
\pgfpathcurveto{\pgfqpoint{6.078151in}{5.782485in}}{\pgfqpoint{6.073761in}{5.793084in}}{\pgfqpoint{6.065947in}{5.800898in}}%
\pgfpathcurveto{\pgfqpoint{6.058134in}{5.808711in}}{\pgfqpoint{6.047535in}{5.813102in}}{\pgfqpoint{6.036485in}{5.813102in}}%
\pgfpathcurveto{\pgfqpoint{6.025435in}{5.813102in}}{\pgfqpoint{6.014836in}{5.808711in}}{\pgfqpoint{6.007022in}{5.800898in}}%
\pgfpathcurveto{\pgfqpoint{5.999208in}{5.793084in}}{\pgfqpoint{5.994818in}{5.782485in}}{\pgfqpoint{5.994818in}{5.771435in}}%
\pgfpathcurveto{\pgfqpoint{5.994818in}{5.760385in}}{\pgfqpoint{5.999208in}{5.749786in}}{\pgfqpoint{6.007022in}{5.741972in}}%
\pgfpathcurveto{\pgfqpoint{6.014836in}{5.734159in}}{\pgfqpoint{6.025435in}{5.729768in}}{\pgfqpoint{6.036485in}{5.729768in}}%
\pgfpathclose%
\pgfusepath{stroke,fill}%
\end{pgfscope}%
\begin{pgfscope}%
\pgfpathrectangle{\pgfqpoint{0.570343in}{0.331635in}}{\pgfqpoint{9.300000in}{7.700000in}}%
\pgfusepath{clip}%
\pgfsetbuttcap%
\pgfsetroundjoin%
\definecolor{currentfill}{rgb}{1.000000,0.705882,0.509804}%
\pgfsetfillcolor{currentfill}%
\pgfsetlinewidth{0.481800pt}%
\definecolor{currentstroke}{rgb}{1.000000,1.000000,1.000000}%
\pgfsetstrokecolor{currentstroke}%
\pgfsetdash{}{0pt}%
\pgfpathmoveto{\pgfqpoint{6.492770in}{1.920532in}}%
\pgfpathcurveto{\pgfqpoint{6.503820in}{1.920532in}}{\pgfqpoint{6.514419in}{1.924923in}}{\pgfqpoint{6.522233in}{1.932736in}}%
\pgfpathcurveto{\pgfqpoint{6.530046in}{1.940550in}}{\pgfqpoint{6.534437in}{1.951149in}}{\pgfqpoint{6.534437in}{1.962199in}}%
\pgfpathcurveto{\pgfqpoint{6.534437in}{1.973249in}}{\pgfqpoint{6.530046in}{1.983848in}}{\pgfqpoint{6.522233in}{1.991662in}}%
\pgfpathcurveto{\pgfqpoint{6.514419in}{1.999475in}}{\pgfqpoint{6.503820in}{2.003866in}}{\pgfqpoint{6.492770in}{2.003866in}}%
\pgfpathcurveto{\pgfqpoint{6.481720in}{2.003866in}}{\pgfqpoint{6.471121in}{1.999475in}}{\pgfqpoint{6.463307in}{1.991662in}}%
\pgfpathcurveto{\pgfqpoint{6.455494in}{1.983848in}}{\pgfqpoint{6.451103in}{1.973249in}}{\pgfqpoint{6.451103in}{1.962199in}}%
\pgfpathcurveto{\pgfqpoint{6.451103in}{1.951149in}}{\pgfqpoint{6.455494in}{1.940550in}}{\pgfqpoint{6.463307in}{1.932736in}}%
\pgfpathcurveto{\pgfqpoint{6.471121in}{1.924923in}}{\pgfqpoint{6.481720in}{1.920532in}}{\pgfqpoint{6.492770in}{1.920532in}}%
\pgfpathclose%
\pgfusepath{stroke,fill}%
\end{pgfscope}%
\begin{pgfscope}%
\pgfpathrectangle{\pgfqpoint{0.570343in}{0.331635in}}{\pgfqpoint{9.300000in}{7.700000in}}%
\pgfusepath{clip}%
\pgfsetbuttcap%
\pgfsetroundjoin%
\definecolor{currentfill}{rgb}{1.000000,0.705882,0.509804}%
\pgfsetfillcolor{currentfill}%
\pgfsetlinewidth{0.481800pt}%
\definecolor{currentstroke}{rgb}{1.000000,1.000000,1.000000}%
\pgfsetstrokecolor{currentstroke}%
\pgfsetdash{}{0pt}%
\pgfpathmoveto{\pgfqpoint{5.959958in}{6.259152in}}%
\pgfpathcurveto{\pgfqpoint{5.971008in}{6.259152in}}{\pgfqpoint{5.981607in}{6.263542in}}{\pgfqpoint{5.989420in}{6.271356in}}%
\pgfpathcurveto{\pgfqpoint{5.997234in}{6.279169in}}{\pgfqpoint{6.001624in}{6.289768in}}{\pgfqpoint{6.001624in}{6.300818in}}%
\pgfpathcurveto{\pgfqpoint{6.001624in}{6.311868in}}{\pgfqpoint{5.997234in}{6.322468in}}{\pgfqpoint{5.989420in}{6.330281in}}%
\pgfpathcurveto{\pgfqpoint{5.981607in}{6.338095in}}{\pgfqpoint{5.971008in}{6.342485in}}{\pgfqpoint{5.959958in}{6.342485in}}%
\pgfpathcurveto{\pgfqpoint{5.948908in}{6.342485in}}{\pgfqpoint{5.938309in}{6.338095in}}{\pgfqpoint{5.930495in}{6.330281in}}%
\pgfpathcurveto{\pgfqpoint{5.922681in}{6.322468in}}{\pgfqpoint{5.918291in}{6.311868in}}{\pgfqpoint{5.918291in}{6.300818in}}%
\pgfpathcurveto{\pgfqpoint{5.918291in}{6.289768in}}{\pgfqpoint{5.922681in}{6.279169in}}{\pgfqpoint{5.930495in}{6.271356in}}%
\pgfpathcurveto{\pgfqpoint{5.938309in}{6.263542in}}{\pgfqpoint{5.948908in}{6.259152in}}{\pgfqpoint{5.959958in}{6.259152in}}%
\pgfpathclose%
\pgfusepath{stroke,fill}%
\end{pgfscope}%
\begin{pgfscope}%
\pgfpathrectangle{\pgfqpoint{0.570343in}{0.331635in}}{\pgfqpoint{9.300000in}{7.700000in}}%
\pgfusepath{clip}%
\pgfsetbuttcap%
\pgfsetroundjoin%
\definecolor{currentfill}{rgb}{1.000000,0.705882,0.509804}%
\pgfsetfillcolor{currentfill}%
\pgfsetlinewidth{0.481800pt}%
\definecolor{currentstroke}{rgb}{1.000000,1.000000,1.000000}%
\pgfsetstrokecolor{currentstroke}%
\pgfsetdash{}{0pt}%
\pgfpathmoveto{\pgfqpoint{6.546796in}{3.063605in}}%
\pgfpathcurveto{\pgfqpoint{6.557846in}{3.063605in}}{\pgfqpoint{6.568445in}{3.067995in}}{\pgfqpoint{6.576259in}{3.075809in}}%
\pgfpathcurveto{\pgfqpoint{6.584072in}{3.083623in}}{\pgfqpoint{6.588463in}{3.094222in}}{\pgfqpoint{6.588463in}{3.105272in}}%
\pgfpathcurveto{\pgfqpoint{6.588463in}{3.116322in}}{\pgfqpoint{6.584072in}{3.126921in}}{\pgfqpoint{6.576259in}{3.134735in}}%
\pgfpathcurveto{\pgfqpoint{6.568445in}{3.142548in}}{\pgfqpoint{6.557846in}{3.146938in}}{\pgfqpoint{6.546796in}{3.146938in}}%
\pgfpathcurveto{\pgfqpoint{6.535746in}{3.146938in}}{\pgfqpoint{6.525147in}{3.142548in}}{\pgfqpoint{6.517333in}{3.134735in}}%
\pgfpathcurveto{\pgfqpoint{6.509520in}{3.126921in}}{\pgfqpoint{6.505129in}{3.116322in}}{\pgfqpoint{6.505129in}{3.105272in}}%
\pgfpathcurveto{\pgfqpoint{6.505129in}{3.094222in}}{\pgfqpoint{6.509520in}{3.083623in}}{\pgfqpoint{6.517333in}{3.075809in}}%
\pgfpathcurveto{\pgfqpoint{6.525147in}{3.067995in}}{\pgfqpoint{6.535746in}{3.063605in}}{\pgfqpoint{6.546796in}{3.063605in}}%
\pgfpathclose%
\pgfusepath{stroke,fill}%
\end{pgfscope}%
\begin{pgfscope}%
\pgfpathrectangle{\pgfqpoint{0.570343in}{0.331635in}}{\pgfqpoint{9.300000in}{7.700000in}}%
\pgfusepath{clip}%
\pgfsetbuttcap%
\pgfsetroundjoin%
\definecolor{currentfill}{rgb}{0.631373,0.788235,0.956863}%
\pgfsetfillcolor{currentfill}%
\pgfsetlinewidth{1.003750pt}%
\definecolor{currentstroke}{rgb}{0.631373,0.788235,0.956863}%
\pgfsetstrokecolor{currentstroke}%
\pgfsetdash{}{0pt}%
\pgfsys@defobject{currentmarker}{\pgfqpoint{-0.041667in}{-0.041667in}}{\pgfqpoint{0.041667in}{0.041667in}}{%
\pgfpathmoveto{\pgfqpoint{0.000000in}{-0.041667in}}%
\pgfpathcurveto{\pgfqpoint{0.011050in}{-0.041667in}}{\pgfqpoint{0.021649in}{-0.037276in}}{\pgfqpoint{0.029463in}{-0.029463in}}%
\pgfpathcurveto{\pgfqpoint{0.037276in}{-0.021649in}}{\pgfqpoint{0.041667in}{-0.011050in}}{\pgfqpoint{0.041667in}{0.000000in}}%
\pgfpathcurveto{\pgfqpoint{0.041667in}{0.011050in}}{\pgfqpoint{0.037276in}{0.021649in}}{\pgfqpoint{0.029463in}{0.029463in}}%
\pgfpathcurveto{\pgfqpoint{0.021649in}{0.037276in}}{\pgfqpoint{0.011050in}{0.041667in}}{\pgfqpoint{0.000000in}{0.041667in}}%
\pgfpathcurveto{\pgfqpoint{-0.011050in}{0.041667in}}{\pgfqpoint{-0.021649in}{0.037276in}}{\pgfqpoint{-0.029463in}{0.029463in}}%
\pgfpathcurveto{\pgfqpoint{-0.037276in}{0.021649in}}{\pgfqpoint{-0.041667in}{0.011050in}}{\pgfqpoint{-0.041667in}{0.000000in}}%
\pgfpathcurveto{\pgfqpoint{-0.041667in}{-0.011050in}}{\pgfqpoint{-0.037276in}{-0.021649in}}{\pgfqpoint{-0.029463in}{-0.029463in}}%
\pgfpathcurveto{\pgfqpoint{-0.021649in}{-0.037276in}}{\pgfqpoint{-0.011050in}{-0.041667in}}{\pgfqpoint{0.000000in}{-0.041667in}}%
\pgfpathclose%
\pgfusepath{stroke,fill}%
}%
\end{pgfscope}%
\begin{pgfscope}%
\pgfpathrectangle{\pgfqpoint{0.570343in}{0.331635in}}{\pgfqpoint{9.300000in}{7.700000in}}%
\pgfusepath{clip}%
\pgfsetbuttcap%
\pgfsetroundjoin%
\definecolor{currentfill}{rgb}{1.000000,0.705882,0.509804}%
\pgfsetfillcolor{currentfill}%
\pgfsetlinewidth{1.003750pt}%
\definecolor{currentstroke}{rgb}{1.000000,0.705882,0.509804}%
\pgfsetstrokecolor{currentstroke}%
\pgfsetdash{}{0pt}%
\pgfsys@defobject{currentmarker}{\pgfqpoint{-0.041667in}{-0.041667in}}{\pgfqpoint{0.041667in}{0.041667in}}{%
\pgfpathmoveto{\pgfqpoint{0.000000in}{-0.041667in}}%
\pgfpathcurveto{\pgfqpoint{0.011050in}{-0.041667in}}{\pgfqpoint{0.021649in}{-0.037276in}}{\pgfqpoint{0.029463in}{-0.029463in}}%
\pgfpathcurveto{\pgfqpoint{0.037276in}{-0.021649in}}{\pgfqpoint{0.041667in}{-0.011050in}}{\pgfqpoint{0.041667in}{0.000000in}}%
\pgfpathcurveto{\pgfqpoint{0.041667in}{0.011050in}}{\pgfqpoint{0.037276in}{0.021649in}}{\pgfqpoint{0.029463in}{0.029463in}}%
\pgfpathcurveto{\pgfqpoint{0.021649in}{0.037276in}}{\pgfqpoint{0.011050in}{0.041667in}}{\pgfqpoint{0.000000in}{0.041667in}}%
\pgfpathcurveto{\pgfqpoint{-0.011050in}{0.041667in}}{\pgfqpoint{-0.021649in}{0.037276in}}{\pgfqpoint{-0.029463in}{0.029463in}}%
\pgfpathcurveto{\pgfqpoint{-0.037276in}{0.021649in}}{\pgfqpoint{-0.041667in}{0.011050in}}{\pgfqpoint{-0.041667in}{0.000000in}}%
\pgfpathcurveto{\pgfqpoint{-0.041667in}{-0.011050in}}{\pgfqpoint{-0.037276in}{-0.021649in}}{\pgfqpoint{-0.029463in}{-0.029463in}}%
\pgfpathcurveto{\pgfqpoint{-0.021649in}{-0.037276in}}{\pgfqpoint{-0.011050in}{-0.041667in}}{\pgfqpoint{0.000000in}{-0.041667in}}%
\pgfpathclose%
\pgfusepath{stroke,fill}%
}%
\end{pgfscope}%
\begin{pgfscope}%
\pgfsetbuttcap%
\pgfsetroundjoin%
\definecolor{currentfill}{rgb}{0.000000,0.000000,0.000000}%
\pgfsetfillcolor{currentfill}%
\pgfsetlinewidth{0.803000pt}%
\definecolor{currentstroke}{rgb}{0.000000,0.000000,0.000000}%
\pgfsetstrokecolor{currentstroke}%
\pgfsetdash{}{0pt}%
\pgfsys@defobject{currentmarker}{\pgfqpoint{0.000000in}{-0.048611in}}{\pgfqpoint{0.000000in}{0.000000in}}{%
\pgfpathmoveto{\pgfqpoint{0.000000in}{0.000000in}}%
\pgfpathlineto{\pgfqpoint{0.000000in}{-0.048611in}}%
\pgfusepath{stroke,fill}%
}%
\begin{pgfscope}%
\pgfsys@transformshift{0.832690in}{0.331635in}%
\pgfsys@useobject{currentmarker}{}%
\end{pgfscope}%
\end{pgfscope}%
\begin{pgfscope}%
\definecolor{textcolor}{rgb}{0.000000,0.000000,0.000000}%
\pgfsetstrokecolor{textcolor}%
\pgfsetfillcolor{textcolor}%
\pgftext[x=0.832690in,y=0.234413in,,top]{\color{textcolor}\sffamily\fontsize{10.000000}{12.000000}\selectfont \ensuremath{-}2500}%
\end{pgfscope}%
\begin{pgfscope}%
\pgfsetbuttcap%
\pgfsetroundjoin%
\definecolor{currentfill}{rgb}{0.000000,0.000000,0.000000}%
\pgfsetfillcolor{currentfill}%
\pgfsetlinewidth{0.803000pt}%
\definecolor{currentstroke}{rgb}{0.000000,0.000000,0.000000}%
\pgfsetstrokecolor{currentstroke}%
\pgfsetdash{}{0pt}%
\pgfsys@defobject{currentmarker}{\pgfqpoint{0.000000in}{-0.048611in}}{\pgfqpoint{0.000000in}{0.000000in}}{%
\pgfpathmoveto{\pgfqpoint{0.000000in}{0.000000in}}%
\pgfpathlineto{\pgfqpoint{0.000000in}{-0.048611in}}%
\pgfusepath{stroke,fill}%
}%
\begin{pgfscope}%
\pgfsys@transformshift{1.906600in}{0.331635in}%
\pgfsys@useobject{currentmarker}{}%
\end{pgfscope}%
\end{pgfscope}%
\begin{pgfscope}%
\definecolor{textcolor}{rgb}{0.000000,0.000000,0.000000}%
\pgfsetstrokecolor{textcolor}%
\pgfsetfillcolor{textcolor}%
\pgftext[x=1.906600in,y=0.234413in,,top]{\color{textcolor}\sffamily\fontsize{10.000000}{12.000000}\selectfont \ensuremath{-}2000}%
\end{pgfscope}%
\begin{pgfscope}%
\pgfsetbuttcap%
\pgfsetroundjoin%
\definecolor{currentfill}{rgb}{0.000000,0.000000,0.000000}%
\pgfsetfillcolor{currentfill}%
\pgfsetlinewidth{0.803000pt}%
\definecolor{currentstroke}{rgb}{0.000000,0.000000,0.000000}%
\pgfsetstrokecolor{currentstroke}%
\pgfsetdash{}{0pt}%
\pgfsys@defobject{currentmarker}{\pgfqpoint{0.000000in}{-0.048611in}}{\pgfqpoint{0.000000in}{0.000000in}}{%
\pgfpathmoveto{\pgfqpoint{0.000000in}{0.000000in}}%
\pgfpathlineto{\pgfqpoint{0.000000in}{-0.048611in}}%
\pgfusepath{stroke,fill}%
}%
\begin{pgfscope}%
\pgfsys@transformshift{2.980510in}{0.331635in}%
\pgfsys@useobject{currentmarker}{}%
\end{pgfscope}%
\end{pgfscope}%
\begin{pgfscope}%
\definecolor{textcolor}{rgb}{0.000000,0.000000,0.000000}%
\pgfsetstrokecolor{textcolor}%
\pgfsetfillcolor{textcolor}%
\pgftext[x=2.980510in,y=0.234413in,,top]{\color{textcolor}\sffamily\fontsize{10.000000}{12.000000}\selectfont \ensuremath{-}1500}%
\end{pgfscope}%
\begin{pgfscope}%
\pgfsetbuttcap%
\pgfsetroundjoin%
\definecolor{currentfill}{rgb}{0.000000,0.000000,0.000000}%
\pgfsetfillcolor{currentfill}%
\pgfsetlinewidth{0.803000pt}%
\definecolor{currentstroke}{rgb}{0.000000,0.000000,0.000000}%
\pgfsetstrokecolor{currentstroke}%
\pgfsetdash{}{0pt}%
\pgfsys@defobject{currentmarker}{\pgfqpoint{0.000000in}{-0.048611in}}{\pgfqpoint{0.000000in}{0.000000in}}{%
\pgfpathmoveto{\pgfqpoint{0.000000in}{0.000000in}}%
\pgfpathlineto{\pgfqpoint{0.000000in}{-0.048611in}}%
\pgfusepath{stroke,fill}%
}%
\begin{pgfscope}%
\pgfsys@transformshift{4.054420in}{0.331635in}%
\pgfsys@useobject{currentmarker}{}%
\end{pgfscope}%
\end{pgfscope}%
\begin{pgfscope}%
\definecolor{textcolor}{rgb}{0.000000,0.000000,0.000000}%
\pgfsetstrokecolor{textcolor}%
\pgfsetfillcolor{textcolor}%
\pgftext[x=4.054420in,y=0.234413in,,top]{\color{textcolor}\sffamily\fontsize{10.000000}{12.000000}\selectfont \ensuremath{-}1000}%
\end{pgfscope}%
\begin{pgfscope}%
\pgfsetbuttcap%
\pgfsetroundjoin%
\definecolor{currentfill}{rgb}{0.000000,0.000000,0.000000}%
\pgfsetfillcolor{currentfill}%
\pgfsetlinewidth{0.803000pt}%
\definecolor{currentstroke}{rgb}{0.000000,0.000000,0.000000}%
\pgfsetstrokecolor{currentstroke}%
\pgfsetdash{}{0pt}%
\pgfsys@defobject{currentmarker}{\pgfqpoint{0.000000in}{-0.048611in}}{\pgfqpoint{0.000000in}{0.000000in}}{%
\pgfpathmoveto{\pgfqpoint{0.000000in}{0.000000in}}%
\pgfpathlineto{\pgfqpoint{0.000000in}{-0.048611in}}%
\pgfusepath{stroke,fill}%
}%
\begin{pgfscope}%
\pgfsys@transformshift{5.128330in}{0.331635in}%
\pgfsys@useobject{currentmarker}{}%
\end{pgfscope}%
\end{pgfscope}%
\begin{pgfscope}%
\definecolor{textcolor}{rgb}{0.000000,0.000000,0.000000}%
\pgfsetstrokecolor{textcolor}%
\pgfsetfillcolor{textcolor}%
\pgftext[x=5.128330in,y=0.234413in,,top]{\color{textcolor}\sffamily\fontsize{10.000000}{12.000000}\selectfont \ensuremath{-}500}%
\end{pgfscope}%
\begin{pgfscope}%
\pgfsetbuttcap%
\pgfsetroundjoin%
\definecolor{currentfill}{rgb}{0.000000,0.000000,0.000000}%
\pgfsetfillcolor{currentfill}%
\pgfsetlinewidth{0.803000pt}%
\definecolor{currentstroke}{rgb}{0.000000,0.000000,0.000000}%
\pgfsetstrokecolor{currentstroke}%
\pgfsetdash{}{0pt}%
\pgfsys@defobject{currentmarker}{\pgfqpoint{0.000000in}{-0.048611in}}{\pgfqpoint{0.000000in}{0.000000in}}{%
\pgfpathmoveto{\pgfqpoint{0.000000in}{0.000000in}}%
\pgfpathlineto{\pgfqpoint{0.000000in}{-0.048611in}}%
\pgfusepath{stroke,fill}%
}%
\begin{pgfscope}%
\pgfsys@transformshift{6.202239in}{0.331635in}%
\pgfsys@useobject{currentmarker}{}%
\end{pgfscope}%
\end{pgfscope}%
\begin{pgfscope}%
\definecolor{textcolor}{rgb}{0.000000,0.000000,0.000000}%
\pgfsetstrokecolor{textcolor}%
\pgfsetfillcolor{textcolor}%
\pgftext[x=6.202239in,y=0.234413in,,top]{\color{textcolor}\sffamily\fontsize{10.000000}{12.000000}\selectfont 0}%
\end{pgfscope}%
\begin{pgfscope}%
\pgfsetbuttcap%
\pgfsetroundjoin%
\definecolor{currentfill}{rgb}{0.000000,0.000000,0.000000}%
\pgfsetfillcolor{currentfill}%
\pgfsetlinewidth{0.803000pt}%
\definecolor{currentstroke}{rgb}{0.000000,0.000000,0.000000}%
\pgfsetstrokecolor{currentstroke}%
\pgfsetdash{}{0pt}%
\pgfsys@defobject{currentmarker}{\pgfqpoint{0.000000in}{-0.048611in}}{\pgfqpoint{0.000000in}{0.000000in}}{%
\pgfpathmoveto{\pgfqpoint{0.000000in}{0.000000in}}%
\pgfpathlineto{\pgfqpoint{0.000000in}{-0.048611in}}%
\pgfusepath{stroke,fill}%
}%
\begin{pgfscope}%
\pgfsys@transformshift{7.276149in}{0.331635in}%
\pgfsys@useobject{currentmarker}{}%
\end{pgfscope}%
\end{pgfscope}%
\begin{pgfscope}%
\definecolor{textcolor}{rgb}{0.000000,0.000000,0.000000}%
\pgfsetstrokecolor{textcolor}%
\pgfsetfillcolor{textcolor}%
\pgftext[x=7.276149in,y=0.234413in,,top]{\color{textcolor}\sffamily\fontsize{10.000000}{12.000000}\selectfont 500}%
\end{pgfscope}%
\begin{pgfscope}%
\pgfsetbuttcap%
\pgfsetroundjoin%
\definecolor{currentfill}{rgb}{0.000000,0.000000,0.000000}%
\pgfsetfillcolor{currentfill}%
\pgfsetlinewidth{0.803000pt}%
\definecolor{currentstroke}{rgb}{0.000000,0.000000,0.000000}%
\pgfsetstrokecolor{currentstroke}%
\pgfsetdash{}{0pt}%
\pgfsys@defobject{currentmarker}{\pgfqpoint{0.000000in}{-0.048611in}}{\pgfqpoint{0.000000in}{0.000000in}}{%
\pgfpathmoveto{\pgfqpoint{0.000000in}{0.000000in}}%
\pgfpathlineto{\pgfqpoint{0.000000in}{-0.048611in}}%
\pgfusepath{stroke,fill}%
}%
\begin{pgfscope}%
\pgfsys@transformshift{8.350059in}{0.331635in}%
\pgfsys@useobject{currentmarker}{}%
\end{pgfscope}%
\end{pgfscope}%
\begin{pgfscope}%
\definecolor{textcolor}{rgb}{0.000000,0.000000,0.000000}%
\pgfsetstrokecolor{textcolor}%
\pgfsetfillcolor{textcolor}%
\pgftext[x=8.350059in,y=0.234413in,,top]{\color{textcolor}\sffamily\fontsize{10.000000}{12.000000}\selectfont 1000}%
\end{pgfscope}%
\begin{pgfscope}%
\pgfsetbuttcap%
\pgfsetroundjoin%
\definecolor{currentfill}{rgb}{0.000000,0.000000,0.000000}%
\pgfsetfillcolor{currentfill}%
\pgfsetlinewidth{0.803000pt}%
\definecolor{currentstroke}{rgb}{0.000000,0.000000,0.000000}%
\pgfsetstrokecolor{currentstroke}%
\pgfsetdash{}{0pt}%
\pgfsys@defobject{currentmarker}{\pgfqpoint{0.000000in}{-0.048611in}}{\pgfqpoint{0.000000in}{0.000000in}}{%
\pgfpathmoveto{\pgfqpoint{0.000000in}{0.000000in}}%
\pgfpathlineto{\pgfqpoint{0.000000in}{-0.048611in}}%
\pgfusepath{stroke,fill}%
}%
\begin{pgfscope}%
\pgfsys@transformshift{9.423969in}{0.331635in}%
\pgfsys@useobject{currentmarker}{}%
\end{pgfscope}%
\end{pgfscope}%
\begin{pgfscope}%
\definecolor{textcolor}{rgb}{0.000000,0.000000,0.000000}%
\pgfsetstrokecolor{textcolor}%
\pgfsetfillcolor{textcolor}%
\pgftext[x=9.423969in,y=0.234413in,,top]{\color{textcolor}\sffamily\fontsize{10.000000}{12.000000}\selectfont 1500}%
\end{pgfscope}%
\begin{pgfscope}%
\pgfsetbuttcap%
\pgfsetroundjoin%
\definecolor{currentfill}{rgb}{0.000000,0.000000,0.000000}%
\pgfsetfillcolor{currentfill}%
\pgfsetlinewidth{0.803000pt}%
\definecolor{currentstroke}{rgb}{0.000000,0.000000,0.000000}%
\pgfsetstrokecolor{currentstroke}%
\pgfsetdash{}{0pt}%
\pgfsys@defobject{currentmarker}{\pgfqpoint{-0.048611in}{0.000000in}}{\pgfqpoint{-0.000000in}{0.000000in}}{%
\pgfpathmoveto{\pgfqpoint{-0.000000in}{0.000000in}}%
\pgfpathlineto{\pgfqpoint{-0.048611in}{0.000000in}}%
\pgfusepath{stroke,fill}%
}%
\begin{pgfscope}%
\pgfsys@transformshift{0.570343in}{0.922984in}%
\pgfsys@useobject{currentmarker}{}%
\end{pgfscope}%
\end{pgfscope}%
\begin{pgfscope}%
\definecolor{textcolor}{rgb}{0.000000,0.000000,0.000000}%
\pgfsetstrokecolor{textcolor}%
\pgfsetfillcolor{textcolor}%
\pgftext[x=0.100000in, y=0.870223in, left, base]{\color{textcolor}\sffamily\fontsize{10.000000}{12.000000}\selectfont \ensuremath{-}200}%
\end{pgfscope}%
\begin{pgfscope}%
\pgfsetbuttcap%
\pgfsetroundjoin%
\definecolor{currentfill}{rgb}{0.000000,0.000000,0.000000}%
\pgfsetfillcolor{currentfill}%
\pgfsetlinewidth{0.803000pt}%
\definecolor{currentstroke}{rgb}{0.000000,0.000000,0.000000}%
\pgfsetstrokecolor{currentstroke}%
\pgfsetdash{}{0pt}%
\pgfsys@defobject{currentmarker}{\pgfqpoint{-0.048611in}{0.000000in}}{\pgfqpoint{-0.000000in}{0.000000in}}{%
\pgfpathmoveto{\pgfqpoint{-0.000000in}{0.000000in}}%
\pgfpathlineto{\pgfqpoint{-0.048611in}{0.000000in}}%
\pgfusepath{stroke,fill}%
}%
\begin{pgfscope}%
\pgfsys@transformshift{0.570343in}{1.792912in}%
\pgfsys@useobject{currentmarker}{}%
\end{pgfscope}%
\end{pgfscope}%
\begin{pgfscope}%
\definecolor{textcolor}{rgb}{0.000000,0.000000,0.000000}%
\pgfsetstrokecolor{textcolor}%
\pgfsetfillcolor{textcolor}%
\pgftext[x=0.100000in, y=1.740151in, left, base]{\color{textcolor}\sffamily\fontsize{10.000000}{12.000000}\selectfont \ensuremath{-}150}%
\end{pgfscope}%
\begin{pgfscope}%
\pgfsetbuttcap%
\pgfsetroundjoin%
\definecolor{currentfill}{rgb}{0.000000,0.000000,0.000000}%
\pgfsetfillcolor{currentfill}%
\pgfsetlinewidth{0.803000pt}%
\definecolor{currentstroke}{rgb}{0.000000,0.000000,0.000000}%
\pgfsetstrokecolor{currentstroke}%
\pgfsetdash{}{0pt}%
\pgfsys@defobject{currentmarker}{\pgfqpoint{-0.048611in}{0.000000in}}{\pgfqpoint{-0.000000in}{0.000000in}}{%
\pgfpathmoveto{\pgfqpoint{-0.000000in}{0.000000in}}%
\pgfpathlineto{\pgfqpoint{-0.048611in}{0.000000in}}%
\pgfusepath{stroke,fill}%
}%
\begin{pgfscope}%
\pgfsys@transformshift{0.570343in}{2.662840in}%
\pgfsys@useobject{currentmarker}{}%
\end{pgfscope}%
\end{pgfscope}%
\begin{pgfscope}%
\definecolor{textcolor}{rgb}{0.000000,0.000000,0.000000}%
\pgfsetstrokecolor{textcolor}%
\pgfsetfillcolor{textcolor}%
\pgftext[x=0.100000in, y=2.610079in, left, base]{\color{textcolor}\sffamily\fontsize{10.000000}{12.000000}\selectfont \ensuremath{-}100}%
\end{pgfscope}%
\begin{pgfscope}%
\pgfsetbuttcap%
\pgfsetroundjoin%
\definecolor{currentfill}{rgb}{0.000000,0.000000,0.000000}%
\pgfsetfillcolor{currentfill}%
\pgfsetlinewidth{0.803000pt}%
\definecolor{currentstroke}{rgb}{0.000000,0.000000,0.000000}%
\pgfsetstrokecolor{currentstroke}%
\pgfsetdash{}{0pt}%
\pgfsys@defobject{currentmarker}{\pgfqpoint{-0.048611in}{0.000000in}}{\pgfqpoint{-0.000000in}{0.000000in}}{%
\pgfpathmoveto{\pgfqpoint{-0.000000in}{0.000000in}}%
\pgfpathlineto{\pgfqpoint{-0.048611in}{0.000000in}}%
\pgfusepath{stroke,fill}%
}%
\begin{pgfscope}%
\pgfsys@transformshift{0.570343in}{3.532768in}%
\pgfsys@useobject{currentmarker}{}%
\end{pgfscope}%
\end{pgfscope}%
\begin{pgfscope}%
\definecolor{textcolor}{rgb}{0.000000,0.000000,0.000000}%
\pgfsetstrokecolor{textcolor}%
\pgfsetfillcolor{textcolor}%
\pgftext[x=0.188365in, y=3.480006in, left, base]{\color{textcolor}\sffamily\fontsize{10.000000}{12.000000}\selectfont \ensuremath{-}50}%
\end{pgfscope}%
\begin{pgfscope}%
\pgfsetbuttcap%
\pgfsetroundjoin%
\definecolor{currentfill}{rgb}{0.000000,0.000000,0.000000}%
\pgfsetfillcolor{currentfill}%
\pgfsetlinewidth{0.803000pt}%
\definecolor{currentstroke}{rgb}{0.000000,0.000000,0.000000}%
\pgfsetstrokecolor{currentstroke}%
\pgfsetdash{}{0pt}%
\pgfsys@defobject{currentmarker}{\pgfqpoint{-0.048611in}{0.000000in}}{\pgfqpoint{-0.000000in}{0.000000in}}{%
\pgfpathmoveto{\pgfqpoint{-0.000000in}{0.000000in}}%
\pgfpathlineto{\pgfqpoint{-0.048611in}{0.000000in}}%
\pgfusepath{stroke,fill}%
}%
\begin{pgfscope}%
\pgfsys@transformshift{0.570343in}{4.402696in}%
\pgfsys@useobject{currentmarker}{}%
\end{pgfscope}%
\end{pgfscope}%
\begin{pgfscope}%
\definecolor{textcolor}{rgb}{0.000000,0.000000,0.000000}%
\pgfsetstrokecolor{textcolor}%
\pgfsetfillcolor{textcolor}%
\pgftext[x=0.384756in, y=4.349934in, left, base]{\color{textcolor}\sffamily\fontsize{10.000000}{12.000000}\selectfont 0}%
\end{pgfscope}%
\begin{pgfscope}%
\pgfsetbuttcap%
\pgfsetroundjoin%
\definecolor{currentfill}{rgb}{0.000000,0.000000,0.000000}%
\pgfsetfillcolor{currentfill}%
\pgfsetlinewidth{0.803000pt}%
\definecolor{currentstroke}{rgb}{0.000000,0.000000,0.000000}%
\pgfsetstrokecolor{currentstroke}%
\pgfsetdash{}{0pt}%
\pgfsys@defobject{currentmarker}{\pgfqpoint{-0.048611in}{0.000000in}}{\pgfqpoint{-0.000000in}{0.000000in}}{%
\pgfpathmoveto{\pgfqpoint{-0.000000in}{0.000000in}}%
\pgfpathlineto{\pgfqpoint{-0.048611in}{0.000000in}}%
\pgfusepath{stroke,fill}%
}%
\begin{pgfscope}%
\pgfsys@transformshift{0.570343in}{5.272624in}%
\pgfsys@useobject{currentmarker}{}%
\end{pgfscope}%
\end{pgfscope}%
\begin{pgfscope}%
\definecolor{textcolor}{rgb}{0.000000,0.000000,0.000000}%
\pgfsetstrokecolor{textcolor}%
\pgfsetfillcolor{textcolor}%
\pgftext[x=0.296390in, y=5.219862in, left, base]{\color{textcolor}\sffamily\fontsize{10.000000}{12.000000}\selectfont 50}%
\end{pgfscope}%
\begin{pgfscope}%
\pgfsetbuttcap%
\pgfsetroundjoin%
\definecolor{currentfill}{rgb}{0.000000,0.000000,0.000000}%
\pgfsetfillcolor{currentfill}%
\pgfsetlinewidth{0.803000pt}%
\definecolor{currentstroke}{rgb}{0.000000,0.000000,0.000000}%
\pgfsetstrokecolor{currentstroke}%
\pgfsetdash{}{0pt}%
\pgfsys@defobject{currentmarker}{\pgfqpoint{-0.048611in}{0.000000in}}{\pgfqpoint{-0.000000in}{0.000000in}}{%
\pgfpathmoveto{\pgfqpoint{-0.000000in}{0.000000in}}%
\pgfpathlineto{\pgfqpoint{-0.048611in}{0.000000in}}%
\pgfusepath{stroke,fill}%
}%
\begin{pgfscope}%
\pgfsys@transformshift{0.570343in}{6.142552in}%
\pgfsys@useobject{currentmarker}{}%
\end{pgfscope}%
\end{pgfscope}%
\begin{pgfscope}%
\definecolor{textcolor}{rgb}{0.000000,0.000000,0.000000}%
\pgfsetstrokecolor{textcolor}%
\pgfsetfillcolor{textcolor}%
\pgftext[x=0.208025in, y=6.089790in, left, base]{\color{textcolor}\sffamily\fontsize{10.000000}{12.000000}\selectfont 100}%
\end{pgfscope}%
\begin{pgfscope}%
\pgfsetbuttcap%
\pgfsetroundjoin%
\definecolor{currentfill}{rgb}{0.000000,0.000000,0.000000}%
\pgfsetfillcolor{currentfill}%
\pgfsetlinewidth{0.803000pt}%
\definecolor{currentstroke}{rgb}{0.000000,0.000000,0.000000}%
\pgfsetstrokecolor{currentstroke}%
\pgfsetdash{}{0pt}%
\pgfsys@defobject{currentmarker}{\pgfqpoint{-0.048611in}{0.000000in}}{\pgfqpoint{-0.000000in}{0.000000in}}{%
\pgfpathmoveto{\pgfqpoint{-0.000000in}{0.000000in}}%
\pgfpathlineto{\pgfqpoint{-0.048611in}{0.000000in}}%
\pgfusepath{stroke,fill}%
}%
\begin{pgfscope}%
\pgfsys@transformshift{0.570343in}{7.012480in}%
\pgfsys@useobject{currentmarker}{}%
\end{pgfscope}%
\end{pgfscope}%
\begin{pgfscope}%
\definecolor{textcolor}{rgb}{0.000000,0.000000,0.000000}%
\pgfsetstrokecolor{textcolor}%
\pgfsetfillcolor{textcolor}%
\pgftext[x=0.208025in, y=6.959718in, left, base]{\color{textcolor}\sffamily\fontsize{10.000000}{12.000000}\selectfont 150}%
\end{pgfscope}%
\begin{pgfscope}%
\pgfsetbuttcap%
\pgfsetroundjoin%
\definecolor{currentfill}{rgb}{0.000000,0.000000,0.000000}%
\pgfsetfillcolor{currentfill}%
\pgfsetlinewidth{0.803000pt}%
\definecolor{currentstroke}{rgb}{0.000000,0.000000,0.000000}%
\pgfsetstrokecolor{currentstroke}%
\pgfsetdash{}{0pt}%
\pgfsys@defobject{currentmarker}{\pgfqpoint{-0.048611in}{0.000000in}}{\pgfqpoint{-0.000000in}{0.000000in}}{%
\pgfpathmoveto{\pgfqpoint{-0.000000in}{0.000000in}}%
\pgfpathlineto{\pgfqpoint{-0.048611in}{0.000000in}}%
\pgfusepath{stroke,fill}%
}%
\begin{pgfscope}%
\pgfsys@transformshift{0.570343in}{7.882408in}%
\pgfsys@useobject{currentmarker}{}%
\end{pgfscope}%
\end{pgfscope}%
\begin{pgfscope}%
\definecolor{textcolor}{rgb}{0.000000,0.000000,0.000000}%
\pgfsetstrokecolor{textcolor}%
\pgfsetfillcolor{textcolor}%
\pgftext[x=0.208025in, y=7.829646in, left, base]{\color{textcolor}\sffamily\fontsize{10.000000}{12.000000}\selectfont 200}%
\end{pgfscope}%
\begin{pgfscope}%
\pgfpathrectangle{\pgfqpoint{0.570343in}{0.331635in}}{\pgfqpoint{9.300000in}{7.700000in}}%
\pgfusepath{clip}%
\pgfsetrectcap%
\pgfsetroundjoin%
\pgfsetlinewidth{1.505625pt}%
\definecolor{currentstroke}{rgb}{0.631373,0.788235,0.956863}%
\pgfsetstrokecolor{currentstroke}%
\pgfsetstrokeopacity{0.800000}%
\pgfsetdash{}{0pt}%
\pgfpathmoveto{\pgfqpoint{5.786839in}{4.786460in}}%
\pgfpathlineto{\pgfqpoint{6.094877in}{4.708927in}}%
\pgfusepath{stroke}%
\end{pgfscope}%
\begin{pgfscope}%
\pgfpathrectangle{\pgfqpoint{0.570343in}{0.331635in}}{\pgfqpoint{9.300000in}{7.700000in}}%
\pgfusepath{clip}%
\pgfsetrectcap%
\pgfsetroundjoin%
\pgfsetlinewidth{1.505625pt}%
\definecolor{currentstroke}{rgb}{0.631373,0.788235,0.956863}%
\pgfsetstrokecolor{currentstroke}%
\pgfsetstrokeopacity{0.800000}%
\pgfsetdash{}{0pt}%
\pgfpathmoveto{\pgfqpoint{6.090094in}{6.743997in}}%
\pgfpathlineto{\pgfqpoint{6.094877in}{4.708927in}}%
\pgfusepath{stroke}%
\end{pgfscope}%
\begin{pgfscope}%
\pgfpathrectangle{\pgfqpoint{0.570343in}{0.331635in}}{\pgfqpoint{9.300000in}{7.700000in}}%
\pgfusepath{clip}%
\pgfsetrectcap%
\pgfsetroundjoin%
\pgfsetlinewidth{1.505625pt}%
\definecolor{currentstroke}{rgb}{0.631373,0.788235,0.956863}%
\pgfsetstrokecolor{currentstroke}%
\pgfsetstrokeopacity{0.800000}%
\pgfsetdash{}{0pt}%
\pgfpathmoveto{\pgfqpoint{6.096102in}{4.037835in}}%
\pgfpathlineto{\pgfqpoint{6.094877in}{4.708927in}}%
\pgfusepath{stroke}%
\end{pgfscope}%
\begin{pgfscope}%
\pgfpathrectangle{\pgfqpoint{0.570343in}{0.331635in}}{\pgfqpoint{9.300000in}{7.700000in}}%
\pgfusepath{clip}%
\pgfsetrectcap%
\pgfsetroundjoin%
\pgfsetlinewidth{1.505625pt}%
\definecolor{currentstroke}{rgb}{0.631373,0.788235,0.956863}%
\pgfsetstrokecolor{currentstroke}%
\pgfsetstrokeopacity{0.800000}%
\pgfsetdash{}{0pt}%
\pgfpathmoveto{\pgfqpoint{6.085130in}{3.591851in}}%
\pgfpathlineto{\pgfqpoint{6.094877in}{4.708927in}}%
\pgfusepath{stroke}%
\end{pgfscope}%
\begin{pgfscope}%
\pgfpathrectangle{\pgfqpoint{0.570343in}{0.331635in}}{\pgfqpoint{9.300000in}{7.700000in}}%
\pgfusepath{clip}%
\pgfsetrectcap%
\pgfsetroundjoin%
\pgfsetlinewidth{1.505625pt}%
\definecolor{currentstroke}{rgb}{0.631373,0.788235,0.956863}%
\pgfsetstrokecolor{currentstroke}%
\pgfsetstrokeopacity{0.800000}%
\pgfsetdash{}{0pt}%
\pgfpathmoveto{\pgfqpoint{5.699074in}{3.917404in}}%
\pgfpathlineto{\pgfqpoint{6.094877in}{4.708927in}}%
\pgfusepath{stroke}%
\end{pgfscope}%
\begin{pgfscope}%
\pgfpathrectangle{\pgfqpoint{0.570343in}{0.331635in}}{\pgfqpoint{9.300000in}{7.700000in}}%
\pgfusepath{clip}%
\pgfsetrectcap%
\pgfsetroundjoin%
\pgfsetlinewidth{1.505625pt}%
\definecolor{currentstroke}{rgb}{0.631373,0.788235,0.956863}%
\pgfsetstrokecolor{currentstroke}%
\pgfsetstrokeopacity{0.800000}%
\pgfsetdash{}{0pt}%
\pgfpathmoveto{\pgfqpoint{6.170060in}{7.112143in}}%
\pgfpathlineto{\pgfqpoint{6.094877in}{4.708927in}}%
\pgfusepath{stroke}%
\end{pgfscope}%
\begin{pgfscope}%
\pgfpathrectangle{\pgfqpoint{0.570343in}{0.331635in}}{\pgfqpoint{9.300000in}{7.700000in}}%
\pgfusepath{clip}%
\pgfsetrectcap%
\pgfsetroundjoin%
\pgfsetlinewidth{1.505625pt}%
\definecolor{currentstroke}{rgb}{0.631373,0.788235,0.956863}%
\pgfsetstrokecolor{currentstroke}%
\pgfsetstrokeopacity{0.800000}%
\pgfsetdash{}{0pt}%
\pgfpathmoveto{\pgfqpoint{5.816201in}{5.353806in}}%
\pgfpathlineto{\pgfqpoint{6.094877in}{4.708927in}}%
\pgfusepath{stroke}%
\end{pgfscope}%
\begin{pgfscope}%
\pgfpathrectangle{\pgfqpoint{0.570343in}{0.331635in}}{\pgfqpoint{9.300000in}{7.700000in}}%
\pgfusepath{clip}%
\pgfsetrectcap%
\pgfsetroundjoin%
\pgfsetlinewidth{1.505625pt}%
\definecolor{currentstroke}{rgb}{0.631373,0.788235,0.956863}%
\pgfsetstrokecolor{currentstroke}%
\pgfsetstrokeopacity{0.800000}%
\pgfsetdash{}{0pt}%
\pgfpathmoveto{\pgfqpoint{6.056307in}{3.122075in}}%
\pgfpathlineto{\pgfqpoint{6.094877in}{4.708927in}}%
\pgfusepath{stroke}%
\end{pgfscope}%
\begin{pgfscope}%
\pgfpathrectangle{\pgfqpoint{0.570343in}{0.331635in}}{\pgfqpoint{9.300000in}{7.700000in}}%
\pgfusepath{clip}%
\pgfsetrectcap%
\pgfsetroundjoin%
\pgfsetlinewidth{1.505625pt}%
\definecolor{currentstroke}{rgb}{0.631373,0.788235,0.956863}%
\pgfsetstrokecolor{currentstroke}%
\pgfsetstrokeopacity{0.800000}%
\pgfsetdash{}{0pt}%
\pgfpathmoveto{\pgfqpoint{6.035930in}{4.663747in}}%
\pgfpathlineto{\pgfqpoint{6.094877in}{4.708927in}}%
\pgfusepath{stroke}%
\end{pgfscope}%
\begin{pgfscope}%
\pgfpathrectangle{\pgfqpoint{0.570343in}{0.331635in}}{\pgfqpoint{9.300000in}{7.700000in}}%
\pgfusepath{clip}%
\pgfsetrectcap%
\pgfsetroundjoin%
\pgfsetlinewidth{1.505625pt}%
\definecolor{currentstroke}{rgb}{0.631373,0.788235,0.956863}%
\pgfsetstrokecolor{currentstroke}%
\pgfsetstrokeopacity{0.800000}%
\pgfsetdash{}{0pt}%
\pgfpathmoveto{\pgfqpoint{6.279500in}{0.832937in}}%
\pgfpathlineto{\pgfqpoint{6.094877in}{4.708927in}}%
\pgfusepath{stroke}%
\end{pgfscope}%
\begin{pgfscope}%
\pgfpathrectangle{\pgfqpoint{0.570343in}{0.331635in}}{\pgfqpoint{9.300000in}{7.700000in}}%
\pgfusepath{clip}%
\pgfsetrectcap%
\pgfsetroundjoin%
\pgfsetlinewidth{1.505625pt}%
\definecolor{currentstroke}{rgb}{0.631373,0.788235,0.956863}%
\pgfsetstrokecolor{currentstroke}%
\pgfsetstrokeopacity{0.800000}%
\pgfsetdash{}{0pt}%
\pgfpathmoveto{\pgfqpoint{6.307476in}{6.145042in}}%
\pgfpathlineto{\pgfqpoint{6.094877in}{4.708927in}}%
\pgfusepath{stroke}%
\end{pgfscope}%
\begin{pgfscope}%
\pgfpathrectangle{\pgfqpoint{0.570343in}{0.331635in}}{\pgfqpoint{9.300000in}{7.700000in}}%
\pgfusepath{clip}%
\pgfsetrectcap%
\pgfsetroundjoin%
\pgfsetlinewidth{1.505625pt}%
\definecolor{currentstroke}{rgb}{0.631373,0.788235,0.956863}%
\pgfsetstrokecolor{currentstroke}%
\pgfsetstrokeopacity{0.800000}%
\pgfsetdash{}{0pt}%
\pgfpathmoveto{\pgfqpoint{6.108461in}{7.532871in}}%
\pgfpathlineto{\pgfqpoint{6.094877in}{4.708927in}}%
\pgfusepath{stroke}%
\end{pgfscope}%
\begin{pgfscope}%
\pgfpathrectangle{\pgfqpoint{0.570343in}{0.331635in}}{\pgfqpoint{9.300000in}{7.700000in}}%
\pgfusepath{clip}%
\pgfsetrectcap%
\pgfsetroundjoin%
\pgfsetlinewidth{1.505625pt}%
\definecolor{currentstroke}{rgb}{0.631373,0.788235,0.956863}%
\pgfsetstrokecolor{currentstroke}%
\pgfsetstrokeopacity{0.800000}%
\pgfsetdash{}{0pt}%
\pgfpathmoveto{\pgfqpoint{6.404411in}{1.075135in}}%
\pgfpathlineto{\pgfqpoint{6.094877in}{4.708927in}}%
\pgfusepath{stroke}%
\end{pgfscope}%
\begin{pgfscope}%
\pgfpathrectangle{\pgfqpoint{0.570343in}{0.331635in}}{\pgfqpoint{9.300000in}{7.700000in}}%
\pgfusepath{clip}%
\pgfsetrectcap%
\pgfsetroundjoin%
\pgfsetlinewidth{1.505625pt}%
\definecolor{currentstroke}{rgb}{0.631373,0.788235,0.956863}%
\pgfsetstrokecolor{currentstroke}%
\pgfsetstrokeopacity{0.800000}%
\pgfsetdash{}{0pt}%
\pgfpathmoveto{\pgfqpoint{6.178369in}{7.681635in}}%
\pgfpathlineto{\pgfqpoint{6.094877in}{4.708927in}}%
\pgfusepath{stroke}%
\end{pgfscope}%
\begin{pgfscope}%
\pgfpathrectangle{\pgfqpoint{0.570343in}{0.331635in}}{\pgfqpoint{9.300000in}{7.700000in}}%
\pgfusepath{clip}%
\pgfsetrectcap%
\pgfsetroundjoin%
\pgfsetlinewidth{1.505625pt}%
\definecolor{currentstroke}{rgb}{0.631373,0.788235,0.956863}%
\pgfsetstrokecolor{currentstroke}%
\pgfsetstrokeopacity{0.800000}%
\pgfsetdash{}{0pt}%
\pgfpathmoveto{\pgfqpoint{6.121234in}{3.101143in}}%
\pgfpathlineto{\pgfqpoint{6.094877in}{4.708927in}}%
\pgfusepath{stroke}%
\end{pgfscope}%
\begin{pgfscope}%
\pgfpathrectangle{\pgfqpoint{0.570343in}{0.331635in}}{\pgfqpoint{9.300000in}{7.700000in}}%
\pgfusepath{clip}%
\pgfsetrectcap%
\pgfsetroundjoin%
\pgfsetlinewidth{1.505625pt}%
\definecolor{currentstroke}{rgb}{0.631373,0.788235,0.956863}%
\pgfsetstrokecolor{currentstroke}%
\pgfsetstrokeopacity{0.800000}%
\pgfsetdash{}{0pt}%
\pgfpathmoveto{\pgfqpoint{6.298085in}{7.670283in}}%
\pgfpathlineto{\pgfqpoint{6.094877in}{4.708927in}}%
\pgfusepath{stroke}%
\end{pgfscope}%
\begin{pgfscope}%
\pgfpathrectangle{\pgfqpoint{0.570343in}{0.331635in}}{\pgfqpoint{9.300000in}{7.700000in}}%
\pgfusepath{clip}%
\pgfsetrectcap%
\pgfsetroundjoin%
\pgfsetlinewidth{1.505625pt}%
\definecolor{currentstroke}{rgb}{0.631373,0.788235,0.956863}%
\pgfsetstrokecolor{currentstroke}%
\pgfsetstrokeopacity{0.800000}%
\pgfsetdash{}{0pt}%
\pgfpathmoveto{\pgfqpoint{6.239187in}{7.324138in}}%
\pgfpathlineto{\pgfqpoint{6.094877in}{4.708927in}}%
\pgfusepath{stroke}%
\end{pgfscope}%
\begin{pgfscope}%
\pgfpathrectangle{\pgfqpoint{0.570343in}{0.331635in}}{\pgfqpoint{9.300000in}{7.700000in}}%
\pgfusepath{clip}%
\pgfsetrectcap%
\pgfsetroundjoin%
\pgfsetlinewidth{1.505625pt}%
\definecolor{currentstroke}{rgb}{0.631373,0.788235,0.956863}%
\pgfsetstrokecolor{currentstroke}%
\pgfsetstrokeopacity{0.800000}%
\pgfsetdash{}{0pt}%
\pgfpathmoveto{\pgfqpoint{5.924096in}{5.585671in}}%
\pgfpathlineto{\pgfqpoint{6.094877in}{4.708927in}}%
\pgfusepath{stroke}%
\end{pgfscope}%
\begin{pgfscope}%
\pgfpathrectangle{\pgfqpoint{0.570343in}{0.331635in}}{\pgfqpoint{9.300000in}{7.700000in}}%
\pgfusepath{clip}%
\pgfsetrectcap%
\pgfsetroundjoin%
\pgfsetlinewidth{1.505625pt}%
\definecolor{currentstroke}{rgb}{0.631373,0.788235,0.956863}%
\pgfsetstrokecolor{currentstroke}%
\pgfsetstrokeopacity{0.800000}%
\pgfsetdash{}{0pt}%
\pgfpathmoveto{\pgfqpoint{5.694819in}{4.453919in}}%
\pgfpathlineto{\pgfqpoint{6.094877in}{4.708927in}}%
\pgfusepath{stroke}%
\end{pgfscope}%
\begin{pgfscope}%
\pgfpathrectangle{\pgfqpoint{0.570343in}{0.331635in}}{\pgfqpoint{9.300000in}{7.700000in}}%
\pgfusepath{clip}%
\pgfsetrectcap%
\pgfsetroundjoin%
\pgfsetlinewidth{1.505625pt}%
\definecolor{currentstroke}{rgb}{0.631373,0.788235,0.956863}%
\pgfsetstrokecolor{currentstroke}%
\pgfsetstrokeopacity{0.800000}%
\pgfsetdash{}{0pt}%
\pgfpathmoveto{\pgfqpoint{6.024489in}{2.457674in}}%
\pgfpathlineto{\pgfqpoint{6.094877in}{4.708927in}}%
\pgfusepath{stroke}%
\end{pgfscope}%
\begin{pgfscope}%
\pgfpathrectangle{\pgfqpoint{0.570343in}{0.331635in}}{\pgfqpoint{9.300000in}{7.700000in}}%
\pgfusepath{clip}%
\pgfsetrectcap%
\pgfsetroundjoin%
\pgfsetlinewidth{1.505625pt}%
\definecolor{currentstroke}{rgb}{0.631373,0.788235,0.956863}%
\pgfsetstrokecolor{currentstroke}%
\pgfsetstrokeopacity{0.800000}%
\pgfsetdash{}{0pt}%
\pgfpathmoveto{\pgfqpoint{6.436145in}{4.397584in}}%
\pgfpathlineto{\pgfqpoint{6.094877in}{4.708927in}}%
\pgfusepath{stroke}%
\end{pgfscope}%
\begin{pgfscope}%
\pgfpathrectangle{\pgfqpoint{0.570343in}{0.331635in}}{\pgfqpoint{9.300000in}{7.700000in}}%
\pgfusepath{clip}%
\pgfsetrectcap%
\pgfsetroundjoin%
\pgfsetlinewidth{1.505625pt}%
\definecolor{currentstroke}{rgb}{0.631373,0.788235,0.956863}%
\pgfsetstrokecolor{currentstroke}%
\pgfsetstrokeopacity{0.800000}%
\pgfsetdash{}{0pt}%
\pgfpathmoveto{\pgfqpoint{6.011284in}{5.203258in}}%
\pgfpathlineto{\pgfqpoint{6.094877in}{4.708927in}}%
\pgfusepath{stroke}%
\end{pgfscope}%
\begin{pgfscope}%
\pgfpathrectangle{\pgfqpoint{0.570343in}{0.331635in}}{\pgfqpoint{9.300000in}{7.700000in}}%
\pgfusepath{clip}%
\pgfsetrectcap%
\pgfsetroundjoin%
\pgfsetlinewidth{1.505625pt}%
\definecolor{currentstroke}{rgb}{0.631373,0.788235,0.956863}%
\pgfsetstrokecolor{currentstroke}%
\pgfsetstrokeopacity{0.800000}%
\pgfsetdash{}{0pt}%
\pgfpathmoveto{\pgfqpoint{6.108299in}{5.963715in}}%
\pgfpathlineto{\pgfqpoint{6.094877in}{4.708927in}}%
\pgfusepath{stroke}%
\end{pgfscope}%
\begin{pgfscope}%
\pgfpathrectangle{\pgfqpoint{0.570343in}{0.331635in}}{\pgfqpoint{9.300000in}{7.700000in}}%
\pgfusepath{clip}%
\pgfsetrectcap%
\pgfsetroundjoin%
\pgfsetlinewidth{1.505625pt}%
\definecolor{currentstroke}{rgb}{0.631373,0.788235,0.956863}%
\pgfsetstrokecolor{currentstroke}%
\pgfsetstrokeopacity{0.800000}%
\pgfsetdash{}{0pt}%
\pgfpathmoveto{\pgfqpoint{6.344974in}{3.806462in}}%
\pgfpathlineto{\pgfqpoint{6.094877in}{4.708927in}}%
\pgfusepath{stroke}%
\end{pgfscope}%
\begin{pgfscope}%
\pgfpathrectangle{\pgfqpoint{0.570343in}{0.331635in}}{\pgfqpoint{9.300000in}{7.700000in}}%
\pgfusepath{clip}%
\pgfsetrectcap%
\pgfsetroundjoin%
\pgfsetlinewidth{1.505625pt}%
\definecolor{currentstroke}{rgb}{0.631373,0.788235,0.956863}%
\pgfsetstrokecolor{currentstroke}%
\pgfsetstrokeopacity{0.800000}%
\pgfsetdash{}{0pt}%
\pgfpathmoveto{\pgfqpoint{6.386006in}{3.100773in}}%
\pgfpathlineto{\pgfqpoint{6.094877in}{4.708927in}}%
\pgfusepath{stroke}%
\end{pgfscope}%
\begin{pgfscope}%
\pgfpathrectangle{\pgfqpoint{0.570343in}{0.331635in}}{\pgfqpoint{9.300000in}{7.700000in}}%
\pgfusepath{clip}%
\pgfsetrectcap%
\pgfsetroundjoin%
\pgfsetlinewidth{1.505625pt}%
\definecolor{currentstroke}{rgb}{0.631373,0.788235,0.956863}%
\pgfsetstrokecolor{currentstroke}%
\pgfsetstrokeopacity{0.800000}%
\pgfsetdash{}{0pt}%
\pgfpathmoveto{\pgfqpoint{5.830354in}{4.699006in}}%
\pgfpathlineto{\pgfqpoint{6.094877in}{4.708927in}}%
\pgfusepath{stroke}%
\end{pgfscope}%
\begin{pgfscope}%
\pgfpathrectangle{\pgfqpoint{0.570343in}{0.331635in}}{\pgfqpoint{9.300000in}{7.700000in}}%
\pgfusepath{clip}%
\pgfsetrectcap%
\pgfsetroundjoin%
\pgfsetlinewidth{1.505625pt}%
\definecolor{currentstroke}{rgb}{0.631373,0.788235,0.956863}%
\pgfsetstrokecolor{currentstroke}%
\pgfsetstrokeopacity{0.800000}%
\pgfsetdash{}{0pt}%
\pgfpathmoveto{\pgfqpoint{6.363645in}{4.779876in}}%
\pgfpathlineto{\pgfqpoint{6.094877in}{4.708927in}}%
\pgfusepath{stroke}%
\end{pgfscope}%
\begin{pgfscope}%
\pgfpathrectangle{\pgfqpoint{0.570343in}{0.331635in}}{\pgfqpoint{9.300000in}{7.700000in}}%
\pgfusepath{clip}%
\pgfsetrectcap%
\pgfsetroundjoin%
\pgfsetlinewidth{1.505625pt}%
\definecolor{currentstroke}{rgb}{0.631373,0.788235,0.956863}%
\pgfsetstrokecolor{currentstroke}%
\pgfsetstrokeopacity{0.800000}%
\pgfsetdash{}{0pt}%
\pgfpathmoveto{\pgfqpoint{6.416113in}{3.402302in}}%
\pgfpathlineto{\pgfqpoint{6.094877in}{4.708927in}}%
\pgfusepath{stroke}%
\end{pgfscope}%
\begin{pgfscope}%
\pgfpathrectangle{\pgfqpoint{0.570343in}{0.331635in}}{\pgfqpoint{9.300000in}{7.700000in}}%
\pgfusepath{clip}%
\pgfsetrectcap%
\pgfsetroundjoin%
\pgfsetlinewidth{1.505625pt}%
\definecolor{currentstroke}{rgb}{0.631373,0.788235,0.956863}%
\pgfsetstrokecolor{currentstroke}%
\pgfsetstrokeopacity{0.800000}%
\pgfsetdash{}{0pt}%
\pgfpathmoveto{\pgfqpoint{5.950563in}{1.908226in}}%
\pgfpathlineto{\pgfqpoint{6.094877in}{4.708927in}}%
\pgfusepath{stroke}%
\end{pgfscope}%
\begin{pgfscope}%
\pgfpathrectangle{\pgfqpoint{0.570343in}{0.331635in}}{\pgfqpoint{9.300000in}{7.700000in}}%
\pgfusepath{clip}%
\pgfsetrectcap%
\pgfsetroundjoin%
\pgfsetlinewidth{1.505625pt}%
\definecolor{currentstroke}{rgb}{0.631373,0.788235,0.956863}%
\pgfsetstrokecolor{currentstroke}%
\pgfsetstrokeopacity{0.800000}%
\pgfsetdash{}{0pt}%
\pgfpathmoveto{\pgfqpoint{6.254873in}{6.716253in}}%
\pgfpathlineto{\pgfqpoint{6.094877in}{4.708927in}}%
\pgfusepath{stroke}%
\end{pgfscope}%
\begin{pgfscope}%
\pgfpathrectangle{\pgfqpoint{0.570343in}{0.331635in}}{\pgfqpoint{9.300000in}{7.700000in}}%
\pgfusepath{clip}%
\pgfsetrectcap%
\pgfsetroundjoin%
\pgfsetlinewidth{1.505625pt}%
\definecolor{currentstroke}{rgb}{0.631373,0.788235,0.956863}%
\pgfsetstrokecolor{currentstroke}%
\pgfsetstrokeopacity{0.800000}%
\pgfsetdash{}{0pt}%
\pgfpathmoveto{\pgfqpoint{6.143314in}{6.552458in}}%
\pgfpathlineto{\pgfqpoint{6.094877in}{4.708927in}}%
\pgfusepath{stroke}%
\end{pgfscope}%
\begin{pgfscope}%
\pgfpathrectangle{\pgfqpoint{0.570343in}{0.331635in}}{\pgfqpoint{9.300000in}{7.700000in}}%
\pgfusepath{clip}%
\pgfsetrectcap%
\pgfsetroundjoin%
\pgfsetlinewidth{1.505625pt}%
\definecolor{currentstroke}{rgb}{0.631373,0.788235,0.956863}%
\pgfsetstrokecolor{currentstroke}%
\pgfsetstrokeopacity{0.800000}%
\pgfsetdash{}{0pt}%
\pgfpathmoveto{\pgfqpoint{5.740772in}{5.184091in}}%
\pgfpathlineto{\pgfqpoint{6.094877in}{4.708927in}}%
\pgfusepath{stroke}%
\end{pgfscope}%
\begin{pgfscope}%
\pgfpathrectangle{\pgfqpoint{0.570343in}{0.331635in}}{\pgfqpoint{9.300000in}{7.700000in}}%
\pgfusepath{clip}%
\pgfsetrectcap%
\pgfsetroundjoin%
\pgfsetlinewidth{1.505625pt}%
\definecolor{currentstroke}{rgb}{0.631373,0.788235,0.956863}%
\pgfsetstrokecolor{currentstroke}%
\pgfsetstrokeopacity{0.800000}%
\pgfsetdash{}{0pt}%
\pgfpathmoveto{\pgfqpoint{6.169499in}{5.979279in}}%
\pgfpathlineto{\pgfqpoint{6.094877in}{4.708927in}}%
\pgfusepath{stroke}%
\end{pgfscope}%
\begin{pgfscope}%
\pgfpathrectangle{\pgfqpoint{0.570343in}{0.331635in}}{\pgfqpoint{9.300000in}{7.700000in}}%
\pgfusepath{clip}%
\pgfsetrectcap%
\pgfsetroundjoin%
\pgfsetlinewidth{1.505625pt}%
\definecolor{currentstroke}{rgb}{0.631373,0.788235,0.956863}%
\pgfsetstrokecolor{currentstroke}%
\pgfsetstrokeopacity{0.800000}%
\pgfsetdash{}{0pt}%
\pgfpathmoveto{\pgfqpoint{6.359818in}{6.197363in}}%
\pgfpathlineto{\pgfqpoint{6.094877in}{4.708927in}}%
\pgfusepath{stroke}%
\end{pgfscope}%
\begin{pgfscope}%
\pgfpathrectangle{\pgfqpoint{0.570343in}{0.331635in}}{\pgfqpoint{9.300000in}{7.700000in}}%
\pgfusepath{clip}%
\pgfsetrectcap%
\pgfsetroundjoin%
\pgfsetlinewidth{1.505625pt}%
\definecolor{currentstroke}{rgb}{0.631373,0.788235,0.956863}%
\pgfsetstrokecolor{currentstroke}%
\pgfsetstrokeopacity{0.800000}%
\pgfsetdash{}{0pt}%
\pgfpathmoveto{\pgfqpoint{5.850943in}{6.816126in}}%
\pgfpathlineto{\pgfqpoint{6.094877in}{4.708927in}}%
\pgfusepath{stroke}%
\end{pgfscope}%
\begin{pgfscope}%
\pgfpathrectangle{\pgfqpoint{0.570343in}{0.331635in}}{\pgfqpoint{9.300000in}{7.700000in}}%
\pgfusepath{clip}%
\pgfsetrectcap%
\pgfsetroundjoin%
\pgfsetlinewidth{1.505625pt}%
\definecolor{currentstroke}{rgb}{0.631373,0.788235,0.956863}%
\pgfsetstrokecolor{currentstroke}%
\pgfsetstrokeopacity{0.800000}%
\pgfsetdash{}{0pt}%
\pgfpathmoveto{\pgfqpoint{6.323724in}{4.397691in}}%
\pgfpathlineto{\pgfqpoint{6.094877in}{4.708927in}}%
\pgfusepath{stroke}%
\end{pgfscope}%
\begin{pgfscope}%
\pgfpathrectangle{\pgfqpoint{0.570343in}{0.331635in}}{\pgfqpoint{9.300000in}{7.700000in}}%
\pgfusepath{clip}%
\pgfsetrectcap%
\pgfsetroundjoin%
\pgfsetlinewidth{1.505625pt}%
\definecolor{currentstroke}{rgb}{0.631373,0.788235,0.956863}%
\pgfsetstrokecolor{currentstroke}%
\pgfsetstrokeopacity{0.800000}%
\pgfsetdash{}{0pt}%
\pgfpathmoveto{\pgfqpoint{5.756078in}{4.117922in}}%
\pgfpathlineto{\pgfqpoint{6.094877in}{4.708927in}}%
\pgfusepath{stroke}%
\end{pgfscope}%
\begin{pgfscope}%
\pgfpathrectangle{\pgfqpoint{0.570343in}{0.331635in}}{\pgfqpoint{9.300000in}{7.700000in}}%
\pgfusepath{clip}%
\pgfsetrectcap%
\pgfsetroundjoin%
\pgfsetlinewidth{1.505625pt}%
\definecolor{currentstroke}{rgb}{0.631373,0.788235,0.956863}%
\pgfsetstrokecolor{currentstroke}%
\pgfsetstrokeopacity{0.800000}%
\pgfsetdash{}{0pt}%
\pgfpathmoveto{\pgfqpoint{6.208523in}{6.272798in}}%
\pgfpathlineto{\pgfqpoint{6.094877in}{4.708927in}}%
\pgfusepath{stroke}%
\end{pgfscope}%
\begin{pgfscope}%
\pgfpathrectangle{\pgfqpoint{0.570343in}{0.331635in}}{\pgfqpoint{9.300000in}{7.700000in}}%
\pgfusepath{clip}%
\pgfsetrectcap%
\pgfsetroundjoin%
\pgfsetlinewidth{1.505625pt}%
\definecolor{currentstroke}{rgb}{0.631373,0.788235,0.956863}%
\pgfsetstrokecolor{currentstroke}%
\pgfsetstrokeopacity{0.800000}%
\pgfsetdash{}{0pt}%
\pgfpathmoveto{\pgfqpoint{5.947265in}{3.012205in}}%
\pgfpathlineto{\pgfqpoint{6.094877in}{4.708927in}}%
\pgfusepath{stroke}%
\end{pgfscope}%
\begin{pgfscope}%
\pgfpathrectangle{\pgfqpoint{0.570343in}{0.331635in}}{\pgfqpoint{9.300000in}{7.700000in}}%
\pgfusepath{clip}%
\pgfsetrectcap%
\pgfsetroundjoin%
\pgfsetlinewidth{1.505625pt}%
\definecolor{currentstroke}{rgb}{0.631373,0.788235,0.956863}%
\pgfsetstrokecolor{currentstroke}%
\pgfsetstrokeopacity{0.800000}%
\pgfsetdash{}{0pt}%
\pgfpathmoveto{\pgfqpoint{5.997400in}{7.481522in}}%
\pgfpathlineto{\pgfqpoint{6.094877in}{4.708927in}}%
\pgfusepath{stroke}%
\end{pgfscope}%
\begin{pgfscope}%
\pgfpathrectangle{\pgfqpoint{0.570343in}{0.331635in}}{\pgfqpoint{9.300000in}{7.700000in}}%
\pgfusepath{clip}%
\pgfsetrectcap%
\pgfsetroundjoin%
\pgfsetlinewidth{1.505625pt}%
\definecolor{currentstroke}{rgb}{0.631373,0.788235,0.956863}%
\pgfsetstrokecolor{currentstroke}%
\pgfsetstrokeopacity{0.800000}%
\pgfsetdash{}{0pt}%
\pgfpathmoveto{\pgfqpoint{6.104525in}{2.688545in}}%
\pgfpathlineto{\pgfqpoint{6.094877in}{4.708927in}}%
\pgfusepath{stroke}%
\end{pgfscope}%
\begin{pgfscope}%
\pgfpathrectangle{\pgfqpoint{0.570343in}{0.331635in}}{\pgfqpoint{9.300000in}{7.700000in}}%
\pgfusepath{clip}%
\pgfsetrectcap%
\pgfsetroundjoin%
\pgfsetlinewidth{1.505625pt}%
\definecolor{currentstroke}{rgb}{0.631373,0.788235,0.956863}%
\pgfsetstrokecolor{currentstroke}%
\pgfsetstrokeopacity{0.800000}%
\pgfsetdash{}{0pt}%
\pgfpathmoveto{\pgfqpoint{5.949816in}{4.799418in}}%
\pgfpathlineto{\pgfqpoint{6.094877in}{4.708927in}}%
\pgfusepath{stroke}%
\end{pgfscope}%
\begin{pgfscope}%
\pgfpathrectangle{\pgfqpoint{0.570343in}{0.331635in}}{\pgfqpoint{9.300000in}{7.700000in}}%
\pgfusepath{clip}%
\pgfsetrectcap%
\pgfsetroundjoin%
\pgfsetlinewidth{1.505625pt}%
\definecolor{currentstroke}{rgb}{0.631373,0.788235,0.956863}%
\pgfsetstrokecolor{currentstroke}%
\pgfsetstrokeopacity{0.800000}%
\pgfsetdash{}{0pt}%
\pgfpathmoveto{\pgfqpoint{6.020291in}{4.081854in}}%
\pgfpathlineto{\pgfqpoint{6.094877in}{4.708927in}}%
\pgfusepath{stroke}%
\end{pgfscope}%
\begin{pgfscope}%
\pgfpathrectangle{\pgfqpoint{0.570343in}{0.331635in}}{\pgfqpoint{9.300000in}{7.700000in}}%
\pgfusepath{clip}%
\pgfsetrectcap%
\pgfsetroundjoin%
\pgfsetlinewidth{1.505625pt}%
\definecolor{currentstroke}{rgb}{0.631373,0.788235,0.956863}%
\pgfsetstrokecolor{currentstroke}%
\pgfsetstrokeopacity{0.800000}%
\pgfsetdash{}{0pt}%
\pgfpathmoveto{\pgfqpoint{5.824951in}{3.468911in}}%
\pgfpathlineto{\pgfqpoint{6.094877in}{4.708927in}}%
\pgfusepath{stroke}%
\end{pgfscope}%
\begin{pgfscope}%
\pgfpathrectangle{\pgfqpoint{0.570343in}{0.331635in}}{\pgfqpoint{9.300000in}{7.700000in}}%
\pgfusepath{clip}%
\pgfsetrectcap%
\pgfsetroundjoin%
\pgfsetlinewidth{1.505625pt}%
\definecolor{currentstroke}{rgb}{0.631373,0.788235,0.956863}%
\pgfsetstrokecolor{currentstroke}%
\pgfsetstrokeopacity{0.800000}%
\pgfsetdash{}{0pt}%
\pgfpathmoveto{\pgfqpoint{6.330522in}{1.708942in}}%
\pgfpathlineto{\pgfqpoint{6.094877in}{4.708927in}}%
\pgfusepath{stroke}%
\end{pgfscope}%
\begin{pgfscope}%
\pgfpathrectangle{\pgfqpoint{0.570343in}{0.331635in}}{\pgfqpoint{9.300000in}{7.700000in}}%
\pgfusepath{clip}%
\pgfsetrectcap%
\pgfsetroundjoin%
\pgfsetlinewidth{1.505625pt}%
\definecolor{currentstroke}{rgb}{0.631373,0.788235,0.956863}%
\pgfsetstrokecolor{currentstroke}%
\pgfsetstrokeopacity{0.800000}%
\pgfsetdash{}{0pt}%
\pgfpathmoveto{\pgfqpoint{6.013977in}{3.431807in}}%
\pgfpathlineto{\pgfqpoint{6.094877in}{4.708927in}}%
\pgfusepath{stroke}%
\end{pgfscope}%
\begin{pgfscope}%
\pgfpathrectangle{\pgfqpoint{0.570343in}{0.331635in}}{\pgfqpoint{9.300000in}{7.700000in}}%
\pgfusepath{clip}%
\pgfsetrectcap%
\pgfsetroundjoin%
\pgfsetlinewidth{1.505625pt}%
\definecolor{currentstroke}{rgb}{0.631373,0.788235,0.956863}%
\pgfsetstrokecolor{currentstroke}%
\pgfsetstrokeopacity{0.800000}%
\pgfsetdash{}{0pt}%
\pgfpathmoveto{\pgfqpoint{5.861429in}{4.141974in}}%
\pgfpathlineto{\pgfqpoint{6.094877in}{4.708927in}}%
\pgfusepath{stroke}%
\end{pgfscope}%
\begin{pgfscope}%
\pgfpathrectangle{\pgfqpoint{0.570343in}{0.331635in}}{\pgfqpoint{9.300000in}{7.700000in}}%
\pgfusepath{clip}%
\pgfsetrectcap%
\pgfsetroundjoin%
\pgfsetlinewidth{1.505625pt}%
\definecolor{currentstroke}{rgb}{0.631373,0.788235,0.956863}%
\pgfsetstrokecolor{currentstroke}%
\pgfsetstrokeopacity{0.800000}%
\pgfsetdash{}{0pt}%
\pgfpathmoveto{\pgfqpoint{6.215095in}{2.976681in}}%
\pgfpathlineto{\pgfqpoint{6.094877in}{4.708927in}}%
\pgfusepath{stroke}%
\end{pgfscope}%
\begin{pgfscope}%
\pgfpathrectangle{\pgfqpoint{0.570343in}{0.331635in}}{\pgfqpoint{9.300000in}{7.700000in}}%
\pgfusepath{clip}%
\pgfsetrectcap%
\pgfsetroundjoin%
\pgfsetlinewidth{1.505625pt}%
\definecolor{currentstroke}{rgb}{0.631373,0.788235,0.956863}%
\pgfsetstrokecolor{currentstroke}%
\pgfsetstrokeopacity{0.800000}%
\pgfsetdash{}{0pt}%
\pgfpathmoveto{\pgfqpoint{6.424663in}{4.034970in}}%
\pgfpathlineto{\pgfqpoint{6.094877in}{4.708927in}}%
\pgfusepath{stroke}%
\end{pgfscope}%
\begin{pgfscope}%
\pgfpathrectangle{\pgfqpoint{0.570343in}{0.331635in}}{\pgfqpoint{9.300000in}{7.700000in}}%
\pgfusepath{clip}%
\pgfsetrectcap%
\pgfsetroundjoin%
\pgfsetlinewidth{1.505625pt}%
\definecolor{currentstroke}{rgb}{0.631373,0.788235,0.956863}%
\pgfsetstrokecolor{currentstroke}%
\pgfsetstrokeopacity{0.800000}%
\pgfsetdash{}{0pt}%
\pgfpathmoveto{\pgfqpoint{5.983136in}{6.934548in}}%
\pgfpathlineto{\pgfqpoint{6.094877in}{4.708927in}}%
\pgfusepath{stroke}%
\end{pgfscope}%
\begin{pgfscope}%
\pgfpathrectangle{\pgfqpoint{0.570343in}{0.331635in}}{\pgfqpoint{9.300000in}{7.700000in}}%
\pgfusepath{clip}%
\pgfsetrectcap%
\pgfsetroundjoin%
\pgfsetlinewidth{1.505625pt}%
\definecolor{currentstroke}{rgb}{1.000000,0.705882,0.509804}%
\pgfsetstrokecolor{currentstroke}%
\pgfsetstrokeopacity{0.800000}%
\pgfsetdash{}{0pt}%
\pgfpathmoveto{\pgfqpoint{6.442953in}{5.837678in}}%
\pgfpathlineto{\pgfqpoint{6.279485in}{4.087478in}}%
\pgfusepath{stroke}%
\end{pgfscope}%
\begin{pgfscope}%
\pgfpathrectangle{\pgfqpoint{0.570343in}{0.331635in}}{\pgfqpoint{9.300000in}{7.700000in}}%
\pgfusepath{clip}%
\pgfsetrectcap%
\pgfsetroundjoin%
\pgfsetlinewidth{1.505625pt}%
\definecolor{currentstroke}{rgb}{1.000000,0.705882,0.509804}%
\pgfsetstrokecolor{currentstroke}%
\pgfsetstrokeopacity{0.800000}%
\pgfsetdash{}{0pt}%
\pgfpathmoveto{\pgfqpoint{6.303552in}{3.055860in}}%
\pgfpathlineto{\pgfqpoint{6.279485in}{4.087478in}}%
\pgfusepath{stroke}%
\end{pgfscope}%
\begin{pgfscope}%
\pgfpathrectangle{\pgfqpoint{0.570343in}{0.331635in}}{\pgfqpoint{9.300000in}{7.700000in}}%
\pgfusepath{clip}%
\pgfsetrectcap%
\pgfsetroundjoin%
\pgfsetlinewidth{1.505625pt}%
\definecolor{currentstroke}{rgb}{1.000000,0.705882,0.509804}%
\pgfsetstrokecolor{currentstroke}%
\pgfsetstrokeopacity{0.800000}%
\pgfsetdash{}{0pt}%
\pgfpathmoveto{\pgfqpoint{6.657819in}{4.310291in}}%
\pgfpathlineto{\pgfqpoint{6.279485in}{4.087478in}}%
\pgfusepath{stroke}%
\end{pgfscope}%
\begin{pgfscope}%
\pgfpathrectangle{\pgfqpoint{0.570343in}{0.331635in}}{\pgfqpoint{9.300000in}{7.700000in}}%
\pgfusepath{clip}%
\pgfsetrectcap%
\pgfsetroundjoin%
\pgfsetlinewidth{1.505625pt}%
\definecolor{currentstroke}{rgb}{1.000000,0.705882,0.509804}%
\pgfsetstrokecolor{currentstroke}%
\pgfsetstrokeopacity{0.800000}%
\pgfsetdash{}{0pt}%
\pgfpathmoveto{\pgfqpoint{6.208605in}{2.020129in}}%
\pgfpathlineto{\pgfqpoint{6.279485in}{4.087478in}}%
\pgfusepath{stroke}%
\end{pgfscope}%
\begin{pgfscope}%
\pgfpathrectangle{\pgfqpoint{0.570343in}{0.331635in}}{\pgfqpoint{9.300000in}{7.700000in}}%
\pgfusepath{clip}%
\pgfsetrectcap%
\pgfsetroundjoin%
\pgfsetlinewidth{1.505625pt}%
\definecolor{currentstroke}{rgb}{1.000000,0.705882,0.509804}%
\pgfsetstrokecolor{currentstroke}%
\pgfsetstrokeopacity{0.800000}%
\pgfsetdash{}{0pt}%
\pgfpathmoveto{\pgfqpoint{6.335839in}{5.394024in}}%
\pgfpathlineto{\pgfqpoint{6.279485in}{4.087478in}}%
\pgfusepath{stroke}%
\end{pgfscope}%
\begin{pgfscope}%
\pgfpathrectangle{\pgfqpoint{0.570343in}{0.331635in}}{\pgfqpoint{9.300000in}{7.700000in}}%
\pgfusepath{clip}%
\pgfsetrectcap%
\pgfsetroundjoin%
\pgfsetlinewidth{1.505625pt}%
\definecolor{currentstroke}{rgb}{1.000000,0.705882,0.509804}%
\pgfsetstrokecolor{currentstroke}%
\pgfsetstrokeopacity{0.800000}%
\pgfsetdash{}{0pt}%
\pgfpathmoveto{\pgfqpoint{7.346402in}{3.742692in}}%
\pgfpathlineto{\pgfqpoint{6.279485in}{4.087478in}}%
\pgfusepath{stroke}%
\end{pgfscope}%
\begin{pgfscope}%
\pgfpathrectangle{\pgfqpoint{0.570343in}{0.331635in}}{\pgfqpoint{9.300000in}{7.700000in}}%
\pgfusepath{clip}%
\pgfsetrectcap%
\pgfsetroundjoin%
\pgfsetlinewidth{1.505625pt}%
\definecolor{currentstroke}{rgb}{1.000000,0.705882,0.509804}%
\pgfsetstrokecolor{currentstroke}%
\pgfsetstrokeopacity{0.800000}%
\pgfsetdash{}{0pt}%
\pgfpathmoveto{\pgfqpoint{6.089195in}{5.095160in}}%
\pgfpathlineto{\pgfqpoint{6.279485in}{4.087478in}}%
\pgfusepath{stroke}%
\end{pgfscope}%
\begin{pgfscope}%
\pgfpathrectangle{\pgfqpoint{0.570343in}{0.331635in}}{\pgfqpoint{9.300000in}{7.700000in}}%
\pgfusepath{clip}%
\pgfsetrectcap%
\pgfsetroundjoin%
\pgfsetlinewidth{1.505625pt}%
\definecolor{currentstroke}{rgb}{1.000000,0.705882,0.509804}%
\pgfsetstrokecolor{currentstroke}%
\pgfsetstrokeopacity{0.800000}%
\pgfsetdash{}{0pt}%
\pgfpathmoveto{\pgfqpoint{6.210432in}{5.191605in}}%
\pgfpathlineto{\pgfqpoint{6.279485in}{4.087478in}}%
\pgfusepath{stroke}%
\end{pgfscope}%
\begin{pgfscope}%
\pgfpathrectangle{\pgfqpoint{0.570343in}{0.331635in}}{\pgfqpoint{9.300000in}{7.700000in}}%
\pgfusepath{clip}%
\pgfsetrectcap%
\pgfsetroundjoin%
\pgfsetlinewidth{1.505625pt}%
\definecolor{currentstroke}{rgb}{1.000000,0.705882,0.509804}%
\pgfsetstrokecolor{currentstroke}%
\pgfsetstrokeopacity{0.800000}%
\pgfsetdash{}{0pt}%
\pgfpathmoveto{\pgfqpoint{0.993071in}{4.948857in}}%
\pgfpathlineto{\pgfqpoint{6.279485in}{4.087478in}}%
\pgfusepath{stroke}%
\end{pgfscope}%
\begin{pgfscope}%
\pgfpathrectangle{\pgfqpoint{0.570343in}{0.331635in}}{\pgfqpoint{9.300000in}{7.700000in}}%
\pgfusepath{clip}%
\pgfsetrectcap%
\pgfsetroundjoin%
\pgfsetlinewidth{1.505625pt}%
\definecolor{currentstroke}{rgb}{1.000000,0.705882,0.509804}%
\pgfsetstrokecolor{currentstroke}%
\pgfsetstrokeopacity{0.800000}%
\pgfsetdash{}{0pt}%
\pgfpathmoveto{\pgfqpoint{6.146652in}{5.080980in}}%
\pgfpathlineto{\pgfqpoint{6.279485in}{4.087478in}}%
\pgfusepath{stroke}%
\end{pgfscope}%
\begin{pgfscope}%
\pgfpathrectangle{\pgfqpoint{0.570343in}{0.331635in}}{\pgfqpoint{9.300000in}{7.700000in}}%
\pgfusepath{clip}%
\pgfsetrectcap%
\pgfsetroundjoin%
\pgfsetlinewidth{1.505625pt}%
\definecolor{currentstroke}{rgb}{1.000000,0.705882,0.509804}%
\pgfsetstrokecolor{currentstroke}%
\pgfsetstrokeopacity{0.800000}%
\pgfsetdash{}{0pt}%
\pgfpathmoveto{\pgfqpoint{6.252198in}{5.825743in}}%
\pgfpathlineto{\pgfqpoint{6.279485in}{4.087478in}}%
\pgfusepath{stroke}%
\end{pgfscope}%
\begin{pgfscope}%
\pgfpathrectangle{\pgfqpoint{0.570343in}{0.331635in}}{\pgfqpoint{9.300000in}{7.700000in}}%
\pgfusepath{clip}%
\pgfsetrectcap%
\pgfsetroundjoin%
\pgfsetlinewidth{1.505625pt}%
\definecolor{currentstroke}{rgb}{1.000000,0.705882,0.509804}%
\pgfsetstrokecolor{currentstroke}%
\pgfsetstrokeopacity{0.800000}%
\pgfsetdash{}{0pt}%
\pgfpathmoveto{\pgfqpoint{6.265171in}{3.613538in}}%
\pgfpathlineto{\pgfqpoint{6.279485in}{4.087478in}}%
\pgfusepath{stroke}%
\end{pgfscope}%
\begin{pgfscope}%
\pgfpathrectangle{\pgfqpoint{0.570343in}{0.331635in}}{\pgfqpoint{9.300000in}{7.700000in}}%
\pgfusepath{clip}%
\pgfsetrectcap%
\pgfsetroundjoin%
\pgfsetlinewidth{1.505625pt}%
\definecolor{currentstroke}{rgb}{1.000000,0.705882,0.509804}%
\pgfsetstrokecolor{currentstroke}%
\pgfsetstrokeopacity{0.800000}%
\pgfsetdash{}{0pt}%
\pgfpathmoveto{\pgfqpoint{6.235593in}{4.091144in}}%
\pgfpathlineto{\pgfqpoint{6.279485in}{4.087478in}}%
\pgfusepath{stroke}%
\end{pgfscope}%
\begin{pgfscope}%
\pgfpathrectangle{\pgfqpoint{0.570343in}{0.331635in}}{\pgfqpoint{9.300000in}{7.700000in}}%
\pgfusepath{clip}%
\pgfsetrectcap%
\pgfsetroundjoin%
\pgfsetlinewidth{1.505625pt}%
\definecolor{currentstroke}{rgb}{1.000000,0.705882,0.509804}%
\pgfsetstrokecolor{currentstroke}%
\pgfsetstrokeopacity{0.800000}%
\pgfsetdash{}{0pt}%
\pgfpathmoveto{\pgfqpoint{5.626465in}{1.821398in}}%
\pgfpathlineto{\pgfqpoint{6.279485in}{4.087478in}}%
\pgfusepath{stroke}%
\end{pgfscope}%
\begin{pgfscope}%
\pgfpathrectangle{\pgfqpoint{0.570343in}{0.331635in}}{\pgfqpoint{9.300000in}{7.700000in}}%
\pgfusepath{clip}%
\pgfsetrectcap%
\pgfsetroundjoin%
\pgfsetlinewidth{1.505625pt}%
\definecolor{currentstroke}{rgb}{1.000000,0.705882,0.509804}%
\pgfsetstrokecolor{currentstroke}%
\pgfsetstrokeopacity{0.800000}%
\pgfsetdash{}{0pt}%
\pgfpathmoveto{\pgfqpoint{6.206878in}{3.687550in}}%
\pgfpathlineto{\pgfqpoint{6.279485in}{4.087478in}}%
\pgfusepath{stroke}%
\end{pgfscope}%
\begin{pgfscope}%
\pgfpathrectangle{\pgfqpoint{0.570343in}{0.331635in}}{\pgfqpoint{9.300000in}{7.700000in}}%
\pgfusepath{clip}%
\pgfsetrectcap%
\pgfsetroundjoin%
\pgfsetlinewidth{1.505625pt}%
\definecolor{currentstroke}{rgb}{1.000000,0.705882,0.509804}%
\pgfsetstrokecolor{currentstroke}%
\pgfsetstrokeopacity{0.800000}%
\pgfsetdash{}{0pt}%
\pgfpathmoveto{\pgfqpoint{6.595639in}{2.761237in}}%
\pgfpathlineto{\pgfqpoint{6.279485in}{4.087478in}}%
\pgfusepath{stroke}%
\end{pgfscope}%
\begin{pgfscope}%
\pgfpathrectangle{\pgfqpoint{0.570343in}{0.331635in}}{\pgfqpoint{9.300000in}{7.700000in}}%
\pgfusepath{clip}%
\pgfsetrectcap%
\pgfsetroundjoin%
\pgfsetlinewidth{1.505625pt}%
\definecolor{currentstroke}{rgb}{1.000000,0.705882,0.509804}%
\pgfsetstrokecolor{currentstroke}%
\pgfsetstrokeopacity{0.800000}%
\pgfsetdash{}{0pt}%
\pgfpathmoveto{\pgfqpoint{6.543720in}{5.606423in}}%
\pgfpathlineto{\pgfqpoint{6.279485in}{4.087478in}}%
\pgfusepath{stroke}%
\end{pgfscope}%
\begin{pgfscope}%
\pgfpathrectangle{\pgfqpoint{0.570343in}{0.331635in}}{\pgfqpoint{9.300000in}{7.700000in}}%
\pgfusepath{clip}%
\pgfsetrectcap%
\pgfsetroundjoin%
\pgfsetlinewidth{1.505625pt}%
\definecolor{currentstroke}{rgb}{1.000000,0.705882,0.509804}%
\pgfsetstrokecolor{currentstroke}%
\pgfsetstrokeopacity{0.800000}%
\pgfsetdash{}{0pt}%
\pgfpathmoveto{\pgfqpoint{6.147267in}{5.534029in}}%
\pgfpathlineto{\pgfqpoint{6.279485in}{4.087478in}}%
\pgfusepath{stroke}%
\end{pgfscope}%
\begin{pgfscope}%
\pgfpathrectangle{\pgfqpoint{0.570343in}{0.331635in}}{\pgfqpoint{9.300000in}{7.700000in}}%
\pgfusepath{clip}%
\pgfsetrectcap%
\pgfsetroundjoin%
\pgfsetlinewidth{1.505625pt}%
\definecolor{currentstroke}{rgb}{1.000000,0.705882,0.509804}%
\pgfsetstrokecolor{currentstroke}%
\pgfsetstrokeopacity{0.800000}%
\pgfsetdash{}{0pt}%
\pgfpathmoveto{\pgfqpoint{6.478915in}{2.463296in}}%
\pgfpathlineto{\pgfqpoint{6.279485in}{4.087478in}}%
\pgfusepath{stroke}%
\end{pgfscope}%
\begin{pgfscope}%
\pgfpathrectangle{\pgfqpoint{0.570343in}{0.331635in}}{\pgfqpoint{9.300000in}{7.700000in}}%
\pgfusepath{clip}%
\pgfsetrectcap%
\pgfsetroundjoin%
\pgfsetlinewidth{1.505625pt}%
\definecolor{currentstroke}{rgb}{1.000000,0.705882,0.509804}%
\pgfsetstrokecolor{currentstroke}%
\pgfsetstrokeopacity{0.800000}%
\pgfsetdash{}{0pt}%
\pgfpathmoveto{\pgfqpoint{6.474508in}{4.952163in}}%
\pgfpathlineto{\pgfqpoint{6.279485in}{4.087478in}}%
\pgfusepath{stroke}%
\end{pgfscope}%
\begin{pgfscope}%
\pgfpathrectangle{\pgfqpoint{0.570343in}{0.331635in}}{\pgfqpoint{9.300000in}{7.700000in}}%
\pgfusepath{clip}%
\pgfsetrectcap%
\pgfsetroundjoin%
\pgfsetlinewidth{1.505625pt}%
\definecolor{currentstroke}{rgb}{1.000000,0.705882,0.509804}%
\pgfsetstrokecolor{currentstroke}%
\pgfsetstrokeopacity{0.800000}%
\pgfsetdash{}{0pt}%
\pgfpathmoveto{\pgfqpoint{6.197218in}{4.751164in}}%
\pgfpathlineto{\pgfqpoint{6.279485in}{4.087478in}}%
\pgfusepath{stroke}%
\end{pgfscope}%
\begin{pgfscope}%
\pgfpathrectangle{\pgfqpoint{0.570343in}{0.331635in}}{\pgfqpoint{9.300000in}{7.700000in}}%
\pgfusepath{clip}%
\pgfsetrectcap%
\pgfsetroundjoin%
\pgfsetlinewidth{1.505625pt}%
\definecolor{currentstroke}{rgb}{1.000000,0.705882,0.509804}%
\pgfsetstrokecolor{currentstroke}%
\pgfsetstrokeopacity{0.800000}%
\pgfsetdash{}{0pt}%
\pgfpathmoveto{\pgfqpoint{5.938120in}{3.878488in}}%
\pgfpathlineto{\pgfqpoint{6.279485in}{4.087478in}}%
\pgfusepath{stroke}%
\end{pgfscope}%
\begin{pgfscope}%
\pgfpathrectangle{\pgfqpoint{0.570343in}{0.331635in}}{\pgfqpoint{9.300000in}{7.700000in}}%
\pgfusepath{clip}%
\pgfsetrectcap%
\pgfsetroundjoin%
\pgfsetlinewidth{1.505625pt}%
\definecolor{currentstroke}{rgb}{1.000000,0.705882,0.509804}%
\pgfsetstrokecolor{currentstroke}%
\pgfsetstrokeopacity{0.800000}%
\pgfsetdash{}{0pt}%
\pgfpathmoveto{\pgfqpoint{6.148378in}{1.324357in}}%
\pgfpathlineto{\pgfqpoint{6.279485in}{4.087478in}}%
\pgfusepath{stroke}%
\end{pgfscope}%
\begin{pgfscope}%
\pgfpathrectangle{\pgfqpoint{0.570343in}{0.331635in}}{\pgfqpoint{9.300000in}{7.700000in}}%
\pgfusepath{clip}%
\pgfsetrectcap%
\pgfsetroundjoin%
\pgfsetlinewidth{1.505625pt}%
\definecolor{currentstroke}{rgb}{1.000000,0.705882,0.509804}%
\pgfsetstrokecolor{currentstroke}%
\pgfsetstrokeopacity{0.800000}%
\pgfsetdash{}{0pt}%
\pgfpathmoveto{\pgfqpoint{6.264144in}{5.298682in}}%
\pgfpathlineto{\pgfqpoint{6.279485in}{4.087478in}}%
\pgfusepath{stroke}%
\end{pgfscope}%
\begin{pgfscope}%
\pgfpathrectangle{\pgfqpoint{0.570343in}{0.331635in}}{\pgfqpoint{9.300000in}{7.700000in}}%
\pgfusepath{clip}%
\pgfsetrectcap%
\pgfsetroundjoin%
\pgfsetlinewidth{1.505625pt}%
\definecolor{currentstroke}{rgb}{1.000000,0.705882,0.509804}%
\pgfsetstrokecolor{currentstroke}%
\pgfsetstrokeopacity{0.800000}%
\pgfsetdash{}{0pt}%
\pgfpathmoveto{\pgfqpoint{6.250398in}{2.295041in}}%
\pgfpathlineto{\pgfqpoint{6.279485in}{4.087478in}}%
\pgfusepath{stroke}%
\end{pgfscope}%
\begin{pgfscope}%
\pgfpathrectangle{\pgfqpoint{0.570343in}{0.331635in}}{\pgfqpoint{9.300000in}{7.700000in}}%
\pgfusepath{clip}%
\pgfsetrectcap%
\pgfsetroundjoin%
\pgfsetlinewidth{1.505625pt}%
\definecolor{currentstroke}{rgb}{1.000000,0.705882,0.509804}%
\pgfsetstrokecolor{currentstroke}%
\pgfsetstrokeopacity{0.800000}%
\pgfsetdash{}{0pt}%
\pgfpathmoveto{\pgfqpoint{6.335293in}{7.013812in}}%
\pgfpathlineto{\pgfqpoint{6.279485in}{4.087478in}}%
\pgfusepath{stroke}%
\end{pgfscope}%
\begin{pgfscope}%
\pgfpathrectangle{\pgfqpoint{0.570343in}{0.331635in}}{\pgfqpoint{9.300000in}{7.700000in}}%
\pgfusepath{clip}%
\pgfsetrectcap%
\pgfsetroundjoin%
\pgfsetlinewidth{1.505625pt}%
\definecolor{currentstroke}{rgb}{1.000000,0.705882,0.509804}%
\pgfsetstrokecolor{currentstroke}%
\pgfsetstrokeopacity{0.800000}%
\pgfsetdash{}{0pt}%
\pgfpathmoveto{\pgfqpoint{9.447616in}{2.070739in}}%
\pgfpathlineto{\pgfqpoint{6.279485in}{4.087478in}}%
\pgfusepath{stroke}%
\end{pgfscope}%
\begin{pgfscope}%
\pgfpathrectangle{\pgfqpoint{0.570343in}{0.331635in}}{\pgfqpoint{9.300000in}{7.700000in}}%
\pgfusepath{clip}%
\pgfsetrectcap%
\pgfsetroundjoin%
\pgfsetlinewidth{1.505625pt}%
\definecolor{currentstroke}{rgb}{1.000000,0.705882,0.509804}%
\pgfsetstrokecolor{currentstroke}%
\pgfsetstrokeopacity{0.800000}%
\pgfsetdash{}{0pt}%
\pgfpathmoveto{\pgfqpoint{6.163528in}{3.896552in}}%
\pgfpathlineto{\pgfqpoint{6.279485in}{4.087478in}}%
\pgfusepath{stroke}%
\end{pgfscope}%
\begin{pgfscope}%
\pgfpathrectangle{\pgfqpoint{0.570343in}{0.331635in}}{\pgfqpoint{9.300000in}{7.700000in}}%
\pgfusepath{clip}%
\pgfsetrectcap%
\pgfsetroundjoin%
\pgfsetlinewidth{1.505625pt}%
\definecolor{currentstroke}{rgb}{1.000000,0.705882,0.509804}%
\pgfsetstrokecolor{currentstroke}%
\pgfsetstrokeopacity{0.800000}%
\pgfsetdash{}{0pt}%
\pgfpathmoveto{\pgfqpoint{6.074829in}{5.559960in}}%
\pgfpathlineto{\pgfqpoint{6.279485in}{4.087478in}}%
\pgfusepath{stroke}%
\end{pgfscope}%
\begin{pgfscope}%
\pgfpathrectangle{\pgfqpoint{0.570343in}{0.331635in}}{\pgfqpoint{9.300000in}{7.700000in}}%
\pgfusepath{clip}%
\pgfsetrectcap%
\pgfsetroundjoin%
\pgfsetlinewidth{1.505625pt}%
\definecolor{currentstroke}{rgb}{1.000000,0.705882,0.509804}%
\pgfsetstrokecolor{currentstroke}%
\pgfsetstrokeopacity{0.800000}%
\pgfsetdash{}{0pt}%
\pgfpathmoveto{\pgfqpoint{6.249575in}{4.595461in}}%
\pgfpathlineto{\pgfqpoint{6.279485in}{4.087478in}}%
\pgfusepath{stroke}%
\end{pgfscope}%
\begin{pgfscope}%
\pgfpathrectangle{\pgfqpoint{0.570343in}{0.331635in}}{\pgfqpoint{9.300000in}{7.700000in}}%
\pgfusepath{clip}%
\pgfsetrectcap%
\pgfsetroundjoin%
\pgfsetlinewidth{1.505625pt}%
\definecolor{currentstroke}{rgb}{1.000000,0.705882,0.509804}%
\pgfsetstrokecolor{currentstroke}%
\pgfsetstrokeopacity{0.800000}%
\pgfsetdash{}{0pt}%
\pgfpathmoveto{\pgfqpoint{6.295225in}{2.309091in}}%
\pgfpathlineto{\pgfqpoint{6.279485in}{4.087478in}}%
\pgfusepath{stroke}%
\end{pgfscope}%
\begin{pgfscope}%
\pgfpathrectangle{\pgfqpoint{0.570343in}{0.331635in}}{\pgfqpoint{9.300000in}{7.700000in}}%
\pgfusepath{clip}%
\pgfsetrectcap%
\pgfsetroundjoin%
\pgfsetlinewidth{1.505625pt}%
\definecolor{currentstroke}{rgb}{1.000000,0.705882,0.509804}%
\pgfsetstrokecolor{currentstroke}%
\pgfsetstrokeopacity{0.800000}%
\pgfsetdash{}{0pt}%
\pgfpathmoveto{\pgfqpoint{6.124439in}{4.538027in}}%
\pgfpathlineto{\pgfqpoint{6.279485in}{4.087478in}}%
\pgfusepath{stroke}%
\end{pgfscope}%
\begin{pgfscope}%
\pgfpathrectangle{\pgfqpoint{0.570343in}{0.331635in}}{\pgfqpoint{9.300000in}{7.700000in}}%
\pgfusepath{clip}%
\pgfsetrectcap%
\pgfsetroundjoin%
\pgfsetlinewidth{1.505625pt}%
\definecolor{currentstroke}{rgb}{1.000000,0.705882,0.509804}%
\pgfsetstrokecolor{currentstroke}%
\pgfsetstrokeopacity{0.800000}%
\pgfsetdash{}{0pt}%
\pgfpathmoveto{\pgfqpoint{6.053358in}{6.362264in}}%
\pgfpathlineto{\pgfqpoint{6.279485in}{4.087478in}}%
\pgfusepath{stroke}%
\end{pgfscope}%
\begin{pgfscope}%
\pgfpathrectangle{\pgfqpoint{0.570343in}{0.331635in}}{\pgfqpoint{9.300000in}{7.700000in}}%
\pgfusepath{clip}%
\pgfsetrectcap%
\pgfsetroundjoin%
\pgfsetlinewidth{1.505625pt}%
\definecolor{currentstroke}{rgb}{1.000000,0.705882,0.509804}%
\pgfsetstrokecolor{currentstroke}%
\pgfsetstrokeopacity{0.800000}%
\pgfsetdash{}{0pt}%
\pgfpathmoveto{\pgfqpoint{6.175797in}{4.281858in}}%
\pgfpathlineto{\pgfqpoint{6.279485in}{4.087478in}}%
\pgfusepath{stroke}%
\end{pgfscope}%
\begin{pgfscope}%
\pgfpathrectangle{\pgfqpoint{0.570343in}{0.331635in}}{\pgfqpoint{9.300000in}{7.700000in}}%
\pgfusepath{clip}%
\pgfsetrectcap%
\pgfsetroundjoin%
\pgfsetlinewidth{1.505625pt}%
\definecolor{currentstroke}{rgb}{1.000000,0.705882,0.509804}%
\pgfsetstrokecolor{currentstroke}%
\pgfsetstrokeopacity{0.800000}%
\pgfsetdash{}{0pt}%
\pgfpathmoveto{\pgfqpoint{6.585001in}{1.419315in}}%
\pgfpathlineto{\pgfqpoint{6.279485in}{4.087478in}}%
\pgfusepath{stroke}%
\end{pgfscope}%
\begin{pgfscope}%
\pgfpathrectangle{\pgfqpoint{0.570343in}{0.331635in}}{\pgfqpoint{9.300000in}{7.700000in}}%
\pgfusepath{clip}%
\pgfsetrectcap%
\pgfsetroundjoin%
\pgfsetlinewidth{1.505625pt}%
\definecolor{currentstroke}{rgb}{1.000000,0.705882,0.509804}%
\pgfsetstrokecolor{currentstroke}%
\pgfsetstrokeopacity{0.800000}%
\pgfsetdash{}{0pt}%
\pgfpathmoveto{\pgfqpoint{6.280311in}{4.942018in}}%
\pgfpathlineto{\pgfqpoint{6.279485in}{4.087478in}}%
\pgfusepath{stroke}%
\end{pgfscope}%
\begin{pgfscope}%
\pgfpathrectangle{\pgfqpoint{0.570343in}{0.331635in}}{\pgfqpoint{9.300000in}{7.700000in}}%
\pgfusepath{clip}%
\pgfsetrectcap%
\pgfsetroundjoin%
\pgfsetlinewidth{1.505625pt}%
\definecolor{currentstroke}{rgb}{1.000000,0.705882,0.509804}%
\pgfsetstrokecolor{currentstroke}%
\pgfsetstrokeopacity{0.800000}%
\pgfsetdash{}{0pt}%
\pgfpathmoveto{\pgfqpoint{6.490755in}{2.851961in}}%
\pgfpathlineto{\pgfqpoint{6.279485in}{4.087478in}}%
\pgfusepath{stroke}%
\end{pgfscope}%
\begin{pgfscope}%
\pgfpathrectangle{\pgfqpoint{0.570343in}{0.331635in}}{\pgfqpoint{9.300000in}{7.700000in}}%
\pgfusepath{clip}%
\pgfsetrectcap%
\pgfsetroundjoin%
\pgfsetlinewidth{1.505625pt}%
\definecolor{currentstroke}{rgb}{1.000000,0.705882,0.509804}%
\pgfsetstrokecolor{currentstroke}%
\pgfsetstrokeopacity{0.800000}%
\pgfsetdash{}{0pt}%
\pgfpathmoveto{\pgfqpoint{6.415715in}{5.409020in}}%
\pgfpathlineto{\pgfqpoint{6.279485in}{4.087478in}}%
\pgfusepath{stroke}%
\end{pgfscope}%
\begin{pgfscope}%
\pgfpathrectangle{\pgfqpoint{0.570343in}{0.331635in}}{\pgfqpoint{9.300000in}{7.700000in}}%
\pgfusepath{clip}%
\pgfsetrectcap%
\pgfsetroundjoin%
\pgfsetlinewidth{1.505625pt}%
\definecolor{currentstroke}{rgb}{1.000000,0.705882,0.509804}%
\pgfsetstrokecolor{currentstroke}%
\pgfsetstrokeopacity{0.800000}%
\pgfsetdash{}{0pt}%
\pgfpathmoveto{\pgfqpoint{6.578206in}{3.619867in}}%
\pgfpathlineto{\pgfqpoint{6.279485in}{4.087478in}}%
\pgfusepath{stroke}%
\end{pgfscope}%
\begin{pgfscope}%
\pgfpathrectangle{\pgfqpoint{0.570343in}{0.331635in}}{\pgfqpoint{9.300000in}{7.700000in}}%
\pgfusepath{clip}%
\pgfsetrectcap%
\pgfsetroundjoin%
\pgfsetlinewidth{1.505625pt}%
\definecolor{currentstroke}{rgb}{1.000000,0.705882,0.509804}%
\pgfsetstrokecolor{currentstroke}%
\pgfsetstrokeopacity{0.800000}%
\pgfsetdash{}{0pt}%
\pgfpathmoveto{\pgfqpoint{6.570870in}{4.691179in}}%
\pgfpathlineto{\pgfqpoint{6.279485in}{4.087478in}}%
\pgfusepath{stroke}%
\end{pgfscope}%
\begin{pgfscope}%
\pgfpathrectangle{\pgfqpoint{0.570343in}{0.331635in}}{\pgfqpoint{9.300000in}{7.700000in}}%
\pgfusepath{clip}%
\pgfsetrectcap%
\pgfsetroundjoin%
\pgfsetlinewidth{1.505625pt}%
\definecolor{currentstroke}{rgb}{1.000000,0.705882,0.509804}%
\pgfsetstrokecolor{currentstroke}%
\pgfsetstrokeopacity{0.800000}%
\pgfsetdash{}{0pt}%
\pgfpathmoveto{\pgfqpoint{6.605909in}{5.201778in}}%
\pgfpathlineto{\pgfqpoint{6.279485in}{4.087478in}}%
\pgfusepath{stroke}%
\end{pgfscope}%
\begin{pgfscope}%
\pgfpathrectangle{\pgfqpoint{0.570343in}{0.331635in}}{\pgfqpoint{9.300000in}{7.700000in}}%
\pgfusepath{clip}%
\pgfsetrectcap%
\pgfsetroundjoin%
\pgfsetlinewidth{1.505625pt}%
\definecolor{currentstroke}{rgb}{1.000000,0.705882,0.509804}%
\pgfsetstrokecolor{currentstroke}%
\pgfsetstrokeopacity{0.800000}%
\pgfsetdash{}{0pt}%
\pgfpathmoveto{\pgfqpoint{6.543745in}{2.423372in}}%
\pgfpathlineto{\pgfqpoint{6.279485in}{4.087478in}}%
\pgfusepath{stroke}%
\end{pgfscope}%
\begin{pgfscope}%
\pgfpathrectangle{\pgfqpoint{0.570343in}{0.331635in}}{\pgfqpoint{9.300000in}{7.700000in}}%
\pgfusepath{clip}%
\pgfsetrectcap%
\pgfsetroundjoin%
\pgfsetlinewidth{1.505625pt}%
\definecolor{currentstroke}{rgb}{1.000000,0.705882,0.509804}%
\pgfsetstrokecolor{currentstroke}%
\pgfsetstrokeopacity{0.800000}%
\pgfsetdash{}{0pt}%
\pgfpathmoveto{\pgfqpoint{6.129683in}{0.681635in}}%
\pgfpathlineto{\pgfqpoint{6.279485in}{4.087478in}}%
\pgfusepath{stroke}%
\end{pgfscope}%
\begin{pgfscope}%
\pgfpathrectangle{\pgfqpoint{0.570343in}{0.331635in}}{\pgfqpoint{9.300000in}{7.700000in}}%
\pgfusepath{clip}%
\pgfsetrectcap%
\pgfsetroundjoin%
\pgfsetlinewidth{1.505625pt}%
\definecolor{currentstroke}{rgb}{1.000000,0.705882,0.509804}%
\pgfsetstrokecolor{currentstroke}%
\pgfsetstrokeopacity{0.800000}%
\pgfsetdash{}{0pt}%
\pgfpathmoveto{\pgfqpoint{6.658623in}{5.234270in}}%
\pgfpathlineto{\pgfqpoint{6.279485in}{4.087478in}}%
\pgfusepath{stroke}%
\end{pgfscope}%
\begin{pgfscope}%
\pgfpathrectangle{\pgfqpoint{0.570343in}{0.331635in}}{\pgfqpoint{9.300000in}{7.700000in}}%
\pgfusepath{clip}%
\pgfsetrectcap%
\pgfsetroundjoin%
\pgfsetlinewidth{1.505625pt}%
\definecolor{currentstroke}{rgb}{1.000000,0.705882,0.509804}%
\pgfsetstrokecolor{currentstroke}%
\pgfsetstrokeopacity{0.800000}%
\pgfsetdash{}{0pt}%
\pgfpathmoveto{\pgfqpoint{6.207206in}{5.550924in}}%
\pgfpathlineto{\pgfqpoint{6.279485in}{4.087478in}}%
\pgfusepath{stroke}%
\end{pgfscope}%
\begin{pgfscope}%
\pgfpathrectangle{\pgfqpoint{0.570343in}{0.331635in}}{\pgfqpoint{9.300000in}{7.700000in}}%
\pgfusepath{clip}%
\pgfsetrectcap%
\pgfsetroundjoin%
\pgfsetlinewidth{1.505625pt}%
\definecolor{currentstroke}{rgb}{1.000000,0.705882,0.509804}%
\pgfsetstrokecolor{currentstroke}%
\pgfsetstrokeopacity{0.800000}%
\pgfsetdash{}{0pt}%
\pgfpathmoveto{\pgfqpoint{6.593399in}{1.999542in}}%
\pgfpathlineto{\pgfqpoint{6.279485in}{4.087478in}}%
\pgfusepath{stroke}%
\end{pgfscope}%
\begin{pgfscope}%
\pgfpathrectangle{\pgfqpoint{0.570343in}{0.331635in}}{\pgfqpoint{9.300000in}{7.700000in}}%
\pgfusepath{clip}%
\pgfsetrectcap%
\pgfsetroundjoin%
\pgfsetlinewidth{1.505625pt}%
\definecolor{currentstroke}{rgb}{1.000000,0.705882,0.509804}%
\pgfsetstrokecolor{currentstroke}%
\pgfsetstrokeopacity{0.800000}%
\pgfsetdash{}{0pt}%
\pgfpathmoveto{\pgfqpoint{6.036485in}{5.771435in}}%
\pgfpathlineto{\pgfqpoint{6.279485in}{4.087478in}}%
\pgfusepath{stroke}%
\end{pgfscope}%
\begin{pgfscope}%
\pgfpathrectangle{\pgfqpoint{0.570343in}{0.331635in}}{\pgfqpoint{9.300000in}{7.700000in}}%
\pgfusepath{clip}%
\pgfsetrectcap%
\pgfsetroundjoin%
\pgfsetlinewidth{1.505625pt}%
\definecolor{currentstroke}{rgb}{1.000000,0.705882,0.509804}%
\pgfsetstrokecolor{currentstroke}%
\pgfsetstrokeopacity{0.800000}%
\pgfsetdash{}{0pt}%
\pgfpathmoveto{\pgfqpoint{6.492770in}{1.962199in}}%
\pgfpathlineto{\pgfqpoint{6.279485in}{4.087478in}}%
\pgfusepath{stroke}%
\end{pgfscope}%
\begin{pgfscope}%
\pgfpathrectangle{\pgfqpoint{0.570343in}{0.331635in}}{\pgfqpoint{9.300000in}{7.700000in}}%
\pgfusepath{clip}%
\pgfsetrectcap%
\pgfsetroundjoin%
\pgfsetlinewidth{1.505625pt}%
\definecolor{currentstroke}{rgb}{1.000000,0.705882,0.509804}%
\pgfsetstrokecolor{currentstroke}%
\pgfsetstrokeopacity{0.800000}%
\pgfsetdash{}{0pt}%
\pgfpathmoveto{\pgfqpoint{5.959958in}{6.300818in}}%
\pgfpathlineto{\pgfqpoint{6.279485in}{4.087478in}}%
\pgfusepath{stroke}%
\end{pgfscope}%
\begin{pgfscope}%
\pgfpathrectangle{\pgfqpoint{0.570343in}{0.331635in}}{\pgfqpoint{9.300000in}{7.700000in}}%
\pgfusepath{clip}%
\pgfsetrectcap%
\pgfsetroundjoin%
\pgfsetlinewidth{1.505625pt}%
\definecolor{currentstroke}{rgb}{1.000000,0.705882,0.509804}%
\pgfsetstrokecolor{currentstroke}%
\pgfsetstrokeopacity{0.800000}%
\pgfsetdash{}{0pt}%
\pgfpathmoveto{\pgfqpoint{6.546796in}{3.105272in}}%
\pgfpathlineto{\pgfqpoint{6.279485in}{4.087478in}}%
\pgfusepath{stroke}%
\end{pgfscope}%
\begin{pgfscope}%
\pgfsetrectcap%
\pgfsetmiterjoin%
\pgfsetlinewidth{0.803000pt}%
\definecolor{currentstroke}{rgb}{0.000000,0.000000,0.000000}%
\pgfsetstrokecolor{currentstroke}%
\pgfsetdash{}{0pt}%
\pgfpathmoveto{\pgfqpoint{0.570343in}{0.331635in}}%
\pgfpathlineto{\pgfqpoint{0.570343in}{8.031635in}}%
\pgfusepath{stroke}%
\end{pgfscope}%
\begin{pgfscope}%
\pgfsetrectcap%
\pgfsetmiterjoin%
\pgfsetlinewidth{0.803000pt}%
\definecolor{currentstroke}{rgb}{0.000000,0.000000,0.000000}%
\pgfsetstrokecolor{currentstroke}%
\pgfsetdash{}{0pt}%
\pgfpathmoveto{\pgfqpoint{9.870343in}{0.331635in}}%
\pgfpathlineto{\pgfqpoint{9.870343in}{8.031635in}}%
\pgfusepath{stroke}%
\end{pgfscope}%
\begin{pgfscope}%
\pgfsetrectcap%
\pgfsetmiterjoin%
\pgfsetlinewidth{0.803000pt}%
\definecolor{currentstroke}{rgb}{0.000000,0.000000,0.000000}%
\pgfsetstrokecolor{currentstroke}%
\pgfsetdash{}{0pt}%
\pgfpathmoveto{\pgfqpoint{0.570343in}{0.331635in}}%
\pgfpathlineto{\pgfqpoint{9.870343in}{0.331635in}}%
\pgfusepath{stroke}%
\end{pgfscope}%
\begin{pgfscope}%
\pgfsetrectcap%
\pgfsetmiterjoin%
\pgfsetlinewidth{0.803000pt}%
\definecolor{currentstroke}{rgb}{0.000000,0.000000,0.000000}%
\pgfsetstrokecolor{currentstroke}%
\pgfsetdash{}{0pt}%
\pgfpathmoveto{\pgfqpoint{0.570343in}{8.031635in}}%
\pgfpathlineto{\pgfqpoint{9.870343in}{8.031635in}}%
\pgfusepath{stroke}%
\end{pgfscope}%
\begin{pgfscope}%
\definecolor{textcolor}{rgb}{0.000000,0.000000,0.000000}%
\pgfsetstrokecolor{textcolor}%
\pgfsetfillcolor{textcolor}%
\pgftext[x=5.220343in,y=8.114968in,,base]{\color{textcolor}\sffamily\fontsize{12.000000}{14.400000}\selectfont T-SNE for chair images (s2r3dfree\_background\_light2)}%
\end{pgfscope}%
\begin{pgfscope}%
\pgfsetbuttcap%
\pgfsetmiterjoin%
\definecolor{currentfill}{rgb}{1.000000,1.000000,1.000000}%
\pgfsetfillcolor{currentfill}%
\pgfsetfillopacity{0.800000}%
\pgfsetlinewidth{1.003750pt}%
\definecolor{currentstroke}{rgb}{0.800000,0.800000,0.800000}%
\pgfsetstrokecolor{currentstroke}%
\pgfsetstrokeopacity{0.800000}%
\pgfsetdash{}{0pt}%
\pgfpathmoveto{\pgfqpoint{9.967566in}{3.955012in}}%
\pgfpathlineto{\pgfqpoint{12.436750in}{3.955012in}}%
\pgfpathquadraticcurveto{\pgfqpoint{12.464527in}{3.955012in}}{\pgfqpoint{12.464527in}{3.982789in}}%
\pgfpathlineto{\pgfqpoint{12.464527in}{4.380481in}}%
\pgfpathquadraticcurveto{\pgfqpoint{12.464527in}{4.408258in}}{\pgfqpoint{12.436750in}{4.408258in}}%
\pgfpathlineto{\pgfqpoint{9.967566in}{4.408258in}}%
\pgfpathquadraticcurveto{\pgfqpoint{9.939788in}{4.408258in}}{\pgfqpoint{9.939788in}{4.380481in}}%
\pgfpathlineto{\pgfqpoint{9.939788in}{3.982789in}}%
\pgfpathquadraticcurveto{\pgfqpoint{9.939788in}{3.955012in}}{\pgfqpoint{9.967566in}{3.955012in}}%
\pgfpathclose%
\pgfusepath{stroke,fill}%
\end{pgfscope}%
\begin{pgfscope}%
\pgfsetbuttcap%
\pgfsetroundjoin%
\definecolor{currentfill}{rgb}{0.631373,0.788235,0.956863}%
\pgfsetfillcolor{currentfill}%
\pgfsetlinewidth{1.003750pt}%
\definecolor{currentstroke}{rgb}{0.631373,0.788235,0.956863}%
\pgfsetstrokecolor{currentstroke}%
\pgfsetdash{}{0pt}%
\pgfsys@defobject{currentmarker}{\pgfqpoint{-0.041667in}{-0.041667in}}{\pgfqpoint{0.041667in}{0.041667in}}{%
\pgfpathmoveto{\pgfqpoint{0.000000in}{-0.041667in}}%
\pgfpathcurveto{\pgfqpoint{0.011050in}{-0.041667in}}{\pgfqpoint{0.021649in}{-0.037276in}}{\pgfqpoint{0.029463in}{-0.029463in}}%
\pgfpathcurveto{\pgfqpoint{0.037276in}{-0.021649in}}{\pgfqpoint{0.041667in}{-0.011050in}}{\pgfqpoint{0.041667in}{0.000000in}}%
\pgfpathcurveto{\pgfqpoint{0.041667in}{0.011050in}}{\pgfqpoint{0.037276in}{0.021649in}}{\pgfqpoint{0.029463in}{0.029463in}}%
\pgfpathcurveto{\pgfqpoint{0.021649in}{0.037276in}}{\pgfqpoint{0.011050in}{0.041667in}}{\pgfqpoint{0.000000in}{0.041667in}}%
\pgfpathcurveto{\pgfqpoint{-0.011050in}{0.041667in}}{\pgfqpoint{-0.021649in}{0.037276in}}{\pgfqpoint{-0.029463in}{0.029463in}}%
\pgfpathcurveto{\pgfqpoint{-0.037276in}{0.021649in}}{\pgfqpoint{-0.041667in}{0.011050in}}{\pgfqpoint{-0.041667in}{0.000000in}}%
\pgfpathcurveto{\pgfqpoint{-0.041667in}{-0.011050in}}{\pgfqpoint{-0.037276in}{-0.021649in}}{\pgfqpoint{-0.029463in}{-0.029463in}}%
\pgfpathcurveto{\pgfqpoint{-0.021649in}{-0.037276in}}{\pgfqpoint{-0.011050in}{-0.041667in}}{\pgfqpoint{0.000000in}{-0.041667in}}%
\pgfpathclose%
\pgfusepath{stroke,fill}%
}%
\begin{pgfscope}%
\pgfsys@transformshift{10.134232in}{4.283638in}%
\pgfsys@useobject{currentmarker}{}%
\end{pgfscope}%
\end{pgfscope}%
\begin{pgfscope}%
\definecolor{textcolor}{rgb}{0.000000,0.000000,0.000000}%
\pgfsetstrokecolor{textcolor}%
\pgfsetfillcolor{textcolor}%
\pgftext[x=10.384232in,y=4.247180in,left,base]{\color{textcolor}\sffamily\fontsize{10.000000}{12.000000}\selectfont Pix3D}%
\end{pgfscope}%
\begin{pgfscope}%
\pgfsetbuttcap%
\pgfsetroundjoin%
\definecolor{currentfill}{rgb}{1.000000,0.705882,0.509804}%
\pgfsetfillcolor{currentfill}%
\pgfsetlinewidth{1.003750pt}%
\definecolor{currentstroke}{rgb}{1.000000,0.705882,0.509804}%
\pgfsetstrokecolor{currentstroke}%
\pgfsetdash{}{0pt}%
\pgfsys@defobject{currentmarker}{\pgfqpoint{-0.041667in}{-0.041667in}}{\pgfqpoint{0.041667in}{0.041667in}}{%
\pgfpathmoveto{\pgfqpoint{0.000000in}{-0.041667in}}%
\pgfpathcurveto{\pgfqpoint{0.011050in}{-0.041667in}}{\pgfqpoint{0.021649in}{-0.037276in}}{\pgfqpoint{0.029463in}{-0.029463in}}%
\pgfpathcurveto{\pgfqpoint{0.037276in}{-0.021649in}}{\pgfqpoint{0.041667in}{-0.011050in}}{\pgfqpoint{0.041667in}{0.000000in}}%
\pgfpathcurveto{\pgfqpoint{0.041667in}{0.011050in}}{\pgfqpoint{0.037276in}{0.021649in}}{\pgfqpoint{0.029463in}{0.029463in}}%
\pgfpathcurveto{\pgfqpoint{0.021649in}{0.037276in}}{\pgfqpoint{0.011050in}{0.041667in}}{\pgfqpoint{0.000000in}{0.041667in}}%
\pgfpathcurveto{\pgfqpoint{-0.011050in}{0.041667in}}{\pgfqpoint{-0.021649in}{0.037276in}}{\pgfqpoint{-0.029463in}{0.029463in}}%
\pgfpathcurveto{\pgfqpoint{-0.037276in}{0.021649in}}{\pgfqpoint{-0.041667in}{0.011050in}}{\pgfqpoint{-0.041667in}{0.000000in}}%
\pgfpathcurveto{\pgfqpoint{-0.041667in}{-0.011050in}}{\pgfqpoint{-0.037276in}{-0.021649in}}{\pgfqpoint{-0.029463in}{-0.029463in}}%
\pgfpathcurveto{\pgfqpoint{-0.021649in}{-0.037276in}}{\pgfqpoint{-0.011050in}{-0.041667in}}{\pgfqpoint{0.000000in}{-0.041667in}}%
\pgfpathclose%
\pgfusepath{stroke,fill}%
}%
\begin{pgfscope}%
\pgfsys@transformshift{10.134232in}{4.079781in}%
\pgfsys@useobject{currentmarker}{}%
\end{pgfscope}%
\end{pgfscope}%
\begin{pgfscope}%
\definecolor{textcolor}{rgb}{0.000000,0.000000,0.000000}%
\pgfsetstrokecolor{textcolor}%
\pgfsetfillcolor{textcolor}%
\pgftext[x=10.384232in,y=4.043322in,left,base]{\color{textcolor}\sffamily\fontsize{10.000000}{12.000000}\selectfont s2r3dfree\_background\_light2}%
\end{pgfscope}%
\end{pgfpicture}%
\makeatother%
\endgroup%
}\\
    \resizebox{0.49\linewidth}{5cm}{%% Creator: Matplotlib, PGF backend
%%
%% To include the figure in your LaTeX document, write
%%   \input{<filename>.pgf}
%%
%% Make sure the required packages are loaded in your preamble
%%   \usepackage{pgf}
%%
%% Figures using additional raster images can only be included by \input if
%% they are in the same directory as the main LaTeX file. For loading figures
%% from other directories you can use the `import` package
%%   \usepackage{import}
%%
%% and then include the figures with
%%   \import{<path to file>}{<filename>.pgf}
%%
%% Matplotlib used the following preamble
%%   \usepackage{fontspec}
%%   \setmainfont{DejaVuSerif.ttf}[Path=\detokenize{/Users/apple/opt/anaconda3/envs/kaolin/lib/python3.7/site-packages/matplotlib/mpl-data/fonts/ttf/}]
%%   \setsansfont{DejaVuSans.ttf}[Path=\detokenize{/Users/apple/opt/anaconda3/envs/kaolin/lib/python3.7/site-packages/matplotlib/mpl-data/fonts/ttf/}]
%%   \setmonofont{DejaVuSansMono.ttf}[Path=\detokenize{/Users/apple/opt/anaconda3/envs/kaolin/lib/python3.7/site-packages/matplotlib/mpl-data/fonts/ttf/}]
%%
\begingroup%
\makeatletter%
\begin{pgfpicture}%
\pgfpathrectangle{\pgfpointorigin}{\pgfqpoint{11.622619in}{8.341596in}}%
\pgfusepath{use as bounding box, clip}%
\begin{pgfscope}%
\pgfsetbuttcap%
\pgfsetmiterjoin%
\definecolor{currentfill}{rgb}{1.000000,1.000000,1.000000}%
\pgfsetfillcolor{currentfill}%
\pgfsetlinewidth{0.000000pt}%
\definecolor{currentstroke}{rgb}{1.000000,1.000000,1.000000}%
\pgfsetstrokecolor{currentstroke}%
\pgfsetdash{}{0pt}%
\pgfpathmoveto{\pgfqpoint{0.000000in}{0.000000in}}%
\pgfpathlineto{\pgfqpoint{11.622619in}{0.000000in}}%
\pgfpathlineto{\pgfqpoint{11.622619in}{8.341596in}}%
\pgfpathlineto{\pgfqpoint{0.000000in}{8.341596in}}%
\pgfpathclose%
\pgfusepath{fill}%
\end{pgfscope}%
\begin{pgfscope}%
\pgfsetbuttcap%
\pgfsetmiterjoin%
\definecolor{currentfill}{rgb}{1.000000,1.000000,1.000000}%
\pgfsetfillcolor{currentfill}%
\pgfsetlinewidth{0.000000pt}%
\definecolor{currentstroke}{rgb}{0.000000,0.000000,0.000000}%
\pgfsetstrokecolor{currentstroke}%
\pgfsetstrokeopacity{0.000000}%
\pgfsetdash{}{0pt}%
\pgfpathmoveto{\pgfqpoint{0.570343in}{0.331635in}}%
\pgfpathlineto{\pgfqpoint{9.870343in}{0.331635in}}%
\pgfpathlineto{\pgfqpoint{9.870343in}{8.031635in}}%
\pgfpathlineto{\pgfqpoint{0.570343in}{8.031635in}}%
\pgfpathclose%
\pgfusepath{fill}%
\end{pgfscope}%
\begin{pgfscope}%
\pgfpathrectangle{\pgfqpoint{0.570343in}{0.331635in}}{\pgfqpoint{9.300000in}{7.700000in}}%
\pgfusepath{clip}%
\pgfsetbuttcap%
\pgfsetroundjoin%
\definecolor{currentfill}{rgb}{0.631373,0.788235,0.956863}%
\pgfsetfillcolor{currentfill}%
\pgfsetlinewidth{0.481800pt}%
\definecolor{currentstroke}{rgb}{1.000000,1.000000,1.000000}%
\pgfsetstrokecolor{currentstroke}%
\pgfsetdash{}{0pt}%
\pgfpathmoveto{\pgfqpoint{2.191177in}{2.268161in}}%
\pgfpathcurveto{\pgfqpoint{2.202227in}{2.268161in}}{\pgfqpoint{2.212826in}{2.272551in}}{\pgfqpoint{2.220640in}{2.280365in}}%
\pgfpathcurveto{\pgfqpoint{2.228454in}{2.288178in}}{\pgfqpoint{2.232844in}{2.298777in}}{\pgfqpoint{2.232844in}{2.309828in}}%
\pgfpathcurveto{\pgfqpoint{2.232844in}{2.320878in}}{\pgfqpoint{2.228454in}{2.331477in}}{\pgfqpoint{2.220640in}{2.339290in}}%
\pgfpathcurveto{\pgfqpoint{2.212826in}{2.347104in}}{\pgfqpoint{2.202227in}{2.351494in}}{\pgfqpoint{2.191177in}{2.351494in}}%
\pgfpathcurveto{\pgfqpoint{2.180127in}{2.351494in}}{\pgfqpoint{2.169528in}{2.347104in}}{\pgfqpoint{2.161714in}{2.339290in}}%
\pgfpathcurveto{\pgfqpoint{2.153901in}{2.331477in}}{\pgfqpoint{2.149511in}{2.320878in}}{\pgfqpoint{2.149511in}{2.309828in}}%
\pgfpathcurveto{\pgfqpoint{2.149511in}{2.298777in}}{\pgfqpoint{2.153901in}{2.288178in}}{\pgfqpoint{2.161714in}{2.280365in}}%
\pgfpathcurveto{\pgfqpoint{2.169528in}{2.272551in}}{\pgfqpoint{2.180127in}{2.268161in}}{\pgfqpoint{2.191177in}{2.268161in}}%
\pgfpathclose%
\pgfusepath{stroke,fill}%
\end{pgfscope}%
\begin{pgfscope}%
\pgfpathrectangle{\pgfqpoint{0.570343in}{0.331635in}}{\pgfqpoint{9.300000in}{7.700000in}}%
\pgfusepath{clip}%
\pgfsetbuttcap%
\pgfsetroundjoin%
\definecolor{currentfill}{rgb}{0.631373,0.788235,0.956863}%
\pgfsetfillcolor{currentfill}%
\pgfsetlinewidth{0.481800pt}%
\definecolor{currentstroke}{rgb}{1.000000,1.000000,1.000000}%
\pgfsetstrokecolor{currentstroke}%
\pgfsetdash{}{0pt}%
\pgfpathmoveto{\pgfqpoint{8.068189in}{5.118661in}}%
\pgfpathcurveto{\pgfqpoint{8.079240in}{5.118661in}}{\pgfqpoint{8.089839in}{5.123051in}}{\pgfqpoint{8.097652in}{5.130865in}}%
\pgfpathcurveto{\pgfqpoint{8.105466in}{5.138678in}}{\pgfqpoint{8.109856in}{5.149278in}}{\pgfqpoint{8.109856in}{5.160328in}}%
\pgfpathcurveto{\pgfqpoint{8.109856in}{5.171378in}}{\pgfqpoint{8.105466in}{5.181977in}}{\pgfqpoint{8.097652in}{5.189790in}}%
\pgfpathcurveto{\pgfqpoint{8.089839in}{5.197604in}}{\pgfqpoint{8.079240in}{5.201994in}}{\pgfqpoint{8.068189in}{5.201994in}}%
\pgfpathcurveto{\pgfqpoint{8.057139in}{5.201994in}}{\pgfqpoint{8.046540in}{5.197604in}}{\pgfqpoint{8.038727in}{5.189790in}}%
\pgfpathcurveto{\pgfqpoint{8.030913in}{5.181977in}}{\pgfqpoint{8.026523in}{5.171378in}}{\pgfqpoint{8.026523in}{5.160328in}}%
\pgfpathcurveto{\pgfqpoint{8.026523in}{5.149278in}}{\pgfqpoint{8.030913in}{5.138678in}}{\pgfqpoint{8.038727in}{5.130865in}}%
\pgfpathcurveto{\pgfqpoint{8.046540in}{5.123051in}}{\pgfqpoint{8.057139in}{5.118661in}}{\pgfqpoint{8.068189in}{5.118661in}}%
\pgfpathclose%
\pgfusepath{stroke,fill}%
\end{pgfscope}%
\begin{pgfscope}%
\pgfpathrectangle{\pgfqpoint{0.570343in}{0.331635in}}{\pgfqpoint{9.300000in}{7.700000in}}%
\pgfusepath{clip}%
\pgfsetbuttcap%
\pgfsetroundjoin%
\definecolor{currentfill}{rgb}{0.631373,0.788235,0.956863}%
\pgfsetfillcolor{currentfill}%
\pgfsetlinewidth{0.481800pt}%
\definecolor{currentstroke}{rgb}{1.000000,1.000000,1.000000}%
\pgfsetstrokecolor{currentstroke}%
\pgfsetdash{}{0pt}%
\pgfpathmoveto{\pgfqpoint{4.775457in}{3.718415in}}%
\pgfpathcurveto{\pgfqpoint{4.786507in}{3.718415in}}{\pgfqpoint{4.797106in}{3.722806in}}{\pgfqpoint{4.804920in}{3.730619in}}%
\pgfpathcurveto{\pgfqpoint{4.812733in}{3.738433in}}{\pgfqpoint{4.817124in}{3.749032in}}{\pgfqpoint{4.817124in}{3.760082in}}%
\pgfpathcurveto{\pgfqpoint{4.817124in}{3.771132in}}{\pgfqpoint{4.812733in}{3.781731in}}{\pgfqpoint{4.804920in}{3.789545in}}%
\pgfpathcurveto{\pgfqpoint{4.797106in}{3.797358in}}{\pgfqpoint{4.786507in}{3.801749in}}{\pgfqpoint{4.775457in}{3.801749in}}%
\pgfpathcurveto{\pgfqpoint{4.764407in}{3.801749in}}{\pgfqpoint{4.753808in}{3.797358in}}{\pgfqpoint{4.745994in}{3.789545in}}%
\pgfpathcurveto{\pgfqpoint{4.738181in}{3.781731in}}{\pgfqpoint{4.733790in}{3.771132in}}{\pgfqpoint{4.733790in}{3.760082in}}%
\pgfpathcurveto{\pgfqpoint{4.733790in}{3.749032in}}{\pgfqpoint{4.738181in}{3.738433in}}{\pgfqpoint{4.745994in}{3.730619in}}%
\pgfpathcurveto{\pgfqpoint{4.753808in}{3.722806in}}{\pgfqpoint{4.764407in}{3.718415in}}{\pgfqpoint{4.775457in}{3.718415in}}%
\pgfpathclose%
\pgfusepath{stroke,fill}%
\end{pgfscope}%
\begin{pgfscope}%
\pgfpathrectangle{\pgfqpoint{0.570343in}{0.331635in}}{\pgfqpoint{9.300000in}{7.700000in}}%
\pgfusepath{clip}%
\pgfsetbuttcap%
\pgfsetroundjoin%
\definecolor{currentfill}{rgb}{0.631373,0.788235,0.956863}%
\pgfsetfillcolor{currentfill}%
\pgfsetlinewidth{0.481800pt}%
\definecolor{currentstroke}{rgb}{1.000000,1.000000,1.000000}%
\pgfsetstrokecolor{currentstroke}%
\pgfsetdash{}{0pt}%
\pgfpathmoveto{\pgfqpoint{4.043501in}{4.342699in}}%
\pgfpathcurveto{\pgfqpoint{4.054551in}{4.342699in}}{\pgfqpoint{4.065150in}{4.347089in}}{\pgfqpoint{4.072964in}{4.354903in}}%
\pgfpathcurveto{\pgfqpoint{4.080777in}{4.362716in}}{\pgfqpoint{4.085168in}{4.373315in}}{\pgfqpoint{4.085168in}{4.384365in}}%
\pgfpathcurveto{\pgfqpoint{4.085168in}{4.395416in}}{\pgfqpoint{4.080777in}{4.406015in}}{\pgfqpoint{4.072964in}{4.413828in}}%
\pgfpathcurveto{\pgfqpoint{4.065150in}{4.421642in}}{\pgfqpoint{4.054551in}{4.426032in}}{\pgfqpoint{4.043501in}{4.426032in}}%
\pgfpathcurveto{\pgfqpoint{4.032451in}{4.426032in}}{\pgfqpoint{4.021852in}{4.421642in}}{\pgfqpoint{4.014038in}{4.413828in}}%
\pgfpathcurveto{\pgfqpoint{4.006225in}{4.406015in}}{\pgfqpoint{4.001834in}{4.395416in}}{\pgfqpoint{4.001834in}{4.384365in}}%
\pgfpathcurveto{\pgfqpoint{4.001834in}{4.373315in}}{\pgfqpoint{4.006225in}{4.362716in}}{\pgfqpoint{4.014038in}{4.354903in}}%
\pgfpathcurveto{\pgfqpoint{4.021852in}{4.347089in}}{\pgfqpoint{4.032451in}{4.342699in}}{\pgfqpoint{4.043501in}{4.342699in}}%
\pgfpathclose%
\pgfusepath{stroke,fill}%
\end{pgfscope}%
\begin{pgfscope}%
\pgfpathrectangle{\pgfqpoint{0.570343in}{0.331635in}}{\pgfqpoint{9.300000in}{7.700000in}}%
\pgfusepath{clip}%
\pgfsetbuttcap%
\pgfsetroundjoin%
\definecolor{currentfill}{rgb}{0.631373,0.788235,0.956863}%
\pgfsetfillcolor{currentfill}%
\pgfsetlinewidth{0.481800pt}%
\definecolor{currentstroke}{rgb}{1.000000,1.000000,1.000000}%
\pgfsetstrokecolor{currentstroke}%
\pgfsetdash{}{0pt}%
\pgfpathmoveto{\pgfqpoint{2.250247in}{2.787165in}}%
\pgfpathcurveto{\pgfqpoint{2.261297in}{2.787165in}}{\pgfqpoint{2.271896in}{2.791555in}}{\pgfqpoint{2.279710in}{2.799369in}}%
\pgfpathcurveto{\pgfqpoint{2.287524in}{2.807183in}}{\pgfqpoint{2.291914in}{2.817782in}}{\pgfqpoint{2.291914in}{2.828832in}}%
\pgfpathcurveto{\pgfqpoint{2.291914in}{2.839882in}}{\pgfqpoint{2.287524in}{2.850481in}}{\pgfqpoint{2.279710in}{2.858295in}}%
\pgfpathcurveto{\pgfqpoint{2.271896in}{2.866108in}}{\pgfqpoint{2.261297in}{2.870498in}}{\pgfqpoint{2.250247in}{2.870498in}}%
\pgfpathcurveto{\pgfqpoint{2.239197in}{2.870498in}}{\pgfqpoint{2.228598in}{2.866108in}}{\pgfqpoint{2.220784in}{2.858295in}}%
\pgfpathcurveto{\pgfqpoint{2.212971in}{2.850481in}}{\pgfqpoint{2.208581in}{2.839882in}}{\pgfqpoint{2.208581in}{2.828832in}}%
\pgfpathcurveto{\pgfqpoint{2.208581in}{2.817782in}}{\pgfqpoint{2.212971in}{2.807183in}}{\pgfqpoint{2.220784in}{2.799369in}}%
\pgfpathcurveto{\pgfqpoint{2.228598in}{2.791555in}}{\pgfqpoint{2.239197in}{2.787165in}}{\pgfqpoint{2.250247in}{2.787165in}}%
\pgfpathclose%
\pgfusepath{stroke,fill}%
\end{pgfscope}%
\begin{pgfscope}%
\pgfpathrectangle{\pgfqpoint{0.570343in}{0.331635in}}{\pgfqpoint{9.300000in}{7.700000in}}%
\pgfusepath{clip}%
\pgfsetbuttcap%
\pgfsetroundjoin%
\definecolor{currentfill}{rgb}{0.631373,0.788235,0.956863}%
\pgfsetfillcolor{currentfill}%
\pgfsetlinewidth{0.481800pt}%
\definecolor{currentstroke}{rgb}{1.000000,1.000000,1.000000}%
\pgfsetstrokecolor{currentstroke}%
\pgfsetdash{}{0pt}%
\pgfpathmoveto{\pgfqpoint{8.033060in}{5.563042in}}%
\pgfpathcurveto{\pgfqpoint{8.044110in}{5.563042in}}{\pgfqpoint{8.054709in}{5.567432in}}{\pgfqpoint{8.062523in}{5.575246in}}%
\pgfpathcurveto{\pgfqpoint{8.070337in}{5.583059in}}{\pgfqpoint{8.074727in}{5.593658in}}{\pgfqpoint{8.074727in}{5.604709in}}%
\pgfpathcurveto{\pgfqpoint{8.074727in}{5.615759in}}{\pgfqpoint{8.070337in}{5.626358in}}{\pgfqpoint{8.062523in}{5.634171in}}%
\pgfpathcurveto{\pgfqpoint{8.054709in}{5.641985in}}{\pgfqpoint{8.044110in}{5.646375in}}{\pgfqpoint{8.033060in}{5.646375in}}%
\pgfpathcurveto{\pgfqpoint{8.022010in}{5.646375in}}{\pgfqpoint{8.011411in}{5.641985in}}{\pgfqpoint{8.003597in}{5.634171in}}%
\pgfpathcurveto{\pgfqpoint{7.995784in}{5.626358in}}{\pgfqpoint{7.991394in}{5.615759in}}{\pgfqpoint{7.991394in}{5.604709in}}%
\pgfpathcurveto{\pgfqpoint{7.991394in}{5.593658in}}{\pgfqpoint{7.995784in}{5.583059in}}{\pgfqpoint{8.003597in}{5.575246in}}%
\pgfpathcurveto{\pgfqpoint{8.011411in}{5.567432in}}{\pgfqpoint{8.022010in}{5.563042in}}{\pgfqpoint{8.033060in}{5.563042in}}%
\pgfpathclose%
\pgfusepath{stroke,fill}%
\end{pgfscope}%
\begin{pgfscope}%
\pgfpathrectangle{\pgfqpoint{0.570343in}{0.331635in}}{\pgfqpoint{9.300000in}{7.700000in}}%
\pgfusepath{clip}%
\pgfsetbuttcap%
\pgfsetroundjoin%
\definecolor{currentfill}{rgb}{0.631373,0.788235,0.956863}%
\pgfsetfillcolor{currentfill}%
\pgfsetlinewidth{0.481800pt}%
\definecolor{currentstroke}{rgb}{1.000000,1.000000,1.000000}%
\pgfsetstrokecolor{currentstroke}%
\pgfsetdash{}{0pt}%
\pgfpathmoveto{\pgfqpoint{2.963896in}{2.251838in}}%
\pgfpathcurveto{\pgfqpoint{2.974946in}{2.251838in}}{\pgfqpoint{2.985545in}{2.256229in}}{\pgfqpoint{2.993358in}{2.264042in}}%
\pgfpathcurveto{\pgfqpoint{3.001172in}{2.271856in}}{\pgfqpoint{3.005562in}{2.282455in}}{\pgfqpoint{3.005562in}{2.293505in}}%
\pgfpathcurveto{\pgfqpoint{3.005562in}{2.304555in}}{\pgfqpoint{3.001172in}{2.315154in}}{\pgfqpoint{2.993358in}{2.322968in}}%
\pgfpathcurveto{\pgfqpoint{2.985545in}{2.330782in}}{\pgfqpoint{2.974946in}{2.335172in}}{\pgfqpoint{2.963896in}{2.335172in}}%
\pgfpathcurveto{\pgfqpoint{2.952845in}{2.335172in}}{\pgfqpoint{2.942246in}{2.330782in}}{\pgfqpoint{2.934433in}{2.322968in}}%
\pgfpathcurveto{\pgfqpoint{2.926619in}{2.315154in}}{\pgfqpoint{2.922229in}{2.304555in}}{\pgfqpoint{2.922229in}{2.293505in}}%
\pgfpathcurveto{\pgfqpoint{2.922229in}{2.282455in}}{\pgfqpoint{2.926619in}{2.271856in}}{\pgfqpoint{2.934433in}{2.264042in}}%
\pgfpathcurveto{\pgfqpoint{2.942246in}{2.256229in}}{\pgfqpoint{2.952845in}{2.251838in}}{\pgfqpoint{2.963896in}{2.251838in}}%
\pgfpathclose%
\pgfusepath{stroke,fill}%
\end{pgfscope}%
\begin{pgfscope}%
\pgfpathrectangle{\pgfqpoint{0.570343in}{0.331635in}}{\pgfqpoint{9.300000in}{7.700000in}}%
\pgfusepath{clip}%
\pgfsetbuttcap%
\pgfsetroundjoin%
\definecolor{currentfill}{rgb}{0.631373,0.788235,0.956863}%
\pgfsetfillcolor{currentfill}%
\pgfsetlinewidth{0.481800pt}%
\definecolor{currentstroke}{rgb}{1.000000,1.000000,1.000000}%
\pgfsetstrokecolor{currentstroke}%
\pgfsetdash{}{0pt}%
\pgfpathmoveto{\pgfqpoint{4.257872in}{3.909406in}}%
\pgfpathcurveto{\pgfqpoint{4.268922in}{3.909406in}}{\pgfqpoint{4.279521in}{3.913796in}}{\pgfqpoint{4.287335in}{3.921610in}}%
\pgfpathcurveto{\pgfqpoint{4.295149in}{3.929423in}}{\pgfqpoint{4.299539in}{3.940022in}}{\pgfqpoint{4.299539in}{3.951073in}}%
\pgfpathcurveto{\pgfqpoint{4.299539in}{3.962123in}}{\pgfqpoint{4.295149in}{3.972722in}}{\pgfqpoint{4.287335in}{3.980535in}}%
\pgfpathcurveto{\pgfqpoint{4.279521in}{3.988349in}}{\pgfqpoint{4.268922in}{3.992739in}}{\pgfqpoint{4.257872in}{3.992739in}}%
\pgfpathcurveto{\pgfqpoint{4.246822in}{3.992739in}}{\pgfqpoint{4.236223in}{3.988349in}}{\pgfqpoint{4.228409in}{3.980535in}}%
\pgfpathcurveto{\pgfqpoint{4.220596in}{3.972722in}}{\pgfqpoint{4.216206in}{3.962123in}}{\pgfqpoint{4.216206in}{3.951073in}}%
\pgfpathcurveto{\pgfqpoint{4.216206in}{3.940022in}}{\pgfqpoint{4.220596in}{3.929423in}}{\pgfqpoint{4.228409in}{3.921610in}}%
\pgfpathcurveto{\pgfqpoint{4.236223in}{3.913796in}}{\pgfqpoint{4.246822in}{3.909406in}}{\pgfqpoint{4.257872in}{3.909406in}}%
\pgfpathclose%
\pgfusepath{stroke,fill}%
\end{pgfscope}%
\begin{pgfscope}%
\pgfpathrectangle{\pgfqpoint{0.570343in}{0.331635in}}{\pgfqpoint{9.300000in}{7.700000in}}%
\pgfusepath{clip}%
\pgfsetbuttcap%
\pgfsetroundjoin%
\definecolor{currentfill}{rgb}{0.631373,0.788235,0.956863}%
\pgfsetfillcolor{currentfill}%
\pgfsetlinewidth{0.481800pt}%
\definecolor{currentstroke}{rgb}{1.000000,1.000000,1.000000}%
\pgfsetstrokecolor{currentstroke}%
\pgfsetdash{}{0pt}%
\pgfpathmoveto{\pgfqpoint{9.312478in}{4.988711in}}%
\pgfpathcurveto{\pgfqpoint{9.323528in}{4.988711in}}{\pgfqpoint{9.334127in}{4.993102in}}{\pgfqpoint{9.341941in}{5.000915in}}%
\pgfpathcurveto{\pgfqpoint{9.349754in}{5.008729in}}{\pgfqpoint{9.354145in}{5.019328in}}{\pgfqpoint{9.354145in}{5.030378in}}%
\pgfpathcurveto{\pgfqpoint{9.354145in}{5.041428in}}{\pgfqpoint{9.349754in}{5.052027in}}{\pgfqpoint{9.341941in}{5.059841in}}%
\pgfpathcurveto{\pgfqpoint{9.334127in}{5.067655in}}{\pgfqpoint{9.323528in}{5.072045in}}{\pgfqpoint{9.312478in}{5.072045in}}%
\pgfpathcurveto{\pgfqpoint{9.301428in}{5.072045in}}{\pgfqpoint{9.290829in}{5.067655in}}{\pgfqpoint{9.283015in}{5.059841in}}%
\pgfpathcurveto{\pgfqpoint{9.275201in}{5.052027in}}{\pgfqpoint{9.270811in}{5.041428in}}{\pgfqpoint{9.270811in}{5.030378in}}%
\pgfpathcurveto{\pgfqpoint{9.270811in}{5.019328in}}{\pgfqpoint{9.275201in}{5.008729in}}{\pgfqpoint{9.283015in}{5.000915in}}%
\pgfpathcurveto{\pgfqpoint{9.290829in}{4.993102in}}{\pgfqpoint{9.301428in}{4.988711in}}{\pgfqpoint{9.312478in}{4.988711in}}%
\pgfpathclose%
\pgfusepath{stroke,fill}%
\end{pgfscope}%
\begin{pgfscope}%
\pgfpathrectangle{\pgfqpoint{0.570343in}{0.331635in}}{\pgfqpoint{9.300000in}{7.700000in}}%
\pgfusepath{clip}%
\pgfsetbuttcap%
\pgfsetroundjoin%
\definecolor{currentfill}{rgb}{0.631373,0.788235,0.956863}%
\pgfsetfillcolor{currentfill}%
\pgfsetlinewidth{0.481800pt}%
\definecolor{currentstroke}{rgb}{1.000000,1.000000,1.000000}%
\pgfsetstrokecolor{currentstroke}%
\pgfsetdash{}{0pt}%
\pgfpathmoveto{\pgfqpoint{8.119150in}{6.376309in}}%
\pgfpathcurveto{\pgfqpoint{8.130200in}{6.376309in}}{\pgfqpoint{8.140799in}{6.380700in}}{\pgfqpoint{8.148613in}{6.388513in}}%
\pgfpathcurveto{\pgfqpoint{8.156426in}{6.396327in}}{\pgfqpoint{8.160817in}{6.406926in}}{\pgfqpoint{8.160817in}{6.417976in}}%
\pgfpathcurveto{\pgfqpoint{8.160817in}{6.429026in}}{\pgfqpoint{8.156426in}{6.439625in}}{\pgfqpoint{8.148613in}{6.447439in}}%
\pgfpathcurveto{\pgfqpoint{8.140799in}{6.455252in}}{\pgfqpoint{8.130200in}{6.459643in}}{\pgfqpoint{8.119150in}{6.459643in}}%
\pgfpathcurveto{\pgfqpoint{8.108100in}{6.459643in}}{\pgfqpoint{8.097501in}{6.455252in}}{\pgfqpoint{8.089687in}{6.447439in}}%
\pgfpathcurveto{\pgfqpoint{8.081874in}{6.439625in}}{\pgfqpoint{8.077483in}{6.429026in}}{\pgfqpoint{8.077483in}{6.417976in}}%
\pgfpathcurveto{\pgfqpoint{8.077483in}{6.406926in}}{\pgfqpoint{8.081874in}{6.396327in}}{\pgfqpoint{8.089687in}{6.388513in}}%
\pgfpathcurveto{\pgfqpoint{8.097501in}{6.380700in}}{\pgfqpoint{8.108100in}{6.376309in}}{\pgfqpoint{8.119150in}{6.376309in}}%
\pgfpathclose%
\pgfusepath{stroke,fill}%
\end{pgfscope}%
\begin{pgfscope}%
\pgfpathrectangle{\pgfqpoint{0.570343in}{0.331635in}}{\pgfqpoint{9.300000in}{7.700000in}}%
\pgfusepath{clip}%
\pgfsetbuttcap%
\pgfsetroundjoin%
\definecolor{currentfill}{rgb}{0.631373,0.788235,0.956863}%
\pgfsetfillcolor{currentfill}%
\pgfsetlinewidth{0.481800pt}%
\definecolor{currentstroke}{rgb}{1.000000,1.000000,1.000000}%
\pgfsetstrokecolor{currentstroke}%
\pgfsetdash{}{0pt}%
\pgfpathmoveto{\pgfqpoint{6.778721in}{5.992995in}}%
\pgfpathcurveto{\pgfqpoint{6.789771in}{5.992995in}}{\pgfqpoint{6.800370in}{5.997385in}}{\pgfqpoint{6.808184in}{6.005199in}}%
\pgfpathcurveto{\pgfqpoint{6.815998in}{6.013012in}}{\pgfqpoint{6.820388in}{6.023611in}}{\pgfqpoint{6.820388in}{6.034661in}}%
\pgfpathcurveto{\pgfqpoint{6.820388in}{6.045711in}}{\pgfqpoint{6.815998in}{6.056310in}}{\pgfqpoint{6.808184in}{6.064124in}}%
\pgfpathcurveto{\pgfqpoint{6.800370in}{6.071938in}}{\pgfqpoint{6.789771in}{6.076328in}}{\pgfqpoint{6.778721in}{6.076328in}}%
\pgfpathcurveto{\pgfqpoint{6.767671in}{6.076328in}}{\pgfqpoint{6.757072in}{6.071938in}}{\pgfqpoint{6.749258in}{6.064124in}}%
\pgfpathcurveto{\pgfqpoint{6.741445in}{6.056310in}}{\pgfqpoint{6.737055in}{6.045711in}}{\pgfqpoint{6.737055in}{6.034661in}}%
\pgfpathcurveto{\pgfqpoint{6.737055in}{6.023611in}}{\pgfqpoint{6.741445in}{6.013012in}}{\pgfqpoint{6.749258in}{6.005199in}}%
\pgfpathcurveto{\pgfqpoint{6.757072in}{5.997385in}}{\pgfqpoint{6.767671in}{5.992995in}}{\pgfqpoint{6.778721in}{5.992995in}}%
\pgfpathclose%
\pgfusepath{stroke,fill}%
\end{pgfscope}%
\begin{pgfscope}%
\pgfpathrectangle{\pgfqpoint{0.570343in}{0.331635in}}{\pgfqpoint{9.300000in}{7.700000in}}%
\pgfusepath{clip}%
\pgfsetbuttcap%
\pgfsetroundjoin%
\definecolor{currentfill}{rgb}{0.631373,0.788235,0.956863}%
\pgfsetfillcolor{currentfill}%
\pgfsetlinewidth{0.481800pt}%
\definecolor{currentstroke}{rgb}{1.000000,1.000000,1.000000}%
\pgfsetstrokecolor{currentstroke}%
\pgfsetdash{}{0pt}%
\pgfpathmoveto{\pgfqpoint{7.814530in}{6.020289in}}%
\pgfpathcurveto{\pgfqpoint{7.825580in}{6.020289in}}{\pgfqpoint{7.836179in}{6.024680in}}{\pgfqpoint{7.843993in}{6.032493in}}%
\pgfpathcurveto{\pgfqpoint{7.851807in}{6.040307in}}{\pgfqpoint{7.856197in}{6.050906in}}{\pgfqpoint{7.856197in}{6.061956in}}%
\pgfpathcurveto{\pgfqpoint{7.856197in}{6.073006in}}{\pgfqpoint{7.851807in}{6.083605in}}{\pgfqpoint{7.843993in}{6.091419in}}%
\pgfpathcurveto{\pgfqpoint{7.836179in}{6.099232in}}{\pgfqpoint{7.825580in}{6.103623in}}{\pgfqpoint{7.814530in}{6.103623in}}%
\pgfpathcurveto{\pgfqpoint{7.803480in}{6.103623in}}{\pgfqpoint{7.792881in}{6.099232in}}{\pgfqpoint{7.785067in}{6.091419in}}%
\pgfpathcurveto{\pgfqpoint{7.777254in}{6.083605in}}{\pgfqpoint{7.772864in}{6.073006in}}{\pgfqpoint{7.772864in}{6.061956in}}%
\pgfpathcurveto{\pgfqpoint{7.772864in}{6.050906in}}{\pgfqpoint{7.777254in}{6.040307in}}{\pgfqpoint{7.785067in}{6.032493in}}%
\pgfpathcurveto{\pgfqpoint{7.792881in}{6.024680in}}{\pgfqpoint{7.803480in}{6.020289in}}{\pgfqpoint{7.814530in}{6.020289in}}%
\pgfpathclose%
\pgfusepath{stroke,fill}%
\end{pgfscope}%
\begin{pgfscope}%
\pgfpathrectangle{\pgfqpoint{0.570343in}{0.331635in}}{\pgfqpoint{9.300000in}{7.700000in}}%
\pgfusepath{clip}%
\pgfsetbuttcap%
\pgfsetroundjoin%
\definecolor{currentfill}{rgb}{0.631373,0.788235,0.956863}%
\pgfsetfillcolor{currentfill}%
\pgfsetlinewidth{0.481800pt}%
\definecolor{currentstroke}{rgb}{1.000000,1.000000,1.000000}%
\pgfsetstrokecolor{currentstroke}%
\pgfsetdash{}{0pt}%
\pgfpathmoveto{\pgfqpoint{6.210576in}{7.639968in}}%
\pgfpathcurveto{\pgfqpoint{6.221626in}{7.639968in}}{\pgfqpoint{6.232225in}{7.644359in}}{\pgfqpoint{6.240039in}{7.652172in}}%
\pgfpathcurveto{\pgfqpoint{6.247852in}{7.659986in}}{\pgfqpoint{6.252243in}{7.670585in}}{\pgfqpoint{6.252243in}{7.681635in}}%
\pgfpathcurveto{\pgfqpoint{6.252243in}{7.692685in}}{\pgfqpoint{6.247852in}{7.703284in}}{\pgfqpoint{6.240039in}{7.711098in}}%
\pgfpathcurveto{\pgfqpoint{6.232225in}{7.718911in}}{\pgfqpoint{6.221626in}{7.723302in}}{\pgfqpoint{6.210576in}{7.723302in}}%
\pgfpathcurveto{\pgfqpoint{6.199526in}{7.723302in}}{\pgfqpoint{6.188927in}{7.718911in}}{\pgfqpoint{6.181113in}{7.711098in}}%
\pgfpathcurveto{\pgfqpoint{6.173299in}{7.703284in}}{\pgfqpoint{6.168909in}{7.692685in}}{\pgfqpoint{6.168909in}{7.681635in}}%
\pgfpathcurveto{\pgfqpoint{6.168909in}{7.670585in}}{\pgfqpoint{6.173299in}{7.659986in}}{\pgfqpoint{6.181113in}{7.652172in}}%
\pgfpathcurveto{\pgfqpoint{6.188927in}{7.644359in}}{\pgfqpoint{6.199526in}{7.639968in}}{\pgfqpoint{6.210576in}{7.639968in}}%
\pgfpathclose%
\pgfusepath{stroke,fill}%
\end{pgfscope}%
\begin{pgfscope}%
\pgfpathrectangle{\pgfqpoint{0.570343in}{0.331635in}}{\pgfqpoint{9.300000in}{7.700000in}}%
\pgfusepath{clip}%
\pgfsetbuttcap%
\pgfsetroundjoin%
\definecolor{currentfill}{rgb}{0.631373,0.788235,0.956863}%
\pgfsetfillcolor{currentfill}%
\pgfsetlinewidth{0.481800pt}%
\definecolor{currentstroke}{rgb}{1.000000,1.000000,1.000000}%
\pgfsetstrokecolor{currentstroke}%
\pgfsetdash{}{0pt}%
\pgfpathmoveto{\pgfqpoint{8.639749in}{5.519270in}}%
\pgfpathcurveto{\pgfqpoint{8.650799in}{5.519270in}}{\pgfqpoint{8.661398in}{5.523660in}}{\pgfqpoint{8.669212in}{5.531474in}}%
\pgfpathcurveto{\pgfqpoint{8.677025in}{5.539287in}}{\pgfqpoint{8.681416in}{5.549886in}}{\pgfqpoint{8.681416in}{5.560936in}}%
\pgfpathcurveto{\pgfqpoint{8.681416in}{5.571986in}}{\pgfqpoint{8.677025in}{5.582586in}}{\pgfqpoint{8.669212in}{5.590399in}}%
\pgfpathcurveto{\pgfqpoint{8.661398in}{5.598213in}}{\pgfqpoint{8.650799in}{5.602603in}}{\pgfqpoint{8.639749in}{5.602603in}}%
\pgfpathcurveto{\pgfqpoint{8.628699in}{5.602603in}}{\pgfqpoint{8.618100in}{5.598213in}}{\pgfqpoint{8.610286in}{5.590399in}}%
\pgfpathcurveto{\pgfqpoint{8.602473in}{5.582586in}}{\pgfqpoint{8.598082in}{5.571986in}}{\pgfqpoint{8.598082in}{5.560936in}}%
\pgfpathcurveto{\pgfqpoint{8.598082in}{5.549886in}}{\pgfqpoint{8.602473in}{5.539287in}}{\pgfqpoint{8.610286in}{5.531474in}}%
\pgfpathcurveto{\pgfqpoint{8.618100in}{5.523660in}}{\pgfqpoint{8.628699in}{5.519270in}}{\pgfqpoint{8.639749in}{5.519270in}}%
\pgfpathclose%
\pgfusepath{stroke,fill}%
\end{pgfscope}%
\begin{pgfscope}%
\pgfpathrectangle{\pgfqpoint{0.570343in}{0.331635in}}{\pgfqpoint{9.300000in}{7.700000in}}%
\pgfusepath{clip}%
\pgfsetbuttcap%
\pgfsetroundjoin%
\definecolor{currentfill}{rgb}{0.631373,0.788235,0.956863}%
\pgfsetfillcolor{currentfill}%
\pgfsetlinewidth{0.481800pt}%
\definecolor{currentstroke}{rgb}{1.000000,1.000000,1.000000}%
\pgfsetstrokecolor{currentstroke}%
\pgfsetdash{}{0pt}%
\pgfpathmoveto{\pgfqpoint{2.502764in}{5.483757in}}%
\pgfpathcurveto{\pgfqpoint{2.513815in}{5.483757in}}{\pgfqpoint{2.524414in}{5.488147in}}{\pgfqpoint{2.532227in}{5.495961in}}%
\pgfpathcurveto{\pgfqpoint{2.540041in}{5.503774in}}{\pgfqpoint{2.544431in}{5.514373in}}{\pgfqpoint{2.544431in}{5.525424in}}%
\pgfpathcurveto{\pgfqpoint{2.544431in}{5.536474in}}{\pgfqpoint{2.540041in}{5.547073in}}{\pgfqpoint{2.532227in}{5.554886in}}%
\pgfpathcurveto{\pgfqpoint{2.524414in}{5.562700in}}{\pgfqpoint{2.513815in}{5.567090in}}{\pgfqpoint{2.502764in}{5.567090in}}%
\pgfpathcurveto{\pgfqpoint{2.491714in}{5.567090in}}{\pgfqpoint{2.481115in}{5.562700in}}{\pgfqpoint{2.473302in}{5.554886in}}%
\pgfpathcurveto{\pgfqpoint{2.465488in}{5.547073in}}{\pgfqpoint{2.461098in}{5.536474in}}{\pgfqpoint{2.461098in}{5.525424in}}%
\pgfpathcurveto{\pgfqpoint{2.461098in}{5.514373in}}{\pgfqpoint{2.465488in}{5.503774in}}{\pgfqpoint{2.473302in}{5.495961in}}%
\pgfpathcurveto{\pgfqpoint{2.481115in}{5.488147in}}{\pgfqpoint{2.491714in}{5.483757in}}{\pgfqpoint{2.502764in}{5.483757in}}%
\pgfpathclose%
\pgfusepath{stroke,fill}%
\end{pgfscope}%
\begin{pgfscope}%
\pgfpathrectangle{\pgfqpoint{0.570343in}{0.331635in}}{\pgfqpoint{9.300000in}{7.700000in}}%
\pgfusepath{clip}%
\pgfsetbuttcap%
\pgfsetroundjoin%
\definecolor{currentfill}{rgb}{0.631373,0.788235,0.956863}%
\pgfsetfillcolor{currentfill}%
\pgfsetlinewidth{0.481800pt}%
\definecolor{currentstroke}{rgb}{1.000000,1.000000,1.000000}%
\pgfsetstrokecolor{currentstroke}%
\pgfsetdash{}{0pt}%
\pgfpathmoveto{\pgfqpoint{7.737428in}{2.270402in}}%
\pgfpathcurveto{\pgfqpoint{7.748478in}{2.270402in}}{\pgfqpoint{7.759077in}{2.274792in}}{\pgfqpoint{7.766891in}{2.282605in}}%
\pgfpathcurveto{\pgfqpoint{7.774705in}{2.290419in}}{\pgfqpoint{7.779095in}{2.301018in}}{\pgfqpoint{7.779095in}{2.312068in}}%
\pgfpathcurveto{\pgfqpoint{7.779095in}{2.323118in}}{\pgfqpoint{7.774705in}{2.333717in}}{\pgfqpoint{7.766891in}{2.341531in}}%
\pgfpathcurveto{\pgfqpoint{7.759077in}{2.349345in}}{\pgfqpoint{7.748478in}{2.353735in}}{\pgfqpoint{7.737428in}{2.353735in}}%
\pgfpathcurveto{\pgfqpoint{7.726378in}{2.353735in}}{\pgfqpoint{7.715779in}{2.349345in}}{\pgfqpoint{7.707965in}{2.341531in}}%
\pgfpathcurveto{\pgfqpoint{7.700152in}{2.333717in}}{\pgfqpoint{7.695762in}{2.323118in}}{\pgfqpoint{7.695762in}{2.312068in}}%
\pgfpathcurveto{\pgfqpoint{7.695762in}{2.301018in}}{\pgfqpoint{7.700152in}{2.290419in}}{\pgfqpoint{7.707965in}{2.282605in}}%
\pgfpathcurveto{\pgfqpoint{7.715779in}{2.274792in}}{\pgfqpoint{7.726378in}{2.270402in}}{\pgfqpoint{7.737428in}{2.270402in}}%
\pgfpathclose%
\pgfusepath{stroke,fill}%
\end{pgfscope}%
\begin{pgfscope}%
\pgfpathrectangle{\pgfqpoint{0.570343in}{0.331635in}}{\pgfqpoint{9.300000in}{7.700000in}}%
\pgfusepath{clip}%
\pgfsetbuttcap%
\pgfsetroundjoin%
\definecolor{currentfill}{rgb}{0.631373,0.788235,0.956863}%
\pgfsetfillcolor{currentfill}%
\pgfsetlinewidth{0.481800pt}%
\definecolor{currentstroke}{rgb}{1.000000,1.000000,1.000000}%
\pgfsetstrokecolor{currentstroke}%
\pgfsetdash{}{0pt}%
\pgfpathmoveto{\pgfqpoint{8.606468in}{5.997684in}}%
\pgfpathcurveto{\pgfqpoint{8.617518in}{5.997684in}}{\pgfqpoint{8.628117in}{6.002074in}}{\pgfqpoint{8.635931in}{6.009888in}}%
\pgfpathcurveto{\pgfqpoint{8.643744in}{6.017702in}}{\pgfqpoint{8.648135in}{6.028301in}}{\pgfqpoint{8.648135in}{6.039351in}}%
\pgfpathcurveto{\pgfqpoint{8.648135in}{6.050401in}}{\pgfqpoint{8.643744in}{6.061000in}}{\pgfqpoint{8.635931in}{6.068813in}}%
\pgfpathcurveto{\pgfqpoint{8.628117in}{6.076627in}}{\pgfqpoint{8.617518in}{6.081017in}}{\pgfqpoint{8.606468in}{6.081017in}}%
\pgfpathcurveto{\pgfqpoint{8.595418in}{6.081017in}}{\pgfqpoint{8.584819in}{6.076627in}}{\pgfqpoint{8.577005in}{6.068813in}}%
\pgfpathcurveto{\pgfqpoint{8.569192in}{6.061000in}}{\pgfqpoint{8.564801in}{6.050401in}}{\pgfqpoint{8.564801in}{6.039351in}}%
\pgfpathcurveto{\pgfqpoint{8.564801in}{6.028301in}}{\pgfqpoint{8.569192in}{6.017702in}}{\pgfqpoint{8.577005in}{6.009888in}}%
\pgfpathcurveto{\pgfqpoint{8.584819in}{6.002074in}}{\pgfqpoint{8.595418in}{5.997684in}}{\pgfqpoint{8.606468in}{5.997684in}}%
\pgfpathclose%
\pgfusepath{stroke,fill}%
\end{pgfscope}%
\begin{pgfscope}%
\pgfpathrectangle{\pgfqpoint{0.570343in}{0.331635in}}{\pgfqpoint{9.300000in}{7.700000in}}%
\pgfusepath{clip}%
\pgfsetbuttcap%
\pgfsetroundjoin%
\definecolor{currentfill}{rgb}{0.631373,0.788235,0.956863}%
\pgfsetfillcolor{currentfill}%
\pgfsetlinewidth{0.481800pt}%
\definecolor{currentstroke}{rgb}{1.000000,1.000000,1.000000}%
\pgfsetstrokecolor{currentstroke}%
\pgfsetdash{}{0pt}%
\pgfpathmoveto{\pgfqpoint{9.447616in}{3.539448in}}%
\pgfpathcurveto{\pgfqpoint{9.458666in}{3.539448in}}{\pgfqpoint{9.469265in}{3.543838in}}{\pgfqpoint{9.477079in}{3.551652in}}%
\pgfpathcurveto{\pgfqpoint{9.484892in}{3.559466in}}{\pgfqpoint{9.489283in}{3.570065in}}{\pgfqpoint{9.489283in}{3.581115in}}%
\pgfpathcurveto{\pgfqpoint{9.489283in}{3.592165in}}{\pgfqpoint{9.484892in}{3.602764in}}{\pgfqpoint{9.477079in}{3.610577in}}%
\pgfpathcurveto{\pgfqpoint{9.469265in}{3.618391in}}{\pgfqpoint{9.458666in}{3.622781in}}{\pgfqpoint{9.447616in}{3.622781in}}%
\pgfpathcurveto{\pgfqpoint{9.436566in}{3.622781in}}{\pgfqpoint{9.425967in}{3.618391in}}{\pgfqpoint{9.418153in}{3.610577in}}%
\pgfpathcurveto{\pgfqpoint{9.410340in}{3.602764in}}{\pgfqpoint{9.405949in}{3.592165in}}{\pgfqpoint{9.405949in}{3.581115in}}%
\pgfpathcurveto{\pgfqpoint{9.405949in}{3.570065in}}{\pgfqpoint{9.410340in}{3.559466in}}{\pgfqpoint{9.418153in}{3.551652in}}%
\pgfpathcurveto{\pgfqpoint{9.425967in}{3.543838in}}{\pgfqpoint{9.436566in}{3.539448in}}{\pgfqpoint{9.447616in}{3.539448in}}%
\pgfpathclose%
\pgfusepath{stroke,fill}%
\end{pgfscope}%
\begin{pgfscope}%
\pgfpathrectangle{\pgfqpoint{0.570343in}{0.331635in}}{\pgfqpoint{9.300000in}{7.700000in}}%
\pgfusepath{clip}%
\pgfsetbuttcap%
\pgfsetroundjoin%
\definecolor{currentfill}{rgb}{0.631373,0.788235,0.956863}%
\pgfsetfillcolor{currentfill}%
\pgfsetlinewidth{0.481800pt}%
\definecolor{currentstroke}{rgb}{1.000000,1.000000,1.000000}%
\pgfsetstrokecolor{currentstroke}%
\pgfsetdash{}{0pt}%
\pgfpathmoveto{\pgfqpoint{1.771086in}{2.836403in}}%
\pgfpathcurveto{\pgfqpoint{1.782136in}{2.836403in}}{\pgfqpoint{1.792735in}{2.840793in}}{\pgfqpoint{1.800548in}{2.848607in}}%
\pgfpathcurveto{\pgfqpoint{1.808362in}{2.856421in}}{\pgfqpoint{1.812752in}{2.867020in}}{\pgfqpoint{1.812752in}{2.878070in}}%
\pgfpathcurveto{\pgfqpoint{1.812752in}{2.889120in}}{\pgfqpoint{1.808362in}{2.899719in}}{\pgfqpoint{1.800548in}{2.907533in}}%
\pgfpathcurveto{\pgfqpoint{1.792735in}{2.915346in}}{\pgfqpoint{1.782136in}{2.919736in}}{\pgfqpoint{1.771086in}{2.919736in}}%
\pgfpathcurveto{\pgfqpoint{1.760035in}{2.919736in}}{\pgfqpoint{1.749436in}{2.915346in}}{\pgfqpoint{1.741623in}{2.907533in}}%
\pgfpathcurveto{\pgfqpoint{1.733809in}{2.899719in}}{\pgfqpoint{1.729419in}{2.889120in}}{\pgfqpoint{1.729419in}{2.878070in}}%
\pgfpathcurveto{\pgfqpoint{1.729419in}{2.867020in}}{\pgfqpoint{1.733809in}{2.856421in}}{\pgfqpoint{1.741623in}{2.848607in}}%
\pgfpathcurveto{\pgfqpoint{1.749436in}{2.840793in}}{\pgfqpoint{1.760035in}{2.836403in}}{\pgfqpoint{1.771086in}{2.836403in}}%
\pgfpathclose%
\pgfusepath{stroke,fill}%
\end{pgfscope}%
\begin{pgfscope}%
\pgfpathrectangle{\pgfqpoint{0.570343in}{0.331635in}}{\pgfqpoint{9.300000in}{7.700000in}}%
\pgfusepath{clip}%
\pgfsetbuttcap%
\pgfsetroundjoin%
\definecolor{currentfill}{rgb}{0.631373,0.788235,0.956863}%
\pgfsetfillcolor{currentfill}%
\pgfsetlinewidth{0.481800pt}%
\definecolor{currentstroke}{rgb}{1.000000,1.000000,1.000000}%
\pgfsetstrokecolor{currentstroke}%
\pgfsetdash{}{0pt}%
\pgfpathmoveto{\pgfqpoint{4.161271in}{6.089642in}}%
\pgfpathcurveto{\pgfqpoint{4.172322in}{6.089642in}}{\pgfqpoint{4.182921in}{6.094033in}}{\pgfqpoint{4.190734in}{6.101846in}}%
\pgfpathcurveto{\pgfqpoint{4.198548in}{6.109660in}}{\pgfqpoint{4.202938in}{6.120259in}}{\pgfqpoint{4.202938in}{6.131309in}}%
\pgfpathcurveto{\pgfqpoint{4.202938in}{6.142359in}}{\pgfqpoint{4.198548in}{6.152958in}}{\pgfqpoint{4.190734in}{6.160772in}}%
\pgfpathcurveto{\pgfqpoint{4.182921in}{6.168585in}}{\pgfqpoint{4.172322in}{6.172976in}}{\pgfqpoint{4.161271in}{6.172976in}}%
\pgfpathcurveto{\pgfqpoint{4.150221in}{6.172976in}}{\pgfqpoint{4.139622in}{6.168585in}}{\pgfqpoint{4.131809in}{6.160772in}}%
\pgfpathcurveto{\pgfqpoint{4.123995in}{6.152958in}}{\pgfqpoint{4.119605in}{6.142359in}}{\pgfqpoint{4.119605in}{6.131309in}}%
\pgfpathcurveto{\pgfqpoint{4.119605in}{6.120259in}}{\pgfqpoint{4.123995in}{6.109660in}}{\pgfqpoint{4.131809in}{6.101846in}}%
\pgfpathcurveto{\pgfqpoint{4.139622in}{6.094033in}}{\pgfqpoint{4.150221in}{6.089642in}}{\pgfqpoint{4.161271in}{6.089642in}}%
\pgfpathclose%
\pgfusepath{stroke,fill}%
\end{pgfscope}%
\begin{pgfscope}%
\pgfpathrectangle{\pgfqpoint{0.570343in}{0.331635in}}{\pgfqpoint{9.300000in}{7.700000in}}%
\pgfusepath{clip}%
\pgfsetbuttcap%
\pgfsetroundjoin%
\definecolor{currentfill}{rgb}{0.631373,0.788235,0.956863}%
\pgfsetfillcolor{currentfill}%
\pgfsetlinewidth{0.481800pt}%
\definecolor{currentstroke}{rgb}{1.000000,1.000000,1.000000}%
\pgfsetstrokecolor{currentstroke}%
\pgfsetdash{}{0pt}%
\pgfpathmoveto{\pgfqpoint{4.477751in}{4.822319in}}%
\pgfpathcurveto{\pgfqpoint{4.488801in}{4.822319in}}{\pgfqpoint{4.499400in}{4.826709in}}{\pgfqpoint{4.507214in}{4.834523in}}%
\pgfpathcurveto{\pgfqpoint{4.515027in}{4.842336in}}{\pgfqpoint{4.519418in}{4.852936in}}{\pgfqpoint{4.519418in}{4.863986in}}%
\pgfpathcurveto{\pgfqpoint{4.519418in}{4.875036in}}{\pgfqpoint{4.515027in}{4.885635in}}{\pgfqpoint{4.507214in}{4.893448in}}%
\pgfpathcurveto{\pgfqpoint{4.499400in}{4.901262in}}{\pgfqpoint{4.488801in}{4.905652in}}{\pgfqpoint{4.477751in}{4.905652in}}%
\pgfpathcurveto{\pgfqpoint{4.466701in}{4.905652in}}{\pgfqpoint{4.456102in}{4.901262in}}{\pgfqpoint{4.448288in}{4.893448in}}%
\pgfpathcurveto{\pgfqpoint{4.440475in}{4.885635in}}{\pgfqpoint{4.436084in}{4.875036in}}{\pgfqpoint{4.436084in}{4.863986in}}%
\pgfpathcurveto{\pgfqpoint{4.436084in}{4.852936in}}{\pgfqpoint{4.440475in}{4.842336in}}{\pgfqpoint{4.448288in}{4.834523in}}%
\pgfpathcurveto{\pgfqpoint{4.456102in}{4.826709in}}{\pgfqpoint{4.466701in}{4.822319in}}{\pgfqpoint{4.477751in}{4.822319in}}%
\pgfpathclose%
\pgfusepath{stroke,fill}%
\end{pgfscope}%
\begin{pgfscope}%
\pgfpathrectangle{\pgfqpoint{0.570343in}{0.331635in}}{\pgfqpoint{9.300000in}{7.700000in}}%
\pgfusepath{clip}%
\pgfsetbuttcap%
\pgfsetroundjoin%
\definecolor{currentfill}{rgb}{0.631373,0.788235,0.956863}%
\pgfsetfillcolor{currentfill}%
\pgfsetlinewidth{0.481800pt}%
\definecolor{currentstroke}{rgb}{1.000000,1.000000,1.000000}%
\pgfsetstrokecolor{currentstroke}%
\pgfsetdash{}{0pt}%
\pgfpathmoveto{\pgfqpoint{6.827660in}{3.250088in}}%
\pgfpathcurveto{\pgfqpoint{6.838710in}{3.250088in}}{\pgfqpoint{6.849309in}{3.254478in}}{\pgfqpoint{6.857123in}{3.262292in}}%
\pgfpathcurveto{\pgfqpoint{6.864936in}{3.270106in}}{\pgfqpoint{6.869327in}{3.280705in}}{\pgfqpoint{6.869327in}{3.291755in}}%
\pgfpathcurveto{\pgfqpoint{6.869327in}{3.302805in}}{\pgfqpoint{6.864936in}{3.313404in}}{\pgfqpoint{6.857123in}{3.321217in}}%
\pgfpathcurveto{\pgfqpoint{6.849309in}{3.329031in}}{\pgfqpoint{6.838710in}{3.333421in}}{\pgfqpoint{6.827660in}{3.333421in}}%
\pgfpathcurveto{\pgfqpoint{6.816610in}{3.333421in}}{\pgfqpoint{6.806011in}{3.329031in}}{\pgfqpoint{6.798197in}{3.321217in}}%
\pgfpathcurveto{\pgfqpoint{6.790384in}{3.313404in}}{\pgfqpoint{6.785993in}{3.302805in}}{\pgfqpoint{6.785993in}{3.291755in}}%
\pgfpathcurveto{\pgfqpoint{6.785993in}{3.280705in}}{\pgfqpoint{6.790384in}{3.270106in}}{\pgfqpoint{6.798197in}{3.262292in}}%
\pgfpathcurveto{\pgfqpoint{6.806011in}{3.254478in}}{\pgfqpoint{6.816610in}{3.250088in}}{\pgfqpoint{6.827660in}{3.250088in}}%
\pgfpathclose%
\pgfusepath{stroke,fill}%
\end{pgfscope}%
\begin{pgfscope}%
\pgfpathrectangle{\pgfqpoint{0.570343in}{0.331635in}}{\pgfqpoint{9.300000in}{7.700000in}}%
\pgfusepath{clip}%
\pgfsetbuttcap%
\pgfsetroundjoin%
\definecolor{currentfill}{rgb}{0.631373,0.788235,0.956863}%
\pgfsetfillcolor{currentfill}%
\pgfsetlinewidth{0.481800pt}%
\definecolor{currentstroke}{rgb}{1.000000,1.000000,1.000000}%
\pgfsetstrokecolor{currentstroke}%
\pgfsetdash{}{0pt}%
\pgfpathmoveto{\pgfqpoint{6.876963in}{5.172517in}}%
\pgfpathcurveto{\pgfqpoint{6.888013in}{5.172517in}}{\pgfqpoint{6.898612in}{5.176907in}}{\pgfqpoint{6.906426in}{5.184721in}}%
\pgfpathcurveto{\pgfqpoint{6.914239in}{5.192535in}}{\pgfqpoint{6.918630in}{5.203134in}}{\pgfqpoint{6.918630in}{5.214184in}}%
\pgfpathcurveto{\pgfqpoint{6.918630in}{5.225234in}}{\pgfqpoint{6.914239in}{5.235833in}}{\pgfqpoint{6.906426in}{5.243647in}}%
\pgfpathcurveto{\pgfqpoint{6.898612in}{5.251460in}}{\pgfqpoint{6.888013in}{5.255850in}}{\pgfqpoint{6.876963in}{5.255850in}}%
\pgfpathcurveto{\pgfqpoint{6.865913in}{5.255850in}}{\pgfqpoint{6.855314in}{5.251460in}}{\pgfqpoint{6.847500in}{5.243647in}}%
\pgfpathcurveto{\pgfqpoint{6.839687in}{5.235833in}}{\pgfqpoint{6.835296in}{5.225234in}}{\pgfqpoint{6.835296in}{5.214184in}}%
\pgfpathcurveto{\pgfqpoint{6.835296in}{5.203134in}}{\pgfqpoint{6.839687in}{5.192535in}}{\pgfqpoint{6.847500in}{5.184721in}}%
\pgfpathcurveto{\pgfqpoint{6.855314in}{5.176907in}}{\pgfqpoint{6.865913in}{5.172517in}}{\pgfqpoint{6.876963in}{5.172517in}}%
\pgfpathclose%
\pgfusepath{stroke,fill}%
\end{pgfscope}%
\begin{pgfscope}%
\pgfpathrectangle{\pgfqpoint{0.570343in}{0.331635in}}{\pgfqpoint{9.300000in}{7.700000in}}%
\pgfusepath{clip}%
\pgfsetbuttcap%
\pgfsetroundjoin%
\definecolor{currentfill}{rgb}{0.631373,0.788235,0.956863}%
\pgfsetfillcolor{currentfill}%
\pgfsetlinewidth{0.481800pt}%
\definecolor{currentstroke}{rgb}{1.000000,1.000000,1.000000}%
\pgfsetstrokecolor{currentstroke}%
\pgfsetdash{}{0pt}%
\pgfpathmoveto{\pgfqpoint{5.974644in}{6.800670in}}%
\pgfpathcurveto{\pgfqpoint{5.985694in}{6.800670in}}{\pgfqpoint{5.996293in}{6.805060in}}{\pgfqpoint{6.004107in}{6.812874in}}%
\pgfpathcurveto{\pgfqpoint{6.011920in}{6.820688in}}{\pgfqpoint{6.016311in}{6.831287in}}{\pgfqpoint{6.016311in}{6.842337in}}%
\pgfpathcurveto{\pgfqpoint{6.016311in}{6.853387in}}{\pgfqpoint{6.011920in}{6.863986in}}{\pgfqpoint{6.004107in}{6.871799in}}%
\pgfpathcurveto{\pgfqpoint{5.996293in}{6.879613in}}{\pgfqpoint{5.985694in}{6.884003in}}{\pgfqpoint{5.974644in}{6.884003in}}%
\pgfpathcurveto{\pgfqpoint{5.963594in}{6.884003in}}{\pgfqpoint{5.952995in}{6.879613in}}{\pgfqpoint{5.945181in}{6.871799in}}%
\pgfpathcurveto{\pgfqpoint{5.937368in}{6.863986in}}{\pgfqpoint{5.932977in}{6.853387in}}{\pgfqpoint{5.932977in}{6.842337in}}%
\pgfpathcurveto{\pgfqpoint{5.932977in}{6.831287in}}{\pgfqpoint{5.937368in}{6.820688in}}{\pgfqpoint{5.945181in}{6.812874in}}%
\pgfpathcurveto{\pgfqpoint{5.952995in}{6.805060in}}{\pgfqpoint{5.963594in}{6.800670in}}{\pgfqpoint{5.974644in}{6.800670in}}%
\pgfpathclose%
\pgfusepath{stroke,fill}%
\end{pgfscope}%
\begin{pgfscope}%
\pgfpathrectangle{\pgfqpoint{0.570343in}{0.331635in}}{\pgfqpoint{9.300000in}{7.700000in}}%
\pgfusepath{clip}%
\pgfsetbuttcap%
\pgfsetroundjoin%
\definecolor{currentfill}{rgb}{0.631373,0.788235,0.956863}%
\pgfsetfillcolor{currentfill}%
\pgfsetlinewidth{0.481800pt}%
\definecolor{currentstroke}{rgb}{1.000000,1.000000,1.000000}%
\pgfsetstrokecolor{currentstroke}%
\pgfsetdash{}{0pt}%
\pgfpathmoveto{\pgfqpoint{0.993071in}{3.079068in}}%
\pgfpathcurveto{\pgfqpoint{1.004121in}{3.079068in}}{\pgfqpoint{1.014720in}{3.083458in}}{\pgfqpoint{1.022533in}{3.091272in}}%
\pgfpathcurveto{\pgfqpoint{1.030347in}{3.099085in}}{\pgfqpoint{1.034737in}{3.109684in}}{\pgfqpoint{1.034737in}{3.120734in}}%
\pgfpathcurveto{\pgfqpoint{1.034737in}{3.131785in}}{\pgfqpoint{1.030347in}{3.142384in}}{\pgfqpoint{1.022533in}{3.150197in}}%
\pgfpathcurveto{\pgfqpoint{1.014720in}{3.158011in}}{\pgfqpoint{1.004121in}{3.162401in}}{\pgfqpoint{0.993071in}{3.162401in}}%
\pgfpathcurveto{\pgfqpoint{0.982020in}{3.162401in}}{\pgfqpoint{0.971421in}{3.158011in}}{\pgfqpoint{0.963608in}{3.150197in}}%
\pgfpathcurveto{\pgfqpoint{0.955794in}{3.142384in}}{\pgfqpoint{0.951404in}{3.131785in}}{\pgfqpoint{0.951404in}{3.120734in}}%
\pgfpathcurveto{\pgfqpoint{0.951404in}{3.109684in}}{\pgfqpoint{0.955794in}{3.099085in}}{\pgfqpoint{0.963608in}{3.091272in}}%
\pgfpathcurveto{\pgfqpoint{0.971421in}{3.083458in}}{\pgfqpoint{0.982020in}{3.079068in}}{\pgfqpoint{0.993071in}{3.079068in}}%
\pgfpathclose%
\pgfusepath{stroke,fill}%
\end{pgfscope}%
\begin{pgfscope}%
\pgfpathrectangle{\pgfqpoint{0.570343in}{0.331635in}}{\pgfqpoint{9.300000in}{7.700000in}}%
\pgfusepath{clip}%
\pgfsetbuttcap%
\pgfsetroundjoin%
\definecolor{currentfill}{rgb}{0.631373,0.788235,0.956863}%
\pgfsetfillcolor{currentfill}%
\pgfsetlinewidth{0.481800pt}%
\definecolor{currentstroke}{rgb}{1.000000,1.000000,1.000000}%
\pgfsetstrokecolor{currentstroke}%
\pgfsetdash{}{0pt}%
\pgfpathmoveto{\pgfqpoint{1.720010in}{2.117634in}}%
\pgfpathcurveto{\pgfqpoint{1.731060in}{2.117634in}}{\pgfqpoint{1.741659in}{2.122024in}}{\pgfqpoint{1.749472in}{2.129838in}}%
\pgfpathcurveto{\pgfqpoint{1.757286in}{2.137652in}}{\pgfqpoint{1.761676in}{2.148251in}}{\pgfqpoint{1.761676in}{2.159301in}}%
\pgfpathcurveto{\pgfqpoint{1.761676in}{2.170351in}}{\pgfqpoint{1.757286in}{2.180950in}}{\pgfqpoint{1.749472in}{2.188764in}}%
\pgfpathcurveto{\pgfqpoint{1.741659in}{2.196577in}}{\pgfqpoint{1.731060in}{2.200967in}}{\pgfqpoint{1.720010in}{2.200967in}}%
\pgfpathcurveto{\pgfqpoint{1.708959in}{2.200967in}}{\pgfqpoint{1.698360in}{2.196577in}}{\pgfqpoint{1.690547in}{2.188764in}}%
\pgfpathcurveto{\pgfqpoint{1.682733in}{2.180950in}}{\pgfqpoint{1.678343in}{2.170351in}}{\pgfqpoint{1.678343in}{2.159301in}}%
\pgfpathcurveto{\pgfqpoint{1.678343in}{2.148251in}}{\pgfqpoint{1.682733in}{2.137652in}}{\pgfqpoint{1.690547in}{2.129838in}}%
\pgfpathcurveto{\pgfqpoint{1.698360in}{2.122024in}}{\pgfqpoint{1.708959in}{2.117634in}}{\pgfqpoint{1.720010in}{2.117634in}}%
\pgfpathclose%
\pgfusepath{stroke,fill}%
\end{pgfscope}%
\begin{pgfscope}%
\pgfpathrectangle{\pgfqpoint{0.570343in}{0.331635in}}{\pgfqpoint{9.300000in}{7.700000in}}%
\pgfusepath{clip}%
\pgfsetbuttcap%
\pgfsetroundjoin%
\definecolor{currentfill}{rgb}{0.631373,0.788235,0.956863}%
\pgfsetfillcolor{currentfill}%
\pgfsetlinewidth{0.481800pt}%
\definecolor{currentstroke}{rgb}{1.000000,1.000000,1.000000}%
\pgfsetstrokecolor{currentstroke}%
\pgfsetdash{}{0pt}%
\pgfpathmoveto{\pgfqpoint{3.244521in}{5.026736in}}%
\pgfpathcurveto{\pgfqpoint{3.255572in}{5.026736in}}{\pgfqpoint{3.266171in}{5.031126in}}{\pgfqpoint{3.273984in}{5.038939in}}%
\pgfpathcurveto{\pgfqpoint{3.281798in}{5.046753in}}{\pgfqpoint{3.286188in}{5.057352in}}{\pgfqpoint{3.286188in}{5.068402in}}%
\pgfpathcurveto{\pgfqpoint{3.286188in}{5.079452in}}{\pgfqpoint{3.281798in}{5.090051in}}{\pgfqpoint{3.273984in}{5.097865in}}%
\pgfpathcurveto{\pgfqpoint{3.266171in}{5.105679in}}{\pgfqpoint{3.255572in}{5.110069in}}{\pgfqpoint{3.244521in}{5.110069in}}%
\pgfpathcurveto{\pgfqpoint{3.233471in}{5.110069in}}{\pgfqpoint{3.222872in}{5.105679in}}{\pgfqpoint{3.215059in}{5.097865in}}%
\pgfpathcurveto{\pgfqpoint{3.207245in}{5.090051in}}{\pgfqpoint{3.202855in}{5.079452in}}{\pgfqpoint{3.202855in}{5.068402in}}%
\pgfpathcurveto{\pgfqpoint{3.202855in}{5.057352in}}{\pgfqpoint{3.207245in}{5.046753in}}{\pgfqpoint{3.215059in}{5.038939in}}%
\pgfpathcurveto{\pgfqpoint{3.222872in}{5.031126in}}{\pgfqpoint{3.233471in}{5.026736in}}{\pgfqpoint{3.244521in}{5.026736in}}%
\pgfpathclose%
\pgfusepath{stroke,fill}%
\end{pgfscope}%
\begin{pgfscope}%
\pgfpathrectangle{\pgfqpoint{0.570343in}{0.331635in}}{\pgfqpoint{9.300000in}{7.700000in}}%
\pgfusepath{clip}%
\pgfsetbuttcap%
\pgfsetroundjoin%
\definecolor{currentfill}{rgb}{0.631373,0.788235,0.956863}%
\pgfsetfillcolor{currentfill}%
\pgfsetlinewidth{0.481800pt}%
\definecolor{currentstroke}{rgb}{1.000000,1.000000,1.000000}%
\pgfsetstrokecolor{currentstroke}%
\pgfsetdash{}{0pt}%
\pgfpathmoveto{\pgfqpoint{1.313094in}{3.337712in}}%
\pgfpathcurveto{\pgfqpoint{1.324144in}{3.337712in}}{\pgfqpoint{1.334744in}{3.342102in}}{\pgfqpoint{1.342557in}{3.349916in}}%
\pgfpathcurveto{\pgfqpoint{1.350371in}{3.357729in}}{\pgfqpoint{1.354761in}{3.368328in}}{\pgfqpoint{1.354761in}{3.379378in}}%
\pgfpathcurveto{\pgfqpoint{1.354761in}{3.390429in}}{\pgfqpoint{1.350371in}{3.401028in}}{\pgfqpoint{1.342557in}{3.408841in}}%
\pgfpathcurveto{\pgfqpoint{1.334744in}{3.416655in}}{\pgfqpoint{1.324144in}{3.421045in}}{\pgfqpoint{1.313094in}{3.421045in}}%
\pgfpathcurveto{\pgfqpoint{1.302044in}{3.421045in}}{\pgfqpoint{1.291445in}{3.416655in}}{\pgfqpoint{1.283632in}{3.408841in}}%
\pgfpathcurveto{\pgfqpoint{1.275818in}{3.401028in}}{\pgfqpoint{1.271428in}{3.390429in}}{\pgfqpoint{1.271428in}{3.379378in}}%
\pgfpathcurveto{\pgfqpoint{1.271428in}{3.368328in}}{\pgfqpoint{1.275818in}{3.357729in}}{\pgfqpoint{1.283632in}{3.349916in}}%
\pgfpathcurveto{\pgfqpoint{1.291445in}{3.342102in}}{\pgfqpoint{1.302044in}{3.337712in}}{\pgfqpoint{1.313094in}{3.337712in}}%
\pgfpathclose%
\pgfusepath{stroke,fill}%
\end{pgfscope}%
\begin{pgfscope}%
\pgfpathrectangle{\pgfqpoint{0.570343in}{0.331635in}}{\pgfqpoint{9.300000in}{7.700000in}}%
\pgfusepath{clip}%
\pgfsetbuttcap%
\pgfsetroundjoin%
\definecolor{currentfill}{rgb}{0.631373,0.788235,0.956863}%
\pgfsetfillcolor{currentfill}%
\pgfsetlinewidth{0.481800pt}%
\definecolor{currentstroke}{rgb}{1.000000,1.000000,1.000000}%
\pgfsetstrokecolor{currentstroke}%
\pgfsetdash{}{0pt}%
\pgfpathmoveto{\pgfqpoint{2.996308in}{4.456367in}}%
\pgfpathcurveto{\pgfqpoint{3.007358in}{4.456367in}}{\pgfqpoint{3.017957in}{4.460757in}}{\pgfqpoint{3.025771in}{4.468571in}}%
\pgfpathcurveto{\pgfqpoint{3.033584in}{4.476384in}}{\pgfqpoint{3.037975in}{4.486983in}}{\pgfqpoint{3.037975in}{4.498033in}}%
\pgfpathcurveto{\pgfqpoint{3.037975in}{4.509083in}}{\pgfqpoint{3.033584in}{4.519682in}}{\pgfqpoint{3.025771in}{4.527496in}}%
\pgfpathcurveto{\pgfqpoint{3.017957in}{4.535310in}}{\pgfqpoint{3.007358in}{4.539700in}}{\pgfqpoint{2.996308in}{4.539700in}}%
\pgfpathcurveto{\pgfqpoint{2.985258in}{4.539700in}}{\pgfqpoint{2.974659in}{4.535310in}}{\pgfqpoint{2.966845in}{4.527496in}}%
\pgfpathcurveto{\pgfqpoint{2.959032in}{4.519682in}}{\pgfqpoint{2.954641in}{4.509083in}}{\pgfqpoint{2.954641in}{4.498033in}}%
\pgfpathcurveto{\pgfqpoint{2.954641in}{4.486983in}}{\pgfqpoint{2.959032in}{4.476384in}}{\pgfqpoint{2.966845in}{4.468571in}}%
\pgfpathcurveto{\pgfqpoint{2.974659in}{4.460757in}}{\pgfqpoint{2.985258in}{4.456367in}}{\pgfqpoint{2.996308in}{4.456367in}}%
\pgfpathclose%
\pgfusepath{stroke,fill}%
\end{pgfscope}%
\begin{pgfscope}%
\pgfpathrectangle{\pgfqpoint{0.570343in}{0.331635in}}{\pgfqpoint{9.300000in}{7.700000in}}%
\pgfusepath{clip}%
\pgfsetbuttcap%
\pgfsetroundjoin%
\definecolor{currentfill}{rgb}{0.631373,0.788235,0.956863}%
\pgfsetfillcolor{currentfill}%
\pgfsetlinewidth{0.481800pt}%
\definecolor{currentstroke}{rgb}{1.000000,1.000000,1.000000}%
\pgfsetstrokecolor{currentstroke}%
\pgfsetdash{}{0pt}%
\pgfpathmoveto{\pgfqpoint{8.746558in}{4.749872in}}%
\pgfpathcurveto{\pgfqpoint{8.757608in}{4.749872in}}{\pgfqpoint{8.768207in}{4.754262in}}{\pgfqpoint{8.776020in}{4.762076in}}%
\pgfpathcurveto{\pgfqpoint{8.783834in}{4.769889in}}{\pgfqpoint{8.788224in}{4.780488in}}{\pgfqpoint{8.788224in}{4.791538in}}%
\pgfpathcurveto{\pgfqpoint{8.788224in}{4.802589in}}{\pgfqpoint{8.783834in}{4.813188in}}{\pgfqpoint{8.776020in}{4.821001in}}%
\pgfpathcurveto{\pgfqpoint{8.768207in}{4.828815in}}{\pgfqpoint{8.757608in}{4.833205in}}{\pgfqpoint{8.746558in}{4.833205in}}%
\pgfpathcurveto{\pgfqpoint{8.735508in}{4.833205in}}{\pgfqpoint{8.724908in}{4.828815in}}{\pgfqpoint{8.717095in}{4.821001in}}%
\pgfpathcurveto{\pgfqpoint{8.709281in}{4.813188in}}{\pgfqpoint{8.704891in}{4.802589in}}{\pgfqpoint{8.704891in}{4.791538in}}%
\pgfpathcurveto{\pgfqpoint{8.704891in}{4.780488in}}{\pgfqpoint{8.709281in}{4.769889in}}{\pgfqpoint{8.717095in}{4.762076in}}%
\pgfpathcurveto{\pgfqpoint{8.724908in}{4.754262in}}{\pgfqpoint{8.735508in}{4.749872in}}{\pgfqpoint{8.746558in}{4.749872in}}%
\pgfpathclose%
\pgfusepath{stroke,fill}%
\end{pgfscope}%
\begin{pgfscope}%
\pgfpathrectangle{\pgfqpoint{0.570343in}{0.331635in}}{\pgfqpoint{9.300000in}{7.700000in}}%
\pgfusepath{clip}%
\pgfsetbuttcap%
\pgfsetroundjoin%
\definecolor{currentfill}{rgb}{0.631373,0.788235,0.956863}%
\pgfsetfillcolor{currentfill}%
\pgfsetlinewidth{0.481800pt}%
\definecolor{currentstroke}{rgb}{1.000000,1.000000,1.000000}%
\pgfsetstrokecolor{currentstroke}%
\pgfsetdash{}{0pt}%
\pgfpathmoveto{\pgfqpoint{7.450593in}{5.331593in}}%
\pgfpathcurveto{\pgfqpoint{7.461643in}{5.331593in}}{\pgfqpoint{7.472242in}{5.335983in}}{\pgfqpoint{7.480056in}{5.343797in}}%
\pgfpathcurveto{\pgfqpoint{7.487870in}{5.351610in}}{\pgfqpoint{7.492260in}{5.362209in}}{\pgfqpoint{7.492260in}{5.373260in}}%
\pgfpathcurveto{\pgfqpoint{7.492260in}{5.384310in}}{\pgfqpoint{7.487870in}{5.394909in}}{\pgfqpoint{7.480056in}{5.402722in}}%
\pgfpathcurveto{\pgfqpoint{7.472242in}{5.410536in}}{\pgfqpoint{7.461643in}{5.414926in}}{\pgfqpoint{7.450593in}{5.414926in}}%
\pgfpathcurveto{\pgfqpoint{7.439543in}{5.414926in}}{\pgfqpoint{7.428944in}{5.410536in}}{\pgfqpoint{7.421131in}{5.402722in}}%
\pgfpathcurveto{\pgfqpoint{7.413317in}{5.394909in}}{\pgfqpoint{7.408927in}{5.384310in}}{\pgfqpoint{7.408927in}{5.373260in}}%
\pgfpathcurveto{\pgfqpoint{7.408927in}{5.362209in}}{\pgfqpoint{7.413317in}{5.351610in}}{\pgfqpoint{7.421131in}{5.343797in}}%
\pgfpathcurveto{\pgfqpoint{7.428944in}{5.335983in}}{\pgfqpoint{7.439543in}{5.331593in}}{\pgfqpoint{7.450593in}{5.331593in}}%
\pgfpathclose%
\pgfusepath{stroke,fill}%
\end{pgfscope}%
\begin{pgfscope}%
\pgfpathrectangle{\pgfqpoint{0.570343in}{0.331635in}}{\pgfqpoint{9.300000in}{7.700000in}}%
\pgfusepath{clip}%
\pgfsetbuttcap%
\pgfsetroundjoin%
\definecolor{currentfill}{rgb}{0.631373,0.788235,0.956863}%
\pgfsetfillcolor{currentfill}%
\pgfsetlinewidth{0.481800pt}%
\definecolor{currentstroke}{rgb}{1.000000,1.000000,1.000000}%
\pgfsetstrokecolor{currentstroke}%
\pgfsetdash{}{0pt}%
\pgfpathmoveto{\pgfqpoint{1.930672in}{1.793024in}}%
\pgfpathcurveto{\pgfqpoint{1.941722in}{1.793024in}}{\pgfqpoint{1.952321in}{1.797414in}}{\pgfqpoint{1.960134in}{1.805228in}}%
\pgfpathcurveto{\pgfqpoint{1.967948in}{1.813042in}}{\pgfqpoint{1.972338in}{1.823641in}}{\pgfqpoint{1.972338in}{1.834691in}}%
\pgfpathcurveto{\pgfqpoint{1.972338in}{1.845741in}}{\pgfqpoint{1.967948in}{1.856340in}}{\pgfqpoint{1.960134in}{1.864154in}}%
\pgfpathcurveto{\pgfqpoint{1.952321in}{1.871967in}}{\pgfqpoint{1.941722in}{1.876358in}}{\pgfqpoint{1.930672in}{1.876358in}}%
\pgfpathcurveto{\pgfqpoint{1.919621in}{1.876358in}}{\pgfqpoint{1.909022in}{1.871967in}}{\pgfqpoint{1.901209in}{1.864154in}}%
\pgfpathcurveto{\pgfqpoint{1.893395in}{1.856340in}}{\pgfqpoint{1.889005in}{1.845741in}}{\pgfqpoint{1.889005in}{1.834691in}}%
\pgfpathcurveto{\pgfqpoint{1.889005in}{1.823641in}}{\pgfqpoint{1.893395in}{1.813042in}}{\pgfqpoint{1.901209in}{1.805228in}}%
\pgfpathcurveto{\pgfqpoint{1.909022in}{1.797414in}}{\pgfqpoint{1.919621in}{1.793024in}}{\pgfqpoint{1.930672in}{1.793024in}}%
\pgfpathclose%
\pgfusepath{stroke,fill}%
\end{pgfscope}%
\begin{pgfscope}%
\pgfpathrectangle{\pgfqpoint{0.570343in}{0.331635in}}{\pgfqpoint{9.300000in}{7.700000in}}%
\pgfusepath{clip}%
\pgfsetbuttcap%
\pgfsetroundjoin%
\definecolor{currentfill}{rgb}{0.631373,0.788235,0.956863}%
\pgfsetfillcolor{currentfill}%
\pgfsetlinewidth{0.481800pt}%
\definecolor{currentstroke}{rgb}{1.000000,1.000000,1.000000}%
\pgfsetstrokecolor{currentstroke}%
\pgfsetdash{}{0pt}%
\pgfpathmoveto{\pgfqpoint{7.537340in}{6.309050in}}%
\pgfpathcurveto{\pgfqpoint{7.548390in}{6.309050in}}{\pgfqpoint{7.558989in}{6.313440in}}{\pgfqpoint{7.566803in}{6.321254in}}%
\pgfpathcurveto{\pgfqpoint{7.574617in}{6.329067in}}{\pgfqpoint{7.579007in}{6.339666in}}{\pgfqpoint{7.579007in}{6.350716in}}%
\pgfpathcurveto{\pgfqpoint{7.579007in}{6.361767in}}{\pgfqpoint{7.574617in}{6.372366in}}{\pgfqpoint{7.566803in}{6.380179in}}%
\pgfpathcurveto{\pgfqpoint{7.558989in}{6.387993in}}{\pgfqpoint{7.548390in}{6.392383in}}{\pgfqpoint{7.537340in}{6.392383in}}%
\pgfpathcurveto{\pgfqpoint{7.526290in}{6.392383in}}{\pgfqpoint{7.515691in}{6.387993in}}{\pgfqpoint{7.507877in}{6.380179in}}%
\pgfpathcurveto{\pgfqpoint{7.500064in}{6.372366in}}{\pgfqpoint{7.495673in}{6.361767in}}{\pgfqpoint{7.495673in}{6.350716in}}%
\pgfpathcurveto{\pgfqpoint{7.495673in}{6.339666in}}{\pgfqpoint{7.500064in}{6.329067in}}{\pgfqpoint{7.507877in}{6.321254in}}%
\pgfpathcurveto{\pgfqpoint{7.515691in}{6.313440in}}{\pgfqpoint{7.526290in}{6.309050in}}{\pgfqpoint{7.537340in}{6.309050in}}%
\pgfpathclose%
\pgfusepath{stroke,fill}%
\end{pgfscope}%
\begin{pgfscope}%
\pgfpathrectangle{\pgfqpoint{0.570343in}{0.331635in}}{\pgfqpoint{9.300000in}{7.700000in}}%
\pgfusepath{clip}%
\pgfsetbuttcap%
\pgfsetroundjoin%
\definecolor{currentfill}{rgb}{0.631373,0.788235,0.956863}%
\pgfsetfillcolor{currentfill}%
\pgfsetlinewidth{0.481800pt}%
\definecolor{currentstroke}{rgb}{1.000000,1.000000,1.000000}%
\pgfsetstrokecolor{currentstroke}%
\pgfsetdash{}{0pt}%
\pgfpathmoveto{\pgfqpoint{4.430255in}{2.646276in}}%
\pgfpathcurveto{\pgfqpoint{4.441305in}{2.646276in}}{\pgfqpoint{4.451904in}{2.650666in}}{\pgfqpoint{4.459717in}{2.658480in}}%
\pgfpathcurveto{\pgfqpoint{4.467531in}{2.666293in}}{\pgfqpoint{4.471921in}{2.676892in}}{\pgfqpoint{4.471921in}{2.687942in}}%
\pgfpathcurveto{\pgfqpoint{4.471921in}{2.698992in}}{\pgfqpoint{4.467531in}{2.709591in}}{\pgfqpoint{4.459717in}{2.717405in}}%
\pgfpathcurveto{\pgfqpoint{4.451904in}{2.725219in}}{\pgfqpoint{4.441305in}{2.729609in}}{\pgfqpoint{4.430255in}{2.729609in}}%
\pgfpathcurveto{\pgfqpoint{4.419205in}{2.729609in}}{\pgfqpoint{4.408606in}{2.725219in}}{\pgfqpoint{4.400792in}{2.717405in}}%
\pgfpathcurveto{\pgfqpoint{4.392978in}{2.709591in}}{\pgfqpoint{4.388588in}{2.698992in}}{\pgfqpoint{4.388588in}{2.687942in}}%
\pgfpathcurveto{\pgfqpoint{4.388588in}{2.676892in}}{\pgfqpoint{4.392978in}{2.666293in}}{\pgfqpoint{4.400792in}{2.658480in}}%
\pgfpathcurveto{\pgfqpoint{4.408606in}{2.650666in}}{\pgfqpoint{4.419205in}{2.646276in}}{\pgfqpoint{4.430255in}{2.646276in}}%
\pgfpathclose%
\pgfusepath{stroke,fill}%
\end{pgfscope}%
\begin{pgfscope}%
\pgfpathrectangle{\pgfqpoint{0.570343in}{0.331635in}}{\pgfqpoint{9.300000in}{7.700000in}}%
\pgfusepath{clip}%
\pgfsetbuttcap%
\pgfsetroundjoin%
\definecolor{currentfill}{rgb}{0.631373,0.788235,0.956863}%
\pgfsetfillcolor{currentfill}%
\pgfsetlinewidth{0.481800pt}%
\definecolor{currentstroke}{rgb}{1.000000,1.000000,1.000000}%
\pgfsetstrokecolor{currentstroke}%
\pgfsetdash{}{0pt}%
\pgfpathmoveto{\pgfqpoint{6.449722in}{7.385043in}}%
\pgfpathcurveto{\pgfqpoint{6.460772in}{7.385043in}}{\pgfqpoint{6.471371in}{7.389434in}}{\pgfqpoint{6.479185in}{7.397247in}}%
\pgfpathcurveto{\pgfqpoint{6.486998in}{7.405061in}}{\pgfqpoint{6.491389in}{7.415660in}}{\pgfqpoint{6.491389in}{7.426710in}}%
\pgfpathcurveto{\pgfqpoint{6.491389in}{7.437760in}}{\pgfqpoint{6.486998in}{7.448359in}}{\pgfqpoint{6.479185in}{7.456173in}}%
\pgfpathcurveto{\pgfqpoint{6.471371in}{7.463987in}}{\pgfqpoint{6.460772in}{7.468377in}}{\pgfqpoint{6.449722in}{7.468377in}}%
\pgfpathcurveto{\pgfqpoint{6.438672in}{7.468377in}}{\pgfqpoint{6.428073in}{7.463987in}}{\pgfqpoint{6.420259in}{7.456173in}}%
\pgfpathcurveto{\pgfqpoint{6.412445in}{7.448359in}}{\pgfqpoint{6.408055in}{7.437760in}}{\pgfqpoint{6.408055in}{7.426710in}}%
\pgfpathcurveto{\pgfqpoint{6.408055in}{7.415660in}}{\pgfqpoint{6.412445in}{7.405061in}}{\pgfqpoint{6.420259in}{7.397247in}}%
\pgfpathcurveto{\pgfqpoint{6.428073in}{7.389434in}}{\pgfqpoint{6.438672in}{7.385043in}}{\pgfqpoint{6.449722in}{7.385043in}}%
\pgfpathclose%
\pgfusepath{stroke,fill}%
\end{pgfscope}%
\begin{pgfscope}%
\pgfpathrectangle{\pgfqpoint{0.570343in}{0.331635in}}{\pgfqpoint{9.300000in}{7.700000in}}%
\pgfusepath{clip}%
\pgfsetbuttcap%
\pgfsetroundjoin%
\definecolor{currentfill}{rgb}{0.631373,0.788235,0.956863}%
\pgfsetfillcolor{currentfill}%
\pgfsetlinewidth{0.481800pt}%
\definecolor{currentstroke}{rgb}{1.000000,1.000000,1.000000}%
\pgfsetstrokecolor{currentstroke}%
\pgfsetdash{}{0pt}%
\pgfpathmoveto{\pgfqpoint{6.510955in}{2.706388in}}%
\pgfpathcurveto{\pgfqpoint{6.522005in}{2.706388in}}{\pgfqpoint{6.532604in}{2.710778in}}{\pgfqpoint{6.540418in}{2.718592in}}%
\pgfpathcurveto{\pgfqpoint{6.548232in}{2.726406in}}{\pgfqpoint{6.552622in}{2.737005in}}{\pgfqpoint{6.552622in}{2.748055in}}%
\pgfpathcurveto{\pgfqpoint{6.552622in}{2.759105in}}{\pgfqpoint{6.548232in}{2.769704in}}{\pgfqpoint{6.540418in}{2.777518in}}%
\pgfpathcurveto{\pgfqpoint{6.532604in}{2.785331in}}{\pgfqpoint{6.522005in}{2.789721in}}{\pgfqpoint{6.510955in}{2.789721in}}%
\pgfpathcurveto{\pgfqpoint{6.499905in}{2.789721in}}{\pgfqpoint{6.489306in}{2.785331in}}{\pgfqpoint{6.481492in}{2.777518in}}%
\pgfpathcurveto{\pgfqpoint{6.473679in}{2.769704in}}{\pgfqpoint{6.469288in}{2.759105in}}{\pgfqpoint{6.469288in}{2.748055in}}%
\pgfpathcurveto{\pgfqpoint{6.469288in}{2.737005in}}{\pgfqpoint{6.473679in}{2.726406in}}{\pgfqpoint{6.481492in}{2.718592in}}%
\pgfpathcurveto{\pgfqpoint{6.489306in}{2.710778in}}{\pgfqpoint{6.499905in}{2.706388in}}{\pgfqpoint{6.510955in}{2.706388in}}%
\pgfpathclose%
\pgfusepath{stroke,fill}%
\end{pgfscope}%
\begin{pgfscope}%
\pgfpathrectangle{\pgfqpoint{0.570343in}{0.331635in}}{\pgfqpoint{9.300000in}{7.700000in}}%
\pgfusepath{clip}%
\pgfsetbuttcap%
\pgfsetroundjoin%
\definecolor{currentfill}{rgb}{0.631373,0.788235,0.956863}%
\pgfsetfillcolor{currentfill}%
\pgfsetlinewidth{0.481800pt}%
\definecolor{currentstroke}{rgb}{1.000000,1.000000,1.000000}%
\pgfsetstrokecolor{currentstroke}%
\pgfsetdash{}{0pt}%
\pgfpathmoveto{\pgfqpoint{1.349034in}{2.477788in}}%
\pgfpathcurveto{\pgfqpoint{1.360084in}{2.477788in}}{\pgfqpoint{1.370683in}{2.482178in}}{\pgfqpoint{1.378497in}{2.489992in}}%
\pgfpathcurveto{\pgfqpoint{1.386311in}{2.497805in}}{\pgfqpoint{1.390701in}{2.508404in}}{\pgfqpoint{1.390701in}{2.519455in}}%
\pgfpathcurveto{\pgfqpoint{1.390701in}{2.530505in}}{\pgfqpoint{1.386311in}{2.541104in}}{\pgfqpoint{1.378497in}{2.548917in}}%
\pgfpathcurveto{\pgfqpoint{1.370683in}{2.556731in}}{\pgfqpoint{1.360084in}{2.561121in}}{\pgfqpoint{1.349034in}{2.561121in}}%
\pgfpathcurveto{\pgfqpoint{1.337984in}{2.561121in}}{\pgfqpoint{1.327385in}{2.556731in}}{\pgfqpoint{1.319572in}{2.548917in}}%
\pgfpathcurveto{\pgfqpoint{1.311758in}{2.541104in}}{\pgfqpoint{1.307368in}{2.530505in}}{\pgfqpoint{1.307368in}{2.519455in}}%
\pgfpathcurveto{\pgfqpoint{1.307368in}{2.508404in}}{\pgfqpoint{1.311758in}{2.497805in}}{\pgfqpoint{1.319572in}{2.489992in}}%
\pgfpathcurveto{\pgfqpoint{1.327385in}{2.482178in}}{\pgfqpoint{1.337984in}{2.477788in}}{\pgfqpoint{1.349034in}{2.477788in}}%
\pgfpathclose%
\pgfusepath{stroke,fill}%
\end{pgfscope}%
\begin{pgfscope}%
\pgfpathrectangle{\pgfqpoint{0.570343in}{0.331635in}}{\pgfqpoint{9.300000in}{7.700000in}}%
\pgfusepath{clip}%
\pgfsetbuttcap%
\pgfsetroundjoin%
\definecolor{currentfill}{rgb}{0.631373,0.788235,0.956863}%
\pgfsetfillcolor{currentfill}%
\pgfsetlinewidth{0.481800pt}%
\definecolor{currentstroke}{rgb}{1.000000,1.000000,1.000000}%
\pgfsetstrokecolor{currentstroke}%
\pgfsetdash{}{0pt}%
\pgfpathmoveto{\pgfqpoint{6.043529in}{4.558192in}}%
\pgfpathcurveto{\pgfqpoint{6.054579in}{4.558192in}}{\pgfqpoint{6.065178in}{4.562583in}}{\pgfqpoint{6.072991in}{4.570396in}}%
\pgfpathcurveto{\pgfqpoint{6.080805in}{4.578210in}}{\pgfqpoint{6.085195in}{4.588809in}}{\pgfqpoint{6.085195in}{4.599859in}}%
\pgfpathcurveto{\pgfqpoint{6.085195in}{4.610909in}}{\pgfqpoint{6.080805in}{4.621508in}}{\pgfqpoint{6.072991in}{4.629322in}}%
\pgfpathcurveto{\pgfqpoint{6.065178in}{4.637135in}}{\pgfqpoint{6.054579in}{4.641526in}}{\pgfqpoint{6.043529in}{4.641526in}}%
\pgfpathcurveto{\pgfqpoint{6.032478in}{4.641526in}}{\pgfqpoint{6.021879in}{4.637135in}}{\pgfqpoint{6.014066in}{4.629322in}}%
\pgfpathcurveto{\pgfqpoint{6.006252in}{4.621508in}}{\pgfqpoint{6.001862in}{4.610909in}}{\pgfqpoint{6.001862in}{4.599859in}}%
\pgfpathcurveto{\pgfqpoint{6.001862in}{4.588809in}}{\pgfqpoint{6.006252in}{4.578210in}}{\pgfqpoint{6.014066in}{4.570396in}}%
\pgfpathcurveto{\pgfqpoint{6.021879in}{4.562583in}}{\pgfqpoint{6.032478in}{4.558192in}}{\pgfqpoint{6.043529in}{4.558192in}}%
\pgfpathclose%
\pgfusepath{stroke,fill}%
\end{pgfscope}%
\begin{pgfscope}%
\pgfpathrectangle{\pgfqpoint{0.570343in}{0.331635in}}{\pgfqpoint{9.300000in}{7.700000in}}%
\pgfusepath{clip}%
\pgfsetbuttcap%
\pgfsetroundjoin%
\definecolor{currentfill}{rgb}{0.631373,0.788235,0.956863}%
\pgfsetfillcolor{currentfill}%
\pgfsetlinewidth{0.481800pt}%
\definecolor{currentstroke}{rgb}{1.000000,1.000000,1.000000}%
\pgfsetstrokecolor{currentstroke}%
\pgfsetdash{}{0pt}%
\pgfpathmoveto{\pgfqpoint{3.476646in}{5.812205in}}%
\pgfpathcurveto{\pgfqpoint{3.487696in}{5.812205in}}{\pgfqpoint{3.498295in}{5.816595in}}{\pgfqpoint{3.506109in}{5.824409in}}%
\pgfpathcurveto{\pgfqpoint{3.513922in}{5.832222in}}{\pgfqpoint{3.518312in}{5.842822in}}{\pgfqpoint{3.518312in}{5.853872in}}%
\pgfpathcurveto{\pgfqpoint{3.518312in}{5.864922in}}{\pgfqpoint{3.513922in}{5.875521in}}{\pgfqpoint{3.506109in}{5.883334in}}%
\pgfpathcurveto{\pgfqpoint{3.498295in}{5.891148in}}{\pgfqpoint{3.487696in}{5.895538in}}{\pgfqpoint{3.476646in}{5.895538in}}%
\pgfpathcurveto{\pgfqpoint{3.465596in}{5.895538in}}{\pgfqpoint{3.454997in}{5.891148in}}{\pgfqpoint{3.447183in}{5.883334in}}%
\pgfpathcurveto{\pgfqpoint{3.439369in}{5.875521in}}{\pgfqpoint{3.434979in}{5.864922in}}{\pgfqpoint{3.434979in}{5.853872in}}%
\pgfpathcurveto{\pgfqpoint{3.434979in}{5.842822in}}{\pgfqpoint{3.439369in}{5.832222in}}{\pgfqpoint{3.447183in}{5.824409in}}%
\pgfpathcurveto{\pgfqpoint{3.454997in}{5.816595in}}{\pgfqpoint{3.465596in}{5.812205in}}{\pgfqpoint{3.476646in}{5.812205in}}%
\pgfpathclose%
\pgfusepath{stroke,fill}%
\end{pgfscope}%
\begin{pgfscope}%
\pgfpathrectangle{\pgfqpoint{0.570343in}{0.331635in}}{\pgfqpoint{9.300000in}{7.700000in}}%
\pgfusepath{clip}%
\pgfsetbuttcap%
\pgfsetroundjoin%
\definecolor{currentfill}{rgb}{0.631373,0.788235,0.956863}%
\pgfsetfillcolor{currentfill}%
\pgfsetlinewidth{0.481800pt}%
\definecolor{currentstroke}{rgb}{1.000000,1.000000,1.000000}%
\pgfsetstrokecolor{currentstroke}%
\pgfsetdash{}{0pt}%
\pgfpathmoveto{\pgfqpoint{7.982207in}{3.310373in}}%
\pgfpathcurveto{\pgfqpoint{7.993257in}{3.310373in}}{\pgfqpoint{8.003856in}{3.314763in}}{\pgfqpoint{8.011670in}{3.322577in}}%
\pgfpathcurveto{\pgfqpoint{8.019483in}{3.330391in}}{\pgfqpoint{8.023874in}{3.340990in}}{\pgfqpoint{8.023874in}{3.352040in}}%
\pgfpathcurveto{\pgfqpoint{8.023874in}{3.363090in}}{\pgfqpoint{8.019483in}{3.373689in}}{\pgfqpoint{8.011670in}{3.381503in}}%
\pgfpathcurveto{\pgfqpoint{8.003856in}{3.389316in}}{\pgfqpoint{7.993257in}{3.393706in}}{\pgfqpoint{7.982207in}{3.393706in}}%
\pgfpathcurveto{\pgfqpoint{7.971157in}{3.393706in}}{\pgfqpoint{7.960558in}{3.389316in}}{\pgfqpoint{7.952744in}{3.381503in}}%
\pgfpathcurveto{\pgfqpoint{7.944931in}{3.373689in}}{\pgfqpoint{7.940540in}{3.363090in}}{\pgfqpoint{7.940540in}{3.352040in}}%
\pgfpathcurveto{\pgfqpoint{7.940540in}{3.340990in}}{\pgfqpoint{7.944931in}{3.330391in}}{\pgfqpoint{7.952744in}{3.322577in}}%
\pgfpathcurveto{\pgfqpoint{7.960558in}{3.314763in}}{\pgfqpoint{7.971157in}{3.310373in}}{\pgfqpoint{7.982207in}{3.310373in}}%
\pgfpathclose%
\pgfusepath{stroke,fill}%
\end{pgfscope}%
\begin{pgfscope}%
\pgfpathrectangle{\pgfqpoint{0.570343in}{0.331635in}}{\pgfqpoint{9.300000in}{7.700000in}}%
\pgfusepath{clip}%
\pgfsetbuttcap%
\pgfsetroundjoin%
\definecolor{currentfill}{rgb}{0.631373,0.788235,0.956863}%
\pgfsetfillcolor{currentfill}%
\pgfsetlinewidth{0.481800pt}%
\definecolor{currentstroke}{rgb}{1.000000,1.000000,1.000000}%
\pgfsetstrokecolor{currentstroke}%
\pgfsetdash{}{0pt}%
\pgfpathmoveto{\pgfqpoint{2.053144in}{5.525128in}}%
\pgfpathcurveto{\pgfqpoint{2.064194in}{5.525128in}}{\pgfqpoint{2.074793in}{5.529518in}}{\pgfqpoint{2.082607in}{5.537332in}}%
\pgfpathcurveto{\pgfqpoint{2.090421in}{5.545145in}}{\pgfqpoint{2.094811in}{5.555744in}}{\pgfqpoint{2.094811in}{5.566795in}}%
\pgfpathcurveto{\pgfqpoint{2.094811in}{5.577845in}}{\pgfqpoint{2.090421in}{5.588444in}}{\pgfqpoint{2.082607in}{5.596257in}}%
\pgfpathcurveto{\pgfqpoint{2.074793in}{5.604071in}}{\pgfqpoint{2.064194in}{5.608461in}}{\pgfqpoint{2.053144in}{5.608461in}}%
\pgfpathcurveto{\pgfqpoint{2.042094in}{5.608461in}}{\pgfqpoint{2.031495in}{5.604071in}}{\pgfqpoint{2.023681in}{5.596257in}}%
\pgfpathcurveto{\pgfqpoint{2.015868in}{5.588444in}}{\pgfqpoint{2.011478in}{5.577845in}}{\pgfqpoint{2.011478in}{5.566795in}}%
\pgfpathcurveto{\pgfqpoint{2.011478in}{5.555744in}}{\pgfqpoint{2.015868in}{5.545145in}}{\pgfqpoint{2.023681in}{5.537332in}}%
\pgfpathcurveto{\pgfqpoint{2.031495in}{5.529518in}}{\pgfqpoint{2.042094in}{5.525128in}}{\pgfqpoint{2.053144in}{5.525128in}}%
\pgfpathclose%
\pgfusepath{stroke,fill}%
\end{pgfscope}%
\begin{pgfscope}%
\pgfpathrectangle{\pgfqpoint{0.570343in}{0.331635in}}{\pgfqpoint{9.300000in}{7.700000in}}%
\pgfusepath{clip}%
\pgfsetbuttcap%
\pgfsetroundjoin%
\definecolor{currentfill}{rgb}{0.631373,0.788235,0.956863}%
\pgfsetfillcolor{currentfill}%
\pgfsetlinewidth{0.481800pt}%
\definecolor{currentstroke}{rgb}{1.000000,1.000000,1.000000}%
\pgfsetstrokecolor{currentstroke}%
\pgfsetdash{}{0pt}%
\pgfpathmoveto{\pgfqpoint{2.355215in}{3.763243in}}%
\pgfpathcurveto{\pgfqpoint{2.366265in}{3.763243in}}{\pgfqpoint{2.376864in}{3.767633in}}{\pgfqpoint{2.384678in}{3.775447in}}%
\pgfpathcurveto{\pgfqpoint{2.392491in}{3.783260in}}{\pgfqpoint{2.396881in}{3.793859in}}{\pgfqpoint{2.396881in}{3.804910in}}%
\pgfpathcurveto{\pgfqpoint{2.396881in}{3.815960in}}{\pgfqpoint{2.392491in}{3.826559in}}{\pgfqpoint{2.384678in}{3.834372in}}%
\pgfpathcurveto{\pgfqpoint{2.376864in}{3.842186in}}{\pgfqpoint{2.366265in}{3.846576in}}{\pgfqpoint{2.355215in}{3.846576in}}%
\pgfpathcurveto{\pgfqpoint{2.344165in}{3.846576in}}{\pgfqpoint{2.333566in}{3.842186in}}{\pgfqpoint{2.325752in}{3.834372in}}%
\pgfpathcurveto{\pgfqpoint{2.317938in}{3.826559in}}{\pgfqpoint{2.313548in}{3.815960in}}{\pgfqpoint{2.313548in}{3.804910in}}%
\pgfpathcurveto{\pgfqpoint{2.313548in}{3.793859in}}{\pgfqpoint{2.317938in}{3.783260in}}{\pgfqpoint{2.325752in}{3.775447in}}%
\pgfpathcurveto{\pgfqpoint{2.333566in}{3.767633in}}{\pgfqpoint{2.344165in}{3.763243in}}{\pgfqpoint{2.355215in}{3.763243in}}%
\pgfpathclose%
\pgfusepath{stroke,fill}%
\end{pgfscope}%
\begin{pgfscope}%
\pgfpathrectangle{\pgfqpoint{0.570343in}{0.331635in}}{\pgfqpoint{9.300000in}{7.700000in}}%
\pgfusepath{clip}%
\pgfsetbuttcap%
\pgfsetroundjoin%
\definecolor{currentfill}{rgb}{0.631373,0.788235,0.956863}%
\pgfsetfillcolor{currentfill}%
\pgfsetlinewidth{0.481800pt}%
\definecolor{currentstroke}{rgb}{1.000000,1.000000,1.000000}%
\pgfsetstrokecolor{currentstroke}%
\pgfsetdash{}{0pt}%
\pgfpathmoveto{\pgfqpoint{5.553256in}{2.854049in}}%
\pgfpathcurveto{\pgfqpoint{5.564306in}{2.854049in}}{\pgfqpoint{5.574905in}{2.858439in}}{\pgfqpoint{5.582718in}{2.866253in}}%
\pgfpathcurveto{\pgfqpoint{5.590532in}{2.874066in}}{\pgfqpoint{5.594922in}{2.884665in}}{\pgfqpoint{5.594922in}{2.895716in}}%
\pgfpathcurveto{\pgfqpoint{5.594922in}{2.906766in}}{\pgfqpoint{5.590532in}{2.917365in}}{\pgfqpoint{5.582718in}{2.925178in}}%
\pgfpathcurveto{\pgfqpoint{5.574905in}{2.932992in}}{\pgfqpoint{5.564306in}{2.937382in}}{\pgfqpoint{5.553256in}{2.937382in}}%
\pgfpathcurveto{\pgfqpoint{5.542206in}{2.937382in}}{\pgfqpoint{5.531607in}{2.932992in}}{\pgfqpoint{5.523793in}{2.925178in}}%
\pgfpathcurveto{\pgfqpoint{5.515979in}{2.917365in}}{\pgfqpoint{5.511589in}{2.906766in}}{\pgfqpoint{5.511589in}{2.895716in}}%
\pgfpathcurveto{\pgfqpoint{5.511589in}{2.884665in}}{\pgfqpoint{5.515979in}{2.874066in}}{\pgfqpoint{5.523793in}{2.866253in}}%
\pgfpathcurveto{\pgfqpoint{5.531607in}{2.858439in}}{\pgfqpoint{5.542206in}{2.854049in}}{\pgfqpoint{5.553256in}{2.854049in}}%
\pgfpathclose%
\pgfusepath{stroke,fill}%
\end{pgfscope}%
\begin{pgfscope}%
\pgfpathrectangle{\pgfqpoint{0.570343in}{0.331635in}}{\pgfqpoint{9.300000in}{7.700000in}}%
\pgfusepath{clip}%
\pgfsetbuttcap%
\pgfsetroundjoin%
\definecolor{currentfill}{rgb}{0.631373,0.788235,0.956863}%
\pgfsetfillcolor{currentfill}%
\pgfsetlinewidth{0.481800pt}%
\definecolor{currentstroke}{rgb}{1.000000,1.000000,1.000000}%
\pgfsetstrokecolor{currentstroke}%
\pgfsetdash{}{0pt}%
\pgfpathmoveto{\pgfqpoint{1.164286in}{3.855313in}}%
\pgfpathcurveto{\pgfqpoint{1.175336in}{3.855313in}}{\pgfqpoint{1.185935in}{3.859704in}}{\pgfqpoint{1.193748in}{3.867517in}}%
\pgfpathcurveto{\pgfqpoint{1.201562in}{3.875331in}}{\pgfqpoint{1.205952in}{3.885930in}}{\pgfqpoint{1.205952in}{3.896980in}}%
\pgfpathcurveto{\pgfqpoint{1.205952in}{3.908030in}}{\pgfqpoint{1.201562in}{3.918629in}}{\pgfqpoint{1.193748in}{3.926443in}}%
\pgfpathcurveto{\pgfqpoint{1.185935in}{3.934257in}}{\pgfqpoint{1.175336in}{3.938647in}}{\pgfqpoint{1.164286in}{3.938647in}}%
\pgfpathcurveto{\pgfqpoint{1.153236in}{3.938647in}}{\pgfqpoint{1.142636in}{3.934257in}}{\pgfqpoint{1.134823in}{3.926443in}}%
\pgfpathcurveto{\pgfqpoint{1.127009in}{3.918629in}}{\pgfqpoint{1.122619in}{3.908030in}}{\pgfqpoint{1.122619in}{3.896980in}}%
\pgfpathcurveto{\pgfqpoint{1.122619in}{3.885930in}}{\pgfqpoint{1.127009in}{3.875331in}}{\pgfqpoint{1.134823in}{3.867517in}}%
\pgfpathcurveto{\pgfqpoint{1.142636in}{3.859704in}}{\pgfqpoint{1.153236in}{3.855313in}}{\pgfqpoint{1.164286in}{3.855313in}}%
\pgfpathclose%
\pgfusepath{stroke,fill}%
\end{pgfscope}%
\begin{pgfscope}%
\pgfpathrectangle{\pgfqpoint{0.570343in}{0.331635in}}{\pgfqpoint{9.300000in}{7.700000in}}%
\pgfusepath{clip}%
\pgfsetbuttcap%
\pgfsetroundjoin%
\definecolor{currentfill}{rgb}{0.631373,0.788235,0.956863}%
\pgfsetfillcolor{currentfill}%
\pgfsetlinewidth{0.481800pt}%
\definecolor{currentstroke}{rgb}{1.000000,1.000000,1.000000}%
\pgfsetstrokecolor{currentstroke}%
\pgfsetdash{}{0pt}%
\pgfpathmoveto{\pgfqpoint{2.482835in}{6.278622in}}%
\pgfpathcurveto{\pgfqpoint{2.493885in}{6.278622in}}{\pgfqpoint{2.504484in}{6.283012in}}{\pgfqpoint{2.512298in}{6.290826in}}%
\pgfpathcurveto{\pgfqpoint{2.520112in}{6.298640in}}{\pgfqpoint{2.524502in}{6.309239in}}{\pgfqpoint{2.524502in}{6.320289in}}%
\pgfpathcurveto{\pgfqpoint{2.524502in}{6.331339in}}{\pgfqpoint{2.520112in}{6.341938in}}{\pgfqpoint{2.512298in}{6.349751in}}%
\pgfpathcurveto{\pgfqpoint{2.504484in}{6.357565in}}{\pgfqpoint{2.493885in}{6.361955in}}{\pgfqpoint{2.482835in}{6.361955in}}%
\pgfpathcurveto{\pgfqpoint{2.471785in}{6.361955in}}{\pgfqpoint{2.461186in}{6.357565in}}{\pgfqpoint{2.453372in}{6.349751in}}%
\pgfpathcurveto{\pgfqpoint{2.445559in}{6.341938in}}{\pgfqpoint{2.441169in}{6.331339in}}{\pgfqpoint{2.441169in}{6.320289in}}%
\pgfpathcurveto{\pgfqpoint{2.441169in}{6.309239in}}{\pgfqpoint{2.445559in}{6.298640in}}{\pgfqpoint{2.453372in}{6.290826in}}%
\pgfpathcurveto{\pgfqpoint{2.461186in}{6.283012in}}{\pgfqpoint{2.471785in}{6.278622in}}{\pgfqpoint{2.482835in}{6.278622in}}%
\pgfpathclose%
\pgfusepath{stroke,fill}%
\end{pgfscope}%
\begin{pgfscope}%
\pgfpathrectangle{\pgfqpoint{0.570343in}{0.331635in}}{\pgfqpoint{9.300000in}{7.700000in}}%
\pgfusepath{clip}%
\pgfsetbuttcap%
\pgfsetroundjoin%
\definecolor{currentfill}{rgb}{0.631373,0.788235,0.956863}%
\pgfsetfillcolor{currentfill}%
\pgfsetlinewidth{0.481800pt}%
\definecolor{currentstroke}{rgb}{1.000000,1.000000,1.000000}%
\pgfsetstrokecolor{currentstroke}%
\pgfsetdash{}{0pt}%
\pgfpathmoveto{\pgfqpoint{3.630943in}{4.181517in}}%
\pgfpathcurveto{\pgfqpoint{3.641994in}{4.181517in}}{\pgfqpoint{3.652593in}{4.185907in}}{\pgfqpoint{3.660406in}{4.193721in}}%
\pgfpathcurveto{\pgfqpoint{3.668220in}{4.201534in}}{\pgfqpoint{3.672610in}{4.212134in}}{\pgfqpoint{3.672610in}{4.223184in}}%
\pgfpathcurveto{\pgfqpoint{3.672610in}{4.234234in}}{\pgfqpoint{3.668220in}{4.244833in}}{\pgfqpoint{3.660406in}{4.252646in}}%
\pgfpathcurveto{\pgfqpoint{3.652593in}{4.260460in}}{\pgfqpoint{3.641994in}{4.264850in}}{\pgfqpoint{3.630943in}{4.264850in}}%
\pgfpathcurveto{\pgfqpoint{3.619893in}{4.264850in}}{\pgfqpoint{3.609294in}{4.260460in}}{\pgfqpoint{3.601481in}{4.252646in}}%
\pgfpathcurveto{\pgfqpoint{3.593667in}{4.244833in}}{\pgfqpoint{3.589277in}{4.234234in}}{\pgfqpoint{3.589277in}{4.223184in}}%
\pgfpathcurveto{\pgfqpoint{3.589277in}{4.212134in}}{\pgfqpoint{3.593667in}{4.201534in}}{\pgfqpoint{3.601481in}{4.193721in}}%
\pgfpathcurveto{\pgfqpoint{3.609294in}{4.185907in}}{\pgfqpoint{3.619893in}{4.181517in}}{\pgfqpoint{3.630943in}{4.181517in}}%
\pgfpathclose%
\pgfusepath{stroke,fill}%
\end{pgfscope}%
\begin{pgfscope}%
\pgfpathrectangle{\pgfqpoint{0.570343in}{0.331635in}}{\pgfqpoint{9.300000in}{7.700000in}}%
\pgfusepath{clip}%
\pgfsetbuttcap%
\pgfsetroundjoin%
\definecolor{currentfill}{rgb}{0.631373,0.788235,0.956863}%
\pgfsetfillcolor{currentfill}%
\pgfsetlinewidth{0.481800pt}%
\definecolor{currentstroke}{rgb}{1.000000,1.000000,1.000000}%
\pgfsetstrokecolor{currentstroke}%
\pgfsetdash{}{0pt}%
\pgfpathmoveto{\pgfqpoint{4.241907in}{3.119631in}}%
\pgfpathcurveto{\pgfqpoint{4.252957in}{3.119631in}}{\pgfqpoint{4.263556in}{3.124021in}}{\pgfqpoint{4.271369in}{3.131834in}}%
\pgfpathcurveto{\pgfqpoint{4.279183in}{3.139648in}}{\pgfqpoint{4.283573in}{3.150247in}}{\pgfqpoint{4.283573in}{3.161297in}}%
\pgfpathcurveto{\pgfqpoint{4.283573in}{3.172347in}}{\pgfqpoint{4.279183in}{3.182946in}}{\pgfqpoint{4.271369in}{3.190760in}}%
\pgfpathcurveto{\pgfqpoint{4.263556in}{3.198574in}}{\pgfqpoint{4.252957in}{3.202964in}}{\pgfqpoint{4.241907in}{3.202964in}}%
\pgfpathcurveto{\pgfqpoint{4.230857in}{3.202964in}}{\pgfqpoint{4.220258in}{3.198574in}}{\pgfqpoint{4.212444in}{3.190760in}}%
\pgfpathcurveto{\pgfqpoint{4.204630in}{3.182946in}}{\pgfqpoint{4.200240in}{3.172347in}}{\pgfqpoint{4.200240in}{3.161297in}}%
\pgfpathcurveto{\pgfqpoint{4.200240in}{3.150247in}}{\pgfqpoint{4.204630in}{3.139648in}}{\pgfqpoint{4.212444in}{3.131834in}}%
\pgfpathcurveto{\pgfqpoint{4.220258in}{3.124021in}}{\pgfqpoint{4.230857in}{3.119631in}}{\pgfqpoint{4.241907in}{3.119631in}}%
\pgfpathclose%
\pgfusepath{stroke,fill}%
\end{pgfscope}%
\begin{pgfscope}%
\pgfpathrectangle{\pgfqpoint{0.570343in}{0.331635in}}{\pgfqpoint{9.300000in}{7.700000in}}%
\pgfusepath{clip}%
\pgfsetbuttcap%
\pgfsetroundjoin%
\definecolor{currentfill}{rgb}{0.631373,0.788235,0.956863}%
\pgfsetfillcolor{currentfill}%
\pgfsetlinewidth{0.481800pt}%
\definecolor{currentstroke}{rgb}{1.000000,1.000000,1.000000}%
\pgfsetstrokecolor{currentstroke}%
\pgfsetdash{}{0pt}%
\pgfpathmoveto{\pgfqpoint{6.802935in}{5.680244in}}%
\pgfpathcurveto{\pgfqpoint{6.813985in}{5.680244in}}{\pgfqpoint{6.824584in}{5.684635in}}{\pgfqpoint{6.832398in}{5.692448in}}%
\pgfpathcurveto{\pgfqpoint{6.840212in}{5.700262in}}{\pgfqpoint{6.844602in}{5.710861in}}{\pgfqpoint{6.844602in}{5.721911in}}%
\pgfpathcurveto{\pgfqpoint{6.844602in}{5.732961in}}{\pgfqpoint{6.840212in}{5.743560in}}{\pgfqpoint{6.832398in}{5.751374in}}%
\pgfpathcurveto{\pgfqpoint{6.824584in}{5.759187in}}{\pgfqpoint{6.813985in}{5.763578in}}{\pgfqpoint{6.802935in}{5.763578in}}%
\pgfpathcurveto{\pgfqpoint{6.791885in}{5.763578in}}{\pgfqpoint{6.781286in}{5.759187in}}{\pgfqpoint{6.773473in}{5.751374in}}%
\pgfpathcurveto{\pgfqpoint{6.765659in}{5.743560in}}{\pgfqpoint{6.761269in}{5.732961in}}{\pgfqpoint{6.761269in}{5.721911in}}%
\pgfpathcurveto{\pgfqpoint{6.761269in}{5.710861in}}{\pgfqpoint{6.765659in}{5.700262in}}{\pgfqpoint{6.773473in}{5.692448in}}%
\pgfpathcurveto{\pgfqpoint{6.781286in}{5.684635in}}{\pgfqpoint{6.791885in}{5.680244in}}{\pgfqpoint{6.802935in}{5.680244in}}%
\pgfpathclose%
\pgfusepath{stroke,fill}%
\end{pgfscope}%
\begin{pgfscope}%
\pgfpathrectangle{\pgfqpoint{0.570343in}{0.331635in}}{\pgfqpoint{9.300000in}{7.700000in}}%
\pgfusepath{clip}%
\pgfsetbuttcap%
\pgfsetroundjoin%
\definecolor{currentfill}{rgb}{0.631373,0.788235,0.956863}%
\pgfsetfillcolor{currentfill}%
\pgfsetlinewidth{0.481800pt}%
\definecolor{currentstroke}{rgb}{1.000000,1.000000,1.000000}%
\pgfsetstrokecolor{currentstroke}%
\pgfsetdash{}{0pt}%
\pgfpathmoveto{\pgfqpoint{3.985351in}{4.786684in}}%
\pgfpathcurveto{\pgfqpoint{3.996401in}{4.786684in}}{\pgfqpoint{4.007000in}{4.791074in}}{\pgfqpoint{4.014814in}{4.798888in}}%
\pgfpathcurveto{\pgfqpoint{4.022628in}{4.806702in}}{\pgfqpoint{4.027018in}{4.817301in}}{\pgfqpoint{4.027018in}{4.828351in}}%
\pgfpathcurveto{\pgfqpoint{4.027018in}{4.839401in}}{\pgfqpoint{4.022628in}{4.850000in}}{\pgfqpoint{4.014814in}{4.857814in}}%
\pgfpathcurveto{\pgfqpoint{4.007000in}{4.865627in}}{\pgfqpoint{3.996401in}{4.870017in}}{\pgfqpoint{3.985351in}{4.870017in}}%
\pgfpathcurveto{\pgfqpoint{3.974301in}{4.870017in}}{\pgfqpoint{3.963702in}{4.865627in}}{\pgfqpoint{3.955888in}{4.857814in}}%
\pgfpathcurveto{\pgfqpoint{3.948075in}{4.850000in}}{\pgfqpoint{3.943685in}{4.839401in}}{\pgfqpoint{3.943685in}{4.828351in}}%
\pgfpathcurveto{\pgfqpoint{3.943685in}{4.817301in}}{\pgfqpoint{3.948075in}{4.806702in}}{\pgfqpoint{3.955888in}{4.798888in}}%
\pgfpathcurveto{\pgfqpoint{3.963702in}{4.791074in}}{\pgfqpoint{3.974301in}{4.786684in}}{\pgfqpoint{3.985351in}{4.786684in}}%
\pgfpathclose%
\pgfusepath{stroke,fill}%
\end{pgfscope}%
\begin{pgfscope}%
\pgfpathrectangle{\pgfqpoint{0.570343in}{0.331635in}}{\pgfqpoint{9.300000in}{7.700000in}}%
\pgfusepath{clip}%
\pgfsetbuttcap%
\pgfsetroundjoin%
\definecolor{currentfill}{rgb}{0.631373,0.788235,0.956863}%
\pgfsetfillcolor{currentfill}%
\pgfsetlinewidth{0.481800pt}%
\definecolor{currentstroke}{rgb}{1.000000,1.000000,1.000000}%
\pgfsetstrokecolor{currentstroke}%
\pgfsetdash{}{0pt}%
\pgfpathmoveto{\pgfqpoint{8.385485in}{2.949338in}}%
\pgfpathcurveto{\pgfqpoint{8.396535in}{2.949338in}}{\pgfqpoint{8.407134in}{2.953728in}}{\pgfqpoint{8.414948in}{2.961542in}}%
\pgfpathcurveto{\pgfqpoint{8.422762in}{2.969355in}}{\pgfqpoint{8.427152in}{2.979954in}}{\pgfqpoint{8.427152in}{2.991004in}}%
\pgfpathcurveto{\pgfqpoint{8.427152in}{3.002055in}}{\pgfqpoint{8.422762in}{3.012654in}}{\pgfqpoint{8.414948in}{3.020467in}}%
\pgfpathcurveto{\pgfqpoint{8.407134in}{3.028281in}}{\pgfqpoint{8.396535in}{3.032671in}}{\pgfqpoint{8.385485in}{3.032671in}}%
\pgfpathcurveto{\pgfqpoint{8.374435in}{3.032671in}}{\pgfqpoint{8.363836in}{3.028281in}}{\pgfqpoint{8.356022in}{3.020467in}}%
\pgfpathcurveto{\pgfqpoint{8.348209in}{3.012654in}}{\pgfqpoint{8.343818in}{3.002055in}}{\pgfqpoint{8.343818in}{2.991004in}}%
\pgfpathcurveto{\pgfqpoint{8.343818in}{2.979954in}}{\pgfqpoint{8.348209in}{2.969355in}}{\pgfqpoint{8.356022in}{2.961542in}}%
\pgfpathcurveto{\pgfqpoint{8.363836in}{2.953728in}}{\pgfqpoint{8.374435in}{2.949338in}}{\pgfqpoint{8.385485in}{2.949338in}}%
\pgfpathclose%
\pgfusepath{stroke,fill}%
\end{pgfscope}%
\begin{pgfscope}%
\pgfpathrectangle{\pgfqpoint{0.570343in}{0.331635in}}{\pgfqpoint{9.300000in}{7.700000in}}%
\pgfusepath{clip}%
\pgfsetbuttcap%
\pgfsetroundjoin%
\definecolor{currentfill}{rgb}{1.000000,0.705882,0.509804}%
\pgfsetfillcolor{currentfill}%
\pgfsetlinewidth{0.481800pt}%
\definecolor{currentstroke}{rgb}{1.000000,1.000000,1.000000}%
\pgfsetstrokecolor{currentstroke}%
\pgfsetdash{}{0pt}%
\pgfpathmoveto{\pgfqpoint{5.756401in}{4.263575in}}%
\pgfpathcurveto{\pgfqpoint{5.767451in}{4.263575in}}{\pgfqpoint{5.778050in}{4.267966in}}{\pgfqpoint{5.785864in}{4.275779in}}%
\pgfpathcurveto{\pgfqpoint{5.793677in}{4.283593in}}{\pgfqpoint{5.798068in}{4.294192in}}{\pgfqpoint{5.798068in}{4.305242in}}%
\pgfpathcurveto{\pgfqpoint{5.798068in}{4.316292in}}{\pgfqpoint{5.793677in}{4.326891in}}{\pgfqpoint{5.785864in}{4.334705in}}%
\pgfpathcurveto{\pgfqpoint{5.778050in}{4.342518in}}{\pgfqpoint{5.767451in}{4.346909in}}{\pgfqpoint{5.756401in}{4.346909in}}%
\pgfpathcurveto{\pgfqpoint{5.745351in}{4.346909in}}{\pgfqpoint{5.734752in}{4.342518in}}{\pgfqpoint{5.726938in}{4.334705in}}%
\pgfpathcurveto{\pgfqpoint{5.719125in}{4.326891in}}{\pgfqpoint{5.714734in}{4.316292in}}{\pgfqpoint{5.714734in}{4.305242in}}%
\pgfpathcurveto{\pgfqpoint{5.714734in}{4.294192in}}{\pgfqpoint{5.719125in}{4.283593in}}{\pgfqpoint{5.726938in}{4.275779in}}%
\pgfpathcurveto{\pgfqpoint{5.734752in}{4.267966in}}{\pgfqpoint{5.745351in}{4.263575in}}{\pgfqpoint{5.756401in}{4.263575in}}%
\pgfpathclose%
\pgfusepath{stroke,fill}%
\end{pgfscope}%
\begin{pgfscope}%
\pgfpathrectangle{\pgfqpoint{0.570343in}{0.331635in}}{\pgfqpoint{9.300000in}{7.700000in}}%
\pgfusepath{clip}%
\pgfsetbuttcap%
\pgfsetroundjoin%
\definecolor{currentfill}{rgb}{1.000000,0.705882,0.509804}%
\pgfsetfillcolor{currentfill}%
\pgfsetlinewidth{0.481800pt}%
\definecolor{currentstroke}{rgb}{1.000000,1.000000,1.000000}%
\pgfsetstrokecolor{currentstroke}%
\pgfsetdash{}{0pt}%
\pgfpathmoveto{\pgfqpoint{6.668584in}{4.338802in}}%
\pgfpathcurveto{\pgfqpoint{6.679634in}{4.338802in}}{\pgfqpoint{6.690233in}{4.343192in}}{\pgfqpoint{6.698047in}{4.351006in}}%
\pgfpathcurveto{\pgfqpoint{6.705860in}{4.358819in}}{\pgfqpoint{6.710250in}{4.369418in}}{\pgfqpoint{6.710250in}{4.380469in}}%
\pgfpathcurveto{\pgfqpoint{6.710250in}{4.391519in}}{\pgfqpoint{6.705860in}{4.402118in}}{\pgfqpoint{6.698047in}{4.409931in}}%
\pgfpathcurveto{\pgfqpoint{6.690233in}{4.417745in}}{\pgfqpoint{6.679634in}{4.422135in}}{\pgfqpoint{6.668584in}{4.422135in}}%
\pgfpathcurveto{\pgfqpoint{6.657534in}{4.422135in}}{\pgfqpoint{6.646935in}{4.417745in}}{\pgfqpoint{6.639121in}{4.409931in}}%
\pgfpathcurveto{\pgfqpoint{6.631307in}{4.402118in}}{\pgfqpoint{6.626917in}{4.391519in}}{\pgfqpoint{6.626917in}{4.380469in}}%
\pgfpathcurveto{\pgfqpoint{6.626917in}{4.369418in}}{\pgfqpoint{6.631307in}{4.358819in}}{\pgfqpoint{6.639121in}{4.351006in}}%
\pgfpathcurveto{\pgfqpoint{6.646935in}{4.343192in}}{\pgfqpoint{6.657534in}{4.338802in}}{\pgfqpoint{6.668584in}{4.338802in}}%
\pgfpathclose%
\pgfusepath{stroke,fill}%
\end{pgfscope}%
\begin{pgfscope}%
\pgfpathrectangle{\pgfqpoint{0.570343in}{0.331635in}}{\pgfqpoint{9.300000in}{7.700000in}}%
\pgfusepath{clip}%
\pgfsetbuttcap%
\pgfsetroundjoin%
\definecolor{currentfill}{rgb}{1.000000,0.705882,0.509804}%
\pgfsetfillcolor{currentfill}%
\pgfsetlinewidth{0.481800pt}%
\definecolor{currentstroke}{rgb}{1.000000,1.000000,1.000000}%
\pgfsetstrokecolor{currentstroke}%
\pgfsetdash{}{0pt}%
\pgfpathmoveto{\pgfqpoint{3.259347in}{1.580683in}}%
\pgfpathcurveto{\pgfqpoint{3.270398in}{1.580683in}}{\pgfqpoint{3.280997in}{1.585073in}}{\pgfqpoint{3.288810in}{1.592887in}}%
\pgfpathcurveto{\pgfqpoint{3.296624in}{1.600700in}}{\pgfqpoint{3.301014in}{1.611299in}}{\pgfqpoint{3.301014in}{1.622349in}}%
\pgfpathcurveto{\pgfqpoint{3.301014in}{1.633400in}}{\pgfqpoint{3.296624in}{1.643999in}}{\pgfqpoint{3.288810in}{1.651812in}}%
\pgfpathcurveto{\pgfqpoint{3.280997in}{1.659626in}}{\pgfqpoint{3.270398in}{1.664016in}}{\pgfqpoint{3.259347in}{1.664016in}}%
\pgfpathcurveto{\pgfqpoint{3.248297in}{1.664016in}}{\pgfqpoint{3.237698in}{1.659626in}}{\pgfqpoint{3.229885in}{1.651812in}}%
\pgfpathcurveto{\pgfqpoint{3.222071in}{1.643999in}}{\pgfqpoint{3.217681in}{1.633400in}}{\pgfqpoint{3.217681in}{1.622349in}}%
\pgfpathcurveto{\pgfqpoint{3.217681in}{1.611299in}}{\pgfqpoint{3.222071in}{1.600700in}}{\pgfqpoint{3.229885in}{1.592887in}}%
\pgfpathcurveto{\pgfqpoint{3.237698in}{1.585073in}}{\pgfqpoint{3.248297in}{1.580683in}}{\pgfqpoint{3.259347in}{1.580683in}}%
\pgfpathclose%
\pgfusepath{stroke,fill}%
\end{pgfscope}%
\begin{pgfscope}%
\pgfpathrectangle{\pgfqpoint{0.570343in}{0.331635in}}{\pgfqpoint{9.300000in}{7.700000in}}%
\pgfusepath{clip}%
\pgfsetbuttcap%
\pgfsetroundjoin%
\definecolor{currentfill}{rgb}{1.000000,0.705882,0.509804}%
\pgfsetfillcolor{currentfill}%
\pgfsetlinewidth{0.481800pt}%
\definecolor{currentstroke}{rgb}{1.000000,1.000000,1.000000}%
\pgfsetstrokecolor{currentstroke}%
\pgfsetdash{}{0pt}%
\pgfpathmoveto{\pgfqpoint{3.583317in}{1.978821in}}%
\pgfpathcurveto{\pgfqpoint{3.594367in}{1.978821in}}{\pgfqpoint{3.604966in}{1.983211in}}{\pgfqpoint{3.612780in}{1.991025in}}%
\pgfpathcurveto{\pgfqpoint{3.620594in}{1.998839in}}{\pgfqpoint{3.624984in}{2.009438in}}{\pgfqpoint{3.624984in}{2.020488in}}%
\pgfpathcurveto{\pgfqpoint{3.624984in}{2.031538in}}{\pgfqpoint{3.620594in}{2.042137in}}{\pgfqpoint{3.612780in}{2.049951in}}%
\pgfpathcurveto{\pgfqpoint{3.604966in}{2.057764in}}{\pgfqpoint{3.594367in}{2.062155in}}{\pgfqpoint{3.583317in}{2.062155in}}%
\pgfpathcurveto{\pgfqpoint{3.572267in}{2.062155in}}{\pgfqpoint{3.561668in}{2.057764in}}{\pgfqpoint{3.553854in}{2.049951in}}%
\pgfpathcurveto{\pgfqpoint{3.546041in}{2.042137in}}{\pgfqpoint{3.541650in}{2.031538in}}{\pgfqpoint{3.541650in}{2.020488in}}%
\pgfpathcurveto{\pgfqpoint{3.541650in}{2.009438in}}{\pgfqpoint{3.546041in}{1.998839in}}{\pgfqpoint{3.553854in}{1.991025in}}%
\pgfpathcurveto{\pgfqpoint{3.561668in}{1.983211in}}{\pgfqpoint{3.572267in}{1.978821in}}{\pgfqpoint{3.583317in}{1.978821in}}%
\pgfpathclose%
\pgfusepath{stroke,fill}%
\end{pgfscope}%
\begin{pgfscope}%
\pgfpathrectangle{\pgfqpoint{0.570343in}{0.331635in}}{\pgfqpoint{9.300000in}{7.700000in}}%
\pgfusepath{clip}%
\pgfsetbuttcap%
\pgfsetroundjoin%
\definecolor{currentfill}{rgb}{1.000000,0.705882,0.509804}%
\pgfsetfillcolor{currentfill}%
\pgfsetlinewidth{0.481800pt}%
\definecolor{currentstroke}{rgb}{1.000000,1.000000,1.000000}%
\pgfsetstrokecolor{currentstroke}%
\pgfsetdash{}{0pt}%
\pgfpathmoveto{\pgfqpoint{6.070729in}{4.056533in}}%
\pgfpathcurveto{\pgfqpoint{6.081779in}{4.056533in}}{\pgfqpoint{6.092378in}{4.060923in}}{\pgfqpoint{6.100192in}{4.068737in}}%
\pgfpathcurveto{\pgfqpoint{6.108006in}{4.076550in}}{\pgfqpoint{6.112396in}{4.087149in}}{\pgfqpoint{6.112396in}{4.098200in}}%
\pgfpathcurveto{\pgfqpoint{6.112396in}{4.109250in}}{\pgfqpoint{6.108006in}{4.119849in}}{\pgfqpoint{6.100192in}{4.127662in}}%
\pgfpathcurveto{\pgfqpoint{6.092378in}{4.135476in}}{\pgfqpoint{6.081779in}{4.139866in}}{\pgfqpoint{6.070729in}{4.139866in}}%
\pgfpathcurveto{\pgfqpoint{6.059679in}{4.139866in}}{\pgfqpoint{6.049080in}{4.135476in}}{\pgfqpoint{6.041266in}{4.127662in}}%
\pgfpathcurveto{\pgfqpoint{6.033453in}{4.119849in}}{\pgfqpoint{6.029063in}{4.109250in}}{\pgfqpoint{6.029063in}{4.098200in}}%
\pgfpathcurveto{\pgfqpoint{6.029063in}{4.087149in}}{\pgfqpoint{6.033453in}{4.076550in}}{\pgfqpoint{6.041266in}{4.068737in}}%
\pgfpathcurveto{\pgfqpoint{6.049080in}{4.060923in}}{\pgfqpoint{6.059679in}{4.056533in}}{\pgfqpoint{6.070729in}{4.056533in}}%
\pgfpathclose%
\pgfusepath{stroke,fill}%
\end{pgfscope}%
\begin{pgfscope}%
\pgfpathrectangle{\pgfqpoint{0.570343in}{0.331635in}}{\pgfqpoint{9.300000in}{7.700000in}}%
\pgfusepath{clip}%
\pgfsetbuttcap%
\pgfsetroundjoin%
\definecolor{currentfill}{rgb}{1.000000,0.705882,0.509804}%
\pgfsetfillcolor{currentfill}%
\pgfsetlinewidth{0.481800pt}%
\definecolor{currentstroke}{rgb}{1.000000,1.000000,1.000000}%
\pgfsetstrokecolor{currentstroke}%
\pgfsetdash{}{0pt}%
\pgfpathmoveto{\pgfqpoint{5.275125in}{5.669655in}}%
\pgfpathcurveto{\pgfqpoint{5.286175in}{5.669655in}}{\pgfqpoint{5.296774in}{5.674045in}}{\pgfqpoint{5.304588in}{5.681859in}}%
\pgfpathcurveto{\pgfqpoint{5.312402in}{5.689672in}}{\pgfqpoint{5.316792in}{5.700272in}}{\pgfqpoint{5.316792in}{5.711322in}}%
\pgfpathcurveto{\pgfqpoint{5.316792in}{5.722372in}}{\pgfqpoint{5.312402in}{5.732971in}}{\pgfqpoint{5.304588in}{5.740784in}}%
\pgfpathcurveto{\pgfqpoint{5.296774in}{5.748598in}}{\pgfqpoint{5.286175in}{5.752988in}}{\pgfqpoint{5.275125in}{5.752988in}}%
\pgfpathcurveto{\pgfqpoint{5.264075in}{5.752988in}}{\pgfqpoint{5.253476in}{5.748598in}}{\pgfqpoint{5.245662in}{5.740784in}}%
\pgfpathcurveto{\pgfqpoint{5.237849in}{5.732971in}}{\pgfqpoint{5.233459in}{5.722372in}}{\pgfqpoint{5.233459in}{5.711322in}}%
\pgfpathcurveto{\pgfqpoint{5.233459in}{5.700272in}}{\pgfqpoint{5.237849in}{5.689672in}}{\pgfqpoint{5.245662in}{5.681859in}}%
\pgfpathcurveto{\pgfqpoint{5.253476in}{5.674045in}}{\pgfqpoint{5.264075in}{5.669655in}}{\pgfqpoint{5.275125in}{5.669655in}}%
\pgfpathclose%
\pgfusepath{stroke,fill}%
\end{pgfscope}%
\begin{pgfscope}%
\pgfpathrectangle{\pgfqpoint{0.570343in}{0.331635in}}{\pgfqpoint{9.300000in}{7.700000in}}%
\pgfusepath{clip}%
\pgfsetbuttcap%
\pgfsetroundjoin%
\definecolor{currentfill}{rgb}{1.000000,0.705882,0.509804}%
\pgfsetfillcolor{currentfill}%
\pgfsetlinewidth{0.481800pt}%
\definecolor{currentstroke}{rgb}{1.000000,1.000000,1.000000}%
\pgfsetstrokecolor{currentstroke}%
\pgfsetdash{}{0pt}%
\pgfpathmoveto{\pgfqpoint{6.812363in}{6.534977in}}%
\pgfpathcurveto{\pgfqpoint{6.823413in}{6.534977in}}{\pgfqpoint{6.834012in}{6.539367in}}{\pgfqpoint{6.841826in}{6.547181in}}%
\pgfpathcurveto{\pgfqpoint{6.849639in}{6.554995in}}{\pgfqpoint{6.854030in}{6.565594in}}{\pgfqpoint{6.854030in}{6.576644in}}%
\pgfpathcurveto{\pgfqpoint{6.854030in}{6.587694in}}{\pgfqpoint{6.849639in}{6.598293in}}{\pgfqpoint{6.841826in}{6.606107in}}%
\pgfpathcurveto{\pgfqpoint{6.834012in}{6.613920in}}{\pgfqpoint{6.823413in}{6.618311in}}{\pgfqpoint{6.812363in}{6.618311in}}%
\pgfpathcurveto{\pgfqpoint{6.801313in}{6.618311in}}{\pgfqpoint{6.790714in}{6.613920in}}{\pgfqpoint{6.782900in}{6.606107in}}%
\pgfpathcurveto{\pgfqpoint{6.775086in}{6.598293in}}{\pgfqpoint{6.770696in}{6.587694in}}{\pgfqpoint{6.770696in}{6.576644in}}%
\pgfpathcurveto{\pgfqpoint{6.770696in}{6.565594in}}{\pgfqpoint{6.775086in}{6.554995in}}{\pgfqpoint{6.782900in}{6.547181in}}%
\pgfpathcurveto{\pgfqpoint{6.790714in}{6.539367in}}{\pgfqpoint{6.801313in}{6.534977in}}{\pgfqpoint{6.812363in}{6.534977in}}%
\pgfpathclose%
\pgfusepath{stroke,fill}%
\end{pgfscope}%
\begin{pgfscope}%
\pgfpathrectangle{\pgfqpoint{0.570343in}{0.331635in}}{\pgfqpoint{9.300000in}{7.700000in}}%
\pgfusepath{clip}%
\pgfsetbuttcap%
\pgfsetroundjoin%
\definecolor{currentfill}{rgb}{1.000000,0.705882,0.509804}%
\pgfsetfillcolor{currentfill}%
\pgfsetlinewidth{0.481800pt}%
\definecolor{currentstroke}{rgb}{1.000000,1.000000,1.000000}%
\pgfsetstrokecolor{currentstroke}%
\pgfsetdash{}{0pt}%
\pgfpathmoveto{\pgfqpoint{8.535292in}{4.068891in}}%
\pgfpathcurveto{\pgfqpoint{8.546342in}{4.068891in}}{\pgfqpoint{8.556941in}{4.073281in}}{\pgfqpoint{8.564755in}{4.081095in}}%
\pgfpathcurveto{\pgfqpoint{8.572569in}{4.088908in}}{\pgfqpoint{8.576959in}{4.099507in}}{\pgfqpoint{8.576959in}{4.110557in}}%
\pgfpathcurveto{\pgfqpoint{8.576959in}{4.121607in}}{\pgfqpoint{8.572569in}{4.132206in}}{\pgfqpoint{8.564755in}{4.140020in}}%
\pgfpathcurveto{\pgfqpoint{8.556941in}{4.147834in}}{\pgfqpoint{8.546342in}{4.152224in}}{\pgfqpoint{8.535292in}{4.152224in}}%
\pgfpathcurveto{\pgfqpoint{8.524242in}{4.152224in}}{\pgfqpoint{8.513643in}{4.147834in}}{\pgfqpoint{8.505829in}{4.140020in}}%
\pgfpathcurveto{\pgfqpoint{8.498016in}{4.132206in}}{\pgfqpoint{8.493626in}{4.121607in}}{\pgfqpoint{8.493626in}{4.110557in}}%
\pgfpathcurveto{\pgfqpoint{8.493626in}{4.099507in}}{\pgfqpoint{8.498016in}{4.088908in}}{\pgfqpoint{8.505829in}{4.081095in}}%
\pgfpathcurveto{\pgfqpoint{8.513643in}{4.073281in}}{\pgfqpoint{8.524242in}{4.068891in}}{\pgfqpoint{8.535292in}{4.068891in}}%
\pgfpathclose%
\pgfusepath{stroke,fill}%
\end{pgfscope}%
\begin{pgfscope}%
\pgfpathrectangle{\pgfqpoint{0.570343in}{0.331635in}}{\pgfqpoint{9.300000in}{7.700000in}}%
\pgfusepath{clip}%
\pgfsetbuttcap%
\pgfsetroundjoin%
\definecolor{currentfill}{rgb}{1.000000,0.705882,0.509804}%
\pgfsetfillcolor{currentfill}%
\pgfsetlinewidth{0.481800pt}%
\definecolor{currentstroke}{rgb}{1.000000,1.000000,1.000000}%
\pgfsetstrokecolor{currentstroke}%
\pgfsetdash{}{0pt}%
\pgfpathmoveto{\pgfqpoint{5.969237in}{6.155259in}}%
\pgfpathcurveto{\pgfqpoint{5.980287in}{6.155259in}}{\pgfqpoint{5.990886in}{6.159650in}}{\pgfqpoint{5.998700in}{6.167463in}}%
\pgfpathcurveto{\pgfqpoint{6.006513in}{6.175277in}}{\pgfqpoint{6.010904in}{6.185876in}}{\pgfqpoint{6.010904in}{6.196926in}}%
\pgfpathcurveto{\pgfqpoint{6.010904in}{6.207976in}}{\pgfqpoint{6.006513in}{6.218575in}}{\pgfqpoint{5.998700in}{6.226389in}}%
\pgfpathcurveto{\pgfqpoint{5.990886in}{6.234202in}}{\pgfqpoint{5.980287in}{6.238593in}}{\pgfqpoint{5.969237in}{6.238593in}}%
\pgfpathcurveto{\pgfqpoint{5.958187in}{6.238593in}}{\pgfqpoint{5.947588in}{6.234202in}}{\pgfqpoint{5.939774in}{6.226389in}}%
\pgfpathcurveto{\pgfqpoint{5.931961in}{6.218575in}}{\pgfqpoint{5.927570in}{6.207976in}}{\pgfqpoint{5.927570in}{6.196926in}}%
\pgfpathcurveto{\pgfqpoint{5.927570in}{6.185876in}}{\pgfqpoint{5.931961in}{6.175277in}}{\pgfqpoint{5.939774in}{6.167463in}}%
\pgfpathcurveto{\pgfqpoint{5.947588in}{6.159650in}}{\pgfqpoint{5.958187in}{6.155259in}}{\pgfqpoint{5.969237in}{6.155259in}}%
\pgfpathclose%
\pgfusepath{stroke,fill}%
\end{pgfscope}%
\begin{pgfscope}%
\pgfpathrectangle{\pgfqpoint{0.570343in}{0.331635in}}{\pgfqpoint{9.300000in}{7.700000in}}%
\pgfusepath{clip}%
\pgfsetbuttcap%
\pgfsetroundjoin%
\definecolor{currentfill}{rgb}{1.000000,0.705882,0.509804}%
\pgfsetfillcolor{currentfill}%
\pgfsetlinewidth{0.481800pt}%
\definecolor{currentstroke}{rgb}{1.000000,1.000000,1.000000}%
\pgfsetstrokecolor{currentstroke}%
\pgfsetdash{}{0pt}%
\pgfpathmoveto{\pgfqpoint{3.243518in}{3.479066in}}%
\pgfpathcurveto{\pgfqpoint{3.254568in}{3.479066in}}{\pgfqpoint{3.265167in}{3.483456in}}{\pgfqpoint{3.272981in}{3.491270in}}%
\pgfpathcurveto{\pgfqpoint{3.280794in}{3.499083in}}{\pgfqpoint{3.285185in}{3.509682in}}{\pgfqpoint{3.285185in}{3.520733in}}%
\pgfpathcurveto{\pgfqpoint{3.285185in}{3.531783in}}{\pgfqpoint{3.280794in}{3.542382in}}{\pgfqpoint{3.272981in}{3.550195in}}%
\pgfpathcurveto{\pgfqpoint{3.265167in}{3.558009in}}{\pgfqpoint{3.254568in}{3.562399in}}{\pgfqpoint{3.243518in}{3.562399in}}%
\pgfpathcurveto{\pgfqpoint{3.232468in}{3.562399in}}{\pgfqpoint{3.221869in}{3.558009in}}{\pgfqpoint{3.214055in}{3.550195in}}%
\pgfpathcurveto{\pgfqpoint{3.206241in}{3.542382in}}{\pgfqpoint{3.201851in}{3.531783in}}{\pgfqpoint{3.201851in}{3.520733in}}%
\pgfpathcurveto{\pgfqpoint{3.201851in}{3.509682in}}{\pgfqpoint{3.206241in}{3.499083in}}{\pgfqpoint{3.214055in}{3.491270in}}%
\pgfpathcurveto{\pgfqpoint{3.221869in}{3.483456in}}{\pgfqpoint{3.232468in}{3.479066in}}{\pgfqpoint{3.243518in}{3.479066in}}%
\pgfpathclose%
\pgfusepath{stroke,fill}%
\end{pgfscope}%
\begin{pgfscope}%
\pgfpathrectangle{\pgfqpoint{0.570343in}{0.331635in}}{\pgfqpoint{9.300000in}{7.700000in}}%
\pgfusepath{clip}%
\pgfsetbuttcap%
\pgfsetroundjoin%
\definecolor{currentfill}{rgb}{1.000000,0.705882,0.509804}%
\pgfsetfillcolor{currentfill}%
\pgfsetlinewidth{0.481800pt}%
\definecolor{currentstroke}{rgb}{1.000000,1.000000,1.000000}%
\pgfsetstrokecolor{currentstroke}%
\pgfsetdash{}{0pt}%
\pgfpathmoveto{\pgfqpoint{3.870316in}{6.831723in}}%
\pgfpathcurveto{\pgfqpoint{3.881366in}{6.831723in}}{\pgfqpoint{3.891965in}{6.836113in}}{\pgfqpoint{3.899779in}{6.843927in}}%
\pgfpathcurveto{\pgfqpoint{3.907593in}{6.851740in}}{\pgfqpoint{3.911983in}{6.862339in}}{\pgfqpoint{3.911983in}{6.873389in}}%
\pgfpathcurveto{\pgfqpoint{3.911983in}{6.884440in}}{\pgfqpoint{3.907593in}{6.895039in}}{\pgfqpoint{3.899779in}{6.902852in}}%
\pgfpathcurveto{\pgfqpoint{3.891965in}{6.910666in}}{\pgfqpoint{3.881366in}{6.915056in}}{\pgfqpoint{3.870316in}{6.915056in}}%
\pgfpathcurveto{\pgfqpoint{3.859266in}{6.915056in}}{\pgfqpoint{3.848667in}{6.910666in}}{\pgfqpoint{3.840853in}{6.902852in}}%
\pgfpathcurveto{\pgfqpoint{3.833040in}{6.895039in}}{\pgfqpoint{3.828649in}{6.884440in}}{\pgfqpoint{3.828649in}{6.873389in}}%
\pgfpathcurveto{\pgfqpoint{3.828649in}{6.862339in}}{\pgfqpoint{3.833040in}{6.851740in}}{\pgfqpoint{3.840853in}{6.843927in}}%
\pgfpathcurveto{\pgfqpoint{3.848667in}{6.836113in}}{\pgfqpoint{3.859266in}{6.831723in}}{\pgfqpoint{3.870316in}{6.831723in}}%
\pgfpathclose%
\pgfusepath{stroke,fill}%
\end{pgfscope}%
\begin{pgfscope}%
\pgfpathrectangle{\pgfqpoint{0.570343in}{0.331635in}}{\pgfqpoint{9.300000in}{7.700000in}}%
\pgfusepath{clip}%
\pgfsetbuttcap%
\pgfsetroundjoin%
\definecolor{currentfill}{rgb}{1.000000,0.705882,0.509804}%
\pgfsetfillcolor{currentfill}%
\pgfsetlinewidth{0.481800pt}%
\definecolor{currentstroke}{rgb}{1.000000,1.000000,1.000000}%
\pgfsetstrokecolor{currentstroke}%
\pgfsetdash{}{0pt}%
\pgfpathmoveto{\pgfqpoint{4.461357in}{0.706063in}}%
\pgfpathcurveto{\pgfqpoint{4.472408in}{0.706063in}}{\pgfqpoint{4.483007in}{0.710454in}}{\pgfqpoint{4.490820in}{0.718267in}}%
\pgfpathcurveto{\pgfqpoint{4.498634in}{0.726081in}}{\pgfqpoint{4.503024in}{0.736680in}}{\pgfqpoint{4.503024in}{0.747730in}}%
\pgfpathcurveto{\pgfqpoint{4.503024in}{0.758780in}}{\pgfqpoint{4.498634in}{0.769379in}}{\pgfqpoint{4.490820in}{0.777193in}}%
\pgfpathcurveto{\pgfqpoint{4.483007in}{0.785007in}}{\pgfqpoint{4.472408in}{0.789397in}}{\pgfqpoint{4.461357in}{0.789397in}}%
\pgfpathcurveto{\pgfqpoint{4.450307in}{0.789397in}}{\pgfqpoint{4.439708in}{0.785007in}}{\pgfqpoint{4.431895in}{0.777193in}}%
\pgfpathcurveto{\pgfqpoint{4.424081in}{0.769379in}}{\pgfqpoint{4.419691in}{0.758780in}}{\pgfqpoint{4.419691in}{0.747730in}}%
\pgfpathcurveto{\pgfqpoint{4.419691in}{0.736680in}}{\pgfqpoint{4.424081in}{0.726081in}}{\pgfqpoint{4.431895in}{0.718267in}}%
\pgfpathcurveto{\pgfqpoint{4.439708in}{0.710454in}}{\pgfqpoint{4.450307in}{0.706063in}}{\pgfqpoint{4.461357in}{0.706063in}}%
\pgfpathclose%
\pgfusepath{stroke,fill}%
\end{pgfscope}%
\begin{pgfscope}%
\pgfpathrectangle{\pgfqpoint{0.570343in}{0.331635in}}{\pgfqpoint{9.300000in}{7.700000in}}%
\pgfusepath{clip}%
\pgfsetbuttcap%
\pgfsetroundjoin%
\definecolor{currentfill}{rgb}{1.000000,0.705882,0.509804}%
\pgfsetfillcolor{currentfill}%
\pgfsetlinewidth{0.481800pt}%
\definecolor{currentstroke}{rgb}{1.000000,1.000000,1.000000}%
\pgfsetstrokecolor{currentstroke}%
\pgfsetdash{}{0pt}%
\pgfpathmoveto{\pgfqpoint{2.074350in}{4.404231in}}%
\pgfpathcurveto{\pgfqpoint{2.085401in}{4.404231in}}{\pgfqpoint{2.096000in}{4.408621in}}{\pgfqpoint{2.103813in}{4.416434in}}%
\pgfpathcurveto{\pgfqpoint{2.111627in}{4.424248in}}{\pgfqpoint{2.116017in}{4.434847in}}{\pgfqpoint{2.116017in}{4.445897in}}%
\pgfpathcurveto{\pgfqpoint{2.116017in}{4.456947in}}{\pgfqpoint{2.111627in}{4.467546in}}{\pgfqpoint{2.103813in}{4.475360in}}%
\pgfpathcurveto{\pgfqpoint{2.096000in}{4.483174in}}{\pgfqpoint{2.085401in}{4.487564in}}{\pgfqpoint{2.074350in}{4.487564in}}%
\pgfpathcurveto{\pgfqpoint{2.063300in}{4.487564in}}{\pgfqpoint{2.052701in}{4.483174in}}{\pgfqpoint{2.044888in}{4.475360in}}%
\pgfpathcurveto{\pgfqpoint{2.037074in}{4.467546in}}{\pgfqpoint{2.032684in}{4.456947in}}{\pgfqpoint{2.032684in}{4.445897in}}%
\pgfpathcurveto{\pgfqpoint{2.032684in}{4.434847in}}{\pgfqpoint{2.037074in}{4.424248in}}{\pgfqpoint{2.044888in}{4.416434in}}%
\pgfpathcurveto{\pgfqpoint{2.052701in}{4.408621in}}{\pgfqpoint{2.063300in}{4.404231in}}{\pgfqpoint{2.074350in}{4.404231in}}%
\pgfpathclose%
\pgfusepath{stroke,fill}%
\end{pgfscope}%
\begin{pgfscope}%
\pgfpathrectangle{\pgfqpoint{0.570343in}{0.331635in}}{\pgfqpoint{9.300000in}{7.700000in}}%
\pgfusepath{clip}%
\pgfsetbuttcap%
\pgfsetroundjoin%
\definecolor{currentfill}{rgb}{1.000000,0.705882,0.509804}%
\pgfsetfillcolor{currentfill}%
\pgfsetlinewidth{0.481800pt}%
\definecolor{currentstroke}{rgb}{1.000000,1.000000,1.000000}%
\pgfsetstrokecolor{currentstroke}%
\pgfsetdash{}{0pt}%
\pgfpathmoveto{\pgfqpoint{7.251452in}{4.895744in}}%
\pgfpathcurveto{\pgfqpoint{7.262502in}{4.895744in}}{\pgfqpoint{7.273101in}{4.900135in}}{\pgfqpoint{7.280915in}{4.907948in}}%
\pgfpathcurveto{\pgfqpoint{7.288728in}{4.915762in}}{\pgfqpoint{7.293119in}{4.926361in}}{\pgfqpoint{7.293119in}{4.937411in}}%
\pgfpathcurveto{\pgfqpoint{7.293119in}{4.948461in}}{\pgfqpoint{7.288728in}{4.959060in}}{\pgfqpoint{7.280915in}{4.966874in}}%
\pgfpathcurveto{\pgfqpoint{7.273101in}{4.974687in}}{\pgfqpoint{7.262502in}{4.979078in}}{\pgfqpoint{7.251452in}{4.979078in}}%
\pgfpathcurveto{\pgfqpoint{7.240402in}{4.979078in}}{\pgfqpoint{7.229803in}{4.974687in}}{\pgfqpoint{7.221989in}{4.966874in}}%
\pgfpathcurveto{\pgfqpoint{7.214175in}{4.959060in}}{\pgfqpoint{7.209785in}{4.948461in}}{\pgfqpoint{7.209785in}{4.937411in}}%
\pgfpathcurveto{\pgfqpoint{7.209785in}{4.926361in}}{\pgfqpoint{7.214175in}{4.915762in}}{\pgfqpoint{7.221989in}{4.907948in}}%
\pgfpathcurveto{\pgfqpoint{7.229803in}{4.900135in}}{\pgfqpoint{7.240402in}{4.895744in}}{\pgfqpoint{7.251452in}{4.895744in}}%
\pgfpathclose%
\pgfusepath{stroke,fill}%
\end{pgfscope}%
\begin{pgfscope}%
\pgfpathrectangle{\pgfqpoint{0.570343in}{0.331635in}}{\pgfqpoint{9.300000in}{7.700000in}}%
\pgfusepath{clip}%
\pgfsetbuttcap%
\pgfsetroundjoin%
\definecolor{currentfill}{rgb}{1.000000,0.705882,0.509804}%
\pgfsetfillcolor{currentfill}%
\pgfsetlinewidth{0.481800pt}%
\definecolor{currentstroke}{rgb}{1.000000,1.000000,1.000000}%
\pgfsetstrokecolor{currentstroke}%
\pgfsetdash{}{0pt}%
\pgfpathmoveto{\pgfqpoint{5.103453in}{4.133952in}}%
\pgfpathcurveto{\pgfqpoint{5.114503in}{4.133952in}}{\pgfqpoint{5.125102in}{4.138342in}}{\pgfqpoint{5.132916in}{4.146155in}}%
\pgfpathcurveto{\pgfqpoint{5.140729in}{4.153969in}}{\pgfqpoint{5.145119in}{4.164568in}}{\pgfqpoint{5.145119in}{4.175618in}}%
\pgfpathcurveto{\pgfqpoint{5.145119in}{4.186668in}}{\pgfqpoint{5.140729in}{4.197267in}}{\pgfqpoint{5.132916in}{4.205081in}}%
\pgfpathcurveto{\pgfqpoint{5.125102in}{4.212895in}}{\pgfqpoint{5.114503in}{4.217285in}}{\pgfqpoint{5.103453in}{4.217285in}}%
\pgfpathcurveto{\pgfqpoint{5.092403in}{4.217285in}}{\pgfqpoint{5.081804in}{4.212895in}}{\pgfqpoint{5.073990in}{4.205081in}}%
\pgfpathcurveto{\pgfqpoint{5.066176in}{4.197267in}}{\pgfqpoint{5.061786in}{4.186668in}}{\pgfqpoint{5.061786in}{4.175618in}}%
\pgfpathcurveto{\pgfqpoint{5.061786in}{4.164568in}}{\pgfqpoint{5.066176in}{4.153969in}}{\pgfqpoint{5.073990in}{4.146155in}}%
\pgfpathcurveto{\pgfqpoint{5.081804in}{4.138342in}}{\pgfqpoint{5.092403in}{4.133952in}}{\pgfqpoint{5.103453in}{4.133952in}}%
\pgfpathclose%
\pgfusepath{stroke,fill}%
\end{pgfscope}%
\begin{pgfscope}%
\pgfpathrectangle{\pgfqpoint{0.570343in}{0.331635in}}{\pgfqpoint{9.300000in}{7.700000in}}%
\pgfusepath{clip}%
\pgfsetbuttcap%
\pgfsetroundjoin%
\definecolor{currentfill}{rgb}{1.000000,0.705882,0.509804}%
\pgfsetfillcolor{currentfill}%
\pgfsetlinewidth{0.481800pt}%
\definecolor{currentstroke}{rgb}{1.000000,1.000000,1.000000}%
\pgfsetstrokecolor{currentstroke}%
\pgfsetdash{}{0pt}%
\pgfpathmoveto{\pgfqpoint{7.313999in}{3.799893in}}%
\pgfpathcurveto{\pgfqpoint{7.325049in}{3.799893in}}{\pgfqpoint{7.335648in}{3.804283in}}{\pgfqpoint{7.343462in}{3.812097in}}%
\pgfpathcurveto{\pgfqpoint{7.351276in}{3.819911in}}{\pgfqpoint{7.355666in}{3.830510in}}{\pgfqpoint{7.355666in}{3.841560in}}%
\pgfpathcurveto{\pgfqpoint{7.355666in}{3.852610in}}{\pgfqpoint{7.351276in}{3.863209in}}{\pgfqpoint{7.343462in}{3.871023in}}%
\pgfpathcurveto{\pgfqpoint{7.335648in}{3.878836in}}{\pgfqpoint{7.325049in}{3.883226in}}{\pgfqpoint{7.313999in}{3.883226in}}%
\pgfpathcurveto{\pgfqpoint{7.302949in}{3.883226in}}{\pgfqpoint{7.292350in}{3.878836in}}{\pgfqpoint{7.284537in}{3.871023in}}%
\pgfpathcurveto{\pgfqpoint{7.276723in}{3.863209in}}{\pgfqpoint{7.272333in}{3.852610in}}{\pgfqpoint{7.272333in}{3.841560in}}%
\pgfpathcurveto{\pgfqpoint{7.272333in}{3.830510in}}{\pgfqpoint{7.276723in}{3.819911in}}{\pgfqpoint{7.284537in}{3.812097in}}%
\pgfpathcurveto{\pgfqpoint{7.292350in}{3.804283in}}{\pgfqpoint{7.302949in}{3.799893in}}{\pgfqpoint{7.313999in}{3.799893in}}%
\pgfpathclose%
\pgfusepath{stroke,fill}%
\end{pgfscope}%
\begin{pgfscope}%
\pgfpathrectangle{\pgfqpoint{0.570343in}{0.331635in}}{\pgfqpoint{9.300000in}{7.700000in}}%
\pgfusepath{clip}%
\pgfsetbuttcap%
\pgfsetroundjoin%
\definecolor{currentfill}{rgb}{1.000000,0.705882,0.509804}%
\pgfsetfillcolor{currentfill}%
\pgfsetlinewidth{0.481800pt}%
\definecolor{currentstroke}{rgb}{1.000000,1.000000,1.000000}%
\pgfsetstrokecolor{currentstroke}%
\pgfsetdash{}{0pt}%
\pgfpathmoveto{\pgfqpoint{7.365854in}{4.148184in}}%
\pgfpathcurveto{\pgfqpoint{7.376904in}{4.148184in}}{\pgfqpoint{7.387503in}{4.152574in}}{\pgfqpoint{7.395317in}{4.160388in}}%
\pgfpathcurveto{\pgfqpoint{7.403130in}{4.168202in}}{\pgfqpoint{7.407521in}{4.178801in}}{\pgfqpoint{7.407521in}{4.189851in}}%
\pgfpathcurveto{\pgfqpoint{7.407521in}{4.200901in}}{\pgfqpoint{7.403130in}{4.211500in}}{\pgfqpoint{7.395317in}{4.219314in}}%
\pgfpathcurveto{\pgfqpoint{7.387503in}{4.227127in}}{\pgfqpoint{7.376904in}{4.231517in}}{\pgfqpoint{7.365854in}{4.231517in}}%
\pgfpathcurveto{\pgfqpoint{7.354804in}{4.231517in}}{\pgfqpoint{7.344205in}{4.227127in}}{\pgfqpoint{7.336391in}{4.219314in}}%
\pgfpathcurveto{\pgfqpoint{7.328577in}{4.211500in}}{\pgfqpoint{7.324187in}{4.200901in}}{\pgfqpoint{7.324187in}{4.189851in}}%
\pgfpathcurveto{\pgfqpoint{7.324187in}{4.178801in}}{\pgfqpoint{7.328577in}{4.168202in}}{\pgfqpoint{7.336391in}{4.160388in}}%
\pgfpathcurveto{\pgfqpoint{7.344205in}{4.152574in}}{\pgfqpoint{7.354804in}{4.148184in}}{\pgfqpoint{7.365854in}{4.148184in}}%
\pgfpathclose%
\pgfusepath{stroke,fill}%
\end{pgfscope}%
\begin{pgfscope}%
\pgfpathrectangle{\pgfqpoint{0.570343in}{0.331635in}}{\pgfqpoint{9.300000in}{7.700000in}}%
\pgfusepath{clip}%
\pgfsetbuttcap%
\pgfsetroundjoin%
\definecolor{currentfill}{rgb}{1.000000,0.705882,0.509804}%
\pgfsetfillcolor{currentfill}%
\pgfsetlinewidth{0.481800pt}%
\definecolor{currentstroke}{rgb}{1.000000,1.000000,1.000000}%
\pgfsetstrokecolor{currentstroke}%
\pgfsetdash{}{0pt}%
\pgfpathmoveto{\pgfqpoint{4.130821in}{2.042572in}}%
\pgfpathcurveto{\pgfqpoint{4.141871in}{2.042572in}}{\pgfqpoint{4.152470in}{2.046962in}}{\pgfqpoint{4.160284in}{2.054776in}}%
\pgfpathcurveto{\pgfqpoint{4.168098in}{2.062589in}}{\pgfqpoint{4.172488in}{2.073188in}}{\pgfqpoint{4.172488in}{2.084238in}}%
\pgfpathcurveto{\pgfqpoint{4.172488in}{2.095288in}}{\pgfqpoint{4.168098in}{2.105888in}}{\pgfqpoint{4.160284in}{2.113701in}}%
\pgfpathcurveto{\pgfqpoint{4.152470in}{2.121515in}}{\pgfqpoint{4.141871in}{2.125905in}}{\pgfqpoint{4.130821in}{2.125905in}}%
\pgfpathcurveto{\pgfqpoint{4.119771in}{2.125905in}}{\pgfqpoint{4.109172in}{2.121515in}}{\pgfqpoint{4.101358in}{2.113701in}}%
\pgfpathcurveto{\pgfqpoint{4.093545in}{2.105888in}}{\pgfqpoint{4.089155in}{2.095288in}}{\pgfqpoint{4.089155in}{2.084238in}}%
\pgfpathcurveto{\pgfqpoint{4.089155in}{2.073188in}}{\pgfqpoint{4.093545in}{2.062589in}}{\pgfqpoint{4.101358in}{2.054776in}}%
\pgfpathcurveto{\pgfqpoint{4.109172in}{2.046962in}}{\pgfqpoint{4.119771in}{2.042572in}}{\pgfqpoint{4.130821in}{2.042572in}}%
\pgfpathclose%
\pgfusepath{stroke,fill}%
\end{pgfscope}%
\begin{pgfscope}%
\pgfpathrectangle{\pgfqpoint{0.570343in}{0.331635in}}{\pgfqpoint{9.300000in}{7.700000in}}%
\pgfusepath{clip}%
\pgfsetbuttcap%
\pgfsetroundjoin%
\definecolor{currentfill}{rgb}{1.000000,0.705882,0.509804}%
\pgfsetfillcolor{currentfill}%
\pgfsetlinewidth{0.481800pt}%
\definecolor{currentstroke}{rgb}{1.000000,1.000000,1.000000}%
\pgfsetstrokecolor{currentstroke}%
\pgfsetdash{}{0pt}%
\pgfpathmoveto{\pgfqpoint{3.194504in}{2.762873in}}%
\pgfpathcurveto{\pgfqpoint{3.205555in}{2.762873in}}{\pgfqpoint{3.216154in}{2.767264in}}{\pgfqpoint{3.223967in}{2.775077in}}%
\pgfpathcurveto{\pgfqpoint{3.231781in}{2.782891in}}{\pgfqpoint{3.236171in}{2.793490in}}{\pgfqpoint{3.236171in}{2.804540in}}%
\pgfpathcurveto{\pgfqpoint{3.236171in}{2.815590in}}{\pgfqpoint{3.231781in}{2.826189in}}{\pgfqpoint{3.223967in}{2.834003in}}%
\pgfpathcurveto{\pgfqpoint{3.216154in}{2.841816in}}{\pgfqpoint{3.205555in}{2.846207in}}{\pgfqpoint{3.194504in}{2.846207in}}%
\pgfpathcurveto{\pgfqpoint{3.183454in}{2.846207in}}{\pgfqpoint{3.172855in}{2.841816in}}{\pgfqpoint{3.165042in}{2.834003in}}%
\pgfpathcurveto{\pgfqpoint{3.157228in}{2.826189in}}{\pgfqpoint{3.152838in}{2.815590in}}{\pgfqpoint{3.152838in}{2.804540in}}%
\pgfpathcurveto{\pgfqpoint{3.152838in}{2.793490in}}{\pgfqpoint{3.157228in}{2.782891in}}{\pgfqpoint{3.165042in}{2.775077in}}%
\pgfpathcurveto{\pgfqpoint{3.172855in}{2.767264in}}{\pgfqpoint{3.183454in}{2.762873in}}{\pgfqpoint{3.194504in}{2.762873in}}%
\pgfpathclose%
\pgfusepath{stroke,fill}%
\end{pgfscope}%
\begin{pgfscope}%
\pgfpathrectangle{\pgfqpoint{0.570343in}{0.331635in}}{\pgfqpoint{9.300000in}{7.700000in}}%
\pgfusepath{clip}%
\pgfsetbuttcap%
\pgfsetroundjoin%
\definecolor{currentfill}{rgb}{1.000000,0.705882,0.509804}%
\pgfsetfillcolor{currentfill}%
\pgfsetlinewidth{0.481800pt}%
\definecolor{currentstroke}{rgb}{1.000000,1.000000,1.000000}%
\pgfsetstrokecolor{currentstroke}%
\pgfsetdash{}{0pt}%
\pgfpathmoveto{\pgfqpoint{6.329979in}{5.367247in}}%
\pgfpathcurveto{\pgfqpoint{6.341029in}{5.367247in}}{\pgfqpoint{6.351628in}{5.371637in}}{\pgfqpoint{6.359441in}{5.379451in}}%
\pgfpathcurveto{\pgfqpoint{6.367255in}{5.387264in}}{\pgfqpoint{6.371645in}{5.397863in}}{\pgfqpoint{6.371645in}{5.408913in}}%
\pgfpathcurveto{\pgfqpoint{6.371645in}{5.419963in}}{\pgfqpoint{6.367255in}{5.430562in}}{\pgfqpoint{6.359441in}{5.438376in}}%
\pgfpathcurveto{\pgfqpoint{6.351628in}{5.446190in}}{\pgfqpoint{6.341029in}{5.450580in}}{\pgfqpoint{6.329979in}{5.450580in}}%
\pgfpathcurveto{\pgfqpoint{6.318928in}{5.450580in}}{\pgfqpoint{6.308329in}{5.446190in}}{\pgfqpoint{6.300516in}{5.438376in}}%
\pgfpathcurveto{\pgfqpoint{6.292702in}{5.430562in}}{\pgfqpoint{6.288312in}{5.419963in}}{\pgfqpoint{6.288312in}{5.408913in}}%
\pgfpathcurveto{\pgfqpoint{6.288312in}{5.397863in}}{\pgfqpoint{6.292702in}{5.387264in}}{\pgfqpoint{6.300516in}{5.379451in}}%
\pgfpathcurveto{\pgfqpoint{6.308329in}{5.371637in}}{\pgfqpoint{6.318928in}{5.367247in}}{\pgfqpoint{6.329979in}{5.367247in}}%
\pgfpathclose%
\pgfusepath{stroke,fill}%
\end{pgfscope}%
\begin{pgfscope}%
\pgfpathrectangle{\pgfqpoint{0.570343in}{0.331635in}}{\pgfqpoint{9.300000in}{7.700000in}}%
\pgfusepath{clip}%
\pgfsetbuttcap%
\pgfsetroundjoin%
\definecolor{currentfill}{rgb}{1.000000,0.705882,0.509804}%
\pgfsetfillcolor{currentfill}%
\pgfsetlinewidth{0.481800pt}%
\definecolor{currentstroke}{rgb}{1.000000,1.000000,1.000000}%
\pgfsetstrokecolor{currentstroke}%
\pgfsetdash{}{0pt}%
\pgfpathmoveto{\pgfqpoint{6.112839in}{4.933124in}}%
\pgfpathcurveto{\pgfqpoint{6.123889in}{4.933124in}}{\pgfqpoint{6.134488in}{4.937515in}}{\pgfqpoint{6.142302in}{4.945328in}}%
\pgfpathcurveto{\pgfqpoint{6.150115in}{4.953142in}}{\pgfqpoint{6.154506in}{4.963741in}}{\pgfqpoint{6.154506in}{4.974791in}}%
\pgfpathcurveto{\pgfqpoint{6.154506in}{4.985841in}}{\pgfqpoint{6.150115in}{4.996440in}}{\pgfqpoint{6.142302in}{5.004254in}}%
\pgfpathcurveto{\pgfqpoint{6.134488in}{5.012067in}}{\pgfqpoint{6.123889in}{5.016458in}}{\pgfqpoint{6.112839in}{5.016458in}}%
\pgfpathcurveto{\pgfqpoint{6.101789in}{5.016458in}}{\pgfqpoint{6.091190in}{5.012067in}}{\pgfqpoint{6.083376in}{5.004254in}}%
\pgfpathcurveto{\pgfqpoint{6.075563in}{4.996440in}}{\pgfqpoint{6.071172in}{4.985841in}}{\pgfqpoint{6.071172in}{4.974791in}}%
\pgfpathcurveto{\pgfqpoint{6.071172in}{4.963741in}}{\pgfqpoint{6.075563in}{4.953142in}}{\pgfqpoint{6.083376in}{4.945328in}}%
\pgfpathcurveto{\pgfqpoint{6.091190in}{4.937515in}}{\pgfqpoint{6.101789in}{4.933124in}}{\pgfqpoint{6.112839in}{4.933124in}}%
\pgfpathclose%
\pgfusepath{stroke,fill}%
\end{pgfscope}%
\begin{pgfscope}%
\pgfpathrectangle{\pgfqpoint{0.570343in}{0.331635in}}{\pgfqpoint{9.300000in}{7.700000in}}%
\pgfusepath{clip}%
\pgfsetbuttcap%
\pgfsetroundjoin%
\definecolor{currentfill}{rgb}{1.000000,0.705882,0.509804}%
\pgfsetfillcolor{currentfill}%
\pgfsetlinewidth{0.481800pt}%
\definecolor{currentstroke}{rgb}{1.000000,1.000000,1.000000}%
\pgfsetstrokecolor{currentstroke}%
\pgfsetdash{}{0pt}%
\pgfpathmoveto{\pgfqpoint{6.735587in}{3.939821in}}%
\pgfpathcurveto{\pgfqpoint{6.746637in}{3.939821in}}{\pgfqpoint{6.757236in}{3.944211in}}{\pgfqpoint{6.765050in}{3.952025in}}%
\pgfpathcurveto{\pgfqpoint{6.772863in}{3.959838in}}{\pgfqpoint{6.777253in}{3.970437in}}{\pgfqpoint{6.777253in}{3.981487in}}%
\pgfpathcurveto{\pgfqpoint{6.777253in}{3.992538in}}{\pgfqpoint{6.772863in}{4.003137in}}{\pgfqpoint{6.765050in}{4.010950in}}%
\pgfpathcurveto{\pgfqpoint{6.757236in}{4.018764in}}{\pgfqpoint{6.746637in}{4.023154in}}{\pgfqpoint{6.735587in}{4.023154in}}%
\pgfpathcurveto{\pgfqpoint{6.724537in}{4.023154in}}{\pgfqpoint{6.713938in}{4.018764in}}{\pgfqpoint{6.706124in}{4.010950in}}%
\pgfpathcurveto{\pgfqpoint{6.698310in}{4.003137in}}{\pgfqpoint{6.693920in}{3.992538in}}{\pgfqpoint{6.693920in}{3.981487in}}%
\pgfpathcurveto{\pgfqpoint{6.693920in}{3.970437in}}{\pgfqpoint{6.698310in}{3.959838in}}{\pgfqpoint{6.706124in}{3.952025in}}%
\pgfpathcurveto{\pgfqpoint{6.713938in}{3.944211in}}{\pgfqpoint{6.724537in}{3.939821in}}{\pgfqpoint{6.735587in}{3.939821in}}%
\pgfpathclose%
\pgfusepath{stroke,fill}%
\end{pgfscope}%
\begin{pgfscope}%
\pgfpathrectangle{\pgfqpoint{0.570343in}{0.331635in}}{\pgfqpoint{9.300000in}{7.700000in}}%
\pgfusepath{clip}%
\pgfsetbuttcap%
\pgfsetroundjoin%
\definecolor{currentfill}{rgb}{1.000000,0.705882,0.509804}%
\pgfsetfillcolor{currentfill}%
\pgfsetlinewidth{0.481800pt}%
\definecolor{currentstroke}{rgb}{1.000000,1.000000,1.000000}%
\pgfsetstrokecolor{currentstroke}%
\pgfsetdash{}{0pt}%
\pgfpathmoveto{\pgfqpoint{3.789899in}{1.578456in}}%
\pgfpathcurveto{\pgfqpoint{3.800949in}{1.578456in}}{\pgfqpoint{3.811548in}{1.582846in}}{\pgfqpoint{3.819362in}{1.590660in}}%
\pgfpathcurveto{\pgfqpoint{3.827176in}{1.598474in}}{\pgfqpoint{3.831566in}{1.609073in}}{\pgfqpoint{3.831566in}{1.620123in}}%
\pgfpathcurveto{\pgfqpoint{3.831566in}{1.631173in}}{\pgfqpoint{3.827176in}{1.641772in}}{\pgfqpoint{3.819362in}{1.649586in}}%
\pgfpathcurveto{\pgfqpoint{3.811548in}{1.657399in}}{\pgfqpoint{3.800949in}{1.661790in}}{\pgfqpoint{3.789899in}{1.661790in}}%
\pgfpathcurveto{\pgfqpoint{3.778849in}{1.661790in}}{\pgfqpoint{3.768250in}{1.657399in}}{\pgfqpoint{3.760436in}{1.649586in}}%
\pgfpathcurveto{\pgfqpoint{3.752623in}{1.641772in}}{\pgfqpoint{3.748233in}{1.631173in}}{\pgfqpoint{3.748233in}{1.620123in}}%
\pgfpathcurveto{\pgfqpoint{3.748233in}{1.609073in}}{\pgfqpoint{3.752623in}{1.598474in}}{\pgfqpoint{3.760436in}{1.590660in}}%
\pgfpathcurveto{\pgfqpoint{3.768250in}{1.582846in}}{\pgfqpoint{3.778849in}{1.578456in}}{\pgfqpoint{3.789899in}{1.578456in}}%
\pgfpathclose%
\pgfusepath{stroke,fill}%
\end{pgfscope}%
\begin{pgfscope}%
\pgfpathrectangle{\pgfqpoint{0.570343in}{0.331635in}}{\pgfqpoint{9.300000in}{7.700000in}}%
\pgfusepath{clip}%
\pgfsetbuttcap%
\pgfsetroundjoin%
\definecolor{currentfill}{rgb}{1.000000,0.705882,0.509804}%
\pgfsetfillcolor{currentfill}%
\pgfsetlinewidth{0.481800pt}%
\definecolor{currentstroke}{rgb}{1.000000,1.000000,1.000000}%
\pgfsetstrokecolor{currentstroke}%
\pgfsetdash{}{0pt}%
\pgfpathmoveto{\pgfqpoint{6.698828in}{4.753589in}}%
\pgfpathcurveto{\pgfqpoint{6.709878in}{4.753589in}}{\pgfqpoint{6.720477in}{4.757979in}}{\pgfqpoint{6.728291in}{4.765793in}}%
\pgfpathcurveto{\pgfqpoint{6.736104in}{4.773607in}}{\pgfqpoint{6.740495in}{4.784206in}}{\pgfqpoint{6.740495in}{4.795256in}}%
\pgfpathcurveto{\pgfqpoint{6.740495in}{4.806306in}}{\pgfqpoint{6.736104in}{4.816905in}}{\pgfqpoint{6.728291in}{4.824718in}}%
\pgfpathcurveto{\pgfqpoint{6.720477in}{4.832532in}}{\pgfqpoint{6.709878in}{4.836922in}}{\pgfqpoint{6.698828in}{4.836922in}}%
\pgfpathcurveto{\pgfqpoint{6.687778in}{4.836922in}}{\pgfqpoint{6.677179in}{4.832532in}}{\pgfqpoint{6.669365in}{4.824718in}}%
\pgfpathcurveto{\pgfqpoint{6.661551in}{4.816905in}}{\pgfqpoint{6.657161in}{4.806306in}}{\pgfqpoint{6.657161in}{4.795256in}}%
\pgfpathcurveto{\pgfqpoint{6.657161in}{4.784206in}}{\pgfqpoint{6.661551in}{4.773607in}}{\pgfqpoint{6.669365in}{4.765793in}}%
\pgfpathcurveto{\pgfqpoint{6.677179in}{4.757979in}}{\pgfqpoint{6.687778in}{4.753589in}}{\pgfqpoint{6.698828in}{4.753589in}}%
\pgfpathclose%
\pgfusepath{stroke,fill}%
\end{pgfscope}%
\begin{pgfscope}%
\pgfpathrectangle{\pgfqpoint{0.570343in}{0.331635in}}{\pgfqpoint{9.300000in}{7.700000in}}%
\pgfusepath{clip}%
\pgfsetbuttcap%
\pgfsetroundjoin%
\definecolor{currentfill}{rgb}{1.000000,0.705882,0.509804}%
\pgfsetfillcolor{currentfill}%
\pgfsetlinewidth{0.481800pt}%
\definecolor{currentstroke}{rgb}{1.000000,1.000000,1.000000}%
\pgfsetstrokecolor{currentstroke}%
\pgfsetdash{}{0pt}%
\pgfpathmoveto{\pgfqpoint{3.751769in}{3.583704in}}%
\pgfpathcurveto{\pgfqpoint{3.762819in}{3.583704in}}{\pgfqpoint{3.773418in}{3.588095in}}{\pgfqpoint{3.781231in}{3.595908in}}%
\pgfpathcurveto{\pgfqpoint{3.789045in}{3.603722in}}{\pgfqpoint{3.793435in}{3.614321in}}{\pgfqpoint{3.793435in}{3.625371in}}%
\pgfpathcurveto{\pgfqpoint{3.793435in}{3.636421in}}{\pgfqpoint{3.789045in}{3.647020in}}{\pgfqpoint{3.781231in}{3.654834in}}%
\pgfpathcurveto{\pgfqpoint{3.773418in}{3.662647in}}{\pgfqpoint{3.762819in}{3.667038in}}{\pgfqpoint{3.751769in}{3.667038in}}%
\pgfpathcurveto{\pgfqpoint{3.740718in}{3.667038in}}{\pgfqpoint{3.730119in}{3.662647in}}{\pgfqpoint{3.722306in}{3.654834in}}%
\pgfpathcurveto{\pgfqpoint{3.714492in}{3.647020in}}{\pgfqpoint{3.710102in}{3.636421in}}{\pgfqpoint{3.710102in}{3.625371in}}%
\pgfpathcurveto{\pgfqpoint{3.710102in}{3.614321in}}{\pgfqpoint{3.714492in}{3.603722in}}{\pgfqpoint{3.722306in}{3.595908in}}%
\pgfpathcurveto{\pgfqpoint{3.730119in}{3.588095in}}{\pgfqpoint{3.740718in}{3.583704in}}{\pgfqpoint{3.751769in}{3.583704in}}%
\pgfpathclose%
\pgfusepath{stroke,fill}%
\end{pgfscope}%
\begin{pgfscope}%
\pgfpathrectangle{\pgfqpoint{0.570343in}{0.331635in}}{\pgfqpoint{9.300000in}{7.700000in}}%
\pgfusepath{clip}%
\pgfsetbuttcap%
\pgfsetroundjoin%
\definecolor{currentfill}{rgb}{1.000000,0.705882,0.509804}%
\pgfsetfillcolor{currentfill}%
\pgfsetlinewidth{0.481800pt}%
\definecolor{currentstroke}{rgb}{1.000000,1.000000,1.000000}%
\pgfsetstrokecolor{currentstroke}%
\pgfsetdash{}{0pt}%
\pgfpathmoveto{\pgfqpoint{7.658417in}{7.059988in}}%
\pgfpathcurveto{\pgfqpoint{7.669468in}{7.059988in}}{\pgfqpoint{7.680067in}{7.064378in}}{\pgfqpoint{7.687880in}{7.072192in}}%
\pgfpathcurveto{\pgfqpoint{7.695694in}{7.080005in}}{\pgfqpoint{7.700084in}{7.090604in}}{\pgfqpoint{7.700084in}{7.101654in}}%
\pgfpathcurveto{\pgfqpoint{7.700084in}{7.112704in}}{\pgfqpoint{7.695694in}{7.123303in}}{\pgfqpoint{7.687880in}{7.131117in}}%
\pgfpathcurveto{\pgfqpoint{7.680067in}{7.138931in}}{\pgfqpoint{7.669468in}{7.143321in}}{\pgfqpoint{7.658417in}{7.143321in}}%
\pgfpathcurveto{\pgfqpoint{7.647367in}{7.143321in}}{\pgfqpoint{7.636768in}{7.138931in}}{\pgfqpoint{7.628955in}{7.131117in}}%
\pgfpathcurveto{\pgfqpoint{7.621141in}{7.123303in}}{\pgfqpoint{7.616751in}{7.112704in}}{\pgfqpoint{7.616751in}{7.101654in}}%
\pgfpathcurveto{\pgfqpoint{7.616751in}{7.090604in}}{\pgfqpoint{7.621141in}{7.080005in}}{\pgfqpoint{7.628955in}{7.072192in}}%
\pgfpathcurveto{\pgfqpoint{7.636768in}{7.064378in}}{\pgfqpoint{7.647367in}{7.059988in}}{\pgfqpoint{7.658417in}{7.059988in}}%
\pgfpathclose%
\pgfusepath{stroke,fill}%
\end{pgfscope}%
\begin{pgfscope}%
\pgfpathrectangle{\pgfqpoint{0.570343in}{0.331635in}}{\pgfqpoint{9.300000in}{7.700000in}}%
\pgfusepath{clip}%
\pgfsetbuttcap%
\pgfsetroundjoin%
\definecolor{currentfill}{rgb}{1.000000,0.705882,0.509804}%
\pgfsetfillcolor{currentfill}%
\pgfsetlinewidth{0.481800pt}%
\definecolor{currentstroke}{rgb}{1.000000,1.000000,1.000000}%
\pgfsetstrokecolor{currentstroke}%
\pgfsetdash{}{0pt}%
\pgfpathmoveto{\pgfqpoint{4.089483in}{5.372542in}}%
\pgfpathcurveto{\pgfqpoint{4.100533in}{5.372542in}}{\pgfqpoint{4.111132in}{5.376932in}}{\pgfqpoint{4.118946in}{5.384746in}}%
\pgfpathcurveto{\pgfqpoint{4.126760in}{5.392560in}}{\pgfqpoint{4.131150in}{5.403159in}}{\pgfqpoint{4.131150in}{5.414209in}}%
\pgfpathcurveto{\pgfqpoint{4.131150in}{5.425259in}}{\pgfqpoint{4.126760in}{5.435858in}}{\pgfqpoint{4.118946in}{5.443672in}}%
\pgfpathcurveto{\pgfqpoint{4.111132in}{5.451485in}}{\pgfqpoint{4.100533in}{5.455876in}}{\pgfqpoint{4.089483in}{5.455876in}}%
\pgfpathcurveto{\pgfqpoint{4.078433in}{5.455876in}}{\pgfqpoint{4.067834in}{5.451485in}}{\pgfqpoint{4.060021in}{5.443672in}}%
\pgfpathcurveto{\pgfqpoint{4.052207in}{5.435858in}}{\pgfqpoint{4.047817in}{5.425259in}}{\pgfqpoint{4.047817in}{5.414209in}}%
\pgfpathcurveto{\pgfqpoint{4.047817in}{5.403159in}}{\pgfqpoint{4.052207in}{5.392560in}}{\pgfqpoint{4.060021in}{5.384746in}}%
\pgfpathcurveto{\pgfqpoint{4.067834in}{5.376932in}}{\pgfqpoint{4.078433in}{5.372542in}}{\pgfqpoint{4.089483in}{5.372542in}}%
\pgfpathclose%
\pgfusepath{stroke,fill}%
\end{pgfscope}%
\begin{pgfscope}%
\pgfpathrectangle{\pgfqpoint{0.570343in}{0.331635in}}{\pgfqpoint{9.300000in}{7.700000in}}%
\pgfusepath{clip}%
\pgfsetbuttcap%
\pgfsetroundjoin%
\definecolor{currentfill}{rgb}{1.000000,0.705882,0.509804}%
\pgfsetfillcolor{currentfill}%
\pgfsetlinewidth{0.481800pt}%
\definecolor{currentstroke}{rgb}{1.000000,1.000000,1.000000}%
\pgfsetstrokecolor{currentstroke}%
\pgfsetdash{}{0pt}%
\pgfpathmoveto{\pgfqpoint{4.728764in}{5.503824in}}%
\pgfpathcurveto{\pgfqpoint{4.739815in}{5.503824in}}{\pgfqpoint{4.750414in}{5.508214in}}{\pgfqpoint{4.758227in}{5.516028in}}%
\pgfpathcurveto{\pgfqpoint{4.766041in}{5.523841in}}{\pgfqpoint{4.770431in}{5.534441in}}{\pgfqpoint{4.770431in}{5.545491in}}%
\pgfpathcurveto{\pgfqpoint{4.770431in}{5.556541in}}{\pgfqpoint{4.766041in}{5.567140in}}{\pgfqpoint{4.758227in}{5.574953in}}%
\pgfpathcurveto{\pgfqpoint{4.750414in}{5.582767in}}{\pgfqpoint{4.739815in}{5.587157in}}{\pgfqpoint{4.728764in}{5.587157in}}%
\pgfpathcurveto{\pgfqpoint{4.717714in}{5.587157in}}{\pgfqpoint{4.707115in}{5.582767in}}{\pgfqpoint{4.699302in}{5.574953in}}%
\pgfpathcurveto{\pgfqpoint{4.691488in}{5.567140in}}{\pgfqpoint{4.687098in}{5.556541in}}{\pgfqpoint{4.687098in}{5.545491in}}%
\pgfpathcurveto{\pgfqpoint{4.687098in}{5.534441in}}{\pgfqpoint{4.691488in}{5.523841in}}{\pgfqpoint{4.699302in}{5.516028in}}%
\pgfpathcurveto{\pgfqpoint{4.707115in}{5.508214in}}{\pgfqpoint{4.717714in}{5.503824in}}{\pgfqpoint{4.728764in}{5.503824in}}%
\pgfpathclose%
\pgfusepath{stroke,fill}%
\end{pgfscope}%
\begin{pgfscope}%
\pgfpathrectangle{\pgfqpoint{0.570343in}{0.331635in}}{\pgfqpoint{9.300000in}{7.700000in}}%
\pgfusepath{clip}%
\pgfsetbuttcap%
\pgfsetroundjoin%
\definecolor{currentfill}{rgb}{1.000000,0.705882,0.509804}%
\pgfsetfillcolor{currentfill}%
\pgfsetlinewidth{0.481800pt}%
\definecolor{currentstroke}{rgb}{1.000000,1.000000,1.000000}%
\pgfsetstrokecolor{currentstroke}%
\pgfsetdash{}{0pt}%
\pgfpathmoveto{\pgfqpoint{5.530826in}{3.793248in}}%
\pgfpathcurveto{\pgfqpoint{5.541876in}{3.793248in}}{\pgfqpoint{5.552475in}{3.797638in}}{\pgfqpoint{5.560289in}{3.805451in}}%
\pgfpathcurveto{\pgfqpoint{5.568103in}{3.813265in}}{\pgfqpoint{5.572493in}{3.823864in}}{\pgfqpoint{5.572493in}{3.834914in}}%
\pgfpathcurveto{\pgfqpoint{5.572493in}{3.845964in}}{\pgfqpoint{5.568103in}{3.856563in}}{\pgfqpoint{5.560289in}{3.864377in}}%
\pgfpathcurveto{\pgfqpoint{5.552475in}{3.872191in}}{\pgfqpoint{5.541876in}{3.876581in}}{\pgfqpoint{5.530826in}{3.876581in}}%
\pgfpathcurveto{\pgfqpoint{5.519776in}{3.876581in}}{\pgfqpoint{5.509177in}{3.872191in}}{\pgfqpoint{5.501363in}{3.864377in}}%
\pgfpathcurveto{\pgfqpoint{5.493550in}{3.856563in}}{\pgfqpoint{5.489160in}{3.845964in}}{\pgfqpoint{5.489160in}{3.834914in}}%
\pgfpathcurveto{\pgfqpoint{5.489160in}{3.823864in}}{\pgfqpoint{5.493550in}{3.813265in}}{\pgfqpoint{5.501363in}{3.805451in}}%
\pgfpathcurveto{\pgfqpoint{5.509177in}{3.797638in}}{\pgfqpoint{5.519776in}{3.793248in}}{\pgfqpoint{5.530826in}{3.793248in}}%
\pgfpathclose%
\pgfusepath{stroke,fill}%
\end{pgfscope}%
\begin{pgfscope}%
\pgfpathrectangle{\pgfqpoint{0.570343in}{0.331635in}}{\pgfqpoint{9.300000in}{7.700000in}}%
\pgfusepath{clip}%
\pgfsetbuttcap%
\pgfsetroundjoin%
\definecolor{currentfill}{rgb}{1.000000,0.705882,0.509804}%
\pgfsetfillcolor{currentfill}%
\pgfsetlinewidth{0.481800pt}%
\definecolor{currentstroke}{rgb}{1.000000,1.000000,1.000000}%
\pgfsetstrokecolor{currentstroke}%
\pgfsetdash{}{0pt}%
\pgfpathmoveto{\pgfqpoint{4.970295in}{0.639968in}}%
\pgfpathcurveto{\pgfqpoint{4.981345in}{0.639968in}}{\pgfqpoint{4.991944in}{0.644359in}}{\pgfqpoint{4.999757in}{0.652172in}}%
\pgfpathcurveto{\pgfqpoint{5.007571in}{0.659986in}}{\pgfqpoint{5.011961in}{0.670585in}}{\pgfqpoint{5.011961in}{0.681635in}}%
\pgfpathcurveto{\pgfqpoint{5.011961in}{0.692685in}}{\pgfqpoint{5.007571in}{0.703284in}}{\pgfqpoint{4.999757in}{0.711098in}}%
\pgfpathcurveto{\pgfqpoint{4.991944in}{0.718911in}}{\pgfqpoint{4.981345in}{0.723302in}}{\pgfqpoint{4.970295in}{0.723302in}}%
\pgfpathcurveto{\pgfqpoint{4.959245in}{0.723302in}}{\pgfqpoint{4.948646in}{0.718911in}}{\pgfqpoint{4.940832in}{0.711098in}}%
\pgfpathcurveto{\pgfqpoint{4.933018in}{0.703284in}}{\pgfqpoint{4.928628in}{0.692685in}}{\pgfqpoint{4.928628in}{0.681635in}}%
\pgfpathcurveto{\pgfqpoint{4.928628in}{0.670585in}}{\pgfqpoint{4.933018in}{0.659986in}}{\pgfqpoint{4.940832in}{0.652172in}}%
\pgfpathcurveto{\pgfqpoint{4.948646in}{0.644359in}}{\pgfqpoint{4.959245in}{0.639968in}}{\pgfqpoint{4.970295in}{0.639968in}}%
\pgfpathclose%
\pgfusepath{stroke,fill}%
\end{pgfscope}%
\begin{pgfscope}%
\pgfpathrectangle{\pgfqpoint{0.570343in}{0.331635in}}{\pgfqpoint{9.300000in}{7.700000in}}%
\pgfusepath{clip}%
\pgfsetbuttcap%
\pgfsetroundjoin%
\definecolor{currentfill}{rgb}{1.000000,0.705882,0.509804}%
\pgfsetfillcolor{currentfill}%
\pgfsetlinewidth{0.481800pt}%
\definecolor{currentstroke}{rgb}{1.000000,1.000000,1.000000}%
\pgfsetstrokecolor{currentstroke}%
\pgfsetdash{}{0pt}%
\pgfpathmoveto{\pgfqpoint{7.708486in}{4.654135in}}%
\pgfpathcurveto{\pgfqpoint{7.719536in}{4.654135in}}{\pgfqpoint{7.730135in}{4.658526in}}{\pgfqpoint{7.737949in}{4.666339in}}%
\pgfpathcurveto{\pgfqpoint{7.745762in}{4.674153in}}{\pgfqpoint{7.750153in}{4.684752in}}{\pgfqpoint{7.750153in}{4.695802in}}%
\pgfpathcurveto{\pgfqpoint{7.750153in}{4.706852in}}{\pgfqpoint{7.745762in}{4.717451in}}{\pgfqpoint{7.737949in}{4.725265in}}%
\pgfpathcurveto{\pgfqpoint{7.730135in}{4.733078in}}{\pgfqpoint{7.719536in}{4.737469in}}{\pgfqpoint{7.708486in}{4.737469in}}%
\pgfpathcurveto{\pgfqpoint{7.697436in}{4.737469in}}{\pgfqpoint{7.686837in}{4.733078in}}{\pgfqpoint{7.679023in}{4.725265in}}%
\pgfpathcurveto{\pgfqpoint{7.671210in}{4.717451in}}{\pgfqpoint{7.666819in}{4.706852in}}{\pgfqpoint{7.666819in}{4.695802in}}%
\pgfpathcurveto{\pgfqpoint{7.666819in}{4.684752in}}{\pgfqpoint{7.671210in}{4.674153in}}{\pgfqpoint{7.679023in}{4.666339in}}%
\pgfpathcurveto{\pgfqpoint{7.686837in}{4.658526in}}{\pgfqpoint{7.697436in}{4.654135in}}{\pgfqpoint{7.708486in}{4.654135in}}%
\pgfpathclose%
\pgfusepath{stroke,fill}%
\end{pgfscope}%
\begin{pgfscope}%
\pgfpathrectangle{\pgfqpoint{0.570343in}{0.331635in}}{\pgfqpoint{9.300000in}{7.700000in}}%
\pgfusepath{clip}%
\pgfsetbuttcap%
\pgfsetroundjoin%
\definecolor{currentfill}{rgb}{1.000000,0.705882,0.509804}%
\pgfsetfillcolor{currentfill}%
\pgfsetlinewidth{0.481800pt}%
\definecolor{currentstroke}{rgb}{1.000000,1.000000,1.000000}%
\pgfsetstrokecolor{currentstroke}%
\pgfsetdash{}{0pt}%
\pgfpathmoveto{\pgfqpoint{5.865552in}{3.196333in}}%
\pgfpathcurveto{\pgfqpoint{5.876602in}{3.196333in}}{\pgfqpoint{5.887201in}{3.200723in}}{\pgfqpoint{5.895015in}{3.208537in}}%
\pgfpathcurveto{\pgfqpoint{5.902828in}{3.216351in}}{\pgfqpoint{5.907218in}{3.226950in}}{\pgfqpoint{5.907218in}{3.238000in}}%
\pgfpathcurveto{\pgfqpoint{5.907218in}{3.249050in}}{\pgfqpoint{5.902828in}{3.259649in}}{\pgfqpoint{5.895015in}{3.267463in}}%
\pgfpathcurveto{\pgfqpoint{5.887201in}{3.275276in}}{\pgfqpoint{5.876602in}{3.279666in}}{\pgfqpoint{5.865552in}{3.279666in}}%
\pgfpathcurveto{\pgfqpoint{5.854502in}{3.279666in}}{\pgfqpoint{5.843903in}{3.275276in}}{\pgfqpoint{5.836089in}{3.267463in}}%
\pgfpathcurveto{\pgfqpoint{5.828275in}{3.259649in}}{\pgfqpoint{5.823885in}{3.249050in}}{\pgfqpoint{5.823885in}{3.238000in}}%
\pgfpathcurveto{\pgfqpoint{5.823885in}{3.226950in}}{\pgfqpoint{5.828275in}{3.216351in}}{\pgfqpoint{5.836089in}{3.208537in}}%
\pgfpathcurveto{\pgfqpoint{5.843903in}{3.200723in}}{\pgfqpoint{5.854502in}{3.196333in}}{\pgfqpoint{5.865552in}{3.196333in}}%
\pgfpathclose%
\pgfusepath{stroke,fill}%
\end{pgfscope}%
\begin{pgfscope}%
\pgfpathrectangle{\pgfqpoint{0.570343in}{0.331635in}}{\pgfqpoint{9.300000in}{7.700000in}}%
\pgfusepath{clip}%
\pgfsetbuttcap%
\pgfsetroundjoin%
\definecolor{currentfill}{rgb}{1.000000,0.705882,0.509804}%
\pgfsetfillcolor{currentfill}%
\pgfsetlinewidth{0.481800pt}%
\definecolor{currentstroke}{rgb}{1.000000,1.000000,1.000000}%
\pgfsetstrokecolor{currentstroke}%
\pgfsetdash{}{0pt}%
\pgfpathmoveto{\pgfqpoint{4.644801in}{4.279907in}}%
\pgfpathcurveto{\pgfqpoint{4.655851in}{4.279907in}}{\pgfqpoint{4.666450in}{4.284297in}}{\pgfqpoint{4.674263in}{4.292111in}}%
\pgfpathcurveto{\pgfqpoint{4.682077in}{4.299924in}}{\pgfqpoint{4.686467in}{4.310523in}}{\pgfqpoint{4.686467in}{4.321574in}}%
\pgfpathcurveto{\pgfqpoint{4.686467in}{4.332624in}}{\pgfqpoint{4.682077in}{4.343223in}}{\pgfqpoint{4.674263in}{4.351036in}}%
\pgfpathcurveto{\pgfqpoint{4.666450in}{4.358850in}}{\pgfqpoint{4.655851in}{4.363240in}}{\pgfqpoint{4.644801in}{4.363240in}}%
\pgfpathcurveto{\pgfqpoint{4.633751in}{4.363240in}}{\pgfqpoint{4.623152in}{4.358850in}}{\pgfqpoint{4.615338in}{4.351036in}}%
\pgfpathcurveto{\pgfqpoint{4.607524in}{4.343223in}}{\pgfqpoint{4.603134in}{4.332624in}}{\pgfqpoint{4.603134in}{4.321574in}}%
\pgfpathcurveto{\pgfqpoint{4.603134in}{4.310523in}}{\pgfqpoint{4.607524in}{4.299924in}}{\pgfqpoint{4.615338in}{4.292111in}}%
\pgfpathcurveto{\pgfqpoint{4.623152in}{4.284297in}}{\pgfqpoint{4.633751in}{4.279907in}}{\pgfqpoint{4.644801in}{4.279907in}}%
\pgfpathclose%
\pgfusepath{stroke,fill}%
\end{pgfscope}%
\begin{pgfscope}%
\pgfpathrectangle{\pgfqpoint{0.570343in}{0.331635in}}{\pgfqpoint{9.300000in}{7.700000in}}%
\pgfusepath{clip}%
\pgfsetbuttcap%
\pgfsetroundjoin%
\definecolor{currentfill}{rgb}{1.000000,0.705882,0.509804}%
\pgfsetfillcolor{currentfill}%
\pgfsetlinewidth{0.481800pt}%
\definecolor{currentstroke}{rgb}{1.000000,1.000000,1.000000}%
\pgfsetstrokecolor{currentstroke}%
\pgfsetdash{}{0pt}%
\pgfpathmoveto{\pgfqpoint{3.446889in}{3.153655in}}%
\pgfpathcurveto{\pgfqpoint{3.457939in}{3.153655in}}{\pgfqpoint{3.468538in}{3.158045in}}{\pgfqpoint{3.476352in}{3.165859in}}%
\pgfpathcurveto{\pgfqpoint{3.484166in}{3.173672in}}{\pgfqpoint{3.488556in}{3.184271in}}{\pgfqpoint{3.488556in}{3.195321in}}%
\pgfpathcurveto{\pgfqpoint{3.488556in}{3.206372in}}{\pgfqpoint{3.484166in}{3.216971in}}{\pgfqpoint{3.476352in}{3.224784in}}%
\pgfpathcurveto{\pgfqpoint{3.468538in}{3.232598in}}{\pgfqpoint{3.457939in}{3.236988in}}{\pgfqpoint{3.446889in}{3.236988in}}%
\pgfpathcurveto{\pgfqpoint{3.435839in}{3.236988in}}{\pgfqpoint{3.425240in}{3.232598in}}{\pgfqpoint{3.417426in}{3.224784in}}%
\pgfpathcurveto{\pgfqpoint{3.409613in}{3.216971in}}{\pgfqpoint{3.405222in}{3.206372in}}{\pgfqpoint{3.405222in}{3.195321in}}%
\pgfpathcurveto{\pgfqpoint{3.405222in}{3.184271in}}{\pgfqpoint{3.409613in}{3.173672in}}{\pgfqpoint{3.417426in}{3.165859in}}%
\pgfpathcurveto{\pgfqpoint{3.425240in}{3.158045in}}{\pgfqpoint{3.435839in}{3.153655in}}{\pgfqpoint{3.446889in}{3.153655in}}%
\pgfpathclose%
\pgfusepath{stroke,fill}%
\end{pgfscope}%
\begin{pgfscope}%
\pgfpathrectangle{\pgfqpoint{0.570343in}{0.331635in}}{\pgfqpoint{9.300000in}{7.700000in}}%
\pgfusepath{clip}%
\pgfsetbuttcap%
\pgfsetroundjoin%
\definecolor{currentfill}{rgb}{1.000000,0.705882,0.509804}%
\pgfsetfillcolor{currentfill}%
\pgfsetlinewidth{0.481800pt}%
\definecolor{currentstroke}{rgb}{1.000000,1.000000,1.000000}%
\pgfsetstrokecolor{currentstroke}%
\pgfsetdash{}{0pt}%
\pgfpathmoveto{\pgfqpoint{5.205704in}{4.583283in}}%
\pgfpathcurveto{\pgfqpoint{5.216754in}{4.583283in}}{\pgfqpoint{5.227353in}{4.587673in}}{\pgfqpoint{5.235167in}{4.595487in}}%
\pgfpathcurveto{\pgfqpoint{5.242980in}{4.603300in}}{\pgfqpoint{5.247370in}{4.613899in}}{\pgfqpoint{5.247370in}{4.624950in}}%
\pgfpathcurveto{\pgfqpoint{5.247370in}{4.636000in}}{\pgfqpoint{5.242980in}{4.646599in}}{\pgfqpoint{5.235167in}{4.654412in}}%
\pgfpathcurveto{\pgfqpoint{5.227353in}{4.662226in}}{\pgfqpoint{5.216754in}{4.666616in}}{\pgfqpoint{5.205704in}{4.666616in}}%
\pgfpathcurveto{\pgfqpoint{5.194654in}{4.666616in}}{\pgfqpoint{5.184055in}{4.662226in}}{\pgfqpoint{5.176241in}{4.654412in}}%
\pgfpathcurveto{\pgfqpoint{5.168427in}{4.646599in}}{\pgfqpoint{5.164037in}{4.636000in}}{\pgfqpoint{5.164037in}{4.624950in}}%
\pgfpathcurveto{\pgfqpoint{5.164037in}{4.613899in}}{\pgfqpoint{5.168427in}{4.603300in}}{\pgfqpoint{5.176241in}{4.595487in}}%
\pgfpathcurveto{\pgfqpoint{5.184055in}{4.587673in}}{\pgfqpoint{5.194654in}{4.583283in}}{\pgfqpoint{5.205704in}{4.583283in}}%
\pgfpathclose%
\pgfusepath{stroke,fill}%
\end{pgfscope}%
\begin{pgfscope}%
\pgfpathrectangle{\pgfqpoint{0.570343in}{0.331635in}}{\pgfqpoint{9.300000in}{7.700000in}}%
\pgfusepath{clip}%
\pgfsetbuttcap%
\pgfsetroundjoin%
\definecolor{currentfill}{rgb}{1.000000,0.705882,0.509804}%
\pgfsetfillcolor{currentfill}%
\pgfsetlinewidth{0.481800pt}%
\definecolor{currentstroke}{rgb}{1.000000,1.000000,1.000000}%
\pgfsetstrokecolor{currentstroke}%
\pgfsetdash{}{0pt}%
\pgfpathmoveto{\pgfqpoint{5.477250in}{4.902321in}}%
\pgfpathcurveto{\pgfqpoint{5.488300in}{4.902321in}}{\pgfqpoint{5.498899in}{4.906712in}}{\pgfqpoint{5.506713in}{4.914525in}}%
\pgfpathcurveto{\pgfqpoint{5.514527in}{4.922339in}}{\pgfqpoint{5.518917in}{4.932938in}}{\pgfqpoint{5.518917in}{4.943988in}}%
\pgfpathcurveto{\pgfqpoint{5.518917in}{4.955038in}}{\pgfqpoint{5.514527in}{4.965637in}}{\pgfqpoint{5.506713in}{4.973451in}}%
\pgfpathcurveto{\pgfqpoint{5.498899in}{4.981264in}}{\pgfqpoint{5.488300in}{4.985655in}}{\pgfqpoint{5.477250in}{4.985655in}}%
\pgfpathcurveto{\pgfqpoint{5.466200in}{4.985655in}}{\pgfqpoint{5.455601in}{4.981264in}}{\pgfqpoint{5.447787in}{4.973451in}}%
\pgfpathcurveto{\pgfqpoint{5.439974in}{4.965637in}}{\pgfqpoint{5.435583in}{4.955038in}}{\pgfqpoint{5.435583in}{4.943988in}}%
\pgfpathcurveto{\pgfqpoint{5.435583in}{4.932938in}}{\pgfqpoint{5.439974in}{4.922339in}}{\pgfqpoint{5.447787in}{4.914525in}}%
\pgfpathcurveto{\pgfqpoint{5.455601in}{4.906712in}}{\pgfqpoint{5.466200in}{4.902321in}}{\pgfqpoint{5.477250in}{4.902321in}}%
\pgfpathclose%
\pgfusepath{stroke,fill}%
\end{pgfscope}%
\begin{pgfscope}%
\pgfpathrectangle{\pgfqpoint{0.570343in}{0.331635in}}{\pgfqpoint{9.300000in}{7.700000in}}%
\pgfusepath{clip}%
\pgfsetbuttcap%
\pgfsetroundjoin%
\definecolor{currentfill}{rgb}{1.000000,0.705882,0.509804}%
\pgfsetfillcolor{currentfill}%
\pgfsetlinewidth{0.481800pt}%
\definecolor{currentstroke}{rgb}{1.000000,1.000000,1.000000}%
\pgfsetstrokecolor{currentstroke}%
\pgfsetdash{}{0pt}%
\pgfpathmoveto{\pgfqpoint{4.499527in}{1.562270in}}%
\pgfpathcurveto{\pgfqpoint{4.510578in}{1.562270in}}{\pgfqpoint{4.521177in}{1.566661in}}{\pgfqpoint{4.528990in}{1.574474in}}%
\pgfpathcurveto{\pgfqpoint{4.536804in}{1.582288in}}{\pgfqpoint{4.541194in}{1.592887in}}{\pgfqpoint{4.541194in}{1.603937in}}%
\pgfpathcurveto{\pgfqpoint{4.541194in}{1.614987in}}{\pgfqpoint{4.536804in}{1.625586in}}{\pgfqpoint{4.528990in}{1.633400in}}%
\pgfpathcurveto{\pgfqpoint{4.521177in}{1.641213in}}{\pgfqpoint{4.510578in}{1.645604in}}{\pgfqpoint{4.499527in}{1.645604in}}%
\pgfpathcurveto{\pgfqpoint{4.488477in}{1.645604in}}{\pgfqpoint{4.477878in}{1.641213in}}{\pgfqpoint{4.470065in}{1.633400in}}%
\pgfpathcurveto{\pgfqpoint{4.462251in}{1.625586in}}{\pgfqpoint{4.457861in}{1.614987in}}{\pgfqpoint{4.457861in}{1.603937in}}%
\pgfpathcurveto{\pgfqpoint{4.457861in}{1.592887in}}{\pgfqpoint{4.462251in}{1.582288in}}{\pgfqpoint{4.470065in}{1.574474in}}%
\pgfpathcurveto{\pgfqpoint{4.477878in}{1.566661in}}{\pgfqpoint{4.488477in}{1.562270in}}{\pgfqpoint{4.499527in}{1.562270in}}%
\pgfpathclose%
\pgfusepath{stroke,fill}%
\end{pgfscope}%
\begin{pgfscope}%
\pgfpathrectangle{\pgfqpoint{0.570343in}{0.331635in}}{\pgfqpoint{9.300000in}{7.700000in}}%
\pgfusepath{clip}%
\pgfsetbuttcap%
\pgfsetroundjoin%
\definecolor{currentfill}{rgb}{1.000000,0.705882,0.509804}%
\pgfsetfillcolor{currentfill}%
\pgfsetlinewidth{0.481800pt}%
\definecolor{currentstroke}{rgb}{1.000000,1.000000,1.000000}%
\pgfsetstrokecolor{currentstroke}%
\pgfsetdash{}{0pt}%
\pgfpathmoveto{\pgfqpoint{4.967473in}{2.141235in}}%
\pgfpathcurveto{\pgfqpoint{4.978523in}{2.141235in}}{\pgfqpoint{4.989122in}{2.145625in}}{\pgfqpoint{4.996935in}{2.153438in}}%
\pgfpathcurveto{\pgfqpoint{5.004749in}{2.161252in}}{\pgfqpoint{5.009139in}{2.171851in}}{\pgfqpoint{5.009139in}{2.182901in}}%
\pgfpathcurveto{\pgfqpoint{5.009139in}{2.193951in}}{\pgfqpoint{5.004749in}{2.204550in}}{\pgfqpoint{4.996935in}{2.212364in}}%
\pgfpathcurveto{\pgfqpoint{4.989122in}{2.220178in}}{\pgfqpoint{4.978523in}{2.224568in}}{\pgfqpoint{4.967473in}{2.224568in}}%
\pgfpathcurveto{\pgfqpoint{4.956422in}{2.224568in}}{\pgfqpoint{4.945823in}{2.220178in}}{\pgfqpoint{4.938010in}{2.212364in}}%
\pgfpathcurveto{\pgfqpoint{4.930196in}{2.204550in}}{\pgfqpoint{4.925806in}{2.193951in}}{\pgfqpoint{4.925806in}{2.182901in}}%
\pgfpathcurveto{\pgfqpoint{4.925806in}{2.171851in}}{\pgfqpoint{4.930196in}{2.161252in}}{\pgfqpoint{4.938010in}{2.153438in}}%
\pgfpathcurveto{\pgfqpoint{4.945823in}{2.145625in}}{\pgfqpoint{4.956422in}{2.141235in}}{\pgfqpoint{4.967473in}{2.141235in}}%
\pgfpathclose%
\pgfusepath{stroke,fill}%
\end{pgfscope}%
\begin{pgfscope}%
\pgfpathrectangle{\pgfqpoint{0.570343in}{0.331635in}}{\pgfqpoint{9.300000in}{7.700000in}}%
\pgfusepath{clip}%
\pgfsetbuttcap%
\pgfsetroundjoin%
\definecolor{currentfill}{rgb}{1.000000,0.705882,0.509804}%
\pgfsetfillcolor{currentfill}%
\pgfsetlinewidth{0.481800pt}%
\definecolor{currentstroke}{rgb}{1.000000,1.000000,1.000000}%
\pgfsetstrokecolor{currentstroke}%
\pgfsetdash{}{0pt}%
\pgfpathmoveto{\pgfqpoint{5.200385in}{5.874887in}}%
\pgfpathcurveto{\pgfqpoint{5.211435in}{5.874887in}}{\pgfqpoint{5.222034in}{5.879277in}}{\pgfqpoint{5.229848in}{5.887091in}}%
\pgfpathcurveto{\pgfqpoint{5.237662in}{5.894904in}}{\pgfqpoint{5.242052in}{5.905503in}}{\pgfqpoint{5.242052in}{5.916554in}}%
\pgfpathcurveto{\pgfqpoint{5.242052in}{5.927604in}}{\pgfqpoint{5.237662in}{5.938203in}}{\pgfqpoint{5.229848in}{5.946016in}}%
\pgfpathcurveto{\pgfqpoint{5.222034in}{5.953830in}}{\pgfqpoint{5.211435in}{5.958220in}}{\pgfqpoint{5.200385in}{5.958220in}}%
\pgfpathcurveto{\pgfqpoint{5.189335in}{5.958220in}}{\pgfqpoint{5.178736in}{5.953830in}}{\pgfqpoint{5.170922in}{5.946016in}}%
\pgfpathcurveto{\pgfqpoint{5.163109in}{5.938203in}}{\pgfqpoint{5.158719in}{5.927604in}}{\pgfqpoint{5.158719in}{5.916554in}}%
\pgfpathcurveto{\pgfqpoint{5.158719in}{5.905503in}}{\pgfqpoint{5.163109in}{5.894904in}}{\pgfqpoint{5.170922in}{5.887091in}}%
\pgfpathcurveto{\pgfqpoint{5.178736in}{5.879277in}}{\pgfqpoint{5.189335in}{5.874887in}}{\pgfqpoint{5.200385in}{5.874887in}}%
\pgfpathclose%
\pgfusepath{stroke,fill}%
\end{pgfscope}%
\begin{pgfscope}%
\pgfpathrectangle{\pgfqpoint{0.570343in}{0.331635in}}{\pgfqpoint{9.300000in}{7.700000in}}%
\pgfusepath{clip}%
\pgfsetbuttcap%
\pgfsetroundjoin%
\definecolor{currentfill}{rgb}{1.000000,0.705882,0.509804}%
\pgfsetfillcolor{currentfill}%
\pgfsetlinewidth{0.481800pt}%
\definecolor{currentstroke}{rgb}{1.000000,1.000000,1.000000}%
\pgfsetstrokecolor{currentstroke}%
\pgfsetdash{}{0pt}%
\pgfpathmoveto{\pgfqpoint{3.099182in}{3.893732in}}%
\pgfpathcurveto{\pgfqpoint{3.110232in}{3.893732in}}{\pgfqpoint{3.120831in}{3.898122in}}{\pgfqpoint{3.128644in}{3.905936in}}%
\pgfpathcurveto{\pgfqpoint{3.136458in}{3.913750in}}{\pgfqpoint{3.140848in}{3.924349in}}{\pgfqpoint{3.140848in}{3.935399in}}%
\pgfpathcurveto{\pgfqpoint{3.140848in}{3.946449in}}{\pgfqpoint{3.136458in}{3.957048in}}{\pgfqpoint{3.128644in}{3.964862in}}%
\pgfpathcurveto{\pgfqpoint{3.120831in}{3.972675in}}{\pgfqpoint{3.110232in}{3.977065in}}{\pgfqpoint{3.099182in}{3.977065in}}%
\pgfpathcurveto{\pgfqpoint{3.088131in}{3.977065in}}{\pgfqpoint{3.077532in}{3.972675in}}{\pgfqpoint{3.069719in}{3.964862in}}%
\pgfpathcurveto{\pgfqpoint{3.061905in}{3.957048in}}{\pgfqpoint{3.057515in}{3.946449in}}{\pgfqpoint{3.057515in}{3.935399in}}%
\pgfpathcurveto{\pgfqpoint{3.057515in}{3.924349in}}{\pgfqpoint{3.061905in}{3.913750in}}{\pgfqpoint{3.069719in}{3.905936in}}%
\pgfpathcurveto{\pgfqpoint{3.077532in}{3.898122in}}{\pgfqpoint{3.088131in}{3.893732in}}{\pgfqpoint{3.099182in}{3.893732in}}%
\pgfpathclose%
\pgfusepath{stroke,fill}%
\end{pgfscope}%
\begin{pgfscope}%
\pgfpathrectangle{\pgfqpoint{0.570343in}{0.331635in}}{\pgfqpoint{9.300000in}{7.700000in}}%
\pgfusepath{clip}%
\pgfsetbuttcap%
\pgfsetroundjoin%
\definecolor{currentfill}{rgb}{1.000000,0.705882,0.509804}%
\pgfsetfillcolor{currentfill}%
\pgfsetlinewidth{0.481800pt}%
\definecolor{currentstroke}{rgb}{1.000000,1.000000,1.000000}%
\pgfsetstrokecolor{currentstroke}%
\pgfsetdash{}{0pt}%
\pgfpathmoveto{\pgfqpoint{5.161067in}{5.103647in}}%
\pgfpathcurveto{\pgfqpoint{5.172117in}{5.103647in}}{\pgfqpoint{5.182716in}{5.108038in}}{\pgfqpoint{5.190529in}{5.115851in}}%
\pgfpathcurveto{\pgfqpoint{5.198343in}{5.123665in}}{\pgfqpoint{5.202733in}{5.134264in}}{\pgfqpoint{5.202733in}{5.145314in}}%
\pgfpathcurveto{\pgfqpoint{5.202733in}{5.156364in}}{\pgfqpoint{5.198343in}{5.166963in}}{\pgfqpoint{5.190529in}{5.174777in}}%
\pgfpathcurveto{\pgfqpoint{5.182716in}{5.182591in}}{\pgfqpoint{5.172117in}{5.186981in}}{\pgfqpoint{5.161067in}{5.186981in}}%
\pgfpathcurveto{\pgfqpoint{5.150016in}{5.186981in}}{\pgfqpoint{5.139417in}{5.182591in}}{\pgfqpoint{5.131604in}{5.174777in}}%
\pgfpathcurveto{\pgfqpoint{5.123790in}{5.166963in}}{\pgfqpoint{5.119400in}{5.156364in}}{\pgfqpoint{5.119400in}{5.145314in}}%
\pgfpathcurveto{\pgfqpoint{5.119400in}{5.134264in}}{\pgfqpoint{5.123790in}{5.123665in}}{\pgfqpoint{5.131604in}{5.115851in}}%
\pgfpathcurveto{\pgfqpoint{5.139417in}{5.108038in}}{\pgfqpoint{5.150016in}{5.103647in}}{\pgfqpoint{5.161067in}{5.103647in}}%
\pgfpathclose%
\pgfusepath{stroke,fill}%
\end{pgfscope}%
\begin{pgfscope}%
\pgfpathrectangle{\pgfqpoint{0.570343in}{0.331635in}}{\pgfqpoint{9.300000in}{7.700000in}}%
\pgfusepath{clip}%
\pgfsetbuttcap%
\pgfsetroundjoin%
\definecolor{currentfill}{rgb}{1.000000,0.705882,0.509804}%
\pgfsetfillcolor{currentfill}%
\pgfsetlinewidth{0.481800pt}%
\definecolor{currentstroke}{rgb}{1.000000,1.000000,1.000000}%
\pgfsetstrokecolor{currentstroke}%
\pgfsetdash{}{0pt}%
\pgfpathmoveto{\pgfqpoint{3.646482in}{2.425017in}}%
\pgfpathcurveto{\pgfqpoint{3.657532in}{2.425017in}}{\pgfqpoint{3.668131in}{2.429407in}}{\pgfqpoint{3.675945in}{2.437220in}}%
\pgfpathcurveto{\pgfqpoint{3.683758in}{2.445034in}}{\pgfqpoint{3.688149in}{2.455633in}}{\pgfqpoint{3.688149in}{2.466683in}}%
\pgfpathcurveto{\pgfqpoint{3.688149in}{2.477733in}}{\pgfqpoint{3.683758in}{2.488332in}}{\pgfqpoint{3.675945in}{2.496146in}}%
\pgfpathcurveto{\pgfqpoint{3.668131in}{2.503960in}}{\pgfqpoint{3.657532in}{2.508350in}}{\pgfqpoint{3.646482in}{2.508350in}}%
\pgfpathcurveto{\pgfqpoint{3.635432in}{2.508350in}}{\pgfqpoint{3.624833in}{2.503960in}}{\pgfqpoint{3.617019in}{2.496146in}}%
\pgfpathcurveto{\pgfqpoint{3.609206in}{2.488332in}}{\pgfqpoint{3.604815in}{2.477733in}}{\pgfqpoint{3.604815in}{2.466683in}}%
\pgfpathcurveto{\pgfqpoint{3.604815in}{2.455633in}}{\pgfqpoint{3.609206in}{2.445034in}}{\pgfqpoint{3.617019in}{2.437220in}}%
\pgfpathcurveto{\pgfqpoint{3.624833in}{2.429407in}}{\pgfqpoint{3.635432in}{2.425017in}}{\pgfqpoint{3.646482in}{2.425017in}}%
\pgfpathclose%
\pgfusepath{stroke,fill}%
\end{pgfscope}%
\begin{pgfscope}%
\pgfpathrectangle{\pgfqpoint{0.570343in}{0.331635in}}{\pgfqpoint{9.300000in}{7.700000in}}%
\pgfusepath{clip}%
\pgfsetbuttcap%
\pgfsetroundjoin%
\definecolor{currentfill}{rgb}{1.000000,0.705882,0.509804}%
\pgfsetfillcolor{currentfill}%
\pgfsetlinewidth{0.481800pt}%
\definecolor{currentstroke}{rgb}{1.000000,1.000000,1.000000}%
\pgfsetstrokecolor{currentstroke}%
\pgfsetdash{}{0pt}%
\pgfpathmoveto{\pgfqpoint{3.852857in}{2.780127in}}%
\pgfpathcurveto{\pgfqpoint{3.863908in}{2.780127in}}{\pgfqpoint{3.874507in}{2.784518in}}{\pgfqpoint{3.882320in}{2.792331in}}%
\pgfpathcurveto{\pgfqpoint{3.890134in}{2.800145in}}{\pgfqpoint{3.894524in}{2.810744in}}{\pgfqpoint{3.894524in}{2.821794in}}%
\pgfpathcurveto{\pgfqpoint{3.894524in}{2.832844in}}{\pgfqpoint{3.890134in}{2.843443in}}{\pgfqpoint{3.882320in}{2.851257in}}%
\pgfpathcurveto{\pgfqpoint{3.874507in}{2.859071in}}{\pgfqpoint{3.863908in}{2.863461in}}{\pgfqpoint{3.852857in}{2.863461in}}%
\pgfpathcurveto{\pgfqpoint{3.841807in}{2.863461in}}{\pgfqpoint{3.831208in}{2.859071in}}{\pgfqpoint{3.823395in}{2.851257in}}%
\pgfpathcurveto{\pgfqpoint{3.815581in}{2.843443in}}{\pgfqpoint{3.811191in}{2.832844in}}{\pgfqpoint{3.811191in}{2.821794in}}%
\pgfpathcurveto{\pgfqpoint{3.811191in}{2.810744in}}{\pgfqpoint{3.815581in}{2.800145in}}{\pgfqpoint{3.823395in}{2.792331in}}%
\pgfpathcurveto{\pgfqpoint{3.831208in}{2.784518in}}{\pgfqpoint{3.841807in}{2.780127in}}{\pgfqpoint{3.852857in}{2.780127in}}%
\pgfpathclose%
\pgfusepath{stroke,fill}%
\end{pgfscope}%
\begin{pgfscope}%
\pgfpathrectangle{\pgfqpoint{0.570343in}{0.331635in}}{\pgfqpoint{9.300000in}{7.700000in}}%
\pgfusepath{clip}%
\pgfsetbuttcap%
\pgfsetroundjoin%
\definecolor{currentfill}{rgb}{1.000000,0.705882,0.509804}%
\pgfsetfillcolor{currentfill}%
\pgfsetlinewidth{0.481800pt}%
\definecolor{currentstroke}{rgb}{1.000000,1.000000,1.000000}%
\pgfsetstrokecolor{currentstroke}%
\pgfsetdash{}{0pt}%
\pgfpathmoveto{\pgfqpoint{5.635152in}{5.298062in}}%
\pgfpathcurveto{\pgfqpoint{5.646202in}{5.298062in}}{\pgfqpoint{5.656801in}{5.302452in}}{\pgfqpoint{5.664615in}{5.310266in}}%
\pgfpathcurveto{\pgfqpoint{5.672429in}{5.318079in}}{\pgfqpoint{5.676819in}{5.328678in}}{\pgfqpoint{5.676819in}{5.339728in}}%
\pgfpathcurveto{\pgfqpoint{5.676819in}{5.350779in}}{\pgfqpoint{5.672429in}{5.361378in}}{\pgfqpoint{5.664615in}{5.369191in}}%
\pgfpathcurveto{\pgfqpoint{5.656801in}{5.377005in}}{\pgfqpoint{5.646202in}{5.381395in}}{\pgfqpoint{5.635152in}{5.381395in}}%
\pgfpathcurveto{\pgfqpoint{5.624102in}{5.381395in}}{\pgfqpoint{5.613503in}{5.377005in}}{\pgfqpoint{5.605689in}{5.369191in}}%
\pgfpathcurveto{\pgfqpoint{5.597876in}{5.361378in}}{\pgfqpoint{5.593485in}{5.350779in}}{\pgfqpoint{5.593485in}{5.339728in}}%
\pgfpathcurveto{\pgfqpoint{5.593485in}{5.328678in}}{\pgfqpoint{5.597876in}{5.318079in}}{\pgfqpoint{5.605689in}{5.310266in}}%
\pgfpathcurveto{\pgfqpoint{5.613503in}{5.302452in}}{\pgfqpoint{5.624102in}{5.298062in}}{\pgfqpoint{5.635152in}{5.298062in}}%
\pgfpathclose%
\pgfusepath{stroke,fill}%
\end{pgfscope}%
\begin{pgfscope}%
\pgfpathrectangle{\pgfqpoint{0.570343in}{0.331635in}}{\pgfqpoint{9.300000in}{7.700000in}}%
\pgfusepath{clip}%
\pgfsetbuttcap%
\pgfsetroundjoin%
\definecolor{currentfill}{rgb}{1.000000,0.705882,0.509804}%
\pgfsetfillcolor{currentfill}%
\pgfsetlinewidth{0.481800pt}%
\definecolor{currentstroke}{rgb}{1.000000,1.000000,1.000000}%
\pgfsetstrokecolor{currentstroke}%
\pgfsetdash{}{0pt}%
\pgfpathmoveto{\pgfqpoint{7.128037in}{4.559279in}}%
\pgfpathcurveto{\pgfqpoint{7.139087in}{4.559279in}}{\pgfqpoint{7.149686in}{4.563670in}}{\pgfqpoint{7.157500in}{4.571483in}}%
\pgfpathcurveto{\pgfqpoint{7.165313in}{4.579297in}}{\pgfqpoint{7.169704in}{4.589896in}}{\pgfqpoint{7.169704in}{4.600946in}}%
\pgfpathcurveto{\pgfqpoint{7.169704in}{4.611996in}}{\pgfqpoint{7.165313in}{4.622595in}}{\pgfqpoint{7.157500in}{4.630409in}}%
\pgfpathcurveto{\pgfqpoint{7.149686in}{4.638222in}}{\pgfqpoint{7.139087in}{4.642613in}}{\pgfqpoint{7.128037in}{4.642613in}}%
\pgfpathcurveto{\pgfqpoint{7.116987in}{4.642613in}}{\pgfqpoint{7.106388in}{4.638222in}}{\pgfqpoint{7.098574in}{4.630409in}}%
\pgfpathcurveto{\pgfqpoint{7.090760in}{4.622595in}}{\pgfqpoint{7.086370in}{4.611996in}}{\pgfqpoint{7.086370in}{4.600946in}}%
\pgfpathcurveto{\pgfqpoint{7.086370in}{4.589896in}}{\pgfqpoint{7.090760in}{4.579297in}}{\pgfqpoint{7.098574in}{4.571483in}}%
\pgfpathcurveto{\pgfqpoint{7.106388in}{4.563670in}}{\pgfqpoint{7.116987in}{4.559279in}}{\pgfqpoint{7.128037in}{4.559279in}}%
\pgfpathclose%
\pgfusepath{stroke,fill}%
\end{pgfscope}%
\begin{pgfscope}%
\pgfpathrectangle{\pgfqpoint{0.570343in}{0.331635in}}{\pgfqpoint{9.300000in}{7.700000in}}%
\pgfusepath{clip}%
\pgfsetbuttcap%
\pgfsetroundjoin%
\definecolor{currentfill}{rgb}{1.000000,0.705882,0.509804}%
\pgfsetfillcolor{currentfill}%
\pgfsetlinewidth{0.481800pt}%
\definecolor{currentstroke}{rgb}{1.000000,1.000000,1.000000}%
\pgfsetstrokecolor{currentstroke}%
\pgfsetdash{}{0pt}%
\pgfpathmoveto{\pgfqpoint{5.097012in}{6.473605in}}%
\pgfpathcurveto{\pgfqpoint{5.108062in}{6.473605in}}{\pgfqpoint{5.118661in}{6.477996in}}{\pgfqpoint{5.126475in}{6.485809in}}%
\pgfpathcurveto{\pgfqpoint{5.134288in}{6.493623in}}{\pgfqpoint{5.138678in}{6.504222in}}{\pgfqpoint{5.138678in}{6.515272in}}%
\pgfpathcurveto{\pgfqpoint{5.138678in}{6.526322in}}{\pgfqpoint{5.134288in}{6.536921in}}{\pgfqpoint{5.126475in}{6.544735in}}%
\pgfpathcurveto{\pgfqpoint{5.118661in}{6.552548in}}{\pgfqpoint{5.108062in}{6.556939in}}{\pgfqpoint{5.097012in}{6.556939in}}%
\pgfpathcurveto{\pgfqpoint{5.085962in}{6.556939in}}{\pgfqpoint{5.075363in}{6.552548in}}{\pgfqpoint{5.067549in}{6.544735in}}%
\pgfpathcurveto{\pgfqpoint{5.059735in}{6.536921in}}{\pgfqpoint{5.055345in}{6.526322in}}{\pgfqpoint{5.055345in}{6.515272in}}%
\pgfpathcurveto{\pgfqpoint{5.055345in}{6.504222in}}{\pgfqpoint{5.059735in}{6.493623in}}{\pgfqpoint{5.067549in}{6.485809in}}%
\pgfpathcurveto{\pgfqpoint{5.075363in}{6.477996in}}{\pgfqpoint{5.085962in}{6.473605in}}{\pgfqpoint{5.097012in}{6.473605in}}%
\pgfpathclose%
\pgfusepath{stroke,fill}%
\end{pgfscope}%
\begin{pgfscope}%
\pgfpathrectangle{\pgfqpoint{0.570343in}{0.331635in}}{\pgfqpoint{9.300000in}{7.700000in}}%
\pgfusepath{clip}%
\pgfsetbuttcap%
\pgfsetroundjoin%
\definecolor{currentfill}{rgb}{1.000000,0.705882,0.509804}%
\pgfsetfillcolor{currentfill}%
\pgfsetlinewidth{0.481800pt}%
\definecolor{currentstroke}{rgb}{1.000000,1.000000,1.000000}%
\pgfsetstrokecolor{currentstroke}%
\pgfsetdash{}{0pt}%
\pgfpathmoveto{\pgfqpoint{4.135589in}{1.137478in}}%
\pgfpathcurveto{\pgfqpoint{4.146639in}{1.137478in}}{\pgfqpoint{4.157238in}{1.141868in}}{\pgfqpoint{4.165052in}{1.149682in}}%
\pgfpathcurveto{\pgfqpoint{4.172866in}{1.157495in}}{\pgfqpoint{4.177256in}{1.168094in}}{\pgfqpoint{4.177256in}{1.179144in}}%
\pgfpathcurveto{\pgfqpoint{4.177256in}{1.190194in}}{\pgfqpoint{4.172866in}{1.200793in}}{\pgfqpoint{4.165052in}{1.208607in}}%
\pgfpathcurveto{\pgfqpoint{4.157238in}{1.216421in}}{\pgfqpoint{4.146639in}{1.220811in}}{\pgfqpoint{4.135589in}{1.220811in}}%
\pgfpathcurveto{\pgfqpoint{4.124539in}{1.220811in}}{\pgfqpoint{4.113940in}{1.216421in}}{\pgfqpoint{4.106126in}{1.208607in}}%
\pgfpathcurveto{\pgfqpoint{4.098313in}{1.200793in}}{\pgfqpoint{4.093923in}{1.190194in}}{\pgfqpoint{4.093923in}{1.179144in}}%
\pgfpathcurveto{\pgfqpoint{4.093923in}{1.168094in}}{\pgfqpoint{4.098313in}{1.157495in}}{\pgfqpoint{4.106126in}{1.149682in}}%
\pgfpathcurveto{\pgfqpoint{4.113940in}{1.141868in}}{\pgfqpoint{4.124539in}{1.137478in}}{\pgfqpoint{4.135589in}{1.137478in}}%
\pgfpathclose%
\pgfusepath{stroke,fill}%
\end{pgfscope}%
\begin{pgfscope}%
\pgfpathrectangle{\pgfqpoint{0.570343in}{0.331635in}}{\pgfqpoint{9.300000in}{7.700000in}}%
\pgfusepath{clip}%
\pgfsetbuttcap%
\pgfsetroundjoin%
\definecolor{currentfill}{rgb}{1.000000,0.705882,0.509804}%
\pgfsetfillcolor{currentfill}%
\pgfsetlinewidth{0.481800pt}%
\definecolor{currentstroke}{rgb}{1.000000,1.000000,1.000000}%
\pgfsetstrokecolor{currentstroke}%
\pgfsetdash{}{0pt}%
\pgfpathmoveto{\pgfqpoint{6.100627in}{3.634431in}}%
\pgfpathcurveto{\pgfqpoint{6.111677in}{3.634431in}}{\pgfqpoint{6.122276in}{3.638821in}}{\pgfqpoint{6.130090in}{3.646635in}}%
\pgfpathcurveto{\pgfqpoint{6.137903in}{3.654449in}}{\pgfqpoint{6.142293in}{3.665048in}}{\pgfqpoint{6.142293in}{3.676098in}}%
\pgfpathcurveto{\pgfqpoint{6.142293in}{3.687148in}}{\pgfqpoint{6.137903in}{3.697747in}}{\pgfqpoint{6.130090in}{3.705561in}}%
\pgfpathcurveto{\pgfqpoint{6.122276in}{3.713374in}}{\pgfqpoint{6.111677in}{3.717764in}}{\pgfqpoint{6.100627in}{3.717764in}}%
\pgfpathcurveto{\pgfqpoint{6.089577in}{3.717764in}}{\pgfqpoint{6.078978in}{3.713374in}}{\pgfqpoint{6.071164in}{3.705561in}}%
\pgfpathcurveto{\pgfqpoint{6.063350in}{3.697747in}}{\pgfqpoint{6.058960in}{3.687148in}}{\pgfqpoint{6.058960in}{3.676098in}}%
\pgfpathcurveto{\pgfqpoint{6.058960in}{3.665048in}}{\pgfqpoint{6.063350in}{3.654449in}}{\pgfqpoint{6.071164in}{3.646635in}}%
\pgfpathcurveto{\pgfqpoint{6.078978in}{3.638821in}}{\pgfqpoint{6.089577in}{3.634431in}}{\pgfqpoint{6.100627in}{3.634431in}}%
\pgfpathclose%
\pgfusepath{stroke,fill}%
\end{pgfscope}%
\begin{pgfscope}%
\pgfpathrectangle{\pgfqpoint{0.570343in}{0.331635in}}{\pgfqpoint{9.300000in}{7.700000in}}%
\pgfusepath{clip}%
\pgfsetbuttcap%
\pgfsetroundjoin%
\definecolor{currentfill}{rgb}{1.000000,0.705882,0.509804}%
\pgfsetfillcolor{currentfill}%
\pgfsetlinewidth{0.481800pt}%
\definecolor{currentstroke}{rgb}{1.000000,1.000000,1.000000}%
\pgfsetstrokecolor{currentstroke}%
\pgfsetdash{}{0pt}%
\pgfpathmoveto{\pgfqpoint{6.017440in}{5.681375in}}%
\pgfpathcurveto{\pgfqpoint{6.028490in}{5.681375in}}{\pgfqpoint{6.039089in}{5.685765in}}{\pgfqpoint{6.046902in}{5.693578in}}%
\pgfpathcurveto{\pgfqpoint{6.054716in}{5.701392in}}{\pgfqpoint{6.059106in}{5.711991in}}{\pgfqpoint{6.059106in}{5.723041in}}%
\pgfpathcurveto{\pgfqpoint{6.059106in}{5.734091in}}{\pgfqpoint{6.054716in}{5.744690in}}{\pgfqpoint{6.046902in}{5.752504in}}%
\pgfpathcurveto{\pgfqpoint{6.039089in}{5.760318in}}{\pgfqpoint{6.028490in}{5.764708in}}{\pgfqpoint{6.017440in}{5.764708in}}%
\pgfpathcurveto{\pgfqpoint{6.006389in}{5.764708in}}{\pgfqpoint{5.995790in}{5.760318in}}{\pgfqpoint{5.987977in}{5.752504in}}%
\pgfpathcurveto{\pgfqpoint{5.980163in}{5.744690in}}{\pgfqpoint{5.975773in}{5.734091in}}{\pgfqpoint{5.975773in}{5.723041in}}%
\pgfpathcurveto{\pgfqpoint{5.975773in}{5.711991in}}{\pgfqpoint{5.980163in}{5.701392in}}{\pgfqpoint{5.987977in}{5.693578in}}%
\pgfpathcurveto{\pgfqpoint{5.995790in}{5.685765in}}{\pgfqpoint{6.006389in}{5.681375in}}{\pgfqpoint{6.017440in}{5.681375in}}%
\pgfpathclose%
\pgfusepath{stroke,fill}%
\end{pgfscope}%
\begin{pgfscope}%
\pgfpathrectangle{\pgfqpoint{0.570343in}{0.331635in}}{\pgfqpoint{9.300000in}{7.700000in}}%
\pgfusepath{clip}%
\pgfsetbuttcap%
\pgfsetroundjoin%
\definecolor{currentfill}{rgb}{1.000000,0.705882,0.509804}%
\pgfsetfillcolor{currentfill}%
\pgfsetlinewidth{0.481800pt}%
\definecolor{currentstroke}{rgb}{1.000000,1.000000,1.000000}%
\pgfsetstrokecolor{currentstroke}%
\pgfsetdash{}{0pt}%
\pgfpathmoveto{\pgfqpoint{5.188705in}{3.475971in}}%
\pgfpathcurveto{\pgfqpoint{5.199755in}{3.475971in}}{\pgfqpoint{5.210354in}{3.480361in}}{\pgfqpoint{5.218168in}{3.488175in}}%
\pgfpathcurveto{\pgfqpoint{5.225982in}{3.495988in}}{\pgfqpoint{5.230372in}{3.506587in}}{\pgfqpoint{5.230372in}{3.517638in}}%
\pgfpathcurveto{\pgfqpoint{5.230372in}{3.528688in}}{\pgfqpoint{5.225982in}{3.539287in}}{\pgfqpoint{5.218168in}{3.547100in}}%
\pgfpathcurveto{\pgfqpoint{5.210354in}{3.554914in}}{\pgfqpoint{5.199755in}{3.559304in}}{\pgfqpoint{5.188705in}{3.559304in}}%
\pgfpathcurveto{\pgfqpoint{5.177655in}{3.559304in}}{\pgfqpoint{5.167056in}{3.554914in}}{\pgfqpoint{5.159242in}{3.547100in}}%
\pgfpathcurveto{\pgfqpoint{5.151429in}{3.539287in}}{\pgfqpoint{5.147039in}{3.528688in}}{\pgfqpoint{5.147039in}{3.517638in}}%
\pgfpathcurveto{\pgfqpoint{5.147039in}{3.506587in}}{\pgfqpoint{5.151429in}{3.495988in}}{\pgfqpoint{5.159242in}{3.488175in}}%
\pgfpathcurveto{\pgfqpoint{5.167056in}{3.480361in}}{\pgfqpoint{5.177655in}{3.475971in}}{\pgfqpoint{5.188705in}{3.475971in}}%
\pgfpathclose%
\pgfusepath{stroke,fill}%
\end{pgfscope}%
\begin{pgfscope}%
\pgfpathrectangle{\pgfqpoint{0.570343in}{0.331635in}}{\pgfqpoint{9.300000in}{7.700000in}}%
\pgfusepath{clip}%
\pgfsetbuttcap%
\pgfsetroundjoin%
\definecolor{currentfill}{rgb}{0.631373,0.788235,0.956863}%
\pgfsetfillcolor{currentfill}%
\pgfsetlinewidth{1.003750pt}%
\definecolor{currentstroke}{rgb}{0.631373,0.788235,0.956863}%
\pgfsetstrokecolor{currentstroke}%
\pgfsetdash{}{0pt}%
\pgfsys@defobject{currentmarker}{\pgfqpoint{-0.041667in}{-0.041667in}}{\pgfqpoint{0.041667in}{0.041667in}}{%
\pgfpathmoveto{\pgfqpoint{0.000000in}{-0.041667in}}%
\pgfpathcurveto{\pgfqpoint{0.011050in}{-0.041667in}}{\pgfqpoint{0.021649in}{-0.037276in}}{\pgfqpoint{0.029463in}{-0.029463in}}%
\pgfpathcurveto{\pgfqpoint{0.037276in}{-0.021649in}}{\pgfqpoint{0.041667in}{-0.011050in}}{\pgfqpoint{0.041667in}{0.000000in}}%
\pgfpathcurveto{\pgfqpoint{0.041667in}{0.011050in}}{\pgfqpoint{0.037276in}{0.021649in}}{\pgfqpoint{0.029463in}{0.029463in}}%
\pgfpathcurveto{\pgfqpoint{0.021649in}{0.037276in}}{\pgfqpoint{0.011050in}{0.041667in}}{\pgfqpoint{0.000000in}{0.041667in}}%
\pgfpathcurveto{\pgfqpoint{-0.011050in}{0.041667in}}{\pgfqpoint{-0.021649in}{0.037276in}}{\pgfqpoint{-0.029463in}{0.029463in}}%
\pgfpathcurveto{\pgfqpoint{-0.037276in}{0.021649in}}{\pgfqpoint{-0.041667in}{0.011050in}}{\pgfqpoint{-0.041667in}{0.000000in}}%
\pgfpathcurveto{\pgfqpoint{-0.041667in}{-0.011050in}}{\pgfqpoint{-0.037276in}{-0.021649in}}{\pgfqpoint{-0.029463in}{-0.029463in}}%
\pgfpathcurveto{\pgfqpoint{-0.021649in}{-0.037276in}}{\pgfqpoint{-0.011050in}{-0.041667in}}{\pgfqpoint{0.000000in}{-0.041667in}}%
\pgfpathclose%
\pgfusepath{stroke,fill}%
}%
\end{pgfscope}%
\begin{pgfscope}%
\pgfpathrectangle{\pgfqpoint{0.570343in}{0.331635in}}{\pgfqpoint{9.300000in}{7.700000in}}%
\pgfusepath{clip}%
\pgfsetbuttcap%
\pgfsetroundjoin%
\definecolor{currentfill}{rgb}{1.000000,0.705882,0.509804}%
\pgfsetfillcolor{currentfill}%
\pgfsetlinewidth{1.003750pt}%
\definecolor{currentstroke}{rgb}{1.000000,0.705882,0.509804}%
\pgfsetstrokecolor{currentstroke}%
\pgfsetdash{}{0pt}%
\pgfsys@defobject{currentmarker}{\pgfqpoint{-0.041667in}{-0.041667in}}{\pgfqpoint{0.041667in}{0.041667in}}{%
\pgfpathmoveto{\pgfqpoint{0.000000in}{-0.041667in}}%
\pgfpathcurveto{\pgfqpoint{0.011050in}{-0.041667in}}{\pgfqpoint{0.021649in}{-0.037276in}}{\pgfqpoint{0.029463in}{-0.029463in}}%
\pgfpathcurveto{\pgfqpoint{0.037276in}{-0.021649in}}{\pgfqpoint{0.041667in}{-0.011050in}}{\pgfqpoint{0.041667in}{0.000000in}}%
\pgfpathcurveto{\pgfqpoint{0.041667in}{0.011050in}}{\pgfqpoint{0.037276in}{0.021649in}}{\pgfqpoint{0.029463in}{0.029463in}}%
\pgfpathcurveto{\pgfqpoint{0.021649in}{0.037276in}}{\pgfqpoint{0.011050in}{0.041667in}}{\pgfqpoint{0.000000in}{0.041667in}}%
\pgfpathcurveto{\pgfqpoint{-0.011050in}{0.041667in}}{\pgfqpoint{-0.021649in}{0.037276in}}{\pgfqpoint{-0.029463in}{0.029463in}}%
\pgfpathcurveto{\pgfqpoint{-0.037276in}{0.021649in}}{\pgfqpoint{-0.041667in}{0.011050in}}{\pgfqpoint{-0.041667in}{0.000000in}}%
\pgfpathcurveto{\pgfqpoint{-0.041667in}{-0.011050in}}{\pgfqpoint{-0.037276in}{-0.021649in}}{\pgfqpoint{-0.029463in}{-0.029463in}}%
\pgfpathcurveto{\pgfqpoint{-0.021649in}{-0.037276in}}{\pgfqpoint{-0.011050in}{-0.041667in}}{\pgfqpoint{0.000000in}{-0.041667in}}%
\pgfpathclose%
\pgfusepath{stroke,fill}%
}%
\end{pgfscope}%
\begin{pgfscope}%
\pgfsetbuttcap%
\pgfsetroundjoin%
\definecolor{currentfill}{rgb}{0.000000,0.000000,0.000000}%
\pgfsetfillcolor{currentfill}%
\pgfsetlinewidth{0.803000pt}%
\definecolor{currentstroke}{rgb}{0.000000,0.000000,0.000000}%
\pgfsetstrokecolor{currentstroke}%
\pgfsetdash{}{0pt}%
\pgfsys@defobject{currentmarker}{\pgfqpoint{0.000000in}{-0.048611in}}{\pgfqpoint{0.000000in}{0.000000in}}{%
\pgfpathmoveto{\pgfqpoint{0.000000in}{0.000000in}}%
\pgfpathlineto{\pgfqpoint{0.000000in}{-0.048611in}}%
\pgfusepath{stroke,fill}%
}%
\begin{pgfscope}%
\pgfsys@transformshift{1.465902in}{0.331635in}%
\pgfsys@useobject{currentmarker}{}%
\end{pgfscope}%
\end{pgfscope}%
\begin{pgfscope}%
\definecolor{textcolor}{rgb}{0.000000,0.000000,0.000000}%
\pgfsetstrokecolor{textcolor}%
\pgfsetfillcolor{textcolor}%
\pgftext[x=1.465902in,y=0.234413in,,top]{\color{textcolor}\sffamily\fontsize{10.000000}{12.000000}\selectfont \ensuremath{-}75}%
\end{pgfscope}%
\begin{pgfscope}%
\pgfsetbuttcap%
\pgfsetroundjoin%
\definecolor{currentfill}{rgb}{0.000000,0.000000,0.000000}%
\pgfsetfillcolor{currentfill}%
\pgfsetlinewidth{0.803000pt}%
\definecolor{currentstroke}{rgb}{0.000000,0.000000,0.000000}%
\pgfsetstrokecolor{currentstroke}%
\pgfsetdash{}{0pt}%
\pgfsys@defobject{currentmarker}{\pgfqpoint{0.000000in}{-0.048611in}}{\pgfqpoint{0.000000in}{0.000000in}}{%
\pgfpathmoveto{\pgfqpoint{0.000000in}{0.000000in}}%
\pgfpathlineto{\pgfqpoint{0.000000in}{-0.048611in}}%
\pgfusepath{stroke,fill}%
}%
\begin{pgfscope}%
\pgfsys@transformshift{2.743766in}{0.331635in}%
\pgfsys@useobject{currentmarker}{}%
\end{pgfscope}%
\end{pgfscope}%
\begin{pgfscope}%
\definecolor{textcolor}{rgb}{0.000000,0.000000,0.000000}%
\pgfsetstrokecolor{textcolor}%
\pgfsetfillcolor{textcolor}%
\pgftext[x=2.743766in,y=0.234413in,,top]{\color{textcolor}\sffamily\fontsize{10.000000}{12.000000}\selectfont \ensuremath{-}50}%
\end{pgfscope}%
\begin{pgfscope}%
\pgfsetbuttcap%
\pgfsetroundjoin%
\definecolor{currentfill}{rgb}{0.000000,0.000000,0.000000}%
\pgfsetfillcolor{currentfill}%
\pgfsetlinewidth{0.803000pt}%
\definecolor{currentstroke}{rgb}{0.000000,0.000000,0.000000}%
\pgfsetstrokecolor{currentstroke}%
\pgfsetdash{}{0pt}%
\pgfsys@defobject{currentmarker}{\pgfqpoint{0.000000in}{-0.048611in}}{\pgfqpoint{0.000000in}{0.000000in}}{%
\pgfpathmoveto{\pgfqpoint{0.000000in}{0.000000in}}%
\pgfpathlineto{\pgfqpoint{0.000000in}{-0.048611in}}%
\pgfusepath{stroke,fill}%
}%
\begin{pgfscope}%
\pgfsys@transformshift{4.021631in}{0.331635in}%
\pgfsys@useobject{currentmarker}{}%
\end{pgfscope}%
\end{pgfscope}%
\begin{pgfscope}%
\definecolor{textcolor}{rgb}{0.000000,0.000000,0.000000}%
\pgfsetstrokecolor{textcolor}%
\pgfsetfillcolor{textcolor}%
\pgftext[x=4.021631in,y=0.234413in,,top]{\color{textcolor}\sffamily\fontsize{10.000000}{12.000000}\selectfont \ensuremath{-}25}%
\end{pgfscope}%
\begin{pgfscope}%
\pgfsetbuttcap%
\pgfsetroundjoin%
\definecolor{currentfill}{rgb}{0.000000,0.000000,0.000000}%
\pgfsetfillcolor{currentfill}%
\pgfsetlinewidth{0.803000pt}%
\definecolor{currentstroke}{rgb}{0.000000,0.000000,0.000000}%
\pgfsetstrokecolor{currentstroke}%
\pgfsetdash{}{0pt}%
\pgfsys@defobject{currentmarker}{\pgfqpoint{0.000000in}{-0.048611in}}{\pgfqpoint{0.000000in}{0.000000in}}{%
\pgfpathmoveto{\pgfqpoint{0.000000in}{0.000000in}}%
\pgfpathlineto{\pgfqpoint{0.000000in}{-0.048611in}}%
\pgfusepath{stroke,fill}%
}%
\begin{pgfscope}%
\pgfsys@transformshift{5.299495in}{0.331635in}%
\pgfsys@useobject{currentmarker}{}%
\end{pgfscope}%
\end{pgfscope}%
\begin{pgfscope}%
\definecolor{textcolor}{rgb}{0.000000,0.000000,0.000000}%
\pgfsetstrokecolor{textcolor}%
\pgfsetfillcolor{textcolor}%
\pgftext[x=5.299495in,y=0.234413in,,top]{\color{textcolor}\sffamily\fontsize{10.000000}{12.000000}\selectfont 0}%
\end{pgfscope}%
\begin{pgfscope}%
\pgfsetbuttcap%
\pgfsetroundjoin%
\definecolor{currentfill}{rgb}{0.000000,0.000000,0.000000}%
\pgfsetfillcolor{currentfill}%
\pgfsetlinewidth{0.803000pt}%
\definecolor{currentstroke}{rgb}{0.000000,0.000000,0.000000}%
\pgfsetstrokecolor{currentstroke}%
\pgfsetdash{}{0pt}%
\pgfsys@defobject{currentmarker}{\pgfqpoint{0.000000in}{-0.048611in}}{\pgfqpoint{0.000000in}{0.000000in}}{%
\pgfpathmoveto{\pgfqpoint{0.000000in}{0.000000in}}%
\pgfpathlineto{\pgfqpoint{0.000000in}{-0.048611in}}%
\pgfusepath{stroke,fill}%
}%
\begin{pgfscope}%
\pgfsys@transformshift{6.577359in}{0.331635in}%
\pgfsys@useobject{currentmarker}{}%
\end{pgfscope}%
\end{pgfscope}%
\begin{pgfscope}%
\definecolor{textcolor}{rgb}{0.000000,0.000000,0.000000}%
\pgfsetstrokecolor{textcolor}%
\pgfsetfillcolor{textcolor}%
\pgftext[x=6.577359in,y=0.234413in,,top]{\color{textcolor}\sffamily\fontsize{10.000000}{12.000000}\selectfont 25}%
\end{pgfscope}%
\begin{pgfscope}%
\pgfsetbuttcap%
\pgfsetroundjoin%
\definecolor{currentfill}{rgb}{0.000000,0.000000,0.000000}%
\pgfsetfillcolor{currentfill}%
\pgfsetlinewidth{0.803000pt}%
\definecolor{currentstroke}{rgb}{0.000000,0.000000,0.000000}%
\pgfsetstrokecolor{currentstroke}%
\pgfsetdash{}{0pt}%
\pgfsys@defobject{currentmarker}{\pgfqpoint{0.000000in}{-0.048611in}}{\pgfqpoint{0.000000in}{0.000000in}}{%
\pgfpathmoveto{\pgfqpoint{0.000000in}{0.000000in}}%
\pgfpathlineto{\pgfqpoint{0.000000in}{-0.048611in}}%
\pgfusepath{stroke,fill}%
}%
\begin{pgfscope}%
\pgfsys@transformshift{7.855223in}{0.331635in}%
\pgfsys@useobject{currentmarker}{}%
\end{pgfscope}%
\end{pgfscope}%
\begin{pgfscope}%
\definecolor{textcolor}{rgb}{0.000000,0.000000,0.000000}%
\pgfsetstrokecolor{textcolor}%
\pgfsetfillcolor{textcolor}%
\pgftext[x=7.855223in,y=0.234413in,,top]{\color{textcolor}\sffamily\fontsize{10.000000}{12.000000}\selectfont 50}%
\end{pgfscope}%
\begin{pgfscope}%
\pgfsetbuttcap%
\pgfsetroundjoin%
\definecolor{currentfill}{rgb}{0.000000,0.000000,0.000000}%
\pgfsetfillcolor{currentfill}%
\pgfsetlinewidth{0.803000pt}%
\definecolor{currentstroke}{rgb}{0.000000,0.000000,0.000000}%
\pgfsetstrokecolor{currentstroke}%
\pgfsetdash{}{0pt}%
\pgfsys@defobject{currentmarker}{\pgfqpoint{0.000000in}{-0.048611in}}{\pgfqpoint{0.000000in}{0.000000in}}{%
\pgfpathmoveto{\pgfqpoint{0.000000in}{0.000000in}}%
\pgfpathlineto{\pgfqpoint{0.000000in}{-0.048611in}}%
\pgfusepath{stroke,fill}%
}%
\begin{pgfscope}%
\pgfsys@transformshift{9.133088in}{0.331635in}%
\pgfsys@useobject{currentmarker}{}%
\end{pgfscope}%
\end{pgfscope}%
\begin{pgfscope}%
\definecolor{textcolor}{rgb}{0.000000,0.000000,0.000000}%
\pgfsetstrokecolor{textcolor}%
\pgfsetfillcolor{textcolor}%
\pgftext[x=9.133088in,y=0.234413in,,top]{\color{textcolor}\sffamily\fontsize{10.000000}{12.000000}\selectfont 75}%
\end{pgfscope}%
\begin{pgfscope}%
\pgfsetbuttcap%
\pgfsetroundjoin%
\definecolor{currentfill}{rgb}{0.000000,0.000000,0.000000}%
\pgfsetfillcolor{currentfill}%
\pgfsetlinewidth{0.803000pt}%
\definecolor{currentstroke}{rgb}{0.000000,0.000000,0.000000}%
\pgfsetstrokecolor{currentstroke}%
\pgfsetdash{}{0pt}%
\pgfsys@defobject{currentmarker}{\pgfqpoint{-0.048611in}{0.000000in}}{\pgfqpoint{-0.000000in}{0.000000in}}{%
\pgfpathmoveto{\pgfqpoint{-0.000000in}{0.000000in}}%
\pgfpathlineto{\pgfqpoint{-0.048611in}{0.000000in}}%
\pgfusepath{stroke,fill}%
}%
\begin{pgfscope}%
\pgfsys@transformshift{0.570343in}{0.520572in}%
\pgfsys@useobject{currentmarker}{}%
\end{pgfscope}%
\end{pgfscope}%
\begin{pgfscope}%
\definecolor{textcolor}{rgb}{0.000000,0.000000,0.000000}%
\pgfsetstrokecolor{textcolor}%
\pgfsetfillcolor{textcolor}%
\pgftext[x=0.100000in, y=0.467810in, left, base]{\color{textcolor}\sffamily\fontsize{10.000000}{12.000000}\selectfont \ensuremath{-}100}%
\end{pgfscope}%
\begin{pgfscope}%
\pgfsetbuttcap%
\pgfsetroundjoin%
\definecolor{currentfill}{rgb}{0.000000,0.000000,0.000000}%
\pgfsetfillcolor{currentfill}%
\pgfsetlinewidth{0.803000pt}%
\definecolor{currentstroke}{rgb}{0.000000,0.000000,0.000000}%
\pgfsetstrokecolor{currentstroke}%
\pgfsetdash{}{0pt}%
\pgfsys@defobject{currentmarker}{\pgfqpoint{-0.048611in}{0.000000in}}{\pgfqpoint{-0.000000in}{0.000000in}}{%
\pgfpathmoveto{\pgfqpoint{-0.000000in}{0.000000in}}%
\pgfpathlineto{\pgfqpoint{-0.048611in}{0.000000in}}%
\pgfusepath{stroke,fill}%
}%
\begin{pgfscope}%
\pgfsys@transformshift{0.570343in}{1.396766in}%
\pgfsys@useobject{currentmarker}{}%
\end{pgfscope}%
\end{pgfscope}%
\begin{pgfscope}%
\definecolor{textcolor}{rgb}{0.000000,0.000000,0.000000}%
\pgfsetstrokecolor{textcolor}%
\pgfsetfillcolor{textcolor}%
\pgftext[x=0.188365in, y=1.344005in, left, base]{\color{textcolor}\sffamily\fontsize{10.000000}{12.000000}\selectfont \ensuremath{-}75}%
\end{pgfscope}%
\begin{pgfscope}%
\pgfsetbuttcap%
\pgfsetroundjoin%
\definecolor{currentfill}{rgb}{0.000000,0.000000,0.000000}%
\pgfsetfillcolor{currentfill}%
\pgfsetlinewidth{0.803000pt}%
\definecolor{currentstroke}{rgb}{0.000000,0.000000,0.000000}%
\pgfsetstrokecolor{currentstroke}%
\pgfsetdash{}{0pt}%
\pgfsys@defobject{currentmarker}{\pgfqpoint{-0.048611in}{0.000000in}}{\pgfqpoint{-0.000000in}{0.000000in}}{%
\pgfpathmoveto{\pgfqpoint{-0.000000in}{0.000000in}}%
\pgfpathlineto{\pgfqpoint{-0.048611in}{0.000000in}}%
\pgfusepath{stroke,fill}%
}%
\begin{pgfscope}%
\pgfsys@transformshift{0.570343in}{2.272961in}%
\pgfsys@useobject{currentmarker}{}%
\end{pgfscope}%
\end{pgfscope}%
\begin{pgfscope}%
\definecolor{textcolor}{rgb}{0.000000,0.000000,0.000000}%
\pgfsetstrokecolor{textcolor}%
\pgfsetfillcolor{textcolor}%
\pgftext[x=0.188365in, y=2.220200in, left, base]{\color{textcolor}\sffamily\fontsize{10.000000}{12.000000}\selectfont \ensuremath{-}50}%
\end{pgfscope}%
\begin{pgfscope}%
\pgfsetbuttcap%
\pgfsetroundjoin%
\definecolor{currentfill}{rgb}{0.000000,0.000000,0.000000}%
\pgfsetfillcolor{currentfill}%
\pgfsetlinewidth{0.803000pt}%
\definecolor{currentstroke}{rgb}{0.000000,0.000000,0.000000}%
\pgfsetstrokecolor{currentstroke}%
\pgfsetdash{}{0pt}%
\pgfsys@defobject{currentmarker}{\pgfqpoint{-0.048611in}{0.000000in}}{\pgfqpoint{-0.000000in}{0.000000in}}{%
\pgfpathmoveto{\pgfqpoint{-0.000000in}{0.000000in}}%
\pgfpathlineto{\pgfqpoint{-0.048611in}{0.000000in}}%
\pgfusepath{stroke,fill}%
}%
\begin{pgfscope}%
\pgfsys@transformshift{0.570343in}{3.149156in}%
\pgfsys@useobject{currentmarker}{}%
\end{pgfscope}%
\end{pgfscope}%
\begin{pgfscope}%
\definecolor{textcolor}{rgb}{0.000000,0.000000,0.000000}%
\pgfsetstrokecolor{textcolor}%
\pgfsetfillcolor{textcolor}%
\pgftext[x=0.188365in, y=3.096395in, left, base]{\color{textcolor}\sffamily\fontsize{10.000000}{12.000000}\selectfont \ensuremath{-}25}%
\end{pgfscope}%
\begin{pgfscope}%
\pgfsetbuttcap%
\pgfsetroundjoin%
\definecolor{currentfill}{rgb}{0.000000,0.000000,0.000000}%
\pgfsetfillcolor{currentfill}%
\pgfsetlinewidth{0.803000pt}%
\definecolor{currentstroke}{rgb}{0.000000,0.000000,0.000000}%
\pgfsetstrokecolor{currentstroke}%
\pgfsetdash{}{0pt}%
\pgfsys@defobject{currentmarker}{\pgfqpoint{-0.048611in}{0.000000in}}{\pgfqpoint{-0.000000in}{0.000000in}}{%
\pgfpathmoveto{\pgfqpoint{-0.000000in}{0.000000in}}%
\pgfpathlineto{\pgfqpoint{-0.048611in}{0.000000in}}%
\pgfusepath{stroke,fill}%
}%
\begin{pgfscope}%
\pgfsys@transformshift{0.570343in}{4.025351in}%
\pgfsys@useobject{currentmarker}{}%
\end{pgfscope}%
\end{pgfscope}%
\begin{pgfscope}%
\definecolor{textcolor}{rgb}{0.000000,0.000000,0.000000}%
\pgfsetstrokecolor{textcolor}%
\pgfsetfillcolor{textcolor}%
\pgftext[x=0.384756in, y=3.972589in, left, base]{\color{textcolor}\sffamily\fontsize{10.000000}{12.000000}\selectfont 0}%
\end{pgfscope}%
\begin{pgfscope}%
\pgfsetbuttcap%
\pgfsetroundjoin%
\definecolor{currentfill}{rgb}{0.000000,0.000000,0.000000}%
\pgfsetfillcolor{currentfill}%
\pgfsetlinewidth{0.803000pt}%
\definecolor{currentstroke}{rgb}{0.000000,0.000000,0.000000}%
\pgfsetstrokecolor{currentstroke}%
\pgfsetdash{}{0pt}%
\pgfsys@defobject{currentmarker}{\pgfqpoint{-0.048611in}{0.000000in}}{\pgfqpoint{-0.000000in}{0.000000in}}{%
\pgfpathmoveto{\pgfqpoint{-0.000000in}{0.000000in}}%
\pgfpathlineto{\pgfqpoint{-0.048611in}{0.000000in}}%
\pgfusepath{stroke,fill}%
}%
\begin{pgfscope}%
\pgfsys@transformshift{0.570343in}{4.901546in}%
\pgfsys@useobject{currentmarker}{}%
\end{pgfscope}%
\end{pgfscope}%
\begin{pgfscope}%
\definecolor{textcolor}{rgb}{0.000000,0.000000,0.000000}%
\pgfsetstrokecolor{textcolor}%
\pgfsetfillcolor{textcolor}%
\pgftext[x=0.296390in, y=4.848784in, left, base]{\color{textcolor}\sffamily\fontsize{10.000000}{12.000000}\selectfont 25}%
\end{pgfscope}%
\begin{pgfscope}%
\pgfsetbuttcap%
\pgfsetroundjoin%
\definecolor{currentfill}{rgb}{0.000000,0.000000,0.000000}%
\pgfsetfillcolor{currentfill}%
\pgfsetlinewidth{0.803000pt}%
\definecolor{currentstroke}{rgb}{0.000000,0.000000,0.000000}%
\pgfsetstrokecolor{currentstroke}%
\pgfsetdash{}{0pt}%
\pgfsys@defobject{currentmarker}{\pgfqpoint{-0.048611in}{0.000000in}}{\pgfqpoint{-0.000000in}{0.000000in}}{%
\pgfpathmoveto{\pgfqpoint{-0.000000in}{0.000000in}}%
\pgfpathlineto{\pgfqpoint{-0.048611in}{0.000000in}}%
\pgfusepath{stroke,fill}%
}%
\begin{pgfscope}%
\pgfsys@transformshift{0.570343in}{5.777741in}%
\pgfsys@useobject{currentmarker}{}%
\end{pgfscope}%
\end{pgfscope}%
\begin{pgfscope}%
\definecolor{textcolor}{rgb}{0.000000,0.000000,0.000000}%
\pgfsetstrokecolor{textcolor}%
\pgfsetfillcolor{textcolor}%
\pgftext[x=0.296390in, y=5.724979in, left, base]{\color{textcolor}\sffamily\fontsize{10.000000}{12.000000}\selectfont 50}%
\end{pgfscope}%
\begin{pgfscope}%
\pgfsetbuttcap%
\pgfsetroundjoin%
\definecolor{currentfill}{rgb}{0.000000,0.000000,0.000000}%
\pgfsetfillcolor{currentfill}%
\pgfsetlinewidth{0.803000pt}%
\definecolor{currentstroke}{rgb}{0.000000,0.000000,0.000000}%
\pgfsetstrokecolor{currentstroke}%
\pgfsetdash{}{0pt}%
\pgfsys@defobject{currentmarker}{\pgfqpoint{-0.048611in}{0.000000in}}{\pgfqpoint{-0.000000in}{0.000000in}}{%
\pgfpathmoveto{\pgfqpoint{-0.000000in}{0.000000in}}%
\pgfpathlineto{\pgfqpoint{-0.048611in}{0.000000in}}%
\pgfusepath{stroke,fill}%
}%
\begin{pgfscope}%
\pgfsys@transformshift{0.570343in}{6.653936in}%
\pgfsys@useobject{currentmarker}{}%
\end{pgfscope}%
\end{pgfscope}%
\begin{pgfscope}%
\definecolor{textcolor}{rgb}{0.000000,0.000000,0.000000}%
\pgfsetstrokecolor{textcolor}%
\pgfsetfillcolor{textcolor}%
\pgftext[x=0.296390in, y=6.601174in, left, base]{\color{textcolor}\sffamily\fontsize{10.000000}{12.000000}\selectfont 75}%
\end{pgfscope}%
\begin{pgfscope}%
\pgfsetbuttcap%
\pgfsetroundjoin%
\definecolor{currentfill}{rgb}{0.000000,0.000000,0.000000}%
\pgfsetfillcolor{currentfill}%
\pgfsetlinewidth{0.803000pt}%
\definecolor{currentstroke}{rgb}{0.000000,0.000000,0.000000}%
\pgfsetstrokecolor{currentstroke}%
\pgfsetdash{}{0pt}%
\pgfsys@defobject{currentmarker}{\pgfqpoint{-0.048611in}{0.000000in}}{\pgfqpoint{-0.000000in}{0.000000in}}{%
\pgfpathmoveto{\pgfqpoint{-0.000000in}{0.000000in}}%
\pgfpathlineto{\pgfqpoint{-0.048611in}{0.000000in}}%
\pgfusepath{stroke,fill}%
}%
\begin{pgfscope}%
\pgfsys@transformshift{0.570343in}{7.530130in}%
\pgfsys@useobject{currentmarker}{}%
\end{pgfscope}%
\end{pgfscope}%
\begin{pgfscope}%
\definecolor{textcolor}{rgb}{0.000000,0.000000,0.000000}%
\pgfsetstrokecolor{textcolor}%
\pgfsetfillcolor{textcolor}%
\pgftext[x=0.208025in, y=7.477369in, left, base]{\color{textcolor}\sffamily\fontsize{10.000000}{12.000000}\selectfont 100}%
\end{pgfscope}%
\begin{pgfscope}%
\pgfpathrectangle{\pgfqpoint{0.570343in}{0.331635in}}{\pgfqpoint{9.300000in}{7.700000in}}%
\pgfusepath{clip}%
\pgfsetrectcap%
\pgfsetroundjoin%
\pgfsetlinewidth{1.505625pt}%
\definecolor{currentstroke}{rgb}{0.631373,0.788235,0.956863}%
\pgfsetstrokecolor{currentstroke}%
\pgfsetstrokeopacity{0.800000}%
\pgfsetdash{}{0pt}%
\pgfpathmoveto{\pgfqpoint{2.191177in}{2.309828in}}%
\pgfpathlineto{\pgfqpoint{5.093443in}{4.458686in}}%
\pgfusepath{stroke}%
\end{pgfscope}%
\begin{pgfscope}%
\pgfpathrectangle{\pgfqpoint{0.570343in}{0.331635in}}{\pgfqpoint{9.300000in}{7.700000in}}%
\pgfusepath{clip}%
\pgfsetrectcap%
\pgfsetroundjoin%
\pgfsetlinewidth{1.505625pt}%
\definecolor{currentstroke}{rgb}{0.631373,0.788235,0.956863}%
\pgfsetstrokecolor{currentstroke}%
\pgfsetstrokeopacity{0.800000}%
\pgfsetdash{}{0pt}%
\pgfpathmoveto{\pgfqpoint{8.068189in}{5.160328in}}%
\pgfpathlineto{\pgfqpoint{5.093443in}{4.458686in}}%
\pgfusepath{stroke}%
\end{pgfscope}%
\begin{pgfscope}%
\pgfpathrectangle{\pgfqpoint{0.570343in}{0.331635in}}{\pgfqpoint{9.300000in}{7.700000in}}%
\pgfusepath{clip}%
\pgfsetrectcap%
\pgfsetroundjoin%
\pgfsetlinewidth{1.505625pt}%
\definecolor{currentstroke}{rgb}{0.631373,0.788235,0.956863}%
\pgfsetstrokecolor{currentstroke}%
\pgfsetstrokeopacity{0.800000}%
\pgfsetdash{}{0pt}%
\pgfpathmoveto{\pgfqpoint{4.775457in}{3.760082in}}%
\pgfpathlineto{\pgfqpoint{5.093443in}{4.458686in}}%
\pgfusepath{stroke}%
\end{pgfscope}%
\begin{pgfscope}%
\pgfpathrectangle{\pgfqpoint{0.570343in}{0.331635in}}{\pgfqpoint{9.300000in}{7.700000in}}%
\pgfusepath{clip}%
\pgfsetrectcap%
\pgfsetroundjoin%
\pgfsetlinewidth{1.505625pt}%
\definecolor{currentstroke}{rgb}{0.631373,0.788235,0.956863}%
\pgfsetstrokecolor{currentstroke}%
\pgfsetstrokeopacity{0.800000}%
\pgfsetdash{}{0pt}%
\pgfpathmoveto{\pgfqpoint{4.043501in}{4.384365in}}%
\pgfpathlineto{\pgfqpoint{5.093443in}{4.458686in}}%
\pgfusepath{stroke}%
\end{pgfscope}%
\begin{pgfscope}%
\pgfpathrectangle{\pgfqpoint{0.570343in}{0.331635in}}{\pgfqpoint{9.300000in}{7.700000in}}%
\pgfusepath{clip}%
\pgfsetrectcap%
\pgfsetroundjoin%
\pgfsetlinewidth{1.505625pt}%
\definecolor{currentstroke}{rgb}{0.631373,0.788235,0.956863}%
\pgfsetstrokecolor{currentstroke}%
\pgfsetstrokeopacity{0.800000}%
\pgfsetdash{}{0pt}%
\pgfpathmoveto{\pgfqpoint{2.250247in}{2.828832in}}%
\pgfpathlineto{\pgfqpoint{5.093443in}{4.458686in}}%
\pgfusepath{stroke}%
\end{pgfscope}%
\begin{pgfscope}%
\pgfpathrectangle{\pgfqpoint{0.570343in}{0.331635in}}{\pgfqpoint{9.300000in}{7.700000in}}%
\pgfusepath{clip}%
\pgfsetrectcap%
\pgfsetroundjoin%
\pgfsetlinewidth{1.505625pt}%
\definecolor{currentstroke}{rgb}{0.631373,0.788235,0.956863}%
\pgfsetstrokecolor{currentstroke}%
\pgfsetstrokeopacity{0.800000}%
\pgfsetdash{}{0pt}%
\pgfpathmoveto{\pgfqpoint{8.033060in}{5.604709in}}%
\pgfpathlineto{\pgfqpoint{5.093443in}{4.458686in}}%
\pgfusepath{stroke}%
\end{pgfscope}%
\begin{pgfscope}%
\pgfpathrectangle{\pgfqpoint{0.570343in}{0.331635in}}{\pgfqpoint{9.300000in}{7.700000in}}%
\pgfusepath{clip}%
\pgfsetrectcap%
\pgfsetroundjoin%
\pgfsetlinewidth{1.505625pt}%
\definecolor{currentstroke}{rgb}{0.631373,0.788235,0.956863}%
\pgfsetstrokecolor{currentstroke}%
\pgfsetstrokeopacity{0.800000}%
\pgfsetdash{}{0pt}%
\pgfpathmoveto{\pgfqpoint{2.963896in}{2.293505in}}%
\pgfpathlineto{\pgfqpoint{5.093443in}{4.458686in}}%
\pgfusepath{stroke}%
\end{pgfscope}%
\begin{pgfscope}%
\pgfpathrectangle{\pgfqpoint{0.570343in}{0.331635in}}{\pgfqpoint{9.300000in}{7.700000in}}%
\pgfusepath{clip}%
\pgfsetrectcap%
\pgfsetroundjoin%
\pgfsetlinewidth{1.505625pt}%
\definecolor{currentstroke}{rgb}{0.631373,0.788235,0.956863}%
\pgfsetstrokecolor{currentstroke}%
\pgfsetstrokeopacity{0.800000}%
\pgfsetdash{}{0pt}%
\pgfpathmoveto{\pgfqpoint{4.257872in}{3.951073in}}%
\pgfpathlineto{\pgfqpoint{5.093443in}{4.458686in}}%
\pgfusepath{stroke}%
\end{pgfscope}%
\begin{pgfscope}%
\pgfpathrectangle{\pgfqpoint{0.570343in}{0.331635in}}{\pgfqpoint{9.300000in}{7.700000in}}%
\pgfusepath{clip}%
\pgfsetrectcap%
\pgfsetroundjoin%
\pgfsetlinewidth{1.505625pt}%
\definecolor{currentstroke}{rgb}{0.631373,0.788235,0.956863}%
\pgfsetstrokecolor{currentstroke}%
\pgfsetstrokeopacity{0.800000}%
\pgfsetdash{}{0pt}%
\pgfpathmoveto{\pgfqpoint{9.312478in}{5.030378in}}%
\pgfpathlineto{\pgfqpoint{5.093443in}{4.458686in}}%
\pgfusepath{stroke}%
\end{pgfscope}%
\begin{pgfscope}%
\pgfpathrectangle{\pgfqpoint{0.570343in}{0.331635in}}{\pgfqpoint{9.300000in}{7.700000in}}%
\pgfusepath{clip}%
\pgfsetrectcap%
\pgfsetroundjoin%
\pgfsetlinewidth{1.505625pt}%
\definecolor{currentstroke}{rgb}{0.631373,0.788235,0.956863}%
\pgfsetstrokecolor{currentstroke}%
\pgfsetstrokeopacity{0.800000}%
\pgfsetdash{}{0pt}%
\pgfpathmoveto{\pgfqpoint{8.119150in}{6.417976in}}%
\pgfpathlineto{\pgfqpoint{5.093443in}{4.458686in}}%
\pgfusepath{stroke}%
\end{pgfscope}%
\begin{pgfscope}%
\pgfpathrectangle{\pgfqpoint{0.570343in}{0.331635in}}{\pgfqpoint{9.300000in}{7.700000in}}%
\pgfusepath{clip}%
\pgfsetrectcap%
\pgfsetroundjoin%
\pgfsetlinewidth{1.505625pt}%
\definecolor{currentstroke}{rgb}{0.631373,0.788235,0.956863}%
\pgfsetstrokecolor{currentstroke}%
\pgfsetstrokeopacity{0.800000}%
\pgfsetdash{}{0pt}%
\pgfpathmoveto{\pgfqpoint{6.778721in}{6.034661in}}%
\pgfpathlineto{\pgfqpoint{5.093443in}{4.458686in}}%
\pgfusepath{stroke}%
\end{pgfscope}%
\begin{pgfscope}%
\pgfpathrectangle{\pgfqpoint{0.570343in}{0.331635in}}{\pgfqpoint{9.300000in}{7.700000in}}%
\pgfusepath{clip}%
\pgfsetrectcap%
\pgfsetroundjoin%
\pgfsetlinewidth{1.505625pt}%
\definecolor{currentstroke}{rgb}{0.631373,0.788235,0.956863}%
\pgfsetstrokecolor{currentstroke}%
\pgfsetstrokeopacity{0.800000}%
\pgfsetdash{}{0pt}%
\pgfpathmoveto{\pgfqpoint{7.814530in}{6.061956in}}%
\pgfpathlineto{\pgfqpoint{5.093443in}{4.458686in}}%
\pgfusepath{stroke}%
\end{pgfscope}%
\begin{pgfscope}%
\pgfpathrectangle{\pgfqpoint{0.570343in}{0.331635in}}{\pgfqpoint{9.300000in}{7.700000in}}%
\pgfusepath{clip}%
\pgfsetrectcap%
\pgfsetroundjoin%
\pgfsetlinewidth{1.505625pt}%
\definecolor{currentstroke}{rgb}{0.631373,0.788235,0.956863}%
\pgfsetstrokecolor{currentstroke}%
\pgfsetstrokeopacity{0.800000}%
\pgfsetdash{}{0pt}%
\pgfpathmoveto{\pgfqpoint{6.210576in}{7.681635in}}%
\pgfpathlineto{\pgfqpoint{5.093443in}{4.458686in}}%
\pgfusepath{stroke}%
\end{pgfscope}%
\begin{pgfscope}%
\pgfpathrectangle{\pgfqpoint{0.570343in}{0.331635in}}{\pgfqpoint{9.300000in}{7.700000in}}%
\pgfusepath{clip}%
\pgfsetrectcap%
\pgfsetroundjoin%
\pgfsetlinewidth{1.505625pt}%
\definecolor{currentstroke}{rgb}{0.631373,0.788235,0.956863}%
\pgfsetstrokecolor{currentstroke}%
\pgfsetstrokeopacity{0.800000}%
\pgfsetdash{}{0pt}%
\pgfpathmoveto{\pgfqpoint{8.639749in}{5.560936in}}%
\pgfpathlineto{\pgfqpoint{5.093443in}{4.458686in}}%
\pgfusepath{stroke}%
\end{pgfscope}%
\begin{pgfscope}%
\pgfpathrectangle{\pgfqpoint{0.570343in}{0.331635in}}{\pgfqpoint{9.300000in}{7.700000in}}%
\pgfusepath{clip}%
\pgfsetrectcap%
\pgfsetroundjoin%
\pgfsetlinewidth{1.505625pt}%
\definecolor{currentstroke}{rgb}{0.631373,0.788235,0.956863}%
\pgfsetstrokecolor{currentstroke}%
\pgfsetstrokeopacity{0.800000}%
\pgfsetdash{}{0pt}%
\pgfpathmoveto{\pgfqpoint{2.502764in}{5.525424in}}%
\pgfpathlineto{\pgfqpoint{5.093443in}{4.458686in}}%
\pgfusepath{stroke}%
\end{pgfscope}%
\begin{pgfscope}%
\pgfpathrectangle{\pgfqpoint{0.570343in}{0.331635in}}{\pgfqpoint{9.300000in}{7.700000in}}%
\pgfusepath{clip}%
\pgfsetrectcap%
\pgfsetroundjoin%
\pgfsetlinewidth{1.505625pt}%
\definecolor{currentstroke}{rgb}{0.631373,0.788235,0.956863}%
\pgfsetstrokecolor{currentstroke}%
\pgfsetstrokeopacity{0.800000}%
\pgfsetdash{}{0pt}%
\pgfpathmoveto{\pgfqpoint{7.737428in}{2.312068in}}%
\pgfpathlineto{\pgfqpoint{5.093443in}{4.458686in}}%
\pgfusepath{stroke}%
\end{pgfscope}%
\begin{pgfscope}%
\pgfpathrectangle{\pgfqpoint{0.570343in}{0.331635in}}{\pgfqpoint{9.300000in}{7.700000in}}%
\pgfusepath{clip}%
\pgfsetrectcap%
\pgfsetroundjoin%
\pgfsetlinewidth{1.505625pt}%
\definecolor{currentstroke}{rgb}{0.631373,0.788235,0.956863}%
\pgfsetstrokecolor{currentstroke}%
\pgfsetstrokeopacity{0.800000}%
\pgfsetdash{}{0pt}%
\pgfpathmoveto{\pgfqpoint{8.606468in}{6.039351in}}%
\pgfpathlineto{\pgfqpoint{5.093443in}{4.458686in}}%
\pgfusepath{stroke}%
\end{pgfscope}%
\begin{pgfscope}%
\pgfpathrectangle{\pgfqpoint{0.570343in}{0.331635in}}{\pgfqpoint{9.300000in}{7.700000in}}%
\pgfusepath{clip}%
\pgfsetrectcap%
\pgfsetroundjoin%
\pgfsetlinewidth{1.505625pt}%
\definecolor{currentstroke}{rgb}{0.631373,0.788235,0.956863}%
\pgfsetstrokecolor{currentstroke}%
\pgfsetstrokeopacity{0.800000}%
\pgfsetdash{}{0pt}%
\pgfpathmoveto{\pgfqpoint{9.447616in}{3.581115in}}%
\pgfpathlineto{\pgfqpoint{5.093443in}{4.458686in}}%
\pgfusepath{stroke}%
\end{pgfscope}%
\begin{pgfscope}%
\pgfpathrectangle{\pgfqpoint{0.570343in}{0.331635in}}{\pgfqpoint{9.300000in}{7.700000in}}%
\pgfusepath{clip}%
\pgfsetrectcap%
\pgfsetroundjoin%
\pgfsetlinewidth{1.505625pt}%
\definecolor{currentstroke}{rgb}{0.631373,0.788235,0.956863}%
\pgfsetstrokecolor{currentstroke}%
\pgfsetstrokeopacity{0.800000}%
\pgfsetdash{}{0pt}%
\pgfpathmoveto{\pgfqpoint{1.771086in}{2.878070in}}%
\pgfpathlineto{\pgfqpoint{5.093443in}{4.458686in}}%
\pgfusepath{stroke}%
\end{pgfscope}%
\begin{pgfscope}%
\pgfpathrectangle{\pgfqpoint{0.570343in}{0.331635in}}{\pgfqpoint{9.300000in}{7.700000in}}%
\pgfusepath{clip}%
\pgfsetrectcap%
\pgfsetroundjoin%
\pgfsetlinewidth{1.505625pt}%
\definecolor{currentstroke}{rgb}{0.631373,0.788235,0.956863}%
\pgfsetstrokecolor{currentstroke}%
\pgfsetstrokeopacity{0.800000}%
\pgfsetdash{}{0pt}%
\pgfpathmoveto{\pgfqpoint{4.161271in}{6.131309in}}%
\pgfpathlineto{\pgfqpoint{5.093443in}{4.458686in}}%
\pgfusepath{stroke}%
\end{pgfscope}%
\begin{pgfscope}%
\pgfpathrectangle{\pgfqpoint{0.570343in}{0.331635in}}{\pgfqpoint{9.300000in}{7.700000in}}%
\pgfusepath{clip}%
\pgfsetrectcap%
\pgfsetroundjoin%
\pgfsetlinewidth{1.505625pt}%
\definecolor{currentstroke}{rgb}{0.631373,0.788235,0.956863}%
\pgfsetstrokecolor{currentstroke}%
\pgfsetstrokeopacity{0.800000}%
\pgfsetdash{}{0pt}%
\pgfpathmoveto{\pgfqpoint{4.477751in}{4.863986in}}%
\pgfpathlineto{\pgfqpoint{5.093443in}{4.458686in}}%
\pgfusepath{stroke}%
\end{pgfscope}%
\begin{pgfscope}%
\pgfpathrectangle{\pgfqpoint{0.570343in}{0.331635in}}{\pgfqpoint{9.300000in}{7.700000in}}%
\pgfusepath{clip}%
\pgfsetrectcap%
\pgfsetroundjoin%
\pgfsetlinewidth{1.505625pt}%
\definecolor{currentstroke}{rgb}{0.631373,0.788235,0.956863}%
\pgfsetstrokecolor{currentstroke}%
\pgfsetstrokeopacity{0.800000}%
\pgfsetdash{}{0pt}%
\pgfpathmoveto{\pgfqpoint{6.827660in}{3.291755in}}%
\pgfpathlineto{\pgfqpoint{5.093443in}{4.458686in}}%
\pgfusepath{stroke}%
\end{pgfscope}%
\begin{pgfscope}%
\pgfpathrectangle{\pgfqpoint{0.570343in}{0.331635in}}{\pgfqpoint{9.300000in}{7.700000in}}%
\pgfusepath{clip}%
\pgfsetrectcap%
\pgfsetroundjoin%
\pgfsetlinewidth{1.505625pt}%
\definecolor{currentstroke}{rgb}{0.631373,0.788235,0.956863}%
\pgfsetstrokecolor{currentstroke}%
\pgfsetstrokeopacity{0.800000}%
\pgfsetdash{}{0pt}%
\pgfpathmoveto{\pgfqpoint{6.876963in}{5.214184in}}%
\pgfpathlineto{\pgfqpoint{5.093443in}{4.458686in}}%
\pgfusepath{stroke}%
\end{pgfscope}%
\begin{pgfscope}%
\pgfpathrectangle{\pgfqpoint{0.570343in}{0.331635in}}{\pgfqpoint{9.300000in}{7.700000in}}%
\pgfusepath{clip}%
\pgfsetrectcap%
\pgfsetroundjoin%
\pgfsetlinewidth{1.505625pt}%
\definecolor{currentstroke}{rgb}{0.631373,0.788235,0.956863}%
\pgfsetstrokecolor{currentstroke}%
\pgfsetstrokeopacity{0.800000}%
\pgfsetdash{}{0pt}%
\pgfpathmoveto{\pgfqpoint{5.974644in}{6.842337in}}%
\pgfpathlineto{\pgfqpoint{5.093443in}{4.458686in}}%
\pgfusepath{stroke}%
\end{pgfscope}%
\begin{pgfscope}%
\pgfpathrectangle{\pgfqpoint{0.570343in}{0.331635in}}{\pgfqpoint{9.300000in}{7.700000in}}%
\pgfusepath{clip}%
\pgfsetrectcap%
\pgfsetroundjoin%
\pgfsetlinewidth{1.505625pt}%
\definecolor{currentstroke}{rgb}{0.631373,0.788235,0.956863}%
\pgfsetstrokecolor{currentstroke}%
\pgfsetstrokeopacity{0.800000}%
\pgfsetdash{}{0pt}%
\pgfpathmoveto{\pgfqpoint{0.993071in}{3.120734in}}%
\pgfpathlineto{\pgfqpoint{5.093443in}{4.458686in}}%
\pgfusepath{stroke}%
\end{pgfscope}%
\begin{pgfscope}%
\pgfpathrectangle{\pgfqpoint{0.570343in}{0.331635in}}{\pgfqpoint{9.300000in}{7.700000in}}%
\pgfusepath{clip}%
\pgfsetrectcap%
\pgfsetroundjoin%
\pgfsetlinewidth{1.505625pt}%
\definecolor{currentstroke}{rgb}{0.631373,0.788235,0.956863}%
\pgfsetstrokecolor{currentstroke}%
\pgfsetstrokeopacity{0.800000}%
\pgfsetdash{}{0pt}%
\pgfpathmoveto{\pgfqpoint{1.720010in}{2.159301in}}%
\pgfpathlineto{\pgfqpoint{5.093443in}{4.458686in}}%
\pgfusepath{stroke}%
\end{pgfscope}%
\begin{pgfscope}%
\pgfpathrectangle{\pgfqpoint{0.570343in}{0.331635in}}{\pgfqpoint{9.300000in}{7.700000in}}%
\pgfusepath{clip}%
\pgfsetrectcap%
\pgfsetroundjoin%
\pgfsetlinewidth{1.505625pt}%
\definecolor{currentstroke}{rgb}{0.631373,0.788235,0.956863}%
\pgfsetstrokecolor{currentstroke}%
\pgfsetstrokeopacity{0.800000}%
\pgfsetdash{}{0pt}%
\pgfpathmoveto{\pgfqpoint{3.244521in}{5.068402in}}%
\pgfpathlineto{\pgfqpoint{5.093443in}{4.458686in}}%
\pgfusepath{stroke}%
\end{pgfscope}%
\begin{pgfscope}%
\pgfpathrectangle{\pgfqpoint{0.570343in}{0.331635in}}{\pgfqpoint{9.300000in}{7.700000in}}%
\pgfusepath{clip}%
\pgfsetrectcap%
\pgfsetroundjoin%
\pgfsetlinewidth{1.505625pt}%
\definecolor{currentstroke}{rgb}{0.631373,0.788235,0.956863}%
\pgfsetstrokecolor{currentstroke}%
\pgfsetstrokeopacity{0.800000}%
\pgfsetdash{}{0pt}%
\pgfpathmoveto{\pgfqpoint{1.313094in}{3.379378in}}%
\pgfpathlineto{\pgfqpoint{5.093443in}{4.458686in}}%
\pgfusepath{stroke}%
\end{pgfscope}%
\begin{pgfscope}%
\pgfpathrectangle{\pgfqpoint{0.570343in}{0.331635in}}{\pgfqpoint{9.300000in}{7.700000in}}%
\pgfusepath{clip}%
\pgfsetrectcap%
\pgfsetroundjoin%
\pgfsetlinewidth{1.505625pt}%
\definecolor{currentstroke}{rgb}{0.631373,0.788235,0.956863}%
\pgfsetstrokecolor{currentstroke}%
\pgfsetstrokeopacity{0.800000}%
\pgfsetdash{}{0pt}%
\pgfpathmoveto{\pgfqpoint{2.996308in}{4.498033in}}%
\pgfpathlineto{\pgfqpoint{5.093443in}{4.458686in}}%
\pgfusepath{stroke}%
\end{pgfscope}%
\begin{pgfscope}%
\pgfpathrectangle{\pgfqpoint{0.570343in}{0.331635in}}{\pgfqpoint{9.300000in}{7.700000in}}%
\pgfusepath{clip}%
\pgfsetrectcap%
\pgfsetroundjoin%
\pgfsetlinewidth{1.505625pt}%
\definecolor{currentstroke}{rgb}{0.631373,0.788235,0.956863}%
\pgfsetstrokecolor{currentstroke}%
\pgfsetstrokeopacity{0.800000}%
\pgfsetdash{}{0pt}%
\pgfpathmoveto{\pgfqpoint{8.746558in}{4.791538in}}%
\pgfpathlineto{\pgfqpoint{5.093443in}{4.458686in}}%
\pgfusepath{stroke}%
\end{pgfscope}%
\begin{pgfscope}%
\pgfpathrectangle{\pgfqpoint{0.570343in}{0.331635in}}{\pgfqpoint{9.300000in}{7.700000in}}%
\pgfusepath{clip}%
\pgfsetrectcap%
\pgfsetroundjoin%
\pgfsetlinewidth{1.505625pt}%
\definecolor{currentstroke}{rgb}{0.631373,0.788235,0.956863}%
\pgfsetstrokecolor{currentstroke}%
\pgfsetstrokeopacity{0.800000}%
\pgfsetdash{}{0pt}%
\pgfpathmoveto{\pgfqpoint{7.450593in}{5.373260in}}%
\pgfpathlineto{\pgfqpoint{5.093443in}{4.458686in}}%
\pgfusepath{stroke}%
\end{pgfscope}%
\begin{pgfscope}%
\pgfpathrectangle{\pgfqpoint{0.570343in}{0.331635in}}{\pgfqpoint{9.300000in}{7.700000in}}%
\pgfusepath{clip}%
\pgfsetrectcap%
\pgfsetroundjoin%
\pgfsetlinewidth{1.505625pt}%
\definecolor{currentstroke}{rgb}{0.631373,0.788235,0.956863}%
\pgfsetstrokecolor{currentstroke}%
\pgfsetstrokeopacity{0.800000}%
\pgfsetdash{}{0pt}%
\pgfpathmoveto{\pgfqpoint{1.930672in}{1.834691in}}%
\pgfpathlineto{\pgfqpoint{5.093443in}{4.458686in}}%
\pgfusepath{stroke}%
\end{pgfscope}%
\begin{pgfscope}%
\pgfpathrectangle{\pgfqpoint{0.570343in}{0.331635in}}{\pgfqpoint{9.300000in}{7.700000in}}%
\pgfusepath{clip}%
\pgfsetrectcap%
\pgfsetroundjoin%
\pgfsetlinewidth{1.505625pt}%
\definecolor{currentstroke}{rgb}{0.631373,0.788235,0.956863}%
\pgfsetstrokecolor{currentstroke}%
\pgfsetstrokeopacity{0.800000}%
\pgfsetdash{}{0pt}%
\pgfpathmoveto{\pgfqpoint{7.537340in}{6.350716in}}%
\pgfpathlineto{\pgfqpoint{5.093443in}{4.458686in}}%
\pgfusepath{stroke}%
\end{pgfscope}%
\begin{pgfscope}%
\pgfpathrectangle{\pgfqpoint{0.570343in}{0.331635in}}{\pgfqpoint{9.300000in}{7.700000in}}%
\pgfusepath{clip}%
\pgfsetrectcap%
\pgfsetroundjoin%
\pgfsetlinewidth{1.505625pt}%
\definecolor{currentstroke}{rgb}{0.631373,0.788235,0.956863}%
\pgfsetstrokecolor{currentstroke}%
\pgfsetstrokeopacity{0.800000}%
\pgfsetdash{}{0pt}%
\pgfpathmoveto{\pgfqpoint{4.430255in}{2.687942in}}%
\pgfpathlineto{\pgfqpoint{5.093443in}{4.458686in}}%
\pgfusepath{stroke}%
\end{pgfscope}%
\begin{pgfscope}%
\pgfpathrectangle{\pgfqpoint{0.570343in}{0.331635in}}{\pgfqpoint{9.300000in}{7.700000in}}%
\pgfusepath{clip}%
\pgfsetrectcap%
\pgfsetroundjoin%
\pgfsetlinewidth{1.505625pt}%
\definecolor{currentstroke}{rgb}{0.631373,0.788235,0.956863}%
\pgfsetstrokecolor{currentstroke}%
\pgfsetstrokeopacity{0.800000}%
\pgfsetdash{}{0pt}%
\pgfpathmoveto{\pgfqpoint{6.449722in}{7.426710in}}%
\pgfpathlineto{\pgfqpoint{5.093443in}{4.458686in}}%
\pgfusepath{stroke}%
\end{pgfscope}%
\begin{pgfscope}%
\pgfpathrectangle{\pgfqpoint{0.570343in}{0.331635in}}{\pgfqpoint{9.300000in}{7.700000in}}%
\pgfusepath{clip}%
\pgfsetrectcap%
\pgfsetroundjoin%
\pgfsetlinewidth{1.505625pt}%
\definecolor{currentstroke}{rgb}{0.631373,0.788235,0.956863}%
\pgfsetstrokecolor{currentstroke}%
\pgfsetstrokeopacity{0.800000}%
\pgfsetdash{}{0pt}%
\pgfpathmoveto{\pgfqpoint{6.510955in}{2.748055in}}%
\pgfpathlineto{\pgfqpoint{5.093443in}{4.458686in}}%
\pgfusepath{stroke}%
\end{pgfscope}%
\begin{pgfscope}%
\pgfpathrectangle{\pgfqpoint{0.570343in}{0.331635in}}{\pgfqpoint{9.300000in}{7.700000in}}%
\pgfusepath{clip}%
\pgfsetrectcap%
\pgfsetroundjoin%
\pgfsetlinewidth{1.505625pt}%
\definecolor{currentstroke}{rgb}{0.631373,0.788235,0.956863}%
\pgfsetstrokecolor{currentstroke}%
\pgfsetstrokeopacity{0.800000}%
\pgfsetdash{}{0pt}%
\pgfpathmoveto{\pgfqpoint{1.349034in}{2.519455in}}%
\pgfpathlineto{\pgfqpoint{5.093443in}{4.458686in}}%
\pgfusepath{stroke}%
\end{pgfscope}%
\begin{pgfscope}%
\pgfpathrectangle{\pgfqpoint{0.570343in}{0.331635in}}{\pgfqpoint{9.300000in}{7.700000in}}%
\pgfusepath{clip}%
\pgfsetrectcap%
\pgfsetroundjoin%
\pgfsetlinewidth{1.505625pt}%
\definecolor{currentstroke}{rgb}{0.631373,0.788235,0.956863}%
\pgfsetstrokecolor{currentstroke}%
\pgfsetstrokeopacity{0.800000}%
\pgfsetdash{}{0pt}%
\pgfpathmoveto{\pgfqpoint{6.043529in}{4.599859in}}%
\pgfpathlineto{\pgfqpoint{5.093443in}{4.458686in}}%
\pgfusepath{stroke}%
\end{pgfscope}%
\begin{pgfscope}%
\pgfpathrectangle{\pgfqpoint{0.570343in}{0.331635in}}{\pgfqpoint{9.300000in}{7.700000in}}%
\pgfusepath{clip}%
\pgfsetrectcap%
\pgfsetroundjoin%
\pgfsetlinewidth{1.505625pt}%
\definecolor{currentstroke}{rgb}{0.631373,0.788235,0.956863}%
\pgfsetstrokecolor{currentstroke}%
\pgfsetstrokeopacity{0.800000}%
\pgfsetdash{}{0pt}%
\pgfpathmoveto{\pgfqpoint{3.476646in}{5.853872in}}%
\pgfpathlineto{\pgfqpoint{5.093443in}{4.458686in}}%
\pgfusepath{stroke}%
\end{pgfscope}%
\begin{pgfscope}%
\pgfpathrectangle{\pgfqpoint{0.570343in}{0.331635in}}{\pgfqpoint{9.300000in}{7.700000in}}%
\pgfusepath{clip}%
\pgfsetrectcap%
\pgfsetroundjoin%
\pgfsetlinewidth{1.505625pt}%
\definecolor{currentstroke}{rgb}{0.631373,0.788235,0.956863}%
\pgfsetstrokecolor{currentstroke}%
\pgfsetstrokeopacity{0.800000}%
\pgfsetdash{}{0pt}%
\pgfpathmoveto{\pgfqpoint{7.982207in}{3.352040in}}%
\pgfpathlineto{\pgfqpoint{5.093443in}{4.458686in}}%
\pgfusepath{stroke}%
\end{pgfscope}%
\begin{pgfscope}%
\pgfpathrectangle{\pgfqpoint{0.570343in}{0.331635in}}{\pgfqpoint{9.300000in}{7.700000in}}%
\pgfusepath{clip}%
\pgfsetrectcap%
\pgfsetroundjoin%
\pgfsetlinewidth{1.505625pt}%
\definecolor{currentstroke}{rgb}{0.631373,0.788235,0.956863}%
\pgfsetstrokecolor{currentstroke}%
\pgfsetstrokeopacity{0.800000}%
\pgfsetdash{}{0pt}%
\pgfpathmoveto{\pgfqpoint{2.053144in}{5.566795in}}%
\pgfpathlineto{\pgfqpoint{5.093443in}{4.458686in}}%
\pgfusepath{stroke}%
\end{pgfscope}%
\begin{pgfscope}%
\pgfpathrectangle{\pgfqpoint{0.570343in}{0.331635in}}{\pgfqpoint{9.300000in}{7.700000in}}%
\pgfusepath{clip}%
\pgfsetrectcap%
\pgfsetroundjoin%
\pgfsetlinewidth{1.505625pt}%
\definecolor{currentstroke}{rgb}{0.631373,0.788235,0.956863}%
\pgfsetstrokecolor{currentstroke}%
\pgfsetstrokeopacity{0.800000}%
\pgfsetdash{}{0pt}%
\pgfpathmoveto{\pgfqpoint{2.355215in}{3.804910in}}%
\pgfpathlineto{\pgfqpoint{5.093443in}{4.458686in}}%
\pgfusepath{stroke}%
\end{pgfscope}%
\begin{pgfscope}%
\pgfpathrectangle{\pgfqpoint{0.570343in}{0.331635in}}{\pgfqpoint{9.300000in}{7.700000in}}%
\pgfusepath{clip}%
\pgfsetrectcap%
\pgfsetroundjoin%
\pgfsetlinewidth{1.505625pt}%
\definecolor{currentstroke}{rgb}{0.631373,0.788235,0.956863}%
\pgfsetstrokecolor{currentstroke}%
\pgfsetstrokeopacity{0.800000}%
\pgfsetdash{}{0pt}%
\pgfpathmoveto{\pgfqpoint{5.553256in}{2.895716in}}%
\pgfpathlineto{\pgfqpoint{5.093443in}{4.458686in}}%
\pgfusepath{stroke}%
\end{pgfscope}%
\begin{pgfscope}%
\pgfpathrectangle{\pgfqpoint{0.570343in}{0.331635in}}{\pgfqpoint{9.300000in}{7.700000in}}%
\pgfusepath{clip}%
\pgfsetrectcap%
\pgfsetroundjoin%
\pgfsetlinewidth{1.505625pt}%
\definecolor{currentstroke}{rgb}{0.631373,0.788235,0.956863}%
\pgfsetstrokecolor{currentstroke}%
\pgfsetstrokeopacity{0.800000}%
\pgfsetdash{}{0pt}%
\pgfpathmoveto{\pgfqpoint{1.164286in}{3.896980in}}%
\pgfpathlineto{\pgfqpoint{5.093443in}{4.458686in}}%
\pgfusepath{stroke}%
\end{pgfscope}%
\begin{pgfscope}%
\pgfpathrectangle{\pgfqpoint{0.570343in}{0.331635in}}{\pgfqpoint{9.300000in}{7.700000in}}%
\pgfusepath{clip}%
\pgfsetrectcap%
\pgfsetroundjoin%
\pgfsetlinewidth{1.505625pt}%
\definecolor{currentstroke}{rgb}{0.631373,0.788235,0.956863}%
\pgfsetstrokecolor{currentstroke}%
\pgfsetstrokeopacity{0.800000}%
\pgfsetdash{}{0pt}%
\pgfpathmoveto{\pgfqpoint{2.482835in}{6.320289in}}%
\pgfpathlineto{\pgfqpoint{5.093443in}{4.458686in}}%
\pgfusepath{stroke}%
\end{pgfscope}%
\begin{pgfscope}%
\pgfpathrectangle{\pgfqpoint{0.570343in}{0.331635in}}{\pgfqpoint{9.300000in}{7.700000in}}%
\pgfusepath{clip}%
\pgfsetrectcap%
\pgfsetroundjoin%
\pgfsetlinewidth{1.505625pt}%
\definecolor{currentstroke}{rgb}{0.631373,0.788235,0.956863}%
\pgfsetstrokecolor{currentstroke}%
\pgfsetstrokeopacity{0.800000}%
\pgfsetdash{}{0pt}%
\pgfpathmoveto{\pgfqpoint{3.630943in}{4.223184in}}%
\pgfpathlineto{\pgfqpoint{5.093443in}{4.458686in}}%
\pgfusepath{stroke}%
\end{pgfscope}%
\begin{pgfscope}%
\pgfpathrectangle{\pgfqpoint{0.570343in}{0.331635in}}{\pgfqpoint{9.300000in}{7.700000in}}%
\pgfusepath{clip}%
\pgfsetrectcap%
\pgfsetroundjoin%
\pgfsetlinewidth{1.505625pt}%
\definecolor{currentstroke}{rgb}{0.631373,0.788235,0.956863}%
\pgfsetstrokecolor{currentstroke}%
\pgfsetstrokeopacity{0.800000}%
\pgfsetdash{}{0pt}%
\pgfpathmoveto{\pgfqpoint{4.241907in}{3.161297in}}%
\pgfpathlineto{\pgfqpoint{5.093443in}{4.458686in}}%
\pgfusepath{stroke}%
\end{pgfscope}%
\begin{pgfscope}%
\pgfpathrectangle{\pgfqpoint{0.570343in}{0.331635in}}{\pgfqpoint{9.300000in}{7.700000in}}%
\pgfusepath{clip}%
\pgfsetrectcap%
\pgfsetroundjoin%
\pgfsetlinewidth{1.505625pt}%
\definecolor{currentstroke}{rgb}{0.631373,0.788235,0.956863}%
\pgfsetstrokecolor{currentstroke}%
\pgfsetstrokeopacity{0.800000}%
\pgfsetdash{}{0pt}%
\pgfpathmoveto{\pgfqpoint{6.802935in}{5.721911in}}%
\pgfpathlineto{\pgfqpoint{5.093443in}{4.458686in}}%
\pgfusepath{stroke}%
\end{pgfscope}%
\begin{pgfscope}%
\pgfpathrectangle{\pgfqpoint{0.570343in}{0.331635in}}{\pgfqpoint{9.300000in}{7.700000in}}%
\pgfusepath{clip}%
\pgfsetrectcap%
\pgfsetroundjoin%
\pgfsetlinewidth{1.505625pt}%
\definecolor{currentstroke}{rgb}{0.631373,0.788235,0.956863}%
\pgfsetstrokecolor{currentstroke}%
\pgfsetstrokeopacity{0.800000}%
\pgfsetdash{}{0pt}%
\pgfpathmoveto{\pgfqpoint{3.985351in}{4.828351in}}%
\pgfpathlineto{\pgfqpoint{5.093443in}{4.458686in}}%
\pgfusepath{stroke}%
\end{pgfscope}%
\begin{pgfscope}%
\pgfpathrectangle{\pgfqpoint{0.570343in}{0.331635in}}{\pgfqpoint{9.300000in}{7.700000in}}%
\pgfusepath{clip}%
\pgfsetrectcap%
\pgfsetroundjoin%
\pgfsetlinewidth{1.505625pt}%
\definecolor{currentstroke}{rgb}{0.631373,0.788235,0.956863}%
\pgfsetstrokecolor{currentstroke}%
\pgfsetstrokeopacity{0.800000}%
\pgfsetdash{}{0pt}%
\pgfpathmoveto{\pgfqpoint{8.385485in}{2.991004in}}%
\pgfpathlineto{\pgfqpoint{5.093443in}{4.458686in}}%
\pgfusepath{stroke}%
\end{pgfscope}%
\begin{pgfscope}%
\pgfpathrectangle{\pgfqpoint{0.570343in}{0.331635in}}{\pgfqpoint{9.300000in}{7.700000in}}%
\pgfusepath{clip}%
\pgfsetrectcap%
\pgfsetroundjoin%
\pgfsetlinewidth{1.505625pt}%
\definecolor{currentstroke}{rgb}{1.000000,0.705882,0.509804}%
\pgfsetstrokecolor{currentstroke}%
\pgfsetstrokeopacity{0.800000}%
\pgfsetdash{}{0pt}%
\pgfpathmoveto{\pgfqpoint{5.756401in}{4.305242in}}%
\pgfpathlineto{\pgfqpoint{5.248298in}{4.065422in}}%
\pgfusepath{stroke}%
\end{pgfscope}%
\begin{pgfscope}%
\pgfpathrectangle{\pgfqpoint{0.570343in}{0.331635in}}{\pgfqpoint{9.300000in}{7.700000in}}%
\pgfusepath{clip}%
\pgfsetrectcap%
\pgfsetroundjoin%
\pgfsetlinewidth{1.505625pt}%
\definecolor{currentstroke}{rgb}{1.000000,0.705882,0.509804}%
\pgfsetstrokecolor{currentstroke}%
\pgfsetstrokeopacity{0.800000}%
\pgfsetdash{}{0pt}%
\pgfpathmoveto{\pgfqpoint{6.668584in}{4.380469in}}%
\pgfpathlineto{\pgfqpoint{5.248298in}{4.065422in}}%
\pgfusepath{stroke}%
\end{pgfscope}%
\begin{pgfscope}%
\pgfpathrectangle{\pgfqpoint{0.570343in}{0.331635in}}{\pgfqpoint{9.300000in}{7.700000in}}%
\pgfusepath{clip}%
\pgfsetrectcap%
\pgfsetroundjoin%
\pgfsetlinewidth{1.505625pt}%
\definecolor{currentstroke}{rgb}{1.000000,0.705882,0.509804}%
\pgfsetstrokecolor{currentstroke}%
\pgfsetstrokeopacity{0.800000}%
\pgfsetdash{}{0pt}%
\pgfpathmoveto{\pgfqpoint{3.259347in}{1.622349in}}%
\pgfpathlineto{\pgfqpoint{5.248298in}{4.065422in}}%
\pgfusepath{stroke}%
\end{pgfscope}%
\begin{pgfscope}%
\pgfpathrectangle{\pgfqpoint{0.570343in}{0.331635in}}{\pgfqpoint{9.300000in}{7.700000in}}%
\pgfusepath{clip}%
\pgfsetrectcap%
\pgfsetroundjoin%
\pgfsetlinewidth{1.505625pt}%
\definecolor{currentstroke}{rgb}{1.000000,0.705882,0.509804}%
\pgfsetstrokecolor{currentstroke}%
\pgfsetstrokeopacity{0.800000}%
\pgfsetdash{}{0pt}%
\pgfpathmoveto{\pgfqpoint{3.583317in}{2.020488in}}%
\pgfpathlineto{\pgfqpoint{5.248298in}{4.065422in}}%
\pgfusepath{stroke}%
\end{pgfscope}%
\begin{pgfscope}%
\pgfpathrectangle{\pgfqpoint{0.570343in}{0.331635in}}{\pgfqpoint{9.300000in}{7.700000in}}%
\pgfusepath{clip}%
\pgfsetrectcap%
\pgfsetroundjoin%
\pgfsetlinewidth{1.505625pt}%
\definecolor{currentstroke}{rgb}{1.000000,0.705882,0.509804}%
\pgfsetstrokecolor{currentstroke}%
\pgfsetstrokeopacity{0.800000}%
\pgfsetdash{}{0pt}%
\pgfpathmoveto{\pgfqpoint{6.070729in}{4.098200in}}%
\pgfpathlineto{\pgfqpoint{5.248298in}{4.065422in}}%
\pgfusepath{stroke}%
\end{pgfscope}%
\begin{pgfscope}%
\pgfpathrectangle{\pgfqpoint{0.570343in}{0.331635in}}{\pgfqpoint{9.300000in}{7.700000in}}%
\pgfusepath{clip}%
\pgfsetrectcap%
\pgfsetroundjoin%
\pgfsetlinewidth{1.505625pt}%
\definecolor{currentstroke}{rgb}{1.000000,0.705882,0.509804}%
\pgfsetstrokecolor{currentstroke}%
\pgfsetstrokeopacity{0.800000}%
\pgfsetdash{}{0pt}%
\pgfpathmoveto{\pgfqpoint{5.275125in}{5.711322in}}%
\pgfpathlineto{\pgfqpoint{5.248298in}{4.065422in}}%
\pgfusepath{stroke}%
\end{pgfscope}%
\begin{pgfscope}%
\pgfpathrectangle{\pgfqpoint{0.570343in}{0.331635in}}{\pgfqpoint{9.300000in}{7.700000in}}%
\pgfusepath{clip}%
\pgfsetrectcap%
\pgfsetroundjoin%
\pgfsetlinewidth{1.505625pt}%
\definecolor{currentstroke}{rgb}{1.000000,0.705882,0.509804}%
\pgfsetstrokecolor{currentstroke}%
\pgfsetstrokeopacity{0.800000}%
\pgfsetdash{}{0pt}%
\pgfpathmoveto{\pgfqpoint{6.812363in}{6.576644in}}%
\pgfpathlineto{\pgfqpoint{5.248298in}{4.065422in}}%
\pgfusepath{stroke}%
\end{pgfscope}%
\begin{pgfscope}%
\pgfpathrectangle{\pgfqpoint{0.570343in}{0.331635in}}{\pgfqpoint{9.300000in}{7.700000in}}%
\pgfusepath{clip}%
\pgfsetrectcap%
\pgfsetroundjoin%
\pgfsetlinewidth{1.505625pt}%
\definecolor{currentstroke}{rgb}{1.000000,0.705882,0.509804}%
\pgfsetstrokecolor{currentstroke}%
\pgfsetstrokeopacity{0.800000}%
\pgfsetdash{}{0pt}%
\pgfpathmoveto{\pgfqpoint{8.535292in}{4.110557in}}%
\pgfpathlineto{\pgfqpoint{5.248298in}{4.065422in}}%
\pgfusepath{stroke}%
\end{pgfscope}%
\begin{pgfscope}%
\pgfpathrectangle{\pgfqpoint{0.570343in}{0.331635in}}{\pgfqpoint{9.300000in}{7.700000in}}%
\pgfusepath{clip}%
\pgfsetrectcap%
\pgfsetroundjoin%
\pgfsetlinewidth{1.505625pt}%
\definecolor{currentstroke}{rgb}{1.000000,0.705882,0.509804}%
\pgfsetstrokecolor{currentstroke}%
\pgfsetstrokeopacity{0.800000}%
\pgfsetdash{}{0pt}%
\pgfpathmoveto{\pgfqpoint{5.969237in}{6.196926in}}%
\pgfpathlineto{\pgfqpoint{5.248298in}{4.065422in}}%
\pgfusepath{stroke}%
\end{pgfscope}%
\begin{pgfscope}%
\pgfpathrectangle{\pgfqpoint{0.570343in}{0.331635in}}{\pgfqpoint{9.300000in}{7.700000in}}%
\pgfusepath{clip}%
\pgfsetrectcap%
\pgfsetroundjoin%
\pgfsetlinewidth{1.505625pt}%
\definecolor{currentstroke}{rgb}{1.000000,0.705882,0.509804}%
\pgfsetstrokecolor{currentstroke}%
\pgfsetstrokeopacity{0.800000}%
\pgfsetdash{}{0pt}%
\pgfpathmoveto{\pgfqpoint{3.243518in}{3.520733in}}%
\pgfpathlineto{\pgfqpoint{5.248298in}{4.065422in}}%
\pgfusepath{stroke}%
\end{pgfscope}%
\begin{pgfscope}%
\pgfpathrectangle{\pgfqpoint{0.570343in}{0.331635in}}{\pgfqpoint{9.300000in}{7.700000in}}%
\pgfusepath{clip}%
\pgfsetrectcap%
\pgfsetroundjoin%
\pgfsetlinewidth{1.505625pt}%
\definecolor{currentstroke}{rgb}{1.000000,0.705882,0.509804}%
\pgfsetstrokecolor{currentstroke}%
\pgfsetstrokeopacity{0.800000}%
\pgfsetdash{}{0pt}%
\pgfpathmoveto{\pgfqpoint{3.870316in}{6.873389in}}%
\pgfpathlineto{\pgfqpoint{5.248298in}{4.065422in}}%
\pgfusepath{stroke}%
\end{pgfscope}%
\begin{pgfscope}%
\pgfpathrectangle{\pgfqpoint{0.570343in}{0.331635in}}{\pgfqpoint{9.300000in}{7.700000in}}%
\pgfusepath{clip}%
\pgfsetrectcap%
\pgfsetroundjoin%
\pgfsetlinewidth{1.505625pt}%
\definecolor{currentstroke}{rgb}{1.000000,0.705882,0.509804}%
\pgfsetstrokecolor{currentstroke}%
\pgfsetstrokeopacity{0.800000}%
\pgfsetdash{}{0pt}%
\pgfpathmoveto{\pgfqpoint{4.461357in}{0.747730in}}%
\pgfpathlineto{\pgfqpoint{5.248298in}{4.065422in}}%
\pgfusepath{stroke}%
\end{pgfscope}%
\begin{pgfscope}%
\pgfpathrectangle{\pgfqpoint{0.570343in}{0.331635in}}{\pgfqpoint{9.300000in}{7.700000in}}%
\pgfusepath{clip}%
\pgfsetrectcap%
\pgfsetroundjoin%
\pgfsetlinewidth{1.505625pt}%
\definecolor{currentstroke}{rgb}{1.000000,0.705882,0.509804}%
\pgfsetstrokecolor{currentstroke}%
\pgfsetstrokeopacity{0.800000}%
\pgfsetdash{}{0pt}%
\pgfpathmoveto{\pgfqpoint{2.074350in}{4.445897in}}%
\pgfpathlineto{\pgfqpoint{5.248298in}{4.065422in}}%
\pgfusepath{stroke}%
\end{pgfscope}%
\begin{pgfscope}%
\pgfpathrectangle{\pgfqpoint{0.570343in}{0.331635in}}{\pgfqpoint{9.300000in}{7.700000in}}%
\pgfusepath{clip}%
\pgfsetrectcap%
\pgfsetroundjoin%
\pgfsetlinewidth{1.505625pt}%
\definecolor{currentstroke}{rgb}{1.000000,0.705882,0.509804}%
\pgfsetstrokecolor{currentstroke}%
\pgfsetstrokeopacity{0.800000}%
\pgfsetdash{}{0pt}%
\pgfpathmoveto{\pgfqpoint{7.251452in}{4.937411in}}%
\pgfpathlineto{\pgfqpoint{5.248298in}{4.065422in}}%
\pgfusepath{stroke}%
\end{pgfscope}%
\begin{pgfscope}%
\pgfpathrectangle{\pgfqpoint{0.570343in}{0.331635in}}{\pgfqpoint{9.300000in}{7.700000in}}%
\pgfusepath{clip}%
\pgfsetrectcap%
\pgfsetroundjoin%
\pgfsetlinewidth{1.505625pt}%
\definecolor{currentstroke}{rgb}{1.000000,0.705882,0.509804}%
\pgfsetstrokecolor{currentstroke}%
\pgfsetstrokeopacity{0.800000}%
\pgfsetdash{}{0pt}%
\pgfpathmoveto{\pgfqpoint{5.103453in}{4.175618in}}%
\pgfpathlineto{\pgfqpoint{5.248298in}{4.065422in}}%
\pgfusepath{stroke}%
\end{pgfscope}%
\begin{pgfscope}%
\pgfpathrectangle{\pgfqpoint{0.570343in}{0.331635in}}{\pgfqpoint{9.300000in}{7.700000in}}%
\pgfusepath{clip}%
\pgfsetrectcap%
\pgfsetroundjoin%
\pgfsetlinewidth{1.505625pt}%
\definecolor{currentstroke}{rgb}{1.000000,0.705882,0.509804}%
\pgfsetstrokecolor{currentstroke}%
\pgfsetstrokeopacity{0.800000}%
\pgfsetdash{}{0pt}%
\pgfpathmoveto{\pgfqpoint{7.313999in}{3.841560in}}%
\pgfpathlineto{\pgfqpoint{5.248298in}{4.065422in}}%
\pgfusepath{stroke}%
\end{pgfscope}%
\begin{pgfscope}%
\pgfpathrectangle{\pgfqpoint{0.570343in}{0.331635in}}{\pgfqpoint{9.300000in}{7.700000in}}%
\pgfusepath{clip}%
\pgfsetrectcap%
\pgfsetroundjoin%
\pgfsetlinewidth{1.505625pt}%
\definecolor{currentstroke}{rgb}{1.000000,0.705882,0.509804}%
\pgfsetstrokecolor{currentstroke}%
\pgfsetstrokeopacity{0.800000}%
\pgfsetdash{}{0pt}%
\pgfpathmoveto{\pgfqpoint{7.365854in}{4.189851in}}%
\pgfpathlineto{\pgfqpoint{5.248298in}{4.065422in}}%
\pgfusepath{stroke}%
\end{pgfscope}%
\begin{pgfscope}%
\pgfpathrectangle{\pgfqpoint{0.570343in}{0.331635in}}{\pgfqpoint{9.300000in}{7.700000in}}%
\pgfusepath{clip}%
\pgfsetrectcap%
\pgfsetroundjoin%
\pgfsetlinewidth{1.505625pt}%
\definecolor{currentstroke}{rgb}{1.000000,0.705882,0.509804}%
\pgfsetstrokecolor{currentstroke}%
\pgfsetstrokeopacity{0.800000}%
\pgfsetdash{}{0pt}%
\pgfpathmoveto{\pgfqpoint{4.130821in}{2.084238in}}%
\pgfpathlineto{\pgfqpoint{5.248298in}{4.065422in}}%
\pgfusepath{stroke}%
\end{pgfscope}%
\begin{pgfscope}%
\pgfpathrectangle{\pgfqpoint{0.570343in}{0.331635in}}{\pgfqpoint{9.300000in}{7.700000in}}%
\pgfusepath{clip}%
\pgfsetrectcap%
\pgfsetroundjoin%
\pgfsetlinewidth{1.505625pt}%
\definecolor{currentstroke}{rgb}{1.000000,0.705882,0.509804}%
\pgfsetstrokecolor{currentstroke}%
\pgfsetstrokeopacity{0.800000}%
\pgfsetdash{}{0pt}%
\pgfpathmoveto{\pgfqpoint{3.194504in}{2.804540in}}%
\pgfpathlineto{\pgfqpoint{5.248298in}{4.065422in}}%
\pgfusepath{stroke}%
\end{pgfscope}%
\begin{pgfscope}%
\pgfpathrectangle{\pgfqpoint{0.570343in}{0.331635in}}{\pgfqpoint{9.300000in}{7.700000in}}%
\pgfusepath{clip}%
\pgfsetrectcap%
\pgfsetroundjoin%
\pgfsetlinewidth{1.505625pt}%
\definecolor{currentstroke}{rgb}{1.000000,0.705882,0.509804}%
\pgfsetstrokecolor{currentstroke}%
\pgfsetstrokeopacity{0.800000}%
\pgfsetdash{}{0pt}%
\pgfpathmoveto{\pgfqpoint{6.329979in}{5.408913in}}%
\pgfpathlineto{\pgfqpoint{5.248298in}{4.065422in}}%
\pgfusepath{stroke}%
\end{pgfscope}%
\begin{pgfscope}%
\pgfpathrectangle{\pgfqpoint{0.570343in}{0.331635in}}{\pgfqpoint{9.300000in}{7.700000in}}%
\pgfusepath{clip}%
\pgfsetrectcap%
\pgfsetroundjoin%
\pgfsetlinewidth{1.505625pt}%
\definecolor{currentstroke}{rgb}{1.000000,0.705882,0.509804}%
\pgfsetstrokecolor{currentstroke}%
\pgfsetstrokeopacity{0.800000}%
\pgfsetdash{}{0pt}%
\pgfpathmoveto{\pgfqpoint{6.112839in}{4.974791in}}%
\pgfpathlineto{\pgfqpoint{5.248298in}{4.065422in}}%
\pgfusepath{stroke}%
\end{pgfscope}%
\begin{pgfscope}%
\pgfpathrectangle{\pgfqpoint{0.570343in}{0.331635in}}{\pgfqpoint{9.300000in}{7.700000in}}%
\pgfusepath{clip}%
\pgfsetrectcap%
\pgfsetroundjoin%
\pgfsetlinewidth{1.505625pt}%
\definecolor{currentstroke}{rgb}{1.000000,0.705882,0.509804}%
\pgfsetstrokecolor{currentstroke}%
\pgfsetstrokeopacity{0.800000}%
\pgfsetdash{}{0pt}%
\pgfpathmoveto{\pgfqpoint{6.735587in}{3.981487in}}%
\pgfpathlineto{\pgfqpoint{5.248298in}{4.065422in}}%
\pgfusepath{stroke}%
\end{pgfscope}%
\begin{pgfscope}%
\pgfpathrectangle{\pgfqpoint{0.570343in}{0.331635in}}{\pgfqpoint{9.300000in}{7.700000in}}%
\pgfusepath{clip}%
\pgfsetrectcap%
\pgfsetroundjoin%
\pgfsetlinewidth{1.505625pt}%
\definecolor{currentstroke}{rgb}{1.000000,0.705882,0.509804}%
\pgfsetstrokecolor{currentstroke}%
\pgfsetstrokeopacity{0.800000}%
\pgfsetdash{}{0pt}%
\pgfpathmoveto{\pgfqpoint{3.789899in}{1.620123in}}%
\pgfpathlineto{\pgfqpoint{5.248298in}{4.065422in}}%
\pgfusepath{stroke}%
\end{pgfscope}%
\begin{pgfscope}%
\pgfpathrectangle{\pgfqpoint{0.570343in}{0.331635in}}{\pgfqpoint{9.300000in}{7.700000in}}%
\pgfusepath{clip}%
\pgfsetrectcap%
\pgfsetroundjoin%
\pgfsetlinewidth{1.505625pt}%
\definecolor{currentstroke}{rgb}{1.000000,0.705882,0.509804}%
\pgfsetstrokecolor{currentstroke}%
\pgfsetstrokeopacity{0.800000}%
\pgfsetdash{}{0pt}%
\pgfpathmoveto{\pgfqpoint{6.698828in}{4.795256in}}%
\pgfpathlineto{\pgfqpoint{5.248298in}{4.065422in}}%
\pgfusepath{stroke}%
\end{pgfscope}%
\begin{pgfscope}%
\pgfpathrectangle{\pgfqpoint{0.570343in}{0.331635in}}{\pgfqpoint{9.300000in}{7.700000in}}%
\pgfusepath{clip}%
\pgfsetrectcap%
\pgfsetroundjoin%
\pgfsetlinewidth{1.505625pt}%
\definecolor{currentstroke}{rgb}{1.000000,0.705882,0.509804}%
\pgfsetstrokecolor{currentstroke}%
\pgfsetstrokeopacity{0.800000}%
\pgfsetdash{}{0pt}%
\pgfpathmoveto{\pgfqpoint{3.751769in}{3.625371in}}%
\pgfpathlineto{\pgfqpoint{5.248298in}{4.065422in}}%
\pgfusepath{stroke}%
\end{pgfscope}%
\begin{pgfscope}%
\pgfpathrectangle{\pgfqpoint{0.570343in}{0.331635in}}{\pgfqpoint{9.300000in}{7.700000in}}%
\pgfusepath{clip}%
\pgfsetrectcap%
\pgfsetroundjoin%
\pgfsetlinewidth{1.505625pt}%
\definecolor{currentstroke}{rgb}{1.000000,0.705882,0.509804}%
\pgfsetstrokecolor{currentstroke}%
\pgfsetstrokeopacity{0.800000}%
\pgfsetdash{}{0pt}%
\pgfpathmoveto{\pgfqpoint{7.658417in}{7.101654in}}%
\pgfpathlineto{\pgfqpoint{5.248298in}{4.065422in}}%
\pgfusepath{stroke}%
\end{pgfscope}%
\begin{pgfscope}%
\pgfpathrectangle{\pgfqpoint{0.570343in}{0.331635in}}{\pgfqpoint{9.300000in}{7.700000in}}%
\pgfusepath{clip}%
\pgfsetrectcap%
\pgfsetroundjoin%
\pgfsetlinewidth{1.505625pt}%
\definecolor{currentstroke}{rgb}{1.000000,0.705882,0.509804}%
\pgfsetstrokecolor{currentstroke}%
\pgfsetstrokeopacity{0.800000}%
\pgfsetdash{}{0pt}%
\pgfpathmoveto{\pgfqpoint{4.089483in}{5.414209in}}%
\pgfpathlineto{\pgfqpoint{5.248298in}{4.065422in}}%
\pgfusepath{stroke}%
\end{pgfscope}%
\begin{pgfscope}%
\pgfpathrectangle{\pgfqpoint{0.570343in}{0.331635in}}{\pgfqpoint{9.300000in}{7.700000in}}%
\pgfusepath{clip}%
\pgfsetrectcap%
\pgfsetroundjoin%
\pgfsetlinewidth{1.505625pt}%
\definecolor{currentstroke}{rgb}{1.000000,0.705882,0.509804}%
\pgfsetstrokecolor{currentstroke}%
\pgfsetstrokeopacity{0.800000}%
\pgfsetdash{}{0pt}%
\pgfpathmoveto{\pgfqpoint{4.728764in}{5.545491in}}%
\pgfpathlineto{\pgfqpoint{5.248298in}{4.065422in}}%
\pgfusepath{stroke}%
\end{pgfscope}%
\begin{pgfscope}%
\pgfpathrectangle{\pgfqpoint{0.570343in}{0.331635in}}{\pgfqpoint{9.300000in}{7.700000in}}%
\pgfusepath{clip}%
\pgfsetrectcap%
\pgfsetroundjoin%
\pgfsetlinewidth{1.505625pt}%
\definecolor{currentstroke}{rgb}{1.000000,0.705882,0.509804}%
\pgfsetstrokecolor{currentstroke}%
\pgfsetstrokeopacity{0.800000}%
\pgfsetdash{}{0pt}%
\pgfpathmoveto{\pgfqpoint{5.530826in}{3.834914in}}%
\pgfpathlineto{\pgfqpoint{5.248298in}{4.065422in}}%
\pgfusepath{stroke}%
\end{pgfscope}%
\begin{pgfscope}%
\pgfpathrectangle{\pgfqpoint{0.570343in}{0.331635in}}{\pgfqpoint{9.300000in}{7.700000in}}%
\pgfusepath{clip}%
\pgfsetrectcap%
\pgfsetroundjoin%
\pgfsetlinewidth{1.505625pt}%
\definecolor{currentstroke}{rgb}{1.000000,0.705882,0.509804}%
\pgfsetstrokecolor{currentstroke}%
\pgfsetstrokeopacity{0.800000}%
\pgfsetdash{}{0pt}%
\pgfpathmoveto{\pgfqpoint{4.970295in}{0.681635in}}%
\pgfpathlineto{\pgfqpoint{5.248298in}{4.065422in}}%
\pgfusepath{stroke}%
\end{pgfscope}%
\begin{pgfscope}%
\pgfpathrectangle{\pgfqpoint{0.570343in}{0.331635in}}{\pgfqpoint{9.300000in}{7.700000in}}%
\pgfusepath{clip}%
\pgfsetrectcap%
\pgfsetroundjoin%
\pgfsetlinewidth{1.505625pt}%
\definecolor{currentstroke}{rgb}{1.000000,0.705882,0.509804}%
\pgfsetstrokecolor{currentstroke}%
\pgfsetstrokeopacity{0.800000}%
\pgfsetdash{}{0pt}%
\pgfpathmoveto{\pgfqpoint{7.708486in}{4.695802in}}%
\pgfpathlineto{\pgfqpoint{5.248298in}{4.065422in}}%
\pgfusepath{stroke}%
\end{pgfscope}%
\begin{pgfscope}%
\pgfpathrectangle{\pgfqpoint{0.570343in}{0.331635in}}{\pgfqpoint{9.300000in}{7.700000in}}%
\pgfusepath{clip}%
\pgfsetrectcap%
\pgfsetroundjoin%
\pgfsetlinewidth{1.505625pt}%
\definecolor{currentstroke}{rgb}{1.000000,0.705882,0.509804}%
\pgfsetstrokecolor{currentstroke}%
\pgfsetstrokeopacity{0.800000}%
\pgfsetdash{}{0pt}%
\pgfpathmoveto{\pgfqpoint{5.865552in}{3.238000in}}%
\pgfpathlineto{\pgfqpoint{5.248298in}{4.065422in}}%
\pgfusepath{stroke}%
\end{pgfscope}%
\begin{pgfscope}%
\pgfpathrectangle{\pgfqpoint{0.570343in}{0.331635in}}{\pgfqpoint{9.300000in}{7.700000in}}%
\pgfusepath{clip}%
\pgfsetrectcap%
\pgfsetroundjoin%
\pgfsetlinewidth{1.505625pt}%
\definecolor{currentstroke}{rgb}{1.000000,0.705882,0.509804}%
\pgfsetstrokecolor{currentstroke}%
\pgfsetstrokeopacity{0.800000}%
\pgfsetdash{}{0pt}%
\pgfpathmoveto{\pgfqpoint{4.644801in}{4.321574in}}%
\pgfpathlineto{\pgfqpoint{5.248298in}{4.065422in}}%
\pgfusepath{stroke}%
\end{pgfscope}%
\begin{pgfscope}%
\pgfpathrectangle{\pgfqpoint{0.570343in}{0.331635in}}{\pgfqpoint{9.300000in}{7.700000in}}%
\pgfusepath{clip}%
\pgfsetrectcap%
\pgfsetroundjoin%
\pgfsetlinewidth{1.505625pt}%
\definecolor{currentstroke}{rgb}{1.000000,0.705882,0.509804}%
\pgfsetstrokecolor{currentstroke}%
\pgfsetstrokeopacity{0.800000}%
\pgfsetdash{}{0pt}%
\pgfpathmoveto{\pgfqpoint{3.446889in}{3.195321in}}%
\pgfpathlineto{\pgfqpoint{5.248298in}{4.065422in}}%
\pgfusepath{stroke}%
\end{pgfscope}%
\begin{pgfscope}%
\pgfpathrectangle{\pgfqpoint{0.570343in}{0.331635in}}{\pgfqpoint{9.300000in}{7.700000in}}%
\pgfusepath{clip}%
\pgfsetrectcap%
\pgfsetroundjoin%
\pgfsetlinewidth{1.505625pt}%
\definecolor{currentstroke}{rgb}{1.000000,0.705882,0.509804}%
\pgfsetstrokecolor{currentstroke}%
\pgfsetstrokeopacity{0.800000}%
\pgfsetdash{}{0pt}%
\pgfpathmoveto{\pgfqpoint{5.205704in}{4.624950in}}%
\pgfpathlineto{\pgfqpoint{5.248298in}{4.065422in}}%
\pgfusepath{stroke}%
\end{pgfscope}%
\begin{pgfscope}%
\pgfpathrectangle{\pgfqpoint{0.570343in}{0.331635in}}{\pgfqpoint{9.300000in}{7.700000in}}%
\pgfusepath{clip}%
\pgfsetrectcap%
\pgfsetroundjoin%
\pgfsetlinewidth{1.505625pt}%
\definecolor{currentstroke}{rgb}{1.000000,0.705882,0.509804}%
\pgfsetstrokecolor{currentstroke}%
\pgfsetstrokeopacity{0.800000}%
\pgfsetdash{}{0pt}%
\pgfpathmoveto{\pgfqpoint{5.477250in}{4.943988in}}%
\pgfpathlineto{\pgfqpoint{5.248298in}{4.065422in}}%
\pgfusepath{stroke}%
\end{pgfscope}%
\begin{pgfscope}%
\pgfpathrectangle{\pgfqpoint{0.570343in}{0.331635in}}{\pgfqpoint{9.300000in}{7.700000in}}%
\pgfusepath{clip}%
\pgfsetrectcap%
\pgfsetroundjoin%
\pgfsetlinewidth{1.505625pt}%
\definecolor{currentstroke}{rgb}{1.000000,0.705882,0.509804}%
\pgfsetstrokecolor{currentstroke}%
\pgfsetstrokeopacity{0.800000}%
\pgfsetdash{}{0pt}%
\pgfpathmoveto{\pgfqpoint{4.499527in}{1.603937in}}%
\pgfpathlineto{\pgfqpoint{5.248298in}{4.065422in}}%
\pgfusepath{stroke}%
\end{pgfscope}%
\begin{pgfscope}%
\pgfpathrectangle{\pgfqpoint{0.570343in}{0.331635in}}{\pgfqpoint{9.300000in}{7.700000in}}%
\pgfusepath{clip}%
\pgfsetrectcap%
\pgfsetroundjoin%
\pgfsetlinewidth{1.505625pt}%
\definecolor{currentstroke}{rgb}{1.000000,0.705882,0.509804}%
\pgfsetstrokecolor{currentstroke}%
\pgfsetstrokeopacity{0.800000}%
\pgfsetdash{}{0pt}%
\pgfpathmoveto{\pgfqpoint{4.967473in}{2.182901in}}%
\pgfpathlineto{\pgfqpoint{5.248298in}{4.065422in}}%
\pgfusepath{stroke}%
\end{pgfscope}%
\begin{pgfscope}%
\pgfpathrectangle{\pgfqpoint{0.570343in}{0.331635in}}{\pgfqpoint{9.300000in}{7.700000in}}%
\pgfusepath{clip}%
\pgfsetrectcap%
\pgfsetroundjoin%
\pgfsetlinewidth{1.505625pt}%
\definecolor{currentstroke}{rgb}{1.000000,0.705882,0.509804}%
\pgfsetstrokecolor{currentstroke}%
\pgfsetstrokeopacity{0.800000}%
\pgfsetdash{}{0pt}%
\pgfpathmoveto{\pgfqpoint{5.200385in}{5.916554in}}%
\pgfpathlineto{\pgfqpoint{5.248298in}{4.065422in}}%
\pgfusepath{stroke}%
\end{pgfscope}%
\begin{pgfscope}%
\pgfpathrectangle{\pgfqpoint{0.570343in}{0.331635in}}{\pgfqpoint{9.300000in}{7.700000in}}%
\pgfusepath{clip}%
\pgfsetrectcap%
\pgfsetroundjoin%
\pgfsetlinewidth{1.505625pt}%
\definecolor{currentstroke}{rgb}{1.000000,0.705882,0.509804}%
\pgfsetstrokecolor{currentstroke}%
\pgfsetstrokeopacity{0.800000}%
\pgfsetdash{}{0pt}%
\pgfpathmoveto{\pgfqpoint{3.099182in}{3.935399in}}%
\pgfpathlineto{\pgfqpoint{5.248298in}{4.065422in}}%
\pgfusepath{stroke}%
\end{pgfscope}%
\begin{pgfscope}%
\pgfpathrectangle{\pgfqpoint{0.570343in}{0.331635in}}{\pgfqpoint{9.300000in}{7.700000in}}%
\pgfusepath{clip}%
\pgfsetrectcap%
\pgfsetroundjoin%
\pgfsetlinewidth{1.505625pt}%
\definecolor{currentstroke}{rgb}{1.000000,0.705882,0.509804}%
\pgfsetstrokecolor{currentstroke}%
\pgfsetstrokeopacity{0.800000}%
\pgfsetdash{}{0pt}%
\pgfpathmoveto{\pgfqpoint{5.161067in}{5.145314in}}%
\pgfpathlineto{\pgfqpoint{5.248298in}{4.065422in}}%
\pgfusepath{stroke}%
\end{pgfscope}%
\begin{pgfscope}%
\pgfpathrectangle{\pgfqpoint{0.570343in}{0.331635in}}{\pgfqpoint{9.300000in}{7.700000in}}%
\pgfusepath{clip}%
\pgfsetrectcap%
\pgfsetroundjoin%
\pgfsetlinewidth{1.505625pt}%
\definecolor{currentstroke}{rgb}{1.000000,0.705882,0.509804}%
\pgfsetstrokecolor{currentstroke}%
\pgfsetstrokeopacity{0.800000}%
\pgfsetdash{}{0pt}%
\pgfpathmoveto{\pgfqpoint{3.646482in}{2.466683in}}%
\pgfpathlineto{\pgfqpoint{5.248298in}{4.065422in}}%
\pgfusepath{stroke}%
\end{pgfscope}%
\begin{pgfscope}%
\pgfpathrectangle{\pgfqpoint{0.570343in}{0.331635in}}{\pgfqpoint{9.300000in}{7.700000in}}%
\pgfusepath{clip}%
\pgfsetrectcap%
\pgfsetroundjoin%
\pgfsetlinewidth{1.505625pt}%
\definecolor{currentstroke}{rgb}{1.000000,0.705882,0.509804}%
\pgfsetstrokecolor{currentstroke}%
\pgfsetstrokeopacity{0.800000}%
\pgfsetdash{}{0pt}%
\pgfpathmoveto{\pgfqpoint{3.852857in}{2.821794in}}%
\pgfpathlineto{\pgfqpoint{5.248298in}{4.065422in}}%
\pgfusepath{stroke}%
\end{pgfscope}%
\begin{pgfscope}%
\pgfpathrectangle{\pgfqpoint{0.570343in}{0.331635in}}{\pgfqpoint{9.300000in}{7.700000in}}%
\pgfusepath{clip}%
\pgfsetrectcap%
\pgfsetroundjoin%
\pgfsetlinewidth{1.505625pt}%
\definecolor{currentstroke}{rgb}{1.000000,0.705882,0.509804}%
\pgfsetstrokecolor{currentstroke}%
\pgfsetstrokeopacity{0.800000}%
\pgfsetdash{}{0pt}%
\pgfpathmoveto{\pgfqpoint{5.635152in}{5.339728in}}%
\pgfpathlineto{\pgfqpoint{5.248298in}{4.065422in}}%
\pgfusepath{stroke}%
\end{pgfscope}%
\begin{pgfscope}%
\pgfpathrectangle{\pgfqpoint{0.570343in}{0.331635in}}{\pgfqpoint{9.300000in}{7.700000in}}%
\pgfusepath{clip}%
\pgfsetrectcap%
\pgfsetroundjoin%
\pgfsetlinewidth{1.505625pt}%
\definecolor{currentstroke}{rgb}{1.000000,0.705882,0.509804}%
\pgfsetstrokecolor{currentstroke}%
\pgfsetstrokeopacity{0.800000}%
\pgfsetdash{}{0pt}%
\pgfpathmoveto{\pgfqpoint{7.128037in}{4.600946in}}%
\pgfpathlineto{\pgfqpoint{5.248298in}{4.065422in}}%
\pgfusepath{stroke}%
\end{pgfscope}%
\begin{pgfscope}%
\pgfpathrectangle{\pgfqpoint{0.570343in}{0.331635in}}{\pgfqpoint{9.300000in}{7.700000in}}%
\pgfusepath{clip}%
\pgfsetrectcap%
\pgfsetroundjoin%
\pgfsetlinewidth{1.505625pt}%
\definecolor{currentstroke}{rgb}{1.000000,0.705882,0.509804}%
\pgfsetstrokecolor{currentstroke}%
\pgfsetstrokeopacity{0.800000}%
\pgfsetdash{}{0pt}%
\pgfpathmoveto{\pgfqpoint{5.097012in}{6.515272in}}%
\pgfpathlineto{\pgfqpoint{5.248298in}{4.065422in}}%
\pgfusepath{stroke}%
\end{pgfscope}%
\begin{pgfscope}%
\pgfpathrectangle{\pgfqpoint{0.570343in}{0.331635in}}{\pgfqpoint{9.300000in}{7.700000in}}%
\pgfusepath{clip}%
\pgfsetrectcap%
\pgfsetroundjoin%
\pgfsetlinewidth{1.505625pt}%
\definecolor{currentstroke}{rgb}{1.000000,0.705882,0.509804}%
\pgfsetstrokecolor{currentstroke}%
\pgfsetstrokeopacity{0.800000}%
\pgfsetdash{}{0pt}%
\pgfpathmoveto{\pgfqpoint{4.135589in}{1.179144in}}%
\pgfpathlineto{\pgfqpoint{5.248298in}{4.065422in}}%
\pgfusepath{stroke}%
\end{pgfscope}%
\begin{pgfscope}%
\pgfpathrectangle{\pgfqpoint{0.570343in}{0.331635in}}{\pgfqpoint{9.300000in}{7.700000in}}%
\pgfusepath{clip}%
\pgfsetrectcap%
\pgfsetroundjoin%
\pgfsetlinewidth{1.505625pt}%
\definecolor{currentstroke}{rgb}{1.000000,0.705882,0.509804}%
\pgfsetstrokecolor{currentstroke}%
\pgfsetstrokeopacity{0.800000}%
\pgfsetdash{}{0pt}%
\pgfpathmoveto{\pgfqpoint{6.100627in}{3.676098in}}%
\pgfpathlineto{\pgfqpoint{5.248298in}{4.065422in}}%
\pgfusepath{stroke}%
\end{pgfscope}%
\begin{pgfscope}%
\pgfpathrectangle{\pgfqpoint{0.570343in}{0.331635in}}{\pgfqpoint{9.300000in}{7.700000in}}%
\pgfusepath{clip}%
\pgfsetrectcap%
\pgfsetroundjoin%
\pgfsetlinewidth{1.505625pt}%
\definecolor{currentstroke}{rgb}{1.000000,0.705882,0.509804}%
\pgfsetstrokecolor{currentstroke}%
\pgfsetstrokeopacity{0.800000}%
\pgfsetdash{}{0pt}%
\pgfpathmoveto{\pgfqpoint{6.017440in}{5.723041in}}%
\pgfpathlineto{\pgfqpoint{5.248298in}{4.065422in}}%
\pgfusepath{stroke}%
\end{pgfscope}%
\begin{pgfscope}%
\pgfpathrectangle{\pgfqpoint{0.570343in}{0.331635in}}{\pgfqpoint{9.300000in}{7.700000in}}%
\pgfusepath{clip}%
\pgfsetrectcap%
\pgfsetroundjoin%
\pgfsetlinewidth{1.505625pt}%
\definecolor{currentstroke}{rgb}{1.000000,0.705882,0.509804}%
\pgfsetstrokecolor{currentstroke}%
\pgfsetstrokeopacity{0.800000}%
\pgfsetdash{}{0pt}%
\pgfpathmoveto{\pgfqpoint{5.188705in}{3.517638in}}%
\pgfpathlineto{\pgfqpoint{5.248298in}{4.065422in}}%
\pgfusepath{stroke}%
\end{pgfscope}%
\begin{pgfscope}%
\pgfsetrectcap%
\pgfsetmiterjoin%
\pgfsetlinewidth{0.803000pt}%
\definecolor{currentstroke}{rgb}{0.000000,0.000000,0.000000}%
\pgfsetstrokecolor{currentstroke}%
\pgfsetdash{}{0pt}%
\pgfpathmoveto{\pgfqpoint{0.570343in}{0.331635in}}%
\pgfpathlineto{\pgfqpoint{0.570343in}{8.031635in}}%
\pgfusepath{stroke}%
\end{pgfscope}%
\begin{pgfscope}%
\pgfsetrectcap%
\pgfsetmiterjoin%
\pgfsetlinewidth{0.803000pt}%
\definecolor{currentstroke}{rgb}{0.000000,0.000000,0.000000}%
\pgfsetstrokecolor{currentstroke}%
\pgfsetdash{}{0pt}%
\pgfpathmoveto{\pgfqpoint{9.870343in}{0.331635in}}%
\pgfpathlineto{\pgfqpoint{9.870343in}{8.031635in}}%
\pgfusepath{stroke}%
\end{pgfscope}%
\begin{pgfscope}%
\pgfsetrectcap%
\pgfsetmiterjoin%
\pgfsetlinewidth{0.803000pt}%
\definecolor{currentstroke}{rgb}{0.000000,0.000000,0.000000}%
\pgfsetstrokecolor{currentstroke}%
\pgfsetdash{}{0pt}%
\pgfpathmoveto{\pgfqpoint{0.570343in}{0.331635in}}%
\pgfpathlineto{\pgfqpoint{9.870343in}{0.331635in}}%
\pgfusepath{stroke}%
\end{pgfscope}%
\begin{pgfscope}%
\pgfsetrectcap%
\pgfsetmiterjoin%
\pgfsetlinewidth{0.803000pt}%
\definecolor{currentstroke}{rgb}{0.000000,0.000000,0.000000}%
\pgfsetstrokecolor{currentstroke}%
\pgfsetdash{}{0pt}%
\pgfpathmoveto{\pgfqpoint{0.570343in}{8.031635in}}%
\pgfpathlineto{\pgfqpoint{9.870343in}{8.031635in}}%
\pgfusepath{stroke}%
\end{pgfscope}%
\begin{pgfscope}%
\definecolor{textcolor}{rgb}{0.000000,0.000000,0.000000}%
\pgfsetstrokecolor{textcolor}%
\pgfsetfillcolor{textcolor}%
\pgftext[x=5.220343in,y=8.114968in,,base]{\color{textcolor}\sffamily\fontsize{12.000000}{14.400000}\selectfont T-SNE for chair images (s2r3dfree\_chair)}%
\end{pgfscope}%
\begin{pgfscope}%
\pgfsetbuttcap%
\pgfsetmiterjoin%
\definecolor{currentfill}{rgb}{1.000000,1.000000,1.000000}%
\pgfsetfillcolor{currentfill}%
\pgfsetfillopacity{0.800000}%
\pgfsetlinewidth{1.003750pt}%
\definecolor{currentstroke}{rgb}{0.800000,0.800000,0.800000}%
\pgfsetstrokecolor{currentstroke}%
\pgfsetstrokeopacity{0.800000}%
\pgfsetdash{}{0pt}%
\pgfpathmoveto{\pgfqpoint{9.967566in}{3.955012in}}%
\pgfpathlineto{\pgfqpoint{11.494842in}{3.955012in}}%
\pgfpathquadraticcurveto{\pgfqpoint{11.522619in}{3.955012in}}{\pgfqpoint{11.522619in}{3.982789in}}%
\pgfpathlineto{\pgfqpoint{11.522619in}{4.380481in}}%
\pgfpathquadraticcurveto{\pgfqpoint{11.522619in}{4.408258in}}{\pgfqpoint{11.494842in}{4.408258in}}%
\pgfpathlineto{\pgfqpoint{9.967566in}{4.408258in}}%
\pgfpathquadraticcurveto{\pgfqpoint{9.939788in}{4.408258in}}{\pgfqpoint{9.939788in}{4.380481in}}%
\pgfpathlineto{\pgfqpoint{9.939788in}{3.982789in}}%
\pgfpathquadraticcurveto{\pgfqpoint{9.939788in}{3.955012in}}{\pgfqpoint{9.967566in}{3.955012in}}%
\pgfpathclose%
\pgfusepath{stroke,fill}%
\end{pgfscope}%
\begin{pgfscope}%
\pgfsetbuttcap%
\pgfsetroundjoin%
\definecolor{currentfill}{rgb}{0.631373,0.788235,0.956863}%
\pgfsetfillcolor{currentfill}%
\pgfsetlinewidth{1.003750pt}%
\definecolor{currentstroke}{rgb}{0.631373,0.788235,0.956863}%
\pgfsetstrokecolor{currentstroke}%
\pgfsetdash{}{0pt}%
\pgfsys@defobject{currentmarker}{\pgfqpoint{-0.041667in}{-0.041667in}}{\pgfqpoint{0.041667in}{0.041667in}}{%
\pgfpathmoveto{\pgfqpoint{0.000000in}{-0.041667in}}%
\pgfpathcurveto{\pgfqpoint{0.011050in}{-0.041667in}}{\pgfqpoint{0.021649in}{-0.037276in}}{\pgfqpoint{0.029463in}{-0.029463in}}%
\pgfpathcurveto{\pgfqpoint{0.037276in}{-0.021649in}}{\pgfqpoint{0.041667in}{-0.011050in}}{\pgfqpoint{0.041667in}{0.000000in}}%
\pgfpathcurveto{\pgfqpoint{0.041667in}{0.011050in}}{\pgfqpoint{0.037276in}{0.021649in}}{\pgfqpoint{0.029463in}{0.029463in}}%
\pgfpathcurveto{\pgfqpoint{0.021649in}{0.037276in}}{\pgfqpoint{0.011050in}{0.041667in}}{\pgfqpoint{0.000000in}{0.041667in}}%
\pgfpathcurveto{\pgfqpoint{-0.011050in}{0.041667in}}{\pgfqpoint{-0.021649in}{0.037276in}}{\pgfqpoint{-0.029463in}{0.029463in}}%
\pgfpathcurveto{\pgfqpoint{-0.037276in}{0.021649in}}{\pgfqpoint{-0.041667in}{0.011050in}}{\pgfqpoint{-0.041667in}{0.000000in}}%
\pgfpathcurveto{\pgfqpoint{-0.041667in}{-0.011050in}}{\pgfqpoint{-0.037276in}{-0.021649in}}{\pgfqpoint{-0.029463in}{-0.029463in}}%
\pgfpathcurveto{\pgfqpoint{-0.021649in}{-0.037276in}}{\pgfqpoint{-0.011050in}{-0.041667in}}{\pgfqpoint{0.000000in}{-0.041667in}}%
\pgfpathclose%
\pgfusepath{stroke,fill}%
}%
\begin{pgfscope}%
\pgfsys@transformshift{10.134232in}{4.283638in}%
\pgfsys@useobject{currentmarker}{}%
\end{pgfscope}%
\end{pgfscope}%
\begin{pgfscope}%
\definecolor{textcolor}{rgb}{0.000000,0.000000,0.000000}%
\pgfsetstrokecolor{textcolor}%
\pgfsetfillcolor{textcolor}%
\pgftext[x=10.384232in,y=4.247180in,left,base]{\color{textcolor}\sffamily\fontsize{10.000000}{12.000000}\selectfont Pix3D}%
\end{pgfscope}%
\begin{pgfscope}%
\pgfsetbuttcap%
\pgfsetroundjoin%
\definecolor{currentfill}{rgb}{1.000000,0.705882,0.509804}%
\pgfsetfillcolor{currentfill}%
\pgfsetlinewidth{1.003750pt}%
\definecolor{currentstroke}{rgb}{1.000000,0.705882,0.509804}%
\pgfsetstrokecolor{currentstroke}%
\pgfsetdash{}{0pt}%
\pgfsys@defobject{currentmarker}{\pgfqpoint{-0.041667in}{-0.041667in}}{\pgfqpoint{0.041667in}{0.041667in}}{%
\pgfpathmoveto{\pgfqpoint{0.000000in}{-0.041667in}}%
\pgfpathcurveto{\pgfqpoint{0.011050in}{-0.041667in}}{\pgfqpoint{0.021649in}{-0.037276in}}{\pgfqpoint{0.029463in}{-0.029463in}}%
\pgfpathcurveto{\pgfqpoint{0.037276in}{-0.021649in}}{\pgfqpoint{0.041667in}{-0.011050in}}{\pgfqpoint{0.041667in}{0.000000in}}%
\pgfpathcurveto{\pgfqpoint{0.041667in}{0.011050in}}{\pgfqpoint{0.037276in}{0.021649in}}{\pgfqpoint{0.029463in}{0.029463in}}%
\pgfpathcurveto{\pgfqpoint{0.021649in}{0.037276in}}{\pgfqpoint{0.011050in}{0.041667in}}{\pgfqpoint{0.000000in}{0.041667in}}%
\pgfpathcurveto{\pgfqpoint{-0.011050in}{0.041667in}}{\pgfqpoint{-0.021649in}{0.037276in}}{\pgfqpoint{-0.029463in}{0.029463in}}%
\pgfpathcurveto{\pgfqpoint{-0.037276in}{0.021649in}}{\pgfqpoint{-0.041667in}{0.011050in}}{\pgfqpoint{-0.041667in}{0.000000in}}%
\pgfpathcurveto{\pgfqpoint{-0.041667in}{-0.011050in}}{\pgfqpoint{-0.037276in}{-0.021649in}}{\pgfqpoint{-0.029463in}{-0.029463in}}%
\pgfpathcurveto{\pgfqpoint{-0.021649in}{-0.037276in}}{\pgfqpoint{-0.011050in}{-0.041667in}}{\pgfqpoint{0.000000in}{-0.041667in}}%
\pgfpathclose%
\pgfusepath{stroke,fill}%
}%
\begin{pgfscope}%
\pgfsys@transformshift{10.134232in}{4.079781in}%
\pgfsys@useobject{currentmarker}{}%
\end{pgfscope}%
\end{pgfscope}%
\begin{pgfscope}%
\definecolor{textcolor}{rgb}{0.000000,0.000000,0.000000}%
\pgfsetstrokecolor{textcolor}%
\pgfsetfillcolor{textcolor}%
\pgftext[x=10.384232in,y=4.043322in,left,base]{\color{textcolor}\sffamily\fontsize{10.000000}{12.000000}\selectfont s2r3dfree\_chair}%
\end{pgfscope}%
\end{pgfpicture}%
\makeatother%
\endgroup%
}
    \caption{T-SNE visualisation for images from individual synthetic chair dataset with different domain randomisation parameter compared with Pix3D chair latent space.
        (Left to right, top to bottom) Textureless, Textureless with light, Textured, Textured with light, Multi-Object}
    \label{fig:tsne per chair dataset}
\end{figure}


\subsubsection{2R:3DFREE\_Textureless}

    For textureless dataset, the chair models were kept at the center of an un-textured room with constant light source.
    A total of 10000 images were generated from different camera viewpoints.

\subsubsection{\Gls{free}\_Textureless\_Light}

    For this dataset, similar to the textureless dataset the chair models were kept at the center of an un-textured room, but with randomised light source.
    The light variation was implemented as in~\ref{subsec:lightings-and-shadows}.
    Along with randomised light source, the camera view points were randomised with a distance in range of 0.75 to 1.5 meters from the model under observation.

\subsubsection{\Gls{free}\_Textured}

    As the name suggests, in this dataset both the model and the default single room were textured randomly for each snapshot as explained in~\ref{subsec:randomised-texture}.
    10000 snapshot of chair models were taken using different camera viewpoints.

\subsubsection{\Gls{free}\_Textured\_Light}

    This is an extension of the above mention \gls{free}\_Textured dataset, with addition of randomized light sources.
    The lights are randomized as implemented in~\ref{subsec:lightings-and-shadows}.

\subsubsection{\gls{free}\_Chair}

    \Gls{free}\_Chair dataset was created using 'Multi Object pipeline' with chair replacing a similar category from scene as implemented in~\ref{subsec:replacing-target-objects}.
    Both the light and camera view points were randomised making sure that the model under observation is not completly occluded.


\begin{figure}
    \begin{tabular}{llll}
        Pix3D & \includegraphics[width=.2\textwidth, height =.19\textwidth]{/Users/apple/OVGU/Thesis/code/3dReconstruction/report/images/evaluation/datasets/pix3d_1} &
        \includegraphics[width=.2\textwidth, height =.19\textwidth]{/Users/apple/OVGU/Thesis/code/3dReconstruction/report/images/evaluation/datasets/pix3d_2} &
        \includegraphics[width=.2\textwidth, height =.19\textwidth]{/Users/apple/OVGU/Thesis/code/3dReconstruction/report/images/evaluation/datasets/pix3d_3}\\

        \Gls{free} Version 1 & \includegraphics[width=.2\textwidth, height =.19\textwidth]{/Users/apple/OVGU/Thesis/code/3dReconstruction/report/images/evaluation/datasets/s2r_v1_1} &
        \includegraphics[width=.2\textwidth, height=.19\textwidth]{/Users/apple/OVGU/Thesis/code/3dReconstruction/report/images/evaluation/datasets/s2r_v1_2} &
        \includegraphics[width=.2\textwidth, height=.19\textwidth]{/Users/apple/OVGU/Thesis/code/3dReconstruction/report/images/evaluation/datasets/s2r_v1_3}\\

        \Gls{free} Version 2 & \includegraphics[width=.19\textwidth, height =.2\textwidth]{/Users/apple/OVGU/Thesis/code/3dReconstruction/report/images/evaluation/datasets/s2r_v3_1} &
        \includegraphics[width=.2\textwidth, height=.19\textwidth]{/Users/apple/OVGU/Thesis/code/3dReconstruction/report/images/evaluation/datasets/s2r_v3_2} &
        \includegraphics[width=.2\textwidth, height=.19\textwidth]{/Users/apple/OVGU/Thesis/code/3dReconstruction/report/images/evaluation/datasets/s2r_v3_3}\\

    \end{tabular}
    \caption{Samples of images from real and synthetic datasets.}
    \label{fig:samples for synthetic and real comparison}
\end{figure}


\begin{figure}
    \begin{tabular}{llll}
        Textureless & \includegraphics[width=.2\linewidth]{/Users/apple/OVGU/Thesis/code/3dReconstruction/report/images/evaluation/datasets/s2r_textureless_1} &
        \includegraphics[width=.2\linewidth]{/Users/apple/OVGU/Thesis/code/3dReconstruction/report/images/evaluation/datasets/s2r_textureless_2} &
        \includegraphics[width=.2\linewidth]{/Users/apple/OVGU/Thesis/code/3dReconstruction/report/images/evaluation/datasets/s2r_textureless_3}\\

        Textureless with light & \includegraphics[width=.2\linewidth]{/Users/apple/OVGU/Thesis/code/3dReconstruction/report/images/evaluation/datasets/s2r_textureless_light_1} &
        \includegraphics[width=.2\linewidth]{/Users/apple/OVGU/Thesis/code/3dReconstruction/report/images/evaluation/datasets/s2r_textureless_light_2} &
        \includegraphics[width=.2\linewidth]{/Users/apple/OVGU/Thesis/code/3dReconstruction/report/images/evaluation/datasets/s2r_textureless_light_3}\\

        Textured & \includegraphics[width=.2\linewidth]{/Users/apple/OVGU/Thesis/code/3dReconstruction/report/images/evaluation/datasets/s2r_textured_1} &
        \includegraphics[width=.2\linewidth]{/Users/apple/OVGU/Thesis/code/3dReconstruction/report/images/evaluation/datasets/s2r_textured_2} &
        \includegraphics[width=.2\linewidth]{/Users/apple/OVGU/Thesis/code/3dReconstruction/report/images/evaluation/datasets/s2r_textured_3}\\

        Textured with light & \includegraphics[width=.2\linewidth]{/Users/apple/OVGU/Thesis/code/3dReconstruction/report/images/evaluation/datasets/s2r_textured_light_1} &
        \includegraphics[width=.2\linewidth]{/Users/apple/OVGU/Thesis/code/3dReconstruction/report/images/evaluation/datasets/s2r_textured_light_2} &
        \includegraphics[width=.2\linewidth]{/Users/apple/OVGU/Thesis/code/3dReconstruction/report/images/evaluation/datasets/s2r_textured_light_4}\\

        Multi-Object & \includegraphics[width=.2\linewidth]{/Users/apple/OVGU/Thesis/code/3dReconstruction/report/images/evaluation/datasets/s2r_chair_1} &
        \includegraphics[width=.2\linewidth]{/Users/apple/OVGU/Thesis/code/3dReconstruction/report/images/evaluation/datasets/s2r_chair_2} &
        \includegraphics[width=.2\linewidth]{/Users/apple/OVGU/Thesis/code/3dReconstruction/report/images/evaluation/datasets/s2r_chair_3}\\

    \end{tabular}
    \caption{Samples of images used for ablation study on chairs with different parameters of domain randomisation.}
    \label{fig:domain randomisation for ablation study}
\end{figure}

\section{Baseline}\label{sec:baseline}

As mentioned in~\ref{subsec:pix2vox-and-pix2vox++}, pix2vox and pix2vox++ are the models which will act as the baselines for all the experiments.
For the dataset from~\ref{sec:datasets}, Pix3D is the real dataset and will be acting as the base dataset.
The models are also compared with and without 2D augmentation.
The 2D augmentations include Random Flip, Random Crop, Color Jitter, RandomPermuteRGB.
A performance test with no augmentation will give us actually idea whether synthetic dataset actually enhances the performance of 3D reconstruction task.

Figure~\ref{fig:baseline1}, represents performance of baseline models on different datasets.
It is seen that 2D augmentation improves \gls{iou}  by 1.2\% for Pix3D on pix2vox++, and 1.15\% on pix2vox.
For the synthetic dataset, the models are trained on 70\% of data and 30\% is used for validation.
For testing we use the same 30\% of real data from Pix3D dataset.
In case of synthetic dataset, 2D augmentation increased the \gls{iou}  by 2.04\% and 4.77\%  for pix2vo++ and pix2vox respectively.
When it comes to whether synthetic dataset gives equivalent performance as real dataset, we can clearly see a dip in the performance,
proving that there is a domain gap between real and synthetic data.

\begin{figure}
    \centering
    \resizebox{\textwidth}{!}{%% Creator: Matplotlib, PGF backend
%%
%% To include the figure in your LaTeX document, write
%%   \input{<filename>.pgf}
%%
%% Make sure the required packages are loaded in your preamble
%%   \usepackage{pgf}
%%
%% Figures using additional raster images can only be included by \input if
%% they are in the same directory as the main LaTeX file. For loading figures
%% from other directories you can use the `import` package
%%   \usepackage{import}
%%
%% and then include the figures with
%%   \import{<path to file>}{<filename>.pgf}
%%
%% Matplotlib used the following preamble
%%   \usepackage{fontspec}
%%   \setmainfont{DejaVuSerif.ttf}[Path=\detokenize{/Users/apple/opt/anaconda3/envs/kaolin/lib/python3.7/site-packages/matplotlib/mpl-data/fonts/ttf/}]
%%   \setsansfont{DejaVuSans.ttf}[Path=\detokenize{/Users/apple/opt/anaconda3/envs/kaolin/lib/python3.7/site-packages/matplotlib/mpl-data/fonts/ttf/}]
%%   \setmonofont{DejaVuSansMono.ttf}[Path=\detokenize{/Users/apple/opt/anaconda3/envs/kaolin/lib/python3.7/site-packages/matplotlib/mpl-data/fonts/ttf/}]
%%
\begingroup%
\makeatletter%
\begin{pgfpicture}%
\pgfpathrectangle{\pgfpointorigin}{\pgfqpoint{7.635323in}{4.684073in}}%
\pgfusepath{use as bounding box, clip}%
\begin{pgfscope}%
\pgfsetbuttcap%
\pgfsetmiterjoin%
\definecolor{currentfill}{rgb}{1.000000,1.000000,1.000000}%
\pgfsetfillcolor{currentfill}%
\pgfsetlinewidth{0.000000pt}%
\definecolor{currentstroke}{rgb}{1.000000,1.000000,1.000000}%
\pgfsetstrokecolor{currentstroke}%
\pgfsetdash{}{0pt}%
\pgfpathmoveto{\pgfqpoint{0.000000in}{0.000000in}}%
\pgfpathlineto{\pgfqpoint{7.635323in}{0.000000in}}%
\pgfpathlineto{\pgfqpoint{7.635323in}{4.684073in}}%
\pgfpathlineto{\pgfqpoint{0.000000in}{4.684073in}}%
\pgfpathclose%
\pgfusepath{fill}%
\end{pgfscope}%
\begin{pgfscope}%
\pgfsetbuttcap%
\pgfsetmiterjoin%
\definecolor{currentfill}{rgb}{1.000000,1.000000,1.000000}%
\pgfsetfillcolor{currentfill}%
\pgfsetlinewidth{0.000000pt}%
\definecolor{currentstroke}{rgb}{0.000000,0.000000,0.000000}%
\pgfsetstrokecolor{currentstroke}%
\pgfsetstrokeopacity{0.000000}%
\pgfsetdash{}{0pt}%
\pgfpathmoveto{\pgfqpoint{0.696435in}{1.033583in}}%
\pgfpathlineto{\pgfqpoint{6.196158in}{1.033583in}}%
\pgfpathlineto{\pgfqpoint{6.196158in}{4.374112in}}%
\pgfpathlineto{\pgfqpoint{0.696435in}{4.374112in}}%
\pgfpathclose%
\pgfusepath{fill}%
\end{pgfscope}%
\begin{pgfscope}%
\pgfpathrectangle{\pgfqpoint{0.696435in}{1.033583in}}{\pgfqpoint{5.499722in}{3.340529in}}%
\pgfusepath{clip}%
\pgfsetbuttcap%
\pgfsetmiterjoin%
\definecolor{currentfill}{rgb}{0.121569,0.466667,0.705882}%
\pgfsetfillcolor{currentfill}%
\pgfsetlinewidth{0.000000pt}%
\definecolor{currentstroke}{rgb}{0.000000,0.000000,0.000000}%
\pgfsetstrokecolor{currentstroke}%
\pgfsetstrokeopacity{0.000000}%
\pgfsetdash{}{0pt}%
\pgfpathmoveto{\pgfqpoint{0.946423in}{1.033583in}}%
\pgfpathlineto{\pgfqpoint{1.405583in}{1.033583in}}%
\pgfpathlineto{\pgfqpoint{1.405583in}{4.002815in}}%
\pgfpathlineto{\pgfqpoint{0.946423in}{4.002815in}}%
\pgfpathclose%
\pgfusepath{fill}%
\end{pgfscope}%
\begin{pgfscope}%
\pgfpathrectangle{\pgfqpoint{0.696435in}{1.033583in}}{\pgfqpoint{5.499722in}{3.340529in}}%
\pgfusepath{clip}%
\pgfsetbuttcap%
\pgfsetmiterjoin%
\definecolor{currentfill}{rgb}{0.121569,0.466667,0.705882}%
\pgfsetfillcolor{currentfill}%
\pgfsetlinewidth{0.000000pt}%
\definecolor{currentstroke}{rgb}{0.000000,0.000000,0.000000}%
\pgfsetstrokecolor{currentstroke}%
\pgfsetstrokeopacity{0.000000}%
\pgfsetdash{}{0pt}%
\pgfpathmoveto{\pgfqpoint{1.966779in}{1.033583in}}%
\pgfpathlineto{\pgfqpoint{2.425940in}{1.033583in}}%
\pgfpathlineto{\pgfqpoint{2.425940in}{4.215039in}}%
\pgfpathlineto{\pgfqpoint{1.966779in}{4.215039in}}%
\pgfpathclose%
\pgfusepath{fill}%
\end{pgfscope}%
\begin{pgfscope}%
\pgfpathrectangle{\pgfqpoint{0.696435in}{1.033583in}}{\pgfqpoint{5.499722in}{3.340529in}}%
\pgfusepath{clip}%
\pgfsetbuttcap%
\pgfsetmiterjoin%
\definecolor{currentfill}{rgb}{0.121569,0.466667,0.705882}%
\pgfsetfillcolor{currentfill}%
\pgfsetlinewidth{0.000000pt}%
\definecolor{currentstroke}{rgb}{0.000000,0.000000,0.000000}%
\pgfsetstrokecolor{currentstroke}%
\pgfsetstrokeopacity{0.000000}%
\pgfsetdash{}{0pt}%
\pgfpathmoveto{\pgfqpoint{2.987136in}{1.033583in}}%
\pgfpathlineto{\pgfqpoint{3.446297in}{1.033583in}}%
\pgfpathlineto{\pgfqpoint{3.446297in}{2.696009in}}%
\pgfpathlineto{\pgfqpoint{2.987136in}{2.696009in}}%
\pgfpathclose%
\pgfusepath{fill}%
\end{pgfscope}%
\begin{pgfscope}%
\pgfpathrectangle{\pgfqpoint{0.696435in}{1.033583in}}{\pgfqpoint{5.499722in}{3.340529in}}%
\pgfusepath{clip}%
\pgfsetbuttcap%
\pgfsetmiterjoin%
\definecolor{currentfill}{rgb}{0.121569,0.466667,0.705882}%
\pgfsetfillcolor{currentfill}%
\pgfsetlinewidth{0.000000pt}%
\definecolor{currentstroke}{rgb}{0.000000,0.000000,0.000000}%
\pgfsetstrokecolor{currentstroke}%
\pgfsetstrokeopacity{0.000000}%
\pgfsetdash{}{0pt}%
\pgfpathmoveto{\pgfqpoint{4.007493in}{1.033583in}}%
\pgfpathlineto{\pgfqpoint{4.466653in}{1.033583in}}%
\pgfpathlineto{\pgfqpoint{4.466653in}{2.891026in}}%
\pgfpathlineto{\pgfqpoint{4.007493in}{2.891026in}}%
\pgfpathclose%
\pgfusepath{fill}%
\end{pgfscope}%
\begin{pgfscope}%
\pgfpathrectangle{\pgfqpoint{0.696435in}{1.033583in}}{\pgfqpoint{5.499722in}{3.340529in}}%
\pgfusepath{clip}%
\pgfsetbuttcap%
\pgfsetmiterjoin%
\definecolor{currentfill}{rgb}{0.121569,0.466667,0.705882}%
\pgfsetfillcolor{currentfill}%
\pgfsetlinewidth{0.000000pt}%
\definecolor{currentstroke}{rgb}{0.000000,0.000000,0.000000}%
\pgfsetstrokecolor{currentstroke}%
\pgfsetstrokeopacity{0.000000}%
\pgfsetdash{}{0pt}%
\pgfpathmoveto{\pgfqpoint{5.027849in}{1.033583in}}%
\pgfpathlineto{\pgfqpoint{5.487010in}{1.033583in}}%
\pgfpathlineto{\pgfqpoint{5.487010in}{2.778222in}}%
\pgfpathlineto{\pgfqpoint{5.027849in}{2.778222in}}%
\pgfpathclose%
\pgfusepath{fill}%
\end{pgfscope}%
\begin{pgfscope}%
\pgfpathrectangle{\pgfqpoint{0.696435in}{1.033583in}}{\pgfqpoint{5.499722in}{3.340529in}}%
\pgfusepath{clip}%
\pgfsetbuttcap%
\pgfsetmiterjoin%
\definecolor{currentfill}{rgb}{1.000000,0.498039,0.054902}%
\pgfsetfillcolor{currentfill}%
\pgfsetlinewidth{0.000000pt}%
\definecolor{currentstroke}{rgb}{0.000000,0.000000,0.000000}%
\pgfsetstrokecolor{currentstroke}%
\pgfsetstrokeopacity{0.000000}%
\pgfsetdash{}{0pt}%
\pgfpathmoveto{\pgfqpoint{1.405583in}{1.033583in}}%
\pgfpathlineto{\pgfqpoint{1.864744in}{1.033583in}}%
\pgfpathlineto{\pgfqpoint{1.864744in}{4.004726in}}%
\pgfpathlineto{\pgfqpoint{1.405583in}{4.004726in}}%
\pgfpathclose%
\pgfusepath{fill}%
\end{pgfscope}%
\begin{pgfscope}%
\pgfpathrectangle{\pgfqpoint{0.696435in}{1.033583in}}{\pgfqpoint{5.499722in}{3.340529in}}%
\pgfusepath{clip}%
\pgfsetbuttcap%
\pgfsetmiterjoin%
\definecolor{currentfill}{rgb}{1.000000,0.498039,0.054902}%
\pgfsetfillcolor{currentfill}%
\pgfsetlinewidth{0.000000pt}%
\definecolor{currentstroke}{rgb}{0.000000,0.000000,0.000000}%
\pgfsetstrokecolor{currentstroke}%
\pgfsetstrokeopacity{0.000000}%
\pgfsetdash{}{0pt}%
\pgfpathmoveto{\pgfqpoint{2.425940in}{1.033583in}}%
\pgfpathlineto{\pgfqpoint{2.885100in}{1.033583in}}%
\pgfpathlineto{\pgfqpoint{2.885100in}{4.121354in}}%
\pgfpathlineto{\pgfqpoint{2.425940in}{4.121354in}}%
\pgfpathclose%
\pgfusepath{fill}%
\end{pgfscope}%
\begin{pgfscope}%
\pgfpathrectangle{\pgfqpoint{0.696435in}{1.033583in}}{\pgfqpoint{5.499722in}{3.340529in}}%
\pgfusepath{clip}%
\pgfsetbuttcap%
\pgfsetmiterjoin%
\definecolor{currentfill}{rgb}{1.000000,0.498039,0.054902}%
\pgfsetfillcolor{currentfill}%
\pgfsetlinewidth{0.000000pt}%
\definecolor{currentstroke}{rgb}{0.000000,0.000000,0.000000}%
\pgfsetstrokecolor{currentstroke}%
\pgfsetstrokeopacity{0.000000}%
\pgfsetdash{}{0pt}%
\pgfpathmoveto{\pgfqpoint{3.446297in}{1.033583in}}%
\pgfpathlineto{\pgfqpoint{3.905457in}{1.033583in}}%
\pgfpathlineto{\pgfqpoint{3.905457in}{2.265824in}}%
\pgfpathlineto{\pgfqpoint{3.446297in}{2.265824in}}%
\pgfpathclose%
\pgfusepath{fill}%
\end{pgfscope}%
\begin{pgfscope}%
\pgfpathrectangle{\pgfqpoint{0.696435in}{1.033583in}}{\pgfqpoint{5.499722in}{3.340529in}}%
\pgfusepath{clip}%
\pgfsetbuttcap%
\pgfsetmiterjoin%
\definecolor{currentfill}{rgb}{1.000000,0.498039,0.054902}%
\pgfsetfillcolor{currentfill}%
\pgfsetlinewidth{0.000000pt}%
\definecolor{currentstroke}{rgb}{0.000000,0.000000,0.000000}%
\pgfsetstrokecolor{currentstroke}%
\pgfsetstrokeopacity{0.000000}%
\pgfsetdash{}{0pt}%
\pgfpathmoveto{\pgfqpoint{4.466653in}{1.033583in}}%
\pgfpathlineto{\pgfqpoint{4.925814in}{1.033583in}}%
\pgfpathlineto{\pgfqpoint{4.925814in}{2.721820in}}%
\pgfpathlineto{\pgfqpoint{4.466653in}{2.721820in}}%
\pgfpathclose%
\pgfusepath{fill}%
\end{pgfscope}%
\begin{pgfscope}%
\pgfpathrectangle{\pgfqpoint{0.696435in}{1.033583in}}{\pgfqpoint{5.499722in}{3.340529in}}%
\pgfusepath{clip}%
\pgfsetbuttcap%
\pgfsetmiterjoin%
\definecolor{currentfill}{rgb}{1.000000,0.498039,0.054902}%
\pgfsetfillcolor{currentfill}%
\pgfsetlinewidth{0.000000pt}%
\definecolor{currentstroke}{rgb}{0.000000,0.000000,0.000000}%
\pgfsetstrokecolor{currentstroke}%
\pgfsetstrokeopacity{0.000000}%
\pgfsetdash{}{0pt}%
\pgfpathmoveto{\pgfqpoint{5.487010in}{1.033583in}}%
\pgfpathlineto{\pgfqpoint{5.946170in}{1.033583in}}%
\pgfpathlineto{\pgfqpoint{5.946170in}{2.807857in}}%
\pgfpathlineto{\pgfqpoint{5.487010in}{2.807857in}}%
\pgfpathclose%
\pgfusepath{fill}%
\end{pgfscope}%
\begin{pgfscope}%
\pgfsetbuttcap%
\pgfsetroundjoin%
\definecolor{currentfill}{rgb}{0.000000,0.000000,0.000000}%
\pgfsetfillcolor{currentfill}%
\pgfsetlinewidth{0.803000pt}%
\definecolor{currentstroke}{rgb}{0.000000,0.000000,0.000000}%
\pgfsetstrokecolor{currentstroke}%
\pgfsetdash{}{0pt}%
\pgfsys@defobject{currentmarker}{\pgfqpoint{0.000000in}{-0.048611in}}{\pgfqpoint{0.000000in}{0.000000in}}{%
\pgfpathmoveto{\pgfqpoint{0.000000in}{0.000000in}}%
\pgfpathlineto{\pgfqpoint{0.000000in}{-0.048611in}}%
\pgfusepath{stroke,fill}%
}%
\begin{pgfscope}%
\pgfsys@transformshift{1.405583in}{1.033583in}%
\pgfsys@useobject{currentmarker}{}%
\end{pgfscope}%
\end{pgfscope}%
\begin{pgfscope}%
\definecolor{textcolor}{rgb}{0.000000,0.000000,0.000000}%
\pgfsetstrokecolor{textcolor}%
\pgfsetfillcolor{textcolor}%
\pgftext[x=1.091558in, y=0.179507in, left, base,rotate=45.000000]{\color{textcolor}\sffamily\fontsize{10.000000}{12.000000}\selectfont Pix3d(no aug)}%
\end{pgfscope}%
\begin{pgfscope}%
\pgfsetbuttcap%
\pgfsetroundjoin%
\definecolor{currentfill}{rgb}{0.000000,0.000000,0.000000}%
\pgfsetfillcolor{currentfill}%
\pgfsetlinewidth{0.803000pt}%
\definecolor{currentstroke}{rgb}{0.000000,0.000000,0.000000}%
\pgfsetstrokecolor{currentstroke}%
\pgfsetdash{}{0pt}%
\pgfsys@defobject{currentmarker}{\pgfqpoint{0.000000in}{-0.048611in}}{\pgfqpoint{0.000000in}{0.000000in}}{%
\pgfpathmoveto{\pgfqpoint{0.000000in}{0.000000in}}%
\pgfpathlineto{\pgfqpoint{0.000000in}{-0.048611in}}%
\pgfusepath{stroke,fill}%
}%
\begin{pgfscope}%
\pgfsys@transformshift{2.425940in}{1.033583in}%
\pgfsys@useobject{currentmarker}{}%
\end{pgfscope}%
\end{pgfscope}%
\begin{pgfscope}%
\definecolor{textcolor}{rgb}{0.000000,0.000000,0.000000}%
\pgfsetstrokecolor{textcolor}%
\pgfsetfillcolor{textcolor}%
\pgftext[x=2.319387in, y=0.594451in, left, base,rotate=45.000000]{\color{textcolor}\sffamily\fontsize{10.000000}{12.000000}\selectfont Pix3d}%
\end{pgfscope}%
\begin{pgfscope}%
\pgfsetbuttcap%
\pgfsetroundjoin%
\definecolor{currentfill}{rgb}{0.000000,0.000000,0.000000}%
\pgfsetfillcolor{currentfill}%
\pgfsetlinewidth{0.803000pt}%
\definecolor{currentstroke}{rgb}{0.000000,0.000000,0.000000}%
\pgfsetstrokecolor{currentstroke}%
\pgfsetdash{}{0pt}%
\pgfsys@defobject{currentmarker}{\pgfqpoint{0.000000in}{-0.048611in}}{\pgfqpoint{0.000000in}{0.000000in}}{%
\pgfpathmoveto{\pgfqpoint{0.000000in}{0.000000in}}%
\pgfpathlineto{\pgfqpoint{0.000000in}{-0.048611in}}%
\pgfusepath{stroke,fill}%
}%
\begin{pgfscope}%
\pgfsys@transformshift{3.446297in}{1.033583in}%
\pgfsys@useobject{currentmarker}{}%
\end{pgfscope}%
\end{pgfscope}%
\begin{pgfscope}%
\definecolor{textcolor}{rgb}{0.000000,0.000000,0.000000}%
\pgfsetstrokecolor{textcolor}%
\pgfsetfillcolor{textcolor}%
\pgftext[x=3.102636in, y=0.123161in, left, base,rotate=45.000000]{\color{textcolor}\sffamily\fontsize{10.000000}{12.000000}\selectfont s2r\_v1(no aug)}%
\end{pgfscope}%
\begin{pgfscope}%
\pgfsetbuttcap%
\pgfsetroundjoin%
\definecolor{currentfill}{rgb}{0.000000,0.000000,0.000000}%
\pgfsetfillcolor{currentfill}%
\pgfsetlinewidth{0.803000pt}%
\definecolor{currentstroke}{rgb}{0.000000,0.000000,0.000000}%
\pgfsetstrokecolor{currentstroke}%
\pgfsetdash{}{0pt}%
\pgfsys@defobject{currentmarker}{\pgfqpoint{0.000000in}{-0.048611in}}{\pgfqpoint{0.000000in}{0.000000in}}{%
\pgfpathmoveto{\pgfqpoint{0.000000in}{0.000000in}}%
\pgfpathlineto{\pgfqpoint{0.000000in}{-0.048611in}}%
\pgfusepath{stroke,fill}%
}%
\begin{pgfscope}%
\pgfsys@transformshift{4.466653in}{1.033583in}%
\pgfsys@useobject{currentmarker}{}%
\end{pgfscope}%
\end{pgfscope}%
\begin{pgfscope}%
\definecolor{textcolor}{rgb}{0.000000,0.000000,0.000000}%
\pgfsetstrokecolor{textcolor}%
\pgfsetfillcolor{textcolor}%
\pgftext[x=4.329649in, y=0.539736in, left, base,rotate=45.000000]{\color{textcolor}\sffamily\fontsize{10.000000}{12.000000}\selectfont s2r\_v1}%
\end{pgfscope}%
\begin{pgfscope}%
\pgfsetbuttcap%
\pgfsetroundjoin%
\definecolor{currentfill}{rgb}{0.000000,0.000000,0.000000}%
\pgfsetfillcolor{currentfill}%
\pgfsetlinewidth{0.803000pt}%
\definecolor{currentstroke}{rgb}{0.000000,0.000000,0.000000}%
\pgfsetstrokecolor{currentstroke}%
\pgfsetdash{}{0pt}%
\pgfsys@defobject{currentmarker}{\pgfqpoint{0.000000in}{-0.048611in}}{\pgfqpoint{0.000000in}{0.000000in}}{%
\pgfpathmoveto{\pgfqpoint{0.000000in}{0.000000in}}%
\pgfpathlineto{\pgfqpoint{0.000000in}{-0.048611in}}%
\pgfusepath{stroke,fill}%
}%
\begin{pgfscope}%
\pgfsys@transformshift{5.487010in}{1.033583in}%
\pgfsys@useobject{currentmarker}{}%
\end{pgfscope}%
\end{pgfscope}%
\begin{pgfscope}%
\definecolor{textcolor}{rgb}{0.000000,0.000000,0.000000}%
\pgfsetstrokecolor{textcolor}%
\pgfsetfillcolor{textcolor}%
\pgftext[x=5.350006in, y=0.539736in, left, base,rotate=45.000000]{\color{textcolor}\sffamily\fontsize{10.000000}{12.000000}\selectfont s2r\_v2}%
\end{pgfscope}%
\begin{pgfscope}%
\pgfsetbuttcap%
\pgfsetroundjoin%
\definecolor{currentfill}{rgb}{0.000000,0.000000,0.000000}%
\pgfsetfillcolor{currentfill}%
\pgfsetlinewidth{0.803000pt}%
\definecolor{currentstroke}{rgb}{0.000000,0.000000,0.000000}%
\pgfsetstrokecolor{currentstroke}%
\pgfsetdash{}{0pt}%
\pgfsys@defobject{currentmarker}{\pgfqpoint{-0.048611in}{0.000000in}}{\pgfqpoint{-0.000000in}{0.000000in}}{%
\pgfpathmoveto{\pgfqpoint{-0.000000in}{0.000000in}}%
\pgfpathlineto{\pgfqpoint{-0.048611in}{0.000000in}}%
\pgfusepath{stroke,fill}%
}%
\begin{pgfscope}%
\pgfsys@transformshift{0.696435in}{1.033583in}%
\pgfsys@useobject{currentmarker}{}%
\end{pgfscope}%
\end{pgfscope}%
\begin{pgfscope}%
\definecolor{textcolor}{rgb}{0.000000,0.000000,0.000000}%
\pgfsetstrokecolor{textcolor}%
\pgfsetfillcolor{textcolor}%
\pgftext[x=0.289968in, y=0.980822in, left, base]{\color{textcolor}\sffamily\fontsize{10.000000}{12.000000}\selectfont 0.00}%
\end{pgfscope}%
\begin{pgfscope}%
\pgfsetbuttcap%
\pgfsetroundjoin%
\definecolor{currentfill}{rgb}{0.000000,0.000000,0.000000}%
\pgfsetfillcolor{currentfill}%
\pgfsetlinewidth{0.803000pt}%
\definecolor{currentstroke}{rgb}{0.000000,0.000000,0.000000}%
\pgfsetstrokecolor{currentstroke}%
\pgfsetdash{}{0pt}%
\pgfsys@defobject{currentmarker}{\pgfqpoint{-0.048611in}{0.000000in}}{\pgfqpoint{-0.000000in}{0.000000in}}{%
\pgfpathmoveto{\pgfqpoint{-0.000000in}{0.000000in}}%
\pgfpathlineto{\pgfqpoint{-0.048611in}{0.000000in}}%
\pgfusepath{stroke,fill}%
}%
\begin{pgfscope}%
\pgfsys@transformshift{0.696435in}{1.511566in}%
\pgfsys@useobject{currentmarker}{}%
\end{pgfscope}%
\end{pgfscope}%
\begin{pgfscope}%
\definecolor{textcolor}{rgb}{0.000000,0.000000,0.000000}%
\pgfsetstrokecolor{textcolor}%
\pgfsetfillcolor{textcolor}%
\pgftext[x=0.289968in, y=1.458805in, left, base]{\color{textcolor}\sffamily\fontsize{10.000000}{12.000000}\selectfont 0.05}%
\end{pgfscope}%
\begin{pgfscope}%
\pgfsetbuttcap%
\pgfsetroundjoin%
\definecolor{currentfill}{rgb}{0.000000,0.000000,0.000000}%
\pgfsetfillcolor{currentfill}%
\pgfsetlinewidth{0.803000pt}%
\definecolor{currentstroke}{rgb}{0.000000,0.000000,0.000000}%
\pgfsetstrokecolor{currentstroke}%
\pgfsetdash{}{0pt}%
\pgfsys@defobject{currentmarker}{\pgfqpoint{-0.048611in}{0.000000in}}{\pgfqpoint{-0.000000in}{0.000000in}}{%
\pgfpathmoveto{\pgfqpoint{-0.000000in}{0.000000in}}%
\pgfpathlineto{\pgfqpoint{-0.048611in}{0.000000in}}%
\pgfusepath{stroke,fill}%
}%
\begin{pgfscope}%
\pgfsys@transformshift{0.696435in}{1.989550in}%
\pgfsys@useobject{currentmarker}{}%
\end{pgfscope}%
\end{pgfscope}%
\begin{pgfscope}%
\definecolor{textcolor}{rgb}{0.000000,0.000000,0.000000}%
\pgfsetstrokecolor{textcolor}%
\pgfsetfillcolor{textcolor}%
\pgftext[x=0.289968in, y=1.936788in, left, base]{\color{textcolor}\sffamily\fontsize{10.000000}{12.000000}\selectfont 0.10}%
\end{pgfscope}%
\begin{pgfscope}%
\pgfsetbuttcap%
\pgfsetroundjoin%
\definecolor{currentfill}{rgb}{0.000000,0.000000,0.000000}%
\pgfsetfillcolor{currentfill}%
\pgfsetlinewidth{0.803000pt}%
\definecolor{currentstroke}{rgb}{0.000000,0.000000,0.000000}%
\pgfsetstrokecolor{currentstroke}%
\pgfsetdash{}{0pt}%
\pgfsys@defobject{currentmarker}{\pgfqpoint{-0.048611in}{0.000000in}}{\pgfqpoint{-0.000000in}{0.000000in}}{%
\pgfpathmoveto{\pgfqpoint{-0.000000in}{0.000000in}}%
\pgfpathlineto{\pgfqpoint{-0.048611in}{0.000000in}}%
\pgfusepath{stroke,fill}%
}%
\begin{pgfscope}%
\pgfsys@transformshift{0.696435in}{2.467533in}%
\pgfsys@useobject{currentmarker}{}%
\end{pgfscope}%
\end{pgfscope}%
\begin{pgfscope}%
\definecolor{textcolor}{rgb}{0.000000,0.000000,0.000000}%
\pgfsetstrokecolor{textcolor}%
\pgfsetfillcolor{textcolor}%
\pgftext[x=0.289968in, y=2.414771in, left, base]{\color{textcolor}\sffamily\fontsize{10.000000}{12.000000}\selectfont 0.15}%
\end{pgfscope}%
\begin{pgfscope}%
\pgfsetbuttcap%
\pgfsetroundjoin%
\definecolor{currentfill}{rgb}{0.000000,0.000000,0.000000}%
\pgfsetfillcolor{currentfill}%
\pgfsetlinewidth{0.803000pt}%
\definecolor{currentstroke}{rgb}{0.000000,0.000000,0.000000}%
\pgfsetstrokecolor{currentstroke}%
\pgfsetdash{}{0pt}%
\pgfsys@defobject{currentmarker}{\pgfqpoint{-0.048611in}{0.000000in}}{\pgfqpoint{-0.000000in}{0.000000in}}{%
\pgfpathmoveto{\pgfqpoint{-0.000000in}{0.000000in}}%
\pgfpathlineto{\pgfqpoint{-0.048611in}{0.000000in}}%
\pgfusepath{stroke,fill}%
}%
\begin{pgfscope}%
\pgfsys@transformshift{0.696435in}{2.945516in}%
\pgfsys@useobject{currentmarker}{}%
\end{pgfscope}%
\end{pgfscope}%
\begin{pgfscope}%
\definecolor{textcolor}{rgb}{0.000000,0.000000,0.000000}%
\pgfsetstrokecolor{textcolor}%
\pgfsetfillcolor{textcolor}%
\pgftext[x=0.289968in, y=2.892754in, left, base]{\color{textcolor}\sffamily\fontsize{10.000000}{12.000000}\selectfont 0.20}%
\end{pgfscope}%
\begin{pgfscope}%
\pgfsetbuttcap%
\pgfsetroundjoin%
\definecolor{currentfill}{rgb}{0.000000,0.000000,0.000000}%
\pgfsetfillcolor{currentfill}%
\pgfsetlinewidth{0.803000pt}%
\definecolor{currentstroke}{rgb}{0.000000,0.000000,0.000000}%
\pgfsetstrokecolor{currentstroke}%
\pgfsetdash{}{0pt}%
\pgfsys@defobject{currentmarker}{\pgfqpoint{-0.048611in}{0.000000in}}{\pgfqpoint{-0.000000in}{0.000000in}}{%
\pgfpathmoveto{\pgfqpoint{-0.000000in}{0.000000in}}%
\pgfpathlineto{\pgfqpoint{-0.048611in}{0.000000in}}%
\pgfusepath{stroke,fill}%
}%
\begin{pgfscope}%
\pgfsys@transformshift{0.696435in}{3.423499in}%
\pgfsys@useobject{currentmarker}{}%
\end{pgfscope}%
\end{pgfscope}%
\begin{pgfscope}%
\definecolor{textcolor}{rgb}{0.000000,0.000000,0.000000}%
\pgfsetstrokecolor{textcolor}%
\pgfsetfillcolor{textcolor}%
\pgftext[x=0.289968in, y=3.370737in, left, base]{\color{textcolor}\sffamily\fontsize{10.000000}{12.000000}\selectfont 0.25}%
\end{pgfscope}%
\begin{pgfscope}%
\pgfsetbuttcap%
\pgfsetroundjoin%
\definecolor{currentfill}{rgb}{0.000000,0.000000,0.000000}%
\pgfsetfillcolor{currentfill}%
\pgfsetlinewidth{0.803000pt}%
\definecolor{currentstroke}{rgb}{0.000000,0.000000,0.000000}%
\pgfsetstrokecolor{currentstroke}%
\pgfsetdash{}{0pt}%
\pgfsys@defobject{currentmarker}{\pgfqpoint{-0.048611in}{0.000000in}}{\pgfqpoint{-0.000000in}{0.000000in}}{%
\pgfpathmoveto{\pgfqpoint{-0.000000in}{0.000000in}}%
\pgfpathlineto{\pgfqpoint{-0.048611in}{0.000000in}}%
\pgfusepath{stroke,fill}%
}%
\begin{pgfscope}%
\pgfsys@transformshift{0.696435in}{3.901482in}%
\pgfsys@useobject{currentmarker}{}%
\end{pgfscope}%
\end{pgfscope}%
\begin{pgfscope}%
\definecolor{textcolor}{rgb}{0.000000,0.000000,0.000000}%
\pgfsetstrokecolor{textcolor}%
\pgfsetfillcolor{textcolor}%
\pgftext[x=0.289968in, y=3.848721in, left, base]{\color{textcolor}\sffamily\fontsize{10.000000}{12.000000}\selectfont 0.30}%
\end{pgfscope}%
\begin{pgfscope}%
\definecolor{textcolor}{rgb}{0.000000,0.000000,0.000000}%
\pgfsetstrokecolor{textcolor}%
\pgfsetfillcolor{textcolor}%
\pgftext[x=0.234413in,y=2.703848in,,bottom,rotate=90.000000]{\color{textcolor}\sffamily\fontsize{10.000000}{12.000000}\selectfont IoU}%
\end{pgfscope}%
\begin{pgfscope}%
\pgfsetrectcap%
\pgfsetmiterjoin%
\pgfsetlinewidth{0.803000pt}%
\definecolor{currentstroke}{rgb}{0.000000,0.000000,0.000000}%
\pgfsetstrokecolor{currentstroke}%
\pgfsetdash{}{0pt}%
\pgfpathmoveto{\pgfqpoint{0.696435in}{1.033583in}}%
\pgfpathlineto{\pgfqpoint{0.696435in}{4.374112in}}%
\pgfusepath{stroke}%
\end{pgfscope}%
\begin{pgfscope}%
\pgfsetrectcap%
\pgfsetmiterjoin%
\pgfsetlinewidth{0.803000pt}%
\definecolor{currentstroke}{rgb}{0.000000,0.000000,0.000000}%
\pgfsetstrokecolor{currentstroke}%
\pgfsetdash{}{0pt}%
\pgfpathmoveto{\pgfqpoint{6.196158in}{1.033583in}}%
\pgfpathlineto{\pgfqpoint{6.196158in}{4.374112in}}%
\pgfusepath{stroke}%
\end{pgfscope}%
\begin{pgfscope}%
\pgfsetrectcap%
\pgfsetmiterjoin%
\pgfsetlinewidth{0.803000pt}%
\definecolor{currentstroke}{rgb}{0.000000,0.000000,0.000000}%
\pgfsetstrokecolor{currentstroke}%
\pgfsetdash{}{0pt}%
\pgfpathmoveto{\pgfqpoint{0.696435in}{1.033583in}}%
\pgfpathlineto{\pgfqpoint{6.196158in}{1.033583in}}%
\pgfusepath{stroke}%
\end{pgfscope}%
\begin{pgfscope}%
\pgfsetrectcap%
\pgfsetmiterjoin%
\pgfsetlinewidth{0.803000pt}%
\definecolor{currentstroke}{rgb}{0.000000,0.000000,0.000000}%
\pgfsetstrokecolor{currentstroke}%
\pgfsetdash{}{0pt}%
\pgfpathmoveto{\pgfqpoint{0.696435in}{4.374112in}}%
\pgfpathlineto{\pgfqpoint{6.196158in}{4.374112in}}%
\pgfusepath{stroke}%
\end{pgfscope}%
\begin{pgfscope}%
\definecolor{textcolor}{rgb}{0.000000,0.000000,0.000000}%
\pgfsetstrokecolor{textcolor}%
\pgfsetfillcolor{textcolor}%
\pgftext[x=1.176003in,y=4.044481in,,bottom]{\color{textcolor}\sffamily\fontsize{10.000000}{12.000000}\selectfont 0.3106}%
\end{pgfscope}%
\begin{pgfscope}%
\definecolor{textcolor}{rgb}{0.000000,0.000000,0.000000}%
\pgfsetstrokecolor{textcolor}%
\pgfsetfillcolor{textcolor}%
\pgftext[x=2.196360in,y=4.256706in,,bottom]{\color{textcolor}\sffamily\fontsize{10.000000}{12.000000}\selectfont 0.3328}%
\end{pgfscope}%
\begin{pgfscope}%
\definecolor{textcolor}{rgb}{0.000000,0.000000,0.000000}%
\pgfsetstrokecolor{textcolor}%
\pgfsetfillcolor{textcolor}%
\pgftext[x=3.216716in,y=2.737675in,,bottom]{\color{textcolor}\sffamily\fontsize{10.000000}{12.000000}\selectfont 0.1739}%
\end{pgfscope}%
\begin{pgfscope}%
\definecolor{textcolor}{rgb}{0.000000,0.000000,0.000000}%
\pgfsetstrokecolor{textcolor}%
\pgfsetfillcolor{textcolor}%
\pgftext[x=4.237073in,y=2.932692in,,bottom]{\color{textcolor}\sffamily\fontsize{10.000000}{12.000000}\selectfont 0.1943}%
\end{pgfscope}%
\begin{pgfscope}%
\definecolor{textcolor}{rgb}{0.000000,0.000000,0.000000}%
\pgfsetstrokecolor{textcolor}%
\pgfsetfillcolor{textcolor}%
\pgftext[x=5.257430in,y=2.819888in,,bottom]{\color{textcolor}\sffamily\fontsize{10.000000}{12.000000}\selectfont 0.1825}%
\end{pgfscope}%
\begin{pgfscope}%
\definecolor{textcolor}{rgb}{0.000000,0.000000,0.000000}%
\pgfsetstrokecolor{textcolor}%
\pgfsetfillcolor{textcolor}%
\pgftext[x=1.635164in,y=4.046393in,,bottom]{\color{textcolor}\sffamily\fontsize{10.000000}{12.000000}\selectfont 0.3108}%
\end{pgfscope}%
\begin{pgfscope}%
\definecolor{textcolor}{rgb}{0.000000,0.000000,0.000000}%
\pgfsetstrokecolor{textcolor}%
\pgfsetfillcolor{textcolor}%
\pgftext[x=2.655520in,y=4.163021in,,bottom]{\color{textcolor}\sffamily\fontsize{10.000000}{12.000000}\selectfont 0.323}%
\end{pgfscope}%
\begin{pgfscope}%
\definecolor{textcolor}{rgb}{0.000000,0.000000,0.000000}%
\pgfsetstrokecolor{textcolor}%
\pgfsetfillcolor{textcolor}%
\pgftext[x=3.675877in,y=2.307491in,,bottom]{\color{textcolor}\sffamily\fontsize{10.000000}{12.000000}\selectfont 0.1289}%
\end{pgfscope}%
\begin{pgfscope}%
\definecolor{textcolor}{rgb}{0.000000,0.000000,0.000000}%
\pgfsetstrokecolor{textcolor}%
\pgfsetfillcolor{textcolor}%
\pgftext[x=4.696233in,y=2.763486in,,bottom]{\color{textcolor}\sffamily\fontsize{10.000000}{12.000000}\selectfont 0.1766}%
\end{pgfscope}%
\begin{pgfscope}%
\definecolor{textcolor}{rgb}{0.000000,0.000000,0.000000}%
\pgfsetstrokecolor{textcolor}%
\pgfsetfillcolor{textcolor}%
\pgftext[x=5.716590in,y=2.849523in,,bottom]{\color{textcolor}\sffamily\fontsize{10.000000}{12.000000}\selectfont 0.1856}%
\end{pgfscope}%
\begin{pgfscope}%
\definecolor{textcolor}{rgb}{0.000000,0.000000,0.000000}%
\pgfsetstrokecolor{textcolor}%
\pgfsetfillcolor{textcolor}%
\pgftext[x=3.446297in,y=4.457445in,,base]{\color{textcolor}\sffamily\fontsize{12.000000}{14.400000}\selectfont Baseline comparison with different datasets}%
\end{pgfscope}%
\begin{pgfscope}%
\pgfsetbuttcap%
\pgfsetmiterjoin%
\definecolor{currentfill}{rgb}{1.000000,1.000000,1.000000}%
\pgfsetfillcolor{currentfill}%
\pgfsetfillopacity{0.800000}%
\pgfsetlinewidth{1.003750pt}%
\definecolor{currentstroke}{rgb}{0.800000,0.800000,0.800000}%
\pgfsetstrokecolor{currentstroke}%
\pgfsetstrokeopacity{0.800000}%
\pgfsetdash{}{0pt}%
\pgfpathmoveto{\pgfqpoint{6.293380in}{2.479157in}}%
\pgfpathlineto{\pgfqpoint{7.507545in}{2.479157in}}%
\pgfpathquadraticcurveto{\pgfqpoint{7.535323in}{2.479157in}}{\pgfqpoint{7.535323in}{2.506935in}}%
\pgfpathlineto{\pgfqpoint{7.535323in}{2.900760in}}%
\pgfpathquadraticcurveto{\pgfqpoint{7.535323in}{2.928538in}}{\pgfqpoint{7.507545in}{2.928538in}}%
\pgfpathlineto{\pgfqpoint{6.293380in}{2.928538in}}%
\pgfpathquadraticcurveto{\pgfqpoint{6.265602in}{2.928538in}}{\pgfqpoint{6.265602in}{2.900760in}}%
\pgfpathlineto{\pgfqpoint{6.265602in}{2.506935in}}%
\pgfpathquadraticcurveto{\pgfqpoint{6.265602in}{2.479157in}}{\pgfqpoint{6.293380in}{2.479157in}}%
\pgfpathclose%
\pgfusepath{stroke,fill}%
\end{pgfscope}%
\begin{pgfscope}%
\pgfsetbuttcap%
\pgfsetmiterjoin%
\definecolor{currentfill}{rgb}{0.121569,0.466667,0.705882}%
\pgfsetfillcolor{currentfill}%
\pgfsetlinewidth{0.000000pt}%
\definecolor{currentstroke}{rgb}{0.000000,0.000000,0.000000}%
\pgfsetstrokecolor{currentstroke}%
\pgfsetstrokeopacity{0.000000}%
\pgfsetdash{}{0pt}%
\pgfpathmoveto{\pgfqpoint{6.321158in}{2.767460in}}%
\pgfpathlineto{\pgfqpoint{6.598935in}{2.767460in}}%
\pgfpathlineto{\pgfqpoint{6.598935in}{2.864682in}}%
\pgfpathlineto{\pgfqpoint{6.321158in}{2.864682in}}%
\pgfpathclose%
\pgfusepath{fill}%
\end{pgfscope}%
\begin{pgfscope}%
\definecolor{textcolor}{rgb}{0.000000,0.000000,0.000000}%
\pgfsetstrokecolor{textcolor}%
\pgfsetfillcolor{textcolor}%
\pgftext[x=6.710047in,y=2.767460in,left,base]{\color{textcolor}\sffamily\fontsize{10.000000}{12.000000}\selectfont Pix2Vox++}%
\end{pgfscope}%
\begin{pgfscope}%
\pgfsetbuttcap%
\pgfsetmiterjoin%
\definecolor{currentfill}{rgb}{1.000000,0.498039,0.054902}%
\pgfsetfillcolor{currentfill}%
\pgfsetlinewidth{0.000000pt}%
\definecolor{currentstroke}{rgb}{0.000000,0.000000,0.000000}%
\pgfsetstrokecolor{currentstroke}%
\pgfsetstrokeopacity{0.000000}%
\pgfsetdash{}{0pt}%
\pgfpathmoveto{\pgfqpoint{6.321158in}{2.563602in}}%
\pgfpathlineto{\pgfqpoint{6.598935in}{2.563602in}}%
\pgfpathlineto{\pgfqpoint{6.598935in}{2.660825in}}%
\pgfpathlineto{\pgfqpoint{6.321158in}{2.660825in}}%
\pgfpathclose%
\pgfusepath{fill}%
\end{pgfscope}%
\begin{pgfscope}%
\definecolor{textcolor}{rgb}{0.000000,0.000000,0.000000}%
\pgfsetstrokecolor{textcolor}%
\pgfsetfillcolor{textcolor}%
\pgftext[x=6.710047in,y=2.563602in,left,base]{\color{textcolor}\sffamily\fontsize{10.000000}{12.000000}\selectfont Pix2Vox}%
\end{pgfscope}%
\end{pgfpicture}%
\makeatother%
\endgroup%
}
    \caption{Bar plot for the IoU for baseline models(Pix2Vox++ and Pix2Vox) trained on dataset mentioned in~\ref{sec:datasets}. }
    \label{fig:baseline1}
\end{figure}

%\todo{replace value of s2r\_v1(currently values are from s2r\_v4)}

\section{Fine Tuning}\label{sec:fine-tuning}
Fine tuning or Transfer Learning is a common way of domain adaptation.
For this experiment, the model is first trained on synthetic dataset and then used as a pre-trained model to be fine tuned using real dataset.

In figure~\ref{fig:finetuning1}, we have a comparison of \gls{iou}  with pure real and pure synthetic dataset, followed by fine-tuning the models with real dataset.
The core comparison is between real data and fine-tuned model.
Models are pre-trained with 2 versions of \gls{free} as mentioned in~\ref{sec:datasets}.
It is noticed that for Version 1, there is an increment of 1.36\% and 1.55\% on pix2vox++ and pix2vox respectively.
For Version 2, an increment of 2.41\% on pix2vox++, but decrement of 1.01\% on pix2vox model.

\begin{figure}
    \centering
    \resizebox{\textwidth}{!}{%% Creator: Matplotlib, PGF backend
%%
%% To include the figure in your LaTeX document, write
%%   \input{<filename>.pgf}
%%
%% Make sure the required packages are loaded in your preamble
%%   \usepackage{pgf}
%%
%% Figures using additional raster images can only be included by \input if
%% they are in the same directory as the main LaTeX file. For loading figures
%% from other directories you can use the `import` package
%%   \usepackage{import}
%%
%% and then include the figures with
%%   \import{<path to file>}{<filename>.pgf}
%%
%% Matplotlib used the following preamble
%%   \usepackage{fontspec}
%%   \setmainfont{DejaVuSerif.ttf}[Path=\detokenize{/Users/apple/opt/anaconda3/envs/kaolin/lib/python3.7/site-packages/matplotlib/mpl-data/fonts/ttf/}]
%%   \setsansfont{DejaVuSans.ttf}[Path=\detokenize{/Users/apple/opt/anaconda3/envs/kaolin/lib/python3.7/site-packages/matplotlib/mpl-data/fonts/ttf/}]
%%   \setmonofont{DejaVuSansMono.ttf}[Path=\detokenize{/Users/apple/opt/anaconda3/envs/kaolin/lib/python3.7/site-packages/matplotlib/mpl-data/fonts/ttf/}]
%%
\begingroup%
\makeatletter%
\begin{pgfpicture}%
\pgfpathrectangle{\pgfpointorigin}{\pgfqpoint{7.635323in}{4.684187in}}%
\pgfusepath{use as bounding box, clip}%
\begin{pgfscope}%
\pgfsetbuttcap%
\pgfsetmiterjoin%
\definecolor{currentfill}{rgb}{1.000000,1.000000,1.000000}%
\pgfsetfillcolor{currentfill}%
\pgfsetlinewidth{0.000000pt}%
\definecolor{currentstroke}{rgb}{1.000000,1.000000,1.000000}%
\pgfsetstrokecolor{currentstroke}%
\pgfsetdash{}{0pt}%
\pgfpathmoveto{\pgfqpoint{0.000000in}{0.000000in}}%
\pgfpathlineto{\pgfqpoint{7.635323in}{0.000000in}}%
\pgfpathlineto{\pgfqpoint{7.635323in}{4.684187in}}%
\pgfpathlineto{\pgfqpoint{0.000000in}{4.684187in}}%
\pgfpathclose%
\pgfusepath{fill}%
\end{pgfscope}%
\begin{pgfscope}%
\pgfsetbuttcap%
\pgfsetmiterjoin%
\definecolor{currentfill}{rgb}{1.000000,1.000000,1.000000}%
\pgfsetfillcolor{currentfill}%
\pgfsetlinewidth{0.000000pt}%
\definecolor{currentstroke}{rgb}{0.000000,0.000000,0.000000}%
\pgfsetstrokecolor{currentstroke}%
\pgfsetstrokeopacity{0.000000}%
\pgfsetdash{}{0pt}%
\pgfpathmoveto{\pgfqpoint{0.696435in}{0.973593in}}%
\pgfpathlineto{\pgfqpoint{6.196158in}{0.973593in}}%
\pgfpathlineto{\pgfqpoint{6.196158in}{4.374226in}}%
\pgfpathlineto{\pgfqpoint{0.696435in}{4.374226in}}%
\pgfpathclose%
\pgfusepath{fill}%
\end{pgfscope}%
\begin{pgfscope}%
\pgfpathrectangle{\pgfqpoint{0.696435in}{0.973593in}}{\pgfqpoint{5.499722in}{3.400633in}}%
\pgfusepath{clip}%
\pgfsetbuttcap%
\pgfsetmiterjoin%
\definecolor{currentfill}{rgb}{0.121569,0.466667,0.705882}%
\pgfsetfillcolor{currentfill}%
\pgfsetlinewidth{0.000000pt}%
\definecolor{currentstroke}{rgb}{0.000000,0.000000,0.000000}%
\pgfsetstrokecolor{currentstroke}%
\pgfsetstrokeopacity{0.000000}%
\pgfsetdash{}{0pt}%
\pgfpathmoveto{\pgfqpoint{0.946423in}{0.973593in}}%
\pgfpathlineto{\pgfqpoint{1.405583in}{0.973593in}}%
\pgfpathlineto{\pgfqpoint{1.405583in}{3.993595in}}%
\pgfpathlineto{\pgfqpoint{0.946423in}{3.993595in}}%
\pgfpathclose%
\pgfusepath{fill}%
\end{pgfscope}%
\begin{pgfscope}%
\pgfpathrectangle{\pgfqpoint{0.696435in}{0.973593in}}{\pgfqpoint{5.499722in}{3.400633in}}%
\pgfusepath{clip}%
\pgfsetbuttcap%
\pgfsetmiterjoin%
\definecolor{currentfill}{rgb}{0.121569,0.466667,0.705882}%
\pgfsetfillcolor{currentfill}%
\pgfsetlinewidth{0.000000pt}%
\definecolor{currentstroke}{rgb}{0.000000,0.000000,0.000000}%
\pgfsetstrokecolor{currentstroke}%
\pgfsetstrokeopacity{0.000000}%
\pgfsetdash{}{0pt}%
\pgfpathmoveto{\pgfqpoint{1.966779in}{0.973593in}}%
\pgfpathlineto{\pgfqpoint{2.425940in}{0.973593in}}%
\pgfpathlineto{\pgfqpoint{2.425940in}{2.736773in}}%
\pgfpathlineto{\pgfqpoint{1.966779in}{2.736773in}}%
\pgfpathclose%
\pgfusepath{fill}%
\end{pgfscope}%
\begin{pgfscope}%
\pgfpathrectangle{\pgfqpoint{0.696435in}{0.973593in}}{\pgfqpoint{5.499722in}{3.400633in}}%
\pgfusepath{clip}%
\pgfsetbuttcap%
\pgfsetmiterjoin%
\definecolor{currentfill}{rgb}{0.121569,0.466667,0.705882}%
\pgfsetfillcolor{currentfill}%
\pgfsetlinewidth{0.000000pt}%
\definecolor{currentstroke}{rgb}{0.000000,0.000000,0.000000}%
\pgfsetstrokecolor{currentstroke}%
\pgfsetstrokeopacity{0.000000}%
\pgfsetdash{}{0pt}%
\pgfpathmoveto{\pgfqpoint{2.987136in}{0.973593in}}%
\pgfpathlineto{\pgfqpoint{3.446297in}{0.973593in}}%
\pgfpathlineto{\pgfqpoint{3.446297in}{2.629694in}}%
\pgfpathlineto{\pgfqpoint{2.987136in}{2.629694in}}%
\pgfpathclose%
\pgfusepath{fill}%
\end{pgfscope}%
\begin{pgfscope}%
\pgfpathrectangle{\pgfqpoint{0.696435in}{0.973593in}}{\pgfqpoint{5.499722in}{3.400633in}}%
\pgfusepath{clip}%
\pgfsetbuttcap%
\pgfsetmiterjoin%
\definecolor{currentfill}{rgb}{0.121569,0.466667,0.705882}%
\pgfsetfillcolor{currentfill}%
\pgfsetlinewidth{0.000000pt}%
\definecolor{currentstroke}{rgb}{0.000000,0.000000,0.000000}%
\pgfsetstrokecolor{currentstroke}%
\pgfsetstrokeopacity{0.000000}%
\pgfsetdash{}{0pt}%
\pgfpathmoveto{\pgfqpoint{4.007493in}{0.973593in}}%
\pgfpathlineto{\pgfqpoint{4.466653in}{0.973593in}}%
\pgfpathlineto{\pgfqpoint{4.466653in}{4.117008in}}%
\pgfpathlineto{\pgfqpoint{4.007493in}{4.117008in}}%
\pgfpathclose%
\pgfusepath{fill}%
\end{pgfscope}%
\begin{pgfscope}%
\pgfpathrectangle{\pgfqpoint{0.696435in}{0.973593in}}{\pgfqpoint{5.499722in}{3.400633in}}%
\pgfusepath{clip}%
\pgfsetbuttcap%
\pgfsetmiterjoin%
\definecolor{currentfill}{rgb}{0.121569,0.466667,0.705882}%
\pgfsetfillcolor{currentfill}%
\pgfsetlinewidth{0.000000pt}%
\definecolor{currentstroke}{rgb}{0.000000,0.000000,0.000000}%
\pgfsetstrokecolor{currentstroke}%
\pgfsetstrokeopacity{0.000000}%
\pgfsetdash{}{0pt}%
\pgfpathmoveto{\pgfqpoint{5.027849in}{0.973593in}}%
\pgfpathlineto{\pgfqpoint{5.487010in}{0.973593in}}%
\pgfpathlineto{\pgfqpoint{5.487010in}{4.212291in}}%
\pgfpathlineto{\pgfqpoint{5.027849in}{4.212291in}}%
\pgfpathclose%
\pgfusepath{fill}%
\end{pgfscope}%
\begin{pgfscope}%
\pgfpathrectangle{\pgfqpoint{0.696435in}{0.973593in}}{\pgfqpoint{5.499722in}{3.400633in}}%
\pgfusepath{clip}%
\pgfsetbuttcap%
\pgfsetmiterjoin%
\definecolor{currentfill}{rgb}{1.000000,0.498039,0.054902}%
\pgfsetfillcolor{currentfill}%
\pgfsetlinewidth{0.000000pt}%
\definecolor{currentstroke}{rgb}{0.000000,0.000000,0.000000}%
\pgfsetstrokecolor{currentstroke}%
\pgfsetstrokeopacity{0.000000}%
\pgfsetdash{}{0pt}%
\pgfpathmoveto{\pgfqpoint{1.405583in}{0.973593in}}%
\pgfpathlineto{\pgfqpoint{1.864744in}{0.973593in}}%
\pgfpathlineto{\pgfqpoint{1.864744in}{3.904665in}}%
\pgfpathlineto{\pgfqpoint{1.405583in}{3.904665in}}%
\pgfpathclose%
\pgfusepath{fill}%
\end{pgfscope}%
\begin{pgfscope}%
\pgfpathrectangle{\pgfqpoint{0.696435in}{0.973593in}}{\pgfqpoint{5.499722in}{3.400633in}}%
\pgfusepath{clip}%
\pgfsetbuttcap%
\pgfsetmiterjoin%
\definecolor{currentfill}{rgb}{1.000000,0.498039,0.054902}%
\pgfsetfillcolor{currentfill}%
\pgfsetlinewidth{0.000000pt}%
\definecolor{currentstroke}{rgb}{0.000000,0.000000,0.000000}%
\pgfsetstrokecolor{currentstroke}%
\pgfsetstrokeopacity{0.000000}%
\pgfsetdash{}{0pt}%
\pgfpathmoveto{\pgfqpoint{2.425940in}{0.973593in}}%
\pgfpathlineto{\pgfqpoint{2.885100in}{0.973593in}}%
\pgfpathlineto{\pgfqpoint{2.885100in}{2.576154in}}%
\pgfpathlineto{\pgfqpoint{2.425940in}{2.576154in}}%
\pgfpathclose%
\pgfusepath{fill}%
\end{pgfscope}%
\begin{pgfscope}%
\pgfpathrectangle{\pgfqpoint{0.696435in}{0.973593in}}{\pgfqpoint{5.499722in}{3.400633in}}%
\pgfusepath{clip}%
\pgfsetbuttcap%
\pgfsetmiterjoin%
\definecolor{currentfill}{rgb}{1.000000,0.498039,0.054902}%
\pgfsetfillcolor{currentfill}%
\pgfsetlinewidth{0.000000pt}%
\definecolor{currentstroke}{rgb}{0.000000,0.000000,0.000000}%
\pgfsetstrokecolor{currentstroke}%
\pgfsetstrokeopacity{0.000000}%
\pgfsetdash{}{0pt}%
\pgfpathmoveto{\pgfqpoint{3.446297in}{0.973593in}}%
\pgfpathlineto{\pgfqpoint{3.905457in}{0.973593in}}%
\pgfpathlineto{\pgfqpoint{3.905457in}{2.657825in}}%
\pgfpathlineto{\pgfqpoint{3.446297in}{2.657825in}}%
\pgfpathclose%
\pgfusepath{fill}%
\end{pgfscope}%
\begin{pgfscope}%
\pgfpathrectangle{\pgfqpoint{0.696435in}{0.973593in}}{\pgfqpoint{5.499722in}{3.400633in}}%
\pgfusepath{clip}%
\pgfsetbuttcap%
\pgfsetmiterjoin%
\definecolor{currentfill}{rgb}{1.000000,0.498039,0.054902}%
\pgfsetfillcolor{currentfill}%
\pgfsetlinewidth{0.000000pt}%
\definecolor{currentstroke}{rgb}{0.000000,0.000000,0.000000}%
\pgfsetstrokecolor{currentstroke}%
\pgfsetstrokeopacity{0.000000}%
\pgfsetdash{}{0pt}%
\pgfpathmoveto{\pgfqpoint{4.466653in}{0.973593in}}%
\pgfpathlineto{\pgfqpoint{4.925814in}{0.973593in}}%
\pgfpathlineto{\pgfqpoint{4.925814in}{4.045320in}}%
\pgfpathlineto{\pgfqpoint{4.466653in}{4.045320in}}%
\pgfpathclose%
\pgfusepath{fill}%
\end{pgfscope}%
\begin{pgfscope}%
\pgfpathrectangle{\pgfqpoint{0.696435in}{0.973593in}}{\pgfqpoint{5.499722in}{3.400633in}}%
\pgfusepath{clip}%
\pgfsetbuttcap%
\pgfsetmiterjoin%
\definecolor{currentfill}{rgb}{1.000000,0.498039,0.054902}%
\pgfsetfillcolor{currentfill}%
\pgfsetlinewidth{0.000000pt}%
\definecolor{currentstroke}{rgb}{0.000000,0.000000,0.000000}%
\pgfsetstrokecolor{currentstroke}%
\pgfsetstrokeopacity{0.000000}%
\pgfsetdash{}{0pt}%
\pgfpathmoveto{\pgfqpoint{5.487010in}{0.973593in}}%
\pgfpathlineto{\pgfqpoint{5.946170in}{0.973593in}}%
\pgfpathlineto{\pgfqpoint{5.946170in}{3.813012in}}%
\pgfpathlineto{\pgfqpoint{5.487010in}{3.813012in}}%
\pgfpathclose%
\pgfusepath{fill}%
\end{pgfscope}%
\begin{pgfscope}%
\pgfsetbuttcap%
\pgfsetroundjoin%
\definecolor{currentfill}{rgb}{0.000000,0.000000,0.000000}%
\pgfsetfillcolor{currentfill}%
\pgfsetlinewidth{0.803000pt}%
\definecolor{currentstroke}{rgb}{0.000000,0.000000,0.000000}%
\pgfsetstrokecolor{currentstroke}%
\pgfsetdash{}{0pt}%
\pgfsys@defobject{currentmarker}{\pgfqpoint{0.000000in}{-0.048611in}}{\pgfqpoint{0.000000in}{0.000000in}}{%
\pgfpathmoveto{\pgfqpoint{0.000000in}{0.000000in}}%
\pgfpathlineto{\pgfqpoint{0.000000in}{-0.048611in}}%
\pgfusepath{stroke,fill}%
}%
\begin{pgfscope}%
\pgfsys@transformshift{1.405583in}{0.973593in}%
\pgfsys@useobject{currentmarker}{}%
\end{pgfscope}%
\end{pgfscope}%
\begin{pgfscope}%
\definecolor{textcolor}{rgb}{0.000000,0.000000,0.000000}%
\pgfsetstrokecolor{textcolor}%
\pgfsetfillcolor{textcolor}%
\pgftext[x=1.292389in, y=0.521178in, left, base,rotate=45.000000]{\color{textcolor}\sffamily\fontsize{10.000000}{12.000000}\selectfont Pix3D}%
\end{pgfscope}%
\begin{pgfscope}%
\pgfsetbuttcap%
\pgfsetroundjoin%
\definecolor{currentfill}{rgb}{0.000000,0.000000,0.000000}%
\pgfsetfillcolor{currentfill}%
\pgfsetlinewidth{0.803000pt}%
\definecolor{currentstroke}{rgb}{0.000000,0.000000,0.000000}%
\pgfsetstrokecolor{currentstroke}%
\pgfsetdash{}{0pt}%
\pgfsys@defobject{currentmarker}{\pgfqpoint{0.000000in}{-0.048611in}}{\pgfqpoint{0.000000in}{0.000000in}}{%
\pgfpathmoveto{\pgfqpoint{0.000000in}{0.000000in}}%
\pgfpathlineto{\pgfqpoint{0.000000in}{-0.048611in}}%
\pgfusepath{stroke,fill}%
}%
\begin{pgfscope}%
\pgfsys@transformshift{2.425940in}{0.973593in}%
\pgfsys@useobject{currentmarker}{}%
\end{pgfscope}%
\end{pgfscope}%
\begin{pgfscope}%
\definecolor{textcolor}{rgb}{0.000000,0.000000,0.000000}%
\pgfsetstrokecolor{textcolor}%
\pgfsetfillcolor{textcolor}%
\pgftext[x=2.288936in, y=0.479746in, left, base,rotate=45.000000]{\color{textcolor}\sffamily\fontsize{10.000000}{12.000000}\selectfont s2r\_v1}%
\end{pgfscope}%
\begin{pgfscope}%
\pgfsetbuttcap%
\pgfsetroundjoin%
\definecolor{currentfill}{rgb}{0.000000,0.000000,0.000000}%
\pgfsetfillcolor{currentfill}%
\pgfsetlinewidth{0.803000pt}%
\definecolor{currentstroke}{rgb}{0.000000,0.000000,0.000000}%
\pgfsetstrokecolor{currentstroke}%
\pgfsetdash{}{0pt}%
\pgfsys@defobject{currentmarker}{\pgfqpoint{0.000000in}{-0.048611in}}{\pgfqpoint{0.000000in}{0.000000in}}{%
\pgfpathmoveto{\pgfqpoint{0.000000in}{0.000000in}}%
\pgfpathlineto{\pgfqpoint{0.000000in}{-0.048611in}}%
\pgfusepath{stroke,fill}%
}%
\begin{pgfscope}%
\pgfsys@transformshift{3.446297in}{0.973593in}%
\pgfsys@useobject{currentmarker}{}%
\end{pgfscope}%
\end{pgfscope}%
\begin{pgfscope}%
\definecolor{textcolor}{rgb}{0.000000,0.000000,0.000000}%
\pgfsetstrokecolor{textcolor}%
\pgfsetfillcolor{textcolor}%
\pgftext[x=3.309293in, y=0.479746in, left, base,rotate=45.000000]{\color{textcolor}\sffamily\fontsize{10.000000}{12.000000}\selectfont s2r\_v2}%
\end{pgfscope}%
\begin{pgfscope}%
\pgfsetbuttcap%
\pgfsetroundjoin%
\definecolor{currentfill}{rgb}{0.000000,0.000000,0.000000}%
\pgfsetfillcolor{currentfill}%
\pgfsetlinewidth{0.803000pt}%
\definecolor{currentstroke}{rgb}{0.000000,0.000000,0.000000}%
\pgfsetstrokecolor{currentstroke}%
\pgfsetdash{}{0pt}%
\pgfsys@defobject{currentmarker}{\pgfqpoint{0.000000in}{-0.048611in}}{\pgfqpoint{0.000000in}{0.000000in}}{%
\pgfpathmoveto{\pgfqpoint{0.000000in}{0.000000in}}%
\pgfpathlineto{\pgfqpoint{0.000000in}{-0.048611in}}%
\pgfusepath{stroke,fill}%
}%
\begin{pgfscope}%
\pgfsys@transformshift{4.466653in}{0.973593in}%
\pgfsys@useobject{currentmarker}{}%
\end{pgfscope}%
\end{pgfscope}%
\begin{pgfscope}%
\definecolor{textcolor}{rgb}{0.000000,0.000000,0.000000}%
\pgfsetstrokecolor{textcolor}%
\pgfsetfillcolor{textcolor}%
\pgftext[x=4.153084in, y=0.123161in, left, base,rotate=45.000000]{\color{textcolor}\sffamily\fontsize{10.000000}{12.000000}\selectfont s2r\_v1+pix3d}%
\end{pgfscope}%
\begin{pgfscope}%
\pgfsetbuttcap%
\pgfsetroundjoin%
\definecolor{currentfill}{rgb}{0.000000,0.000000,0.000000}%
\pgfsetfillcolor{currentfill}%
\pgfsetlinewidth{0.803000pt}%
\definecolor{currentstroke}{rgb}{0.000000,0.000000,0.000000}%
\pgfsetstrokecolor{currentstroke}%
\pgfsetdash{}{0pt}%
\pgfsys@defobject{currentmarker}{\pgfqpoint{0.000000in}{-0.048611in}}{\pgfqpoint{0.000000in}{0.000000in}}{%
\pgfpathmoveto{\pgfqpoint{0.000000in}{0.000000in}}%
\pgfpathlineto{\pgfqpoint{0.000000in}{-0.048611in}}%
\pgfusepath{stroke,fill}%
}%
\begin{pgfscope}%
\pgfsys@transformshift{5.487010in}{0.973593in}%
\pgfsys@useobject{currentmarker}{}%
\end{pgfscope}%
\end{pgfscope}%
\begin{pgfscope}%
\definecolor{textcolor}{rgb}{0.000000,0.000000,0.000000}%
\pgfsetstrokecolor{textcolor}%
\pgfsetfillcolor{textcolor}%
\pgftext[x=5.173440in, y=0.123161in, left, base,rotate=45.000000]{\color{textcolor}\sffamily\fontsize{10.000000}{12.000000}\selectfont s2r\_v2+pix3d}%
\end{pgfscope}%
\begin{pgfscope}%
\pgfsetbuttcap%
\pgfsetroundjoin%
\definecolor{currentfill}{rgb}{0.000000,0.000000,0.000000}%
\pgfsetfillcolor{currentfill}%
\pgfsetlinewidth{0.803000pt}%
\definecolor{currentstroke}{rgb}{0.000000,0.000000,0.000000}%
\pgfsetstrokecolor{currentstroke}%
\pgfsetdash{}{0pt}%
\pgfsys@defobject{currentmarker}{\pgfqpoint{-0.048611in}{0.000000in}}{\pgfqpoint{-0.000000in}{0.000000in}}{%
\pgfpathmoveto{\pgfqpoint{-0.000000in}{0.000000in}}%
\pgfpathlineto{\pgfqpoint{-0.048611in}{0.000000in}}%
\pgfusepath{stroke,fill}%
}%
\begin{pgfscope}%
\pgfsys@transformshift{0.696435in}{0.973593in}%
\pgfsys@useobject{currentmarker}{}%
\end{pgfscope}%
\end{pgfscope}%
\begin{pgfscope}%
\definecolor{textcolor}{rgb}{0.000000,0.000000,0.000000}%
\pgfsetstrokecolor{textcolor}%
\pgfsetfillcolor{textcolor}%
\pgftext[x=0.289968in, y=0.920832in, left, base]{\color{textcolor}\sffamily\fontsize{10.000000}{12.000000}\selectfont 0.00}%
\end{pgfscope}%
\begin{pgfscope}%
\pgfsetbuttcap%
\pgfsetroundjoin%
\definecolor{currentfill}{rgb}{0.000000,0.000000,0.000000}%
\pgfsetfillcolor{currentfill}%
\pgfsetlinewidth{0.803000pt}%
\definecolor{currentstroke}{rgb}{0.000000,0.000000,0.000000}%
\pgfsetstrokecolor{currentstroke}%
\pgfsetdash{}{0pt}%
\pgfsys@defobject{currentmarker}{\pgfqpoint{-0.048611in}{0.000000in}}{\pgfqpoint{-0.000000in}{0.000000in}}{%
\pgfpathmoveto{\pgfqpoint{-0.000000in}{0.000000in}}%
\pgfpathlineto{\pgfqpoint{-0.048611in}{0.000000in}}%
\pgfusepath{stroke,fill}%
}%
\begin{pgfscope}%
\pgfsys@transformshift{0.696435in}{1.427319in}%
\pgfsys@useobject{currentmarker}{}%
\end{pgfscope}%
\end{pgfscope}%
\begin{pgfscope}%
\definecolor{textcolor}{rgb}{0.000000,0.000000,0.000000}%
\pgfsetstrokecolor{textcolor}%
\pgfsetfillcolor{textcolor}%
\pgftext[x=0.289968in, y=1.374558in, left, base]{\color{textcolor}\sffamily\fontsize{10.000000}{12.000000}\selectfont 0.05}%
\end{pgfscope}%
\begin{pgfscope}%
\pgfsetbuttcap%
\pgfsetroundjoin%
\definecolor{currentfill}{rgb}{0.000000,0.000000,0.000000}%
\pgfsetfillcolor{currentfill}%
\pgfsetlinewidth{0.803000pt}%
\definecolor{currentstroke}{rgb}{0.000000,0.000000,0.000000}%
\pgfsetstrokecolor{currentstroke}%
\pgfsetdash{}{0pt}%
\pgfsys@defobject{currentmarker}{\pgfqpoint{-0.048611in}{0.000000in}}{\pgfqpoint{-0.000000in}{0.000000in}}{%
\pgfpathmoveto{\pgfqpoint{-0.000000in}{0.000000in}}%
\pgfpathlineto{\pgfqpoint{-0.048611in}{0.000000in}}%
\pgfusepath{stroke,fill}%
}%
\begin{pgfscope}%
\pgfsys@transformshift{0.696435in}{1.881046in}%
\pgfsys@useobject{currentmarker}{}%
\end{pgfscope}%
\end{pgfscope}%
\begin{pgfscope}%
\definecolor{textcolor}{rgb}{0.000000,0.000000,0.000000}%
\pgfsetstrokecolor{textcolor}%
\pgfsetfillcolor{textcolor}%
\pgftext[x=0.289968in, y=1.828284in, left, base]{\color{textcolor}\sffamily\fontsize{10.000000}{12.000000}\selectfont 0.10}%
\end{pgfscope}%
\begin{pgfscope}%
\pgfsetbuttcap%
\pgfsetroundjoin%
\definecolor{currentfill}{rgb}{0.000000,0.000000,0.000000}%
\pgfsetfillcolor{currentfill}%
\pgfsetlinewidth{0.803000pt}%
\definecolor{currentstroke}{rgb}{0.000000,0.000000,0.000000}%
\pgfsetstrokecolor{currentstroke}%
\pgfsetdash{}{0pt}%
\pgfsys@defobject{currentmarker}{\pgfqpoint{-0.048611in}{0.000000in}}{\pgfqpoint{-0.000000in}{0.000000in}}{%
\pgfpathmoveto{\pgfqpoint{-0.000000in}{0.000000in}}%
\pgfpathlineto{\pgfqpoint{-0.048611in}{0.000000in}}%
\pgfusepath{stroke,fill}%
}%
\begin{pgfscope}%
\pgfsys@transformshift{0.696435in}{2.334772in}%
\pgfsys@useobject{currentmarker}{}%
\end{pgfscope}%
\end{pgfscope}%
\begin{pgfscope}%
\definecolor{textcolor}{rgb}{0.000000,0.000000,0.000000}%
\pgfsetstrokecolor{textcolor}%
\pgfsetfillcolor{textcolor}%
\pgftext[x=0.289968in, y=2.282010in, left, base]{\color{textcolor}\sffamily\fontsize{10.000000}{12.000000}\selectfont 0.15}%
\end{pgfscope}%
\begin{pgfscope}%
\pgfsetbuttcap%
\pgfsetroundjoin%
\definecolor{currentfill}{rgb}{0.000000,0.000000,0.000000}%
\pgfsetfillcolor{currentfill}%
\pgfsetlinewidth{0.803000pt}%
\definecolor{currentstroke}{rgb}{0.000000,0.000000,0.000000}%
\pgfsetstrokecolor{currentstroke}%
\pgfsetdash{}{0pt}%
\pgfsys@defobject{currentmarker}{\pgfqpoint{-0.048611in}{0.000000in}}{\pgfqpoint{-0.000000in}{0.000000in}}{%
\pgfpathmoveto{\pgfqpoint{-0.000000in}{0.000000in}}%
\pgfpathlineto{\pgfqpoint{-0.048611in}{0.000000in}}%
\pgfusepath{stroke,fill}%
}%
\begin{pgfscope}%
\pgfsys@transformshift{0.696435in}{2.788498in}%
\pgfsys@useobject{currentmarker}{}%
\end{pgfscope}%
\end{pgfscope}%
\begin{pgfscope}%
\definecolor{textcolor}{rgb}{0.000000,0.000000,0.000000}%
\pgfsetstrokecolor{textcolor}%
\pgfsetfillcolor{textcolor}%
\pgftext[x=0.289968in, y=2.735737in, left, base]{\color{textcolor}\sffamily\fontsize{10.000000}{12.000000}\selectfont 0.20}%
\end{pgfscope}%
\begin{pgfscope}%
\pgfsetbuttcap%
\pgfsetroundjoin%
\definecolor{currentfill}{rgb}{0.000000,0.000000,0.000000}%
\pgfsetfillcolor{currentfill}%
\pgfsetlinewidth{0.803000pt}%
\definecolor{currentstroke}{rgb}{0.000000,0.000000,0.000000}%
\pgfsetstrokecolor{currentstroke}%
\pgfsetdash{}{0pt}%
\pgfsys@defobject{currentmarker}{\pgfqpoint{-0.048611in}{0.000000in}}{\pgfqpoint{-0.000000in}{0.000000in}}{%
\pgfpathmoveto{\pgfqpoint{-0.000000in}{0.000000in}}%
\pgfpathlineto{\pgfqpoint{-0.048611in}{0.000000in}}%
\pgfusepath{stroke,fill}%
}%
\begin{pgfscope}%
\pgfsys@transformshift{0.696435in}{3.242224in}%
\pgfsys@useobject{currentmarker}{}%
\end{pgfscope}%
\end{pgfscope}%
\begin{pgfscope}%
\definecolor{textcolor}{rgb}{0.000000,0.000000,0.000000}%
\pgfsetstrokecolor{textcolor}%
\pgfsetfillcolor{textcolor}%
\pgftext[x=0.289968in, y=3.189463in, left, base]{\color{textcolor}\sffamily\fontsize{10.000000}{12.000000}\selectfont 0.25}%
\end{pgfscope}%
\begin{pgfscope}%
\pgfsetbuttcap%
\pgfsetroundjoin%
\definecolor{currentfill}{rgb}{0.000000,0.000000,0.000000}%
\pgfsetfillcolor{currentfill}%
\pgfsetlinewidth{0.803000pt}%
\definecolor{currentstroke}{rgb}{0.000000,0.000000,0.000000}%
\pgfsetstrokecolor{currentstroke}%
\pgfsetdash{}{0pt}%
\pgfsys@defobject{currentmarker}{\pgfqpoint{-0.048611in}{0.000000in}}{\pgfqpoint{-0.000000in}{0.000000in}}{%
\pgfpathmoveto{\pgfqpoint{-0.000000in}{0.000000in}}%
\pgfpathlineto{\pgfqpoint{-0.048611in}{0.000000in}}%
\pgfusepath{stroke,fill}%
}%
\begin{pgfscope}%
\pgfsys@transformshift{0.696435in}{3.695951in}%
\pgfsys@useobject{currentmarker}{}%
\end{pgfscope}%
\end{pgfscope}%
\begin{pgfscope}%
\definecolor{textcolor}{rgb}{0.000000,0.000000,0.000000}%
\pgfsetstrokecolor{textcolor}%
\pgfsetfillcolor{textcolor}%
\pgftext[x=0.289968in, y=3.643189in, left, base]{\color{textcolor}\sffamily\fontsize{10.000000}{12.000000}\selectfont 0.30}%
\end{pgfscope}%
\begin{pgfscope}%
\pgfsetbuttcap%
\pgfsetroundjoin%
\definecolor{currentfill}{rgb}{0.000000,0.000000,0.000000}%
\pgfsetfillcolor{currentfill}%
\pgfsetlinewidth{0.803000pt}%
\definecolor{currentstroke}{rgb}{0.000000,0.000000,0.000000}%
\pgfsetstrokecolor{currentstroke}%
\pgfsetdash{}{0pt}%
\pgfsys@defobject{currentmarker}{\pgfqpoint{-0.048611in}{0.000000in}}{\pgfqpoint{-0.000000in}{0.000000in}}{%
\pgfpathmoveto{\pgfqpoint{-0.000000in}{0.000000in}}%
\pgfpathlineto{\pgfqpoint{-0.048611in}{0.000000in}}%
\pgfusepath{stroke,fill}%
}%
\begin{pgfscope}%
\pgfsys@transformshift{0.696435in}{4.149677in}%
\pgfsys@useobject{currentmarker}{}%
\end{pgfscope}%
\end{pgfscope}%
\begin{pgfscope}%
\definecolor{textcolor}{rgb}{0.000000,0.000000,0.000000}%
\pgfsetstrokecolor{textcolor}%
\pgfsetfillcolor{textcolor}%
\pgftext[x=0.289968in, y=4.096915in, left, base]{\color{textcolor}\sffamily\fontsize{10.000000}{12.000000}\selectfont 0.35}%
\end{pgfscope}%
\begin{pgfscope}%
\definecolor{textcolor}{rgb}{0.000000,0.000000,0.000000}%
\pgfsetstrokecolor{textcolor}%
\pgfsetfillcolor{textcolor}%
\pgftext[x=0.234413in,y=2.673910in,,bottom,rotate=90.000000]{\color{textcolor}\sffamily\fontsize{10.000000}{12.000000}\selectfont IoU}%
\end{pgfscope}%
\begin{pgfscope}%
\pgfsetrectcap%
\pgfsetmiterjoin%
\pgfsetlinewidth{0.803000pt}%
\definecolor{currentstroke}{rgb}{0.000000,0.000000,0.000000}%
\pgfsetstrokecolor{currentstroke}%
\pgfsetdash{}{0pt}%
\pgfpathmoveto{\pgfqpoint{0.696435in}{0.973593in}}%
\pgfpathlineto{\pgfqpoint{0.696435in}{4.374226in}}%
\pgfusepath{stroke}%
\end{pgfscope}%
\begin{pgfscope}%
\pgfsetrectcap%
\pgfsetmiterjoin%
\pgfsetlinewidth{0.803000pt}%
\definecolor{currentstroke}{rgb}{0.000000,0.000000,0.000000}%
\pgfsetstrokecolor{currentstroke}%
\pgfsetdash{}{0pt}%
\pgfpathmoveto{\pgfqpoint{6.196158in}{0.973593in}}%
\pgfpathlineto{\pgfqpoint{6.196158in}{4.374226in}}%
\pgfusepath{stroke}%
\end{pgfscope}%
\begin{pgfscope}%
\pgfsetrectcap%
\pgfsetmiterjoin%
\pgfsetlinewidth{0.803000pt}%
\definecolor{currentstroke}{rgb}{0.000000,0.000000,0.000000}%
\pgfsetstrokecolor{currentstroke}%
\pgfsetdash{}{0pt}%
\pgfpathmoveto{\pgfqpoint{0.696435in}{0.973593in}}%
\pgfpathlineto{\pgfqpoint{6.196158in}{0.973593in}}%
\pgfusepath{stroke}%
\end{pgfscope}%
\begin{pgfscope}%
\pgfsetrectcap%
\pgfsetmiterjoin%
\pgfsetlinewidth{0.803000pt}%
\definecolor{currentstroke}{rgb}{0.000000,0.000000,0.000000}%
\pgfsetstrokecolor{currentstroke}%
\pgfsetdash{}{0pt}%
\pgfpathmoveto{\pgfqpoint{0.696435in}{4.374226in}}%
\pgfpathlineto{\pgfqpoint{6.196158in}{4.374226in}}%
\pgfusepath{stroke}%
\end{pgfscope}%
\begin{pgfscope}%
\definecolor{textcolor}{rgb}{0.000000,0.000000,0.000000}%
\pgfsetstrokecolor{textcolor}%
\pgfsetfillcolor{textcolor}%
\pgftext[x=1.176003in,y=4.035262in,,bottom]{\color{textcolor}\sffamily\fontsize{10.000000}{12.000000}\selectfont 0.3328}%
\end{pgfscope}%
\begin{pgfscope}%
\definecolor{textcolor}{rgb}{0.000000,0.000000,0.000000}%
\pgfsetstrokecolor{textcolor}%
\pgfsetfillcolor{textcolor}%
\pgftext[x=2.196360in,y=2.778440in,,bottom]{\color{textcolor}\sffamily\fontsize{10.000000}{12.000000}\selectfont 0.1943}%
\end{pgfscope}%
\begin{pgfscope}%
\definecolor{textcolor}{rgb}{0.000000,0.000000,0.000000}%
\pgfsetstrokecolor{textcolor}%
\pgfsetfillcolor{textcolor}%
\pgftext[x=3.216716in,y=2.671361in,,bottom]{\color{textcolor}\sffamily\fontsize{10.000000}{12.000000}\selectfont 0.1825}%
\end{pgfscope}%
\begin{pgfscope}%
\definecolor{textcolor}{rgb}{0.000000,0.000000,0.000000}%
\pgfsetstrokecolor{textcolor}%
\pgfsetfillcolor{textcolor}%
\pgftext[x=4.237073in,y=4.158675in,,bottom]{\color{textcolor}\sffamily\fontsize{10.000000}{12.000000}\selectfont 0.3464}%
\end{pgfscope}%
\begin{pgfscope}%
\definecolor{textcolor}{rgb}{0.000000,0.000000,0.000000}%
\pgfsetstrokecolor{textcolor}%
\pgfsetfillcolor{textcolor}%
\pgftext[x=5.257430in,y=4.253958in,,bottom]{\color{textcolor}\sffamily\fontsize{10.000000}{12.000000}\selectfont 0.3569}%
\end{pgfscope}%
\begin{pgfscope}%
\definecolor{textcolor}{rgb}{0.000000,0.000000,0.000000}%
\pgfsetstrokecolor{textcolor}%
\pgfsetfillcolor{textcolor}%
\pgftext[x=1.635164in,y=3.946331in,,bottom]{\color{textcolor}\sffamily\fontsize{10.000000}{12.000000}\selectfont 0.323}%
\end{pgfscope}%
\begin{pgfscope}%
\definecolor{textcolor}{rgb}{0.000000,0.000000,0.000000}%
\pgfsetstrokecolor{textcolor}%
\pgfsetfillcolor{textcolor}%
\pgftext[x=2.655520in,y=2.617821in,,bottom]{\color{textcolor}\sffamily\fontsize{10.000000}{12.000000}\selectfont 0.1766}%
\end{pgfscope}%
\begin{pgfscope}%
\definecolor{textcolor}{rgb}{0.000000,0.000000,0.000000}%
\pgfsetstrokecolor{textcolor}%
\pgfsetfillcolor{textcolor}%
\pgftext[x=3.675877in,y=2.699492in,,bottom]{\color{textcolor}\sffamily\fontsize{10.000000}{12.000000}\selectfont 0.1856}%
\end{pgfscope}%
\begin{pgfscope}%
\definecolor{textcolor}{rgb}{0.000000,0.000000,0.000000}%
\pgfsetstrokecolor{textcolor}%
\pgfsetfillcolor{textcolor}%
\pgftext[x=4.696233in,y=4.086986in,,bottom]{\color{textcolor}\sffamily\fontsize{10.000000}{12.000000}\selectfont 0.3385}%
\end{pgfscope}%
\begin{pgfscope}%
\definecolor{textcolor}{rgb}{0.000000,0.000000,0.000000}%
\pgfsetstrokecolor{textcolor}%
\pgfsetfillcolor{textcolor}%
\pgftext[x=5.716590in,y=3.854679in,,bottom]{\color{textcolor}\sffamily\fontsize{10.000000}{12.000000}\selectfont 0.3129}%
\end{pgfscope}%
\begin{pgfscope}%
\definecolor{textcolor}{rgb}{0.000000,0.000000,0.000000}%
\pgfsetstrokecolor{textcolor}%
\pgfsetfillcolor{textcolor}%
\pgftext[x=3.446297in,y=4.457559in,,base]{\color{textcolor}\sffamily\fontsize{12.000000}{14.400000}\selectfont Baseline comparison with finetuning of different datasets}%
\end{pgfscope}%
\begin{pgfscope}%
\pgfsetbuttcap%
\pgfsetmiterjoin%
\definecolor{currentfill}{rgb}{1.000000,1.000000,1.000000}%
\pgfsetfillcolor{currentfill}%
\pgfsetfillopacity{0.800000}%
\pgfsetlinewidth{1.003750pt}%
\definecolor{currentstroke}{rgb}{0.800000,0.800000,0.800000}%
\pgfsetstrokecolor{currentstroke}%
\pgfsetstrokeopacity{0.800000}%
\pgfsetdash{}{0pt}%
\pgfpathmoveto{\pgfqpoint{6.293380in}{2.449219in}}%
\pgfpathlineto{\pgfqpoint{7.507545in}{2.449219in}}%
\pgfpathquadraticcurveto{\pgfqpoint{7.535323in}{2.449219in}}{\pgfqpoint{7.535323in}{2.476997in}}%
\pgfpathlineto{\pgfqpoint{7.535323in}{2.870822in}}%
\pgfpathquadraticcurveto{\pgfqpoint{7.535323in}{2.898600in}}{\pgfqpoint{7.507545in}{2.898600in}}%
\pgfpathlineto{\pgfqpoint{6.293380in}{2.898600in}}%
\pgfpathquadraticcurveto{\pgfqpoint{6.265602in}{2.898600in}}{\pgfqpoint{6.265602in}{2.870822in}}%
\pgfpathlineto{\pgfqpoint{6.265602in}{2.476997in}}%
\pgfpathquadraticcurveto{\pgfqpoint{6.265602in}{2.449219in}}{\pgfqpoint{6.293380in}{2.449219in}}%
\pgfpathclose%
\pgfusepath{stroke,fill}%
\end{pgfscope}%
\begin{pgfscope}%
\pgfsetbuttcap%
\pgfsetmiterjoin%
\definecolor{currentfill}{rgb}{0.121569,0.466667,0.705882}%
\pgfsetfillcolor{currentfill}%
\pgfsetlinewidth{0.000000pt}%
\definecolor{currentstroke}{rgb}{0.000000,0.000000,0.000000}%
\pgfsetstrokecolor{currentstroke}%
\pgfsetstrokeopacity{0.000000}%
\pgfsetdash{}{0pt}%
\pgfpathmoveto{\pgfqpoint{6.321158in}{2.737522in}}%
\pgfpathlineto{\pgfqpoint{6.598935in}{2.737522in}}%
\pgfpathlineto{\pgfqpoint{6.598935in}{2.834744in}}%
\pgfpathlineto{\pgfqpoint{6.321158in}{2.834744in}}%
\pgfpathclose%
\pgfusepath{fill}%
\end{pgfscope}%
\begin{pgfscope}%
\definecolor{textcolor}{rgb}{0.000000,0.000000,0.000000}%
\pgfsetstrokecolor{textcolor}%
\pgfsetfillcolor{textcolor}%
\pgftext[x=6.710047in,y=2.737522in,left,base]{\color{textcolor}\sffamily\fontsize{10.000000}{12.000000}\selectfont Pix2Vox++}%
\end{pgfscope}%
\begin{pgfscope}%
\pgfsetbuttcap%
\pgfsetmiterjoin%
\definecolor{currentfill}{rgb}{1.000000,0.498039,0.054902}%
\pgfsetfillcolor{currentfill}%
\pgfsetlinewidth{0.000000pt}%
\definecolor{currentstroke}{rgb}{0.000000,0.000000,0.000000}%
\pgfsetstrokecolor{currentstroke}%
\pgfsetstrokeopacity{0.000000}%
\pgfsetdash{}{0pt}%
\pgfpathmoveto{\pgfqpoint{6.321158in}{2.533664in}}%
\pgfpathlineto{\pgfqpoint{6.598935in}{2.533664in}}%
\pgfpathlineto{\pgfqpoint{6.598935in}{2.630887in}}%
\pgfpathlineto{\pgfqpoint{6.321158in}{2.630887in}}%
\pgfpathclose%
\pgfusepath{fill}%
\end{pgfscope}%
\begin{pgfscope}%
\definecolor{textcolor}{rgb}{0.000000,0.000000,0.000000}%
\pgfsetstrokecolor{textcolor}%
\pgfsetfillcolor{textcolor}%
\pgftext[x=6.710047in,y=2.533664in,left,base]{\color{textcolor}\sffamily\fontsize{10.000000}{12.000000}\selectfont Pix2Vox}%
\end{pgfscope}%
\end{pgfpicture}%
\makeatother%
\endgroup%
}
    \caption{Bar plot for the \gls{iou}  for baseline models(Pix2Vox++ and Pix2Vox) trained on synthetic and fine-tuned with real dataset. }
    \label{fig:finetuning1}
\end{figure}

\section{Mixed Training}\label{sec:mixed-training}
For mixed training, we mix synthetic and real dataset with different ratio in each of the minibatches.
The ratios used ware 0.15,0.25,0.5,0.75 and 0,9.
Higher the ratio, closer the mixed dataset becomes a real dataset.
Both real and synthetic dataset drawn according to these ratio for each minibatch.
As synthetic dataset is much more abundant than real, the real dataset will be oversampled to achieve the mentioned ratios.

In figure~\ref{fig:mixed1}, with each of the ratios the performance is better than the baseline of 0.3328 for pix2vox++,
except for a ratio 15\% where pix2vox++ on V2 dataset has slightly lower \gls{iou}  value.
Most significant difference is seen for at 50\% using V2 dataset with an increase of 3.04\%.
As we noted in ~\ref{sec:fine-tuning}, a maximum of 2.41\% increment was observed on pix2vox++.

\todo{check for pix2vox, already in queue}

\begin{figure}
    \centering
    \resizebox{\textwidth}{!}{%% Creator: Matplotlib, PGF backend
%%
%% To include the figure in your LaTeX document, write
%%   \input{<filename>.pgf}
%%
%% Make sure the required packages are loaded in your preamble
%%   \usepackage{pgf}
%%
%% Figures using additional raster images can only be included by \input if
%% they are in the same directory as the main LaTeX file. For loading figures
%% from other directories you can use the `import` package
%%   \usepackage{import}
%%
%% and then include the figures with
%%   \import{<path to file>}{<filename>.pgf}
%%
%% Matplotlib used the following preamble
%%   \usepackage{fontspec}
%%   \setmainfont{DejaVuSerif.ttf}[Path=\detokenize{/Users/apple/opt/anaconda3/envs/kaolin/lib/python3.7/site-packages/matplotlib/mpl-data/fonts/ttf/}]
%%   \setsansfont{DejaVuSans.ttf}[Path=\detokenize{/Users/apple/opt/anaconda3/envs/kaolin/lib/python3.7/site-packages/matplotlib/mpl-data/fonts/ttf/}]
%%   \setmonofont{DejaVuSansMono.ttf}[Path=\detokenize{/Users/apple/opt/anaconda3/envs/kaolin/lib/python3.7/site-packages/matplotlib/mpl-data/fonts/ttf/}]
%%
\begingroup%
\makeatletter%
\begin{pgfpicture}%
\pgfpathrectangle{\pgfpointorigin}{\pgfqpoint{8.079998in}{4.683527in}}%
\pgfusepath{use as bounding box, clip}%
\begin{pgfscope}%
\pgfsetbuttcap%
\pgfsetmiterjoin%
\definecolor{currentfill}{rgb}{1.000000,1.000000,1.000000}%
\pgfsetfillcolor{currentfill}%
\pgfsetlinewidth{0.000000pt}%
\definecolor{currentstroke}{rgb}{1.000000,1.000000,1.000000}%
\pgfsetstrokecolor{currentstroke}%
\pgfsetdash{}{0pt}%
\pgfpathmoveto{\pgfqpoint{0.000000in}{0.000000in}}%
\pgfpathlineto{\pgfqpoint{8.079998in}{0.000000in}}%
\pgfpathlineto{\pgfqpoint{8.079998in}{4.683527in}}%
\pgfpathlineto{\pgfqpoint{0.000000in}{4.683527in}}%
\pgfpathclose%
\pgfusepath{fill}%
\end{pgfscope}%
\begin{pgfscope}%
\pgfsetbuttcap%
\pgfsetmiterjoin%
\definecolor{currentfill}{rgb}{1.000000,1.000000,1.000000}%
\pgfsetfillcolor{currentfill}%
\pgfsetlinewidth{0.000000pt}%
\definecolor{currentstroke}{rgb}{0.000000,0.000000,0.000000}%
\pgfsetstrokecolor{currentstroke}%
\pgfsetstrokeopacity{0.000000}%
\pgfsetdash{}{0pt}%
\pgfpathmoveto{\pgfqpoint{0.696435in}{0.510552in}}%
\pgfpathlineto{\pgfqpoint{6.196158in}{0.510552in}}%
\pgfpathlineto{\pgfqpoint{6.196158in}{4.373566in}}%
\pgfpathlineto{\pgfqpoint{0.696435in}{4.373566in}}%
\pgfpathclose%
\pgfusepath{fill}%
\end{pgfscope}%
\begin{pgfscope}%
\pgfpathrectangle{\pgfqpoint{0.696435in}{0.510552in}}{\pgfqpoint{5.499722in}{3.863014in}}%
\pgfusepath{clip}%
\pgfsetbuttcap%
\pgfsetmiterjoin%
\definecolor{currentfill}{rgb}{0.121569,0.466667,0.705882}%
\pgfsetfillcolor{currentfill}%
\pgfsetlinewidth{0.000000pt}%
\definecolor{currentstroke}{rgb}{0.000000,0.000000,0.000000}%
\pgfsetstrokecolor{currentstroke}%
\pgfsetstrokeopacity{0.000000}%
\pgfsetdash{}{0pt}%
\pgfpathmoveto{\pgfqpoint{0.946423in}{0.510552in}}%
\pgfpathlineto{\pgfqpoint{1.405583in}{0.510552in}}%
\pgfpathlineto{\pgfqpoint{1.405583in}{4.064006in}}%
\pgfpathlineto{\pgfqpoint{0.946423in}{4.064006in}}%
\pgfpathclose%
\pgfusepath{fill}%
\end{pgfscope}%
\begin{pgfscope}%
\pgfpathrectangle{\pgfqpoint{0.696435in}{0.510552in}}{\pgfqpoint{5.499722in}{3.863014in}}%
\pgfusepath{clip}%
\pgfsetbuttcap%
\pgfsetmiterjoin%
\definecolor{currentfill}{rgb}{0.121569,0.466667,0.705882}%
\pgfsetfillcolor{currentfill}%
\pgfsetlinewidth{0.000000pt}%
\definecolor{currentstroke}{rgb}{0.000000,0.000000,0.000000}%
\pgfsetstrokecolor{currentstroke}%
\pgfsetstrokeopacity{0.000000}%
\pgfsetdash{}{0pt}%
\pgfpathmoveto{\pgfqpoint{1.966779in}{0.510552in}}%
\pgfpathlineto{\pgfqpoint{2.425940in}{0.510552in}}%
\pgfpathlineto{\pgfqpoint{2.425940in}{3.875596in}}%
\pgfpathlineto{\pgfqpoint{1.966779in}{3.875596in}}%
\pgfpathclose%
\pgfusepath{fill}%
\end{pgfscope}%
\begin{pgfscope}%
\pgfpathrectangle{\pgfqpoint{0.696435in}{0.510552in}}{\pgfqpoint{5.499722in}{3.863014in}}%
\pgfusepath{clip}%
\pgfsetbuttcap%
\pgfsetmiterjoin%
\definecolor{currentfill}{rgb}{0.121569,0.466667,0.705882}%
\pgfsetfillcolor{currentfill}%
\pgfsetlinewidth{0.000000pt}%
\definecolor{currentstroke}{rgb}{0.000000,0.000000,0.000000}%
\pgfsetstrokecolor{currentstroke}%
\pgfsetstrokeopacity{0.000000}%
\pgfsetdash{}{0pt}%
\pgfpathmoveto{\pgfqpoint{2.987136in}{0.510552in}}%
\pgfpathlineto{\pgfqpoint{3.446297in}{0.510552in}}%
\pgfpathlineto{\pgfqpoint{3.446297in}{4.056916in}}%
\pgfpathlineto{\pgfqpoint{2.987136in}{4.056916in}}%
\pgfpathclose%
\pgfusepath{fill}%
\end{pgfscope}%
\begin{pgfscope}%
\pgfpathrectangle{\pgfqpoint{0.696435in}{0.510552in}}{\pgfqpoint{5.499722in}{3.863014in}}%
\pgfusepath{clip}%
\pgfsetbuttcap%
\pgfsetmiterjoin%
\definecolor{currentfill}{rgb}{0.121569,0.466667,0.705882}%
\pgfsetfillcolor{currentfill}%
\pgfsetlinewidth{0.000000pt}%
\definecolor{currentstroke}{rgb}{0.000000,0.000000,0.000000}%
\pgfsetstrokecolor{currentstroke}%
\pgfsetstrokeopacity{0.000000}%
\pgfsetdash{}{0pt}%
\pgfpathmoveto{\pgfqpoint{4.007493in}{0.510552in}}%
\pgfpathlineto{\pgfqpoint{4.466653in}{0.510552in}}%
\pgfpathlineto{\pgfqpoint{4.466653in}{4.117693in}}%
\pgfpathlineto{\pgfqpoint{4.007493in}{4.117693in}}%
\pgfpathclose%
\pgfusepath{fill}%
\end{pgfscope}%
\begin{pgfscope}%
\pgfpathrectangle{\pgfqpoint{0.696435in}{0.510552in}}{\pgfqpoint{5.499722in}{3.863014in}}%
\pgfusepath{clip}%
\pgfsetbuttcap%
\pgfsetmiterjoin%
\definecolor{currentfill}{rgb}{0.121569,0.466667,0.705882}%
\pgfsetfillcolor{currentfill}%
\pgfsetlinewidth{0.000000pt}%
\definecolor{currentstroke}{rgb}{0.000000,0.000000,0.000000}%
\pgfsetstrokecolor{currentstroke}%
\pgfsetstrokeopacity{0.000000}%
\pgfsetdash{}{0pt}%
\pgfpathmoveto{\pgfqpoint{5.027849in}{0.510552in}}%
\pgfpathlineto{\pgfqpoint{5.487010in}{0.510552in}}%
\pgfpathlineto{\pgfqpoint{5.487010in}{4.014371in}}%
\pgfpathlineto{\pgfqpoint{5.027849in}{4.014371in}}%
\pgfpathclose%
\pgfusepath{fill}%
\end{pgfscope}%
\begin{pgfscope}%
\pgfpathrectangle{\pgfqpoint{0.696435in}{0.510552in}}{\pgfqpoint{5.499722in}{3.863014in}}%
\pgfusepath{clip}%
\pgfsetbuttcap%
\pgfsetmiterjoin%
\definecolor{currentfill}{rgb}{1.000000,0.498039,0.054902}%
\pgfsetfillcolor{currentfill}%
\pgfsetlinewidth{0.000000pt}%
\definecolor{currentstroke}{rgb}{0.000000,0.000000,0.000000}%
\pgfsetstrokecolor{currentstroke}%
\pgfsetstrokeopacity{0.000000}%
\pgfsetdash{}{0pt}%
\pgfpathmoveto{\pgfqpoint{1.405583in}{0.510552in}}%
\pgfpathlineto{\pgfqpoint{1.864744in}{0.510552in}}%
\pgfpathlineto{\pgfqpoint{1.864744in}{3.733782in}}%
\pgfpathlineto{\pgfqpoint{1.405583in}{3.733782in}}%
\pgfpathclose%
\pgfusepath{fill}%
\end{pgfscope}%
\begin{pgfscope}%
\pgfpathrectangle{\pgfqpoint{0.696435in}{0.510552in}}{\pgfqpoint{5.499722in}{3.863014in}}%
\pgfusepath{clip}%
\pgfsetbuttcap%
\pgfsetmiterjoin%
\definecolor{currentfill}{rgb}{1.000000,0.498039,0.054902}%
\pgfsetfillcolor{currentfill}%
\pgfsetlinewidth{0.000000pt}%
\definecolor{currentstroke}{rgb}{0.000000,0.000000,0.000000}%
\pgfsetstrokecolor{currentstroke}%
\pgfsetstrokeopacity{0.000000}%
\pgfsetdash{}{0pt}%
\pgfpathmoveto{\pgfqpoint{2.425940in}{0.510552in}}%
\pgfpathlineto{\pgfqpoint{2.885100in}{0.510552in}}%
\pgfpathlineto{\pgfqpoint{2.885100in}{3.978918in}}%
\pgfpathlineto{\pgfqpoint{2.425940in}{3.978918in}}%
\pgfpathclose%
\pgfusepath{fill}%
\end{pgfscope}%
\begin{pgfscope}%
\pgfpathrectangle{\pgfqpoint{0.696435in}{0.510552in}}{\pgfqpoint{5.499722in}{3.863014in}}%
\pgfusepath{clip}%
\pgfsetbuttcap%
\pgfsetmiterjoin%
\definecolor{currentfill}{rgb}{1.000000,0.498039,0.054902}%
\pgfsetfillcolor{currentfill}%
\pgfsetlinewidth{0.000000pt}%
\definecolor{currentstroke}{rgb}{0.000000,0.000000,0.000000}%
\pgfsetstrokecolor{currentstroke}%
\pgfsetstrokeopacity{0.000000}%
\pgfsetdash{}{0pt}%
\pgfpathmoveto{\pgfqpoint{3.446297in}{0.510552in}}%
\pgfpathlineto{\pgfqpoint{3.905457in}{0.510552in}}%
\pgfpathlineto{\pgfqpoint{3.905457in}{4.189613in}}%
\pgfpathlineto{\pgfqpoint{3.446297in}{4.189613in}}%
\pgfpathclose%
\pgfusepath{fill}%
\end{pgfscope}%
\begin{pgfscope}%
\pgfpathrectangle{\pgfqpoint{0.696435in}{0.510552in}}{\pgfqpoint{5.499722in}{3.863014in}}%
\pgfusepath{clip}%
\pgfsetbuttcap%
\pgfsetmiterjoin%
\definecolor{currentfill}{rgb}{1.000000,0.498039,0.054902}%
\pgfsetfillcolor{currentfill}%
\pgfsetlinewidth{0.000000pt}%
\definecolor{currentstroke}{rgb}{0.000000,0.000000,0.000000}%
\pgfsetstrokecolor{currentstroke}%
\pgfsetstrokeopacity{0.000000}%
\pgfsetdash{}{0pt}%
\pgfpathmoveto{\pgfqpoint{4.466653in}{0.510552in}}%
\pgfpathlineto{\pgfqpoint{4.925814in}{0.510552in}}%
\pgfpathlineto{\pgfqpoint{4.925814in}{4.075149in}}%
\pgfpathlineto{\pgfqpoint{4.466653in}{4.075149in}}%
\pgfpathclose%
\pgfusepath{fill}%
\end{pgfscope}%
\begin{pgfscope}%
\pgfpathrectangle{\pgfqpoint{0.696435in}{0.510552in}}{\pgfqpoint{5.499722in}{3.863014in}}%
\pgfusepath{clip}%
\pgfsetbuttcap%
\pgfsetmiterjoin%
\definecolor{currentfill}{rgb}{1.000000,0.498039,0.054902}%
\pgfsetfillcolor{currentfill}%
\pgfsetlinewidth{0.000000pt}%
\definecolor{currentstroke}{rgb}{0.000000,0.000000,0.000000}%
\pgfsetstrokecolor{currentstroke}%
\pgfsetstrokeopacity{0.000000}%
\pgfsetdash{}{0pt}%
\pgfpathmoveto{\pgfqpoint{5.487010in}{0.510552in}}%
\pgfpathlineto{\pgfqpoint{5.946170in}{0.510552in}}%
\pgfpathlineto{\pgfqpoint{5.946170in}{4.082239in}}%
\pgfpathlineto{\pgfqpoint{5.487010in}{4.082239in}}%
\pgfpathclose%
\pgfusepath{fill}%
\end{pgfscope}%
\begin{pgfscope}%
\pgfsetbuttcap%
\pgfsetroundjoin%
\definecolor{currentfill}{rgb}{0.000000,0.000000,0.000000}%
\pgfsetfillcolor{currentfill}%
\pgfsetlinewidth{0.803000pt}%
\definecolor{currentstroke}{rgb}{0.000000,0.000000,0.000000}%
\pgfsetstrokecolor{currentstroke}%
\pgfsetdash{}{0pt}%
\pgfsys@defobject{currentmarker}{\pgfqpoint{0.000000in}{-0.048611in}}{\pgfqpoint{0.000000in}{0.000000in}}{%
\pgfpathmoveto{\pgfqpoint{0.000000in}{0.000000in}}%
\pgfpathlineto{\pgfqpoint{0.000000in}{-0.048611in}}%
\pgfusepath{stroke,fill}%
}%
\begin{pgfscope}%
\pgfsys@transformshift{1.405583in}{0.510552in}%
\pgfsys@useobject{currentmarker}{}%
\end{pgfscope}%
\end{pgfscope}%
\begin{pgfscope}%
\definecolor{textcolor}{rgb}{0.000000,0.000000,0.000000}%
\pgfsetstrokecolor{textcolor}%
\pgfsetfillcolor{textcolor}%
\pgftext[x=1.323535in, y=0.120428in, left, base,rotate=45.000000]{\color{textcolor}\sffamily\fontsize{10.000000}{12.000000}\selectfont 15\%}%
\end{pgfscope}%
\begin{pgfscope}%
\pgfsetbuttcap%
\pgfsetroundjoin%
\definecolor{currentfill}{rgb}{0.000000,0.000000,0.000000}%
\pgfsetfillcolor{currentfill}%
\pgfsetlinewidth{0.803000pt}%
\definecolor{currentstroke}{rgb}{0.000000,0.000000,0.000000}%
\pgfsetstrokecolor{currentstroke}%
\pgfsetdash{}{0pt}%
\pgfsys@defobject{currentmarker}{\pgfqpoint{0.000000in}{-0.048611in}}{\pgfqpoint{0.000000in}{0.000000in}}{%
\pgfpathmoveto{\pgfqpoint{0.000000in}{0.000000in}}%
\pgfpathlineto{\pgfqpoint{0.000000in}{-0.048611in}}%
\pgfusepath{stroke,fill}%
}%
\begin{pgfscope}%
\pgfsys@transformshift{2.425940in}{0.510552in}%
\pgfsys@useobject{currentmarker}{}%
\end{pgfscope}%
\end{pgfscope}%
\begin{pgfscope}%
\definecolor{textcolor}{rgb}{0.000000,0.000000,0.000000}%
\pgfsetstrokecolor{textcolor}%
\pgfsetfillcolor{textcolor}%
\pgftext[x=2.343891in, y=0.120428in, left, base,rotate=45.000000]{\color{textcolor}\sffamily\fontsize{10.000000}{12.000000}\selectfont 25\%}%
\end{pgfscope}%
\begin{pgfscope}%
\pgfsetbuttcap%
\pgfsetroundjoin%
\definecolor{currentfill}{rgb}{0.000000,0.000000,0.000000}%
\pgfsetfillcolor{currentfill}%
\pgfsetlinewidth{0.803000pt}%
\definecolor{currentstroke}{rgb}{0.000000,0.000000,0.000000}%
\pgfsetstrokecolor{currentstroke}%
\pgfsetdash{}{0pt}%
\pgfsys@defobject{currentmarker}{\pgfqpoint{0.000000in}{-0.048611in}}{\pgfqpoint{0.000000in}{0.000000in}}{%
\pgfpathmoveto{\pgfqpoint{0.000000in}{0.000000in}}%
\pgfpathlineto{\pgfqpoint{0.000000in}{-0.048611in}}%
\pgfusepath{stroke,fill}%
}%
\begin{pgfscope}%
\pgfsys@transformshift{3.446297in}{0.510552in}%
\pgfsys@useobject{currentmarker}{}%
\end{pgfscope}%
\end{pgfscope}%
\begin{pgfscope}%
\definecolor{textcolor}{rgb}{0.000000,0.000000,0.000000}%
\pgfsetstrokecolor{textcolor}%
\pgfsetfillcolor{textcolor}%
\pgftext[x=3.364248in, y=0.120428in, left, base,rotate=45.000000]{\color{textcolor}\sffamily\fontsize{10.000000}{12.000000}\selectfont 50\%}%
\end{pgfscope}%
\begin{pgfscope}%
\pgfsetbuttcap%
\pgfsetroundjoin%
\definecolor{currentfill}{rgb}{0.000000,0.000000,0.000000}%
\pgfsetfillcolor{currentfill}%
\pgfsetlinewidth{0.803000pt}%
\definecolor{currentstroke}{rgb}{0.000000,0.000000,0.000000}%
\pgfsetstrokecolor{currentstroke}%
\pgfsetdash{}{0pt}%
\pgfsys@defobject{currentmarker}{\pgfqpoint{0.000000in}{-0.048611in}}{\pgfqpoint{0.000000in}{0.000000in}}{%
\pgfpathmoveto{\pgfqpoint{0.000000in}{0.000000in}}%
\pgfpathlineto{\pgfqpoint{0.000000in}{-0.048611in}}%
\pgfusepath{stroke,fill}%
}%
\begin{pgfscope}%
\pgfsys@transformshift{4.466653in}{0.510552in}%
\pgfsys@useobject{currentmarker}{}%
\end{pgfscope}%
\end{pgfscope}%
\begin{pgfscope}%
\definecolor{textcolor}{rgb}{0.000000,0.000000,0.000000}%
\pgfsetstrokecolor{textcolor}%
\pgfsetfillcolor{textcolor}%
\pgftext[x=4.384604in, y=0.120428in, left, base,rotate=45.000000]{\color{textcolor}\sffamily\fontsize{10.000000}{12.000000}\selectfont 75\%}%
\end{pgfscope}%
\begin{pgfscope}%
\pgfsetbuttcap%
\pgfsetroundjoin%
\definecolor{currentfill}{rgb}{0.000000,0.000000,0.000000}%
\pgfsetfillcolor{currentfill}%
\pgfsetlinewidth{0.803000pt}%
\definecolor{currentstroke}{rgb}{0.000000,0.000000,0.000000}%
\pgfsetstrokecolor{currentstroke}%
\pgfsetdash{}{0pt}%
\pgfsys@defobject{currentmarker}{\pgfqpoint{0.000000in}{-0.048611in}}{\pgfqpoint{0.000000in}{0.000000in}}{%
\pgfpathmoveto{\pgfqpoint{0.000000in}{0.000000in}}%
\pgfpathlineto{\pgfqpoint{0.000000in}{-0.048611in}}%
\pgfusepath{stroke,fill}%
}%
\begin{pgfscope}%
\pgfsys@transformshift{5.487010in}{0.510552in}%
\pgfsys@useobject{currentmarker}{}%
\end{pgfscope}%
\end{pgfscope}%
\begin{pgfscope}%
\definecolor{textcolor}{rgb}{0.000000,0.000000,0.000000}%
\pgfsetstrokecolor{textcolor}%
\pgfsetfillcolor{textcolor}%
\pgftext[x=5.404961in, y=0.120428in, left, base,rotate=45.000000]{\color{textcolor}\sffamily\fontsize{10.000000}{12.000000}\selectfont 90\%}%
\end{pgfscope}%
\begin{pgfscope}%
\pgfsetbuttcap%
\pgfsetroundjoin%
\definecolor{currentfill}{rgb}{0.000000,0.000000,0.000000}%
\pgfsetfillcolor{currentfill}%
\pgfsetlinewidth{0.803000pt}%
\definecolor{currentstroke}{rgb}{0.000000,0.000000,0.000000}%
\pgfsetstrokecolor{currentstroke}%
\pgfsetdash{}{0pt}%
\pgfsys@defobject{currentmarker}{\pgfqpoint{-0.048611in}{0.000000in}}{\pgfqpoint{-0.000000in}{0.000000in}}{%
\pgfpathmoveto{\pgfqpoint{-0.000000in}{0.000000in}}%
\pgfpathlineto{\pgfqpoint{-0.048611in}{0.000000in}}%
\pgfusepath{stroke,fill}%
}%
\begin{pgfscope}%
\pgfsys@transformshift{0.696435in}{0.510552in}%
\pgfsys@useobject{currentmarker}{}%
\end{pgfscope}%
\end{pgfscope}%
\begin{pgfscope}%
\definecolor{textcolor}{rgb}{0.000000,0.000000,0.000000}%
\pgfsetstrokecolor{textcolor}%
\pgfsetfillcolor{textcolor}%
\pgftext[x=0.289968in, y=0.457790in, left, base]{\color{textcolor}\sffamily\fontsize{10.000000}{12.000000}\selectfont 0.00}%
\end{pgfscope}%
\begin{pgfscope}%
\pgfsetbuttcap%
\pgfsetroundjoin%
\definecolor{currentfill}{rgb}{0.000000,0.000000,0.000000}%
\pgfsetfillcolor{currentfill}%
\pgfsetlinewidth{0.803000pt}%
\definecolor{currentstroke}{rgb}{0.000000,0.000000,0.000000}%
\pgfsetstrokecolor{currentstroke}%
\pgfsetdash{}{0pt}%
\pgfsys@defobject{currentmarker}{\pgfqpoint{-0.048611in}{0.000000in}}{\pgfqpoint{-0.000000in}{0.000000in}}{%
\pgfpathmoveto{\pgfqpoint{-0.000000in}{0.000000in}}%
\pgfpathlineto{\pgfqpoint{-0.048611in}{0.000000in}}%
\pgfusepath{stroke,fill}%
}%
\begin{pgfscope}%
\pgfsys@transformshift{0.696435in}{1.017031in}%
\pgfsys@useobject{currentmarker}{}%
\end{pgfscope}%
\end{pgfscope}%
\begin{pgfscope}%
\definecolor{textcolor}{rgb}{0.000000,0.000000,0.000000}%
\pgfsetstrokecolor{textcolor}%
\pgfsetfillcolor{textcolor}%
\pgftext[x=0.289968in, y=0.964269in, left, base]{\color{textcolor}\sffamily\fontsize{10.000000}{12.000000}\selectfont 0.05}%
\end{pgfscope}%
\begin{pgfscope}%
\pgfsetbuttcap%
\pgfsetroundjoin%
\definecolor{currentfill}{rgb}{0.000000,0.000000,0.000000}%
\pgfsetfillcolor{currentfill}%
\pgfsetlinewidth{0.803000pt}%
\definecolor{currentstroke}{rgb}{0.000000,0.000000,0.000000}%
\pgfsetstrokecolor{currentstroke}%
\pgfsetdash{}{0pt}%
\pgfsys@defobject{currentmarker}{\pgfqpoint{-0.048611in}{0.000000in}}{\pgfqpoint{-0.000000in}{0.000000in}}{%
\pgfpathmoveto{\pgfqpoint{-0.000000in}{0.000000in}}%
\pgfpathlineto{\pgfqpoint{-0.048611in}{0.000000in}}%
\pgfusepath{stroke,fill}%
}%
\begin{pgfscope}%
\pgfsys@transformshift{0.696435in}{1.523509in}%
\pgfsys@useobject{currentmarker}{}%
\end{pgfscope}%
\end{pgfscope}%
\begin{pgfscope}%
\definecolor{textcolor}{rgb}{0.000000,0.000000,0.000000}%
\pgfsetstrokecolor{textcolor}%
\pgfsetfillcolor{textcolor}%
\pgftext[x=0.289968in, y=1.470748in, left, base]{\color{textcolor}\sffamily\fontsize{10.000000}{12.000000}\selectfont 0.10}%
\end{pgfscope}%
\begin{pgfscope}%
\pgfsetbuttcap%
\pgfsetroundjoin%
\definecolor{currentfill}{rgb}{0.000000,0.000000,0.000000}%
\pgfsetfillcolor{currentfill}%
\pgfsetlinewidth{0.803000pt}%
\definecolor{currentstroke}{rgb}{0.000000,0.000000,0.000000}%
\pgfsetstrokecolor{currentstroke}%
\pgfsetdash{}{0pt}%
\pgfsys@defobject{currentmarker}{\pgfqpoint{-0.048611in}{0.000000in}}{\pgfqpoint{-0.000000in}{0.000000in}}{%
\pgfpathmoveto{\pgfqpoint{-0.000000in}{0.000000in}}%
\pgfpathlineto{\pgfqpoint{-0.048611in}{0.000000in}}%
\pgfusepath{stroke,fill}%
}%
\begin{pgfscope}%
\pgfsys@transformshift{0.696435in}{2.029988in}%
\pgfsys@useobject{currentmarker}{}%
\end{pgfscope}%
\end{pgfscope}%
\begin{pgfscope}%
\definecolor{textcolor}{rgb}{0.000000,0.000000,0.000000}%
\pgfsetstrokecolor{textcolor}%
\pgfsetfillcolor{textcolor}%
\pgftext[x=0.289968in, y=1.977226in, left, base]{\color{textcolor}\sffamily\fontsize{10.000000}{12.000000}\selectfont 0.15}%
\end{pgfscope}%
\begin{pgfscope}%
\pgfsetbuttcap%
\pgfsetroundjoin%
\definecolor{currentfill}{rgb}{0.000000,0.000000,0.000000}%
\pgfsetfillcolor{currentfill}%
\pgfsetlinewidth{0.803000pt}%
\definecolor{currentstroke}{rgb}{0.000000,0.000000,0.000000}%
\pgfsetstrokecolor{currentstroke}%
\pgfsetdash{}{0pt}%
\pgfsys@defobject{currentmarker}{\pgfqpoint{-0.048611in}{0.000000in}}{\pgfqpoint{-0.000000in}{0.000000in}}{%
\pgfpathmoveto{\pgfqpoint{-0.000000in}{0.000000in}}%
\pgfpathlineto{\pgfqpoint{-0.048611in}{0.000000in}}%
\pgfusepath{stroke,fill}%
}%
\begin{pgfscope}%
\pgfsys@transformshift{0.696435in}{2.536467in}%
\pgfsys@useobject{currentmarker}{}%
\end{pgfscope}%
\end{pgfscope}%
\begin{pgfscope}%
\definecolor{textcolor}{rgb}{0.000000,0.000000,0.000000}%
\pgfsetstrokecolor{textcolor}%
\pgfsetfillcolor{textcolor}%
\pgftext[x=0.289968in, y=2.483705in, left, base]{\color{textcolor}\sffamily\fontsize{10.000000}{12.000000}\selectfont 0.20}%
\end{pgfscope}%
\begin{pgfscope}%
\pgfsetbuttcap%
\pgfsetroundjoin%
\definecolor{currentfill}{rgb}{0.000000,0.000000,0.000000}%
\pgfsetfillcolor{currentfill}%
\pgfsetlinewidth{0.803000pt}%
\definecolor{currentstroke}{rgb}{0.000000,0.000000,0.000000}%
\pgfsetstrokecolor{currentstroke}%
\pgfsetdash{}{0pt}%
\pgfsys@defobject{currentmarker}{\pgfqpoint{-0.048611in}{0.000000in}}{\pgfqpoint{-0.000000in}{0.000000in}}{%
\pgfpathmoveto{\pgfqpoint{-0.000000in}{0.000000in}}%
\pgfpathlineto{\pgfqpoint{-0.048611in}{0.000000in}}%
\pgfusepath{stroke,fill}%
}%
\begin{pgfscope}%
\pgfsys@transformshift{0.696435in}{3.042945in}%
\pgfsys@useobject{currentmarker}{}%
\end{pgfscope}%
\end{pgfscope}%
\begin{pgfscope}%
\definecolor{textcolor}{rgb}{0.000000,0.000000,0.000000}%
\pgfsetstrokecolor{textcolor}%
\pgfsetfillcolor{textcolor}%
\pgftext[x=0.289968in, y=2.990184in, left, base]{\color{textcolor}\sffamily\fontsize{10.000000}{12.000000}\selectfont 0.25}%
\end{pgfscope}%
\begin{pgfscope}%
\pgfsetbuttcap%
\pgfsetroundjoin%
\definecolor{currentfill}{rgb}{0.000000,0.000000,0.000000}%
\pgfsetfillcolor{currentfill}%
\pgfsetlinewidth{0.803000pt}%
\definecolor{currentstroke}{rgb}{0.000000,0.000000,0.000000}%
\pgfsetstrokecolor{currentstroke}%
\pgfsetdash{}{0pt}%
\pgfsys@defobject{currentmarker}{\pgfqpoint{-0.048611in}{0.000000in}}{\pgfqpoint{-0.000000in}{0.000000in}}{%
\pgfpathmoveto{\pgfqpoint{-0.000000in}{0.000000in}}%
\pgfpathlineto{\pgfqpoint{-0.048611in}{0.000000in}}%
\pgfusepath{stroke,fill}%
}%
\begin{pgfscope}%
\pgfsys@transformshift{0.696435in}{3.549424in}%
\pgfsys@useobject{currentmarker}{}%
\end{pgfscope}%
\end{pgfscope}%
\begin{pgfscope}%
\definecolor{textcolor}{rgb}{0.000000,0.000000,0.000000}%
\pgfsetstrokecolor{textcolor}%
\pgfsetfillcolor{textcolor}%
\pgftext[x=0.289968in, y=3.496662in, left, base]{\color{textcolor}\sffamily\fontsize{10.000000}{12.000000}\selectfont 0.30}%
\end{pgfscope}%
\begin{pgfscope}%
\pgfsetbuttcap%
\pgfsetroundjoin%
\definecolor{currentfill}{rgb}{0.000000,0.000000,0.000000}%
\pgfsetfillcolor{currentfill}%
\pgfsetlinewidth{0.803000pt}%
\definecolor{currentstroke}{rgb}{0.000000,0.000000,0.000000}%
\pgfsetstrokecolor{currentstroke}%
\pgfsetdash{}{0pt}%
\pgfsys@defobject{currentmarker}{\pgfqpoint{-0.048611in}{0.000000in}}{\pgfqpoint{-0.000000in}{0.000000in}}{%
\pgfpathmoveto{\pgfqpoint{-0.000000in}{0.000000in}}%
\pgfpathlineto{\pgfqpoint{-0.048611in}{0.000000in}}%
\pgfusepath{stroke,fill}%
}%
\begin{pgfscope}%
\pgfsys@transformshift{0.696435in}{4.055903in}%
\pgfsys@useobject{currentmarker}{}%
\end{pgfscope}%
\end{pgfscope}%
\begin{pgfscope}%
\definecolor{textcolor}{rgb}{0.000000,0.000000,0.000000}%
\pgfsetstrokecolor{textcolor}%
\pgfsetfillcolor{textcolor}%
\pgftext[x=0.289968in, y=4.003141in, left, base]{\color{textcolor}\sffamily\fontsize{10.000000}{12.000000}\selectfont 0.35}%
\end{pgfscope}%
\begin{pgfscope}%
\definecolor{textcolor}{rgb}{0.000000,0.000000,0.000000}%
\pgfsetstrokecolor{textcolor}%
\pgfsetfillcolor{textcolor}%
\pgftext[x=0.234413in,y=2.442059in,,bottom,rotate=90.000000]{\color{textcolor}\sffamily\fontsize{10.000000}{12.000000}\selectfont IoU}%
\end{pgfscope}%
\begin{pgfscope}%
\pgfsetrectcap%
\pgfsetmiterjoin%
\pgfsetlinewidth{0.803000pt}%
\definecolor{currentstroke}{rgb}{0.000000,0.000000,0.000000}%
\pgfsetstrokecolor{currentstroke}%
\pgfsetdash{}{0pt}%
\pgfpathmoveto{\pgfqpoint{0.696435in}{0.510552in}}%
\pgfpathlineto{\pgfqpoint{0.696435in}{4.373566in}}%
\pgfusepath{stroke}%
\end{pgfscope}%
\begin{pgfscope}%
\pgfsetrectcap%
\pgfsetmiterjoin%
\pgfsetlinewidth{0.803000pt}%
\definecolor{currentstroke}{rgb}{0.000000,0.000000,0.000000}%
\pgfsetstrokecolor{currentstroke}%
\pgfsetdash{}{0pt}%
\pgfpathmoveto{\pgfqpoint{6.196158in}{0.510552in}}%
\pgfpathlineto{\pgfqpoint{6.196158in}{4.373566in}}%
\pgfusepath{stroke}%
\end{pgfscope}%
\begin{pgfscope}%
\pgfsetrectcap%
\pgfsetmiterjoin%
\pgfsetlinewidth{0.803000pt}%
\definecolor{currentstroke}{rgb}{0.000000,0.000000,0.000000}%
\pgfsetstrokecolor{currentstroke}%
\pgfsetdash{}{0pt}%
\pgfpathmoveto{\pgfqpoint{0.696435in}{0.510552in}}%
\pgfpathlineto{\pgfqpoint{6.196158in}{0.510552in}}%
\pgfusepath{stroke}%
\end{pgfscope}%
\begin{pgfscope}%
\pgfsetrectcap%
\pgfsetmiterjoin%
\pgfsetlinewidth{0.803000pt}%
\definecolor{currentstroke}{rgb}{0.000000,0.000000,0.000000}%
\pgfsetstrokecolor{currentstroke}%
\pgfsetdash{}{0pt}%
\pgfpathmoveto{\pgfqpoint{0.696435in}{4.373566in}}%
\pgfpathlineto{\pgfqpoint{6.196158in}{4.373566in}}%
\pgfusepath{stroke}%
\end{pgfscope}%
\begin{pgfscope}%
\definecolor{textcolor}{rgb}{0.000000,0.000000,0.000000}%
\pgfsetstrokecolor{textcolor}%
\pgfsetfillcolor{textcolor}%
\pgftext[x=1.176003in,y=4.105673in,,bottom]{\color{textcolor}\sffamily\fontsize{10.000000}{12.000000}\selectfont 0.3508}%
\end{pgfscope}%
\begin{pgfscope}%
\definecolor{textcolor}{rgb}{0.000000,0.000000,0.000000}%
\pgfsetstrokecolor{textcolor}%
\pgfsetfillcolor{textcolor}%
\pgftext[x=2.196360in,y=3.917263in,,bottom]{\color{textcolor}\sffamily\fontsize{10.000000}{12.000000}\selectfont 0.3322}%
\end{pgfscope}%
\begin{pgfscope}%
\definecolor{textcolor}{rgb}{0.000000,0.000000,0.000000}%
\pgfsetstrokecolor{textcolor}%
\pgfsetfillcolor{textcolor}%
\pgftext[x=3.216716in,y=4.098582in,,bottom]{\color{textcolor}\sffamily\fontsize{10.000000}{12.000000}\selectfont 0.3501}%
\end{pgfscope}%
\begin{pgfscope}%
\definecolor{textcolor}{rgb}{0.000000,0.000000,0.000000}%
\pgfsetstrokecolor{textcolor}%
\pgfsetfillcolor{textcolor}%
\pgftext[x=4.237073in,y=4.159360in,,bottom]{\color{textcolor}\sffamily\fontsize{10.000000}{12.000000}\selectfont 0.3561}%
\end{pgfscope}%
\begin{pgfscope}%
\definecolor{textcolor}{rgb}{0.000000,0.000000,0.000000}%
\pgfsetstrokecolor{textcolor}%
\pgfsetfillcolor{textcolor}%
\pgftext[x=5.257430in,y=4.056038in,,bottom]{\color{textcolor}\sffamily\fontsize{10.000000}{12.000000}\selectfont 0.3459}%
\end{pgfscope}%
\begin{pgfscope}%
\definecolor{textcolor}{rgb}{0.000000,0.000000,0.000000}%
\pgfsetstrokecolor{textcolor}%
\pgfsetfillcolor{textcolor}%
\pgftext[x=1.635164in,y=3.775449in,,bottom]{\color{textcolor}\sffamily\fontsize{10.000000}{12.000000}\selectfont 0.3182}%
\end{pgfscope}%
\begin{pgfscope}%
\definecolor{textcolor}{rgb}{0.000000,0.000000,0.000000}%
\pgfsetstrokecolor{textcolor}%
\pgfsetfillcolor{textcolor}%
\pgftext[x=2.655520in,y=4.020584in,,bottom]{\color{textcolor}\sffamily\fontsize{10.000000}{12.000000}\selectfont 0.3424}%
\end{pgfscope}%
\begin{pgfscope}%
\definecolor{textcolor}{rgb}{0.000000,0.000000,0.000000}%
\pgfsetstrokecolor{textcolor}%
\pgfsetfillcolor{textcolor}%
\pgftext[x=3.675877in,y=4.231280in,,bottom]{\color{textcolor}\sffamily\fontsize{10.000000}{12.000000}\selectfont 0.3632}%
\end{pgfscope}%
\begin{pgfscope}%
\definecolor{textcolor}{rgb}{0.000000,0.000000,0.000000}%
\pgfsetstrokecolor{textcolor}%
\pgfsetfillcolor{textcolor}%
\pgftext[x=4.696233in,y=4.116815in,,bottom]{\color{textcolor}\sffamily\fontsize{10.000000}{12.000000}\selectfont 0.3519}%
\end{pgfscope}%
\begin{pgfscope}%
\definecolor{textcolor}{rgb}{0.000000,0.000000,0.000000}%
\pgfsetstrokecolor{textcolor}%
\pgfsetfillcolor{textcolor}%
\pgftext[x=5.716590in,y=4.123906in,,bottom]{\color{textcolor}\sffamily\fontsize{10.000000}{12.000000}\selectfont 0.3526}%
\end{pgfscope}%
\begin{pgfscope}%
\definecolor{textcolor}{rgb}{0.000000,0.000000,0.000000}%
\pgfsetstrokecolor{textcolor}%
\pgfsetfillcolor{textcolor}%
\pgftext[x=3.446297in,y=4.456899in,,base]{\color{textcolor}\sffamily\fontsize{12.000000}{14.400000}\selectfont Mixed training on Pix2Vox++ using 2 versions of S2R:3DFREE}%
\end{pgfscope}%
\begin{pgfscope}%
\pgfsetbuttcap%
\pgfsetmiterjoin%
\definecolor{currentfill}{rgb}{1.000000,1.000000,1.000000}%
\pgfsetfillcolor{currentfill}%
\pgfsetfillopacity{0.800000}%
\pgfsetlinewidth{1.003750pt}%
\definecolor{currentstroke}{rgb}{0.800000,0.800000,0.800000}%
\pgfsetstrokecolor{currentstroke}%
\pgfsetstrokeopacity{0.800000}%
\pgfsetdash{}{0pt}%
\pgfpathmoveto{\pgfqpoint{6.293380in}{2.217368in}}%
\pgfpathlineto{\pgfqpoint{7.952220in}{2.217368in}}%
\pgfpathquadraticcurveto{\pgfqpoint{7.979998in}{2.217368in}}{\pgfqpoint{7.979998in}{2.245146in}}%
\pgfpathlineto{\pgfqpoint{7.979998in}{2.638972in}}%
\pgfpathquadraticcurveto{\pgfqpoint{7.979998in}{2.666749in}}{\pgfqpoint{7.952220in}{2.666749in}}%
\pgfpathlineto{\pgfqpoint{6.293380in}{2.666749in}}%
\pgfpathquadraticcurveto{\pgfqpoint{6.265602in}{2.666749in}}{\pgfqpoint{6.265602in}{2.638972in}}%
\pgfpathlineto{\pgfqpoint{6.265602in}{2.245146in}}%
\pgfpathquadraticcurveto{\pgfqpoint{6.265602in}{2.217368in}}{\pgfqpoint{6.293380in}{2.217368in}}%
\pgfpathclose%
\pgfusepath{stroke,fill}%
\end{pgfscope}%
\begin{pgfscope}%
\pgfsetbuttcap%
\pgfsetmiterjoin%
\definecolor{currentfill}{rgb}{0.121569,0.466667,0.705882}%
\pgfsetfillcolor{currentfill}%
\pgfsetlinewidth{0.000000pt}%
\definecolor{currentstroke}{rgb}{0.000000,0.000000,0.000000}%
\pgfsetstrokecolor{currentstroke}%
\pgfsetstrokeopacity{0.000000}%
\pgfsetdash{}{0pt}%
\pgfpathmoveto{\pgfqpoint{6.321158in}{2.505671in}}%
\pgfpathlineto{\pgfqpoint{6.598935in}{2.505671in}}%
\pgfpathlineto{\pgfqpoint{6.598935in}{2.602893in}}%
\pgfpathlineto{\pgfqpoint{6.321158in}{2.602893in}}%
\pgfpathclose%
\pgfusepath{fill}%
\end{pgfscope}%
\begin{pgfscope}%
\definecolor{textcolor}{rgb}{0.000000,0.000000,0.000000}%
\pgfsetstrokecolor{textcolor}%
\pgfsetfillcolor{textcolor}%
\pgftext[x=6.710047in,y=2.505671in,left,base]{\color{textcolor}\sffamily\fontsize{10.000000}{12.000000}\selectfont V1 on Pix2Vox++}%
\end{pgfscope}%
\begin{pgfscope}%
\pgfsetbuttcap%
\pgfsetmiterjoin%
\definecolor{currentfill}{rgb}{1.000000,0.498039,0.054902}%
\pgfsetfillcolor{currentfill}%
\pgfsetlinewidth{0.000000pt}%
\definecolor{currentstroke}{rgb}{0.000000,0.000000,0.000000}%
\pgfsetstrokecolor{currentstroke}%
\pgfsetstrokeopacity{0.000000}%
\pgfsetdash{}{0pt}%
\pgfpathmoveto{\pgfqpoint{6.321158in}{2.301814in}}%
\pgfpathlineto{\pgfqpoint{6.598935in}{2.301814in}}%
\pgfpathlineto{\pgfqpoint{6.598935in}{2.399036in}}%
\pgfpathlineto{\pgfqpoint{6.321158in}{2.399036in}}%
\pgfpathclose%
\pgfusepath{fill}%
\end{pgfscope}%
\begin{pgfscope}%
\definecolor{textcolor}{rgb}{0.000000,0.000000,0.000000}%
\pgfsetstrokecolor{textcolor}%
\pgfsetfillcolor{textcolor}%
\pgftext[x=6.710047in,y=2.301814in,left,base]{\color{textcolor}\sffamily\fontsize{10.000000}{12.000000}\selectfont V2 on Pix2Vox++}%
\end{pgfscope}%
\end{pgfpicture}%
\makeatother%
\endgroup%
}
    \caption{Bar plot for the \gls{iou}  for baselines trained on different ratios of synthetic and real dataset}
    \label{fig:mixed1}
\end{figure}

\section{Ablation study on chairs}\label{sec:ablation-study-on-chairs}
In this section, we conduct ablation study oon chairs by changing domain randomisation property.
The dataset used for the study is explained in subsection~\ref{subsec:s2r:3dfree-ablation}.
The samples for  each of the dataset used is as shown in~\ref{fig:domain randomisation for ablation study}.

\subsection{Domain randomisation on chair dataset}\label{subsec:domain-randomisation-on-chair-dataset}
For comparison with real dataset, we extract only the chair models from Pix3D and compare the results of with and without 2D augmentation with synthetic dataset.
In figure~\ref{fig:ablation1}, we see that textureless chair dataset out performs rest of the dataset.
Contrary to the hypothesis that more the randomisation, better the performance, we see that as the randomisation is increased the performance has decreased.

The baseline on real chair dataset from Pix3D is 0.3035 and 0.3308, without and with 2D augmentations on pix2vox++.
The textureless chair dataset from \gls{free} gives an \gls{iou}  of 0.2492 and 0.2435 with light for pix2vox++.
When added light to this dataset, the performance decreases by 0.6\%.
It further decreases 0.8\% nd 0.6\% when texture and texture with light is added.
The \gls{iou}  is reached 0.2406 for muli-object with all the randomisation.

In case of pix2vox, similar behavior is seen, except that is an increase of 5.9\% when light is added to textureless.
The values decrease by 2.02\%, 5.86\% when texture and texture with light is added.
For multi-object there is an increase 6.8\% when compared to Texture with light.

\begin{figure}
    \centering
    \resizebox{\textwidth}{!}{%% Creator: Matplotlib, PGF backend
%%
%% To include the figure in your LaTeX document, write
%%   \input{<filename>.pgf}
%%
%% Make sure the required packages are loaded in your preamble
%%   \usepackage{pgf}
%%
%% Figures using additional raster images can only be included by \input if
%% they are in the same directory as the main LaTeX file. For loading figures
%% from other directories you can use the `import` package
%%   \usepackage{import}
%%
%% and then include the figures with
%%   \import{<path to file>}{<filename>.pgf}
%%
%% Matplotlib used the following preamble
%%   \usepackage{fontspec}
%%   \setmainfont{DejaVuSerif.ttf}[Path=\detokenize{/Users/apple/opt/anaconda3/envs/kaolin/lib/python3.7/site-packages/matplotlib/mpl-data/fonts/ttf/}]
%%   \setsansfont{DejaVuSans.ttf}[Path=\detokenize{/Users/apple/opt/anaconda3/envs/kaolin/lib/python3.7/site-packages/matplotlib/mpl-data/fonts/ttf/}]
%%   \setmonofont{DejaVuSansMono.ttf}[Path=\detokenize{/Users/apple/opt/anaconda3/envs/kaolin/lib/python3.7/site-packages/matplotlib/mpl-data/fonts/ttf/}]
%%
\begingroup%
\makeatletter%
\begin{pgfpicture}%
\pgfpathrectangle{\pgfpointorigin}{\pgfqpoint{7.635323in}{4.681204in}}%
\pgfusepath{use as bounding box, clip}%
\begin{pgfscope}%
\pgfsetbuttcap%
\pgfsetmiterjoin%
\definecolor{currentfill}{rgb}{1.000000,1.000000,1.000000}%
\pgfsetfillcolor{currentfill}%
\pgfsetlinewidth{0.000000pt}%
\definecolor{currentstroke}{rgb}{1.000000,1.000000,1.000000}%
\pgfsetstrokecolor{currentstroke}%
\pgfsetdash{}{0pt}%
\pgfpathmoveto{\pgfqpoint{0.000000in}{0.000000in}}%
\pgfpathlineto{\pgfqpoint{7.635323in}{0.000000in}}%
\pgfpathlineto{\pgfqpoint{7.635323in}{4.681204in}}%
\pgfpathlineto{\pgfqpoint{0.000000in}{4.681204in}}%
\pgfpathclose%
\pgfusepath{fill}%
\end{pgfscope}%
\begin{pgfscope}%
\pgfsetbuttcap%
\pgfsetmiterjoin%
\definecolor{currentfill}{rgb}{1.000000,1.000000,1.000000}%
\pgfsetfillcolor{currentfill}%
\pgfsetlinewidth{0.000000pt}%
\definecolor{currentstroke}{rgb}{0.000000,0.000000,0.000000}%
\pgfsetstrokecolor{currentstroke}%
\pgfsetstrokeopacity{0.000000}%
\pgfsetdash{}{0pt}%
\pgfpathmoveto{\pgfqpoint{0.696435in}{1.249807in}}%
\pgfpathlineto{\pgfqpoint{6.196158in}{1.249807in}}%
\pgfpathlineto{\pgfqpoint{6.196158in}{4.371243in}}%
\pgfpathlineto{\pgfqpoint{0.696435in}{4.371243in}}%
\pgfpathclose%
\pgfusepath{fill}%
\end{pgfscope}%
\begin{pgfscope}%
\pgfpathrectangle{\pgfqpoint{0.696435in}{1.249807in}}{\pgfqpoint{5.499722in}{3.121436in}}%
\pgfusepath{clip}%
\pgfsetbuttcap%
\pgfsetmiterjoin%
\definecolor{currentfill}{rgb}{0.121569,0.466667,0.705882}%
\pgfsetfillcolor{currentfill}%
\pgfsetlinewidth{0.000000pt}%
\definecolor{currentstroke}{rgb}{0.000000,0.000000,0.000000}%
\pgfsetstrokecolor{currentstroke}%
\pgfsetstrokeopacity{0.000000}%
\pgfsetdash{}{0pt}%
\pgfpathmoveto{\pgfqpoint{0.946423in}{1.249807in}}%
\pgfpathlineto{\pgfqpoint{1.272493in}{1.249807in}}%
\pgfpathlineto{\pgfqpoint{1.272493in}{3.977266in}}%
\pgfpathlineto{\pgfqpoint{0.946423in}{3.977266in}}%
\pgfpathclose%
\pgfusepath{fill}%
\end{pgfscope}%
\begin{pgfscope}%
\pgfpathrectangle{\pgfqpoint{0.696435in}{1.249807in}}{\pgfqpoint{5.499722in}{3.121436in}}%
\pgfusepath{clip}%
\pgfsetbuttcap%
\pgfsetmiterjoin%
\definecolor{currentfill}{rgb}{0.121569,0.466667,0.705882}%
\pgfsetfillcolor{currentfill}%
\pgfsetlinewidth{0.000000pt}%
\definecolor{currentstroke}{rgb}{0.000000,0.000000,0.000000}%
\pgfsetstrokecolor{currentstroke}%
\pgfsetstrokeopacity{0.000000}%
\pgfsetdash{}{0pt}%
\pgfpathmoveto{\pgfqpoint{1.671024in}{1.249807in}}%
\pgfpathlineto{\pgfqpoint{1.997094in}{1.249807in}}%
\pgfpathlineto{\pgfqpoint{1.997094in}{4.222603in}}%
\pgfpathlineto{\pgfqpoint{1.671024in}{4.222603in}}%
\pgfpathclose%
\pgfusepath{fill}%
\end{pgfscope}%
\begin{pgfscope}%
\pgfpathrectangle{\pgfqpoint{0.696435in}{1.249807in}}{\pgfqpoint{5.499722in}{3.121436in}}%
\pgfusepath{clip}%
\pgfsetbuttcap%
\pgfsetmiterjoin%
\definecolor{currentfill}{rgb}{0.121569,0.466667,0.705882}%
\pgfsetfillcolor{currentfill}%
\pgfsetlinewidth{0.000000pt}%
\definecolor{currentstroke}{rgb}{0.000000,0.000000,0.000000}%
\pgfsetstrokecolor{currentstroke}%
\pgfsetstrokeopacity{0.000000}%
\pgfsetdash{}{0pt}%
\pgfpathmoveto{\pgfqpoint{2.395625in}{1.249807in}}%
\pgfpathlineto{\pgfqpoint{2.721695in}{1.249807in}}%
\pgfpathlineto{\pgfqpoint{2.721695in}{3.489289in}}%
\pgfpathlineto{\pgfqpoint{2.395625in}{3.489289in}}%
\pgfpathclose%
\pgfusepath{fill}%
\end{pgfscope}%
\begin{pgfscope}%
\pgfpathrectangle{\pgfqpoint{0.696435in}{1.249807in}}{\pgfqpoint{5.499722in}{3.121436in}}%
\pgfusepath{clip}%
\pgfsetbuttcap%
\pgfsetmiterjoin%
\definecolor{currentfill}{rgb}{0.121569,0.466667,0.705882}%
\pgfsetfillcolor{currentfill}%
\pgfsetlinewidth{0.000000pt}%
\definecolor{currentstroke}{rgb}{0.000000,0.000000,0.000000}%
\pgfsetstrokecolor{currentstroke}%
\pgfsetstrokeopacity{0.000000}%
\pgfsetdash{}{0pt}%
\pgfpathmoveto{\pgfqpoint{3.120226in}{1.249807in}}%
\pgfpathlineto{\pgfqpoint{3.446297in}{1.249807in}}%
\pgfpathlineto{\pgfqpoint{3.446297in}{3.438065in}}%
\pgfpathlineto{\pgfqpoint{3.120226in}{3.438065in}}%
\pgfpathclose%
\pgfusepath{fill}%
\end{pgfscope}%
\begin{pgfscope}%
\pgfpathrectangle{\pgfqpoint{0.696435in}{1.249807in}}{\pgfqpoint{5.499722in}{3.121436in}}%
\pgfusepath{clip}%
\pgfsetbuttcap%
\pgfsetmiterjoin%
\definecolor{currentfill}{rgb}{0.121569,0.466667,0.705882}%
\pgfsetfillcolor{currentfill}%
\pgfsetlinewidth{0.000000pt}%
\definecolor{currentstroke}{rgb}{0.000000,0.000000,0.000000}%
\pgfsetstrokecolor{currentstroke}%
\pgfsetstrokeopacity{0.000000}%
\pgfsetdash{}{0pt}%
\pgfpathmoveto{\pgfqpoint{3.844827in}{1.249807in}}%
\pgfpathlineto{\pgfqpoint{4.170898in}{1.249807in}}%
\pgfpathlineto{\pgfqpoint{4.170898in}{3.366172in}}%
\pgfpathlineto{\pgfqpoint{3.844827in}{3.366172in}}%
\pgfpathclose%
\pgfusepath{fill}%
\end{pgfscope}%
\begin{pgfscope}%
\pgfpathrectangle{\pgfqpoint{0.696435in}{1.249807in}}{\pgfqpoint{5.499722in}{3.121436in}}%
\pgfusepath{clip}%
\pgfsetbuttcap%
\pgfsetmiterjoin%
\definecolor{currentfill}{rgb}{0.121569,0.466667,0.705882}%
\pgfsetfillcolor{currentfill}%
\pgfsetlinewidth{0.000000pt}%
\definecolor{currentstroke}{rgb}{0.000000,0.000000,0.000000}%
\pgfsetstrokecolor{currentstroke}%
\pgfsetstrokeopacity{0.000000}%
\pgfsetdash{}{0pt}%
\pgfpathmoveto{\pgfqpoint{4.569428in}{1.249807in}}%
\pgfpathlineto{\pgfqpoint{4.895499in}{1.249807in}}%
\pgfpathlineto{\pgfqpoint{4.895499in}{3.223283in}}%
\pgfpathlineto{\pgfqpoint{4.569428in}{3.223283in}}%
\pgfpathclose%
\pgfusepath{fill}%
\end{pgfscope}%
\begin{pgfscope}%
\pgfpathrectangle{\pgfqpoint{0.696435in}{1.249807in}}{\pgfqpoint{5.499722in}{3.121436in}}%
\pgfusepath{clip}%
\pgfsetbuttcap%
\pgfsetmiterjoin%
\definecolor{currentfill}{rgb}{0.121569,0.466667,0.705882}%
\pgfsetfillcolor{currentfill}%
\pgfsetlinewidth{0.000000pt}%
\definecolor{currentstroke}{rgb}{0.000000,0.000000,0.000000}%
\pgfsetstrokecolor{currentstroke}%
\pgfsetstrokeopacity{0.000000}%
\pgfsetdash{}{0pt}%
\pgfpathmoveto{\pgfqpoint{5.294029in}{1.249807in}}%
\pgfpathlineto{\pgfqpoint{5.620100in}{1.249807in}}%
\pgfpathlineto{\pgfqpoint{5.620100in}{3.412004in}}%
\pgfpathlineto{\pgfqpoint{5.294029in}{3.412004in}}%
\pgfpathclose%
\pgfusepath{fill}%
\end{pgfscope}%
\begin{pgfscope}%
\pgfpathrectangle{\pgfqpoint{0.696435in}{1.249807in}}{\pgfqpoint{5.499722in}{3.121436in}}%
\pgfusepath{clip}%
\pgfsetbuttcap%
\pgfsetmiterjoin%
\definecolor{currentfill}{rgb}{1.000000,0.498039,0.054902}%
\pgfsetfillcolor{currentfill}%
\pgfsetlinewidth{0.000000pt}%
\definecolor{currentstroke}{rgb}{0.000000,0.000000,0.000000}%
\pgfsetstrokecolor{currentstroke}%
\pgfsetstrokeopacity{0.000000}%
\pgfsetdash{}{0pt}%
\pgfpathmoveto{\pgfqpoint{1.272493in}{1.249807in}}%
\pgfpathlineto{\pgfqpoint{1.598564in}{1.249807in}}%
\pgfpathlineto{\pgfqpoint{1.598564in}{3.643860in}}%
\pgfpathlineto{\pgfqpoint{1.272493in}{3.643860in}}%
\pgfpathclose%
\pgfusepath{fill}%
\end{pgfscope}%
\begin{pgfscope}%
\pgfpathrectangle{\pgfqpoint{0.696435in}{1.249807in}}{\pgfqpoint{5.499722in}{3.121436in}}%
\pgfusepath{clip}%
\pgfsetbuttcap%
\pgfsetmiterjoin%
\definecolor{currentfill}{rgb}{1.000000,0.498039,0.054902}%
\pgfsetfillcolor{currentfill}%
\pgfsetlinewidth{0.000000pt}%
\definecolor{currentstroke}{rgb}{0.000000,0.000000,0.000000}%
\pgfsetstrokecolor{currentstroke}%
\pgfsetstrokeopacity{0.000000}%
\pgfsetdash{}{0pt}%
\pgfpathmoveto{\pgfqpoint{1.997094in}{1.249807in}}%
\pgfpathlineto{\pgfqpoint{2.323165in}{1.249807in}}%
\pgfpathlineto{\pgfqpoint{2.323165in}{3.862237in}}%
\pgfpathlineto{\pgfqpoint{1.997094in}{3.862237in}}%
\pgfpathclose%
\pgfusepath{fill}%
\end{pgfscope}%
\begin{pgfscope}%
\pgfpathrectangle{\pgfqpoint{0.696435in}{1.249807in}}{\pgfqpoint{5.499722in}{3.121436in}}%
\pgfusepath{clip}%
\pgfsetbuttcap%
\pgfsetmiterjoin%
\definecolor{currentfill}{rgb}{1.000000,0.498039,0.054902}%
\pgfsetfillcolor{currentfill}%
\pgfsetlinewidth{0.000000pt}%
\definecolor{currentstroke}{rgb}{0.000000,0.000000,0.000000}%
\pgfsetstrokecolor{currentstroke}%
\pgfsetstrokeopacity{0.000000}%
\pgfsetdash{}{0pt}%
\pgfpathmoveto{\pgfqpoint{2.721695in}{1.249807in}}%
\pgfpathlineto{\pgfqpoint{3.047766in}{1.249807in}}%
\pgfpathlineto{\pgfqpoint{3.047766in}{2.908749in}}%
\pgfpathlineto{\pgfqpoint{2.721695in}{2.908749in}}%
\pgfpathclose%
\pgfusepath{fill}%
\end{pgfscope}%
\begin{pgfscope}%
\pgfpathrectangle{\pgfqpoint{0.696435in}{1.249807in}}{\pgfqpoint{5.499722in}{3.121436in}}%
\pgfusepath{clip}%
\pgfsetbuttcap%
\pgfsetmiterjoin%
\definecolor{currentfill}{rgb}{1.000000,0.498039,0.054902}%
\pgfsetfillcolor{currentfill}%
\pgfsetlinewidth{0.000000pt}%
\definecolor{currentstroke}{rgb}{0.000000,0.000000,0.000000}%
\pgfsetstrokecolor{currentstroke}%
\pgfsetstrokeopacity{0.000000}%
\pgfsetdash{}{0pt}%
\pgfpathmoveto{\pgfqpoint{3.446297in}{1.249807in}}%
\pgfpathlineto{\pgfqpoint{3.772367in}{1.249807in}}%
\pgfpathlineto{\pgfqpoint{3.772367in}{3.420092in}}%
\pgfpathlineto{\pgfqpoint{3.446297in}{3.420092in}}%
\pgfpathclose%
\pgfusepath{fill}%
\end{pgfscope}%
\begin{pgfscope}%
\pgfpathrectangle{\pgfqpoint{0.696435in}{1.249807in}}{\pgfqpoint{5.499722in}{3.121436in}}%
\pgfusepath{clip}%
\pgfsetbuttcap%
\pgfsetmiterjoin%
\definecolor{currentfill}{rgb}{1.000000,0.498039,0.054902}%
\pgfsetfillcolor{currentfill}%
\pgfsetlinewidth{0.000000pt}%
\definecolor{currentstroke}{rgb}{0.000000,0.000000,0.000000}%
\pgfsetstrokecolor{currentstroke}%
\pgfsetstrokeopacity{0.000000}%
\pgfsetdash{}{0pt}%
\pgfpathmoveto{\pgfqpoint{4.170898in}{1.249807in}}%
\pgfpathlineto{\pgfqpoint{4.496968in}{1.249807in}}%
\pgfpathlineto{\pgfqpoint{4.496968in}{3.163971in}}%
\pgfpathlineto{\pgfqpoint{4.170898in}{3.163971in}}%
\pgfpathclose%
\pgfusepath{fill}%
\end{pgfscope}%
\begin{pgfscope}%
\pgfpathrectangle{\pgfqpoint{0.696435in}{1.249807in}}{\pgfqpoint{5.499722in}{3.121436in}}%
\pgfusepath{clip}%
\pgfsetbuttcap%
\pgfsetmiterjoin%
\definecolor{currentfill}{rgb}{1.000000,0.498039,0.054902}%
\pgfsetfillcolor{currentfill}%
\pgfsetlinewidth{0.000000pt}%
\definecolor{currentstroke}{rgb}{0.000000,0.000000,0.000000}%
\pgfsetstrokecolor{currentstroke}%
\pgfsetstrokeopacity{0.000000}%
\pgfsetdash{}{0pt}%
\pgfpathmoveto{\pgfqpoint{4.895499in}{1.249807in}}%
\pgfpathlineto{\pgfqpoint{5.221569in}{1.249807in}}%
\pgfpathlineto{\pgfqpoint{5.221569in}{2.637351in}}%
\pgfpathlineto{\pgfqpoint{4.895499in}{2.637351in}}%
\pgfpathclose%
\pgfusepath{fill}%
\end{pgfscope}%
\begin{pgfscope}%
\pgfpathrectangle{\pgfqpoint{0.696435in}{1.249807in}}{\pgfqpoint{5.499722in}{3.121436in}}%
\pgfusepath{clip}%
\pgfsetbuttcap%
\pgfsetmiterjoin%
\definecolor{currentfill}{rgb}{1.000000,0.498039,0.054902}%
\pgfsetfillcolor{currentfill}%
\pgfsetlinewidth{0.000000pt}%
\definecolor{currentstroke}{rgb}{0.000000,0.000000,0.000000}%
\pgfsetstrokecolor{currentstroke}%
\pgfsetstrokeopacity{0.000000}%
\pgfsetdash{}{0pt}%
\pgfpathmoveto{\pgfqpoint{5.620100in}{1.249807in}}%
\pgfpathlineto{\pgfqpoint{5.946170in}{1.249807in}}%
\pgfpathlineto{\pgfqpoint{5.946170in}{3.250243in}}%
\pgfpathlineto{\pgfqpoint{5.620100in}{3.250243in}}%
\pgfpathclose%
\pgfusepath{fill}%
\end{pgfscope}%
\begin{pgfscope}%
\pgfsetbuttcap%
\pgfsetroundjoin%
\definecolor{currentfill}{rgb}{0.000000,0.000000,0.000000}%
\pgfsetfillcolor{currentfill}%
\pgfsetlinewidth{0.803000pt}%
\definecolor{currentstroke}{rgb}{0.000000,0.000000,0.000000}%
\pgfsetstrokecolor{currentstroke}%
\pgfsetdash{}{0pt}%
\pgfsys@defobject{currentmarker}{\pgfqpoint{0.000000in}{-0.048611in}}{\pgfqpoint{0.000000in}{0.000000in}}{%
\pgfpathmoveto{\pgfqpoint{0.000000in}{0.000000in}}%
\pgfpathlineto{\pgfqpoint{0.000000in}{-0.048611in}}%
\pgfusepath{stroke,fill}%
}%
\begin{pgfscope}%
\pgfsys@transformshift{1.272493in}{1.249807in}%
\pgfsys@useobject{currentmarker}{}%
\end{pgfscope}%
\end{pgfscope}%
\begin{pgfscope}%
\definecolor{textcolor}{rgb}{0.000000,0.000000,0.000000}%
\pgfsetstrokecolor{textcolor}%
\pgfsetfillcolor{textcolor}%
\pgftext[x=0.820817in, y=0.120428in, left, base,rotate=45.000000]{\color{textcolor}\sffamily\fontsize{10.000000}{12.000000}\selectfont Pix3d(chair,no aug)}%
\end{pgfscope}%
\begin{pgfscope}%
\pgfsetbuttcap%
\pgfsetroundjoin%
\definecolor{currentfill}{rgb}{0.000000,0.000000,0.000000}%
\pgfsetfillcolor{currentfill}%
\pgfsetlinewidth{0.803000pt}%
\definecolor{currentstroke}{rgb}{0.000000,0.000000,0.000000}%
\pgfsetstrokecolor{currentstroke}%
\pgfsetdash{}{0pt}%
\pgfsys@defobject{currentmarker}{\pgfqpoint{0.000000in}{-0.048611in}}{\pgfqpoint{0.000000in}{0.000000in}}{%
\pgfpathmoveto{\pgfqpoint{0.000000in}{0.000000in}}%
\pgfpathlineto{\pgfqpoint{0.000000in}{-0.048611in}}%
\pgfusepath{stroke,fill}%
}%
\begin{pgfscope}%
\pgfsys@transformshift{1.997094in}{1.249807in}%
\pgfsys@useobject{currentmarker}{}%
\end{pgfscope}%
\end{pgfscope}%
\begin{pgfscope}%
\definecolor{textcolor}{rgb}{0.000000,0.000000,0.000000}%
\pgfsetstrokecolor{textcolor}%
\pgfsetfillcolor{textcolor}%
\pgftext[x=1.730184in, y=0.489960in, left, base,rotate=45.000000]{\color{textcolor}\sffamily\fontsize{10.000000}{12.000000}\selectfont Pix3d(chair)}%
\end{pgfscope}%
\begin{pgfscope}%
\pgfsetbuttcap%
\pgfsetroundjoin%
\definecolor{currentfill}{rgb}{0.000000,0.000000,0.000000}%
\pgfsetfillcolor{currentfill}%
\pgfsetlinewidth{0.803000pt}%
\definecolor{currentstroke}{rgb}{0.000000,0.000000,0.000000}%
\pgfsetstrokecolor{currentstroke}%
\pgfsetdash{}{0pt}%
\pgfsys@defobject{currentmarker}{\pgfqpoint{0.000000in}{-0.048611in}}{\pgfqpoint{0.000000in}{0.000000in}}{%
\pgfpathmoveto{\pgfqpoint{0.000000in}{0.000000in}}%
\pgfpathlineto{\pgfqpoint{0.000000in}{-0.048611in}}%
\pgfusepath{stroke,fill}%
}%
\begin{pgfscope}%
\pgfsys@transformshift{2.721695in}{1.249807in}%
\pgfsys@useobject{currentmarker}{}%
\end{pgfscope}%
\end{pgfscope}%
\begin{pgfscope}%
\definecolor{textcolor}{rgb}{0.000000,0.000000,0.000000}%
\pgfsetstrokecolor{textcolor}%
\pgfsetfillcolor{textcolor}%
\pgftext[x=2.474014in, y=0.528419in, left, base,rotate=45.000000]{\color{textcolor}\sffamily\fontsize{10.000000}{12.000000}\selectfont Textureless}%
\end{pgfscope}%
\begin{pgfscope}%
\pgfsetbuttcap%
\pgfsetroundjoin%
\definecolor{currentfill}{rgb}{0.000000,0.000000,0.000000}%
\pgfsetfillcolor{currentfill}%
\pgfsetlinewidth{0.803000pt}%
\definecolor{currentstroke}{rgb}{0.000000,0.000000,0.000000}%
\pgfsetstrokecolor{currentstroke}%
\pgfsetdash{}{0pt}%
\pgfsys@defobject{currentmarker}{\pgfqpoint{0.000000in}{-0.048611in}}{\pgfqpoint{0.000000in}{0.000000in}}{%
\pgfpathmoveto{\pgfqpoint{0.000000in}{0.000000in}}%
\pgfpathlineto{\pgfqpoint{0.000000in}{-0.048611in}}%
\pgfusepath{stroke,fill}%
}%
\begin{pgfscope}%
\pgfsys@transformshift{3.446297in}{1.249807in}%
\pgfsys@useobject{currentmarker}{}%
\end{pgfscope}%
\end{pgfscope}%
\begin{pgfscope}%
\definecolor{textcolor}{rgb}{0.000000,0.000000,0.000000}%
\pgfsetstrokecolor{textcolor}%
\pgfsetfillcolor{textcolor}%
\pgftext[x=3.034925in, y=0.201038in, left, base,rotate=45.000000]{\color{textcolor}\sffamily\fontsize{10.000000}{12.000000}\selectfont Textureless+Light}%
\end{pgfscope}%
\begin{pgfscope}%
\pgfsetbuttcap%
\pgfsetroundjoin%
\definecolor{currentfill}{rgb}{0.000000,0.000000,0.000000}%
\pgfsetfillcolor{currentfill}%
\pgfsetlinewidth{0.803000pt}%
\definecolor{currentstroke}{rgb}{0.000000,0.000000,0.000000}%
\pgfsetstrokecolor{currentstroke}%
\pgfsetdash{}{0pt}%
\pgfsys@defobject{currentmarker}{\pgfqpoint{0.000000in}{-0.048611in}}{\pgfqpoint{0.000000in}{0.000000in}}{%
\pgfpathmoveto{\pgfqpoint{0.000000in}{0.000000in}}%
\pgfpathlineto{\pgfqpoint{0.000000in}{-0.048611in}}%
\pgfusepath{stroke,fill}%
}%
\begin{pgfscope}%
\pgfsys@transformshift{4.170898in}{1.249807in}%
\pgfsys@useobject{currentmarker}{}%
\end{pgfscope}%
\end{pgfscope}%
\begin{pgfscope}%
\definecolor{textcolor}{rgb}{0.000000,0.000000,0.000000}%
\pgfsetstrokecolor{textcolor}%
\pgfsetfillcolor{textcolor}%
\pgftext[x=3.987067in, y=0.656119in, left, base,rotate=45.000000]{\color{textcolor}\sffamily\fontsize{10.000000}{12.000000}\selectfont Textured}%
\end{pgfscope}%
\begin{pgfscope}%
\pgfsetbuttcap%
\pgfsetroundjoin%
\definecolor{currentfill}{rgb}{0.000000,0.000000,0.000000}%
\pgfsetfillcolor{currentfill}%
\pgfsetlinewidth{0.803000pt}%
\definecolor{currentstroke}{rgb}{0.000000,0.000000,0.000000}%
\pgfsetstrokecolor{currentstroke}%
\pgfsetdash{}{0pt}%
\pgfsys@defobject{currentmarker}{\pgfqpoint{0.000000in}{-0.048611in}}{\pgfqpoint{0.000000in}{0.000000in}}{%
\pgfpathmoveto{\pgfqpoint{0.000000in}{0.000000in}}%
\pgfpathlineto{\pgfqpoint{0.000000in}{-0.048611in}}%
\pgfusepath{stroke,fill}%
}%
\begin{pgfscope}%
\pgfsys@transformshift{4.895499in}{1.249807in}%
\pgfsys@useobject{currentmarker}{}%
\end{pgfscope}%
\end{pgfscope}%
\begin{pgfscope}%
\definecolor{textcolor}{rgb}{0.000000,0.000000,0.000000}%
\pgfsetstrokecolor{textcolor}%
\pgfsetfillcolor{textcolor}%
\pgftext[x=4.547978in, y=0.328739in, left, base,rotate=45.000000]{\color{textcolor}\sffamily\fontsize{10.000000}{12.000000}\selectfont Textured+Light}%
\end{pgfscope}%
\begin{pgfscope}%
\pgfsetbuttcap%
\pgfsetroundjoin%
\definecolor{currentfill}{rgb}{0.000000,0.000000,0.000000}%
\pgfsetfillcolor{currentfill}%
\pgfsetlinewidth{0.803000pt}%
\definecolor{currentstroke}{rgb}{0.000000,0.000000,0.000000}%
\pgfsetstrokecolor{currentstroke}%
\pgfsetdash{}{0pt}%
\pgfsys@defobject{currentmarker}{\pgfqpoint{0.000000in}{-0.048611in}}{\pgfqpoint{0.000000in}{0.000000in}}{%
\pgfpathmoveto{\pgfqpoint{0.000000in}{0.000000in}}%
\pgfpathlineto{\pgfqpoint{0.000000in}{-0.048611in}}%
\pgfusepath{stroke,fill}%
}%
\begin{pgfscope}%
\pgfsys@transformshift{5.620100in}{1.249807in}%
\pgfsys@useobject{currentmarker}{}%
\end{pgfscope}%
\end{pgfscope}%
\begin{pgfscope}%
\definecolor{textcolor}{rgb}{0.000000,0.000000,0.000000}%
\pgfsetstrokecolor{textcolor}%
\pgfsetfillcolor{textcolor}%
\pgftext[x=5.348154in, y=0.479889in, left, base,rotate=45.000000]{\color{textcolor}\sffamily\fontsize{10.000000}{12.000000}\selectfont Multi-Object}%
\end{pgfscope}%
\begin{pgfscope}%
\pgfsetbuttcap%
\pgfsetroundjoin%
\definecolor{currentfill}{rgb}{0.000000,0.000000,0.000000}%
\pgfsetfillcolor{currentfill}%
\pgfsetlinewidth{0.803000pt}%
\definecolor{currentstroke}{rgb}{0.000000,0.000000,0.000000}%
\pgfsetstrokecolor{currentstroke}%
\pgfsetdash{}{0pt}%
\pgfsys@defobject{currentmarker}{\pgfqpoint{-0.048611in}{0.000000in}}{\pgfqpoint{-0.000000in}{0.000000in}}{%
\pgfpathmoveto{\pgfqpoint{-0.000000in}{0.000000in}}%
\pgfpathlineto{\pgfqpoint{-0.048611in}{0.000000in}}%
\pgfusepath{stroke,fill}%
}%
\begin{pgfscope}%
\pgfsys@transformshift{0.696435in}{1.249807in}%
\pgfsys@useobject{currentmarker}{}%
\end{pgfscope}%
\end{pgfscope}%
\begin{pgfscope}%
\definecolor{textcolor}{rgb}{0.000000,0.000000,0.000000}%
\pgfsetstrokecolor{textcolor}%
\pgfsetfillcolor{textcolor}%
\pgftext[x=0.289968in, y=1.197045in, left, base]{\color{textcolor}\sffamily\fontsize{10.000000}{12.000000}\selectfont 0.00}%
\end{pgfscope}%
\begin{pgfscope}%
\pgfsetbuttcap%
\pgfsetroundjoin%
\definecolor{currentfill}{rgb}{0.000000,0.000000,0.000000}%
\pgfsetfillcolor{currentfill}%
\pgfsetlinewidth{0.803000pt}%
\definecolor{currentstroke}{rgb}{0.000000,0.000000,0.000000}%
\pgfsetstrokecolor{currentstroke}%
\pgfsetdash{}{0pt}%
\pgfsys@defobject{currentmarker}{\pgfqpoint{-0.048611in}{0.000000in}}{\pgfqpoint{-0.000000in}{0.000000in}}{%
\pgfpathmoveto{\pgfqpoint{-0.000000in}{0.000000in}}%
\pgfpathlineto{\pgfqpoint{-0.048611in}{0.000000in}}%
\pgfusepath{stroke,fill}%
}%
\begin{pgfscope}%
\pgfsys@transformshift{0.696435in}{1.699141in}%
\pgfsys@useobject{currentmarker}{}%
\end{pgfscope}%
\end{pgfscope}%
\begin{pgfscope}%
\definecolor{textcolor}{rgb}{0.000000,0.000000,0.000000}%
\pgfsetstrokecolor{textcolor}%
\pgfsetfillcolor{textcolor}%
\pgftext[x=0.289968in, y=1.646380in, left, base]{\color{textcolor}\sffamily\fontsize{10.000000}{12.000000}\selectfont 0.05}%
\end{pgfscope}%
\begin{pgfscope}%
\pgfsetbuttcap%
\pgfsetroundjoin%
\definecolor{currentfill}{rgb}{0.000000,0.000000,0.000000}%
\pgfsetfillcolor{currentfill}%
\pgfsetlinewidth{0.803000pt}%
\definecolor{currentstroke}{rgb}{0.000000,0.000000,0.000000}%
\pgfsetstrokecolor{currentstroke}%
\pgfsetdash{}{0pt}%
\pgfsys@defobject{currentmarker}{\pgfqpoint{-0.048611in}{0.000000in}}{\pgfqpoint{-0.000000in}{0.000000in}}{%
\pgfpathmoveto{\pgfqpoint{-0.000000in}{0.000000in}}%
\pgfpathlineto{\pgfqpoint{-0.048611in}{0.000000in}}%
\pgfusepath{stroke,fill}%
}%
\begin{pgfscope}%
\pgfsys@transformshift{0.696435in}{2.148476in}%
\pgfsys@useobject{currentmarker}{}%
\end{pgfscope}%
\end{pgfscope}%
\begin{pgfscope}%
\definecolor{textcolor}{rgb}{0.000000,0.000000,0.000000}%
\pgfsetstrokecolor{textcolor}%
\pgfsetfillcolor{textcolor}%
\pgftext[x=0.289968in, y=2.095714in, left, base]{\color{textcolor}\sffamily\fontsize{10.000000}{12.000000}\selectfont 0.10}%
\end{pgfscope}%
\begin{pgfscope}%
\pgfsetbuttcap%
\pgfsetroundjoin%
\definecolor{currentfill}{rgb}{0.000000,0.000000,0.000000}%
\pgfsetfillcolor{currentfill}%
\pgfsetlinewidth{0.803000pt}%
\definecolor{currentstroke}{rgb}{0.000000,0.000000,0.000000}%
\pgfsetstrokecolor{currentstroke}%
\pgfsetdash{}{0pt}%
\pgfsys@defobject{currentmarker}{\pgfqpoint{-0.048611in}{0.000000in}}{\pgfqpoint{-0.000000in}{0.000000in}}{%
\pgfpathmoveto{\pgfqpoint{-0.000000in}{0.000000in}}%
\pgfpathlineto{\pgfqpoint{-0.048611in}{0.000000in}}%
\pgfusepath{stroke,fill}%
}%
\begin{pgfscope}%
\pgfsys@transformshift{0.696435in}{2.597810in}%
\pgfsys@useobject{currentmarker}{}%
\end{pgfscope}%
\end{pgfscope}%
\begin{pgfscope}%
\definecolor{textcolor}{rgb}{0.000000,0.000000,0.000000}%
\pgfsetstrokecolor{textcolor}%
\pgfsetfillcolor{textcolor}%
\pgftext[x=0.289968in, y=2.545048in, left, base]{\color{textcolor}\sffamily\fontsize{10.000000}{12.000000}\selectfont 0.15}%
\end{pgfscope}%
\begin{pgfscope}%
\pgfsetbuttcap%
\pgfsetroundjoin%
\definecolor{currentfill}{rgb}{0.000000,0.000000,0.000000}%
\pgfsetfillcolor{currentfill}%
\pgfsetlinewidth{0.803000pt}%
\definecolor{currentstroke}{rgb}{0.000000,0.000000,0.000000}%
\pgfsetstrokecolor{currentstroke}%
\pgfsetdash{}{0pt}%
\pgfsys@defobject{currentmarker}{\pgfqpoint{-0.048611in}{0.000000in}}{\pgfqpoint{-0.000000in}{0.000000in}}{%
\pgfpathmoveto{\pgfqpoint{-0.000000in}{0.000000in}}%
\pgfpathlineto{\pgfqpoint{-0.048611in}{0.000000in}}%
\pgfusepath{stroke,fill}%
}%
\begin{pgfscope}%
\pgfsys@transformshift{0.696435in}{3.047144in}%
\pgfsys@useobject{currentmarker}{}%
\end{pgfscope}%
\end{pgfscope}%
\begin{pgfscope}%
\definecolor{textcolor}{rgb}{0.000000,0.000000,0.000000}%
\pgfsetstrokecolor{textcolor}%
\pgfsetfillcolor{textcolor}%
\pgftext[x=0.289968in, y=2.994383in, left, base]{\color{textcolor}\sffamily\fontsize{10.000000}{12.000000}\selectfont 0.20}%
\end{pgfscope}%
\begin{pgfscope}%
\pgfsetbuttcap%
\pgfsetroundjoin%
\definecolor{currentfill}{rgb}{0.000000,0.000000,0.000000}%
\pgfsetfillcolor{currentfill}%
\pgfsetlinewidth{0.803000pt}%
\definecolor{currentstroke}{rgb}{0.000000,0.000000,0.000000}%
\pgfsetstrokecolor{currentstroke}%
\pgfsetdash{}{0pt}%
\pgfsys@defobject{currentmarker}{\pgfqpoint{-0.048611in}{0.000000in}}{\pgfqpoint{-0.000000in}{0.000000in}}{%
\pgfpathmoveto{\pgfqpoint{-0.000000in}{0.000000in}}%
\pgfpathlineto{\pgfqpoint{-0.048611in}{0.000000in}}%
\pgfusepath{stroke,fill}%
}%
\begin{pgfscope}%
\pgfsys@transformshift{0.696435in}{3.496479in}%
\pgfsys@useobject{currentmarker}{}%
\end{pgfscope}%
\end{pgfscope}%
\begin{pgfscope}%
\definecolor{textcolor}{rgb}{0.000000,0.000000,0.000000}%
\pgfsetstrokecolor{textcolor}%
\pgfsetfillcolor{textcolor}%
\pgftext[x=0.289968in, y=3.443717in, left, base]{\color{textcolor}\sffamily\fontsize{10.000000}{12.000000}\selectfont 0.25}%
\end{pgfscope}%
\begin{pgfscope}%
\pgfsetbuttcap%
\pgfsetroundjoin%
\definecolor{currentfill}{rgb}{0.000000,0.000000,0.000000}%
\pgfsetfillcolor{currentfill}%
\pgfsetlinewidth{0.803000pt}%
\definecolor{currentstroke}{rgb}{0.000000,0.000000,0.000000}%
\pgfsetstrokecolor{currentstroke}%
\pgfsetdash{}{0pt}%
\pgfsys@defobject{currentmarker}{\pgfqpoint{-0.048611in}{0.000000in}}{\pgfqpoint{-0.000000in}{0.000000in}}{%
\pgfpathmoveto{\pgfqpoint{-0.000000in}{0.000000in}}%
\pgfpathlineto{\pgfqpoint{-0.048611in}{0.000000in}}%
\pgfusepath{stroke,fill}%
}%
\begin{pgfscope}%
\pgfsys@transformshift{0.696435in}{3.945813in}%
\pgfsys@useobject{currentmarker}{}%
\end{pgfscope}%
\end{pgfscope}%
\begin{pgfscope}%
\definecolor{textcolor}{rgb}{0.000000,0.000000,0.000000}%
\pgfsetstrokecolor{textcolor}%
\pgfsetfillcolor{textcolor}%
\pgftext[x=0.289968in, y=3.893051in, left, base]{\color{textcolor}\sffamily\fontsize{10.000000}{12.000000}\selectfont 0.30}%
\end{pgfscope}%
\begin{pgfscope}%
\definecolor{textcolor}{rgb}{0.000000,0.000000,0.000000}%
\pgfsetstrokecolor{textcolor}%
\pgfsetfillcolor{textcolor}%
\pgftext[x=0.234413in,y=2.810525in,,bottom,rotate=90.000000]{\color{textcolor}\sffamily\fontsize{10.000000}{12.000000}\selectfont IoU}%
\end{pgfscope}%
\begin{pgfscope}%
\pgfsetrectcap%
\pgfsetmiterjoin%
\pgfsetlinewidth{0.803000pt}%
\definecolor{currentstroke}{rgb}{0.000000,0.000000,0.000000}%
\pgfsetstrokecolor{currentstroke}%
\pgfsetdash{}{0pt}%
\pgfpathmoveto{\pgfqpoint{0.696435in}{1.249807in}}%
\pgfpathlineto{\pgfqpoint{0.696435in}{4.371243in}}%
\pgfusepath{stroke}%
\end{pgfscope}%
\begin{pgfscope}%
\pgfsetrectcap%
\pgfsetmiterjoin%
\pgfsetlinewidth{0.803000pt}%
\definecolor{currentstroke}{rgb}{0.000000,0.000000,0.000000}%
\pgfsetstrokecolor{currentstroke}%
\pgfsetdash{}{0pt}%
\pgfpathmoveto{\pgfqpoint{6.196158in}{1.249807in}}%
\pgfpathlineto{\pgfqpoint{6.196158in}{4.371243in}}%
\pgfusepath{stroke}%
\end{pgfscope}%
\begin{pgfscope}%
\pgfsetrectcap%
\pgfsetmiterjoin%
\pgfsetlinewidth{0.803000pt}%
\definecolor{currentstroke}{rgb}{0.000000,0.000000,0.000000}%
\pgfsetstrokecolor{currentstroke}%
\pgfsetdash{}{0pt}%
\pgfpathmoveto{\pgfqpoint{0.696435in}{1.249807in}}%
\pgfpathlineto{\pgfqpoint{6.196158in}{1.249807in}}%
\pgfusepath{stroke}%
\end{pgfscope}%
\begin{pgfscope}%
\pgfsetrectcap%
\pgfsetmiterjoin%
\pgfsetlinewidth{0.803000pt}%
\definecolor{currentstroke}{rgb}{0.000000,0.000000,0.000000}%
\pgfsetstrokecolor{currentstroke}%
\pgfsetdash{}{0pt}%
\pgfpathmoveto{\pgfqpoint{0.696435in}{4.371243in}}%
\pgfpathlineto{\pgfqpoint{6.196158in}{4.371243in}}%
\pgfusepath{stroke}%
\end{pgfscope}%
\begin{pgfscope}%
\definecolor{textcolor}{rgb}{0.000000,0.000000,0.000000}%
\pgfsetstrokecolor{textcolor}%
\pgfsetfillcolor{textcolor}%
\pgftext[x=1.109458in,y=4.018933in,,bottom]{\color{textcolor}\sffamily\fontsize{10.000000}{12.000000}\selectfont 0.3035}%
\end{pgfscope}%
\begin{pgfscope}%
\definecolor{textcolor}{rgb}{0.000000,0.000000,0.000000}%
\pgfsetstrokecolor{textcolor}%
\pgfsetfillcolor{textcolor}%
\pgftext[x=1.834059in,y=4.264270in,,bottom]{\color{textcolor}\sffamily\fontsize{10.000000}{12.000000}\selectfont 0.3308}%
\end{pgfscope}%
\begin{pgfscope}%
\definecolor{textcolor}{rgb}{0.000000,0.000000,0.000000}%
\pgfsetstrokecolor{textcolor}%
\pgfsetfillcolor{textcolor}%
\pgftext[x=2.558660in,y=3.530956in,,bottom]{\color{textcolor}\sffamily\fontsize{10.000000}{12.000000}\selectfont 0.2492}%
\end{pgfscope}%
\begin{pgfscope}%
\definecolor{textcolor}{rgb}{0.000000,0.000000,0.000000}%
\pgfsetstrokecolor{textcolor}%
\pgfsetfillcolor{textcolor}%
\pgftext[x=3.283261in,y=3.479732in,,bottom]{\color{textcolor}\sffamily\fontsize{10.000000}{12.000000}\selectfont 0.2435}%
\end{pgfscope}%
\begin{pgfscope}%
\definecolor{textcolor}{rgb}{0.000000,0.000000,0.000000}%
\pgfsetstrokecolor{textcolor}%
\pgfsetfillcolor{textcolor}%
\pgftext[x=4.007862in,y=3.407838in,,bottom]{\color{textcolor}\sffamily\fontsize{10.000000}{12.000000}\selectfont 0.2355}%
\end{pgfscope}%
\begin{pgfscope}%
\definecolor{textcolor}{rgb}{0.000000,0.000000,0.000000}%
\pgfsetstrokecolor{textcolor}%
\pgfsetfillcolor{textcolor}%
\pgftext[x=4.732463in,y=3.264950in,,bottom]{\color{textcolor}\sffamily\fontsize{10.000000}{12.000000}\selectfont 0.2196}%
\end{pgfscope}%
\begin{pgfscope}%
\definecolor{textcolor}{rgb}{0.000000,0.000000,0.000000}%
\pgfsetstrokecolor{textcolor}%
\pgfsetfillcolor{textcolor}%
\pgftext[x=5.457065in,y=3.453670in,,bottom]{\color{textcolor}\sffamily\fontsize{10.000000}{12.000000}\selectfont 0.2406}%
\end{pgfscope}%
\begin{pgfscope}%
\definecolor{textcolor}{rgb}{0.000000,0.000000,0.000000}%
\pgfsetstrokecolor{textcolor}%
\pgfsetfillcolor{textcolor}%
\pgftext[x=1.435529in,y=3.685527in,,bottom]{\color{textcolor}\sffamily\fontsize{10.000000}{12.000000}\selectfont 0.2664}%
\end{pgfscope}%
\begin{pgfscope}%
\definecolor{textcolor}{rgb}{0.000000,0.000000,0.000000}%
\pgfsetstrokecolor{textcolor}%
\pgfsetfillcolor{textcolor}%
\pgftext[x=2.160130in,y=3.903903in,,bottom]{\color{textcolor}\sffamily\fontsize{10.000000}{12.000000}\selectfont 0.2907}%
\end{pgfscope}%
\begin{pgfscope}%
\definecolor{textcolor}{rgb}{0.000000,0.000000,0.000000}%
\pgfsetstrokecolor{textcolor}%
\pgfsetfillcolor{textcolor}%
\pgftext[x=2.884731in,y=2.950416in,,bottom]{\color{textcolor}\sffamily\fontsize{10.000000}{12.000000}\selectfont 0.1846}%
\end{pgfscope}%
\begin{pgfscope}%
\definecolor{textcolor}{rgb}{0.000000,0.000000,0.000000}%
\pgfsetstrokecolor{textcolor}%
\pgfsetfillcolor{textcolor}%
\pgftext[x=3.609332in,y=3.461758in,,bottom]{\color{textcolor}\sffamily\fontsize{10.000000}{12.000000}\selectfont 0.2415}%
\end{pgfscope}%
\begin{pgfscope}%
\definecolor{textcolor}{rgb}{0.000000,0.000000,0.000000}%
\pgfsetstrokecolor{textcolor}%
\pgfsetfillcolor{textcolor}%
\pgftext[x=4.333933in,y=3.205638in,,bottom]{\color{textcolor}\sffamily\fontsize{10.000000}{12.000000}\selectfont 0.213}%
\end{pgfscope}%
\begin{pgfscope}%
\definecolor{textcolor}{rgb}{0.000000,0.000000,0.000000}%
\pgfsetstrokecolor{textcolor}%
\pgfsetfillcolor{textcolor}%
\pgftext[x=5.058534in,y=2.679018in,,bottom]{\color{textcolor}\sffamily\fontsize{10.000000}{12.000000}\selectfont 0.1544}%
\end{pgfscope}%
\begin{pgfscope}%
\definecolor{textcolor}{rgb}{0.000000,0.000000,0.000000}%
\pgfsetstrokecolor{textcolor}%
\pgfsetfillcolor{textcolor}%
\pgftext[x=5.783135in,y=3.291910in,,bottom]{\color{textcolor}\sffamily\fontsize{10.000000}{12.000000}\selectfont 0.2226}%
\end{pgfscope}%
\begin{pgfscope}%
\definecolor{textcolor}{rgb}{0.000000,0.000000,0.000000}%
\pgfsetstrokecolor{textcolor}%
\pgfsetfillcolor{textcolor}%
\pgftext[x=3.446297in,y=4.454576in,,base]{\color{textcolor}\sffamily\fontsize{12.000000}{14.400000}\selectfont Abalation study on chairs}%
\end{pgfscope}%
\begin{pgfscope}%
\pgfsetbuttcap%
\pgfsetmiterjoin%
\definecolor{currentfill}{rgb}{1.000000,1.000000,1.000000}%
\pgfsetfillcolor{currentfill}%
\pgfsetfillopacity{0.800000}%
\pgfsetlinewidth{1.003750pt}%
\definecolor{currentstroke}{rgb}{0.800000,0.800000,0.800000}%
\pgfsetstrokecolor{currentstroke}%
\pgfsetstrokeopacity{0.800000}%
\pgfsetdash{}{0pt}%
\pgfpathmoveto{\pgfqpoint{6.293380in}{2.585834in}}%
\pgfpathlineto{\pgfqpoint{7.507545in}{2.585834in}}%
\pgfpathquadraticcurveto{\pgfqpoint{7.535323in}{2.585834in}}{\pgfqpoint{7.535323in}{2.613612in}}%
\pgfpathlineto{\pgfqpoint{7.535323in}{3.007438in}}%
\pgfpathquadraticcurveto{\pgfqpoint{7.535323in}{3.035215in}}{\pgfqpoint{7.507545in}{3.035215in}}%
\pgfpathlineto{\pgfqpoint{6.293380in}{3.035215in}}%
\pgfpathquadraticcurveto{\pgfqpoint{6.265602in}{3.035215in}}{\pgfqpoint{6.265602in}{3.007438in}}%
\pgfpathlineto{\pgfqpoint{6.265602in}{2.613612in}}%
\pgfpathquadraticcurveto{\pgfqpoint{6.265602in}{2.585834in}}{\pgfqpoint{6.293380in}{2.585834in}}%
\pgfpathclose%
\pgfusepath{stroke,fill}%
\end{pgfscope}%
\begin{pgfscope}%
\pgfsetbuttcap%
\pgfsetmiterjoin%
\definecolor{currentfill}{rgb}{0.121569,0.466667,0.705882}%
\pgfsetfillcolor{currentfill}%
\pgfsetlinewidth{0.000000pt}%
\definecolor{currentstroke}{rgb}{0.000000,0.000000,0.000000}%
\pgfsetstrokecolor{currentstroke}%
\pgfsetstrokeopacity{0.000000}%
\pgfsetdash{}{0pt}%
\pgfpathmoveto{\pgfqpoint{6.321158in}{2.874137in}}%
\pgfpathlineto{\pgfqpoint{6.598935in}{2.874137in}}%
\pgfpathlineto{\pgfqpoint{6.598935in}{2.971359in}}%
\pgfpathlineto{\pgfqpoint{6.321158in}{2.971359in}}%
\pgfpathclose%
\pgfusepath{fill}%
\end{pgfscope}%
\begin{pgfscope}%
\definecolor{textcolor}{rgb}{0.000000,0.000000,0.000000}%
\pgfsetstrokecolor{textcolor}%
\pgfsetfillcolor{textcolor}%
\pgftext[x=6.710047in,y=2.874137in,left,base]{\color{textcolor}\sffamily\fontsize{10.000000}{12.000000}\selectfont Pix2Vox++}%
\end{pgfscope}%
\begin{pgfscope}%
\pgfsetbuttcap%
\pgfsetmiterjoin%
\definecolor{currentfill}{rgb}{1.000000,0.498039,0.054902}%
\pgfsetfillcolor{currentfill}%
\pgfsetlinewidth{0.000000pt}%
\definecolor{currentstroke}{rgb}{0.000000,0.000000,0.000000}%
\pgfsetstrokecolor{currentstroke}%
\pgfsetstrokeopacity{0.000000}%
\pgfsetdash{}{0pt}%
\pgfpathmoveto{\pgfqpoint{6.321158in}{2.670280in}}%
\pgfpathlineto{\pgfqpoint{6.598935in}{2.670280in}}%
\pgfpathlineto{\pgfqpoint{6.598935in}{2.767502in}}%
\pgfpathlineto{\pgfqpoint{6.321158in}{2.767502in}}%
\pgfpathclose%
\pgfusepath{fill}%
\end{pgfscope}%
\begin{pgfscope}%
\definecolor{textcolor}{rgb}{0.000000,0.000000,0.000000}%
\pgfsetstrokecolor{textcolor}%
\pgfsetfillcolor{textcolor}%
\pgftext[x=6.710047in,y=2.670280in,left,base]{\color{textcolor}\sffamily\fontsize{10.000000}{12.000000}\selectfont Pix2Vox}%
\end{pgfscope}%
\end{pgfpicture}%
\makeatother%
\endgroup%
}
    \caption{Bar plot for the \gls{iou}  for baseline trained on chair dataset with different parameters and tested on real dataset.}
    \label{fig:ablation1}
\end{figure}
\todo{make sure bar values are not overlapping}

\subsection{Domain randomisation with Mixed training}\label{subsec:domain-randomisation-with-mixed-training}

To reiterate,the performance on real chair dataset from Pix3D is 0.3035 and 0.3308, 0.2664 and 0.2907 without and with 2D augmentations on pix2vox++ and pix2vox respectively.
In the mixed training with ratio of 50\%, we see a maximum increase of 3.4\% increment in pix2vox++ and 5.24\% in pix2vox.
The behavior of models for the different randomisation parameters is similar to what we observed in ~\ref{subsec:domain-randomisation-on-chair-dataset}.
For pix2vox++, the textureless chair dataset gives the best performance, with gradual decrease with addition of each parameter and slight increase for multi-object dataset.
Similarly, for pix2vox the gradual decrease is observed, but multi-object gives better performance than the textureless dataset.
These observations are seen in figure~\ref{fig:ablation2}.

\begin{figure}
    \centering
    \resizebox{\textwidth}{!}{%% Creator: Matplotlib, PGF backend
%%
%% To include the figure in your LaTeX document, write
%%   \input{<filename>.pgf}
%%
%% Make sure the required packages are loaded in your preamble
%%   \usepackage{pgf}
%%
%% Figures using additional raster images can only be included by \input if
%% they are in the same directory as the main LaTeX file. For loading figures
%% from other directories you can use the `import` package
%%   \usepackage{import}
%%
%% and then include the figures with
%%   \import{<path to file>}{<filename>.pgf}
%%
%% Matplotlib used the following preamble
%%   \usepackage{fontspec}
%%   \setmainfont{DejaVuSerif.ttf}[Path=\detokenize{/Users/apple/opt/anaconda3/envs/kaolin/lib/python3.7/site-packages/matplotlib/mpl-data/fonts/ttf/}]
%%   \setsansfont{DejaVuSans.ttf}[Path=\detokenize{/Users/apple/opt/anaconda3/envs/kaolin/lib/python3.7/site-packages/matplotlib/mpl-data/fonts/ttf/}]
%%   \setmonofont{DejaVuSansMono.ttf}[Path=\detokenize{/Users/apple/opt/anaconda3/envs/kaolin/lib/python3.7/site-packages/matplotlib/mpl-data/fonts/ttf/}]
%%
\begingroup%
\makeatletter%
\begin{pgfpicture}%
\pgfpathrectangle{\pgfpointorigin}{\pgfqpoint{7.635323in}{4.683679in}}%
\pgfusepath{use as bounding box, clip}%
\begin{pgfscope}%
\pgfsetbuttcap%
\pgfsetmiterjoin%
\definecolor{currentfill}{rgb}{1.000000,1.000000,1.000000}%
\pgfsetfillcolor{currentfill}%
\pgfsetlinewidth{0.000000pt}%
\definecolor{currentstroke}{rgb}{1.000000,1.000000,1.000000}%
\pgfsetstrokecolor{currentstroke}%
\pgfsetdash{}{0pt}%
\pgfpathmoveto{\pgfqpoint{0.000000in}{0.000000in}}%
\pgfpathlineto{\pgfqpoint{7.635323in}{0.000000in}}%
\pgfpathlineto{\pgfqpoint{7.635323in}{4.683679in}}%
\pgfpathlineto{\pgfqpoint{0.000000in}{4.683679in}}%
\pgfpathclose%
\pgfusepath{fill}%
\end{pgfscope}%
\begin{pgfscope}%
\pgfsetbuttcap%
\pgfsetmiterjoin%
\definecolor{currentfill}{rgb}{1.000000,1.000000,1.000000}%
\pgfsetfillcolor{currentfill}%
\pgfsetlinewidth{0.000000pt}%
\definecolor{currentstroke}{rgb}{0.000000,0.000000,0.000000}%
\pgfsetstrokecolor{currentstroke}%
\pgfsetstrokeopacity{0.000000}%
\pgfsetdash{}{0pt}%
\pgfpathmoveto{\pgfqpoint{0.696435in}{1.169197in}}%
\pgfpathlineto{\pgfqpoint{6.196158in}{1.169197in}}%
\pgfpathlineto{\pgfqpoint{6.196158in}{4.373718in}}%
\pgfpathlineto{\pgfqpoint{0.696435in}{4.373718in}}%
\pgfpathclose%
\pgfusepath{fill}%
\end{pgfscope}%
\begin{pgfscope}%
\pgfpathrectangle{\pgfqpoint{0.696435in}{1.169197in}}{\pgfqpoint{5.499722in}{3.204521in}}%
\pgfusepath{clip}%
\pgfsetbuttcap%
\pgfsetmiterjoin%
\definecolor{currentfill}{rgb}{0.121569,0.466667,0.705882}%
\pgfsetfillcolor{currentfill}%
\pgfsetlinewidth{0.000000pt}%
\definecolor{currentstroke}{rgb}{0.000000,0.000000,0.000000}%
\pgfsetstrokecolor{currentstroke}%
\pgfsetstrokeopacity{0.000000}%
\pgfsetdash{}{0pt}%
\pgfpathmoveto{\pgfqpoint{0.946423in}{1.169197in}}%
\pgfpathlineto{\pgfqpoint{1.405583in}{1.169197in}}%
\pgfpathlineto{\pgfqpoint{1.405583in}{4.221121in}}%
\pgfpathlineto{\pgfqpoint{0.946423in}{4.221121in}}%
\pgfpathclose%
\pgfusepath{fill}%
\end{pgfscope}%
\begin{pgfscope}%
\pgfpathrectangle{\pgfqpoint{0.696435in}{1.169197in}}{\pgfqpoint{5.499722in}{3.204521in}}%
\pgfusepath{clip}%
\pgfsetbuttcap%
\pgfsetmiterjoin%
\definecolor{currentfill}{rgb}{0.121569,0.466667,0.705882}%
\pgfsetfillcolor{currentfill}%
\pgfsetlinewidth{0.000000pt}%
\definecolor{currentstroke}{rgb}{0.000000,0.000000,0.000000}%
\pgfsetstrokecolor{currentstroke}%
\pgfsetstrokeopacity{0.000000}%
\pgfsetdash{}{0pt}%
\pgfpathmoveto{\pgfqpoint{1.966779in}{1.169197in}}%
\pgfpathlineto{\pgfqpoint{2.425940in}{1.169197in}}%
\pgfpathlineto{\pgfqpoint{2.425940in}{4.194350in}}%
\pgfpathlineto{\pgfqpoint{1.966779in}{4.194350in}}%
\pgfpathclose%
\pgfusepath{fill}%
\end{pgfscope}%
\begin{pgfscope}%
\pgfpathrectangle{\pgfqpoint{0.696435in}{1.169197in}}{\pgfqpoint{5.499722in}{3.204521in}}%
\pgfusepath{clip}%
\pgfsetbuttcap%
\pgfsetmiterjoin%
\definecolor{currentfill}{rgb}{0.121569,0.466667,0.705882}%
\pgfsetfillcolor{currentfill}%
\pgfsetlinewidth{0.000000pt}%
\definecolor{currentstroke}{rgb}{0.000000,0.000000,0.000000}%
\pgfsetstrokecolor{currentstroke}%
\pgfsetstrokeopacity{0.000000}%
\pgfsetdash{}{0pt}%
\pgfpathmoveto{\pgfqpoint{2.987136in}{1.169197in}}%
\pgfpathlineto{\pgfqpoint{3.446297in}{1.169197in}}%
\pgfpathlineto{\pgfqpoint{3.446297in}{4.080572in}}%
\pgfpathlineto{\pgfqpoint{2.987136in}{4.080572in}}%
\pgfpathclose%
\pgfusepath{fill}%
\end{pgfscope}%
\begin{pgfscope}%
\pgfpathrectangle{\pgfqpoint{0.696435in}{1.169197in}}{\pgfqpoint{5.499722in}{3.204521in}}%
\pgfusepath{clip}%
\pgfsetbuttcap%
\pgfsetmiterjoin%
\definecolor{currentfill}{rgb}{0.121569,0.466667,0.705882}%
\pgfsetfillcolor{currentfill}%
\pgfsetlinewidth{0.000000pt}%
\definecolor{currentstroke}{rgb}{0.000000,0.000000,0.000000}%
\pgfsetstrokecolor{currentstroke}%
\pgfsetstrokeopacity{0.000000}%
\pgfsetdash{}{0pt}%
\pgfpathmoveto{\pgfqpoint{4.007493in}{1.169197in}}%
\pgfpathlineto{\pgfqpoint{4.466653in}{1.169197in}}%
\pgfpathlineto{\pgfqpoint{4.466653in}{4.034559in}}%
\pgfpathlineto{\pgfqpoint{4.007493in}{4.034559in}}%
\pgfpathclose%
\pgfusepath{fill}%
\end{pgfscope}%
\begin{pgfscope}%
\pgfpathrectangle{\pgfqpoint{0.696435in}{1.169197in}}{\pgfqpoint{5.499722in}{3.204521in}}%
\pgfusepath{clip}%
\pgfsetbuttcap%
\pgfsetmiterjoin%
\definecolor{currentfill}{rgb}{0.121569,0.466667,0.705882}%
\pgfsetfillcolor{currentfill}%
\pgfsetlinewidth{0.000000pt}%
\definecolor{currentstroke}{rgb}{0.000000,0.000000,0.000000}%
\pgfsetstrokecolor{currentstroke}%
\pgfsetstrokeopacity{0.000000}%
\pgfsetdash{}{0pt}%
\pgfpathmoveto{\pgfqpoint{5.027849in}{1.169197in}}%
\pgfpathlineto{\pgfqpoint{5.487010in}{1.169197in}}%
\pgfpathlineto{\pgfqpoint{5.487010in}{4.099814in}}%
\pgfpathlineto{\pgfqpoint{5.027849in}{4.099814in}}%
\pgfpathclose%
\pgfusepath{fill}%
\end{pgfscope}%
\begin{pgfscope}%
\pgfpathrectangle{\pgfqpoint{0.696435in}{1.169197in}}{\pgfqpoint{5.499722in}{3.204521in}}%
\pgfusepath{clip}%
\pgfsetbuttcap%
\pgfsetmiterjoin%
\definecolor{currentfill}{rgb}{1.000000,0.498039,0.054902}%
\pgfsetfillcolor{currentfill}%
\pgfsetlinewidth{0.000000pt}%
\definecolor{currentstroke}{rgb}{0.000000,0.000000,0.000000}%
\pgfsetstrokecolor{currentstroke}%
\pgfsetstrokeopacity{0.000000}%
\pgfsetdash{}{0pt}%
\pgfpathmoveto{\pgfqpoint{1.405583in}{1.169197in}}%
\pgfpathlineto{\pgfqpoint{1.864744in}{1.169197in}}%
\pgfpathlineto{\pgfqpoint{1.864744in}{4.022010in}}%
\pgfpathlineto{\pgfqpoint{1.405583in}{4.022010in}}%
\pgfpathclose%
\pgfusepath{fill}%
\end{pgfscope}%
\begin{pgfscope}%
\pgfpathrectangle{\pgfqpoint{0.696435in}{1.169197in}}{\pgfqpoint{5.499722in}{3.204521in}}%
\pgfusepath{clip}%
\pgfsetbuttcap%
\pgfsetmiterjoin%
\definecolor{currentfill}{rgb}{1.000000,0.498039,0.054902}%
\pgfsetfillcolor{currentfill}%
\pgfsetlinewidth{0.000000pt}%
\definecolor{currentstroke}{rgb}{0.000000,0.000000,0.000000}%
\pgfsetstrokecolor{currentstroke}%
\pgfsetstrokeopacity{0.000000}%
\pgfsetdash{}{0pt}%
\pgfpathmoveto{\pgfqpoint{2.425940in}{1.169197in}}%
\pgfpathlineto{\pgfqpoint{2.885100in}{1.169197in}}%
\pgfpathlineto{\pgfqpoint{2.885100in}{4.005278in}}%
\pgfpathlineto{\pgfqpoint{2.425940in}{4.005278in}}%
\pgfpathclose%
\pgfusepath{fill}%
\end{pgfscope}%
\begin{pgfscope}%
\pgfpathrectangle{\pgfqpoint{0.696435in}{1.169197in}}{\pgfqpoint{5.499722in}{3.204521in}}%
\pgfusepath{clip}%
\pgfsetbuttcap%
\pgfsetmiterjoin%
\definecolor{currentfill}{rgb}{1.000000,0.498039,0.054902}%
\pgfsetfillcolor{currentfill}%
\pgfsetlinewidth{0.000000pt}%
\definecolor{currentstroke}{rgb}{0.000000,0.000000,0.000000}%
\pgfsetstrokecolor{currentstroke}%
\pgfsetstrokeopacity{0.000000}%
\pgfsetdash{}{0pt}%
\pgfpathmoveto{\pgfqpoint{3.446297in}{1.169197in}}%
\pgfpathlineto{\pgfqpoint{3.905457in}{1.169197in}}%
\pgfpathlineto{\pgfqpoint{3.905457in}{4.016990in}}%
\pgfpathlineto{\pgfqpoint{3.446297in}{4.016990in}}%
\pgfpathclose%
\pgfusepath{fill}%
\end{pgfscope}%
\begin{pgfscope}%
\pgfpathrectangle{\pgfqpoint{0.696435in}{1.169197in}}{\pgfqpoint{5.499722in}{3.204521in}}%
\pgfusepath{clip}%
\pgfsetbuttcap%
\pgfsetmiterjoin%
\definecolor{currentfill}{rgb}{1.000000,0.498039,0.054902}%
\pgfsetfillcolor{currentfill}%
\pgfsetlinewidth{0.000000pt}%
\definecolor{currentstroke}{rgb}{0.000000,0.000000,0.000000}%
\pgfsetstrokecolor{currentstroke}%
\pgfsetstrokeopacity{0.000000}%
\pgfsetdash{}{0pt}%
\pgfpathmoveto{\pgfqpoint{4.466653in}{1.169197in}}%
\pgfpathlineto{\pgfqpoint{4.925814in}{1.169197in}}%
\pgfpathlineto{\pgfqpoint{4.925814in}{3.930820in}}%
\pgfpathlineto{\pgfqpoint{4.466653in}{3.930820in}}%
\pgfpathclose%
\pgfusepath{fill}%
\end{pgfscope}%
\begin{pgfscope}%
\pgfpathrectangle{\pgfqpoint{0.696435in}{1.169197in}}{\pgfqpoint{5.499722in}{3.204521in}}%
\pgfusepath{clip}%
\pgfsetbuttcap%
\pgfsetmiterjoin%
\definecolor{currentfill}{rgb}{1.000000,0.498039,0.054902}%
\pgfsetfillcolor{currentfill}%
\pgfsetlinewidth{0.000000pt}%
\definecolor{currentstroke}{rgb}{0.000000,0.000000,0.000000}%
\pgfsetstrokecolor{currentstroke}%
\pgfsetstrokeopacity{0.000000}%
\pgfsetdash{}{0pt}%
\pgfpathmoveto{\pgfqpoint{5.487010in}{1.169197in}}%
\pgfpathlineto{\pgfqpoint{5.946170in}{1.169197in}}%
\pgfpathlineto{\pgfqpoint{5.946170in}{4.039579in}}%
\pgfpathlineto{\pgfqpoint{5.487010in}{4.039579in}}%
\pgfpathclose%
\pgfusepath{fill}%
\end{pgfscope}%
\begin{pgfscope}%
\pgfsetbuttcap%
\pgfsetroundjoin%
\definecolor{currentfill}{rgb}{0.000000,0.000000,0.000000}%
\pgfsetfillcolor{currentfill}%
\pgfsetlinewidth{0.803000pt}%
\definecolor{currentstroke}{rgb}{0.000000,0.000000,0.000000}%
\pgfsetstrokecolor{currentstroke}%
\pgfsetdash{}{0pt}%
\pgfsys@defobject{currentmarker}{\pgfqpoint{0.000000in}{-0.048611in}}{\pgfqpoint{0.000000in}{0.000000in}}{%
\pgfpathmoveto{\pgfqpoint{0.000000in}{0.000000in}}%
\pgfpathlineto{\pgfqpoint{0.000000in}{-0.048611in}}%
\pgfusepath{stroke,fill}%
}%
\begin{pgfscope}%
\pgfsys@transformshift{1.405583in}{1.169197in}%
\pgfsys@useobject{currentmarker}{}%
\end{pgfscope}%
\end{pgfscope}%
\begin{pgfscope}%
\definecolor{textcolor}{rgb}{0.000000,0.000000,0.000000}%
\pgfsetstrokecolor{textcolor}%
\pgfsetfillcolor{textcolor}%
\pgftext[x=1.157902in, y=0.447808in, left, base,rotate=45.000000]{\color{textcolor}\sffamily\fontsize{10.000000}{12.000000}\selectfont Textureless}%
\end{pgfscope}%
\begin{pgfscope}%
\pgfsetbuttcap%
\pgfsetroundjoin%
\definecolor{currentfill}{rgb}{0.000000,0.000000,0.000000}%
\pgfsetfillcolor{currentfill}%
\pgfsetlinewidth{0.803000pt}%
\definecolor{currentstroke}{rgb}{0.000000,0.000000,0.000000}%
\pgfsetstrokecolor{currentstroke}%
\pgfsetdash{}{0pt}%
\pgfsys@defobject{currentmarker}{\pgfqpoint{0.000000in}{-0.048611in}}{\pgfqpoint{0.000000in}{0.000000in}}{%
\pgfpathmoveto{\pgfqpoint{0.000000in}{0.000000in}}%
\pgfpathlineto{\pgfqpoint{0.000000in}{-0.048611in}}%
\pgfusepath{stroke,fill}%
}%
\begin{pgfscope}%
\pgfsys@transformshift{2.425940in}{1.169197in}%
\pgfsys@useobject{currentmarker}{}%
\end{pgfscope}%
\end{pgfscope}%
\begin{pgfscope}%
\definecolor{textcolor}{rgb}{0.000000,0.000000,0.000000}%
\pgfsetstrokecolor{textcolor}%
\pgfsetfillcolor{textcolor}%
\pgftext[x=2.014569in, y=0.120428in, left, base,rotate=45.000000]{\color{textcolor}\sffamily\fontsize{10.000000}{12.000000}\selectfont Textureless+Light}%
\end{pgfscope}%
\begin{pgfscope}%
\pgfsetbuttcap%
\pgfsetroundjoin%
\definecolor{currentfill}{rgb}{0.000000,0.000000,0.000000}%
\pgfsetfillcolor{currentfill}%
\pgfsetlinewidth{0.803000pt}%
\definecolor{currentstroke}{rgb}{0.000000,0.000000,0.000000}%
\pgfsetstrokecolor{currentstroke}%
\pgfsetdash{}{0pt}%
\pgfsys@defobject{currentmarker}{\pgfqpoint{0.000000in}{-0.048611in}}{\pgfqpoint{0.000000in}{0.000000in}}{%
\pgfpathmoveto{\pgfqpoint{0.000000in}{0.000000in}}%
\pgfpathlineto{\pgfqpoint{0.000000in}{-0.048611in}}%
\pgfusepath{stroke,fill}%
}%
\begin{pgfscope}%
\pgfsys@transformshift{3.446297in}{1.169197in}%
\pgfsys@useobject{currentmarker}{}%
\end{pgfscope}%
\end{pgfscope}%
\begin{pgfscope}%
\definecolor{textcolor}{rgb}{0.000000,0.000000,0.000000}%
\pgfsetstrokecolor{textcolor}%
\pgfsetfillcolor{textcolor}%
\pgftext[x=3.262466in, y=0.575509in, left, base,rotate=45.000000]{\color{textcolor}\sffamily\fontsize{10.000000}{12.000000}\selectfont Textured}%
\end{pgfscope}%
\begin{pgfscope}%
\pgfsetbuttcap%
\pgfsetroundjoin%
\definecolor{currentfill}{rgb}{0.000000,0.000000,0.000000}%
\pgfsetfillcolor{currentfill}%
\pgfsetlinewidth{0.803000pt}%
\definecolor{currentstroke}{rgb}{0.000000,0.000000,0.000000}%
\pgfsetstrokecolor{currentstroke}%
\pgfsetdash{}{0pt}%
\pgfsys@defobject{currentmarker}{\pgfqpoint{0.000000in}{-0.048611in}}{\pgfqpoint{0.000000in}{0.000000in}}{%
\pgfpathmoveto{\pgfqpoint{0.000000in}{0.000000in}}%
\pgfpathlineto{\pgfqpoint{0.000000in}{-0.048611in}}%
\pgfusepath{stroke,fill}%
}%
\begin{pgfscope}%
\pgfsys@transformshift{4.466653in}{1.169197in}%
\pgfsys@useobject{currentmarker}{}%
\end{pgfscope}%
\end{pgfscope}%
\begin{pgfscope}%
\definecolor{textcolor}{rgb}{0.000000,0.000000,0.000000}%
\pgfsetstrokecolor{textcolor}%
\pgfsetfillcolor{textcolor}%
\pgftext[x=4.119132in, y=0.248129in, left, base,rotate=45.000000]{\color{textcolor}\sffamily\fontsize{10.000000}{12.000000}\selectfont Textured+Light}%
\end{pgfscope}%
\begin{pgfscope}%
\pgfsetbuttcap%
\pgfsetroundjoin%
\definecolor{currentfill}{rgb}{0.000000,0.000000,0.000000}%
\pgfsetfillcolor{currentfill}%
\pgfsetlinewidth{0.803000pt}%
\definecolor{currentstroke}{rgb}{0.000000,0.000000,0.000000}%
\pgfsetstrokecolor{currentstroke}%
\pgfsetdash{}{0pt}%
\pgfsys@defobject{currentmarker}{\pgfqpoint{0.000000in}{-0.048611in}}{\pgfqpoint{0.000000in}{0.000000in}}{%
\pgfpathmoveto{\pgfqpoint{0.000000in}{0.000000in}}%
\pgfpathlineto{\pgfqpoint{0.000000in}{-0.048611in}}%
\pgfusepath{stroke,fill}%
}%
\begin{pgfscope}%
\pgfsys@transformshift{5.487010in}{1.169197in}%
\pgfsys@useobject{currentmarker}{}%
\end{pgfscope}%
\end{pgfscope}%
\begin{pgfscope}%
\definecolor{textcolor}{rgb}{0.000000,0.000000,0.000000}%
\pgfsetstrokecolor{textcolor}%
\pgfsetfillcolor{textcolor}%
\pgftext[x=5.215064in, y=0.399279in, left, base,rotate=45.000000]{\color{textcolor}\sffamily\fontsize{10.000000}{12.000000}\selectfont Multi-Object}%
\end{pgfscope}%
\begin{pgfscope}%
\pgfsetbuttcap%
\pgfsetroundjoin%
\definecolor{currentfill}{rgb}{0.000000,0.000000,0.000000}%
\pgfsetfillcolor{currentfill}%
\pgfsetlinewidth{0.803000pt}%
\definecolor{currentstroke}{rgb}{0.000000,0.000000,0.000000}%
\pgfsetstrokecolor{currentstroke}%
\pgfsetdash{}{0pt}%
\pgfsys@defobject{currentmarker}{\pgfqpoint{-0.048611in}{0.000000in}}{\pgfqpoint{-0.000000in}{0.000000in}}{%
\pgfpathmoveto{\pgfqpoint{-0.000000in}{0.000000in}}%
\pgfpathlineto{\pgfqpoint{-0.048611in}{0.000000in}}%
\pgfusepath{stroke,fill}%
}%
\begin{pgfscope}%
\pgfsys@transformshift{0.696435in}{1.169197in}%
\pgfsys@useobject{currentmarker}{}%
\end{pgfscope}%
\end{pgfscope}%
\begin{pgfscope}%
\definecolor{textcolor}{rgb}{0.000000,0.000000,0.000000}%
\pgfsetstrokecolor{textcolor}%
\pgfsetfillcolor{textcolor}%
\pgftext[x=0.289968in, y=1.116435in, left, base]{\color{textcolor}\sffamily\fontsize{10.000000}{12.000000}\selectfont 0.00}%
\end{pgfscope}%
\begin{pgfscope}%
\pgfsetbuttcap%
\pgfsetroundjoin%
\definecolor{currentfill}{rgb}{0.000000,0.000000,0.000000}%
\pgfsetfillcolor{currentfill}%
\pgfsetlinewidth{0.803000pt}%
\definecolor{currentstroke}{rgb}{0.000000,0.000000,0.000000}%
\pgfsetstrokecolor{currentstroke}%
\pgfsetdash{}{0pt}%
\pgfsys@defobject{currentmarker}{\pgfqpoint{-0.048611in}{0.000000in}}{\pgfqpoint{-0.000000in}{0.000000in}}{%
\pgfpathmoveto{\pgfqpoint{-0.000000in}{0.000000in}}%
\pgfpathlineto{\pgfqpoint{-0.048611in}{0.000000in}}%
\pgfusepath{stroke,fill}%
}%
\begin{pgfscope}%
\pgfsys@transformshift{0.696435in}{1.587498in}%
\pgfsys@useobject{currentmarker}{}%
\end{pgfscope}%
\end{pgfscope}%
\begin{pgfscope}%
\definecolor{textcolor}{rgb}{0.000000,0.000000,0.000000}%
\pgfsetstrokecolor{textcolor}%
\pgfsetfillcolor{textcolor}%
\pgftext[x=0.289968in, y=1.534736in, left, base]{\color{textcolor}\sffamily\fontsize{10.000000}{12.000000}\selectfont 0.05}%
\end{pgfscope}%
\begin{pgfscope}%
\pgfsetbuttcap%
\pgfsetroundjoin%
\definecolor{currentfill}{rgb}{0.000000,0.000000,0.000000}%
\pgfsetfillcolor{currentfill}%
\pgfsetlinewidth{0.803000pt}%
\definecolor{currentstroke}{rgb}{0.000000,0.000000,0.000000}%
\pgfsetstrokecolor{currentstroke}%
\pgfsetdash{}{0pt}%
\pgfsys@defobject{currentmarker}{\pgfqpoint{-0.048611in}{0.000000in}}{\pgfqpoint{-0.000000in}{0.000000in}}{%
\pgfpathmoveto{\pgfqpoint{-0.000000in}{0.000000in}}%
\pgfpathlineto{\pgfqpoint{-0.048611in}{0.000000in}}%
\pgfusepath{stroke,fill}%
}%
\begin{pgfscope}%
\pgfsys@transformshift{0.696435in}{2.005799in}%
\pgfsys@useobject{currentmarker}{}%
\end{pgfscope}%
\end{pgfscope}%
\begin{pgfscope}%
\definecolor{textcolor}{rgb}{0.000000,0.000000,0.000000}%
\pgfsetstrokecolor{textcolor}%
\pgfsetfillcolor{textcolor}%
\pgftext[x=0.289968in, y=1.953037in, left, base]{\color{textcolor}\sffamily\fontsize{10.000000}{12.000000}\selectfont 0.10}%
\end{pgfscope}%
\begin{pgfscope}%
\pgfsetbuttcap%
\pgfsetroundjoin%
\definecolor{currentfill}{rgb}{0.000000,0.000000,0.000000}%
\pgfsetfillcolor{currentfill}%
\pgfsetlinewidth{0.803000pt}%
\definecolor{currentstroke}{rgb}{0.000000,0.000000,0.000000}%
\pgfsetstrokecolor{currentstroke}%
\pgfsetdash{}{0pt}%
\pgfsys@defobject{currentmarker}{\pgfqpoint{-0.048611in}{0.000000in}}{\pgfqpoint{-0.000000in}{0.000000in}}{%
\pgfpathmoveto{\pgfqpoint{-0.000000in}{0.000000in}}%
\pgfpathlineto{\pgfqpoint{-0.048611in}{0.000000in}}%
\pgfusepath{stroke,fill}%
}%
\begin{pgfscope}%
\pgfsys@transformshift{0.696435in}{2.424100in}%
\pgfsys@useobject{currentmarker}{}%
\end{pgfscope}%
\end{pgfscope}%
\begin{pgfscope}%
\definecolor{textcolor}{rgb}{0.000000,0.000000,0.000000}%
\pgfsetstrokecolor{textcolor}%
\pgfsetfillcolor{textcolor}%
\pgftext[x=0.289968in, y=2.371338in, left, base]{\color{textcolor}\sffamily\fontsize{10.000000}{12.000000}\selectfont 0.15}%
\end{pgfscope}%
\begin{pgfscope}%
\pgfsetbuttcap%
\pgfsetroundjoin%
\definecolor{currentfill}{rgb}{0.000000,0.000000,0.000000}%
\pgfsetfillcolor{currentfill}%
\pgfsetlinewidth{0.803000pt}%
\definecolor{currentstroke}{rgb}{0.000000,0.000000,0.000000}%
\pgfsetstrokecolor{currentstroke}%
\pgfsetdash{}{0pt}%
\pgfsys@defobject{currentmarker}{\pgfqpoint{-0.048611in}{0.000000in}}{\pgfqpoint{-0.000000in}{0.000000in}}{%
\pgfpathmoveto{\pgfqpoint{-0.000000in}{0.000000in}}%
\pgfpathlineto{\pgfqpoint{-0.048611in}{0.000000in}}%
\pgfusepath{stroke,fill}%
}%
\begin{pgfscope}%
\pgfsys@transformshift{0.696435in}{2.842401in}%
\pgfsys@useobject{currentmarker}{}%
\end{pgfscope}%
\end{pgfscope}%
\begin{pgfscope}%
\definecolor{textcolor}{rgb}{0.000000,0.000000,0.000000}%
\pgfsetstrokecolor{textcolor}%
\pgfsetfillcolor{textcolor}%
\pgftext[x=0.289968in, y=2.789639in, left, base]{\color{textcolor}\sffamily\fontsize{10.000000}{12.000000}\selectfont 0.20}%
\end{pgfscope}%
\begin{pgfscope}%
\pgfsetbuttcap%
\pgfsetroundjoin%
\definecolor{currentfill}{rgb}{0.000000,0.000000,0.000000}%
\pgfsetfillcolor{currentfill}%
\pgfsetlinewidth{0.803000pt}%
\definecolor{currentstroke}{rgb}{0.000000,0.000000,0.000000}%
\pgfsetstrokecolor{currentstroke}%
\pgfsetdash{}{0pt}%
\pgfsys@defobject{currentmarker}{\pgfqpoint{-0.048611in}{0.000000in}}{\pgfqpoint{-0.000000in}{0.000000in}}{%
\pgfpathmoveto{\pgfqpoint{-0.000000in}{0.000000in}}%
\pgfpathlineto{\pgfqpoint{-0.048611in}{0.000000in}}%
\pgfusepath{stroke,fill}%
}%
\begin{pgfscope}%
\pgfsys@transformshift{0.696435in}{3.260702in}%
\pgfsys@useobject{currentmarker}{}%
\end{pgfscope}%
\end{pgfscope}%
\begin{pgfscope}%
\definecolor{textcolor}{rgb}{0.000000,0.000000,0.000000}%
\pgfsetstrokecolor{textcolor}%
\pgfsetfillcolor{textcolor}%
\pgftext[x=0.289968in, y=3.207940in, left, base]{\color{textcolor}\sffamily\fontsize{10.000000}{12.000000}\selectfont 0.25}%
\end{pgfscope}%
\begin{pgfscope}%
\pgfsetbuttcap%
\pgfsetroundjoin%
\definecolor{currentfill}{rgb}{0.000000,0.000000,0.000000}%
\pgfsetfillcolor{currentfill}%
\pgfsetlinewidth{0.803000pt}%
\definecolor{currentstroke}{rgb}{0.000000,0.000000,0.000000}%
\pgfsetstrokecolor{currentstroke}%
\pgfsetdash{}{0pt}%
\pgfsys@defobject{currentmarker}{\pgfqpoint{-0.048611in}{0.000000in}}{\pgfqpoint{-0.000000in}{0.000000in}}{%
\pgfpathmoveto{\pgfqpoint{-0.000000in}{0.000000in}}%
\pgfpathlineto{\pgfqpoint{-0.048611in}{0.000000in}}%
\pgfusepath{stroke,fill}%
}%
\begin{pgfscope}%
\pgfsys@transformshift{0.696435in}{3.679003in}%
\pgfsys@useobject{currentmarker}{}%
\end{pgfscope}%
\end{pgfscope}%
\begin{pgfscope}%
\definecolor{textcolor}{rgb}{0.000000,0.000000,0.000000}%
\pgfsetstrokecolor{textcolor}%
\pgfsetfillcolor{textcolor}%
\pgftext[x=0.289968in, y=3.626242in, left, base]{\color{textcolor}\sffamily\fontsize{10.000000}{12.000000}\selectfont 0.30}%
\end{pgfscope}%
\begin{pgfscope}%
\pgfsetbuttcap%
\pgfsetroundjoin%
\definecolor{currentfill}{rgb}{0.000000,0.000000,0.000000}%
\pgfsetfillcolor{currentfill}%
\pgfsetlinewidth{0.803000pt}%
\definecolor{currentstroke}{rgb}{0.000000,0.000000,0.000000}%
\pgfsetstrokecolor{currentstroke}%
\pgfsetdash{}{0pt}%
\pgfsys@defobject{currentmarker}{\pgfqpoint{-0.048611in}{0.000000in}}{\pgfqpoint{-0.000000in}{0.000000in}}{%
\pgfpathmoveto{\pgfqpoint{-0.000000in}{0.000000in}}%
\pgfpathlineto{\pgfqpoint{-0.048611in}{0.000000in}}%
\pgfusepath{stroke,fill}%
}%
\begin{pgfscope}%
\pgfsys@transformshift{0.696435in}{4.097304in}%
\pgfsys@useobject{currentmarker}{}%
\end{pgfscope}%
\end{pgfscope}%
\begin{pgfscope}%
\definecolor{textcolor}{rgb}{0.000000,0.000000,0.000000}%
\pgfsetstrokecolor{textcolor}%
\pgfsetfillcolor{textcolor}%
\pgftext[x=0.289968in, y=4.044543in, left, base]{\color{textcolor}\sffamily\fontsize{10.000000}{12.000000}\selectfont 0.35}%
\end{pgfscope}%
\begin{pgfscope}%
\definecolor{textcolor}{rgb}{0.000000,0.000000,0.000000}%
\pgfsetstrokecolor{textcolor}%
\pgfsetfillcolor{textcolor}%
\pgftext[x=0.234413in,y=2.771457in,,bottom,rotate=90.000000]{\color{textcolor}\sffamily\fontsize{10.000000}{12.000000}\selectfont IoU}%
\end{pgfscope}%
\begin{pgfscope}%
\pgfsetrectcap%
\pgfsetmiterjoin%
\pgfsetlinewidth{0.803000pt}%
\definecolor{currentstroke}{rgb}{0.000000,0.000000,0.000000}%
\pgfsetstrokecolor{currentstroke}%
\pgfsetdash{}{0pt}%
\pgfpathmoveto{\pgfqpoint{0.696435in}{1.169197in}}%
\pgfpathlineto{\pgfqpoint{0.696435in}{4.373718in}}%
\pgfusepath{stroke}%
\end{pgfscope}%
\begin{pgfscope}%
\pgfsetrectcap%
\pgfsetmiterjoin%
\pgfsetlinewidth{0.803000pt}%
\definecolor{currentstroke}{rgb}{0.000000,0.000000,0.000000}%
\pgfsetstrokecolor{currentstroke}%
\pgfsetdash{}{0pt}%
\pgfpathmoveto{\pgfqpoint{6.196158in}{1.169197in}}%
\pgfpathlineto{\pgfqpoint{6.196158in}{4.373718in}}%
\pgfusepath{stroke}%
\end{pgfscope}%
\begin{pgfscope}%
\pgfsetrectcap%
\pgfsetmiterjoin%
\pgfsetlinewidth{0.803000pt}%
\definecolor{currentstroke}{rgb}{0.000000,0.000000,0.000000}%
\pgfsetstrokecolor{currentstroke}%
\pgfsetdash{}{0pt}%
\pgfpathmoveto{\pgfqpoint{0.696435in}{1.169197in}}%
\pgfpathlineto{\pgfqpoint{6.196158in}{1.169197in}}%
\pgfusepath{stroke}%
\end{pgfscope}%
\begin{pgfscope}%
\pgfsetrectcap%
\pgfsetmiterjoin%
\pgfsetlinewidth{0.803000pt}%
\definecolor{currentstroke}{rgb}{0.000000,0.000000,0.000000}%
\pgfsetstrokecolor{currentstroke}%
\pgfsetdash{}{0pt}%
\pgfpathmoveto{\pgfqpoint{0.696435in}{4.373718in}}%
\pgfpathlineto{\pgfqpoint{6.196158in}{4.373718in}}%
\pgfusepath{stroke}%
\end{pgfscope}%
\begin{pgfscope}%
\definecolor{textcolor}{rgb}{0.000000,0.000000,0.000000}%
\pgfsetstrokecolor{textcolor}%
\pgfsetfillcolor{textcolor}%
\pgftext[x=1.176003in,y=4.262788in,,bottom]{\color{textcolor}\sffamily\fontsize{10.000000}{12.000000}\selectfont 0.3648}%
\end{pgfscope}%
\begin{pgfscope}%
\definecolor{textcolor}{rgb}{0.000000,0.000000,0.000000}%
\pgfsetstrokecolor{textcolor}%
\pgfsetfillcolor{textcolor}%
\pgftext[x=2.196360in,y=4.236017in,,bottom]{\color{textcolor}\sffamily\fontsize{10.000000}{12.000000}\selectfont 0.3616}%
\end{pgfscope}%
\begin{pgfscope}%
\definecolor{textcolor}{rgb}{0.000000,0.000000,0.000000}%
\pgfsetstrokecolor{textcolor}%
\pgfsetfillcolor{textcolor}%
\pgftext[x=3.216716in,y=4.122239in,,bottom]{\color{textcolor}\sffamily\fontsize{10.000000}{12.000000}\selectfont 0.348}%
\end{pgfscope}%
\begin{pgfscope}%
\definecolor{textcolor}{rgb}{0.000000,0.000000,0.000000}%
\pgfsetstrokecolor{textcolor}%
\pgfsetfillcolor{textcolor}%
\pgftext[x=4.237073in,y=4.076226in,,bottom]{\color{textcolor}\sffamily\fontsize{10.000000}{12.000000}\selectfont 0.3425}%
\end{pgfscope}%
\begin{pgfscope}%
\definecolor{textcolor}{rgb}{0.000000,0.000000,0.000000}%
\pgfsetstrokecolor{textcolor}%
\pgfsetfillcolor{textcolor}%
\pgftext[x=5.257430in,y=4.141481in,,bottom]{\color{textcolor}\sffamily\fontsize{10.000000}{12.000000}\selectfont 0.3503}%
\end{pgfscope}%
\begin{pgfscope}%
\definecolor{textcolor}{rgb}{0.000000,0.000000,0.000000}%
\pgfsetstrokecolor{textcolor}%
\pgfsetfillcolor{textcolor}%
\pgftext[x=1.635164in,y=4.063677in,,bottom]{\color{textcolor}\sffamily\fontsize{10.000000}{12.000000}\selectfont 0.341}%
\end{pgfscope}%
\begin{pgfscope}%
\definecolor{textcolor}{rgb}{0.000000,0.000000,0.000000}%
\pgfsetstrokecolor{textcolor}%
\pgfsetfillcolor{textcolor}%
\pgftext[x=2.655520in,y=4.046945in,,bottom]{\color{textcolor}\sffamily\fontsize{10.000000}{12.000000}\selectfont 0.339}%
\end{pgfscope}%
\begin{pgfscope}%
\definecolor{textcolor}{rgb}{0.000000,0.000000,0.000000}%
\pgfsetstrokecolor{textcolor}%
\pgfsetfillcolor{textcolor}%
\pgftext[x=3.675877in,y=4.058657in,,bottom]{\color{textcolor}\sffamily\fontsize{10.000000}{12.000000}\selectfont 0.3404}%
\end{pgfscope}%
\begin{pgfscope}%
\definecolor{textcolor}{rgb}{0.000000,0.000000,0.000000}%
\pgfsetstrokecolor{textcolor}%
\pgfsetfillcolor{textcolor}%
\pgftext[x=4.696233in,y=3.972487in,,bottom]{\color{textcolor}\sffamily\fontsize{10.000000}{12.000000}\selectfont 0.3301}%
\end{pgfscope}%
\begin{pgfscope}%
\definecolor{textcolor}{rgb}{0.000000,0.000000,0.000000}%
\pgfsetstrokecolor{textcolor}%
\pgfsetfillcolor{textcolor}%
\pgftext[x=5.716590in,y=4.081245in,,bottom]{\color{textcolor}\sffamily\fontsize{10.000000}{12.000000}\selectfont 0.3431}%
\end{pgfscope}%
\begin{pgfscope}%
\definecolor{textcolor}{rgb}{0.000000,0.000000,0.000000}%
\pgfsetstrokecolor{textcolor}%
\pgfsetfillcolor{textcolor}%
\pgftext[x=3.446297in,y=4.457051in,,base]{\color{textcolor}\sffamily\fontsize{12.000000}{14.400000}\selectfont Abalation study on chairs with mixed training}%
\end{pgfscope}%
\begin{pgfscope}%
\pgfsetbuttcap%
\pgfsetmiterjoin%
\definecolor{currentfill}{rgb}{1.000000,1.000000,1.000000}%
\pgfsetfillcolor{currentfill}%
\pgfsetfillopacity{0.800000}%
\pgfsetlinewidth{1.003750pt}%
\definecolor{currentstroke}{rgb}{0.800000,0.800000,0.800000}%
\pgfsetstrokecolor{currentstroke}%
\pgfsetstrokeopacity{0.800000}%
\pgfsetdash{}{0pt}%
\pgfpathmoveto{\pgfqpoint{6.293380in}{2.546766in}}%
\pgfpathlineto{\pgfqpoint{7.507545in}{2.546766in}}%
\pgfpathquadraticcurveto{\pgfqpoint{7.535323in}{2.546766in}}{\pgfqpoint{7.535323in}{2.574544in}}%
\pgfpathlineto{\pgfqpoint{7.535323in}{2.968370in}}%
\pgfpathquadraticcurveto{\pgfqpoint{7.535323in}{2.996148in}}{\pgfqpoint{7.507545in}{2.996148in}}%
\pgfpathlineto{\pgfqpoint{6.293380in}{2.996148in}}%
\pgfpathquadraticcurveto{\pgfqpoint{6.265602in}{2.996148in}}{\pgfqpoint{6.265602in}{2.968370in}}%
\pgfpathlineto{\pgfqpoint{6.265602in}{2.574544in}}%
\pgfpathquadraticcurveto{\pgfqpoint{6.265602in}{2.546766in}}{\pgfqpoint{6.293380in}{2.546766in}}%
\pgfpathclose%
\pgfusepath{stroke,fill}%
\end{pgfscope}%
\begin{pgfscope}%
\pgfsetbuttcap%
\pgfsetmiterjoin%
\definecolor{currentfill}{rgb}{0.121569,0.466667,0.705882}%
\pgfsetfillcolor{currentfill}%
\pgfsetlinewidth{0.000000pt}%
\definecolor{currentstroke}{rgb}{0.000000,0.000000,0.000000}%
\pgfsetstrokecolor{currentstroke}%
\pgfsetstrokeopacity{0.000000}%
\pgfsetdash{}{0pt}%
\pgfpathmoveto{\pgfqpoint{6.321158in}{2.835069in}}%
\pgfpathlineto{\pgfqpoint{6.598935in}{2.835069in}}%
\pgfpathlineto{\pgfqpoint{6.598935in}{2.932291in}}%
\pgfpathlineto{\pgfqpoint{6.321158in}{2.932291in}}%
\pgfpathclose%
\pgfusepath{fill}%
\end{pgfscope}%
\begin{pgfscope}%
\definecolor{textcolor}{rgb}{0.000000,0.000000,0.000000}%
\pgfsetstrokecolor{textcolor}%
\pgfsetfillcolor{textcolor}%
\pgftext[x=6.710047in,y=2.835069in,left,base]{\color{textcolor}\sffamily\fontsize{10.000000}{12.000000}\selectfont Pix2Vox++}%
\end{pgfscope}%
\begin{pgfscope}%
\pgfsetbuttcap%
\pgfsetmiterjoin%
\definecolor{currentfill}{rgb}{1.000000,0.498039,0.054902}%
\pgfsetfillcolor{currentfill}%
\pgfsetlinewidth{0.000000pt}%
\definecolor{currentstroke}{rgb}{0.000000,0.000000,0.000000}%
\pgfsetstrokecolor{currentstroke}%
\pgfsetstrokeopacity{0.000000}%
\pgfsetdash{}{0pt}%
\pgfpathmoveto{\pgfqpoint{6.321158in}{2.631212in}}%
\pgfpathlineto{\pgfqpoint{6.598935in}{2.631212in}}%
\pgfpathlineto{\pgfqpoint{6.598935in}{2.728434in}}%
\pgfpathlineto{\pgfqpoint{6.321158in}{2.728434in}}%
\pgfpathclose%
\pgfusepath{fill}%
\end{pgfscope}%
\begin{pgfscope}%
\definecolor{textcolor}{rgb}{0.000000,0.000000,0.000000}%
\pgfsetstrokecolor{textcolor}%
\pgfsetfillcolor{textcolor}%
\pgftext[x=6.710047in,y=2.631212in,left,base]{\color{textcolor}\sffamily\fontsize{10.000000}{12.000000}\selectfont Pix2Vox}%
\end{pgfscope}%
\end{pgfpicture}%
\makeatother%
\endgroup%
}
    \caption{Bar plot for the \gls{iou}  for baseline trained by mixing chair dataset from real and synthetic dataset with ratio of 50\%}
    \label{fig:ablation2}
\end{figure}

\chapter{\iftoggle{german}{Fazit}{Conclusion}}\label{ch:conclusion}

In this chapter we summarize~\ref{sec:Summary} the entire thesis and answer the research questions discussed in~\ref{sec:goal}.
We will further discuss the limitation of the work indulged in this thesis in ~\ref{sec:limitations-of-game-engines2}.
We will also mention possible future scopes and ideas which we came across while conducting this thesis.


\section{\iftoggle{german}{Zukünftige Arbeiten}{Summary}}\label{sec:Summary}

In chapter~\ref{ch:introduction} we describe the main aim of the thesis to create a synthetic dataset that can be used for 3D reconstruction tasks
for furniture as there are very limited number of dataset open to researchers.
We also explained why synthetic datasets are important and how it will be handy for future tasks.
We also introduced game engines and how we intend to use the Unity framework in our pipeline.

In chapter~\ref{ch:related_work}, we provided an overview of existing proclaimed photorealistic datasets along their use cases.
These datasets were further used in survey to verify the photorealism from human perspective the images for which can be seen in ~\ref{fig:photorealistic images comparison}.
The tools to create synthetic datasets was discussed in ~\ref{subsec:tools-to-create-synthetic}, where in we discussed the underlying framework used build the tool.
Then we came across deep learning techniques and models used for 3D reconstruction with different representations.
In the same section we discussed some similar works where in synthetic datasets were used to enhance the performance of real dataset by techniques like fine-tuning and mixed training.

In chapter~\ref{ch:concept}, we discussed about the benchmark datasets Pix3D~\cite{pix3d} and SceneNet~\cite{McCormac:etal:ICCV2017}, and how we use them in our synthetic dataset generation pipeline.
We then discussed hot unity game engine be used as a framework to domain randomisation and what types of randomisation can be used to generate synthetic dataset.
Then came the introduction of S2R:3DFREE dataset and its distribution.
We then moved to deep learning aspect of the thesis and discussed baseline models Pix2Vox and Pix2Vox++ used for 3D reconstruction task.

For implementation, we used Unity game engine, and the application details were provided in chapter~\ref{ch:implementation}.
Using different modes of operation and importing models from Pix3D and SceneNet, the pipeline can create an ersatz environment for our dataset.
We then discussed domain randomisation achieved from camera viewpoints, textures , lighting and shadows and replacing category objects.
The snapshots were taken by using Unity technologies' library called ML-ImageSynthesis which provides option to generate 2.5D images, i.e. Depth, normals, flows, segmentation.
Then we see the Deep Learning framework and its configurations.

We dedicate chapter~\ref{ch:evaluation} for experiments and evaluation.
We evaluate the photorealism of datasets in comparison with real dataset(Pix3D) from a survey conducted, which answers our first research question:
\textbf{\emph{Are game engines a good medium to create photorealistic synthetic dataset?}} From the results of survey as discussed in~\ref{subsec:survey-results},
we came to a conclusion that the proposed dataset using unity game engine, although comparable with other automated dataset collection,
still lags the photorealism from real dataset or even the manually created datasets by architects.
The S2R:3DFREE outperformed only some synthetic dataset when it came to individual rating, but was in top 5 out of 9 datasets while having a comparative study.

For our second research question: \textbf{\emph{Can Ersatz environment from a game engine like Unity replace real data for training in 3d reconstruction task?}}
We trained baseline models for 3D reconstruction task, with real dataset and also with synthetic dataset and tested them on real dataset.
In section~\ref{sec:baseline}, we see that there is a drop of around 14\% in performance when we use only sythetic dataset.

To investigate for our third research question: \textbf{\emph{To what extent can the performance of model Pre-trained with real dataset improved with Synthetic dataset from game engine with size of synthetic dataset being
10\todo{(depends on how much data is generated)} times that of real dataset?}}, we tried domain adaptation techniques.
When we fine-tuned a model pretrained on synthetic dataset using real dataset, we see an increase of 1.36\% to 2.41\%.
We further investigate the domain adaptation, by mix training the models with both synthetic and real dataset with different ratios and see a further 3\% improvement in performance.
To conclude, synthetic data does improve the performance on top of real data for 3D reconstruction task.

As stated in~\ref{sec:contributions}, we have conducted a comprehensive study on mixed training with different ratios and comparing it with fine-tuning or transfer learning approach.
We saw that mixed training performed better than fine-tuning of the model.
A study was also conducted to see the domain randomisation by using only the chair dataset with different parameters of randomisation.
Contrary to our initial hypothesis, adding more randomisation decreased the performance of the models as it gave the maximum performance with the dataset was textureless and with constant light as seen in ~\ref{sec:ablation-study-on-chairs}.
We now hypothesis this result is because of limited dataset created, and the limited real data present to evaluation, meaning the variance of real dataset might be limited and additional randomisation is not helping the cause.

\section{\iftoggle{german}{Zukünftige Arbeiten}{Limitations}}\label{sec:limitations-of-game-engines2}

In this section we discuss the limitations observed while conducting experiments for the thesis.

\subsection{Unity Game Engine}\label{subsec:unity-game-engine}
Unity game engine does not give control to the user for GPU usage.
It is entirely under the control of framework and the user can just hope for faster performance.
We observed that it took around 8hours to create a 3000 images using Ml-Image Synthesis library on a CPU.
Comparing this to Blenderproc, which takes 20 minutes for a single image on CPU, but proclaimed to create 3000 images in 1 hour with GPU usage.
Blenderproc has python as its underlying framework and hence can access GPUs easily.

\subsection{Configuring synthetic dataset generation pipeline}\label{subsec:configuring-synthetic-dataset-generation-pipeline}
We had to try different configuration to try to get optimum solution manually.
And since the shader layers used by unity are not differentiable, we can not train a model to optimise the input parameters like
light intensity, light color, darkness of shadow, texture colors,texture brightness, etc.
These parameters are randomised and not fine-tuned to increase the performance.
Though blenderproc does not do this, since it uses python, users may try to optimise these parameters.

\subsection{High Definition Render Pipeline}\label{subsec:high-definition-render-pipeline}
Unity has introduced 'High Definition Render Pipeline'(HDRP) which provides more realistic lighting and natural looking scenes.
But this pipeline is not cross-compatible and the models/assets which are not generated for HDRP do not get rendered, as shown in figure~\ref{fig:hdrp}.
The shaders are not made public either, and hence can not be assigned to models at runtime.
To use these pipeline we will have to manually modify the models and make them compatible.

\section{\iftoggle{german}{Zukünftige Arbeiten}{Future Scope}}\label{sec:Future_scope}

In this section we discuss future prospects of the generated dataset, and the ideas which were not implemented.

\subsection{Domain adaptation using Generative models}
A study can be conducted to see the affects of domain adaptation or style transfer using Generative models like CyceleGAN~\cite{CycleGAN2017} to see if the performance increases after conversion.
An attempt was made by creating a pipeline for CycelGAN, but we did not get favourable style translation.
Unity provides a library to inference Deep Neural Networks, which can be integrated with the camera for style transfer.
In this case, the user need not fine-tune the configuration for generating images and only needs to provide few reference indoor room reference images for style optimisation.

\subsection{A study on Style transfer with G-buffers}
The proposed S2R:3DFREE dataset provides with various G-buffers like depth, normal, semantic and instance segmentations and optical flows.
These buffer images can be used to enhance the photorealism of images as proved in ~\cite{Richter_2021}.
In the cited publication, the authors use G-buffer from GTA-V game and enhance the photorealism which can be used for training autonomous vehicles.
The dataset generated in this thesis can also be used in similar way to train autonomous bots for indoor environment.

\subsection{Integrating the framework with Unity Cloud}
Unity has provided the functionality of integrating simulations on Amazon Web Service, and hence can be run on cloud.
By achieving this, the user will not have to worry about running the data generator for days or weeks on local machine.
Though, running models on AWS will come at a cost.

\subsection{Other forms of representations}
In this thesis, we focused to work on Voxel representation as indicated in~\ref{sec:Volumetric representation}.
It would be interesting to check if the performance increases significantly if the model representation is changed to mesh or pointclouds.

\subsection{Increasing the synthetic dataset}
Though we have increased the dataset size by creating synthetically generated images.
The quantity is still limited.
For domain randomisation to actually increase the performance we might need a very large dataset.
Another dataset can be created using the same framework to check if larger number of images are needed for domain randomisation to show its impact.



\appendix
\chapter*{\iftoggle{german}{Anhang}{Appendix}}
\markboth{}{}
\setcounter{chapter}{1}
\addcontentsline{toc}{chapter}{\iftoggle{german}{Anhang}{Appendix}}

\section{Study with F-Score}\label{sec:study-with-f-score}

In this section, we conduct a contemporary study on the performance of Deep Learning models using the \gls{f1} as an evaluation measure instead of \gls{iou}.
\gls{iou} tends to penalize a single wrong classification more than Dice score, and hence we can say that the \gls{f1} gives a measure that is closer to the average,
whereas \gls{iou} gives a score for the worst possible scenario.
Even though both \gls{iou} and \gls{f1} are positively correlated, a study with \gls{f1} might support the dataset with better average.

\subsection{Baseline}\label{subsec:baseline_dice}
The setup for this experiment is explained in \autoref{sec:baseline}, where the evaluation was done using \gls{iou}.
In \autoref{fig:baseline_dice1}, we see the comparison of models trained on a real and a synthetic dataset.
The performance is similar to what we see in \autoref{fig:baseline1}.

\begin{figure}[ht]
    \centering
    \resizebox{0.75\linewidth}{!}{%% Creator: Matplotlib, PGF backend
%%
%% To include the figure in your LaTeX document, write
%%   \input{<filename>.pgf}
%%
%% Make sure the required packages are loaded in your preamble
%%   \usepackage{pgf}
%%
%% Figures using additional raster images can only be included by \input if
%% they are in the same directory as the main LaTeX file. For loading figures
%% from other directories you can use the `import` package
%%   \usepackage{import}
%%
%% and then include the figures with
%%   \import{<path to file>}{<filename>.pgf}
%%
%% Matplotlib used the following preamble
%%   \usepackage{fontspec}
%%   \setmainfont{DejaVuSerif.ttf}[Path=\detokenize{/Users/apple/opt/anaconda3/envs/kaolin/lib/python3.7/site-packages/matplotlib/mpl-data/fonts/ttf/}]
%%   \setsansfont{DejaVuSans.ttf}[Path=\detokenize{/Users/apple/opt/anaconda3/envs/kaolin/lib/python3.7/site-packages/matplotlib/mpl-data/fonts/ttf/}]
%%   \setmonofont{DejaVuSansMono.ttf}[Path=\detokenize{/Users/apple/opt/anaconda3/envs/kaolin/lib/python3.7/site-packages/matplotlib/mpl-data/fonts/ttf/}]
%%
\begingroup%
\makeatletter%
\begin{pgfpicture}%
\pgfpathrectangle{\pgfpointorigin}{\pgfqpoint{6.294042in}{4.697602in}}%
\pgfusepath{use as bounding box, clip}%
\begin{pgfscope}%
\pgfsetbuttcap%
\pgfsetmiterjoin%
\definecolor{currentfill}{rgb}{1.000000,1.000000,1.000000}%
\pgfsetfillcolor{currentfill}%
\pgfsetlinewidth{0.000000pt}%
\definecolor{currentstroke}{rgb}{1.000000,1.000000,1.000000}%
\pgfsetstrokecolor{currentstroke}%
\pgfsetdash{}{0pt}%
\pgfpathmoveto{\pgfqpoint{0.000000in}{0.000000in}}%
\pgfpathlineto{\pgfqpoint{6.294042in}{0.000000in}}%
\pgfpathlineto{\pgfqpoint{6.294042in}{4.697602in}}%
\pgfpathlineto{\pgfqpoint{0.000000in}{4.697602in}}%
\pgfpathclose%
\pgfusepath{fill}%
\end{pgfscope}%
\begin{pgfscope}%
\pgfsetbuttcap%
\pgfsetmiterjoin%
\definecolor{currentfill}{rgb}{1.000000,1.000000,1.000000}%
\pgfsetfillcolor{currentfill}%
\pgfsetlinewidth{0.000000pt}%
\definecolor{currentstroke}{rgb}{0.000000,0.000000,0.000000}%
\pgfsetstrokecolor{currentstroke}%
\pgfsetstrokeopacity{0.000000}%
\pgfsetdash{}{0pt}%
\pgfpathmoveto{\pgfqpoint{0.608070in}{1.223552in}}%
\pgfpathlineto{\pgfqpoint{6.194042in}{1.223552in}}%
\pgfpathlineto{\pgfqpoint{6.194042in}{4.387641in}}%
\pgfpathlineto{\pgfqpoint{0.608070in}{4.387641in}}%
\pgfpathclose%
\pgfusepath{fill}%
\end{pgfscope}%
\begin{pgfscope}%
\pgfpathrectangle{\pgfqpoint{0.608070in}{1.223552in}}{\pgfqpoint{5.585972in}{3.164089in}}%
\pgfusepath{clip}%
\pgfsetbuttcap%
\pgfsetmiterjoin%
\definecolor{currentfill}{rgb}{0.121569,0.466667,0.705882}%
\pgfsetfillcolor{currentfill}%
\pgfsetlinewidth{0.000000pt}%
\definecolor{currentstroke}{rgb}{0.000000,0.000000,0.000000}%
\pgfsetstrokecolor{currentstroke}%
\pgfsetstrokeopacity{0.000000}%
\pgfsetdash{}{0pt}%
\pgfpathmoveto{\pgfqpoint{0.861978in}{1.223552in}}%
\pgfpathlineto{\pgfqpoint{1.328339in}{1.223552in}}%
\pgfpathlineto{\pgfqpoint{1.328339in}{3.616869in}}%
\pgfpathlineto{\pgfqpoint{0.861978in}{3.616869in}}%
\pgfpathclose%
\pgfusepath{fill}%
\end{pgfscope}%
\begin{pgfscope}%
\pgfpathrectangle{\pgfqpoint{0.608070in}{1.223552in}}{\pgfqpoint{5.585972in}{3.164089in}}%
\pgfusepath{clip}%
\pgfsetbuttcap%
\pgfsetmiterjoin%
\definecolor{currentfill}{rgb}{0.121569,0.466667,0.705882}%
\pgfsetfillcolor{currentfill}%
\pgfsetlinewidth{0.000000pt}%
\definecolor{currentstroke}{rgb}{0.000000,0.000000,0.000000}%
\pgfsetstrokecolor{currentstroke}%
\pgfsetstrokeopacity{0.000000}%
\pgfsetdash{}{0pt}%
\pgfpathmoveto{\pgfqpoint{1.898336in}{1.223552in}}%
\pgfpathlineto{\pgfqpoint{2.364698in}{1.223552in}}%
\pgfpathlineto{\pgfqpoint{2.364698in}{4.021239in}}%
\pgfpathlineto{\pgfqpoint{1.898336in}{4.021239in}}%
\pgfpathclose%
\pgfusepath{fill}%
\end{pgfscope}%
\begin{pgfscope}%
\pgfpathrectangle{\pgfqpoint{0.608070in}{1.223552in}}{\pgfqpoint{5.585972in}{3.164089in}}%
\pgfusepath{clip}%
\pgfsetbuttcap%
\pgfsetmiterjoin%
\definecolor{currentfill}{rgb}{0.121569,0.466667,0.705882}%
\pgfsetfillcolor{currentfill}%
\pgfsetlinewidth{0.000000pt}%
\definecolor{currentstroke}{rgb}{0.000000,0.000000,0.000000}%
\pgfsetstrokecolor{currentstroke}%
\pgfsetstrokeopacity{0.000000}%
\pgfsetdash{}{0pt}%
\pgfpathmoveto{\pgfqpoint{2.934695in}{1.223552in}}%
\pgfpathlineto{\pgfqpoint{3.401056in}{1.223552in}}%
\pgfpathlineto{\pgfqpoint{3.401056in}{1.890542in}}%
\pgfpathlineto{\pgfqpoint{2.934695in}{1.890542in}}%
\pgfpathclose%
\pgfusepath{fill}%
\end{pgfscope}%
\begin{pgfscope}%
\pgfpathrectangle{\pgfqpoint{0.608070in}{1.223552in}}{\pgfqpoint{5.585972in}{3.164089in}}%
\pgfusepath{clip}%
\pgfsetbuttcap%
\pgfsetmiterjoin%
\definecolor{currentfill}{rgb}{0.121569,0.466667,0.705882}%
\pgfsetfillcolor{currentfill}%
\pgfsetlinewidth{0.000000pt}%
\definecolor{currentstroke}{rgb}{0.000000,0.000000,0.000000}%
\pgfsetstrokecolor{currentstroke}%
\pgfsetstrokeopacity{0.000000}%
\pgfsetdash{}{0pt}%
\pgfpathmoveto{\pgfqpoint{3.971053in}{1.223552in}}%
\pgfpathlineto{\pgfqpoint{4.437415in}{1.223552in}}%
\pgfpathlineto{\pgfqpoint{4.437415in}{2.158224in}}%
\pgfpathlineto{\pgfqpoint{3.971053in}{2.158224in}}%
\pgfpathclose%
\pgfusepath{fill}%
\end{pgfscope}%
\begin{pgfscope}%
\pgfpathrectangle{\pgfqpoint{0.608070in}{1.223552in}}{\pgfqpoint{5.585972in}{3.164089in}}%
\pgfusepath{clip}%
\pgfsetbuttcap%
\pgfsetmiterjoin%
\definecolor{currentfill}{rgb}{0.121569,0.466667,0.705882}%
\pgfsetfillcolor{currentfill}%
\pgfsetlinewidth{0.000000pt}%
\definecolor{currentstroke}{rgb}{0.000000,0.000000,0.000000}%
\pgfsetstrokecolor{currentstroke}%
\pgfsetstrokeopacity{0.000000}%
\pgfsetdash{}{0pt}%
\pgfpathmoveto{\pgfqpoint{5.007412in}{1.223552in}}%
\pgfpathlineto{\pgfqpoint{5.473773in}{1.223552in}}%
\pgfpathlineto{\pgfqpoint{5.473773in}{2.594235in}}%
\pgfpathlineto{\pgfqpoint{5.007412in}{2.594235in}}%
\pgfpathclose%
\pgfusepath{fill}%
\end{pgfscope}%
\begin{pgfscope}%
\pgfpathrectangle{\pgfqpoint{0.608070in}{1.223552in}}{\pgfqpoint{5.585972in}{3.164089in}}%
\pgfusepath{clip}%
\pgfsetbuttcap%
\pgfsetmiterjoin%
\definecolor{currentfill}{rgb}{1.000000,0.498039,0.054902}%
\pgfsetfillcolor{currentfill}%
\pgfsetlinewidth{0.000000pt}%
\definecolor{currentstroke}{rgb}{0.000000,0.000000,0.000000}%
\pgfsetstrokecolor{currentstroke}%
\pgfsetstrokeopacity{0.000000}%
\pgfsetdash{}{0pt}%
\pgfpathmoveto{\pgfqpoint{1.328339in}{1.223552in}}%
\pgfpathlineto{\pgfqpoint{1.794700in}{1.223552in}}%
\pgfpathlineto{\pgfqpoint{1.794700in}{3.809878in}}%
\pgfpathlineto{\pgfqpoint{1.328339in}{3.809878in}}%
\pgfpathclose%
\pgfusepath{fill}%
\end{pgfscope}%
\begin{pgfscope}%
\pgfpathrectangle{\pgfqpoint{0.608070in}{1.223552in}}{\pgfqpoint{5.585972in}{3.164089in}}%
\pgfusepath{clip}%
\pgfsetbuttcap%
\pgfsetmiterjoin%
\definecolor{currentfill}{rgb}{1.000000,0.498039,0.054902}%
\pgfsetfillcolor{currentfill}%
\pgfsetlinewidth{0.000000pt}%
\definecolor{currentstroke}{rgb}{0.000000,0.000000,0.000000}%
\pgfsetstrokecolor{currentstroke}%
\pgfsetstrokeopacity{0.000000}%
\pgfsetdash{}{0pt}%
\pgfpathmoveto{\pgfqpoint{2.364698in}{1.223552in}}%
\pgfpathlineto{\pgfqpoint{2.831059in}{1.223552in}}%
\pgfpathlineto{\pgfqpoint{2.831059in}{3.932645in}}%
\pgfpathlineto{\pgfqpoint{2.364698in}{3.932645in}}%
\pgfpathclose%
\pgfusepath{fill}%
\end{pgfscope}%
\begin{pgfscope}%
\pgfpathrectangle{\pgfqpoint{0.608070in}{1.223552in}}{\pgfqpoint{5.585972in}{3.164089in}}%
\pgfusepath{clip}%
\pgfsetbuttcap%
\pgfsetmiterjoin%
\definecolor{currentfill}{rgb}{1.000000,0.498039,0.054902}%
\pgfsetfillcolor{currentfill}%
\pgfsetlinewidth{0.000000pt}%
\definecolor{currentstroke}{rgb}{0.000000,0.000000,0.000000}%
\pgfsetstrokecolor{currentstroke}%
\pgfsetstrokeopacity{0.000000}%
\pgfsetdash{}{0pt}%
\pgfpathmoveto{\pgfqpoint{3.401056in}{1.223552in}}%
\pgfpathlineto{\pgfqpoint{3.867417in}{1.223552in}}%
\pgfpathlineto{\pgfqpoint{3.867417in}{1.989894in}}%
\pgfpathlineto{\pgfqpoint{3.401056in}{1.989894in}}%
\pgfpathclose%
\pgfusepath{fill}%
\end{pgfscope}%
\begin{pgfscope}%
\pgfpathrectangle{\pgfqpoint{0.608070in}{1.223552in}}{\pgfqpoint{5.585972in}{3.164089in}}%
\pgfusepath{clip}%
\pgfsetbuttcap%
\pgfsetmiterjoin%
\definecolor{currentfill}{rgb}{1.000000,0.498039,0.054902}%
\pgfsetfillcolor{currentfill}%
\pgfsetlinewidth{0.000000pt}%
\definecolor{currentstroke}{rgb}{0.000000,0.000000,0.000000}%
\pgfsetstrokecolor{currentstroke}%
\pgfsetstrokeopacity{0.000000}%
\pgfsetdash{}{0pt}%
\pgfpathmoveto{\pgfqpoint{4.437415in}{1.223552in}}%
\pgfpathlineto{\pgfqpoint{4.903776in}{1.223552in}}%
\pgfpathlineto{\pgfqpoint{4.903776in}{1.881682in}}%
\pgfpathlineto{\pgfqpoint{4.437415in}{1.881682in}}%
\pgfpathclose%
\pgfusepath{fill}%
\end{pgfscope}%
\begin{pgfscope}%
\pgfpathrectangle{\pgfqpoint{0.608070in}{1.223552in}}{\pgfqpoint{5.585972in}{3.164089in}}%
\pgfusepath{clip}%
\pgfsetbuttcap%
\pgfsetmiterjoin%
\definecolor{currentfill}{rgb}{1.000000,0.498039,0.054902}%
\pgfsetfillcolor{currentfill}%
\pgfsetlinewidth{0.000000pt}%
\definecolor{currentstroke}{rgb}{0.000000,0.000000,0.000000}%
\pgfsetstrokecolor{currentstroke}%
\pgfsetstrokeopacity{0.000000}%
\pgfsetdash{}{0pt}%
\pgfpathmoveto{\pgfqpoint{5.473773in}{1.223552in}}%
\pgfpathlineto{\pgfqpoint{5.940134in}{1.223552in}}%
\pgfpathlineto{\pgfqpoint{5.940134in}{2.913175in}}%
\pgfpathlineto{\pgfqpoint{5.473773in}{2.913175in}}%
\pgfpathclose%
\pgfusepath{fill}%
\end{pgfscope}%
\begin{pgfscope}%
\pgfsetbuttcap%
\pgfsetroundjoin%
\definecolor{currentfill}{rgb}{0.000000,0.000000,0.000000}%
\pgfsetfillcolor{currentfill}%
\pgfsetlinewidth{0.803000pt}%
\definecolor{currentstroke}{rgb}{0.000000,0.000000,0.000000}%
\pgfsetstrokecolor{currentstroke}%
\pgfsetdash{}{0pt}%
\pgfsys@defobject{currentmarker}{\pgfqpoint{0.000000in}{-0.048611in}}{\pgfqpoint{0.000000in}{0.000000in}}{%
\pgfpathmoveto{\pgfqpoint{0.000000in}{0.000000in}}%
\pgfpathlineto{\pgfqpoint{0.000000in}{-0.048611in}}%
\pgfusepath{stroke,fill}%
}%
\begin{pgfscope}%
\pgfsys@transformshift{1.328339in}{1.223552in}%
\pgfsys@useobject{currentmarker}{}%
\end{pgfscope}%
\end{pgfscope}%
\begin{pgfscope}%
\definecolor{textcolor}{rgb}{0.000000,0.000000,0.000000}%
\pgfsetstrokecolor{textcolor}%
\pgfsetfillcolor{textcolor}%
\pgftext[x=1.014314in, y=0.369475in, left, base,rotate=45.000000]{\color{textcolor}\sffamily\fontsize{10.000000}{12.000000}\selectfont Pix3d(no aug)}%
\end{pgfscope}%
\begin{pgfscope}%
\pgfsetbuttcap%
\pgfsetroundjoin%
\definecolor{currentfill}{rgb}{0.000000,0.000000,0.000000}%
\pgfsetfillcolor{currentfill}%
\pgfsetlinewidth{0.803000pt}%
\definecolor{currentstroke}{rgb}{0.000000,0.000000,0.000000}%
\pgfsetstrokecolor{currentstroke}%
\pgfsetdash{}{0pt}%
\pgfsys@defobject{currentmarker}{\pgfqpoint{0.000000in}{-0.048611in}}{\pgfqpoint{0.000000in}{0.000000in}}{%
\pgfpathmoveto{\pgfqpoint{0.000000in}{0.000000in}}%
\pgfpathlineto{\pgfqpoint{0.000000in}{-0.048611in}}%
\pgfusepath{stroke,fill}%
}%
\begin{pgfscope}%
\pgfsys@transformshift{2.364698in}{1.223552in}%
\pgfsys@useobject{currentmarker}{}%
\end{pgfscope}%
\end{pgfscope}%
\begin{pgfscope}%
\definecolor{textcolor}{rgb}{0.000000,0.000000,0.000000}%
\pgfsetstrokecolor{textcolor}%
\pgfsetfillcolor{textcolor}%
\pgftext[x=2.258144in, y=0.784419in, left, base,rotate=45.000000]{\color{textcolor}\sffamily\fontsize{10.000000}{12.000000}\selectfont Pix3d}%
\end{pgfscope}%
\begin{pgfscope}%
\pgfsetbuttcap%
\pgfsetroundjoin%
\definecolor{currentfill}{rgb}{0.000000,0.000000,0.000000}%
\pgfsetfillcolor{currentfill}%
\pgfsetlinewidth{0.803000pt}%
\definecolor{currentstroke}{rgb}{0.000000,0.000000,0.000000}%
\pgfsetstrokecolor{currentstroke}%
\pgfsetdash{}{0pt}%
\pgfsys@defobject{currentmarker}{\pgfqpoint{0.000000in}{-0.048611in}}{\pgfqpoint{0.000000in}{0.000000in}}{%
\pgfpathmoveto{\pgfqpoint{0.000000in}{0.000000in}}%
\pgfpathlineto{\pgfqpoint{0.000000in}{-0.048611in}}%
\pgfusepath{stroke,fill}%
}%
\begin{pgfscope}%
\pgfsys@transformshift{3.401056in}{1.223552in}%
\pgfsys@useobject{currentmarker}{}%
\end{pgfscope}%
\end{pgfscope}%
\begin{pgfscope}%
\definecolor{textcolor}{rgb}{0.000000,0.000000,0.000000}%
\pgfsetstrokecolor{textcolor}%
\pgfsetfillcolor{textcolor}%
\pgftext[x=3.057396in, y=0.313130in, left, base,rotate=45.000000]{\color{textcolor}\sffamily\fontsize{10.000000}{12.000000}\selectfont s2r\_v1(no aug)}%
\end{pgfscope}%
\begin{pgfscope}%
\pgfsetbuttcap%
\pgfsetroundjoin%
\definecolor{currentfill}{rgb}{0.000000,0.000000,0.000000}%
\pgfsetfillcolor{currentfill}%
\pgfsetlinewidth{0.803000pt}%
\definecolor{currentstroke}{rgb}{0.000000,0.000000,0.000000}%
\pgfsetstrokecolor{currentstroke}%
\pgfsetdash{}{0pt}%
\pgfsys@defobject{currentmarker}{\pgfqpoint{0.000000in}{-0.048611in}}{\pgfqpoint{0.000000in}{0.000000in}}{%
\pgfpathmoveto{\pgfqpoint{0.000000in}{0.000000in}}%
\pgfpathlineto{\pgfqpoint{0.000000in}{-0.048611in}}%
\pgfusepath{stroke,fill}%
}%
\begin{pgfscope}%
\pgfsys@transformshift{4.437415in}{1.223552in}%
\pgfsys@useobject{currentmarker}{}%
\end{pgfscope}%
\end{pgfscope}%
\begin{pgfscope}%
\definecolor{textcolor}{rgb}{0.000000,0.000000,0.000000}%
\pgfsetstrokecolor{textcolor}%
\pgfsetfillcolor{textcolor}%
\pgftext[x=4.300411in, y=0.729704in, left, base,rotate=45.000000]{\color{textcolor}\sffamily\fontsize{10.000000}{12.000000}\selectfont s2r\_v1}%
\end{pgfscope}%
\begin{pgfscope}%
\pgfsetbuttcap%
\pgfsetroundjoin%
\definecolor{currentfill}{rgb}{0.000000,0.000000,0.000000}%
\pgfsetfillcolor{currentfill}%
\pgfsetlinewidth{0.803000pt}%
\definecolor{currentstroke}{rgb}{0.000000,0.000000,0.000000}%
\pgfsetstrokecolor{currentstroke}%
\pgfsetdash{}{0pt}%
\pgfsys@defobject{currentmarker}{\pgfqpoint{0.000000in}{-0.048611in}}{\pgfqpoint{0.000000in}{0.000000in}}{%
\pgfpathmoveto{\pgfqpoint{0.000000in}{0.000000in}}%
\pgfpathlineto{\pgfqpoint{0.000000in}{-0.048611in}}%
\pgfusepath{stroke,fill}%
}%
\begin{pgfscope}%
\pgfsys@transformshift{5.473773in}{1.223552in}%
\pgfsys@useobject{currentmarker}{}%
\end{pgfscope}%
\end{pgfscope}%
\begin{pgfscope}%
\definecolor{textcolor}{rgb}{0.000000,0.000000,0.000000}%
\pgfsetstrokecolor{textcolor}%
\pgfsetfillcolor{textcolor}%
\pgftext[x=5.336769in, y=0.729704in, left, base,rotate=45.000000]{\color{textcolor}\sffamily\fontsize{10.000000}{12.000000}\selectfont s2r\_v2}%
\end{pgfscope}%
\begin{pgfscope}%
\definecolor{textcolor}{rgb}{0.000000,0.000000,0.000000}%
\pgfsetstrokecolor{textcolor}%
\pgfsetfillcolor{textcolor}%
\pgftext[x=3.401056in,y=0.234413in,,top]{\color{textcolor}\sffamily\fontsize{10.000000}{12.000000}\selectfont Dataset}%
\end{pgfscope}%
\begin{pgfscope}%
\pgfsetbuttcap%
\pgfsetroundjoin%
\definecolor{currentfill}{rgb}{0.000000,0.000000,0.000000}%
\pgfsetfillcolor{currentfill}%
\pgfsetlinewidth{0.803000pt}%
\definecolor{currentstroke}{rgb}{0.000000,0.000000,0.000000}%
\pgfsetstrokecolor{currentstroke}%
\pgfsetdash{}{0pt}%
\pgfsys@defobject{currentmarker}{\pgfqpoint{-0.048611in}{0.000000in}}{\pgfqpoint{-0.000000in}{0.000000in}}{%
\pgfpathmoveto{\pgfqpoint{-0.000000in}{0.000000in}}%
\pgfpathlineto{\pgfqpoint{-0.048611in}{0.000000in}}%
\pgfusepath{stroke,fill}%
}%
\begin{pgfscope}%
\pgfsys@transformshift{0.608070in}{1.223552in}%
\pgfsys@useobject{currentmarker}{}%
\end{pgfscope}%
\end{pgfscope}%
\begin{pgfscope}%
\definecolor{textcolor}{rgb}{0.000000,0.000000,0.000000}%
\pgfsetstrokecolor{textcolor}%
\pgfsetfillcolor{textcolor}%
\pgftext[x=0.289968in, y=1.170790in, left, base]{\color{textcolor}\sffamily\fontsize{10.000000}{12.000000}\selectfont 0.0}%
\end{pgfscope}%
\begin{pgfscope}%
\pgfsetbuttcap%
\pgfsetroundjoin%
\definecolor{currentfill}{rgb}{0.000000,0.000000,0.000000}%
\pgfsetfillcolor{currentfill}%
\pgfsetlinewidth{0.803000pt}%
\definecolor{currentstroke}{rgb}{0.000000,0.000000,0.000000}%
\pgfsetstrokecolor{currentstroke}%
\pgfsetdash{}{0pt}%
\pgfsys@defobject{currentmarker}{\pgfqpoint{-0.048611in}{0.000000in}}{\pgfqpoint{-0.000000in}{0.000000in}}{%
\pgfpathmoveto{\pgfqpoint{-0.000000in}{0.000000in}}%
\pgfpathlineto{\pgfqpoint{-0.048611in}{0.000000in}}%
\pgfusepath{stroke,fill}%
}%
\begin{pgfscope}%
\pgfsys@transformshift{0.608070in}{1.856369in}%
\pgfsys@useobject{currentmarker}{}%
\end{pgfscope}%
\end{pgfscope}%
\begin{pgfscope}%
\definecolor{textcolor}{rgb}{0.000000,0.000000,0.000000}%
\pgfsetstrokecolor{textcolor}%
\pgfsetfillcolor{textcolor}%
\pgftext[x=0.289968in, y=1.803608in, left, base]{\color{textcolor}\sffamily\fontsize{10.000000}{12.000000}\selectfont 0.1}%
\end{pgfscope}%
\begin{pgfscope}%
\pgfsetbuttcap%
\pgfsetroundjoin%
\definecolor{currentfill}{rgb}{0.000000,0.000000,0.000000}%
\pgfsetfillcolor{currentfill}%
\pgfsetlinewidth{0.803000pt}%
\definecolor{currentstroke}{rgb}{0.000000,0.000000,0.000000}%
\pgfsetstrokecolor{currentstroke}%
\pgfsetdash{}{0pt}%
\pgfsys@defobject{currentmarker}{\pgfqpoint{-0.048611in}{0.000000in}}{\pgfqpoint{-0.000000in}{0.000000in}}{%
\pgfpathmoveto{\pgfqpoint{-0.000000in}{0.000000in}}%
\pgfpathlineto{\pgfqpoint{-0.048611in}{0.000000in}}%
\pgfusepath{stroke,fill}%
}%
\begin{pgfscope}%
\pgfsys@transformshift{0.608070in}{2.489187in}%
\pgfsys@useobject{currentmarker}{}%
\end{pgfscope}%
\end{pgfscope}%
\begin{pgfscope}%
\definecolor{textcolor}{rgb}{0.000000,0.000000,0.000000}%
\pgfsetstrokecolor{textcolor}%
\pgfsetfillcolor{textcolor}%
\pgftext[x=0.289968in, y=2.436426in, left, base]{\color{textcolor}\sffamily\fontsize{10.000000}{12.000000}\selectfont 0.2}%
\end{pgfscope}%
\begin{pgfscope}%
\pgfsetbuttcap%
\pgfsetroundjoin%
\definecolor{currentfill}{rgb}{0.000000,0.000000,0.000000}%
\pgfsetfillcolor{currentfill}%
\pgfsetlinewidth{0.803000pt}%
\definecolor{currentstroke}{rgb}{0.000000,0.000000,0.000000}%
\pgfsetstrokecolor{currentstroke}%
\pgfsetdash{}{0pt}%
\pgfsys@defobject{currentmarker}{\pgfqpoint{-0.048611in}{0.000000in}}{\pgfqpoint{-0.000000in}{0.000000in}}{%
\pgfpathmoveto{\pgfqpoint{-0.000000in}{0.000000in}}%
\pgfpathlineto{\pgfqpoint{-0.048611in}{0.000000in}}%
\pgfusepath{stroke,fill}%
}%
\begin{pgfscope}%
\pgfsys@transformshift{0.608070in}{3.122005in}%
\pgfsys@useobject{currentmarker}{}%
\end{pgfscope}%
\end{pgfscope}%
\begin{pgfscope}%
\definecolor{textcolor}{rgb}{0.000000,0.000000,0.000000}%
\pgfsetstrokecolor{textcolor}%
\pgfsetfillcolor{textcolor}%
\pgftext[x=0.289968in, y=3.069244in, left, base]{\color{textcolor}\sffamily\fontsize{10.000000}{12.000000}\selectfont 0.3}%
\end{pgfscope}%
\begin{pgfscope}%
\pgfsetbuttcap%
\pgfsetroundjoin%
\definecolor{currentfill}{rgb}{0.000000,0.000000,0.000000}%
\pgfsetfillcolor{currentfill}%
\pgfsetlinewidth{0.803000pt}%
\definecolor{currentstroke}{rgb}{0.000000,0.000000,0.000000}%
\pgfsetstrokecolor{currentstroke}%
\pgfsetdash{}{0pt}%
\pgfsys@defobject{currentmarker}{\pgfqpoint{-0.048611in}{0.000000in}}{\pgfqpoint{-0.000000in}{0.000000in}}{%
\pgfpathmoveto{\pgfqpoint{-0.000000in}{0.000000in}}%
\pgfpathlineto{\pgfqpoint{-0.048611in}{0.000000in}}%
\pgfusepath{stroke,fill}%
}%
\begin{pgfscope}%
\pgfsys@transformshift{0.608070in}{3.754823in}%
\pgfsys@useobject{currentmarker}{}%
\end{pgfscope}%
\end{pgfscope}%
\begin{pgfscope}%
\definecolor{textcolor}{rgb}{0.000000,0.000000,0.000000}%
\pgfsetstrokecolor{textcolor}%
\pgfsetfillcolor{textcolor}%
\pgftext[x=0.289968in, y=3.702061in, left, base]{\color{textcolor}\sffamily\fontsize{10.000000}{12.000000}\selectfont 0.4}%
\end{pgfscope}%
\begin{pgfscope}%
\pgfsetbuttcap%
\pgfsetroundjoin%
\definecolor{currentfill}{rgb}{0.000000,0.000000,0.000000}%
\pgfsetfillcolor{currentfill}%
\pgfsetlinewidth{0.803000pt}%
\definecolor{currentstroke}{rgb}{0.000000,0.000000,0.000000}%
\pgfsetstrokecolor{currentstroke}%
\pgfsetdash{}{0pt}%
\pgfsys@defobject{currentmarker}{\pgfqpoint{-0.048611in}{0.000000in}}{\pgfqpoint{-0.000000in}{0.000000in}}{%
\pgfpathmoveto{\pgfqpoint{-0.000000in}{0.000000in}}%
\pgfpathlineto{\pgfqpoint{-0.048611in}{0.000000in}}%
\pgfusepath{stroke,fill}%
}%
\begin{pgfscope}%
\pgfsys@transformshift{0.608070in}{4.387641in}%
\pgfsys@useobject{currentmarker}{}%
\end{pgfscope}%
\end{pgfscope}%
\begin{pgfscope}%
\definecolor{textcolor}{rgb}{0.000000,0.000000,0.000000}%
\pgfsetstrokecolor{textcolor}%
\pgfsetfillcolor{textcolor}%
\pgftext[x=0.289968in, y=4.334879in, left, base]{\color{textcolor}\sffamily\fontsize{10.000000}{12.000000}\selectfont 0.5}%
\end{pgfscope}%
\begin{pgfscope}%
\definecolor{textcolor}{rgb}{0.000000,0.000000,0.000000}%
\pgfsetstrokecolor{textcolor}%
\pgfsetfillcolor{textcolor}%
\pgftext[x=0.234413in,y=2.805596in,,bottom,rotate=90.000000]{\color{textcolor}\sffamily\fontsize{10.000000}{12.000000}\selectfont F1 Score}%
\end{pgfscope}%
\begin{pgfscope}%
\pgfsetrectcap%
\pgfsetmiterjoin%
\pgfsetlinewidth{0.803000pt}%
\definecolor{currentstroke}{rgb}{0.000000,0.000000,0.000000}%
\pgfsetstrokecolor{currentstroke}%
\pgfsetdash{}{0pt}%
\pgfpathmoveto{\pgfqpoint{0.608070in}{1.223552in}}%
\pgfpathlineto{\pgfqpoint{0.608070in}{4.387641in}}%
\pgfusepath{stroke}%
\end{pgfscope}%
\begin{pgfscope}%
\pgfsetrectcap%
\pgfsetmiterjoin%
\pgfsetlinewidth{0.803000pt}%
\definecolor{currentstroke}{rgb}{0.000000,0.000000,0.000000}%
\pgfsetstrokecolor{currentstroke}%
\pgfsetdash{}{0pt}%
\pgfpathmoveto{\pgfqpoint{6.194042in}{1.223552in}}%
\pgfpathlineto{\pgfqpoint{6.194042in}{4.387641in}}%
\pgfusepath{stroke}%
\end{pgfscope}%
\begin{pgfscope}%
\pgfsetrectcap%
\pgfsetmiterjoin%
\pgfsetlinewidth{0.803000pt}%
\definecolor{currentstroke}{rgb}{0.000000,0.000000,0.000000}%
\pgfsetstrokecolor{currentstroke}%
\pgfsetdash{}{0pt}%
\pgfpathmoveto{\pgfqpoint{0.608070in}{1.223552in}}%
\pgfpathlineto{\pgfqpoint{6.194042in}{1.223552in}}%
\pgfusepath{stroke}%
\end{pgfscope}%
\begin{pgfscope}%
\pgfsetrectcap%
\pgfsetmiterjoin%
\pgfsetlinewidth{0.803000pt}%
\definecolor{currentstroke}{rgb}{0.000000,0.000000,0.000000}%
\pgfsetstrokecolor{currentstroke}%
\pgfsetdash{}{0pt}%
\pgfpathmoveto{\pgfqpoint{0.608070in}{4.387641in}}%
\pgfpathlineto{\pgfqpoint{6.194042in}{4.387641in}}%
\pgfusepath{stroke}%
\end{pgfscope}%
\begin{pgfscope}%
\definecolor{textcolor}{rgb}{0.000000,0.000000,0.000000}%
\pgfsetstrokecolor{textcolor}%
\pgfsetfillcolor{textcolor}%
\pgftext[x=1.095158in,y=3.658535in,,bottom]{\color{textcolor}\sffamily\fontsize{9.000000}{10.800000}\selectfont 0.3782}%
\end{pgfscope}%
\begin{pgfscope}%
\definecolor{textcolor}{rgb}{0.000000,0.000000,0.000000}%
\pgfsetstrokecolor{textcolor}%
\pgfsetfillcolor{textcolor}%
\pgftext[x=2.131517in,y=4.062906in,,bottom]{\color{textcolor}\sffamily\fontsize{9.000000}{10.800000}\selectfont 0.4421}%
\end{pgfscope}%
\begin{pgfscope}%
\definecolor{textcolor}{rgb}{0.000000,0.000000,0.000000}%
\pgfsetstrokecolor{textcolor}%
\pgfsetfillcolor{textcolor}%
\pgftext[x=3.167875in,y=1.932208in,,bottom]{\color{textcolor}\sffamily\fontsize{9.000000}{10.800000}\selectfont 0.1054}%
\end{pgfscope}%
\begin{pgfscope}%
\definecolor{textcolor}{rgb}{0.000000,0.000000,0.000000}%
\pgfsetstrokecolor{textcolor}%
\pgfsetfillcolor{textcolor}%
\pgftext[x=4.204234in,y=2.199890in,,bottom]{\color{textcolor}\sffamily\fontsize{9.000000}{10.800000}\selectfont 0.1477}%
\end{pgfscope}%
\begin{pgfscope}%
\definecolor{textcolor}{rgb}{0.000000,0.000000,0.000000}%
\pgfsetstrokecolor{textcolor}%
\pgfsetfillcolor{textcolor}%
\pgftext[x=5.240592in,y=2.635902in,,bottom]{\color{textcolor}\sffamily\fontsize{9.000000}{10.800000}\selectfont 0.2166}%
\end{pgfscope}%
\begin{pgfscope}%
\definecolor{textcolor}{rgb}{0.000000,0.000000,0.000000}%
\pgfsetstrokecolor{textcolor}%
\pgfsetfillcolor{textcolor}%
\pgftext[x=1.561520in,y=3.851545in,,bottom]{\color{textcolor}\sffamily\fontsize{9.000000}{10.800000}\selectfont 0.4087}%
\end{pgfscope}%
\begin{pgfscope}%
\definecolor{textcolor}{rgb}{0.000000,0.000000,0.000000}%
\pgfsetstrokecolor{textcolor}%
\pgfsetfillcolor{textcolor}%
\pgftext[x=2.597878in,y=3.974311in,,bottom]{\color{textcolor}\sffamily\fontsize{9.000000}{10.800000}\selectfont 0.4281}%
\end{pgfscope}%
\begin{pgfscope}%
\definecolor{textcolor}{rgb}{0.000000,0.000000,0.000000}%
\pgfsetstrokecolor{textcolor}%
\pgfsetfillcolor{textcolor}%
\pgftext[x=3.634237in,y=2.031561in,,bottom]{\color{textcolor}\sffamily\fontsize{9.000000}{10.800000}\selectfont 0.1211}%
\end{pgfscope}%
\begin{pgfscope}%
\definecolor{textcolor}{rgb}{0.000000,0.000000,0.000000}%
\pgfsetstrokecolor{textcolor}%
\pgfsetfillcolor{textcolor}%
\pgftext[x=4.670595in,y=1.923349in,,bottom]{\color{textcolor}\sffamily\fontsize{9.000000}{10.800000}\selectfont 0.104}%
\end{pgfscope}%
\begin{pgfscope}%
\definecolor{textcolor}{rgb}{0.000000,0.000000,0.000000}%
\pgfsetstrokecolor{textcolor}%
\pgfsetfillcolor{textcolor}%
\pgftext[x=5.706954in,y=2.954842in,,bottom]{\color{textcolor}\sffamily\fontsize{9.000000}{10.800000}\selectfont 0.267}%
\end{pgfscope}%
\begin{pgfscope}%
\definecolor{textcolor}{rgb}{0.000000,0.000000,0.000000}%
\pgfsetstrokecolor{textcolor}%
\pgfsetfillcolor{textcolor}%
\pgftext[x=3.401056in,y=4.470974in,,base]{\color{textcolor}\sffamily\fontsize{12.000000}{14.400000}\selectfont Baselines trained on Pix3D and S2R:3DFREE}%
\end{pgfscope}%
\begin{pgfscope}%
\pgfsetbuttcap%
\pgfsetmiterjoin%
\definecolor{currentfill}{rgb}{1.000000,1.000000,1.000000}%
\pgfsetfillcolor{currentfill}%
\pgfsetfillopacity{0.800000}%
\pgfsetlinewidth{1.003750pt}%
\definecolor{currentstroke}{rgb}{0.800000,0.800000,0.800000}%
\pgfsetstrokecolor{currentstroke}%
\pgfsetstrokeopacity{0.800000}%
\pgfsetdash{}{0pt}%
\pgfpathmoveto{\pgfqpoint{4.882654in}{3.868815in}}%
\pgfpathlineto{\pgfqpoint{6.096820in}{3.868815in}}%
\pgfpathquadraticcurveto{\pgfqpoint{6.124598in}{3.868815in}}{\pgfqpoint{6.124598in}{3.896593in}}%
\pgfpathlineto{\pgfqpoint{6.124598in}{4.290419in}}%
\pgfpathquadraticcurveto{\pgfqpoint{6.124598in}{4.318196in}}{\pgfqpoint{6.096820in}{4.318196in}}%
\pgfpathlineto{\pgfqpoint{4.882654in}{4.318196in}}%
\pgfpathquadraticcurveto{\pgfqpoint{4.854877in}{4.318196in}}{\pgfqpoint{4.854877in}{4.290419in}}%
\pgfpathlineto{\pgfqpoint{4.854877in}{3.896593in}}%
\pgfpathquadraticcurveto{\pgfqpoint{4.854877in}{3.868815in}}{\pgfqpoint{4.882654in}{3.868815in}}%
\pgfpathclose%
\pgfusepath{stroke,fill}%
\end{pgfscope}%
\begin{pgfscope}%
\pgfsetbuttcap%
\pgfsetmiterjoin%
\definecolor{currentfill}{rgb}{0.121569,0.466667,0.705882}%
\pgfsetfillcolor{currentfill}%
\pgfsetlinewidth{0.000000pt}%
\definecolor{currentstroke}{rgb}{0.000000,0.000000,0.000000}%
\pgfsetstrokecolor{currentstroke}%
\pgfsetstrokeopacity{0.000000}%
\pgfsetdash{}{0pt}%
\pgfpathmoveto{\pgfqpoint{4.910432in}{4.157118in}}%
\pgfpathlineto{\pgfqpoint{5.188210in}{4.157118in}}%
\pgfpathlineto{\pgfqpoint{5.188210in}{4.254340in}}%
\pgfpathlineto{\pgfqpoint{4.910432in}{4.254340in}}%
\pgfpathclose%
\pgfusepath{fill}%
\end{pgfscope}%
\begin{pgfscope}%
\definecolor{textcolor}{rgb}{0.000000,0.000000,0.000000}%
\pgfsetstrokecolor{textcolor}%
\pgfsetfillcolor{textcolor}%
\pgftext[x=5.299321in,y=4.157118in,left,base]{\color{textcolor}\sffamily\fontsize{10.000000}{12.000000}\selectfont Pix2Vox++}%
\end{pgfscope}%
\begin{pgfscope}%
\pgfsetbuttcap%
\pgfsetmiterjoin%
\definecolor{currentfill}{rgb}{1.000000,0.498039,0.054902}%
\pgfsetfillcolor{currentfill}%
\pgfsetlinewidth{0.000000pt}%
\definecolor{currentstroke}{rgb}{0.000000,0.000000,0.000000}%
\pgfsetstrokecolor{currentstroke}%
\pgfsetstrokeopacity{0.000000}%
\pgfsetdash{}{0pt}%
\pgfpathmoveto{\pgfqpoint{4.910432in}{3.953260in}}%
\pgfpathlineto{\pgfqpoint{5.188210in}{3.953260in}}%
\pgfpathlineto{\pgfqpoint{5.188210in}{4.050483in}}%
\pgfpathlineto{\pgfqpoint{4.910432in}{4.050483in}}%
\pgfpathclose%
\pgfusepath{fill}%
\end{pgfscope}%
\begin{pgfscope}%
\definecolor{textcolor}{rgb}{0.000000,0.000000,0.000000}%
\pgfsetstrokecolor{textcolor}%
\pgfsetfillcolor{textcolor}%
\pgftext[x=5.299321in,y=3.953260in,left,base]{\color{textcolor}\sffamily\fontsize{10.000000}{12.000000}\selectfont Pix2Vox}%
\end{pgfscope}%
\end{pgfpicture}%
\makeatother%
\endgroup%
}
    \caption{Bar plot for the \gls{f1}  for baselines trained on real and synthetic datasets, with and without 2D augmentation.
    We see that ~\gls{free} does not perform adequately on its own. \gls{s2rv2} contributes slightly better than \gls{s2rv1}.}
    \label{fig:baseline_dice1}
\end{figure}

There is a difference of 6.39\% and 1.94\% for Pix2vox++ and Pix2vox when trained with and without 2D augmentation, which shows its importance.
There is a massive dip in performance when we train the models only on the proposed synthetic dataset (\gls(s2rv1) and \gls(s2rv2)).
Among the two versions, \gls(s2rv2) 6.8\% and 1.6\% more than \gls{s2rv1}, for pix2vox++ and pix2vox, respectively, proving that multi-object is the better dataset.
%
%\begin{figure}
%    \centering
%    \resizebox{0.6\linewidth}{!}{\input{/Users/apple/OVGU/Thesis/code/3dReconstruction/report/images/evaluation/performance/baseline_dice_linegraph1.pgf}}
%    \caption{Line plot for the \gls{f1}  for baselines trained on real and synthetic datasets, with and without 2D augmentation.
%        We see that ~\gls{free} does not perform adequately on its own. \gls{s2rv2} contributes slightly better than \gls{s2rv1}.}
%    \label{fig:baseline_dice1}
%\end{figure}

\subsection{Fine-Tuning}\label{subsec:fine-tuning-dice}

This section is contemporary of what we experimented in \autoref{sec:fine-tuning}.
\autoref{fig:finetuning_dice1} compares models trained with and without fine-tuning with real datasets and models trained on only a real dataset.
We see an increment of only 0.3\% for pix2vox trained on \gls{s2rv2} and fine-tuned with Pix3d.
In all other cases, the performance is decreased instead of improving.

We see the same results in  \autoref{fig:finetuning_dice2} and \autoref{fig:finetuning_dice3},
where the \gls{f1} for each category with fine-tuned models is less than the models trained only on the real dataset.
%
%\begin{figure}
%    \centering
%    \resizebox{0.6\textwidth}{!}{%% Creator: Matplotlib, PGF backend
%%
%% To include the figure in your LaTeX document, write
%%   \input{<filename>.pgf}
%%
%% Make sure the required packages are loaded in your preamble
%%   \usepackage{pgf}
%%
%% Figures using additional raster images can only be included by \input if
%% they are in the same directory as the main LaTeX file. For loading figures
%% from other directories you can use the `import` package
%%   \usepackage{import}
%%
%% and then include the figures with
%%   \import{<path to file>}{<filename>.pgf}
%%
%% Matplotlib used the following preamble
%%   \usepackage{fontspec}
%%   \setmainfont{DejaVuSerif.ttf}[Path=\detokenize{/Users/apple/opt/anaconda3/envs/kaolin/lib/python3.7/site-packages/matplotlib/mpl-data/fonts/ttf/}]
%%   \setsansfont{DejaVuSans.ttf}[Path=\detokenize{/Users/apple/opt/anaconda3/envs/kaolin/lib/python3.7/site-packages/matplotlib/mpl-data/fonts/ttf/}]
%%   \setmonofont{DejaVuSansMono.ttf}[Path=\detokenize{/Users/apple/opt/anaconda3/envs/kaolin/lib/python3.7/site-packages/matplotlib/mpl-data/fonts/ttf/}]
%%
\begingroup%
\makeatletter%
\begin{pgfpicture}%
\pgfpathrectangle{\pgfpointorigin}{\pgfqpoint{5.830801in}{5.012323in}}%
\pgfusepath{use as bounding box, clip}%
\begin{pgfscope}%
\pgfsetbuttcap%
\pgfsetmiterjoin%
\definecolor{currentfill}{rgb}{1.000000,1.000000,1.000000}%
\pgfsetfillcolor{currentfill}%
\pgfsetlinewidth{0.000000pt}%
\definecolor{currentstroke}{rgb}{1.000000,1.000000,1.000000}%
\pgfsetstrokecolor{currentstroke}%
\pgfsetdash{}{0pt}%
\pgfpathmoveto{\pgfqpoint{0.000000in}{0.000000in}}%
\pgfpathlineto{\pgfqpoint{5.830801in}{0.000000in}}%
\pgfpathlineto{\pgfqpoint{5.830801in}{5.012323in}}%
\pgfpathlineto{\pgfqpoint{0.000000in}{5.012323in}}%
\pgfpathclose%
\pgfusepath{fill}%
\end{pgfscope}%
\begin{pgfscope}%
\pgfsetbuttcap%
\pgfsetmiterjoin%
\definecolor{currentfill}{rgb}{1.000000,1.000000,1.000000}%
\pgfsetfillcolor{currentfill}%
\pgfsetlinewidth{0.000000pt}%
\definecolor{currentstroke}{rgb}{0.000000,0.000000,0.000000}%
\pgfsetstrokecolor{currentstroke}%
\pgfsetstrokeopacity{0.000000}%
\pgfsetdash{}{0pt}%
\pgfpathmoveto{\pgfqpoint{0.608070in}{1.163562in}}%
\pgfpathlineto{\pgfqpoint{5.568070in}{1.163562in}}%
\pgfpathlineto{\pgfqpoint{5.568070in}{4.859562in}}%
\pgfpathlineto{\pgfqpoint{0.608070in}{4.859562in}}%
\pgfpathclose%
\pgfusepath{fill}%
\end{pgfscope}%
\begin{pgfscope}%
\pgfsetbuttcap%
\pgfsetroundjoin%
\definecolor{currentfill}{rgb}{0.000000,0.000000,0.000000}%
\pgfsetfillcolor{currentfill}%
\pgfsetlinewidth{0.803000pt}%
\definecolor{currentstroke}{rgb}{0.000000,0.000000,0.000000}%
\pgfsetstrokecolor{currentstroke}%
\pgfsetdash{}{0pt}%
\pgfsys@defobject{currentmarker}{\pgfqpoint{0.000000in}{-0.048611in}}{\pgfqpoint{0.000000in}{0.000000in}}{%
\pgfpathmoveto{\pgfqpoint{0.000000in}{0.000000in}}%
\pgfpathlineto{\pgfqpoint{0.000000in}{-0.048611in}}%
\pgfusepath{stroke,fill}%
}%
\begin{pgfscope}%
\pgfsys@transformshift{0.833525in}{1.163562in}%
\pgfsys@useobject{currentmarker}{}%
\end{pgfscope}%
\end{pgfscope}%
\begin{pgfscope}%
\definecolor{textcolor}{rgb}{0.000000,0.000000,0.000000}%
\pgfsetstrokecolor{textcolor}%
\pgfsetfillcolor{textcolor}%
\pgftext[x=0.720330in, y=0.711146in, left, base,rotate=45.000000]{\color{textcolor}\sffamily\fontsize{10.000000}{12.000000}\selectfont Pix3D}%
\end{pgfscope}%
\begin{pgfscope}%
\pgfsetbuttcap%
\pgfsetroundjoin%
\definecolor{currentfill}{rgb}{0.000000,0.000000,0.000000}%
\pgfsetfillcolor{currentfill}%
\pgfsetlinewidth{0.803000pt}%
\definecolor{currentstroke}{rgb}{0.000000,0.000000,0.000000}%
\pgfsetstrokecolor{currentstroke}%
\pgfsetdash{}{0pt}%
\pgfsys@defobject{currentmarker}{\pgfqpoint{0.000000in}{-0.048611in}}{\pgfqpoint{0.000000in}{0.000000in}}{%
\pgfpathmoveto{\pgfqpoint{0.000000in}{0.000000in}}%
\pgfpathlineto{\pgfqpoint{0.000000in}{-0.048611in}}%
\pgfusepath{stroke,fill}%
}%
\begin{pgfscope}%
\pgfsys@transformshift{1.960797in}{1.163562in}%
\pgfsys@useobject{currentmarker}{}%
\end{pgfscope}%
\end{pgfscope}%
\begin{pgfscope}%
\definecolor{textcolor}{rgb}{0.000000,0.000000,0.000000}%
\pgfsetstrokecolor{textcolor}%
\pgfsetfillcolor{textcolor}%
\pgftext[x=1.823794in, y=0.669714in, left, base,rotate=45.000000]{\color{textcolor}\sffamily\fontsize{10.000000}{12.000000}\selectfont s2r\_v1}%
\end{pgfscope}%
\begin{pgfscope}%
\pgfsetbuttcap%
\pgfsetroundjoin%
\definecolor{currentfill}{rgb}{0.000000,0.000000,0.000000}%
\pgfsetfillcolor{currentfill}%
\pgfsetlinewidth{0.803000pt}%
\definecolor{currentstroke}{rgb}{0.000000,0.000000,0.000000}%
\pgfsetstrokecolor{currentstroke}%
\pgfsetdash{}{0pt}%
\pgfsys@defobject{currentmarker}{\pgfqpoint{0.000000in}{-0.048611in}}{\pgfqpoint{0.000000in}{0.000000in}}{%
\pgfpathmoveto{\pgfqpoint{0.000000in}{0.000000in}}%
\pgfpathlineto{\pgfqpoint{0.000000in}{-0.048611in}}%
\pgfusepath{stroke,fill}%
}%
\begin{pgfscope}%
\pgfsys@transformshift{3.088070in}{1.163562in}%
\pgfsys@useobject{currentmarker}{}%
\end{pgfscope}%
\end{pgfscope}%
\begin{pgfscope}%
\definecolor{textcolor}{rgb}{0.000000,0.000000,0.000000}%
\pgfsetstrokecolor{textcolor}%
\pgfsetfillcolor{textcolor}%
\pgftext[x=2.951066in, y=0.669714in, left, base,rotate=45.000000]{\color{textcolor}\sffamily\fontsize{10.000000}{12.000000}\selectfont s2r\_v2}%
\end{pgfscope}%
\begin{pgfscope}%
\pgfsetbuttcap%
\pgfsetroundjoin%
\definecolor{currentfill}{rgb}{0.000000,0.000000,0.000000}%
\pgfsetfillcolor{currentfill}%
\pgfsetlinewidth{0.803000pt}%
\definecolor{currentstroke}{rgb}{0.000000,0.000000,0.000000}%
\pgfsetstrokecolor{currentstroke}%
\pgfsetdash{}{0pt}%
\pgfsys@defobject{currentmarker}{\pgfqpoint{0.000000in}{-0.048611in}}{\pgfqpoint{0.000000in}{0.000000in}}{%
\pgfpathmoveto{\pgfqpoint{0.000000in}{0.000000in}}%
\pgfpathlineto{\pgfqpoint{0.000000in}{-0.048611in}}%
\pgfusepath{stroke,fill}%
}%
\begin{pgfscope}%
\pgfsys@transformshift{4.215343in}{1.163562in}%
\pgfsys@useobject{currentmarker}{}%
\end{pgfscope}%
\end{pgfscope}%
\begin{pgfscope}%
\definecolor{textcolor}{rgb}{0.000000,0.000000,0.000000}%
\pgfsetstrokecolor{textcolor}%
\pgfsetfillcolor{textcolor}%
\pgftext[x=3.901773in, y=0.313130in, left, base,rotate=45.000000]{\color{textcolor}\sffamily\fontsize{10.000000}{12.000000}\selectfont s2r\_v1+pix3d}%
\end{pgfscope}%
\begin{pgfscope}%
\pgfsetbuttcap%
\pgfsetroundjoin%
\definecolor{currentfill}{rgb}{0.000000,0.000000,0.000000}%
\pgfsetfillcolor{currentfill}%
\pgfsetlinewidth{0.803000pt}%
\definecolor{currentstroke}{rgb}{0.000000,0.000000,0.000000}%
\pgfsetstrokecolor{currentstroke}%
\pgfsetdash{}{0pt}%
\pgfsys@defobject{currentmarker}{\pgfqpoint{0.000000in}{-0.048611in}}{\pgfqpoint{0.000000in}{0.000000in}}{%
\pgfpathmoveto{\pgfqpoint{0.000000in}{0.000000in}}%
\pgfpathlineto{\pgfqpoint{0.000000in}{-0.048611in}}%
\pgfusepath{stroke,fill}%
}%
\begin{pgfscope}%
\pgfsys@transformshift{5.342615in}{1.163562in}%
\pgfsys@useobject{currentmarker}{}%
\end{pgfscope}%
\end{pgfscope}%
\begin{pgfscope}%
\definecolor{textcolor}{rgb}{0.000000,0.000000,0.000000}%
\pgfsetstrokecolor{textcolor}%
\pgfsetfillcolor{textcolor}%
\pgftext[x=5.029046in, y=0.313130in, left, base,rotate=45.000000]{\color{textcolor}\sffamily\fontsize{10.000000}{12.000000}\selectfont s2r\_v2+pix3d}%
\end{pgfscope}%
\begin{pgfscope}%
\definecolor{textcolor}{rgb}{0.000000,0.000000,0.000000}%
\pgfsetstrokecolor{textcolor}%
\pgfsetfillcolor{textcolor}%
\pgftext[x=3.088070in,y=0.234413in,,top]{\color{textcolor}\sffamily\fontsize{10.000000}{12.000000}\selectfont Datasets}%
\end{pgfscope}%
\begin{pgfscope}%
\pgfsetbuttcap%
\pgfsetroundjoin%
\definecolor{currentfill}{rgb}{0.000000,0.000000,0.000000}%
\pgfsetfillcolor{currentfill}%
\pgfsetlinewidth{0.803000pt}%
\definecolor{currentstroke}{rgb}{0.000000,0.000000,0.000000}%
\pgfsetstrokecolor{currentstroke}%
\pgfsetdash{}{0pt}%
\pgfsys@defobject{currentmarker}{\pgfqpoint{-0.048611in}{0.000000in}}{\pgfqpoint{-0.000000in}{0.000000in}}{%
\pgfpathmoveto{\pgfqpoint{-0.000000in}{0.000000in}}%
\pgfpathlineto{\pgfqpoint{-0.048611in}{0.000000in}}%
\pgfusepath{stroke,fill}%
}%
\begin{pgfscope}%
\pgfsys@transformshift{0.608070in}{1.163562in}%
\pgfsys@useobject{currentmarker}{}%
\end{pgfscope}%
\end{pgfscope}%
\begin{pgfscope}%
\definecolor{textcolor}{rgb}{0.000000,0.000000,0.000000}%
\pgfsetstrokecolor{textcolor}%
\pgfsetfillcolor{textcolor}%
\pgftext[x=0.289968in, y=1.110800in, left, base]{\color{textcolor}\sffamily\fontsize{10.000000}{12.000000}\selectfont 0.0}%
\end{pgfscope}%
\begin{pgfscope}%
\pgfsetbuttcap%
\pgfsetroundjoin%
\definecolor{currentfill}{rgb}{0.000000,0.000000,0.000000}%
\pgfsetfillcolor{currentfill}%
\pgfsetlinewidth{0.803000pt}%
\definecolor{currentstroke}{rgb}{0.000000,0.000000,0.000000}%
\pgfsetstrokecolor{currentstroke}%
\pgfsetdash{}{0pt}%
\pgfsys@defobject{currentmarker}{\pgfqpoint{-0.048611in}{0.000000in}}{\pgfqpoint{-0.000000in}{0.000000in}}{%
\pgfpathmoveto{\pgfqpoint{-0.000000in}{0.000000in}}%
\pgfpathlineto{\pgfqpoint{-0.048611in}{0.000000in}}%
\pgfusepath{stroke,fill}%
}%
\begin{pgfscope}%
\pgfsys@transformshift{0.608070in}{1.902762in}%
\pgfsys@useobject{currentmarker}{}%
\end{pgfscope}%
\end{pgfscope}%
\begin{pgfscope}%
\definecolor{textcolor}{rgb}{0.000000,0.000000,0.000000}%
\pgfsetstrokecolor{textcolor}%
\pgfsetfillcolor{textcolor}%
\pgftext[x=0.289968in, y=1.850000in, left, base]{\color{textcolor}\sffamily\fontsize{10.000000}{12.000000}\selectfont 0.1}%
\end{pgfscope}%
\begin{pgfscope}%
\pgfsetbuttcap%
\pgfsetroundjoin%
\definecolor{currentfill}{rgb}{0.000000,0.000000,0.000000}%
\pgfsetfillcolor{currentfill}%
\pgfsetlinewidth{0.803000pt}%
\definecolor{currentstroke}{rgb}{0.000000,0.000000,0.000000}%
\pgfsetstrokecolor{currentstroke}%
\pgfsetdash{}{0pt}%
\pgfsys@defobject{currentmarker}{\pgfqpoint{-0.048611in}{0.000000in}}{\pgfqpoint{-0.000000in}{0.000000in}}{%
\pgfpathmoveto{\pgfqpoint{-0.000000in}{0.000000in}}%
\pgfpathlineto{\pgfqpoint{-0.048611in}{0.000000in}}%
\pgfusepath{stroke,fill}%
}%
\begin{pgfscope}%
\pgfsys@transformshift{0.608070in}{2.641962in}%
\pgfsys@useobject{currentmarker}{}%
\end{pgfscope}%
\end{pgfscope}%
\begin{pgfscope}%
\definecolor{textcolor}{rgb}{0.000000,0.000000,0.000000}%
\pgfsetstrokecolor{textcolor}%
\pgfsetfillcolor{textcolor}%
\pgftext[x=0.289968in, y=2.589200in, left, base]{\color{textcolor}\sffamily\fontsize{10.000000}{12.000000}\selectfont 0.2}%
\end{pgfscope}%
\begin{pgfscope}%
\pgfsetbuttcap%
\pgfsetroundjoin%
\definecolor{currentfill}{rgb}{0.000000,0.000000,0.000000}%
\pgfsetfillcolor{currentfill}%
\pgfsetlinewidth{0.803000pt}%
\definecolor{currentstroke}{rgb}{0.000000,0.000000,0.000000}%
\pgfsetstrokecolor{currentstroke}%
\pgfsetdash{}{0pt}%
\pgfsys@defobject{currentmarker}{\pgfqpoint{-0.048611in}{0.000000in}}{\pgfqpoint{-0.000000in}{0.000000in}}{%
\pgfpathmoveto{\pgfqpoint{-0.000000in}{0.000000in}}%
\pgfpathlineto{\pgfqpoint{-0.048611in}{0.000000in}}%
\pgfusepath{stroke,fill}%
}%
\begin{pgfscope}%
\pgfsys@transformshift{0.608070in}{3.381162in}%
\pgfsys@useobject{currentmarker}{}%
\end{pgfscope}%
\end{pgfscope}%
\begin{pgfscope}%
\definecolor{textcolor}{rgb}{0.000000,0.000000,0.000000}%
\pgfsetstrokecolor{textcolor}%
\pgfsetfillcolor{textcolor}%
\pgftext[x=0.289968in, y=3.328400in, left, base]{\color{textcolor}\sffamily\fontsize{10.000000}{12.000000}\selectfont 0.3}%
\end{pgfscope}%
\begin{pgfscope}%
\pgfsetbuttcap%
\pgfsetroundjoin%
\definecolor{currentfill}{rgb}{0.000000,0.000000,0.000000}%
\pgfsetfillcolor{currentfill}%
\pgfsetlinewidth{0.803000pt}%
\definecolor{currentstroke}{rgb}{0.000000,0.000000,0.000000}%
\pgfsetstrokecolor{currentstroke}%
\pgfsetdash{}{0pt}%
\pgfsys@defobject{currentmarker}{\pgfqpoint{-0.048611in}{0.000000in}}{\pgfqpoint{-0.000000in}{0.000000in}}{%
\pgfpathmoveto{\pgfqpoint{-0.000000in}{0.000000in}}%
\pgfpathlineto{\pgfqpoint{-0.048611in}{0.000000in}}%
\pgfusepath{stroke,fill}%
}%
\begin{pgfscope}%
\pgfsys@transformshift{0.608070in}{4.120362in}%
\pgfsys@useobject{currentmarker}{}%
\end{pgfscope}%
\end{pgfscope}%
\begin{pgfscope}%
\definecolor{textcolor}{rgb}{0.000000,0.000000,0.000000}%
\pgfsetstrokecolor{textcolor}%
\pgfsetfillcolor{textcolor}%
\pgftext[x=0.289968in, y=4.067600in, left, base]{\color{textcolor}\sffamily\fontsize{10.000000}{12.000000}\selectfont 0.4}%
\end{pgfscope}%
\begin{pgfscope}%
\pgfsetbuttcap%
\pgfsetroundjoin%
\definecolor{currentfill}{rgb}{0.000000,0.000000,0.000000}%
\pgfsetfillcolor{currentfill}%
\pgfsetlinewidth{0.803000pt}%
\definecolor{currentstroke}{rgb}{0.000000,0.000000,0.000000}%
\pgfsetstrokecolor{currentstroke}%
\pgfsetdash{}{0pt}%
\pgfsys@defobject{currentmarker}{\pgfqpoint{-0.048611in}{0.000000in}}{\pgfqpoint{-0.000000in}{0.000000in}}{%
\pgfpathmoveto{\pgfqpoint{-0.000000in}{0.000000in}}%
\pgfpathlineto{\pgfqpoint{-0.048611in}{0.000000in}}%
\pgfusepath{stroke,fill}%
}%
\begin{pgfscope}%
\pgfsys@transformshift{0.608070in}{4.859562in}%
\pgfsys@useobject{currentmarker}{}%
\end{pgfscope}%
\end{pgfscope}%
\begin{pgfscope}%
\definecolor{textcolor}{rgb}{0.000000,0.000000,0.000000}%
\pgfsetstrokecolor{textcolor}%
\pgfsetfillcolor{textcolor}%
\pgftext[x=0.289968in, y=4.806800in, left, base]{\color{textcolor}\sffamily\fontsize{10.000000}{12.000000}\selectfont 0.5}%
\end{pgfscope}%
\begin{pgfscope}%
\definecolor{textcolor}{rgb}{0.000000,0.000000,0.000000}%
\pgfsetstrokecolor{textcolor}%
\pgfsetfillcolor{textcolor}%
\pgftext[x=0.234413in,y=3.011562in,,bottom,rotate=90.000000]{\color{textcolor}\sffamily\fontsize{10.000000}{12.000000}\selectfont F1 Score}%
\end{pgfscope}%
\begin{pgfscope}%
\pgfpathrectangle{\pgfqpoint{0.608070in}{1.163562in}}{\pgfqpoint{4.960000in}{3.696000in}}%
\pgfusepath{clip}%
\pgfsetrectcap%
\pgfsetroundjoin%
\pgfsetlinewidth{1.505625pt}%
\definecolor{currentstroke}{rgb}{0.121569,0.466667,0.705882}%
\pgfsetstrokecolor{currentstroke}%
\pgfsetdash{}{0pt}%
\pgfpathmoveto{\pgfqpoint{0.833525in}{4.431565in}}%
\pgfpathlineto{\pgfqpoint{1.960797in}{2.255360in}}%
\pgfpathlineto{\pgfqpoint{3.088070in}{2.764595in}}%
\pgfpathlineto{\pgfqpoint{4.215343in}{4.251200in}}%
\pgfpathlineto{\pgfqpoint{5.342615in}{4.168410in}}%
\pgfusepath{stroke}%
\end{pgfscope}%
\begin{pgfscope}%
\pgfpathrectangle{\pgfqpoint{0.608070in}{1.163562in}}{\pgfqpoint{4.960000in}{3.696000in}}%
\pgfusepath{clip}%
\pgfsetbuttcap%
\pgfsetroundjoin%
\definecolor{currentfill}{rgb}{0.121569,0.466667,0.705882}%
\pgfsetfillcolor{currentfill}%
\pgfsetlinewidth{1.003750pt}%
\definecolor{currentstroke}{rgb}{0.121569,0.466667,0.705882}%
\pgfsetstrokecolor{currentstroke}%
\pgfsetdash{}{0pt}%
\pgfsys@defobject{currentmarker}{\pgfqpoint{-0.041667in}{-0.041667in}}{\pgfqpoint{0.041667in}{0.041667in}}{%
\pgfpathmoveto{\pgfqpoint{0.000000in}{-0.041667in}}%
\pgfpathcurveto{\pgfqpoint{0.011050in}{-0.041667in}}{\pgfqpoint{0.021649in}{-0.037276in}}{\pgfqpoint{0.029463in}{-0.029463in}}%
\pgfpathcurveto{\pgfqpoint{0.037276in}{-0.021649in}}{\pgfqpoint{0.041667in}{-0.011050in}}{\pgfqpoint{0.041667in}{0.000000in}}%
\pgfpathcurveto{\pgfqpoint{0.041667in}{0.011050in}}{\pgfqpoint{0.037276in}{0.021649in}}{\pgfqpoint{0.029463in}{0.029463in}}%
\pgfpathcurveto{\pgfqpoint{0.021649in}{0.037276in}}{\pgfqpoint{0.011050in}{0.041667in}}{\pgfqpoint{0.000000in}{0.041667in}}%
\pgfpathcurveto{\pgfqpoint{-0.011050in}{0.041667in}}{\pgfqpoint{-0.021649in}{0.037276in}}{\pgfqpoint{-0.029463in}{0.029463in}}%
\pgfpathcurveto{\pgfqpoint{-0.037276in}{0.021649in}}{\pgfqpoint{-0.041667in}{0.011050in}}{\pgfqpoint{-0.041667in}{0.000000in}}%
\pgfpathcurveto{\pgfqpoint{-0.041667in}{-0.011050in}}{\pgfqpoint{-0.037276in}{-0.021649in}}{\pgfqpoint{-0.029463in}{-0.029463in}}%
\pgfpathcurveto{\pgfqpoint{-0.021649in}{-0.037276in}}{\pgfqpoint{-0.011050in}{-0.041667in}}{\pgfqpoint{0.000000in}{-0.041667in}}%
\pgfpathclose%
\pgfusepath{stroke,fill}%
}%
\begin{pgfscope}%
\pgfsys@transformshift{0.833525in}{4.431565in}%
\pgfsys@useobject{currentmarker}{}%
\end{pgfscope}%
\begin{pgfscope}%
\pgfsys@transformshift{1.960797in}{2.255360in}%
\pgfsys@useobject{currentmarker}{}%
\end{pgfscope}%
\begin{pgfscope}%
\pgfsys@transformshift{3.088070in}{2.764595in}%
\pgfsys@useobject{currentmarker}{}%
\end{pgfscope}%
\begin{pgfscope}%
\pgfsys@transformshift{4.215343in}{4.251200in}%
\pgfsys@useobject{currentmarker}{}%
\end{pgfscope}%
\begin{pgfscope}%
\pgfsys@transformshift{5.342615in}{4.168410in}%
\pgfsys@useobject{currentmarker}{}%
\end{pgfscope}%
\end{pgfscope}%
\begin{pgfscope}%
\pgfpathrectangle{\pgfqpoint{0.608070in}{1.163562in}}{\pgfqpoint{4.960000in}{3.696000in}}%
\pgfusepath{clip}%
\pgfsetrectcap%
\pgfsetroundjoin%
\pgfsetlinewidth{1.505625pt}%
\definecolor{currentstroke}{rgb}{1.000000,0.498039,0.054902}%
\pgfsetstrokecolor{currentstroke}%
\pgfsetdash{}{0pt}%
\pgfpathmoveto{\pgfqpoint{0.833525in}{4.328077in}}%
\pgfpathlineto{\pgfqpoint{1.960797in}{1.932330in}}%
\pgfpathlineto{\pgfqpoint{3.088070in}{3.137226in}}%
\pgfpathlineto{\pgfqpoint{4.215343in}{4.234938in}}%
\pgfpathlineto{\pgfqpoint{5.342615in}{4.349514in}}%
\pgfusepath{stroke}%
\end{pgfscope}%
\begin{pgfscope}%
\pgfpathrectangle{\pgfqpoint{0.608070in}{1.163562in}}{\pgfqpoint{4.960000in}{3.696000in}}%
\pgfusepath{clip}%
\pgfsetbuttcap%
\pgfsetmiterjoin%
\definecolor{currentfill}{rgb}{1.000000,0.498039,0.054902}%
\pgfsetfillcolor{currentfill}%
\pgfsetlinewidth{1.003750pt}%
\definecolor{currentstroke}{rgb}{1.000000,0.498039,0.054902}%
\pgfsetstrokecolor{currentstroke}%
\pgfsetdash{}{0pt}%
\pgfsys@defobject{currentmarker}{\pgfqpoint{-0.041667in}{-0.041667in}}{\pgfqpoint{0.041667in}{0.041667in}}{%
\pgfpathmoveto{\pgfqpoint{-0.000000in}{-0.041667in}}%
\pgfpathlineto{\pgfqpoint{0.041667in}{0.041667in}}%
\pgfpathlineto{\pgfqpoint{-0.041667in}{0.041667in}}%
\pgfpathclose%
\pgfusepath{stroke,fill}%
}%
\begin{pgfscope}%
\pgfsys@transformshift{0.833525in}{4.328077in}%
\pgfsys@useobject{currentmarker}{}%
\end{pgfscope}%
\begin{pgfscope}%
\pgfsys@transformshift{1.960797in}{1.932330in}%
\pgfsys@useobject{currentmarker}{}%
\end{pgfscope}%
\begin{pgfscope}%
\pgfsys@transformshift{3.088070in}{3.137226in}%
\pgfsys@useobject{currentmarker}{}%
\end{pgfscope}%
\begin{pgfscope}%
\pgfsys@transformshift{4.215343in}{4.234938in}%
\pgfsys@useobject{currentmarker}{}%
\end{pgfscope}%
\begin{pgfscope}%
\pgfsys@transformshift{5.342615in}{4.349514in}%
\pgfsys@useobject{currentmarker}{}%
\end{pgfscope}%
\end{pgfscope}%
\begin{pgfscope}%
\pgfsetrectcap%
\pgfsetmiterjoin%
\pgfsetlinewidth{0.803000pt}%
\definecolor{currentstroke}{rgb}{0.000000,0.000000,0.000000}%
\pgfsetstrokecolor{currentstroke}%
\pgfsetdash{}{0pt}%
\pgfpathmoveto{\pgfqpoint{0.608070in}{1.163562in}}%
\pgfpathlineto{\pgfqpoint{0.608070in}{4.859562in}}%
\pgfusepath{stroke}%
\end{pgfscope}%
\begin{pgfscope}%
\pgfsetrectcap%
\pgfsetmiterjoin%
\pgfsetlinewidth{0.803000pt}%
\definecolor{currentstroke}{rgb}{0.000000,0.000000,0.000000}%
\pgfsetstrokecolor{currentstroke}%
\pgfsetdash{}{0pt}%
\pgfpathmoveto{\pgfqpoint{5.568070in}{1.163562in}}%
\pgfpathlineto{\pgfqpoint{5.568070in}{4.859562in}}%
\pgfusepath{stroke}%
\end{pgfscope}%
\begin{pgfscope}%
\pgfsetrectcap%
\pgfsetmiterjoin%
\pgfsetlinewidth{0.803000pt}%
\definecolor{currentstroke}{rgb}{0.000000,0.000000,0.000000}%
\pgfsetstrokecolor{currentstroke}%
\pgfsetdash{}{0pt}%
\pgfpathmoveto{\pgfqpoint{0.608070in}{1.163562in}}%
\pgfpathlineto{\pgfqpoint{5.568070in}{1.163562in}}%
\pgfusepath{stroke}%
\end{pgfscope}%
\begin{pgfscope}%
\pgfsetrectcap%
\pgfsetmiterjoin%
\pgfsetlinewidth{0.803000pt}%
\definecolor{currentstroke}{rgb}{0.000000,0.000000,0.000000}%
\pgfsetstrokecolor{currentstroke}%
\pgfsetdash{}{0pt}%
\pgfpathmoveto{\pgfqpoint{0.608070in}{4.859562in}}%
\pgfpathlineto{\pgfqpoint{5.568070in}{4.859562in}}%
\pgfusepath{stroke}%
\end{pgfscope}%
\begin{pgfscope}%
\pgfsetbuttcap%
\pgfsetmiterjoin%
\definecolor{currentfill}{rgb}{1.000000,1.000000,1.000000}%
\pgfsetfillcolor{currentfill}%
\pgfsetfillopacity{0.800000}%
\pgfsetlinewidth{1.003750pt}%
\definecolor{currentstroke}{rgb}{0.800000,0.800000,0.800000}%
\pgfsetstrokecolor{currentstroke}%
\pgfsetstrokeopacity{0.800000}%
\pgfsetdash{}{0pt}%
\pgfpathmoveto{\pgfqpoint{0.705292in}{1.233006in}}%
\pgfpathlineto{\pgfqpoint{1.919458in}{1.233006in}}%
\pgfpathquadraticcurveto{\pgfqpoint{1.947236in}{1.233006in}}{\pgfqpoint{1.947236in}{1.260784in}}%
\pgfpathlineto{\pgfqpoint{1.947236in}{1.654609in}}%
\pgfpathquadraticcurveto{\pgfqpoint{1.947236in}{1.682387in}}{\pgfqpoint{1.919458in}{1.682387in}}%
\pgfpathlineto{\pgfqpoint{0.705292in}{1.682387in}}%
\pgfpathquadraticcurveto{\pgfqpoint{0.677514in}{1.682387in}}{\pgfqpoint{0.677514in}{1.654609in}}%
\pgfpathlineto{\pgfqpoint{0.677514in}{1.260784in}}%
\pgfpathquadraticcurveto{\pgfqpoint{0.677514in}{1.233006in}}{\pgfqpoint{0.705292in}{1.233006in}}%
\pgfpathclose%
\pgfusepath{stroke,fill}%
\end{pgfscope}%
\begin{pgfscope}%
\pgfsetrectcap%
\pgfsetroundjoin%
\pgfsetlinewidth{1.505625pt}%
\definecolor{currentstroke}{rgb}{0.121569,0.466667,0.705882}%
\pgfsetstrokecolor{currentstroke}%
\pgfsetdash{}{0pt}%
\pgfpathmoveto{\pgfqpoint{0.733070in}{1.569920in}}%
\pgfpathlineto{\pgfqpoint{1.010848in}{1.569920in}}%
\pgfusepath{stroke}%
\end{pgfscope}%
\begin{pgfscope}%
\pgfsetbuttcap%
\pgfsetroundjoin%
\definecolor{currentfill}{rgb}{0.121569,0.466667,0.705882}%
\pgfsetfillcolor{currentfill}%
\pgfsetlinewidth{1.003750pt}%
\definecolor{currentstroke}{rgb}{0.121569,0.466667,0.705882}%
\pgfsetstrokecolor{currentstroke}%
\pgfsetdash{}{0pt}%
\pgfsys@defobject{currentmarker}{\pgfqpoint{-0.041667in}{-0.041667in}}{\pgfqpoint{0.041667in}{0.041667in}}{%
\pgfpathmoveto{\pgfqpoint{0.000000in}{-0.041667in}}%
\pgfpathcurveto{\pgfqpoint{0.011050in}{-0.041667in}}{\pgfqpoint{0.021649in}{-0.037276in}}{\pgfqpoint{0.029463in}{-0.029463in}}%
\pgfpathcurveto{\pgfqpoint{0.037276in}{-0.021649in}}{\pgfqpoint{0.041667in}{-0.011050in}}{\pgfqpoint{0.041667in}{0.000000in}}%
\pgfpathcurveto{\pgfqpoint{0.041667in}{0.011050in}}{\pgfqpoint{0.037276in}{0.021649in}}{\pgfqpoint{0.029463in}{0.029463in}}%
\pgfpathcurveto{\pgfqpoint{0.021649in}{0.037276in}}{\pgfqpoint{0.011050in}{0.041667in}}{\pgfqpoint{0.000000in}{0.041667in}}%
\pgfpathcurveto{\pgfqpoint{-0.011050in}{0.041667in}}{\pgfqpoint{-0.021649in}{0.037276in}}{\pgfqpoint{-0.029463in}{0.029463in}}%
\pgfpathcurveto{\pgfqpoint{-0.037276in}{0.021649in}}{\pgfqpoint{-0.041667in}{0.011050in}}{\pgfqpoint{-0.041667in}{0.000000in}}%
\pgfpathcurveto{\pgfqpoint{-0.041667in}{-0.011050in}}{\pgfqpoint{-0.037276in}{-0.021649in}}{\pgfqpoint{-0.029463in}{-0.029463in}}%
\pgfpathcurveto{\pgfqpoint{-0.021649in}{-0.037276in}}{\pgfqpoint{-0.011050in}{-0.041667in}}{\pgfqpoint{0.000000in}{-0.041667in}}%
\pgfpathclose%
\pgfusepath{stroke,fill}%
}%
\begin{pgfscope}%
\pgfsys@transformshift{0.871959in}{1.569920in}%
\pgfsys@useobject{currentmarker}{}%
\end{pgfscope}%
\end{pgfscope}%
\begin{pgfscope}%
\definecolor{textcolor}{rgb}{0.000000,0.000000,0.000000}%
\pgfsetstrokecolor{textcolor}%
\pgfsetfillcolor{textcolor}%
\pgftext[x=1.121959in,y=1.521309in,left,base]{\color{textcolor}\sffamily\fontsize{10.000000}{12.000000}\selectfont Pix2Vox++}%
\end{pgfscope}%
\begin{pgfscope}%
\pgfsetrectcap%
\pgfsetroundjoin%
\pgfsetlinewidth{1.505625pt}%
\definecolor{currentstroke}{rgb}{1.000000,0.498039,0.054902}%
\pgfsetstrokecolor{currentstroke}%
\pgfsetdash{}{0pt}%
\pgfpathmoveto{\pgfqpoint{0.733070in}{1.366062in}}%
\pgfpathlineto{\pgfqpoint{1.010848in}{1.366062in}}%
\pgfusepath{stroke}%
\end{pgfscope}%
\begin{pgfscope}%
\pgfsetbuttcap%
\pgfsetmiterjoin%
\definecolor{currentfill}{rgb}{1.000000,0.498039,0.054902}%
\pgfsetfillcolor{currentfill}%
\pgfsetlinewidth{1.003750pt}%
\definecolor{currentstroke}{rgb}{1.000000,0.498039,0.054902}%
\pgfsetstrokecolor{currentstroke}%
\pgfsetdash{}{0pt}%
\pgfsys@defobject{currentmarker}{\pgfqpoint{-0.041667in}{-0.041667in}}{\pgfqpoint{0.041667in}{0.041667in}}{%
\pgfpathmoveto{\pgfqpoint{-0.000000in}{-0.041667in}}%
\pgfpathlineto{\pgfqpoint{0.041667in}{0.041667in}}%
\pgfpathlineto{\pgfqpoint{-0.041667in}{0.041667in}}%
\pgfpathclose%
\pgfusepath{stroke,fill}%
}%
\begin{pgfscope}%
\pgfsys@transformshift{0.871959in}{1.366062in}%
\pgfsys@useobject{currentmarker}{}%
\end{pgfscope}%
\end{pgfscope}%
\begin{pgfscope}%
\definecolor{textcolor}{rgb}{0.000000,0.000000,0.000000}%
\pgfsetstrokecolor{textcolor}%
\pgfsetfillcolor{textcolor}%
\pgftext[x=1.121959in,y=1.317451in,left,base]{\color{textcolor}\sffamily\fontsize{10.000000}{12.000000}\selectfont Pix2Vox}%
\end{pgfscope}%
\end{pgfpicture}%
\makeatother%
\endgroup%
}
%    \caption{Line plot for the \gls{f1} for baseline models(Pix2Vox++ and Pix2Vox) trained on synthetic and fine-tuned with real dataset.
%    We see that even after fine-tuning both the models do not perform as good as models trained on only real dataset.}
%    \label{fig:finetuning_dice1}
%\end{figure}


\begin{figure}[ht]
    \centering
    \resizebox{0.75\textwidth}{!}{%% Creator: Matplotlib, PGF backend
%%
%% To include the figure in your LaTeX document, write
%%   \input{<filename>.pgf}
%%
%% Make sure the required packages are loaded in your preamble
%%   \usepackage{pgf}
%%
%% Figures using additional raster images can only be included by \input if
%% they are in the same directory as the main LaTeX file. For loading figures
%% from other directories you can use the `import` package
%%   \usepackage{import}
%%
%% and then include the figures with
%%   \import{<path to file>}{<filename>.pgf}
%%
%% Matplotlib used the following preamble
%%   \usepackage{fontspec}
%%   \setmainfont{DejaVuSerif.ttf}[Path=\detokenize{/Users/apple/opt/anaconda3/envs/kaolin/lib/python3.7/site-packages/matplotlib/mpl-data/fonts/ttf/}]
%%   \setsansfont{DejaVuSans.ttf}[Path=\detokenize{/Users/apple/opt/anaconda3/envs/kaolin/lib/python3.7/site-packages/matplotlib/mpl-data/fonts/ttf/}]
%%   \setmonofont{DejaVuSansMono.ttf}[Path=\detokenize{/Users/apple/opt/anaconda3/envs/kaolin/lib/python3.7/site-packages/matplotlib/mpl-data/fonts/ttf/}]
%%
\begingroup%
\makeatletter%
\begin{pgfpicture}%
\pgfpathrectangle{\pgfpointorigin}{\pgfqpoint{6.294042in}{4.695948in}}%
\pgfusepath{use as bounding box, clip}%
\begin{pgfscope}%
\pgfsetbuttcap%
\pgfsetmiterjoin%
\definecolor{currentfill}{rgb}{1.000000,1.000000,1.000000}%
\pgfsetfillcolor{currentfill}%
\pgfsetlinewidth{0.000000pt}%
\definecolor{currentstroke}{rgb}{1.000000,1.000000,1.000000}%
\pgfsetstrokecolor{currentstroke}%
\pgfsetdash{}{0pt}%
\pgfpathmoveto{\pgfqpoint{0.000000in}{0.000000in}}%
\pgfpathlineto{\pgfqpoint{6.294042in}{0.000000in}}%
\pgfpathlineto{\pgfqpoint{6.294042in}{4.695948in}}%
\pgfpathlineto{\pgfqpoint{0.000000in}{4.695948in}}%
\pgfpathclose%
\pgfusepath{fill}%
\end{pgfscope}%
\begin{pgfscope}%
\pgfsetbuttcap%
\pgfsetmiterjoin%
\definecolor{currentfill}{rgb}{1.000000,1.000000,1.000000}%
\pgfsetfillcolor{currentfill}%
\pgfsetlinewidth{0.000000pt}%
\definecolor{currentstroke}{rgb}{0.000000,0.000000,0.000000}%
\pgfsetstrokecolor{currentstroke}%
\pgfsetstrokeopacity{0.000000}%
\pgfsetdash{}{0pt}%
\pgfpathmoveto{\pgfqpoint{0.608070in}{1.163562in}}%
\pgfpathlineto{\pgfqpoint{6.194042in}{1.163562in}}%
\pgfpathlineto{\pgfqpoint{6.194042in}{4.385987in}}%
\pgfpathlineto{\pgfqpoint{0.608070in}{4.385987in}}%
\pgfpathclose%
\pgfusepath{fill}%
\end{pgfscope}%
\begin{pgfscope}%
\pgfpathrectangle{\pgfqpoint{0.608070in}{1.163562in}}{\pgfqpoint{5.585972in}{3.222425in}}%
\pgfusepath{clip}%
\pgfsetbuttcap%
\pgfsetmiterjoin%
\definecolor{currentfill}{rgb}{0.121569,0.466667,0.705882}%
\pgfsetfillcolor{currentfill}%
\pgfsetlinewidth{0.000000pt}%
\definecolor{currentstroke}{rgb}{0.000000,0.000000,0.000000}%
\pgfsetstrokecolor{currentstroke}%
\pgfsetstrokeopacity{0.000000}%
\pgfsetdash{}{0pt}%
\pgfpathmoveto{\pgfqpoint{0.861978in}{1.163562in}}%
\pgfpathlineto{\pgfqpoint{1.328339in}{1.163562in}}%
\pgfpathlineto{\pgfqpoint{1.328339in}{3.537952in}}%
\pgfpathlineto{\pgfqpoint{0.861978in}{3.537952in}}%
\pgfpathclose%
\pgfusepath{fill}%
\end{pgfscope}%
\begin{pgfscope}%
\pgfpathrectangle{\pgfqpoint{0.608070in}{1.163562in}}{\pgfqpoint{5.585972in}{3.222425in}}%
\pgfusepath{clip}%
\pgfsetbuttcap%
\pgfsetmiterjoin%
\definecolor{currentfill}{rgb}{0.121569,0.466667,0.705882}%
\pgfsetfillcolor{currentfill}%
\pgfsetlinewidth{0.000000pt}%
\definecolor{currentstroke}{rgb}{0.000000,0.000000,0.000000}%
\pgfsetstrokecolor{currentstroke}%
\pgfsetstrokeopacity{0.000000}%
\pgfsetdash{}{0pt}%
\pgfpathmoveto{\pgfqpoint{1.898336in}{1.163562in}}%
\pgfpathlineto{\pgfqpoint{2.364698in}{1.163562in}}%
\pgfpathlineto{\pgfqpoint{2.364698in}{1.956815in}}%
\pgfpathlineto{\pgfqpoint{1.898336in}{1.956815in}}%
\pgfpathclose%
\pgfusepath{fill}%
\end{pgfscope}%
\begin{pgfscope}%
\pgfpathrectangle{\pgfqpoint{0.608070in}{1.163562in}}{\pgfqpoint{5.585972in}{3.222425in}}%
\pgfusepath{clip}%
\pgfsetbuttcap%
\pgfsetmiterjoin%
\definecolor{currentfill}{rgb}{0.121569,0.466667,0.705882}%
\pgfsetfillcolor{currentfill}%
\pgfsetlinewidth{0.000000pt}%
\definecolor{currentstroke}{rgb}{0.000000,0.000000,0.000000}%
\pgfsetstrokecolor{currentstroke}%
\pgfsetstrokeopacity{0.000000}%
\pgfsetdash{}{0pt}%
\pgfpathmoveto{\pgfqpoint{2.934695in}{1.163562in}}%
\pgfpathlineto{\pgfqpoint{3.401056in}{1.163562in}}%
\pgfpathlineto{\pgfqpoint{3.401056in}{2.326803in}}%
\pgfpathlineto{\pgfqpoint{2.934695in}{2.326803in}}%
\pgfpathclose%
\pgfusepath{fill}%
\end{pgfscope}%
\begin{pgfscope}%
\pgfpathrectangle{\pgfqpoint{0.608070in}{1.163562in}}{\pgfqpoint{5.585972in}{3.222425in}}%
\pgfusepath{clip}%
\pgfsetbuttcap%
\pgfsetmiterjoin%
\definecolor{currentfill}{rgb}{0.121569,0.466667,0.705882}%
\pgfsetfillcolor{currentfill}%
\pgfsetlinewidth{0.000000pt}%
\definecolor{currentstroke}{rgb}{0.000000,0.000000,0.000000}%
\pgfsetstrokecolor{currentstroke}%
\pgfsetstrokeopacity{0.000000}%
\pgfsetdash{}{0pt}%
\pgfpathmoveto{\pgfqpoint{3.971053in}{1.163562in}}%
\pgfpathlineto{\pgfqpoint{4.437415in}{1.163562in}}%
\pgfpathlineto{\pgfqpoint{4.437415in}{3.406907in}}%
\pgfpathlineto{\pgfqpoint{3.971053in}{3.406907in}}%
\pgfpathclose%
\pgfusepath{fill}%
\end{pgfscope}%
\begin{pgfscope}%
\pgfpathrectangle{\pgfqpoint{0.608070in}{1.163562in}}{\pgfqpoint{5.585972in}{3.222425in}}%
\pgfusepath{clip}%
\pgfsetbuttcap%
\pgfsetmiterjoin%
\definecolor{currentfill}{rgb}{0.121569,0.466667,0.705882}%
\pgfsetfillcolor{currentfill}%
\pgfsetlinewidth{0.000000pt}%
\definecolor{currentstroke}{rgb}{0.000000,0.000000,0.000000}%
\pgfsetstrokecolor{currentstroke}%
\pgfsetstrokeopacity{0.000000}%
\pgfsetdash{}{0pt}%
\pgfpathmoveto{\pgfqpoint{5.007412in}{1.163562in}}%
\pgfpathlineto{\pgfqpoint{5.473773in}{1.163562in}}%
\pgfpathlineto{\pgfqpoint{5.473773in}{3.346755in}}%
\pgfpathlineto{\pgfqpoint{5.007412in}{3.346755in}}%
\pgfpathclose%
\pgfusepath{fill}%
\end{pgfscope}%
\begin{pgfscope}%
\pgfpathrectangle{\pgfqpoint{0.608070in}{1.163562in}}{\pgfqpoint{5.585972in}{3.222425in}}%
\pgfusepath{clip}%
\pgfsetbuttcap%
\pgfsetmiterjoin%
\definecolor{currentfill}{rgb}{1.000000,0.498039,0.054902}%
\pgfsetfillcolor{currentfill}%
\pgfsetlinewidth{0.000000pt}%
\definecolor{currentstroke}{rgb}{0.000000,0.000000,0.000000}%
\pgfsetstrokecolor{currentstroke}%
\pgfsetstrokeopacity{0.000000}%
\pgfsetdash{}{0pt}%
\pgfpathmoveto{\pgfqpoint{1.328339in}{1.163562in}}%
\pgfpathlineto{\pgfqpoint{1.794700in}{1.163562in}}%
\pgfpathlineto{\pgfqpoint{1.794700in}{3.462762in}}%
\pgfpathlineto{\pgfqpoint{1.328339in}{3.462762in}}%
\pgfpathclose%
\pgfusepath{fill}%
\end{pgfscope}%
\begin{pgfscope}%
\pgfpathrectangle{\pgfqpoint{0.608070in}{1.163562in}}{\pgfqpoint{5.585972in}{3.222425in}}%
\pgfusepath{clip}%
\pgfsetbuttcap%
\pgfsetmiterjoin%
\definecolor{currentfill}{rgb}{1.000000,0.498039,0.054902}%
\pgfsetfillcolor{currentfill}%
\pgfsetlinewidth{0.000000pt}%
\definecolor{currentstroke}{rgb}{0.000000,0.000000,0.000000}%
\pgfsetstrokecolor{currentstroke}%
\pgfsetstrokeopacity{0.000000}%
\pgfsetdash{}{0pt}%
\pgfpathmoveto{\pgfqpoint{2.364698in}{1.163562in}}%
\pgfpathlineto{\pgfqpoint{2.831059in}{1.163562in}}%
\pgfpathlineto{\pgfqpoint{2.831059in}{1.722115in}}%
\pgfpathlineto{\pgfqpoint{2.364698in}{1.722115in}}%
\pgfpathclose%
\pgfusepath{fill}%
\end{pgfscope}%
\begin{pgfscope}%
\pgfpathrectangle{\pgfqpoint{0.608070in}{1.163562in}}{\pgfqpoint{5.585972in}{3.222425in}}%
\pgfusepath{clip}%
\pgfsetbuttcap%
\pgfsetmiterjoin%
\definecolor{currentfill}{rgb}{1.000000,0.498039,0.054902}%
\pgfsetfillcolor{currentfill}%
\pgfsetlinewidth{0.000000pt}%
\definecolor{currentstroke}{rgb}{0.000000,0.000000,0.000000}%
\pgfsetstrokecolor{currentstroke}%
\pgfsetstrokeopacity{0.000000}%
\pgfsetdash{}{0pt}%
\pgfpathmoveto{\pgfqpoint{3.401056in}{1.163562in}}%
\pgfpathlineto{\pgfqpoint{3.867417in}{1.163562in}}%
\pgfpathlineto{\pgfqpoint{3.867417in}{2.597541in}}%
\pgfpathlineto{\pgfqpoint{3.401056in}{2.597541in}}%
\pgfpathclose%
\pgfusepath{fill}%
\end{pgfscope}%
\begin{pgfscope}%
\pgfpathrectangle{\pgfqpoint{0.608070in}{1.163562in}}{\pgfqpoint{5.585972in}{3.222425in}}%
\pgfusepath{clip}%
\pgfsetbuttcap%
\pgfsetmiterjoin%
\definecolor{currentfill}{rgb}{1.000000,0.498039,0.054902}%
\pgfsetfillcolor{currentfill}%
\pgfsetlinewidth{0.000000pt}%
\definecolor{currentstroke}{rgb}{0.000000,0.000000,0.000000}%
\pgfsetstrokecolor{currentstroke}%
\pgfsetstrokeopacity{0.000000}%
\pgfsetdash{}{0pt}%
\pgfpathmoveto{\pgfqpoint{4.437415in}{1.163562in}}%
\pgfpathlineto{\pgfqpoint{4.903776in}{1.163562in}}%
\pgfpathlineto{\pgfqpoint{4.903776in}{3.395091in}}%
\pgfpathlineto{\pgfqpoint{4.437415in}{3.395091in}}%
\pgfpathclose%
\pgfusepath{fill}%
\end{pgfscope}%
\begin{pgfscope}%
\pgfpathrectangle{\pgfqpoint{0.608070in}{1.163562in}}{\pgfqpoint{5.585972in}{3.222425in}}%
\pgfusepath{clip}%
\pgfsetbuttcap%
\pgfsetmiterjoin%
\definecolor{currentfill}{rgb}{1.000000,0.498039,0.054902}%
\pgfsetfillcolor{currentfill}%
\pgfsetlinewidth{0.000000pt}%
\definecolor{currentstroke}{rgb}{0.000000,0.000000,0.000000}%
\pgfsetstrokecolor{currentstroke}%
\pgfsetstrokeopacity{0.000000}%
\pgfsetdash{}{0pt}%
\pgfpathmoveto{\pgfqpoint{5.473773in}{1.163562in}}%
\pgfpathlineto{\pgfqpoint{5.940134in}{1.163562in}}%
\pgfpathlineto{\pgfqpoint{5.940134in}{3.478337in}}%
\pgfpathlineto{\pgfqpoint{5.473773in}{3.478337in}}%
\pgfpathclose%
\pgfusepath{fill}%
\end{pgfscope}%
\begin{pgfscope}%
\pgfsetbuttcap%
\pgfsetroundjoin%
\definecolor{currentfill}{rgb}{0.000000,0.000000,0.000000}%
\pgfsetfillcolor{currentfill}%
\pgfsetlinewidth{0.803000pt}%
\definecolor{currentstroke}{rgb}{0.000000,0.000000,0.000000}%
\pgfsetstrokecolor{currentstroke}%
\pgfsetdash{}{0pt}%
\pgfsys@defobject{currentmarker}{\pgfqpoint{0.000000in}{-0.048611in}}{\pgfqpoint{0.000000in}{0.000000in}}{%
\pgfpathmoveto{\pgfqpoint{0.000000in}{0.000000in}}%
\pgfpathlineto{\pgfqpoint{0.000000in}{-0.048611in}}%
\pgfusepath{stroke,fill}%
}%
\begin{pgfscope}%
\pgfsys@transformshift{1.328339in}{1.163562in}%
\pgfsys@useobject{currentmarker}{}%
\end{pgfscope}%
\end{pgfscope}%
\begin{pgfscope}%
\definecolor{textcolor}{rgb}{0.000000,0.000000,0.000000}%
\pgfsetstrokecolor{textcolor}%
\pgfsetfillcolor{textcolor}%
\pgftext[x=1.215144in, y=0.711146in, left, base,rotate=45.000000]{\color{textcolor}\sffamily\fontsize{10.000000}{12.000000}\selectfont Pix3D}%
\end{pgfscope}%
\begin{pgfscope}%
\pgfsetbuttcap%
\pgfsetroundjoin%
\definecolor{currentfill}{rgb}{0.000000,0.000000,0.000000}%
\pgfsetfillcolor{currentfill}%
\pgfsetlinewidth{0.803000pt}%
\definecolor{currentstroke}{rgb}{0.000000,0.000000,0.000000}%
\pgfsetstrokecolor{currentstroke}%
\pgfsetdash{}{0pt}%
\pgfsys@defobject{currentmarker}{\pgfqpoint{0.000000in}{-0.048611in}}{\pgfqpoint{0.000000in}{0.000000in}}{%
\pgfpathmoveto{\pgfqpoint{0.000000in}{0.000000in}}%
\pgfpathlineto{\pgfqpoint{0.000000in}{-0.048611in}}%
\pgfusepath{stroke,fill}%
}%
\begin{pgfscope}%
\pgfsys@transformshift{2.364698in}{1.163562in}%
\pgfsys@useobject{currentmarker}{}%
\end{pgfscope}%
\end{pgfscope}%
\begin{pgfscope}%
\definecolor{textcolor}{rgb}{0.000000,0.000000,0.000000}%
\pgfsetstrokecolor{textcolor}%
\pgfsetfillcolor{textcolor}%
\pgftext[x=2.227694in, y=0.669714in, left, base,rotate=45.000000]{\color{textcolor}\sffamily\fontsize{10.000000}{12.000000}\selectfont s2r\_v1}%
\end{pgfscope}%
\begin{pgfscope}%
\pgfsetbuttcap%
\pgfsetroundjoin%
\definecolor{currentfill}{rgb}{0.000000,0.000000,0.000000}%
\pgfsetfillcolor{currentfill}%
\pgfsetlinewidth{0.803000pt}%
\definecolor{currentstroke}{rgb}{0.000000,0.000000,0.000000}%
\pgfsetstrokecolor{currentstroke}%
\pgfsetdash{}{0pt}%
\pgfsys@defobject{currentmarker}{\pgfqpoint{0.000000in}{-0.048611in}}{\pgfqpoint{0.000000in}{0.000000in}}{%
\pgfpathmoveto{\pgfqpoint{0.000000in}{0.000000in}}%
\pgfpathlineto{\pgfqpoint{0.000000in}{-0.048611in}}%
\pgfusepath{stroke,fill}%
}%
\begin{pgfscope}%
\pgfsys@transformshift{3.401056in}{1.163562in}%
\pgfsys@useobject{currentmarker}{}%
\end{pgfscope}%
\end{pgfscope}%
\begin{pgfscope}%
\definecolor{textcolor}{rgb}{0.000000,0.000000,0.000000}%
\pgfsetstrokecolor{textcolor}%
\pgfsetfillcolor{textcolor}%
\pgftext[x=3.264052in, y=0.669714in, left, base,rotate=45.000000]{\color{textcolor}\sffamily\fontsize{10.000000}{12.000000}\selectfont s2r\_v2}%
\end{pgfscope}%
\begin{pgfscope}%
\pgfsetbuttcap%
\pgfsetroundjoin%
\definecolor{currentfill}{rgb}{0.000000,0.000000,0.000000}%
\pgfsetfillcolor{currentfill}%
\pgfsetlinewidth{0.803000pt}%
\definecolor{currentstroke}{rgb}{0.000000,0.000000,0.000000}%
\pgfsetstrokecolor{currentstroke}%
\pgfsetdash{}{0pt}%
\pgfsys@defobject{currentmarker}{\pgfqpoint{0.000000in}{-0.048611in}}{\pgfqpoint{0.000000in}{0.000000in}}{%
\pgfpathmoveto{\pgfqpoint{0.000000in}{0.000000in}}%
\pgfpathlineto{\pgfqpoint{0.000000in}{-0.048611in}}%
\pgfusepath{stroke,fill}%
}%
\begin{pgfscope}%
\pgfsys@transformshift{4.437415in}{1.163562in}%
\pgfsys@useobject{currentmarker}{}%
\end{pgfscope}%
\end{pgfscope}%
\begin{pgfscope}%
\definecolor{textcolor}{rgb}{0.000000,0.000000,0.000000}%
\pgfsetstrokecolor{textcolor}%
\pgfsetfillcolor{textcolor}%
\pgftext[x=4.123845in, y=0.313130in, left, base,rotate=45.000000]{\color{textcolor}\sffamily\fontsize{10.000000}{12.000000}\selectfont s2r\_v1+pix3d}%
\end{pgfscope}%
\begin{pgfscope}%
\pgfsetbuttcap%
\pgfsetroundjoin%
\definecolor{currentfill}{rgb}{0.000000,0.000000,0.000000}%
\pgfsetfillcolor{currentfill}%
\pgfsetlinewidth{0.803000pt}%
\definecolor{currentstroke}{rgb}{0.000000,0.000000,0.000000}%
\pgfsetstrokecolor{currentstroke}%
\pgfsetdash{}{0pt}%
\pgfsys@defobject{currentmarker}{\pgfqpoint{0.000000in}{-0.048611in}}{\pgfqpoint{0.000000in}{0.000000in}}{%
\pgfpathmoveto{\pgfqpoint{0.000000in}{0.000000in}}%
\pgfpathlineto{\pgfqpoint{0.000000in}{-0.048611in}}%
\pgfusepath{stroke,fill}%
}%
\begin{pgfscope}%
\pgfsys@transformshift{5.473773in}{1.163562in}%
\pgfsys@useobject{currentmarker}{}%
\end{pgfscope}%
\end{pgfscope}%
\begin{pgfscope}%
\definecolor{textcolor}{rgb}{0.000000,0.000000,0.000000}%
\pgfsetstrokecolor{textcolor}%
\pgfsetfillcolor{textcolor}%
\pgftext[x=5.160204in, y=0.313130in, left, base,rotate=45.000000]{\color{textcolor}\sffamily\fontsize{10.000000}{12.000000}\selectfont s2r\_v2+pix3d}%
\end{pgfscope}%
\begin{pgfscope}%
\definecolor{textcolor}{rgb}{0.000000,0.000000,0.000000}%
\pgfsetstrokecolor{textcolor}%
\pgfsetfillcolor{textcolor}%
\pgftext[x=3.401056in,y=0.234413in,,top]{\color{textcolor}\sffamily\fontsize{10.000000}{12.000000}\bfseries\selectfont Dataset}%
\end{pgfscope}%
\begin{pgfscope}%
\pgfsetbuttcap%
\pgfsetroundjoin%
\definecolor{currentfill}{rgb}{0.000000,0.000000,0.000000}%
\pgfsetfillcolor{currentfill}%
\pgfsetlinewidth{0.803000pt}%
\definecolor{currentstroke}{rgb}{0.000000,0.000000,0.000000}%
\pgfsetstrokecolor{currentstroke}%
\pgfsetdash{}{0pt}%
\pgfsys@defobject{currentmarker}{\pgfqpoint{-0.048611in}{0.000000in}}{\pgfqpoint{-0.000000in}{0.000000in}}{%
\pgfpathmoveto{\pgfqpoint{-0.000000in}{0.000000in}}%
\pgfpathlineto{\pgfqpoint{-0.048611in}{0.000000in}}%
\pgfusepath{stroke,fill}%
}%
\begin{pgfscope}%
\pgfsys@transformshift{0.608070in}{1.163562in}%
\pgfsys@useobject{currentmarker}{}%
\end{pgfscope}%
\end{pgfscope}%
\begin{pgfscope}%
\definecolor{textcolor}{rgb}{0.000000,0.000000,0.000000}%
\pgfsetstrokecolor{textcolor}%
\pgfsetfillcolor{textcolor}%
\pgftext[x=0.289968in, y=1.110800in, left, base]{\color{textcolor}\sffamily\fontsize{10.000000}{12.000000}\selectfont 0.0}%
\end{pgfscope}%
\begin{pgfscope}%
\pgfsetbuttcap%
\pgfsetroundjoin%
\definecolor{currentfill}{rgb}{0.000000,0.000000,0.000000}%
\pgfsetfillcolor{currentfill}%
\pgfsetlinewidth{0.803000pt}%
\definecolor{currentstroke}{rgb}{0.000000,0.000000,0.000000}%
\pgfsetstrokecolor{currentstroke}%
\pgfsetdash{}{0pt}%
\pgfsys@defobject{currentmarker}{\pgfqpoint{-0.048611in}{0.000000in}}{\pgfqpoint{-0.000000in}{0.000000in}}{%
\pgfpathmoveto{\pgfqpoint{-0.000000in}{0.000000in}}%
\pgfpathlineto{\pgfqpoint{-0.048611in}{0.000000in}}%
\pgfusepath{stroke,fill}%
}%
\begin{pgfscope}%
\pgfsys@transformshift{0.608070in}{1.700632in}%
\pgfsys@useobject{currentmarker}{}%
\end{pgfscope}%
\end{pgfscope}%
\begin{pgfscope}%
\definecolor{textcolor}{rgb}{0.000000,0.000000,0.000000}%
\pgfsetstrokecolor{textcolor}%
\pgfsetfillcolor{textcolor}%
\pgftext[x=0.289968in, y=1.647871in, left, base]{\color{textcolor}\sffamily\fontsize{10.000000}{12.000000}\selectfont 0.1}%
\end{pgfscope}%
\begin{pgfscope}%
\pgfsetbuttcap%
\pgfsetroundjoin%
\definecolor{currentfill}{rgb}{0.000000,0.000000,0.000000}%
\pgfsetfillcolor{currentfill}%
\pgfsetlinewidth{0.803000pt}%
\definecolor{currentstroke}{rgb}{0.000000,0.000000,0.000000}%
\pgfsetstrokecolor{currentstroke}%
\pgfsetdash{}{0pt}%
\pgfsys@defobject{currentmarker}{\pgfqpoint{-0.048611in}{0.000000in}}{\pgfqpoint{-0.000000in}{0.000000in}}{%
\pgfpathmoveto{\pgfqpoint{-0.000000in}{0.000000in}}%
\pgfpathlineto{\pgfqpoint{-0.048611in}{0.000000in}}%
\pgfusepath{stroke,fill}%
}%
\begin{pgfscope}%
\pgfsys@transformshift{0.608070in}{2.237703in}%
\pgfsys@useobject{currentmarker}{}%
\end{pgfscope}%
\end{pgfscope}%
\begin{pgfscope}%
\definecolor{textcolor}{rgb}{0.000000,0.000000,0.000000}%
\pgfsetstrokecolor{textcolor}%
\pgfsetfillcolor{textcolor}%
\pgftext[x=0.289968in, y=2.184942in, left, base]{\color{textcolor}\sffamily\fontsize{10.000000}{12.000000}\selectfont 0.2}%
\end{pgfscope}%
\begin{pgfscope}%
\pgfsetbuttcap%
\pgfsetroundjoin%
\definecolor{currentfill}{rgb}{0.000000,0.000000,0.000000}%
\pgfsetfillcolor{currentfill}%
\pgfsetlinewidth{0.803000pt}%
\definecolor{currentstroke}{rgb}{0.000000,0.000000,0.000000}%
\pgfsetstrokecolor{currentstroke}%
\pgfsetdash{}{0pt}%
\pgfsys@defobject{currentmarker}{\pgfqpoint{-0.048611in}{0.000000in}}{\pgfqpoint{-0.000000in}{0.000000in}}{%
\pgfpathmoveto{\pgfqpoint{-0.000000in}{0.000000in}}%
\pgfpathlineto{\pgfqpoint{-0.048611in}{0.000000in}}%
\pgfusepath{stroke,fill}%
}%
\begin{pgfscope}%
\pgfsys@transformshift{0.608070in}{2.774774in}%
\pgfsys@useobject{currentmarker}{}%
\end{pgfscope}%
\end{pgfscope}%
\begin{pgfscope}%
\definecolor{textcolor}{rgb}{0.000000,0.000000,0.000000}%
\pgfsetstrokecolor{textcolor}%
\pgfsetfillcolor{textcolor}%
\pgftext[x=0.289968in, y=2.722013in, left, base]{\color{textcolor}\sffamily\fontsize{10.000000}{12.000000}\selectfont 0.3}%
\end{pgfscope}%
\begin{pgfscope}%
\pgfsetbuttcap%
\pgfsetroundjoin%
\definecolor{currentfill}{rgb}{0.000000,0.000000,0.000000}%
\pgfsetfillcolor{currentfill}%
\pgfsetlinewidth{0.803000pt}%
\definecolor{currentstroke}{rgb}{0.000000,0.000000,0.000000}%
\pgfsetstrokecolor{currentstroke}%
\pgfsetdash{}{0pt}%
\pgfsys@defobject{currentmarker}{\pgfqpoint{-0.048611in}{0.000000in}}{\pgfqpoint{-0.000000in}{0.000000in}}{%
\pgfpathmoveto{\pgfqpoint{-0.000000in}{0.000000in}}%
\pgfpathlineto{\pgfqpoint{-0.048611in}{0.000000in}}%
\pgfusepath{stroke,fill}%
}%
\begin{pgfscope}%
\pgfsys@transformshift{0.608070in}{3.311845in}%
\pgfsys@useobject{currentmarker}{}%
\end{pgfscope}%
\end{pgfscope}%
\begin{pgfscope}%
\definecolor{textcolor}{rgb}{0.000000,0.000000,0.000000}%
\pgfsetstrokecolor{textcolor}%
\pgfsetfillcolor{textcolor}%
\pgftext[x=0.289968in, y=3.259084in, left, base]{\color{textcolor}\sffamily\fontsize{10.000000}{12.000000}\selectfont 0.4}%
\end{pgfscope}%
\begin{pgfscope}%
\pgfsetbuttcap%
\pgfsetroundjoin%
\definecolor{currentfill}{rgb}{0.000000,0.000000,0.000000}%
\pgfsetfillcolor{currentfill}%
\pgfsetlinewidth{0.803000pt}%
\definecolor{currentstroke}{rgb}{0.000000,0.000000,0.000000}%
\pgfsetstrokecolor{currentstroke}%
\pgfsetdash{}{0pt}%
\pgfsys@defobject{currentmarker}{\pgfqpoint{-0.048611in}{0.000000in}}{\pgfqpoint{-0.000000in}{0.000000in}}{%
\pgfpathmoveto{\pgfqpoint{-0.000000in}{0.000000in}}%
\pgfpathlineto{\pgfqpoint{-0.048611in}{0.000000in}}%
\pgfusepath{stroke,fill}%
}%
\begin{pgfscope}%
\pgfsys@transformshift{0.608070in}{3.848916in}%
\pgfsys@useobject{currentmarker}{}%
\end{pgfscope}%
\end{pgfscope}%
\begin{pgfscope}%
\definecolor{textcolor}{rgb}{0.000000,0.000000,0.000000}%
\pgfsetstrokecolor{textcolor}%
\pgfsetfillcolor{textcolor}%
\pgftext[x=0.289968in, y=3.796155in, left, base]{\color{textcolor}\sffamily\fontsize{10.000000}{12.000000}\selectfont 0.5}%
\end{pgfscope}%
\begin{pgfscope}%
\pgfsetbuttcap%
\pgfsetroundjoin%
\definecolor{currentfill}{rgb}{0.000000,0.000000,0.000000}%
\pgfsetfillcolor{currentfill}%
\pgfsetlinewidth{0.803000pt}%
\definecolor{currentstroke}{rgb}{0.000000,0.000000,0.000000}%
\pgfsetstrokecolor{currentstroke}%
\pgfsetdash{}{0pt}%
\pgfsys@defobject{currentmarker}{\pgfqpoint{-0.048611in}{0.000000in}}{\pgfqpoint{-0.000000in}{0.000000in}}{%
\pgfpathmoveto{\pgfqpoint{-0.000000in}{0.000000in}}%
\pgfpathlineto{\pgfqpoint{-0.048611in}{0.000000in}}%
\pgfusepath{stroke,fill}%
}%
\begin{pgfscope}%
\pgfsys@transformshift{0.608070in}{4.385987in}%
\pgfsys@useobject{currentmarker}{}%
\end{pgfscope}%
\end{pgfscope}%
\begin{pgfscope}%
\definecolor{textcolor}{rgb}{0.000000,0.000000,0.000000}%
\pgfsetstrokecolor{textcolor}%
\pgfsetfillcolor{textcolor}%
\pgftext[x=0.289968in, y=4.333225in, left, base]{\color{textcolor}\sffamily\fontsize{10.000000}{12.000000}\selectfont 0.6}%
\end{pgfscope}%
\begin{pgfscope}%
\definecolor{textcolor}{rgb}{0.000000,0.000000,0.000000}%
\pgfsetstrokecolor{textcolor}%
\pgfsetfillcolor{textcolor}%
\pgftext[x=0.234413in,y=2.774774in,,bottom,rotate=90.000000]{\color{textcolor}\sffamily\fontsize{10.000000}{12.000000}\bfseries\selectfont F1 Score}%
\end{pgfscope}%
\begin{pgfscope}%
\pgfsetrectcap%
\pgfsetmiterjoin%
\pgfsetlinewidth{0.803000pt}%
\definecolor{currentstroke}{rgb}{0.000000,0.000000,0.000000}%
\pgfsetstrokecolor{currentstroke}%
\pgfsetdash{}{0pt}%
\pgfpathmoveto{\pgfqpoint{0.608070in}{1.163562in}}%
\pgfpathlineto{\pgfqpoint{0.608070in}{4.385987in}}%
\pgfusepath{stroke}%
\end{pgfscope}%
\begin{pgfscope}%
\pgfsetrectcap%
\pgfsetmiterjoin%
\pgfsetlinewidth{0.803000pt}%
\definecolor{currentstroke}{rgb}{0.000000,0.000000,0.000000}%
\pgfsetstrokecolor{currentstroke}%
\pgfsetdash{}{0pt}%
\pgfpathmoveto{\pgfqpoint{6.194042in}{1.163562in}}%
\pgfpathlineto{\pgfqpoint{6.194042in}{4.385987in}}%
\pgfusepath{stroke}%
\end{pgfscope}%
\begin{pgfscope}%
\pgfsetrectcap%
\pgfsetmiterjoin%
\pgfsetlinewidth{0.803000pt}%
\definecolor{currentstroke}{rgb}{0.000000,0.000000,0.000000}%
\pgfsetstrokecolor{currentstroke}%
\pgfsetdash{}{0pt}%
\pgfpathmoveto{\pgfqpoint{0.608070in}{1.163562in}}%
\pgfpathlineto{\pgfqpoint{6.194042in}{1.163562in}}%
\pgfusepath{stroke}%
\end{pgfscope}%
\begin{pgfscope}%
\pgfsetrectcap%
\pgfsetmiterjoin%
\pgfsetlinewidth{0.803000pt}%
\definecolor{currentstroke}{rgb}{0.000000,0.000000,0.000000}%
\pgfsetstrokecolor{currentstroke}%
\pgfsetdash{}{0pt}%
\pgfpathmoveto{\pgfqpoint{0.608070in}{4.385987in}}%
\pgfpathlineto{\pgfqpoint{6.194042in}{4.385987in}}%
\pgfusepath{stroke}%
\end{pgfscope}%
\begin{pgfscope}%
\definecolor{textcolor}{rgb}{0.000000,0.000000,0.000000}%
\pgfsetstrokecolor{textcolor}%
\pgfsetfillcolor{textcolor}%
\pgftext[x=1.095158in,y=3.579619in,,bottom]{\color{textcolor}\sffamily\fontsize{9.000000}{10.800000}\selectfont 0.4421}%
\end{pgfscope}%
\begin{pgfscope}%
\definecolor{textcolor}{rgb}{0.000000,0.000000,0.000000}%
\pgfsetstrokecolor{textcolor}%
\pgfsetfillcolor{textcolor}%
\pgftext[x=2.131517in,y=1.998482in,,bottom]{\color{textcolor}\sffamily\fontsize{9.000000}{10.800000}\selectfont 0.1477}%
\end{pgfscope}%
\begin{pgfscope}%
\definecolor{textcolor}{rgb}{0.000000,0.000000,0.000000}%
\pgfsetstrokecolor{textcolor}%
\pgfsetfillcolor{textcolor}%
\pgftext[x=3.167875in,y=2.368470in,,bottom]{\color{textcolor}\sffamily\fontsize{9.000000}{10.800000}\selectfont 0.21659}%
\end{pgfscope}%
\begin{pgfscope}%
\definecolor{textcolor}{rgb}{0.000000,0.000000,0.000000}%
\pgfsetstrokecolor{textcolor}%
\pgfsetfillcolor{textcolor}%
\pgftext[x=4.204234in,y=3.448573in,,bottom]{\color{textcolor}\sffamily\fontsize{9.000000}{10.800000}\selectfont 0.4177}%
\end{pgfscope}%
\begin{pgfscope}%
\definecolor{textcolor}{rgb}{0.000000,0.000000,0.000000}%
\pgfsetstrokecolor{textcolor}%
\pgfsetfillcolor{textcolor}%
\pgftext[x=5.240592in,y=3.388421in,,bottom]{\color{textcolor}\sffamily\fontsize{9.000000}{10.800000}\selectfont 0.4065}%
\end{pgfscope}%
\begin{pgfscope}%
\definecolor{textcolor}{rgb}{0.000000,0.000000,0.000000}%
\pgfsetstrokecolor{textcolor}%
\pgfsetfillcolor{textcolor}%
\pgftext[x=1.561520in,y=3.504429in,,bottom]{\color{textcolor}\sffamily\fontsize{9.000000}{10.800000}\selectfont 0.4281}%
\end{pgfscope}%
\begin{pgfscope}%
\definecolor{textcolor}{rgb}{0.000000,0.000000,0.000000}%
\pgfsetstrokecolor{textcolor}%
\pgfsetfillcolor{textcolor}%
\pgftext[x=2.597878in,y=1.763782in,,bottom]{\color{textcolor}\sffamily\fontsize{9.000000}{10.800000}\selectfont 0.104}%
\end{pgfscope}%
\begin{pgfscope}%
\definecolor{textcolor}{rgb}{0.000000,0.000000,0.000000}%
\pgfsetstrokecolor{textcolor}%
\pgfsetfillcolor{textcolor}%
\pgftext[x=3.634237in,y=2.639208in,,bottom]{\color{textcolor}\sffamily\fontsize{9.000000}{10.800000}\selectfont 0.267}%
\end{pgfscope}%
\begin{pgfscope}%
\definecolor{textcolor}{rgb}{0.000000,0.000000,0.000000}%
\pgfsetstrokecolor{textcolor}%
\pgfsetfillcolor{textcolor}%
\pgftext[x=4.670595in,y=3.436758in,,bottom]{\color{textcolor}\sffamily\fontsize{9.000000}{10.800000}\selectfont 0.4155}%
\end{pgfscope}%
\begin{pgfscope}%
\definecolor{textcolor}{rgb}{0.000000,0.000000,0.000000}%
\pgfsetstrokecolor{textcolor}%
\pgfsetfillcolor{textcolor}%
\pgftext[x=5.706954in,y=3.520004in,,bottom]{\color{textcolor}\sffamily\fontsize{9.000000}{10.800000}\selectfont 0.431}%
\end{pgfscope}%
\begin{pgfscope}%
\definecolor{textcolor}{rgb}{0.000000,0.000000,0.000000}%
\pgfsetstrokecolor{textcolor}%
\pgfsetfillcolor{textcolor}%
\pgftext[x=3.401056in,y=4.469320in,,base]{\color{textcolor}\sffamily\fontsize{12.000000}{14.400000}\selectfont Baselines trained on S2R:3DFREE and fine-tuned with Pix3D}%
\end{pgfscope}%
\begin{pgfscope}%
\pgfsetbuttcap%
\pgfsetmiterjoin%
\definecolor{currentfill}{rgb}{1.000000,1.000000,1.000000}%
\pgfsetfillcolor{currentfill}%
\pgfsetfillopacity{0.800000}%
\pgfsetlinewidth{1.003750pt}%
\definecolor{currentstroke}{rgb}{0.800000,0.800000,0.800000}%
\pgfsetstrokecolor{currentstroke}%
\pgfsetstrokeopacity{0.800000}%
\pgfsetdash{}{0pt}%
\pgfpathmoveto{\pgfqpoint{4.882654in}{3.867161in}}%
\pgfpathlineto{\pgfqpoint{6.096820in}{3.867161in}}%
\pgfpathquadraticcurveto{\pgfqpoint{6.124598in}{3.867161in}}{\pgfqpoint{6.124598in}{3.894939in}}%
\pgfpathlineto{\pgfqpoint{6.124598in}{4.288765in}}%
\pgfpathquadraticcurveto{\pgfqpoint{6.124598in}{4.316543in}}{\pgfqpoint{6.096820in}{4.316543in}}%
\pgfpathlineto{\pgfqpoint{4.882654in}{4.316543in}}%
\pgfpathquadraticcurveto{\pgfqpoint{4.854877in}{4.316543in}}{\pgfqpoint{4.854877in}{4.288765in}}%
\pgfpathlineto{\pgfqpoint{4.854877in}{3.894939in}}%
\pgfpathquadraticcurveto{\pgfqpoint{4.854877in}{3.867161in}}{\pgfqpoint{4.882654in}{3.867161in}}%
\pgfpathclose%
\pgfusepath{stroke,fill}%
\end{pgfscope}%
\begin{pgfscope}%
\pgfsetbuttcap%
\pgfsetmiterjoin%
\definecolor{currentfill}{rgb}{0.121569,0.466667,0.705882}%
\pgfsetfillcolor{currentfill}%
\pgfsetlinewidth{0.000000pt}%
\definecolor{currentstroke}{rgb}{0.000000,0.000000,0.000000}%
\pgfsetstrokecolor{currentstroke}%
\pgfsetstrokeopacity{0.000000}%
\pgfsetdash{}{0pt}%
\pgfpathmoveto{\pgfqpoint{4.910432in}{4.155464in}}%
\pgfpathlineto{\pgfqpoint{5.188210in}{4.155464in}}%
\pgfpathlineto{\pgfqpoint{5.188210in}{4.252686in}}%
\pgfpathlineto{\pgfqpoint{4.910432in}{4.252686in}}%
\pgfpathclose%
\pgfusepath{fill}%
\end{pgfscope}%
\begin{pgfscope}%
\definecolor{textcolor}{rgb}{0.000000,0.000000,0.000000}%
\pgfsetstrokecolor{textcolor}%
\pgfsetfillcolor{textcolor}%
\pgftext[x=5.299321in,y=4.155464in,left,base]{\color{textcolor}\sffamily\fontsize{10.000000}{12.000000}\selectfont Pix2Vox++}%
\end{pgfscope}%
\begin{pgfscope}%
\pgfsetbuttcap%
\pgfsetmiterjoin%
\definecolor{currentfill}{rgb}{1.000000,0.498039,0.054902}%
\pgfsetfillcolor{currentfill}%
\pgfsetlinewidth{0.000000pt}%
\definecolor{currentstroke}{rgb}{0.000000,0.000000,0.000000}%
\pgfsetstrokecolor{currentstroke}%
\pgfsetstrokeopacity{0.000000}%
\pgfsetdash{}{0pt}%
\pgfpathmoveto{\pgfqpoint{4.910432in}{3.951607in}}%
\pgfpathlineto{\pgfqpoint{5.188210in}{3.951607in}}%
\pgfpathlineto{\pgfqpoint{5.188210in}{4.048829in}}%
\pgfpathlineto{\pgfqpoint{4.910432in}{4.048829in}}%
\pgfpathclose%
\pgfusepath{fill}%
\end{pgfscope}%
\begin{pgfscope}%
\definecolor{textcolor}{rgb}{0.000000,0.000000,0.000000}%
\pgfsetstrokecolor{textcolor}%
\pgfsetfillcolor{textcolor}%
\pgftext[x=5.299321in,y=3.951607in,left,base]{\color{textcolor}\sffamily\fontsize{10.000000}{12.000000}\selectfont Pix2Vox}%
\end{pgfscope}%
\end{pgfpicture}%
\makeatother%
\endgroup%
}
    \caption{Bar plot for the \gls{f1} for baseline models(Pix2Vox++ and Pix2Vox) trained on synthetic and fine-tuned with real dataset.
    We see that even after fine-tuning both the models do not perform as good as models trained on only real dataset.}
    \label{fig:finetuning_dice1}
\end{figure}

\begin{figure}
    \centering
    \resizebox{0.75\textwidth}{!}{\input{/Users/apple/OVGU/Thesis/code/3dReconstruction/report/images/evaluation/performance/finetune_dice_barplot_pix2voxpp.pgf}}
    \caption{Bar plot for the \gls{f1} for baseline pix2vox++ trained on (\gls{s2rv1}, \gls{s2rv2}) and fine-tuned with pix3d.
    The categories are listed along with the number of images.
    The performance of \texbf{pix2vox++} mixed with both the synthetic dataset is less than model trained on only pix3d, for majority of the categories.}
    \label{fig:finetuning_dice2}
\end{figure}

\begin{figure}
    \centering
    \resizebox{0.75\textwidth}{!}{\input{/Users/apple/OVGU/Thesis/code/3dReconstruction/report/images/evaluation/performance/finetune_dice_barplot_pix2vox.pgf}}
    \caption{Bar plot for the \gls{f1} for baseline \texbf{pix2vox} trained on (\gls{s2rv1}, \gls{s2rv2}) and fine-tuned with pix3d.
    The categories are listed along with the number of images.
    The performance of pix2vox mixed with both the synthetic dataset is less than model trained on only pix3d, for majority the categories.}
    \label{fig:finetuning_dice3}
\end{figure}

\subsection{Mixed training}\label{subsec:mixed-training-dice}

Similar to the experiments conducted in \autoref{sec:mixed-training}, we also note the \gls{f1} for training executed with different mixing ratios per mini-batch.
The results in \autoref{fig:mixed_dice1} are consistent with previous experiment(\autoref{fig:mixed1}).
The performance increases slightly with an increase in the ratio of real data in mini-batch and decreases performance gradually until it is 100\% real dataset.
\autoref{fig:mixed_dice2} and \autoref{fig:mixed_dice3} are \gls{f1} per category when trained with a ratio of 50\% of real data per mini-batch, on pix2vox++ and pix2vox, respectively.
%\begin{figure}
%    \centering
%    \resizebox{0.49\linewidth}{!}{\input{/Users/apple/OVGU/Thesis/code/3dReconstruction/report/images/evaluation/performance/mixed_dice_linegraph1.pgf}}
%    \resizebox{0.49\linewidth}{!}{\input{/Users/apple/OVGU/Thesis/code/3dReconstruction/report/images/evaluation/performance/mixed_dice_linegraph2.pgf}}
%    \caption{Line plot for the \gls{f1}  for baselines trained on different ratios of synthetic and real dataset.
%        (Left)Mixed training on Pix2Vox++, (right)Mixed training on Pix2Vox. In both cases we see a slight increase in \gls{f1} with addition of real data, and a gradual decrease till it reaches 100\% real data}
%    \label{fig:mixed_dice1}
%\end{figure}

\begin{figure}
    \centering
    \resizebox{0.49\linewidth}{0.45\linewidth}{\input{/Users/apple/OVGU/Thesis/code/3dReconstruction/report/images/evaluation/performance/mixed_dice_barplot1.pgf}}
    \resizebox{0.49\linewidth}{0.45\linewidth}{%% Creator: Matplotlib, PGF backend
%%
%% To include the figure in your LaTeX document, write
%%   \input{<filename>.pgf}
%%
%% Make sure the required packages are loaded in your preamble
%%   \usepackage{pgf}
%%
%% Figures using additional raster images can only be included by \input if
%% they are in the same directory as the main LaTeX file. For loading figures
%% from other directories you can use the `import` package
%%   \usepackage{import}
%%
%% and then include the figures with
%%   \import{<path to file>}{<filename>.pgf}
%%
%% Matplotlib used the following preamble
%%   \usepackage{fontspec}
%%   \setmainfont{DejaVuSerif.ttf}[Path=\detokenize{/Users/apple/opt/anaconda3/envs/kaolin/lib/python3.7/site-packages/matplotlib/mpl-data/fonts/ttf/}]
%%   \setsansfont{DejaVuSans.ttf}[Path=\detokenize{/Users/apple/opt/anaconda3/envs/kaolin/lib/python3.7/site-packages/matplotlib/mpl-data/fonts/ttf/}]
%%   \setmonofont{DejaVuSansMono.ttf}[Path=\detokenize{/Users/apple/opt/anaconda3/envs/kaolin/lib/python3.7/site-packages/matplotlib/mpl-data/fonts/ttf/}]
%%
\begingroup%
\makeatletter%
\begin{pgfpicture}%
\pgfpathrectangle{\pgfpointorigin}{\pgfqpoint{6.293658in}{4.697056in}}%
\pgfusepath{use as bounding box, clip}%
\begin{pgfscope}%
\pgfsetbuttcap%
\pgfsetmiterjoin%
\definecolor{currentfill}{rgb}{1.000000,1.000000,1.000000}%
\pgfsetfillcolor{currentfill}%
\pgfsetlinewidth{0.000000pt}%
\definecolor{currentstroke}{rgb}{1.000000,1.000000,1.000000}%
\pgfsetstrokecolor{currentstroke}%
\pgfsetdash{}{0pt}%
\pgfpathmoveto{\pgfqpoint{0.000000in}{0.000000in}}%
\pgfpathlineto{\pgfqpoint{6.293658in}{0.000000in}}%
\pgfpathlineto{\pgfqpoint{6.293658in}{4.697056in}}%
\pgfpathlineto{\pgfqpoint{0.000000in}{4.697056in}}%
\pgfpathclose%
\pgfusepath{fill}%
\end{pgfscope}%
\begin{pgfscope}%
\pgfsetbuttcap%
\pgfsetmiterjoin%
\definecolor{currentfill}{rgb}{1.000000,1.000000,1.000000}%
\pgfsetfillcolor{currentfill}%
\pgfsetlinewidth{0.000000pt}%
\definecolor{currentstroke}{rgb}{0.000000,0.000000,0.000000}%
\pgfsetstrokecolor{currentstroke}%
\pgfsetstrokeopacity{0.000000}%
\pgfsetdash{}{0pt}%
\pgfpathmoveto{\pgfqpoint{0.696435in}{0.700520in}}%
\pgfpathlineto{\pgfqpoint{6.193658in}{0.700520in}}%
\pgfpathlineto{\pgfqpoint{6.193658in}{4.387095in}}%
\pgfpathlineto{\pgfqpoint{0.696435in}{4.387095in}}%
\pgfpathclose%
\pgfusepath{fill}%
\end{pgfscope}%
\begin{pgfscope}%
\pgfpathrectangle{\pgfqpoint{0.696435in}{0.700520in}}{\pgfqpoint{5.497222in}{3.686575in}}%
\pgfusepath{clip}%
\pgfsetbuttcap%
\pgfsetmiterjoin%
\definecolor{currentfill}{rgb}{0.121569,0.466667,0.705882}%
\pgfsetfillcolor{currentfill}%
\pgfsetlinewidth{0.000000pt}%
\definecolor{currentstroke}{rgb}{0.000000,0.000000,0.000000}%
\pgfsetstrokecolor{currentstroke}%
\pgfsetstrokeopacity{0.000000}%
\pgfsetdash{}{0pt}%
\pgfpathmoveto{\pgfqpoint{0.946309in}{-7.901488in}}%
\pgfpathlineto{\pgfqpoint{1.405261in}{-7.901488in}}%
\pgfpathlineto{\pgfqpoint{1.405261in}{3.099251in}}%
\pgfpathlineto{\pgfqpoint{0.946309in}{3.099251in}}%
\pgfpathclose%
\pgfusepath{fill}%
\end{pgfscope}%
\begin{pgfscope}%
\pgfpathrectangle{\pgfqpoint{0.696435in}{0.700520in}}{\pgfqpoint{5.497222in}{3.686575in}}%
\pgfusepath{clip}%
\pgfsetbuttcap%
\pgfsetmiterjoin%
\definecolor{currentfill}{rgb}{0.121569,0.466667,0.705882}%
\pgfsetfillcolor{currentfill}%
\pgfsetlinewidth{0.000000pt}%
\definecolor{currentstroke}{rgb}{0.000000,0.000000,0.000000}%
\pgfsetstrokecolor{currentstroke}%
\pgfsetstrokeopacity{0.000000}%
\pgfsetdash{}{0pt}%
\pgfpathmoveto{\pgfqpoint{1.966202in}{-7.901488in}}%
\pgfpathlineto{\pgfqpoint{2.425154in}{-7.901488in}}%
\pgfpathlineto{\pgfqpoint{2.425154in}{2.981281in}}%
\pgfpathlineto{\pgfqpoint{1.966202in}{2.981281in}}%
\pgfpathclose%
\pgfusepath{fill}%
\end{pgfscope}%
\begin{pgfscope}%
\pgfpathrectangle{\pgfqpoint{0.696435in}{0.700520in}}{\pgfqpoint{5.497222in}{3.686575in}}%
\pgfusepath{clip}%
\pgfsetbuttcap%
\pgfsetmiterjoin%
\definecolor{currentfill}{rgb}{0.121569,0.466667,0.705882}%
\pgfsetfillcolor{currentfill}%
\pgfsetlinewidth{0.000000pt}%
\definecolor{currentstroke}{rgb}{0.000000,0.000000,0.000000}%
\pgfsetstrokecolor{currentstroke}%
\pgfsetstrokeopacity{0.000000}%
\pgfsetdash{}{0pt}%
\pgfpathmoveto{\pgfqpoint{2.986095in}{-7.901488in}}%
\pgfpathlineto{\pgfqpoint{3.445047in}{-7.901488in}}%
\pgfpathlineto{\pgfqpoint{3.445047in}{3.155779in}}%
\pgfpathlineto{\pgfqpoint{2.986095in}{3.155779in}}%
\pgfpathclose%
\pgfusepath{fill}%
\end{pgfscope}%
\begin{pgfscope}%
\pgfpathrectangle{\pgfqpoint{0.696435in}{0.700520in}}{\pgfqpoint{5.497222in}{3.686575in}}%
\pgfusepath{clip}%
\pgfsetbuttcap%
\pgfsetmiterjoin%
\definecolor{currentfill}{rgb}{0.121569,0.466667,0.705882}%
\pgfsetfillcolor{currentfill}%
\pgfsetlinewidth{0.000000pt}%
\definecolor{currentstroke}{rgb}{0.000000,0.000000,0.000000}%
\pgfsetstrokecolor{currentstroke}%
\pgfsetstrokeopacity{0.000000}%
\pgfsetdash{}{0pt}%
\pgfpathmoveto{\pgfqpoint{4.005988in}{-7.901488in}}%
\pgfpathlineto{\pgfqpoint{4.464939in}{-7.901488in}}%
\pgfpathlineto{\pgfqpoint{4.464939in}{2.890346in}}%
\pgfpathlineto{\pgfqpoint{4.005988in}{2.890346in}}%
\pgfpathclose%
\pgfusepath{fill}%
\end{pgfscope}%
\begin{pgfscope}%
\pgfpathrectangle{\pgfqpoint{0.696435in}{0.700520in}}{\pgfqpoint{5.497222in}{3.686575in}}%
\pgfusepath{clip}%
\pgfsetbuttcap%
\pgfsetmiterjoin%
\definecolor{currentfill}{rgb}{0.121569,0.466667,0.705882}%
\pgfsetfillcolor{currentfill}%
\pgfsetlinewidth{0.000000pt}%
\definecolor{currentstroke}{rgb}{0.000000,0.000000,0.000000}%
\pgfsetstrokecolor{currentstroke}%
\pgfsetstrokeopacity{0.000000}%
\pgfsetdash{}{0pt}%
\pgfpathmoveto{\pgfqpoint{5.025880in}{-7.901488in}}%
\pgfpathlineto{\pgfqpoint{5.484832in}{-7.901488in}}%
\pgfpathlineto{\pgfqpoint{5.484832in}{2.801868in}}%
\pgfpathlineto{\pgfqpoint{5.025880in}{2.801868in}}%
\pgfpathclose%
\pgfusepath{fill}%
\end{pgfscope}%
\begin{pgfscope}%
\pgfpathrectangle{\pgfqpoint{0.696435in}{0.700520in}}{\pgfqpoint{5.497222in}{3.686575in}}%
\pgfusepath{clip}%
\pgfsetbuttcap%
\pgfsetmiterjoin%
\definecolor{currentfill}{rgb}{1.000000,0.498039,0.054902}%
\pgfsetfillcolor{currentfill}%
\pgfsetlinewidth{0.000000pt}%
\definecolor{currentstroke}{rgb}{0.000000,0.000000,0.000000}%
\pgfsetstrokecolor{currentstroke}%
\pgfsetstrokeopacity{0.000000}%
\pgfsetdash{}{0pt}%
\pgfpathmoveto{\pgfqpoint{1.405261in}{-7.901488in}}%
\pgfpathlineto{\pgfqpoint{1.864213in}{-7.901488in}}%
\pgfpathlineto{\pgfqpoint{1.864213in}{2.076841in}}%
\pgfpathlineto{\pgfqpoint{1.405261in}{2.076841in}}%
\pgfpathclose%
\pgfusepath{fill}%
\end{pgfscope}%
\begin{pgfscope}%
\pgfpathrectangle{\pgfqpoint{0.696435in}{0.700520in}}{\pgfqpoint{5.497222in}{3.686575in}}%
\pgfusepath{clip}%
\pgfsetbuttcap%
\pgfsetmiterjoin%
\definecolor{currentfill}{rgb}{1.000000,0.498039,0.054902}%
\pgfsetfillcolor{currentfill}%
\pgfsetlinewidth{0.000000pt}%
\definecolor{currentstroke}{rgb}{0.000000,0.000000,0.000000}%
\pgfsetstrokecolor{currentstroke}%
\pgfsetstrokeopacity{0.000000}%
\pgfsetdash{}{0pt}%
\pgfpathmoveto{\pgfqpoint{2.425154in}{-7.901488in}}%
\pgfpathlineto{\pgfqpoint{2.884105in}{-7.901488in}}%
\pgfpathlineto{\pgfqpoint{2.884105in}{3.190187in}}%
\pgfpathlineto{\pgfqpoint{2.425154in}{3.190187in}}%
\pgfpathclose%
\pgfusepath{fill}%
\end{pgfscope}%
\begin{pgfscope}%
\pgfpathrectangle{\pgfqpoint{0.696435in}{0.700520in}}{\pgfqpoint{5.497222in}{3.686575in}}%
\pgfusepath{clip}%
\pgfsetbuttcap%
\pgfsetmiterjoin%
\definecolor{currentfill}{rgb}{1.000000,0.498039,0.054902}%
\pgfsetfillcolor{currentfill}%
\pgfsetlinewidth{0.000000pt}%
\definecolor{currentstroke}{rgb}{0.000000,0.000000,0.000000}%
\pgfsetstrokecolor{currentstroke}%
\pgfsetstrokeopacity{0.000000}%
\pgfsetdash{}{0pt}%
\pgfpathmoveto{\pgfqpoint{3.445047in}{-7.901488in}}%
\pgfpathlineto{\pgfqpoint{3.903998in}{-7.901488in}}%
\pgfpathlineto{\pgfqpoint{3.903998in}{2.794495in}}%
\pgfpathlineto{\pgfqpoint{3.445047in}{2.794495in}}%
\pgfpathclose%
\pgfusepath{fill}%
\end{pgfscope}%
\begin{pgfscope}%
\pgfpathrectangle{\pgfqpoint{0.696435in}{0.700520in}}{\pgfqpoint{5.497222in}{3.686575in}}%
\pgfusepath{clip}%
\pgfsetbuttcap%
\pgfsetmiterjoin%
\definecolor{currentfill}{rgb}{1.000000,0.498039,0.054902}%
\pgfsetfillcolor{currentfill}%
\pgfsetlinewidth{0.000000pt}%
\definecolor{currentstroke}{rgb}{0.000000,0.000000,0.000000}%
\pgfsetstrokecolor{currentstroke}%
\pgfsetstrokeopacity{0.000000}%
\pgfsetdash{}{0pt}%
\pgfpathmoveto{\pgfqpoint{4.464939in}{-7.901488in}}%
\pgfpathlineto{\pgfqpoint{4.923891in}{-7.901488in}}%
\pgfpathlineto{\pgfqpoint{4.923891in}{2.846107in}}%
\pgfpathlineto{\pgfqpoint{4.464939in}{2.846107in}}%
\pgfpathclose%
\pgfusepath{fill}%
\end{pgfscope}%
\begin{pgfscope}%
\pgfpathrectangle{\pgfqpoint{0.696435in}{0.700520in}}{\pgfqpoint{5.497222in}{3.686575in}}%
\pgfusepath{clip}%
\pgfsetbuttcap%
\pgfsetmiterjoin%
\definecolor{currentfill}{rgb}{1.000000,0.498039,0.054902}%
\pgfsetfillcolor{currentfill}%
\pgfsetlinewidth{0.000000pt}%
\definecolor{currentstroke}{rgb}{0.000000,0.000000,0.000000}%
\pgfsetstrokecolor{currentstroke}%
\pgfsetstrokeopacity{0.000000}%
\pgfsetdash{}{0pt}%
\pgfpathmoveto{\pgfqpoint{5.484832in}{-7.901488in}}%
\pgfpathlineto{\pgfqpoint{5.943784in}{-7.901488in}}%
\pgfpathlineto{\pgfqpoint{5.943784in}{2.887888in}}%
\pgfpathlineto{\pgfqpoint{5.484832in}{2.887888in}}%
\pgfpathclose%
\pgfusepath{fill}%
\end{pgfscope}%
\begin{pgfscope}%
\pgfsetbuttcap%
\pgfsetroundjoin%
\definecolor{currentfill}{rgb}{0.000000,0.000000,0.000000}%
\pgfsetfillcolor{currentfill}%
\pgfsetlinewidth{0.803000pt}%
\definecolor{currentstroke}{rgb}{0.000000,0.000000,0.000000}%
\pgfsetstrokecolor{currentstroke}%
\pgfsetdash{}{0pt}%
\pgfsys@defobject{currentmarker}{\pgfqpoint{0.000000in}{-0.048611in}}{\pgfqpoint{0.000000in}{0.000000in}}{%
\pgfpathmoveto{\pgfqpoint{0.000000in}{0.000000in}}%
\pgfpathlineto{\pgfqpoint{0.000000in}{-0.048611in}}%
\pgfusepath{stroke,fill}%
}%
\begin{pgfscope}%
\pgfsys@transformshift{1.405261in}{0.700520in}%
\pgfsys@useobject{currentmarker}{}%
\end{pgfscope}%
\end{pgfscope}%
\begin{pgfscope}%
\definecolor{textcolor}{rgb}{0.000000,0.000000,0.000000}%
\pgfsetstrokecolor{textcolor}%
\pgfsetfillcolor{textcolor}%
\pgftext[x=1.323212in, y=0.310396in, left, base,rotate=45.000000]{\color{textcolor}\sffamily\fontsize{10.000000}{12.000000}\selectfont 15\%}%
\end{pgfscope}%
\begin{pgfscope}%
\pgfsetbuttcap%
\pgfsetroundjoin%
\definecolor{currentfill}{rgb}{0.000000,0.000000,0.000000}%
\pgfsetfillcolor{currentfill}%
\pgfsetlinewidth{0.803000pt}%
\definecolor{currentstroke}{rgb}{0.000000,0.000000,0.000000}%
\pgfsetstrokecolor{currentstroke}%
\pgfsetdash{}{0pt}%
\pgfsys@defobject{currentmarker}{\pgfqpoint{0.000000in}{-0.048611in}}{\pgfqpoint{0.000000in}{0.000000in}}{%
\pgfpathmoveto{\pgfqpoint{0.000000in}{0.000000in}}%
\pgfpathlineto{\pgfqpoint{0.000000in}{-0.048611in}}%
\pgfusepath{stroke,fill}%
}%
\begin{pgfscope}%
\pgfsys@transformshift{2.425154in}{0.700520in}%
\pgfsys@useobject{currentmarker}{}%
\end{pgfscope}%
\end{pgfscope}%
\begin{pgfscope}%
\definecolor{textcolor}{rgb}{0.000000,0.000000,0.000000}%
\pgfsetstrokecolor{textcolor}%
\pgfsetfillcolor{textcolor}%
\pgftext[x=2.343105in, y=0.310396in, left, base,rotate=45.000000]{\color{textcolor}\sffamily\fontsize{10.000000}{12.000000}\selectfont 25\%}%
\end{pgfscope}%
\begin{pgfscope}%
\pgfsetbuttcap%
\pgfsetroundjoin%
\definecolor{currentfill}{rgb}{0.000000,0.000000,0.000000}%
\pgfsetfillcolor{currentfill}%
\pgfsetlinewidth{0.803000pt}%
\definecolor{currentstroke}{rgb}{0.000000,0.000000,0.000000}%
\pgfsetstrokecolor{currentstroke}%
\pgfsetdash{}{0pt}%
\pgfsys@defobject{currentmarker}{\pgfqpoint{0.000000in}{-0.048611in}}{\pgfqpoint{0.000000in}{0.000000in}}{%
\pgfpathmoveto{\pgfqpoint{0.000000in}{0.000000in}}%
\pgfpathlineto{\pgfqpoint{0.000000in}{-0.048611in}}%
\pgfusepath{stroke,fill}%
}%
\begin{pgfscope}%
\pgfsys@transformshift{3.445047in}{0.700520in}%
\pgfsys@useobject{currentmarker}{}%
\end{pgfscope}%
\end{pgfscope}%
\begin{pgfscope}%
\definecolor{textcolor}{rgb}{0.000000,0.000000,0.000000}%
\pgfsetstrokecolor{textcolor}%
\pgfsetfillcolor{textcolor}%
\pgftext[x=3.362998in, y=0.310396in, left, base,rotate=45.000000]{\color{textcolor}\sffamily\fontsize{10.000000}{12.000000}\selectfont 50\%}%
\end{pgfscope}%
\begin{pgfscope}%
\pgfsetbuttcap%
\pgfsetroundjoin%
\definecolor{currentfill}{rgb}{0.000000,0.000000,0.000000}%
\pgfsetfillcolor{currentfill}%
\pgfsetlinewidth{0.803000pt}%
\definecolor{currentstroke}{rgb}{0.000000,0.000000,0.000000}%
\pgfsetstrokecolor{currentstroke}%
\pgfsetdash{}{0pt}%
\pgfsys@defobject{currentmarker}{\pgfqpoint{0.000000in}{-0.048611in}}{\pgfqpoint{0.000000in}{0.000000in}}{%
\pgfpathmoveto{\pgfqpoint{0.000000in}{0.000000in}}%
\pgfpathlineto{\pgfqpoint{0.000000in}{-0.048611in}}%
\pgfusepath{stroke,fill}%
}%
\begin{pgfscope}%
\pgfsys@transformshift{4.464939in}{0.700520in}%
\pgfsys@useobject{currentmarker}{}%
\end{pgfscope}%
\end{pgfscope}%
\begin{pgfscope}%
\definecolor{textcolor}{rgb}{0.000000,0.000000,0.000000}%
\pgfsetstrokecolor{textcolor}%
\pgfsetfillcolor{textcolor}%
\pgftext[x=4.382891in, y=0.310396in, left, base,rotate=45.000000]{\color{textcolor}\sffamily\fontsize{10.000000}{12.000000}\selectfont 75\%}%
\end{pgfscope}%
\begin{pgfscope}%
\pgfsetbuttcap%
\pgfsetroundjoin%
\definecolor{currentfill}{rgb}{0.000000,0.000000,0.000000}%
\pgfsetfillcolor{currentfill}%
\pgfsetlinewidth{0.803000pt}%
\definecolor{currentstroke}{rgb}{0.000000,0.000000,0.000000}%
\pgfsetstrokecolor{currentstroke}%
\pgfsetdash{}{0pt}%
\pgfsys@defobject{currentmarker}{\pgfqpoint{0.000000in}{-0.048611in}}{\pgfqpoint{0.000000in}{0.000000in}}{%
\pgfpathmoveto{\pgfqpoint{0.000000in}{0.000000in}}%
\pgfpathlineto{\pgfqpoint{0.000000in}{-0.048611in}}%
\pgfusepath{stroke,fill}%
}%
\begin{pgfscope}%
\pgfsys@transformshift{5.484832in}{0.700520in}%
\pgfsys@useobject{currentmarker}{}%
\end{pgfscope}%
\end{pgfscope}%
\begin{pgfscope}%
\definecolor{textcolor}{rgb}{0.000000,0.000000,0.000000}%
\pgfsetstrokecolor{textcolor}%
\pgfsetfillcolor{textcolor}%
\pgftext[x=5.402783in, y=0.310396in, left, base,rotate=45.000000]{\color{textcolor}\sffamily\fontsize{10.000000}{12.000000}\selectfont 90\%}%
\end{pgfscope}%
\begin{pgfscope}%
\definecolor{textcolor}{rgb}{0.000000,0.000000,0.000000}%
\pgfsetstrokecolor{textcolor}%
\pgfsetfillcolor{textcolor}%
\pgftext[x=3.445047in,y=0.234413in,,top]{\color{textcolor}\sffamily\fontsize{10.000000}{12.000000}\bfseries\selectfont Dataset}%
\end{pgfscope}%
\begin{pgfscope}%
\pgfsetbuttcap%
\pgfsetroundjoin%
\definecolor{currentfill}{rgb}{0.000000,0.000000,0.000000}%
\pgfsetfillcolor{currentfill}%
\pgfsetlinewidth{0.803000pt}%
\definecolor{currentstroke}{rgb}{0.000000,0.000000,0.000000}%
\pgfsetstrokecolor{currentstroke}%
\pgfsetdash{}{0pt}%
\pgfsys@defobject{currentmarker}{\pgfqpoint{-0.048611in}{0.000000in}}{\pgfqpoint{-0.000000in}{0.000000in}}{%
\pgfpathmoveto{\pgfqpoint{-0.000000in}{0.000000in}}%
\pgfpathlineto{\pgfqpoint{-0.048611in}{0.000000in}}%
\pgfusepath{stroke,fill}%
}%
\begin{pgfscope}%
\pgfsys@transformshift{0.696435in}{0.946292in}%
\pgfsys@useobject{currentmarker}{}%
\end{pgfscope}%
\end{pgfscope}%
\begin{pgfscope}%
\definecolor{textcolor}{rgb}{0.000000,0.000000,0.000000}%
\pgfsetstrokecolor{textcolor}%
\pgfsetfillcolor{textcolor}%
\pgftext[x=0.289968in, y=0.893530in, left, base]{\color{textcolor}\sffamily\fontsize{10.000000}{12.000000}\selectfont 0.36}%
\end{pgfscope}%
\begin{pgfscope}%
\pgfsetbuttcap%
\pgfsetroundjoin%
\definecolor{currentfill}{rgb}{0.000000,0.000000,0.000000}%
\pgfsetfillcolor{currentfill}%
\pgfsetlinewidth{0.803000pt}%
\definecolor{currentstroke}{rgb}{0.000000,0.000000,0.000000}%
\pgfsetstrokecolor{currentstroke}%
\pgfsetdash{}{0pt}%
\pgfsys@defobject{currentmarker}{\pgfqpoint{-0.048611in}{0.000000in}}{\pgfqpoint{-0.000000in}{0.000000in}}{%
\pgfpathmoveto{\pgfqpoint{-0.000000in}{0.000000in}}%
\pgfpathlineto{\pgfqpoint{-0.048611in}{0.000000in}}%
\pgfusepath{stroke,fill}%
}%
\begin{pgfscope}%
\pgfsys@transformshift{0.696435in}{1.437835in}%
\pgfsys@useobject{currentmarker}{}%
\end{pgfscope}%
\end{pgfscope}%
\begin{pgfscope}%
\definecolor{textcolor}{rgb}{0.000000,0.000000,0.000000}%
\pgfsetstrokecolor{textcolor}%
\pgfsetfillcolor{textcolor}%
\pgftext[x=0.289968in, y=1.385074in, left, base]{\color{textcolor}\sffamily\fontsize{10.000000}{12.000000}\selectfont 0.38}%
\end{pgfscope}%
\begin{pgfscope}%
\pgfsetbuttcap%
\pgfsetroundjoin%
\definecolor{currentfill}{rgb}{0.000000,0.000000,0.000000}%
\pgfsetfillcolor{currentfill}%
\pgfsetlinewidth{0.803000pt}%
\definecolor{currentstroke}{rgb}{0.000000,0.000000,0.000000}%
\pgfsetstrokecolor{currentstroke}%
\pgfsetdash{}{0pt}%
\pgfsys@defobject{currentmarker}{\pgfqpoint{-0.048611in}{0.000000in}}{\pgfqpoint{-0.000000in}{0.000000in}}{%
\pgfpathmoveto{\pgfqpoint{-0.000000in}{0.000000in}}%
\pgfpathlineto{\pgfqpoint{-0.048611in}{0.000000in}}%
\pgfusepath{stroke,fill}%
}%
\begin{pgfscope}%
\pgfsys@transformshift{0.696435in}{1.929378in}%
\pgfsys@useobject{currentmarker}{}%
\end{pgfscope}%
\end{pgfscope}%
\begin{pgfscope}%
\definecolor{textcolor}{rgb}{0.000000,0.000000,0.000000}%
\pgfsetstrokecolor{textcolor}%
\pgfsetfillcolor{textcolor}%
\pgftext[x=0.289968in, y=1.876617in, left, base]{\color{textcolor}\sffamily\fontsize{10.000000}{12.000000}\selectfont 0.40}%
\end{pgfscope}%
\begin{pgfscope}%
\pgfsetbuttcap%
\pgfsetroundjoin%
\definecolor{currentfill}{rgb}{0.000000,0.000000,0.000000}%
\pgfsetfillcolor{currentfill}%
\pgfsetlinewidth{0.803000pt}%
\definecolor{currentstroke}{rgb}{0.000000,0.000000,0.000000}%
\pgfsetstrokecolor{currentstroke}%
\pgfsetdash{}{0pt}%
\pgfsys@defobject{currentmarker}{\pgfqpoint{-0.048611in}{0.000000in}}{\pgfqpoint{-0.000000in}{0.000000in}}{%
\pgfpathmoveto{\pgfqpoint{-0.000000in}{0.000000in}}%
\pgfpathlineto{\pgfqpoint{-0.048611in}{0.000000in}}%
\pgfusepath{stroke,fill}%
}%
\begin{pgfscope}%
\pgfsys@transformshift{0.696435in}{2.420922in}%
\pgfsys@useobject{currentmarker}{}%
\end{pgfscope}%
\end{pgfscope}%
\begin{pgfscope}%
\definecolor{textcolor}{rgb}{0.000000,0.000000,0.000000}%
\pgfsetstrokecolor{textcolor}%
\pgfsetfillcolor{textcolor}%
\pgftext[x=0.289968in, y=2.368160in, left, base]{\color{textcolor}\sffamily\fontsize{10.000000}{12.000000}\selectfont 0.42}%
\end{pgfscope}%
\begin{pgfscope}%
\pgfsetbuttcap%
\pgfsetroundjoin%
\definecolor{currentfill}{rgb}{0.000000,0.000000,0.000000}%
\pgfsetfillcolor{currentfill}%
\pgfsetlinewidth{0.803000pt}%
\definecolor{currentstroke}{rgb}{0.000000,0.000000,0.000000}%
\pgfsetstrokecolor{currentstroke}%
\pgfsetdash{}{0pt}%
\pgfsys@defobject{currentmarker}{\pgfqpoint{-0.048611in}{0.000000in}}{\pgfqpoint{-0.000000in}{0.000000in}}{%
\pgfpathmoveto{\pgfqpoint{-0.000000in}{0.000000in}}%
\pgfpathlineto{\pgfqpoint{-0.048611in}{0.000000in}}%
\pgfusepath{stroke,fill}%
}%
\begin{pgfscope}%
\pgfsys@transformshift{0.696435in}{2.912465in}%
\pgfsys@useobject{currentmarker}{}%
\end{pgfscope}%
\end{pgfscope}%
\begin{pgfscope}%
\definecolor{textcolor}{rgb}{0.000000,0.000000,0.000000}%
\pgfsetstrokecolor{textcolor}%
\pgfsetfillcolor{textcolor}%
\pgftext[x=0.289968in, y=2.859703in, left, base]{\color{textcolor}\sffamily\fontsize{10.000000}{12.000000}\selectfont 0.44}%
\end{pgfscope}%
\begin{pgfscope}%
\pgfsetbuttcap%
\pgfsetroundjoin%
\definecolor{currentfill}{rgb}{0.000000,0.000000,0.000000}%
\pgfsetfillcolor{currentfill}%
\pgfsetlinewidth{0.803000pt}%
\definecolor{currentstroke}{rgb}{0.000000,0.000000,0.000000}%
\pgfsetstrokecolor{currentstroke}%
\pgfsetdash{}{0pt}%
\pgfsys@defobject{currentmarker}{\pgfqpoint{-0.048611in}{0.000000in}}{\pgfqpoint{-0.000000in}{0.000000in}}{%
\pgfpathmoveto{\pgfqpoint{-0.000000in}{0.000000in}}%
\pgfpathlineto{\pgfqpoint{-0.048611in}{0.000000in}}%
\pgfusepath{stroke,fill}%
}%
\begin{pgfscope}%
\pgfsys@transformshift{0.696435in}{3.404008in}%
\pgfsys@useobject{currentmarker}{}%
\end{pgfscope}%
\end{pgfscope}%
\begin{pgfscope}%
\definecolor{textcolor}{rgb}{0.000000,0.000000,0.000000}%
\pgfsetstrokecolor{textcolor}%
\pgfsetfillcolor{textcolor}%
\pgftext[x=0.289968in, y=3.351247in, left, base]{\color{textcolor}\sffamily\fontsize{10.000000}{12.000000}\selectfont 0.46}%
\end{pgfscope}%
\begin{pgfscope}%
\pgfsetbuttcap%
\pgfsetroundjoin%
\definecolor{currentfill}{rgb}{0.000000,0.000000,0.000000}%
\pgfsetfillcolor{currentfill}%
\pgfsetlinewidth{0.803000pt}%
\definecolor{currentstroke}{rgb}{0.000000,0.000000,0.000000}%
\pgfsetstrokecolor{currentstroke}%
\pgfsetdash{}{0pt}%
\pgfsys@defobject{currentmarker}{\pgfqpoint{-0.048611in}{0.000000in}}{\pgfqpoint{-0.000000in}{0.000000in}}{%
\pgfpathmoveto{\pgfqpoint{-0.000000in}{0.000000in}}%
\pgfpathlineto{\pgfqpoint{-0.048611in}{0.000000in}}%
\pgfusepath{stroke,fill}%
}%
\begin{pgfscope}%
\pgfsys@transformshift{0.696435in}{3.895552in}%
\pgfsys@useobject{currentmarker}{}%
\end{pgfscope}%
\end{pgfscope}%
\begin{pgfscope}%
\definecolor{textcolor}{rgb}{0.000000,0.000000,0.000000}%
\pgfsetstrokecolor{textcolor}%
\pgfsetfillcolor{textcolor}%
\pgftext[x=0.289968in, y=3.842790in, left, base]{\color{textcolor}\sffamily\fontsize{10.000000}{12.000000}\selectfont 0.48}%
\end{pgfscope}%
\begin{pgfscope}%
\pgfsetbuttcap%
\pgfsetroundjoin%
\definecolor{currentfill}{rgb}{0.000000,0.000000,0.000000}%
\pgfsetfillcolor{currentfill}%
\pgfsetlinewidth{0.803000pt}%
\definecolor{currentstroke}{rgb}{0.000000,0.000000,0.000000}%
\pgfsetstrokecolor{currentstroke}%
\pgfsetdash{}{0pt}%
\pgfsys@defobject{currentmarker}{\pgfqpoint{-0.048611in}{0.000000in}}{\pgfqpoint{-0.000000in}{0.000000in}}{%
\pgfpathmoveto{\pgfqpoint{-0.000000in}{0.000000in}}%
\pgfpathlineto{\pgfqpoint{-0.048611in}{0.000000in}}%
\pgfusepath{stroke,fill}%
}%
\begin{pgfscope}%
\pgfsys@transformshift{0.696435in}{4.387095in}%
\pgfsys@useobject{currentmarker}{}%
\end{pgfscope}%
\end{pgfscope}%
\begin{pgfscope}%
\definecolor{textcolor}{rgb}{0.000000,0.000000,0.000000}%
\pgfsetstrokecolor{textcolor}%
\pgfsetfillcolor{textcolor}%
\pgftext[x=0.289968in, y=4.334333in, left, base]{\color{textcolor}\sffamily\fontsize{10.000000}{12.000000}\selectfont 0.50}%
\end{pgfscope}%
\begin{pgfscope}%
\definecolor{textcolor}{rgb}{0.000000,0.000000,0.000000}%
\pgfsetstrokecolor{textcolor}%
\pgfsetfillcolor{textcolor}%
\pgftext[x=0.234413in,y=2.543808in,,bottom,rotate=90.000000]{\color{textcolor}\sffamily\fontsize{10.000000}{12.000000}\bfseries\selectfont F1 Score}%
\end{pgfscope}%
\begin{pgfscope}%
\pgfsetrectcap%
\pgfsetmiterjoin%
\pgfsetlinewidth{0.803000pt}%
\definecolor{currentstroke}{rgb}{0.000000,0.000000,0.000000}%
\pgfsetstrokecolor{currentstroke}%
\pgfsetdash{}{0pt}%
\pgfpathmoveto{\pgfqpoint{0.696435in}{0.700520in}}%
\pgfpathlineto{\pgfqpoint{0.696435in}{4.387095in}}%
\pgfusepath{stroke}%
\end{pgfscope}%
\begin{pgfscope}%
\pgfsetrectcap%
\pgfsetmiterjoin%
\pgfsetlinewidth{0.803000pt}%
\definecolor{currentstroke}{rgb}{0.000000,0.000000,0.000000}%
\pgfsetstrokecolor{currentstroke}%
\pgfsetdash{}{0pt}%
\pgfpathmoveto{\pgfqpoint{6.193658in}{0.700520in}}%
\pgfpathlineto{\pgfqpoint{6.193658in}{4.387095in}}%
\pgfusepath{stroke}%
\end{pgfscope}%
\begin{pgfscope}%
\pgfsetrectcap%
\pgfsetmiterjoin%
\pgfsetlinewidth{0.803000pt}%
\definecolor{currentstroke}{rgb}{0.000000,0.000000,0.000000}%
\pgfsetstrokecolor{currentstroke}%
\pgfsetdash{}{0pt}%
\pgfpathmoveto{\pgfqpoint{0.696435in}{0.700520in}}%
\pgfpathlineto{\pgfqpoint{6.193658in}{0.700520in}}%
\pgfusepath{stroke}%
\end{pgfscope}%
\begin{pgfscope}%
\pgfsetrectcap%
\pgfsetmiterjoin%
\pgfsetlinewidth{0.803000pt}%
\definecolor{currentstroke}{rgb}{0.000000,0.000000,0.000000}%
\pgfsetstrokecolor{currentstroke}%
\pgfsetdash{}{0pt}%
\pgfpathmoveto{\pgfqpoint{0.696435in}{4.387095in}}%
\pgfpathlineto{\pgfqpoint{6.193658in}{4.387095in}}%
\pgfusepath{stroke}%
\end{pgfscope}%
\begin{pgfscope}%
\definecolor{textcolor}{rgb}{0.000000,0.000000,0.000000}%
\pgfsetstrokecolor{textcolor}%
\pgfsetfillcolor{textcolor}%
\pgftext[x=1.175785in,y=3.140918in,,bottom]{\color{textcolor}\sffamily\fontsize{9.000000}{10.800000}\selectfont 0.4476}%
\end{pgfscope}%
\begin{pgfscope}%
\definecolor{textcolor}{rgb}{0.000000,0.000000,0.000000}%
\pgfsetstrokecolor{textcolor}%
\pgfsetfillcolor{textcolor}%
\pgftext[x=2.195678in,y=3.022948in,,bottom]{\color{textcolor}\sffamily\fontsize{9.000000}{10.800000}\selectfont 0.4428}%
\end{pgfscope}%
\begin{pgfscope}%
\definecolor{textcolor}{rgb}{0.000000,0.000000,0.000000}%
\pgfsetstrokecolor{textcolor}%
\pgfsetfillcolor{textcolor}%
\pgftext[x=3.215571in,y=3.197446in,,bottom]{\color{textcolor}\sffamily\fontsize{9.000000}{10.800000}\selectfont 0.4499}%
\end{pgfscope}%
\begin{pgfscope}%
\definecolor{textcolor}{rgb}{0.000000,0.000000,0.000000}%
\pgfsetstrokecolor{textcolor}%
\pgfsetfillcolor{textcolor}%
\pgftext[x=4.235463in,y=2.932012in,,bottom]{\color{textcolor}\sffamily\fontsize{9.000000}{10.800000}\selectfont 0.4391}%
\end{pgfscope}%
\begin{pgfscope}%
\definecolor{textcolor}{rgb}{0.000000,0.000000,0.000000}%
\pgfsetstrokecolor{textcolor}%
\pgfsetfillcolor{textcolor}%
\pgftext[x=5.255356in,y=2.843534in,,bottom]{\color{textcolor}\sffamily\fontsize{9.000000}{10.800000}\selectfont 0.4355}%
\end{pgfscope}%
\begin{pgfscope}%
\definecolor{textcolor}{rgb}{0.000000,0.000000,0.000000}%
\pgfsetstrokecolor{textcolor}%
\pgfsetfillcolor{textcolor}%
\pgftext[x=1.634737in,y=2.118508in,,bottom]{\color{textcolor}\sffamily\fontsize{9.000000}{10.800000}\selectfont 0.406}%
\end{pgfscope}%
\begin{pgfscope}%
\definecolor{textcolor}{rgb}{0.000000,0.000000,0.000000}%
\pgfsetstrokecolor{textcolor}%
\pgfsetfillcolor{textcolor}%
\pgftext[x=2.654630in,y=3.231854in,,bottom]{\color{textcolor}\sffamily\fontsize{9.000000}{10.800000}\selectfont 0.4513}%
\end{pgfscope}%
\begin{pgfscope}%
\definecolor{textcolor}{rgb}{0.000000,0.000000,0.000000}%
\pgfsetstrokecolor{textcolor}%
\pgfsetfillcolor{textcolor}%
\pgftext[x=3.674522in,y=2.836161in,,bottom]{\color{textcolor}\sffamily\fontsize{9.000000}{10.800000}\selectfont 0.4352}%
\end{pgfscope}%
\begin{pgfscope}%
\definecolor{textcolor}{rgb}{0.000000,0.000000,0.000000}%
\pgfsetstrokecolor{textcolor}%
\pgfsetfillcolor{textcolor}%
\pgftext[x=4.694415in,y=2.887773in,,bottom]{\color{textcolor}\sffamily\fontsize{9.000000}{10.800000}\selectfont 0.4373}%
\end{pgfscope}%
\begin{pgfscope}%
\definecolor{textcolor}{rgb}{0.000000,0.000000,0.000000}%
\pgfsetstrokecolor{textcolor}%
\pgfsetfillcolor{textcolor}%
\pgftext[x=5.714308in,y=2.929555in,,bottom]{\color{textcolor}\sffamily\fontsize{9.000000}{10.800000}\selectfont 0.439}%
\end{pgfscope}%
\begin{pgfscope}%
\definecolor{textcolor}{rgb}{0.000000,0.000000,0.000000}%
\pgfsetstrokecolor{textcolor}%
\pgfsetfillcolor{textcolor}%
\pgftext[x=3.445047in,y=4.470428in,,base]{\color{textcolor}\sffamily\fontsize{12.000000}{14.400000}\selectfont Mixed training on Pix2Vox using 2 versions of S2R:3DFREE}%
\end{pgfscope}%
\begin{pgfscope}%
\pgfsetbuttcap%
\pgfsetmiterjoin%
\definecolor{currentfill}{rgb}{1.000000,1.000000,1.000000}%
\pgfsetfillcolor{currentfill}%
\pgfsetfillopacity{0.800000}%
\pgfsetlinewidth{1.003750pt}%
\definecolor{currentstroke}{rgb}{0.800000,0.800000,0.800000}%
\pgfsetstrokecolor{currentstroke}%
\pgfsetstrokeopacity{0.800000}%
\pgfsetdash{}{0pt}%
\pgfpathmoveto{\pgfqpoint{4.670342in}{3.868269in}}%
\pgfpathlineto{\pgfqpoint{6.096435in}{3.868269in}}%
\pgfpathquadraticcurveto{\pgfqpoint{6.124213in}{3.868269in}}{\pgfqpoint{6.124213in}{3.896047in}}%
\pgfpathlineto{\pgfqpoint{6.124213in}{4.289873in}}%
\pgfpathquadraticcurveto{\pgfqpoint{6.124213in}{4.317650in}}{\pgfqpoint{6.096435in}{4.317650in}}%
\pgfpathlineto{\pgfqpoint{4.670342in}{4.317650in}}%
\pgfpathquadraticcurveto{\pgfqpoint{4.642564in}{4.317650in}}{\pgfqpoint{4.642564in}{4.289873in}}%
\pgfpathlineto{\pgfqpoint{4.642564in}{3.896047in}}%
\pgfpathquadraticcurveto{\pgfqpoint{4.642564in}{3.868269in}}{\pgfqpoint{4.670342in}{3.868269in}}%
\pgfpathclose%
\pgfusepath{stroke,fill}%
\end{pgfscope}%
\begin{pgfscope}%
\pgfsetbuttcap%
\pgfsetmiterjoin%
\definecolor{currentfill}{rgb}{0.121569,0.466667,0.705882}%
\pgfsetfillcolor{currentfill}%
\pgfsetlinewidth{0.000000pt}%
\definecolor{currentstroke}{rgb}{0.000000,0.000000,0.000000}%
\pgfsetstrokecolor{currentstroke}%
\pgfsetstrokeopacity{0.000000}%
\pgfsetdash{}{0pt}%
\pgfpathmoveto{\pgfqpoint{4.698120in}{4.156572in}}%
\pgfpathlineto{\pgfqpoint{4.975898in}{4.156572in}}%
\pgfpathlineto{\pgfqpoint{4.975898in}{4.253794in}}%
\pgfpathlineto{\pgfqpoint{4.698120in}{4.253794in}}%
\pgfpathclose%
\pgfusepath{fill}%
\end{pgfscope}%
\begin{pgfscope}%
\definecolor{textcolor}{rgb}{0.000000,0.000000,0.000000}%
\pgfsetstrokecolor{textcolor}%
\pgfsetfillcolor{textcolor}%
\pgftext[x=5.087009in,y=4.156572in,left,base]{\color{textcolor}\sffamily\fontsize{10.000000}{12.000000}\selectfont V1 on Pix2Vox}%
\end{pgfscope}%
\begin{pgfscope}%
\pgfsetbuttcap%
\pgfsetmiterjoin%
\definecolor{currentfill}{rgb}{1.000000,0.498039,0.054902}%
\pgfsetfillcolor{currentfill}%
\pgfsetlinewidth{0.000000pt}%
\definecolor{currentstroke}{rgb}{0.000000,0.000000,0.000000}%
\pgfsetstrokecolor{currentstroke}%
\pgfsetstrokeopacity{0.000000}%
\pgfsetdash{}{0pt}%
\pgfpathmoveto{\pgfqpoint{4.698120in}{3.952715in}}%
\pgfpathlineto{\pgfqpoint{4.975898in}{3.952715in}}%
\pgfpathlineto{\pgfqpoint{4.975898in}{4.049937in}}%
\pgfpathlineto{\pgfqpoint{4.698120in}{4.049937in}}%
\pgfpathclose%
\pgfusepath{fill}%
\end{pgfscope}%
\begin{pgfscope}%
\definecolor{textcolor}{rgb}{0.000000,0.000000,0.000000}%
\pgfsetstrokecolor{textcolor}%
\pgfsetfillcolor{textcolor}%
\pgftext[x=5.087009in,y=3.952715in,left,base]{\color{textcolor}\sffamily\fontsize{10.000000}{12.000000}\selectfont V2 on Pix2Vox}%
\end{pgfscope}%
\end{pgfpicture}%
\makeatother%
\endgroup%
}
    \caption{Bar plot for the \gls{f1}  for baselines trained on different ratios of synthetic and real dataset.
        (left)Mixed training on Pix2Vox++, (right)Mixed training on Pix2Vox. In both cases we see a slight increase in \gls{f1} with addition of real data, and a gradual decrease till it reaches 100\% real data}
    \label{fig:mixed_dice1}
\end{figure}

\begin{figure}
    \centering
    \resizebox{0.7\textwidth}{!}{\input{/Users/apple/OVGU/Thesis/code/3dReconstruction/report/images/evaluation/performance/mixed_dice_barplot_pix2voxpp.pgf}}
    \caption{Bar plot for the \gls{f1} for baseline \texbf{pix2vox++} trained on 50\% of mixed dataset(\gls{s2rv1}, \gls{s2rv2}) and with pix3d.
    The categories are listed along with the number of images.
    The performance of pix2vox++ mixed with both the synthetic dataset is more than model trained on only pix3d, for most of the categories.
    }
    \label{fig:mixed_dice2}
\end{figure}

\begin{figure}
    \centering
    \resizebox{0.7\textwidth}{!}{\input{/Users/apple/OVGU/Thesis/code/3dReconstruction/report/images/evaluation/performance/mixed_dice_barplot_pix2vox.pgf}}
    \caption{Bar plot for the \gls{f1} for baseline \texbf{pix2vox} trained on 50\% of mixed dataset(\gls{s2rv1}, \gls{s2rv2}) and with pix3d.
    The categories are listed along with the number of images.
    The performance of pix2vox mixed with both the synthetic dataset is more than model trained on only pix3d, for all the categories.}
    \label{fig:mixed_dice3}
\end{figure}

\subsection{Ablation study on chairs}\label{subsec:ablation-study-on-chairs}

For the ablation study on chairs, we created different datasets with different domain randomization components as in \autoref{subsec:s2r:3dfree-ablation}.
\autoref{fig:ablation_dice1} shows the performance of baseline models with each of the randomization components.
Pix2Vox++ seems to give a \gls{f1} closer to the real dataset when trained on \gls{s2rv2} than Pix2vox.
Overall the performance still does not match the real dataset on its own.

\begin{figure}[ht]
    \centering
    \resizebox{0.7\textwidth}{!}{%% Creator: Matplotlib, PGF backend
%%
%% To include the figure in your LaTeX document, write
%%   \input{<filename>.pgf}
%%
%% Make sure the required packages are loaded in your preamble
%%   \usepackage{pgf}
%%
%% Figures using additional raster images can only be included by \input if
%% they are in the same directory as the main LaTeX file. For loading figures
%% from other directories you can use the `import` package
%%   \usepackage{import}
%%
%% and then include the figures with
%%   \import{<path to file>}{<filename>.pgf}
%%
%% Matplotlib used the following preamble
%%   \usepackage{fontspec}
%%   \setmainfont{DejaVuSerif.ttf}[Path=\detokenize{/Users/apple/opt/anaconda3/envs/kaolin/lib/python3.7/site-packages/matplotlib/mpl-data/fonts/ttf/}]
%%   \setsansfont{DejaVuSans.ttf}[Path=\detokenize{/Users/apple/opt/anaconda3/envs/kaolin/lib/python3.7/site-packages/matplotlib/mpl-data/fonts/ttf/}]
%%   \setmonofont{DejaVuSansMono.ttf}[Path=\detokenize{/Users/apple/opt/anaconda3/envs/kaolin/lib/python3.7/site-packages/matplotlib/mpl-data/fonts/ttf/}]
%%
\begingroup%
\makeatletter%
\begin{pgfpicture}%
\pgfpathrectangle{\pgfpointorigin}{\pgfqpoint{6.294042in}{4.693849in}}%
\pgfusepath{use as bounding box, clip}%
\begin{pgfscope}%
\pgfsetbuttcap%
\pgfsetmiterjoin%
\definecolor{currentfill}{rgb}{1.000000,1.000000,1.000000}%
\pgfsetfillcolor{currentfill}%
\pgfsetlinewidth{0.000000pt}%
\definecolor{currentstroke}{rgb}{1.000000,1.000000,1.000000}%
\pgfsetstrokecolor{currentstroke}%
\pgfsetdash{}{0pt}%
\pgfpathmoveto{\pgfqpoint{0.000000in}{0.000000in}}%
\pgfpathlineto{\pgfqpoint{6.294042in}{0.000000in}}%
\pgfpathlineto{\pgfqpoint{6.294042in}{4.693849in}}%
\pgfpathlineto{\pgfqpoint{0.000000in}{4.693849in}}%
\pgfpathclose%
\pgfusepath{fill}%
\end{pgfscope}%
\begin{pgfscope}%
\pgfsetbuttcap%
\pgfsetmiterjoin%
\definecolor{currentfill}{rgb}{1.000000,1.000000,1.000000}%
\pgfsetfillcolor{currentfill}%
\pgfsetlinewidth{0.000000pt}%
\definecolor{currentstroke}{rgb}{0.000000,0.000000,0.000000}%
\pgfsetstrokecolor{currentstroke}%
\pgfsetstrokeopacity{0.000000}%
\pgfsetdash{}{0pt}%
\pgfpathmoveto{\pgfqpoint{0.608070in}{1.439775in}}%
\pgfpathlineto{\pgfqpoint{6.194042in}{1.439775in}}%
\pgfpathlineto{\pgfqpoint{6.194042in}{4.383888in}}%
\pgfpathlineto{\pgfqpoint{0.608070in}{4.383888in}}%
\pgfpathclose%
\pgfusepath{fill}%
\end{pgfscope}%
\begin{pgfscope}%
\pgfpathrectangle{\pgfqpoint{0.608070in}{1.439775in}}{\pgfqpoint{5.585972in}{2.944113in}}%
\pgfusepath{clip}%
\pgfsetbuttcap%
\pgfsetmiterjoin%
\definecolor{currentfill}{rgb}{0.121569,0.466667,0.705882}%
\pgfsetfillcolor{currentfill}%
\pgfsetlinewidth{0.000000pt}%
\definecolor{currentstroke}{rgb}{0.000000,0.000000,0.000000}%
\pgfsetstrokecolor{currentstroke}%
\pgfsetstrokeopacity{0.000000}%
\pgfsetdash{}{0pt}%
\pgfpathmoveto{\pgfqpoint{0.861978in}{1.439775in}}%
\pgfpathlineto{\pgfqpoint{1.193162in}{1.439775in}}%
\pgfpathlineto{\pgfqpoint{1.193162in}{3.822740in}}%
\pgfpathlineto{\pgfqpoint{0.861978in}{3.822740in}}%
\pgfpathclose%
\pgfusepath{fill}%
\end{pgfscope}%
\begin{pgfscope}%
\pgfpathrectangle{\pgfqpoint{0.608070in}{1.439775in}}{\pgfqpoint{5.585972in}{2.944113in}}%
\pgfusepath{clip}%
\pgfsetbuttcap%
\pgfsetmiterjoin%
\definecolor{currentfill}{rgb}{0.121569,0.466667,0.705882}%
\pgfsetfillcolor{currentfill}%
\pgfsetlinewidth{0.000000pt}%
\definecolor{currentstroke}{rgb}{0.000000,0.000000,0.000000}%
\pgfsetstrokecolor{currentstroke}%
\pgfsetstrokeopacity{0.000000}%
\pgfsetdash{}{0pt}%
\pgfpathmoveto{\pgfqpoint{1.597943in}{1.439775in}}%
\pgfpathlineto{\pgfqpoint{1.929127in}{1.439775in}}%
\pgfpathlineto{\pgfqpoint{1.929127in}{4.117151in}}%
\pgfpathlineto{\pgfqpoint{1.597943in}{4.117151in}}%
\pgfpathclose%
\pgfusepath{fill}%
\end{pgfscope}%
\begin{pgfscope}%
\pgfpathrectangle{\pgfqpoint{0.608070in}{1.439775in}}{\pgfqpoint{5.585972in}{2.944113in}}%
\pgfusepath{clip}%
\pgfsetbuttcap%
\pgfsetmiterjoin%
\definecolor{currentfill}{rgb}{0.121569,0.466667,0.705882}%
\pgfsetfillcolor{currentfill}%
\pgfsetlinewidth{0.000000pt}%
\definecolor{currentstroke}{rgb}{0.000000,0.000000,0.000000}%
\pgfsetstrokecolor{currentstroke}%
\pgfsetstrokeopacity{0.000000}%
\pgfsetdash{}{0pt}%
\pgfpathmoveto{\pgfqpoint{2.333907in}{1.439775in}}%
\pgfpathlineto{\pgfqpoint{2.665091in}{1.439775in}}%
\pgfpathlineto{\pgfqpoint{2.665091in}{3.166792in}}%
\pgfpathlineto{\pgfqpoint{2.333907in}{3.166792in}}%
\pgfpathclose%
\pgfusepath{fill}%
\end{pgfscope}%
\begin{pgfscope}%
\pgfpathrectangle{\pgfqpoint{0.608070in}{1.439775in}}{\pgfqpoint{5.585972in}{2.944113in}}%
\pgfusepath{clip}%
\pgfsetbuttcap%
\pgfsetmiterjoin%
\definecolor{currentfill}{rgb}{0.121569,0.466667,0.705882}%
\pgfsetfillcolor{currentfill}%
\pgfsetlinewidth{0.000000pt}%
\definecolor{currentstroke}{rgb}{0.000000,0.000000,0.000000}%
\pgfsetstrokecolor{currentstroke}%
\pgfsetstrokeopacity{0.000000}%
\pgfsetdash{}{0pt}%
\pgfpathmoveto{\pgfqpoint{3.069872in}{1.439775in}}%
\pgfpathlineto{\pgfqpoint{3.401056in}{1.439775in}}%
\pgfpathlineto{\pgfqpoint{3.401056in}{2.864726in}}%
\pgfpathlineto{\pgfqpoint{3.069872in}{2.864726in}}%
\pgfpathclose%
\pgfusepath{fill}%
\end{pgfscope}%
\begin{pgfscope}%
\pgfpathrectangle{\pgfqpoint{0.608070in}{1.439775in}}{\pgfqpoint{5.585972in}{2.944113in}}%
\pgfusepath{clip}%
\pgfsetbuttcap%
\pgfsetmiterjoin%
\definecolor{currentfill}{rgb}{0.121569,0.466667,0.705882}%
\pgfsetfillcolor{currentfill}%
\pgfsetlinewidth{0.000000pt}%
\definecolor{currentstroke}{rgb}{0.000000,0.000000,0.000000}%
\pgfsetstrokecolor{currentstroke}%
\pgfsetstrokeopacity{0.000000}%
\pgfsetdash{}{0pt}%
\pgfpathmoveto{\pgfqpoint{3.805837in}{1.439775in}}%
\pgfpathlineto{\pgfqpoint{4.137021in}{1.439775in}}%
\pgfpathlineto{\pgfqpoint{4.137021in}{3.508309in}}%
\pgfpathlineto{\pgfqpoint{3.805837in}{3.508309in}}%
\pgfpathclose%
\pgfusepath{fill}%
\end{pgfscope}%
\begin{pgfscope}%
\pgfpathrectangle{\pgfqpoint{0.608070in}{1.439775in}}{\pgfqpoint{5.585972in}{2.944113in}}%
\pgfusepath{clip}%
\pgfsetbuttcap%
\pgfsetmiterjoin%
\definecolor{currentfill}{rgb}{0.121569,0.466667,0.705882}%
\pgfsetfillcolor{currentfill}%
\pgfsetlinewidth{0.000000pt}%
\definecolor{currentstroke}{rgb}{0.000000,0.000000,0.000000}%
\pgfsetstrokecolor{currentstroke}%
\pgfsetstrokeopacity{0.000000}%
\pgfsetdash{}{0pt}%
\pgfpathmoveto{\pgfqpoint{4.541801in}{1.439775in}}%
\pgfpathlineto{\pgfqpoint{4.872986in}{1.439775in}}%
\pgfpathlineto{\pgfqpoint{4.872986in}{3.511842in}}%
\pgfpathlineto{\pgfqpoint{4.541801in}{3.511842in}}%
\pgfpathclose%
\pgfusepath{fill}%
\end{pgfscope}%
\begin{pgfscope}%
\pgfpathrectangle{\pgfqpoint{0.608070in}{1.439775in}}{\pgfqpoint{5.585972in}{2.944113in}}%
\pgfusepath{clip}%
\pgfsetbuttcap%
\pgfsetmiterjoin%
\definecolor{currentfill}{rgb}{0.121569,0.466667,0.705882}%
\pgfsetfillcolor{currentfill}%
\pgfsetlinewidth{0.000000pt}%
\definecolor{currentstroke}{rgb}{0.000000,0.000000,0.000000}%
\pgfsetstrokecolor{currentstroke}%
\pgfsetstrokeopacity{0.000000}%
\pgfsetdash{}{0pt}%
\pgfpathmoveto{\pgfqpoint{5.277766in}{1.439775in}}%
\pgfpathlineto{\pgfqpoint{5.608950in}{1.439775in}}%
\pgfpathlineto{\pgfqpoint{5.608950in}{3.621363in}}%
\pgfpathlineto{\pgfqpoint{5.277766in}{3.621363in}}%
\pgfpathclose%
\pgfusepath{fill}%
\end{pgfscope}%
\begin{pgfscope}%
\pgfpathrectangle{\pgfqpoint{0.608070in}{1.439775in}}{\pgfqpoint{5.585972in}{2.944113in}}%
\pgfusepath{clip}%
\pgfsetbuttcap%
\pgfsetmiterjoin%
\definecolor{currentfill}{rgb}{1.000000,0.498039,0.054902}%
\pgfsetfillcolor{currentfill}%
\pgfsetlinewidth{0.000000pt}%
\definecolor{currentstroke}{rgb}{0.000000,0.000000,0.000000}%
\pgfsetstrokecolor{currentstroke}%
\pgfsetstrokeopacity{0.000000}%
\pgfsetdash{}{0pt}%
\pgfpathmoveto{\pgfqpoint{1.193162in}{1.439775in}}%
\pgfpathlineto{\pgfqpoint{1.524346in}{1.439775in}}%
\pgfpathlineto{\pgfqpoint{1.524346in}{3.762091in}}%
\pgfpathlineto{\pgfqpoint{1.193162in}{3.762091in}}%
\pgfpathclose%
\pgfusepath{fill}%
\end{pgfscope}%
\begin{pgfscope}%
\pgfpathrectangle{\pgfqpoint{0.608070in}{1.439775in}}{\pgfqpoint{5.585972in}{2.944113in}}%
\pgfusepath{clip}%
\pgfsetbuttcap%
\pgfsetmiterjoin%
\definecolor{currentfill}{rgb}{1.000000,0.498039,0.054902}%
\pgfsetfillcolor{currentfill}%
\pgfsetlinewidth{0.000000pt}%
\definecolor{currentstroke}{rgb}{0.000000,0.000000,0.000000}%
\pgfsetstrokecolor{currentstroke}%
\pgfsetstrokeopacity{0.000000}%
\pgfsetdash{}{0pt}%
\pgfpathmoveto{\pgfqpoint{1.929127in}{1.439775in}}%
\pgfpathlineto{\pgfqpoint{2.260311in}{1.439775in}}%
\pgfpathlineto{\pgfqpoint{2.260311in}{3.898109in}}%
\pgfpathlineto{\pgfqpoint{1.929127in}{3.898109in}}%
\pgfpathclose%
\pgfusepath{fill}%
\end{pgfscope}%
\begin{pgfscope}%
\pgfpathrectangle{\pgfqpoint{0.608070in}{1.439775in}}{\pgfqpoint{5.585972in}{2.944113in}}%
\pgfusepath{clip}%
\pgfsetbuttcap%
\pgfsetmiterjoin%
\definecolor{currentfill}{rgb}{1.000000,0.498039,0.054902}%
\pgfsetfillcolor{currentfill}%
\pgfsetlinewidth{0.000000pt}%
\definecolor{currentstroke}{rgb}{0.000000,0.000000,0.000000}%
\pgfsetstrokecolor{currentstroke}%
\pgfsetstrokeopacity{0.000000}%
\pgfsetdash{}{0pt}%
\pgfpathmoveto{\pgfqpoint{2.665091in}{1.439775in}}%
\pgfpathlineto{\pgfqpoint{2.996276in}{1.439775in}}%
\pgfpathlineto{\pgfqpoint{2.996276in}{2.243518in}}%
\pgfpathlineto{\pgfqpoint{2.665091in}{2.243518in}}%
\pgfpathclose%
\pgfusepath{fill}%
\end{pgfscope}%
\begin{pgfscope}%
\pgfpathrectangle{\pgfqpoint{0.608070in}{1.439775in}}{\pgfqpoint{5.585972in}{2.944113in}}%
\pgfusepath{clip}%
\pgfsetbuttcap%
\pgfsetmiterjoin%
\definecolor{currentfill}{rgb}{1.000000,0.498039,0.054902}%
\pgfsetfillcolor{currentfill}%
\pgfsetlinewidth{0.000000pt}%
\definecolor{currentstroke}{rgb}{0.000000,0.000000,0.000000}%
\pgfsetstrokecolor{currentstroke}%
\pgfsetstrokeopacity{0.000000}%
\pgfsetdash{}{0pt}%
\pgfpathmoveto{\pgfqpoint{3.401056in}{1.439775in}}%
\pgfpathlineto{\pgfqpoint{3.732240in}{1.439775in}}%
\pgfpathlineto{\pgfqpoint{3.732240in}{2.496712in}}%
\pgfpathlineto{\pgfqpoint{3.401056in}{2.496712in}}%
\pgfpathclose%
\pgfusepath{fill}%
\end{pgfscope}%
\begin{pgfscope}%
\pgfpathrectangle{\pgfqpoint{0.608070in}{1.439775in}}{\pgfqpoint{5.585972in}{2.944113in}}%
\pgfusepath{clip}%
\pgfsetbuttcap%
\pgfsetmiterjoin%
\definecolor{currentfill}{rgb}{1.000000,0.498039,0.054902}%
\pgfsetfillcolor{currentfill}%
\pgfsetlinewidth{0.000000pt}%
\definecolor{currentstroke}{rgb}{0.000000,0.000000,0.000000}%
\pgfsetstrokecolor{currentstroke}%
\pgfsetstrokeopacity{0.000000}%
\pgfsetdash{}{0pt}%
\pgfpathmoveto{\pgfqpoint{4.137021in}{1.439775in}}%
\pgfpathlineto{\pgfqpoint{4.468205in}{1.439775in}}%
\pgfpathlineto{\pgfqpoint{4.468205in}{2.593278in}}%
\pgfpathlineto{\pgfqpoint{4.137021in}{2.593278in}}%
\pgfpathclose%
\pgfusepath{fill}%
\end{pgfscope}%
\begin{pgfscope}%
\pgfpathrectangle{\pgfqpoint{0.608070in}{1.439775in}}{\pgfqpoint{5.585972in}{2.944113in}}%
\pgfusepath{clip}%
\pgfsetbuttcap%
\pgfsetmiterjoin%
\definecolor{currentfill}{rgb}{1.000000,0.498039,0.054902}%
\pgfsetfillcolor{currentfill}%
\pgfsetlinewidth{0.000000pt}%
\definecolor{currentstroke}{rgb}{0.000000,0.000000,0.000000}%
\pgfsetstrokecolor{currentstroke}%
\pgfsetstrokeopacity{0.000000}%
\pgfsetdash{}{0pt}%
\pgfpathmoveto{\pgfqpoint{4.872986in}{1.439775in}}%
\pgfpathlineto{\pgfqpoint{5.204170in}{1.439775in}}%
\pgfpathlineto{\pgfqpoint{5.204170in}{2.481402in}}%
\pgfpathlineto{\pgfqpoint{4.872986in}{2.481402in}}%
\pgfpathclose%
\pgfusepath{fill}%
\end{pgfscope}%
\begin{pgfscope}%
\pgfpathrectangle{\pgfqpoint{0.608070in}{1.439775in}}{\pgfqpoint{5.585972in}{2.944113in}}%
\pgfusepath{clip}%
\pgfsetbuttcap%
\pgfsetmiterjoin%
\definecolor{currentfill}{rgb}{1.000000,0.498039,0.054902}%
\pgfsetfillcolor{currentfill}%
\pgfsetlinewidth{0.000000pt}%
\definecolor{currentstroke}{rgb}{0.000000,0.000000,0.000000}%
\pgfsetstrokecolor{currentstroke}%
\pgfsetstrokeopacity{0.000000}%
\pgfsetdash{}{0pt}%
\pgfpathmoveto{\pgfqpoint{5.608950in}{1.439775in}}%
\pgfpathlineto{\pgfqpoint{5.940134in}{1.439775in}}%
\pgfpathlineto{\pgfqpoint{5.940134in}{2.428997in}}%
\pgfpathlineto{\pgfqpoint{5.608950in}{2.428997in}}%
\pgfpathclose%
\pgfusepath{fill}%
\end{pgfscope}%
\begin{pgfscope}%
\pgfsetbuttcap%
\pgfsetroundjoin%
\definecolor{currentfill}{rgb}{0.000000,0.000000,0.000000}%
\pgfsetfillcolor{currentfill}%
\pgfsetlinewidth{0.803000pt}%
\definecolor{currentstroke}{rgb}{0.000000,0.000000,0.000000}%
\pgfsetstrokecolor{currentstroke}%
\pgfsetdash{}{0pt}%
\pgfsys@defobject{currentmarker}{\pgfqpoint{0.000000in}{-0.048611in}}{\pgfqpoint{0.000000in}{0.000000in}}{%
\pgfpathmoveto{\pgfqpoint{0.000000in}{0.000000in}}%
\pgfpathlineto{\pgfqpoint{0.000000in}{-0.048611in}}%
\pgfusepath{stroke,fill}%
}%
\begin{pgfscope}%
\pgfsys@transformshift{1.193162in}{1.439775in}%
\pgfsys@useobject{currentmarker}{}%
\end{pgfscope}%
\end{pgfscope}%
\begin{pgfscope}%
\definecolor{textcolor}{rgb}{0.000000,0.000000,0.000000}%
\pgfsetstrokecolor{textcolor}%
\pgfsetfillcolor{textcolor}%
\pgftext[x=0.741486in, y=0.310396in, left, base,rotate=45.000000]{\color{textcolor}\sffamily\fontsize{10.000000}{12.000000}\selectfont Pix3d(chair,no aug)}%
\end{pgfscope}%
\begin{pgfscope}%
\pgfsetbuttcap%
\pgfsetroundjoin%
\definecolor{currentfill}{rgb}{0.000000,0.000000,0.000000}%
\pgfsetfillcolor{currentfill}%
\pgfsetlinewidth{0.803000pt}%
\definecolor{currentstroke}{rgb}{0.000000,0.000000,0.000000}%
\pgfsetstrokecolor{currentstroke}%
\pgfsetdash{}{0pt}%
\pgfsys@defobject{currentmarker}{\pgfqpoint{0.000000in}{-0.048611in}}{\pgfqpoint{0.000000in}{0.000000in}}{%
\pgfpathmoveto{\pgfqpoint{0.000000in}{0.000000in}}%
\pgfpathlineto{\pgfqpoint{0.000000in}{-0.048611in}}%
\pgfusepath{stroke,fill}%
}%
\begin{pgfscope}%
\pgfsys@transformshift{1.929127in}{1.439775in}%
\pgfsys@useobject{currentmarker}{}%
\end{pgfscope}%
\end{pgfscope}%
\begin{pgfscope}%
\definecolor{textcolor}{rgb}{0.000000,0.000000,0.000000}%
\pgfsetstrokecolor{textcolor}%
\pgfsetfillcolor{textcolor}%
\pgftext[x=1.662216in, y=0.679928in, left, base,rotate=45.000000]{\color{textcolor}\sffamily\fontsize{10.000000}{12.000000}\selectfont Pix3d(chair)}%
\end{pgfscope}%
\begin{pgfscope}%
\pgfsetbuttcap%
\pgfsetroundjoin%
\definecolor{currentfill}{rgb}{0.000000,0.000000,0.000000}%
\pgfsetfillcolor{currentfill}%
\pgfsetlinewidth{0.803000pt}%
\definecolor{currentstroke}{rgb}{0.000000,0.000000,0.000000}%
\pgfsetstrokecolor{currentstroke}%
\pgfsetdash{}{0pt}%
\pgfsys@defobject{currentmarker}{\pgfqpoint{0.000000in}{-0.048611in}}{\pgfqpoint{0.000000in}{0.000000in}}{%
\pgfpathmoveto{\pgfqpoint{0.000000in}{0.000000in}}%
\pgfpathlineto{\pgfqpoint{0.000000in}{-0.048611in}}%
\pgfusepath{stroke,fill}%
}%
\begin{pgfscope}%
\pgfsys@transformshift{2.665091in}{1.439775in}%
\pgfsys@useobject{currentmarker}{}%
\end{pgfscope}%
\end{pgfscope}%
\begin{pgfscope}%
\definecolor{textcolor}{rgb}{0.000000,0.000000,0.000000}%
\pgfsetstrokecolor{textcolor}%
\pgfsetfillcolor{textcolor}%
\pgftext[x=2.417410in, y=0.718387in, left, base,rotate=45.000000]{\color{textcolor}\sffamily\fontsize{10.000000}{12.000000}\selectfont Textureless}%
\end{pgfscope}%
\begin{pgfscope}%
\pgfsetbuttcap%
\pgfsetroundjoin%
\definecolor{currentfill}{rgb}{0.000000,0.000000,0.000000}%
\pgfsetfillcolor{currentfill}%
\pgfsetlinewidth{0.803000pt}%
\definecolor{currentstroke}{rgb}{0.000000,0.000000,0.000000}%
\pgfsetstrokecolor{currentstroke}%
\pgfsetdash{}{0pt}%
\pgfsys@defobject{currentmarker}{\pgfqpoint{0.000000in}{-0.048611in}}{\pgfqpoint{0.000000in}{0.000000in}}{%
\pgfpathmoveto{\pgfqpoint{0.000000in}{0.000000in}}%
\pgfpathlineto{\pgfqpoint{0.000000in}{-0.048611in}}%
\pgfusepath{stroke,fill}%
}%
\begin{pgfscope}%
\pgfsys@transformshift{3.401056in}{1.439775in}%
\pgfsys@useobject{currentmarker}{}%
\end{pgfscope}%
\end{pgfscope}%
\begin{pgfscope}%
\definecolor{textcolor}{rgb}{0.000000,0.000000,0.000000}%
\pgfsetstrokecolor{textcolor}%
\pgfsetfillcolor{textcolor}%
\pgftext[x=2.989685in, y=0.391007in, left, base,rotate=45.000000]{\color{textcolor}\sffamily\fontsize{10.000000}{12.000000}\selectfont Textureless+Light}%
\end{pgfscope}%
\begin{pgfscope}%
\pgfsetbuttcap%
\pgfsetroundjoin%
\definecolor{currentfill}{rgb}{0.000000,0.000000,0.000000}%
\pgfsetfillcolor{currentfill}%
\pgfsetlinewidth{0.803000pt}%
\definecolor{currentstroke}{rgb}{0.000000,0.000000,0.000000}%
\pgfsetstrokecolor{currentstroke}%
\pgfsetdash{}{0pt}%
\pgfsys@defobject{currentmarker}{\pgfqpoint{0.000000in}{-0.048611in}}{\pgfqpoint{0.000000in}{0.000000in}}{%
\pgfpathmoveto{\pgfqpoint{0.000000in}{0.000000in}}%
\pgfpathlineto{\pgfqpoint{0.000000in}{-0.048611in}}%
\pgfusepath{stroke,fill}%
}%
\begin{pgfscope}%
\pgfsys@transformshift{4.137021in}{1.439775in}%
\pgfsys@useobject{currentmarker}{}%
\end{pgfscope}%
\end{pgfscope}%
\begin{pgfscope}%
\definecolor{textcolor}{rgb}{0.000000,0.000000,0.000000}%
\pgfsetstrokecolor{textcolor}%
\pgfsetfillcolor{textcolor}%
\pgftext[x=3.953190in, y=0.846088in, left, base,rotate=45.000000]{\color{textcolor}\sffamily\fontsize{10.000000}{12.000000}\selectfont Textured}%
\end{pgfscope}%
\begin{pgfscope}%
\pgfsetbuttcap%
\pgfsetroundjoin%
\definecolor{currentfill}{rgb}{0.000000,0.000000,0.000000}%
\pgfsetfillcolor{currentfill}%
\pgfsetlinewidth{0.803000pt}%
\definecolor{currentstroke}{rgb}{0.000000,0.000000,0.000000}%
\pgfsetstrokecolor{currentstroke}%
\pgfsetdash{}{0pt}%
\pgfsys@defobject{currentmarker}{\pgfqpoint{0.000000in}{-0.048611in}}{\pgfqpoint{0.000000in}{0.000000in}}{%
\pgfpathmoveto{\pgfqpoint{0.000000in}{0.000000in}}%
\pgfpathlineto{\pgfqpoint{0.000000in}{-0.048611in}}%
\pgfusepath{stroke,fill}%
}%
\begin{pgfscope}%
\pgfsys@transformshift{4.872986in}{1.439775in}%
\pgfsys@useobject{currentmarker}{}%
\end{pgfscope}%
\end{pgfscope}%
\begin{pgfscope}%
\definecolor{textcolor}{rgb}{0.000000,0.000000,0.000000}%
\pgfsetstrokecolor{textcolor}%
\pgfsetfillcolor{textcolor}%
\pgftext[x=4.525465in, y=0.518708in, left, base,rotate=45.000000]{\color{textcolor}\sffamily\fontsize{10.000000}{12.000000}\selectfont Textured+Light}%
\end{pgfscope}%
\begin{pgfscope}%
\pgfsetbuttcap%
\pgfsetroundjoin%
\definecolor{currentfill}{rgb}{0.000000,0.000000,0.000000}%
\pgfsetfillcolor{currentfill}%
\pgfsetlinewidth{0.803000pt}%
\definecolor{currentstroke}{rgb}{0.000000,0.000000,0.000000}%
\pgfsetstrokecolor{currentstroke}%
\pgfsetdash{}{0pt}%
\pgfsys@defobject{currentmarker}{\pgfqpoint{0.000000in}{-0.048611in}}{\pgfqpoint{0.000000in}{0.000000in}}{%
\pgfpathmoveto{\pgfqpoint{0.000000in}{0.000000in}}%
\pgfpathlineto{\pgfqpoint{0.000000in}{-0.048611in}}%
\pgfusepath{stroke,fill}%
}%
\begin{pgfscope}%
\pgfsys@transformshift{5.608950in}{1.439775in}%
\pgfsys@useobject{currentmarker}{}%
\end{pgfscope}%
\end{pgfscope}%
\begin{pgfscope}%
\definecolor{textcolor}{rgb}{0.000000,0.000000,0.000000}%
\pgfsetstrokecolor{textcolor}%
\pgfsetfillcolor{textcolor}%
\pgftext[x=5.337005in, y=0.669858in, left, base,rotate=45.000000]{\color{textcolor}\sffamily\fontsize{10.000000}{12.000000}\selectfont Multi-Object}%
\end{pgfscope}%
\begin{pgfscope}%
\definecolor{textcolor}{rgb}{0.000000,0.000000,0.000000}%
\pgfsetstrokecolor{textcolor}%
\pgfsetfillcolor{textcolor}%
\pgftext[x=3.401056in,y=0.234413in,,top]{\color{textcolor}\sffamily\fontsize{10.000000}{12.000000}\selectfont Dataset}%
\end{pgfscope}%
\begin{pgfscope}%
\pgfsetbuttcap%
\pgfsetroundjoin%
\definecolor{currentfill}{rgb}{0.000000,0.000000,0.000000}%
\pgfsetfillcolor{currentfill}%
\pgfsetlinewidth{0.803000pt}%
\definecolor{currentstroke}{rgb}{0.000000,0.000000,0.000000}%
\pgfsetstrokecolor{currentstroke}%
\pgfsetdash{}{0pt}%
\pgfsys@defobject{currentmarker}{\pgfqpoint{-0.048611in}{0.000000in}}{\pgfqpoint{-0.000000in}{0.000000in}}{%
\pgfpathmoveto{\pgfqpoint{-0.000000in}{0.000000in}}%
\pgfpathlineto{\pgfqpoint{-0.048611in}{0.000000in}}%
\pgfusepath{stroke,fill}%
}%
\begin{pgfscope}%
\pgfsys@transformshift{0.608070in}{1.439775in}%
\pgfsys@useobject{currentmarker}{}%
\end{pgfscope}%
\end{pgfscope}%
\begin{pgfscope}%
\definecolor{textcolor}{rgb}{0.000000,0.000000,0.000000}%
\pgfsetstrokecolor{textcolor}%
\pgfsetfillcolor{textcolor}%
\pgftext[x=0.289968in, y=1.387014in, left, base]{\color{textcolor}\sffamily\fontsize{10.000000}{12.000000}\selectfont 0.0}%
\end{pgfscope}%
\begin{pgfscope}%
\pgfsetbuttcap%
\pgfsetroundjoin%
\definecolor{currentfill}{rgb}{0.000000,0.000000,0.000000}%
\pgfsetfillcolor{currentfill}%
\pgfsetlinewidth{0.803000pt}%
\definecolor{currentstroke}{rgb}{0.000000,0.000000,0.000000}%
\pgfsetstrokecolor{currentstroke}%
\pgfsetdash{}{0pt}%
\pgfsys@defobject{currentmarker}{\pgfqpoint{-0.048611in}{0.000000in}}{\pgfqpoint{-0.000000in}{0.000000in}}{%
\pgfpathmoveto{\pgfqpoint{-0.000000in}{0.000000in}}%
\pgfpathlineto{\pgfqpoint{-0.048611in}{0.000000in}}%
\pgfusepath{stroke,fill}%
}%
\begin{pgfscope}%
\pgfsys@transformshift{0.608070in}{2.028598in}%
\pgfsys@useobject{currentmarker}{}%
\end{pgfscope}%
\end{pgfscope}%
\begin{pgfscope}%
\definecolor{textcolor}{rgb}{0.000000,0.000000,0.000000}%
\pgfsetstrokecolor{textcolor}%
\pgfsetfillcolor{textcolor}%
\pgftext[x=0.289968in, y=1.975836in, left, base]{\color{textcolor}\sffamily\fontsize{10.000000}{12.000000}\selectfont 0.1}%
\end{pgfscope}%
\begin{pgfscope}%
\pgfsetbuttcap%
\pgfsetroundjoin%
\definecolor{currentfill}{rgb}{0.000000,0.000000,0.000000}%
\pgfsetfillcolor{currentfill}%
\pgfsetlinewidth{0.803000pt}%
\definecolor{currentstroke}{rgb}{0.000000,0.000000,0.000000}%
\pgfsetstrokecolor{currentstroke}%
\pgfsetdash{}{0pt}%
\pgfsys@defobject{currentmarker}{\pgfqpoint{-0.048611in}{0.000000in}}{\pgfqpoint{-0.000000in}{0.000000in}}{%
\pgfpathmoveto{\pgfqpoint{-0.000000in}{0.000000in}}%
\pgfpathlineto{\pgfqpoint{-0.048611in}{0.000000in}}%
\pgfusepath{stroke,fill}%
}%
\begin{pgfscope}%
\pgfsys@transformshift{0.608070in}{2.617420in}%
\pgfsys@useobject{currentmarker}{}%
\end{pgfscope}%
\end{pgfscope}%
\begin{pgfscope}%
\definecolor{textcolor}{rgb}{0.000000,0.000000,0.000000}%
\pgfsetstrokecolor{textcolor}%
\pgfsetfillcolor{textcolor}%
\pgftext[x=0.289968in, y=2.564659in, left, base]{\color{textcolor}\sffamily\fontsize{10.000000}{12.000000}\selectfont 0.2}%
\end{pgfscope}%
\begin{pgfscope}%
\pgfsetbuttcap%
\pgfsetroundjoin%
\definecolor{currentfill}{rgb}{0.000000,0.000000,0.000000}%
\pgfsetfillcolor{currentfill}%
\pgfsetlinewidth{0.803000pt}%
\definecolor{currentstroke}{rgb}{0.000000,0.000000,0.000000}%
\pgfsetstrokecolor{currentstroke}%
\pgfsetdash{}{0pt}%
\pgfsys@defobject{currentmarker}{\pgfqpoint{-0.048611in}{0.000000in}}{\pgfqpoint{-0.000000in}{0.000000in}}{%
\pgfpathmoveto{\pgfqpoint{-0.000000in}{0.000000in}}%
\pgfpathlineto{\pgfqpoint{-0.048611in}{0.000000in}}%
\pgfusepath{stroke,fill}%
}%
\begin{pgfscope}%
\pgfsys@transformshift{0.608070in}{3.206243in}%
\pgfsys@useobject{currentmarker}{}%
\end{pgfscope}%
\end{pgfscope}%
\begin{pgfscope}%
\definecolor{textcolor}{rgb}{0.000000,0.000000,0.000000}%
\pgfsetstrokecolor{textcolor}%
\pgfsetfillcolor{textcolor}%
\pgftext[x=0.289968in, y=3.153481in, left, base]{\color{textcolor}\sffamily\fontsize{10.000000}{12.000000}\selectfont 0.3}%
\end{pgfscope}%
\begin{pgfscope}%
\pgfsetbuttcap%
\pgfsetroundjoin%
\definecolor{currentfill}{rgb}{0.000000,0.000000,0.000000}%
\pgfsetfillcolor{currentfill}%
\pgfsetlinewidth{0.803000pt}%
\definecolor{currentstroke}{rgb}{0.000000,0.000000,0.000000}%
\pgfsetstrokecolor{currentstroke}%
\pgfsetdash{}{0pt}%
\pgfsys@defobject{currentmarker}{\pgfqpoint{-0.048611in}{0.000000in}}{\pgfqpoint{-0.000000in}{0.000000in}}{%
\pgfpathmoveto{\pgfqpoint{-0.000000in}{0.000000in}}%
\pgfpathlineto{\pgfqpoint{-0.048611in}{0.000000in}}%
\pgfusepath{stroke,fill}%
}%
\begin{pgfscope}%
\pgfsys@transformshift{0.608070in}{3.795065in}%
\pgfsys@useobject{currentmarker}{}%
\end{pgfscope}%
\end{pgfscope}%
\begin{pgfscope}%
\definecolor{textcolor}{rgb}{0.000000,0.000000,0.000000}%
\pgfsetstrokecolor{textcolor}%
\pgfsetfillcolor{textcolor}%
\pgftext[x=0.289968in, y=3.742304in, left, base]{\color{textcolor}\sffamily\fontsize{10.000000}{12.000000}\selectfont 0.4}%
\end{pgfscope}%
\begin{pgfscope}%
\pgfsetbuttcap%
\pgfsetroundjoin%
\definecolor{currentfill}{rgb}{0.000000,0.000000,0.000000}%
\pgfsetfillcolor{currentfill}%
\pgfsetlinewidth{0.803000pt}%
\definecolor{currentstroke}{rgb}{0.000000,0.000000,0.000000}%
\pgfsetstrokecolor{currentstroke}%
\pgfsetdash{}{0pt}%
\pgfsys@defobject{currentmarker}{\pgfqpoint{-0.048611in}{0.000000in}}{\pgfqpoint{-0.000000in}{0.000000in}}{%
\pgfpathmoveto{\pgfqpoint{-0.000000in}{0.000000in}}%
\pgfpathlineto{\pgfqpoint{-0.048611in}{0.000000in}}%
\pgfusepath{stroke,fill}%
}%
\begin{pgfscope}%
\pgfsys@transformshift{0.608070in}{4.383888in}%
\pgfsys@useobject{currentmarker}{}%
\end{pgfscope}%
\end{pgfscope}%
\begin{pgfscope}%
\definecolor{textcolor}{rgb}{0.000000,0.000000,0.000000}%
\pgfsetstrokecolor{textcolor}%
\pgfsetfillcolor{textcolor}%
\pgftext[x=0.289968in, y=4.331126in, left, base]{\color{textcolor}\sffamily\fontsize{10.000000}{12.000000}\selectfont 0.5}%
\end{pgfscope}%
\begin{pgfscope}%
\definecolor{textcolor}{rgb}{0.000000,0.000000,0.000000}%
\pgfsetstrokecolor{textcolor}%
\pgfsetfillcolor{textcolor}%
\pgftext[x=0.234413in,y=2.911831in,,bottom,rotate=90.000000]{\color{textcolor}\sffamily\fontsize{10.000000}{12.000000}\selectfont F1 Score}%
\end{pgfscope}%
\begin{pgfscope}%
\pgfsetrectcap%
\pgfsetmiterjoin%
\pgfsetlinewidth{0.803000pt}%
\definecolor{currentstroke}{rgb}{0.000000,0.000000,0.000000}%
\pgfsetstrokecolor{currentstroke}%
\pgfsetdash{}{0pt}%
\pgfpathmoveto{\pgfqpoint{0.608070in}{1.439775in}}%
\pgfpathlineto{\pgfqpoint{0.608070in}{4.383888in}}%
\pgfusepath{stroke}%
\end{pgfscope}%
\begin{pgfscope}%
\pgfsetrectcap%
\pgfsetmiterjoin%
\pgfsetlinewidth{0.803000pt}%
\definecolor{currentstroke}{rgb}{0.000000,0.000000,0.000000}%
\pgfsetstrokecolor{currentstroke}%
\pgfsetdash{}{0pt}%
\pgfpathmoveto{\pgfqpoint{6.194042in}{1.439775in}}%
\pgfpathlineto{\pgfqpoint{6.194042in}{4.383888in}}%
\pgfusepath{stroke}%
\end{pgfscope}%
\begin{pgfscope}%
\pgfsetrectcap%
\pgfsetmiterjoin%
\pgfsetlinewidth{0.803000pt}%
\definecolor{currentstroke}{rgb}{0.000000,0.000000,0.000000}%
\pgfsetstrokecolor{currentstroke}%
\pgfsetdash{}{0pt}%
\pgfpathmoveto{\pgfqpoint{0.608070in}{1.439775in}}%
\pgfpathlineto{\pgfqpoint{6.194042in}{1.439775in}}%
\pgfusepath{stroke}%
\end{pgfscope}%
\begin{pgfscope}%
\pgfsetrectcap%
\pgfsetmiterjoin%
\pgfsetlinewidth{0.803000pt}%
\definecolor{currentstroke}{rgb}{0.000000,0.000000,0.000000}%
\pgfsetstrokecolor{currentstroke}%
\pgfsetdash{}{0pt}%
\pgfpathmoveto{\pgfqpoint{0.608070in}{4.383888in}}%
\pgfpathlineto{\pgfqpoint{6.194042in}{4.383888in}}%
\pgfusepath{stroke}%
\end{pgfscope}%
\begin{pgfscope}%
\definecolor{textcolor}{rgb}{0.000000,0.000000,0.000000}%
\pgfsetstrokecolor{textcolor}%
\pgfsetfillcolor{textcolor}%
\pgftext[x=1.027570in,y=3.864407in,,bottom]{\color{textcolor}\sffamily\fontsize{9.000000}{10.800000}\selectfont 0.4047}%
\end{pgfscope}%
\begin{pgfscope}%
\definecolor{textcolor}{rgb}{0.000000,0.000000,0.000000}%
\pgfsetstrokecolor{textcolor}%
\pgfsetfillcolor{textcolor}%
\pgftext[x=1.763535in,y=4.158818in,,bottom]{\color{textcolor}\sffamily\fontsize{9.000000}{10.800000}\selectfont 0.4547}%
\end{pgfscope}%
\begin{pgfscope}%
\definecolor{textcolor}{rgb}{0.000000,0.000000,0.000000}%
\pgfsetstrokecolor{textcolor}%
\pgfsetfillcolor{textcolor}%
\pgftext[x=2.499499in,y=3.208458in,,bottom]{\color{textcolor}\sffamily\fontsize{9.000000}{10.800000}\selectfont 0.2933}%
\end{pgfscope}%
\begin{pgfscope}%
\definecolor{textcolor}{rgb}{0.000000,0.000000,0.000000}%
\pgfsetstrokecolor{textcolor}%
\pgfsetfillcolor{textcolor}%
\pgftext[x=3.235464in,y=2.906392in,,bottom]{\color{textcolor}\sffamily\fontsize{9.000000}{10.800000}\selectfont 0.242}%
\end{pgfscope}%
\begin{pgfscope}%
\definecolor{textcolor}{rgb}{0.000000,0.000000,0.000000}%
\pgfsetstrokecolor{textcolor}%
\pgfsetfillcolor{textcolor}%
\pgftext[x=3.971429in,y=3.549975in,,bottom]{\color{textcolor}\sffamily\fontsize{9.000000}{10.800000}\selectfont 0.3513}%
\end{pgfscope}%
\begin{pgfscope}%
\definecolor{textcolor}{rgb}{0.000000,0.000000,0.000000}%
\pgfsetstrokecolor{textcolor}%
\pgfsetfillcolor{textcolor}%
\pgftext[x=4.707393in,y=3.553508in,,bottom]{\color{textcolor}\sffamily\fontsize{9.000000}{10.800000}\selectfont 0.3519}%
\end{pgfscope}%
\begin{pgfscope}%
\definecolor{textcolor}{rgb}{0.000000,0.000000,0.000000}%
\pgfsetstrokecolor{textcolor}%
\pgfsetfillcolor{textcolor}%
\pgftext[x=5.443358in,y=3.663029in,,bottom]{\color{textcolor}\sffamily\fontsize{9.000000}{10.800000}\selectfont 0.3705}%
\end{pgfscope}%
\begin{pgfscope}%
\definecolor{textcolor}{rgb}{0.000000,0.000000,0.000000}%
\pgfsetstrokecolor{textcolor}%
\pgfsetfillcolor{textcolor}%
\pgftext[x=1.358754in,y=3.803758in,,bottom]{\color{textcolor}\sffamily\fontsize{9.000000}{10.800000}\selectfont 0.3944}%
\end{pgfscope}%
\begin{pgfscope}%
\definecolor{textcolor}{rgb}{0.000000,0.000000,0.000000}%
\pgfsetstrokecolor{textcolor}%
\pgfsetfillcolor{textcolor}%
\pgftext[x=2.094719in,y=3.939776in,,bottom]{\color{textcolor}\sffamily\fontsize{9.000000}{10.800000}\selectfont 0.4175}%
\end{pgfscope}%
\begin{pgfscope}%
\definecolor{textcolor}{rgb}{0.000000,0.000000,0.000000}%
\pgfsetstrokecolor{textcolor}%
\pgfsetfillcolor{textcolor}%
\pgftext[x=2.830683in,y=2.285185in,,bottom]{\color{textcolor}\sffamily\fontsize{9.000000}{10.800000}\selectfont 0.1365}%
\end{pgfscope}%
\begin{pgfscope}%
\definecolor{textcolor}{rgb}{0.000000,0.000000,0.000000}%
\pgfsetstrokecolor{textcolor}%
\pgfsetfillcolor{textcolor}%
\pgftext[x=3.566648in,y=2.538378in,,bottom]{\color{textcolor}\sffamily\fontsize{9.000000}{10.800000}\selectfont 0.1795}%
\end{pgfscope}%
\begin{pgfscope}%
\definecolor{textcolor}{rgb}{0.000000,0.000000,0.000000}%
\pgfsetstrokecolor{textcolor}%
\pgfsetfillcolor{textcolor}%
\pgftext[x=4.302613in,y=2.634945in,,bottom]{\color{textcolor}\sffamily\fontsize{9.000000}{10.800000}\selectfont 0.1959}%
\end{pgfscope}%
\begin{pgfscope}%
\definecolor{textcolor}{rgb}{0.000000,0.000000,0.000000}%
\pgfsetstrokecolor{textcolor}%
\pgfsetfillcolor{textcolor}%
\pgftext[x=5.038578in,y=2.523069in,,bottom]{\color{textcolor}\sffamily\fontsize{9.000000}{10.800000}\selectfont 0.1769}%
\end{pgfscope}%
\begin{pgfscope}%
\definecolor{textcolor}{rgb}{0.000000,0.000000,0.000000}%
\pgfsetstrokecolor{textcolor}%
\pgfsetfillcolor{textcolor}%
\pgftext[x=5.774542in,y=2.470664in,,bottom]{\color{textcolor}\sffamily\fontsize{9.000000}{10.800000}\selectfont 0.168}%
\end{pgfscope}%
\begin{pgfscope}%
\definecolor{textcolor}{rgb}{0.000000,0.000000,0.000000}%
\pgfsetstrokecolor{textcolor}%
\pgfsetfillcolor{textcolor}%
\pgftext[x=3.401056in,y=4.467221in,,base]{\color{textcolor}\sffamily\fontsize{12.000000}{14.400000}\selectfont Abalation study on chairs}%
\end{pgfscope}%
\begin{pgfscope}%
\pgfsetbuttcap%
\pgfsetmiterjoin%
\definecolor{currentfill}{rgb}{1.000000,1.000000,1.000000}%
\pgfsetfillcolor{currentfill}%
\pgfsetfillopacity{0.800000}%
\pgfsetlinewidth{1.003750pt}%
\definecolor{currentstroke}{rgb}{0.800000,0.800000,0.800000}%
\pgfsetstrokecolor{currentstroke}%
\pgfsetstrokeopacity{0.800000}%
\pgfsetdash{}{0pt}%
\pgfpathmoveto{\pgfqpoint{4.882654in}{3.865062in}}%
\pgfpathlineto{\pgfqpoint{6.096820in}{3.865062in}}%
\pgfpathquadraticcurveto{\pgfqpoint{6.124598in}{3.865062in}}{\pgfqpoint{6.124598in}{3.892840in}}%
\pgfpathlineto{\pgfqpoint{6.124598in}{4.286666in}}%
\pgfpathquadraticcurveto{\pgfqpoint{6.124598in}{4.314443in}}{\pgfqpoint{6.096820in}{4.314443in}}%
\pgfpathlineto{\pgfqpoint{4.882654in}{4.314443in}}%
\pgfpathquadraticcurveto{\pgfqpoint{4.854877in}{4.314443in}}{\pgfqpoint{4.854877in}{4.286666in}}%
\pgfpathlineto{\pgfqpoint{4.854877in}{3.892840in}}%
\pgfpathquadraticcurveto{\pgfqpoint{4.854877in}{3.865062in}}{\pgfqpoint{4.882654in}{3.865062in}}%
\pgfpathclose%
\pgfusepath{stroke,fill}%
\end{pgfscope}%
\begin{pgfscope}%
\pgfsetbuttcap%
\pgfsetmiterjoin%
\definecolor{currentfill}{rgb}{0.121569,0.466667,0.705882}%
\pgfsetfillcolor{currentfill}%
\pgfsetlinewidth{0.000000pt}%
\definecolor{currentstroke}{rgb}{0.000000,0.000000,0.000000}%
\pgfsetstrokecolor{currentstroke}%
\pgfsetstrokeopacity{0.000000}%
\pgfsetdash{}{0pt}%
\pgfpathmoveto{\pgfqpoint{4.910432in}{4.153365in}}%
\pgfpathlineto{\pgfqpoint{5.188210in}{4.153365in}}%
\pgfpathlineto{\pgfqpoint{5.188210in}{4.250587in}}%
\pgfpathlineto{\pgfqpoint{4.910432in}{4.250587in}}%
\pgfpathclose%
\pgfusepath{fill}%
\end{pgfscope}%
\begin{pgfscope}%
\definecolor{textcolor}{rgb}{0.000000,0.000000,0.000000}%
\pgfsetstrokecolor{textcolor}%
\pgfsetfillcolor{textcolor}%
\pgftext[x=5.299321in,y=4.153365in,left,base]{\color{textcolor}\sffamily\fontsize{10.000000}{12.000000}\selectfont Pix2Vox++}%
\end{pgfscope}%
\begin{pgfscope}%
\pgfsetbuttcap%
\pgfsetmiterjoin%
\definecolor{currentfill}{rgb}{1.000000,0.498039,0.054902}%
\pgfsetfillcolor{currentfill}%
\pgfsetlinewidth{0.000000pt}%
\definecolor{currentstroke}{rgb}{0.000000,0.000000,0.000000}%
\pgfsetstrokecolor{currentstroke}%
\pgfsetstrokeopacity{0.000000}%
\pgfsetdash{}{0pt}%
\pgfpathmoveto{\pgfqpoint{4.910432in}{3.949507in}}%
\pgfpathlineto{\pgfqpoint{5.188210in}{3.949507in}}%
\pgfpathlineto{\pgfqpoint{5.188210in}{4.046730in}}%
\pgfpathlineto{\pgfqpoint{4.910432in}{4.046730in}}%
\pgfpathclose%
\pgfusepath{fill}%
\end{pgfscope}%
\begin{pgfscope}%
\definecolor{textcolor}{rgb}{0.000000,0.000000,0.000000}%
\pgfsetstrokecolor{textcolor}%
\pgfsetfillcolor{textcolor}%
\pgftext[x=5.299321in,y=3.949507in,left,base]{\color{textcolor}\sffamily\fontsize{10.000000}{12.000000}\selectfont Pix2Vox}%
\end{pgfscope}%
\end{pgfpicture}%
\makeatother%
\endgroup%
}
    \caption{Bar plot for the \gls{f1}  for baseline trained on chair dataset with different domain randomization parameters and tested on real dataset.
    We see a dip in performance near textureless dataset, but it gradually increases with addition of domain randomization parameter.}
    \label{fig:ablation_dice1}
\end{figure}

To further improve the performance(as in \autoref{fig:ablation_dice2}), we do a mixed training with 50\% of real data per mini-batch to witness an increment of average \gls{f1} in the range of 3 to 7\%,
While the baseline Pix2vox++ give a \gls{f1} of 0.4547 on real chairs, with the textureless chair we get a \gks{f1} of 0.4879.
For Pix2Vox, the difference is slightly lesser, as it is 0.4175and 0.4659 on real and textureless chairs.
With the combined dataset we see that Pix2Vox++ has \gls{f1} of 0.4962 and Pix2Vox has 0.4812, which is 4.15\% and 6.37\% improvement over real dataset.

%\begin{figure}
%    \centering
%    \resizebox{0.7\textwidth}{!}{\input{/Users/apple/OVGU/Thesis/code/3dReconstruction/report/images/evaluation/performance/ablation_dice_linegraph1.pgf}}
%    \caption{Line plot for the \gls{f1}  for baseline trained on chair dataset with different domain randomization parameters and tested on real dataset.
%    We see a dip in performance near textureless dataset, but it gradually increases with addition of domain randomization parameter.}
%    \label{fig:ablation_dice1}
%\end{figure}


%\begin{figure}
%    \centering
%    \resizebox{0.7\textwidth}{!}{%% Creator: Matplotlib, PGF backend
%%
%% To include the figure in your LaTeX document, write
%%   \input{<filename>.pgf}
%%
%% Make sure the required packages are loaded in your preamble
%%   \usepackage{pgf}
%%
%% Figures using additional raster images can only be included by \input if
%% they are in the same directory as the main LaTeX file. For loading figures
%% from other directories you can use the `import` package
%%   \usepackage{import}
%%
%% and then include the figures with
%%   \import{<path to file>}{<filename>.pgf}
%%
%% Matplotlib used the following preamble
%%   \usepackage{fontspec}
%%   \setmainfont{DejaVuSerif.ttf}[Path=\detokenize{/Users/apple/opt/anaconda3/envs/kaolin/lib/python3.7/site-packages/matplotlib/mpl-data/fonts/ttf/}]
%%   \setsansfont{DejaVuSans.ttf}[Path=\detokenize{/Users/apple/opt/anaconda3/envs/kaolin/lib/python3.7/site-packages/matplotlib/mpl-data/fonts/ttf/}]
%%   \setmonofont{DejaVuSansMono.ttf}[Path=\detokenize{/Users/apple/opt/anaconda3/envs/kaolin/lib/python3.7/site-packages/matplotlib/mpl-data/fonts/ttf/}]
%%
\begingroup%
\makeatletter%
\begin{pgfpicture}%
\pgfpathrectangle{\pgfpointorigin}{\pgfqpoint{5.827982in}{5.207926in}}%
\pgfusepath{use as bounding box, clip}%
\begin{pgfscope}%
\pgfsetbuttcap%
\pgfsetmiterjoin%
\definecolor{currentfill}{rgb}{1.000000,1.000000,1.000000}%
\pgfsetfillcolor{currentfill}%
\pgfsetlinewidth{0.000000pt}%
\definecolor{currentstroke}{rgb}{1.000000,1.000000,1.000000}%
\pgfsetstrokecolor{currentstroke}%
\pgfsetdash{}{0pt}%
\pgfpathmoveto{\pgfqpoint{0.000000in}{0.000000in}}%
\pgfpathlineto{\pgfqpoint{5.827982in}{0.000000in}}%
\pgfpathlineto{\pgfqpoint{5.827982in}{5.207926in}}%
\pgfpathlineto{\pgfqpoint{0.000000in}{5.207926in}}%
\pgfpathclose%
\pgfusepath{fill}%
\end{pgfscope}%
\begin{pgfscope}%
\pgfsetbuttcap%
\pgfsetmiterjoin%
\definecolor{currentfill}{rgb}{1.000000,1.000000,1.000000}%
\pgfsetfillcolor{currentfill}%
\pgfsetlinewidth{0.000000pt}%
\definecolor{currentstroke}{rgb}{0.000000,0.000000,0.000000}%
\pgfsetstrokecolor{currentstroke}%
\pgfsetstrokeopacity{0.000000}%
\pgfsetdash{}{0pt}%
\pgfpathmoveto{\pgfqpoint{0.696435in}{1.359165in}}%
\pgfpathlineto{\pgfqpoint{5.656435in}{1.359165in}}%
\pgfpathlineto{\pgfqpoint{5.656435in}{5.055165in}}%
\pgfpathlineto{\pgfqpoint{0.696435in}{5.055165in}}%
\pgfpathclose%
\pgfusepath{fill}%
\end{pgfscope}%
\begin{pgfscope}%
\pgfsetbuttcap%
\pgfsetroundjoin%
\definecolor{currentfill}{rgb}{0.000000,0.000000,0.000000}%
\pgfsetfillcolor{currentfill}%
\pgfsetlinewidth{0.803000pt}%
\definecolor{currentstroke}{rgb}{0.000000,0.000000,0.000000}%
\pgfsetstrokecolor{currentstroke}%
\pgfsetdash{}{0pt}%
\pgfsys@defobject{currentmarker}{\pgfqpoint{0.000000in}{-0.048611in}}{\pgfqpoint{0.000000in}{0.000000in}}{%
\pgfpathmoveto{\pgfqpoint{0.000000in}{0.000000in}}%
\pgfpathlineto{\pgfqpoint{0.000000in}{-0.048611in}}%
\pgfusepath{stroke,fill}%
}%
\begin{pgfscope}%
\pgfsys@transformshift{0.921890in}{1.359165in}%
\pgfsys@useobject{currentmarker}{}%
\end{pgfscope}%
\end{pgfscope}%
\begin{pgfscope}%
\definecolor{textcolor}{rgb}{0.000000,0.000000,0.000000}%
\pgfsetstrokecolor{textcolor}%
\pgfsetfillcolor{textcolor}%
\pgftext[x=0.654979in, y=0.599318in, left, base,rotate=45.000000]{\color{textcolor}\sffamily\fontsize{10.000000}{12.000000}\selectfont Pix3d(chair)}%
\end{pgfscope}%
\begin{pgfscope}%
\pgfsetbuttcap%
\pgfsetroundjoin%
\definecolor{currentfill}{rgb}{0.000000,0.000000,0.000000}%
\pgfsetfillcolor{currentfill}%
\pgfsetlinewidth{0.803000pt}%
\definecolor{currentstroke}{rgb}{0.000000,0.000000,0.000000}%
\pgfsetstrokecolor{currentstroke}%
\pgfsetdash{}{0pt}%
\pgfsys@defobject{currentmarker}{\pgfqpoint{0.000000in}{-0.048611in}}{\pgfqpoint{0.000000in}{0.000000in}}{%
\pgfpathmoveto{\pgfqpoint{0.000000in}{0.000000in}}%
\pgfpathlineto{\pgfqpoint{0.000000in}{-0.048611in}}%
\pgfusepath{stroke,fill}%
}%
\begin{pgfscope}%
\pgfsys@transformshift{1.673405in}{1.359165in}%
\pgfsys@useobject{currentmarker}{}%
\end{pgfscope}%
\end{pgfscope}%
\begin{pgfscope}%
\definecolor{textcolor}{rgb}{0.000000,0.000000,0.000000}%
\pgfsetstrokecolor{textcolor}%
\pgfsetfillcolor{textcolor}%
\pgftext[x=1.425724in, y=0.637777in, left, base,rotate=45.000000]{\color{textcolor}\sffamily\fontsize{10.000000}{12.000000}\selectfont Textureless}%
\end{pgfscope}%
\begin{pgfscope}%
\pgfsetbuttcap%
\pgfsetroundjoin%
\definecolor{currentfill}{rgb}{0.000000,0.000000,0.000000}%
\pgfsetfillcolor{currentfill}%
\pgfsetlinewidth{0.803000pt}%
\definecolor{currentstroke}{rgb}{0.000000,0.000000,0.000000}%
\pgfsetstrokecolor{currentstroke}%
\pgfsetdash{}{0pt}%
\pgfsys@defobject{currentmarker}{\pgfqpoint{0.000000in}{-0.048611in}}{\pgfqpoint{0.000000in}{0.000000in}}{%
\pgfpathmoveto{\pgfqpoint{0.000000in}{0.000000in}}%
\pgfpathlineto{\pgfqpoint{0.000000in}{-0.048611in}}%
\pgfusepath{stroke,fill}%
}%
\begin{pgfscope}%
\pgfsys@transformshift{2.424920in}{1.359165in}%
\pgfsys@useobject{currentmarker}{}%
\end{pgfscope}%
\end{pgfscope}%
\begin{pgfscope}%
\definecolor{textcolor}{rgb}{0.000000,0.000000,0.000000}%
\pgfsetstrokecolor{textcolor}%
\pgfsetfillcolor{textcolor}%
\pgftext[x=2.013549in, y=0.310396in, left, base,rotate=45.000000]{\color{textcolor}\sffamily\fontsize{10.000000}{12.000000}\selectfont Textureless+Light}%
\end{pgfscope}%
\begin{pgfscope}%
\pgfsetbuttcap%
\pgfsetroundjoin%
\definecolor{currentfill}{rgb}{0.000000,0.000000,0.000000}%
\pgfsetfillcolor{currentfill}%
\pgfsetlinewidth{0.803000pt}%
\definecolor{currentstroke}{rgb}{0.000000,0.000000,0.000000}%
\pgfsetstrokecolor{currentstroke}%
\pgfsetdash{}{0pt}%
\pgfsys@defobject{currentmarker}{\pgfqpoint{0.000000in}{-0.048611in}}{\pgfqpoint{0.000000in}{0.000000in}}{%
\pgfpathmoveto{\pgfqpoint{0.000000in}{0.000000in}}%
\pgfpathlineto{\pgfqpoint{0.000000in}{-0.048611in}}%
\pgfusepath{stroke,fill}%
}%
\begin{pgfscope}%
\pgfsys@transformshift{3.176435in}{1.359165in}%
\pgfsys@useobject{currentmarker}{}%
\end{pgfscope}%
\end{pgfscope}%
\begin{pgfscope}%
\definecolor{textcolor}{rgb}{0.000000,0.000000,0.000000}%
\pgfsetstrokecolor{textcolor}%
\pgfsetfillcolor{textcolor}%
\pgftext[x=2.992605in, y=0.765478in, left, base,rotate=45.000000]{\color{textcolor}\sffamily\fontsize{10.000000}{12.000000}\selectfont Textured}%
\end{pgfscope}%
\begin{pgfscope}%
\pgfsetbuttcap%
\pgfsetroundjoin%
\definecolor{currentfill}{rgb}{0.000000,0.000000,0.000000}%
\pgfsetfillcolor{currentfill}%
\pgfsetlinewidth{0.803000pt}%
\definecolor{currentstroke}{rgb}{0.000000,0.000000,0.000000}%
\pgfsetstrokecolor{currentstroke}%
\pgfsetdash{}{0pt}%
\pgfsys@defobject{currentmarker}{\pgfqpoint{0.000000in}{-0.048611in}}{\pgfqpoint{0.000000in}{0.000000in}}{%
\pgfpathmoveto{\pgfqpoint{0.000000in}{0.000000in}}%
\pgfpathlineto{\pgfqpoint{0.000000in}{-0.048611in}}%
\pgfusepath{stroke,fill}%
}%
\begin{pgfscope}%
\pgfsys@transformshift{3.927951in}{1.359165in}%
\pgfsys@useobject{currentmarker}{}%
\end{pgfscope}%
\end{pgfscope}%
\begin{pgfscope}%
\definecolor{textcolor}{rgb}{0.000000,0.000000,0.000000}%
\pgfsetstrokecolor{textcolor}%
\pgfsetfillcolor{textcolor}%
\pgftext[x=3.580430in, y=0.438097in, left, base,rotate=45.000000]{\color{textcolor}\sffamily\fontsize{10.000000}{12.000000}\selectfont Textured+Light}%
\end{pgfscope}%
\begin{pgfscope}%
\pgfsetbuttcap%
\pgfsetroundjoin%
\definecolor{currentfill}{rgb}{0.000000,0.000000,0.000000}%
\pgfsetfillcolor{currentfill}%
\pgfsetlinewidth{0.803000pt}%
\definecolor{currentstroke}{rgb}{0.000000,0.000000,0.000000}%
\pgfsetstrokecolor{currentstroke}%
\pgfsetdash{}{0pt}%
\pgfsys@defobject{currentmarker}{\pgfqpoint{0.000000in}{-0.048611in}}{\pgfqpoint{0.000000in}{0.000000in}}{%
\pgfpathmoveto{\pgfqpoint{0.000000in}{0.000000in}}%
\pgfpathlineto{\pgfqpoint{0.000000in}{-0.048611in}}%
\pgfusepath{stroke,fill}%
}%
\begin{pgfscope}%
\pgfsys@transformshift{4.679466in}{1.359165in}%
\pgfsys@useobject{currentmarker}{}%
\end{pgfscope}%
\end{pgfscope}%
\begin{pgfscope}%
\definecolor{textcolor}{rgb}{0.000000,0.000000,0.000000}%
\pgfsetstrokecolor{textcolor}%
\pgfsetfillcolor{textcolor}%
\pgftext[x=4.407520in, y=0.589247in, left, base,rotate=45.000000]{\color{textcolor}\sffamily\fontsize{10.000000}{12.000000}\selectfont Multi-Object}%
\end{pgfscope}%
\begin{pgfscope}%
\pgfsetbuttcap%
\pgfsetroundjoin%
\definecolor{currentfill}{rgb}{0.000000,0.000000,0.000000}%
\pgfsetfillcolor{currentfill}%
\pgfsetlinewidth{0.803000pt}%
\definecolor{currentstroke}{rgb}{0.000000,0.000000,0.000000}%
\pgfsetstrokecolor{currentstroke}%
\pgfsetdash{}{0pt}%
\pgfsys@defobject{currentmarker}{\pgfqpoint{0.000000in}{-0.048611in}}{\pgfqpoint{0.000000in}{0.000000in}}{%
\pgfpathmoveto{\pgfqpoint{0.000000in}{0.000000in}}%
\pgfpathlineto{\pgfqpoint{0.000000in}{-0.048611in}}%
\pgfusepath{stroke,fill}%
}%
\begin{pgfscope}%
\pgfsys@transformshift{5.430981in}{1.359165in}%
\pgfsys@useobject{currentmarker}{}%
\end{pgfscope}%
\end{pgfscope}%
\begin{pgfscope}%
\definecolor{textcolor}{rgb}{0.000000,0.000000,0.000000}%
\pgfsetstrokecolor{textcolor}%
\pgfsetfillcolor{textcolor}%
\pgftext[x=5.208595in, y=0.688368in, left, base,rotate=45.000000]{\color{textcolor}\sffamily\fontsize{10.000000}{12.000000}\selectfont Combined}%
\end{pgfscope}%
\begin{pgfscope}%
\definecolor{textcolor}{rgb}{0.000000,0.000000,0.000000}%
\pgfsetstrokecolor{textcolor}%
\pgfsetfillcolor{textcolor}%
\pgftext[x=3.176435in,y=0.234413in,,top]{\color{textcolor}\sffamily\fontsize{10.000000}{12.000000}\selectfont Datasets}%
\end{pgfscope}%
\begin{pgfscope}%
\pgfsetbuttcap%
\pgfsetroundjoin%
\definecolor{currentfill}{rgb}{0.000000,0.000000,0.000000}%
\pgfsetfillcolor{currentfill}%
\pgfsetlinewidth{0.803000pt}%
\definecolor{currentstroke}{rgb}{0.000000,0.000000,0.000000}%
\pgfsetstrokecolor{currentstroke}%
\pgfsetdash{}{0pt}%
\pgfsys@defobject{currentmarker}{\pgfqpoint{-0.048611in}{0.000000in}}{\pgfqpoint{-0.000000in}{0.000000in}}{%
\pgfpathmoveto{\pgfqpoint{-0.000000in}{0.000000in}}%
\pgfpathlineto{\pgfqpoint{-0.048611in}{0.000000in}}%
\pgfusepath{stroke,fill}%
}%
\begin{pgfscope}%
\pgfsys@transformshift{0.696435in}{1.359165in}%
\pgfsys@useobject{currentmarker}{}%
\end{pgfscope}%
\end{pgfscope}%
\begin{pgfscope}%
\definecolor{textcolor}{rgb}{0.000000,0.000000,0.000000}%
\pgfsetstrokecolor{textcolor}%
\pgfsetfillcolor{textcolor}%
\pgftext[x=0.289968in, y=1.306403in, left, base]{\color{textcolor}\sffamily\fontsize{10.000000}{12.000000}\selectfont 0.30}%
\end{pgfscope}%
\begin{pgfscope}%
\pgfsetbuttcap%
\pgfsetroundjoin%
\definecolor{currentfill}{rgb}{0.000000,0.000000,0.000000}%
\pgfsetfillcolor{currentfill}%
\pgfsetlinewidth{0.803000pt}%
\definecolor{currentstroke}{rgb}{0.000000,0.000000,0.000000}%
\pgfsetstrokecolor{currentstroke}%
\pgfsetdash{}{0pt}%
\pgfsys@defobject{currentmarker}{\pgfqpoint{-0.048611in}{0.000000in}}{\pgfqpoint{-0.000000in}{0.000000in}}{%
\pgfpathmoveto{\pgfqpoint{-0.000000in}{0.000000in}}%
\pgfpathlineto{\pgfqpoint{-0.048611in}{0.000000in}}%
\pgfusepath{stroke,fill}%
}%
\begin{pgfscope}%
\pgfsys@transformshift{0.696435in}{1.975165in}%
\pgfsys@useobject{currentmarker}{}%
\end{pgfscope}%
\end{pgfscope}%
\begin{pgfscope}%
\definecolor{textcolor}{rgb}{0.000000,0.000000,0.000000}%
\pgfsetstrokecolor{textcolor}%
\pgfsetfillcolor{textcolor}%
\pgftext[x=0.289968in, y=1.922403in, left, base]{\color{textcolor}\sffamily\fontsize{10.000000}{12.000000}\selectfont 0.35}%
\end{pgfscope}%
\begin{pgfscope}%
\pgfsetbuttcap%
\pgfsetroundjoin%
\definecolor{currentfill}{rgb}{0.000000,0.000000,0.000000}%
\pgfsetfillcolor{currentfill}%
\pgfsetlinewidth{0.803000pt}%
\definecolor{currentstroke}{rgb}{0.000000,0.000000,0.000000}%
\pgfsetstrokecolor{currentstroke}%
\pgfsetdash{}{0pt}%
\pgfsys@defobject{currentmarker}{\pgfqpoint{-0.048611in}{0.000000in}}{\pgfqpoint{-0.000000in}{0.000000in}}{%
\pgfpathmoveto{\pgfqpoint{-0.000000in}{0.000000in}}%
\pgfpathlineto{\pgfqpoint{-0.048611in}{0.000000in}}%
\pgfusepath{stroke,fill}%
}%
\begin{pgfscope}%
\pgfsys@transformshift{0.696435in}{2.591165in}%
\pgfsys@useobject{currentmarker}{}%
\end{pgfscope}%
\end{pgfscope}%
\begin{pgfscope}%
\definecolor{textcolor}{rgb}{0.000000,0.000000,0.000000}%
\pgfsetstrokecolor{textcolor}%
\pgfsetfillcolor{textcolor}%
\pgftext[x=0.289968in, y=2.538403in, left, base]{\color{textcolor}\sffamily\fontsize{10.000000}{12.000000}\selectfont 0.40}%
\end{pgfscope}%
\begin{pgfscope}%
\pgfsetbuttcap%
\pgfsetroundjoin%
\definecolor{currentfill}{rgb}{0.000000,0.000000,0.000000}%
\pgfsetfillcolor{currentfill}%
\pgfsetlinewidth{0.803000pt}%
\definecolor{currentstroke}{rgb}{0.000000,0.000000,0.000000}%
\pgfsetstrokecolor{currentstroke}%
\pgfsetdash{}{0pt}%
\pgfsys@defobject{currentmarker}{\pgfqpoint{-0.048611in}{0.000000in}}{\pgfqpoint{-0.000000in}{0.000000in}}{%
\pgfpathmoveto{\pgfqpoint{-0.000000in}{0.000000in}}%
\pgfpathlineto{\pgfqpoint{-0.048611in}{0.000000in}}%
\pgfusepath{stroke,fill}%
}%
\begin{pgfscope}%
\pgfsys@transformshift{0.696435in}{3.207165in}%
\pgfsys@useobject{currentmarker}{}%
\end{pgfscope}%
\end{pgfscope}%
\begin{pgfscope}%
\definecolor{textcolor}{rgb}{0.000000,0.000000,0.000000}%
\pgfsetstrokecolor{textcolor}%
\pgfsetfillcolor{textcolor}%
\pgftext[x=0.289968in, y=3.154403in, left, base]{\color{textcolor}\sffamily\fontsize{10.000000}{12.000000}\selectfont 0.45}%
\end{pgfscope}%
\begin{pgfscope}%
\pgfsetbuttcap%
\pgfsetroundjoin%
\definecolor{currentfill}{rgb}{0.000000,0.000000,0.000000}%
\pgfsetfillcolor{currentfill}%
\pgfsetlinewidth{0.803000pt}%
\definecolor{currentstroke}{rgb}{0.000000,0.000000,0.000000}%
\pgfsetstrokecolor{currentstroke}%
\pgfsetdash{}{0pt}%
\pgfsys@defobject{currentmarker}{\pgfqpoint{-0.048611in}{0.000000in}}{\pgfqpoint{-0.000000in}{0.000000in}}{%
\pgfpathmoveto{\pgfqpoint{-0.000000in}{0.000000in}}%
\pgfpathlineto{\pgfqpoint{-0.048611in}{0.000000in}}%
\pgfusepath{stroke,fill}%
}%
\begin{pgfscope}%
\pgfsys@transformshift{0.696435in}{3.823165in}%
\pgfsys@useobject{currentmarker}{}%
\end{pgfscope}%
\end{pgfscope}%
\begin{pgfscope}%
\definecolor{textcolor}{rgb}{0.000000,0.000000,0.000000}%
\pgfsetstrokecolor{textcolor}%
\pgfsetfillcolor{textcolor}%
\pgftext[x=0.289968in, y=3.770403in, left, base]{\color{textcolor}\sffamily\fontsize{10.000000}{12.000000}\selectfont 0.50}%
\end{pgfscope}%
\begin{pgfscope}%
\pgfsetbuttcap%
\pgfsetroundjoin%
\definecolor{currentfill}{rgb}{0.000000,0.000000,0.000000}%
\pgfsetfillcolor{currentfill}%
\pgfsetlinewidth{0.803000pt}%
\definecolor{currentstroke}{rgb}{0.000000,0.000000,0.000000}%
\pgfsetstrokecolor{currentstroke}%
\pgfsetdash{}{0pt}%
\pgfsys@defobject{currentmarker}{\pgfqpoint{-0.048611in}{0.000000in}}{\pgfqpoint{-0.000000in}{0.000000in}}{%
\pgfpathmoveto{\pgfqpoint{-0.000000in}{0.000000in}}%
\pgfpathlineto{\pgfqpoint{-0.048611in}{0.000000in}}%
\pgfusepath{stroke,fill}%
}%
\begin{pgfscope}%
\pgfsys@transformshift{0.696435in}{4.439165in}%
\pgfsys@useobject{currentmarker}{}%
\end{pgfscope}%
\end{pgfscope}%
\begin{pgfscope}%
\definecolor{textcolor}{rgb}{0.000000,0.000000,0.000000}%
\pgfsetstrokecolor{textcolor}%
\pgfsetfillcolor{textcolor}%
\pgftext[x=0.289968in, y=4.386403in, left, base]{\color{textcolor}\sffamily\fontsize{10.000000}{12.000000}\selectfont 0.55}%
\end{pgfscope}%
\begin{pgfscope}%
\pgfsetbuttcap%
\pgfsetroundjoin%
\definecolor{currentfill}{rgb}{0.000000,0.000000,0.000000}%
\pgfsetfillcolor{currentfill}%
\pgfsetlinewidth{0.803000pt}%
\definecolor{currentstroke}{rgb}{0.000000,0.000000,0.000000}%
\pgfsetstrokecolor{currentstroke}%
\pgfsetdash{}{0pt}%
\pgfsys@defobject{currentmarker}{\pgfqpoint{-0.048611in}{0.000000in}}{\pgfqpoint{-0.000000in}{0.000000in}}{%
\pgfpathmoveto{\pgfqpoint{-0.000000in}{0.000000in}}%
\pgfpathlineto{\pgfqpoint{-0.048611in}{0.000000in}}%
\pgfusepath{stroke,fill}%
}%
\begin{pgfscope}%
\pgfsys@transformshift{0.696435in}{5.055165in}%
\pgfsys@useobject{currentmarker}{}%
\end{pgfscope}%
\end{pgfscope}%
\begin{pgfscope}%
\definecolor{textcolor}{rgb}{0.000000,0.000000,0.000000}%
\pgfsetstrokecolor{textcolor}%
\pgfsetfillcolor{textcolor}%
\pgftext[x=0.289968in, y=5.002403in, left, base]{\color{textcolor}\sffamily\fontsize{10.000000}{12.000000}\selectfont 0.60}%
\end{pgfscope}%
\begin{pgfscope}%
\definecolor{textcolor}{rgb}{0.000000,0.000000,0.000000}%
\pgfsetstrokecolor{textcolor}%
\pgfsetfillcolor{textcolor}%
\pgftext[x=0.234413in,y=3.207165in,,bottom,rotate=90.000000]{\color{textcolor}\sffamily\fontsize{10.000000}{12.000000}\selectfont F1 Score}%
\end{pgfscope}%
\begin{pgfscope}%
\pgfpathrectangle{\pgfqpoint{0.696435in}{1.359165in}}{\pgfqpoint{4.960000in}{3.696000in}}%
\pgfusepath{clip}%
\pgfsetrectcap%
\pgfsetroundjoin%
\pgfsetlinewidth{1.505625pt}%
\definecolor{currentstroke}{rgb}{0.121569,0.466667,0.705882}%
\pgfsetstrokecolor{currentstroke}%
\pgfsetdash{}{0pt}%
\pgfpathmoveto{\pgfqpoint{0.921890in}{3.265069in}}%
\pgfpathlineto{\pgfqpoint{1.673405in}{3.674093in}}%
\pgfpathlineto{\pgfqpoint{2.424920in}{3.602637in}}%
\pgfpathlineto{\pgfqpoint{3.176435in}{3.417837in}}%
\pgfpathlineto{\pgfqpoint{3.927951in}{3.045773in}}%
\pgfpathlineto{\pgfqpoint{4.679466in}{3.039613in}}%
\pgfpathlineto{\pgfqpoint{5.430981in}{3.776349in}}%
\pgfusepath{stroke}%
\end{pgfscope}%
\begin{pgfscope}%
\pgfpathrectangle{\pgfqpoint{0.696435in}{1.359165in}}{\pgfqpoint{4.960000in}{3.696000in}}%
\pgfusepath{clip}%
\pgfsetbuttcap%
\pgfsetroundjoin%
\definecolor{currentfill}{rgb}{0.121569,0.466667,0.705882}%
\pgfsetfillcolor{currentfill}%
\pgfsetlinewidth{1.003750pt}%
\definecolor{currentstroke}{rgb}{0.121569,0.466667,0.705882}%
\pgfsetstrokecolor{currentstroke}%
\pgfsetdash{}{0pt}%
\pgfsys@defobject{currentmarker}{\pgfqpoint{-0.041667in}{-0.041667in}}{\pgfqpoint{0.041667in}{0.041667in}}{%
\pgfpathmoveto{\pgfqpoint{0.000000in}{-0.041667in}}%
\pgfpathcurveto{\pgfqpoint{0.011050in}{-0.041667in}}{\pgfqpoint{0.021649in}{-0.037276in}}{\pgfqpoint{0.029463in}{-0.029463in}}%
\pgfpathcurveto{\pgfqpoint{0.037276in}{-0.021649in}}{\pgfqpoint{0.041667in}{-0.011050in}}{\pgfqpoint{0.041667in}{0.000000in}}%
\pgfpathcurveto{\pgfqpoint{0.041667in}{0.011050in}}{\pgfqpoint{0.037276in}{0.021649in}}{\pgfqpoint{0.029463in}{0.029463in}}%
\pgfpathcurveto{\pgfqpoint{0.021649in}{0.037276in}}{\pgfqpoint{0.011050in}{0.041667in}}{\pgfqpoint{0.000000in}{0.041667in}}%
\pgfpathcurveto{\pgfqpoint{-0.011050in}{0.041667in}}{\pgfqpoint{-0.021649in}{0.037276in}}{\pgfqpoint{-0.029463in}{0.029463in}}%
\pgfpathcurveto{\pgfqpoint{-0.037276in}{0.021649in}}{\pgfqpoint{-0.041667in}{0.011050in}}{\pgfqpoint{-0.041667in}{0.000000in}}%
\pgfpathcurveto{\pgfqpoint{-0.041667in}{-0.011050in}}{\pgfqpoint{-0.037276in}{-0.021649in}}{\pgfqpoint{-0.029463in}{-0.029463in}}%
\pgfpathcurveto{\pgfqpoint{-0.021649in}{-0.037276in}}{\pgfqpoint{-0.011050in}{-0.041667in}}{\pgfqpoint{0.000000in}{-0.041667in}}%
\pgfpathclose%
\pgfusepath{stroke,fill}%
}%
\begin{pgfscope}%
\pgfsys@transformshift{0.921890in}{3.265069in}%
\pgfsys@useobject{currentmarker}{}%
\end{pgfscope}%
\begin{pgfscope}%
\pgfsys@transformshift{1.673405in}{3.674093in}%
\pgfsys@useobject{currentmarker}{}%
\end{pgfscope}%
\begin{pgfscope}%
\pgfsys@transformshift{2.424920in}{3.602637in}%
\pgfsys@useobject{currentmarker}{}%
\end{pgfscope}%
\begin{pgfscope}%
\pgfsys@transformshift{3.176435in}{3.417837in}%
\pgfsys@useobject{currentmarker}{}%
\end{pgfscope}%
\begin{pgfscope}%
\pgfsys@transformshift{3.927951in}{3.045773in}%
\pgfsys@useobject{currentmarker}{}%
\end{pgfscope}%
\begin{pgfscope}%
\pgfsys@transformshift{4.679466in}{3.039613in}%
\pgfsys@useobject{currentmarker}{}%
\end{pgfscope}%
\begin{pgfscope}%
\pgfsys@transformshift{5.430981in}{3.776349in}%
\pgfsys@useobject{currentmarker}{}%
\end{pgfscope}%
\end{pgfscope}%
\begin{pgfscope}%
\pgfpathrectangle{\pgfqpoint{0.696435in}{1.359165in}}{\pgfqpoint{4.960000in}{3.696000in}}%
\pgfusepath{clip}%
\pgfsetrectcap%
\pgfsetroundjoin%
\pgfsetlinewidth{1.505625pt}%
\definecolor{currentstroke}{rgb}{1.000000,0.498039,0.054902}%
\pgfsetstrokecolor{currentstroke}%
\pgfsetdash{}{0pt}%
\pgfpathmoveto{\pgfqpoint{0.921890in}{2.806765in}}%
\pgfpathlineto{\pgfqpoint{1.673405in}{3.403053in}}%
\pgfpathlineto{\pgfqpoint{2.424920in}{3.380877in}}%
\pgfpathlineto{\pgfqpoint{3.176435in}{3.419069in}}%
\pgfpathlineto{\pgfqpoint{3.927951in}{3.266301in}}%
\pgfpathlineto{\pgfqpoint{4.679466in}{3.410445in}}%
\pgfpathlineto{\pgfqpoint{5.430981in}{3.591549in}}%
\pgfusepath{stroke}%
\end{pgfscope}%
\begin{pgfscope}%
\pgfpathrectangle{\pgfqpoint{0.696435in}{1.359165in}}{\pgfqpoint{4.960000in}{3.696000in}}%
\pgfusepath{clip}%
\pgfsetbuttcap%
\pgfsetmiterjoin%
\definecolor{currentfill}{rgb}{1.000000,0.498039,0.054902}%
\pgfsetfillcolor{currentfill}%
\pgfsetlinewidth{1.003750pt}%
\definecolor{currentstroke}{rgb}{1.000000,0.498039,0.054902}%
\pgfsetstrokecolor{currentstroke}%
\pgfsetdash{}{0pt}%
\pgfsys@defobject{currentmarker}{\pgfqpoint{-0.041667in}{-0.041667in}}{\pgfqpoint{0.041667in}{0.041667in}}{%
\pgfpathmoveto{\pgfqpoint{-0.000000in}{-0.041667in}}%
\pgfpathlineto{\pgfqpoint{0.041667in}{0.041667in}}%
\pgfpathlineto{\pgfqpoint{-0.041667in}{0.041667in}}%
\pgfpathclose%
\pgfusepath{stroke,fill}%
}%
\begin{pgfscope}%
\pgfsys@transformshift{0.921890in}{2.806765in}%
\pgfsys@useobject{currentmarker}{}%
\end{pgfscope}%
\begin{pgfscope}%
\pgfsys@transformshift{1.673405in}{3.403053in}%
\pgfsys@useobject{currentmarker}{}%
\end{pgfscope}%
\begin{pgfscope}%
\pgfsys@transformshift{2.424920in}{3.380877in}%
\pgfsys@useobject{currentmarker}{}%
\end{pgfscope}%
\begin{pgfscope}%
\pgfsys@transformshift{3.176435in}{3.419069in}%
\pgfsys@useobject{currentmarker}{}%
\end{pgfscope}%
\begin{pgfscope}%
\pgfsys@transformshift{3.927951in}{3.266301in}%
\pgfsys@useobject{currentmarker}{}%
\end{pgfscope}%
\begin{pgfscope}%
\pgfsys@transformshift{4.679466in}{3.410445in}%
\pgfsys@useobject{currentmarker}{}%
\end{pgfscope}%
\begin{pgfscope}%
\pgfsys@transformshift{5.430981in}{3.591549in}%
\pgfsys@useobject{currentmarker}{}%
\end{pgfscope}%
\end{pgfscope}%
\begin{pgfscope}%
\pgfsetrectcap%
\pgfsetmiterjoin%
\pgfsetlinewidth{0.803000pt}%
\definecolor{currentstroke}{rgb}{0.000000,0.000000,0.000000}%
\pgfsetstrokecolor{currentstroke}%
\pgfsetdash{}{0pt}%
\pgfpathmoveto{\pgfqpoint{0.696435in}{1.359165in}}%
\pgfpathlineto{\pgfqpoint{0.696435in}{5.055165in}}%
\pgfusepath{stroke}%
\end{pgfscope}%
\begin{pgfscope}%
\pgfsetrectcap%
\pgfsetmiterjoin%
\pgfsetlinewidth{0.803000pt}%
\definecolor{currentstroke}{rgb}{0.000000,0.000000,0.000000}%
\pgfsetstrokecolor{currentstroke}%
\pgfsetdash{}{0pt}%
\pgfpathmoveto{\pgfqpoint{5.656435in}{1.359165in}}%
\pgfpathlineto{\pgfqpoint{5.656435in}{5.055165in}}%
\pgfusepath{stroke}%
\end{pgfscope}%
\begin{pgfscope}%
\pgfsetrectcap%
\pgfsetmiterjoin%
\pgfsetlinewidth{0.803000pt}%
\definecolor{currentstroke}{rgb}{0.000000,0.000000,0.000000}%
\pgfsetstrokecolor{currentstroke}%
\pgfsetdash{}{0pt}%
\pgfpathmoveto{\pgfqpoint{0.696435in}{1.359165in}}%
\pgfpathlineto{\pgfqpoint{5.656435in}{1.359165in}}%
\pgfusepath{stroke}%
\end{pgfscope}%
\begin{pgfscope}%
\pgfsetrectcap%
\pgfsetmiterjoin%
\pgfsetlinewidth{0.803000pt}%
\definecolor{currentstroke}{rgb}{0.000000,0.000000,0.000000}%
\pgfsetstrokecolor{currentstroke}%
\pgfsetdash{}{0pt}%
\pgfpathmoveto{\pgfqpoint{0.696435in}{5.055165in}}%
\pgfpathlineto{\pgfqpoint{5.656435in}{5.055165in}}%
\pgfusepath{stroke}%
\end{pgfscope}%
\begin{pgfscope}%
\pgfsetbuttcap%
\pgfsetmiterjoin%
\definecolor{currentfill}{rgb}{1.000000,1.000000,1.000000}%
\pgfsetfillcolor{currentfill}%
\pgfsetfillopacity{0.800000}%
\pgfsetlinewidth{1.003750pt}%
\definecolor{currentstroke}{rgb}{0.800000,0.800000,0.800000}%
\pgfsetstrokecolor{currentstroke}%
\pgfsetstrokeopacity{0.800000}%
\pgfsetdash{}{0pt}%
\pgfpathmoveto{\pgfqpoint{4.345048in}{4.536339in}}%
\pgfpathlineto{\pgfqpoint{5.559213in}{4.536339in}}%
\pgfpathquadraticcurveto{\pgfqpoint{5.586991in}{4.536339in}}{\pgfqpoint{5.586991in}{4.564117in}}%
\pgfpathlineto{\pgfqpoint{5.586991in}{4.957943in}}%
\pgfpathquadraticcurveto{\pgfqpoint{5.586991in}{4.985720in}}{\pgfqpoint{5.559213in}{4.985720in}}%
\pgfpathlineto{\pgfqpoint{4.345048in}{4.985720in}}%
\pgfpathquadraticcurveto{\pgfqpoint{4.317270in}{4.985720in}}{\pgfqpoint{4.317270in}{4.957943in}}%
\pgfpathlineto{\pgfqpoint{4.317270in}{4.564117in}}%
\pgfpathquadraticcurveto{\pgfqpoint{4.317270in}{4.536339in}}{\pgfqpoint{4.345048in}{4.536339in}}%
\pgfpathclose%
\pgfusepath{stroke,fill}%
\end{pgfscope}%
\begin{pgfscope}%
\pgfsetrectcap%
\pgfsetroundjoin%
\pgfsetlinewidth{1.505625pt}%
\definecolor{currentstroke}{rgb}{0.121569,0.466667,0.705882}%
\pgfsetstrokecolor{currentstroke}%
\pgfsetdash{}{0pt}%
\pgfpathmoveto{\pgfqpoint{4.372825in}{4.873253in}}%
\pgfpathlineto{\pgfqpoint{4.650603in}{4.873253in}}%
\pgfusepath{stroke}%
\end{pgfscope}%
\begin{pgfscope}%
\pgfsetbuttcap%
\pgfsetroundjoin%
\definecolor{currentfill}{rgb}{0.121569,0.466667,0.705882}%
\pgfsetfillcolor{currentfill}%
\pgfsetlinewidth{1.003750pt}%
\definecolor{currentstroke}{rgb}{0.121569,0.466667,0.705882}%
\pgfsetstrokecolor{currentstroke}%
\pgfsetdash{}{0pt}%
\pgfsys@defobject{currentmarker}{\pgfqpoint{-0.041667in}{-0.041667in}}{\pgfqpoint{0.041667in}{0.041667in}}{%
\pgfpathmoveto{\pgfqpoint{0.000000in}{-0.041667in}}%
\pgfpathcurveto{\pgfqpoint{0.011050in}{-0.041667in}}{\pgfqpoint{0.021649in}{-0.037276in}}{\pgfqpoint{0.029463in}{-0.029463in}}%
\pgfpathcurveto{\pgfqpoint{0.037276in}{-0.021649in}}{\pgfqpoint{0.041667in}{-0.011050in}}{\pgfqpoint{0.041667in}{0.000000in}}%
\pgfpathcurveto{\pgfqpoint{0.041667in}{0.011050in}}{\pgfqpoint{0.037276in}{0.021649in}}{\pgfqpoint{0.029463in}{0.029463in}}%
\pgfpathcurveto{\pgfqpoint{0.021649in}{0.037276in}}{\pgfqpoint{0.011050in}{0.041667in}}{\pgfqpoint{0.000000in}{0.041667in}}%
\pgfpathcurveto{\pgfqpoint{-0.011050in}{0.041667in}}{\pgfqpoint{-0.021649in}{0.037276in}}{\pgfqpoint{-0.029463in}{0.029463in}}%
\pgfpathcurveto{\pgfqpoint{-0.037276in}{0.021649in}}{\pgfqpoint{-0.041667in}{0.011050in}}{\pgfqpoint{-0.041667in}{0.000000in}}%
\pgfpathcurveto{\pgfqpoint{-0.041667in}{-0.011050in}}{\pgfqpoint{-0.037276in}{-0.021649in}}{\pgfqpoint{-0.029463in}{-0.029463in}}%
\pgfpathcurveto{\pgfqpoint{-0.021649in}{-0.037276in}}{\pgfqpoint{-0.011050in}{-0.041667in}}{\pgfqpoint{0.000000in}{-0.041667in}}%
\pgfpathclose%
\pgfusepath{stroke,fill}%
}%
\begin{pgfscope}%
\pgfsys@transformshift{4.511714in}{4.873253in}%
\pgfsys@useobject{currentmarker}{}%
\end{pgfscope}%
\end{pgfscope}%
\begin{pgfscope}%
\definecolor{textcolor}{rgb}{0.000000,0.000000,0.000000}%
\pgfsetstrokecolor{textcolor}%
\pgfsetfillcolor{textcolor}%
\pgftext[x=4.761714in,y=4.824642in,left,base]{\color{textcolor}\sffamily\fontsize{10.000000}{12.000000}\selectfont Pix2Vox++}%
\end{pgfscope}%
\begin{pgfscope}%
\pgfsetrectcap%
\pgfsetroundjoin%
\pgfsetlinewidth{1.505625pt}%
\definecolor{currentstroke}{rgb}{1.000000,0.498039,0.054902}%
\pgfsetstrokecolor{currentstroke}%
\pgfsetdash{}{0pt}%
\pgfpathmoveto{\pgfqpoint{4.372825in}{4.669396in}}%
\pgfpathlineto{\pgfqpoint{4.650603in}{4.669396in}}%
\pgfusepath{stroke}%
\end{pgfscope}%
\begin{pgfscope}%
\pgfsetbuttcap%
\pgfsetmiterjoin%
\definecolor{currentfill}{rgb}{1.000000,0.498039,0.054902}%
\pgfsetfillcolor{currentfill}%
\pgfsetlinewidth{1.003750pt}%
\definecolor{currentstroke}{rgb}{1.000000,0.498039,0.054902}%
\pgfsetstrokecolor{currentstroke}%
\pgfsetdash{}{0pt}%
\pgfsys@defobject{currentmarker}{\pgfqpoint{-0.041667in}{-0.041667in}}{\pgfqpoint{0.041667in}{0.041667in}}{%
\pgfpathmoveto{\pgfqpoint{-0.000000in}{-0.041667in}}%
\pgfpathlineto{\pgfqpoint{0.041667in}{0.041667in}}%
\pgfpathlineto{\pgfqpoint{-0.041667in}{0.041667in}}%
\pgfpathclose%
\pgfusepath{stroke,fill}%
}%
\begin{pgfscope}%
\pgfsys@transformshift{4.511714in}{4.669396in}%
\pgfsys@useobject{currentmarker}{}%
\end{pgfscope}%
\end{pgfscope}%
\begin{pgfscope}%
\definecolor{textcolor}{rgb}{0.000000,0.000000,0.000000}%
\pgfsetstrokecolor{textcolor}%
\pgfsetfillcolor{textcolor}%
\pgftext[x=4.761714in,y=4.620785in,left,base]{\color{textcolor}\sffamily\fontsize{10.000000}{12.000000}\selectfont Pix2Vox}%
\end{pgfscope}%
\end{pgfpicture}%
\makeatother%
\endgroup%
}
%    \caption{Line plot for the \gls{f1}  for baseline trained by mixing chair dataset from real and synthetic dataset with ratio of 50\%.
%    Observe that the \fls{IoU} is consistent for all types of randomization proving that mixed training negates loss from randomization.}
%    \label{fig:ablation_dice2}
%\end{figure}
\begin{figure}[ht]
    \centering
    \resizebox{0.7\textwidth}{!}{%% Creator: Matplotlib, PGF backend
%%
%% To include the figure in your LaTeX document, write
%%   \input{<filename>.pgf}
%%
%% Make sure the required packages are loaded in your preamble
%%   \usepackage{pgf}
%%
%% Figures using additional raster images can only be included by \input if
%% they are in the same directory as the main LaTeX file. For loading figures
%% from other directories you can use the `import` package
%%   \usepackage{import}
%%
%% and then include the figures with
%%   \import{<path to file>}{<filename>.pgf}
%%
%% Matplotlib used the following preamble
%%   \usepackage{fontspec}
%%   \setmainfont{DejaVuSerif.ttf}[Path=\detokenize{/Users/apple/opt/anaconda3/envs/kaolin/lib/python3.7/site-packages/matplotlib/mpl-data/fonts/ttf/}]
%%   \setsansfont{DejaVuSans.ttf}[Path=\detokenize{/Users/apple/opt/anaconda3/envs/kaolin/lib/python3.7/site-packages/matplotlib/mpl-data/fonts/ttf/}]
%%   \setmonofont{DejaVuSansMono.ttf}[Path=\detokenize{/Users/apple/opt/anaconda3/envs/kaolin/lib/python3.7/site-packages/matplotlib/mpl-data/fonts/ttf/}]
%%
\begingroup%
\makeatletter%
\begin{pgfpicture}%
\pgfpathrectangle{\pgfpointorigin}{\pgfqpoint{6.293658in}{4.696324in}}%
\pgfusepath{use as bounding box, clip}%
\begin{pgfscope}%
\pgfsetbuttcap%
\pgfsetmiterjoin%
\definecolor{currentfill}{rgb}{1.000000,1.000000,1.000000}%
\pgfsetfillcolor{currentfill}%
\pgfsetlinewidth{0.000000pt}%
\definecolor{currentstroke}{rgb}{1.000000,1.000000,1.000000}%
\pgfsetstrokecolor{currentstroke}%
\pgfsetdash{}{0pt}%
\pgfpathmoveto{\pgfqpoint{0.000000in}{0.000000in}}%
\pgfpathlineto{\pgfqpoint{6.293658in}{0.000000in}}%
\pgfpathlineto{\pgfqpoint{6.293658in}{4.696324in}}%
\pgfpathlineto{\pgfqpoint{0.000000in}{4.696324in}}%
\pgfpathclose%
\pgfusepath{fill}%
\end{pgfscope}%
\begin{pgfscope}%
\pgfsetbuttcap%
\pgfsetmiterjoin%
\definecolor{currentfill}{rgb}{1.000000,1.000000,1.000000}%
\pgfsetfillcolor{currentfill}%
\pgfsetlinewidth{0.000000pt}%
\definecolor{currentstroke}{rgb}{0.000000,0.000000,0.000000}%
\pgfsetstrokecolor{currentstroke}%
\pgfsetstrokeopacity{0.000000}%
\pgfsetdash{}{0pt}%
\pgfpathmoveto{\pgfqpoint{0.696435in}{1.359165in}}%
\pgfpathlineto{\pgfqpoint{6.193658in}{1.359165in}}%
\pgfpathlineto{\pgfqpoint{6.193658in}{4.386363in}}%
\pgfpathlineto{\pgfqpoint{0.696435in}{4.386363in}}%
\pgfpathclose%
\pgfusepath{fill}%
\end{pgfscope}%
\begin{pgfscope}%
\pgfpathrectangle{\pgfqpoint{0.696435in}{1.359165in}}{\pgfqpoint{5.497222in}{3.027198in}}%
\pgfusepath{clip}%
\pgfsetbuttcap%
\pgfsetmiterjoin%
\definecolor{currentfill}{rgb}{0.121569,0.466667,0.705882}%
\pgfsetfillcolor{currentfill}%
\pgfsetlinewidth{0.000000pt}%
\definecolor{currentstroke}{rgb}{0.000000,0.000000,0.000000}%
\pgfsetstrokecolor{currentstroke}%
\pgfsetstrokeopacity{0.000000}%
\pgfsetdash{}{0pt}%
\pgfpathmoveto{\pgfqpoint{0.946309in}{-1.668033in}}%
\pgfpathlineto{\pgfqpoint{1.272231in}{-1.668033in}}%
\pgfpathlineto{\pgfqpoint{1.272231in}{2.920190in}}%
\pgfpathlineto{\pgfqpoint{0.946309in}{2.920190in}}%
\pgfpathclose%
\pgfusepath{fill}%
\end{pgfscope}%
\begin{pgfscope}%
\pgfpathrectangle{\pgfqpoint{0.696435in}{1.359165in}}{\pgfqpoint{5.497222in}{3.027198in}}%
\pgfusepath{clip}%
\pgfsetbuttcap%
\pgfsetmiterjoin%
\definecolor{currentfill}{rgb}{0.121569,0.466667,0.705882}%
\pgfsetfillcolor{currentfill}%
\pgfsetlinewidth{0.000000pt}%
\definecolor{currentstroke}{rgb}{0.000000,0.000000,0.000000}%
\pgfsetstrokecolor{currentstroke}%
\pgfsetstrokeopacity{0.000000}%
\pgfsetdash{}{0pt}%
\pgfpathmoveto{\pgfqpoint{1.670581in}{-1.668033in}}%
\pgfpathlineto{\pgfqpoint{1.996503in}{-1.668033in}}%
\pgfpathlineto{\pgfqpoint{1.996503in}{3.255200in}}%
\pgfpathlineto{\pgfqpoint{1.670581in}{3.255200in}}%
\pgfpathclose%
\pgfusepath{fill}%
\end{pgfscope}%
\begin{pgfscope}%
\pgfpathrectangle{\pgfqpoint{0.696435in}{1.359165in}}{\pgfqpoint{5.497222in}{3.027198in}}%
\pgfusepath{clip}%
\pgfsetbuttcap%
\pgfsetmiterjoin%
\definecolor{currentfill}{rgb}{0.121569,0.466667,0.705882}%
\pgfsetfillcolor{currentfill}%
\pgfsetlinewidth{0.000000pt}%
\definecolor{currentstroke}{rgb}{0.000000,0.000000,0.000000}%
\pgfsetstrokecolor{currentstroke}%
\pgfsetstrokeopacity{0.000000}%
\pgfsetdash{}{0pt}%
\pgfpathmoveto{\pgfqpoint{2.394853in}{-1.668033in}}%
\pgfpathlineto{\pgfqpoint{2.720775in}{-1.668033in}}%
\pgfpathlineto{\pgfqpoint{2.720775in}{3.196674in}}%
\pgfpathlineto{\pgfqpoint{2.394853in}{3.196674in}}%
\pgfpathclose%
\pgfusepath{fill}%
\end{pgfscope}%
\begin{pgfscope}%
\pgfpathrectangle{\pgfqpoint{0.696435in}{1.359165in}}{\pgfqpoint{5.497222in}{3.027198in}}%
\pgfusepath{clip}%
\pgfsetbuttcap%
\pgfsetmiterjoin%
\definecolor{currentfill}{rgb}{0.121569,0.466667,0.705882}%
\pgfsetfillcolor{currentfill}%
\pgfsetlinewidth{0.000000pt}%
\definecolor{currentstroke}{rgb}{0.000000,0.000000,0.000000}%
\pgfsetstrokecolor{currentstroke}%
\pgfsetstrokeopacity{0.000000}%
\pgfsetdash{}{0pt}%
\pgfpathmoveto{\pgfqpoint{3.119124in}{-1.668033in}}%
\pgfpathlineto{\pgfqpoint{3.445047in}{-1.668033in}}%
\pgfpathlineto{\pgfqpoint{3.445047in}{3.045314in}}%
\pgfpathlineto{\pgfqpoint{3.119124in}{3.045314in}}%
\pgfpathclose%
\pgfusepath{fill}%
\end{pgfscope}%
\begin{pgfscope}%
\pgfpathrectangle{\pgfqpoint{0.696435in}{1.359165in}}{\pgfqpoint{5.497222in}{3.027198in}}%
\pgfusepath{clip}%
\pgfsetbuttcap%
\pgfsetmiterjoin%
\definecolor{currentfill}{rgb}{0.121569,0.466667,0.705882}%
\pgfsetfillcolor{currentfill}%
\pgfsetlinewidth{0.000000pt}%
\definecolor{currentstroke}{rgb}{0.000000,0.000000,0.000000}%
\pgfsetstrokecolor{currentstroke}%
\pgfsetstrokeopacity{0.000000}%
\pgfsetdash{}{0pt}%
\pgfpathmoveto{\pgfqpoint{3.843396in}{-1.668033in}}%
\pgfpathlineto{\pgfqpoint{4.169318in}{-1.668033in}}%
\pgfpathlineto{\pgfqpoint{4.169318in}{2.740576in}}%
\pgfpathlineto{\pgfqpoint{3.843396in}{2.740576in}}%
\pgfpathclose%
\pgfusepath{fill}%
\end{pgfscope}%
\begin{pgfscope}%
\pgfpathrectangle{\pgfqpoint{0.696435in}{1.359165in}}{\pgfqpoint{5.497222in}{3.027198in}}%
\pgfusepath{clip}%
\pgfsetbuttcap%
\pgfsetmiterjoin%
\definecolor{currentfill}{rgb}{0.121569,0.466667,0.705882}%
\pgfsetfillcolor{currentfill}%
\pgfsetlinewidth{0.000000pt}%
\definecolor{currentstroke}{rgb}{0.000000,0.000000,0.000000}%
\pgfsetstrokecolor{currentstroke}%
\pgfsetstrokeopacity{0.000000}%
\pgfsetdash{}{0pt}%
\pgfpathmoveto{\pgfqpoint{4.567668in}{-1.668033in}}%
\pgfpathlineto{\pgfqpoint{4.893590in}{-1.668033in}}%
\pgfpathlineto{\pgfqpoint{4.893590in}{2.735531in}}%
\pgfpathlineto{\pgfqpoint{4.567668in}{2.735531in}}%
\pgfpathclose%
\pgfusepath{fill}%
\end{pgfscope}%
\begin{pgfscope}%
\pgfpathrectangle{\pgfqpoint{0.696435in}{1.359165in}}{\pgfqpoint{5.497222in}{3.027198in}}%
\pgfusepath{clip}%
\pgfsetbuttcap%
\pgfsetmiterjoin%
\definecolor{currentfill}{rgb}{0.121569,0.466667,0.705882}%
\pgfsetfillcolor{currentfill}%
\pgfsetlinewidth{0.000000pt}%
\definecolor{currentstroke}{rgb}{0.000000,0.000000,0.000000}%
\pgfsetstrokecolor{currentstroke}%
\pgfsetstrokeopacity{0.000000}%
\pgfsetdash{}{0pt}%
\pgfpathmoveto{\pgfqpoint{5.291939in}{-1.668033in}}%
\pgfpathlineto{\pgfqpoint{5.617862in}{-1.668033in}}%
\pgfpathlineto{\pgfqpoint{5.617862in}{3.338952in}}%
\pgfpathlineto{\pgfqpoint{5.291939in}{3.338952in}}%
\pgfpathclose%
\pgfusepath{fill}%
\end{pgfscope}%
\begin{pgfscope}%
\pgfpathrectangle{\pgfqpoint{0.696435in}{1.359165in}}{\pgfqpoint{5.497222in}{3.027198in}}%
\pgfusepath{clip}%
\pgfsetbuttcap%
\pgfsetmiterjoin%
\definecolor{currentfill}{rgb}{1.000000,0.498039,0.054902}%
\pgfsetfillcolor{currentfill}%
\pgfsetlinewidth{0.000000pt}%
\definecolor{currentstroke}{rgb}{0.000000,0.000000,0.000000}%
\pgfsetstrokecolor{currentstroke}%
\pgfsetstrokeopacity{0.000000}%
\pgfsetdash{}{0pt}%
\pgfpathmoveto{\pgfqpoint{1.272231in}{-1.668033in}}%
\pgfpathlineto{\pgfqpoint{1.598154in}{-1.668033in}}%
\pgfpathlineto{\pgfqpoint{1.598154in}{2.544817in}}%
\pgfpathlineto{\pgfqpoint{1.272231in}{2.544817in}}%
\pgfpathclose%
\pgfusepath{fill}%
\end{pgfscope}%
\begin{pgfscope}%
\pgfpathrectangle{\pgfqpoint{0.696435in}{1.359165in}}{\pgfqpoint{5.497222in}{3.027198in}}%
\pgfusepath{clip}%
\pgfsetbuttcap%
\pgfsetmiterjoin%
\definecolor{currentfill}{rgb}{1.000000,0.498039,0.054902}%
\pgfsetfillcolor{currentfill}%
\pgfsetlinewidth{0.000000pt}%
\definecolor{currentstroke}{rgb}{0.000000,0.000000,0.000000}%
\pgfsetstrokecolor{currentstroke}%
\pgfsetstrokeopacity{0.000000}%
\pgfsetdash{}{0pt}%
\pgfpathmoveto{\pgfqpoint{1.996503in}{-1.668033in}}%
\pgfpathlineto{\pgfqpoint{2.322425in}{-1.668033in}}%
\pgfpathlineto{\pgfqpoint{2.322425in}{3.033205in}}%
\pgfpathlineto{\pgfqpoint{1.996503in}{3.033205in}}%
\pgfpathclose%
\pgfusepath{fill}%
\end{pgfscope}%
\begin{pgfscope}%
\pgfpathrectangle{\pgfqpoint{0.696435in}{1.359165in}}{\pgfqpoint{5.497222in}{3.027198in}}%
\pgfusepath{clip}%
\pgfsetbuttcap%
\pgfsetmiterjoin%
\definecolor{currentfill}{rgb}{1.000000,0.498039,0.054902}%
\pgfsetfillcolor{currentfill}%
\pgfsetlinewidth{0.000000pt}%
\definecolor{currentstroke}{rgb}{0.000000,0.000000,0.000000}%
\pgfsetstrokecolor{currentstroke}%
\pgfsetstrokeopacity{0.000000}%
\pgfsetdash{}{0pt}%
\pgfpathmoveto{\pgfqpoint{2.720775in}{-1.668033in}}%
\pgfpathlineto{\pgfqpoint{3.046697in}{-1.668033in}}%
\pgfpathlineto{\pgfqpoint{3.046697in}{3.015042in}}%
\pgfpathlineto{\pgfqpoint{2.720775in}{3.015042in}}%
\pgfpathclose%
\pgfusepath{fill}%
\end{pgfscope}%
\begin{pgfscope}%
\pgfpathrectangle{\pgfqpoint{0.696435in}{1.359165in}}{\pgfqpoint{5.497222in}{3.027198in}}%
\pgfusepath{clip}%
\pgfsetbuttcap%
\pgfsetmiterjoin%
\definecolor{currentfill}{rgb}{1.000000,0.498039,0.054902}%
\pgfsetfillcolor{currentfill}%
\pgfsetlinewidth{0.000000pt}%
\definecolor{currentstroke}{rgb}{0.000000,0.000000,0.000000}%
\pgfsetstrokecolor{currentstroke}%
\pgfsetstrokeopacity{0.000000}%
\pgfsetdash{}{0pt}%
\pgfpathmoveto{\pgfqpoint{3.445047in}{-1.668033in}}%
\pgfpathlineto{\pgfqpoint{3.770969in}{-1.668033in}}%
\pgfpathlineto{\pgfqpoint{3.770969in}{3.046323in}}%
\pgfpathlineto{\pgfqpoint{3.445047in}{3.046323in}}%
\pgfpathclose%
\pgfusepath{fill}%
\end{pgfscope}%
\begin{pgfscope}%
\pgfpathrectangle{\pgfqpoint{0.696435in}{1.359165in}}{\pgfqpoint{5.497222in}{3.027198in}}%
\pgfusepath{clip}%
\pgfsetbuttcap%
\pgfsetmiterjoin%
\definecolor{currentfill}{rgb}{1.000000,0.498039,0.054902}%
\pgfsetfillcolor{currentfill}%
\pgfsetlinewidth{0.000000pt}%
\definecolor{currentstroke}{rgb}{0.000000,0.000000,0.000000}%
\pgfsetstrokecolor{currentstroke}%
\pgfsetstrokeopacity{0.000000}%
\pgfsetdash{}{0pt}%
\pgfpathmoveto{\pgfqpoint{4.169318in}{-1.668033in}}%
\pgfpathlineto{\pgfqpoint{4.495240in}{-1.668033in}}%
\pgfpathlineto{\pgfqpoint{4.495240in}{2.921199in}}%
\pgfpathlineto{\pgfqpoint{4.169318in}{2.921199in}}%
\pgfpathclose%
\pgfusepath{fill}%
\end{pgfscope}%
\begin{pgfscope}%
\pgfpathrectangle{\pgfqpoint{0.696435in}{1.359165in}}{\pgfqpoint{5.497222in}{3.027198in}}%
\pgfusepath{clip}%
\pgfsetbuttcap%
\pgfsetmiterjoin%
\definecolor{currentfill}{rgb}{1.000000,0.498039,0.054902}%
\pgfsetfillcolor{currentfill}%
\pgfsetlinewidth{0.000000pt}%
\definecolor{currentstroke}{rgb}{0.000000,0.000000,0.000000}%
\pgfsetstrokecolor{currentstroke}%
\pgfsetstrokeopacity{0.000000}%
\pgfsetdash{}{0pt}%
\pgfpathmoveto{\pgfqpoint{4.893590in}{-1.668033in}}%
\pgfpathlineto{\pgfqpoint{5.219512in}{-1.668033in}}%
\pgfpathlineto{\pgfqpoint{5.219512in}{3.039260in}}%
\pgfpathlineto{\pgfqpoint{4.893590in}{3.039260in}}%
\pgfpathclose%
\pgfusepath{fill}%
\end{pgfscope}%
\begin{pgfscope}%
\pgfpathrectangle{\pgfqpoint{0.696435in}{1.359165in}}{\pgfqpoint{5.497222in}{3.027198in}}%
\pgfusepath{clip}%
\pgfsetbuttcap%
\pgfsetmiterjoin%
\definecolor{currentfill}{rgb}{1.000000,0.498039,0.054902}%
\pgfsetfillcolor{currentfill}%
\pgfsetlinewidth{0.000000pt}%
\definecolor{currentstroke}{rgb}{0.000000,0.000000,0.000000}%
\pgfsetstrokecolor{currentstroke}%
\pgfsetstrokeopacity{0.000000}%
\pgfsetdash{}{0pt}%
\pgfpathmoveto{\pgfqpoint{5.617862in}{-1.668033in}}%
\pgfpathlineto{\pgfqpoint{5.943784in}{-1.668033in}}%
\pgfpathlineto{\pgfqpoint{5.943784in}{3.187592in}}%
\pgfpathlineto{\pgfqpoint{5.617862in}{3.187592in}}%
\pgfpathclose%
\pgfusepath{fill}%
\end{pgfscope}%
\begin{pgfscope}%
\pgfsetbuttcap%
\pgfsetroundjoin%
\definecolor{currentfill}{rgb}{0.000000,0.000000,0.000000}%
\pgfsetfillcolor{currentfill}%
\pgfsetlinewidth{0.803000pt}%
\definecolor{currentstroke}{rgb}{0.000000,0.000000,0.000000}%
\pgfsetstrokecolor{currentstroke}%
\pgfsetdash{}{0pt}%
\pgfsys@defobject{currentmarker}{\pgfqpoint{0.000000in}{-0.048611in}}{\pgfqpoint{0.000000in}{0.000000in}}{%
\pgfpathmoveto{\pgfqpoint{0.000000in}{0.000000in}}%
\pgfpathlineto{\pgfqpoint{0.000000in}{-0.048611in}}%
\pgfusepath{stroke,fill}%
}%
\begin{pgfscope}%
\pgfsys@transformshift{1.272231in}{1.359165in}%
\pgfsys@useobject{currentmarker}{}%
\end{pgfscope}%
\end{pgfscope}%
\begin{pgfscope}%
\definecolor{textcolor}{rgb}{0.000000,0.000000,0.000000}%
\pgfsetstrokecolor{textcolor}%
\pgfsetfillcolor{textcolor}%
\pgftext[x=1.005321in, y=0.599318in, left, base,rotate=45.000000]{\color{textcolor}\sffamily\fontsize{10.000000}{12.000000}\selectfont Pix3d(chair)}%
\end{pgfscope}%
\begin{pgfscope}%
\pgfsetbuttcap%
\pgfsetroundjoin%
\definecolor{currentfill}{rgb}{0.000000,0.000000,0.000000}%
\pgfsetfillcolor{currentfill}%
\pgfsetlinewidth{0.803000pt}%
\definecolor{currentstroke}{rgb}{0.000000,0.000000,0.000000}%
\pgfsetstrokecolor{currentstroke}%
\pgfsetdash{}{0pt}%
\pgfsys@defobject{currentmarker}{\pgfqpoint{0.000000in}{-0.048611in}}{\pgfqpoint{0.000000in}{0.000000in}}{%
\pgfpathmoveto{\pgfqpoint{0.000000in}{0.000000in}}%
\pgfpathlineto{\pgfqpoint{0.000000in}{-0.048611in}}%
\pgfusepath{stroke,fill}%
}%
\begin{pgfscope}%
\pgfsys@transformshift{1.996503in}{1.359165in}%
\pgfsys@useobject{currentmarker}{}%
\end{pgfscope}%
\end{pgfscope}%
\begin{pgfscope}%
\definecolor{textcolor}{rgb}{0.000000,0.000000,0.000000}%
\pgfsetstrokecolor{textcolor}%
\pgfsetfillcolor{textcolor}%
\pgftext[x=1.748822in, y=0.637777in, left, base,rotate=45.000000]{\color{textcolor}\sffamily\fontsize{10.000000}{12.000000}\selectfont Textureless}%
\end{pgfscope}%
\begin{pgfscope}%
\pgfsetbuttcap%
\pgfsetroundjoin%
\definecolor{currentfill}{rgb}{0.000000,0.000000,0.000000}%
\pgfsetfillcolor{currentfill}%
\pgfsetlinewidth{0.803000pt}%
\definecolor{currentstroke}{rgb}{0.000000,0.000000,0.000000}%
\pgfsetstrokecolor{currentstroke}%
\pgfsetdash{}{0pt}%
\pgfsys@defobject{currentmarker}{\pgfqpoint{0.000000in}{-0.048611in}}{\pgfqpoint{0.000000in}{0.000000in}}{%
\pgfpathmoveto{\pgfqpoint{0.000000in}{0.000000in}}%
\pgfpathlineto{\pgfqpoint{0.000000in}{-0.048611in}}%
\pgfusepath{stroke,fill}%
}%
\begin{pgfscope}%
\pgfsys@transformshift{2.720775in}{1.359165in}%
\pgfsys@useobject{currentmarker}{}%
\end{pgfscope}%
\end{pgfscope}%
\begin{pgfscope}%
\definecolor{textcolor}{rgb}{0.000000,0.000000,0.000000}%
\pgfsetstrokecolor{textcolor}%
\pgfsetfillcolor{textcolor}%
\pgftext[x=2.309404in, y=0.310396in, left, base,rotate=45.000000]{\color{textcolor}\sffamily\fontsize{10.000000}{12.000000}\selectfont Textureless+Light}%
\end{pgfscope}%
\begin{pgfscope}%
\pgfsetbuttcap%
\pgfsetroundjoin%
\definecolor{currentfill}{rgb}{0.000000,0.000000,0.000000}%
\pgfsetfillcolor{currentfill}%
\pgfsetlinewidth{0.803000pt}%
\definecolor{currentstroke}{rgb}{0.000000,0.000000,0.000000}%
\pgfsetstrokecolor{currentstroke}%
\pgfsetdash{}{0pt}%
\pgfsys@defobject{currentmarker}{\pgfqpoint{0.000000in}{-0.048611in}}{\pgfqpoint{0.000000in}{0.000000in}}{%
\pgfpathmoveto{\pgfqpoint{0.000000in}{0.000000in}}%
\pgfpathlineto{\pgfqpoint{0.000000in}{-0.048611in}}%
\pgfusepath{stroke,fill}%
}%
\begin{pgfscope}%
\pgfsys@transformshift{3.445047in}{1.359165in}%
\pgfsys@useobject{currentmarker}{}%
\end{pgfscope}%
\end{pgfscope}%
\begin{pgfscope}%
\definecolor{textcolor}{rgb}{0.000000,0.000000,0.000000}%
\pgfsetstrokecolor{textcolor}%
\pgfsetfillcolor{textcolor}%
\pgftext[x=3.261216in, y=0.765478in, left, base,rotate=45.000000]{\color{textcolor}\sffamily\fontsize{10.000000}{12.000000}\selectfont Textured}%
\end{pgfscope}%
\begin{pgfscope}%
\pgfsetbuttcap%
\pgfsetroundjoin%
\definecolor{currentfill}{rgb}{0.000000,0.000000,0.000000}%
\pgfsetfillcolor{currentfill}%
\pgfsetlinewidth{0.803000pt}%
\definecolor{currentstroke}{rgb}{0.000000,0.000000,0.000000}%
\pgfsetstrokecolor{currentstroke}%
\pgfsetdash{}{0pt}%
\pgfsys@defobject{currentmarker}{\pgfqpoint{0.000000in}{-0.048611in}}{\pgfqpoint{0.000000in}{0.000000in}}{%
\pgfpathmoveto{\pgfqpoint{0.000000in}{0.000000in}}%
\pgfpathlineto{\pgfqpoint{0.000000in}{-0.048611in}}%
\pgfusepath{stroke,fill}%
}%
\begin{pgfscope}%
\pgfsys@transformshift{4.169318in}{1.359165in}%
\pgfsys@useobject{currentmarker}{}%
\end{pgfscope}%
\end{pgfscope}%
\begin{pgfscope}%
\definecolor{textcolor}{rgb}{0.000000,0.000000,0.000000}%
\pgfsetstrokecolor{textcolor}%
\pgfsetfillcolor{textcolor}%
\pgftext[x=3.821798in, y=0.438097in, left, base,rotate=45.000000]{\color{textcolor}\sffamily\fontsize{10.000000}{12.000000}\selectfont Textured+Light}%
\end{pgfscope}%
\begin{pgfscope}%
\pgfsetbuttcap%
\pgfsetroundjoin%
\definecolor{currentfill}{rgb}{0.000000,0.000000,0.000000}%
\pgfsetfillcolor{currentfill}%
\pgfsetlinewidth{0.803000pt}%
\definecolor{currentstroke}{rgb}{0.000000,0.000000,0.000000}%
\pgfsetstrokecolor{currentstroke}%
\pgfsetdash{}{0pt}%
\pgfsys@defobject{currentmarker}{\pgfqpoint{0.000000in}{-0.048611in}}{\pgfqpoint{0.000000in}{0.000000in}}{%
\pgfpathmoveto{\pgfqpoint{0.000000in}{0.000000in}}%
\pgfpathlineto{\pgfqpoint{0.000000in}{-0.048611in}}%
\pgfusepath{stroke,fill}%
}%
\begin{pgfscope}%
\pgfsys@transformshift{4.893590in}{1.359165in}%
\pgfsys@useobject{currentmarker}{}%
\end{pgfscope}%
\end{pgfscope}%
\begin{pgfscope}%
\definecolor{textcolor}{rgb}{0.000000,0.000000,0.000000}%
\pgfsetstrokecolor{textcolor}%
\pgfsetfillcolor{textcolor}%
\pgftext[x=4.621644in, y=0.589247in, left, base,rotate=45.000000]{\color{textcolor}\sffamily\fontsize{10.000000}{12.000000}\selectfont Multi-Object}%
\end{pgfscope}%
\begin{pgfscope}%
\pgfsetbuttcap%
\pgfsetroundjoin%
\definecolor{currentfill}{rgb}{0.000000,0.000000,0.000000}%
\pgfsetfillcolor{currentfill}%
\pgfsetlinewidth{0.803000pt}%
\definecolor{currentstroke}{rgb}{0.000000,0.000000,0.000000}%
\pgfsetstrokecolor{currentstroke}%
\pgfsetdash{}{0pt}%
\pgfsys@defobject{currentmarker}{\pgfqpoint{0.000000in}{-0.048611in}}{\pgfqpoint{0.000000in}{0.000000in}}{%
\pgfpathmoveto{\pgfqpoint{0.000000in}{0.000000in}}%
\pgfpathlineto{\pgfqpoint{0.000000in}{-0.048611in}}%
\pgfusepath{stroke,fill}%
}%
\begin{pgfscope}%
\pgfsys@transformshift{5.617862in}{1.359165in}%
\pgfsys@useobject{currentmarker}{}%
\end{pgfscope}%
\end{pgfscope}%
\begin{pgfscope}%
\definecolor{textcolor}{rgb}{0.000000,0.000000,0.000000}%
\pgfsetstrokecolor{textcolor}%
\pgfsetfillcolor{textcolor}%
\pgftext[x=5.395476in, y=0.688368in, left, base,rotate=45.000000]{\color{textcolor}\sffamily\fontsize{10.000000}{12.000000}\selectfont Combined}%
\end{pgfscope}%
\begin{pgfscope}%
\definecolor{textcolor}{rgb}{0.000000,0.000000,0.000000}%
\pgfsetstrokecolor{textcolor}%
\pgfsetfillcolor{textcolor}%
\pgftext[x=3.445047in,y=0.234413in,,top]{\color{textcolor}\sffamily\fontsize{10.000000}{12.000000}\bfseries\selectfont Dataset}%
\end{pgfscope}%
\begin{pgfscope}%
\pgfsetbuttcap%
\pgfsetroundjoin%
\definecolor{currentfill}{rgb}{0.000000,0.000000,0.000000}%
\pgfsetfillcolor{currentfill}%
\pgfsetlinewidth{0.803000pt}%
\definecolor{currentstroke}{rgb}{0.000000,0.000000,0.000000}%
\pgfsetstrokecolor{currentstroke}%
\pgfsetdash{}{0pt}%
\pgfsys@defobject{currentmarker}{\pgfqpoint{-0.048611in}{0.000000in}}{\pgfqpoint{-0.000000in}{0.000000in}}{%
\pgfpathmoveto{\pgfqpoint{-0.000000in}{0.000000in}}%
\pgfpathlineto{\pgfqpoint{-0.048611in}{0.000000in}}%
\pgfusepath{stroke,fill}%
}%
\begin{pgfscope}%
\pgfsys@transformshift{0.696435in}{1.359165in}%
\pgfsys@useobject{currentmarker}{}%
\end{pgfscope}%
\end{pgfscope}%
\begin{pgfscope}%
\definecolor{textcolor}{rgb}{0.000000,0.000000,0.000000}%
\pgfsetstrokecolor{textcolor}%
\pgfsetfillcolor{textcolor}%
\pgftext[x=0.289968in, y=1.306403in, left, base]{\color{textcolor}\sffamily\fontsize{10.000000}{12.000000}\selectfont 0.30}%
\end{pgfscope}%
\begin{pgfscope}%
\pgfsetbuttcap%
\pgfsetroundjoin%
\definecolor{currentfill}{rgb}{0.000000,0.000000,0.000000}%
\pgfsetfillcolor{currentfill}%
\pgfsetlinewidth{0.803000pt}%
\definecolor{currentstroke}{rgb}{0.000000,0.000000,0.000000}%
\pgfsetstrokecolor{currentstroke}%
\pgfsetdash{}{0pt}%
\pgfsys@defobject{currentmarker}{\pgfqpoint{-0.048611in}{0.000000in}}{\pgfqpoint{-0.000000in}{0.000000in}}{%
\pgfpathmoveto{\pgfqpoint{-0.000000in}{0.000000in}}%
\pgfpathlineto{\pgfqpoint{-0.048611in}{0.000000in}}%
\pgfusepath{stroke,fill}%
}%
\begin{pgfscope}%
\pgfsys@transformshift{0.696435in}{1.863698in}%
\pgfsys@useobject{currentmarker}{}%
\end{pgfscope}%
\end{pgfscope}%
\begin{pgfscope}%
\definecolor{textcolor}{rgb}{0.000000,0.000000,0.000000}%
\pgfsetstrokecolor{textcolor}%
\pgfsetfillcolor{textcolor}%
\pgftext[x=0.289968in, y=1.810936in, left, base]{\color{textcolor}\sffamily\fontsize{10.000000}{12.000000}\selectfont 0.35}%
\end{pgfscope}%
\begin{pgfscope}%
\pgfsetbuttcap%
\pgfsetroundjoin%
\definecolor{currentfill}{rgb}{0.000000,0.000000,0.000000}%
\pgfsetfillcolor{currentfill}%
\pgfsetlinewidth{0.803000pt}%
\definecolor{currentstroke}{rgb}{0.000000,0.000000,0.000000}%
\pgfsetstrokecolor{currentstroke}%
\pgfsetdash{}{0pt}%
\pgfsys@defobject{currentmarker}{\pgfqpoint{-0.048611in}{0.000000in}}{\pgfqpoint{-0.000000in}{0.000000in}}{%
\pgfpathmoveto{\pgfqpoint{-0.000000in}{0.000000in}}%
\pgfpathlineto{\pgfqpoint{-0.048611in}{0.000000in}}%
\pgfusepath{stroke,fill}%
}%
\begin{pgfscope}%
\pgfsys@transformshift{0.696435in}{2.368231in}%
\pgfsys@useobject{currentmarker}{}%
\end{pgfscope}%
\end{pgfscope}%
\begin{pgfscope}%
\definecolor{textcolor}{rgb}{0.000000,0.000000,0.000000}%
\pgfsetstrokecolor{textcolor}%
\pgfsetfillcolor{textcolor}%
\pgftext[x=0.289968in, y=2.315469in, left, base]{\color{textcolor}\sffamily\fontsize{10.000000}{12.000000}\selectfont 0.40}%
\end{pgfscope}%
\begin{pgfscope}%
\pgfsetbuttcap%
\pgfsetroundjoin%
\definecolor{currentfill}{rgb}{0.000000,0.000000,0.000000}%
\pgfsetfillcolor{currentfill}%
\pgfsetlinewidth{0.803000pt}%
\definecolor{currentstroke}{rgb}{0.000000,0.000000,0.000000}%
\pgfsetstrokecolor{currentstroke}%
\pgfsetdash{}{0pt}%
\pgfsys@defobject{currentmarker}{\pgfqpoint{-0.048611in}{0.000000in}}{\pgfqpoint{-0.000000in}{0.000000in}}{%
\pgfpathmoveto{\pgfqpoint{-0.000000in}{0.000000in}}%
\pgfpathlineto{\pgfqpoint{-0.048611in}{0.000000in}}%
\pgfusepath{stroke,fill}%
}%
\begin{pgfscope}%
\pgfsys@transformshift{0.696435in}{2.872764in}%
\pgfsys@useobject{currentmarker}{}%
\end{pgfscope}%
\end{pgfscope}%
\begin{pgfscope}%
\definecolor{textcolor}{rgb}{0.000000,0.000000,0.000000}%
\pgfsetstrokecolor{textcolor}%
\pgfsetfillcolor{textcolor}%
\pgftext[x=0.289968in, y=2.820002in, left, base]{\color{textcolor}\sffamily\fontsize{10.000000}{12.000000}\selectfont 0.45}%
\end{pgfscope}%
\begin{pgfscope}%
\pgfsetbuttcap%
\pgfsetroundjoin%
\definecolor{currentfill}{rgb}{0.000000,0.000000,0.000000}%
\pgfsetfillcolor{currentfill}%
\pgfsetlinewidth{0.803000pt}%
\definecolor{currentstroke}{rgb}{0.000000,0.000000,0.000000}%
\pgfsetstrokecolor{currentstroke}%
\pgfsetdash{}{0pt}%
\pgfsys@defobject{currentmarker}{\pgfqpoint{-0.048611in}{0.000000in}}{\pgfqpoint{-0.000000in}{0.000000in}}{%
\pgfpathmoveto{\pgfqpoint{-0.000000in}{0.000000in}}%
\pgfpathlineto{\pgfqpoint{-0.048611in}{0.000000in}}%
\pgfusepath{stroke,fill}%
}%
\begin{pgfscope}%
\pgfsys@transformshift{0.696435in}{3.377297in}%
\pgfsys@useobject{currentmarker}{}%
\end{pgfscope}%
\end{pgfscope}%
\begin{pgfscope}%
\definecolor{textcolor}{rgb}{0.000000,0.000000,0.000000}%
\pgfsetstrokecolor{textcolor}%
\pgfsetfillcolor{textcolor}%
\pgftext[x=0.289968in, y=3.324535in, left, base]{\color{textcolor}\sffamily\fontsize{10.000000}{12.000000}\selectfont 0.50}%
\end{pgfscope}%
\begin{pgfscope}%
\pgfsetbuttcap%
\pgfsetroundjoin%
\definecolor{currentfill}{rgb}{0.000000,0.000000,0.000000}%
\pgfsetfillcolor{currentfill}%
\pgfsetlinewidth{0.803000pt}%
\definecolor{currentstroke}{rgb}{0.000000,0.000000,0.000000}%
\pgfsetstrokecolor{currentstroke}%
\pgfsetdash{}{0pt}%
\pgfsys@defobject{currentmarker}{\pgfqpoint{-0.048611in}{0.000000in}}{\pgfqpoint{-0.000000in}{0.000000in}}{%
\pgfpathmoveto{\pgfqpoint{-0.000000in}{0.000000in}}%
\pgfpathlineto{\pgfqpoint{-0.048611in}{0.000000in}}%
\pgfusepath{stroke,fill}%
}%
\begin{pgfscope}%
\pgfsys@transformshift{0.696435in}{3.881830in}%
\pgfsys@useobject{currentmarker}{}%
\end{pgfscope}%
\end{pgfscope}%
\begin{pgfscope}%
\definecolor{textcolor}{rgb}{0.000000,0.000000,0.000000}%
\pgfsetstrokecolor{textcolor}%
\pgfsetfillcolor{textcolor}%
\pgftext[x=0.289968in, y=3.829068in, left, base]{\color{textcolor}\sffamily\fontsize{10.000000}{12.000000}\selectfont 0.55}%
\end{pgfscope}%
\begin{pgfscope}%
\pgfsetbuttcap%
\pgfsetroundjoin%
\definecolor{currentfill}{rgb}{0.000000,0.000000,0.000000}%
\pgfsetfillcolor{currentfill}%
\pgfsetlinewidth{0.803000pt}%
\definecolor{currentstroke}{rgb}{0.000000,0.000000,0.000000}%
\pgfsetstrokecolor{currentstroke}%
\pgfsetdash{}{0pt}%
\pgfsys@defobject{currentmarker}{\pgfqpoint{-0.048611in}{0.000000in}}{\pgfqpoint{-0.000000in}{0.000000in}}{%
\pgfpathmoveto{\pgfqpoint{-0.000000in}{0.000000in}}%
\pgfpathlineto{\pgfqpoint{-0.048611in}{0.000000in}}%
\pgfusepath{stroke,fill}%
}%
\begin{pgfscope}%
\pgfsys@transformshift{0.696435in}{4.386363in}%
\pgfsys@useobject{currentmarker}{}%
\end{pgfscope}%
\end{pgfscope}%
\begin{pgfscope}%
\definecolor{textcolor}{rgb}{0.000000,0.000000,0.000000}%
\pgfsetstrokecolor{textcolor}%
\pgfsetfillcolor{textcolor}%
\pgftext[x=0.289968in, y=4.333601in, left, base]{\color{textcolor}\sffamily\fontsize{10.000000}{12.000000}\selectfont 0.60}%
\end{pgfscope}%
\begin{pgfscope}%
\definecolor{textcolor}{rgb}{0.000000,0.000000,0.000000}%
\pgfsetstrokecolor{textcolor}%
\pgfsetfillcolor{textcolor}%
\pgftext[x=0.234413in,y=2.872764in,,bottom,rotate=90.000000]{\color{textcolor}\sffamily\fontsize{10.000000}{12.000000}\bfseries\selectfont F1 Score}%
\end{pgfscope}%
\begin{pgfscope}%
\pgfsetrectcap%
\pgfsetmiterjoin%
\pgfsetlinewidth{0.803000pt}%
\definecolor{currentstroke}{rgb}{0.000000,0.000000,0.000000}%
\pgfsetstrokecolor{currentstroke}%
\pgfsetdash{}{0pt}%
\pgfpathmoveto{\pgfqpoint{0.696435in}{1.359165in}}%
\pgfpathlineto{\pgfqpoint{0.696435in}{4.386363in}}%
\pgfusepath{stroke}%
\end{pgfscope}%
\begin{pgfscope}%
\pgfsetrectcap%
\pgfsetmiterjoin%
\pgfsetlinewidth{0.803000pt}%
\definecolor{currentstroke}{rgb}{0.000000,0.000000,0.000000}%
\pgfsetstrokecolor{currentstroke}%
\pgfsetdash{}{0pt}%
\pgfpathmoveto{\pgfqpoint{6.193658in}{1.359165in}}%
\pgfpathlineto{\pgfqpoint{6.193658in}{4.386363in}}%
\pgfusepath{stroke}%
\end{pgfscope}%
\begin{pgfscope}%
\pgfsetrectcap%
\pgfsetmiterjoin%
\pgfsetlinewidth{0.803000pt}%
\definecolor{currentstroke}{rgb}{0.000000,0.000000,0.000000}%
\pgfsetstrokecolor{currentstroke}%
\pgfsetdash{}{0pt}%
\pgfpathmoveto{\pgfqpoint{0.696435in}{1.359165in}}%
\pgfpathlineto{\pgfqpoint{6.193658in}{1.359165in}}%
\pgfusepath{stroke}%
\end{pgfscope}%
\begin{pgfscope}%
\pgfsetrectcap%
\pgfsetmiterjoin%
\pgfsetlinewidth{0.803000pt}%
\definecolor{currentstroke}{rgb}{0.000000,0.000000,0.000000}%
\pgfsetstrokecolor{currentstroke}%
\pgfsetdash{}{0pt}%
\pgfpathmoveto{\pgfqpoint{0.696435in}{4.386363in}}%
\pgfpathlineto{\pgfqpoint{6.193658in}{4.386363in}}%
\pgfusepath{stroke}%
\end{pgfscope}%
\begin{pgfscope}%
\definecolor{textcolor}{rgb}{0.000000,0.000000,0.000000}%
\pgfsetstrokecolor{textcolor}%
\pgfsetfillcolor{textcolor}%
\pgftext[x=1.109270in,y=2.961856in,,bottom]{\color{textcolor}\sffamily\fontsize{9.000000}{10.800000}\selectfont 0.4547}%
\end{pgfscope}%
\begin{pgfscope}%
\definecolor{textcolor}{rgb}{0.000000,0.000000,0.000000}%
\pgfsetstrokecolor{textcolor}%
\pgfsetfillcolor{textcolor}%
\pgftext[x=1.833542in,y=3.296866in,,bottom]{\color{textcolor}\sffamily\fontsize{9.000000}{10.800000}\selectfont 0.4879}%
\end{pgfscope}%
\begin{pgfscope}%
\definecolor{textcolor}{rgb}{0.000000,0.000000,0.000000}%
\pgfsetstrokecolor{textcolor}%
\pgfsetfillcolor{textcolor}%
\pgftext[x=2.557814in,y=3.238341in,,bottom]{\color{textcolor}\sffamily\fontsize{9.000000}{10.800000}\selectfont 0.4821}%
\end{pgfscope}%
\begin{pgfscope}%
\definecolor{textcolor}{rgb}{0.000000,0.000000,0.000000}%
\pgfsetstrokecolor{textcolor}%
\pgfsetfillcolor{textcolor}%
\pgftext[x=3.282085in,y=3.086981in,,bottom]{\color{textcolor}\sffamily\fontsize{9.000000}{10.800000}\selectfont 0.4671}%
\end{pgfscope}%
\begin{pgfscope}%
\definecolor{textcolor}{rgb}{0.000000,0.000000,0.000000}%
\pgfsetstrokecolor{textcolor}%
\pgfsetfillcolor{textcolor}%
\pgftext[x=4.006357in,y=2.782243in,,bottom]{\color{textcolor}\sffamily\fontsize{9.000000}{10.800000}\selectfont 0.4369}%
\end{pgfscope}%
\begin{pgfscope}%
\definecolor{textcolor}{rgb}{0.000000,0.000000,0.000000}%
\pgfsetstrokecolor{textcolor}%
\pgfsetfillcolor{textcolor}%
\pgftext[x=4.730629in,y=2.777197in,,bottom]{\color{textcolor}\sffamily\fontsize{9.000000}{10.800000}\selectfont 0.4364}%
\end{pgfscope}%
\begin{pgfscope}%
\definecolor{textcolor}{rgb}{0.000000,0.000000,0.000000}%
\pgfsetstrokecolor{textcolor}%
\pgfsetfillcolor{textcolor}%
\pgftext[x=5.454901in,y=3.380619in,,bottom]{\color{textcolor}\sffamily\fontsize{9.000000}{10.800000}\selectfont 0.4962}%
\end{pgfscope}%
\begin{pgfscope}%
\definecolor{textcolor}{rgb}{0.000000,0.000000,0.000000}%
\pgfsetstrokecolor{textcolor}%
\pgfsetfillcolor{textcolor}%
\pgftext[x=1.435193in,y=2.586484in,,bottom]{\color{textcolor}\sffamily\fontsize{9.000000}{10.800000}\selectfont 0.4175}%
\end{pgfscope}%
\begin{pgfscope}%
\definecolor{textcolor}{rgb}{0.000000,0.000000,0.000000}%
\pgfsetstrokecolor{textcolor}%
\pgfsetfillcolor{textcolor}%
\pgftext[x=2.159464in,y=3.074872in,,bottom]{\color{textcolor}\sffamily\fontsize{9.000000}{10.800000}\selectfont 0.4659}%
\end{pgfscope}%
\begin{pgfscope}%
\definecolor{textcolor}{rgb}{0.000000,0.000000,0.000000}%
\pgfsetstrokecolor{textcolor}%
\pgfsetfillcolor{textcolor}%
\pgftext[x=2.883736in,y=3.056709in,,bottom]{\color{textcolor}\sffamily\fontsize{9.000000}{10.800000}\selectfont 0.4641}%
\end{pgfscope}%
\begin{pgfscope}%
\definecolor{textcolor}{rgb}{0.000000,0.000000,0.000000}%
\pgfsetstrokecolor{textcolor}%
\pgfsetfillcolor{textcolor}%
\pgftext[x=3.608008in,y=3.087990in,,bottom]{\color{textcolor}\sffamily\fontsize{9.000000}{10.800000}\selectfont 0.4672}%
\end{pgfscope}%
\begin{pgfscope}%
\definecolor{textcolor}{rgb}{0.000000,0.000000,0.000000}%
\pgfsetstrokecolor{textcolor}%
\pgfsetfillcolor{textcolor}%
\pgftext[x=4.332279in,y=2.962866in,,bottom]{\color{textcolor}\sffamily\fontsize{9.000000}{10.800000}\selectfont 0.4548}%
\end{pgfscope}%
\begin{pgfscope}%
\definecolor{textcolor}{rgb}{0.000000,0.000000,0.000000}%
\pgfsetstrokecolor{textcolor}%
\pgfsetfillcolor{textcolor}%
\pgftext[x=5.056551in,y=3.080926in,,bottom]{\color{textcolor}\sffamily\fontsize{9.000000}{10.800000}\selectfont 0.4665}%
\end{pgfscope}%
\begin{pgfscope}%
\definecolor{textcolor}{rgb}{0.000000,0.000000,0.000000}%
\pgfsetstrokecolor{textcolor}%
\pgfsetfillcolor{textcolor}%
\pgftext[x=5.780823in,y=3.229259in,,bottom]{\color{textcolor}\sffamily\fontsize{9.000000}{10.800000}\selectfont 0.4812}%
\end{pgfscope}%
\begin{pgfscope}%
\definecolor{textcolor}{rgb}{0.000000,0.000000,0.000000}%
\pgfsetstrokecolor{textcolor}%
\pgfsetfillcolor{textcolor}%
\pgftext[x=3.445047in,y=4.469696in,,base]{\color{textcolor}\sffamily\fontsize{12.000000}{14.400000}\selectfont Abalation study on chairs with mixed training}%
\end{pgfscope}%
\begin{pgfscope}%
\pgfsetbuttcap%
\pgfsetmiterjoin%
\definecolor{currentfill}{rgb}{1.000000,1.000000,1.000000}%
\pgfsetfillcolor{currentfill}%
\pgfsetfillopacity{0.800000}%
\pgfsetlinewidth{1.003750pt}%
\definecolor{currentstroke}{rgb}{0.800000,0.800000,0.800000}%
\pgfsetstrokecolor{currentstroke}%
\pgfsetstrokeopacity{0.800000}%
\pgfsetdash{}{0pt}%
\pgfpathmoveto{\pgfqpoint{4.882270in}{3.867537in}}%
\pgfpathlineto{\pgfqpoint{6.096435in}{3.867537in}}%
\pgfpathquadraticcurveto{\pgfqpoint{6.124213in}{3.867537in}}{\pgfqpoint{6.124213in}{3.895315in}}%
\pgfpathlineto{\pgfqpoint{6.124213in}{4.289140in}}%
\pgfpathquadraticcurveto{\pgfqpoint{6.124213in}{4.316918in}}{\pgfqpoint{6.096435in}{4.316918in}}%
\pgfpathlineto{\pgfqpoint{4.882270in}{4.316918in}}%
\pgfpathquadraticcurveto{\pgfqpoint{4.854492in}{4.316918in}}{\pgfqpoint{4.854492in}{4.289140in}}%
\pgfpathlineto{\pgfqpoint{4.854492in}{3.895315in}}%
\pgfpathquadraticcurveto{\pgfqpoint{4.854492in}{3.867537in}}{\pgfqpoint{4.882270in}{3.867537in}}%
\pgfpathclose%
\pgfusepath{stroke,fill}%
\end{pgfscope}%
\begin{pgfscope}%
\pgfsetbuttcap%
\pgfsetmiterjoin%
\definecolor{currentfill}{rgb}{0.121569,0.466667,0.705882}%
\pgfsetfillcolor{currentfill}%
\pgfsetlinewidth{0.000000pt}%
\definecolor{currentstroke}{rgb}{0.000000,0.000000,0.000000}%
\pgfsetstrokecolor{currentstroke}%
\pgfsetstrokeopacity{0.000000}%
\pgfsetdash{}{0pt}%
\pgfpathmoveto{\pgfqpoint{4.910048in}{4.155839in}}%
\pgfpathlineto{\pgfqpoint{5.187825in}{4.155839in}}%
\pgfpathlineto{\pgfqpoint{5.187825in}{4.253062in}}%
\pgfpathlineto{\pgfqpoint{4.910048in}{4.253062in}}%
\pgfpathclose%
\pgfusepath{fill}%
\end{pgfscope}%
\begin{pgfscope}%
\definecolor{textcolor}{rgb}{0.000000,0.000000,0.000000}%
\pgfsetstrokecolor{textcolor}%
\pgfsetfillcolor{textcolor}%
\pgftext[x=5.298937in,y=4.155839in,left,base]{\color{textcolor}\sffamily\fontsize{10.000000}{12.000000}\selectfont Pix2Vox++}%
\end{pgfscope}%
\begin{pgfscope}%
\pgfsetbuttcap%
\pgfsetmiterjoin%
\definecolor{currentfill}{rgb}{1.000000,0.498039,0.054902}%
\pgfsetfillcolor{currentfill}%
\pgfsetlinewidth{0.000000pt}%
\definecolor{currentstroke}{rgb}{0.000000,0.000000,0.000000}%
\pgfsetstrokecolor{currentstroke}%
\pgfsetstrokeopacity{0.000000}%
\pgfsetdash{}{0pt}%
\pgfpathmoveto{\pgfqpoint{4.910048in}{3.951982in}}%
\pgfpathlineto{\pgfqpoint{5.187825in}{3.951982in}}%
\pgfpathlineto{\pgfqpoint{5.187825in}{4.049204in}}%
\pgfpathlineto{\pgfqpoint{4.910048in}{4.049204in}}%
\pgfpathclose%
\pgfusepath{fill}%
\end{pgfscope}%
\begin{pgfscope}%
\definecolor{textcolor}{rgb}{0.000000,0.000000,0.000000}%
\pgfsetstrokecolor{textcolor}%
\pgfsetfillcolor{textcolor}%
\pgftext[x=5.298937in,y=3.951982in,left,base]{\color{textcolor}\sffamily\fontsize{10.000000}{12.000000}\selectfont Pix2Vox}%
\end{pgfscope}%
\end{pgfpicture}%
\makeatother%
\endgroup%
}
    \caption{Bar plot for the \gls{f1}  for baseline trained by mixing chair dataset from real and synthetic dataset with ratio of 50\%.
    Observe that the \fls{IoU} is consistent for all types of randomization proving that mixed training negates loss from randomization.}
    \label{fig:ablation_dice2}
\end{figure}

\section{Additional outputs}\label{sec:additional-outputs}

This section displays more outputs discussed in \autoref{sec:baseline}, \autoref{sec:fine-tuning}, and \autoref{sec:mixed-training}.
The models trained on only synthetic datasets fail to reconstruct 3D voxels as seen in \autoref{fig:baseline_more_images1}.
By fine-tuning the models with a real dataset, the 3D reconstruction improves, as seen \autoref{fig:finetuning_more_images1}.
MMixed training further improves the model's output, as seen in \autoref{fig:mixed_more_images1}.

\begin{figure}
    \begin{tabular}{llll}
        Pix3D images & \includegraphics[width=.2\linewidth]{/Users/apple/OVGU/Thesis/code/3dReconstruction/report/images/evaluation/reconstruction/baseline/bookcase2} &
        \includegraphics[width=.2\linewidth]{/Users/apple/OVGU/Thesis/code/3dReconstruction/report/images/evaluation/reconstruction/baseline/sofa2} &
        \includegraphics[width=.2\linewidth]{/Users/apple/OVGU/Thesis/code/3dReconstruction/report/images/evaluation/reconstruction/baseline/desk1}\\

        Ground Truth & \includegraphics[trim={0 0 {.1\width} 0},clip,width=.2\linewidth]{/Users/apple/OVGU/Thesis/code/3dReconstruction/report/images/evaluation/reconstruction/baseline/bookcase2_original} &
        \includegraphics[trim={0 0 {.1\width} 0},clip,width=.2\linewidth]{/Users/apple/OVGU/Thesis/code/3dReconstruction/report/images/evaluation/reconstruction/baseline/sofa2_original} &
        \includegraphics[trim={0 0 {.1\width} 0},clip,width=.2\linewidth]{/Users/apple/OVGU/Thesis/code/3dReconstruction/report/images/evaluation/reconstruction/baseline/desk1_original}\\

        Output1 & \includegraphics[width=.2\linewidth]{/Users/apple/OVGU/Thesis/code/3dReconstruction/report/images/evaluation/reconstruction/baseline/pix3d_p2vpp_bookcase2_output} &
        \includegraphics[width=.2\linewidth]{/Users/apple/OVGU/Thesis/code/3dReconstruction/report/images/evaluation/reconstruction/baseline/pix3d_p2vpp_sofa2_output} &
        \includegraphics[width=.2\linewidth]{/Users/apple/OVGU/Thesis/code/3dReconstruction/report/images/evaluation/reconstruction/baseline/pix3d_p2vpp_desk1_output}\\

        Output2 & \includegraphics[width=.2\linewidth]{/Users/apple/OVGU/Thesis/code/3dReconstruction/report/images/evaluation/reconstruction/baseline/pix3d_p2v_bookcase2} &
        \includegraphics[width=.2\linewidth]{/Users/apple/OVGU/Thesis/code/3dReconstruction/report/images/evaluation/reconstruction/baseline/pix3d_p2v_sofa2} &
        \includegraphics[width=.2\linewidth]{/Users/apple/OVGU/Thesis/code/3dReconstruction/report/images/evaluation/reconstruction/baseline/pix3d_p2v_desk1}\\

        Output3 & \includegraphics[width=.2\linewidth]{/Users/apple/OVGU/Thesis/code/3dReconstruction/report/images/evaluation/reconstruction/baseline/s2rv3_p2vpp_bookcase2} &
        \includegraphics[width=.2\linewidth]{/Users/apple/OVGU/Thesis/code/3dReconstruction/report/images/evaluation/reconstruction/baseline/s2rv3_p2vpp_sofa2} &
        \includegraphics[width=.2\linewidth]{/Users/apple/OVGU/Thesis/code/3dReconstruction/report/images/evaluation/reconstruction/baseline/s2rv3_p2vpp_desk1}\\

        Output4 & \includegraphics[width=.2\linewidth]{/Users/apple/OVGU/Thesis/code/3dReconstruction/report/images/evaluation/reconstruction/baseline/s2rv3_p2v_bookcase2} &
        \includegraphics[width=.2\linewidth]{/Users/apple/OVGU/Thesis/code/3dReconstruction/report/images/evaluation/reconstruction/baseline/s2rv3_p2v_sofa2} &
        \includegraphics[width=.2\linewidth]{/Users/apple/OVGU/Thesis/code/3dReconstruction/report/images/evaluation/reconstruction/baseline/s2rv3_p2v_desk1}\\

    \end{tabular}
    \caption{3D reconstruction outputs for models trained on \textbf{only real dataset}, and \textbf{only synthetic dataset}. Output1-2: Pix2Vox++ and Pix2Vox trained on Pix3D(real dataset).
    Output3-4: Pix2Vox++ and Pix2Vox trained with only \gls{s2rv2} synthetic dataset. This corresponds to the bad \gls{iou} when trained on only synthetic dataset.}
    \label{fig:baseline_more_images1}
\end{figure}


\begin{figure}
    \begin{tabular}{llll}
        Pix3D images & \includegraphics[width=.2\linewidth]{/Users/apple/OVGU/Thesis/code/3dReconstruction/report/images/evaluation/reconstruction/baseline/bookcase2} &
        \includegraphics[width=.2\linewidth]{/Users/apple/OVGU/Thesis/code/3dReconstruction/report/images/evaluation/reconstruction/baseline/sofa2} &
        \includegraphics[width=.2\linewidth]{/Users/apple/OVGU/Thesis/code/3dReconstruction/report/images/evaluation/reconstruction/baseline/desk1}\\

        Ground Truth & \includegraphics[trim={0 0 {.1\width} 0},clip,width=.2\linewidth]{/Users/apple/OVGU/Thesis/code/3dReconstruction/report/images/evaluation/reconstruction/baseline/bookcase2_original} &
        \includegraphics[trim={0 0 {.1\width} 0},clip,width=.2\linewidth]{/Users/apple/OVGU/Thesis/code/3dReconstruction/report/images/evaluation/reconstruction/baseline/sofa2_original} &
        \includegraphics[trim={0 0 {.1\width} 0},clip,width=.2\linewidth]{/Users/apple/OVGU/Thesis/code/3dReconstruction/report/images/evaluation/reconstruction/baseline/desk1_original}\\

        Output1 & \includegraphics[width=.2\linewidth]{/Users/apple/OVGU/Thesis/code/3dReconstruction/report/images/evaluation/reconstruction/baseline/pix3d_p2vpp_bookcase2_output} &
        \includegraphics[width=.2\linewidth]{/Users/apple/OVGU/Thesis/code/3dReconstruction/report/images/evaluation/reconstruction/baseline/pix3d_p2vpp_sofa2_output} &
        \includegraphics[width=.2\linewidth]{/Users/apple/OVGU/Thesis/code/3dReconstruction/report/images/evaluation/reconstruction/baseline/pix3d_p2vpp_desk1_output}\\

        Output2 & \includegraphics[width=.2\linewidth]{/Users/apple/OVGU/Thesis/code/3dReconstruction/report/images/evaluation/reconstruction/baseline/pix3d_p2v_bookcase2} &
        \includegraphics[width=.2\linewidth]{/Users/apple/OVGU/Thesis/code/3dReconstruction/report/images/evaluation/reconstruction/baseline/pix3d_p2v_sofa2} &
        \includegraphics[width=.2\linewidth]{/Users/apple/OVGU/Thesis/code/3dReconstruction/report/images/evaluation/reconstruction/baseline/pix3d_p2v_desk1}\\

        Output3 & \includegraphics[width=.2\linewidth]{/Users/apple/OVGU/Thesis/code/3dReconstruction/report/images/evaluation/reconstruction/finetuning/s2rv3_p2vpp_bookcase2} &
        \includegraphics[width=.2\linewidth]{/Users/apple/OVGU/Thesis/code/3dReconstruction/report/images/evaluation/reconstruction/finetuning/s2rv3_p2vpp_sofa2} &
        \includegraphics[width=.2\linewidth]{/Users/apple/OVGU/Thesis/code/3dReconstruction/report/images/evaluation/reconstruction/finetuning/s2rv3_p2vpp_desk1}\\

        Output4 & \includegraphics[width=.2\linewidth]{/Users/apple/OVGU/Thesis/code/3dReconstruction/report/images/evaluation/reconstruction/finetuning/s2rv3_p2v_bookcase2} &
        \includegraphics[width=.2\linewidth]{/Users/apple/OVGU/Thesis/code/3dReconstruction/report/images/evaluation/reconstruction/finetuning/s2rv3_p2v_sofa2} &
        \includegraphics[width=.2\linewidth]{/Users/apple/OVGU/Thesis/code/3dReconstruction/report/images/evaluation/reconstruction/finetuning/s2rv3_p2v_desk1}\\

    \end{tabular}
    \caption{3D reconstruction outputs for models trained on real dataset and synthetic datasets with \textbf{fine-tuning}. Output1-2: Pix2Vox++ and Pix2Vox trained on Pix3D(real dataset).
    Output3-4: Pix2Vox++ and Pix2Vox pre-trained with only \gls{s2rv2} synthetic dataset and then fine-tuned with Pix3d. The reconstruction is better than models trained on only synthetic dataset.}
    \label{fig:finetuning_more_images1}
\end{figure}


\begin{figure}
    \begin{tabular}{llll}
        Pix3D images & \includegraphics[width=.2\linewidth]{/Users/apple/OVGU/Thesis/code/3dReconstruction/report/images/evaluation/reconstruction/baseline/bookcase2} &
        \includegraphics[width=.2\linewidth]{/Users/apple/OVGU/Thesis/code/3dReconstruction/report/images/evaluation/reconstruction/baseline/sofa2} &
        \includegraphics[width=.2\linewidth]{/Users/apple/OVGU/Thesis/code/3dReconstruction/report/images/evaluation/reconstruction/baseline/desk1}\\

        Ground Truth & \includegraphics[trim={0 0 {.1\width} 0},clip,width=.2\linewidth]{/Users/apple/OVGU/Thesis/code/3dReconstruction/report/images/evaluation/reconstruction/mixed/bookcase2_original} &
        \includegraphics[trim={0 0 {.1\width} 0},clip,width=.2\linewidth]{/Users/apple/OVGU/Thesis/code/3dReconstruction/report/images/evaluation/reconstruction/mixed/sofa2_original} &
        \includegraphics[trim={0 0 {.1\width} 0},clip,width=.2\linewidth]{/Users/apple/OVGU/Thesis/code/3dReconstruction/report/images/evaluation/reconstruction/mixed/desk2_original}\\

        Output1 & \includegraphics[width=.2\linewidth]{/Users/apple/OVGU/Thesis/code/3dReconstruction/report/images/evaluation/reconstruction/mixed/mixed1_p2vpp_bookcase2} &
        \includegraphics[width=.2\linewidth]{/Users/apple/OVGU/Thesis/code/3dReconstruction/report/images/evaluation/reconstruction/mixed/mixed1_p2vpp_sofa2} &
        \includegraphics[width=.2\linewidth]{/Users/apple/OVGU/Thesis/code/3dReconstruction/report/images/evaluation/reconstruction/mixed/mixed1_p2vpp_desk1}\\

        Output2 & \includegraphics[width=.2\linewidth]{/Users/apple/OVGU/Thesis/code/3dReconstruction/report/images/evaluation/reconstruction/mixed/mixed1_p2v_bookcase2} &
        \includegraphics[width=.2\linewidth]{/Users/apple/OVGU/Thesis/code/3dReconstruction/report/images/evaluation/reconstruction/mixed/mixed1_p2v_sofa2} &
        \includegraphics[width=.2\linewidth]{/Users/apple/OVGU/Thesis/code/3dReconstruction/report/images/evaluation/reconstruction/mixed/mixed1_p2v_desk1}\\

        Output3 & \includegraphics[width=.2\linewidth]{/Users/apple/OVGU/Thesis/code/3dReconstruction/report/images/evaluation/reconstruction/mixed/mixed2_p2vpp_bookcase2} &
        \includegraphics[width=.2\linewidth]{/Users/apple/OVGU/Thesis/code/3dReconstruction/report/images/evaluation/reconstruction/mixed/mixed2_p2vpp_sofa1} &
        \includegraphics[width=.2\linewidth]{/Users/apple/OVGU/Thesis/code/3dReconstruction/report/images/evaluation/reconstruction/mixed/mixed2_p2vpp_desk1}\\

        Output4 & \includegraphics[width=.2\linewidth]{/Users/apple/OVGU/Thesis/code/3dReconstruction/report/images/evaluation/reconstruction/mixed/mixed2_p2v_bookcase2} &
        \includegraphics[width=.2\linewidth]{/Users/apple/OVGU/Thesis/code/3dReconstruction/report/images/evaluation/reconstruction/mixed/mixed2_p2v_sofa2} &
        \includegraphics[width=.2\linewidth]{/Users/apple/OVGU/Thesis/code/3dReconstruction/report/images/evaluation/reconstruction/mixed/mixed2_p2v_desk1}\\

    \end{tabular}
    \caption{3D reconstruction outputs for best \textbf{mixed training}(50\% per mini-batch) models. Output1-2: Pix2Vox++ and Pix2Vox mixed trained with \gls{s2rv1}.
    Output3-4:Pix2Vox++ and Pix2Vox mixed trained with \gls{s2rv2}}
    \label{fig:mixed_more_images1}
\end{figure}

\clearpage

\section{Additional attempts}\label{sec:some-failed-attempts}

This section briefly discusses other attempts made to solve the research question related to this thesis.

\subsection{High Definition Render Pipeline(HDRP)}\label{subsec:high-definition-render-pipeline(hdrp)}

Unity provides three types of rendering pipelines: Built-in Render Pipeline, Universal Render Pipeline and \gls{hdrp}.
\gls{hdrp} has high-fidelity and high graphics and hence needs compatible GPU and other hardware and software requirements.
Each pipeline is not compatible with the other, and hence the developer has to pre-decide which pipeline to use.
Even the assets and the materials are dependent on the pipeline, and hence assets of one pipeline can not be used in the other.
In \autoref{fig:hdrp} we can see the furniture not being able to render in \gls{hdrp}.
Using \gls{hdrp} to improve the rendering would be interesting future work.

\begin{figure}[ht]
    \centering
    \includegraphics[width=.4\textwidth, height = .3\textwidth]{/Users/apple/OVGU/Thesis/code/3dReconstruction/report/images/appendix/hdrp1}
    \includegraphics[width=.4\textwidth, height = .3\textwidth,valign=m]{/Users/apple/OVGU/Thesis/code/3dReconstruction/report/images/appendix/hdrp2}\\
    \vspace{0.1cm}
    \includegraphics[width=.4\textwidth, height = .3\textwidth]{/Users/apple/OVGU/Thesis/code/3dReconstruction/report/images/appendix/hdrp3}
    \includegraphics[width=.4\textwidth, height = .3\textwidth]{/Users/apple/OVGU/Thesis/code/3dReconstruction/report/images/appendix/hdrp4}\\
    \caption{Sample images for models rendered in High Definition Render Pipeline of Unity}
    \label{fig:hdrp}
\end{figure}


%\subsection{Different loss functions}

\subsection{Empty image test}\label{subsec:empty-image-test}
In all the experiments conducted, the 2D image always had a furniture model in the frame.
This section determines if the model predicts something when there is no furniture in the frame, with an empty room.
This is just an attempt to see if the model hallucinates any furniture in an empty room.

Though a piece of furniture is not reconstructed with complete detail, models predict voxels that resemble the furniture category.
We can imagine some of them to be bed, sofa, bookshelf, and table.

\begin{figure}
    \begin{tabular}{llll}
%        \includegraphics[width=.2\linewidth]{/Users/apple/OVGU/Thesis/code/3dReconstruction/report/images/evaluation/reconstruction/empty/empty_room_1} &
        \includegraphics[width=.2\linewidth]{/Users/apple/OVGU/Thesis/code/3dReconstruction/report/images/evaluation/reconstruction/empty/empty_room_2} &
        \includegraphics[width=.2\linewidth]{/Users/apple/OVGU/Thesis/code/3dReconstruction/report/images/evaluation/reconstruction/empty/empty_room_3} &
        \includegraphics[width=.2\linewidth]{/Users/apple/OVGU/Thesis/code/3dReconstruction/report/images/evaluation/reconstruction/empty/empty_room_4} &
        \includegraphics[width=.2\linewidth]{/Users/apple/OVGU/Thesis/code/3dReconstruction/report/images/evaluation/reconstruction/empty/empty_room_5}\\

%        \includegraphics[width=.2\linewidth]{/Users/apple/OVGU/Thesis/code/3dReconstruction/report/images/evaluation/reconstruction/empty/p2vpp_empty_output_1} &
        \includegraphics[width=.2\linewidth]{/Users/apple/OVGU/Thesis/code/3dReconstruction/report/images/evaluation/reconstruction/empty/p2vpp_empty_output_2} &
        \includegraphics[width=.2\linewidth]{/Users/apple/OVGU/Thesis/code/3dReconstruction/report/images/evaluation/reconstruction/empty/p2vpp_empty_output_3} &
        \includegraphics[width=.2\linewidth]{/Users/apple/OVGU/Thesis/code/3dReconstruction/report/images/evaluation/reconstruction/empty/p2vpp_empty_output_4} &
        \includegraphics[width=.2\linewidth]{/Users/apple/OVGU/Thesis/code/3dReconstruction/report/images/evaluation/reconstruction/empty/p2vpp_empty_output_5}\\

%        \includegraphics[width=.2\linewidth]{/Users/apple/OVGU/Thesis/code/3dReconstruction/report/images/evaluation/reconstruction/empty/p2v_empty_output_1} &
        \includegraphics[width=.2\linewidth]{/Users/apple/OVGU/Thesis/code/3dReconstruction/report/images/evaluation/reconstruction/empty/p2v_empty_output_2} &
        \includegraphics[width=.2\linewidth]{/Users/apple/OVGU/Thesis/code/3dReconstruction/report/images/evaluation/reconstruction/empty/p2v_empty_output_3} &
        \includegraphics[width=.2\linewidth]{/Users/apple/OVGU/Thesis/code/3dReconstruction/report/images/evaluation/reconstruction/empty/p2v_empty_output_4} &
        \includegraphics[width=.2\linewidth]{/Users/apple/OVGU/Thesis/code/3dReconstruction/report/images/evaluation/reconstruction/empty/p2v_empty_output_5}\\

    \end{tabular}
    \caption{3D reconstruction outputs for best \textbf{mixed training}(50\% per mini-batch) models for empty rooms.
    We do not see a detailed reconstruction, but some of them resemble furniture category.}
    \label{fig:empty_images1}
\end{figure}

\clearpage
\newpage
\section{Samples from Survey}\label{sec:samples-from-survey}
We surveyed to realize the photorealism of the proposed \gls{free} dataset with other proclaimed photorealistic datasets.
The survey is discussed in detail in \autoref{subsec:survey-results}.
In this section, we display a sample each for all three sections from the collection.
\autoref{fig:survey_sample_section1}, \autoref{fig:survey_sample_section2}, \autoref{fig:survey_sample_section3} are one examples from each section.

\begin{figure}[ht]
    \centering
    \includegraphics[width=1\textwidth]{/Users/apple/OVGU/Thesis/code/3dReconstruction/report/images/appendix/survey/survey_section1}
    \caption{A sample question used for survey section 1. For section 1, participants were asked to select 'Real' or 'Not real' for 27 images, 3 images per dataset.}
    \label{fig:survey_sample_section1}
\end{figure}

\begin{figure}[ht]
    \centering
    \includegraphics[width=1\textwidth]{/Users/apple/OVGU/Thesis/code/3dReconstruction/report/images/appendix/survey/survey_section2}
    \caption{A sample question used for survey section 2. For section 2, participants were asked rate the 27 images from 1 to 10. We had 3 images per dataset.}
    \label{fig:survey_sample_section2}
\end{figure}

\begin{figure}[ht]
    \centering
    \includegraphics[width=1\textwidth]{/Users/apple/OVGU/Thesis/code/3dReconstruction/report/images/appendix/survey/survey_section3}
    \caption{A sample question used for survey section 3. For section 3, the users were shown 9 images together and asked to rank them on descending order of photorealism. }
    \label{fig:survey_sample_section3}
\end{figure}


\clearpage

\section{Tables}\label{sec:tables}

\begin{table}[ht]
    \centering
    \resizebox{\columnwidth}{!}{%
        \begin{tabular}{|c |c |c |c |c |c |c |c |c |}
            \hline
            Dataset & Chair & Wardrobe & Table & Bed &	Desk & Sofa & Bookcase & Avg.\\ [0.5ex]
            \hline\hline
            Pix3D & \textbf{0.2887} &\textbf{0.6782}	&\textbf{0.1888} &\textbf{0.3739} &\textbf{0.211}2 &\textbf{0.618} &\textbf{0.291} &\textbf{0.3443} \\
            \hline
            Pix3D(no aug.) & 0.2306 &0.5901 &0.1494	&0.2744	&0.1734	&0.5182 &0.2099 &0.2777 \\
            \hline
            \gls{s2rv1}(no aug.) &0.0455 &0.05362 &0.064 &0.0308 &0.1277 &0.0695 &0.03995 &0.0577 \\
            \hline
            \gls{s2rv1} &0.0793 &0.0947 &0.0553 &0.0674 &0.1074 &0.1237 &0.0869 &0.0847 \\
            \hline
            \gls{s2rv2} &0.0916	&0.3604	&0.0822	&0.1263	&0.1079	&0.23022 &0.1686	&0.1297 \\
            \hline
        \end{tabular}%
    }
    \caption{Table represents \gls{iou} values for each category when trained on Pix2Vox++.
    The models trained on synthetic data is significantly bad when tested on real-data.}
    \label{tab:baseline_categories_all_iou_pix2voxpp}
\end{table}


\begin{table}[ht]
    \centering
    \resizebox{\columnwidth}{!}{%
        \begin{tabular}{|c |c |c |c |c |c |c |c |c |}
            \hline
            Dataset & Chair & Wardrobe & Table & Bed &	Desk & Sofa & Bookcase & Avg.\\ [0.5ex]
            \hline\hline
            Pix3D & \textbf{0.2715}	&\textbf{0.7287}	&\textbf{0.1831}	&\textbf{0.355}	&\textbf{0.2004}	&\textbf{0.5703}	&\textbf{0.2747}	&\textbf{0.3254} \\
            \hline
            Pix3D(no aug.) & 0.2493	&0.6615	&0.1774	&0.3216	&0.1838	&0.5857	&0.231 &0.3108 \\
            \hline
            \gls{s2rv1}(no aug.) &0.0763	&0.0597	&0.0187	&0.0541	&0.03774 &0.1202	&0.0549 &0.0679  \\
            \hline
            \gls{s2rv1} &0.04	&0.1431	&0.0485	&0.0404 &0.10411	&0.0723	&0.1067 &0.0571\\
            \hline
            \gls{s2rv2} &0.1323	&0.3747	&0.0959	&0.1505	&0.1197	&0.305	&0.1206 &0.1643\\
            \hline
        \end{tabular}%
    }
    \caption{Table represents \gls{iou} values for each category when trained on Pix2Vox.
    The models trained on synthetic data is significantly bad when tested on real-data.}
    \label{tab:baseline_categories_all_iou_pix2vox}
\end{table}


\begin{table}[ht]
    \centering
    \resizebox{\columnwidth}{!}{%
        \begin{tabular}{|c |c |c |c |c |c |c |c |c |}
            \hline
            Dataset & Chair & Wardrobe & Table & Bed &	Desk & Sofa & Bookcase & Avg.\\ [0.5ex]
            \hline\hline
            Pix3D & \textbf{0.2887} &\textbf{0.6782}	&\textbf{0.1888} &\textbf{0.3739} &\textbf{0.211}2 &\textbf{0.618} &\textbf{0.291} &\textbf{0.3443} \\
            \hline
            \gls{s2rv1}+Pix3D  &0.2589 &0.6454 &0.1715	&0.3382	&0.2096 &0.5897	&0.2006	&0.3161\\
            \hline
            \gls{s2rv2}+Pix3D &0.2605	&0.63 &0.1579 &0.3051 &0.185 &0.5813 &0.2388	&0.3089\\

            \hline
        \end{tabular}%
    }
    \caption{Table represents \gls{iou} values for each category when trained on Pix2Vox++.
    The models trained on synthetic data and then fine-tuned with real data. No improvement is seen with fine-tuning.}
    \label{tab:finetuning_categories_all_iou_pix2voxpp}
\end{table}

\begin{table}[ht]
    \centering
    \resizebox{\columnwidth}{!}{%
        \begin{tabular}{|c |c |c |c |c |c |c |c |c |}
            \hline
            Dataset & Chair & Wardrobe & Table & Bed &	Desk & Sofa & Bookcase & Avg.\\ [0.5ex]
            \hline\hline
            Pix3D & \textbf{0.2715}	&\textbf{0.7287} &\textbf{0.1831} &0.355	&0.2004	&0.5703	&0.2747	&0.3254 \\
            \hline
            \gls{s2rv1}+Pix3D  &0.25347	&0.6806	&0.1735	&0.3176	&\textbf{0.2006}	&0.5824	&0.23386	&0.3125\\
            \hline
            \gls{s2rv2}+Pix3D &0.2684	&0.6965	&0.1765	&\textbf{0.3722}	&0.1963	&\textbf{0.6202}	&\textbf{0.296}	&\textbf{0.3338}\\

            \hline
        \end{tabular}%
    }
    \caption{Table represents \gls{iou} values for each category when trained on Pix2Vox.
    The models trained on synthetic dataand fine-tuned with real-data. A slight improvement is seen in some of the categories and the average \gls{iou}}
    \label{tab:finetuning_categories_all_iou_pix2vox}
\end{table}

\begin{table}[ht]
    \centering
    \resizebox{\columnwidth}{!}{%
        \begin{tabular}{|c |c |c |c |c |c |c |c |c |}
            \hline
            Dataset & Chair & Wardrobe & Table & Bed &	Desk & Sofa & Bookcase & Avg.\\ [0.5ex]
            \hline\hline
            Pix3D & 0.2887 &0.6782	&0.1888 &0.3739 &0.2112 &0.618 &0.291 &0.3443 \\
            \hline
            \gls{s2rv1}+0.15\%  &0.2972	&0.7229 &\textbf{0.2008} &0.3591 &0.2394	&0.6011	&0.2789	&0.3479\\
            \hline
            \gls{s2rv1}+0.25\% &\textbf{0.3114}	&0.6989	&0.1956	&\textbf{0.4034}	&0.2338	&0.6304	&0.2599	&\textbf{0.3607}\\
            \hline
            \gls{s2rv1}+0.5\% &0.3063	&0.7219	&0.1964	&0.397	&0.2533	&\textbf{0.6401}	&0.3051	&0.3636\\
            \hline
            \gls{s2rv1}+0.75\% &0.3005	&0.6787	&0.1797	&0.3867	&0.2241	&0.6168	&0.3049 &0.3498\\
            \hline
            \gls{s2rv1}+0.90\% &0.2933	&0.6514	&0.1901	&0.3692	&0.2135	&0.6388	&0.3021	&0.3499\\
            \hline
            \gls{s2rv2}+0.15\%  &0.2559	&0.7651	&0.1708	&0.3042	&0.2297	&0.554	&0.3159	&0.3134 \\
            \hline
            \gls{s2rv2}+0.25\% &0.3017	&0.7221	&0.1897	&0.3649	&0.2606	&0.6138	&0.372	&0.3556 \\
            \hline
            \gls{s2rv2}+0.5\% &0.3077	&0.7582	&0.1859	&0.3493	&\textbf{0.2664}	&0.6194	&\textbf{0.3874}	&0.3587 \\
            \hline
            \gls{s2rv2}+0.75\% &0.2944	&\textbf{0.7697} &0.1872 &0.3712	&0.2527	&0.6013	&0.3291	&0.3494 \\
            \hline
            \gls{s2rv2}+0.90\% &0.2947	&0.6893	&0.1794	&0.4013	&0.1914	&0.6139	&0.3558	&0.3482\\
            \hline
        \end{tabular}%
    }
    \caption{Table represents \gls{iou} values for each category when trained on Pix2Vox++.
    The first column indicates the synthetic dataset and ratio of real data used in each mini-batch.}
    \label{tab:mixed_categories_all_iou_pix2voxpp}
\end{table}

\begin{table}[ht]
    \centering
    \resizebox{\columnwidth}{!}{%
        \begin{tabular}{|c |c |c |c |c |c |c |c |c |}
            \hline
            Dataset & Chair & Wardrobe & Table & Bed &	Desk & Sofa & Bookcase & Avg.\\ [0.5ex]
            \hline\hline
            Pix3D & 0.2715	&0.7287	&0.1831 &0.355	&0.2004	&0.5703	&0.2747	&0.3254 \\
            \hline
            \gls{s2rv1}+0.15\%  &\textbf{0.2961}	&0.7001	&0.1918	&0.3894	&0.209	&0.6099	&0.294	&0.3483 \\
            \hline
            \gls{s2rv1}+0.25\% &0.2911	&0.7323	&\textbf{0.2026}	&0.367	&0.2297	&0.5889	&0.2644	&0.3432 \\
            \hline
            \gls{s2rv1}+0.5\% &0.2756	&0.7087	&0.1877	&\textbf{0.4072} &0.2365 &\textbf{0.652}	&0.3007	&\textbf{0.3514} \\
            \hline
            \gls{s2rv1}+0.75\% &0.281	&0.6571	&0.1778	&0.3869	&0.2110	&0.6369	&0.2769	&0.343\\
            \hline
            \gls{s2rv1}+0.90\% &0.2756	&0.7258	&0.1769	&0.3611	&0.218	&0.6192	&0.3263	&0.3389 \\
            \hline
            \gls{s2rv2}+0.15\%  &0.2447	&0.7086	&0.1481	&0.3199	&0.1923	&0.5589	&0.3191	&0.3027 \\
            \hline
            \gls{s2rv2}+0.25\% &0.2889	&0.741	&0.1823	&0.375	&0.2226	&0.6268	&\textbf{0.3966}	&0.3513  \\
            \hline
            \gls{s2rv2}+0.5\% &0.268	&0.7479	&0.1790	&0.362	&\textbf{0.2404}	&0.6337	&0.3393	&0.3416 \\
            \hline
            \gls{s2rv2}+0.75\% &0.2837	&0.743	&0.1784	&0.359	&0.1958	&0.6076	&0.3908	&0.3413 \\
            \hline
            \gls{s2rv2}+0.90\% &0.2764	&\textbf{0.7497}	&0.1821	&0.3776	&0.2006	&0.6227	&0.3701	&0.3434\\
            \hline
        \end{tabular}%
    }
    \caption{Table represents \gls{iou} values for each category when trained on Pix2Vox.
    The first column indicates the synthetic dataset and ratio of real data used in each mini-batch.}
    \label{tab:mixed_categories_all_iou_pix2vox}
\end{table}


\begin{table}[ht]
    \centering
    \resizebox{\columnwidth}{!}{%

        \begin{tabular}{|c |c |c | c| c|}
            \hline
            Dataset & Pix2Vox++ & Pix2Vox & Pix2Vox++ with 50\% & Pix2Vox with 50\%\\ [0.5ex]
            \hline\hline
            Pix3D Chair(no aug)  &\textbf{0.2797} &0.2694 &x &x\\
            \hline
            Pix3D Chair &\textbf{0.3305} &0.2918 & x & x\\
            \hline
            S2R\textunderscore Textureless  &0.1798 &0.0748 &\textbf{0.366} &0.3385\\
            \hline
            S2R \textunderscore Textureless\textunderscore Light &0.143 &0.1026 &\textbf{0.362}  &0.338\\
            \hline
            S2R\textunderscore Textured  &0.2217 &0.112 &\textbf{0.3426} &0.3404\\
            \hline
            S2R\textunderscore Textured\textunderscore Light &0.2221 &0.0991 &0.312 &\textbf{0.3303}\\
            \hline
            S2R\textunderscore Multi\textunderscore Object &0.2372 &0.0943 &0.3104 &\textbf{0.3429}\\
            \hline
        \end{tabular}%
    }
    \caption{Table represents \gls{iou} values for each of the dataset created with different domain randomization component.
    The last 2 colums is mixed training with 50\% of mini-batches being Pix3D.
    We see that mixed training consistently increases the performance of the models.}
    \label{tab:ablation1}
\end{table}




%Dice tables
\begin{table}[ht]
    \centering
    \resizebox{\columnwidth}{!}{%
        \begin{tabular}{|c |c |c |c |c |c |c |c |c |}
            \hline
            Dataset & Chair & Wardrobe & Table & Bed &	Desk & Sofa & Bookcase & Avg.\\ [0.5ex]
            \hline\hline
            Pix3D & \textbf{0.4016} &\textbf{0.7452} &\textbf{0.251} &\textbf{0.4783} &\textbf{0.2973}2 &\textbf{0.7234} &\textbf{0.3881} &\textbf{0.4421} \\
            \hline
            Pix3D(no aug.) &0.3392	&0.6894	&0.2114	&0.3831	&0.2534	0.&6442	&0.3041	&0.3782 \\
            \hline
            \gls{s2rv1}(no aug.) &0.0849 &0.1012	&0.1152	&0.0586	&0.2201	&0.1279	&0.07491	&0.1054\\
            \hline
            \gls{s2rv1} &0.1406	&0.1696	&0.0996	&0.1182	&0.1847	&0.2073	&0.1541	&0.1477    \\
            \hline
            \gls{s2rv2} &0.1619	&0.5132	&0.1438	&0.2156	&0.1855	&0.3667	&0.2745	&0.21659 \\
            \hline
        \end{tabular}%
    }
    \caption{Table represents \gls{f1} values for each category when trained on Pix2Vox++.
    The models trained on synthetic data is significantly bad when tested on real-data.}
    \label{tab:baseline_categories_all_f1_pix2voxpp}
\end{table}

\begin{table}[ht]
    \centering
    \resizebox{\columnwidth}{!}{%
        \begin{tabular}{|c |c |c |c |c |c |c |c |c |}
            \hline
            Dataset & Chair & Wardrobe & Table & Bed &	Desk & Sofa & Bookcase & Avg.\\ [0.5ex]
            \hline\hline
            Pix3D & \textbf{0.3883}	&\textbf{0.8013}	&\textbf{0.2456}	&\textbf{0.4656}	&\textbf{0.2856}	&0.6882	&\textbf{0.3807}	&\textbf{0.4281} \\
            \hline
            Pix3D(no aug.) & 0.3585	& 0.7439	& 0.2375	& 0.4258	&0.2625	&\textbf{0.7033}	&0.3256	&0.4087 \\
            \hline
            \gls{s2rv1}(no aug.) &0.1359	&0.1119	&0.0353	&0.0984	&0.0695	&0.2094	&0.103	&0.1211  \\
            \hline
            \gls{s2rv1} &0.0744	&0.2421	&0.0903	&0.0764	&0.1837	&0.13071	&0.1883	&0.104\\
            \hline
            \gls{s2rv2}  &0.2255	&0.5402	&0.165	&0.2562	&0.2052	&0.4608	&0.2112	&0.267
            \\
            \hline
        \end{tabular}%
    }
    \caption{Table represents \gls{f1} values for each category when trained on Pix2Vox.
    The models trained on synthetic data is significantly bad when tested on real-data.}
    \label{tab:baseline_categories_all_f1_pix2vox}
\end{table}


\begin{table}[ht]
    \centering
    \resizebox{\columnwidth}{!}{%
        \begin{tabular}{|c |c |c |c |c |c |c |c |c |}
            \hline
            Dataset & Chair & Wardrobe & Table & Bed &	Desk & Sofa & Bookcase & Avg.\\ [0.5ex]
            \hline\hline
            Pix3D & \textbf{0.4016} &\textbf{0.7452} &\textbf{0.251} &\textbf{0.4783} &\textbf{0.2973} &\textbf{0.7234} &\textbf{0.3881} &\textbf{0.4421} \\
            \hline
            \gls{s2rv1}+Pix3D  &0.3766	&0.7284	&0.2326	&0.4488	&0.2952	&0.7048	&0.2826	&0.4177\\
            \hline
            \gls{s2rv2}+Pix3D &0.374	&0.7064	&0.212	&0.4083	&0.2656	&0.69468	&0.3334	&0.4065\\

            \hline
        \end{tabular}%
    }
    \caption{Table represents \gls{f1} values for each category when trained on Pix2Vox++.
    The models trained on synthetic data and then fine-tuned with real data. No improvement is seen with fine-tuning.}
    \label{tab:finetuning_categories_all_f1_pix2voxpp}
\end{table}



\begin{table}[ht]
    \centering
    \resizebox{\columnwidth}{!}{%
        \begin{tabular}{|c |c |c |c |c |c |c |c |c |}
            \hline
            Dataset & Chair & Wardrobe & Table & Bed &	Desk & Sofa & Bookcase & Avg.\\ [0.5ex]
            \hline\hline
            Pix3D & \textbf{0.3883}	&\textbf{0.8013}	&\textbf{0.2456}	&0.4656	&\textbf{0.2856}	&0.6882	&0.3807	&0.4281 \\
            \hline
            \gls{s2rv1}+Pix3D  &0.3703	&0.7694	&0.2371	&0.4309	&0.2883	&\textbf{0.6977}	&0.335	&0.4155\\
            \hline
            \gls{s2rv2}+Pix3D &0.3798	&0.6646	&0.2311	&\textbf{0.4822}	&0.2757	&0.7327	&\textbf{0.3941}	&\textbf{0.431}\\

            \hline
        \end{tabular}%
    }
    \caption{Table represents \gls{f1} values for each category when trained on Pix2Vox.
    The models trained on synthetic dataand fine-tuned with real-data. A slight improvement is seen in some of the categories and the average \gls{iou}}
    \label{tab:finetuning_categories_all_f1_pix2vox}
\end{table}

\begin{table}[ht]
    \centering
    \resizebox{\columnwidth}{!}{%
        \begin{tabular}{|c |c |c |c |c |c |c |c |c |}
            \hline
            Dataset & Chair & Wardrobe & Table & Bed &	Desk & Sofa & Bookcase & Avg.\\ [0.5ex]
            \hline\hline
            Pix3D & 0.4016 &0.7452 &0.251 &0.4783 &0.2973 &0.7234 &0.3881 &0.4421 \\
            \hline
            \gls{s2rv1}+0.15\%  &0.4178	&0.7787	&\textbf{0.2638}	&0.4618	&0.3278	&0.7116	&0.3711	&0.4493 \\
            \hline
            \gls{s2rv1}+0.25\% &\textbf{0.4326}	&0.7635	&0.2551	&\textbf{0.5076}	&0.3193	&0.7364	&0.3483	&0.4608\\
            \hline
            \gls{s2rv1}+0.5\% &0.4253	&0.789	&0.2564	&0.5056	&0.3429	&\textbf{0.7468}	&0.4066	&\textbf{0.4644}\\
            \hline
            \gls{s2rv1}+0.75\% &0.4057	&0.7329	&0.2278	&0.3303	&0.2946	&0.7145	&0.3931	&0.4377 \\
            \hline
            \gls{s2rv1}+0.90\% &0.4091	&0.7212	&0.2495	&0.4719	&0.2952	&0.7413	&0.3968	&0.4472\\
            \hline
            \gls{s2rv2}+0.15\%  &0.3712	&0.8269	&0.2378	&0.4067	&0.3212	&0.6659	&0.4105	&0.4142 \\
            \hline
            \gls{s2rv2}+0.25\% &0.415	&0.6844	&0.2440	&0.4689	&0.3459	&0.7157	&0.4692	&0.4511 \\
            \hline
            \gls{s2rv2}+0.5\% &0.4211	&0.8135	&0.2422	&0.4485	&\textbf{0.3535} &0.7222	&\textbf{0.485}	&0.4543\\
            \hline
            \gls{s2rv2}+0.75\% &0.4089	&\textbf{0.8313}	&0.249	&0.4747	&0.3384	&0.709	&0.4273	&0.4479 \\
            \hline
            \gls{s2rv2}+0.90\% &0.4022	&0.7502	&0.2279	&0.5018	&0.2614	&0.7148	&0.4497	&0.4386\\
            \hline
        \end{tabular}%
    }
    \caption{Table represents \gls{f1} values for each category when trained on Pix2Vox++.
    The first column indicates the synthetic dataset and ratio of real data used in each mini-batch.}
    \label{tab:mixed_categories_all_f1_pix2voxpp}
\end{table}

\begin{table}[ht]
    \centering
    \resizebox{\columnwidth}{!}{%
        \begin{tabular}{|c |c |c |c |c |c |c |c |c |}
            \hline
            Dataset & Chair & Wardrobe & Table & Bed &	Desk & Sofa & Bookcase & Avg.\\ [0.5ex]
            \hline\hline
            Pix3D & 0.3883	&0.8013	&0.2456	&0.4656	&0.2856	&0.6882	&0.3807	&0.4281 \\
            \hline
            \gls{s2rv1}+0.15\%  &\textbf{0.4137}	&0.7597	&0.2518	&0.4914	&0.29474 &0.7195	&0.3862	&0.4476 \\
            \hline
            \gls{s2rv1}+0.25\% &0.4074	&0.7888	&\textbf{0.2691}	&0.4681	&0.3173	&0.6971	&0.3528	&0.4428 \\
            \hline
            \gls{s2rv1}+0.5\% &0.3894	&0.7748	&0.2458	&\textbf{0.519}	&0.3228	&\textbf{0.7603}	&0.3977	&0.4499 \\
            \hline
            \gls{s2rv1}+0.75\% &0.3931	&0.7284	&0.2338	&0.4916	&0.2917	&0.6704	&0.3657	&0.4391\\
            \hline
            \gls{s2rv1}+0.90\% &0.3851	&0.794	&0.2346	&0.4628	&0.2993	&0.7312	&0.4344	&0.4355 \\
            \hline
            \gls{s2rv2}+0.15\%  &0.3614	&0.7796	&0.2128	&0.4274	&0.2809	&0.6787	&0.4074	&0.406 \\
            \hline
            \gls{s2rv2}+0.25\% &0.4066	&0.7992	&0.2477	&0.304	&0.3074	&0.7337	&0.4904	&\textbf{0.4513}  \\
            \hline
            \gls{s2rv2}+0.5\% &0.3752	&0.8061	&0.2347	&0.4684	&\textbf{0.3247}	&0.7382	&0.427	&0.4352 \\
            \hline
            \gls{s2rv2}+0.75\% &0.3958	&0.8015	&0.2329	&0.4653	&0.268	&0.7172	&\textbf{0.4927}	&0.4373 \\
            \hline
            \gls{s2rv2}+0.90\% &0.3869	&\textbf{0.8116}	&0.238	&0.4823	&0.2778	&0.7298	&0.4769	&0.439\\
            \hline
        \end{tabular}%
    }
    \caption{Table represents \gls{iou} values for each category when trained on Pix2Vox.
    The first column indicates the synthetic dataset and ratio of real data used in each mini-batch.}
    \label{tab:mixed_categories_all_f1_pix2vox}
\end{table}


\begin{table}[ht]
    \centering
    \resizebox{\columnwidth}{!}{%

        \begin{tabular}{|c |c |c | c| c|}
            \hline
            Dataset & Pix2Vox++ & Pix2Vox & Pix2Vox++ with 50\% & Pix2Vox with 50\%\\ [0.5ex]
            \hline\hline
            Pix3D Chair(no aug)  &0.4047 &0.3944 &x &x\\
            \hline
            Pix3D Chair &\textbf{0.4175} &\textbf{0.4547} & x & x\\
            \hline
            S2R\textunderscore Textureless  &0.2933 &0.1365 &0.4879 &0.4659\\
            \hline
            S2R \textunderscore Textureless\textunderscore Light &0.242 &0.1795 &0.4821  &0.4641\\
            \hline
            S2R\textunderscore Textured  &0.3513 &0.1959 &0.4671 &0.4672\\
            \hline
            S2R\textunderscore Textured\textunderscore Light &0.35197 &0.1769 &0.4369 &0.4548\\
            \hline
            S2R\textunderscore Multi\textunderscore Object &0.3705 &0.168 &0.4364 &0.4665\\
            \hline
            S2R\textunderscore Combined &0.3851 &0.2289 &\textbf{0.4962} &\textbf{0.4812}\\
            \hline
        \end{tabular}%
    }
    \caption{Table represents \gls{f1} values for each of the dataset created with different domain randomization component.
    The last 2 colums is mixed training with 50\% of mini-batches being Pix3D.
    We see that mixed training consistently increases the performance of the models.}
    \label{tab:ablation1_f1}
\end{table}

% Usually, you will create your bibliography with BibTeX files, which can be exported from many literature management tools. For an example, see thesis.bib.
\cleardoublepage
\phantomsection
\bibliographystyle{literature/plainnat}
\bibliography{literature/thesis}

\cleardoublepage
\backmatter
\phantomsection
\printindex

% You need to sign this when submitting your thesis to the examination office.
\thispagestyle{empty}
\vspace*{38\baselineskip}
\hbox to \textwidth{\hrulefill}
\par

\iftoggle{german}{
    Hiermit erkläre ich, dass ich die vorliegende Arbeit selbstständig verfasst und keine anderen als die angegebenen Quellen und Hilfsmittel benutzt habe.
    Weiterhin wurde die Arbeit bisher weder in einem anderen Prüfungsverfahren vorgelegt noch anderweitig veröffentlicht.
    Textpassagen, die wörtlich oder dem Sinn nach auf anderen Quellen beruhen, sind als solche kenntlich gemacht.}{
    I herewith assure that I wrote the present thesis independently, that the thesis has not been partially or fully submitted as graded academic work and that I have used no other means than the ones indicated.
    I have indicated all parts of the work in which sources are used according to their wording or to their meaning.}

\signatureplace, \signaturedate

\clearpage

\end{document}
