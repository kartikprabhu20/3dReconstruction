\chapter{\iftoggle{german}{Konzept}{Tutorial}}\label{ch:tutorial}

Acronyms like \glspl{spl} and \gls{fol} are automatically added to the list of acronyms above.
To reference a chapter, subsection, figure, etc., put a label after the command (see above) and then use \autoref{ch:introduction}.

\todo{You can mark stuff as TODO, which is useful for writing drafts.\footnote{Footnotes are also possible, but are rarely used in computer science.}}
\marginnote{\footnotesize Or put notes to your advisors in the margin.}

\begin{theorem}{My Cool Theorem}{theoremname}
    You can also add definitions and theorems in mathematical theses.
\end{theorem}

\begin{proof}
    You can also add proofs, though most theses do not require any.
\end{proof}

In \autoref{ex:tictactoe}, we show an example of how to typeset a complex figure only with \LaTeX{} by using TikZ.

\newcommand{\rot}[1]{\multicolumn{1}{l}{\rlap{\hspace{-1mm}\rotatebox{60}{#1}}}}
\newcommand{\methmark}[2]{\tikz[remember picture, anchor=center] \node[inner sep=0,outer sep=-1pt] (#1){#2};}
\newcommand{\methmarkpad}[2]{\tikz[remember picture, anchor=center] \node[inner sep=0,outer sep=4pt] (#1){#2};}
\newcommand{\ex}{\raisebox{0.4pt}{\footnotesize{$\times$}}}
\newcommand{\oh}{\text{\scriptsize \ensuremath{\Circle}}}
\newcommand{\ohfill}{\text{\scriptsize \ensuremath{\CIRCLE}}}
\newcommand{\tictactoe}[9]{
    \begin{tabular}{rc|c|c}
        & \rot{\textbf{List}}       & \rot{\textbf{Ordered}}       & \rot{\textbf{Set}}       \\
        #1 \ & \methmark{InsertList}{#4} & \methmark{InsertOrdered}{#5} & \methmark{InsertSet}{#6} \\\hhline{~-|-|-}
        #2 \ & \methmark{SearchList}{#7} & \methmark{SearchOrdered}{#8} & \ex{}                    \\\hhline{~-|-|-}
        #3 \ & \ex{}                     & \methmark{SortOrdered}{#9}   & \ex{}
    \end{tabular}}
\newcommand{\link}[3]{
    \begin{tikzpicture}[remember picture, overlay, >=stealth, shift={(0,0)}]
    \draw[->, #3] (#1) to (#2);
    \end{tikzpicture}}
\newcommand{\linkoriginalmethod}[2]{\link{#1}{#2}{bend right, color=red}}
\newcommand{\linkoriginal}[2]{\link{#1}{#2}{bend right, color=red, dashed}}
\newcommand{\linkmethod}[2]{\link{#1}{#2}{bend left, color=blue}}
\newcommand{\linkself}[1]{\link{#1}{#1}{loop, min distance=6mm, out=320, in=220, color=blue}}

\begin{figure}
    \begin{subfigure}{0.29\textwidth}
        \tictactoe{\textbf{Insert}}{\textbf{Search}}{\textbf{Sort}}
        {\ohfill}{\oh}{\oh}
        {\ohfill}{\oh}
        {\oh}
        \caption{{{List}}}\label{ex:tictactoe1}
    \end{subfigure}
    \begin{subfigure}{0.21\textwidth}
        \tictactoe{}{}{}
        {\ohfill}{\ohfill}{\oh}
        {\oh}    {\ohfill}
        {\ohfill}
        \linkoriginalmethod{InsertOrdered}{InsertList}
        \linkoriginal{SearchOrdered}{SearchList}
        \linkmethod{InsertOrdered}{SortOrdered}
        \linkself{SortOrdered}
        \caption{{{List}, {Ordered}}}\label{ex:tictactoe2}
    \end{subfigure}
    \begin{subfigure}{0.19\textwidth}
        \tictactoe{}{}{}
        {\ohfill}{\oh}{\ohfill}
        {\ohfill}{\oh}
        {\oh}
        \linkoriginalmethod{InsertSet}{InsertList}
        \linkmethod{InsertSet}{SearchList}
        \caption{{{List}, {Set}}}\label{ex:tictactoe3}
    \end{subfigure}
    \begin{subfigure}{0.26\textwidth}
        \tictactoe{}{}{}
        {\ohfill}{\ohfill}{\ohfill}
        {\oh}    {\ohfill}
        {\ohfill}
        \linkoriginalmethod{InsertSet}{InsertOrdered}
        \linkoriginalmethod{InsertOrdered}{InsertList}
        \linkmethod{InsertSet}{SearchOrdered}
        \linkoriginal{SearchOrdered}{SearchList}
        \linkmethod{InsertOrdered}{SortOrdered}
        \linkself{SortOrdered}
        \caption{{{List}, {Ordered}, {Set}}}\label{ex:tictactoe4}
    \end{subfigure}
    \caption[Drawing with tables and TikZ]{Drawing with tables and TikZ.}\label{ex:tictactoe}
\end{figure}

\section{This is a section}

For citing, put entries in the BibTeX file (one example is there) and use the following to reference the authors directly:~\cite{Heradio2016Bibliometric} conducted a \gls{slr} (this is an acronym defined in the main file), or use the following for references in parenthesis~\citep{Heradio2016Bibliometric}.
~\cite{pix3d} ~\cite{Lim2013ParsingIO}
\subsection{This is a subsection}

\begin{table}[t]
    \caption[Short caption for list of tables]{Long caption for a table.}
    \label{tab:dummyTable}
    \centering \begin{adjustbox}{max width = \linewidth}
                   \begin{tabular}{lrcp{7cm}}
                       \toprule
                       Column1 & Column2 & Column3 & Column4 \\
                       \midrule
                       Left & Right & Centered & Left, fixed width \\
                       \multicolumn{4}{c}{Multi column} \\
                       \multirow{2}{*}{Multi row} & X & & \\
                       & Y & & \\
                       \bottomrule
                   \end{tabular}
    \end{adjustbox}
\end{table}

Numerated list
\begin{enumerate}
    \item One
    \item Two
    \item Three
\end{enumerate}

Bullet list
\begin{itemize}
    \item One
    \item Two
    \item Three
\end{itemize}