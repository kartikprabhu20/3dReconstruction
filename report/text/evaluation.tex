\chapter{\iftoggle{german}{Evaluierung}{Evaluation}}\label{ch:evaluation}

\section{Domain gaps - Qualitative}
Visualize domain gap between S2R:3D-FREE and real image(pix3d)

Visualize domain gap between other synthetic dataset like 3dfront, scenenet
tsne

\begin{figure}
    \centering
    \includegraphics[width=\linewidth]{/Users/apple/OVGU/Thesis/code/3dReconstruction/out/images/tsne/photorealisit_dataset_2}
    \caption{T-SNE visualisation for images from various photo-realistic synthetic dataset. Pix3d and S2R3DFREE is highlighted with bolder colors.}
    \label{fig:photorealistic tsne}
\end{figure}


\section{Domain gaps - Quantitative}
Maximum Mean Discrepancy
Kullback-Leibler divergence

\section{Performance}

Performance of model trained on synthetic dataset by testing model with real dataset(pix3d)
Qualitative: Voxel output comparisons
Quantitative: IOU, Dice

\section{Domain shift technique}

Compare the differences in domain shift of models learnt on S2R:3D-FREE and a traditional domain shift learning.

\section{Ablation study on chairs}
Check if randomising textures of chair and background helps improve performance.  Example: Create dataset with constant room texture vs randomising texture,
Check if lighting helps (indoor, outdoor). Example: Create a dataset with constant lighting(only outdoor lighting) vs indoor lighting.
Iou per category: as in https://arxiv.org/pdf/1905.03678.pdf
